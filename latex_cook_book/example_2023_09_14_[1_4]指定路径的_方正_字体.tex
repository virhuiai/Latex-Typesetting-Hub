\PassOptionsToPackage{AutoFakeBold=true,AutoFakeSlant=true}{xeCJK}
\documentclass{ctexart}
%  \documentclass[a4paper,openany]{book}

% \setCJKmainfont{ZhuqueFangsong}[
% Path=/Users/virhuiai/hlProjects/Latex-Typesetting-Hub/font/朱雀仿宋/,
% Extension      = .ttf,
% UprightFont    = *-Regular,
% BoldFont       = *-Regular,
% % ItalicFont
% % BoldItalicFont

% % BoldFont = ⟨font name⟩ 
% % ItalicFont = ⟨font name⟩ 
% % BoldItalicFont = ⟨font name⟩ 
% % SlantedFont = ⟨font name⟩ 
% % BoldSlantedFont = ⟨font name⟩ 
% % SwashFont = ⟨font name⟩ 
% % BoldSwashFont = ⟨font name⟩ 
% % SmallCapsFont = ⟨font name⟩ 
% % UprightFont = ⟨font name⟩
% ]



\setCJKmainfont[Path=/Users/virhuiai/hlProjects/Latex-Typesetting-Hub/font/方正/]{FangZhengShuSong-GBK-1.ttf}
\setCJKsansfont[Path=/Users/virhuiai/hlProjects/Latex-Typesetting-Hub/font/方正/]{FangZhengHeiTi-GBK-1.ttf}
\setCJKmonofont[Path=/Users/virhuiai/hlProjects/Latex-Typesetting-Hub/font/方正/]{FangZhengFangSong-GBK-1.ttf}
% 我们首先设置了三种主要的字体:正文 (mainfont)、无衬线字体 (sansfont) 和等宽字体 (monofont)。
\begin{document}

% \newfontfamily\myfont{方正仿宋_GBK}


\section{mainfont}

\newCJKfontfamily\KaiTi[Path=/Users/virhuiai/hlProjects/Latex-Typesetting-Hub/font/方正/]{FangZhengKaiTi-GBK-1.ttf}
\newCJKfontfamily\FangSong[Path=/Users/virhuiai/hlProjects/Latex-Typesetting-Hub/font/方正/]{FangZhengFangSong-GBK-1.ttf}
\newCJKfontfamily\HeiTi[Path=/Users/virhuiai/hlProjects/Latex-Typesetting-Hub/font/方正/]{FangZhengHeiTi-GBK-1.ttf}
\newCJKfontfamily\ShuSong[Path=/Users/virhuiai/hlProjects/Latex-Typesetting-Hub/font/方正/]{FangZhengShuSong-GBK-1.ttf}
\newCJKfontfamily\FZJiaGuWen[Path=/Users/virhuiai/hlProjects/Latex-Typesetting-Hub/font/方正/]{方正甲骨文.ttf}


% Hello,world!
你好,世界!

% \myfont 
你好,世界!

\IfFontExistsTF{/Users/virhuiai/hlProjects/Latex-Typesetting-Hub/font/澳声通拼音文楷/ToneOZ-Pinyin-WenKai-Light.ttf}{T}{F}

{\KaiTi 这是一段使用方正楷体的文本。}

{\FangSong 这是一段使用方正仿宋的文本。}

{\HeiTi 这是一段使用方正黑体的文本。}

{\ShuSong 这是一段使用方正书宋的文本。}

{\FZJiaGuWen 这是一段使用方正甲骨文的文本。}



% \section{sansfont}

% {\sf 
% Hello,world!
% 你好,世界!
% }

% \section{monofont}

% {\tt
% Hello,world!
% 你好,世界!
% }

\end{document}