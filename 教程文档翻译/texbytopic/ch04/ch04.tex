
% %%%% end of input file [char]

% %\InputFile:fontfam
% %%%% this is input file [fontfam]
% %\subject[font] Fonts
% \endofchapter
% \chapter{Fonts\\字体}\label{font}

% In text mode \TeX\ takes characters from a `current font'.
% \term fonts\par
% This chapter describes how fonts are identified to \TeX,
% and what attributes a font can have.

% 在文本模式下,\TeX\ 从“当前字体”中获取字符。
% 本章介绍了如何将字体标识给 \TeX,以及字体可能具有的属性。


% \begin{inventory}
% \item [\cs{font}] 
%       Declare the identifying control sequence of a font.

%       声明字体的标识控制序列。
% \item [\cs{fontname}] 
%       The external name of a font.

%       字体的外部名称。
% \item [\cs{nullfont}] 
%       Name of an empty font that \TeX\ uses in emergencies.

%       \TeX\ 在紧急情况下使用的空字体的名称。
% \item [\cs{hyphenchar}] 
%       Number of the hyphen character of a font.

%       字体的连字符字符的编号。
% \item [\cs{defaulthyphenchar}] 
%       Value of \cs{hyphenchar} when a font is loaded.
%       Plain \TeX\ default:~\verb>`\->.

%       在加载字体时 \cs{hyphenchar} 的值。
% Plain \TeX\ 的默认值为~\verb>`->。
% \item [\cs{fontdimen}] 
%       Access various parameters of fonts.

%       访问字体的各种参数。
% \item [\cs{char47}]
%       Italic correction.

%       斜体校正。
% \item [\cs{noboundary}] 
%       Omit implicit boundary character.

%       忽略隐式边界字符。
% \end{inventory}



% %\point Fonts
% \section{Fonts\\字体}

% In  \TeX\ terminology a font is the set of characters that
% is contained in one external font file. 
% During processing, \TeX\ decides from
% what font a character should be taken. This decision is
% taken separately for text mode and math mode.

% 在 \TeX\ 的术语中,字体是包含在一个外部字体文件中的字符集合。
% 在处理过程中,\TeX\ 决定从哪个字体中取出一个字符。这个决定针对文本模式和数学模式分别进行。


% When \TeX\ is processing ordinary text, characters are taken
% from the `current font'. 
% External font file names are coupled to  control sequences
% by   statements such as

% 当 \TeX\ 处理普通文本时,字符是从“当前字体”中取出的。
% 外部字体文件名与控制序列相关联,类似于以下语句:

% \begin{verbatim}
% \font\MyFont=myfont10
% \end{verbatim}
% which makes \TeX\ load the file \n{myfont10.tfm}.
% Switching the current font to the font described in that file
% is then done by

% 这使得 \TeX\ 加载文件 \n{myfont10.tfm}。
% 然后,可以通过以下方式将当前字体切换为该文件描述的字体:

% \begin{verbatim}
% \MyFont
% \end{verbatim}
% The status of the current font
% can be queried: the sequence 

% 可以查询当前字体的状态:序列
% \begin{verbatim}
% \the\font
% \end{verbatim}
% produces the control sequence for the current font.

% 会生成当前字体的控制序列。

% Math mode completely ignores the current font. Instead
% it looks  at the `current family', which can contain
% three fonts: one for text style, one for script style,
% and one for scriptscript style. This is treated
% in Chapter~\ref{mathchar}.
% \awp

% 数学模式完全忽略当前字体。相反,它关注的是“当前族”,其中可以包含三种字体:文本样式、上标样式和次上标样式的字体。这在第~\ref{mathchar}~章中讨论。

% See \cite{S} for a consistent terminology of fonts and typefaces.

% 有关字体和字型的一致术语,请参阅 \cite{S}。

% With `virtual fonts' (see~\cite{K:virt}) it is possible that
% what looks like one font to \TeX\ resides in more than
% one physical font file.
% \alt
% See further page~\pageref{virtual:fonts}.

% 使用“虚拟字体”(参见~\cite{K:virt})可以实现以下情况:在 \TeX\ 看来,一个字体可能存在于多个物理字体文件中。

% 请参阅第~\pageref{virtual:fonts}~页上的进一步内容。


% %\point Font declaration
% \section{Font declaration\\字体声明}

% Somewhere during a run of \TeX\ or \IniTeX\ 
% \cstoidx font\par
% the coupling between an internal identifying control sequence
% and the external file name of a font has to be made.
% The syntax of the command for this is

% 在 \TeX\ 或 \IniTeX\ 运行期间的某个位置,需要将一个内部标识的控制序列与字体的外部文件名进行关联。
% 此命令的语法如下:
% \begin{disp}\cs{font}\gr{control sequence}\gr{equals}%
% \gr{file name}\gr{at clause}\end{disp} 
% where 其中,
% \begin{disp}\gr{at clause} $\longrightarrow$ \n{at} \gr{dimen}
% $|$ \n{scaled} \gr{number} $|$ \gr{optional spaces}\end{disp}
% Font declarations are local to a group.

% 字体声明在一个组内生效。


% By the \gr{at clause} the user specifies that some
% magnified version of the font is wanted. The \gr{at clause} comes
% in two forms: if the font is given \n{scaled}~{\italic f\/} \TeX\
% multiplies all its font dimensions for that font by~$f/1000$; 
% if the font
% has a design size~{\italic d\/}\n{pt} and 
% the \gr{at clause} is \n{at}~{\italic p\/}\n{pt}
% \TeX\ multiplies all font data by~$p/d$.
% The presence of an \gr{at clause} makes no difference for
% the external font file (the \n{.tfm} file)
% that \TeX\ reads for the font; it just multiplies
% the font dimensions by a constant.

% 通过 \gr{at clause},用户可以指定要使用字体的某个放大版本。
% \gr{at clause} 有两种形式:如果字体给定了 \n{scaled}{\italic f/},\TeX\ 会将该字体的所有字体尺寸乘以 $f/1000$;
% 如果字体具有设计尺寸{\italic d/}\n{pt},并且 \gr{at clause} 为 \n{at}~{\italic p/}\n{pt},
% \TeX\ 会将所有字体数据乘以 $p/d$。
% \gr{at clause} 的存在对于 \TeX\ 读取的外部字体文件(\n{.tfm} 文件)没有影响;它只是将字体尺寸乘以一个常数。


% After such a font declaration, using the defined control sequence
% will set the current font to the font of the 
% control sequence.

% 在这样的字体声明之后,使用所定义的控制序列将设置当前字体为该控制序列所对应的字体。
% %\spoint Fonts and \n{tfm} files
% \subsection{Fonts and \n{tfm} files\\字体和 \n{tfm} 文件}

% The external file needed for the font is a \n{tfm} 
% (\TeX\ font metrics) file,
% which is taken independent of any  \gr{at clause}
% in the \cs{font} declaration. If the \n{tfm}
% file has been loaded already (for instance by \IniTeX\
% when it constructed the format),
% an assignment of that font file can be reexecuted
% without needing recourse to the \n{tfm} file.

% 字体所需的外部文件是一个 \n{tfm}(\TeX 字体度量)文件,它与 \cs{font} 声明中的任何 \gr{at clause} 无关。如果 \n{tfm} 文件已经被加载(例如由 \IniTeX 在构建格式时加载),则可以重新执行该字体文件的赋值,而无需再次访问 \n{tfm} 文件。


% Font design sizes are given in the font metrics files.
% The \n{cmr10} font, for instance, has a design size
% of 10~point. However, there is not much in the font
% that actually has a size of 10~points: the opening and closing
% parentheses are two examples, but capital
% letters are considerably smaller.

% 字体设计尺寸在字体度量文件中给出。例如,\n{cmr10} 字体的设计尺寸为 10 点。然而,在字体中实际上没有太多尺寸为 10 点的内容:开闭括号就是其中的两个例子,但大写字母要小得多。

