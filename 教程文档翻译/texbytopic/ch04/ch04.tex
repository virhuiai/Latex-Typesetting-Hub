
% %%%% end of input file [char]

% %\InputFile:fontfam
% %%%% this is input file [fontfam]
% %\subject[font] Fonts
% \endofchapter
% \chapter{Fonts\\字体}\label{font}

% In text mode \TeX\ takes characters from a `current font'.
% \term fonts\par
% This chapter describes how fonts are identified to \TeX,
% and what attributes a font can have.

% 在文本模式下,\TeX\ 从“当前字体”中获取字符。
% 本章介绍了如何将字体标识给 \TeX,以及字体可能具有的属性。


% \begin{inventory}
% \item [\cs{font}] 
%       Declare the identifying control sequence of a font.

%       声明字体的标识控制序列。
% \item [\cs{fontname}] 
%       The external name of a font.

%       字体的外部名称。
% \item [\cs{nullfont}] 
%       Name of an empty font that \TeX\ uses in emergencies.

%       \TeX\ 在紧急情况下使用的空字体的名称。
% \item [\cs{hyphenchar}] 
%       Number of the hyphen character of a font.

%       字体的连字符字符的编号。
% \item [\cs{defaulthyphenchar}] 
%       Value of \cs{hyphenchar} when a font is loaded.
%       Plain \TeX\ default:~\verb>`\->.

%       在加载字体时 \cs{hyphenchar} 的值。
% Plain \TeX\ 的默认值为~\verb>`->。
% \item [\cs{fontdimen}] 
%       Access various parameters of fonts.

%       访问字体的各种参数。
% \item [\cs{char47}]
%       Italic correction.

%       斜体校正。
% \item [\cs{noboundary}] 
%       Omit implicit boundary character.

%       忽略隐式边界字符。
% \end{inventory}



% %\point Fonts
% \section{Fonts\\字体}

% In  \TeX\ terminology a font is the set of characters that
% is contained in one external font file. 
% During processing, \TeX\ decides from
% what font a character should be taken. This decision is
% taken separately for text mode and math mode.

% 在 \TeX\ 的术语中,字体是包含在一个外部字体文件中的字符集合。
% 在处理过程中,\TeX\ 决定从哪个字体中取出一个字符。这个决定针对文本模式和数学模式分别进行。


% When \TeX\ is processing ordinary text, characters are taken
% from the `current font'. 
% External font file names are coupled to  control sequences
% by   statements such as

% 当 \TeX\ 处理普通文本时,字符是从“当前字体”中取出的。
% 外部字体文件名与控制序列相关联,类似于以下语句:

% \begin{verbatim}
% \font\MyFont=myfont10
% \end{verbatim}
% which makes \TeX\ load the file \n{myfont10.tfm}.
% Switching the current font to the font described in that file
% is then done by

% 这使得 \TeX\ 加载文件 \n{myfont10.tfm}。
% 然后,可以通过以下方式将当前字体切换为该文件描述的字体:

% \begin{verbatim}
% \MyFont
% \end{verbatim}
% The status of the current font
% can be queried: the sequence 

% 可以查询当前字体的状态:序列
% \begin{verbatim}
% \the\font
% \end{verbatim}
% produces the control sequence for the current font.

% 会生成当前字体的控制序列。

% Math mode completely ignores the current font. Instead
% it looks  at the `current family', which can contain
% three fonts: one for text style, one for script style,
% and one for scriptscript style. This is treated
% in Chapter~\ref{mathchar}.
% \awp

% 数学模式完全忽略当前字体。相反,它关注的是“当前族”,其中可以包含三种字体:文本样式、上标样式和次上标样式的字体。这在第~\ref{mathchar}~章中讨论。

% See \cite{S} for a consistent terminology of fonts and typefaces.

% 有关字体和字型的一致术语,请参阅 \cite{S}。

% With `virtual fonts' (see~\cite{K:virt}) it is possible that
% what looks like one font to \TeX\ resides in more than
% one physical font file.
% \alt
% See further page~\pageref{virtual:fonts}.

% 使用“虚拟字体”(参见~\cite{K:virt})可以实现以下情况:在 \TeX\ 看来,一个字体可能存在于多个物理字体文件中。

% 请参阅第~\pageref{virtual:fonts}~页上的进一步内容。


% %\point Font declaration
% \section{Font declaration\\字体声明}

% Somewhere during a run of \TeX\ or \IniTeX\ 
% \cstoidx font\par
% the coupling between an internal identifying control sequence
% and the external file name of a font has to be made.
% The syntax of the command for this is

% 在 \TeX\ 或 \IniTeX\ 运行期间的某个位置,需要将一个内部标识的控制序列与字体的外部文件名进行关联。
% 此命令的语法如下:
% \begin{disp}\cs{font}\gr{control sequence}\gr{equals}%
% \gr{file name}\gr{at clause}\end{disp} 
% where 其中,
% \begin{disp}\gr{at clause} $\longrightarrow$ \n{at} \gr{dimen}
% $|$ \n{scaled} \gr{number} $|$ \gr{optional spaces}\end{disp}
% Font declarations are local to a group.

% 字体声明在一个组内生效。


% By the \gr{at clause} the user specifies that some
% magnified version of the font is wanted. The \gr{at clause} comes
% in two forms: if the font is given \n{scaled}~{\italic f\/} \TeX\
% multiplies all its font dimensions for that font by~$f/1000$; 
% if the font
% has a design size~{\italic d\/}\n{pt} and 
% the \gr{at clause} is \n{at}~{\italic p\/}\n{pt}
% \TeX\ multiplies all font data by~$p/d$.
% The presence of an \gr{at clause} makes no difference for
% the external font file (the \n{.tfm} file)
% that \TeX\ reads for the font; it just multiplies
% the font dimensions by a constant.

% 通过 \gr{at clause},用户可以指定要使用字体的某个放大版本。
% \gr{at clause} 有两种形式:如果字体给定了 \n{scaled}{\italic f/},\TeX\ 会将该字体的所有字体尺寸乘以 $f/1000$;
% 如果字体具有设计尺寸{\italic d/}\n{pt},并且 \gr{at clause} 为 \n{at}~{\italic p/}\n{pt},
% \TeX\ 会将所有字体数据乘以 $p/d$。
% \gr{at clause} 的存在对于 \TeX\ 读取的外部字体文件(\n{.tfm} 文件)没有影响;它只是将字体尺寸乘以一个常数。


% After such a font declaration, using the defined control sequence
% will set the current font to the font of the 
% control sequence.

% 在这样的字体声明之后,使用所定义的控制序列将设置当前字体为该控制序列所对应的字体。
% %\spoint Fonts and \n{tfm} files
% \subsection{Fonts and \n{tfm} files\\字体和 \n{tfm} 文件}

% The external file needed for the font is a \n{tfm} 
% (\TeX\ font metrics) file,
% which is taken independent of any  \gr{at clause}
% in the \cs{font} declaration. If the \n{tfm}
% file has been loaded already (for instance by \IniTeX\
% when it constructed the format),
% an assignment of that font file can be reexecuted
% without needing recourse to the \n{tfm} file.

% 字体所需的外部文件是一个 \n{tfm}(\TeX 字体度量)文件,它与 \cs{font} 声明中的任何 \gr{at clause} 无关。如果 \n{tfm} 文件已经被加载(例如由 \IniTeX 在构建格式时加载),则可以重新执行该字体文件的赋值,而无需再次访问 \n{tfm} 文件。


% Font design sizes are given in the font metrics files.
% The \n{cmr10} font, for instance, has a design size
% of 10~point. However, there is not much in the font
% that actually has a size of 10~points: the opening and closing
% parentheses are two examples, but capital
% letters are considerably smaller.

% 字体设计尺寸在字体度量文件中给出。例如,\n{cmr10} 字体的设计尺寸为 10 点。然而,在字体中实际上没有太多尺寸为 10 点的内容:开闭括号就是其中的两个例子,但大写字母要小得多。


% %\spoint Querying the current font and font names
% \subsection{Querying the current font and font names\\查询当前字体和字体名称}

% It was already mentioned above that the control sequence
% which set the current font can be retrieved by the
% command \verb>\the\font>. This is a special case of

% 上面已经提到,可以通过命令 \verb>\the\font> 获取设置当前字体的控制序列。这是
% \begin{Disp}\cs{the}\gr{font}\end{Disp} where  的一个特例,其中
% \begin{disp}\gr{font} $\longrightarrow$
% \cs{font} $|$ \gr{fontdef token} $|$ \gr{family member}\nl
% \gr{family member} $\longrightarrow$ 
% \gr{font range}\gr{4-bit number}\nl
% \gr{font range} $\longrightarrow$ 
% \cs{textfont} $|$ \cs{scriptfont} $|$ \cs{scriptscriptfont}\end{disp}
% \awp
% A \gr{fontdef token} is a control sequence defined by \cs{font},
% or the predefined control sequence \cs{nullfont}.
% The concept of \gr{family member} is only 
% relevant in math mode.

% \gr{fontdef token} 是由 \cs{font} 定义的控制序列,或者是预定义的控制序列 \cs{nullfont}。在数学模式中,\gr{family member} 的概念只有在数学模式中才相关。

% Also, the 
% \cstoidx fontname\par
% external name of fonts can be retrieved:

% 此外,还可以获取字体的外部名称:
% \begin{Disp}\cs{fontname}\gr{font}\end{Disp}
% gives a sequence of character tokens of category~12
% (but space characters get category~10) that spells the font file
% name, plus an \gr{at clause} if applicable.

% 返回一个类别码为 12 的字符记号序列(但空格字符的类别码为 10),拼写字体文件名,如果适用,则包含一个 \gr{at clause}。

% \begin{example} After 在
% \begin{verbatim}
% \font\tenroman=cmr10 \tenroman
% \end{verbatim}
% the calls
% \verb>\the\font> and \verb>\the\tenroman> both give \cs{tenroman}.
% The call \verb>\fontname\tenroman> gives \n{cmr10}.

% 之后,调用 \verb>\the\font> 和 \verb>\the\tenroman> 都给出 \cs{tenroman}。调用 \verb>\fontname\tenroman> 给出 \n{cmr10}。
% \end{example}



% %\spoint \cs{nullfont}
% \subsection{\cs{nullfont}\\\cs{nullfont}}

% \TeX\ always knows a font that has no characters: the \csidx{nullfont}.
% If no font has been specified, or if in math mode a family member
% is needed that has not been specified, 
% \TeX\ will take its characters from the nullfont.
% This control sequence qualifies as a \gr{fontdef token}:
% it acts like any other control sequence that stands for a font;
% it just does not have an associated \n{tfm} file.

% \TeX\ 总是知道一个没有字符的字体:\csidx{nullfont}。
% 如果没有指定字体,或者在数学模式下需要一个未指定的字族成员,
% \TeX\ 将从 nullfont 获取其字符。
% 这个控制序列符合 \gr{fontdef token}:
% 它的行为类似于代表字体的任何其他控制序列;
% 只是它没有关联的 \n{tfm} 文件。

% %\point Font information
% \section{Font information\\字体信息}

% During a run of \TeX\ the main information needed about the
% \term \n{tfm} files\par
% font consists of the dimensions of the characters.
% \TeX\ finds these in the font metrics files, which usually have
% extension \n{.tfm}. Such files
% contain 

% 在 \TeX\ 的运行过程中,有关字体的主要信息是字符的尺寸。
% \TeX\ 从字体度量文件中获取这些信息,这些文件通常具有扩展名 \n{.tfm}。
% 这些文件包含以下内容:
% \begin{itemize} \item global information: the \cs{fontdimen}
% parameters, and some other information,

% 全局信息:\cs{fontdimen} 参数和其他一些信息,
% \item dimensions and the italic corrections of characters, and
% \altt 

% 字符的尺寸和斜体校正,
% \item ligature and kerning programs for characters.

% 字符的连字和紧排程序。
% 	\end{itemize}
% Also, the design size of a font is specified in the \n{tfm} file;
% see above. The definition of the \n{tfm} format can be found
% in~\cite{Knuth:TeXprogram}.

% 此外,字体的设计尺寸也在 \n{tfm} 文件中指定;
% 请参见上文。关于 \n{tfm} 格式的定义可以在~\cite{Knuth:TeXprogram} 中找到。


% %\spoint[font:dims] Font dimensions
% \subsection{Font dimensions\\字体尺寸}
% \label{font:dims}

% Text fonts need to have at least seven \csidx{fontdimen} parameters
% (but \TeX\ will take zero for unspecified parameters);
% \term font! dimensions\par
% math symbol and math extension fonts have more
% (see page~\pageref{fam23:fontdims}).
% For text fonts the minimal set of seven comprises the following:

% 文本字体至少需要有七个 \csidx{fontdimen} 参数(但对于未指定参数的情况,\TeX\ 会取零);
% 数学符号字体和数学扩展字体有更多参数(参见第~\pageref{fam23:fontdims}页)。
% 对于文本字体,最小的七个参数包括以下内容:
% \begin{enumerate} \item the slant per point; this dimension is used
%     for the proper horizontal positioning of accents;
% \awp

% 每点的倾斜度;这个维度用于正确水平定位重音符号;
% \item the interword space: this is used unless the user
%     specifies an explicit \cs{spaceskip};
%     see Chapter~\ref{space};

%     单词间距:如果用户没有指定显式的 \cs{spaceskip},将使用此间距;请参见第\ref{space}~章;
% \item interword stretch: the stretch component of the interword
%     space;

%     单词间距的可伸展部分;
% \item interword shrink: the shrink component of
%     the interword space;

%     单词间距的可收缩部分;
% \item the x-height: the value of
%     the \gr{internal unit} \n{ex}, which is usually about the
%     height of the lowercase letter~`x'; 

%     小写字母“x”的高度:\gr{内部单位} \n{ex} 的值,通常大约等于小写字母“x”的高度;
% \item the quad width:
%     the value of the \gr{internal unit} \n{em}, which is
%     approximately the width of the capital letter~`M'; and

%     四分宽度:\gr{内部单位} \n{em} 的值,大致等于大写字母“M”的宽度;
% \item the extra space: the space added to the interword space
% at the end of sentences (that is, when \cs{spacefactor}${}\geq2000$)
% unless the user specifies an explicit \cs{x\-space\-skip}.

% 额外空格:在句子末尾(即当 \cs{spacefactor}${}\geq2000$ 时)加到单词间距的空格,除非用户指定了显式的 \cs{x-space-skip}。
% \end{enumerate}

% Parameters 1 and~5 are purely information about the font
% and there is no point in varying them.
% The values of other parameters can be changed in order to
% adjust spacing; see Chapter~\ref{space} for examples
% of changing parameters 2, 3, 4, and~7.

% 参数 1 和 5 纯粹是字体的信息,没有改变它们的意义。
% 可以改变其他参数的值以调整间距;请参见第~\ref{space}~章中更改参数 2、3、4 和 7 的示例。

% Font dimensions can be altered in a \gr{font assignment},
% which is a \gr{global assignment} (see page~\pageref{global:assign}):

% 字体维度可以在\gr{字体赋值}中进行更改,这是一个\gr{全局赋值}(参见第~\pageref{global:assign}~页):
% \begin{Disp}\cs{fontdimen}\gr{number}\gr{font}\gr{equals}\gr{dimen}
% \end{Disp} See above for the definition of \gr{font}.

% \gr{字体}的定义请参见上文。

