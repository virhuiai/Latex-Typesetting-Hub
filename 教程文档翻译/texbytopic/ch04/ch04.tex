
% %%%% end of input file [char]

% %\InputFile:fontfam
% %%%% this is input file [fontfam]
% %\subject[font] Fonts
% \endofchapter
% \chapter{Fonts\\字体}\label{font}

% In text mode \TeX\ takes characters from a `current font'.
% \term fonts\par
% This chapter describes how fonts are identified to \TeX,
% and what attributes a font can have.

% 在文本模式下,\TeX\ 从“当前字体”中获取字符。
% 本章介绍了如何将字体标识给 \TeX,以及字体可能具有的属性。


% \begin{inventory}
% \item [\cs{font}] 
%       Declare the identifying control sequence of a font.

%       声明字体的标识控制序列。
% \item [\cs{fontname}] 
%       The external name of a font.

%       字体的外部名称。
% \item [\cs{nullfont}] 
%       Name of an empty font that \TeX\ uses in emergencies.

%       \TeX\ 在紧急情况下使用的空字体的名称。
% \item [\cs{hyphenchar}] 
%       Number of the hyphen character of a font.

%       字体的连字符字符的编号。
% \item [\cs{defaulthyphenchar}] 
%       Value of \cs{hyphenchar} when a font is loaded.
%       Plain \TeX\ default:~\verb>`\->.

%       在加载字体时 \cs{hyphenchar} 的值。
% Plain \TeX\ 的默认值为~\verb>`->。
% \item [\cs{fontdimen}] 
%       Access various parameters of fonts.

%       访问字体的各种参数。
% \item [\cs{char47}]
%       Italic correction.

%       斜体校正。
% \item [\cs{noboundary}] 
%       Omit implicit boundary character.

%       忽略隐式边界字符。
% \end{inventory}

