
%\InputFile:rules
%%%% this is input file [rules]
%\subject[rules]  Rules and Leaders
% \endofchapter
\chapter{Rules and Leaders\\边框和引导线}\label{rules}

Rules and leaders are two ways of getting \TeX\ to draw a line.
Leaders are more general than rules: they can also fill
available space with copies of a certain box. This chapter
explain how rules and leaders work, and how they interact with modes.

边框和引导线是 \TeX\ 绘制线条的两种方式。
引导线比边框更通用:它们还可以使用某个盒子的副本填充可用空间。
本章将解释边框和引导线的工作原理,以及它们与模式的交互方式。
\begin{inventory}
\item [\cs{hrule}] 
      Rule that spreads in horizontal direction.

      水平方向展开的边框。
\item [\cs{vrule}] 
      Rule that spreads in vertical direction.

      垂直方向展开的边框。
\item [\cs{leaders}] 
      Fill a specified amount of space with a rule or copies of box.

      使用边框或盒子的副本来填充指定数量的空间。
\item [\cs{cleaders}] 
      Like \verb=\leaders=, but with box leaders 
      any excess space is split equally before and after the leaders.

      类似于 \verb=\leaders=,但如果使用盒子作为引导线,则多余的空间会在引导线前后平均分配。
\item [\cs{xleaders}] 
      Like \verb=\leaders=, but with box leaders any excess space is 
      spread equally before, after, and between the boxes.

      类似于 \verb=\leaders=,但如果使用盒子作为引导线,则多余的空间会在引导线前、后和盒子之间均匀分布。
\end{inventory}

%\point Rules
\section{Rules\\边框}

\TeX's rule commands give
\term rules\par
rectangular black patches with horizontal and vertical sides.
Most of the times, a rule command will give output that
looks like a rule, but~\hbox{\vrule height 1.5ex width 1.5ex}
can also be produced by a rule.

\TeX\ 的边框命令生成具有水平和垂直边的矩形黑色块。
大多数情况下,边框命令会生成看起来像边框的输出,
但~\hbox{\vrule height 1.5ex width 1.5ex} 也可以由边框生成。

\TeX\ has both horizontal and vertical rules, 
but the names do not necessarily imply anything about the shape.
They do, however, imply something about modes:
an \csidx{hrule} command can only be used in vertical mode,
and a \csidx{vrule} only in horizontal mode.
In fact, an \cs{hrule} is a \gr{vertical command}, and a \cs{vrule}
is a \gr{horizontal command}, so \TeX\ may change
modes when encountering these commands.

\TeX\ 同时具有水平和垂直的边框命令,
但是名称并不一定对形状有任何暗示。
然而,它们确实对模式有所暗示:
\csidx{hrule} 命令只能在垂直模式中使用,
而 \csidx{vrule} 命令只能在水平模式中使用。
实际上,\cs{hrule} 是一个 \gr{vertical command},\cs{vrule} 是一个 \gr{horizontal command},
因此当遇到这些命令时,\TeX\ 可能会改变模式。

Why then is a \cs{vrule} called a {\em vertical\/} rule?
The reason is that a \cs{vrule} can expand arbitrarily
far in the vertical direction: if its height and depth are not
specified explicitly it will take as much room as its
surroundings allow\altt.

为什么 \cs{vrule} 被称为{\em 垂直}标线呢?原因是 \cs{vrule} 可以在垂直方向上无限扩展:如果没有明确指定其高度和深度,它将占据其所在环境允许的所有空间\altt。

\begin{example}
\begin{verbatim}
\hbox{\vrule\ text \vrule}
\end{verbatim}
looks like 

的效果如下:\begin{disp}\leavevmode\hbox{\vrule\ text \vrule}\end{disp}
and 
\begin{verbatim}
\hbox{\vrule\ A gogo! \vrule}
\end{verbatim}
looks like

的效果如下:\begin{disp}\leavevmode\hbox{\vrule\ A gogo! \vrule}\end{disp}
\end{example}

For the \cs{hrule} command a similar statement is true:
a horizontal rule can spread to assume the width of
its surroundings. Thus 

对于 \cs{hrule} 命令,类似的情况也成立:水平标线可以展开以占据其所在环境的宽度。因此,
\begin{verbatim}
\vbox{\hbox{One line of text}\hrule}
\end{verbatim}
looks like
\begin{disp}\leavevmode\vtop{\hbox{One line of text}\hrule}\end{disp}


%\spoint Rule dimensions
\subsection{Rule dimensions\\标尺尺}

Horizontal and vertical rules have a default thickness:

水平和垂直标线有一个默认的粗细:
\begin{Disp} \cs{hrule}\quad is the same as\quad \verb-\hrule height.4pt depth0pt-
\end{Disp}
and 
\begin{Disp} \cs{vrule}\quad is the same as\quad \verb-\vrule width.4pt- \end{Disp}
and if the remaining dimension remains unspecified, the rule
extends in that direction to fill the enclosing box.

如果剩余的维度未指定,标线将在该方向上扩展以填充封闭的盒子。

Here is the formal specification of how to indicate rule sizes:

以下是关于如何指定规则大小的正式规范:
\begin{disp}\gr{vertical rule} $\longrightarrow$ 
                        \cs{vrule}\gr{rule specification}\nl
     \gr{horizontal rule} $\longrightarrow$
                        \cs{hrule}\gr{rule specification}\nl
     \gr{rule specification} $\longrightarrow$
                        \gr{optional spaces} \nl \indent$|$
                        \gr{rule dimensions}\gr{rule specification}\nl
     \gr{rule dimension} $\longrightarrow$
                        \n{width}\gr{dimen} $|$ \n{height}\gr{dimen} $|$
                        \n{depth}\gr{dimen}
     \end{disp}
If a rule dimension is specified twice, the second instance
takes precedence over the first. This makes it possible
to override the default dimensions. For instance,
after
\alt
\howto Change the default dimensions of rules\par

如果规定了两次规则尺寸,第二次的规定将覆盖第一次的规定。这样可以覆盖默认的尺寸。
例如,在下面的代码之后:
\alt
\howto 更改规则的默认尺寸\par
\begin{verbatim}
\let\xhrule\hrule  \def\hrule{\xhrule height .8pt}
\end{verbatim}
the macro \cs{hrule} gives a horizontal rule
of double the original height, and it is still possible
with 

宏 \cs{hrule} 给出了原始高度的两倍的水平规则,仍然可以通过:
\begin{verbatim}
\hrule height 2pt
\end{verbatim}
to specify other heights.

来指定其他高度。

It is possible to specify all three dimensions; then

可以指定所有三个尺寸;因此,
\begin{verbatim}
\vrule height1ex depth0pt width1ex
\end{verbatim}
and
\begin{verbatim}
\hrule height1ex depth0pt width1ex
\end{verbatim}
look the same.
Still, each of them can be used only in the appropriate mode.

看起来是一样的。
然而,每个规则只能在适当的模式下使用。


%\point Leaders
\section{Leaders\\指引线}

Rules are intimately connected to modes, which makes it easy
\term leaders\par
to obtain some effects. For instance, a typical application
of a vertical rule looks like

规则与模式紧密相关,这使得很容易获得一些效果。例如,垂直规则的典型应用如下:
\begin{verbatim}
\hbox{\vrule width1pt\ Important text! \vrule width 1pt}
\end{verbatim}
which gives
它会生成: \begin{disp}\leavevmode\hbox{\vrule width 1pt\ Important text! 
                      \vrule width 1pt}\end{disp}
However, one might want to have a horizontal rule
in horizontal mode for effects such as

然而,有时我们可能希望在水平模式中使用水平规则,以获得效果,如:
\begin{disp}\leavevmode
\vbox{\hbox to 5cm{$\longleftarrow$\hfil 5cm\hfil$\longrightarrow$}
    \hbox to 5cm{from here\leaders\hrule\hfil to there}}\end{disp}
An \cs{hrule} can not be used in horizontal mode, and
a vertical rule will not spread automatically.

在水平模式下不能使用 \cs{hrule} 命令,垂直规则也不会自动扩展。

However, there is a way to use an \cs{hrule} command in
horizontal mode and a \cs{vrule} in vertical mode,
and that is with `leaders', so called because
they lead your eye across the page. 
A~leader command tells \TeX\
to fill a~specified space, in whatever mode it is in,
with as many copies of some box or rule specification
as are needed. For instance, the above example
was given as

然而,有一种方法可以在水平模式中使用 \cs{hrule} 命令,在垂直模式中使用 \cs{vrule} 命令,那就是使用`指引线'(leaders),因为它们引导你的目光穿过页面。
指引线命令告诉 \TeX\ 以所需的数量复制某个盒子或规则的规范,以填充指定的空间,无论它处于哪种模式中。例如,上面的例子可以表示为:

\begin{disp}\verb>\hbox to 5cm{from here\leaders\hrule\hfil to there}>\end{disp}
that is, with an \cs{hrule} that was allowed to stretch along
an \cs{hfil}.
Note that the leader was given a horizontal skip,
corresponding to the horizontal mode in which it appeared.

即,使用一个可以沿着 \cs{hfil} 伸展的 \cs{hrule}。
请注意,指引线给出的是一个水平的间距,对应于它所在的水平模式。

A general leader command looks like

通用的指引线命令的形式如下:
\begin{Disp} \gr{leaders}\gr{box or rule}%
      \gr{vertical/horizontal/mathematical skip}\end{Disp}
where \gr{leaders} is \cs{leaders}, \cs{cleaders}, 
or~\cs{xleaders}, a \gr{box~or~rule}
is a~\gr{box}, \cs{vrule}, or~\cs{hrule}, and the
lists of horizontal and vertical skips appear in Chapter~\ref{hvmode};
a~mathematical skip is either a horizontal skip or an~\cs{mskip}
(see page~\pageref{muglue}).
Leaders can thus be used in all three modes. Of course, the
appropriate kind of skip must be specified. 

其中,\gr{leaders} 可以是 \cs{leaders}、\cs{cleaders} 或 \cs{xleaders},\gr{box or rule} 是一个 \gr{box}、\cs{vrule} 或 \cs{hrule},水平和垂直间距的列表详见第~\ref{hvmode}章;数学间距则是水平间距或 \cs{mskip}(参见第\pageref{muglue}~页)。
因此,指引线可以在三种模式下使用。当然,必须指定适当类型的间距。

A horizontal (vertical) box containing leaders has at least
the height and depth (width) of the \gr{box~or~rule} used
in the leaders, even if, as can happen in the case of box leaders,
no actual leaders are placed.

包含指引线的水平(垂直)盒子至少具有指引线中使用的 \gr{box or rule} 的高度和深度(宽度),即使在盒子指引线中没有放置实际的指引线,这在盒子指引线的情况下可能发生。



%\spoint Rule leaders
\subsection{Rule leaders}

Rule leaders fill the specified amount of space with a rule
\term leaders !rule\par\cstoidx leaders\par
extending in the direction of the skip specified.
The other dimensions of the resulting rule leader
are determined by the sort of rule that is used:
either dimensions can be specified explicitly, or
the default values can be used.

规则指引(Rule leaders)使用一条规则来填充指定量的空间,规则的延伸方向由指定的间距决定。生成的规则指引的其他尺寸取决于所使用的规则类型:可以明确指定尺寸,也可以使用默认值。

For instance, 

例如,
\begin{verbatim}
\hbox{g\leaders\hrule\hskip20pt f}
\end{verbatim}
gives 生成\begin{disp}\leavevmode\hbox{g\leaders\hrule\hskip20pt f}\end{disp}
because a horizontal rule has a default height of~\n{.4pt}.
On the other hand,

因为水平规则的默认高度为~\n{.4pt}。
另一方面,
\begin{verbatim}
\hbox{g\leaders\vrule\hskip20pt f}
\end{verbatim}
gives 生成\begin{disp}\leavevmode\hbox{g\leaders\vrule\hskip20pt f}\end{disp}
because the height and depth of a vertical rule
by default fill the surrounding box.

因为垂直规则的高度和深度默认填充周围的盒子。

Spurious rule dimensions are ignored: in horizontal mode

无效的规则尺寸会被忽略:在水平模式下,
\begin{verbatim}
\leaders\hrule width 10pt \hskip 20pt
\end{verbatim}
is equivalent to
\begin{verbatim}
\leaders\hrule \hskip 20pt
\end{verbatim}

If the width or height-plus-depth
of either the skip or the box is negative, 
\TeX\ uses ordinary glue instead of leaders.

如果间距或盒子的宽度、高度加深度为负数,则 \TeX\ 使用普通粘连而不是指引。


%\spoint Box leaders
\subsection{Box leaders\\盒子引导线}

Box leaders fill the available spaces with copies of
a given box, instead of with a rule. 

盒子引导线使用给定的盒子的副本来填充可用空间,而不是使用边框。

\newbox\centerdot  \setbox\centerdot=\hbox{\hskip.7em.\hskip.7em}

For all of the following examples, assume that a box register
has been allocated:

对于以下所有示例,请假设已经分配了一个盒子寄存器:
\begin{verbatim}
\newbox\centerdot  \setbox\centerdot=\hbox{\hskip.7em.\hskip.7em}
\end{verbatim}
Now the output of 现在,下面的输出
\begin{verbatim}
\hbox to 8cm {here\leaders\copy\centerdot\hfil there}
\end{verbatim}
is 
\begin{disp}\leavevmode\hbox to 8cm {here\leaders\copy\centerdot\hfil there}
\end{disp} That is, copies of the box register fill up the
available space.

也就是说,盒子寄存器的副本填满了可用空间。

Dot leaders, as in the above example, are often used for
tables of contents. In such applications it is desirable that
dots on subsequent lines are vertically aligned.
The \cs{leaders} command does this automatically:

像上面的示例中的点引导线通常用于目录。
在这种应用中,希望后续行的点在垂直方向上对齐。
\cs{leaders} 命令会自动完成这个操作:
\begin{verbatim}
\hbox to 8cm {here\leaders\copy\centerdot\hfil there}
\hbox to 8cm {over here\leaders\copy\centerdot\hfil over there}
\end{verbatim}
gives \begin{disp}\leavevmode
\vtop{\hbox to 8cm {here\leaders\copy\centerdot\hfil there}
\hbox to 8cm {over here\leaders\copy\centerdot\hfil over there\strut}}
\end{disp}
The mechanism behind this is the following:
\TeX\ acts as if an infinite row of boxes starts (invisibly) at 
the left edge of the surrounding box, 
and the row of copies actually placed is 
merely the part of this row that is not obscured by
the other contents of the box.

背后的机制如下:
\TeX\ 会假定一个无限行的盒子(在不可见的情况下)从周围盒子的左边缘开始,
而实际放置的盒子行仅为不被盒子的其他内容遮挡的部分。

Stated differently, box leaders are a window on an infinite
row of boxes, and the row starts at the left edge of the
surrounding box. Consider the following example:

换句话说,盒子引导线是无限行盒子的一扇窗户,而该行从周围盒子的左边缘开始。
考虑以下示例:
\begin{verbatim}
\hbox to 8cm {\leaders\copy\centerdot\hfil}
\hbox to 8cm {word\leaders\copy\centerdot\hfil}
\end{verbatim}
which gives
\begin{disp}\leavevmode\vtop{\hbox to 8cm {\leaders\copy\centerdot\hfil}
\hbox to 8cm {word\leaders\copy\centerdot\hfil\strut}}\end{disp}
The row of leaders boxes becomes visible as soon as it
does not coincide with other material.

当行引导线与其他内容不重合时,行引导线成为可见。

The above discussion only talked about leaders in horizontal
mode. Leaders can equally well be placed in vertical mode;
for box leaders the `infinite row' then starts at the top
of the surrounding box.

上述讨论仅涉及水平模式中的引导线。
引导线同样可以放置在垂直模式中;
对于盒子引导线,这个“无限行”则从周围盒子的顶部开始。


%\spoint Evenly spaced leaders
\subsection{Evenly spaced leaders\\均匀间隔的引导线}

Aligning subsequent box leaders in the way described above
means that the white space before and after the
leaders will in general be different.
If vertical alignment is not
an issue it may be aesthetically more pleasing to have
the leaders evenly spaced.
The \csidx{cleaders} command is like \cs{leaders},
except that it splits excess space before and after the leaders
into two equal parts, centring the row of boxes in the
available space.

按照上面描述的方式对后续盒子引导线进行对齐意味着引导线之前和之后的空白通常是不同的。
如果垂直对齐不是问题,那么让引导线均匀间隔可能在美学上更令人愉悦。
\csidx{cleaders} 命令类似于 \cs{leaders},但它将多余的空间在引导线之前和之后均匀分割为两个相等的部分,
将盒子行居中放置在可用空间中。


\begin{example}\message{check verbatim indentation}
\begin{verbatim}
\hbox to 7.8cm {here\cleaders\copy\centerdot\hfil there}
\hbox to 7.8cm {here is\cleaders\copy\centerdot\hfil there}
\end{verbatim}
gives \begin{disp}\leavevmode\vbox{
\hbox to 7.8cm {here\cleaders\copy\centerdot\hfil there}
\hbox to 7.8cm {here is\cleaders\copy\centerdot\hfil there\strut}
}\end{disp}
The `expanding leaders' \csidx{xleaders} spread excess space evenly
between the boxes, with equal globs of glue before, after,
and in between leader boxes.

“扩展引导线” \csidx{xleaders} 将多余的空间均匀分布在盒子之间,
在引导线之前、之后和两个盒子之间使用相等的粘连。
\end{example}

\end{document}

\begin{example} \begin{verbatim}
\hbox to 7.8cm{here\hskip.7em
      \xleaders\copy\centerdot\hfil  \hskip.7em there}
\end{verbatim}
gives \begin{disp}\leavevmode
\hbox to 7.8cm {here\hskip.7em\xleaders\copy\centerdot\hfil\hskip.7em there}
\end{disp} Note that the glue in the leader box is balanced here
with explicit glue before and after the leaders;
leaving out these glue items, as in

注意,这里的引导线盒子中的粘连与引导线之前和之后的显式粘连平衡;
如果省略这些粘连项,如
\begin{verbatim}
\hbox to 7.8cm {here\xleaders\copy\centerdot\hfil there}
\end{verbatim}
gives \begin{disp}\leavevmode
\hbox to 7.8cm {here\xleaders\copy\centerdot\hfil there}
\end{disp}
which is clearly not what was intended.

显然不是预期的结果。
\end{example}


%\point Assorted remarks
\section{Assorted remarks\\其他说明}

%\spoint Rules and modes
\subsection{Rules and modes\\标线和模式}

Above it was explained how rules can only occur in the 
appropriate modes. Rules also influence mode-specific
quantities:
no baselineskip is added before rules in 
vertical mode. In order to prevent glue after rules,
\TeX\ sets \cs{prevdepth} to
\n{\hbox{-}1000pt}
(see Chapter~\ref{baseline}).
Similarly the \cs{spacefactor} is set to 1000 after a \cs{vrule}
in horizontal mode (see Chapter~\ref{line:break}).

前面已经解释过,标线只能出现在相应的模式中。标线也会影响模式特定的量:在垂直模式下,标线前不会添加基线间距。为了防止标线后的粘连,\TeX 会将 \cs{prevdepth} 设置为 \n{\hbox{-}1000pt}(见第~\ref{baseline} 章)。类似地,在水平模式下的 \cs{vrule} 后,\TeX 会将 \cs{spacefactor} 设置为 1000(见第~\ref{line:break} 章)。

%\spoint[par:leaders:end] Ending a paragraph with leaders
\subsection{Ending a paragraph with leaders\\以标线结束段落}
\label{par:leaders:end}

An attempt to simulate an \cs{hrule} at the end of a paragraph by

通过以下方式尝试在段落末尾模拟一个 \cs{hrule}:
\howto End a paragraph with leaders\par

\begin{verbatim}
\nobreak\leaders\hrule\hfill\par
\end{verbatim}
does not work. The reason for this is that \TeX\
performs an \cs{unskip} at the end of a paragraph,
which removes the leaders. Normally this \cs{unskip} removes
any space token inserted by the input processor after the
last line. Remedy: stick an \verb.\hbox{}. at the end of
the leaders.

这种方法行不通。原因是,在段落的末尾,\TeX 会执行 \cs{unskip},它会移除标线。通常情况下,\cs{unskip} 会移除输入处理器在最后一行后插入的任何空格记号。解决办法是,在标线的末尾添加一个 \verb.\hbox{}.。
%\spoint Leaders and box registers
\subsection{Leaders and box registers\\指引线和盒子寄存器}

In the above examples the leader box was inserted with
\cs{copy}. The output of

在上面的示例中,指引线盒子是使用 \cs{copy} 插入的。
以下代码的输出:
\begin{verbatim}
\hbox to 8cm {here\leaders\box\centerdot\hfil there}
\hbox to 8cm {over here\leaders\box\centerdot\hfil 
                   over there}
\end{verbatim}
is
\begin{disp}\leavevmode
     \vtop{\hbox to 8cm {here\leaders\box\centerdot\hfil there}
           \hbox to 8cm {over here\leaders\box\centerdot\hfil over there}
           }\end{disp}
The box register is emptied after the first leader command,
but more than one copy is placed in that first command.

在第一个指引线命令之后,盒子寄存器被清空,但多个副本被放置在第一个命令中。
%\spoint Output in leader boxes
\subsection{Output in leader boxes\\指引线盒子中的输出}

Any \cs{write}, \cs{openout}, or \cs{closeout} operation
appearing in leader boxes is ignored. 
Otherwise such an operation would be executed once for every
copy of the box that would be shipped out.

出现在指引线盒子中的任何 \cs{write}、\cs{openout} 或 \cs{closeout} 操作都会被忽略。
否则,这样的操作将针对将要发送的每个盒子副本执行一次。

%\spoint Box leaders in trace output
\subsection{Box leaders in trace output\\跟踪输出中的盒子指引线}

The dumped box representation obtained from,
for instance, \cs{tracingoutput}
does not write out box leaders in full: only the total size and
one copy of the box used are dumped. In particular,
the surrounding white space before and after the leaders
is not indicated.

从 \cs{tracingoutput} 等获取的转储盒子表示不会完整地写出盒子指引线:
只转储盒子的总大小和使用的一个副本。特别地,
不会指示指引线前后的空白空间。

%\spoint Leaders and shifted margins
\subsection{Leaders and shifted margins\\指引线和偏移边距}

If margins have been shifted,
leaders may look different
depending on how the shift has been realized.
For an illustration of how \cs{hangindent} and \cs{leftskip}
influence the look of leaders, consider the following
examples, where

如果边距已经被偏移,指引线的外观可能会有所不同,具体取决于偏移是如何实现的。
为了说明 \cs{hangindent} 和 \cs{leftskip} 如何影响指引线的外观,考虑以下示例,其中
\begin{verbatim}
\setbox0=\hbox{K o }
\end{verbatim}
The horizontal boxes above  the leaders
\altt
serve to indicate the starting point of the row of leaders.

上面的水平盒子表示指引线的起点。



First 首先是
\begin{verbatim}
\hbox{\leaders\copy0\hskip5cm}
\noindent\advance\leftskip 1em
      \leaders\copy0\hskip5cm\hbox{}\par
\end{verbatim}
gives\message{examples on}
\begin{disp}\leavevmode\vbox{\leftskip=0pt \hsize=7cm
\setbox0=\hbox{K o }
\hbox{\leaders\copy0\hskip5cm}
\noindent\advance\leftskip 1em
      \leaders\copy0\hskip5cm\hbox{}\par
    }\end{disp}
Then 然后是
\begin{verbatim}
\hbox{\kern1em\hbox{\leaders\copy0\hskip5cm}}
\hangindent=1em \hangafter=-1 \noindent
      \leaders\copy0\hskip5cm\hbox{}\par
\end{verbatim}
gives (note the shift with respect to the previous example)

给出(请注意与前一个示例相比的偏移)
\begin{disp}\leavevmode\vbox{\leftskip=0pt \hsize=7cm
\setbox0=\hbox{K o }
\hbox{\kern1em\hbox{\leaders\copy0\hskip5cm}}
\hangindent=1em \hangafter=-1 \noindent
      \leaders\copy0\hskip5cm\hbox{}\par}\end{disp}
\message{one page}
In the first paragraph the \cs{leftskip} glue only obscures
the first leader box; in the second paragraph the hanging
indentation actually shifts the orientation point for the 
row of leaders. Hanging indentation is performed in \TeX\
by a \cs{moveright} of the boxes containing the lines
of the paragraph.

在第一个段落中,\cs{leftskip} 仅遮挡了第一个指引线盒子;在第二个段落中,悬挂缩进实际上通过 \cs{moveright} 移动了包含段落行的盒子的定位点。在 \TeX\ 中,悬挂缩进是通过对包含段落行的盒子进行 \cs{moveright} 来进行的。 

%%%% end of input file [rules]