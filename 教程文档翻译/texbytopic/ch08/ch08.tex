
% %\InputFile:glue
% %%%% this is input file [glue]
% %\subject[glue]  Dimensions and Glue
% \endofchapter
% \chapter{Dimensions and Glue\\尺寸和粘连}\label{glue}

% In \TeX\ vertical and horizontal white space
% can have a possibility to adjust itself through `stretching' or
% \term glue\par
% `shrinking'. An~adjustable white space is called `glue'.
% This chapter treats all technical concepts related to
% dimensions and glue, and it explains how the badness of stretching or shrinking
% a  certain amount is calculated.

% 在 \TeX\ 中,垂直和水平的空白可以通过“伸展”或“收缩”来自动调整。可调整的空白被称为“粘连”。本章介绍与尺寸和粘连相关的所有技术概念,并解释了计算伸展或收缩一定量的劣度的方法。
% \begin{inventory}
% \item [\cs{dimen}] 
%       Dimension register prefix.

%       尺寸寄存器的前缀。
% \item [\cs{dimendef}] 
%       Define a control sequence to be a synonym for
%       a~\cs{dimen} register.

%       将一个控制序列定义为\cs{dimen}寄存器的同义词。
% \item [\cs{newdimen}] 
%       Allocate an unused dimen register. 

%       分配一个未使用的尺寸寄存器。
% \item [\cs{skip}] 
%       Skip register prefix.

%       粘连寄存器的前缀。
% \item [\cs{skipdef}] 
%       Define a control sequence to be a synonym for
%       a~\cs{skip} register.

%       将一个控制序列定义为\cs{skip}寄存器的同义词。
% \item [\cs{newskip}]
%       Allocate an unused skip register.

%       分配一个未使用的粘连寄存器。
% \item [\cs{ifdim}] 
%       Compare two dimensions. 

%       比较两个尺寸。
% \item [\cs{hskip}]  
%       Insert in horizontal mode a glue item.

%       在水平模式中插入一个粘连项目。
% \item [\csidx{hfil}] 
%       Equivalent to 

%       等价于
%       \verb-\hskip 0cm plus 1fil-.
% \item [\csidx{hfilneg}] 
%       Equivalent to 

%       等价于

%       \verb-\hskip 0cm minus 1fil-.

% \item [\csidx{hfill}] 
%       Equivalent to 

%       等价于

%       \verb-\hskip 0cm plus 1fill-.

% \item [\csidx{hss}] 
%       Equivalent to 

%       等价于

%       \verb-\hskip 0cm plus 1fil minus 1fil-.

% \item [\cs{vskip}]  
%       Insert in vertical mode a glue item.

%       在垂直模式中插入一个粘连项目。
% \item [\csidx{vfil}] 
%       Equivalent to 

%       等价于

%       \verb-\vskip 0cm plus 1fil-.

% \item [\csidx{vfill}] 
%       Equivalent to 

%       等价于

%       \verb-\vskip 0cm plus 1fill-.

% \item [\csidx{vfilneg}] 
%       Equivalent to 

%       等价于

%       \verb-\vskip 0cm minus 1fil-.

% \item [\csidx{vss}] 
%       Equivalent to 
      
%       等价于

%       \verb-\vskip 0cm plus 1fil minus 1fil-.

% \item [\cs{kern}]  
%       Add a kern item to the current horizontal or vertical list.

%       在当前水平或垂直列表中添加一个紧排项目。
% \item [\cs{lastkern}] 
%       If the last item on the current list was a kern, the size of it.

%       如果当前列表的最后一个项目是紧排,则返回其尺寸。
% \item [\cs{lastskip}] 
%       If the last item on the current list was a~glue, the size of it.

%       如果当前列表的最后一个项目是粘连,则返回其尺寸。
% \item [\cs{unkern}] 
%       If the last item of the current list was a~kern, remove it.

%       如果当前列表的最后一个项目是紧排,则删除它。
% \item [\cs{unskip}] 
%       If the last item of the current list was a~glue, remove it.

%       如果当前列表的最后一个项目是粘连,则删除它。
% \item [\cs{removelastskip}]
%       Macro to append the negative of the \cs{lastskip}.

%       将\cs{lastskip}的负值附加到当前列表。
% \item [\cs{advance}] 
%       Arithmetic command to add to or subtract from
%       a~\gr{numeric variable}.

%       算术命令,用于对\gr{numeric variable}进行加法或减法运算。
% \item [\cs{multiply}] 
%       Arithmetic command to multiply a~\gr{numeric variable}.

%       算术命令,用于将\gr{numeric variable}乘以一个数。
% \item [\cs{divide}] 
%       Arithmetic command to divide a~\gr{numeric variable}.

%       算术命令,用于将\gr{numeric variable}除以一个数。
% \end{inventory} 


% %\point Definition of \gr{glue} and \gr{dimen}
% \section{Definition of \gr{glue} and \gr{dimen}\\定义 \gr{glue} 和 \gr{dimen}}

% This section gives
% the syntax of the quantities
% \gr{dimen} and \gr{glue}. 
% In the next section the practical aspects of glue are treated.

% 本节介绍 \gr{dimen} 和 \gr{glue} 这两个量的语法。
% 下一节将介绍粘连的实际应用。

% Unfortunately the terminology for glue is slightly confusing.
% The syntactical quantity~\gr{glue} is a dimension (a distance) with
% \mdqon
% possibly a stretch and/""or shrink component.
% \mdqoff
% In order to add a glob of `glue' (a white space) to a list one has to
% let a \gr{glue} be preceded by commands such as \cs{vskip}.

% 不幸的是,粘连的术语有些令人困惑。
% 语法上的量 \gr{glue} 是一个尺寸(距离),\mdqon 它可能具有伸展和/或收缩部分。\mdqoff
% 为了在列表中添加一块“粘连”(空白),需要在 \gr{glue} 之前添加诸如 \cs{vskip} 的命令。

% %\spoint Definition of dimensions
% \subsection{Definition of dimensions\\尺寸的定义}

% A~\gr{dimen} is what \TeX\ expects to see when
% it needs to indicate a dimension; it can be positive or negative.

% \gr{dimen} 是 \TeX\ 在需要表示尺寸时期望看到的内容;它可以是正数或负数。
% \begin{disp}\gr{dimen} $\longrightarrow$ \gr{optional signs}%
%      \gr{unsigned dimen}\end{disp}
% The unsigned part of a \gr{dimen} can be

% \gr{dimen} 的无符号部分可以是:
% \begin{disp}\gr{unsigned dimen} $\longrightarrow$ \gr{normal dimen}
%      $|$ \gr{coerced dimen}\nl
%      \gr{normal dimen} $\longrightarrow$ \gr{internal dimen}
%      $|$ \gr{factor}\gr{unit of measure}\nl
%      \gr{coerced dimen} $\longrightarrow$ \gr{internal glue}
%      \end{disp}
% That is, we have the following three cases:

% 也就是说,我们有以下三种情况:
% \begin{itemize} \item an \gr{internal dimen}; this is
%  any register or parameter of \TeX\ that has a \gr{dimen} value:

%  \gr{internal dimen};这是 \TeX\ 的任何具有 \gr{dimen} 值的寄存器或参数:
%  \begin{disp}\PopIndentLevel\gr{internal dimen} $\longrightarrow$
%       \gr{dimen parameter}\nl
%       \indent $|$ \gr{special dimen} $|$ \cs{lastkern}\nl
%       \indent $|$ \gr{dimendef token} $|$ \cs{dimen}\gr{8-bit number}\nl
%       \indent $|$ \cs{fontdimen}\gr{number}\gr{font}\nl
%       \indent $|$ \gr{box dimension}\gr{8-bit number}\nl
%       \gr{dimen parameter} $\longrightarrow$ \cs{boxmaxdepth}\nl
%       \indent $|$ \cs{delimitershortfall} $|$ \cs{displayindent}\nl
%       \indent $|$ \cs{displaywidth} $|$ \cs{hangindent}\nl
%       \indent $|$ \cs{hfuzz} $|$ \cs{hoffset} $|$ \cs{hsize}\nl
%       \indent $|$ \cs{lineskiplimit} $|$ \cs{mathsurround}\nl
%       \indent $|$ \cs{maxdepth} $|$ \cs{nulldelimiterspace}\nl
%       \indent $|$ \cs{overfullrule} $|$ \cs{parindent}\nl
%       \indent $|$ \cs{predisplaysize} $|$ \cs{scriptspace}\nl
%       \indent $|$ \cs{splitmaxdepth} $|$ \cs{vfuzz}\nl
%       \indent $|$ \cs{voffset} $|$ \cs{vsize}
%  \end{disp}
% \item  a dimension denotation, 
%  consisting of \gr{factor}\gr{unit of measure},
%  for example \verb>0.7\vsize>; or

%  尺寸表示,由 \gr{factor}\gr{unit of measure} 组成,例如 \verb>0.7\vsize>;
% \item an \gr{internal glue} (see below) 
%  coerced to a dimension by omitting
%  the stretch and shrink components, for example \cs{parfillskip}.

%  通过省略伸展和收缩部分将 \gr{internal glue} 强制转换为尺寸,例如 \cs{parfillskip}。
% \end{itemize}

% A dimension denotation is a somewhat complicated entity:

% 尺寸表示法是一个相对复杂的实体:
% \begin{itemize} \item a \gr{factor} is an integer denotation,
%  a decimal constant denotation (a number with an integral and
%  a fractional part),
%  or an \gr{internal integer}

%  \gr{factor} 可以是一个整数表示法、一个十进制常量表示法(一个由整数和小数部分组成的数字)或一个 \gr{internal integer}。
%  \begin{disp}\PopIndentLevel
%       \gr{factor} $\longrightarrow$ \gr{normal integer} 
%       $|$ \gr{decimal constant}\nl
%       \gr{normal integer} $\longrightarrow$ \gr{integer denotation}\nl
%       \indent $|$ \gr{internal integer}\nl
%       \gr{decimal constant} $\longrightarrow$ \n{.$_{12}$}
%       $|$ \n{,$_{12}$}\nl
%       \indent $|$ \gr{digit}\gr{decimal constant}\nl
%       \indent $|$ \gr{decimal constant}\gr{digit}
%  \end{disp}
%  An internal integer is a parameter that is `really' an
% \alt
%  integer (for instance, \cs{count0}), and not coerced from a dimension or glue.
%  See Chapter~\ref{number}
%  for the definition of various kinds of integers.

%  \gr{internal integer} 是一个“真正”的整数参数(例如,\cs{count0}),而不是从尺寸或粘连强制转换而来。有关各种整数的定义,请参见第~\ref{number} 章。
% \item a \gr{unit of measure} can be 
%  a \gr{physical unit}, that is, an ordinary unit such as~\n{cm} 
%  (possibly preceded by \n{true}),
%  an internal unit such as~\n{em}, but also an \gr{internal integer}
%  (by conversion to scaled points),
%  an \gr{internal dimen}, or an \gr{internal glue}.

%  \gr{unit of measure} 可以是 \gr{physical unit},即普通的单位,例如 \n{cm}(可能前面带有 \n{true}),也可以是内部单位,例如 \n{em},还可以是 \gr{internal integer}(通过转换为 scaled point),\gr{internal dimen} 或 \gr{internal glue}。
%  \begin{disp}\PopIndentLevel
%       \gr{unit of measure} $\longrightarrow$
%       \gr{optional spaces}\gr{internal unit}\nl
%       \indent $|$ 
%       \gr{optional \n{true}}\gr{physical unit}\gr{one optional space}\nl 
%       \gr{internal unit} $\longrightarrow$ 
%       \n{em}\gr{one optional space}\nl
%       \indent $|$ \n{ex}\gr{one optional space}
%               $|$ \gr{internal integer}\nl
%       \indent $|$ \gr{internal dimen} $|$ \gr{internal glue}
%       \end{disp}
% \end{itemize}

% Some \gr{dimen}s are called \gr{special dimen}s:\label{special:dimen:list}

% 一些 \gr{dimen} 被称为 \gr{special dimen}:
% \begin{disp}\gr{special dimen} $\longrightarrow$ \cs{prevdepth}\nl
%      \indent $|$ \cs{pagegoal} $|$ \cs{pagetotal} $|$ \cs{pagestretch}\nl
%      \indent $|$ \cs{pagefilstretch} $|$ \cs{pagefillstretch}\nl
%      \indent $|$ \cs{pagefilllstretch} $|$ \cs{pageshrink} $|$ \cs{pagedepth}
%      \end{disp}
% An assignment to any of these is
% called an \gr{intimate assignment}, and it is automatically
% global (see Chapter~\ref{group}). The meaning of these 
% dimensions is explained in Chapter \ref{page:break}, with the
% exception of \cs{prevdepth} which is treated in
% Chapter~\ref{baseline}.

% 对这些任何一个的赋值被称为 \gr{intimate assignment},它是自动全局的(见第~\ref{group} 章)。这些尺寸的含义在第~\ref{page:break} 章中解释,\cs{prevdepth} 是一个例外,它在第~\ref{baseline} 章中处理。


% %\spoint Definition of glue
% \subsection{Definition of glue\\粘连的定义}

% A \gr{glue} is either some form of glue variable, or
% a glue denotation with explicitly indicated stretch and
% shrink. Specifically,

% \gr{glue} 可以是某种形式的粘连变量,或者带有明确指定的伸长量和收缩量的粘连表示。
% 具体而言,
% \begin{disp}\gr{glue} $\longrightarrow$ \gr{optional signs}\gr{internal glue}
%      $|$ \gr{dimen}\gr{stretch}\gr{shrink}\nl
%      \gr{internal glue} $\longrightarrow$ \gr{glue parameter}
%      $|$ \cs{lastskip}\nl 
%      \indent $|$ \gr{skipdef token} $|$ \cs{skip}\gr{8-bit number}\nl
%      \gr{glue parameter} $\longrightarrow$ \cs{abovedisplayshortskip}\nl
%      \indent $|$ \cs{abovedisplayskip} $|$ \cs{baselineskip}\nl
%      \indent $|$ \cs{belowdisplayshortskip} $|$ \cs{belowdisplayskip}\nl
%      \indent $|$ \cs{leftskip} $|$ \cs{lineskip} $|$ \cs{parfillskip}
%              $|$ \cs{parskip}\nl
%      \indent $|$ \cs{rightskip} $|$ \cs{spaceskip}
%              $|$ \cs{splittopskip} $|$ \cs{tabskip}\nl
%      \indent $|$ \cs{topskip} $|$ \cs{xspaceskip}
% \end{disp}
% The stretch and shrink components in a glue denotation
% are optional, but when both are specified they have to
% be given in sequence; they are defined as

% 粘连表示中的伸长量和收缩量是可选的,但当两者都指定时,必须按顺序给出;
% 它们的定义如下:
% \begin{disp}
% \gr{stretch} $\longrightarrow$ \n{plus} \gr{dimen}
%       $|$ \n{plus}\gr{fil dimen} $|$ \gr{optional spaces}\nl
% \gr{shrink} $\longrightarrow$ \n{minus} \gr{dimen}
%       $|$ \n{minus}\gr{fil dimen} $|$ \gr{optional spaces}\nl
% \gr{fil dimen} $\longrightarrow$ \gr{optional signs}\gr{factor}%
%      \gr{fil unit}\gr{optional spaces}\nl
% \gr{fil unit} $\longrightarrow$ \n{ $|$ fil $|$ fill $|$ filll}
% \end{disp}

% The actual definition of \gr{fil unit} is recursive
% (see Chapter~\ref{gramm}), but these are the only valid
% possibilities.

% \gr{fil unit} 的实际定义是递归的(见第~\ref{gramm} 章),
% 但这些是唯一有效的可能性。
% %\spoint Conversion of \gr{glue} to \gr{dimen}
% \subsection{Conversion of \gr{glue} to \gr{dimen}\\\gr{glue} 转换为 \gr{dimen}}

% The grammar rule

% 语法规则
% \begin{disp}\gr{dimen} $\longrightarrow$
%      \gr{factor}\gr{unit of measure}
% \end{disp}
% has some noteworthy consequences, caused by the fact
% that a \gr{unit of measure} need not look like a `unit of measure'
% at all (see the list above).

% 有一些值得注意的后果,这是因为 \gr{unit of measure} 不一定看起来像一个“单位”(见上面的列表)。

% For instance, from this definition we conclude that the statement

% 例如,根据这个定义,我们可以得出以下结论:
% \begin{verbatim}
% \dimen0=\lastpenalty\lastpenalty
% \end{verbatim}
% is
% syntactically correct because \cs{lastpenalty} can function
% both as an integer and as \gr{unit of measure} by taking
% its value in scaled points.
% After \verb>\penalty8> the \cs{dimen0} thus defined will
% have a size of~\n{64sp}.

% 在语法上是正确的,因为 \cs{lastpenalty} 可以同时作为整数和\gr{unit of measure},
% 它可以在缩放点中取值。
% 在 \verb>\penalty8> 之后,这样定义的 \cs{dimen0} 将具有大小 \n{64sp}。

% More importantly, consider the case where the \gr{unit of measure} is
% an \gr{internal glue}, that is, any sort of glue parameter.
% Prefixing such a glue with a number (the \gr{factor})
% makes it a valid \gr{dimen} specification.
% Thus 

% 此外,考虑\gr{unit of measure}是\gr{internal glue}(即任何类型的粘连参数)的情况。在粘连前面加上一个数字(\gr{factor})将其作为有效的\gr{dimen}规范。因此,
% \begin{verbatim}
% \skip0=\skip1
% \end{verbatim}
% is very different
% from 

% 非常不同于\begin{verbatim}
% \skip0=1\skip1
% \end{verbatim}
% The first statement makes
% \cs{skip0} equal to \cs{skip1}, the second converts
% the \cs{skip1} to a \gr{dimen} before assigning it.
% In other words, the \cs{skip0} defined by the second statement
% has no stretch or shrink.

% 第一个语句将\cs{skip0}设置为等于\cs{skip1},而第二个语句在赋值之前将\cs{skip1}转换为\gr{dimen}。换句话说,由第二个语句定义的\cs{skip0}没有伸长或收缩的能力。


% %\spoint Registers for \cs{dimen} and \cs{skip}
% \subsection{Registers for \cs{dimen} and \cs{skip}\\用于 \cs{dimen} 和 \cs{skip} 的寄存器}

% \TeX\ has registers for storing \gr{dimen} and \gr{glue}
% values: the \csidx{dimen} and \csidx{skip} registers
% respectively. These are accessible by the expressions

% \TeX\ 有用于存储\gr{dimen}和\gr{glue}值的寄存器:\csidx{dimen} 寄存器和 \csidx{skip} 寄存器。可以通过以下表达式访问这些寄存器:
% \begin{disp}\cs{dimen}\gr{number}\end{disp} and
% \begin{disp}\cs{skip}\gr{number}\end{disp}
% As with all registers of \TeX, these registers are
% numbered~0--255.

% 与 \TeX\ 的所有寄存器一样,这些寄存器的编号为 0--255。

% Synonyms for registers can be made with the \csidx{dimendef} and
% \csidx{skipdef} commands. Their syntax is

% 可以使用 \csidx{dimendef} 和 \csidx{skipdef} 命令创建寄存器的同义词。它们的语法为:
% \begin{Disp}\cs{dimendef}\gr{control sequence}\gr{equals}\gr{8-bit number}
% \end{Disp}
% and 
% \begin{Disp}\cs{skipdef}\gr{control sequence}\gr{equals}\gr{8-bit number}\end{Disp}
% For example, after \verb-\skipdef\foo=13- using \cs{foo}
% is equivalent to using \cs{skip13}.

% 例如,通过执行 \verb-\skipdef\foo=13-,可以使用 \cs{foo} 来代替 \cs{skip13}。

% Macros \csidx{newdimen} and \csidx{newskip} exist in plain \TeX
% for allocating an unused dimen or skip register.
% These macros are defined to be \cs{outer} in the plain format.

% 在 plain \TeX\ 中,有宏 \csidx{newdimen} 和 \csidx{newskip} 用于分配未使用的尺寸寄存器或粘连寄存器。在 plain format 中,这些宏被定义为\cs{outer}。


% %\spoint Arithmetic: addition 
% \subsection{Arithmetic: addition\\算术:加法}

% As for integer variables, arithmetic operations exist for
% \cstoidx advance\par\term  glue!arithmetic on\par\term arithmetic! on glue\par
% dimen, glue, and muglue (mathematical glue; see page~\pageref{muglue})
% variables.

% 对于整数变量,可以进行尺寸、粘连和数学粘连(mathematical glue)变量的算术运算。

% The expressions

% 表达式
% \begin{Disp}\cs{advance}\gr{dimen variable}\gr{optional \n{by}}%
%      \gr{dimen}\nl
%      \cs{advance}\gr{glue variable}\gr{optional \n{by}}%
%      \gr{glue}\nl
%      \cs{advance}\gr{muglue variable}\gr{optional \n{by}}%
%      \gr{muglue}\end{Disp}
% add to the size of a dimen, glue, or muglue.

% 可以将尺寸、粘连或数学粘连的大小增加。


% Advancing a \gr{glue variable} by \gr{glue} is done by
% adding the natural sizes, and the stretch and shrink components.
% Because \TeX\ converts between \gr{glue} and \gr{dimen},
% it is possible to write for instance

% 通过将自然尺寸、伸展和收缩部分相加,可以将 \gr{glue variable} 的粘连增加 \gr{glue}。
% 由于 \TeX\ 可以在 \gr{glue} 和 \gr{dimen} 之间进行转换,
% 因此可以写成如下的形式:
% \begin{verbatim}
% \advance\skip1 by \dimen1
% \end{verbatim}
% or
% \begin{verbatim}
% \advance\dimen1 by \skip1
% \end{verbatim}
% In the first case  \cs{dimen1} is coerced to \gr{glue} without
% stretch or shrink; in the second case the \cs{skip1} is coerced
% to a \gr{dimen} by taking its natural size.

% 在第一种情况下,\cs{dimen1} 被强制转换为没有伸展或收缩的 \gr{glue};
% 在第二种情况下,\cs{skip1} 被强制转换为 \gr{dimen},通过取其自然尺寸。



% %\spoint Arithmetic: multiplication and division
% \subsection{Arithmetic: multiplication and division\\算术:乘法和除法}

% Multiplication and division operations exist for glue
% \cstoidx multiply\par\cstoidx divide\par
% and dimensions. One may for instance write

% 粘连和尺寸的乘法和除法操作存在。
% 例如,可以写成如下形式:
% \begin{verbatim}
% \multiply\skip1 by 2
% \end{verbatim}
% which multiplies the natural size, and the stretch and shrink
% components of \cs{skip1} by~2.

% 这将把 \cs{skip1} 的自然尺寸、伸展和收缩部分乘以 2。

% The second operand of a \cs{multiply} or \cs{divide}
% operation can only be a \gr{number}, that is, an integer.
% Introducing the notion of \gr{numeric variable}:

% \cs{multiply} 或 \cs{divide} 操作的第二个操作数只能是 \gr{number},即整数。
% 引入 \gr{numeric variable} 的概念:
% \begin{disp}\gr{numeric variable} $\longrightarrow$ \gr{integer variable}
%      $|$ \gr{dimen variable} \nl
%      \indent $|$ \gr{glue variable} $|$ \gr{muglue variable}\end{disp}
% these operations take the form

% 这些操作的形式如下:
% \begin{Disp}\cs{multiply}\gr{numeric variable}\gr{optional \n{by}}%
% \gr{number}\end{Disp} 
% and
% \begin{Disp}\cs{divide}\gr{numeric variable}\gr{optional \n{by}}%
% \gr{number}\end{Disp}

% Glue and dimen can be multiplied by 
% non-integer quantities:

% 可以使用非整数的数量来乘以粘连和尺寸:
% \begin{verbatim}
% \skip1=2.5\skip2
% \dimen1=.78\dimen2
% \end{verbatim}
% However, in the first line the \cs{skip2} is first coerced
% to a \gr{dimen} value by omitting its stretch and shrink.

% 然而,在第一行中,\cs{skip2} 首先被强制转换为 \gr{dimen} 值,通过省略其伸展和收缩。


% %\point More about dimensions
% \section{More about dimensions\\关于尺寸的更多信息}

% %\spoint Units of measurement
% \subsection{Units of measurement\\度量单位}

% In \TeX\ dimensions can be indicated in
% \term units of measurement\par

% 在 \TeX 中,尺寸可以用以下度量单位表示:
% \begin{description} \item [centimetre]
%     denoted \n{cm} or 

%     [厘米] 用 \n{cm} 表示,这是国际标准单位制(SI)的单位,即国际标准度量单位制度。
% \item [millimetre]
% 	denoted \n{mm}; these are SI~units ({\italic Syst\`eme International
% 	d'Unit\'es}, the
% 	international system of standard units of measurements).

%       [毫米] 用 \n{mm} 表示;这在英美世界更常见。一毫米等于0.1厘米。
% \item [inch]
% \n{in}; more common in the Anglo-American world.
% One inch is 2.54~centimetres.

% [英寸] 用 \n{in} 表示;这在英美世界更常见。一英寸等于2.54厘米。
% \item [pica]
%     denoted \n{pc}; one pica is 12~points.

%     [派卡] 用 \n{pc} 表示;一派卡等于12点。
% \item [point]
%     denoted \n{pt}; the common system
% for Anglo-American printers. One inch is 72.27 points.

% [点] 用 \n{pt} 表示;这是英美打印机的常用单位。一英寸等于72.27点。
% \item [didot point]
%     denoted \n{dd}; the common system for continental European printers.
%     Furthermore, 1157 didot points are 1238~points.

%     [迪多点] 用 \n{dd} 表示;这是欧洲大陆打印机的常用单位。此外,1157迪多点等于1238点。
% \item [cicero]
%     denoted \n{cc}; one cicero is 12~didot points.

%     [西塞罗] 用 \n{cc} 表示;一西塞罗等于12迪多点。
% \item [big point]
%     denoted \n{bp}; one inch is 72 big points.

%     [大点] 用 \n{bp} 表示;一英寸等于72大点。
% \item [scaled point]
%     denoted \n{sp}; this is the smallest unit in \TeX, and all measurements
%     are integral multiples of one scaled point.
%     There are $65\,536$ scaled points in a~point.

%     [缩放点] 用 \n{sp} 表示;这是 \TeX 中最小的单位,所有尺寸都是缩放点的整数倍。一点中有$65,536$个缩放点。
% \end{description}

% Decimal fractions can be written using both the
% Anglo-American system with the decimal point
% (for example, \n{1in}=\n{72.27pt})
% and the continental European system with a decimal
% comma; \n{1in}=\n{72,27pt}.

% 小数可以用英美制度的小数点(例如,\n{1in}=\n{72.27pt})或欧洲大陆制度的小数逗号表示;\n{1in}=\n{72,27pt}。

% Internally \TeX\ works with multiples of a smallest 
% dimension: the  scaled point.
% Dimensions larger (in absolute value) than $2^{30}-1$\n{sp},
% which is about 5.75~metres or 18.9~feet, are illegal.

% 在内部,\TeX 使用最小尺寸的倍数:缩放点。绝对值大于$2^{30}-1$\n{sp}的尺寸是非法的,这大约相当于5.75米或18.9英尺。

% Both the pica system and the didot system are of French
% origin: in 1737 the type founder Pierre Simon Fournier
% introduced typographical points based on the French foot.
% Although at first he introduced a system based on lines and
% points, he later took the point as unit:
% there are 72 points in an inch,
% which is one-twelfth of a foot. 
% About 1770 another founder, Fran\c{c}ois Ambroise Didot, introduced
% points based on the more common, and slightly longer,
% `pied du roi'.

% 派卡制度和迪多制度都源自法国:1737年,字体铸造者皮埃尔·西蒙·福尼埃引入了基于法国尺的印刷点。尽管他最初引入了基于行和点的系统,但后来他采用了点作为单位:一英寸等于72点,即一英尺的十二分之一。大约在1770年,另一位字体铸造者弗朗索瓦·安布瓦兹·迪多引入了基于更普遍、稍长的“国王尺”的点。
 


% %\spoint Dimension testing
% \subsection{Dimension testing\\尺寸测试}

% Dimensions and natural sizes of glue can be compared with
% the \cs{ifdim} test. This takes the form

% 可以使用 \cs{ifdim} 测试来比较尺寸和粘连的自然尺寸。其形式为:
% \begin{disp}\cs{ifdim}\gr{dimen$_1$}\gr{relation}\gr{dimen$_2$}\end{disp}
% where the relation can be an \n>, \n<, or~\n= token, 
% all of category~12.

% 其中关系可以是 \n>、\n< 或 \n= 符号,它们的类别码均为 12。


% %\spoint Defined dimensions
% \subsection{Defined dimensions\\已定义的尺寸}

% \begin{inventory}
% \item [\cs{z@}]
%  \n{0pt}

% \item [\cs{maxdimen}] 
%       \n{16383.99999pt}; the largest legal dimension.

%       最大合法尺寸。
% \end{inventory}

% These \gr{dimen}s are predefined in the plain format;
% for instance 

% 这些 \gr{dimen} 在 plain 格式中是预定义的;
% 例如:\begin{verbatim}
% \newdimen\z@ \z@=0pt
% \end{verbatim}
% Using such abbreviations for commonly used dimensions
% has at least two advantages. First of all it saves main memory
% if such a dimension occurs in a macro: a control sequence
% is one token, whereas a string such as \n{0pt} takes three.
% Secondly, it saves time in processing, as \TeX\ does not need
% to perform conversions to arrive at the correct type of
% object.

% 使用这样的缩写形式表示常用尺寸至少有两个优点。
% 首先,如果这样的尺寸出现在宏中,它可以节省主存储器:
% 控制序列只是一个记号,而像 \n{0pt} 这样的字符串需要三个记号。
% 其次,它可以节省处理时间,因为 \TeX\ 不需要进行转换以确定正确的对象类型。

% Control sequences such as \cs{z@}
% are only available to a user who changes the
% category code of the `at' sign. Ordinarily, these control sequences
% appear only in the macros defined in packages such as the
% plain format.

% 诸如 \cs{z@} 的控制序列仅对更改“at”符号的类别码的用户可用。
% 通常,这些控制序列仅出现在像 plain 格式等包中定义的宏中。


% %\point More about glue
% \section{More about glue\\关于粘连的更多内容}

% Glue items can be added to a vertical list with one of the
% \alt
% commands \csidx{vskip}\gr{glue}, \cs{vfil}, \cs{vfill}, \cs{vss} or
% \cs{vfilneg}; 
% glue items can be added to a horizontal list with one of the
% commands \csidx{hskip}\gr{glue}, \cs{hfil}, \cs{hfill}, \cs{hss} or
% \cs{hfilneg}. We will now treat the properties of glue.

% 可以使用 \csidx{vskip}\gr{glue}、\cs{vfil}、\cs{vfill}、\cs{vss} 或 \cs{vfilneg} 等命令将粘连项添加到垂直列表;
% 可以使用 \csidx{hskip}\gr{glue}、\cs{hfil}、\cs{hfill}、\cs{hss} 或 \cs{hfilneg} 等命令将粘连项添加到水平列表。
% 我们将进一步介绍粘连的属性。
% %\spoint Stretch and shrink
% \subsection{Stretch and shrink\\伸长和收缩}

% In the syntax given above, \gr{glue} was defined as having
% \term stretch\par\term shrink\par
% \term glue!stretch component of\par\term glue!shrink component of\par

% 在上面给出的语法中,\gr{glue} 被定义为具有以下组成部分:
% \begin{itemize}\item a `natural size', which is a \gr{dimen}, and optionally

%       自然尺寸',是一个 \gr{dimen},可以选择性地添加;
% \item a `stretch' and `shrink' component built out of a \gr{fil dimen}.

% 由 \gr{fil dimen} 组成的 伸长' 和 `收缩' 组成部分。
% \end{itemize}

% Each list that \TeX\ builds has amounts of stretch and shrink
% (possibly zero),
% which are the sum of the
% stretch and shrink components of individual pieces of glue in the list. 
% Stretch and shrink are used if the context in which the list
% appears requires it to assume a size that is different from
% its natural size.

% \TeX\ 构建的每个列表都有伸长和收缩的量(可能为零),这些量是列表中各个粘连的伸长和收缩组成部分的总和。
% 如果列表所在的上下文要求它具有与自然尺寸不同的尺寸,那么将使用伸长和收缩。

% There is an important difference in behaviour between stretch
% and shrink components when they are finite \ldash that is,
% when the \gr{fildimen} is not \n{fil}(\n{l}(\n{l})). 
% A~finite amount of shrink is indeed the maximum shrink
% that \TeX\ will take: the amount of glue specified
% as 

% 当伸长和收缩组成部分有限时(即 \gr{fildimen} 不是 \n{fil}(\n{l}(\n{l})),
% 它们的行为存在重要差异。有限量的收缩是 \TeX\ 允许的最大收缩量:
% 例如,指定为 
% \begin{verbatim}
% 5pt minus 3pt
% \end{verbatim}
% can shrink to \n{2pt}, but not further.
% In contrast to this, a finite amount of stretch 
% can be stretched arbitrarily far. 
% Such arbitrary stretching
% has a large `badness', however.
% Badness calculation is treated below.

% 的粘连可以收缩到 \n{2pt},但不能进一步收缩。
% 与此相反,有限量的伸长可以被无限伸展。然而,这种无限伸展会产生很大的`劣度'。
% 劣度的计算将在下面介绍。

% \begin{example}
% The sequence with natural size \n{20pt}

% 具有自然尺寸 \n{20pt} 的序列
% \begin{verbatim}
% \hskip 10pt plus 2pt \hskip 10pt plus 3pt
% \end{verbatim}
% has \n{5pt} of stretch, but it has no shrink. In

% 有 \n{5pt} 的伸长,但没有收缩。而在
% \begin{verbatim}
% \hskip 10pt minus 2pt \hskip 10pt plus 3pt
% \end{verbatim}
% there is \n{3pt} of stretch, and \n{2pt} of shrink,
% so its minimal size is~\n{18pt}. 

% 中,有 \n{3pt} 的伸长和 \n{2pt} 的收缩,因此其最小尺寸为 \n{18pt}。

% Positive shrink is not the same as negative stretch:

% 正的收缩与负的伸长不同:
% \begin{verbatim}
% \hskip 10pt plus -2pt \hskip 10pt plus 3pt
% \end{verbatim}
% looks a lot like the previous example, but it cannot
% be shrunk as there are no \hbox{\n{minus}\gr{dimen}}
% specifications. It does have \n{1pt} of stretch, however.

% 与前面的例子非常相似,但它不能收缩,因为没有 \hbox{\n{minus}\gr{dimen}} 的规定。然而,它确实有 \n{1pt} 的伸长。

% This is another example of negative amounts of shrink and stretch.
% It is not possible to stretch
% glue (in the informal sense) by shrinking it (in the technical
% sense): 

% 这是负的伸展和收缩量的另一个例子。
% 无法通过收缩(在技术上)来拉伸粘连(在非正式意义上):\begin{verbatim}
% \hbox to 5cm{a\hskip 0cm minus -1fil}
% \end{verbatim}
% is an underfull box, because \TeX\ looks for a \n{plus}~\gr{dimen}
% specification when it needs to stretch the contents.

% 是一个不充实的盒子,因为当需要拉伸盒子的内容时,\TeX\ 会查找 \n{plus}~\gr{dimen} 规范。

% Finally, \begin{verbatim}
% \hskip 10pt plus -3pt \hskip 10pt plus 3pt
% \end{verbatim}
% can neither stretch nor shrink.
% The fact that there is only stretch
% available means that the sequence cannot
% shrink. However, the stretch components cancel out: the 
% total stretch is zero. Another way of looking at this
% is to consider that for each point that the second glue item would
% stretch, the first one would `stretch back' one point.

% 既不能伸展也不能收缩。
% 只有伸展可用的事实意味着该序列不能收缩。
% 然而,伸展部分相互抵消:总伸展为零。
% 从另一个角度来看,可以考虑对于第二个粘连项的每一个点的伸展,第一个粘连项会“伸展回来”一个点。
% \end{example}

% Any amount of infinite stretch or shrink overpowers all
% finite stretch or shrink available:

% 任何无限伸展或收缩的量都能克服所有有限伸展或收缩的量:
% \begin{verbatim}
% \hbox to 5cm{\hskip 0cm plus 16384pt 
%               text\hskip 0cm plus 0.0001fil}
% \end{verbatim}
% has the \n{text} at the extreme left of the box.
% There are three orders of `infinity', each  one infinitely
% stronger than the previous one:

% 将 \n{text} 放在盒子的极左侧。
% 有三个“无穷大”的级别,每个级别都比前一个级别强得多:
% \begin{verbatim}
% \hbox to 5cm{\hskip 0cm plus 16384fil
%               text\hskip 0cm plus 0.0001fill}
% \end{verbatim}
% and
% \begin{verbatim}
% \hbox to 5cm{\hskip 0cm plus 16384fill
%               text\hskip 0cm plus 0.0001filll}
% \end{verbatim}
% both have the \n{text} at the left end of the box.

% 都将 \n{text} 放在盒子的左端。 


% %\spoint Glue setting
% \subsection{Glue setting\\粘连设置}

% In the process of `glue setting', the desired width (or height)
% \term glue! setting\par
% of a box is compared with the natural dimension of its contents,
% which is the sum of all natural dimensions of boxes and globs of glue.
% If the two differ, any available stretchability or shrinkability is used
% to bridge the gap.
% To attain the desired dimension of the box
% only the glue of the highest available order is set:
% each piece of glue of that order is stretched or shrunk by the
% same ratio.

% 在“粘连设置”过程中,将盒子的期望宽度(或高度)与其内容的自然尺寸进行比较,
% 其中自然尺寸是盒子和粘连的所有自然尺寸的总和。
% 如果这两者不一致,则使用任何可用的伸展或收缩来填补差距。
% 为了达到盒子的期望尺寸,只设置最高级别的粘连:
% 该级别的每个粘连部分都按相同的比例进行伸展或收缩。

% For example, in

% 例如,在
% \begin{verbatim}
% \hbox to 6pt{\hskip 0pt plus 3pt \hskip 0pt plus 9pt}
% \end{verbatim}
% the natural size of the box is~\n{0pt}, and
% the total stretch is~\n{12pt}. In order to obtain a box
% of~\n{6pt} each glue item is set with a stretch ratio
% of~$1/2$. Thus the result is equivalent to

% 中,盒子的自然尺寸为~\n{0pt},总伸展为~\n{12pt}。
% 为了获得一个尺寸为~\n{6pt} 的盒子,每个粘连项都以~$1/2$ 的伸展比例进行设置。
% 因此,结果等同于
% \begin{verbatim}
% \hbox {\hskip 1.5pt \hskip 4.5pt}
% \end{verbatim}
% Only the highest order of stretch or shrink is used:
% in 

% 只使用最高级别的伸展或收缩:\begin{verbatim}
% \hbox to 6pt{\hskip 0pt plus 1fil \hskip 0pt plus 9pt}
% \end{verbatim}
% the second glue  will assume its natural size of~\n{0pt},
% and only the first   glue will be stretched.

% 中,第二个粘连项将保持其自然尺寸为~\n{0pt},只有第一个粘连项会被伸展。

% \TeX\ will never exceed the maximum value of a finite
% amount of shrink.
% A~box that cannot be shrunk enough is called `overfull'.
% Finite stretchability can be exceeded to provide an
% escape in difficult situations; however, \TeX\ is likely 
% to give an \verb-Underfull \hbox- message about this
% (see page~\pageref{over/underfull}).
% For an example of infinite shrink see page~\pageref{rlap}.

% \TeX\ 永远不会超过有限收缩的最大值。
% 无法收缩足够的盒子称为“过充实”。
% 可以超过有限伸展性以提供在困难情况下的逃逸;
% 然而,\TeX\ 可能会给出关于此的 \verb-Underfull \hbox- 消息(参见第~\pageref{over/underfull}页)。
% 关于无穷收缩的示例,请参见第\pageref{rlap}~页。

% %\spoint Badness
% \subsection{Badness\\劣度}

% When stretching or shrinking a list \TeX\ calculates 
% \term badness! calculation\par
% badness based on the
% ratio between actual stretch and the amount of stretch
% present in the line. See Chapter~\ref{line:break}
% for the application  of badness to the paragraph algorithm.

% 在拉伸或收缩列表时,\TeX 根据实际拉伸量与行中的可拉伸量之比来计算劣度。有关劣度在段落算法中的应用,请参见第~\ref{line:break} 章。

% %\tracingmacros=2 \tracingcommands\tracingmacros
% The formula for badness of a list that is stretched (shrunk) is
% \label{bad:form}\message{Check roman min}

% 拉伸(收缩)列表的劣度公式为
% \begin{disp} $\displaystyle b=\hbox{min}\left(10\,000,
% 100\times \left({\hbox{actual amount stretched (shrunk)}
% \over\hbox{possible amount of stretch (shrink)}}\right)^3\right)$\end{disp}
% In reality \TeX\ uses a slightly different formula that is
% easier to calculate, but behaves the same. Since glue setting is
% one of the main activities of \TeX, this must be performed
% as efficiently as possible.

% 实际上,\TeX 使用的公式略有不同,更容易计算,但行为相同。由于粘连设置是 \TeX 的主要活动之一,必须尽可能高效地执行。

% This formula lets the badness be a reasonably small number
% if the glue set ratio (the fraction in the above expression)
% is reasonably small, but will let it grow rapidly once
% the ratio is more than~1. Badness is infinite if the
% glue would have to shrink more than the allotted amount;
% stretching glue beyond its maximum is possible, so this
% provides an  escape for very difficult lines of text or pages.

% 该公式使劣度保持在一个相当小的数字,如果粘连设置比(上述表达式中的分数)相对较小,则劣度将保持在相当小的数字。但一旦比率超过 1,劣度将迅速增加。如果粘连必须收缩超过分配的量,劣度为无穷大;超过其最大值的拉伸粘连是可能的,因此这为非常困难的文本行或页面提供了一个逃逸途径。

% In \TeX3, the \cs{badness} parameter records the badness
% of the most recently formed box.

% 在 \TeX3 中,\cs{badness} 参数记录了最近形成的盒子的劣度。


% %\spoint Glue and breaking
% \subsection{Glue and breaking\\粘连和断行}

% \TeX\ can break lines and pages in several kinds of places.
% One of these places is before a glue item. 
% The glue is then discarded. For line breaks this is treated
% in Chapter~\ref{line:break}, 
% for page breaks see Chapter~\ref{page:break}.

% \TeX 可以在几种位置断行和分页。其中之一是在粘连项之前。此时,粘连将被丢弃。对于断行,这在第~\ref{line:break} 章中处理;对于分页,请参见第~\ref{page:break} 章。

% There are two macros in plain \TeX, \csidx{hglue} and \csidx{vglue},
% that give non-disappearing glue in horizontal and
% vertical mode respectively. For the horizontal case this is
% accomplished by
% placing:

% 在 plain \TeX 中,有两个宏 \csidx{hglue} 和 \csidx{vglue} 分别在水平模式和垂直模式下提供非消失的粘连。对于水平情况,通过放置:
% \begin{verbatim}
% \vrule width 0pt \nobreak \hskip ...
% \end{verbatim}
% Because \TeX\ breaks at the front end of glue,
% this glue will always stay attached to the rule,
% and will therefore never disappear.
% The actual macro definitions are somewhat more complicated,
% because they take care to preserve the \cs{spacefactor} and the
% \cs{prevdepth}.

% 来实现。由于 \TeX 在粘连的前端断行,因此这个粘连将始终保持与标线连接,并且因此永远不会消失。实际的宏定义稍微复杂一些,因为它们确保保留了 \cs{spacefactor} 和 \cs{prevdepth}。

% %\spoint \cs{kern}
% \subsection{\cs{kern}}

% The \csidx{kern} command specifies
% a~kern item in whatever mode \TeX\ is currently
% in. A~kern item is much like a glue item without
% stretch or shrink.
% It differs from glue in that it is
% in general not a legal breakpoint. Thus in

% \csidx{kern} 命令可以在 \TeX\ 当前的任何模式中指定 kern 项。
% kern 项类似于没有伸长和收缩的粘连项。
% 与粘连的区别在于,kern 通常不能作为合法的断点。
% 因此,在下面的代码中:
% \begin{verbatim}
% .. text .. \hbox{a}\kern0pt\hbox{b}
% \end{verbatim}
% \TeX\ will not break lines in between the boxes; in

% \TeX\ 不会在盒子之间断行;而在下面的代码中:
% \begin{verbatim}
% .. text .. \hbox{a}\hskip0pt\hbox{b}
% \end{verbatim}
% a line can be broken in between the boxes.

% 可以在盒子之间断行。

% However, if a kern is followed by glue, \TeX\ can break at the
% kern (provided that it is not in math mode). 
% In horizontal mode
% both the kern and the glue then disappear in the break.
% In vertical mode they are discarded when they are moved to
% the (empty) current page after the material before
% the break has been disposed of by the output routine 
% (see Chapter~\ref{page:break}).

% 然而,如果一个 kern 后面跟着粘连,\TeX\ 可以在 kern 处断行(前提是不在数学模式下)。
% 在水平模式下,kern 和粘连都会在断行时消失。
% 在垂直模式下,它们在移动到(空的)当前页面后被丢弃,
% 此前的内容已由输出例程处理掉(请参见第~\ref{page:break} 章)。

% %\spoint Glue and modes
% \subsection{Glue and modes\\粘连和模式}

% All horizontal skip commands are \gr{horizontal command}s and
% all vertical skip commands are \gr{vertical commands}s.
% This means that, for instance, an \cs{hskip} command
% makes \TeX\ start a paragraph if it is given in vertical mode.
% The \cs{kern} command can be given in both modes.

% 所有水平间距命令都是\gr{horizontal command},
% 而所有垂直间距命令都是\gr{vertical command}。
% 这意味着,例如,如果在垂直模式下给出 \cs{hskip} 命令,\TeX\ 将开始一个段落。
% \cs{kern} 命令可以在两种模式中使用。
% %\spoint The last  glue item in a list: backspacing
% \subsection{The last  glue item in a list: backspacing\\列表中的最后一个粘连项:回退}

% The last glue item in a list can be measured, and
% it can be removed in all modes but external vertical mode.
% The internal variables
% \csidx{lastskip} and  \csidx{lastkern} can be used
% to measure the last glob of glue in all modes;
% if the last glue was not a skip or kern respectively
% they give~\n{0pt}.
% In math mode the \cs{lastskip}
% functions as \gr{internal muglue}, but in general
% it classifies as \gr{internal glue}.
% The \cs{lastskip} and \cs{lastkern}
% are also \n{0pt} if that was the size of the last glue or
% kern item on the list.

% 可以测量列表中的最后一个粘连项,并且可以在除外部垂直模式外的所有模式中将其删除。
% 内部变量 \csidx{lastskip} 和 \csidx{lastkern} 可以用于测量所有模式中的最后一段粘连;
% 如果最后一个粘连不是 skip 或 kern,则它们的值为 \n{0pt}。
% 在数学模式中,\cs{lastskip} 作为 \gr{internal muglue},但通常归类为 \gr{internal glue}。
% 如果列表上的最后一个粘连或 kern 的大小为 \n{0pt},则 \cs{lastskip} 和 \cs{lastkern} 也为 \n{0pt}。

% The operations\label{unskip}
% \csidx{unskip} and \csidx{unkern} remove the last item of a list,
% if this is a glue or kern respectively. They have no effect
% in external vertical mode; in that case the
% best substitute is 
% \verb=\vskip-\lastskip= 
% and~\verb=\kern-\lastkern=.

% \csidx{unskip} 和 \csidx{unkern} 操作会删除列表的最后一个项,前提是它是粘连或 kern。
% 在外部垂直模式中,它们不起作用;在这种情况下,最佳替代方法是使用 \verb=\vskip-\lastskip= 和 \verb=\kern-\lastkern=。

% In the process of paragraph building \TeX\ itself performs
% an important \cs{unskip}: a~paragraph ending with a
% white line will have a space token inserted by \TeX's input processor.
% This is removed by an \cs{unskip} before the \cs{parfillskip} glue
% (see Chapter~\ref{par:end}) is inserted.

% 在构建段落的过程中,\TeX\ 本身会执行一个重要的 \cs{unskip}:
% 以空白行结尾的段落将由 \TeX\ 的输入处理器插入一个空格记号。
% 在插入 \cs{parfillskip} 粘连之前,这个空格记号会被 \cs{unskip} 移除(请参见第~\ref{par:end} 章)。

% Glue is treated by \TeX\ as a special case of leaders,
% which becomes apparent when \cs{unskip} is applied to
% leaders: they are removed.

% \TeX\ 将粘连视为指引线(leaders)的一种特殊情况,
% 这在应用 \cs{unskip} 到指引线时会变得明显:它们会被删除。

