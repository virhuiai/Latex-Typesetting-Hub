
% %\InputFile:glue
% %%%% this is input file [glue]
% %\subject[glue]  Dimensions and Glue
% \endofchapter
% \chapter{Dimensions and Glue\\尺寸和粘连}\label{glue}

% In \TeX\ vertical and horizontal white space
% can have a possibility to adjust itself through `stretching' or
% \term glue\par
% `shrinking'. An~adjustable white space is called `glue'.
% This chapter treats all technical concepts related to
% dimensions and glue, and it explains how the badness of stretching or shrinking
% a  certain amount is calculated.

% 在 \TeX\ 中,垂直和水平的空白可以通过“伸展”或“收缩”来自动调整。可调整的空白被称为“粘连”。本章介绍与尺寸和粘连相关的所有技术概念,并解释了计算伸展或收缩一定量的劣度的方法。
% \begin{inventory}
% \item [\cs{dimen}] 
%       Dimension register prefix.

%       尺寸寄存器的前缀。
% \item [\cs{dimendef}] 
%       Define a control sequence to be a synonym for
%       a~\cs{dimen} register.

%       将一个控制序列定义为\cs{dimen}寄存器的同义词。
% \item [\cs{newdimen}] 
%       Allocate an unused dimen register. 

%       分配一个未使用的尺寸寄存器。
% \item [\cs{skip}] 
%       Skip register prefix.

%       粘连寄存器的前缀。
% \item [\cs{skipdef}] 
%       Define a control sequence to be a synonym for
%       a~\cs{skip} register.

%       将一个控制序列定义为\cs{skip}寄存器的同义词。
% \item [\cs{newskip}]
%       Allocate an unused skip register.

%       分配一个未使用的粘连寄存器。
% \item [\cs{ifdim}] 
%       Compare two dimensions. 

%       比较两个尺寸。
% \item [\cs{hskip}]  
%       Insert in horizontal mode a glue item.

%       在水平模式中插入一个粘连项目。
% \item [\csidx{hfil}] 
%       Equivalent to 

%       等价于
%       \verb-\hskip 0cm plus 1fil-.
% \item [\csidx{hfilneg}] 
%       Equivalent to 

%       等价于

%       \verb-\hskip 0cm minus 1fil-.

% \item [\csidx{hfill}] 
%       Equivalent to 

%       等价于

%       \verb-\hskip 0cm plus 1fill-.

% \item [\csidx{hss}] 
%       Equivalent to 

%       等价于

%       \verb-\hskip 0cm plus 1fil minus 1fil-.

% \item [\cs{vskip}]  
%       Insert in vertical mode a glue item.

%       在垂直模式中插入一个粘连项目。
% \item [\csidx{vfil}] 
%       Equivalent to 

%       等价于

%       \verb-\vskip 0cm plus 1fil-.

% \item [\csidx{vfill}] 
%       Equivalent to 

%       等价于

%       \verb-\vskip 0cm plus 1fill-.

% \item [\csidx{vfilneg}] 
%       Equivalent to 

%       等价于

%       \verb-\vskip 0cm minus 1fil-.

% \item [\csidx{vss}] 
%       Equivalent to 
      
%       等价于

%       \verb-\vskip 0cm plus 1fil minus 1fil-.

% \item [\cs{kern}]  
%       Add a kern item to the current horizontal or vertical list.

%       在当前水平或垂直列表中添加一个紧排项目。
% \item [\cs{lastkern}] 
%       If the last item on the current list was a kern, the size of it.

%       如果当前列表的最后一个项目是紧排,则返回其尺寸。
% \item [\cs{lastskip}] 
%       If the last item on the current list was a~glue, the size of it.

%       如果当前列表的最后一个项目是粘连,则返回其尺寸。
% \item [\cs{unkern}] 
%       If the last item of the current list was a~kern, remove it.

%       如果当前列表的最后一个项目是紧排,则删除它。
% \item [\cs{unskip}] 
%       If the last item of the current list was a~glue, remove it.

%       如果当前列表的最后一个项目是粘连,则删除它。
% \item [\cs{removelastskip}]
%       Macro to append the negative of the \cs{lastskip}.

%       将\cs{lastskip}的负值附加到当前列表。
% \item [\cs{advance}] 
%       Arithmetic command to add to or subtract from
%       a~\gr{numeric variable}.

%       算术命令,用于对\gr{numeric variable}进行加法或减法运算。
% \item [\cs{multiply}] 
%       Arithmetic command to multiply a~\gr{numeric variable}.

%       算术命令,用于将\gr{numeric variable}乘以一个数。
% \item [\cs{divide}] 
%       Arithmetic command to divide a~\gr{numeric variable}.

%       算术命令,用于将\gr{numeric variable}除以一个数。
% \end{inventory} 


% %\point Definition of \gr{glue} and \gr{dimen}
% \section{Definition of \gr{glue} and \gr{dimen}\\定义 \gr{glue} 和 \gr{dimen}}

% This section gives
% the syntax of the quantities
% \gr{dimen} and \gr{glue}. 
% In the next section the practical aspects of glue are treated.

% 本节介绍 \gr{dimen} 和 \gr{glue} 这两个量的语法。
% 下一节将介绍粘连的实际应用。

% Unfortunately the terminology for glue is slightly confusing.
% The syntactical quantity~\gr{glue} is a dimension (a distance) with
% \mdqon
% possibly a stretch and/""or shrink component.
% \mdqoff
% In order to add a glob of `glue' (a white space) to a list one has to
% let a \gr{glue} be preceded by commands such as \cs{vskip}.

% 不幸的是,粘连的术语有些令人困惑。
% 语法上的量 \gr{glue} 是一个尺寸(距离),\mdqon 它可能具有伸展和/或收缩部分。\mdqoff
% 为了在列表中添加一块“粘连”(空白),需要在 \gr{glue} 之前添加诸如 \cs{vskip} 的命令。

% %\spoint Definition of dimensions
% \subsection{Definition of dimensions\\尺寸的定义}

% A~\gr{dimen} is what \TeX\ expects to see when
% it needs to indicate a dimension; it can be positive or negative.

% \gr{dimen} 是 \TeX\ 在需要表示尺寸时期望看到的内容;它可以是正数或负数。
% \begin{disp}\gr{dimen} $\longrightarrow$ \gr{optional signs}%
%      \gr{unsigned dimen}\end{disp}
% The unsigned part of a \gr{dimen} can be

% \gr{dimen} 的无符号部分可以是:
% \begin{disp}\gr{unsigned dimen} $\longrightarrow$ \gr{normal dimen}
%      $|$ \gr{coerced dimen}\nl
%      \gr{normal dimen} $\longrightarrow$ \gr{internal dimen}
%      $|$ \gr{factor}\gr{unit of measure}\nl
%      \gr{coerced dimen} $\longrightarrow$ \gr{internal glue}
%      \end{disp}
% That is, we have the following three cases:

% 也就是说,我们有以下三种情况:
% \begin{itemize} \item an \gr{internal dimen}; this is
%  any register or parameter of \TeX\ that has a \gr{dimen} value:

%  \gr{internal dimen};这是 \TeX\ 的任何具有 \gr{dimen} 值的寄存器或参数:
%  \begin{disp}\PopIndentLevel\gr{internal dimen} $\longrightarrow$
%       \gr{dimen parameter}\nl
%       \indent $|$ \gr{special dimen} $|$ \cs{lastkern}\nl
%       \indent $|$ \gr{dimendef token} $|$ \cs{dimen}\gr{8-bit number}\nl
%       \indent $|$ \cs{fontdimen}\gr{number}\gr{font}\nl
%       \indent $|$ \gr{box dimension}\gr{8-bit number}\nl
%       \gr{dimen parameter} $\longrightarrow$ \cs{boxmaxdepth}\nl
%       \indent $|$ \cs{delimitershortfall} $|$ \cs{displayindent}\nl
%       \indent $|$ \cs{displaywidth} $|$ \cs{hangindent}\nl
%       \indent $|$ \cs{hfuzz} $|$ \cs{hoffset} $|$ \cs{hsize}\nl
%       \indent $|$ \cs{lineskiplimit} $|$ \cs{mathsurround}\nl
%       \indent $|$ \cs{maxdepth} $|$ \cs{nulldelimiterspace}\nl
%       \indent $|$ \cs{overfullrule} $|$ \cs{parindent}\nl
%       \indent $|$ \cs{predisplaysize} $|$ \cs{scriptspace}\nl
%       \indent $|$ \cs{splitmaxdepth} $|$ \cs{vfuzz}\nl
%       \indent $|$ \cs{voffset} $|$ \cs{vsize}
%  \end{disp}
% \item  a dimension denotation, 
%  consisting of \gr{factor}\gr{unit of measure},
%  for example \verb>0.7\vsize>; or

%  尺寸表示,由 \gr{factor}\gr{unit of measure} 组成,例如 \verb>0.7\vsize>;
% \item an \gr{internal glue} (see below) 
%  coerced to a dimension by omitting
%  the stretch and shrink components, for example \cs{parfillskip}.

%  通过省略伸展和收缩部分将 \gr{internal glue} 强制转换为尺寸,例如 \cs{parfillskip}。
% \end{itemize}

% A dimension denotation is a somewhat complicated entity:

% 尺寸表示法是一个相对复杂的实体:
% \begin{itemize} \item a \gr{factor} is an integer denotation,
%  a decimal constant denotation (a number with an integral and
%  a fractional part),
%  or an \gr{internal integer}

%  \gr{factor} 可以是一个整数表示法、一个十进制常量表示法(一个由整数和小数部分组成的数字)或一个 \gr{internal integer}。
%  \begin{disp}\PopIndentLevel
%       \gr{factor} $\longrightarrow$ \gr{normal integer} 
%       $|$ \gr{decimal constant}\nl
%       \gr{normal integer} $\longrightarrow$ \gr{integer denotation}\nl
%       \indent $|$ \gr{internal integer}\nl
%       \gr{decimal constant} $\longrightarrow$ \n{.$_{12}$}
%       $|$ \n{,$_{12}$}\nl
%       \indent $|$ \gr{digit}\gr{decimal constant}\nl
%       \indent $|$ \gr{decimal constant}\gr{digit}
%  \end{disp}
%  An internal integer is a parameter that is `really' an
% \alt
%  integer (for instance, \cs{count0}), and not coerced from a dimension or glue.
%  See Chapter~\ref{number}
%  for the definition of various kinds of integers.

%  \gr{internal integer} 是一个“真正”的整数参数(例如,\cs{count0}),而不是从尺寸或粘连强制转换而来。有关各种整数的定义,请参见第~\ref{number} 章。
% \item a \gr{unit of measure} can be 
%  a \gr{physical unit}, that is, an ordinary unit such as~\n{cm} 
%  (possibly preceded by \n{true}),
%  an internal unit such as~\n{em}, but also an \gr{internal integer}
%  (by conversion to scaled points),
%  an \gr{internal dimen}, or an \gr{internal glue}.

%  \gr{unit of measure} 可以是 \gr{physical unit},即普通的单位,例如 \n{cm}(可能前面带有 \n{true}),也可以是内部单位,例如 \n{em},还可以是 \gr{internal integer}(通过转换为 scaled point),\gr{internal dimen} 或 \gr{internal glue}。
%  \begin{disp}\PopIndentLevel
%       \gr{unit of measure} $\longrightarrow$
%       \gr{optional spaces}\gr{internal unit}\nl
%       \indent $|$ 
%       \gr{optional \n{true}}\gr{physical unit}\gr{one optional space}\nl 
%       \gr{internal unit} $\longrightarrow$ 
%       \n{em}\gr{one optional space}\nl
%       \indent $|$ \n{ex}\gr{one optional space}
%               $|$ \gr{internal integer}\nl
%       \indent $|$ \gr{internal dimen} $|$ \gr{internal glue}
%       \end{disp}
% \end{itemize}

% Some \gr{dimen}s are called \gr{special dimen}s:\label{special:dimen:list}

% 一些 \gr{dimen} 被称为 \gr{special dimen}:
% \begin{disp}\gr{special dimen} $\longrightarrow$ \cs{prevdepth}\nl
%      \indent $|$ \cs{pagegoal} $|$ \cs{pagetotal} $|$ \cs{pagestretch}\nl
%      \indent $|$ \cs{pagefilstretch} $|$ \cs{pagefillstretch}\nl
%      \indent $|$ \cs{pagefilllstretch} $|$ \cs{pageshrink} $|$ \cs{pagedepth}
%      \end{disp}
% An assignment to any of these is
% called an \gr{intimate assignment}, and it is automatically
% global (see Chapter~\ref{group}). The meaning of these 
% dimensions is explained in Chapter \ref{page:break}, with the
% exception of \cs{prevdepth} which is treated in
% Chapter~\ref{baseline}.

% 对这些任何一个的赋值被称为 \gr{intimate assignment},它是自动全局的(见第~\ref{group} 章)。这些尺寸的含义在第~\ref{page:break} 章中解释,\cs{prevdepth} 是一个例外,它在第~\ref{baseline} 章中处理。


% %\spoint Definition of glue
% \subsection{Definition of glue\\粘连的定义}

% A \gr{glue} is either some form of glue variable, or
% a glue denotation with explicitly indicated stretch and
% shrink. Specifically,

% \gr{glue} 可以是某种形式的粘连变量,或者带有明确指定的伸长量和收缩量的粘连表示。
% 具体而言,
% \begin{disp}\gr{glue} $\longrightarrow$ \gr{optional signs}\gr{internal glue}
%      $|$ \gr{dimen}\gr{stretch}\gr{shrink}\nl
%      \gr{internal glue} $\longrightarrow$ \gr{glue parameter}
%      $|$ \cs{lastskip}\nl 
%      \indent $|$ \gr{skipdef token} $|$ \cs{skip}\gr{8-bit number}\nl
%      \gr{glue parameter} $\longrightarrow$ \cs{abovedisplayshortskip}\nl
%      \indent $|$ \cs{abovedisplayskip} $|$ \cs{baselineskip}\nl
%      \indent $|$ \cs{belowdisplayshortskip} $|$ \cs{belowdisplayskip}\nl
%      \indent $|$ \cs{leftskip} $|$ \cs{lineskip} $|$ \cs{parfillskip}
%              $|$ \cs{parskip}\nl
%      \indent $|$ \cs{rightskip} $|$ \cs{spaceskip}
%              $|$ \cs{splittopskip} $|$ \cs{tabskip}\nl
%      \indent $|$ \cs{topskip} $|$ \cs{xspaceskip}
% \end{disp}
% The stretch and shrink components in a glue denotation
% are optional, but when both are specified they have to
% be given in sequence; they are defined as

% 粘连表示中的伸长量和收缩量是可选的,但当两者都指定时,必须按顺序给出;
% 它们的定义如下:
% \begin{disp}
% \gr{stretch} $\longrightarrow$ \n{plus} \gr{dimen}
%       $|$ \n{plus}\gr{fil dimen} $|$ \gr{optional spaces}\nl
% \gr{shrink} $\longrightarrow$ \n{minus} \gr{dimen}
%       $|$ \n{minus}\gr{fil dimen} $|$ \gr{optional spaces}\nl
% \gr{fil dimen} $\longrightarrow$ \gr{optional signs}\gr{factor}%
%      \gr{fil unit}\gr{optional spaces}\nl
% \gr{fil unit} $\longrightarrow$ \n{ $|$ fil $|$ fill $|$ filll}
% \end{disp}

% The actual definition of \gr{fil unit} is recursive
% (see Chapter~\ref{gramm}), but these are the only valid
% possibilities.

% \gr{fil unit} 的实际定义是递归的(见第~\ref{gramm} 章),
% 但这些是唯一有效的可能性。
% %\spoint Conversion of \gr{glue} to \gr{dimen}
% \subsection{Conversion of \gr{glue} to \gr{dimen}\\\gr{glue} 转换为 \gr{dimen}}

% The grammar rule

% 语法规则
% \begin{disp}\gr{dimen} $\longrightarrow$
%      \gr{factor}\gr{unit of measure}
% \end{disp}
% has some noteworthy consequences, caused by the fact
% that a \gr{unit of measure} need not look like a `unit of measure'
% at all (see the list above).

% 有一些值得注意的后果,这是因为 \gr{unit of measure} 不一定看起来像一个“单位”(见上面的列表)。

% For instance, from this definition we conclude that the statement

% 例如,根据这个定义,我们可以得出以下结论:
% \begin{verbatim}
% \dimen0=\lastpenalty\lastpenalty
% \end{verbatim}
% is
% syntactically correct because \cs{lastpenalty} can function
% both as an integer and as \gr{unit of measure} by taking
% its value in scaled points.
% After \verb>\penalty8> the \cs{dimen0} thus defined will
% have a size of~\n{64sp}.

% 在语法上是正确的,因为 \cs{lastpenalty} 可以同时作为整数和\gr{unit of measure},
% 它可以在缩放点中取值。
% 在 \verb>\penalty8> 之后,这样定义的 \cs{dimen0} 将具有大小 \n{64sp}。

% More importantly, consider the case where the \gr{unit of measure} is
% an \gr{internal glue}, that is, any sort of glue parameter.
% Prefixing such a glue with a number (the \gr{factor})
% makes it a valid \gr{dimen} specification.
% Thus 

% 此外,考虑\gr{unit of measure}是\gr{internal glue}(即任何类型的粘连参数)的情况。在粘连前面加上一个数字(\gr{factor})将其作为有效的\gr{dimen}规范。因此,
% \begin{verbatim}
% \skip0=\skip1
% \end{verbatim}
% is very different
% from 

% 非常不同于\begin{verbatim}
% \skip0=1\skip1
% \end{verbatim}
% The first statement makes
% \cs{skip0} equal to \cs{skip1}, the second converts
% the \cs{skip1} to a \gr{dimen} before assigning it.
% In other words, the \cs{skip0} defined by the second statement
% has no stretch or shrink.

% 第一个语句将\cs{skip0}设置为等于\cs{skip1},而第二个语句在赋值之前将\cs{skip1}转换为\gr{dimen}。换句话说,由第二个语句定义的\cs{skip0}没有伸长或收缩的能力。


% %\spoint Registers for \cs{dimen} and \cs{skip}
% \subsection{Registers for \cs{dimen} and \cs{skip}\\用于 \cs{dimen} 和 \cs{skip} 的寄存器}

% \TeX\ has registers for storing \gr{dimen} and \gr{glue}
% values: the \csidx{dimen} and \csidx{skip} registers
% respectively. These are accessible by the expressions

% \TeX\ 有用于存储\gr{dimen}和\gr{glue}值的寄存器:\csidx{dimen} 寄存器和 \csidx{skip} 寄存器。可以通过以下表达式访问这些寄存器:
% \begin{disp}\cs{dimen}\gr{number}\end{disp} and
% \begin{disp}\cs{skip}\gr{number}\end{disp}
% As with all registers of \TeX, these registers are
% numbered~0--255.

% 与 \TeX\ 的所有寄存器一样,这些寄存器的编号为 0--255。

% Synonyms for registers can be made with the \csidx{dimendef} and
% \csidx{skipdef} commands. Their syntax is

% 可以使用 \csidx{dimendef} 和 \csidx{skipdef} 命令创建寄存器的同义词。它们的语法为:
% \begin{Disp}\cs{dimendef}\gr{control sequence}\gr{equals}\gr{8-bit number}
% \end{Disp}
% and 
% \begin{Disp}\cs{skipdef}\gr{control sequence}\gr{equals}\gr{8-bit number}\end{Disp}
% For example, after \verb-\skipdef\foo=13- using \cs{foo}
% is equivalent to using \cs{skip13}.

% 例如,通过执行 \verb-\skipdef\foo=13-,可以使用 \cs{foo} 来代替 \cs{skip13}。

% Macros \csidx{newdimen} and \csidx{newskip} exist in plain \TeX
% for allocating an unused dimen or skip register.
% These macros are defined to be \cs{outer} in the plain format.

% 在 plain \TeX\ 中,有宏 \csidx{newdimen} 和 \csidx{newskip} 用于分配未使用的尺寸寄存器或粘连寄存器。在 plain format 中,这些宏被定义为\cs{outer}。


% %\spoint Arithmetic: addition 
% \subsection{Arithmetic: addition\\算术:加法}

% As for integer variables, arithmetic operations exist for
% \cstoidx advance\par\term  glue!arithmetic on\par\term arithmetic! on glue\par
% dimen, glue, and muglue (mathematical glue; see page~\pageref{muglue})
% variables.

% 对于整数变量,可以进行尺寸、粘连和数学粘连(mathematical glue)变量的算术运算。

% The expressions

% 表达式
% \begin{Disp}\cs{advance}\gr{dimen variable}\gr{optional \n{by}}%
%      \gr{dimen}\nl
%      \cs{advance}\gr{glue variable}\gr{optional \n{by}}%
%      \gr{glue}\nl
%      \cs{advance}\gr{muglue variable}\gr{optional \n{by}}%
%      \gr{muglue}\end{Disp}
% add to the size of a dimen, glue, or muglue.

% 可以将尺寸、粘连或数学粘连的大小增加。


% Advancing a \gr{glue variable} by \gr{glue} is done by
% adding the natural sizes, and the stretch and shrink components.
% Because \TeX\ converts between \gr{glue} and \gr{dimen},
% it is possible to write for instance

% 通过将自然尺寸、伸展和收缩部分相加,可以将 \gr{glue variable} 的粘连增加 \gr{glue}。
% 由于 \TeX\ 可以在 \gr{glue} 和 \gr{dimen} 之间进行转换,
% 因此可以写成如下的形式:
% \begin{verbatim}
% \advance\skip1 by \dimen1
% \end{verbatim}
% or
% \begin{verbatim}
% \advance\dimen1 by \skip1
% \end{verbatim}
% In the first case  \cs{dimen1} is coerced to \gr{glue} without
% stretch or shrink; in the second case the \cs{skip1} is coerced
% to a \gr{dimen} by taking its natural size.

% 在第一种情况下,\cs{dimen1} 被强制转换为没有伸展或收缩的 \gr{glue};
% 在第二种情况下,\cs{skip1} 被强制转换为 \gr{dimen},通过取其自然尺寸。



% %\spoint Arithmetic: multiplication and division
% \subsection{Arithmetic: multiplication and division\\算术:乘法和除法}

% Multiplication and division operations exist for glue
% \cstoidx multiply\par\cstoidx divide\par
% and dimensions. One may for instance write

% 粘连和尺寸的乘法和除法操作存在。
% 例如,可以写成如下形式:
% \begin{verbatim}
% \multiply\skip1 by 2
% \end{verbatim}
% which multiplies the natural size, and the stretch and shrink
% components of \cs{skip1} by~2.

% 这将把 \cs{skip1} 的自然尺寸、伸展和收缩部分乘以 2。

% The second operand of a \cs{multiply} or \cs{divide}
% operation can only be a \gr{number}, that is, an integer.
% Introducing the notion of \gr{numeric variable}:

% \cs{multiply} 或 \cs{divide} 操作的第二个操作数只能是 \gr{number},即整数。
% 引入 \gr{numeric variable} 的概念:
% \begin{disp}\gr{numeric variable} $\longrightarrow$ \gr{integer variable}
%      $|$ \gr{dimen variable} \nl
%      \indent $|$ \gr{glue variable} $|$ \gr{muglue variable}\end{disp}
% these operations take the form

% 这些操作的形式如下:
% \begin{Disp}\cs{multiply}\gr{numeric variable}\gr{optional \n{by}}%
% \gr{number}\end{Disp} 
% and
% \begin{Disp}\cs{divide}\gr{numeric variable}\gr{optional \n{by}}%
% \gr{number}\end{Disp}

% Glue and dimen can be multiplied by 
% non-integer quantities:

% 可以使用非整数的数量来乘以粘连和尺寸:
% \begin{verbatim}
% \skip1=2.5\skip2
% \dimen1=.78\dimen2
% \end{verbatim}
% However, in the first line the \cs{skip2} is first coerced
% to a \gr{dimen} value by omitting its stretch and shrink.

% 然而,在第一行中,\cs{skip2} 首先被强制转换为 \gr{dimen} 值,通过省略其伸展和收缩。


% %\point More about dimensions
% \section{More about dimensions\\关于尺寸的更多信息}

% %\spoint Units of measurement
% \subsection{Units of measurement\\度量单位}

% In \TeX\ dimensions can be indicated in
% \term units of measurement\par

% 在 \TeX 中,尺寸可以用以下度量单位表示:
% \begin{description} \item [centimetre]
%     denoted \n{cm} or 

%     [厘米] 用 \n{cm} 表示,这是国际标准单位制(SI)的单位,即国际标准度量单位制度。
% \item [millimetre]
% 	denoted \n{mm}; these are SI~units ({\italic Syst\`eme International
% 	d'Unit\'es}, the
% 	international system of standard units of measurements).

%       [毫米] 用 \n{mm} 表示;这在英美世界更常见。一毫米等于0.1厘米。
% \item [inch]
% \n{in}; more common in the Anglo-American world.
% One inch is 2.54~centimetres.

% [英寸] 用 \n{in} 表示;这在英美世界更常见。一英寸等于2.54厘米。
% \item [pica]
%     denoted \n{pc}; one pica is 12~points.

%     [派卡] 用 \n{pc} 表示;一派卡等于12点。
% \item [point]
%     denoted \n{pt}; the common system
% for Anglo-American printers. One inch is 72.27 points.

% [点] 用 \n{pt} 表示;这是英美打印机的常用单位。一英寸等于72.27点。
% \item [didot point]
%     denoted \n{dd}; the common system for continental European printers.
%     Furthermore, 1157 didot points are 1238~points.

%     [迪多点] 用 \n{dd} 表示;这是欧洲大陆打印机的常用单位。此外,1157迪多点等于1238点。
% \item [cicero]
%     denoted \n{cc}; one cicero is 12~didot points.

%     [西塞罗] 用 \n{cc} 表示;一西塞罗等于12迪多点。
% \item [big point]
%     denoted \n{bp}; one inch is 72 big points.

%     [大点] 用 \n{bp} 表示;一英寸等于72大点。
% \item [scaled point]
%     denoted \n{sp}; this is the smallest unit in \TeX, and all measurements
%     are integral multiples of one scaled point.
%     There are $65\,536$ scaled points in a~point.

%     [缩放点] 用 \n{sp} 表示;这是 \TeX 中最小的单位,所有尺寸都是缩放点的整数倍。一点中有$65,536$个缩放点。
% \end{description}

% Decimal fractions can be written using both the
% Anglo-American system with the decimal point
% (for example, \n{1in}=\n{72.27pt})
% and the continental European system with a decimal
% comma; \n{1in}=\n{72,27pt}.

% 小数可以用英美制度的小数点(例如,\n{1in}=\n{72.27pt})或欧洲大陆制度的小数逗号表示;\n{1in}=\n{72,27pt}。

% Internally \TeX\ works with multiples of a smallest 
% dimension: the  scaled point.
% Dimensions larger (in absolute value) than $2^{30}-1$\n{sp},
% which is about 5.75~metres or 18.9~feet, are illegal.

% 在内部,\TeX 使用最小尺寸的倍数:缩放点。绝对值大于$2^{30}-1$\n{sp}的尺寸是非法的,这大约相当于5.75米或18.9英尺。

% Both the pica system and the didot system are of French
% origin: in 1737 the type founder Pierre Simon Fournier
% introduced typographical points based on the French foot.
% Although at first he introduced a system based on lines and
% points, he later took the point as unit:
% there are 72 points in an inch,
% which is one-twelfth of a foot. 
% About 1770 another founder, Fran\c{c}ois Ambroise Didot, introduced
% points based on the more common, and slightly longer,
% `pied du roi'.

% 派卡制度和迪多制度都源自法国:1737年,字体铸造者皮埃尔·西蒙·福尼埃引入了基于法国尺的印刷点。尽管他最初引入了基于行和点的系统,但后来他采用了点作为单位:一英寸等于72点,即一英尺的十二分之一。大约在1770年,另一位字体铸造者弗朗索瓦·安布瓦兹·迪多引入了基于更普遍、稍长的“国王尺”的点。
 


% %\spoint Dimension testing
% \subsection{Dimension testing\\尺寸测试}

% Dimensions and natural sizes of glue can be compared with
% the \cs{ifdim} test. This takes the form

% 可以使用 \cs{ifdim} 测试来比较尺寸和粘连的自然尺寸。其形式为:
% \begin{disp}\cs{ifdim}\gr{dimen$_1$}\gr{relation}\gr{dimen$_2$}\end{disp}
% where the relation can be an \n>, \n<, or~\n= token, 
% all of category~12.

% 其中关系可以是 \n>、\n< 或 \n= 符号,它们的类别码均为 12。


% %\spoint Defined dimensions
% \subsection{Defined dimensions\\已定义的尺寸}

% \begin{inventory}
% \item [\cs{z@}]
%  \n{0pt}

% \item [\cs{maxdimen}] 
%       \n{16383.99999pt}; the largest legal dimension.

%       最大合法尺寸。
% \end{inventory}

% These \gr{dimen}s are predefined in the plain format;
% for instance 

% 这些 \gr{dimen} 在 plain 格式中是预定义的;
% 例如:\begin{verbatim}
% \newdimen\z@ \z@=0pt
% \end{verbatim}
% Using such abbreviations for commonly used dimensions
% has at least two advantages. First of all it saves main memory
% if such a dimension occurs in a macro: a control sequence
% is one token, whereas a string such as \n{0pt} takes three.
% Secondly, it saves time in processing, as \TeX\ does not need
% to perform conversions to arrive at the correct type of
% object.

% 使用这样的缩写形式表示常用尺寸至少有两个优点。
% 首先,如果这样的尺寸出现在宏中,它可以节省主存储器:
% 控制序列只是一个记号,而像 \n{0pt} 这样的字符串需要三个记号。
% 其次,它可以节省处理时间,因为 \TeX\ 不需要进行转换以确定正确的对象类型。

% Control sequences such as \cs{z@}
% are only available to a user who changes the
% category code of the `at' sign. Ordinarily, these control sequences
% appear only in the macros defined in packages such as the
% plain format.

% 诸如 \cs{z@} 的控制序列仅对更改“at”符号的类别码的用户可用。
% 通常,这些控制序列仅出现在像 plain 格式等包中定义的宏中。

