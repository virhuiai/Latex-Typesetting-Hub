
% %\InputFile:macro
% %%%% this is input file [macro]
% %\subject[macro]  Macros
% \endofchapter
% \chapter{Macros\\宏}\label{macro}

% Macros are \TeX's abbreviation mechanism for sequences of commands
% that are needed more than once,
% somewhat like procedures in ordinary programming languages.
% \TeX's parameter mechanism, however, is quite unusual.
% This chapter explains how \TeX\ macros work. It also
% treats the commands \cs{let} and~\cs{futurelet}.

% 宏是 \TeX 中用于表示需要多次使用的命令序列的缩写机制,有点类似于普通编程语言中的过程。然而,\TeX 的参数机制非常特殊。本章将解释 \TeX 的宏如何工作。还将介绍命令 \cs{let} 和 \cs{futurelet}。
% \begin{inventory}
% \item [\cs{def}] 
%       Start a macro definition.

%       开始宏定义。
% \item [\cs{gdef}] 
%       Synonym for \verb-\global\def-.

%       \verb-\global\def- 的同义词。
% \item [\cs{edef}] 
%       Start a macro definition; 
%       the replacement text is expanded at definition time.
%       This command is treated also in the next chapter.

%       开始宏定义;替代文本在定义时会被展开。
% 这个命令也在下一章中介绍。
% \item [\cs{xdef}] 
%       Synonym for \verb-\global\edef-.

%       \verb-\global\edef- 的同义词。
% \item [\cs{csname}] 
%       Start forming the name of a control sequence.

%       开始构建控制序列的名称。
% \item [\cs{endcsname}] 
%       Stop forming the name of a control sequence.

%       停止构建控制序列的名称。
% \item [\cs{global}] 
%       Make the next definition, arithmetic statement,
%       or assignment global.

%       使下一个定义、算术语句或赋值成为全局的。
% \item [\cs{outer}] 
%       Prefix indicating that the macro being defined 
%       can be used on the `outer' level only.

%       前缀,表示正在定义的宏只能在“外部”级别使用。
% \item [\cs{long}] 
%       Prefix indicating that the arguments of the macro being defined
%       may contain \cs{par} tokens.

%       前缀,表示正在定义的宏的参数可能包含 \cs{par} 记号。
% \item [\cs{let}] 
%       Define a control sequence to be equivalent to the next token.

%       将一个控制序列定义为与下一个记号等价。
% \item [\cs{futurelet}] 
%       Define a control sequence to be equivalent to
%       the token after the next token.

%       将一个控制序列定义为与下一个记号之后的记号等价。
% \end{inventory}


% %\point Introduction
% \section{Introduction\\引言}

% A macro is basically a sequence of tokens that has
% \term macro\par
% been abbreviated into a control sequence.
% Statements starting with (among others) \cs{def}
% are called {\italic macro definitions}\alt, and
% writing 

% 宏基本上是一系列记号的缩写,这些记号被缩写为一个控制序列。
% 以 \cs{def} 等开头的语句称为{\em 宏定义}\alt,而写作:\begin{verbatim}
% \def\abc{\de f\g}
% \end{verbatim}
% defines the macro \cs{abc},
% with the {\italic replacement text\/} \verb>\de f\g>.
% Macros can be used in this way to abbreviate
% pieces of text or sequences of commands
% that have to be given more than once.
% Any time that \TeX's expansion processor
% encounters the control sequence \cs{abc},
% it replaces it by the replacement text.

% 则定义了宏 \cs{abc},其{\em 替换文本}为 \verb>\de f\g>。
% 通过这种方式,可以使用宏来缩写需要多次给出的文本片段或命令序列。
% 每当 \TeX\ 的展开处理器遇到控制序列 \cs{abc} 时,它将其替换为替换文本。

% If a macro should be sensitive to the context
% where it is used, it can be defined with parameters:

% 如果宏应该对其使用的上下文敏感,可以使用参数来定义宏:
% \begin{verbatim}
% \def\PickTwo#1#2{(#1,#2)}
% \end{verbatim}
% takes two arguments and reproduces them in parentheses.
% The call \cs{PickTwo 12} gives `(1,2)'.

% 该宏接受两个参数,并在括号中重现它们。
% 调用 \cs{PickTwo 12} 将得到 `(1,2)'。

% The activity of substituting the replacement text
% for a macro is called {\italic macro expansion}.

% 将替换文本代替宏的过程称为{\em 宏展开}。

% %\point Layout of a macro definition
% \section{Layout of a macro definition\\宏定义的布局}

% A macro definition consists of, in sequence,
% \term definition !macro\par

% 宏定义由以下内容按顺序组成:
% \begin{enumerate} \item any number of \cs{global},
% \cs{long}, and \cs{outer} prefixes,

% 任意数量的 \cs{global}、\cs{long} 和 \cs{outer} 前缀;
% \item a \gr{def} control sequence, or anything
% that has been \cs{let} to one,

% 一个 \gr{def} 控制序列,或者是通过 \cs{let} 定义的任何内容;
% \item a control sequence or active character to be defined, 

% 要定义的控制序列或活动字符;
% \item possibly a \gr{parameter text} specifying among other things
% how many parameters the macro has, and

% 可能的 \gr{parameter text},其中包括宏的参数个数等信息;
% \item a replacement text enclosed in explicit character tokens
% with category codes 1 and~2, by default \verb-{- and~\verb-}-
% in plain \TeX.

% 用类别码为 1 和 2 的显式字符记号括起来的替换文本,默认情况下,在 plain \TeX\ 中为 \verb-{- 和 \verb-}-。
% \end{enumerate}

% After a macro definition is completed, any saved \cs{afterassignment}
% token (see section~\ref{sec:afterassignment}) is inserted.

% 在完成宏定义后,将插入任何已保存的 \cs{afterassignment} 记号(见第~\ref{sec:afterassignment} 节)。

% The `expanding' definitions \cs{edef} and \cs{xdef}
% are treated in Chapter~\ref{expand}.

% `展开式'定义 \cs{edef} 和 \cs{xdef} 将在第~\ref{expand} 章中介绍。




% %\point Prefixes
% \section{Prefixes\\前缀}

% There are three prefixes that alter the status of the
% \term prefixes !macro\par
% macro definition: 

% 有三个前缀可以改变宏定义的状态:\begin{description}
% \item [\csidx{global}]
% If the definition occurs inside a  group, this prefix
% makes the definition global.
% This prefix can also be used for assignments other than
% macro definitions; in fact,
% for macro definitions abbreviations exist obviating the
% use of \cs{global}:

% 如果定义发生在一个组内,这个前缀将使得定义变为全局的。
% 这个前缀也可以用于除宏定义之外的赋值;实际上,对于宏定义,存在缩写形式,可以避免使用 \cs{global}:
% \begin{disp}\verb>\gdef\foo...>\quad is equivalent to\quad \verb>\global\def\foo...>
% \end{disp} and
% \begin{disp}\verb>\xdef\foo...>\quad is equivalent to\quad \verb>\global\edef\foo...>
% \end{disp}

% If the parameter \cs{globaldefs}
% is positive, all assignments are
% implicitly global;
% if \cs{globaldefs} is negative any \cs{global} prefixes are
% ignored,
% and \cs{gdef} and \cs{xdef} make local definitions
% (see Chapter~\ref{group}).

% 如果参数 \cs{globaldefs} 是正值,所有的赋值都是隐式全局的;如果 \cs{globaldefs} 是负值,则忽略任何 \cs{global} 前缀,并且 \cs{gdef} 和 \cs{xdef} 的定义是局部的(参见第~\ref{group}~章)。


% \item [\cs{outer}]
% The mechanism of `outer' macros is supposed to facilitate
% \term macro !outer\par\cstoidx outer\par
% locating (among other errors) unbalanced braces: an \cs{outer}
% macro is supposed
% to  appear only in non-embedded contexts.
% To be precise, it is not allowed to occur 

% `outer' 宏的机制旨在方便定位(除其他错误之外)不平衡的大括号:\cs{outer} 宏应仅出现在非嵌套的上下文中。
% 更加准确地说,它不允许出现在以下情况中:
% \begin{itemize}
% \item in macro replacement texts (but it can appear in
%     for instance \cs{edef} after 
%     \cs{noexpand}, and after \cs{meaning}),

%     宏替换文本中(但它可以出现在例如 \cs{noexpand} 之后的 \cs{edef} 中,以及在 \cs{meaning} 之后);
% \item in parameter texts,

% 参数文本中;
% \item in skipped conditional text,

% 被跳过的条件文本中;
% \item in alignment preambles, and

% 对齐的导言中;
% \item in the \gram{balanced text} of a \cs{message}, \cs{write},
% et cetera. 

% \cs{message}、\cs{write} 等命令的 \gram{balanced text} 中。
% \end{itemize}
% For certain applications, however, it is inconvenient
% that some of the plain macros  are outer, 
% in particular macros such as \cs{newskip}. One remedy is to
% redefine them, without the `outer' option, which
% is done for instance in \LaTeX, but  cleverer tricks are possible.

% 然而,对于某些应用,某些 plain 宏是 outer 的,并不方便,特别是像 \cs{newskip} 这样的宏。一种解决方法是重新定义它们,去掉 `outer' 选项,例如在 \LaTeX\ 中就是这样做的,但是还有更巧妙的技巧可用。
% \item [\cs{long}]
% Ordinarily, macro parameters are not supposed to contain
% \cstoidx long\par
% \cs{par} tokens. This restriction is useful (much more so
% than the \cs{outer} definitions) in locating
% forgotten closing braces. 
% For example, \TeX\ will complain about a `runaway argument'
% \message{Example on}
% in the following sequence:

% 通常,宏参数不应包含 \cstoidx long\par \cs{par} 记号。这个限制在查找漏掉的闭合花括号方面非常有用(比 \cs{outer} 定义更有用)。例如,在以下序列中,\TeX\ 会报错 "runaway argument"(无限逃逸的参数):
% \begin{verbatim}
% \def\a#1{ ... #1 ... }
% \a {This sentence should be in braces.

% And this is not supposed to be part of the argument
% \end{verbatim}
% \message{one page}
% The empty line generates a \cs{par}, which most of the times
% means that a closing brace has been forgotten.

% 空行生成一个 \cs{par},大多数情况下表示忘记了闭合花括号。

% If arguments to a particular macro should be allowed
% to contain \cs{par} tokens,  then the macro must be declared
% to be \cs{long}. 

% 如果某个特定宏的参数允许包含 \cs{par} 记号,则该宏必须声明为 \cs{long}。
% \end{description}

% The \cs{ifx} test for equality of tokens 
% (see Chapter~\ref{if}) takes prefixes into
% account when testing whether two tokens have the same definition.

% \cs{ifx} 测试两个记号是否相等(参见第~\ref{if}~章)时,会考虑前缀,以确定两个记号是否具有相同的定义。
% \begin{comment}
% With a little ingenuity it is possible 
% for \cs{par} tokens to sneak into macro arguments anyway.
% Consider the example

% 通过一些巧妙的方法,\cs{par} 记号仍然有可能 sneaking into macro arguments。考虑以下示例:
% \begin{verbatim}
% \def\a#1\par!{ ... }
% \a bc\par ef\par!
% \end{verbatim}
% Here the macro \cs{a} is not \cs{long}, but the argument
% is \verb>bc\par ef>, which contains a \cs{par} token.
% However,
% this is of no importance in general.

% 这里宏 \cs{a} 不是 \cs{long} 的,但参数是 \verb>bc\par ef>,其中包含一个 \cs{par} 记号。
% 然而,这在一般情况下并不重要。
% \end{comment}

