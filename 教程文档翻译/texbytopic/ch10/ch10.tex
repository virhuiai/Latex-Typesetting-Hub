
%\InputFile:group
%%%% this is input file [group]
%\subject[group] Grouping
\endofchapter
\chapter{Grouping\\分组}\label{group}

\TeX\ has a grouping mechanism that is able to confine most
changes to a~particular locality. This chapter explains
what sort of actions can be local, and how groups are formed.

\TeX\ 具有一种分组机制,能够将大多数更改限制在特定的局部范围内。本章解释了哪些操作可以是局部的,以及如何形成分组。
\begin{inventory}
\item [\cs{bgroup}] 
Implicit beginning of group character.

隐式开始分组字符。
\item [\cs{egroup}] 
Implicit end of group character.

隐式结束分组字符。
\item [\cs{begingroup}] 
Open a group that must be closed with \cs{endgroup}.

打开一个必须用 \cs{endgroup} 关闭的分组。
\item [\cs{endgroup}] 
Close a group that was opened with \cs{begingroup}.

关闭用 \cs{begingroup} 打开的分组。
\item [\cs{aftergroup}] 
Save the next token for insertion after the current group ends.

保存下一个记号,在当前分组结束后插入。
\item [\cs{global}] 
Make assignments, macro definitions, and arithmetic global.

使赋值、宏定义和算术操作具有全局作用域。
\item [\cs{globaldefs}] 
Parameter for overriding \cs{global} prefixes.
\IniTeX\ default:~0.

用于覆盖 \cs{global} 前缀的参数。
\IniTeX\ 默认值为~0。
\end{inventory}



%\point The grouping mechanism
\section{The grouping mechanism\\分组机制}

A group is a sequence of tokens starting with a
\term grouping\par
`beginning of group' token,
and ending with an `end of group'
token, and in which all such tokens are properly balanced. 

分组是由“开始分组”记号开始、以“结束分组”记号结束的一系列记号,并且其中所有这些记号都是正确配对的。

The grouping mechanism of \TeX\ is not the same as
the block structure
of ordinary programming languages.
Most languages with block structure are only able  to have
local definitions. \TeX's grouping mechanism is stronger: 
most assignments made inside a group
are local to that group unless explicitly indicated otherwise,
and outside the group old values are restored.

\TeX\ 的分组机制与普通编程语言的块结构不同。
大多数具有块结构的语言只能具有局部定义。
\TeX\ 的分组机制更强大:在分组内进行的大多数赋值
在该分组内是局部的,除非另有明确指示,
而在分组外部会恢复旧值。

An example of local definitions

局部定义的示例
\begin{verbatim}
{\def\a{b}}\a
\end{verbatim}
gives an `undefined control sequence'
message because \cs{a} is only defined inside the group.
Similarly, the code

会显示“undefined control sequence”消息,因为 \cs{a} 仅在该分组内定义。
类似地,代码
\begin{verbatim}
\count0=1 {\count0=2 } \showthe\count0
\end{verbatim}
will display the value~1; the assignment made inside the group
is undone at the end of the group.

将显示值为~1;在分组内进行的赋值在分组结束时被撤销。

Bookkeeping of values that are to be restored outside the group
is done through the mechanism
\term save stack\par
of the `save stack'. Overflow of the save stack is treated
in Chapter~\ref{error}. The save stack is also used for
a few other purposes: in calls such as \hbox{\verb>\hbox to 100pt{...}>}
the specification \hbox{\n{to 100pt}} is put on the save
stack before a new level of grouping is opened.

用于在分组外部恢复值的值管理是通过“保存栈”的机制完成的。
保存栈的溢出在第~\ref{error}~章中进行了处理。
保存栈还用于其他一些目的:在诸如 \hbox{\verb>\hbox to 100pt{...}>} 的调用中,
规范 \hbox{\n{to 100pt}} 被放在保存栈上,然后再打开一个新级别的分组。

In order to prevent a lot of trouble with the save stack,
\IniTeX\ does not allow dumping a format inside a group.
The \cs{end} command is allowed to occur inside a group,
but \TeX\ will give a diagnostic message about this.

为了防止保存栈出现大量问题,
\IniTeX\ 不允许在分组内转储格式。
\cs{end} 命令允许出现在分组内,
但是 \TeX\ 会给出关于此的诊断消息。

The \cs{aftergroup} control sequence saves a token for
insertion after the current group. Several tokens can be
set aside by this command, and they are inserted in the left-to-right
order in which they were stated.
This is treated in Chapter~\ref{expand}.

\cs{aftergroup} 控制序列保存一个记号,在当前分组之后插入该记号。
可以用这个命令设置多个记号,并按照它们声明的从左到右的顺序插入。
这将在第~\ref{expand}~章中进行讨论。


%\point[global:assign] Local and global assignments
\section{Local and global assignments\\局部和全局赋值}
\label{global:assign}

An assignment or macro definition
is usually made global by prefixing it with \csidx{global},
\term statements !local\par\term statements !global\par
\term local statements\par\term global statements\par
but non-zero values of the \gr{integer parameter}
\csidx{globaldefs} override \cs{global}
specifications: if \cs{globaldefs} is positive every assignment
is implicitly prefixed with \cs{global}, and if
\cs{globaldefs} is negative, \cs{global} is
ignored. Ordinarily this parameter is zero.

通常,通过在赋值或宏定义前加上 \csidx{global} 来使其成为全局的,
但是当 \gr{integer parameter} \csidx{globaldefs} 的非零值覆盖 \cs{global} 的规定时,
赋值操作会受到影响:
如果 \cs{globaldefs} 是正数,则每个赋值都会隐式加上 \cs{global};
如果 \cs{globaldefs} 是负数,则会忽略 \cs{global}。
通常,此参数的值为零。

Some assignment are always global: the \gr{global assignment}s are

某些赋值始终是全局的:\gr{global assignment} 包括以下内容:
\begin{description}%\FlushRight:no
\item [\gr{font assignment}]
assignments involving \cs{fontdimen}, \cs{hyphenchar}, 
and \cs{skew\-char}.

涉及到 \cs{fontdimen}、\cs{hyphenchar} 和 \cs{skew\-char} 的赋值。
\item [\gr{hyphenation assignment}]
\cs{hyphenation} and \cs{patterns} commands
(see Chapter~\ref{line:break}).

\cs{hyphenation} 和 \cs{patterns} 命令
(见第~\ref{line:break} 章)。
\item [\gr{box size assignment}]
altering box dimensions with \cs{ht}, \cs{dp}, and~\cs{wd}
(see Chapter~\ref{boxes}).

使用 \cs{ht}、\cs{dp} 和 \cs{wd} 改变盒子的尺寸
(见第~\ref{boxes} 章)。
\item [\gr{interaction mode assignment}]
run modes for a \TeX\ job (see Chapter~\ref{run}).

\TeX 作业的运行模式(见第~\ref{run} 章)。
\item [\gr{intimate assignment}]
assignments to a \gr{special integer} or \gr{special dimen};
see %Chapters \ref{number} and~\ref{glue}.
pages \pageref{special:int:list} and~\pageref{special:dimen:list}.

对 \gr{special integer} 或 \gr{special dimen} 的赋值;
参见第~\pageref{special:int:list} 页和第~\pageref{special:dimen:list} 页。 
\end{description}


%\point Group delimiters
\section{Group delimiters\\组分隔符}

A group can be delimited by character tokens of category code~1 
\term delimiter! group\par
for `beginning of group' and code~2 for `end of  group', or
control sequence tokens that are \cs{let} to such characters,
the \cs{bgroup} and \cs{egroup} in plain \TeX.
Implicit and explicit braces can match to delimit
a group.

组可以由类别码为 1 的字符记号来界定 \term delimiter! group\par,
对应于“组的起始”,以及类别码为 2 的字符记号对应于“组的结束”,
或者将控制序列记号 \cs{let} 为这些字符,
在 plain \TeX\ 中为 \cs{bgroup} 和 \cs{egroup}。
隐式和显式的花括号可以匹配来界定一个组。

Groups can also be delimited by \csidx{begingroup} and
\csidx{endgroup}. These two control sequences must
be used together: they cannot be matched with implicit
or explicit braces, nor can they function as the braces
surrounding, for instance, boxed material.

组还可以由 \csidx{begingroup} 和 \csidx{endgroup} 来界定。
这两个控制序列必须成对使用:
它们不能与隐式或显式的花括号匹配,
也不能作为用于包围盒子内容的花括号。

Delimiting with \cs{begingroup} and \cs{endgroup} can
\label{begin:end:macros}%
provide a limited form of run-time error checking. 
In between these two group delimiters an excess
open or close brace would result in

使用 \cs{begingroup} 和 \cs{endgroup} 进行界定可以提供有限的运行时错误检查。
在这两个组分隔符之间,如果存在多余的开花括号或闭花括号,
将导致以下结果:
\begin{verbatim}
\begingroup ... } ... \endgroup
\end{verbatim}
or
\begin{verbatim}
\begingroup ... { ... \endgroup
\end{verbatim}
In both cases \TeX\ gives an error message about improper
balancing. Using \cs{bgroup} and \cs{egroup} here would
make an error much harder to find, because of the incorrect
matching that would occur. This idea is used in the environment
macros of several formats.

在两种情况下,\TeX\ 将发出关于不正确配对的错误信息。
在这里使用 \cs{bgroup} 和 \cs{egroup} 会使错误更难发现,
因为会出现不正确的匹配。这个想法在几个格式的环境宏中得到了应用。

The choice of the brace characters for the beginning and end of group
characters is not hard-wired in \TeX. It is arranged
\cstoidx bgroup\par\cstoidx egroup\par
like this in the plain format:

选择用于开始和结束分组的大括号字符并不是在 \TeX\ 中硬编码的。在 plain format 中,它的安排如下:
\begin{verbatim}
\catcode`\{=1 % left brace is begin-group character
\catcode`\}=2 % right brace is end-group character
\end{verbatim}
Implicit braces have also been defined in the plain format:

隐式大括号也已经在 plain format 中定义如下:
\begin{verbatim}
\let\bgroup={ \let\egroup=}
\end{verbatim}

Special cases are the following:

特殊情况如下:
\begin{itemize} \item The replacement text of a macro must be enclosed
in  explicit beginning and end of group character tokens.

宏的替换文本必须用显式的开始和结束分组字符记号括起来。
\item  The open and close braces for boxes, \cs{vadjust},
and \cs{insert} can be implicit. This makes it possible
to define, for instance

对于盒子、\cs{vadjust} 和 \cs{insert} 的开放和关闭大括号可以是隐式的。这使得可以定义例如:
\begin{verbatim}
\def\openbox#1{\setbox#1=\hbox\bgroup}
\def\closebox#1{\egroup\box#1}
\openbox{15}Foo bar\closebox{15}
\end{verbatim}
\item The right-hand side of a token list assignment and the
argument of the commands \cs{write}, \cs{message}, \cs{errmessage}, 
\cs{uppercase}, \cs{lowercase}, 
\cs{special}, and \cs{mark} is a \gr{general text}, defined
as

令牌列表的赋值右边和 \cs{write}、\cs{message}、\cs{errmessage}、\cs{uppercase}、\cs{lowercase}、\cs{special} 和 \cs{mark} 命令的参数是 \gr{general text},定义如下:
\begin{Disp} \gr{general text} $\longrightarrow$ \gr{filler}\lb
      \gr{balanced text}\gr{right brace}\end{Disp}
meaning that the left brace can be implicit, but the closing
right brace must be an explicit character token with category
code~2. 

意味着左大括号可以是隐式的,但是右边的右大括号必须是具有类别码为2的显式字符记号。
\end{itemize}

In cases where an implicit left brace suffices, and where
expansion is not explicitly inhibited, \TeX\ will
expand tokens until a left brace is encountered. This
is the basis for such constructs as
\verb=\uppercase\expandafter{\romannumeral80}=,
which in this unexpanded form do not adhere to the
syntax. If the first unexpandable token is not a left
brace \TeX\ gives an error message.

在隐式左大括号足够的情况下,并且未显式禁止展开时,\TeX\ 将展开标记,直到遇到一个左大括号。这是诸如 \verb=\uppercase\expandafter{\romannumeral80}= 这样的结构的基础,这样的结构在未展开的形式下不符合语法。如果第一个不可展开的记号不是左大括号,\TeX\ 将给出一个错误消息。

The grammar of \TeX\ (see Chapter~\ref{gramm})  uses
\gr{left brace} and \gr{right brace} for explicit
characters, that is, character tokens,
and \n{\lb} and~\n{\rb} 
for possibly implicit characters,
\altt
that is, control sequences that have been \cs{let} to such
explicit characters.

\TeX\ 的语法(见第~\ref{gramm}~章)使用 \gr{left brace} 和 \gr{right brace} 来表示显式字符,也就是字符记号,而使用 \n{\lb} 和 \n{\rb} 来表示可能是隐式字符,也就是已经使用 \cs{let} 定义为这种显式字符的控制序列。 



%\point More about braces
\section{More about braces\\关于花括号的更多内容}


%\spoint Brace counters
\subsection{Brace counters\\花括号计数器}

\TeX\ has two counters for keeping  track of grouping levels:
\term braces\par
the {\it master counter} and the {\it balance counter}.
Both of these counters are syntactic counters: they count the
explicit brace character tokens, but are not affected by implicit
braces (such as \cs{bgroup}) that are semantically equivalent
to an explicit brace.

\TeX\ 有两个计数器用于跟踪分组级别:{\it 主计数器}和{\it 平衡计数器}。
这两个计数器都是语法计数器:它们计算显式的花括号字符记号,但不受与显式花括号在语义上等效的隐式花括号(如 \cs{bgroup})的影响。

The balance counter handles braces in all cases except in
alignment. Its workings are intuitively clear: it goes up
by one for every opening and down for every closing
brace that is not being skipped. Thus

平衡计数器处理除对齐外的所有情况中的花括号。它的工作原理直观明了:对于每个打开的花括号,计数器加一;对于每个不被跳过的关闭花括号,计数器减一。因此,
\begin{verbatim}
\iffalse{\fi
\end{verbatim}
increases the balance counter if
this statement is merely scanned (for instance if it
appears in a macro definition text); if this statement
is executed the brace is skipped, so there is no effect on
the balance counter.

如果仅仅是被扫描的语句(例如出现在宏定义文本中),会增加平衡计数器;如果该语句被执行,花括号会被跳过,因此对平衡计数器没有影响。

The master counter is more tricky;
it is used in alignments instead of the balance counter.
This counter records all braces, even when they are skipped
such as in \verb>\iffalse{\fi>.
For this counter uncounted skipped braces are still possible:
the alphabetic constants \n{`\lb} and \n{`\rb} have
no effect on this counter when they are
use by the execution processor as a~\gr{number};
they do affect this counter when they are seen by the 
input processor (which merely sees characters, and not
the context).

主计数器更为棘手;它在对齐中代替平衡计数器使用。该计数器记录所有花括号,即使它们被跳过,比如 \verb>\iffalse{\fi>。
对于该计数器,未计数的被跳过的花括号仍然可能存在:
当字母常量 \n{\lb} 和 \n{\rb} 作为一个\gr{number}被执行处理器使用时,它们对该计数器没有影响;
当它们被输入处理器看到时(输入处理器仅看到字符,而不看上下文),它们会影响该计数器。



%\spoint The brace as a token
\subsection{The brace as a token\\花括号作为记号}

Explicit braces are character tokens, and as such they are
unexpandable. This implies that they survive until the
last stages of \TeX\ processing. For example,

显式的花括号是字符记号,因此它们是不可展开的。
这意味着它们一直保留到 \TeX\ 处理的最后阶段。
例如,
\begin{verbatim}
\count255=1{2}
\end{verbatim}
will assign~1 to \cs{count255},
and print~`2', because the
opening brace functions as a delimiter for the number~1.
Similarly 

将把1 赋给 \cs{count255},
并打印“2”,因为开括号在此处充当了数字~1 的定界符。
类似地,\begin{verbatim}
f{f}
\end{verbatim}
will prevent \TeX\ from forming
an `\hbox{ff}' ligature.

将阻止 \TeX\ 形成“ff”的连字。

From the fact that braces are unexpandable,
it follows that their nesting is independent
of the nesting of conditionals. For instance

由于花括号是不可展开的,它们的嵌套与条件嵌套无关。
例如,
\begin{verbatim}
\iftrue{\else}\fi
\end{verbatim}
will give an open brace,
as conditionals are handled by expansion. The closing
brace is simply skipped as part of the \gr{false text};
any consequences it has for grouping only come into
play in a later stage of \TeX\ processing.

将给出一个开括号,
因为条件处理是通过展开来完成的。
结束括号只是作为 \gr{false text} 的一部分被跳过;
它对分组的任何影响只在 \TeX\ 处理的后期阶段中发挥作用。

Undelimited macro arguments are either single tokens
or groups of tokens enclosed in explicit braces.
Thus it is not possible for an explicit open or close brace
to be a macro argument. However, braces can be assigned
with \cs{let}, for instance as in 

未定界的宏参数要么是单个记号,要么是用显式花括号括起来的记号组。
因此,显式开括号或闭括号不能作为宏参数。
但是,可以使用 \cs{let} 分配花括号,例如:
\begin{verbatim}
\let\bgroup={
\end{verbatim}
This is used in the plain \cs{footnote} macro
(see page~\pageref{footnote:ex}).

这在 plain \cs{footnote} 宏中使用(参见第~\pageref{footnote:ex}~页)。

%\spoint \csc{\char 123} and \csc{\char 125}
\subsection{Open and closing brace control symbols\\开括号和闭括号控制符}
% \csc{\char 123} and \csc{\char 125}}

The control sequences \verb-\{- and \verb-\}- do not really belong
\cstoidx\char123\par\cstoidx\char125\par
in this chapter,  not being concerned with grouping.
They have been defined with \cs{let} as synonyms of
\cs{lbrace} and \cs{rbrace} respectively,
and these control sequences are \cs{delimiter} instructions
(see Chapter~\ref{mathchar}).

控制序列 \verb-{- 和 \verb-}- 并不真正属于本章,因为它们与分组无关。
它们通过 \cs{let} 定义为 \cs{lbrace} 和 \cs{rbrace} 的同义词,
而这些控制序列是 \cs{delimiter} 指令(参见第~\ref{mathchar}~章)。

The Computer Modern Roman font has no braces, but there are
braces in the typewriter font, and for mathematics 
there are braces of different sizes \ldash and extendable ones \rdash in
the extension font.

计算机现代罗马字体中没有花括号,但打字机字体中有花括号,
而对于数学公式来说,不同尺寸的花括号是可用的,并且扩展性的花括号在扩展字体中。

%%% end of input file [group]