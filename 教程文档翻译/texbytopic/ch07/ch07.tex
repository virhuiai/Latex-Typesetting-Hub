
% %\InputFile:number
% %%%% this is input file [number]
% %\subject[number]  Numbers
% \endofchapter
% \chapter{Numbers\\数字}\label{number}

% In this chapter integers and their
% denotations will be treated,
% the conversions that are possible either way, 
% allocation and use of \cs{count} registers, and
% arithmetic with integers.

% 在本章中,将介绍整数及其表示法,以及双向转换、\cs{count} 寄存器的分配和使用,以及整数的算术运算。
% \begin{inventory} 
% \item [\cs{number}] 
%       Convert a \gr{number} to decimal representation. 

%       将 \gr{number} 转换为十进制表示形式。
% \item [\cs{romannumeral}] 
%       Convert a positive \gr{number} to lowercase roman representation.

%       将正 \gr{number} 转换为小写罗马表示形式。
% \item [\cs{ifnum}] 
%       Test relations between numbers.

%       比较数字之间的关系。
% \item [\cs{ifodd}] 
%       Test whether a number is odd.

%       测试一个数字是否为奇数。
% \item [\cs{ifcase}] 
%       Enumerated case statement.

%       枚举的 case 语句。
% \item [\cs{count}] 
%       Prefix for count registers. 

%       \cs{count} 寄存器的前缀。
% \item [\cs{countdef}] 
%       Define a control sequence to be a synonym for
%       a~\cs{count} register.

%       将控制序列定义为 \cs{count} 寄存器的别名。
% \item [\cs{newcount}] 
%       Allocate an unused \cs{count} register. 

%       分配一个未使用的 \cs{count} 寄存器。
% \item [\cs{advance}] 
%       Arithmetic command to add to or subtract from 
%       a~\gr{numeric variable}.

%       算术命令,用于将值加到或从 \gr{numeric variable} 中减去。
% \item [\cs{multiply}] 
%       Arithmetic command to multiply a \gr{numeric variable}.

%       算术命令,用于将 \gr{numeric variable} 乘以一个数。
% \item [\cs{divide}] 
%       Arithmetic command to divide a \gr{numeric variable}.

%       算术命令,用于将 \gr{numeric variable} 除以一个数。
% \end{inventory}

