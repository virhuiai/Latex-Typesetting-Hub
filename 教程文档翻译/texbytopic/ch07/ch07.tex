
% %\InputFile:number
% %%%% this is input file [number]
% %\subject[number]  Numbers
% \endofchapter
% \chapter{Numbers\\数字}\label{number}

% In this chapter integers and their
% denotations will be treated,
% the conversions that are possible either way, 
% allocation and use of \cs{count} registers, and
% arithmetic with integers.

% 在本章中,将介绍整数及其表示法,以及双向转换、\cs{count} 寄存器的分配和使用,以及整数的算术运算。
% \begin{inventory} 
% \item [\cs{number}] 
%       Convert a \gr{number} to decimal representation. 

%       将 \gr{number} 转换为十进制表示形式。
% \item [\cs{romannumeral}] 
%       Convert a positive \gr{number} to lowercase roman representation.

%       将正 \gr{number} 转换为小写罗马表示形式。
% \item [\cs{ifnum}] 
%       Test relations between numbers.

%       比较数字之间的关系。
% \item [\cs{ifodd}] 
%       Test whether a number is odd.

%       测试一个数字是否为奇数。
% \item [\cs{ifcase}] 
%       Enumerated case statement.

%       枚举的 case 语句。
% \item [\cs{count}] 
%       Prefix for count registers. 

%       \cs{count} 寄存器的前缀。
% \item [\cs{countdef}] 
%       Define a control sequence to be a synonym for
%       a~\cs{count} register.

%       将控制序列定义为 \cs{count} 寄存器的别名。
% \item [\cs{newcount}] 
%       Allocate an unused \cs{count} register. 

%       分配一个未使用的 \cs{count} 寄存器。
% \item [\cs{advance}] 
%       Arithmetic command to add to or subtract from 
%       a~\gr{numeric variable}.

%       算术命令,用于将值加到或从 \gr{numeric variable} 中减去。
% \item [\cs{multiply}] 
%       Arithmetic command to multiply a \gr{numeric variable}.

%       算术命令,用于将 \gr{numeric variable} 乘以一个数。
% \item [\cs{divide}] 
%       Arithmetic command to divide a \gr{numeric variable}.

%       算术命令,用于将 \gr{numeric variable} 除以一个数。
% \end{inventory}


% %\point Numbers and \gr{number}s
% \section{Numbers and \gr{number}s\\数字和 \gr{number}}

% An important part of the grammar of \TeX\ 
% \term numbers\par\term integers\par
% is the rigorous definition of a \gr{number}, the syntactic
% entity that \TeX\ expects when semantically an integer is
% expected. This definition will take the largest part of this
% chapter. Towards the end, \cs{count} registers, arithmetic,
% and tests for numbers are treated.

% \TeX 的语法中的一个重要部分是对 \gr{number} 的严格定义,即在语义上期望一个整数的语法实体。这个定义将占据本章的大部分内容。在最后,将处理 \cs{count} 寄存器、算术和数字测试。

% For clarity of discussion a distinction will be made
% here between integers and numbers, 
% but note that a \gr{number} can be both
% an `integer' and a `number'.
% `Integer'  will be taken to denote a mathematical number:
% a~quantity that can be added or multiplied.
% `Number' will be taken to refer to the printed representation
% of an integer: a string of digits, in other words.

% 为了清晰讨论,这里将在整数和数字之间进行区分,但请注意,\gr{number} 可以同时是“整数”和“数字”。这里,“整数”将表示一个数学数字:可以进行加法或乘法的数量。“数字”将表示整数的打印表示:换句话说,一个数字是一串数字。


% %\point Integers
% \section{Integers\\整数}

% Quite a few different sorts of objects can function
% as integers in \TeX. In this section they will all
% be treated, accompanied by the relevant lines from
% the grammar of \TeX.
% \awp

% 在 \TeX 中,有许多不同类型的对象可以作为整数。在本节中,它们将全部涵盖,并附上来自 \TeX 语法的相关行。

% First of all, an integer can be positive or negative:

% 首先,整数可以是正数或负数:
% \begin{disp}\gr{number} $\longrightarrow$ 
% \gr{optional signs}\gr{unsigned number}\nl
% \gr{optional signs} $\longrightarrow$ \gr{optional spaces}\nl
% \indent $|$ \gr{optional signs}\gr{plus or minus}\gr{optional spaces}
% \end{disp}

% A first possibility for an unsigned integer is a string of digits
% in decimal, octal, or hexadecimal notation.
% Together with the alphabetic constants these will be named
% here \gr{integer denotation}.
% Another possibility for an integer is an
% internal integer quantity, an \gr{internal integer};
% together with the denotations these form the 
% \gr{normal integer}s.
% Lastly an integer can be a \gr{coerced integer}: 
% an internal \gr{dimen} or \gr{glue}
% quantity that is converted to an integer value.

% 一个无符号整数的第一个可能性是十进制、八进制或十六进制表示法中的数字字符串。加上字母常量,这里将它们称为 \gr{integer denotation}。整数的另一个可能性是内部整数数量,即 \gr{internal integer};与表示法一起,它们构成了 \gr{normal integer}。最后,整数可以是强制转换的整数,即将内部 \gr{dimen} 或 \gr{glue} 数量转换为整数值。
% \begin{disp}\gr{unsigned number} $\longrightarrow$ \gr{normal integer}
% $|$ \gr{coerced integer}\nl
% \gr{normal integer} $\longrightarrow$ \gr{integer denotation}
% $|$ \gr{internal integer}\nl
% \gr{coerced integer} $\longrightarrow$ \gr{internal dimen}
% $|$ \gr{internal glue}\end{disp}
% All of these possibilities will be treated in sequence.

% 所有这些可能性将按顺序进行讨论。


% %\spoint[int:denotation] Denotations: integers
% \subsection{Denotations: integers\\表示:整数}
% \label{int:denotation}

% Anything that looks like a number
% can be used as a \gr{number}: thus \verb-42- is a number.
% However, bases other than decimal can also be used:


% 任何看起来像一个数字的东西都可以用作 \gr{number}:因此 \verb-42- 是一个数字。
% 然而,除了十进制之外,还可以使用其他进制:
% \begin{verbatim}
% '123
% \end{verbatim}
% is the octal notation for $1\times8^2+2\times8^1+3\times8^0=83$,
% and 

% 是八进制表示法,表示 $1\times8^2+2\times8^1+3\times8^0=83$,
% 而\begin{verbatim}
% "123
% \end{verbatim}
% is the hexadecimal notation
% for $1\times16^2+2\times16^1+3\times16^0=291$.

% 是十六进制表示法,
% 表示 $1\times16^2+2\times16^1+3\times16^0=291$。
% \begin{disp}\gr{integer denotation} $\longrightarrow$
% \gr{integer constant}\gr{one optional space} \nl
% \indent $|$ \n{\char`\'}\gr{octal constant}\gr{one optional space}\nl
% \indent $|$ \n{\char`\"}\gr{hexadecimal constant}\gr{one optional space}
% \end{disp}
% The octal digits are \n0--\n7; a~digit \n8 or~\n9 following an
% octal denotation is not part of the number: 
% after 

% 八进制数字为 \n0--\n7;在八进制表示法之后的数字 \n8 或 \n9 不是数字的一部分:
% 在\begin{verbatim}
% \count0='078
% \end{verbatim}
% the \cs{count0} will have the value~7, and the 
% digit~\n8 is typeset.

% 之后,\cs{count0} 的值将是 7,而数字 \n8 被排版出来。

% The hexadecimal digits are \n0--\n9, \n A--\n F, 
% where the \n A--\n F can
% have category code 11 or~12. The latter has a somewhat
% far-fetched justification: the characters resulting from a
% \cs{string} operation have category code~12.
% Lowercase \n a--\n f are not
% hexadecimal digits, although (in \TeX3) they are used 
% for hexadecimal notation in
% the `circumflex method' for accessing all character codes
% (see Chapter~\ref{char}).

% 十六进制数字为 \n0--\n9、\n A--\n F,其中 \n A--\n F 可以具有类别码 11 或 12。
% 后者的理由略微牵强:\cs{string} 操作生成的字符具有类别码 12。
% 小写字母 \n a--\n f 不是十六进制数字,尽管(在 \TeX3 中)它们用于以“插入符号方法”访问所有字符码的十六进制表示法
% (请参见第~\ref{char} 章)。

