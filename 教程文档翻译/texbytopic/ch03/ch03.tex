
% %\InputFile:char
% %%%% this is input file [char]
% %\subject[char] Characters
% \endofchapter
% \chapter{Characters\\字符}\label{char}

% Internally, \TeX\ represents characters by their (integer) 
% character code. This chapter treats those codes, and the
% commands that have access to them.

% 在内部,\TeX 用它们的(整数)字符码表示字符。本章讨论这些字符码以及能够访问它们的命令。

% \begin{inventory}
% \item [\cs{char}]
%       Explicit denotation of a character to be typeset. 

%       显式表示要排版的字符。
% \item [\cs{chardef}] 
%       Define a control sequence to be a synonym for
%       a~character code.

%       将控制序列定义为字符码的别名。
% \item [\cs{accent}] 
%       Command to place accent characters.

%       用于放置重音字符的命令。
% \item [\cs{if}]
%       Test equality of character codes. 

%       比较字符码是否相等。
% \item [\cs{ifx}]
%       Test equality of both character and category codes.

%       比较字符码和类别码是否都相等。
% \item [\cs{let}]
%       Define a control sequence to be a synonym of a token.

%       将控制序列定义为一个记号的别名。
% \item [\cs{uccode}] 
%       Query or set
%       the character code that is the uppercase variant of a given code.

%       查询或设置给定字符码的大写形式的字符码。
% \item [\cs{lccode}]
%       Query or set
%       the character code that is the lowercase variant of a given code.

%       查询或设置给定字符码的小写形式的字符码。
% \item [\cs{uppercase}]
%       Convert the \gr{general text} argument to its uppercase form.

%       将 \gr{general text} 参数转换为大写形式。
% \item [\cs{lowercase}] 
%       Convert the \gr{general text} argument to its lowercase form.

%       将 \gr{general text} 参数转换为小写形式。
% \item [\cs{string}]
%       Convert a token to a string of one or more characters.

%       将记号转换为一个或多个字符的字符串。
% \item [\cs{escapechar}]
%       Number of the character that is to be used 
%       for the escape character
%       when control sequences are being converted
%       into character tokens. \IniTeX\ default:~92~(\cs{}).

%       当将控制序列转换为字符记号时,用于作为转义字符的字符的编号。在初始的 \TeX 中,默认值为 92(\cs{})。
% \end{inventory}


% %\point[char:code] Character codes
% \section{Character codes\\字符码}
% \label{char:code}

% Conceptually it is easiest to think that \TeX\ works with
% \term character! codes\par
% characters internally, but in fact
% \TeX\ works with integers: the `character codes'. 

% 从概念上来说,最容易理解的是 \TeX\ 内部使用\term character! codes\par 字符,但实际上 \TeX\ 使用的是整数:即“字符码”。

% The way characters are encoded in a computer may differ
% from system to system.
% Therefore \TeX\ uses its own scheme of character codes.
% Any character that is read from a file (or from the user terminal)
% is converted to a character code according to the
% character code table.
% A~category code is then assigned based on this (see Chapter~\ref{mouth}).
% The character code table is based on the 7-bit \ascii{} table
% for numbers under~128 (see Chapter~\ref{table}).

% 不同的计算机系统对字符的编码方式可能有所不同。因此,\TeX\ 使用自己的字符码方案。从文件(或用户终端)读取的任何字符都会根据字符码表转换为字符码。然后,根据字符码分配类别码(参见第~\ref{mouth}章)。字符码表基于 7 位 \ascii{} 表,在数字小于 128 时使用(参见第\ref{table}~章)。

% There is an explicit conversion between characters
% (better:  character tokens)
% and  character codes  using the left quote (grave, back quote)
% character~\n{`{}}:
% at all places where \TeX\ expects a \gram{number} you
% can use the left quote followed by a character
% token or
% a single-character control sequence.
% Thus both \verb.\count`a. and \verb.\count`\a. are synonyms
% \awp
% for \verb.\count97.. See also Chapter~\ref{number}.

% 可以使用左引号(反引号)字符 \n{{}} 对字符(更准确地说是字符记号)和字符码进行显式转换:在 \TeX\ 需要一个 \gram{number} 的任何地方,都可以使用左引号后跟一个字符记号或一个单字符控制序列。因此,\verb.\counta. 和 \verb.\count`\a. 都是 \verb.\count97. 的同义词。详见第~\ref{number}~章。


% The possibility of a single-character control
% sequence is necessary in certain cases such as

% 在某些情况下,需要使用单字符控制序列,例如:
% \begin{disp}\verb>\catcode`\%=11>\quad or\quad \verb>\def\CommentSign{\char`\%}>\end{disp}
% which would be misunderstood if the backslash were left out.
% For instance 

% 如果省略了反斜杠,则会引起误解。例如:
% \begin{verbatim}
% \catcode`%=11
% \end{verbatim}
% would consider
% the \n{=11} to be a comment.
% Single-character
% control sequences can be formed from characters with any
% category code.

% 会将 \n{=11} 视为注释。单字符控制序列可以由任何类别码的字符构成。

% After the conversion to character codes any connection
% with external representations has disappeared. Of course,
% for most characters  the visible output will `equal' the input
% (that is, an `\n{a}' causes an~`a').
% There are exceptions, however, even among the common symbols.
% In the Computer Modern
% roman fonts there are no `less than' and `greater than'
% \message{Check <>! Dammit!}%
% signs, so the input `\verb.<>.' will give `<>' in the output.
% %{\MathRMx<>}

% 在转换为字符码之后,与外部表示的任何联系都已消失。当然,对于大多数字符,可见的输出将与输入“相等”(即,\n{a}' 会输出为 a')。然而,即使在常见的符号中也存在例外情况。在 Computer Modern Roman 字体中,没有“小于”和“大于”符号,因此输入 \verb.<>.' 会在输出中显示为 <> '。

% In order to make \TeX\ machine independent at the output
% side, the character codes are also used in the \n{dvi} file:
% opcodes $n=0\ldots127$ denote simply the instruction `take
% character $n$ from the current font'. The complete definition
% of the opcodes in a \n{dvi} file can be found in~\cite{Knuth:TeXprogram}.

% 为了使 \TeX\ 在输出方面具有机器无关性,字符码也用于 \n{dvi} 文件:操作码 $n=0\ldots127$ 简单地表示“从当前字体中获取字符 $n$”。有关 \n{dvi} 文件中操作码的完整定义可以在~\cite{Knuth:TeXprogram} 中找到。

