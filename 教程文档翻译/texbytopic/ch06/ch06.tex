
% %\InputFile:modes
% %%%% this is input file [modes] 
% %\subject[hvmode]  Horizontal and \nl Vertical Mode
% \endofchapter
% \chapter{Horizontal and Vertical Mode\\水平模式和垂直模式}\label{hvmode}


% At any point in its processing \TeX\ is in some mode.
% \term mode\par
% There are six modes, divided in three categories:

% 在其处理的任何时刻,\TeX\ 处于某种模式中。
% 有六种模式,分为三个类别:
% \begin{enumerate} \item horizontal mode and restricted horizontal
% mode, 

% 水平模式和受限水平模式,
% \item vertical mode and internal vertical mode, and

% 垂直模式和内部垂直模式,
% \item math mode and display math mode.

% 数学模式和陈列数学模式。\end{enumerate}
% The math modes will be treated elsewhere (see page~\pageref{math:modes}).
% Here we shall look
% at the horizontal and vertical modes, the kinds of objects
% that can occur in the corresponding lists, and the
% commands that are exclusive for one mode or the other. 

% 数学模式将在其他地方处理(见第~\pageref{math:modes} 页)。
% 在这里,我们将看一下水平模式和垂直模式,对应列表中可能出现的对象以及专为一个模式而设计的命令。


% \begin{inventory}
% \item [\cs{ifhmode}] 
%       Test whether the current mode is (possibly restricted) horizontal mode.

%       检查当前模式是否为(可能为受限制的)水平模式。
% \item [\cs{ifvmode}] 
%       Test whether the current mode is (possibly internal) vertical mode.

%       检查当前模式是否为(可能为内部的)垂直模式。
% \item [\cs{ifinner}] 
%       Test whether the current mode is an internal mode.

%       检查当前模式是否为内部模式。
% \item [\cs{vadjust}] 
%       Specify vertical material for the enclosing vertical list
%       while in horizontal mode.

%       在水平模式中为封闭的垂直列表指定垂直材料。
% \item [\cs{showlists}] 
%       Write to the log file the contents of the partial lists 
%       currently being built in all modes.

%       将当前正在构建中的所有模式下的部分列表的内容写入日志文件。
% \end{inventory}


% %\point Horizontal and vertical mode
% \section{Horizontal and vertical mode\\水平和垂直模式}

% When not typesetting mathematics, \TeX\ is in horizontal
% or vertical mode, building horizontal or vertical lists
% respectively. Horizontal mode is typically used to
% make lines of text; vertical mode is typically used
% to stack the lines of a paragraph on top of each other.
% Note that
% these modes
% are different from the internal states of \TeX's input processor
% (see page~\pageref{input:states}).

% 在非数学排版时,\TeX\ 处于水平模式或垂直模式,分别构建水平列表或垂直列表。水平模式通常用于生成文本行;垂直模式通常用于将段落的行堆叠在一起。请注意,这些模式与 \TeX 的输入处理器的内部状态不同(参见第~\pageref{input:states}~页)。
% %\spoint Horizontal mode
% \subsection{Horizontal mode\\水平模式}

% The main activity in horizontal mode is building lines of text.
% \term mode !horizontal\par
% Text on the page and text in a \cs{vbox} or \cs{vtop} is built in
% horizontal mode (this might be called `paragraph mode');
% if the text is in an \cs{hbox} there is only one line
% of text, and the corresponding mode is the restricted
% \awp
% horizontal mode.

% 水平模式的主要活动是构建文本行。页面上的文本和 \cs{vbox} 或 \cs{vtop} 中的文本都是在水平模式下构建的(这可能被称为“段落模式”);如果文本位于 \cs{hbox} 中,那么只有一行文本,相应的模式是受限制的水平模式。

% In horizontal mode all material is added to a horizontal list.
% If this list is built in unrestricted horizontal mode, it
% will later be broken into lines and added to the surrounding vertical list.

% 在水平模式下,所有内容都被添加到水平列表中。
% 如果此列表是在无限制的水平模式下构建的,它将在稍后被分成行,并添加到周围的垂直列表中。

% Each element of a horizontal list is one of the following:
% \term list !horizontal\par

% 水平列表的每个元素都是以下之一:
% \begin{itemize} \item a box (a character, ligature, \cs{vrule},
% or a \gr{box}),

% 盒子(字符、连字、\cs{vrule} 或一个 \gr{box}),
% \item a discretionary break,

% 不连断的断点,
% \item a whatsit (see Chapter~\ref{io}),

% 活动内容(见第~\ref{io}章),
% \item vertical material enclosed in \cs{mark},
% \cs{vadjust}, or \cs{insert},

% 用 \cs{mark}、\cs{vadjust} 或 \cs{insert} 封装的垂直材料,
% \item 
% \mdqon
% glue or leaders, a kern, a penalty, or a math-on/""off item.

% 粘连、指引线、紧排、惩罚或打开/关闭数学公式。
% \mdqoff
% \end{itemize}
% The items in the last point are all discardable.
% Discardable items are called that, because they disappear in
% \term discardable items\par
% a break. Breaking of horizontal
% lists is treated in Chapter~\ref{line:break}.

% 最后一点中的项目都是可丢弃的。
% 可丢弃的项目之所以被称为可丢弃,是因为它们在换行处消失。
% 水平列表的断行在第\ref{line:break}~章中进行了讨论。



% %\spoint Vertical mode
% \subsection{Vertical mode\\垂直模式}

% Vertical mode can be used to stack items on top of one another.
% \term mode !vertical\par
% Most of the time, these items are boxes 
% containing the lines of paragraphs.

% 垂直模式可用于将项目堆叠在一起。
% 大多数情况下,这些项目是包含段落行的盒子。

% Stacking material can take place inside a 
% vertical box, but the
% items that are stacked can also 
% appear by themselves on the page. In the latter case
% \TeX\ is in vertical mode; in the former case, inside a
% vertical box, \TeX\ operates in internal vertical mode.

% 堆叠材料可以出现在垂直盒子中,但堆叠的项目也可以单独出现在页面上。
% 在后一种情况下,\TeX\ 处于垂直模式;在前一种情况下,即在垂直盒子内部,\TeX\ 处于内部垂直模式。

% In vertical mode all material is added to a vertical list.
% If this list is built in external vertical mode, it
% will later be broken when pages are formed.

% 在垂直模式下,所有内容都被添加到垂直列表中。
% 如果此列表是在外部垂直模式下构建的,它将在形成页面时分割。

% Each element of a vertical list is one of the following:
% \term list !vertical\par

% 垂直列表的每个元素都是以下之一:
% \begin{itemize} \item a box (a horizontal or vertical box or 
% an \cs{hrule}),

% 盒子(水平盒子、垂直盒子或 \cs{hrule}),
% \item a whatsit,

% 活动内容,
% \item a mark,

% 标记,
% \item glue or leaders, a kern, or a penalty.

% 粘连、指引线、紧排或惩罚。\end{itemize}
% The items in the last point are all discardable.
% Breaking of vertical lists
% is treated in Chapter~\ref{page:break}.

% 最后一点中的项目都是可丢弃的。
% 垂直列表的分割在第~\ref{page:break}~章中进行了讨论。

% There are a few exceptional conditions at the beginning
% of  a vertical list: the value of \cs{prevdepth} is set
% to \n{-1000pt}. Furthermore, no \cs{parskip} glue is added
% at the top of an internal vertical list; 
% at the top of the main vertical list (the top of the
% `current page') no glue or other discardable items
% are added, and \cs{topskip} glue is added when the
% first box is placed on this list
% (see Chapters \ref{page:shape} and~\ref{page:break}).

% 在垂直列表的开头有一些特殊条件:\cs{prevdepth} 的值被设置为 \n{-1000pt}。此外,在内部垂直列表的顶部不会添加 \cs{parskip} 粘连;在主垂直列表(即“当前页面”的顶部)的顶部不会添加粘连或其他可丢弃项,并且当第一个盒子放置在此列表时,会添加 \cs{topskip} 粘连(参见第 \ref{page:shape} 和第 \ref{page:break} 章)。

% %\point Horizontal and vertical commands
% \section{Horizontal and vertical commands\\水平和垂直命令}

% Some commands are so intrinsically horizontal or vertical
% in nature that they force \TeX\ to go into that mode, if
% possible. A~command that forces \TeX\ into horizontal mode
% is called a \gr{horizontal command}; similarly a command that
% forces \TeX\ into vertical mode is called a
% \awp
% \gr{vertical command}.

% 某些命令在本质上是水平或垂直的,如果可能,它们会强制 \TeX 进入相应的模式。将 \TeX 强制进入水平模式的命令称为 \gr{horizontal command};类似地,将 \TeX 强制进入垂直模式的命令称为 \gr{vertical command}。

% However, not all transitions are possible:
% \TeX\ can switch from both vertical modes to 
% (unrestricted) horizontal mode and back
% through horizontal and vertical commands, but no transitions
% to or from restricted horizontal mode are possible
% (other than by enclosing horizontal boxes in vertical boxes or
% the other way around).
% A~vertical command in restricted horizontal mode thus gives
% an error; the \cs{par} command in restricted horizontal mode
% has no effect.

% 然而,并非所有的转换都是可能的:\TeX 可以通过水平和垂直命令从两种垂直模式切换到(不受限制的)水平模式以及相反,但是不可能进行限制水平模式的转换(除非将水平盒子封装在垂直盒子中,或者反过来)。因此,在限制水平模式中使用垂直命令会导致错误;在限制水平模式中,\cs{par} 命令没有效果。

% The horizontal commands are the following:

% 以下是水平命令的一些示例:
% \label{h:com:list}\term horizontal commands\par
% \begin{itemize}
% \item any \gr{letter}, \gr{otherchar}, \cs{char}, 
% a control sequence defined by \cs{chardef}, or \cs{noboundary};

% 任何 \gr{letter}、\gr{otherchar}、\cs{char}、由 \cs{chardef} 定义的控制序列或 \cs{noboundary};
% \item \cs{accent}, \cs{discretionary}, the discretionary
% hyphen~\verb|\-| and control space~\verb|\|\n{\char32};

% \cs{accent}、\cs{discretionary}、连字符~\verb|-| 和控制空格~\verb||\n{\char32};
% \item \cs{unhbox} and \cs{unhcopy};

% \cs{unhbox} 和 \cs{unhcopy};
% \item \cs{vrule} and the
% \gr{horizontal skip} commands
% \cs{hskip}, \cs{hfil}, \cs{hfill}, \cs{hss}, and \cs{hfilneg};

% \cs{vrule} 和 \gr{horizontal skip} 命令 \cs{hskip}、\cs{hfil}、\cs{hfill}、\cs{hss} 和 \cs{hfilneg};
% \item \cs{valign};
% \item math shift (\n\$).

% 数学模式切换符(\n\$).
% \end{itemize}

% The vertical commands are the following:

% 垂直命令如下:
% \label{v:com:list}\term vertical! commands\par
% \begin{itemize}
% \item \cs{unvbox} and \cs{unvcopy};
% \item \cs{hrule} and the \gr{vertical skip} commands
%  \cs{vskip}, \cs{vfil}, \cs{vfill}, \cs{vss}, and \cs{vfilneg};
% \item \cs{halign};
% \item \cs{end} and \cs{dump}.
% \end{itemize}
% Note that the vertical commands do not include \cs{par};
% nor are \cs{indent} and \cs{noindent} horizontal commands.

% 请注意,垂直命令不包括 \cs{par};
% \cs{indent} 和 \cs{noindent} 也不是水平命令。


% The connection between boxes and modes is explored below;
% see Chapter~\ref{rules} for more on the connection between
% rules and modes.

% 下面将探讨盒子和模式之间的关系;
% 有关规则和模式之间的关系,请参见第~\ref{rules} 章。

