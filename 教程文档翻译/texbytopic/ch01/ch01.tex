% \chapter{The Structure of the \TeX\ Processor\\\TeX\ 处理器的结构}

% This book treats the various aspects of \TeX\ in chapters
% that are concerned with relatively small, well-delineated,
% topics. In this chapter, therefore, 
% a global picture of the way \TeX\ operates will be given.
% Of necessity, many details will be omitted here, but all of
% these are treated in later chapters. On the other hand,
% the few examples given in this chapter will be repeated
% in the appropriate places later on; they are included here
% to make this chapter self-contained.

% 本书将以涉及相对小而清晰的主题的章节来介绍 \TeX 的各个方面。因此,在本章中,将给出 \TeX\ 运行方式的整体图景。当然,这里有很多细节被省略了,但所有这些细节在后面的章节中都有涉及。另一方面,本章中给出的一些示例将在适当的地方再次出现;它们包含在这里是为了使本章内容自成一体。



% %\point Four \TeX\ processors
% \section{Four \TeX\protect\ processors\\四个 \TeX\ 处理器}

% The way \TeX\ processes its input can be viewed as
% happening on four levels. One might  say that
% the \TeX\ processor is split into four separate units,
% each one accepting the output of the previous stage, and
% delivering the input for the next stage. The input of
% the first stage is then the \n{.tex} input file; the output
% of the last stage is a \n{.dvi} file.

% \TeX\ 处理输入的方式可以看作是在四个层次上进行的。可以说 \TeX\ 处理器被分为四个独立的单元,每个单元接受前一阶段的输出,并产生下一阶段的输入。第一阶段的输入是 \n{.tex} 输入文件;最后一阶段的输出是一个 \n{.dvi} 文件。

% For many purposes it is most convenient, and most insightful,
% to consider these four levels of processing as happening
% after one another, each one accepting the {\em completed\/}
% output of the previous level. In reality this is not true:
% all levels are simultaneously
% active, and there is interaction between them.

% 对于许多情况,将这四个处理层次依次考虑是最方便和最有洞察力的,每个层次接受上一层次的{\em 完整}输出。实际上,这并不完全正确:所有层次都是同时活动的,并且它们之间存在相互作用。



% The four levels are (corresponding roughly
% to the `eyes', `mouth', `stomach', and `bowels' respectively
% in Knuth's original terminology) as follows.

% 这四个层次(与 Knuth 最初的术语“眼睛”、“嘴巴”、“胃”和“肠道”相对应)如下所示:
