% \chapter{The Structure of the \TeX\ Processor\\\TeX\ 处理器的结构}

% This book treats the various aspects of \TeX\ in chapters
% that are concerned with relatively small, well-delineated,
% topics. In this chapter, therefore, 
% a global picture of the way \TeX\ operates will be given.
% Of necessity, many details will be omitted here, but all of
% these are treated in later chapters. On the other hand,
% the few examples given in this chapter will be repeated
% in the appropriate places later on; they are included here
% to make this chapter self-contained.

% 本书将以涉及相对小而清晰的主题的章节来介绍 \TeX 的各个方面。因此,在本章中,将给出 \TeX\ 运行方式的整体图景。当然,这里有很多细节被省略了,但所有这些细节在后面的章节中都有涉及。另一方面,本章中给出的一些示例将在适当的地方再次出现;它们包含在这里是为了使本章内容自成一体。



% %\point Four \TeX\ processors
% \section{Four \TeX\protect\ processors\\四个 \TeX\ 处理器}

% The way \TeX\ processes its input can be viewed as
% happening on four levels. One might  say that
% the \TeX\ processor is split into four separate units,
% each one accepting the output of the previous stage, and
% delivering the input for the next stage. The input of
% the first stage is then the \n{.tex} input file; the output
% of the last stage is a \n{.dvi} file.

% \TeX\ 处理输入的方式可以看作是在四个层次上进行的。可以说 \TeX\ 处理器被分为四个独立的单元,每个单元接受前一阶段的输出,并产生下一阶段的输入。第一阶段的输入是 \n{.tex} 输入文件;最后一阶段的输出是一个 \n{.dvi} 文件。

% For many purposes it is most convenient, and most insightful,
% to consider these four levels of processing as happening
% after one another, each one accepting the {\em completed\/}
% output of the previous level. In reality this is not true:
% all levels are simultaneously
% active, and there is interaction between them.

% 对于许多情况,将这四个处理层次依次考虑是最方便和最有洞察力的,每个层次接受上一层次的{\em 完整}输出。实际上,这并不完全正确:所有层次都是同时活动的,并且它们之间存在相互作用。



% The four levels are (corresponding roughly
% to the `eyes', `mouth', `stomach', and `bowels' respectively
% in Knuth's original terminology) as follows.

% 这四个层次(与 Knuth 最初的术语“眼睛”、“嘴巴”、“胃”和“肠道”相对应)如下所示:




% \begin{enumerate}\item
% The input processor. This is the piece of \TeX\ that
% accepts input lines from the file system of whatever computer
% \TeX\ runs on, and turns them into tokens.
% Tokens are the internal objects of \TeX:
% there are character tokens that constitute the typeset
% text, and control sequence tokens that are commands 
% to be processed by the next two levels.

% 输入处理器。这是 \TeX\ 的一部分,它从 \TeX\ 运行所在计算机的文件系统接受输入行,并将它们转换为记号。记号是 \TeX\ 的内部对象:其中有构成排版文本的字符记号,还有作为下两个层次处理的命令的控制序列记号。
% \item The expansion processor. 
% Some but not all of the tokens generated in the first level
% \ldash macros, conditionals, and a number
% of primitive \TeX\ commands \rdash  are subject to expansion.
% Expansion is the process that replaces some (sequences of)
% tokens by other (or no) tokens.

% 展开处理器。在第一层次生成的记号中,一些(但不是全部)\ldash 宏、条件语句和一些原始 \TeX\ 命令 \rdash 可以进行展开。展开是将一些(序列的)记号替换为其他(或不替换)记号的过程。
% \item The execution processor. 
% Control sequences that are not expandable are executable,
% and this execution takes place on the third level of the
% \TeX\ processor.

% 执行处理器。不可展开的控制序列是可执行的,并且这个执行过程发生在 \TeX\ 处理器的第三层次上。


% One part of the activity here concerns changes to
% \TeX's internal state: assignments (including
% macro definitions) are typical activities in this
% \awp
% category. The other major thing happening on this level
% is the construction of horizontal, vertical, and
% mathematical lists.

% 活动的一部分涉及到 \TeX\ 的内部状态的更改:赋值(包括宏定义)是这一类别中的典型活动。在此层次上发生的另一项主要事务是构建水平、垂直和数学列表。
% \item The visual processor. 
% In the final level of processing
% the visual part of \TeX\ processing is performed. Here
% horizontal lists are broken into paragraphs, 
% vertical lists are broken into pages,
% and  formulas are built out of math lists. 
% Also the output to the \n{dvi} file takes place on this level.
% The algorithms working here are not accessible to the user,
% but they can be influenced by a number of parameters.

% 视觉处理器。在处理的最后一层中,执行 \TeX\ 的视觉处理。在这里,水平列表被分割为段落,垂直列表被分割为页面,并且数学列表被构建为公式。此外,输出到 \n{dvi} 文件也在此层次上进行。在此处工作的算法用户无法访问,但可以通过多个参数进行调整。
% \end{enumerate}



% %\point The input processor
% \section{The input processor\\输入处理器}

% The input processor of \TeX\ is that part of \TeX\ that
% translates whatever characters it gets from the input file
% into tokens. The output of this processor is a stream
% of tokens: a token list. Most tokens fall into one of two categories:
% character tokens and control sequence tokens. 
% The remaining category is that of the parameter tokens;
% these will not be treated in this chapter.

% \TeX\ 的输入处理器是将输入文件中获取的字符转换为记号的 \TeX\ 的一部分。该处理器的输出是一系列记号:一个记号列表。大多数记号属于以下两个类别之一:字符记号和控制序列记号。剩下的类别是参数记号;这些在本章中不进行讨论。

% %\spoint Character input
% \subsection{Character input\\字符输入}

% For simple input text, characters are made into
% character tokens. However, \TeX\ can ignore input characters:
% a row of spaces in the input is usually equivalent to just one
% space. Also, \TeX\ itself can insert tokens that do not correspond
% to any character in the input, for instance the space token
% at the end of the line, or the \cs{par} token after an empty line.

% 对于简单的输入文本,字符被转换为字符记号。但是,\TeX\ 可以忽略输入字符:输入中的多个空格通常等效为一个空格。此外,\TeX\ 本身可以插入不对应任何输入字符的记号,例如行末的空格记号,或空行后的 \cs{par} 记号。

