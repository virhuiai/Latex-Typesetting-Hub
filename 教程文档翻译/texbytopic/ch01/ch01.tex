% \chapter{The Structure of the \TeX\ Processor\\\TeX\ 处理器的结构}

% This book treats the various aspects of \TeX\ in chapters
% that are concerned with relatively small, well-delineated,
% topics. In this chapter, therefore, 
% a global picture of the way \TeX\ operates will be given.
% Of necessity, many details will be omitted here, but all of
% these are treated in later chapters. On the other hand,
% the few examples given in this chapter will be repeated
% in the appropriate places later on; they are included here
% to make this chapter self-contained.

% 本书将以涉及相对小而清晰的主题的章节来介绍 \TeX 的各个方面。因此,在本章中,将给出 \TeX\ 运行方式的整体图景。当然,这里有很多细节被省略了,但所有这些细节在后面的章节中都有涉及。另一方面,本章中给出的一些示例将在适当的地方再次出现;它们包含在这里是为了使本章内容自成一体。



% %\point Four \TeX\ processors
% \section{Four \TeX\protect\ processors\\四个 \TeX\ 处理器}

% The way \TeX\ processes its input can be viewed as
% happening on four levels. One might  say that
% the \TeX\ processor is split into four separate units,
% each one accepting the output of the previous stage, and
% delivering the input for the next stage. The input of
% the first stage is then the \n{.tex} input file; the output
% of the last stage is a \n{.dvi} file.

% \TeX\ 处理输入的方式可以看作是在四个层次上进行的。可以说 \TeX\ 处理器被分为四个独立的单元,每个单元接受前一阶段的输出,并产生下一阶段的输入。第一阶段的输入是 \n{.tex} 输入文件;最后一阶段的输出是一个 \n{.dvi} 文件。

% For many purposes it is most convenient, and most insightful,
% to consider these four levels of processing as happening
% after one another, each one accepting the {\em completed\/}
% output of the previous level. In reality this is not true:
% all levels are simultaneously
% active, and there is interaction between them.

% 对于许多情况,将这四个处理层次依次考虑是最方便和最有洞察力的,每个层次接受上一层次的{\em 完整}输出。实际上,这并不完全正确:所有层次都是同时活动的,并且它们之间存在相互作用。



% The four levels are (corresponding roughly
% to the `eyes', `mouth', `stomach', and `bowels' respectively
% in Knuth's original terminology) as follows.

% 这四个层次(与 Knuth 最初的术语“眼睛”、“嘴巴”、“胃”和“肠道”相对应)如下所示:




% \begin{enumerate}\item
% The input processor. This is the piece of \TeX\ that
% accepts input lines from the file system of whatever computer
% \TeX\ runs on, and turns them into tokens.
% Tokens are the internal objects of \TeX:
% there are character tokens that constitute the typeset
% text, and control sequence tokens that are commands 
% to be processed by the next two levels.

% 输入处理器。这是 \TeX\ 的一部分,它从 \TeX\ 运行所在计算机的文件系统接受输入行,并将它们转换为记号。记号是 \TeX\ 的内部对象:其中有构成排版文本的字符记号,还有作为下两个层次处理的命令的控制序列记号。
% \item The expansion processor. 
% Some but not all of the tokens generated in the first level
% \ldash macros, conditionals, and a number
% of primitive \TeX\ commands \rdash  are subject to expansion.
% Expansion is the process that replaces some (sequences of)
% tokens by other (or no) tokens.

% 展开处理器。在第一层次生成的记号中,一些(但不是全部)\ldash 宏、条件语句和一些原始 \TeX\ 命令 \rdash 可以进行展开。展开是将一些(序列的)记号替换为其他(或不替换)记号的过程。
% \item The execution processor. 
% Control sequences that are not expandable are executable,
% and this execution takes place on the third level of the
% \TeX\ processor.

% 执行处理器。不可展开的控制序列是可执行的,并且这个执行过程发生在 \TeX\ 处理器的第三层次上。


% One part of the activity here concerns changes to
% \TeX's internal state: assignments (including
% macro definitions) are typical activities in this
% \awp
% category. The other major thing happening on this level
% is the construction of horizontal, vertical, and
% mathematical lists.

% 活动的一部分涉及到 \TeX\ 的内部状态的更改:赋值(包括宏定义)是这一类别中的典型活动。在此层次上发生的另一项主要事务是构建水平、垂直和数学列表。
% \item The visual processor. 
% In the final level of processing
% the visual part of \TeX\ processing is performed. Here
% horizontal lists are broken into paragraphs, 
% vertical lists are broken into pages,
% and  formulas are built out of math lists. 
% Also the output to the \n{dvi} file takes place on this level.
% The algorithms working here are not accessible to the user,
% but they can be influenced by a number of parameters.

% 视觉处理器。在处理的最后一层中,执行 \TeX\ 的视觉处理。在这里,水平列表被分割为段落,垂直列表被分割为页面,并且数学列表被构建为公式。此外,输出到 \n{dvi} 文件也在此层次上进行。在此处工作的算法用户无法访问,但可以通过多个参数进行调整。
% \end{enumerate}



% %\point The input processor
% \section{The input processor\\输入处理器}

% The input processor of \TeX\ is that part of \TeX\ that
% translates whatever characters it gets from the input file
% into tokens. The output of this processor is a stream
% of tokens: a token list. Most tokens fall into one of two categories:
% character tokens and control sequence tokens. 
% The remaining category is that of the parameter tokens;
% these will not be treated in this chapter.

% \TeX\ 的输入处理器是将输入文件中获取的字符转换为记号的 \TeX\ 的一部分。该处理器的输出是一系列记号:一个记号列表。大多数记号属于以下两个类别之一:字符记号和控制序列记号。剩下的类别是参数记号;这些在本章中不进行讨论。

% %\spoint Character input
% \subsection{Character input\\字符输入}

% For simple input text, characters are made into
% character tokens. However, \TeX\ can ignore input characters:
% a row of spaces in the input is usually equivalent to just one
% space. Also, \TeX\ itself can insert tokens that do not correspond
% to any character in the input, for instance the space token
% at the end of the line, or the \cs{par} token after an empty line.

% 对于简单的输入文本,字符被转换为字符记号。但是,\TeX\ 可以忽略输入字符:输入中的多个空格通常等效为一个空格。此外,\TeX\ 本身可以插入不对应任何输入字符的记号,例如行末的空格记号,或空行后的 \cs{par} 记号。





% Not all character tokens signify characters to be typeset.
% \altt
% Characters fall into sixteen categories \ldash each one
% specifying a certain function that a character can have \rdash 
% of which only two contain the characters that will be
% typeset. The other categories contain such characters 
% as~\n{\char`\{}, \n{\char`\}}, 
% \n\&, and~\n\#. A~character token can be considered
% as a pair of numbers: the character code \ldash typically the \ascii\
% code \rdash  and the category code.
% It is possible to change
% the category code that is associated with a particular
% character code.

% 并非所有字符记号都表示要排版的字符。\altt
% 字符分为十六个类别 \ldash 每个类别指定字符可以具有的某个功能 \rdash 其中只有两个类别包含将要排版的字符。%
% 其他类别包含诸如 ~\n{\char`\{}, \n{\char`\}}, \n\&, 和~\n\# 等字符。%
% 字符记号可以被视为一对数:字符码(通常是 \ascii\ 码)和类别码。可以更改与特定字符码相关联的类别码。

% When the escape character (by default~\cs{}$\,$) appears in the input,
% \TeX's behaviour in forming tokens is more complicated. 
% Basically,
% \TeX\ builds a control sequence by taking a number of characters
% from the input and lumping them together into a single token.

% 当转义字符(默认为 \cs{}$,$)出现在输入中时,\TeX\ 形成记号的行为更加复杂。基本上,\TeX\ 通过从输入中取出一些字符并将它们组合成一个单独的记号来构建控制序列。



% The behaviour with which \TeX's input processor 
% reacts to category codes can be described
% as a machine that switches between three internal states:
% $N$,~new line; $M$,~middle of line; $S$,~skipping spaces.
% These states and the transitions between them are treated
% in Chapter~\ref{mouth}.

% \TeX\ 的输入处理器对类别码的反应可以被描述为在三个内部状态之间切换的机器:$N$,新行;$M$,行中间;$S$,跳过空格。这些状态及其之间的转换在第~\ref{mouth}~章中进行讨论。



% %\spoint Two-level input processing
% \subsection{Two-level input processing\\两级输入处理}

% \TeX's input processor is in fact itself a two-level processor.
% Because of limitations of the terminal, the editor, or the operating
% \awp
% system, the user may not be able to input certain desired characters.
% Therefore, \TeX\ provides a mechanism to access
% with two superscript characters all of the available character
% positions. This may be considered
% a separate stage of \TeX\ processing, taking place
% prior to the three-state machine mentioned above.

% 实际上,\TeX\ 的输入处理器本身就是一个两级处理器。由于终端、编辑器或操作系统的限制,用户可能无法输入某些期望的字符。因此,\TeX\ 提供了一种机制,通过两个上标字符来访问所有可用的字符位置。这可以被视为 \TeX\ 处理的一个单独阶段,在上述三状态机之前进行。


% For instance, the sequence \verb>^^+> is replaced by~\n{k} because
% the \ascii{} codes of \n k and \n + differ by~64. 
% Since this replacement takes place before tokens are formed,
% writing \verb>\vs^^+ip 5cm> has the same effect as
% \verb>\vskip 5cm>. Examples more useful than this exist.

% 例如,序列 \verb>^^+> 会被替换为~\n{k},因为 \n k 和 \n + 的 \ascii{} 码之差为~64。由于这种替换发生在记号形成之前,编写 \verb>\vs^^+ip 5cm> 的效果与 \verb>\vskip 5cm> 相同。存在比这更有用的示例。


% Note that this first stage is a transformation from
% characters to characters, without considering category
% codes. These come into play only in the second phase
% of input processing where characters are converted
% to character tokens by coupling the category code
% to the character code.

% 注意,这个第一阶段是从字符到字符的转换,而不考虑类别码。只有在输入处理的第二阶段,字符才通过将类别码与字符码相结合,转换为字符记号时起作用。


% %\point The expansion processor
% \section{The expansion processor\\展开处理器}

% \TeX's  expansion processor accepts a stream of tokens
% and, if possible, 
% expands the tokens in this stream one by one
% until only unexpandable tokens remain.
% Macro expansion is the clearest example of this:
% if a control sequence is a macro name, it is replaced
% (together possibly with parameter tokens) by 
% the definition text of the macro.

% \TeX\ 的展开处理器接受一系列记号,并在可能的情况下逐个展开这些记号,直到只剩下不可展开的记号为止。宏展开是最明显的例子:如果一个控制序列是宏名,它将被宏的定义文本(可能连同参数记号)替换。

% Input for the expansion processor is provided mainly
% by the input processor. The stream of tokens coming
% from the first stage of \TeX\ processing is subject
% to the expansion process, and the result is a stream
% of unexpandable tokens which is fed to the execution processor.

% 展开处理器的输入主要由输入处理器提供。来自 \TeX\ 处理的第一阶段的记号流会经过展开处理,结果是一系列不可展开的记号,这些记号会被馈送给执行处理器。

% However, the expansion processor comes into play 
% also when (among others) an \cs{edef} or \cs{write} is processed.
% The parameter token list of these commands is
% expanded very much as if the lists had been
% on the top level, instead of the argument to a command.

% 然而,展开处理器还在处理(等等)\cs{edef} 或 \cs{write} 等命令时发挥作用。这些命令的参数记号列表的展开非常类似于列表位于顶层,而不是命令的参数中。




% %\spoint The process of expansion
% \subsection{The process of expansion\\扩展过程}

% Expanding a token consists of the following steps:

% 扩展一个记号(token)包括以下步骤:
% \begin{enumerate}
% \item See whether the token is expandable. 

% 检查记号是否可扩展。
% \item If the token is unexpandable, pass it to the token
%       list currently being built, and take on the next token. 

%       如果记号不可扩展,将其传递给当前正在构建的记号列表,并继续处理下一个记号。
% \item If the token is expandable, replace it by its expansion.
%       For macros without parameters, and a few primitive commands
%       such as \cs{jobname}, this is indeed a simple replacement.
%       Usually, however, \TeX\ needs to absorb some argument tokens from
%       the stream in order to be able to form the replacement
%       of the current token.
%       For instance, if the token was a macro with parameters,
%       sufficiently many tokens need to be absorbed to form
%       the arguments corresponding to  these parameters.

%       如果记号可扩展,用其扩展内容替换它。
% 对于没有参数的宏和一些原始命令(如\cs{jobname}),这确实是一个简单的替换。
% 然而,通常情况下,\TeX 需要从流中吸收一些参数记号,以便能够形成当前记号的替换内容。
% 例如,如果记号是带有参数的宏,需要吸收足够多的记号来形成与这些参数相对应的参数值。
% \item Go on expanding, starting with the first token of the
%       expansion. 

%       继续扩展,从扩展内容的第一个记号开始。
% \end{enumerate}
% %
% Deciding whether a token is expandable is
% a simple decision. Macros and active characters, 
% conditionals, and a number of primitive \TeX\ commands
% \awp
% (see the list on page~\pageref{expand:lijst})
% are expandable, other tokens are not.
% Thus the expansion processor replaces macros by their expansion,
% it evaluates conditionals and eliminates any irrelevant parts of 
% these, but tokens such as \cs{vskip} and character tokens,
% including characters such as dollars and braces, are passed untouched.

% 决定一个记号是否可展开是一个简单的决策。宏和活动字符、条件语句以及一些原始的 \TeX\ 命令
% \awp
% (参见第~\pageref{expand:lijst} 页上的列表)
% 是可展开的,其他的记号则不可展开。
% 因此,展开处理器会用宏的展开部分替换宏,评估条件语句并消除其中的任何无关部分,
% 但是像 \cs{vskip} 这样的记号和字符记号,包括美元符号和花括号等字符,会原样传递。



% %\endinput
% %\spoint Special cases: \cs{expandafter}, \cs{noexpand}, and \cs{the}
% \subsection{Special cases: \cs{expandafter}, \cs{noexpand}, and \cs{the}\\特殊情况:\cs{expandafter}、\cs{noexpand} 和 \cs{the}}

% As stated above,
% after a token has been expanded, \TeX\ will start expanding
% the resulting tokens. At first sight the \cs{expandafter}
% command would seem to be an exception to this rule, because
% it expands only one step. What actually happens is that
% the sequence 

% 如上所述,
% 在记号展开后,\TeX\ 将开始展开所得到的记号。乍一看,\cs{expandafter}
% 命令似乎是一个例外,因为它只展开一步。实际上发生的是,
% 序列\begin{disp}\cs{expandafter}\gr{token$_1$}\gr{token$_2$}\end{disp}
% 被替换为 \hfill is replaced by 
% \begin{disp}\gr{token$_1$}\gr{\italic expansion of token$_2$}\end{disp}
% and this replacement is in fact reexamined by the expansion
% processor.

% 而这个替换实际上会被展开处理器重新检查。

% Real exceptions do exist, however. If the 
% current token is the \cs{noexpand} command, the next
% token is considered for the moment to be unexpandable:
% it is handled as if it were \cs{relax}, and it is
% passed to the token list being built.

% 然而,确实存在一些真正的例外情况。如果
% 当前记号是 \cs{noexpand} 命令,那么下一个
% 记号在此刻被视为不可展开的:
% 它被处理时就好像它是 \cs{relax},并且它被
% 传递给正在构建的记号列表。


% 例如,
% 在宏定义中\hfill
% For example,
% in the macro definition
% \begin{verbatim}
% \edef\a{\noexpand\b}
% \end{verbatim}
% the replacement text \verb>\noexpand\b> is expanded at definition 
% time. The expansion of \cs{noexpand} is the next token, with
% a temporary meaning of \cs{relax}. Thus, when the expansion
% processor tackles the next token, the~\cs{b}, it will consider
% that to be unexpandable, and just pass it to the token list
% being built, which is the replacement text of the macro.

% 替换文本 \verb>\noexpand\b> 在定义时被展开。
% \cs{noexpand} 的展开是下一个记号,并且临时具有 \cs{relax} 的含义。
% 因此,当展开处理器处理下一个记号时,即 \cs{b},它将被视为不可展开的,并且会直接传递给正在构建的记号列表,也就是宏的替换文本。

% Another exception is that the tokens
% resulting from \cs{the}\gr{token variable}
% are not expanded further if this statement occurs
% inside an \cs{edef} macro definition.

% 另一个例外是,由 \cs{the}\gr{token variable} 产生的记号
% 如果出现在 \cs{edef} 宏定义内部,则不会进一步展开。



% %\spoint Braces in the expansion processor
% \subsection{Braces in the expansion processor\\展开处理器中的花括号}

% Above, it was said that braces are passed as unexpandable
% character tokens. In general this is true. For instance,
% the \cs{romannumeral} command is handled by the expansion
% processor; when confronted with 

% 前面提到,花括号作为不可展开的字符记号传递。一般来说,这是正确的。例如,
% \cs{romannumeral} 命令由展开处理器处理;当遇到
% \begin{verbatim}
% \romannumeral1\number\count2 3{4 ...
% \end{verbatim} 
% \TeX\ will expand until the brace is encountered:
% if \cs{count2} has the value of zero, the result will be
% the roman numeral representation of~\n{103}.

% \TeX\ 会展开直到遇到花括号为止:
% 如果 \cs{count2} 的值为零,则结果将是
% 数字 103 的罗马数字表示。

% As another example, 

% 作为另一个例子,\begin{verbatim}
% \iftrue {\else }\fi
% \end{verbatim}
% is handled by the expansion processor 
% completely analogous to

% 在展开处理器中的处理方式与\begin{disp}\cs{iftrue} {\italic a}\cs{else} {\italic b}\cs{fi}\end{disp}
% \awp
% The result is a character token, independent of its category.

% 完全类似。结果是一个字符记号,与其类别无关。
      

% However, in the context of macro expansion 
% the expansion  processor will 
% recognize braces. 
% First of all, a balanced pair of braces marks off a group of tokens
% to be passed as one argument.
% If a macro has an argument 

% 然而,在宏展开的上下文中,展开处理器会识别花括号。首先,一对平衡的花括号将一组记号标记为作为一个参数传递。如果一个宏带有参数,如\begin{verbatim}
% \def\macro#1{ ... }
% \end{verbatim}
% one can call it with a single token, as in

% 可以使用单个记号调用它,例如
% \begin{verbatim}
% \macro 1 \macro \$
% \end{verbatim}
% or with a group of tokens, surrounded by braces

% 或者使用由花括号括起来的记号组
% \begin{verbatim}
% \macro {abc} \macro {d{ef}g}
% \end{verbatim}



% Secondly, when the arguments for a macro with
% parameters are read, no expressions with unbalanced braces
% are accepted. In 

% 其次,在读取带参数的宏的实参时,不接受有不平衡花括号的表达式。在
% \begin{verbatim}
% \def\a#1\stop{ ... }
% \end{verbatim}
% the argument consists of all
% tokens up to the first occurrence of \cs{stop}
% that is not in braces: in

% 中,实参包括从第一个不在花括号中的 \cs{stop} 开始的所有记号:在
% \begin{verbatim}
% \a bc{d\stop}e\stop
% \end{verbatim}
% the argument of~\cs{a} is \verb>bc{d\stop}e>.
% Only balanced expressions
% are accepted here.

% 中,\cs{a} 的实参是 \verb>bc{d\stop}e>。这里只接受平衡的表达式。
