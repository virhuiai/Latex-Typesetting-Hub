
% %\InputFile:boxes
% %%%% this is input file [boxes]
% %\tracingmacros=2 \tracingcommands\tracingmacros
% %\subject[boxes] Boxes
% \endofchapter
% \chapter{Boxes\\盒子}\label{boxes}

% The horizontal and vertical boxes of \TeX\ are containers for
% \term box\par
% pieces of horizontal and vertical lists.
% Boxes can be stored in box registers. 
% This chapter treats box registers and such
% aspects of boxes as their dimensions, and the way their components
% are placed relative to each other.

% \TeX\ 的水平盒子和垂直盒子是水平和垂直列表的容器。
% 盒子可以存储在盒子寄存器中。
% 本章介绍盒子寄存器以及盒子的尺寸和它们的组成部分相对于彼此的放置方式等方面。

% \begin{inventory}
% \item [\cs{hbox}] 
%       Construct a horizontal box.

%       构建一个水平盒子。
% \item [\cs{vbox}] 
%       Construct a vertical box with reference point of the last item.

%       构建一个垂直盒子,其参考点位于最后一个项目。
% \item [\cs{vtop}] 
%       Construct a vertical box with reference point of the first item.

%       构建一个垂直盒子,其参考点位于第一个项目。
% \item [\cs{vcenter}] 
%       Construct a vertical box vertically centred
%       on the math axis; this command can only be used in math mode.

%       构建一个在数学轴上垂直居中的垂直盒子;此命令只能在数学模式中使用。
% \item [\cs{vsplit}] 
%       Split off the top part of a vertical box. 

%       将垂直盒子的顶部部分分离出来。
% \item [\cs{box}] 
%       Use a box register, emptying it. 

%       使用一个盒子寄存器,并清空其内容。
% \item [\cs{setbox}] 
%       Assign a box to a box register.

%       将盒子赋值给一个盒子寄存器。
% \item [\cs{copy}] 
%       Use a box register, but retain the contents. 

%       使用一个盒子寄存器,但保留其内容。
% \item [\cs{ifhbox \cs{ifvbox}}]
% \mdqon
%       Test whether a box register contains a horizontal/""vertical box.

%       检查一个盒子寄存器是否包含一个水平/垂直盒子。
% \mdqoff

% \item [\cs{ifvoid}] 
%       Test whether a box register is empty.

%       检查一个盒子寄存器是否为空。
% \item [\cs{newbox}] 
%       Allocate a new box register. 

%       分配一个新的盒子寄存器。
% \item [\cs{unhbox \cs{unvbox}}]
%       Unpack a box register containing a horizontal/vertical box,
%       adding the contents to the current horizontal/vertical list,
%       and emptying the register. 

% 解开包含水平/垂直盒子的盒子寄存器,将其内容添加到当前的水平/垂直列表中,并清空寄存器。
% \item [\cs{unhcopy \cs{unvcopy}}]
%       The same as \cs{unhbox}$\,$/$\,$\cs{unvbox},
%       but do not empty the register. 

%       与 \cs{unhbox}$,$/$,$\cs{unvbox} 相同,但不清空寄存器。
% \item [\cs{ht \cs{dp} \cs{wd}}]
%       Height/depth/width of the box in a box register. 

%       盒子寄存器中盒子的高度/深度/宽度。
% \item [\cs{boxmaxdepth}] 
%       Maximum allowed depth of boxes.
%       Plain \TeX\ default:~\cs{maxdimen}.

%       盒子的最大允许深度。Plain \TeX\ 的默认值:\cs{maxdimen}。
% \item [\cs{splitmaxdepth}]
%       Maximum allowed depth of boxes generated by \cs{vsplit}.

%       \cs{vsplit} 生成的盒子的最大允许深度。
% \item [\cs{badness}] 
%       Badness of the most recently constructed box.

%       最近构建的盒子的劣度。
% \item [\cs{hfuzz \cs{vfuzz}}]
%       Excess size that \TeX\ tolerates before it considers  

%       在 \TeX\ 认为一个水平/垂直盒子过满之前,允许的多余尺寸。
% \mdqon
%       a horizontal/""vertical box overfull.
% \mdqoff

% \item [\cs{hbadness \cs{vbadness}}]
%       Amount of tolerance before \TeX\ reports an underfull 

%       在 \TeX\ 报告一个水平/垂直盒子不充分或过充分之前的容忍度。
% \mdqon
%       or overfull  horizontal/""vertical box.
% \mdqoff

% \item [\cs{overfullrule}] 
%       Width of the rule that is printed to indicate 
%       overfull horizontal boxes.

%       用于指示过度充满水平盒子的标尺的宽度。 
% \item [\cs{hsize}] 
%       Line width used for text typesetting inside a vertical box.
% \awp

% 用于垂直盒子内文本排版的行宽。
% \item [\cs{vsize}] 
%       Height of the page box.

%       页面盒子的高度。
% \item [\cs{lastbox}] 
%       Register containing the last item added to the current list, 
%       if this was a box.

%       寄存器,其中包含添加到当前列表中的最后一个项目,如果该项目是一个盒子。
% \item [\cs{raise \cs{lower}}]
%       Adjust vertical positioning of a box in horizontal mode. 

%       调整水平模式下盒子的垂直位置。
% \item [\cs{moveleft \cs{moveright}}]
%       Adjust horizontal positioning of a box in vertical mode. 

%       调整垂直模式下盒子的水平位置。
% \item [\cs{everyhbox \cs{everyvbox}}]
% \mdqon
%       Token list inserted at the start of a horizontal/""vertical box.

%       插入到水平/垂直盒子开始处的记号列表。
% \mdqoff

% \end{inventory}

