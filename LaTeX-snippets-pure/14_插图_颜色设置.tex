%%%%%% grffile package extends the file name processing 
%%%%%% of package graphics to support a larger range of file names.
\usepackage{graphicx,grffile}
%%%%%% set default figure placement to htbp
\makeatletter
\def\fps@figure{htbp}
\makeatother
%%%%%% 插图路径命令
%\graphicspath{{}}

%%%%%% 浮动体宏包
\usepackage{floatrow}
%%%%%% 旋转宏包
\usepackage{rotating}

%\includepdf[pages={3,5}]{..../13-1.pdf};
%可使用合并命令\includepdfmerge将多页外部文件自动缩小合并后插入一页之中。
\usepackage{pdfpages}

\usepackage{picinpar}
%\begin{window}[行数,位置,{绕排对象},{标题}] 绕排文本
%\end{window}

%\begin{figwindow}[行数,位置,{绕排对象},{标题}] 绕排文本
%\end{figwindow}

%\begin{tabwindow}[行数,位置,{绕排对象},{标题}] 绕排文本
%\end{tabwindow}

%只是后两种可以在绕排对象的标题前自动添加标 题标志和分隔符号。

%右侧 imgRight
%\begin{window}[0,r,{\includegraphics[width=0.5\textwidth]{图}},{}]
%
%\end{window}

% \usepackage[dvipsnames]{xcolor}
%-----------------------------------------------------------定义颜色---------------
\usepackage{pgf-umlcd}
\newlength{\attribute宽度}
\newlength{\attribute宽度临时}
\newcommand\attribute宽度计算准备{\setlength{\attribute宽度}{0pt}}
\newcommand\attribute宽度计算[1]{%
\settowidth{\attribute宽度临时}{#1}%
\ifthenelse{\lengthtest{\attribute宽度临时 > \attribute宽度}}{\settowidth{\attribute宽度}{#1}}{}%
}%
% \attribute宽度计算准备
% \attribute宽度计算{{\tt +} Public}
% \attribute宽度计算{{\tt \#} Protected}
% \attribute宽度计算{{\tt -} Private}
% \attribute宽度计算{$\sim$ Package}

% \usepackage{graphpap}%TODO 待确定功能,旧文件是放第一行

%%%///////////////////////////////////////////////////////
%%%%%% 定义行间图命令
\newcommand{\vInlineImg}[2]{%
\begin{minipage}[c]{#1}%
    \includegraphics[width=#1]{#2}%
\end{minipage}%
}


\newcommand{\vImgColorBox}[2][0.5em]{\fboxrule=1.2pt \fboxsep=#1\fcolorbox{gray}{white}{#2}}
%%%%%%%%%%%%%%%%%%%%%%%%%%%%%%%%%%%%%%%%%%%%%%%%%%%%%%%%%%%例子 %\vImgColorBox[3em]{\includegraphics{abstract-env-noframe}}

\definecolor{blueblack}{cmyk}{0,0,0,0.35}%浅黑
\definecolor{darkblue}{cmyk}{1,0,0,0}%纯蓝
\definecolor{lightblue}{cmyk}{0.15,0,0,0}%浅蓝

% success
\definecolor{colorSuccess}{HTML}{28a745}
\definecolor{colorSuccessBorder}{HTML}{c3e6cb}
\definecolor{colorSuccessBackground}{HTML}{d4edda}
\definecolor{colorSuccessText}{HTML}{155724}

% info
\definecolor{colorInfo}{HTML}{17a2b8}
\definecolor{colorInfoBorder}{HTML}{bee5eb}
\definecolor{colorInfoBackground}{HTML}{d1ecf1}
\definecolor{colorInfoText}{HTML}{0c5460}

% danger
\definecolor{colorDanger}{HTML}{dc3545}
\definecolor{colorDangerBorder}{HTML}{f5c6cb}
\definecolor{colorDangerBackground}{HTML}{f8d7da}
\definecolor{colorDangerText}{HTML}{721c24}

% warning
\definecolor{colorWarning}{HTML}{ffc107}
\definecolor{colorWarningBorder}{HTML}{ffeeba}
\definecolor{colorWarningBackground}{HTML}{fff3cd}
\definecolor{colorWarningText}{HTML}{856404}

\definecolor{mpurple}{RGB}{48,10,36}
\definecolor{mgray}{RGB}{70,72,67}
\definecolor{ogray}{RGB}{148,147,141}
\definecolor{oorange}{RGB}{233,101,56}
\definecolor{termimal}{RGB}{80,78,70}
\definecolor{linux}{RGB}{0,39,51}
\definecolor{windows}{HTML}{00B294}
\definecolor{cvgrayc}{RGB}{247,247,247}
\definecolor{cvgray}{RGB}{220,220,220}
\definecolor{cvgrayb}{RGB}{153,153,153}
\definecolor{cvblue}{RGB}{223,238,255}
\definecolor{zhanqing}{RGB}{0,51,113}
\definecolor{chengse}{RGB}{250,140,53}

\definecolor{AppleRed}{RGB}{255,95,86}
\definecolor{AppleYellow}{RGB}{255,189,46}
\definecolor{AppleGreen}{RGB}{39,201,63}
\definecolor{AppleGray}{HTML}{D8D6D9}

\definecolor{WinGray}{HTML}{FFFFFF}
\definecolor{WinBlue}{HTML}{1883D7}