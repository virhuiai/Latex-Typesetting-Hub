\def\advisespace{\hbox{}\qquad}
\def\labeladvise{$\to$}

\makeatletter
\def\advise{\par\list\labeladvise
    {\advance\linewidth\@totalleftmargin
     \@totalleftmargin\z@
     \@listi
     \let\small\footnotesize \small\sffamily
     \parsep \z@ \@plus\z@ \@minus\z@
     \topsep6\p@ \@plus1\p@\@minus2\p@
     \def\makelabel##1{\hss\llap{##1}}}}
\let\endadvise\endlist
\makeatother

%\begin{advise}
%\item Must I do that really?
%       \advisespace
%       Yes and no. Some books about programming say this is good.
%       What a mistake! Typing takes time---which is wasted if the code is clear to
%       you. And if you need that time to understand what is going on, the
%       author of the book should reconsider the concept of presenting the
%       crucial things---you might want to say that about this guide even---or
%       you're simply inexperienced with programming. If only the latter case
%       applies, you should spend more time on reading (good) books about
%       programming, (good) documentations, and (good) source code from other
%       people. Of course you should also make your own experiments.
%       You will learn a lot. However, running the example through \LaTeX\
%       shows whether the \packagename{listings} package is installed correctly.
%\end{advise}
