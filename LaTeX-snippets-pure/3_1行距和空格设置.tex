\usepackage{parskip}

% virhuiai@bogon Downloads % latexdef -c ctexbook -v parskip                
% \the\parskip:
% 0.0pt plus 1.0pt
% virhuiai@bogon Downloads % latexdef -c ctexbook -p parskip -v parskip
% \the\parskip:
% 8.22072pt plus 2.0pt

%%%%%%%%%%%%%%%%%%%%%%%%%%%%%%%%%%%%%%%%%%%%%%%%%%%%%%%%%%%%%%%%
%%%%%% 本书所用 行距%4.2.11
%%%%%%   行距设置
%%%%%%     由于本书的标题行和命令行遗留的空白较多,故将行距设置为\linespread{1.245},
%%%%%%     其行距系数略小于中文字体宏包默认的1.3,可使每页五号字文本排版整40行。
%%%%%%   在ctex宏包调用命令(如果已调用)之后,使用系统提供的行距系数命令 linespread
%%%%%%
%%%%%% 系统默认的行距约为字体尺寸的1.2倍,也称单倍行距;
%%%%%% 在调用ctex宏包后,行距被放宽到约为字体尺寸的1.56倍,
%%%%%% 这是因为汉字的高低基本相同,没有英文那种大小写之分,
%%%%%% 若是仍然使用系统默认的行距,那中文就挤作一团了。
%%%%%%%%%%%%%%%%%%%%%%%%%%%%%%%%%%%%%%%%%%%%%%%%%%%%%%%%%%%%%%%%
\linespread{1.245}

%%%%%%%%%%%%%%%%%%%%%%%%%%%%%%%%%%%%%%%%%%%%%%%%%%%%%%%%%%%%%%%%
%%%%%%本书所用 空格命令
%%%%%%    \xspace 可⾃动⽣成⼀个空格,除⾮其后是标点符号。
%%%%%%%%%%%%%%%%%%%%%%%%%%%%%%%%%%%%%%%%%%%%%%%%%%%%%%%%%%%%%%%%
%%%%%% 实测只对英文有效 2019-08-13 08:54:31
\usepackage{xspace}