%%%%%%%%%%%%%%%%%%%%%%%%%%%%%%%%%%%%%%%%%%%%%%%%%%%%%%%%%%%%%%%%
%%%%%%  列表宏包
%%%%%%    及其
%%%%%%  设置
%%%%%%%%%%%%%%%%%%%%%%%%%%%%%%%%%%%%%%%%%%%%%%%%%%%%%%%%%%%%%%%%

%%%%%% 提供各种符号的编号 可用于改换列表的标志符号
\usepackage{pifont}
\usepackage{mflogo,metalogo,mflogo}
\usepackage{texnames}
\usepackage{bbding}
\usepackage{amssymb,latexsym,textcomp,mathrsfs,euscript,yhmath} % 与默认字体不冲突的一些符号包

%%%%%% 给itemize、enumerate和description环境命令添加了一个可选参数,
%%%%%% 可对它们的排版格式进行全面设置。
\usepackage{enumerate}

%%%%%% enumitem enumerate 有什么关系? todo
\usepackage{enumitem}

%%%%%% 使用olditem,oldenum不改变,paralist有借鉴enumerate来着
%[olditem,oldenum]
\usepackage[olditem,oldenum]{paralist}%LaTeX Error: Option clash for package paralist
% \usepackage{paralist}
% \renewenvironment{itemize}{\begin{compactitem}}{\end{compactitem}}
% \renewenvironment{compactenum}{\begin{compactenum}}{\end{compactenum}}
% \renewenvironment{description}{\begin{compactdesc}}{\end{compactdesc}}

%%%%%% 提供itemize、enumerate和description*共3个带星号的列表环境,
%%%%%% 它们与系统提供的3个无星号同名列表环境的区别是所有条目之间的距离等于文本行距,
%%%%%% 这样列表文本与上下文本更为协调。
%%%%%% todo 该宏包还提供了一个可变化三种词条格式的 basedescript 解说列表环境  ???


\usepackage{mdwlist}

\usepackage{bbding}

\usepackage{manfnt}%%\dbend
\usepackage{eurosym}

% \renewenvironment{itemize}%
% {\begin{compactitem}}%
% {\end{compactitem}}%
% \begin{compactitem}
% \end{compactitem}

%\usepackage{mdwlist}
%下一行标签:如果标签太宽而无法与第一行文本相邻,则将其单独放置在一行上; 正文以通常的缩进从下一行开始。
\newenvironment{descNL}[1][2em]{%
\begin{basedescript}{\desclabelwidth{#1}\desclabelstyle{\nextlinelabel}}}%
{\end{basedescript}}

%多行标签:标签在具有适当宽度的 parbox 中排版; 如果它不适合一行,那么文本将被拆分到后续行。
%对中文处理不好
\newenvironment{descML}[1][2em]{%
\begin{basedescript}{\desclabelwidth{#1}\desclabelstyle{\multilinelabel}}}%
{\end{basedescript}}

%推标签:如果标签太宽而无法容纳分配给它的空间,则项目文本的开头将被“推”到右侧以为标签提供空间。 这是标准的 LATEX description 行为。
\newenvironment{descPL}[1][2em]{%
\begin{basedescript}{\desclabelwidth{#1}\desclabelstyle{\pushlabel}}}{\end{basedescript}}


\newlength{\basedescriptDesclabelwidth}
\newenvironment{dl}[1][标签]{%
\settowidth{\basedescriptDesclabelwidth}{\tt #1}
\begin{basedescript}{%
\renewcommand{\makelabel}[1]{\tt ##1}%
\desclabelwidth{\basedescriptDesclabelwidth}%
\desclabelstyle{\pushlabel}}}{\end{basedescript}}
% \begin{dl}
% \end{dl}

% \begin{compactitem}
% \end{compactitem}

% \begin{compactenum}
% \end{compactenum}

% \begin{compactdesc}
% \end{compactdesc}

% \begin{descNL}
% \item [地震烈度] 地震发生时,在波及范围内一定地点%
% 地面震动的激烈程度。
% \item [地球介质的流变性] 地球介质在外力作用下的流%
% 动和变形的性质。 流变性与物质所处的环境温度、压%
% 力和外力作用时间的长短有关。
% \end{descNL}

% \begin{descML}
% \item [地震烈度] 地震发生时,在波及范围内一定地点%
% 地面震动的激烈程度。
% \item [地球介质的流变性] 地球介质在外力作用下的流%
% 动和变形的性质。 流变性与物质所处的环境温度、压%
% 力和外力作用时间的长短有关。
% \end{descML}

% \begin{descPL}
% \item [地震烈度] 地震发生时,在波及范围内一定地点%
% 地面震动的激烈程度。
% \item [地球介质的流变性] 地球介质在外力作用下的流%
% 动和变形的性质。 流变性与物质所处的环境温度、压%
% 力和外力作用时间的长短有关。
% \end{descPL}

% 在导言中调用符号宏包 pifont,并使用其提供的编号为 172 的带圈阿 拉伯数字符号作为排序列表第一个条目的标号
% 每种带圆圈的数字符号 pifont 宏包只提供了 1~10
\newcounter{带圈文字}
\newcommand\带圈文字[1]{
\protect\setcounter{带圈文字}{171+#1}
\protect\ding{\value{带圈文字}}}