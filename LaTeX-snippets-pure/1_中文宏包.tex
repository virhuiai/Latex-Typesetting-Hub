\PassOptionsToPackage{AutoFakeBold=true,AutoFakeSlant=true}{xeCJK}
\PassOptionsToPackage{no-math}{fontspec}

%  加入这一行,就OK了 https://blog.csdn.net/weixin_39278265/article/details/125257636
% Option clash for package xcolor.
\PassOptionsToPackage{prologue,dvipsnames}{xcolor}

% \documentclass[a4paper,openany]{book}
\documentclass[a4paper,oneside,
zihao=-4,space,scheme=chinese,heading=true,
hyperref, 
%添加fntef选项,它将会自动加载下画线宏包CJKfntef
fntef,
fancyhdr, 
% ctex 宏包的字号设定会让许多固定字号的字体宏包报太多 warning,
% 应当抑制 中文正文字体使用 Fandol
fontset=none
]{ctexbook}

% Fandol字体是一款有争议的字体。猫啃网的建议是:对于存在争议的字体,不建议商业使用,避免不必要的法律纠纷。
% 而代替的字体,猫啃网推荐使用方正黑体、方正仿宋、方正楷体、方正书宋,这几款方正的字体是确定可以免费商用的

% virhuiai@virhuiaideMacBook-Pro ~ % fc-list |grep 方正
% /Users/virhuiai/Library/Fonts/FangZhengShuSong-GBK-1.ttf: 方正书宋_GBK,FZShuSong\-Z01:style=Regular
% /Users/virhuiai/Library/Fonts/FangZhengHeiTi-GBK-1.ttf: 方正黑体_GBK,FZHei\-B01:style=Regular
% /Users/virhuiai/Library/Fonts/FangZhengFangSong-GBK-1.ttf: 方正仿宋_GBK,FZFangSong\-Z02:style=Regular
% /Users/virhuiai/Library/Fonts/FangZhengKaiTi-GBK-1.ttf: 方正楷体_GBK,FZKai\-Z03:style=Regular

% /Users/virhuiai/Library/Fonts/FangZhengShuSongJianTi-1.ttf: 方正书宋简体,FZShuSong\-Z01S:style=Regular
% /Users/virhuiai/Library/Fonts/FangZhengShuSongFanTi-1.ttf: 方正书宋繁体,FZShuSong\-Z01T:style=Regular
% /Users/virhuiai/Library/Fonts/FangZhengHeiTiJianTi-1.ttf: 方正黑体简体,FZHei\-B01S:style=Regular
% /Users/virhuiai/Library/Fonts/FangZhengHeiTiFanTi-1.ttf: 方正黑体繁体,FZHei\-B01T:style=Regular
% /Users/virhuiai/Library/Fonts/FangZhengFangSongJianTi-1.ttf: 方正仿宋简体,FZFangSong\-Z02S:style=Regular
% /Users/virhuiai/Library/Fonts/FangZhengFangSongFanTi-1.ttf: 方正仿宋繁体,FZFangSong\-Z02T:style=Regular
% /Users/virhuiai/Library/Fonts/FangZhengKaiTiJianTi-1.ttf: 方正楷体简体,FZKai\-Z03S:style=Regular
% /Users/virhuiai/Library/Fonts/FangZhengKaiTiFanTi-1.ttf: 方正楷体繁体,FZKai\-Z03T:style=Regular



\setCJKmainfont{方正书宋_GBK}
\setCJKsansfont{方正黑体_GBK}
\setCJKmonofont{方正仿宋_GBK}

\setCJKfamilyfont{宋}{方正书宋_GBK}
\newcommand{\songti}{\CJKfamily{宋}}
\setCJKfamilyfont{黑}{方正黑体_GBK}
\newcommand{\heiti}{\CJKfamily{黑}}
\setCJKfamilyfont{仿宋}{方正仿宋_GBK}
\newcommand{\fangsong}{\CJKfamily{仿宋}}
\setCJKfamilyfont{楷}{方正楷体_GBK}
\newcommand{\kaishu}{\CJKfamily{楷}}



%
% 字体和符号宏包
% virhuiai@virhuiaideMacBook-Pro 2022_Perl % cd /usr/local/texlive/2022/texmf-dist/fonts/opentype/public/cm-unicode/
% virhuiai@virhuiaideMacBook-Pro cm-unicode % ls
% cmunbbx.otf	cmunbso.otf	cmunbxo.otf	cmunorm.otf	cmunsl.otf	cmunsx.otf	cmunui.otf
% cmunbi.otf	cmunbsr.otf	cmunci.otf	cmunoti.otf	cmunso.otf	cmuntb.otf	cmunvi.otf
% cmunbl.otf	cmunbtl.otf	cmunit.otf	cmunrb.otf	cmunss.otf	cmunti.otf	cmunvt.otf
% cmunbmo.otf	cmunbto.otf	cmunobi.otf	cmunrm.otf	cmunssdc.otf	cmuntt.otf
% cmunbmr.otf	cmunbx.otf	cmunobx.otf	cmunsi.otf	cmunst.otf	cmuntx.otf
% 
% \setmainfont{ } % 论文中西文部分默认使用的字体。
%通常到 Word 2003 为止,这里的默认字体都会是 Times New Roman。Linux 下也有同名字体。
\setmainfont{cmun}[
  Extension       = .otf,
  UprightFont     = *rm,
  ItalicFont      = *ti,
  SlantedFont     = *sl,
  BoldFont        = *bx,
  BoldItalicFont  = *bi,
  BoldSlantedFont = *bl,
]
%\setsansfont{ } %是西文默认无衬线字体。一般可能出现在大标题等显眼的位置。
\setsansfont{cmun}[
  Extension      = .otf,
  UprightFont    = *ss,
  ItalicFont     = *si,
  BoldFont       = *sx,
  BoldItalicFont = *so,
]
%\setmonofont{ }%是西文默认的等宽字体。一般用于排版程序代码。
%Courier 或者 Courier New 是常见的 Word 选项。Linux 下一般会有 Courier,但很少能看见 Courier New。
\setmonofont{cmun}[
  Extension      = .otf,
  UprightFont    = *btl,% light version
  ItalicFont     = *bto,%  light version
  BoldFont       = *tb,
  BoldItalicFont = *tx,
]


% 消除 \t 命令的字体 warning
\AtBeginDocument{%}
  \renewcommand*\t[1]{{\edef\restore@font{\the\font}\usefont{OML}{cmm}{m}{it}\accent"7F\restore@font#1}}
}%“\AtBeginDocument{…}“和”\AtEndDocument{…}"