%%%%%%    可用于排版表格的环境很多,但在一篇论文中不宜使用多种表格环境,%
%%%%%%    以免在调整与标题或上文间距时相互影响。

% 1. 表格大都采用 longtable 环境编制,( 另一个跨页的是 \usepackage{supertabular})
\usepackage{longtable}
% 2. 表格线使用 booktabs 宏包提供的三种画线命令绘制;
\usepackage{booktabs}
% 这两种宏包分别提供有调整表格与上下文间距、表格与标题间距和表格左右位置的命令;
% 3. 并且调用 tabularx 宏包,用以增强 longtable 环境中列格式的功能。
\usepackage{tabularx}
\usepackage{tabulary}%总宽可设、列宽自动确定 \begin{tabulary}{宽度}{列格式} \end{tabulary} % 列格式有 L、C、R 和 J 
\usepackage{ltablex}%扩展了 tabularx 环境功能,能自动分页,可用于排版跨页长表格。
% 在14color中。。。[table] 
\usepackage{colortbl}
\usepackage{bigstrut}
%%%%%% 跨行表格宏包multirow
\usepackage{multirow}

%%%%%%  Frank Mittelbach 等人编写的 array 宏包
%%%%%%  对 3 个表格环境(tabular tabular* array)的制表功能做了重大改进和扩展
%%%%%% 主要是增加和增强了列格式功能,
%%%%%% 还增加了许多表格参数的调整功能。
\usepackage{array}
%%%%%%    dcolumn小数点对齐宏包,
%%%%%%    它专为tabular和array表格环境定义了
%%%%%%    一个用于对齐小数点或逗号等标点符号的列格式选项
\usepackage{dcolumn}
%%%%%% 彩色表格宏包colortbl
%在调用 xcolor 宏包时,添加 table 选项就可以加载 colortbl
\usepackage{colortbl}
%%%%%%    对角线宏包slashbox
\usepackage{slashbox}

%%%%%% 虚线表格宏包arydshln
%如果已经调用或是还需要调用与表格有关的 array、colortab、colortbl 或 longtable 宏包,
%应将 arydshln 的调用命令放在这些宏包的调用命令之后,否则可能会造成莫名的错误
\usepackage{arydshln}

%////////bigstrut


%\begin{longtable*}%
%{@{\extracolsep{\fill}}*{}{>{\tt}l}@{}}
%    %\caption{}\label{tbl:}%
%    \\\toprule[1pt]
%    %\multicolumn{1}{c}{计数器名} &
%    %\multicolumn{1}{c}{用途} \\\midrule
%    %chapter        & 章序号计数器\\
%    %section        & 节序号计数器\\
%    %subsubsection  & 小小节序号计数器\\
%    %\bottomrule[1pt]
%\end{longtable*}

