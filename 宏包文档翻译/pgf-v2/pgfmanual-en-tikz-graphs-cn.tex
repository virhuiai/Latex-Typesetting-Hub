\setcounter{section}{18}
\setcounter{subsection}{3}
\setcounter{subsubsection}{0}
% Copyright 2019 by Till Tantau
% Copyright 2019 by Jannis Pohlmann
%
% This file may be distributed and/or modified
%
% 1. under the LaTeX Project Public License and/or
% 2. under the GNU Free Documentation License.
%
% See the file doc/generic/pgf/licenses/LICENSE for more details.


\section{Specifying Graphs\\图形规范}
\label{section-library-graphs}

\subsection{Overview}

\tikzname\ offers a powerful path command for specifying how the nodes in a
graph are connected by edges and arcs: The |graph| path command, which becomes
available when you load the |graphs| library.

\tikzname\ 提供了一个强大的路径命令,用于指定图中节点之间的边和弧的连接方式:|graph| 路径命令,在加载 |graphs| 库时可用。

\begin{tikzlibrary}{graphs}
    The package must be loaded to use the |graph| path command.

    在使用 |graph| 路径命令之前,必须加载该宏包。
  \end{tikzlibrary}

In this section, by \emph{graph} we refer to a set of nodes together with some
edges (sometimes also called arcs, in case they are directed) such as the
following:

在本节中,我们使用术语\emph{图}来指代一组节点以及它们之间的一些边(有时也称为弧,如果它们是有向的),例如以下示例:

%
\begin{codeexample}[preamble={\usetikzlibrary{graphs}}]
\tikz \graph { a -> {b, c} -> d };
\end{codeexample}

\begin{codeexample}[preamble={\usetikzlibrary{graphs.standard}}]
\tikz \graph {
  subgraph I_nm [V={a, b, c}, W={1,...,4}];

  a -> { 1, 2, 3 };
  b -> { 1, 4 };
  c -> { 2 [>green!75!black], 3, 4 [>red]}
};
\end{codeexample}

\begin{codeexample}[preamble={\usetikzlibrary{graphs}}]
\tikz
  \graph [nodes={draw, circle}, clockwise, radius=.5cm, empty nodes, n=5] {
    subgraph I_n [name=inner] --[complete bipartite]
    subgraph I_n [name=outer]
  };
\end{codeexample}

\begin{codeexample}[
    preamble={\usetikzlibrary{graphs}},
    pre={\definecolor{graphicbackground}{rgb}{0.96,0.96,0.8}},
]
\tikz
  \graph [nodes={draw, circle}, clockwise, radius=.75cm, empty nodes, n=8] {
    subgraph C_n [name=inner] <->[shorten <=1pt, shorten >=1pt]
    subgraph C_n [name=outer]
  };
\end{codeexample}

\begin{codeexample}[width=6.6cm,preamble={\usetikzlibrary{graphs}}]
\tikz [>={To[sep]}, rotate=90, xscale=-1,
       mark/.style={fill=black!50}, mark/.default=]
  \graph [trie, simple,
          nodes={circle,draw},
          edges={nodes={
              inner sep=1pt, anchor=mid,
              fill=graphicbackground}}, % yellowish background
          put node text on incoming edges]
    {
      root[mark] -> {
        a -> n -> {
          g [mark],
          f -> a -> n -> g [mark]
        },
        f -> a -> n -> g [mark],
        g[mark],
        n -> {
          g[mark],
          f -> a -> n -> g[mark]
        }
      },
      { [edges=red] % highlight one path
        root -> f -> a -> n
      }
    };
\end{codeexample}

The nodes of a graph are normal \tikzname\ nodes, the edges are normal lines
drawn between nodes. There is nothing in the |graphs| library that you cannot
do using the normal |\node| and the |edge| commands. Rather, its purpose is to
offer a concise and powerful way of \emph{specifying} which nodes are present
and how they are connected. The |graphs| library only offers simple methods for
specifying \emph{where} the nodes should be shown, its main strength is in
specifying which nodes and edges are present in principle. The problem of
finding ``good positions on the canvas'' for the nodes of a graph is left to
\emph{graph drawing algorithms}, which are covered in Part~\ref{part-gd} of
this manual and which are not part of the |graphs| library; indeed, these
algorithms can be used also with graphs specified using |node| and |edge|
commands.

图的节点是普通的 \tikzname\ 节点,边是在节点之间绘制的普通线段。使用 |graphs| 库可以使用普通的 |\node| 和 |edge| 命令完成的所有操作。它的目的是提供一种简明而强大的方式来\emph{指定}存在哪些节点以及它们如何连接。|graphs| 库仅提供了简单的方法来指定节点的\emph{位置},其主要优势在于指定了原则上存在哪些节点和边。将图中节点的“良好位置”问题留给\emph{图绘制算法}处理,这在本手册的第~\ref{part-gd}~部分中介绍,并且不属于 |graphs| 库的一部分;实际上,使用 |node| 和 |edge| 命令指定的图也可以使用这些算法。


\ifluatex
As an example, consider the above drawing of a trie, which is drawn without
using the graph drawing libraries. Its layout can be somewhat improved by
loading the |layered| graph drawing library, saying |\tikz[layered layout,...|,
and then using Lua\TeX, resulting in the following drawing of the same graph:

例如,考虑上面绘制的一棵字典树,它是不使用图绘制库绘制的。通过加载 |layered| 图绘制库,例如 |\tikz[layered layout,...]|,然后使用 Lua\TeX,可以稍微改善其布局,从而得到相同图形的以下绘制结果:
\medskip

\tikz [layered layout, >={To[sep]}, rotate=90, xscale=-1,
       mark/.style={fill=black!50}, mark/.default=]
  \graph [trie, simple, sibling distance=8mm,
          nodes={circle,draw},
          edges={nodes={
              inner sep=1pt, anchor=mid, fill=white}},
          put node text on incoming edges]
    {
      root[mark] -> {
        a -> n -> {
          g [mark],
          f -> a -> n -> g [mark]
        },
        f -> a -> n -> g [mark],
        g[mark],
        n -> {
          g[mark],
          f -> a -> n -> g[mark]
        }
      },
      { [edges=red] % highlight one path
        root -> f -> a -> n
      }
    };
\medskip
\fi

The |graphs| library uses a syntax that is quite different from the normal
\tikzname\ syntax for specifying nodes. The reason for this is that for many
medium-sized graphs it can become quite cumbersome to specify all the nodes
using |\node| repeatedly and then using a great number of |edge| command;
possibly with complicated |\foreach| statements. Instead, the syntax of the
|graphs| library is loosely inspired by the \textsc{dot} format, which is quite
useful for specifying medium-sized graphs, with some extensions on top.

|graphs|库使用的语法与普通的\tikzname\ 节点指定语法相当不同。之所以如此,是因为对于许多中等规模的图形来说,重复使用|\node|并使用大量的|edge|命令可能会变得非常麻烦;可能还会涉及复杂的|\foreach|语句。相反,|graphs|库的语法在很大程度上受到了\textsc{dot}格式的启发,该格式非常适用于指定中等规模的图形,并在此基础上进行了一些扩展。


\subsection{Concepts\\概念}

The present section aims at giving a quick overview of the main concepts behind
the |graph| command. The exact syntax is explained in more detail in later
sections.

本节旨在快速概述|graph|命令背后的主要概念。详细的语法将在后面的章节中进行解释。


\subsubsection{Concept: Node Chains\\概念:节点链}

The basic way of specifying a graph is to write down a \emph{node chain} as in
the following example:

指定图形的基本方法是编写一个\emph{节点链},如下例所示:
\begin{codeexample}[preamble={\usetikzlibrary{graphs}}]
\tikz [every node/.style = draw]
  \graph { foo -> bar -> blub };
\end{codeexample}

As can be seen, the text |foo -> bar -> blub| creates three nodes, one with
the text |foo|, one with |bar| and one with the text |blub|. These nodes are
connected by arrows, which are caused by the |->| between the node texts. Such
a sequence of node texts and arrows between them is called a \emph{chain} in
the following.

正如可以看到的那样,文本|foo -> bar -> blub|创建了三个节点,一个带有文本|foo|,一个带有文本|bar|,一个带有文本|blub|。这些节点通过箭头连接在一起,箭头是由节点文本之间的|->|引起的。在接下来的内容中,将把这样一系列节点文本和它们之间的箭头称为\emph{链}。

Inside a graph there can be more than one chain:

在一个图形中可以有多个链:
%
\begin{codeexample}[preamble={\usetikzlibrary{graphs}}]
\tikz \graph {
  a -> b -> c;
  d -> e -> f;
  g -> f;
};
\end{codeexample}

Multiple chains are separated by a semicolon or a comma (both have exactly the
same effect). As the example shows, when a node text is seen for the second
time, instead of creating a new node, a connection is created to the already
existing node.

多个链之间用分号或逗号分隔(两者效果完全相同)。正如示例所示,当第二次看到一个节点文本时,不会创建一个新节点,而是创建到已经存在的节点的连接。

When a node like |f| is created, both the node name and the node text are
identical by default. This is not always desirable and can be changed by using
the |as| key or by providing another text after a slash:

当创建一个像|f|这样的节点时,默认情况下节点名称和节点文本是相同的。这并不总是理想的,可以使用|as|关键字或在斜杠后面提供其他文本来更改:
%
\begin{codeexample}[preamble={\usetikzlibrary{graphs}}]
\tikz \graph {
  x1/$x_1$ -> x2 [as=$x_2$, red] -> x34/{$x_3,x_4$};
  x1 -> [bend left] x34;
};
\end{codeexample}

When you wish to use a node name that contains special symbols like commas or
dashes, you must surround the node name by quotes. This allows you to use quite
arbitrary text as a ``node name'':

当您希望使用包含逗号或破折号等特殊符号的节点名称时,必须用引号括住节点名称。这样您就可以使用任意文本作为“节点名称”:
%
\begin{codeexample}[preamble={\usetikzlibrary{graphs}}]
\tikz \graph {
  "$x_1$" -> "$x_2$"[red] -> "$x_3,x_4$";
  "$x_1$" ->[bend left] "$x_3,x_4$";
};
\end{codeexample}


\subsubsection{Concept: Chain Groups\\概念:链组}

Multiple chains that are separated by a semicolon or a comma and that are
surrounded by curly braces form what will be called a \emph{chain group} or
just a \emph{group}. A group in itself has no special effect. However, things
get interesting when you write down a node or even a whole group and connect it
to another group. In this case, the ``exit points'' of the first node or group
get connected to the ``entry points'' of the second node or group:

由大括号括起来的以分号或逗号分隔的多个链组成了所谓的\emph{链组}或简称\emph{组}。组本身没有特殊效果。然而,当您编写一个节点甚至整个组并将其连接到另一个组时,事情变得有趣起来。在这种情况下,第一个节点或组的“出口点”与第二个节点或组的“入口点”相连接:

%
\begin{codeexample}[preamble={\usetikzlibrary{graphs}}]
\tikz \graph {
  a -> {
    b -> c,
    d -> e
  } -> f
};
\end{codeexample}

Chain groups make it easy to create tree structures:

链组使创建树状结构变得容易:

%
\begin{codeexample}[width=10cm,preamble={\usetikzlibrary{graphs}}]
\tikz
  \graph [grow down,
          branch right=2.5cm] {
  root -> {
    child 1,
    child 2 -> {
      grand child 1,
      grand child 2
    },
    child 3 -> {
      grand child 3
    }
  }
};
\end{codeexample}

As can be seen, the placement is not particularly nice by default, use the
algorithms from the graph drawing libraries to get a better layout. For
instance, adding |tree layout| to the above code (and
|\usetikzlibrary{graphdrawing}| as well as |\usegdlibrary{trees}| to the
preamble) results in the following somewhat more pleasing rendering:

正如可以看到的那样,默认情况下布局不是特别好,可以使用图形绘制库中的算法来获得更好的布局。例如,将|tree layout|添加到上述代码中(以及将|\usetikzlibrary{graphdrawing}|以及|\usegdlibrary{trees}|添加到导言部分),将得到以下稍微更加令人愉悦的渲染:
%
\ifluatex
\medskip

\tikz \graph [grow down, branch right=2.5cm, tree layout] {
  root -> {
    child 1,
    child 2 -> {
      grand child 1,
      grand child 2
    },
    child 3 -> {
      grand child 3
    }
  }
};
\else
    (You need to use Lua\TeX\ to typeset this graphic.)
\fi


\subsubsection{Concept: Edge Labels and Styles\\概念:边标签和样式}

When connectors like |->| or |--| are used to connect nodes or whole chain
groups, one or more edges will typically be created. These edges can be styled
easily by providing options in square brackets directly after these connectors:

当使用 |->| 或 |--| 等连接器连接节点或整个链组时,通常会创建一个或多个边。通过在这些连接器后面的方括号中提供选项,可以轻松地为这些边添加样式:

%
\begin{codeexample}[preamble={\usetikzlibrary{graphs}}]
\tikz \graph {
  a ->[red] b --[thick] {c, d};
};
\end{codeexample}

Using the quotes syntax, see Section~\ref{section-label-quotes}, you can even
add labels to the edges easily by putting the labels in quotes:

使用引号语法,参见第~\ref{section-label-quotes} 节,您甚至可以通过将标签放在引号中轻松地为边添加标签:

%
\begin{codeexample}[preamble={\usetikzlibrary{graphs,quotes}}]
\tikz \graph {
  a ->[red, "foo"] b --[thick, "bar"] {c, d};
};
\end{codeexample}

For the first edge, the effect is as desired, however between |b| and the group
|{c,d}| two edges are inserted and the options |thick| and the label option
|"bar"| is applied to both of them. While this is the correct and consistent
behavior, we typically might wish to specify different labels for the edge
going from |b| to |c| and the edge going from |b| to |d|. To achieve this
effect, we can no longer specify the label as part of the options of |--|.
Rather, we must pass the desired label to the nodes |c| and |d|, but we must
somehow also indicate that these options actually ``belong'' to the edge
``leading'' to nodes. This is achieved by preceding the options with a
greater-than sign:

对于第一条边,效果如期望的那样,然而在 |b| 和组 |{c,d}| 之间插入了两条边,并且选项 |thick| 和标签选项 |"bar"| 都被应用于这两条边。虽然这是正确和一致的行为,但通常我们可能希望为从 |b| 到 |c| 的边和从 |b| 到 |d| 的边指定不同的标签。为了实现这种效果,我们不能再将标签指定为 |--| 选项的一部分。相反,我们必须将所需的标签传递给节点 |c| 和 |d|,但我们必须以某种方式表明这些选项实际上``属于''通往节点的边。这是通过在选项前加上大于号来实现的:

%
\begin{codeexample}[preamble={\usetikzlibrary{graphs,quotes}}]
\tikz \graph {
  a -> b -- {c [> "foo"], d [> "bar"']};
};
\end{codeexample}

Symmetrically, preceding the options by |<| causes the options and labels to
apply to the ``outgoing'' edges of the node:

对称地,通过在选项前加上 |<|,使选项和标签适用于节点的``出边'':

%
\begin{codeexample}[preamble={\usetikzlibrary{graphs,quotes}}]
\tikz \graph {
  a [< red] -> b -- {c [> blue], d [> "bar"']};
};
\end{codeexample}

This syntax allows you to easily create trees with special edge labels as in
the following example of a treap:

使用这种语法,您可以轻松创建具有特殊边标签的树,例如下面的 treap 示例:

%
\begin{codeexample}[preamble={\usetikzlibrary{graphs,quotes}}]
\tikz
  \graph [edge quotes={fill=white,inner sep=1pt},
          grow down, branch right, nodes={circle,draw}] {
    "" -> h [>"9"] -> {
      c [>"4"] -> {
        a [>"2"],
        e [>"0"]
      },
      j [>"7"]
    }
  };
\end{codeexample}


\subsubsection{Concept: Node Sets\\概念:节点集合}

When you write down some node text inside a |graph| command, a new node is
created by default unless this node has already been created inside the same
|graph| command. In particular, if a node has already been declared outside of
the current |graph| command, a new node of the same name gets created.

当您在 |graph| 命令中写入一些节点文本时,默认情况下会创建一个新节点,除非该节点在同一个 |graph| 命令中已经被创建过。特别是,如果一个节点在当前 |graph| 命令外部已经声明过,那么将创建一个同名的新节点。

This is not always the desired behavior. Often, you may wish to make nodes part
of a graph than have already been defined prior to the use of the |graph|
command. For this, simply surround a node name by parentheses. This will cause
a reference to be created to an already existing node:

这并不总是期望的行为。通常,您可能希望将节点作为图的一部分,而这些节点在使用 |graph| 命令之前已经定义。为此,只需将节点名称括在括号中。这将导致创建对已存在节点的引用:

%
\begin{codeexample}[preamble={\usetikzlibrary{graphs}}]
\tikz {
  \node (a) at (0,0) {A};
  \node (b) at (1,0) {B};
  \node (c) at (2,0) {C};

  \graph { (a) -> (b) -> (c) };
}
\end{codeexample}

You can even go a step further: A whole collection of nodes can all be flagged
to belong to a \emph{node set} by adding the option |set=|\meta{node set name}.
Then, inside a |graph| command, you can collectively refer to these nodes by
surrounding the node set name in parentheses:

您甚至可以更进一步:一整组节点可以通过添加选项 |set=|\meta{node set name} 来标记为属于一个\emph{节点集合}。然后,在 |graph| 命令中,您可以通过将节点集合名称括在括号中来集体引用这些节点:
%
\begin{codeexample}[preamble={\usetikzlibrary{graphs,shapes.geometric}}]
\tikz [new set=my nodes] {
  \node [set=my nodes, circle,    draw] at (1,1)   {A};
  \node [set=my nodes, rectangle, draw] at (1.5,0) {B};
  \node [set=my nodes, diamond,   draw] at (1,-1)  {C};
  \node (d)           [star,      draw] at (3,0)   {D};

  \graph { X -> (my nodes) -> (d) };
}
\end{codeexample}


\subsubsection{Concept: Graph Macros\\概念:图宏}

Often, a graph will consist -- at least in parts -- of standard parts. For
instance, a graph might contain a cycle of certain size or a path or a clique.
To facilitate specifying such graphs, you can define a \emph{graph macro}. Once
a graph macro has been defined, you can use the name of the graph to make a
copy of the graph part of the graph currently being specified:

通常,一个图将由标准部分(至少部分如此)组成。例如,一个图可能包含某个大小的循环、路径或团。为了便于指定这样的图形,可以定义一个\emph{图宏}。一旦定义了图宏,就可以使用图的名称来复制当前正在指定的图的一部分:
%
\begin{codeexample}[preamble={\usetikzlibrary{graphs.standard}}]
\tikz \graph { subgraph K_n [n=6, clockwise] };
\end{codeexample}

\begin{codeexample}[preamble={\usetikzlibrary{graphs.standard}}]
\tikz \graph { subgraph C_n [n=5, clockwise] -> mid };
\end{codeexample}

The library |graphs.standard| defines a number of such graphs, including the
complete clique $K_n$ on $n$ nodes, the complete bipartite graph $K_{n,m}$ with
shores sized $n$ and $m$, the cycle $C_n$ on $n$ nodes, the path $P_n$ on $n$
nodes, and the independent set $I_n$ on $n$ nodes.

库 |graphs.standard| 定义了许多这样的图形,包括完全团 $K_n$($n$ 个节点)、具有大小为 $n$ 和 $m$ 的两个分区的完全二分图 $K_{n,m}$、具有 $n$ 个节点的循环 $C_n$、具有 $n$ 个节点的路径 $P_n$,以及具有 $n$ 个节点的独立集 $I_n$。

\subsubsection{Concept: Graph Expressions and Color Classes\\概念:图表达式和颜色类}

When a graph is being constructed using the |graph| command, it is constructed
recursively by uniting smaller graphs to larger graphs. During this recursive
union process the nodes of the graph get implicitly \emph{colored}
(conceptually) and you can also explicitly assign colors to individual nodes
and even change the colors as the graph is being specified. All nodes having
the same color form what is called a \emph{color class}.

当使用 |graph| 命令构建图时,它是通过将较小的图合并为较大的图递归地构建的。在这个递归合并过程中,图的节点被隐式地\emph{着色}(在概念上),您还可以显式地为单个节点分配颜色,甚至在指定图时更改颜色。具有相同颜色的所有节点形成所谓的\emph{颜色类}。

The power of color class is that special \emph{connector operators} allow you
to add edges between nodes having certain colors. For instance, saying
|clique=red| at the beginning of a group will cause all nodes that have been
flagged as being (conceptually) ``red'' to be connected as a clique. Similarly,
saying |complete bipartite={red}{green}| will cause edges to be added between
all red and all green nodes. More advanced connectors, like the |butterfly|
connector, allow you to add edges between color classes in a fancy manner.

颜色类的强大之处在于特殊的\emph{连接符操作符}允许您在具有特定颜色的节点之间添加边。例如,在一个组的开头说|clique=red|将导致所有被标记为(概念上)“红色”的节点作为一个团连接起来。类似地,说|complete bipartite={red}{green}|将在所有红色节点和所有绿色节点之间添加边。更高级的连接符,如|butterfly|连接符,可以以炫酷的方式在颜色类之间添加边。

%
\begin{codeexample}[preamble={\usetikzlibrary{graphs}}]
\tikz [x=8mm, y=6mm, circle]
  \graph [nodes={fill=blue!70}, empty nodes, n=8] {
    subgraph I_n [name=A] --[butterfly={level=4}]
    subgraph I_n [name=B] --[butterfly={level=2}]
    subgraph I_n [name=C] --[butterfly]
    subgraph I_n [name=D] --
    subgraph I_n [name=E]
  };
\end{codeexample}


\subsection{Syntax of the Graph Path Command\\图路径命令的语法}

\subsubsection{The Graph Command\\图命令}

In order to construct a graph, you should use the |graph| path command, which
can be used anywhere on a path at any place where you could also use a command
like, say, |plot| or |--|.

为了构建一个图,您应该使用|graph|路径命令,它可以在路径上的任何位置使用,就像您可以使用|plot|或|--|命令一样。

\begin{command}{\graph}
    Inside a |{tikzpicture}| this is an abbreviation for |\path graph|.

    在|{tikzpicture}|中,这是|\path graph|的缩写。
  \end{command}

\begin{pathoperation}{graph}{\opt{\oarg{options}}\meta{group specification}}
    When this command is encountered on a path, the construction of the current
    path is suspended (similarly to an |edge| command or a |node| command). In
    a local scope, the \meta{options} are first executed with the key path
    |/tikz/graphs| using the following command:
    
    当在路径上遇到此命令时,当前路径的构造将被挂起(类似于|edge|命令或|node|命令)。在一个局部作用域中,首先使用以下命令以键路径|/tikz/graphs|执行\meta{options}:
%
    \begin{command}{\tikzgraphsset\marg{options}}
        Executes the \meta{options} with the path prefix |/tikz/graphs|.

        使用路径前缀|/tikz/graphs|执行\meta{options}。


    \end{command}
    %
    Apart from the keys explained in the following, further permissible keys
    will be listed during the course of the rest of this section.

    除了以下解释的键外,其他可接受的键将在本节其余部分中列出。


    \begin{stylekey}{/tikz/graphs/every graph}
        This style is executed at the beginning of every |graph| path command
        prior to the \meta{options}.

        此样式在每个|graph|路径命令之前执行的一开始。

    \end{stylekey}

    Once the scope has been set up and once the \meta{options} have been
    executed, a parser starts to parse the \meta{group specification}. The
    exact syntax of such a group specification in explained in detail in
    Section~\ref{section-library-graphs-group-spec}. Basically, a group
    specification is a list of chain specifications, separated by commas or
    semicolons.

    一旦设置好作用域并执行了\meta{options},解析器将开始解析\meta{group specification}。这种组规范的确切语法在第\ref{section-library-graphs-group-spec}节中详细解释。基本上,组规范是由逗号或分号分隔的一系列链规范列表。

    Depending on the content of the \meta{group specification}, two things will
    happen:

    根据\meta{group specification}的内容,将发生以下两种情况:


    %
    \begin{enumerate}
        \item A number of new nodes may be created. These will be inserted into
            the picture in the same order as if they had been created using
            multiple |node| path commands at the place where the |graph| path
            command was used. In other words, all nodes created in a |graph|
            path command will be painted on top of any nodes created earlier in
            the path and behind any nodes created later in the path. Like
            normal nodes, the newly created nodes always lie on top of the path
            that is currently being created (which is often empty, for instance
            when the |\graph| command is used).

            可能会创建一些新节点。它们将按照与在使用|graph|路径命令的位置创建多个|node|路径命令时相同的顺序插入到图中。换句话说,在|graph|路径命令中创建的所有节点都将在路径中较早创建的节点之上,并在路径中较晚创建的节点之后。与普通节点一样,新创建的节点始终位于当前正在创建的路径的顶部(通常为空,例如当使用|\graph|命令时)。


        \item Edges between the nodes may be added. They are added in the same
            order as if the |edge| command had been used at the position where
            the |graph| command is being used.

            节点之间的边可能会被添加。它们的添加顺序与在使用|edge|命令的位置相同,即在使用|graph|命令的位置。
    \end{enumerate}

    Let us now have a look at some common keys that may be used inside the

    现在让我们看一下可能在\meta{options}中使用的一些常见键:


    \meta{options}:
    %
    \begin{key}{/tikz/graphs/nodes=\meta{options}}
        This option causes the \meta{options} to be applied to each newly
        created node inside the \meta{group specification}.
        
        此选项使得\meta{group specification}中新创建的每个节点应用\meta{options}。
%
\begin{codeexample}[preamble={\usetikzlibrary{graphs}}]
\tikz \graph [nodes=red] { a -> b -> c };
\end{codeexample}
        %
        Multiple uses of this key accumulate.

        多次使用此键会累积效果。
    \end{key}
    %
    \begin{key}{/tikz/graphs/edges=\meta{options}}
        This option causes the \meta{options} to be applied to each newly
        created edge inside the \meta{group specification}.
        
        此选项使得\meta{group specification}中新创建的每条边应用\meta{options}。

%
\begin{codeexample}[preamble={\usetikzlibrary{graphs}}]
\tikz \graph [edges={red,thick}] { a -> b -> c };
\end{codeexample}
        %
        Again, multiple uses of this key accumulate.

        同样,多次使用此键会累积效果。


    \end{key}
    %
    \begin{key}{/tikz/graphs/edge=\meta{options}}
        This is an alias for |edges|.

        这是|edges|的别名。
    \end{key}

    \begin{key}{/tikz/graphs/edge node=\meta{node specification}}
        This key specifies that the \meta{node specification} should be added
        to each newly created edge as an implicitly placed node.
        
        该键指定将 \meta{node specification} 作为隐式放置的节点添加到每个新创建的边上。

%
\begin{codeexample}[preamble={\usetikzlibrary{graphs}}]
\tikz \graph [edge node={node [red, near end] {X}}] { a -> b -> c };
\end{codeexample}
        %
        Again, multiple uses of this key accumulate.
        
        此键的多次使用会累积。
%
\begin{codeexample}[preamble={\usetikzlibrary{graphs}}]
\tikz \graph [edge node={node [near end] {X}},
              edge node={node [near start] {Y}}] { a -> b -> c };
\end{codeexample}
    \end{key}

    \begin{key}{/tikz/graphs/edge label=\meta{text}}
        This key is an abbreviation for |edge node=node[auto]{|\meta{text}|}|.
        The net effect is that the |text| is placed next to the newly created
        edges.
        
        该键是 |edge node=node[auto]{|\meta{text}|}| 的简写。其效果是将 |text| 放置在新创建的边旁边。

%
\begin{codeexample}[preamble={\usetikzlibrary{graphs}}]
\tikz \graph [edge label=x] { a -> b -> {c,d} };
\end{codeexample}
    \end{key}

    \begin{key}{/tikz/graphs/edge label'=\meta{text}}
        This key is an abbreviation for |edge node=node[auto,swap]{|\meta{text}|}|.
        
        该键是 |edge node=node[auto,swap]{|\meta{text}|}| 的简写。

%
\begin{codeexample}[preamble={\usetikzlibrary{graphs.standard}}]
\tikz \graph [edge label=out, edge label'=in]
  { subgraph C_n [clockwise, n=5] };
\end{codeexample}
    \end{key}
\end{pathoperation}


\subsubsection{Syntax of Group Specifications\\组规范的语法}
\label{section-library-graphs-group-spec}

A \meta{group specification} inside a |graph| path command has the following
syntax:

在 |graph| 路径命令中,\meta{group specification} 具有以下语法:
%
\begin{quote}
    |{|\opt{\oarg{options}}\meta{list of chain specifications}|}|
\end{quote}
%
The \meta{chain specifications} must contain chain specifications, whose syntax
is detailed in the next section, separated by either commas or semicolons; you
can freely mix them. It is permissible to use empty lines (which are mapped to
|\par| commands internally) to structure the chains visually, they are simply
ignored by the parser.

\meta{chain specifications} 必须包含链规范,其语法在下一节中详细介绍,用逗号或分号分隔;可以自由混合使用。可以使用空行(在内部映射为 |\par| 命令)在视觉上结构化链,解析器会忽略它们。

In the following example, the group specification consists of three chain
specifications, namely of |a -> b|, then |c| alone, and finally |d -> e -> f|:

在下面的示例中,组规范由三个链规范组成,即 |a -> b|,然后是单独的 |c|,最后是 |d -> e -> f|:

%
\begin{codeexample}[preamble={\usetikzlibrary{graphs}}]
\tikz \graph {
  a -> b,
  c;

  d -> e -> f
};
\end{codeexample}
%
The above has the same effect as the more compact group specification
|{a->b,c,d->e->f}|.

上述示例与更紧凑的组规范 |{a->b,c,d->e->f}| 具有相同的效果。

Commas are used to detect where chain specifications end. However, you will
often wish to use a comma also inside the options of a single node like in the
following example:

逗号用于检测链规范的结束。但是,您通常希望在单个节点的选项中也使用逗号,如下面的示例所示:

%
\begin{codeexample}[preamble={\usetikzlibrary{graphs}}]
\tikz \graph {
  a [red, draw] -> b [blue, draw],
  c [brown, draw, circle]
};
\end{codeexample}

Note that the above example works as expected: The first comma inside the
option list of |a| is \emph{not} interpreted as the end of the chain
specification ``|a [red|''. Rather, commas inside square brackets are
``protected'' against being interpreted as separators of group specifications.

请注意,上述示例按预期工作:选项列表中 |a| 的选项列表内的第一个逗号\emph{不会}被解释为链规范的结尾 ``|a [red|''。相反,方括号内的逗号在被解释为组规范的分隔符时会被“保护”。

The \meta{options} that can be given at the beginning of a group specification
are local to the group. They are executed with the path prefix |/tikz/graphs|.
Note that for the outermost group specification of a graph it makes no
difference whether the options are passed to the |graph| command or whether
they are given at the beginning of this group. However, for groups nested
inside other groups, it does make a difference:

在组规范的开头给出的 \meta{options} 局部应用于该组。它们会在路径前缀 |/tikz/graphs| 下执行。请注意,对于图的最外层组规范,无论选项是传递给 |graph| 命令还是在此组的开头给出,都没有区别。但是,对于嵌套在其他组中的组,情况会有所不同:

%
\begin{codeexample}[preamble={\usetikzlibrary{graphs}}]
\tikz \graph {
  a -> { [nodes=red] % the option is local to these nodes:
    b, c
  } ->
  d
};
\end{codeexample}

\medskip
\textbf{Using foreach.}
There is special support for the |\foreach| statement inside groups: You may
use the statement inside a group specification at any place where a \meta{chain
specification} would normally go. In this case, the |\foreach| statement is
executed and for each iteration the content of the statement's body is treated
and parsed as a new chain specification.

\textbf{使用 foreach。}
在组内部,对 |\foreach| 语句有特殊支持:可以在组规范中的任何可以放置 \meta{chain specification} 的地方使用该语句。在这种情况下,|\foreach| 语句会被执行,并且对于每次迭代,语句体的内容会被视为新的链规范并进行解析。%
\begin{codeexample}[preamble={\usetikzlibrary{graphs}}]
\tikz \graph [math nodes, branch down=5mm] {
  a -> {
    \foreach \i in {1,2,3} {
      a_\i -> { x_\i, y_\i }
    },
    b
  }
};
\end{codeexample}

\medskip
\textbf{Using macros.}
In some cases you may wish to use macros and \TeX\ code to compute which nodes
and edges are present in a group. You cannot use macros in the normal way
inside a graph specification since the parser does not expand macros as it
scans for the start and end of groups and node names. Rather, only after
commas, semicolons, and hyphens have already been detected and only after all
other parsing decisions have been made will macros be expanded. At this point,
when a macro expands to, say |a,b|, this will not result in two nodes to be
created since the parsing is already done. For these reasons, a special key is
needed to make it possible to ``compute'' which nodes should be present in a
group.

\textbf{使用宏。}
在某些情况下,您可能希望使用宏和 \TeX\ 代码来计算组中存在的节点和边。您不能在图形规范中以正常的方式使用宏,因为解析器在扫描组和节点名称的开始和结束时不会展开宏。相反,在检测到逗号、分号和连字符之后,并且在做出所有其他解析决策之后,才会展开宏。此时,当宏展开为,例如 |a,b| 时,这不会导致创建两个节点,因为解析已经完成。出于这些原因,需要一种特殊的键来使“计算”哪些节点应该存在于组中成为可能。

\begin{key}{/tikz/graph/parse=\meta{text}}
    This key can only be used inside the \meta{options} of a \meta{group
    specification}. Its effect is that the \meta{text} is inserted at the
    beginning of the current group as if you had entered it there. Naturally,
    it makes little sense to just write down some static \meta{text} since you
    could just as well directly place it at the beginning of the group. The
    real power of this command stems from the fact that the keys mechanism
    allows you to say, for instance, |parse/.expand once| to insert the text
    stored in some macro into the group.
    
    该键只能在 \meta{group specification} 的 \meta{options} 内部使用。它的效果是将 \meta{text} 插入到当前组的开头,就像您在那里输入了它一样。显然,仅仅写下一些静态的 \meta{text} 是没有什么意义的,因为您可以直接将其放置在组的开头。这个命令的真正威力来自于键机制,允许您使用例如 |parse/.expand once| 来将存储在某个宏中的文本插入到组中。

%
\begin{codeexample}[preamble={\usetikzlibrary{graphs}}]
\def\mychain{ a -> b -> c; }
\tikz \graph { [parse/.expand once=\mychain] d -> e };
\end{codeexample}
    %
    In the following, more fancy example we use a loop to create a chain of
    dynamic length.
    
    在以下更复杂的示例中,我们使用循环来创建一个动态长度的链表。

%
\begin{codeexample}[preamble={\usetikzlibrary{graphs}}]
\def\mychain#1{
  \def\mytext{1}
  \foreach \i in {2,...,#1} {
    \xdef\mytext{\mytext -> \i}
  }
}
\tikzgraphsset{my chain/.style={
    /utils/exec=\mychain{#1},
    parse/.expand once=\mytext}
}
\tikz \graph { [my chain=4] };
\end{codeexample}
    %
    Multiple uses of this key accumulate, that is, all the \text{text}s given
    in the different uses is inserted in the order it is given.

    此键的多次使用会累积,也就是说,不同使用中给定的所有 \text{text} 将按照给定的顺序插入。


\end{key}


\subsubsection{Syntax of Chain Specifications\\链表规范的语法}

A \meta{chain specification} has the following syntax: It consists of a
sequence of \meta{node specifications}, where subsequent node specifications
are separated by \meta{edge specifications}. Node specifications, which
typically consist of some text, are discussed in the next section in more
detail. They normally represent a single node that is either newly created or
exists already, but they may also specify a whole set of nodes.

\meta{chain specification} 的语法如下:它由一系列 \meta{node specification} 组成,后续的节点规范由 \meta{edge specification} 分隔。节点规范通常由一些文本组成,下一节将更详细地讨论它们。它们通常表示一个已经存在或者新创建的单个节点,但也可以指定整个节点集。

An \meta{edge specification} specifies \emph{which} of the node(s) to the left
of the edge specification should be connected to which node(s) to the right of
it and it also specifies in which direction the connections go. In the
following, we only discuss how the direction is chosen, the powerful mechanism
behind choosing which nodes should be connect is detailed in
Section~\ref{section-library-graphs-color-classes}.

\meta{edge specification} 指定了边规范左侧的节点与右侧节点之间的连接方式,还指定了连接的方向。在下面,我们只讨论如何选择方向,选择应该连接哪些节点的强大机制在第~\ref{section-library-graphs-color-classes} 节中详细介绍。

The syntax of an edge specification is always one of the following five
possibilities:

边规范的语法始终是以下五种可能性之一:

%
\begin{quote}
    |->| \opt{\oarg{options}}\\
    |--| \opt{\oarg{options}}\\
    |<-| \opt{\oarg{options}}\\
    |<->| \opt{\oarg{options}}\\
    |-!-| \opt{\oarg{options}}
\end{quote}

The first four correspond to a directed edge, an undirected edge, a
``backward'' directed edge, and a bidirected edge, respectively. The fifth edge
specification means that there should be no edge (this specification can be
used together with the |simple| option to remove edges that have previously
been added, see Section~\ref{section-library-graphs-simple}).

前四个分别对应有向边、无向边、``反向''有向边和双向边。第五个边规范表示不应该有边(此规范可以与 |simple| 选项一起使用,以删除先前添加的边,参见第~\ref{section-library-graphs-simple}节)。

Suppose the nodes \meta{left nodes} are to the left of the \meta{edge
specification} and \meta{right nodes} are to the right and suppose we have
written |->| between them. Then the following happens:

假设节点 \meta{left nodes} 在 \meta{edge specification} 的左侧,而 \meta{right nodes} 在右侧,并假设我们在它们之间写了 |->|。然后会发生以下情况:

%
\begin{enumerate}
    \item The \meta{options} are executed (inside a local scope) with the path
        |/tikz/graphs|.  These options may setup the connector algorithm (see
        below) and may also use keys like |edge| or |edge label| to specify how
        the edge should look like. As a convenience, whenever an unknown key is
        encountered for the path |/tikz/graphs|, the key is passed to the
        |edge| key. This means that you can directly use options like |thick|
        or |red| inside the \meta{options} and they will apply to the edge as
        expected.

        \meta{options}(在本地作用域内)被执行,其位于路径 |/tikz/graphs| 内。这些选项可以设置连接器算法(参见下文)并且还可以使用诸如 |edge| 或 |edge label| 的键来指定边的样式。为了方便起见,当遇到路径 |/tikz/graphs| 的未知键时,该键被传递给 |edge| 键。这意味着您可以直接在 \meta{options} 中使用诸如 |thick| 或 |red| 的选项,并且它们会按预期应用于边上。


    \item The chosen connector algorithm, see
        Section~\ref{section-library-graphs-color-classes}, is used to compute
        from which of the \meta{left nodes} an edge should lead to which of the
        \meta{right nodes}. Suppose that $(l_1,r_1)$, \dots, $(l_n,r_n)$ is the
        list of node pairs that result (so there should be an edge between
        $l_1$ and $r_1$ and another edge between $l_2$ and $r_2$ and so on).

        所选择的连接器算法(参见第~\ref{section-library-graphs-color-classes}节)用于计算从哪个 \meta{left nodes} 到哪个 \meta{right nodes} 应该有边。假设 $(l_1,r_1)$、\dots、$(l_n,r_n)$ 是结果的节点对列表(因此应该在 $l_1$ 和 $r_1$ 之间有一条边,在 $l_2$ 和 $r_2$ 之间有另一条边,依此类推)。


    \item For each pair $(l_i,r_i)$ an edge is created. This is done by calling
        the following key (for the edge specification |->|, other keys are
        executed for the other kinds of specifications):
        
        
        对于每对 $(l_i,r_i)$,都会创建一条边。这通过调用以下键来完成(对于边规范 |->|,对于其他类型的规范,会执行其他键):

%
        \begin{key}{/tikz/graphs/new ->=\marg{left node}\marg{right node}\marg{edge options}\marg{edge nodes}}
            This key will be called for a |->| edge specification with the
            following four parameters:
            
            该键将使用以下四个参数来调用 |->| 边规范:

%
            \begin{enumerate}
                \item \meta{left node} is the name of the ``left'' node, that
                    is, the name of $l_i$.

                    \meta{left node} 是``左侧''节点的名称,即 $l_i$ 的名称。


                \item \meta{right node} is the name of the right node.

                \meta{right node} 是右侧节点的名称。


                \item \meta{edge options} are the accumulated options from all
                    calls of |/tikz/graph/edges| in groups that surround the
                    edge specification.

                    \meta{edge options} 是所有调用 |/tikz/graph/edges| 的组中累积的选项,这些组围绕边规范进行。


                \item \meta{edge nodes} is text like |node {A} node {B}| that
                    specifies some nodes that should be put as labels on the
                    edge using \tikzname's implicit positioning mechanism.

                    \meta{edge nodes} 是像 |node {A} node {B}| 这样的文本,它使用 \tikzname 的隐式定位机制指定应在边上作为标签放置的一些节点。
            \end{enumerate}
            %
            By default, the key executes the following code:
            
            默认情况下,键执行以下代码:

%
            \begin{quote}
                |\path [->,every new ->]|\\
                \hbox{}\quad|(|\meta{left node}|\tikzgraphleftanchor) edge [|%
                \meta{edge options}|]| \meta{edge nodes}||\\
                \hbox{}\quad|(|\meta{right node}|\tikzgraphrightanchor);|
            \end{quote}
            %
            You are welcome to change the code underlying the key.
            
            欢迎您更改该键的底层代码。

%
            \begin{stylekey}{/tikz/every new ->}
                This key gets executed by default for a |new ->|.

                默认情况下,该键会为 |new ->| 执行。

            \end{stylekey}
        \end{key}
        %
        \begin{key}{/tikz/graphs/left anchor=\meta{anchor}}
            This anchor is used for the node that is to the left of an edge
            specification. Setting this anchor to the empty string means that
            no special anchor is used (which is the default). The \meta{anchor}
            is stored in the macro |\tikzgraphleftanchor| with a leading dot.
            
            该锚点用于位于边规范左侧的节点。将该锚点设置为空字符串表示不使用特殊锚点(这是默认设置)。带有前导点的 \meta{anchor} 存储在宏 |\tikzgraphleftanchor| 中。

%
\begin{codeexample}[preamble={\usetikzlibrary{graphs}}]
\tikz \graph {
  {a,b,c} -> [complete bipartite] {e,f,g}
};
\end{codeexample}
            %
\begin{codeexample}[preamble={\usetikzlibrary{graphs}}]
\tikz \graph [left anchor=east, right anchor=west] {
  {a,b,c} -- [complete bipartite] {e,f,g}
};
\end{codeexample}
        \end{key}
        %
        \begin{key}{/tikz/graphs/right anchor=\meta{anchor}}
            Works like |left anchor|, only for |\tikzgraphrightanchor|.

            与 |left anchor| 类似,只是适用于 |\tikzgraphrightanchor|。
        \end{key}
        %
        For the other three kinds of edge specifications, the following keys
        will be called:

        对于其他三种类型的边规范,将调用以下键:

        %
        \begin{key}{/tikz/graphs/new --=\marg{left node}\marg{right node}\marg{edge options}\marg{edge nodes}}
            This key is called for |--| with the same parameters as above. The
            only difference in the definition is that in the |\path| command
            the |->| gets replaced by |-|.

            对于 |--|,与上述参数相同调用该键。定义中唯一的不同之处在于 |\path| 命令中的 |->| 被替换为 |-|。


            %
            \begin{stylekey}{/tikz/every new --}
            \end{stylekey}
        \end{key}
        %
        \begin{key}{/tikz/graphs/new <->=\marg{left node}\marg{right node}\marg{edge options}\marg{edge nodes}}
            Called for |<->| with the same parameters as above. The |->| is
            replaced by |<-|

            对于 |<->|,与上述参数相同调用该键。将 |->| 替换为 |<-|。


            %
            \begin{stylekey}{/tikz/every new <->}
            \end{stylekey}
        \end{key}
        %
        \begin{key}{/tikz/graphs/new <-=\marg{left node}\marg{right node}\marg{edge options}\marg{edge nodes}}
            Called for |<-| with the same parameters as above.%

            对于 |<-|,与上述参数相同调用该键。


            \footnote{%
                You might wonder why this key is needed: It seems more logical
                at first sight to just call |new edge directed| with swapped
                first parameters. However, a positioning algorithm might wish
                to take the fact into account that an edge is ``backward''
                rather than ``forward'' in order to improve the layout. Also,
                different arrow heads might be used.

                你可能想知道为什么需要这个键:乍一看,只需用交换的第一个参数调用 |new edge directed| 更合乎逻辑。然而,定位算法可能希望考虑到边是“反向”的事实,而不是“正向”的,以改善布局。此外,可能使用不同的箭头样式。

            }
            %
            \begin{stylekey}{/tikz/every new <-}
            \end{stylekey}
        \end{key}
        %
        \begin{key}{/tikz/graphs/new -\protect\exclamationmarktext-=\marg{left node}\marg{right node}\marg{edge options}\marg{edge nodes}}
            Called for |-!-| with the same parameters as above. Does nothing by
            default.

            使用与上面相同的参数调用 |-!-|。默认情况下不执行任何操作。
        \end{key}
\end{enumerate}

Here is an example that shows the default rendering of the different edge
specifications:

以下是显示不同边缘规范的默认渲染的示例:
%
\begin{codeexample}[preamble={\usetikzlibrary{graphs}}]
\tikz \graph [branch down=5mm] {
  a -> b;
  c -- d;
  e <- f;
  g <-> h;
  i -!- j;
};
\end{codeexample}


\subsubsection{Syntax of Node Specifications\\节点规范的语法}
\label{section-library-graphs-node-spec}

Node specifications are the basic building blocks of a graph specification.
There are three different possible kinds of node specifications, each of which
has a different syntax:

节点规范是图规范的基本构建块。有三种不同类型的节点规范,每种规范都有不同的语法:
%
\begin{description}
    \item[Direct Node Specification 直接节点规范]
        \ \\
        \opt{|"|}\meta{node name}\opt{|"|}\opt{|/|\opt{|"|}\meta{text}\opt{|"|}} \opt{\oarg{options}}\\
        (note that the quotation marks are optional and only needed when the
        \meta{node name} contains special symbols)

        (注意引号是可选的,只在\meta{节点名称}包含特殊符号时需要)
    \item[Reference Node Specification引用节点规范]
        \ \\
        |(|\meta{node name or node set name}|)|
    \item[Group Node Specification分组节点规范]
        \ \\
        \meta{group specification}
\end{description}

The rule for determining which of the possible kinds is meant is as follows: If
the node specification starts with an opening parenthesis, a reference node
specification is meant; if it starts with an opening curly brace, a group
specification is meant; and in all other cases a direct node specification is
meant.

确定所指的可能类型的规则如下:如果节点规范以开头的括号开始,则表示引用节点规范;如果以开头的花括号开始,则表示分组规范;在所有其他情况下,表示直接节点规范。

\medskip
\textbf{Direct Node Specifications.} If after reading the first symbol of a
node specification is has been detected to be \emph{direct}, \tikzname\ will
collect all text up to the next edge specification and store it as the
\meta{node name}; however, square brackets are used to indicate options and a
slash ends the \meta{node name} and start a special \meta{text} that is used as
a rendering text instead of the original \meta{node name}.

\textbf{直接节点规范。}如果在读取节点规范的第一个符号后发现它是\emph{直接}的,\tikzname\ 将收集直到下一个边缘规范的所有文本,并将其存储为\meta{节点名称};然而,方括号用于指示选项,斜杠结束\meta{节点名称}并开始一个特殊的\meta{文本},该文本用作呈现文本,而不是原始的\meta{节点名称}。Due to the way the parsing works and due to the restrictions on node names,

most special characters are forbidding inside the \meta{node name}, including
commas, semicolons, hyphens, braces, dots, parentheses, slashes, dashes, and
more (but spaces, single underscores, and the hat character \emph{are}
allowed). To use special characters in the name of a node, you can optionally
surround the \meta{node name} and/or the \meta{text} by quotation marks. In
this case, you can use all of the special symbols once more. The details of
what happens, exactly, when the \meta{node name} is surrounded by quotation
marks is explained later; surrounding the \meta{text} by quotation marks has
essentially the same effect as surrounding it by curly braces.

由于解析工作的方式以及节点名称的限制,大多数特殊字符在\meta{节点名称}内是禁止的,包括逗号、分号、连字符、大括号、点、括号、斜杠、短划线等(但允许使用空格、单下划线和帽子字符)。要在节点名称中使用特殊字符,可以选择用引号引住\meta{节点名称}和/或\meta{文本}。在这种情况下,您可以再次使用所有特殊符号。稍后将解释\meta{节点名称}被引号引住时确切发生了什么;用引号引住\meta{文本}的效果基本上与用花括号引住它的效果相同。

Once the node name has been determined, it is checked whether the same node
name was already used inside the current graph. If this is the case, then we
say that the already existing node is \emph{referenced}; otherwise we say that
the node is \emph{fresh}.

一旦确定了节点名称,就会检查当前图形中是否已经使用了相同的节点名称。如果是这样,我们称已存在的节点为\emph{引用}节点;否则我们称该节点为\emph{新建}节点。


%
\begin{codeexample}[preamble={\usetikzlibrary{graphs}}]
\tikz \graph {
  a -> b; % both are fresh
  c -> a; % only c is fresh, a is referenced
};
\end{codeexample}

This behavior of deciding whether a node is fresh or referenced can, however,
be modified by using the following keys:

然而,可以通过使用以下关键字来修改节点是新建还是引用的行为:

%
\begin{key}{/tikz/graphs/use existing nodes=\opt{\meta{true or false}} (default true)}
    When this key is set to |true|, all nodes will be considered to the
    referenced, no node will be fresh. This option is useful if you have
    already created all the nodes of a graph prior to using the |graph| command
    and you now only wish to connect the nodes. It also implies that an error
    is raised if you reference a node which has not been defined previously.

    当将此关键字设置为|true|时,所有节点都将被视为引用节点,没有节点是新建的。如果在使用|graph|命令之前已经创建了图形的所有节点,并且现在只想连接这些节点,那么此选项非常有用。它还意味着如果引用了先前未定义的节点,将会引发错误。

  \end{key}

\begin{key}{/tikz/graphs/fresh nodes=\opt{\meta{true or false}} (default true)}
    When this key is set to |true|, all nodes will be considered to be fresh.
    This option is useful when you create for instance a tree with many
    identical nodes.

当将此关键字设置为|true|时,所有节点都将被视为新建节点。这个选项在创建具有许多相同节点的树时非常有用。


    When a node name is encountered that was already used previously, a new
    name is chosen is follows: An apostrophe (|'|) is appended repeatedly until
    a node name is found that has not yet been used:
    %

    当遇到已经先前使用过的节点名称时,将选择一个新名称,方法如下:连续附加撇号(|'|),直到找到尚未使用的节点名称为止:
\begin{codeexample}[preamble={\usetikzlibrary{graphs}}]
\tikz \graph [branch down=5mm] {
  { [fresh nodes]
    a -> {
      b -> {c, c},
      b -> {c, c},
      b -> {c, c},
    }
  },
  b' -- b''
};
\end{codeexample}
    %
\end{key}

\begin{key}{/tikz/graphs/number nodes=\opt{\meta{start number}} (default 1)}
    When this key is used in a scope, each encountered node name will get
    appended a new number, starting with \meta{start}. Typically, this ensures
    that all node names are different. Between the original node name and the
    appended number, the setting of the following will be inserted:
    
    当此关键字在作用域中使用时,每个遇到的节点名称都将附加一个新的数字,从\meta{start}开始。通常,这可以确保所有节点名称都不同。在原始节点名称和附加的数字之间,将插入以下设置:
    \begin{key}{/tikz/graphs/number nodes sep=\meta{text} (initially \normalfont space)}
    \end{key}
    %
\begin{codeexample}[preamble={\usetikzlibrary{graphs}}]
\tikz \graph [branch down=5mm] {
  { [number nodes]
    a -> {
      b -> {c, c},
      b -> {c, c},
      b -> {c, c},
    }
  },
  b 2 -- b 5
};
\end{codeexample}
    %
\end{key}

When a fresh node has been detected, a new node is created in the inside a
protecting scope. For this, the current placement strategy is asked to compute
a default position for the node, see
Section~\ref{section-library-graphs-placement} for details. Then, the command

当检测到新建节点时,在保护作用域内创建一个新节点。为此,当前的放置策略被要求计算节点的默认位置,详见第~\ref{section-library-graphs-placement}节。然后,调用以下命令:
\begin{quote}
    |\node (|\meta{full node name}|) [|\meta{node options}|] {|\meta{text}|};|
\end{quote}
%
is called. The different parameters are as follows:

各个参数的含义如下:
\begin{itemize}
    \item The \meta{full node name} is normally the \meta{node name} that has
        been determined as described before. However, there are two exceptions:

        \meta{full node name}通常是根据前面描述的确定的\meta{node name}。然而,有两个例外:

        First, if the \meta{node name} is empty (which happens when there is no
        \meta{node name} before the slash), then a fresh internal node name is
        created and used as \meta{full node name}. This name is guaranteed to
        be different from all node names used in this or any other graph. Thus,
        a direct node starting with a slash represents an anonymous fresh node.

        首先,如果\meta{node name}为空(即在斜杠之前没有\meta{node name}),那么将创建一个新的内部节点名称,并将其用作\meta{full node name}。此名称保证与此图或任何其他图中使用的所有节点名称不同。因此,以斜杠开头的直接节点表示一个匿名的新建节点。


        Second, you can use the following key to prefix the \meta{node name}
        inside the \meta{full node name}:

        其次,可以使用以下关键字在\meta{full node name}中添加前缀的\meta{node name}:


        \begin{key}{/tikz/graphs/name=\meta{text}}
            This key prepends the \meta{text}, followed by a separating symbol
            (a space by default), to all \meta{node name}s inside a \meta{full
            node name}. Repeated calls of this key accumulate, leading to
            ever-longer ``name paths'':
            
            此关键字将\meta{text}添加到\meta{full node name}中的所有\meta{node name}之前,后面跟着一个分隔符(默认为一个空格)。重复调用此关键字会累积,导致越来越长的“名称路径”:
\begin{codeexample}[preamble={\usetikzlibrary{graphs}}]
\begin{tikzpicture}
  \graph {
    { [name=first]  1, 2, 3} --
    { [name=second] 1, 2, 3}
  };
  \draw [red] (second 1) circle [radius=3mm];
\end{tikzpicture}
\end{codeexample}
            %
            Note that, indeed, in the above example six nodes are created even
            though the first and second set of nodes have the same \meta{node
            name}. The reason is that the full names of the six nodes are all
            different. Also note that only the \meta{node name} is used as the
            node text, not the full name. This can be changed as described
            later on.

            请注意,实际上,在上面的示例中创建了六个节点,即使第一组和第二组节点具有相同的\meta{node name}。原因是六个节点的完整名称都是不同的。还请注意,只有\meta{node name}被用作节点文本,而不是完整名称。稍后可以根据需要进行更改。

            This key can be used repeatedly, leading to ever longer node names.

            此关键字可以重复使用,从而得到越来越长的节点名称。
        \end{key}

        \begin{key}{/tikz/graphs/name separator=\meta{symbols} (initially \string\space)}
            Changes the symbol that is used to separate the \meta{text} from
            the \meta{node name}. The default is |\space|, resulting in a
            space.
            
            更改用于将\meta{text}与\meta{node name}分隔的符号。默认值为|\space|,即空格。
\begin{codeexample}[preamble={\usetikzlibrary{graphs}}]
\begin{tikzpicture}
  \graph [name separator=] { % no separator
    { [name=first]  1, 2, 3} --
    { [name=second] 1, 2, 3}
  };
  \draw [red] (second1) circle [radius=3mm];
\end{tikzpicture}
\end{codeexample}
            %
\begin{codeexample}[preamble={\usetikzlibrary{graphs}}]
\begin{tikzpicture}
  \graph [name separator=-] {
    { [name=first]  1, 2, 3} --
    { [name=second] 1, 2, 3}
  };
  \draw [red] (second-1) circle [radius=3mm];
\end{tikzpicture}
\end{codeexample}
        \end{key}
    \item The \meta{node options} are

    \meta{node options} 是:
        %
        \begin{enumerate}
            \item The options that have accumulated in calls to |nodes| from
                the surrounding scopes.

                来自周围作用域中对 |nodes| 的调用中累积的选项。
            \item The local \meta{options}.

            本地的 \meta{options}。
        \end{enumerate}
        %
        The options are executed with the path prefix |/tikz/graphs|, but any
        unknown key is executed with the prefix |/tikz|. This means, in
        essence, that some esoteric keys are more difficult to use inside the
        options and that any key with the prefix |/tikz/graphs| will take
        precedence over a key with the prefix |/tikz|.

        这些选项使用路径前缀 |/tikz/graphs| 执行,但任何未知的键都会使用前缀 |/tikz| 执行。这意味着,从本质上讲,在选项中使用某些奇特的键更加困难,并且具有前缀 |/tikz/graphs| 的键将优先于具有前缀 |/tikz| 的键。
    \item The \meta{text} that is passed to the |\node| command is computed as
        follows: First, you can use the following key to directly set the
        \meta{text}:
        
        传递给 |\node| 命令的 \meta{text} 计算如下:首先,您可以使用以下键直接设置 \meta{text}:

        \begin{key}{/tikz/graphs/as=\meta{text}}
            The \meta{text} is used as the text of the node. This allows you to
            provide a text for the node that differs arbitrarily from the name
            of the node.
            
            \meta{text} 用作节点的文本。这允许您为节点提供与节点名称任意不同的文本。
\begin{codeexample}[preamble={\usetikzlibrary{graphs}}]
\tikz \graph { a [as=$x$] -- b [as=$y_5$] -> c [red, as={a--b}] };
\end{codeexample}
            %
            This key always takes precedence over all of the mechanisms
            described below.

            此键始终优先于下面描述的所有机制。
        \end{key}
        %
        In case the |as| key is not used, a default text is chosen as follows:
        First, when a direct node specification contains a slash (or, for
        historical reasons, a double underscore), the text to the right of the
        slash (or double underscore) is stored in the macro
        |\tikzgraphnodetext|; if there is no slash, the \meta{node name} is
        stored in |\tikzgraphnodetext|, instead. Then, the current value of the
        following key is used as \meta{text}:
        
        如果未使用 |as| 键,则默认选择的文本如下:首先,当直接节点规范包含斜杠(或出于历史原因,双下划线)时,斜杠(或双下划线)右侧的文本存储在宏 |\tikzgraphnodetext| 中;如果没有斜杠,则将 \meta{node name} 存储在 |\tikzgraphnodetext| 中。然后,使用以下键的当前值作为 \meta{text}:
        \begin{key}{/tikz/graphs/typeset=\meta{code}}
            The macro or code stored in this key is used as the \meta{text} of
            the node. Inside the \meta{code}, the following macros are
            available:
            
            使用该键中存储的宏或代码作为节点的 \meta{text}。在 \meta{code} 中,可以使用以下宏:
            \begin{command}{\tikzgraphnodetext}
                This macro expands to the \meta{text} to the right of the
                double underscore or slash in a direct node specification or,
                if there is no slash, to the \meta{node name}.

                此宏展开为直接节点规范中双下划线或斜杠右侧的 \meta{text},如果没有斜杠,则展开为 \meta{node name}。
            \end{command}
            %
            \begin{command}{\tikzgraphnodename}
                This macro expands to the name of the current node without the
                path.

                此宏展开为当前节点的名称,不包括路径。
            \end{command}
            %
            \begin{command}{\tikzgraphnodepath}
                This macro expands to the current path of the node. These paths
                result from the use of the |name| key as described above.

                此宏展开为节点的当前路径。这些路径由上述使用 |name| 键而产生。
            \end{command}
            %
            \begin{command}{\tikzgraphnodefullname}
                This macro contains the concatenation of the above two.

                此宏包含上述两者的连接。
            \end{command}
        \end{key}
        %
        By default, the typesetter is just set to |\tikzgraphnodetext|, which
        means that the default text of a node is its name. However, it may be
        useful to change this: For instance, you might wish that the text of
        all graph nodes is, say, surrounded by parentheses:
        
        默认情况下,排版器仅设置为 |\tikzgraphnodetext|,这意味着节点的默认文本是其名称。然而,更改这一点可能会很有用:例如,您可能希望所有图形节点的文本用括号括起来:
\begin{codeexample}[preamble={\usetikzlibrary{graphs}}]
\tikz \graph [typeset=(\tikzgraphnodetext)]
  { a -> b -> c };
\end{codeexample}
        %
        A more advanced macro might take apart the node text and render it
        differently:
        
        更高级的宏可能会拆分节点文本并以不同方式呈现它:
\begin{codeexample}[preamble={\usetikzlibrary{graphs}}]
\def\mytypesetter{\expandafter\myparser\tikzgraphnodetext\relax}
\def\myparser#1 #2 #3\relax{%
  $#1_{#2,\dots,#3}$
}
\tikz \graph [typeset=\mytypesetter, grow down]
  { a 1 n -> b 2 m -> c 4 nm };
\end{codeexample}
        %
        The following styles install useful predefined typesetting macros:
        
        以下样式安装了有用的预定义排版宏:
        \begin{key}{/tikz/graphs/empty nodes}
            Just sets |typeset| to nothing, which causes all nodes to have an
            empty text (unless, of course, the |as| option is used):
            
            仅将 |typeset| 设置为空,这会导致所有节点的文本为空(除非当然使用了 |as| 选项):
\begin{codeexample}[preamble={\usetikzlibrary{graphs}}]
\tikz \graph [empty nodes, nodes={circle, draw}] { a -> {b, c} };
\end{codeexample}
        \end{key}
        %
        \begin{key}{/tikz/graphs/math nodes}
            Sets |typeset| to |$\tikzgraphnodetext$|, which causes all nodes
            names to be typeset in math mode:
            
            将 |typeset| 设置为 |$\tikzgraphnodetext$|,这会导致所有节点名称以数学模式排版:
\begin{codeexample}[preamble={\usetikzlibrary{graphs}}]
\tikz \graph [math nodes, nodes={circle, draw}] { a_1 -> {b^2, c_3^n} };
\end{codeexample}
        \end{key}
\end{itemize}

If a node is referenced instead of fresh, then this node becomes the node that
will be connected by the preceding or following edge specification to other
nodes. The \meta{options} are executed even for a referenced node, but they
cannot be used to change the appearance of the node (because the node exists
already). Rather, the \meta{options} can only be used to change the logical
coloring of the node, see Section~\ref{section-library-graphs-color-classes}
for details.

如果引用了一个节点而不是新节点,则该节点将成为由前面或后面的边规范连接到其他节点的节点。即使对于引用的节点,也会执行 \meta{options},但它们无法用于更改节点的外观(因为节点已经存在)。相反,\meta{options} 只能用于更改节点的逻辑着色,请参见第~\ref{section-library-graphs-color-classes} 节了解详情。
\medskip
\textbf{Quoted Node Names.} When the \meta{node name} and/or the \meta{text} of
a node is surrounded by quotation marks, you can use all sorts of special
symbols as part of the text that are normally forbidden:

\textbf{引号节点名称。} 当节点的\meta{node name}和/或\meta{text}被引号括起来时,您可以使用各种通常被禁止的特殊符号作为文本的一部分:

\begin{codeexample}[preamble={\usetikzlibrary{graphs}}]
\begin{tikzpicture}
  \graph [grow right=2cm] {
    "Hi, World!"       -> "It's \emph{important}!"[red,rotate=-45];
    "name"/actual text -> "It's \emph{important}!";
  };
  \draw (name) circle [radius=3pt];
\end{tikzpicture}
\end{codeexample}

In detail, for the following happens when quotation marks are encountered at
the beginning of a node name or its text:

具体而言,当在节点名称或其文本的开头遇到引号时,以下情况发生:

\begin{itemize}
    \item Everything following the quotation mark up to the next single
        quotation mark is collected into a macro \meta{collected}. All sorts of
        special characters, including commas, square brackets, dashes, and even
        backslashes are allowed here. Basically, the only restriction is that
        braces must be balanced.

        从引号到下一个单引号之间的所有内容都被收集到一个宏\meta{collected}中。在此处允许使用各种特殊字符,包括逗号、方括号、破折号,甚至反斜杠。基本上,唯一的限制是大括号必须平衡。

    \item A double quotation mark (|""|) does not count as the ``next single
        quotation mark''. Rather, it is replaced by a single quotation mark.
        For instance, |"He said, ""Hello world."""| would be stored inside
        \meta{collected} as |He said, "Hello world."| However, this rule
        applies only on the outer-most level of braces. Thus, in
        %

        双引号(|""|)不被视为“下一个单引号”。相反,它将被替换为单引号。例如,|“He said, ""Hello world."""|会被存储在\meta{collected}中,作为|He said, "Hello world."|。然而,此规则仅适用于括号的最外层级。因此,在下面的示例中:

\begin{codeexample}[code only]
"He {said, ""Hello world.""}"
\end{codeexample}
        %
        we would get |He {said, ""Hello world.""}| as \meta{collected}.

        我们将得到|He {said, ""Hello world.""}|作为\meta{collected}。
    \item ``The next single quotation mark'' refers to the next quotation mark
        on the current level of braces, so in |"hello {"} world"|, the next
        quotation mark would be the one following |world|.

        “下一个单引号”指的是当前级别括号中的下一个引号,因此在|“hello {"} world"|中,下一个引号是在|world|后面的引号。
\end{itemize}

Now, once the \meta{collected} text has been gather, it is used as follows:
When used as \meta{text} (what is actually displayed), it is just used ``as
is''. When it is used as \meta{node name}, however, the following happens:
Every ``special character'' in \meta{collected} is replaced by its Unicode
name, surrounded by |@|-signs. For instance, if \meta{collected} is
|Hello, world!|, the \meta{node name} is the somewhat longer text
|Hello@COMMA@ world@EXCLAMATION MARK@|. Admittedly, referencing such a node
from outside the graph is cumbersome, but when you use exactly the same
\meta{collected} text once more, the same \meta{node name} will result. The
following characters are considered ``special'':

现在,一旦\meta{collected}文本被收集,它将按以下方式使用:

当用作\meta{text}(实际显示的内容)时,它将按原样使用。然而,当它用作\meta{node name}时,将发生以下情况:

\meta{collected}中的每个“特殊字符”都将替换为其Unicode名称,用|@|符号括起来。例如,如果\meta{collected}是|Hello, world!|,则\meta{node name}是稍长一些的文本|Hello@COMMA@ world@EXCLAMATION MARK@|。诚然,从图外部引用这样的节点很麻烦,但当您再次使用完全相同的\meta{collected}文本时,将得到相同的\meta{node name}。以下字符被视为“特殊”:


\begin{quote}
    \texttt{\char`\|}|$&^~_[](){}/.-,+*'`!":;<=>?@#%\{}|%$
\end{quote}
%
These are exactly the Unicode character with a decimal code number between 33
and 126 that are neither digits nor letters.

这些字符正好是Unicode字符,其十进制代码编号介于33和126之间,既不是数字也不是字母。


\medskip
\textbf{Reference Node Specifications.} A reference node specification is a
node specification that starts with an opening parenthesis. In this case,
parentheses must surround a \meta{name} as in |(foo)|, where |foo| is the
\meta{name}. The following will now happen:
%

\textbf{参考节点规范。} 参考节点规范是以开括号开始的节点规范。在这种情况下,括号必须将\meta{name}括在其中,如|(foo)|,其中|foo|是\meta{name}。接下来将发生以下情况:

\begin{enumerate}
    \item It is tested whether \meta{name} is the name of a currently active
        \emph{node set}. This case will be discussed in a moment.

        首先测试 \meta{name} 是否是当前活动的\emph{节点集}的名称。这种情况将在稍后讨论。
    \item Otherwise, the \meta{name} is interpreted and treated as a referenced
        node, but independently of whether the node has already been fresh in
        the current graph or not. In other words, the node must have been
        defined either already inside the graph (in which case the parenthesis
        are more or less superfluous) or it must have been defined outside the
        current picture.

        否则,将解释和处理 \meta{name} 作为引用的节点,但不管该节点是否已经存在于当前图形中。换句话说,该节点必须已经在图形内部定义(在这种情况下,括号更多或更少是多余的),或者它必须在当前图之外定义。

        The way the referenced node is handled is the same way as for a direct
        node that is a referenced node.

        对于引用的节点的处理方式与直接节点的引用节点相同。

        If the node does not already exist, an error message is printed.

        如果该节点尚不存在,则会打印错误消息。
\end{enumerate}

Let us now have a look at node sets. Inside a |{tikzpicture}| you can locally
define a \emph{node set} by using the following key:

现在让我们来看看节点集。在 |{tikzpicture}| 内部,您可以使用以下键局部定义一个\emph{节点集}:

\begin{key}{/tikz/new set=\meta{set name}}
    This will setup a node set named \meta{set name} within the current scope.
    Inside the scope, you can add nodes to the node set using the |set| key. If
    a node set of the same name already exists in the current scope, it will be
    reset and made empty for the current scope.

    这将在当前作用域内设置一个名为 \meta{set name} 的节点集。在作用域内,您可以使用 |set| 键将节点添加到节点集中。如果同一名称的节点集已经存在于当前作用域中,则将重置该节点集,并在当前作用域中将其设置为空。

    Note that this command has the path |/tikz| and is normally used
    \emph{outside} the |graph| command.

    请注意,此命令具有路径 |/tikz|,通常用于在 |graph| 命令之外使用。
\end{key}
%
\begin{key}{/tikz/set=\meta{set name}}
    This key can be used as an option with a |node| command. The \meta{set
    name} must be the name of a node set that has previously been created
    inside some enclosing scope via the |new set| key. The effect is that the
    current node is added to the node set.

    此键可以作为 |node| 命令的选项使用。 \meta{set name} 必须是之前通过 |new set| 键在某个封闭作用域内创建的节点集的名称。效果是将当前节点添加到节点集中。
\end{key}

When you use a |graph| command inside a scope where some node set called
\meta{set name} is defined, then inside this |graph| command you use
|(|\meta{set name}|)| to reference \emph{all} of the nodes in the node set. The
effect is the same as if instead of the reference to the set name you had
created a group specification containing a list of references to all the nodes
that are part of the node set.

当您在定义了名为 \meta{set name} 的节点集的某个封闭作用域内使用 |graph| 命令时,您可以在此 |graph| 命令内部使用 |(|\meta{set name}|)| 引用节点集中的\emph{所有}节点。效果是与您创建一个包含对节点集中所有节点引用的组规范相同。
\begin{codeexample}[preamble={\usetikzlibrary{graphs}}]
\begin{tikzpicture}[new set=red, new set=green, shorten >=2pt]
  \foreach \i in {1,2,3} {
    \node [draw, red!80,         set=red]   (r\i) at (\i,1) {$r_\i$};
    \node [draw, green!50!black, set=green] (g\i) at (\i,2) {$g_\i$};
  }
  \graph {
    root [xshift=2cm] ->
    (red)             -> [complete bipartite, right anchor=south]
    (green)
  };
\end{tikzpicture}
\end{codeexample}

There is an interesting caveat with referencing node sets: Suppose that at the
beginning of a graph you just say |(foo);| where |foo| is a set name. Unless
you have specified special options, this will cause the following to happen: A
group is created whose members are all the nodes of the node set |foo|. These
nodes become referenced nodes, but otherwise nothing happens since, by default,
the nodes of a group are not connected automatically. However, the referenced
nodes have now been referenced inside the graph, you can thus subsequently
access them as if they had been defined inside the graph. Here is an example
showing how you can create nodes outside a |graph| command and then connect
them inside as if they had been declared inside:

在引用节点集时有一个有趣的注意事项:假设在图的开头只使用 |(foo);|,其中 |foo| 是一个集名称。除非您指定了特殊选项,否则将发生以下情况:创建一个组,其成员是节点集 |foo| 的所有节点。这些节点变成了引用节点,但除此之外没有其他操作,因为默认情况下,组的节点不会自动连接。但是,引用节点现在已在图中被引用,因此您随后可以像它们在图内部定义一样访问它们。下面是一个示例,演示如何在 |graph| 命令之外创建节点,然后在内部连接它们,就好像它们在内部声明一样:


\begin{codeexample}[preamble={\usetikzlibrary{graphs}}]
\begin{tikzpicture}[new set=import nodes]
  \begin{scope}[nodes={set=import nodes}] % make all nodes part of this set
    \node [red] (a) at (0,1) {$a$};
    \node [red] (b) at (1,1) {$b$};
    \node [red] (d) at (2,1) {$d$};
  \end{scope}

  \graph {
    (import nodes);         % "import" the nodes

    a -> b -> c -> d -> e;  % only c and e are new
  };
\end{tikzpicture}
\end{codeexample}

\medskip
\textbf{Group Node Specifications.} At a place where a node specification
should go, you can also instead provide a group specification. Since nodes
specifications are part of chain specifications, which in turn are part of
group specifications, this is a recursive definition.

\textbf{组节点规范。} 在节点规范应该出现的位置,您也可以提供组规范作为替代。由于节点规范是链规范的一部分,而链规范又是组规范的一部分,这是一个递归定义。


\begin{codeexample}[preamble={\usetikzlibrary{graphs}}]
\tikz \graph { a -> {b,c,d} -> {e -> {f,g}, h} };
\end{codeexample}

As can be seen in the above example, when two groups of nodes are connected via
an edge specification, it is not immediately obvious which connecting edges are
added. This is detailed in Section~\ref{section-library-graphs-color-classes}.

正如上面的示例所示,当通过边规范连接两组节点时,不会立即清楚添加了哪些连接边。这在第~\ref{section-library-graphs-color-classes} 节中详细说明。


\subsubsection{Specifying Tries\\指定 Trie}

In computer science, a \emph{trie} is a special kind of tree, where for each
node and each symbol of an alphabet, there is at most one child of the node
labeled with this symbol.

在计算机科学中,\emph{trie} 是一种特殊类型的树,对于每个节点和每个字母表符号,节点上最多只有一个用该符号标记的子节点。

The |trie| key is useful for drawing tries, but it can also be used in other
situations. What it does, essentially, is to prepend the node names of all
nodes \emph{before} the current node of the current chain to the node's name.
This will often make it easier or more natural to specify graphs in which
several nodes have the same label.

|trie| 键用于绘制 trie,但它也可以用于其他情况。它的实质是将当前链的当前节点之前的所有节点名称的节点名称前缀到节点的名称。这通常使得在指定具有多个相同标签的节点的图形时更容易或更自然。


\begin{key}{/tikz/graphs/trie=\opt{\meta{true or false}} (default true, initially false)}
    If this key is set to |true|, after a node has been created on a chain, the
    |name| key is executed with the node's \meta{node name}. Thus, all nodes
    later on this chain have the ``path'' of nodes leading to this node as
    their name. This means, in particular, that
    
    如果将此键设置为 |true|,在链上创建了一个节点之后,将使用节点的 \meta{node name} 执行 |name| 键。因此,链上的所有后续节点都将其名称设置为导致该节点的节点的“路径”。这意味着,特别是:

    \begin{enumerate}
        \item two nodes of the same name but in different parts of a chain will
            be different,

            链的不同部分上具有相同名称的两个节点将是不同的。

        \item while if another chain starts with the same nodes, no new nodes
            get created.

            如果另一条链以相同的节点开始,将不会创建新的节点。
    \end{enumerate}
    %
    In total, this is exactly the behavior you would expect of a trie:
    
    总的来说,这正是您对 trie 期望的行为:
\begin{codeexample}[preamble={\usetikzlibrary{graphs}}]
\tikz \graph [trie] {
  a -> {
    a,
    c -> {a, b},
    b
  }
};
\end{codeexample}
    %
    You can even ``reiterate'' over a path in conjunction with the |simple|
    option. However, in this case, the default placement strategies will not
    work and you will need options like |layered layout| from the graph drawing
    libraries, which need Lua\TeX.
    
    您甚至可以在与 |simple| 选项结合使用的路径上“重复迭代”。然而,在这种情况下,默认的放置策略将无法工作,您将需要使用图形绘制库中的选项(如 |layered layout|),这需要 Lua\TeX。

\ifluatex
\begin{codeexample}[preamble={\usetikzlibrary{graphs,graphdrawing}\usegdlibrary{layered}}]
\tikz \graph [trie, simple, layered layout] {
  a -> b -> a,
  a -> b -> c,
  a -> {d,a}
};
\end{codeexample}
    %
    In the following example, we setup the |typeset| key so that it shows the
    complete names of the nodes:
    
    在下面的示例中,我们设置了 |typeset| 键,以显示节点的完整名称:
\begin{codeexample}[preamble={\usetikzlibrary{graphs,graphdrawing}\usegdlibrary{layered}}]
\tikz \graph [trie, simple, layered layout,
              typeset=\tikzgraphnodefullname] {
  a -> b -> a,
  a -> b -> c,
  a -> {d,a}
};
\end{codeexample}
\fi
    %
    You can also use the |trie| key locally and later reference nodes using
    their full name:
    
    您还可以在本地使用 |trie| 键,并稍后使用它们的完整名称引用节点:
\begin{codeexample}[preamble={\usetikzlibrary{graphs}}]
\tikz \graph {
  { [trie, simple]
    a -> {
      b,
      c -> a
    }
  },
  a b ->[red] a c a
};
\end{codeexample}
    %
\end{key}


\subsection{Quick Graphs\\快速图形}
\label{section-library-graphs-quick}

The graph syntax is powerful, but this power comes at a price: parsing the
graph syntax, which is done by \TeX, can take some time. Normally, the parsing
is fast enough that you will not notice it, but it can be bothersome when you
have graphs with hundreds of nodes as happens frequently when nodes are
generated algorithmically by some other program. Fortunately, when another
program generated a graph specification, we typically do not need the full
power of the graph syntax. Rather, a small subset of the graph syntax would
suffice that allows to specify nodes and edges. For these reasons, the is a
special ``quick'' version of the graph syntax.

图形语法功能强大,但是这种功能的实现需要代价:解析图形语法,这由 \TeX\ 完成,可能需要一些时间。通常情况下,解析速度足够快,您不会注意到它,但是当图形具有数百个节点时可能会令人讨厌,这在节点由其他程序以某种算法方式生成时经常发生。幸运的是,当另一个程序生成了图形规范时,通常我们不需要图形语法的全部功能。相反,只需要一小部分的图形语法即可,该部分允许指定节点和边。出于这些原因,有一种特殊的“快速”版本的图形语法。

Note, however, that using this syntax will usually at most halve the time
needed to parse a graph. Thus, it really mostly makes sense in conjunction with
large, algorithmically generated graphs.

然而,请注意,使用此语法通常最多可以将解析图形所需的时间减少一半。因此,它实际上在大型、算法生成的图形中才有意义。

\begin{key}{/tikz/graphs/quick}
    When you provide this key with a graph, the syntax of graph specifications
    gets restricted. You are no longer allowed to use certain features of the
    graph syntax; but all features that are still allowed are also allowed in
    the same way when you do not provide the |quick| option. Thus, leaving out
    the |quick| option will never hurt.

    当您使用此键提供图形时,图形规范的语法将受到限制。您不再允许使用图形语法的某些功能;但是,当您不提供 |quick| 选项时,所有仍然允许的功能也以相同的方式允许使用。因此,省略 |quick| 选项永远不会有问题。

    Since the syntax is so severely restricted, it is easier to explain which
    aspects of the graph syntax \emph{will} still work:
    
    由于语法受到如此严格的限制,更容易解释哪些图形语法的方面仍然有效:
    \begin{enumerate}
        \item A quick graph consists of a sequence of either nodes, edges
            sequences, or groups. These are separated by commas or semicolons.

            快速图形由节点、边序列或组的序列组成。它们由逗号或分号分隔。
        \item Every node is of the form

        每个节点的形式为
            %
            \begin{quote}
                |"|\meta{node name}|"|\opt{|/"|\meta{node text}|"[|\meta{options}|]|}
            \end{quote}

            The quotation marks are mandatory. The part |/"|\meta{node text}|"|
            may  be missing, in which case the node name is used as the node
            text. The \meta{options} may also be missing. The \meta{node name}
            may not contain any ``funny'' characters (unlike in the normal
            graph command).

            引号是强制性的。部分 |/"|\meta{node text}|"| 可以省略,在这种情况下,节点名称将用作节点文本。选项 \meta{options} 也可以省略。节点名称不得包含任何``奇怪''的字符(与正常的图形命令不同)。

        \item Every chain is of the form

        每个链的形式为
            %
            \begin{quote}
                \meta{node spec} \meta{connector} \meta{node spec}
                \meta{connector} \dots \meta{connector} \meta{node spec}|;|
            \end{quote}

            Here, the \meta{node spec} are node specifications as described
            above, the \meta{connector} is one of the four connectors |->|,
            |<-|, |--|, and |<->| (the connector |-!-| is not allowed since the
            |simple| option is also not allowed). Each connector may be
            followed by options in square brackets. The semicolon may be
            replaced by a comma.

            在这里,\meta{node spec} 是如上所述的节点规范,\meta{connector} 是四个连接器之一:|->|、|<-|、|--| 和 |<->|(由于也不允许使用 |simple| 选项,不允许使用连接器 |-!-|)。每个连接器后面可以跟随方括号中的选项。分号可以替换为逗号。

        \item Every group is of the form

        每个组的形式为

            %
            \begin{quote}
                |{ [|\meta{options}|]| \meta{chains and groups} |};|
            \end{quote}
            %
            The \meta{options} are compulsory. The semicolon can, again, be
            replaced by a comma.

            \meta{options} 是强制性的。分号可以再次替换为逗号。

        \item The |number nodes| option will work as expected.

        |number nodes| 选项将按预期工作。
    \end{enumerate}

    Here is a typical way this syntax might be used:
    
    以下是使用此语法的典型方式:
\begin{codeexample}[preamble={\usetikzlibrary{graphs,quotes}}]
\tikz \graph [quick] { "a" --["foo"] "b"[x=1] };
\end{codeexample}

\begin{codeexample}[preamble={\usetikzlibrary{graphs}}]
\tikz \graph [quick] {
  "a"/"$a$" -- "b"[x=1] --[red] "c"[x=2];
  { [nodes=blue] "a" -- "d"[y=1]; };
};
\end{codeexample}

    Let us now have a look at the most important things that will \emph{not}
    work when the |quick| option is used:

    现在让我们来看一下当使用 |quick| 选项时将\emph{不起作用}的最重要的事情:

    \begin{itemize}
        \item Connecting a node and a group as in |a->{b,c}|.

        连接节点和组,例如 |a->{b,c}|。

        \item Node names without quotation marks as in |a--b|.

        节点名称不带引号,例如 |a--b|。


        \item Everything described in subsequent subsections, which includes
            subgraphs (graph macros), graph sets, graph color classes,
            anonymous nodes, the |fresh nodes| option, sublayouts, simple
            graphs, edge annotations.

            后续小节中描述的所有内容,包括子图(图宏)、图形集、图形颜色类、匿名节点、|fresh nodes| 选项、子布局、简单图形、边注释。


        \item Placement strategies -- you either have to define all node
            positions explicitly using |at=| or |x=| and |y=| or you must use a
            graph drawing algorithm like |layered layout|.

            放置策略 -- 您必须明确地使用 |at=| 或 |x=| 和 |y=| 定义所有节点位置,或者您必须使用像 |layered layout| 这样的图形绘制算法。


    \end{itemize}
\end{key}


\subsection{Simple Versus Multi-Graphs\\简单图与多重图}
\label{section-library-graphs-simple}

The |graphs| library allows you to construct both simple graphs and
multi-graphs. In a simple graph there can be at most one edge between any two
vertices, while in a multi-graph there can be multiple edges (hence the name).
The two keys |multi| and |simple| allow you to switch (even locally inside on
of the graph's scopes) between which kind of graph is being constructed. By
default, the |graph| command produces a multi-graph since these are faster to
construct.

|graphs| 库允许您构建简单图和多重图。在简单图中,任意两个顶点之间最多只能有一条边,而在多重图中可以有多条边(因此得名)。两个键 |multi| 和 |simple| 允许您在构建图时切换(甚至可以在图的作用域内局部切换)。默认情况下,|graph| 命令生成的是多重图,因为它们构建起来更快。

\begin{key}{/tikz/graphs/multi}
    When this edge is set for a whole graph (which is the default) or just for
    a group (which is useful if the whole graph is simple in general, but a
    part is a multi-graph), then when you specify an edge between two nodes
    several times, several such edges get created:
    
    当这个边设置为整个图(默认情况),或者仅设置为一个分组(如果整个图通常是简单图,但部分是多重图),那么当您多次指定两个节点之间的边时,将创建多个这样的边:
\begin{codeexample}[preamble={\usetikzlibrary{graphs}}]
\tikz \graph [multi] { % "multi" is not really necessary here
  a ->[bend left,  red]  b;
  a ->[bend right, blue] b;
};
\end{codeexample}
    %
    In case |multi| is used for a scope inside a larger scope where the
    |simple| option is specified, then inside the local |multi| scope edges are
    immediately created and they are completely ignored when it comes to
    deciding which kind of edges should be present in the surrounding simple
    graph. From the surrounding scope's point of view it is as if the local
    |multi| graph contained no edges at all.

    如果在一个大的作用域中使用 |simple| 选项,而在其中的一个作用域内使用了 |multi|,那么在本地的 |multi| 作用域内立即创建边,并且在决定周围简单图中应该存在哪种类型的边时,完全忽略它们。从周围作用域的观点来看,好像本地的 |multi| 图根本没有边。

This means, in particular, that you can use the |multi| option with a
    single edge to ``enforce'' this edge to be present in a simple graph.

    这意味着,特别是您可以使用 |multi| 选项和一条边来“强制”这条边在简单图中存在。
\end{key}

\begin{key}{/tikz/graphs/simple}
    In contrast a multi-graph, in a simple graph, at most one edge gets created
    for every pair of vertices:
    
    相反,对于简单图而言,在每对顶点之间最多只会创建一条边:%
\begin{codeexample}[preamble={\usetikzlibrary{graphs}}]
\tikz \graph [simple]{
  a ->[bend left,  red]  b;
  a ->[bend right, blue] b;
};
\end{codeexample}
    %
    As can be seen, the second edge ``wins'' over the first edge. The general
    rule is as follows: In a simple graph, whenever an edge between two
    vertices is specified multiple times, only the very last specification and
    its options will actually be executed.

    可以看到,第二条边“胜出”第一条边。一般规则如下:在简单图中,只有当指定了两个顶点之间的多条边时,只有最后一个规范及其选项将被实际执行。


    The real power of the |simple| option lies in the fact that you can first
    create a complicated graph and then later redirect and otherwise modify
    edges easily:
    
    |simple| 选项的真正威力在于您可以先创建一个复杂的图,然后轻松地重定向和修改边缘:

\begin{codeexample}[preamble={\usetikzlibrary{graphs}}]
\tikz \graph [simple, grow right=2cm] {
  {a,b,c,d} ->[complete bipartite] {e,f,g,h};

  { [edges={red,thick}] a -> e -> d -> g -> a };
};
\end{codeexample}

    One particularly interesting kind of edge specification for a simple graph
    is |-!-|. Recall that this is used to indicate that ``no edge'' should be
    added between certain nodes. In a multi-graph, this key usually has no
    effect (unless the key |new -!-| has been redefined) and is pretty
    superfluous. In a simple graph, however, it counts as an edge kind and you
    can thus use it to remove an edge that been added previously:
    
    对于简单图而言,一种特别有趣的边缘规范是 |-!-|。请记住,这用于指示在某些节点之间“没有边”。在多重图中,这个键通常没有效果(除非重新定义了键 |new -!-|),并且是多余的。然而,在简单图中,它被视为一种边缘类型,因此您可以使用它来删除先前添加的边缘:


\begin{codeexample}[preamble={\usetikzlibrary{graphs.standard}}]
\tikz \graph [simple] {
  subgraph K_n [n=8, clockwise];
  % Get rid of the following edges:
  1 -!- 2;
  3 -!- 4;
  6 -!- 8;
  % And make one edge red:
  1 --[red] 3;
};
\end{codeexample}

    Creating a graph such as the above in other fashions is pretty awkward.

    以其他方式创建上述图形相当麻烦。

    For every unordered pair $\{u,v\}$ of vertices at most one edge will be
    created in a simple graph. In particular, when you say |a -> b| and later
    also |a <- b|, then only the edge |a <- b| will be created. Similarly, when
    you say |a -> b| and later |b -> a|, then only the edge |b -> a| will be
    created.

    对于每对无序顶点 ${u,v}$,在简单图中最多会创建一条边。特别地,当您说 |a -> b|,然后又说 |a <- b| 时,只会创建边缘 |a <- b|。同样地,当您说 |a -> b|,然后又说 |b -> a| 时,只会创建边缘 |b -> a|。

    The power of the |simple| command comes at a certain cost: As the graph is
    being constructed, a (sparse) array is created that keeps track for each
    edge of the last edge being specified. Then, at the end of the scope
    containing the |simple| command, for every pair of vertices the edge is
    created. This is implemented by two nested loops iterating over all
    possible pairs of vertices -- which may take quite a while in a graph of,
    say, 1000 vertices. Internally, the |simple| command is implemented as an
    operator that adds the edges when it is called, but this should be
    unimportant in normal situations.

    |simple| 命令的威力是以一定的代价而来:在构建图时,会创建一个(稀疏)数组,用于跟踪每条边的最后一条被指定的边。然后,在包含 |simple| 命令的作用域结束时,为每对顶点创建边缘。这通过两个嵌套循环实现,迭代所有可能的顶点对——在具有,比如,1000个顶点的图中可能需要相当长的时间。从内部来看,|simple| 命令被实现为一个在调用时添加边缘的运算符,但在正常情况下,这不重要。
\end{key}


\subsection{Graph Edges: Labeling and Styling\\图的边缘:标记和样式}

When the |graphs| library creates an edge between two nodes in a graph, the
appearance (called ``styling'' in \tikzname) can be specified in different
ways. Sometimes you will simply wish to say ``the edges between these two
groups of node should be red'', but sometimes you may wish to say ``this
particular edge going into this node should be red''. In the following,
different ways of specifying such styling requirements are discussed. Note that
adding labels to edges is, from \tikzname's point of view, almost the same as
styling edges, since they are also specified using options.

当|graphs|库在图中的两个节点之间创建边缘时,可以以不同的方式指定外观(在TikZ 中称为“样式”)的方式。有时您只希望说“这两组节点之间的边缘应为红色”,但有时您可能希望说“这个特定的边缘进入这个节点应为红色”。以下讨论了指定此类样式要求的不同方式。请注意,从 TikZ 的角度来看,添加边缘标签与样式边缘几乎相同,因为它们也是使用选项来指定的。

\subsubsection{Options For All Edges Between Two Groups\\两个组之间所有边的选项}

When you write |... ->[options] ...| somewhere inside your graph specification,
this typically cause one or more edges to be created between the nodes in the
chain group before the |->| and the nodes in the chain group following it. The
|options| are applied to all of them. In particular, if you use the |quotes|
library and you write some text in quotes inside the |options|, this text will
be added as a label to each edge:

当您在图形规范的某个位置写入|... ->[options] ...|时,这通常会在|->|之前的链组中的节点和之后的链组中的节点之间创建一个或多个边。这些|options|将应用于所有边。特别是,如果您使用了|quotes|库,并在|options|中的引号内写入了一些文本,则此文本将被添加为每个边的标签:

\begin{codeexample}[preamble={\usetikzlibrary{graphs,quotes}}]
\tikz
  \graph [edge quotes=near start] {
    { a, b } -> [red, "x", complete bipartite] { c, d };
  };
\end{codeexample}

As documented in the |quotes| library in more detail, you can easily modify the
appearance of edge labels created using the quotes syntax by adding options
after the closing quotes:

如在|quotes|库中更详细地记录的那样,您可以通过在闭引号后添加选项来轻松修改使用引号语法创建的边标签的外观:

\begin{codeexample}[preamble={\usetikzlibrary{graphs,quotes}}]
\tikz \graph {
  a ->["x"] b ->["y"'] c ->["z" red] d;
};
\end{codeexample}

The following options make it easy to setup the styling of nodes created in
this way:

以下选项可方便地设置以此方式创建的节点的样式:

\begin{key}{/tikz/graphs/edge quotes=\opt{\meta{options}}}
    A shorthand for setting the style |every edge quotes| to \meta{options}.
    
    将样式|every edge quotes|设置为\meta{options}的速记方式。
\begin{codeexample}[preamble={\usetikzlibrary{graphs,quotes}}]
  \tikz \graph [edge quotes={blue,auto}] {
  a ->["x"] b ->["y"'] c ->["b" red] d;
};
\end{codeexample}
    %
\end{key}

\begin{key}{/tikz/graphs/edge quotes center}
    A shorthand for |edge quotes| to |anchor=center|.
    
    将|edge quotes|设置为|anchor=center|的速记方式。
\begin{codeexample}[preamble={\usetikzlibrary{graphs,quotes}}]
\tikz \graph [edge quotes center] {
  a ->["x"] b ->["y"] c ->["z" red] d;
};
\end{codeexample}
    %
\end{key}

\begin{key}{/tikz/graphs/edge quotes mid}
    A shorthand for |edge quotes| to |anchor=mid|.
    
    将|edge quotes|设置为|anchor=mid|的速记方式。
\begin{codeexample}[preamble={\usetikzlibrary{graphs,quotes}}]
\tikz \graph [edge quotes mid] {
  a ->["x"] b ->["y"] c ->["z" red] d;
};
\end{codeexample}
    %
\end{key}


\subsubsection{Changing Options For Certain Edges\\更改特定边的选项}

Consider the following tree-like graph:

考虑以下类似树状的图形:

\begin{codeexample}[preamble={\usetikzlibrary{graphs}}]
\tikz \graph { a -> {b,c} };
\end{codeexample}

Suppose we wish to specify that the edge from |a| to |b| should be red, while
the edge from |a| to |c| should be blue. The difficulty lies in the fact that
\emph{both} edges are created by the single |->| operator and we can only add
one of these option |red| or |blue| to the operator.

假设我们希望指定从|a|到|b|的边为红色,而从|a|到|c|的边为蓝色。困难在于\emph{两个}边都是由单个|->|运算符创建的,我们只能在该运算符中添加这些选项之一|red|或|blue|。

There are several ways to solve this problem. First, we can simply split up the
specification and specify the two edges separately:

解决这个问题有几种方法。首先,我们可以简单地拆分规范,并分别指定这两个边:

%
\begin{codeexample}[preamble={\usetikzlibrary{graphs}}]
\tikz \graph {
  a -> [red]  b;
  a -> [blue] c;
};
\end{codeexample}
%
While this works quite well, we can no longer use the nice chain group syntax
of the |graphs| library. For the rather simple graph |a->{b,c}| this is not a
big problem, but if you specify a tree with, say, 30 nodes it is really
worthwhile being able to specify the tree ``in its natural form in the \TeX\
code'' rather than having to list all of the edges explicitly. Also, as can be
seen in the above example, the node placement is changed, which is not always
desirable.

虽然这种方法很好用,但我们不再能够使用|graphs|库的优雅链组语法。对于相当简单的图形|a->{b,c}|,这并不是一个大问题,但如果您指定一个具有30个节点的树,能够以“在\TeX\ 代码中以其自然形式指定树”而不必列举所有边是非常有价值的。此外,如上面的示例所示,节点的布局发生了变化,这并不总是理想的。

One can sidestep this problem using the |simple| option: This option allows you
to first specify a graph and then, later on, replace edges by other edges and,
thereby, provide new options:

通过使用|simple|选项可以避开这个问题:此选项允许您首先指定一个图形,然后在以后用其他边替换边,从而提供新的选项:
\begin{codeexample}[preamble={\usetikzlibrary{graphs}}]
\tikz \graph [simple] {
  a -> {b,c};
  a -> [red]  b;
  a -> [blue] c;
};
\end{codeexample}

The first line is the original specification of the tree, while the following
two lines replace some edges of the tree (in this case, all of them) by edges
with special options. While this method is slower and in the above example
creates even longer code, it is very useful if you wish to, say, highlight a
path in a larger tree: First specify the tree normally and, then, ``respecify''
the path or paths with some other edge options in force. In the following
example, we use this to highlight a whole subtree of a larger tree:
    
第一行是树的原始规范,而接下来的两行将树的一些边(在本例中是全部边)替换为具有特殊选项的边。虽然这种方法较慢,并且在上面的示例中创建了更长的代码,但如果您希望在较大树中突出显示一个路径,这非常有用:首先正常指定树,然后使用某些其他边选项“重新指定”路径或路径。在下面的示例中,我们使用此方法来突出显示较大树的整个子树:


\begin{codeexample}[preamble={\usetikzlibrary{graphs}}]
\tikz \graph [simple] {
  % The larger tree, no special options in force
  a -> {
    b -> {c,d},
    e -> {f,g},
    h
  },
  { [edges=red] % Now highlight a part of the tree
    a -> e -> {f,g}
  }
};
\end{codeexample}


\subsubsection{Options For Incoming and Outgoing Edges\\入边和出边的选项}

When you use the syntax |... ->[options] ...| to specify options, you specify
options for the ``connections between two sets of nodes''. In many cases,
however, it will be more natural to specify options ``for the edges lead to or
coming from a certain node'' and you will want to specify these options ``at
the node''. Returning to the example of the graph |a->{b,c}| where we want a
red edge between |a| and |b| and a blue edge between |a| and |c|, this could
also be phrased as follows: ``Make the edge leading to |b| red and make the
edge leading to |c| blue''.

当您使用语法|... ->[options] ...|来指定选项时,您指定的是“两组节点之间的连接”的选项。然而,在许多情况下,更自然的做法是指定“指向或来自特定节点的边”的选项,并且您希望在“节点处”指定这些选项。回到图形|a->{b,c}|的示例中,我们希望在|a|和|b|之间有一条红色边,在|a|和|c|之间有一条蓝色边,这也可以表述为:“使指向|b|的边为红色,并使指向|c|的边为蓝色”。

For this situation, the |graphs| library offers a number of special keys, which
are documented in the following. However, most of the time you will not use
these keys directly, but, rather, use a special syntax explained in
Section~\ref{section-syntax-outgoing-incoming}.

针对这种情况,|graphs| 库提供了一些特殊的键,这些键在下面进行了文档化。然而,大多数情况下,您不会直接使用这些键,而是使用第~\ref{section-syntax-outgoing-incoming} 节中解释的特殊语法。


\begin{key}{/tikz/graphs/target edge style=\meta{options}}
    This key can (only) be used with a \emph{node} inside a graph
    specification. When used, the \meta{options} will be added to every edge
    that is created by a connector like |->| in which the node is a
    \emph{target}. Consider the following example:
    
    此键仅可在图形规范内的\emph{节点}中使用。使用时,\meta{options} 将添加到由 |->| 等连接器创建的每条以节点作为\emph{目标}的边上。考虑以下示例:
\begin{codeexample}[preamble={\usetikzlibrary{graphs}}]
\tikz \graph {
  { a, b } ->
  { c [target edge style=red], d } ->
  { e, f }
};
\end{codeexample}
    %
    In the example, only when the edge from |a| to |c| is created, |c| is the
    ``target'' of the edge. Thus, only this edge becomes red.

    在示例中,只有在从 |a| 到 |c| 的边创建时,|c| 是边的``目标''。因此,只有这条边变为红色。

    When an edge already has options set directly, the \meta{options} are
    executed after these direct options, thus, they ``overrule'' them:
    
    当边已经直接设置了选项时,\meta{options} 会在这些直接选项之后执行,因此它们会``覆盖''它们:

\begin{codeexample}[preamble={\usetikzlibrary{graphs}}]
\tikz \graph {
  { a, b } -> [blue, thick]
  { c [target edge style=red], d } ->
  { e, f }
};
\end{codeexample}

    The \meta{options} set in this way will stay attached to the node, so also
    for edges created later on that lead to the node will have these options
    set:
    
    以这种方式设置的 \meta{options} 将保持附加到节点上,因此后续创建的指向该节点的边也会具有这些选项:
\begin{codeexample}[preamble={\usetikzlibrary{graphs}}]
\tikz \graph {
  { a, b } ->
  { c [target edge style=red], d } ->
  { e, f },
  b -> c
};
\end{codeexample}

    Multiple uses of this key accumulate. However, you may sometimes also wish
    to ``clear'' these options for a key since at some later point you no
    longer wish the \meta{options} to be added when some further edges are
    added. This can be achieved using the following key:
    
    多次使用此键会累积效果。然而,有时您可能也希望``清除''某个键的这些选项,因为在后续添加某些边时不再希望添加 \meta{options}。可以使用以下键来实现:

    \begin{key}{/tikz/graphs/target edge clear}
        Clears all \meta{options} for edges with the node as a target and
        also edge labels (see below) for this node.

        清除节点作为目标的所有边的所有 \meta{options},还包括该节点的边标签(见下文)。
    \end{key}
    %
\begin{codeexample}[preamble={\usetikzlibrary{graphs}}]
\tikz \graph {
  { a, b } ->
  { c [target edge style=red], d },
  b -> c[target edge clear]
};
\end{codeexample}
    %
\end{key}

\begin{key}{/tikz/graphs/target edge node=\meta{node specification}}
    This key works like |target edge style|, only the \meta{node specification}
    will not be added as options to any newly created edges with the current
    node as their target, but rather it will be added as a node specification.
    
    该键的工作方式类似于 |target edge style|,只是 \meta{node specification} 不会作为选项添加到以当前节点为目标的任何新创建的边,而是会作为节点规范添加。
\begin{codeexample}[preamble={\usetikzlibrary{graphs}}]
\tikz \graph {
  { a, b } ->
  { c [target edge node=node{X}], d } ->
  { e, f }
};
\end{codeexample}
    %
    As for |target edge style| multiple uses of this key accumulate and the key
    |target edge clear| will (also) clear all target edge nodes that have been
    set for a node earlier on.

    对于 |target edge style|,多次使用此键会累积效果,而键 |target edge clear| 将(也)清除之前为节点设置的所有目标边节点。

\end{key}

\begin{key}{/tikz/graphs/source edge style=\meta{options}}
    Works exactly like |target edge style|, only now the \meta{options} are
    only added when the node is a source of a newly created edge:
    
    与 |target edge style| 的工作方式完全相同,只是现在只有当节点是新创建边的源时才会添加 \meta{options}:
\begin{codeexample}[preamble={\usetikzlibrary{graphs}}]
\tikz \graph {
  { a, b } ->
  { c [source edge style=red], d } ->
  { e, f }
};
\end{codeexample}
    %
    If both for the source and also for the target of an edge \meta{options}
    have been specified, the options are applied in the following order:
    
    如果对边的源和目标都指定了 \meta{options},则按以下顺序应用选项:
    \begin{enumerate}
        \item First come the options from the edge itself.

        首先是边本身的选项。

        \item Then come the options contributed by the source node using this
            key.

            然后是源节点使用此键贡献的选项。
        \item Then come the options contributed by the target node using
            |target node style|.

            然后是目标节点使用 |target node style| 贡献的选项。
    \end{enumerate}
    %
\begin{codeexample}[preamble={\usetikzlibrary{graphs}}]
\tikz \graph {
  a [source edge style=red] ->[green]
  b [target edge style=blue]  % blue wins
};
\end{codeexample}
    %
\end{key}

\begin{key}{/tikz/graphs/source edge node=\meta{node specification}}
    Works like |source edge style| and |target edge node|.

    与 |source edge style| 和 |target edge node| 的工作方式相同。
\end{key}

\begin{key}{/tikz/graphs/source edge clear=\meta{node specification}}
    Works like |target edge clear|.

    与 |target edge clear| 的工作方式相同。
  \end{key}


\subsubsection{Special Syntax for Options For Incoming and Outgoing Edges\\用于入边和出边的选项的特殊语法}
\label{section-syntax-outgoing-incoming}

The keys |target node style| and its friends are powerful, but a bit cumbersome
to write down. For this reason, the |graphs| library introduces a special
syntax that is based on what I call the ``first-char syntax'' of keys. Inside
the options of a node inside a graph, the following special rules apply:

|target node style| 等键非常强大,但写起来有点麻烦。因此,|graphs| 库引入了一种特殊的语法,该语法基于我称之为键的``首字符语法''。在图中节点的选项中,以下特殊规则适用:

\begin{enumerate}
    \item Whenever an option starts with |>|, the rest of the options are
        passed to |target edge style|. For instance, when you write |a[>red]|,
        then this has the same effect as if you had written
        
        每当一个选项以 |>| 开头时,其后的所有选项都传递给 |target edge style|。例如,当您写入 |a[>red]| 时,效果与您写入以下内容相同:
\begin{codeexample}[code only]
a[target edge style={red}]
\end{codeexample}
        %
    \item Whenever an options starts with |<|, the rest of the options are
        passed to |source edge style|.

        每当选项以 |<| 开头时,其后的所有选项都传递给 |source edge style|。
    \item In both of the above case, in case the options following the |>| or
        |<| sign start with a quote, the created edge label is passed to
        |source edge node| or |target edge node|, respectively.

        在上述两种情况下,如果紧跟 |>| 或 |<| 符号之后的选项以引号开头,则创建的边标签分别传递给 |source edge node| 或 |target edge node|。

        This is exactly what you want to happen.

        这正是您希望发生的情况。
\end{enumerate}
%
Additionally, the following styles provide shorthands for ``clearing'' the
target and source options:

此外,以下样式提供了用于``清除''目标和源选项的简写方式:

\begin{key}{/tikz/graphs/clear >}
    A more easy-to-remember shorthand for |target edge clear|.

    更易记忆的|target edge clear|的简写。
\end{key}
%
\begin{key}{/tikz/graphs/clear <}
    A more easy-to-remember shorthand for |source edge clear|.

    更易记忆的|source edge clear|的简写。

  \end{key}

These mechanisms make it especially easy to create trees in which the edges are
labeled in some special way:

这些机制使得创建带有特殊标签边的树特别容易:
%
\begin{codeexample}[preamble={\usetikzlibrary{graphs,quotes}}]
\tikz
  \graph [edge quotes={fill=white,inner sep=1pt},
          grow down, branch right] {
    / -> h [>"9"] -> {
      c [>"4" text=red,] -> {
        a [>"2", >thick],
        e [>"0"]
      },
      j [>"7"]
    }
  };
\end{codeexample}


\subsubsection{Placing Node Texts on Incoming Edges\\将节点文本放置在入边上}

Normally, the text of a node is shown (only) inside the node. In some case, for
instance when drawing certain kind of trees, the nodes themselves should not
get any text, but rather the edge leading to the node should be labeled as in
the following example:

通常,节点的文本仅显示在节点内部。在某些情况下,例如绘制某种类型的树时,节点本身不应该有任何文本,而是应该将边引导到节点的边缘应该被标记,如下面的例子所示:%
\begin{codeexample}[preamble={\usetikzlibrary{graphs,quotes}}]
\tikz \graph [empty nodes]
{
  root -> {
    a [>"a"],
    b [>"b"] -> {
      c [>"c"],
      d [>"d"]
    }
  }
};
\end{codeexample}
%
As the example shows, it is a bit cumbersome that we have to label the nodes
and then specify the same text once more using the incoming edge syntax.

正如示例所示,我们必须标记节点然后使用入边语法再次指定相同的文本,这有点繁琐。

For these cases, it would be better if the text of the node where not used with
the node but, rather, be passed directly to the incoming or the outgoing edge.
The following styles do exactly this:

对于这些情况,如果节点的文本不是与节点一起使用,而是直接传递给入边或出边,将会更好。以下样式正是这样做的:

\begin{key}{/tikz/graphs/put node text on incoming edges=\opt{\meta{options}}}
    When this key is used with a node or a group, the following happens:
    
    当将此键与节点或组一起使用时,将发生以下情况:
%
    \begin{enumerate}
        \item The command
            |target edge node={node[|\meta{options}|]{\tikzgraphnodetext}}| is
            executed. This means that all incoming edges of the node get a
            label with the text that would usually be displayed in the node.
            You can use keys like |math nodes| normally.

            执行命令|target edge node={node[|\meta{options}|]{\tikzgraphnodetext}}|。这意味着节点的所有入边都带有通常显示在节点中的文本的标签。您可以像通常一样使用|math nodes|等键。
        \item The command |as={}| is executed. This means that the node itself
            will display nothing.

            执行命令|as={}|。这意味着节点本身不显示任何内容。
    \end{enumerate}
    %
    Here is an example that show how this command is used.
    
    这是一个示例,显示了该命令的使用方式。
\begin{codeexample}[preamble={\usetikzlibrary{graphs}}]
\tikz \graph [put node text on incoming edges,
              math nodes, nodes={circle,draw}]
  { a -> b -> {c, d} };
\end{codeexample}
    %
\end{key}

\begin{key}{/tikz/graphs/put node text on outgoing edges=\opt{\meta{options}}}
    Works like the previous key, only with |target| replaced by |source|.

    与前一个键类似,只是将|target|替换为|source|。
\end{key}


\subsection{Graph Operators, Color Classes, and Graph Expressions\\图运算符、颜色类和图表达式}
\label{section-library-graphs-color-classes}

\tikzname's |graph| command employs a powerful mechanism for adding edges
between nodes and sets of nodes. To a graph theorist, this mechanism may be
known as a \emph{graph expression}: A graph is specified by starting with small
graphs and then applying \emph{operators} to them that form larger graphs and
that connect and recolor colored subsets of the graph's node in different ways.

\tikzname 的|graph|命令采用了一种强大的机制,用于在节点和节点集之间添加边缘。对于图论学家来说,这种机制可能被称为\emph{图表达式}:通过从小图开始并对它们应用形成更大图表的\emph{运算符},以及以不同方式连接和重新着色图表节点的已着色子集来指定图表。


\subsubsection{Color Classes\\颜色类}
\label{section-library-graph-coloring}

\tikzname\ keeps track of a \emph{(multi)coloring} of the graph as it is being
constructed. This does not mean that the actual color of the nodes on the page
will be different, rather, in the following we refer to ``logical'' colors in
the way graph theoreticians do. These ``logical'' colors are only important
while the graph is being constructed and they are ``thrown away'' at the end of
the construction. The actual (``physical'') colors of the nodes are set
independently of these logical colors.

在构建图表时,\tikzname 会跟踪图表的\emph{(多)着色}。这并不意味着页面上节点的实际颜色会有所不同,相反,在接下来的内容中,我们将按照图论学家的方式引用“逻辑”颜色。这些“逻辑”颜色仅在构建图表时重要,并且在构建结束时被“丢弃”。节点的实际(“物理”)颜色是独立于这些逻辑颜色设置的。

As a graph is being constructed, each node can be part of one or more
overlapping \emph{color classes}. So, unlike what is sometimes called a
\emph{legal coloring}, the logical colorings that \tikzname\ keeps track of may
assign multiple colors to the same node and two nodes connected by an edge may
well have the same color.

在构建图表时,每个节点可以是一个或多个重叠的\emph{颜色类}的一部分。因此,与有时称为\emph{合法着色}不同,\tikzname 跟踪的逻辑着色可以为同一节点分配多个颜色,并且通过边缘相连的两个节点可能具有相同的颜色。

Color classes must be declared prior to use. This is done using the following
key:

在使用之前必须声明颜色类。可以使用以下键进行声明:

%
\begin{key}{/tikz/graphs/color class=\meta{color class name}}
    This sets up a new color class called \meta{color class name}. Nodes and
    whole groups of nodes can now be colored with \meta{color class name}. This
    is done using the following keys, which become
    available inside the current scope:

    这将设置一个名为 \meta{color class name} 的新颜色类。现在,节点和整个节点组可以使用 \meta{color class name} 进行着色。可以使用以下键在当前作用域内进行着色:
    %
    \begin{key}{/tikz/graphs/\meta{color class name}}
        This key internally uses the |operator| command to setup an operator
        that will cause all nodes of the current group to get the ``logical
        color'' \meta{color class name}. Nodes retain this color in all
        encompassing scopes, unless it is explicitly changed (see below) or
        unset (again, see below).
        
        此键在内部使用 |operator| 命令设置一个运算符,该运算符将导致当前组的所有节点获得“逻辑颜色” \meta{color class name}。除非显式更改(见下文)或取消设置(同样见下文),节点在所有包围的作用域中保留此颜色。
\begin{codeexample}[preamble={\usetikzlibrary{graphs}}]
\tikz \graph [color class=red] {
  [cycle=red]  % causes all "logically" red nodes to be connected in
               % a cycle
  a,
  b [red],
  { [red] c ->[bend right] d },
  e
};
\end{codeexample}
        %
\begin{codeexample}[preamble={\usetikzlibrary{graphs}}]
\tikz \graph [color class=red, color class=green,
              math nodes, clockwise, n=5] {
  [complete bipartite={red}{green}]
  { [red]   r_1, r_2 },
  { [green] g_1, g_2, g_3 }
};
\end{codeexample}
    \end{key}
    %
    \begin{key}{/tikz/graphs/not \meta{color class name}}
        Sets up an operator for the current scope so that all nodes in it loose
        the color \meta{color class name}. You can also use |!|\meta{color
        class name} as an alias for this key.
        
        设置一个运算符,使当前作用域中的所有节点不再具有颜色 \meta{color class name}。您还可以使用 |!|\meta{color class name} 作为该键的别名。
\begin{codeexample}[preamble={\usetikzlibrary{graphs}}]
\tikz \graph [color class=red, color class=green,
              math nodes, clockwise, n=5] {
  [complete bipartite={red}{green}]
  { [red]   r_1, r_2 },
  { [green] g_1, g_2, g_3 },
  g_2 [not green]
};
\end{codeexample}
    \end{key}
    %
    \begin{key}{/tikz/graphs/recolor \meta{color class name} by=\meta{new color}}
        Causes all keys having color \meta{color class name} to get \meta{new
        color} instead. They loose having color \meta{color class name}, but
        other colors are not affected.
        
        导致具有颜色 \meta{color class name} 的所有键改为使用 \meta{new color}。它们不再具有颜色 \meta{color class name},但其他颜色不受影响。
\begin{codeexample}[preamble={\usetikzlibrary{graphs}}]
\tikz \graph [color class=red, color class=green,
              math nodes, clockwise, n=5] {
  [complete bipartite={red}{green}]
  { [red]   r_1, r_2 },
  { [green] g_1, g_2, g_3 },
  g_2 [recolor green by=red]
};
\end{codeexample}
    \end{key}
\end{key}

The following color classes are available by default:

默认情况下,以下颜色类可用:

\begin{itemize}
\item Color class |all|. Every node is part of this class by default. This
is useful to access all nodes of a (sub)graph, since you can simply
access all nodes of this color class.

颜色类 |all|。默认情况下,每个节点都属于此类。这对于访问(子)图的所有节点非常有用,因为您可以直接访问此颜色类的所有节点。

\item Color classes |source| and |target|. These classes are used to
identify nodes that lead ``into'' a group of nodes and nodes from which
paths should ``leave'' the group. Details on how these colors are
assigned are explained in Section~\ref{section-library-graphs-join}. By
saying |not source| or |not target| with a node, you can influence how
it is connected:

颜色类 |source| 和 |target|。这些类用于标识导致节点组“进入”的节点和路径应从中“离开”组的节点。有关分配这些颜色的详细信息,请参见第~\ref{section-library-graphs-join} 节。通过使用带有节点的 |not source| 或 |not target|,您可以影响它们的连接方式:


\begin{codeexample}[preamble={\usetikzlibrary{graphs}}]
\tikz \graph { a -> { b, c, d } -> e };
\end{codeexample}
        %
\begin{codeexample}[preamble={\usetikzlibrary{graphs}}]
\tikz \graph { a -> { b[not source], c, d[not target] } -> e };
\end{codeexample}
        %
    \item Color classes |source'| and |target'|. These are temporary colors
        that are also explained in Section~\ref{section-library-graphs-join}.

颜色类 |source'| 和 |target'|。这些是临时颜色,也在第~\ref{section-library-graphs-join} 节中有解释。
\end{itemize}


\subsubsection{Graph Operators on Groups of Nodes\\图中节点组的运算符}

Recall that the |graph| command constructs graphs recursively from nested
\meta{group specifications}. Each such \meta{group specification} describes a
subset of the nodes of the final graph. A \emph{graph operator} is an algorithm
that gets the nodes of a group as input and (typically) adds edges between
these nodes in some sensible way. For instance, the |clique| operator will
simply add edges between all nodes of the group.

回想一下,|graph| 命令从嵌套的 \meta{group specifications} 递归地构造图。每个 \meta{group specification} 描述最终图形的一部分节点。图运算符是一种算法,它将组的节点作为输入,并(通常)以某种合理的方式在这些节点之间添加边缘。例如,|clique| 运算符将简单地在组的所有节点之间添加边缘。

\begin{key}{/tikz/graphs/operator=\meta{code}}
    This key has an effect in three places:
    
    此键在三个位置起作用:

%
    \begin{enumerate}
        \item It can be used in the \meta{options} of a \meta{direct node
            specification}.

            可以在 \meta{direct node specification} 的 \meta{options} 中使用。
        \item It can be used in the \meta{options} of a \meta{group
            specification}.

            可以在 \meta{group specification} 的 \meta{options} 中使用。
        \item It can be used in the \meta{options} of an \meta{edge
            specification}.

            可以在 \meta{edge specification} 的 \meta{options} 中使用。


    \end{enumerate}
    %
    The first case is a special case of the second, since it is treated like a
    group specification containing a single node. The last case is more
    complicated and discussed in the next section. So, let us focus on the
    second case.

    第一种情况是第二种情况的特殊情况,因为它被视为包含单个节点的组规范。最后一种情况更加复杂,将在下一节中讨论。因此,让我们专注于第二种情况。



    Even though the \meta{options} of a group are given at the beginning of the
    \meta{group specification}, the \meta{code} is only executed when the group
    has been parsed completely and all its nodes have been identified. If you
    use the |operator| multiple times in the \meta{options}, the effect
    accumulates, that is, all code passed to the different calls of |operator|
    gets executed in the order it is encountered.

    尽管组的 \meta{options} 是在 \meta{group specification} 的开头给出的,但是在解析完整个组并识别出所有节点之后才执行 \meta{code}。如果在 \meta{options} 的多个调用中多次使用 |operator|,则其效果是累积的,也就是说,所有传递给不同的 |operator| 调用的代码将按照遇到的顺序执行。

    The \meta{code} can do ``whatever it wants'', but it will typically add
    edges between certain nodes. You can configure what kind of edges
    (directed, undirected, etc.) are created by using the following keys:

    \meta{code} 可以做“任何它想做的事情”,但通常它会在某些节点之间添加边。您可以使用以下键配置要创建的边的类型(有向、无向等):

    %
    \begin{key}{/tikz/graphs/default edge kind=\meta{value} (initially -\/-)}
        This key stores one of the five edge kinds |--|, |<-|, |->|, |<->|, and
        |-!-|. When an operator wishes to create a new edge, it should
        typically set

        该键存储五种边类型之一:|--|、|<-|、|->|、|<->| 和 |-!-|。当运算符希望创建新的边时,通常应设置
        %
\begin{codeexample}[code only]
\tikzgraphsset{new \pfkeysvalueof{/tikz/graphs/default edge kind}=...}
\end{codeexample}
        %
        While this key can be set explicitly, it may be more convenient to use
        the abbreviating keys listed below. Also, this key is automatically set
        to the current value of \meta{edge specification} when a joining
        operator is called, see the discussion of joining operators in
        Section~\ref{section-library-graphs-join}.

        尽管可以显式设置此键,但使用下面列出的缩写键可能更方便。此外,当调用连接运算符时,此键会自动设置为 \meta{edge specification} 的当前值,请参见第~\ref{section-library-graphs-join} 节中关于连接运算符的讨论。

      \end{key}
    %
    \begin{key}{/tikz/graphs/--}
        Sets the |default edge kind| to |--|.
        
        将 |default edge kind| 设置为 |--|。


\begin{codeexample}[preamble={\usetikzlibrary{graphs.standard}}]
\tikz \graph { subgraph K_n [--, n=5, clockwise, radius=6mm] };
\end{codeexample}
    \end{key}
    %
    \begin{key}{/tikz/graphs/->}
        Sets the |default edge kind| to |->|.
        
        将 |default edge kind| 设置为 |->|。


\begin{codeexample}[preamble={\usetikzlibrary{graphs.standard}}]
\tikz \graph { subgraph K_n [->, n=5, clockwise, radius=6mm] };
\end{codeexample}
    \end{key}
    %
    \begin{key}{/tikz/graphs/<-}
        Sets the |default edge kind| to |<-|.
        
        将 |default edge kind| 设置为 |<-|。


\begin{codeexample}[preamble={\usetikzlibrary{graphs.standard}}]
\tikz \graph { subgraph K_n [<-, n=5, clockwise, radius=6mm] };
\end{codeexample}
    \end{key}
    %
    \begin{key}{/tikz/graphs/<->}
        Sets the |default edge kind| to |<->|.
        
        将 |default edge kind| 设置为 |<->|。


\begin{codeexample}[preamble={\usetikzlibrary{graphs.standard}}]
\tikz \graph { subgraph K_n [<->, n=5, clockwise, radius=6mm] };
\end{codeexample}
    \end{key}
    %
    \begin{key}{/tikz/graphs/-\protect\exclamationmarktext-}
        Sets the |default edge kind| to |-!-|.

        将 |default edge kind| 设置为 |-!-|。

    \end{key}

    When the \meta{code} of an operator is executed, the following commands can
    be used to find the nodes that should be connected:
    
    在执行运算符的 \meta{code} 时,可以使用以下命令找到应连接的节点:
%
    \begin{command}{\tikzgraphforeachcolorednode\marg{color name}\marg{macro}}
        When this command is called inside \meta{code}, the following will
        happen: \tikzname\ will iterate over all nodes inside the
        just-specified group that have the color \meta{color name}. The order
        in which they are iterated over is the order in which they appear
        inside the group specification (if a node is encountered several times
        inside the specification, only the first occurrence counts). Then, for
        each node the \meta{macro} is executed with the node's name as the only
        argument.

        当在 \meta{code} 中调用此命令时,将发生以下情况:\tikzname\ 将遍历刚指定组内具有颜色 \meta{color name} 的所有节点。它们被遍历的顺序是它们在组规范中的出现顺序(如果节点在规范中出现多次,则只计算第一次出现)。然后,对于每个节点,将使用节点的名称作为唯一参数执行 \meta{macro}。


        In the following example we use an operator to connect every node
        colored |all| inside the subgroup to he node |root|.
        
        在下面的示例中,我们使用运算符将子组中着色为 |all| 的每个节点连接到节点 |root|。
%
\begin{codeexample}[preamble={\usetikzlibrary{graphs}}]
\def\myconnect#1{\tikzset{graphs/new ->={root}{#1}{}{}}}

\begin{tikzpicture}
  \node (root) at (-1,-1) {root};

  \graph {
    x,
    {
      [operator=\tikzgraphforeachcolorednode{all}{\myconnect}]
      a, b, c
    }
  };
\end{tikzpicture}
\end{codeexample}
    \end{command}

    \begin{command}{\tikzgraphpreparecolor\marg{color name}\marg{counter}\marg{prefix}}
        This command is used to ``prepare'' the nodes of a certain color for
        random access. The effect is the following: It is counted how many
        nodes there are having color \meta{color name} in the current group and
        the result is stored in \meta{counter}. Next, macros named
        \meta{prefix}|1|, \meta{prefix}|2|, and so on are defined, that store
        the names of the first, second, third, and so on node having the color
        \meta{color name}.

        该命令用于为特定颜色的节点准备随机访问。其效果如下:计算当前组中具有颜色 \meta{color name} 的节点数量,并将结果存储在 \meta{counter} 中。接下来,定义名为 \meta{prefix}|1|、\meta{prefix}|2| 等的宏,它们存储具有颜色 \meta{color name} 的第一个、第二个、第三个等节点的名称。

        The net effect is that after you have prepared a color, you can quickly
        iterate over them. This is especially useful when you iterate over
        several color at the same time.

        效果是,在准备好颜色后,您可以快速迭代它们。当您同时迭代多个颜色时,这尤其有用。

        As an example, let us create an operator then adds a zig-zag path
        between two color classes:
        
        举个例子,让我们创建一个运算符,它在两个颜色类之间添加了一条锯齿路径:
\begin{codeexample}[preamble={\usetikzlibrary{graphs}}]
\newcount\leftshorecount   \newcount\rightshorecount
\newcount\mycount          \newcount\myothercount
\def\zigzag{
  \tikzgraphpreparecolor{left shore}\leftshorecount{left shore prefix}
  \tikzgraphpreparecolor{right shore}\rightshorecount{right shore prefix}
  \mycount=0\relax
  \loop
    \advance\mycount by 1\relax%
    % Add the "forward" edge
    \tikzgraphsset{new ->=
      {\csname left shore prefix\the\mycount\endcsname}
      {\csname right shore prefix\the\mycount\endcsname}{}{}}
    \myothercount=\mycount\relax%
    \advance\myothercount by1\relax%
    \tikzgraphsset{new <-=
      {\csname left shore prefix\the\myothercount\endcsname}
      {\csname right shore prefix\the\mycount\endcsname}{}{}}
  \ifnum\myothercount<\leftshorecount\relax
  \repeat
}
\begin{tikzpicture}
  \graph [color class=left shore, color class=right shore]
  { [operator=\zigzag]
    { [left shore, Cartesian placement]                      a, b, c },
    { [right shore, Cartesian placement, nodes={xshift=1cm}] d, e, f }
  };
\end{tikzpicture}
\end{codeexample}
        %
        Naturally, in order to turn the above code into a usable operator, some
        more code would be needed (like default values and taking care of
        shores of different sizes).

        当然,为了将上述代码转化为可用的运算符,还需要一些代码(如默认值和处理不同大小的边界)。
    \end{command}
\end{key}

There are a number of predefined operators, like |clique| or |cycle|, see the
reference Section~\ref{section-library-graphs-reference} for a complete list.

有许多预定义的运算符,例如 |clique| 或 |cycle|,完整列表请参见参考文献第~\ref{section-library-graphs-reference}节。

\subsubsection{Graph Operators for Joining Groups\\用于连接组的图形运算符}
\label{section-library-graphs-join}

When you join two nodes |foo| and |bar| by the edge specification |->|, it is
fairly obvious, what should happen: An edge from |(foo)| to |(bar)| should be
created. However, suppose we use an edge specification between two node sets
like |{a,b,c}| and |{d,e,f}|. In this case, it is not so clear which edges
should be created. One might argue that all possible edges from any node in the
first set to any node in the second set should be added. On the other hand, one
might also argue that only a matching between these two sets should be created.
Things get even more muddy when a longer chain of node sets are joined.

当您通过边规范 |->| 将两个节点 |foo| 和 |bar| 连接起来时,应该发生的事情是相当明显的:应该创建一条从 |(foo)| 到 |(bar)| 的边。然而,假设我们在两个节点集之间使用边规范,例如 |{a,b,c}| 和 |{d,e,f}|。在这种情况下,不太清楚应该创建哪些边。可以说应该添加从第一个集合中的任何节点到第二个集合中的任何节点的所有可能的边。另一方面,也可以说应该只创建这两个集合之间的匹配。当连接更长的节点集链时,情况变得更加复杂。

Instead of fixing how edges are created between two node sets, \tikzname\ takes
a somewhat more general, but also more complicated approach, which can be
broken into two parts. In the following, assume that the following chain
specification is given:

与其固定如何在两个节点集之间创建边,\tikzname 采取了一种更通用但也更复杂的方法,可以分为两个部分。在下面的内容中,假设给定了以下链规范:
%
\begin{quote}
    \meta{spec$_1$} \meta{edge specification} \meta{spec$_2$}
\end{quote}
%
An example might be |{a,b,c} -> {d, e->f}|.

一个例子可能是 |{a,b,c} -> {d, e->f}|。
\medskip
\textbf{The source and target vertices.} Let us start with the question of
which vertices of the first node set should be connected to vertices in the
second node set.

\textbf{源节点和目标节点。}让我们从第一个节点集的哪些节点应连接到第二个节点集的节点开始讨论。

There are two predefined special color classes that are used for this: |source|
and |target|. For every group specification, some vertices are colored as
|source| vertices and some vertices are |target| vertices (a node can both be a
target and a source). Initially, every vertex is both a source and a target,
but that can change as we will see in a moment.

这里有两个预定义的特殊颜色类用于此目的:|source| 和 |target|。对于每个组规范,一些节点被标记为 |source| 节点,一些节点被标记为 |target| 节点(一个节点既可以是 target 也可以是 source)。最初,每个节点都同时是 source 和 target,但是随着稍后将看到的变化。

The intuition behind source and target vertices is that, in some sense, edges
``from the outside'' lead into the group via the source vertices and lead out
of the group via the target vertices. To be more precise, the following
happens:

源节点和目标节点背后的直觉是,在某种意义上,边缘``从外部''通过源节点进入组,并通过目标节点离开组。更准确地说,发生了以下情况:

%
\begin{enumerate}
    \item The target vertices of the first group are connected to the source
        vertices of the second group.

        第一个组的目标节点与第二个组的源节点连接。


    \item In the group resulting from the union of the nodes from
        \meta{spec$_1$} and \meta{spec$_2$}, the source vertices are only those
        from the first group, and the target vertices are only those from the
        second group.

        在来自 \meta{spec$_1$} 和 \meta{spec$_2$} 的节点的并集组中,源节点只来自第一个组,目标节点只来自第二个组。
\end{enumerate}

Let us go over the effect of these rules for the example
|{a,b,c} -> {d, e->f}|. First, each individual node is initially both a
|source| and a |target| vertex. Then, in |{a,b,c}| all nodes are still both
source and target vertices since just grouping vertices does not change their
colors. Now, in |e->f| something interesting happens for the first time: the
target vertices of the ``group'' |e| (which is just the node |e|) are connected
to the source vertices of the ``group'' |f|. This means, that an edge is added
from |e| to |f|. Then, in the resulting group |e->f| the only source vertex is
|e| and the only target vertex is |f|. This implies that in the group
|{d,e->f}| the sources are |d| and |e| and the targets are |d| and~|f|.

让我们来看一下对于示例 |{a,b,c} -> {d, e->f}| 这些规则的效果。首先,每个单独的节点最初既是 |source| 顶点,也是 |target| 顶点。然后,在 |{a,b,c}| 中,所有节点仍然既是源顶点又是目标顶点,因为仅仅对节点进行分组不会改变它们的颜色。现在,在 |e->f| 中,有一些有趣的事情第一次发生:组'' |e| 的目标顶点(即节点 |e|)与组'' |f| 的源顶点连接。这意味着从 |e| 到 |f| 添加了一条边。然后,在结果组 |e->f| 中,唯一的源顶点是 |e|,唯一的目标顶点是 |f|。这意味着在组 |{d,e->f}| 中,源为 |d| 和 |e|,目标为 |d| 和 |f|。

Now, in |{a,b,c} -> {d,e->f}| the targets  of |{a,b,c}| (which are all three of
them) are connected to the sources of |{d,e->f}| (which are just |d| and~|e|).
Finally, in the whole graph only |a|, |b|, and |c| are sources while only  |d|
and |f| are targets.

现在,在 |{a,b,c} -> {d,e->f}| 中,|{a,b,c}| 的目标(即它们三个)与 |{d,e->f}| 的源(即 |d| 和 |e|)相连。最后,在整个图中,只有 |a|、|b| 和 |c| 是源,而只有 |d| 和 |f| 是目标。

%
\begin{codeexample}[preamble={\usetikzlibrary{graphs}}]
\def\hilightsource#1{\fill [green, opacity=.25] (#1) circle [radius=2mm]; }
\def\hilighttarget#1{\fill [red,   opacity=.25] (#1) circle [radius=2mm]; }
\tikz \graph
  [operator=\tikzgraphforeachcolorednode{source}{\hilightsource},
   operator=\tikzgraphforeachcolorednode{target}{\hilighttarget}]
  { {a,b,c} -> {d, e->f} };
\end{codeexample}

The next objective is to make more precise what it means that ``the targets of
the first graph'' and the ``sources of the second graph'' should be connected.
We know already of a general way of connecting nodes of a graph: operators!
Thus, we use an operator for this job. For instance, the |complete bipartite|
operator adds an edge from every node having a certain color to every node have
a certain other color. This is exactly what we need here: The first color is
``the color |target| restricted to the nodes of the first graph'' and the
second color is ``the color |source| restricted to the nodes of the second
graph''.

下一个目标是更精确地说明第一个图的目标''和第二个图的源''之间应该如何连接。我们已经知道了一种连接图的一般方法:运算符!因此,我们使用一个运算符来完成此任务。例如,|complete bipartite| 运算符会从每个具有特定颜色的节点添加一条边到每个具有另一种特定颜色的节点。这正是我们在这里需要的:第一种颜色是第一个图节点上的 |target| 颜色限制'',第二种颜色是第二个图节点上的 |source| 颜色限制''。

However, we cannot really specify that only nodes from a certain subgraph are
meant -- the |operator| machinery only operates on all nodes of the current
graph. For this reason, what really happens is the following: When the |graph|
command encounters \meta{spec$_1$} \meta{edge specification} \meta{spec$_2$},
it first computes and colors the nodes of the first and the second
specification independently. Then, the |target| nodes of the first graph are
recolored to |target'| and the |source| nodes of the second graph are recolored
to |source'|. Then, the two graphs are united into one graph and a
\emph{joining operator} is executed, which should add edges between |target'|
and |source'|. Once this is done, the colors |target'| and |source'| get
erased. Note that in the resulting graph only the |source| nodes from the first
graph are still |source| nodes and likewise for the |target| nodes of the
second graph.

然而,我们实际上不能指定只有来自某个子图的节点是指定的——|operator| 机制仅对当前图的所有节点起作用。因此,实际发生的情况如下:当 |graph| 命令遇到 \meta{spec$_1$} \meta{edge specification} \meta{spec$_2$} 时,它首先独立计算和着色第一个和第二个规范的节点。然后,第一个图的 |target| 节点被重新着色为 |target'|,第二个图的 |source| 节点被重新着色为 |source'|。然后,将两个图合并为一个图,并执行一个\emph{连接运算符},它应该在 |target'| 和 |source'| 之间添加边缘。完成此操作后,颜色 |target'| 和 |source'| 被擦除。请注意,在结果图中,只有第一个图的 |source| 节点仍然是 |source| 节点,第二个图的 |target| 节点也是如此。

\medskip
\textbf{The joining operators.} The job of a joining operator is to add edges
between nodes colored |target'| and |source'|. The following rule is used to
determine which operator should be chosen for performing this job:

\textbf{连接运算符。}连接运算符的任务是在颜色为 |target'| 和 |source'| 的节点之间添加边缘。使用以下规则来确定应选择哪个运算符来执行此任务:%

\begin{enumerate}
    \item If the \meta{edge specification} explicitly sets the |operator| key
        to something non-empty (and also not to |\relax|), then the \meta{code}
        of this |operator| call is used.

        如果 \meta{edge specification} 显式地将 |operator| 键设置为非空值(且不是 |\relax|),则使用此 |operator| 调用的 \meta{code}。
    \item Otherwise, the current value of the following key is used:
        %

        否则,使用以下键的当前值:
        \begin{key}{/tikz/graphs/default edge operator=\meta{key} (initially matching and star)}
            This key stores the name of a \meta{key} that is executed for every
            \meta{edge specification} whose \meta{options} do not contain the
            |operator| key.
            
            此键存储要执行的 \meta{key} 的名称,该键对于不包含 |operator| 键的 \meta{edge specification} 的每个 \meta{options} 都会执行。
\begin{codeexample}[preamble={\usetikzlibrary{graphs}}]
\tikz \graph [default edge operator=matching] {
  {a, b}    ->[matching and star]
  {c, d, e} --[complete bipartite]
  {f, g, h} --
  {i, j, k}
};
\end{codeexample}
        \end{key}
\end{enumerate}

A typical joining operator is |complete bipartite|. It takes the names of two
color classes as input and adds edges from all vertices of the first class to
all vertices of the second class. Now, the trick is that the default value for
the |complete bipartite| key is |{target'}{source'}|. Thus, if you just write
|->[complete bipartite]|, the same happens as if you had written

典型的连接运算符是|complete bipartite|。它接受两个颜色类的名称作为输入,并将第一类中的所有顶点与第二类中的所有顶点之间添加边。现在,关键字|complete bipartite|的默认值是|{target'}{source'}|。因此,如果您只写|->[complete bipartite]|,则与下面的效果相同:

%
\begin{quote}
    |->[complete bipartite={target'}{source'}]|
\end{quote}
%
This is exactly what we want to happen. The same default values are also set
for other joining operators like |matching| or |butterfly|.

这正是我们想要的效果。对于其他连接运算符如|matching|或|butterfly|,也设置了相同的默认值。

Even though an operator like |complete bipartite| is typically used together
with an edge specification, it can also be used as a normal operator together
with a group specification. In this case, however, the color classes must be
named explicitly:

尽管像|complete bipartite|这样的运算符通常与边规范一起使用,但它也可以作为普通运算符与组规范一起使用。然而,在这种情况下,颜色类必须明确命名:

%
\begin{codeexample}[preamble={\usetikzlibrary{graphs}}]
\begin{tikzpicture}
  \graph [color class=red, color class=green, math nodes]
  { [complete bipartite={red}{green}]
    { [red,   Cartesian placement]                      r_1, r_2, r_3 },
    { [green, Cartesian placement, nodes={xshift=1cm}]  g_1, g_2, g_3 }
  };
\end{tikzpicture}
\end{codeexample}

A list of predefined joining operators can be found in the reference
Section~\ref{section-library-graphs-reference}.

在参考文献第~\ref{section-library-graphs-reference}节中可以找到预定义的连接运算符列表。

The fact that joining operators can also be used as normal operators leads to a
subtle problem: A normal operator will typically use the current value of
|default edge kind| to decide which kind of edges should be put between the
identified vertices, while a joining operator should, naturally, use the kind
of edge specified by the \meta{edge specification}. This problem is solved as
follows: Like a normal operator, a joining operator should also use the current
value of |default edge kind| for the edges it produces. The trick is that this
will automatically be set to the current \meta{edge specification} when the
operator explicitly in the \meta{options} of the edge specification or
implicitly in the |default edge operator|.

连接运算符也可以作为普通运算符使用,这导致了一个微妙的问题:普通运算符通常使用|default edge kind|的当前值来决定应该在识别的顶点之间放置哪种类型的边,而连接运算符自然应该使用由\meta{edge specification}指定的边的类型。这个问题的解决方法如下:与普通运算符一样,连接运算符也应该使用它生成的边的当前|default edge kind|的值。关键在于当运算符在\meta{edge specification}的\meta{options}中显式地或在|default edge operator|中隐式地设置时,它将自动设置为当前的\meta{edge specification}。


\subsection{Graph Macros\\图形宏}
\label{section-library-graphs-macros}

A \emph{graph macro} is a small graph that is inserted at some point into the
graph that is currently being constructed. There is special support for such
graph macros in \tikzname. You might wonder why this is necessary -- can't one
use \TeX's normal macro mechanism? The answer is ``no'': one cannot insert new
nodes into a graph using normal macros because the chains, groups, and nodes
are determined prior to macro expansion. Thus, any macro encountered where some
node text should go will only be expanded when this node is being named and
typeset.

\emph{图形宏}是插入到当前正在构建的图形中的小图形。在\tikzname 中对这样的图形宏有特殊支持。你可能会想知道为什么需要这个——难道不能使用\TeX 的正常宏机制吗?答案是“不行”:使用正常宏无法插入新的节点到图形中,因为链、组和节点是在宏展开之前确定的。因此,当遇到某个节点应该放置的位置有一个宏时,只有在为该节点命名和排版时才会展开该宏。

A graph macro is declared using the following key:

使用以下关键字来声明图形宏:

\begin{key}{/tikz/graphs/declare=\marg{graph name}\marg{specification}}
    This key declares that \meta{graph name} can subsequently be used as a
    replacement for a \meta{node name}. Whenever the \meta{graph name} is used
    in the following, a graph group will be inserted instead whose content is
    exactly \meta{specification}. In case \meta{graph name} is used together
    with some \meta{options}, they are executed prior to inserting the
    \meta{specification}.
    
    此关键字声明\meta{graph name}随后可以用作\meta{node name}的替代。每当在后续中使用\meta{graph name}时,将插入一个图形组,其内容正好为\meta{specification}。如果\meta{graph name}与某些\meta{options}一起使用,则它们将在插入\meta{specification}之前执行。

\begin{codeexample}[preamble={\usetikzlibrary{graphs}}]
\tikz \graph [branch down=4mm, declare={claw}{1 -- {2,3,4}}] {
  a;
  claw;
  b;
};
\end{codeexample}
    %
    In the next example, we use a key to configure a subgraph:
    
    在下一个示例中,我们使用一个关键字来配置一个子图:
\begin{codeexample}[preamble={\usetikzlibrary{graphs}}]
\tikz \graph [ n/.code=\def\n{#1}, branch down=4mm,
               declare={star}{root -- { \foreach \i in {1,...,\n} {\i} }}]
{ star [n=5]; };
\end{codeexample}
    %
    Actually, the |n| key is already defined internally for a similar purpose.

    实际上,关键字|n|已经在内部定义了类似的用途。

    As a last example, let us define a somewhat more complicated graph macro.
    
    作为最后一个示例,让我们定义一个稍微复杂些的图形宏。
\begin{codeexample}[preamble={\usetikzlibrary{graphs}}]
\newcount\mycount
\tikzgraphsset{
  levels/.store in=\tikzgraphlevel,
  levels=1,
  declare={bintree}{%
    [/utils/exec={%
      \ifnum\tikzgraphlevel=1\relax%
        \def\childtrees{ / }%
      \else%
        \mycount=\tikzgraphlevel%
        \advance\mycount by-1\relax%
        \edef\childtrees{
          / -> {
            bintree[levels=\the\mycount],
            bintree[levels=\the\mycount]
          }}
      \fi%
    },
    parse/.expand once=\childtrees
    ]
    % Everything is inside the \childtrees...
  }
}
\tikz \graph [grow down=5mm, branch right=5mm] { bintree [levels=5] };
\end{codeexample}
    %
\end{key}

Note that when you use a graph macro several time inside the same graph, you
will typically have to use the |name| option so that different copies of the
subgraph are created:

请注意,当您在同一图形中多次使用图形宏时,通常需要使用|name|选项,以创建不同的子图副本:

\begin{codeexample}[preamble={\usetikzlibrary{graphs}}]
\tikz \graph [branch down=4mm, declare={claw}{1 -- {2,3,4}}] {
  claw [name=left],
  claw [name=right]
};
\end{codeexample}

You will find a list of useful graph macros in the reference section,
Section~\ref{section-library-graphs-reference-macros}.

您可以在参考文献的第~\ref{section-library-graphs-reference-macros}节中找到一系列有用的图形宏。


\subsection{Online Placement Strategies\\在线放置策略}
\label{section-library-graphs-placement}

The main job of the |graphs| library is to make it easy to specify which nodes
are present in a graph and how they are connected. In contrast, it is
\emph{not} the primary job of the library to compute good positions for nodes
in a graph -- use for instance a |\matrix|, specify good positions ``by hand''
or use the graph drawing facilities. Nevertheless, some basic support for
automatic node placement is provided for simple cases. The |graphs| library
will provide you with information about the position of nodes inside their
groups and chains.

|graphs| 库的主要工作是使得指定图中存在哪些节点以及它们如何连接变得容易。相比之下,该库的主要工作\emph{不是}计算节点在图中的良好位置——可以使用 |\matrix|、手动指定良好位置或使用图形绘制功能。尽管如此,对于简单情况,该库提供了一些基本的自动节点放置支持。|graphs| 库将为您提供有关节点在其组和链中的位置的信息。

As a graph is being constructed, a \emph{placement strategy} is used to
determine a (reasonably good) position for the nodes as they are created. These
placement strategies get some information about what \tikzname\ has already
seen concerning the already constructed nodes, but it gets no information
concerning the upcoming nodes. Because of this lack of information concerning
the future, the strategies need to be what is called an \emph{online strategy}
in computer science. (The opposite are \emph{offline strategies}, which get
information about the whole graph and all the sizes of the nodes in it. The
graph drawing libraries employ such offline strategies.)

在构建图形时,将使用\emph{放置策略}为节点确定(相对良好的)位置,这些节点在创建时逐个添加。这些放置策略会获取一些关于\tikzname\ 已经看到的有关已构建节点的信息,但是它们不会获取有关即将到来的节点的信息。由于缺乏有关未来的信息,这些策略需要在计算机科学中称为\emph{在线策略}。(相反,\emph{离线策略}会获得有关整个图形及其中所有节点大小的信息。图形绘制库使用这些离线策略。)

Strategies are selected using keys like |no placement| or
|Cartesian placement|. It is permissible to use different strategies inside
different parts of a graph, even though the different strategies do not always
work together in perfect harmony.

使用像 |no placement| 或 |Cartesian placement| 这样的键选择策略。可以在图的不同部分使用不同的策略,尽管不同的策略并不总是能够完美协同工作。


\subsubsection{Manual Placement\\手动放置}
\label{section-graphs-xy}

\begin{key}{/tikz/graphs/no placement}
    This strategy simply ``switches off'' the whole placement mechanism,
    causing all nodes to be placed at the origin by default. You need to use
    this strategy if you position nodes ``by hand''. For this, you can use the
    |at| key, the |shift| keys:
    
    此策略只是``关闭''整个放置机制,默认情况下所有节点都会放置在原点。如果您手动定位节点,则需要使用此策略。为此,可以使用 |at| 键、|shift| 键:
\begin{codeexample}[preamble={\usetikzlibrary{graphs}}]
\tikz \graph [no placement]
{
  a[at={(0:0)}] -> b[at={(1,0)}] -> c[yshift=1cm];
};
\end{codeexample}
    %
    Since the syntax and the many braces and parentheses are a bit cumbersome,
    the following two keys might also be useful:
    
    由于语法和许多大括号和圆括号有点繁琐,下面的两个键可能也会有用:

    \begin{key}{/tikz/graphs/x=\meta{x dimension}}
        When you use this key, it will have the same effect as if you had
        written |at={(|\meta{x dimension}|,|\meta{y dimension}|)}|, where
        \meta{y dimension} is a value set using the |y| key:
        
        使用此键时,效果与写入 |at={(|\meta{x dimension}|,|\meta{y dimension}|)}| 相同,其中 \meta{y dimension} 是使用 |y| 键设置的值:
\begin{codeexample}[preamble={\usetikzlibrary{graphs}}]
\tikz \graph [no placement]
{
  a[x=0,y=0] -> b[x=1,y=0] -> c[x=0,y=1];
};
\end{codeexample}
        %
        Note that you can specify an |x| or a |y| key for a whole scope and
        then vary only the other key:
        
        注意,您可以为整个作用域指定 |x| 或 |y| 键,然后仅变化另一个键:
\begin{codeexample}[preamble={\usetikzlibrary{graphs}}]
\tikz \graph [no placement]
{
  a ->
  { [x=1] % group option
    b [y=0] -> c[y=1]
  };
};
\end{codeexample}
        %
        Note that these keys have the path |/tikz/graphs/|, so they will be
        available inside |graph|s and will not clash with the usual |x| and |y|
        keys of \tikzname, which are used to specify the basic lengths of
        vectors.

        注意,这些键具有路径 |/tikz/graphs/|,因此它们将在 |graph| 内部可用,并且不会与\tikzname 的常规 |x| 和 |y| 键冲突,后者用于指定向量的基本长度。
    \end{key}
    %
    \begin{key}{/tikz/graphs/y=\meta{y dimension}}
        See above.见上文。
    \end{key}
\end{key}


\subsubsection{Placement on a Grid\\在网格上的放置}

\begin{key}{/tikz/graphs/Cartesian placement}
    This strategy is the default strategy. It works, roughly, as follows: For
    each new node on a chain, advance a ``logical width'' counter and for each
    new node in a group, advance a ``logical depth'' counter. When a chain
    contains a whole group, then the ``logical width'' taken up by the group is
    the maximum over the logical widths taken up by the chains inside the
    group; and symmetrically the logical depth of a chain is the maximum of the
    depths of the groups inside it.

    这种策略是默认策略。大致工作方式如下:对于链上的每个新节点,增加一个“逻辑宽度”计数器;对于组中的每个新节点,增加一个“逻辑深度”计数器。当一个链包含整个组时,组占据的“逻辑宽度”是组内链占据的逻辑宽度的最大值;类似地,链的逻辑深度是其中组的深度的最大值。

    This slightly confusing explanation is perhaps best exemplified. In the
    below example, the two numbers indicate the two logical width and depth of
    each node as computed by the |graphs| library. Just ignore the arcane code
    that is used to print these numbers.

    这个稍微令人困惑的解释可以通过下面的例子来更好地说明。下面的例子中的两个数字表示每个节点的两个逻辑宽度和深度,这些数字是由|graphs|库计算得出的。忽略用于打印这些数字的奇怪代码。

    %
\begin{codeexample}[preamble={\usetikzlibrary{graphs}}]
\tikz
  \graph [nodes={align=center, inner sep=1pt}, grow right=7mm,
          typeset={\tikzgraphnodetext\\[-4pt]
                   \tiny\mywidth\\[-6pt]\tiny\mydepth},
          placement/compute position/.append code=
            \pgfkeysgetvalue{/tikz/graphs/placement/width}{\mywidth}
            \pgfkeysgetvalue{/tikz/graphs/placement/depth}{\mydepth}]
{
  a,
  b,
  c -> d -> {
    e -> f -> g,
    h -> i
  } -> j,
  k -> l
};
\end{codeexample}
    %
    You will find a detailed description of how these logical units are
    computed, exactly, in Section~\ref{section-library-graphs-new-online}.

    关于如何计算这些逻辑单位的详细描述可以在第\ref{section-library-graphs-new-online}节中找到。

    Now, even though we talk about ``widths'' and ``depths'' and even though by
    default a graph ``grows'' to the right and down, this is by no means fixed.
    Instead, you can use the following keys to change how widths and heights
    are interpreted:
    
    现在,尽管我们谈论“宽度”和“深度”,并且默认情况下图表“向右”和“向下”“增长”,但这决不是固定的。相反,您可以使用以下键来更改如何解释宽度和高度:
%
    \begin{key}{/tikz/graphs/chain shift=\meta{coordinate} (initially {(1,0)})}
        Under the regime of the |Cartesian placement| strategy, each node is
        shifted by the current logical width times this \meta{coordinate}.
        
        在|Cartesian placement|策略下,每个节点都会按当前逻辑宽度乘以此\meta{coordinate}进行移动。%
\begin{codeexample}[preamble={\usetikzlibrary{graphs}}]
\tikz \graph [chain shift=(45:1)] {
  a -> b -> c;
  d -> e;
  f -> g -> h;
};
\end{codeexample}
    \end{key}
    %
    \begin{key}{/tikz/graphs/group shift=\meta{coordinate} (initially {(0,-1)})}
        Like for |chain shift|, each node is shifted by the current logical
        depth times this \meta{coordinate}.
        
        与|chain shift|类似,每个节点都会按当前逻辑深度乘以此\meta{coordinate}进行移动。%
\begin{codeexample}[preamble={\usetikzlibrary{graphs}}]
\tikz \graph [chain shift=(45:7mm), group shift=(-45:7mm)] {
  a -> b -> c;
  d -> e;
  f -> g -> h;
};
\end{codeexample}
    \end{key}
\end{key}

\begin{key}{/tikz/graphs/grow up=\meta{distance} (default 1)}
    Sets the |chain shift| to |(0,|\meta{distance}|)|, so that chains ``grow
    upward''. The distance by which the center of each new element is removed
    from the center of the previous one is \meta{distance}.
    
    将|chain shift|设置为|(0,|\meta{distance}|)|,以便链条“向上增长”。每个新元素的中心与前一个元素的中心之间的距离为\meta{distance}。%
\begin{codeexample}[preamble={\usetikzlibrary{graphs}}]
\tikz \graph [grow up=7mm] { a -> b -> c};
\end{codeexample}
    %
\end{key}

\begin{key}{/tikz/graphs/grow down=\meta{distance} (default 1)}
    Like |grow up|.
    
    与|grow up|类似。
%
\begin{codeexample}[preamble={\usetikzlibrary{graphs}}]
\tikz \graph [grow down=7mm] { a -> b -> c};
\end{codeexample}
    %
\end{key}

\begin{key}{/tikz/graphs/grow left=\meta{distance} (default 1)}
    Like |grow up|. 与|grow up|类似。

    %
\begin{codeexample}[preamble={\usetikzlibrary{graphs}}]
\tikz \graph [grow left=7mm] { a -> b -> c};
\end{codeexample}
    %
\end{key}

\begin{key}{/tikz/graphs/grow right=\meta{distance} (default 1)}
    Like |grow up|.与|grow up|类似。

    %
\begin{codeexample}[preamble={\usetikzlibrary{graphs}}]
\tikz \graph [grow right=7mm] { a -> b -> c};
\end{codeexample}
    %
\end{key}

\begin{key}{/tikz/graphs/branch up=\meta{distance} (default 1)}
    Sets the |group shift| so that groups ``branch upward''.  The distance by
    which the center of each new element is removed from the center of the
    previous one is \meta{distance}.
    
    设置|group shift|,使组“向上分支”。每个新元素的中心与前一个元素的中心之间的距离为\meta{distance}。
%
\begin{codeexample}[preamble={\usetikzlibrary{graphs}}]
\tikz \graph [branch up=7mm] { a -> b -> {c, d, e} };
\end{codeexample}
    %
    Note that when you draw a tree, the |branch ...| keys specify how siblings
    (or adjacent branches) are arranged, while the |grow ...| keys specify in
    which direction the branches ``grow''.

    请注意,在绘制树时,|branch ...|键指定兄弟节点(或相邻分支)的排列方式,而|grow ...|键指定分支“增长”的方向。
  \end{key}

\begin{key}{/tikz/graphs/branch down=\meta{distance} (default 1)}
%
\begin{codeexample}[preamble={\usetikzlibrary{graphs}}]
\tikz \graph [branch down=7mm] { a -> b -> {c, d, e}};
\end{codeexample}
%
\end{key}

\begin{key}{/tikz/graphs/branch left=\meta{distance} (default 1)}
%
\begin{codeexample}[preamble={\usetikzlibrary{graphs}}]
\tikz \graph [branch left=7mm, grow down=7mm] { a -> b -> {c, d, e}};
\end{codeexample}
%
\end{key}

\begin{key}{/tikz/graphs/branch right=\meta{distance} (default 1)}
%
\begin{codeexample}[preamble={\usetikzlibrary{graphs}}]
\tikz \graph [branch right=7mm, grow down=7mm] { a -> b -> {c, d, e}};
\end{codeexample}
%
\end{key}

The following keys place nodes in a $N\times M$ grid.

以下键将节点放置在$N\times M$网格中。
%
\begin{key}{/tikz/graphs/grid placement}
    This key works similar to |Cartesian placement|. As for that placement
    strategy, a node has logical width and depth 1. However, the computed total
    width and depth are mapped to a $N\times M$ grid. The values of $N$ and $M$
    depend on the size of the graph and the value of |wrap after|. The number
    of columns $M$ is either set to |wrap after| explicitly or computed
    automatically as $\sqrt{\texttt{\string|V\string|}}$. $N$ is the number of
    rows needed to lay out the graph in a grid with $M$ columns.
    
    此键与|Cartesian placement|类似。对于该放置策略,节点具有逻辑宽度和深度1。然而,计算得到的总宽度和深度将映射到一个$N\times M$的网格中。$N$和$M$的值取决于图表的大小和|wrap after|的值。列数$M$要么明确设置为|wrap after|,要么根据$\sqrt{\texttt{\string|V\string|}}$自动计算。$N$是在具有$M$列的网格中布局图表所需的行数。%
\begin{codeexample}[preamble={\usetikzlibrary{graphs.standard}}]
% An example with 6 nodes, 3 columns and therefor 2 rows
\tikz \graph [grid placement] { subgraph I_n[n=6, wrap after=3] };
\end{codeexample}
    %
\begin{codeexample}[preamble={\usetikzlibrary{graphs.standard}}]
% An example with 9 nodes with columns and rows computed automatically
\tikz \graph [grid placement] { subgraph Grid_n [n=9] };
\end{codeexample}
    %
\begin{codeexample}[preamble={\usetikzlibrary{graphs.standard}}]
% Directions can be changed
\tikz \graph [grid placement, branch up, grow left] { subgraph Grid_n [n=9] };
\end{codeexample}
    %
    In case a user-defined graph instead of a pre-defined |subgraph| is to be
    laid out using |grid placement|, |n| has to be specified explicitly:
    
    如果要使用|grid placement|布局用户定义的图表而不是预定义的|subgraph|,则必须明确指定|n|:
%
\begin{codeexample}[preamble={\usetikzlibrary{graphs}}]
\tikz \graph [grid placement] {
  [n=6, wrap after=3]
  a -- b -- c -- d -- e -- f
};
\end{codeexample}
    %
\end{key}


\subsubsection{Placement Taking Node Sizes Into Account\\考虑节点大小的放置方式}

Options like |grow up| or |branch right| do not take the sizes of the
to-be-positioned nodes into account -- all nodes are placed quite ``dumbly'' at
grid positions. It turns out that the |Cartesian placement| can also be used to
place nodes in such a way that their height and/or width is taken into account.
Note, however, that while the following options may yield an adequate placement
in many situations, when you need advanced alignments you should use a |matrix|
or advanced offline strategies to place the nodes.

像 |grow up| 或 |branch right| 这样的选项不考虑待定位置的节点的大小 - 所有节点都被相当“愚蠢地”放置在网格位置。然而,事实证明,|Cartesian placement| 也可以用于以考虑其高度和/或宽度的方式放置节点。然而,请注意,虽然下面的选项在许多情况下可能会产生合适的放置,但当您需要高级对齐时,应使用 |matrix| 或高级离线策略来放置节点。

\begin{key}{/tikz/graphs/grow right sep=\meta{distance} (default 1em)}
    This key has several effects, but let us start with the bottom line: Nodes
    along a chain are placed in such a way that the left end of a new node is
    \meta{distance} from the right end of the previous node:
    
    此键具有多个效果,但让我们从最重要的事情开始:链上的节点按照这样的方式放置,即新节点的左端与前一个节点的右端相距 \meta{distance}:

%
\begin{codeexample}[preamble={\usetikzlibrary{graphs}}]
\tikz \graph [grow right sep, left anchor=east, right anchor=west] {
  start -- {
    long text -- {short, very long text} -- more text,
    long -- longer -- longest
  } -- end
};
\end{codeexample}
    %
    What happens internally is the following: First, the |anchor| of the nodes
    is set to |west| (or |north west| or |south west|, see below). Second, the
    logical width of a node is no longer |1|, but set to the actual width of
    the node (which we define as the horizontal difference between the |west|
    anchor and the |east| anchor) in points. Third, the |chain shift| is set to
    |(1pt,0pt)|.

    内部发生的情况如下:首先,节点的 |anchor| 被设置为 |west|(或 |north west| 或 |south west|,见下文)。其次,节点的逻辑宽度不再为 |1|,而是设置为节点的实际宽度(我们将其定义为 |west| 锚点和 |east| 锚点之间的水平差异)以点为单位。第三,|chain shift| 被设置为 |(1pt,0pt)|。

\end{key}

\begin{key}{/tikz/graphs/grow left sep=\meta{distance} (default 1em)}
%
\begin{codeexample}[preamble={\usetikzlibrary{graphs}}]
\tikz \graph [grow left sep] { long -- longer -- longest };
\end{codeexample}
%
\end{key}

\begin{key}{/tikz/graphs/grow up sep=\meta{distance} (default 1em)}
%
\begin{codeexample}[preamble={\usetikzlibrary{graphs}}]
\tikz \graph [grow up sep] {
  a / $a=x$ --
  b / {$b=\displaystyle \int_0^1 x dx$} --
  c [draw, circle, inner sep=7mm]
};
\end{codeexample}
%
\end{key}

\begin{key}{/tikz/graphs/grow down sep=\meta{distance} (default 1em)}
    As above.

    与上面相同。


\end{key}

\begin{key}{/tikz/graphs/branch right sep=\meta{distance} (default 1em)}
    This key works like |grow right sep|, only it affects groups rather than
    chains.
    
    此键的工作方式类似于 |grow right sep|,只是它影响的是组而不是链。
\begin{codeexample}[preamble={\usetikzlibrary{graphs}}]
\tikz \graph [grow down, branch right sep] {
  start -- {
    an even longer text -- {short, very long text} -- more text,
    long -- longer -- longest,
    some text -- a -- b
  } -- end
};
\end{codeexample}
    %
    When both this key and, say, |grow down sep| are set, instead of the |west|
    anchor, the |north west| anchor will be selected automatically.

    当同时设置了此键和,例如,|grow down sep|,将自动选择 |north west| 锚点,而不是 |west| 锚点。
\end{key}

\begin{key}{/tikz/graphs/branch left sep=\meta{distance} (default 1em)}
%
\begin{codeexample}[preamble={\usetikzlibrary{graphs}}]
\tikz \graph [grow down sep, branch left sep] {
  start -- {
    an even longer text -- {short, very long text} -- more text,
    long -- longer,
    some text -- a -- b
  } -- end
};
\end{codeexample}
%
\end{key}

\begin{key}{/tikz/graphs/branch up sep=\meta{distance} (default 1em)}
%
\begin{codeexample}[preamble={\usetikzlibrary{graphs}}]
\tikz \graph [branch up sep] { a, b, c[draw, circle, inner sep=7mm] };
\end{codeexample}
%
\end{key}

\begin{key}{/tikz/graphs/branch down sep=\meta{distance} (default 1em)}
\end{key}


\subsubsection{Placement On a Circle\\在圆上的放置}

The following keys place nodes on circles. Note that, typically, you do not use
|circular placement| directly, but rather use one of the two keys |clockwise|
or |counterclockwise|.

以下键将节点放置在圆上。请注意,通常不直接使用 |circular placement|,而是使用 |clockwise| 或 |counterclockwise| 之一。

\begin{key}{/tikz/graphs/circular placement}
    This key works quite similar to |Cartesian placement|. As for that
    placement strategy, a node has logical width and depth |1|. However, the
    computed total width and depth are mapped to polar coordinates rather than
    Cartesian coordinates.

    此键的工作方式与 |Cartesian placement| 非常相似。对于该放置策略,节点具有逻辑宽度和深度 |1|。然而,计算得到的总宽度和深度被映射到极坐标而不是笛卡尔坐标。

    \begin{key}{/tikz/graphs/chain polar shift=|(|\meta{angle}|:|\meta{radius}|)| (initially {(0:1)})}
        Under the regime of the |circular placement| strategy, each node on a
        chain is shifted by
        |(|\meta{logical width}\meta{angle}|:|\meta{logical width}\meta{angle}|)|.
        %

        在 |circular placement| 策略下,链上的每个节点都会通过 |(|\meta{logical width}\meta{angle}|:|\meta{logical width}\meta{angle}|)| 进行平移。
\begin{codeexample}[preamble={\usetikzlibrary{graphs}}]
\tikz \graph [circular placement] {
  a -> b -> c;
  d -> e;
  f ->  g -> h;
};
\end{codeexample}
        %
    \end{key}
    %
    \begin{key}{/tikz/graphs/group polar shift=|(|\meta{angle}|:|\meta{radius}|)| (initially {(45:0)})}
        Like for |group shift|, each node on a chain is shifted by
        |(|\meta{logical depth}\meta{angle}|:|\meta{logical depth}\meta{angle}|)|.
        %

        与 |group shift| 类似,链上的每个节点都会通过 |(|\meta{logical depth}\meta{angle}|:|\meta{logical depth}\meta{angle}|)| 进行平移。
\begin{codeexample}[preamble={\usetikzlibrary{graphs}}]
\tikz \graph [circular placement, group polar shift=(30:0)] {
  a -> b -> c;
  d -> e;
  f -> g -> h;
};
\end{codeexample}
        %
\begin{codeexample}[preamble={\usetikzlibrary{graphs}}]
\tikz \graph [circular placement,
              chain polar shift=(30:0),
              group polar shift=(0:1cm)] {
  a -- b -- c;
  d -- e;
  f -- g -- h;
};
\end{codeexample}
    \end{key}
    %
    \begin{key}{/tikz/graphs/radius=\meta{dimension} (initially 1cm)}
        This is an initial value that is added to the total computed radius
        when the polar shift of a node has been calculated. Essentially, this
        key allows you to set the \meta{radius} of the innermost circle.
        
        这是一个初始值,当计算出一个节点的极坐标平移时,它将添加到总计算半径中。本质上,该键允许您设置最内圈的 \meta{radius}。%
\begin{codeexample}[preamble={\usetikzlibrary{graphs}}]
\tikz \graph [circular placement, radius=5mm] { a, b, c, d };
\end{codeexample}
        %
\begin{codeexample}[preamble={\usetikzlibrary{graphs}}]
\tikz \graph [circular placement, radius=1cm] { a, b, c, d };
\end{codeexample}
    \end{key}
    %
    \begin{key}{/tikz/graphs/phase=\meta{angle} (initially 90)}
        This is an initial value that is added to the total computed angle when
        the polar shift of a node has been calculated.
        
        这是一个初始值,当计算出一个节点的极坐标平移时,它将添加到总计算角度中。
\begin{codeexample}[preamble={\usetikzlibrary{graphs}}]
\tikz \graph [circular placement] { a, b, c, d };
\end{codeexample}
        %
\begin{codeexample}[preamble={\usetikzlibrary{graphs}}]
\tikz \graph [circular placement, phase=0] { a, b, c, d };
\end{codeexample}
    \end{key}
\end{key}

\label{key-graphs-clockwise}%
\begin{key}{/tikz/graphs/clockwise=\meta{number} (default \string\tikzgraphVnum)}
    This key sets the |group shift| so that if there are exactly \meta{number}
    many nodes in a group, they will form a complete circle. If you do not
    provide a \meta{number}, the current value of |\tikzgraphVnum| is used,
    which is exactly what you want when you use predefined graph macros like
    |subgraph K_n|.
    
    此键设置 |group shift|,以便如果组中恰好有 \meta{number} 个节点,它们将形成一个完整的圆。如果不提供 \meta{number},将使用 |\tikzgraphVnum| 的当前值,当您使用预定义的图宏如 |subgraph K_n| 时,这正是您想要的。
\begin{codeexample}[preamble={\usetikzlibrary{graphs}}]
\tikz \graph [clockwise=4] { a, b, c, d };
\end{codeexample}
    %
\begin{codeexample}[preamble={\usetikzlibrary{graphs.standard}}]
\tikz \graph [clockwise] { subgraph K_n [n=5] };
\end{codeexample}
    %
\end{key}

\label{key-graphs-counterclockwise}%
\begin{key}{/tikz/graphs/counterclockwise=\meta{number} (default \string\tikzgraphVnum)}
    Works like |clockwise|, only the direction is inverted.

    与 |clockwise| 类似,只是方向相反。
\end{key}


\subsubsection{Levels and Level Styles\\层级和层级样式}

As a graph is being parsed, the |graph| command keeps track of a parameter
called the \emph{level} of a node. Provided that the graph is actually
constructed in a tree-like manner, the level is exactly equal to the level of
the node inside this tree.

在解析图形时,|graph| 命令会跟踪一个参数,称为节点的\emph{层级}。只要图形实际上是按照类似树状结构的方式构建的,层级就等于节点在这棵树中的层级。

\begin{key}{/tikz/graphs/placement/level}
    This key stores a number that is increased for each element on a chain, but
    gets reset at the end of a group:
    
    该键存储一个数字,该数字在链上的每个元素上递增,但在组的末尾重置:%
\begin{codeexample}[preamble={\usetikzlibrary{graphs}}]
\tikz \graph [ branch down=5mm, typeset=
    \tikzgraphnodetext:\pgfkeysvalueof{/tikz/graphs/placement/level}]
{
  a -> {
    b,
    c -> {
      d,
      e -> {f,g},
      h
    },
    j
  }
};
\end{codeexample}
    %
    Unlike the parameters |depth| and |width| described in the next section,
    the key |level| is always available.

    与下一节中描述的 |depth| 和 |width| 参数不同,|level| 键始终可用。

\end{key}

In addition to keeping track of the value of the |level| key, the |graph|
command also executes the following keys whenever it creates a node:

除了跟踪 |level| 键的值之外,|graph| 命令还在创建节点时执行以下键:

\begin{stylekey}{/tikz/graph/level=\meta{level}}
    This key gets executed for each newly created node with \meta{level} set to
    the current level of the node. You can use this key to, say, reconfigure
    the node distance or the node color.

    该键在每个新创建的节点上执行,其中的 \meta{level} 设置为节点的当前层级。您可以使用此键来重新配置节点间距或节点颜色。

  \end{stylekey}

\begin{stylekey}{/tikz/graph/level \meta{level}}
    This key also gets executed for each newly created node with \meta{level}
    set to the current level of the node.
     
    该键也在每个新创建的节点上执行,其中的 \meta{level} 设置为节点的当前层级。

    \begin{codeexample}[preamble={\usetikzlibrary{graphs}}]
\tikz \graph [
  branch down=5mm,
  level 1/.style={nodes=red},
  level 2/.style={nodes=green!50!black},
  level 3/.style={nodes=blue}]
{
  a -> {
    b,
    c -> {
      d,
      e -> {f,g},
      h
    },
    j
  }
};
\end{codeexample}
    %
\begin{codeexample}[preamble={\usetikzlibrary{graphs}}]
\tikz \graph [
  branch down=5mm,
  level 1/.style={grow right=2cm},
  level 2/.style={grow right=1cm},
  level 3/.style={grow right=5mm}]
{
  a -> {
    b,
    c -> {
      d,
      e -> {f,g},
      h
    },
    j
  }
};
\end{codeexample}
    %
\end{stylekey}


\subsubsection{Defining New Online Placement Strategies\\定义新的在线放置策略}
\label{section-library-graphs-new-online}

In the following the details of how to define a new placement strategy are
explained. Most readers may wish to skip this section.

下面详细介绍了如何定义新的放置策略。大多数读者可能希望跳过本节。

As a graph specification is being parsed, the |graphs| library will keep track
of different numbers that identify the positions of the nodes. Let us start
with what happens on a chain. First, the following counter is increased for
each element of the chain:

在解析图形规范时,|graphs| 库会跟踪标识节点位置的不同数字。让我们从链表的情况开始。首先,对链表的每个元素递增以下计数器:

%
\begin{key}{/tikz/graphs/placement/element count}
    This key stores a number that tells us the position of the node on the
    current chain. However, you only have access to this value inside the code
    passed to the macro |compute position|, explained later on.
     
    该键存储一个数字,告诉我们节点在当前链表中的位置。但是,只能在传递给宏 |compute position| 的代码中访问此值,稍后会解释。

    \begin{codeexample}[preamble={\usetikzlibrary{graphs}}]
\tikz \graph [
  grow right sep, typeset=\tikzgraphnodetext:\mynum,
  placement/compute position/.append code=
    \pgfkeysgetvalue{/tikz/graphs/placement/element count}{\mynum}]
{
  a -> b -> c,
  d -> {e, f->h} -> j
};
\end{codeexample}
    %
    As can be seen, each group resets the element counter.

    如上所示,每个组都会重置元素计数器。

  \end{key}

The second value that is computed is more complicated to explain, but it also
gives more interesting information:

计算的第二个值更难以解释,但也提供了更有趣的信息:

\begin{key}{/tikz/graphs/placement/width}
    This key stores the ``logical width'' of the nodes parsed up to now in the
    current group or chain (more precisely, parsed since the last call of
    |place| in an enclosing group). This is not necessarily the ``total
    physical width'' of the nodes, but rather a number representing how ``big''
    the elements prior to the current element were. This \emph{may} be their
    width, but it may also be their height or even their number (which,
    incidentally, is the default). You can use the |width| to perform shifts or
    rotations of to-be-created nodes (to be explained later).

    该键存储到目前为止在当前组或链表中解析的节点的“逻辑宽度”(更准确地说,是自最后在外部组中的 |place| 调用以来解析的节点的宽度)。这不一定是节点的“总物理宽度”,而是表示先前元素有多“大”的一个数字。这\emph{可能}是它们的宽度,但也可能是它们的高度甚至是它们的数量(顺便说一句,默认情况下是数量)。您可以使用 |width| 执行节点的平移或旋转(稍后将解释)。

    The logical width is defined recursively as follows. First, the width of a
    single node is computed by calling the following key:

    逻辑宽度的定义如下。首先,通过调用以下键计算单个节点的宽度:

    %
    \begin{key}{/tikz/graphs/placement/logical node width=\meta{full node name}}
        This key is called to compute a physical or logical width of the node
        \meta{full node name}. You can change the code of this key. The code
        should return the computed value in the macro |\pgfmathresult|. By
        default, this key returns |1|.

        通过调用此键来计算节点 \meta{full node name} 的物理或逻辑宽度。您可以更改此键的代码。代码应在宏 |\pgfmathresult| 中返回计算的值。默认情况下,此键返回 |1|。

      \end{key}
    %
    The width of a chain is the sum of the widths of its elements. The width of
    a group is the maximum of the widths of its elements.

    链的宽度是其元素宽度的总和。组的宽度是其元素宽度的最大值。

    To get a feeling what the above rules imply in practice, let us first have
    a look at an example where each node has logical width and height |1|
    (which is the default). The arcane options at the beginning of the code
    just setup things so that the computed width and depth of each node is
    displayed at the bottom of each node.

    为了对上述规则在实践中的含义有所了解,让我们首先看一个示例,其中每个节点的逻辑宽度和高度为 |1|(这是默认值)。代码开头的奇怪选项只是设置了每个节点的计算宽度和深度,并在每个节点底部显示出来。

    %
\begin{codeexample}[preamble={\usetikzlibrary{graphs}}]
\tikz
  \graph [nodes={align=center, inner sep=1pt}, grow right=7mm,
          typeset={\tikzgraphnodetext\\[-4pt]
                   \tiny\mywidth\\[-6pt]\tiny\mydepth},
          placement/compute position/.append code=
            \pgfkeysgetvalue{/tikz/graphs/placement/width}{\mywidth}
            \pgfkeysgetvalue{/tikz/graphs/placement/depth}{\mydepth}]
{
  a,
  b,
  c -> d -> {
    e -> f -> g,
    h -> i
  } -> j,
  k -> l
};
\end{codeexample}
    %
    In the next example the ``logical'' width and depth actually match the
    ``physical'' width and height. This is caused by the |grow right sep|
    option, which internally sets the |logical node width| key so that it
    returns the width of its parameter in points.
     
    在下一个示例中,“逻辑”宽度和深度实际上与“物理”宽度和高度相匹配。这是由于 |grow right sep| 选项,该选项在内部设置了 |logical node width| 键,以便以点为单位返回其参数的宽度。

    \begin{codeexample}[preamble={\usetikzlibrary{graphs}}]
\tikz
  \graph [grow right sep, branch down sep, nodes={align=left, inner sep=1pt},
          typeset={\tikzgraphnodetext\\[-4pt] \tiny Width: \mywidth\\[-6pt] \tiny Depth: \mydepth},
          placement/compute position/.append code=
            \pgfkeysgetvalue{/tikz/graphs/placement/width}{\mywidth}
            \pgfkeysgetvalue{/tikz/graphs/placement/depth}{\mydepth}]
{
  a,
  b,
  c -> d -> {
    e -> f -> g,
    h -> i
  } -> j,
  k -> l
};
\end{codeexample}
    %
\end{key}

Symmetrically to chains, as a group is being constructed, counters are
available for the number of chains encountered so far in the current group and
for the logical depth of the current group:

与链类似,当构建组时,可以使用计数器来计算当前组中到目前为止遇到的链的数量以及当前组的逻辑深度:


\begin{key}{/tikz/graphs/placement/chain count}
    This key stores a number that tells us the sequence number of the
    chain in the current group.
    
    该键存储一个数字,告诉我们当前组中链的序号。
\begin{codeexample}[preamble={\usetikzlibrary{graphs}}]
\tikz \graph [
  grow right sep, branch down=5mm, typeset=\tikzgraphnodetext:\mynum,
  placement/compute position/.append code=
    \pgfkeysgetvalue{/tikz/graphs/placement/chain count}{\mynum}]
{
  a -> b -> {c,d,e},
  f,
  g -> h
};
\end{codeexample}
    %
\end{key}

\begin{key}{/tikz/graphs/placement/depth}
    Similarly to the |width| key, this key stores the ``logical depth'' of the
    nodes parsed up to now in the current group or chain and, also similarly,
    this key may or may not be related to the actual depth/height of the
    current node. As for the |width|, the exact definition is as follows: For a
    single node, the depth is computed by the following key:
    
    与 |width| 键类似,该键存储到目前为止在当前组或链中解析的节点的“逻辑深度”,并且与当前节点的实际深度/高度可能或可能不相关。与 |width| 一样,确切的定义如下:对于单个节点,深度由以下键计算:    
    \begin{key}{/tikz/graphs/placement/logical node depth=\meta{full node name}}
        The code behind this key should return the ``logical height'' of the
        node \meta{full node name} in the macro |\pgfmathresult|.

        该键的代码应返回宏 |\pgfmathresult| 中的节点 \meta{full node name} 的“逻辑高度”。
      \end{key}
    %
    Second, the depth of a group is the sum of the depths of its elements.
    Third, the depth of a chain is the maximum of the depth of its elements.

    其次,组的深度是其元素深度的总和。第三,链的深度是其元素深度的最大值。
\end{key}

The |width|, |depth|, |element count|, and |chain count| keys get updated
automatically, but do not have an effect by themselves. This is to the
following two keys:

|width|、|depth|、|element count| 和 |chain count| 键会自动更新,但本身不会产生效果。这是由以下两个键控制的:
\begin{key}{/tikz/graphs/placement/compute position=\meta{code}}
    The \meta{code} is called by the |graph| command just prior to creating a
    new node (the exact moment when this key is called is detailed in the
    description of the |place| key). When the \meta{code} is called, all of the
    keys described above will hold numbers computed in the way described above.


    \meta{code} 在创建新节点之前由 |graph| 命令调用(调用此键的确切时刻在 |place| 键的描述中详细说明)。当调用 \meta{code} 时,上述描述的所有键都将保存以上述方式计算的数字。



    The job of the \meta{code} is to setup node options appropriately so that
    the to-be-created node will be placed correctly. Thus, the \meta{code}
    should typically set the key |nodes={shift=|\meta{coordinate}|}| where
    \meta{coordinate} is the computed position for the node. The \meta{code}
    could also set other options like, say, the color of a node depending on
    its depth.

    \meta{code} 的任务是适当地设置节点选项,以便将要创建的节点被正确放置。因此,\meta{code} 通常应设置键 |nodes={shift=|\meta{coordinate}|}|,其中 \meta{coordinate} 是节点的计算位置。\meta{code} 还可以设置其他选项,例如根据节点的深度设置节点的颜色。



    The following example appends some code to the standard code of
    |compute position| so that ``deeper'' nodes of a tree are lighter.
    (Naturally, the same effect could be achieved much more easily using the
    |level| key.)
    
    以下示例在 |compute position| 的标准代码后附加了一些代码,以便树的“更深”节点更亮。 (当然,使用 |level| 键可以更轻松地实现相同的效果。)%
\begin{codeexample}[preamble={\usetikzlibrary{graphs}}]
\newcount\mycount
\def\lightendeepernodes{
  \pgfmathsetcount{\mycount}{
    100-20*\pgfkeysvalueof{/tikz/graphs/placement/width}
  }
  \edef\mydepth{\the\mycount}
  \tikzset{nodes={fill=red!\mydepth,circle,text=white}}
}
\tikz
  \graph [placement/compute position/.append code=\lightendeepernodes]
   {
     a -> {
       b -> c -> d,
       e -> {
         f,
         g
       },
       h
     }
   };
\end{codeexample}
    %
\end{key}

\begin{key}{/tikz/graphs/placement/place}
    Executing this key has two effects: First, the key |compute position| is
    called to compute a good position for future nodes (usually, these ``future
    nodes'' are just a single node that is created immediately). Second, all of
    the above counters like |depth| or |width| are reset (but not |level|).

    执行此键有两个效果:首先,调用键 |compute position| 来计算未来节点的良好位置(通常,这些“未来节点”只是立即创建的单个节点)。其次,重置所有上述计数器,如 |depth| 或 |width|(但不包括 |level|)。


    There are two places where this key is sensibly called: First, just prior
    to creating a node, which happens automatically. Second, when you change
    the online strategy. In this case, the computed width and depth values from
    one strategy typically make no sense in the other strategy, which is why
    the new strategy should proceed ``from a fresh start''. In this case, the
    implicit call of |compute position| ensures that the new strategy gets the
    last place the old strategy would have used as its starting point, while
    the computation of its positions is now relative to this new starting
    point.

    在两个情况下,这个键被合理地调用:首先,在创建节点之前,这是自动发生的。其次,在更改在线策略时。在这种情况下,从一个策略计算出的宽度和深度值通常在另一个策略中没有意义,这就是为什么新策略应该从“全新的起点”开始进行的原因。在这种情况下,隐式调用 |compute position| 确保新策略获得旧策略将作为其起点使用的最后位置,而其位置的计算现在是相对于这个新起点的。

    For these reasons, when an online strategy like |Cartesian placement| is
    called, this key gets called implicitly. You will rarely need to call this
    key directly, except when you define a new online strategy.

    出于这些原因,当调用像 |Cartesian placement| 这样的在线策略时,这个键会被隐式调用。除非您定义了一个新的在线策略,否则您很少需要直接调用此键。


\end{key}


\subsection{Reference: Predefined Elements\\参考:预定义元素}
\label{section-library-graphs-reference}

\subsubsection{Graph Macros\\图形宏}
\label{section-library-graphs-reference-macros}

\begin{tikzlibrary}{graphs.standard}
    This library defines a number of graph macros that are often used in the
    literature. When new graphs are added to this collection, they will follow
    the definitions in the Mathematica program, see
    \url{mathworld.wolfram.com/topics/SimpleGraphs.html}.

    这个库定义了一些在文献中经常使用的图形宏。当向该集合添加新的图形时,它们将遵循Mathematica程序中的定义,请参见\url{mathworld.wolfram.com/topics/SimpleGraphs.html}。
\end{tikzlibrary}

\begin{graph}{subgraph I\_n}
    This graph consists just of $n$ unconnected vertices. The following key is
    used to specify the set of these vertices:
    
    该图仅由 $n$ 个不相连的顶点组成。以下键用于指定这些顶点的集合:

    \begin{key}{/tikz/graphs/V=\marg{list of vertices}}
        Sets a list of vertex names for use with graphs like |subgraph I_n| and
        also other graphs. This list is available in the macro |\tikzgraphV|.
        The number of elements of this list is available in |\tikzgraphVnum|.

        设置一个顶点名称列表,用于类似于 |subgraph I_n| 和其他图形的图。此列表可在宏 |\tikzgraphV| 中使用。该列表的元素数量可在 |\tikzgraphVnum| 中使用。


    \end{key}
    %
    \begin{key}{/tikz/graphs/n=\meta{number}}
        This is an abbreviation for

        这是以下缩写形式:

        |V={1,...,|\meta{number}|}, name shore V/.style={name=V}|.
    \end{key}
    %
\begin{codeexample}[preamble={\usetikzlibrary{graphs.standard}}]
\tikz \graph [branch right, nodes={draw, circle}]
  { subgraph I_n [V={a,b,c}] };
\end{codeexample}
    %
    This graph is not particularly exciting by itself. However, it is often
    used to introduce nodes into a graph that are then connected as in the
    following example:
    
    这个图本身并不特别令人兴奋。然而,它经常用于将节点引入图中,然后按照以下示例进行连接:
\begin{codeexample}[preamble={\usetikzlibrary{graphs.standard}}]
\tikz \graph [clockwise, clique] { subgraph I_n [n=4] };
\end{codeexample}
    %
\end{graph}

\begin{graph}{subgraph I\_nm}
    This graph consists of two sets of once $n$ unconnected vertices and then
    $m$ unconnected vertices. The first set consists of the vertices set by the
    key |V|, the other set consists of the vertices set by the key |W|.
    
    该图由两组顶点组成,一开始是 $n$ 个不相连的顶点,然后是 $m$ 个不相连的顶点。第一组由键 |V| 设置的顶点组成,另一组由键 |W| 设置的顶点组成。

\begin{codeexample}[preamble={\usetikzlibrary{graphs.standard}}]
\tikz \graph { subgraph I_nm [V={1,2,3}, W={a,b,c}] };
\end{codeexample}
    %
    In order to set the graph path name of the two sets, the following keys get
    executed:
    
    为了设置这两组的图路径名称,将执行以下键:
    \begin{stylekey}{/tikz/graphs/name shore V (initially \normalfont empty)}
        Set this style to, say, |name=my V set| in order to set a name for the
        |V| set.

        将此样式设置为,例如,|name=my V set|,以便为 |V| 组设置一个名称。

    \end{stylekey}
    %
    \begin{stylekey}{/tikz/graphs/name shore W (initially \normalfont empty)}
        Same as for |name shore V|.

        与 |name shore V| 相同。
    \end{stylekey}
    %
    \begin{key}{/tikz/graphs/W=\marg{list of vertices}}
        Sets the list of vertices for the |W| set. The elements and their
        number are available in the macros |\tikzgraphW| and |\tikzgraphWnum|,
        respectively.

        设置 |W| 组的顶点列表。元素及其数量可分别在宏 |\tikzgraphW| 和 |\tikzgraphWnum| 中使用。
    \end{key}
    %
    \begin{key}{/tikz/graphs/m=\meta{number}}
        This is an abbreviation for

        这是以下缩写形式:

        |W={1,...,|\meta{number}|}, name shore W/.style={name=W}|.
    \end{key}
    %
    The main purpose of this subgraph is to setup the nodes in a bipartite
    graph:

    这个子图的主要目的是设置二分图中的节点:

    %
\begin{codeexample}[preamble={\usetikzlibrary{graphs.standard}}]
\tikz \graph {
  subgraph I_nm [n=3, m=4];

  V 1 -- { W 2, W 3 };
  V 2 -- { W 1, W 3 };
  V 3 -- { W 1, W 4 };
};
\end{codeexample}
    %
\end{graph}

\begin{graph}{subgraph K\_n}
    This graph is the complete clique on the vertices from the |V| key.
    
    该图是由键 |V| 中的顶点构成的完全团。
\begin{codeexample}[preamble={\usetikzlibrary{graphs.standard}}]
\tikz \graph [clockwise] { subgraph K_n [n=7] };
\end{codeexample}
    %
\end{graph}

\begin{graph}{subgraph K\_nm}
    This graph is the complete bipartite graph with the two shores |V| and |W|
    as in |subgraph I_nm|.
    
    该图是具有两个岸 |V| 和 |W| 的完全二分图,就像 |subgraph I_nm| 中一样。
\begin{codeexample}[preamble={\usetikzlibrary{graphs.standard}}]
\tikz \graph [branch right, grow down]
  { subgraph K_nm [V={6,...,9}, W={b,...,e}] };
\end{codeexample}
    %
\begin{codeexample}[preamble={\usetikzlibrary{graphs.standard}}]
\tikz \graph [simple, branch right, grow down]
{
  subgraph K_nm [V={1,2,3}, W={a,b,c,d}, ->];
  subgraph K_nm [V={2,3},   W={b,c},     <-];
};
\end{codeexample}
    %
\end{graph}

\begin{graph}{subgraph P\_n}
    This graph is the path on the vertices in |V|.
    
    该图是在 |V| 中的顶点上的路径图。
\begin{codeexample}[preamble={\usetikzlibrary{graphs.standard}}]
\tikz \graph [branch right] { subgraph P_n [n=3] };
\end{codeexample}
    %
\end{graph}

\begin{graph}{subgraph C\_n}
    This graph is the cycle on the vertices in |V|.
    
    该图是在 |V| 中的顶点上的环图。
\begin{codeexample}[preamble={\usetikzlibrary{graphs.standard}}]
\tikz \graph [clockwise] { subgraph C_n [n=7, ->] };
\end{codeexample}
    %
\end{graph}

\begin{graph}{subgraph Grid\_n}
    This graph is a grid of the vertices in |V|.

    该图是在 |V| 中的顶点上的网格图。
    %
    \begin{key}{/tikz/graphs/wrap after=\meta{number}}
        Defines the number of nodes placed in a single row of the grid. This
        value implicitly defines the number of grid columns as well. In the
        following example a |grid placement| is used to visualize the edges
        created between the nodes of a |Grid_n| |subgraph| using different
        values for |wrap after|.

        定义在网格中放置的单行节点数。这个值隐式地定义了网格的列数。在下面的示例中,使用 |grid placement| 来可视化使用不同的 |wrap after| 值创建的 |Grid_n| |subgraph| 节点之间的边。
        %
\begin{codeexample}[preamble={\usetikzlibrary{graphs.standard}}]
\tikz \graph [grid placement] { subgraph Grid_n [n=3,wrap after=1] };
\tikz \graph [grid placement] { subgraph Grid_n [n=3,wrap after=3] };
\end{codeexample}
        %
\begin{codeexample}[preamble={\usetikzlibrary{graphs.standard}}]
\tikz \graph [grid placement] { subgraph Grid_n [n=4,wrap after=2] };
\tikz \graph [grid placement] { subgraph Grid_n [n=4] };
\end{codeexample}
  \end{key}
\end{graph}

% TODO: Implement the Grid_nm subgraph described here: 待办事项:实现此处描述的 Grid_nm 子图:
%
%\begin{graph}{subgraph Grid\_nm}
%  This graph is a grid built from the cartesian product of the two node
%  sets |V| and |W| which are either defined using the keys
%  |/tikz/graphs/V| and |/tikz/graphs/W| or |/tikz/graphs/n| and
%  |/tikz/graphs/m| or a mixture of both.
%
%该图形是由两个节点集合 |V| 和 |W| 的笛卡尔积构成的网格,可以使用关键字 |tikz/graphs/V| 和 |/tikz/graphs/W| 或者 |/tikz/graphs/n| 和 |/tikz/graphs/m| 或两者混合来定义这两个节点集合。
%
%  The resulting |Grid_nm| subgraph has $n$ ``rows'' and $m$ ``columns'' and
%  the nodes are named |V i W j| with $1\le i\le n$ and $1\le j\le n$.
%  The names of the two shores |V| and |W| can be changed as described in
%  the documentation of the keys |/tikz/graphs/name shore V| and
%  |/tikz/graphs/name shore W|.
%
%得到的 |Grid_nm| 子图具有 $n$ 个“行”和 $m$ 个“列”,节点的名称为 |V i W j|,其中 $1\le i\le n$ 且 $1\le j\le n$。
%两个边界 |V| 和 |W| 的名称可以根据关键字 |/tikz/graphs/name shore V| 和 |/tikz/graphs/name shore W| 中的文档描述进行更改。
%  \begin{codeexample}[preamble={\usetikzlibrary{graphs}}]
%\tikz \graph [grid placement] { subgraph Grid_nm [V={1,2,3}, W={4, 5, 6}] };
%  \end{codeexample}
%\end{graph}


\subsubsection{Group Operators\\组运算符}

The following keys use the |operator| key to setup operators that connect the
vertices of the current group having a certain color in a specific way.

以下关键字使用 |operator| 关键字来设置以特定方式连接当前组中具有特定颜色的顶点的运算符。

\begin{key}{/tikz/graphs/clique=\meta{color} (default all)}
    Adds an edge between all vertices of the current group having the (logical)
    color \meta{color}. Since, by default, this color is set to |all|, which is
    a color that all nodes get by default, when you do not specify anything,
    all nodes will be connected.
    
    在当前组中具有(逻辑上的)颜色 \meta{color} 的所有顶点之间添加边。由于默认情况下,此颜色设置为 |all|,它是所有节点默认获得的颜色,因此当您没有指定任何内容时,将连接所有节点。
\begin{codeexample}[preamble={\usetikzlibrary{graphs}}]
\tikz \graph [clockwise, n=5] {
  a,
  b,
  {
    [clique]
    c, d, e
  }
};
\end{codeexample}
    %
\begin{codeexample}[preamble={\usetikzlibrary{graphs}}]
\tikz \graph [color class=red, clockwise, n=5] {
  [clique=red, ->]
  a, b[red], c[red], d, e[red]
};
\end{codeexample}
    %
\end{key}

\begin{key}{/tikz/graphs/induced independent set=\meta{color} (default all)}
    This key is the ``opposite'' of a |clique|: It removes all edges in the
    current group having belonging to color class \meta{color}. More precisely,
    an edge of kind |-!-| is added for each pair of vertices. This means that
    edge only get removed if you specify the |simple| option.
    
    此关键字是 |clique| 的“相反”:它删除当前组中属于颜色类别 \meta{color} 的所有边。更准确地说,对于每对顶点,将添加一种为 |-!-| 的边。这意味着只有在指定了 |simple| 选项时,边才会被删除。
\begin{codeexample}[preamble={\usetikzlibrary{graphs.standard}}]
\tikz \graph [simple] {
  subgraph K_n [<->, n=7, clockwise]; % create lots of edges

  { [induced independent set] 1, 3, 4, 5, 6 }
};
\end{codeexample}
    %
\end{key}

\begin{key}{/tikz/graphs/cycle=\meta{color} (default all)}
    Connects the nodes colored \meta{color} is a cyclic fashion. The ordering
    is the ordering in which they appear in the whole graph specification.
    
    以循环方式连接具有颜色 \meta{color} 的节点。顺序是它们在整个图形规范中出现的顺序。
\begin{codeexample}[preamble={\usetikzlibrary{graphs}}]
\tikz \graph [clockwise, n=6, phase=60] {
  { [cycle, ->] a, b, c },
  { [cycle, <-] d, e, f }
};
\end{codeexample}
    %
\end{key}

\begin{key}{/tikz/graphs/induced cycle=\meta{color} (default all)}
    While the |cycle| command will only add edges, this key will also remove
    all other edges between the nodes of the cycle, provided we are
    constructing a |simple| graph.
    
    虽然 |cycle| 命令只会添加边,但此关键字也会删除构成循环的节点之间的所有其他边,前提是我们正在构造一个 |simple| 图。
\begin{codeexample}[preamble={\usetikzlibrary{graphs.standard}}]
\tikz \graph [simple] {
  subgraph K_n [n=7, clockwise]; % create lots of edges

  { [induced cycle, ->, edge=red] 2, 3, 4, 6, 7 },
};
\end{codeexample}
    %
\end{key}

\begin{key}{/tikz/graphs/path=\meta{color} (default all)}
    Works like |cycle|, only there is no edge from the last to the first
    vertex.
    
    与 |cycle| 相同,只是最后一个顶点没有指向第一个顶点的边。
\begin{codeexample}[preamble={\usetikzlibrary{graphs}}]
\tikz \graph [clockwise, n=6] {
  { [path, ->] a, b, c },
  { [path, <-] d, e, f }
};
\end{codeexample}
    %
\end{key}

\begin{key}{/tikz/graphs/induced path=\meta{color} (default all)}
    Works like |induced cycle|, only there is no edge from the last to the
    first vertex.
    
    与 |induced cycle| 相同,只是最后一个顶点没有指向第一个顶点的边。
\begin{codeexample}[preamble={\usetikzlibrary{graphs.standard}}]
\tikz \graph [simple] {
  subgraph K_n [n=7, clockwise]; % create lots of edges

  { [induced path, ->, edges=red] 2, 3, 4, 6, 7 },
};
\end{codeexample}
    %
\end{key}


\subsubsection{Joining Operators\\连接运算符}

The following keys are typically used as options of an \meta{edge
specification}, but can also be called in a group specification (however, then,
the colors need to be set explicitly).

以下键通常用作 \meta{edge specification} 的选项,但也可以在组规范中调用(但是,此时颜色需要显式设置)。

\begin{key}{/tikz/graphs/complete bipartite=\meta{from color}\meta{to color} (default \char`\{source'\char`\}\char`\{target'\char`\})}
    Adds all possible edges from every node having color \meta{from color} to
    every node having color \meta{to color}:

    添加所有从颜色为 \meta{from color} 的节点到颜色为 \meta{to color} 的节点的可能边:    %
\begin{codeexample}[preamble={\usetikzlibrary{graphs}}]
\tikz \graph { {a, b}       ->[complete bipartite]
               {c, d, e}    --[complete bipartite]
               {g, h, i, j} --[complete bipartite]
               k };
\end{codeexample}
    %
\begin{codeexample}[preamble={\usetikzlibrary{graphs}}]
\tikz \graph [color class=red, color class=green, clockwise, n=6] {
  [complete bipartite={red}{green}, ->]
  a [red], b[red], c[red], d[green], e[green], f[green]
};
\end{codeexample}
    %
\end{key}

\begin{key}{/tikz/graphs/induced complete bipartite}
    Works like the |complete bipartite| operator, but in a |simple| graph any
    edges between the vertices in either shore are removed (more precisely,
    they get replaced by |-!-| edges).
    
    与 |complete bipartite| 连接符类似,但在 |simple| 图中,任何两个分区中的顶点之间的边都会被移除(更准确地说,它们会被 |-!-| 边替代)。
\begin{codeexample}[preamble={\usetikzlibrary{graphs.standard}}]
\tikz \graph [simple] {
  subgraph K_n [n=5, clockwise];  % Lots of edges

  {2, 3} ->[induced complete bipartite] {4, 5}
};
\end{codeexample}
    %
\end{key}

\begin{key}{/tikz/graphs/matching=\meta{from color}\meta{to color} (default \char`\{source'\char`\}\char`\{target'\char`\})}
    This joining operator forms a maximum \emph{matching} between the nodes of
    the two sets of nodes having colors \meta{from color} and \meta{to color},
    respectively. The first node of the from set is connected to the first node
    of to set, the second node of the from set is connected to the second node
    of the to set, and so on. If the sets have the same size, what results is
    what graph theoreticians call a \emph{perfect matching}, otherwise only a
    maximum, but not perfect matching results.
    
    此连接符在具有颜色为 \meta{from color} 和 \meta{to color} 的两组节点之间形成最大\emph{匹配}。来自组的第一个节点连接到目标组的第一个节点,来自组的第二个节点连接到目标组的第二个节点,依此类推。如果这两组的大小相同,结果就是图论中称为\emph{完美匹配}的情况,否则只会得到最大匹配,但不是完美匹配。
\begin{codeexample}[preamble={\usetikzlibrary{graphs}}]
\tikz \graph {
  {a, b, c} ->[matching]
  {d, e, f} --[matching]
  {g, h}    --[matching]
  {i, j, k}
};
\end{codeexample}
    %
\end{key}

\begin{key}{/tikz/graphs/matching and star=\meta{from color}\meta{to color} (default \char`\{source'\char`\}\char`\{target'\char`\})}
    The |matching and star| connector works like the |matching| connector, only
    it behaves differently when the two to-be-connected sets have different
    size. In this case, all the surplus nodes get connected to the last node of
    the other set, resulting in what is known as a \emph{star} in graph theory.
    This simple rule allows for some powerful effects (since this connector is
    the one initially set, there is no need to add it here):
    
    |matching and star| 连接符与 |matching| 连接符类似,只是当要连接的两个集合大小不同时,它的行为会有所不同。在这种情况下,所有多余的节点都连接到另一个集合的最后一个节点,从而形成图论中所称的\emph{星形图}。这个简单的规则可以实现一些强大的效果(由于此连接符最初设置,因此无需在此处添加它):
\begin{codeexample}[preamble={\usetikzlibrary{graphs}}]
\tikz \graph { a -> {b, c} -> {d, e} -- f};
\end{codeexample}
    %
    The |matching and star| connector also makes it easy to create trees and
    series-parallel graphs.

    |matching and star| 连接符还使得创建树和串行-并行图变得容易。
\end{key}

\begin{key}{/tikz/graphs/butterfly=\opt{\meta{options}}}
    The |butterfly| connector is used to create the kind of connections present
    between layers of a so-called \emph{butterfly network}. As for other
    connectors, two sets of nodes are connected, which are the nodes having
    color |target'| and |source'| by default. In a \emph{level $l$} connection,
    the first $l$ nodes of the first set are connected to the second $l$ nodes
    of the second set, while the second $l$ nodes of the first set get
    connected to the first $l$ nodes of the second set. Then, for next $2l$
    nodes of both sets a similar kind of connection is installed. Additionally,
    each node gets connected to the corresponding node in the other set with
    the same index (as in a |matching|):
    
    |butterfly| 连接符用于创建所谓的\emph{蝶形网络}中层之间存在的连接。与其他连接符一样,连接了两组节点,这两组节点默认情况下是具有 |target'| 和 |source'| 颜色的节点。在\emph{第 $l$ 层}连接中,第一组的前 $l$ 个节点连接到第二组的后 $l$ 个节点,而第一组的后 $l$ 个节点连接到第二组的前 $l$ 个节点。然后,对于两组的下一个 $2l$ 个节点,安装类似类型的连接。此外,每个节点都与另一组的相应节点以相同的索引进行连接(如 |matching|):
\begin{codeexample}[preamble={\usetikzlibrary{graphs.standard}}]
\tikz \graph [left anchor=east, right anchor=west,
              branch down=4mm, grow right=15mm] {
  subgraph I_n [n=12, name=A] --[butterfly={level=3}]
  subgraph I_n [n=12, name=B] --[butterfly={level=2}]
  subgraph I_n [n=12, name=C]
};
\end{codeexample}
    %
    Unlike most joining operators, the colors of the nodes in the first and the
    second set are not passed as parameters to the |butterfly| key. Rather,
    they can be set using the \meta{options}, which are executed with the path
    prefix |/tikz/graphs/butterfly|.
    
    与大多数连接运算符不同,第一组和第二组中节点的颜色不作为参数传递给 |butterfly| 键。相反,可以使用 \meta{options} 设置它们,这些选项在路径前缀 |/tikz/graphs/butterfly| 下执行。
    \begin{key}{/tikz/graphs/butterfly/level=\meta{level} (initially 1)}
        Sets the level $l$ for the connections.

        设置连接的层级 $l$。

    \end{key}
    %
    \begin{key}{/tikz/graphs/butterfly/from=\meta{color} (initially target')}
        Sets the color class of the from nodes.

        设置源节点的颜色类。


    \end{key}
    %
    \begin{key}{/tikz/graphs/butterfly/to=\meta{color} (initially source')}
        Sets the color class of the to nodes.

        设置目标节点的颜色类。


    \end{key}
\end{key}


%%% Local Variables:
%%% mode: latex
%%% TeX-master: "pgfmanual-pdftex-version"
%%% End:
