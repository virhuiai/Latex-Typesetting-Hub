\setcounter{section}{21}
\setcounter{subsection}{6}
\setcounter{subsubsection}{0}
% Copyright 2019 by Till Tantau
%
% This file may be distributed and/or modified
%
% 1. under the LaTeX Project Public License and/or
% 2. under the GNU Free Documentation License.
%
% See the file doc/generic/pgf/licenses/LICENSE for more details.


\section{Plots of Functions\\函数的绘图}
\label{section-tikz-plots}

A warning before we get started: \emph{If you are looking for an easy way to
create a normal plot of a function with scientific axes, ignore this section
and instead look at the |pgfplots| package or at the |datavisualization|
command from Part~\ref{part-dv}.}

在我们开始之前,先提个警告:\emph{如果您想要一种简单的方法来创建具有科学坐标轴的函数图形,请忽略本节,而是查看 |pgfplots| 宏包或第~\ref{part-dv}~部分的 |datavisualization| 命令。}

\subsection{Overview\\概述}
\label{section-why-pgname-for-plots}

\tikzname\ can be used to create plots of functions, a job that is normally
handled by powerful programs like \textsc{gnuplot} or \textsc{mathematica}.
These programs can produce two different kinds of output: First, they can
output a complete plot picture in a certain format (like \pdf) that includes
all low-level commands necessary for drawing the complete plot (including axes
and labels). Second, they can usually also produce ``just plain data'' in the
form of a long list of coordinates. Most of the powerful programs consider it a
to be ``a bit boring'' to just output tabled data and very much prefer to
produce fancy pictures. Nevertheless, when coaxed, they can also provide the
plain data.

\tikzname\ 可以用于创建函数的图形,这通常由强大的程序(如 \textsc{gnuplot} 或 \textsc{mathematica})处理。这些程序可以生成两种不同类型的输出:首先,它们可以输出以某种格式(如 \pdf)表示的完整绘图图像,该图像包含绘制完整图形所需的所有低级命令(包括坐标轴和标签)。其次,它们通常还可以以“纯粹的数据”形式输出长列表的坐标。大多数强大的程序认为仅输出表格数据有点“无聊”,而且更喜欢生成花哨的图片。尽管如此,一旦被激活,它们也可以提供纯粹的数据。

The advantage of creating plots directly using \tikzname\ is
\emph{consistency:} Plots created using \tikzname\ will automatically have the
same styling and fonts as those used in the rest of a document -- something
that is hard to do right when an external program gets involved. Other problems
people encounter with external programs include: Formulas will look different,
if they can be rendered at all; line widths will usually be too thick or too
thin; scaling effects upon inclusion can create a mismatch between sizes in the
plot and sizes in the text; the automatic grid generated by most programs is
mostly distracting; the automatic ticks generated by most programs are cryptic
numerics (try adding a tick reading ``$\pi$'' at the right point); most
programs make it very easy to create ``chart junk'' in a most convenient
fashion; arrows and plot marks will almost never match the arrows used in the
rest of the document. This list is not exhaustive, unfortunately.

直接使用 \tikzname\ 创建绘图的优势在于\emph{一致性}:使用 \tikzname\ 创建的绘图将自动具有与文档中其他部分相同的样式和字体,而这在涉及外部程序时很难做到正确。人们在使用外部程序时遇到的其他问题包括:公式的外观将有所不同,如果可以渲染的话;线条宽度通常过粗或过细;包含时的缩放效果可能导致绘图和文本中的大小不匹配;大多数程序生成的自动网格通常会分散注意力;大多数程序生成的自动刻度是令人费解的数字(尝试在正确的位置添加一个标记为“$\pi$”的刻度);大多数程序非常容易以最方便的方式创建“图表垃圾”;箭头和绘图标记几乎永远与文档中使用的箭头不匹配。遗憾的是,这个列表并不详尽。

There are basically three ways of creating plots using \tikzname:

基本上有三种使用 \tikzname\ 创建绘图的方法:


\begin{enumerate}
    \item Use the |plot| path operation. How this works is explained in the
        present section. This is the most ``basic'' of the three options and
        forces you to do a lot of things ``by hand'' like adding axes or ticks.

        使用 |plot| 路径操作。在本节中将介绍其工作原理。这是三种选项中最“基本”的一种,强制您手动完成许多事情,如添加坐标轴或刻度。

        \item Use the |datavisualization| path command, which is documented in
        Part~\ref{part-dv}. This command is much more powerful than the |plot|
        path operation and produces complete plots including axes and ticks.
        The downside is that you cannot use it to ``just'' quickly plot a
        simple curve (or, more precisely, it is hard to use it in this way).

        使用 |datavisualization| 路径命令,该命令在第~\ref{part-dv}~部分进行了文档化。这个命令比 |plot| 路径操作更强大,可以生成包括坐标轴和刻度在内的完整绘图。缺点是您不能仅仅“快速”地绘制一个简单的曲线(或更准确地说,很难以这种方式使用它)。

        \item Use the |pgfplots| package, which is basically an alternative to the
        |datavisualization| command. While the underlying philosophy of this
        package is not as ``ambitious'' as that of the command
        |datavisualization|, it is somewhat more mature, has a simpler design,
        and wider support base.

        使用 |pgfplots| 宏包,它基本上是 |datavisualization| 命令的替代品。尽管该宏包的底层理念不如 |datavisualization| 命令“雄心勃勃”,但它更成熟一些,设计更简单,支持更广泛。

    \end{enumerate}


\subsection{The Plot Path Operation\\绘图路径操作}

The |plot| path operation can be used to append a line or curve to the path
that goes through a large number of coordinates. These coordinates are either
given in a simple list of coordinates, read from some file, or they are
computed on the fly.

|plot| 路径操作可用于在路径中添加穿过大量坐标的线条或曲线。这些坐标可以是简单的坐标列表,从文件中读取,或者是动态计算的。

The syntax of the |plot| comes in different versions.

|plot|的语法有不同的版本。

\begin{pathoperation}{--plot}{\meta{further arguments}}
    This operation plots the curve through the coordinates specified in the
    \meta{further arguments}. The current (sub)path is simply continued, that
    is, a line-to operation to the first point of the curve is implicitly
    added. The details of the \meta{further arguments} will be explained in a
    moment.

    此操作通过\meta{further arguments}中指定的坐标绘制曲线。当前的(子)路径会简单地延续,也就是说,会隐式添加到曲线的第一个点的线操作。稍后将解释\meta{further arguments}的详细信息。

\end{pathoperation}

\begin{pathoperation}{plot}{\meta{further arguments}}
    This operation plots the curve through the coordinates specified in the
    \meta{further arguments} by first ``moving'' to the first coordinate of the
    curve.

    此操作通过首先“移动”到曲线的第一个坐标来绘制曲线。

\end{pathoperation}

The \meta{further arguments} are used in different ways to specifying the
coordinates of the points to be plotted:

\meta{further arguments}以不同的方式用于指定要绘制的点的坐标:

%
\begin{enumerate}
    \item \opt{|--|}|plot|\oarg{local
        options}\declare{|coordinates{|\meta{coordinate 1}\meta{coordinate
        2}\dots\meta{coordinate $n$}|}|}
    \item \opt{|--|}|plot|\oarg{local
        options}\declare{|file{|\meta{filename}|}|}
    \item \opt{|--|}|plot|\oarg{local options}\declare{\meta{coordinate
        expression}}
    \item \opt{|--|}|plot|\oarg{local options}\declare{|function{|\meta{gnuplot
        formula}|}|}
\end{enumerate}

These different ways are explained in the following.

这些不同的方式在下面进行了解释。


\subsection{Plotting Points Given Inline\\给定内联的绘制点}

Points can be given directly in the \TeX-file as in the following example:

可以直接在\TeX 文件中给出点,如下例所示:

\begin{codeexample}[]
\tikz \draw plot coordinates {(0,0) (1,1) (2,0) (3,1) (2,1) (10:2cm)};
\end{codeexample}

Here is an example showing the difference between |plot| and |--plot|:

以下是显示 |plot| 和 |--plot| 之间差异的示例:

\begin{codeexample}[]
\begin{tikzpicture}
  \draw (0,0) -- (1,1) plot coordinates {(2,0)  (4,0)};
  \draw[color=red,xshift=5cm]
        (0,0) -- (1,1) -- plot coordinates {(2,0)  (4,0)};
\end{tikzpicture}
\end{codeexample}


\subsection{Plotting Points Read From an External File\\从外部文件中读取绘制点}

The second way of specifying points is to put them in an external file named
\meta{filename}. Currently, the only file format that \tikzname\ allows is the
following: Each line of the \meta{filename} should contain one line starting
with two numbers, separated by a space. A line may also be empty or, if it
starts with |#| or |%| it is considered empty. For such lines, a ``new data
set'' is started, typically resulting in a new subpath being started in the
plot (see Section~\ref{section-plot-jumps} on how to change this behavior, if
necessary). For lines containing two numbers, they must be separated by a
space. They may be following by arbitrary text, which is ignored, \emph{except}
if it is |o| or |u|. In the first case, the point is considered to be an
\emph{outlier} and normally also results in a new subpath being started. In the
second case, the point is considered to be \emph{undefined}, which also results
in a new subpath being started. Again, see Section~\ref{section-plot-jumps} on
how to change this, if necessary. (This is exactly the format that
\textsc{gnuplot} produces when you say |set table|.)

第二种指定点的方法是将它们放在名为 \meta{filename} 的外部文件中。目前,\tikzname\ 允许的唯一文件格式如下:\meta{filename} 的每一行应该以两个用空格分隔的数字开始。一行也可以是空行,或者如果以 |#| 或 |%|%
开头,则被视为空行。对于这些行,通常会开始一个“新的数据集”,这通常会导致绘图中开始一个新的子路径(如果有必要,可以参见第~\ref{section-plot-jumps}节以更改此行为)。对于包含两个数字的行,它们必须以空格分隔。它们可以后面跟着任意文本,该文本会被忽略,\emph{但是}如果它是 |o| 或 |u|,则除外。在第一种情况下,该点被视为\emph{异常值},通常也会导致开始一个新的子路径。在第二种情况下,该点被视为\emph{未定义},这也会导致开始一个新的子路径。如果有必要,可以参见第\ref{section-plot-jumps}~节以更改此行为(这正是当您使用 |set table| 时 \textsc{gnuplot} 生成的格式)。


%
\begin{codeexample}[]
\tikz \draw plot[mark=x,smooth] file {plots/pgfmanual-sine.table};
\end{codeexample}

The file |plots/pgfmanual-sine.table| reads:

文件 |plots/pgfmanual-sine.table| 的内容如下:

\begin{codeexample}[code only]
#Curve 0, 20 points
#x y type
0.00000 0.00000  i
0.52632 0.50235  i
1.05263 0.86873  i
1.57895 0.99997  i
...
9.47368 -0.04889  i
10.00000 -0.54402  i
\end{codeexample}
%
It was produced from the following source, using |gnuplot|:

它是从以下源代码中使用 |gnuplot| 生成的:


\begin{codeexample}[code only]
set table  "../plots/pgfmanual-sine.table"
set format "%.5f"
set samples 20
plot [x=0:10] sin(x)
\end{codeexample}

The \meta{local options} of the |plot| operation are local to each plot and do
not affect other plots ``on the same path''. For example, |plot[yshift=1cm]|
will locally shift the plot 1cm upward. Remember, however, that most options
can only be applied to paths as a whole. For example, |plot[red]| does not have
the effect of making the plot red. After all, you are trying to ``locally''
make part of the path red, which is not possible.

|plot|操作的\meta{local options}只对每个绘制的图形局部有效,并不影响其他“在同一路径上”的绘制。例如,|plot[yshift=1cm]|会将绘图局部向上移动1cm。但是请记住,大多数选项只能应用于整个路径。例如,|plot[red]|并不能使绘图变为红色。毕竟,您试图“局部地”使路径的一部分变为红色,这是不可能的。


\subsection{Plotting a Function\\绘制函数}
\label{section-tikz-plot}

When you plot a function, the coordinates of the plot data can be computed by
evaluating a mathematical expression. Since \pgfname\ comes with a mathematical
engine, you can specify this expression and then have \tikzname\ produce the
desired coordinates for you, automatically.

在绘制函数时,绘图数据的坐标可以通过计算数学表达式来获得。由于\pgfname\ 附带了一个数学引擎,您可以指定这个表达式,然后让\tikzname\ 自动为您生成所需的坐标。

Since this case is quite common when plotting a function, the syntax is easy:
Following the |plot| command and its local options, you directly provide a
\meta{coordinate expression}. It looks like a normal coordinate, but inside you
may use a special macro, which is |\x| by default, but this can be changed
using the |variable| option. The \meta{coordinate expression} is then evaluated
for different values for |\x| and the resulting coordinates are plotted.

由于在绘制函数时这种情况非常常见,所以语法很简单:在|plot|命令及其本地选项之后,直接提供一个\meta{坐标表达式}。它看起来像一个普通的坐标,但在内部可以使用一个特殊的宏,该宏默认为|\x|,但可以使用|variable|选项进行更改。然后,对于不同的|\x|值计算并绘制结果坐标。

Note that you will often have to put the $x$- or $y$-coordinate inside braces,
namely whenever you use an expression involving a parenthesis.

请注意,通常需要将$x$-或$y$-坐标放在花括号中,即在使用带有括号的表达式时。

The following options influence how the \meta{coordinate expression} is
evaluated:

以下选项会影响如何计算\meta{坐标表达式}:


\begin{key}{/tikz/variable=\meta{macro} (initially \string\x)}
    Sets the macro whose value is set to the different values when
    \meta{coordinate expression} is evaluated.

    设置在计算\meta{坐标表达式}时\meta{宏}的值不同。

\end{key}

\begin{key}{/tikz/samples=\meta{number} (initially 25)}
    Sets the number of samples used in the plot.

    设置绘图中使用的样本数。

\end{key}

\begin{key}{/tikz/domain=\meta{start}|:|\meta{end} (initially -5:5)}
    Sets the domain from which the samples are taken.

    设置从中获取样本的定义域。

\end{key}

\begin{key}{/tikz/samples at=\meta{sample list}}
    This option specifies a list of positions for which the variable should be
    evaluated. For instance, you can say |samples at={1,2,8,9,10}| to have the
    variable evaluated exactly for values $1$, $2$, $8$, $9$, and $10$. You can
    use the |\foreach| syntax, so you can use |...| inside the \meta{sample
    list}.

    此选项指定要计算变量的位置列表。例如,您可以使用|samples at={1,2,8,9,10}|,以便准确地计算变量的值为$1$、$2$、$8$、$9$和$10$。您可以使用|\foreach|语法,因此可以在\meta{样本列表}中使用|...|。

    When this option is used, the |samples| and |domain| option are overruled.
    The other way round, setting either |samples| or |domain| will overrule
    this option.

    当使用此选项时,|samples|和|domain|选项将被覆盖。反之,设置|samples|或|domain|将覆盖此选项。

\end{key}
%
\begin{codeexample}[]
\begin{tikzpicture}[domain=0:4]
  \draw[very thin,color=gray] (-0.1,-1.1) grid (3.9,3.9);

  \draw[->] (-0.2,0) -- (4.2,0) node[right] {$x$};
  \draw[->] (0,-1.2) -- (0,4.2) node[above] {$f(x)$};

  \draw[color=red]    plot (\x,\x)             node[right] {$f(x) =x$};
  % \x r means to convert '\x' from degrees to _r_adians:
  \draw[color=blue]   plot (\x,{sin(\x r)})    node[right] {$f(x) = \sin x$};
  \draw[color=orange] plot (\x,{0.05*exp(\x)}) node[right] {$f(x) = \frac{1}{20} \mathrm e^x$};
\end{tikzpicture}
\end{codeexample}

\begin{codeexample}[]
\tikz \draw[scale=0.5,domain=-3.141:3.141,smooth,variable=\t]
  plot ({\t*sin(\t r)},{\t*cos(\t r)});
\end{codeexample}

\begin{codeexample}[]
\tikz \draw[domain=0:360,smooth,variable=\t]
  plot ({sin(\t)},\t/360,{cos(\t)});
\end{codeexample}


\subsection{Plotting a Function Using Gnuplot\\使用Gnuplot绘制函数}
\label{section-tikz-gnuplot}

Often, you will want to plot points that are given via a function like $f(x) =
x \sin x$. Unfortunately, \TeX\ does not really have enough computational power
to generate the points of such a function efficiently (it is a text processing
program, after all). However, if you allow it, \TeX\ can try to call external
programs that can easily produce the necessary points. Currently, \tikzname\
knows how to call \textsc{gnuplot}.

通常,您会希望绘制通过函数给出的点,例如$f(x) = x \sin x$。不幸的是,\TeX\ 实际上没有足够的计算能力来高效地生成这种函数的点(毕竟它是一个文本处理程序)。但是,如果允许,\TeX\ 可以尝试调用能够轻松生成所需点的外部程序。目前,\tikzname\ 知道如何调用\textsc{gnuplot}。

When \tikzname\ encounters your operation
|plot[id=|\meta{id}|] function{x*sin(x)}| for the first time, it will create a
file called \meta{prefix}\meta{id}|.gnuplot|, where \meta{prefix} is
|\jobname.| by default, that is, the name of your main |.tex| file. If no
\meta{id} is given, it will be empty, which is alright, but it is better when
each plot has a unique \meta{id} for reasons explained in a moment. Next,
\tikzname\ writes some initialization code into this file followed by
|plot x*sin(x)|. The initialization code sets up things such that the |plot|
operation will write the coordinates into another file called
\meta{prefix}\meta{id}|.table|. Finally, this table file is read as if you had
said |plot file{|\meta{prefix}\meta{id}|.table}|.

当\tikzname\ 首次遇到形如|plot[id=|\meta{id}|] function{xsin(x)}|的操作时,它将创建一个名为\meta{prefix}\meta{id}|.gnuplot|的文件,其中\meta{prefix}默认为|\jobname.|,即您主要的|.tex|文件的名称。如果没有提供\meta{id},它将为空,这是可以的,但是为了稍后的原因,每个绘图都有一个唯一的\meta{id}更好。接下来,\tikzname\ 将一些初始化代码写入此文件,后面是|plot xsin(x)|。初始化代码设置了一些使|plot|操作将坐标写入另一个名为\meta{prefix}\meta{id}|.table|的文件的内容。最后,将像使用|plot file{|\meta{prefix}\meta{id}|.table}|一样读取此表文件。

For the plotting mechanism to work, two conditions must be met:

为使绘图机制起作用,必须满足两个条件:

%
\begin{enumerate}
    \item You must have allowed \TeX\ to call external programs. This is often
        switched off by default since this is a security risk (you might,
        without knowing, run a \TeX\ file that calls all sorts of ``bad''
        commands). To enable this ``calling external programs'' a command line
        option must be given to the \TeX\ program. Usually, it is called
        something like |shell-escape| or |enable-write18|. For example, for my
        |pdflatex| the option |--shell-escape| can be given.

        您必须允许\TeX 调用外部程序。由于这是一种安全风险(您可能在不知情的情况下运行调用各种“不良”命令的\TeX 文件),因此通常默认情况下会禁用此功能。要启用此“调用外部程序”,必须向\TeX 程序提供一个命令行选项。通常,它被称为|shell-escape|或|enable-write18|。例如,对于我的|pdflatex|,可以给出选项|--shell-escape|。

        \item You must have installed the |gnuplot| program and \TeX\ must find it
        when compiling your file.

        您必须安装|gnuplot|程序,并且在编译文件时\TeX 必须找到它。

    \end{enumerate}

Unfortunately, these conditions will not always be met. Especially if you pass
some source to a coauthor and the coauthor does not have \textsc{gnuplot}
installed, he or she will have trouble compiling your files.

不幸的是,这些条件并不总是能够满足。特别是如果您将一些源代码传递给合作者,而合作者没有安装\textsc{gnuplot},他们将难以编译您的文件。

For this reason, \tikzname\ behaves differently when you compile your graphic
for the second time: If upon reaching |plot[id=|\meta{id}|] function{...}| the
file \meta{prefix}\meta{id}|.table| already exists \emph{and} if the
\meta{prefix}\meta{id}|.gnuplot| file contains what \tikzname\ thinks that it
``should'' contain, the |.table| file is immediately read without trying to
call a |gnuplot| program. This approach has the following advantages:

因此,当您第二次编译图形时,\tikzname 的行为会有所不同:如果在到达|plot[id=|\meta{id}|] function{...}|时,文件\meta{prefix}\meta{id}|.table|已经存在\emph{并且}如果\meta{prefix}\meta{id}|.gnuplot|文件包含了\tikzname 认为它“应该”包含的内容,则立即读取|.table|文件,而无需尝试调用|gnuplot|程序。这种方法具有以下优点:


\begin{enumerate}
    \item If you pass a bundle of your |.tex| file and all |.gnuplot| and
        |.table| files to someone else, that person can \TeX\ the |.tex| file
        without having to have |gnuplot| installed.

        如果您将一束包括您的|.tex|文件和所有的|.gnuplot|和|.table|文件的文件传递给他人,那个人可以在不必安装|gnuplot|的情况下编译|.tex|文件。

        \item If the |\write18| feature is switched off for security reasons (a
        good idea), then, upon the first compilation of the |.tex| file, the
        |.gnuplot| will still be generated, but not the |.table| file. You can
        then simply call |gnuplot| ``by hand'' for each |.gnuplot| file, which
        will produce all necessary |.table| files.

        如果出于安全原因关闭了|\write18|功能(这是一个很好的想法),则在第一次编译|.tex|文件时,仍然会生成|.gnuplot|文件,但不会生成|.table|文件。然后,您只需手动为每个|.gnuplot|文件调用|gnuplot|,这将生成所有必需的|.table|文件。

        \item If you change the function that you wish to plot or its domain,
        \tikzname\ will automatically try to regenerate the |.table| file.

        如果您更改要绘制的函数或其定义域,\tikzname 将自动尝试重新生成|.table|文件。

        \item If, out of laziness, you do not provide an |id|, the same |.gnuplot|
        will be used for different plots, but this is not a problem since the
        |.table| will automatically be regenerated for each plot on-the-fly.
        \emph{Note: If you intend to share your files with someone else, always
        use an id, so that the file can by typeset without having
        \textsc{gnuplot} installed.} Also, having unique ids for each plot will
        improve compilation speed since no external programs need to be called,
        unless it is really necessary.

        如果由于懒惰而没有提供一个|id|,则对于不同的绘图使用相同的|.gnuplot|,但这不是一个问题,因为|.table|将在每个绘图上自动进行即时重新生成。注意:如果您打算与他人共享文件,请始终使用一个id,以便在没有安装\textsc{gnuplot}的情况下可以排版文件。此外,为每个绘图使用唯一的id将提高编译速度,因为不需要调用外部程序,除非确实有必要。

    \end{enumerate}

When you use |plot function{|\meta{gnuplot formula}|}|, the \meta{gnuplot
formula} must be given in the |gnuplot| syntax, whose details are beyond the
scope of this manual. Here is the ultra-condensed essence: Use |x| as the
variable and use the C-syntax for normal plots, use |t| as the variable for
parametric plots. Here are some examples:

当您使用|plot function{|\meta{gnuplot formula}|}|时,必须以|gnuplot|语法给出\meta{gnuplot formula},其详细信息超出了本手册的范围。以下是超紧凑的要点:将|x|作为变量,并使用C语法进行常规绘图;将|t|作为变量用于参数绘图。以下是一些示例:


\begin{codeexample}[]
\begin{tikzpicture}[domain=0:4]
  \draw[very thin,color=gray] (-0.1,-1.1) grid (3.9,3.9);

  \draw[->] (-0.2,0) -- (4.2,0) node[right] {$x$};
  \draw[->] (0,-1.2) -- (0,4.2) node[above] {$f(x)$};

  \draw[color=red]    plot[id=x]   function{x}           node[right] {$f(x) =x$};
  \draw[color=blue]   plot[id=sin] function{sin(x)}      node[right] {$f(x) = \sin x$};
  \draw[color=orange] plot[id=exp] function{0.05*exp(x)} node[right] {$f(x) = \frac{1}{20} \mathrm e^x$};
\end{tikzpicture}
\end{codeexample}

The plot is influenced by the following options: First, the options |samples|
and |domain| explained earlier. Second, there are some more specialized
options.

绘图受以下选项的影响:首先是之前解释的|samples|和|domain|选项。其次,还有一些更专门的选项。

\begin{key}{/tikz/parametric=\meta{boolean} (default true)}
    Sets whether the plot is a parametric plot. If true, then |t| must be used
    instead of |x| as the parameter and two comma-separated functions must be
    given in the \meta{gnuplot formula}. An example is the following:
    

    设置绘图是否为参数绘图。如果为真,则必须使用|t|作为参数,而不是|x|,并且在\meta{gnuplot formula}中必须给出两个逗号分隔的函数。以下是一个示例:

\begin{codeexample}[]
\tikz \draw[scale=0.5,domain=-3.141:3.141,smooth]
  plot[parametric,id=parametric-example] function{t*sin(t),t*cos(t)};
\end{codeexample}
    %
\end{key}

\begin{key}{/tikz/range=\meta{start}|:|\meta{end}}
    This key sets the range of the plot. If set, all points whose
    $y$-coordinates lie outside this range will be considered to be outliers
    and will cause jumps in the plot, by default:
    
    此键设置绘图的范围。如果设置了,所有$y$坐标位于此范围之外的点将被视为异常值,并导致绘图跳跃,默认情况下:

\begin{codeexample}[]
\tikz \draw[scale=0.5,domain=-3.141:3.141, samples=100, smooth, range=-3:3]
  plot[id=tan-example] function{tan(x)};
\end{codeexample}
    %
\end{key}

\begin{key}{/tikz/yrange=\meta{start}|:|\meta{end}}
    Same as |range|.

    与|range|相同。
\end{key}

\begin{key}{/tikz/xrange=\meta{start}|:|\meta{end}}
    Set the $x$-range. This makes sense only for parametric plots.
    
    设置$x$范围。这只对参数绘图有意义。
\begin{codeexample}[]
\tikz \draw[scale=0.5,domain=-3.141:3.141,smooth,xrange=0:1]
  plot[parametric,id=parametric-example-cut] function{t*sin(t),t*cos(t)};
\end{codeexample}
    %
\end{key}

\begin{key}{/tikz/id=\meta{id}}
    Sets the identifier of the current plot. This should be a unique identifier
    for each plot (though things will also work if it is not, but not as well,
    see the explanations above). The \meta{id} will be part of a filename, so
    it should not contain anything fancy like |*| or |$|.%$

    设置当前绘图的标识符。这应该是每个绘图的唯一标识符(尽管如果不是,事情也能正常工作,但不太好,参见上面的解释)。\meta{id}将成为文件名的一部分,因此它不应包含任何特殊字符,如|*|或|$|。
\end{key}

\begin{key}{/tikz/prefix=\meta{prefix}}
    The \meta{prefix} is put before each plot file name. The default is
    |\jobname.|, but if you have many plots, it might be better to use, say
    |plots/| and have all plots placed in a directory. You have to create the
    directory yourself.

    \meta{prefix}被放置在每个绘图文件名之前。默认值为|\jobname.|,但如果您有许多绘图,最好使用,例如|plots/|,并将所有绘图放置在一个目录中。您需要自己创建该目录。
\end{key}

\begin{key}{/tikz/raw gnuplot}
    This key causes the \meta{gnuplot formula} to be passed on to
    \textsc{gnuplot} without setting up the samples or the |plot| operation.
    Thus, you could write

    此键使得\meta{gnuplot formula}被传递给\textsc{gnuplot},而不设置样本或|plot|操作。因此,您可以这样写:
\begin{codeexample}[code only]
plot[raw gnuplot,id=raw-example] function{set samples 25; plot sin(x)}
\end{codeexample}
    %
    This can be useful for complicated things that need to be passed to
    \textsc{gnuplot}. However, for really complicated situations you should
    create a special external generating \textsc{gnuplot} file and use the
    |file|-syntax to include the table ``by hand''.

    这对于需要传递给\textsc{gnuplot}的复杂问题可能很有用。然而,对于真正复杂的情况,您应该创建一个特殊的外部生成的\textsc{gnuplot}文件,并使用|file|语法“手动”包含表格。

\end{key}

The following styles influence the plot:

以下样式影响绘图:


\begin{stylekey}{/tikz/every plot (initially \normalfont empty)}
    This style is installed in each plot, that is, as if you always said
    
    此样式被安装在每个绘图中,就好像您总是这样写:
\begin{codeexample}[code only]
  plot[every plot,...]
\end{codeexample}
    %
    This is most useful for globally setting a prefix for all plots by saying:
    
    这对于通过以下方式全局设置所有绘图的前缀非常有用:
\begin{codeexample}[code only]
\tikzset{every plot/.style={prefix=plots/}}
\end{codeexample}
    %
\end{stylekey}


\subsection{Placing Marks on the Plot\\放置标记在绘图上}

As we saw already, it is possible to add \emph{marks} to a plot using the
|mark| option. When this option is used, a copy of the plot mark is placed on
each point of the plot. Note that the marks are placed \emph{after} the whole
path has been drawn/filled/shaded. In this respect, they are handled like text
nodes.

正如我们已经看到的,可以使用|mark|选项在绘图中添加\emph{标记}。当使用此选项时,绘图标记的副本被放置在绘图的每个点上。请注意,标记是在整个路径绘制/填充/着色之后放置的。在这方面,它们与文本节点的处理方式相同。

In detail, the following options govern how marks are drawn:

具体而言,以下选项控制如何绘制标记:

\begin{key}{/tikz/mark=\meta{mark mnemonic}}
    Sets the mark to a mnemonic that has previously been defined using the
    |\pgfdeclareplotmark|. By default, |*|, |+|, and |x| are available, which
    draw a filled circle, a plus, and a cross as marks. Many more marks become
    available when the library |plotmarks| is loaded.
    Section~\ref{section-plot-marks} lists the available plot marks.

    将标记设置为之前使用|\pgfdeclareplotmark|定义的助记符。默认情况下,有|*|、|+|和|x|可用,它们分别绘制填充的圆圈、加号和叉号作为标记。当加载|plotmarks|库时,会有更多的标记可用。第~\ref{section-plot-marks}节列出了可用的绘图标记。

    One plot mark is special: the |ball| plot mark is available only in
    \tikzname. The |ball color| option determines the balls's color. Do not use
    this option with a large number of marks since it will take very long to
    render in PostScript.

    有一个特殊的绘图标记:|ball|绘图标记仅在\tikzname 中可用。|ball color|选项确定球的颜色。不要在大量标记中使用此选项,因为它会花费很长时间在PostScript中渲染。

    \begin{tabular}{lc}
        Option & Effect \\
            \hline
        \vrule height14pt width0pt \plotmarkentrytikz{ball}
    \end{tabular}
\end{key}

\begin{key}{/tikz/mark repeat=\meta{r}}
    This option tells \tikzname\ that only every $r$th mark should be drawn.
    
    此选项告诉\tikzname 仅绘制每个第$r$个标记。

    \begin{codeexample}[]
\tikz \draw plot[mark=x,mark repeat=3,smooth] file {plots/pgfmanual-sine.table};
\end{codeexample}
    %
\end{key}

\begin{key}{/tikz/mark phase=\meta{p}}
    This option tells \tikzname\ that the first mark to be draw should be the
    $p$th, followed by the $(p+r)$th, then the $(p+2r)$th, and so on.
    
    此选项告诉\tikzname 首先绘制第$p$个标记,然后是第$(p+r)$个标记,然后是第$(p+2r)$个标记,依此类推。

    \begin{codeexample}[]
\tikz \draw plot[mark=x,mark repeat=3,mark phase=6,smooth] file {plots/pgfmanual-sine.table};
\end{codeexample}
    %
\end{key}

\begin{key}{/tikz/mark indices=\meta{list}}
    This option allows you to specify explicitly the indices at which a mark
    should be placed. Counting starts with 1. You can use the |\foreach|
    syntax, that is, |...| can be used.
    
    此选项允许您明确指定应放置标记的索引。计数从1开始。您可以使用|\foreach|语法,即可以使用|...|。

    \begin{codeexample}[]
\tikz \draw plot[mark=x,mark indices={1,4,...,10,11,12,...,16,20},smooth]
  file {plots/pgfmanual-sine.table};
\end{codeexample}
    %
\end{key}

\begin{key}{/tikz/mark size=\meta{dimension}}
    Sets the size of the plot marks. For circular plot marks, \meta{dimension}
    is the radius, for other plot marks \meta{dimension} should be about half
    the width and height.

    设置绘图标记的大小。对于圆形绘图标记,\meta{dimension}为半径,对于其他绘图标记,\meta{dimension}应为宽度和高度的一半。

    This option is not really necessary, since you achieve the same effect by
    specifying |scale=|\meta{factor} as a local option, where \meta{factor} is
    the quotient of the desired size and the default size. However, using
    |mark size| is a bit faster and more natural.

    此选项实际上并不是必需的,因为您可以在局部选项中指定 |scale=|\meta{factor} 来实现相同的效果,其中 \meta{factor} 是所需大小与默认大小的商。然而,使用 |mark size| 更快速和更自然。

\end{key}

\begin{stylekey}{/tikz/every mark}
    This style is installed before drawing plot marks. For example, you can
    scale (or otherwise transform) the plot mark or set its color.

    此样式在绘制绘图标记之前安装。例如,您可以缩放(或以其他方式变换)绘图标记或设置其颜色。

\end{stylekey}

\begin{key}{/tikz/mark options=\meta{options}}
    Redefines |every mark| such that it sets \marg{options}.
    
    重新定义 |every mark|,使其设置 \marg{options}。
\begin{codeexample}[]
\tikz \fill[fill=blue!20]
  plot[mark=triangle*,mark options={color=blue,rotate=180}]
    file{plots/pgfmanual-sine.table} |- (0,0);
\end{codeexample}
    %
\end{key}

\begin{stylekey}{/tikz/no marks}
    Disables markers (the same as |mark=none|).

    禁用标记(与 |mark=none| 相同)。
\end{stylekey}
%
\begin{stylekey}{/tikz/no markers}
    Disables markers (the same as |mark=none|).

    禁用标记(与 |mark=none| 相同)。
\end{stylekey}


\subsection{Smooth Plots, Sharp Plots, Jump Plots, Comb Plots and Bar Plots\\平滑绘图、尖锐绘图、跳跃绘图、组合绘图和条形图}

There are different things the |plot| operation can do with the points it reads
from a file or from the inlined list of points. By default, it will connect
these points by straight lines. However, you can also use options to change the
behavior of |plot|.

|plot| 操作可以对从文件或内联点列表读取的点执行不同的操作。默认情况下,它将通过直线连接这些点。然而,您也可以使用选项来更改 |plot| 的行为。

\begin{key}{/tikz/sharp plot}
    This is the default and causes the points to be connected by straight
    lines. This option is included only so that you can ``switch back'' if you
    ``globally'' install, say, |smooth|.

    这是默认选项,使得点之间通过直线连接。此选项仅包括,以便您可以在“全局”安装,例如 |smooth| 时“切换回”默认行为。

\end{key}

\begin{key}{/tikz/smooth}
    This option causes the points on the path to be connected using a smooth
    curve:

    此选项使得路径上的点通过光滑曲线连接:

    \begin{codeexample}[]
\tikz\draw plot[smooth] file{plots/pgfmanual-sine.table};
\end{codeexample}

    Note that the smoothing algorithm is not very intelligent. You will get the
    best results if the bending angles are small, that is, less than about
    $30^\circ$ and, even more importantly, if the distances between points are
    about the same all over the plotting path.

    请注意,平滑算法并不是非常智能。如果弯曲角度较小,即小于约 $30^\circ$,并且更重要的是,如果点之间的距离在整个绘图路径上大致相同,将获得最佳结果。

\end{key}

\begin{key}{/tikz/tension=\meta{value}}
    This option influences how ``tight'' the smoothing is. A lower value will
    result in sharper corners, a higher value in more ``round'' curves. A value
    of $1$ results in a circle if four points at quarter-positions on a circle
    are given. The default is $0.55$. The ``correct'' value depends on the
    details of plot.
    
    此选项影响平滑程度。较小的值会导致更锐利的拐角,较大的值会产生更“圆”的曲线。如果给定四个位于圆的四分之一位置的点,值为 $1$ 的结果是一个圆。默认值为 $0.55$。正确的值取决于绘图的细节。

    \begin{codeexample}[]
\begin{tikzpicture}[smooth cycle]
  \draw                 plot[tension=0.2]
    coordinates{(0,0) (1,1) (2,0) (1,-1)};
  \draw[yshift=-2.25cm] plot[tension=0.5]
    coordinates{(0,0) (1,1) (2,0) (1,-1)};
  \draw[yshift=-4.5cm]  plot[tension=1]
    coordinates{(0,0) (1,1) (2,0) (1,-1)};
\end{tikzpicture}
\end{codeexample}
    %
\end{key}

\begin{key}{/tikz/smooth cycle}
    This option causes the points on the path to be connected using a closed
    smooth curve.
 
    此选项使得路径上的点通过闭合光滑曲线连接。

    \begin{codeexample}[]
\tikz[scale=0.5]
  \draw plot[smooth cycle] coordinates{(0,0) (1,0) (2,1) (1,2)}
        plot               coordinates{(0,0) (1,0) (2,1) (1,2)} -- cycle;
\end{codeexample}
    %
\end{key}

\begin{key}{/tikz/const plot}
    This option causes the points on the path to be connected using piecewise
    constant series of lines:
    
    此选项使得路径上的点通过分段常数直线连接:


\begin{codeexample}[]
\tikz\draw plot[const plot] file{plots/pgfmanual-sine.table};
\end{codeexample}
    %
\end{key}

\begin{key}{/tikz/const plot mark left}
    Just an alias for |/tikz/const plot|.
    
    它只是 |/tikz/const plot| 的别名。


\begin{codeexample}[]
\tikz\draw plot[const plot mark left,mark=*] file{plots/pgfmanual-sine.table};
\end{codeexample}
    %
\end{key}

\begin{key}{/tikz/const plot mark right}
    A variant of |/tikz/const plot| which places its mark on the right ends:
    
    此选项是 |/tikz/const plot| 的变体,它将标记放在右端点。


\begin{codeexample}[]
\tikz\draw plot[const plot mark right,mark=*] file{plots/pgfmanual-sine.table};
\end{codeexample}
    %
\end{key}

\begin{key}{/tikz/const plot mark mid}
    A variant of |/tikz/const plot| which places its mark in the middle of the
    horizontal lines:
    
    此选项是 |/tikz/const plot| 的变体,它将标记放在水平线的中间:
\begin{codeexample}[]
\tikz\draw plot[const plot mark mid,mark=*] file{plots/pgfmanual-sine.table};
\end{codeexample}
    %
    More precisely, it generates vertical lines in the middle between each pair
    of consecutive points. If the mesh width is constant, this leads to
    symmetrically placed marks (``middle'').

    更准确地说,它在每对相邻点之间的中点生成垂直线。如果网格宽度是恒定的,则会得到对称放置的标记(“中间”)。
\end{key}

\begin{key}{/tikz/jump mark left}
    This option causes the points on the path to be drawn using piecewise
    constant, non-connected series of lines. If there are any marks, they will
    be placed on left open ends:
    
    此选项使得路径上的点使用分段常数、非连接的线系绘制。如果有任何标记,它们将放置在左开端:
\begin{codeexample}[]
\tikz\draw plot[jump mark left, mark=*] file{plots/pgfmanual-sine.table};
\end{codeexample}
    %
\end{key}

\begin{key}{/tikz/jump mark right}
    This option causes the points on the path to be drawn using piecewise
    constant, non-connected series of lines. If there are any marks, they will
    be placed on right open ends:
    
    此选项使得路径上的点使用分段常数、非连接的线系绘制。如果有任何标记,它们将放置在右开端:
\begin{codeexample}[]
\tikz\draw plot[jump mark right, mark=*] file{plots/pgfmanual-sine.table};
\end{codeexample}
    %
\end{key}

\begin{key}{/tikz/jump mark mid}
    This option causes the points on the path to be drawn using piecewise
    constant, non-connected series of lines. If there are any marks, they will
    be placed in the middle of the horizontal line segments:
    
    该选项会使用分段常数、非连续的线段来绘制路径上的点。如果存在标记,它们将放置在水平线段的中间位置:

\begin{codeexample}[]
\tikz\draw plot[jump mark mid, mark=*] file{plots/pgfmanual-sine.table};
\end{codeexample}

    In case of non-constant mesh widths, the same remarks as for
    |const plot mark mid| apply.

    对于非常数网格宽度,适用与 |const plot mark mid| 相同的注意事项。

\end{key}

\begin{key}{/tikz/ycomb}
    This option causes the |plot| operation to interpret the plotting points
    differently. Instead of connecting them, for each point of the plot a
    straight line is added to the path from the $x$-axis to the point,
    resulting in a sort of ``comb'' or ``bar diagram''.
    
    该选项使得 |plot| 操作以不同的方式解释绘图点。而不是将它们连接起来,对于绘图的每个点,从 $x$ 轴到该点添加一条直线到路径中,从而形成一种“梳状”或“条形图”。
\begin{codeexample}[]
\tikz\draw[ultra thick] plot[ycomb,thin,mark=*] file{plots/pgfmanual-sine.table};
\end{codeexample}

\begin{codeexample}[]
\begin{tikzpicture}[ycomb]
  \draw[color=red,line width=6pt]
    plot coordinates{(0,1) (.5,1.2) (1,.6) (1.5,.7) (2,.9)};
  \draw[color=red!50,line width=4pt,xshift=3pt]
    plot coordinates{(0,1.2) (.5,1.3) (1,.5) (1.5,.2) (2,.5)};
\end{tikzpicture}
\end{codeexample}
    %
\end{key}

\begin{key}{/tikz/xcomb}
    This option works like |ycomb| except that the bars are horizontal.
    
    该选项与 |ycomb| 类似,只是条形是水平的。
\begin{codeexample}[]
\tikz \draw plot[xcomb,mark=x] coordinates{(1,0) (0.8,0.2) (0.6,0.4) (0.2,1)};
\end{codeexample}
    %
\end{key}

\begin{key}{/tikz/polar comb}
    This option causes a line from the origin to the point to be added to the
    path for each plot point.
    
    该选项使得从原点到每个绘图点添加一条线到路径中。
\begin{codeexample}[]
\tikz \draw plot[polar comb,
     mark=pentagon*,mark options={fill=white,draw=red},mark size=4pt]
   coordinates {(0:1cm) (30:1.5cm) (160:.5cm) (250:2cm) (-60:.8cm)};
\end{codeexample}
    %
\end{key}

\begin{key}{/tikz/ybar}
    This option produces fillable bar plots. It is thus very similar to
    |ycomb|, but it employs rectangular shapes instead of line-to operations.
    It thus allows to use any fill or pattern style.
    
    该选项生成可填充的条形图。因此它与 |ycomb| 非常类似,但它使用矩形形状而不是线段操作。因此,它允许使用任何填充或图案样式。
\begin{codeexample}[]
\tikz\draw[draw=blue,fill=blue!60!black] plot[ybar] file{plots/pgfmanual-sine.table};
\end{codeexample}

\begin{codeexample}[]
\begin{tikzpicture}[ybar]
  \draw[color=red,fill=red!80,bar width=6pt]
    plot coordinates{(0,1) (.5,1.2) (1,.6) (1.5,.7) (2,.9)};
  \draw[color=red!50,fill=red!20,bar width=4pt,bar shift=3pt]
    plot coordinates{(0,1.2) (.5,1.3) (1,.5) (1.5,.2) (2,.5)};
\end{tikzpicture}
\end{codeexample}
    %
    The use of |bar width| and |bar shift| is explained in the |plothandlers|
    library documentation, section~\ref{section-plotlib-bar-handlers}. Please
    refer to page~\pageref{key-bar-width}.

    关于 |bar width| 和 |bar shift| 的用法在 |plothandlers| 库的文档中有解释,参见第~\ref{section-plotlib-bar-handlers}节。请参考第~\pageref{key-bar-width}页。

\end{key}

\begin{key}{/tikz/xbar}
    This option works like |ybar| except that the bars are horizontal.
    
    该选项与 |ybar| 类似,只是条形是水平的。

\begin{codeexample}[preamble={\usetikzlibrary{patterns}}]
\tikz \draw[pattern=north west lines] plot[xbar]
   coordinates{(1,0) (0.4,1) (1.7,2) (1.6,3)};
\end{codeexample}
    %
\end{key}

\begin{key}{/tikz/ybar interval}
    As |/tikz/ybar|, this options produces vertical bars. However, bars are
    centered at coordinate \emph{intervals} instead of interval edges, and the
    bar's width is also determined relatively to the interval's length:
    
    与 |/tikz/ybar| 相同,该选项生成垂直条形。但是,条形是居中于坐标\emph{区间}而不是区间边缘,并且条形的宽度也与区间的长度相关:

\begin{codeexample}[]
\begin{tikzpicture}[ybar interval,x=10pt]
  \draw[color=red,fill=red!80]
    plot coordinates{(0,2) (2,1.2) (3,.3) (5,1.7) (8,.9) (9,.9)};
\end{tikzpicture}
\end{codeexample}
    %
    Since there are $N$ intervals $[x_i,x_{i+1}]$ for given $N+1$ coordinates,
    you will always have one coordinate more than bars. The last $y$ value will
    be ignored.

    由于给定 $N+1$ 个坐标时有 $N$ 个区间 $[x_i,x_{i+1}]$,所以条形数目比坐标数目多一个。最后一个 $y$ 值将被忽略。

    You can configure relative shifts and relative bar widths, which is
    explained in the |plothandlers| library documentation,
    section~\ref{section-plotlib-bar-handlers}. Please refer to
    page~\pageref{key-bar-interval-width}.

    可以配置相对偏移和相对条形宽度,这在 |plothandlers| 库的文档中有解释,参见第~\ref{section-plotlib-bar-handlers}节。请参考第~\pageref{key-bar-interval-width}页。

\end{key}

\begin{key}{/tikz/xbar interval}
    Works like |ybar interval|, but for horizontal bar plots.
    
    与 |ybar interval| 类似,但用于水平条形图。


\begin{codeexample}[]
\begin{tikzpicture}[xbar interval,x=0.5cm,y=0.5cm]
  \draw[color=red,fill=red!80]
    plot coordinates {(3,0) (2,1) (4,1.5) (1,4) (2,6) (2,7)};
\end{tikzpicture}
\end{codeexample}
    %
\end{key}

\begin{key}{/tikz/only marks}
    This option causes only marks to be shown; no path segments are added to
    the actual path. This can be useful for quickly adding some marks to a
    path.

    该选项仅显示标记;不会将路径段添加到实际路径中。这对于快速向路径添加一些标记很有用。

    %
\begin{codeexample}[]
\tikz \draw (0,0) sin (1,1) cos (2,0)
  plot[only marks,mark=x] coordinates{(0,0) (1,1) (2,0) (3,-1)};
\end{codeexample}
\end{key}
 