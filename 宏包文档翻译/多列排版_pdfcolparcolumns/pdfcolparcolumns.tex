\documentclass{ltxdoc}
\usepackage{holtxdoc}[2011/11/22]
\usepackage[,scheme=chinese%中文方案
,fontset=none%不使用默认的字体设置
,space=auto%自动调整中英文间距
]{ctex}
\setCJKmainfont{方正书宋_GBK}%方正书宋_GBK.TTF  设置文本的中文有衬线字体为“方正书宋_GBK”
\setCJKsansfont{方正黑体简体}%方正黑体_GBK.TTF  设置文本的中文无衬线字体为“方正黑体简体”
\setCJKmonofont{方正书宋简体}%方正仿宋_GBK.TTF  设置文本的中文等宽字体为“方正书宋简体”
\usepackage{pdfcolparcolumns}
\begin{document}
\parindent=0pt
  \DocInput{pdfcolparcolumns.dtx}%   
\end{document}
% xelatex -file-line-error -synctex=1  -shell-escape -output-directory=/Volumes/RamDisk/latex_output 