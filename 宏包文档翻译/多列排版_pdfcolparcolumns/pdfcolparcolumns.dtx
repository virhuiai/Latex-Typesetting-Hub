% \iffalse meta-comment
%
% File: pdfcolparcolumns.dtx
% Version: 2019/12/29 v1.5
% Info: Color stacks for parcolumns
%
% Copyright (C)
%    2007, 2008, 2010 Heiko Oberdiek
%    2016-2019 Oberdiek Package Support Group
%    https://github.com/ho-tex/oberdiek/issues
%
% This work may be distributed and/or modified under the
% conditions of the LaTeX Project Public License, either
% version 1.3c of this license or (at your option) any later
% version. This version of this license is in
%    https://www.latex-project.org/lppl/lppl-1-3c.txt
% and the latest version of this license is in
%    https://www.latex-project.org/lppl.txt
% and version 1.3 or later is part of all distributions of
% LaTeX version 2005/12/01 or later.
%
% This work has the LPPL maintenance status "maintained".
%
% The Current Maintainers of this work are
% Heiko Oberdiek and the Oberdiek Package Support Group
% https://github.com/ho-tex/oberdiek/issues
%
% This work consists of the main source file pdfcolparcolumns.dtx
% and the derived files
%    pdfcolparcolumns.sty, pdfcolparcolumns.pdf, pdfcolparcolumns.ins,
%    pdfcolparcolumns.drv, pdfcolparcolumns-test1.tex.
%
% Distribution:
%    CTAN:macros/latex/contrib/oberdiek/pdfcolparcolumns.dtx
%    CTAN:macros/latex/contrib/oberdiek/pdfcolparcolumns.pdf
%
% Unpacking:
%    (a) If pdfcolparcolumns.ins is present:
%           tex pdfcolparcolumns.ins
%    (b) Without pdfcolparcolumns.ins:
%           tex pdfcolparcolumns.dtx
%    (c) If you insist on using LaTeX
%           latex \let\install=y% \iffalse meta-comment
%
% File: pdfcolparcolumns.dtx
% Version: 2019/12/29 v1.5
% Info: Color stacks for parcolumns
%
% Copyright (C)
%    2007, 2008, 2010 Heiko Oberdiek
%    2016-2019 Oberdiek Package Support Group
%    https://github.com/ho-tex/oberdiek/issues
%
% This work may be distributed and/or modified under the
% conditions of the LaTeX Project Public License, either
% version 1.3c of this license or (at your option) any later
% version. This version of this license is in
%    https://www.latex-project.org/lppl/lppl-1-3c.txt
% and the latest version of this license is in
%    https://www.latex-project.org/lppl.txt
% and version 1.3 or later is part of all distributions of
% LaTeX version 2005/12/01 or later.
%
% This work has the LPPL maintenance status "maintained".
%
% The Current Maintainers of this work are
% Heiko Oberdiek and the Oberdiek Package Support Group
% https://github.com/ho-tex/oberdiek/issues
%
% This work consists of the main source file pdfcolparcolumns.dtx
% and the derived files
%    pdfcolparcolumns.sty, pdfcolparcolumns.pdf, pdfcolparcolumns.ins,
%    pdfcolparcolumns.drv, pdfcolparcolumns-test1.tex.
%
% Distribution:
%    CTAN:macros/latex/contrib/oberdiek/pdfcolparcolumns.dtx
%    CTAN:macros/latex/contrib/oberdiek/pdfcolparcolumns.pdf
%
% Unpacking:
%    (a) If pdfcolparcolumns.ins is present:
%           tex pdfcolparcolumns.ins
%    (b) Without pdfcolparcolumns.ins:
%           tex pdfcolparcolumns.dtx
%    (c) If you insist on using LaTeX
%           latex \let\install=y% \iffalse meta-comment
%
% File: pdfcolparcolumns.dtx
% Version: 2019/12/29 v1.5
% Info: Color stacks for parcolumns
%
% Copyright (C)
%    2007, 2008, 2010 Heiko Oberdiek
%    2016-2019 Oberdiek Package Support Group
%    https://github.com/ho-tex/oberdiek/issues
%
% This work may be distributed and/or modified under the
% conditions of the LaTeX Project Public License, either
% version 1.3c of this license or (at your option) any later
% version. This version of this license is in
%    https://www.latex-project.org/lppl/lppl-1-3c.txt
% and the latest version of this license is in
%    https://www.latex-project.org/lppl.txt
% and version 1.3 or later is part of all distributions of
% LaTeX version 2005/12/01 or later.
%
% This work has the LPPL maintenance status "maintained".
%
% The Current Maintainers of this work are
% Heiko Oberdiek and the Oberdiek Package Support Group
% https://github.com/ho-tex/oberdiek/issues
%
% This work consists of the main source file pdfcolparcolumns.dtx
% and the derived files
%    pdfcolparcolumns.sty, pdfcolparcolumns.pdf, pdfcolparcolumns.ins,
%    pdfcolparcolumns.drv, pdfcolparcolumns-test1.tex.
%
% Distribution:
%    CTAN:macros/latex/contrib/oberdiek/pdfcolparcolumns.dtx
%    CTAN:macros/latex/contrib/oberdiek/pdfcolparcolumns.pdf
%
% Unpacking:
%    (a) If pdfcolparcolumns.ins is present:
%           tex pdfcolparcolumns.ins
%    (b) Without pdfcolparcolumns.ins:
%           tex pdfcolparcolumns.dtx
%    (c) If you insist on using LaTeX
%           latex \let\install=y% \iffalse meta-comment
%
% File: pdfcolparcolumns.dtx
% Version: 2019/12/29 v1.5
% Info: Color stacks for parcolumns
%
% Copyright (C)
%    2007, 2008, 2010 Heiko Oberdiek
%    2016-2019 Oberdiek Package Support Group
%    https://github.com/ho-tex/oberdiek/issues
%
% This work may be distributed and/or modified under the
% conditions of the LaTeX Project Public License, either
% version 1.3c of this license or (at your option) any later
% version. This version of this license is in
%    https://www.latex-project.org/lppl/lppl-1-3c.txt
% and the latest version of this license is in
%    https://www.latex-project.org/lppl.txt
% and version 1.3 or later is part of all distributions of
% LaTeX version 2005/12/01 or later.
%
% This work has the LPPL maintenance status "maintained".
%
% The Current Maintainers of this work are
% Heiko Oberdiek and the Oberdiek Package Support Group
% https://github.com/ho-tex/oberdiek/issues
%
% This work consists of the main source file pdfcolparcolumns.dtx
% and the derived files
%    pdfcolparcolumns.sty, pdfcolparcolumns.pdf, pdfcolparcolumns.ins,
%    pdfcolparcolumns.drv, pdfcolparcolumns-test1.tex.
%
% Distribution:
%    CTAN:macros/latex/contrib/oberdiek/pdfcolparcolumns.dtx
%    CTAN:macros/latex/contrib/oberdiek/pdfcolparcolumns.pdf
%
% Unpacking:
%    (a) If pdfcolparcolumns.ins is present:
%           tex pdfcolparcolumns.ins
%    (b) Without pdfcolparcolumns.ins:
%           tex pdfcolparcolumns.dtx
%    (c) If you insist on using LaTeX
%           latex \let\install=y\input{pdfcolparcolumns.dtx}
%        (quote the arguments according to the demands of your shell)
%
% Documentation:
%    (a) If pdfcolparcolumns.drv is present:
%           latex pdfcolparcolumns.drv
%    (b) Without pdfcolparcolumns.drv:
%           latex pdfcolparcolumns.dtx; ...
%    The class ltxdoc loads the configuration file ltxdoc.cfg
%    if available. Here you can specify further options, e.g.
%    use A4 as paper format:
%       \PassOptionsToClass{a4paper}{article}
%
%    Programm calls to get the documentation (example):
%       pdflatex pdfcolparcolumns.dtx
%       makeindex -s gind.ist pdfcolparcolumns.idx
%       pdflatex pdfcolparcolumns.dtx
%       makeindex -s gind.ist pdfcolparcolumns.idx
%       pdflatex pdfcolparcolumns.dtx
%
% Installation:
%    TDS:tex/latex/oberdiek/pdfcolparcolumns.sty
%    TDS:doc/latex/oberdiek/pdfcolparcolumns.pdf
%    TDS:source/latex/oberdiek/pdfcolparcolumns.dtx
%
%<*ignore>
\begingroup
  \catcode123=1 %
  \catcode125=2 %
  \def\x{LaTeX2e}%
\expandafter\endgroup
\ifcase 0\ifx\install y1\fi\expandafter
         \ifx\csname processbatchFile\endcsname\relax\else1\fi
         \ifx\fmtname\x\else 1\fi\relax
\else\csname fi\endcsname
%</ignore>
%<*install>
\input docstrip.tex
\Msg{************************************************************************}
\Msg{* Installation}
\Msg{* Package: pdfcolparcolumns 2019/12/29 v1.5 Color stacks for parcolumns (HO)}
\Msg{************************************************************************}

\keepsilent
\askforoverwritefalse

\let\MetaPrefix\relax
\preamble

This is a generated file.

Project: pdfcolparcolumns
Version: 2019/12/29 v1.5

Copyright (C)
   2007, 2008, 2010 Heiko Oberdiek
   2016-2019 Oberdiek Package Support Group

This work may be distributed and/or modified under the
conditions of the LaTeX Project Public License, either
version 1.3c of this license or (at your option) any later
version. This version of this license is in
   https://www.latex-project.org/lppl/lppl-1-3c.txt
and the latest version of this license is in
   https://www.latex-project.org/lppl.txt
and version 1.3 or later is part of all distributions of
LaTeX version 2005/12/01 or later.

This work has the LPPL maintenance status "maintained".

The Current Maintainers of this work are
Heiko Oberdiek and the Oberdiek Package Support Group
https://github.com/ho-tex/oberdiek/issues


This work consists of the main source file pdfcolparcolumns.dtx
and the derived files
   pdfcolparcolumns.sty, pdfcolparcolumns.pdf, pdfcolparcolumns.ins,
   pdfcolparcolumns.drv, pdfcolparcolumns-test1.tex.

\endpreamble
\let\MetaPrefix\DoubleperCent

\generate{%
  \file{pdfcolparcolumns.ins}{\from{pdfcolparcolumns.dtx}{install}}%
  \file{pdfcolparcolumns.drv}{\from{pdfcolparcolumns.dtx}{driver}}%
  \usedir{tex/latex/oberdiek}%
  \file{pdfcolparcolumns.sty}{\from{pdfcolparcolumns.dtx}{package}}%
%  \usedir{doc/latex/oberdiek/test}%
%  \file{pdfcolparcolumns-test1.tex}{\from{pdfcolparcolumns.dtx}{test1}}%
}

\catcode32=13\relax% active space
\let =\space%
\Msg{************************************************************************}
\Msg{*}
\Msg{* To finish the installation you have to move the following}
\Msg{* file into a directory searched by TeX:}
\Msg{*}
\Msg{*     pdfcolparcolumns.sty}
\Msg{*}
\Msg{* To produce the documentation run the file `pdfcolparcolumns.drv'}
\Msg{* through LaTeX.}
\Msg{*}
\Msg{* Happy TeXing!}
\Msg{*}
\Msg{************************************************************************}

\endbatchfile
%</install>
%<*ignore>
\fi
%</ignore>
%<*driver>
\NeedsTeXFormat{LaTeX2e}
\ProvidesFile{pdfcolparcolumns.drv}%
  [2019/12/29 v1.5 Color stacks for parcolumns (HO)]%
\documentclass{ltxdoc}
\usepackage{holtxdoc}[2011/11/22]
\usepackage[scheme=plain]{ctex}
\setCJKmainfont{方正书宋_GBK}%方正书宋_GBK.TTF  设置文本的中文有衬线字体为“方正书宋_GBK”
\setCJKsansfont{方正黑体简体}%方正黑体_GBK.TTF  设置文本的中文无衬线字体为“方正黑体简体”
\setCJKmonofont{方正书宋简体}%方正仿宋_GBK.TTF  设置文本的中文等宽字体为“方正书宋简体”
\begin{document}
  \DocInput{pdfcolparcolumns.dtx}%
\end{document}
%</driver>
% \fi
%
%
%
% \GetFileInfo{pdfcolparcolumns.drv}
%
% \title{The \xpackage{pdfcolparcolumns} package}
% \date{2019/12/29 v1.5}
% \author{Heiko Oberdiek\thanks
% {Please report any issues at \url{https://github.com/ho-tex/oberdiek/issues}}\and 翻译\\{\tt virhuiai@qq.com}}
%
% \maketitle
%
% \begin{abstract}
%\makebox[0pt]{}
% \begin{parcolumns}[nofirstindent]{2}
%\colchunk{Since version 1.40 \pdfTeX\ supports several color stacks.
%This package uses them to fix color problems in
%package \xpackage{parcolumns}.}
%\colchunk{自从版本1.40起,\pdfTeX 支持多个颜色栈。
%此包使用它们来解决 \xpackage{parcolumns} 包中的颜色问题。}
%\end{parcolumns}



% \end{abstract}
%
% \tableofcontents
%
% \section{Usage\\用法}
%
% \begin{quote}
% |\usepackage{pdfcolparcolumns}|
% \end{quote}
% The package \xpackage{pdfcolparcolumns} loads package \xpackage{parcolums}
% \cite{parcolumns}. If color stacks are available then the
% macros of \xpackage{parcolumns} are patched to add support
% for color stacks.

%包 \xpackage{pdfcolparcolumns} 载入了 \xpackage{parcolumns} \cite{parcolumns} 包。如果有可用的颜色栈,则会对 \xpackage{parcolumns} 的宏进行修补,以添加对颜色栈的支持。

%
% \subsection{Option \xoption{rulebetweencolor}\\选项 \xoption{rulebetweencolor}}
%
% Package \xpackage{pdfcolparcolumns} also fixes the color for the
% rule between columns (if \xoption{rulebetween} is set).
% Default color is \cs{normalcolor}. But this can be changed by using
% option \xoption{rulebetweencolor}. It takes a color specification
% as value. If the value is empty, then the default (\cs{normalcolor})
% is used.
% Examples:

% 包 \xpackage{pdfcolparcolumns} 还修复了列之间分隔线的颜色(如果设置了 \xoption{rulebetween})。默认颜色为 \cs{normalcolor}。但是,可以使用选项 \xoption{rulebetweencolor} 来更改它。它接受一个颜色规范作为值。如果该值为空,则使用默认值(\cs{normalcolor})。例如:

% \begin{quote}
%   |rulebetweencolor=blue|,\\
%   |rulebetweencolor={red}|,\\
%   |rulebetweencolor={}|, \textit{\% \cs{normalcolor} is used}\\
%   |rulebetweencolor=[rgb]{1,0,.5}| \textit{\% see below}
% \end{quote}
% If used inside the optional argument of environment \textsf{parcolumns}
% and the value contains an optional argument, then whole value
% must be put in curly braces:

% 如果在 \textsf{parcolumns} 环境的可选参数中使用,并且值包含可选参数,则整个值必须用花括号括起来:
%\begin{quote}
%\begin{verbatim}
%\begin{parcolumns}[
%  rulebetween,
%  rulebetweencolor={[rgb]{1,0,.5}},
%]{2}
%  ...
%\end{parcolumns}
%\end{verbatim}
%\end{quote}
% This option \xoption{rulebetweencolor} can also be set using
% \cs{setkeys}:

%也可以使用 \cs{setkeys} 来设置选项 \xoption{rulebetweencolor}:
%\begin{quote}
%\begin{verbatim}
%\setkeys{parcolumns}{rulebetweencolor=green}
%\end{verbatim}
%\end{quote}
%
% \subsection{Future\\未来展望}
%
% Currently package \xpackage{parcolumns} does not seem to be
% maintained. Nevertheless if there will be a new version that
% adds support for color stacks, then this package may become
% obsolete.
%
%目前,\xpackage{parcolumns} 包似乎没有维护了。尽管如此,如果出现了新版本并支持颜色栈,那么这个包可能会变得过时。
% \StopEventually{
% }
%
% \section{Implementation\\实现}
%
% \subsection{Identification\\标识}
%
%    \begin{macrocode}
%<*package>
\NeedsTeXFormat{LaTeX2e}
\ProvidesPackage{pdfcolparcolumns}%
  [2019/12/29 v1.5 Color stacks for parcolumns (HO)]%
%    \end{macrocode}
%
% \subsection{Load packages\\加载宏包}
%
% \subsubsection{Package \xpackage{parcolumns}\\宏包 \xpackage{parcolumns}}
%
%    Currently package \xpackage{parcolumns} does not define options.
%    Thus it is just a precaution that the options of
%    package \xpackage{pdfcolparcolumns} are passed to
%    package \xpackage{parcolumns}.

%目前,宏包 \xpackage{parcolumns} 没有定义选项。因此,它只是一个预防措施,以确保将宏包 \xpackage{pdfcolparcolumns} 的选项传递给宏包 \xpackage{parcolumns}。
%    \begin{macrocode}
\DeclareOption*{%
  \PassOptionsToPackage{\CurrentOption}{parcolumns}%
}
\ProcessOptions\relax
\RequirePackage{parcolumns}[2004/11/25]
%    \end{macrocode}
%
% \subsubsection{Package \xpackage{pdfcol}\\宏包 \xpackage{pdfcol}}
%
%    \begin{macrocode}
\RequirePackage{pdfcol}[2007/09/09]
\ifpdfcolAvailable
\else
  \PackageInfo{pdfcolparcolumns}{%
    Loading aborted, because color stacks are not available%
  }%
  \expandafter\endinput
\fi
%    \end{macrocode}
%
% \subsubsection{Package \xpackage{infwarerr}\\宏包 \xpackage{infwarerr}}
%
%    \begin{macrocode}
\RequirePackage{infwarerr}[2007/09/09]
%    \end{macrocode}
%
% \subsection{Color stack macros\\颜色堆栈宏}
%
%    \begin{macro}{\pcpc@MaxStack}
%    Macro \cs{pcpc@MaxStack} holds the highest number of
%    allocated stacks.

%宏 \cs{pcpc@MaxStack} 存储已分配的堆栈中最大的编号。
%    \begin{macrocode}
\global\chardef\pcpc@MaxStack=\z@
%    \end{macrocode}
%    \end{macro}
%    \begin{macro}{\pcpc@InitStacks}
%    Macro \cs{pcpc@InitStacks} takes the number of columns
%    as argument and ensures that there are enough color
%    stacks for all columns.
%
%宏 \cs{pcpc@InitStacks} 接受列数作为参数,并确保为所有列分配了足够的颜色堆栈。
%    \begin{macrocode}
\def\pcpc@InitStacks#1{%
  \ifnum#1>\pcpc@MaxStack
    \begingroup
      \count@\pcpc@MaxStack
      \loop
        \advance\count@\@ne
        \pdfcolInitStack{pcpc@\the\count@}%
      \ifnum#1>\count@
      \repeat
      \global\chardef\pcpc@MaxStack=\count@
    \endgroup
  \fi
}
%    \end{macrocode}
%    \end{macro}
%
%    \begin{macro}{\pcpc@SwitchStack}
%    \begin{macrocode}
\def\pcpc@SwitchStack#1{%
  \pdfcolSwitchStack{pcpc@\number#1}%
}
%    \end{macrocode}
%    \end{macro}
%
%    \begin{macro}{\pcpc@SetCurrent}
%    \begin{macrocode}
\def\pcpc@SetCurrent#1{%
  \pdfcolSetCurrent{pcpc@\number#1}%
}
%    \end{macrocode}
%    \end{macro}
%
% \subsection{Patches\\补丁}
%
%     Now the color stack macros are patched into the macros
%     of package \xpackage{parcolumns}.
%
%现在颜色堆栈宏已经被打补丁到了 \xpackage{parcolumns} 宏包的宏中。
% \subsubsection{Init stacks\\初始化堆栈}
%
%    \cs{pcpc@InitStacks} should go into the definition of
%    environment |parcolumns|. \cs{pc@alloccolumns} is executed
%    there and nowhere else, thus we hook into it.

%\cs{pcpc@InitStacks} 应该被放在环境 |parcolumns| 的定义中。\cs{pc@alloccolumns} 仅在那里执行,因此我们将其连接起来。
%    \begin{macrocode}
\g@addto@macro\pc@alloccolumns{%
  \pcpc@InitStacks\pc@columncount
}
%    \end{macrocode}
%
% \subsubsection{Switch stack\\切换堆栈}
%
%    \cs{pcpc@SwitchStack} should be called by marco \cs{colchunk@}.
%    However it is easier to patch \cs{pc@setcolumnwidth} that
%    is executed in \cs{colchunk@} only.
%
%\cs{pcpc@SwitchStack} 应该被宏 \cs{colchunk@} 调用。不过,我们更容易修补仅在 \cs{colchunk@} 中执行的 \cs{pc@setcolumnwidth}。
%    \begin{macrocode}
\g@addto@macro\pc@setcolumnwidth{%
  \pcpc@SwitchStack\pc@columnctr
}
%    \end{macrocode}
%
% \subsubsection{Set current stack color\\设置当前堆栈颜色}
%
%    \cs{pcpc@SetCurrent} is set at the begin of each line.
%    It must be inserted into \cs{pc@placeboxes}. Unhappily
%    there is no easy way. Therefore we check and
%    redefine \cs{pc@placeboxes}.
%
%\cs{pcpc@SetCurrent} 在每行开始时设置。它必须插入到 \cs{pc@placeboxes} 中。不幸的是,没有简单的方法。因此,我们检查并重新定义 \cs{pc@placeboxes}。
%    \begin{macrocode}
\begingroup
  \def\x{%
    \global\let\@tempa\relax
    \count@\z@
    \hb@xt@\linewidth{%
      \vfuzz30ex %
      \vbadness\@M
      \splittopskip\z@skip
      \loop
      \ifnum\count@<\pc@columncount
        \advance\count@\@ne
        \expandafter\ifvoid\csname pc@column@\number\count@\endcsname
          \hskip\csname pc@column@width@\number\count@\endcsname
        \else
          \expandafter\setbox\expandafter\@tempboxa\expandafter
          \vsplit\csname pc@column@\number\count@\endcsname
              to \dp\strutbox
          \vbox{%
            \unvbox\@tempboxa
          }%
        \fi
        \expandafter\ifvoid\csname pc@column@\number\count@\endcsname
        \else
          \global\let\@tempa\pc@placeboxes
        \fi
        \ifnum\count@<\pc@columncount
          \strut
          \hfill
          \ifpc@rulebetween
            \vrule
            \hfill
          \fi
        \fi
      \repeat
    }%
    \@tempa
  }%
  \ifx\x\pc@placeboxes
  \else
    \@PackageWarningNoLine{pdfcolparcolumns}{%
      Command \string\pc@placeboxes\space has changed.\MessageBreak
      Supported versions of package `parcolumns':\MessageBreak
      \space\space 2004/08/05.\MessageBreak
      The redefinition of \string\pc@placeboxes\space may not%
      \MessageBreak
      behave correctly depending on the changes%
    }%
  \fi
\endgroup
%    \end{macrocode}
%    \begin{macro}{\pc@placeboxes}
%    \begin{macrocode}
\renewcommand*{\pc@placeboxes}{%
  \global\let\@tempa\relax
  \count@\z@
  \hb@xt@\linewidth{%
    \vfuzz30ex %
    \vbadness\@M
    \splittopskip\z@skip
    \loop
    \ifnum\count@<\pc@columncount
      \advance\count@\@ne
      \expandafter\ifvoid\csname pc@column@\number\count@\endcsname
        \hskip\csname pc@column@width@\number\count@\endcsname
      \else
        \expandafter\setbox\expandafter\@tempboxa\expandafter
        \vsplit\csname pc@column@\number\count@\endcsname
            to \dp\strutbox
        \vbox{%
          \pcpc@SetCurrent\count@
          \unvbox\@tempboxa
        }%
      \fi
      \expandafter\ifvoid\csname pc@column@\number\count@\endcsname
      \else
        \global\let\@tempa\pc@placeboxes
      \fi
      \ifnum\count@<\pc@columncount
        \strut
        \hfill
        \ifpc@rulebetween
          \begingroup
            \pcpc@RuleBetweenColor
            \vrule
          \endgroup
          \hfill
        \fi
      \fi
    \repeat
  }%
  \@tempa
}
%    \end{macrocode}
%    \end{macro}
%    \begin{macro}{\pcpc@RuleBetweenColorDefault}
%    \begin{macrocode}
\def\pcpc@RuleBetweenColorDefault{%
  \normalcolor
}
%    \end{macrocode}
%    \end{macro}
%    \begin{macro}{\pcpc@RuleBetweenColor}
%    \begin{macrocode}
\let\pcpc@RuleBetweenColor\pcpc@RuleBetweenColorDefault
%    \end{macrocode}
%    \end{macro}
%    \begin{macrocode}
\define@key{parcolumns}{rulebetweencolor}{%
  \edef\pcpc@temp{#1}%
  \ifx\pcpc@temp\@empty
    \let\pcpc@RuleBetweenColor\pcpc@RuleBetweenColorDefault
  \else
    \edef\pcpc@temp{%
      \noexpand\@ifnextchar[{%
        \def\noexpand\pcpc@RuleBetweenColor{%
          \noexpand\color\pcpc@temp
        }%
        \noexpand\pcpc@GobbleNil
      }{%
        \def\noexpand\pcpc@RuleBetweenColor{%
          \noexpand\color{\pcpc@temp}%
        }%
        \noexpand\pcpc@GobbleNil
      }%
      \pcpc@temp\noexpand\@nil
    }%
    \pcpc@temp
  \fi
}
%    \end{macrocode}
%    \begin{macro}{\pcpc@GobbleNil}
%    \begin{macrocode}
\long\def\pcpc@GobbleNil#1\@nil{}
%    \end{macrocode}
%    \end{macro}
%
%    \begin{macrocode}
%</package>
%    \end{macrocode}
%% \section{Installation\\安装}
%
% \subsection{Download\\下载}
%
% \paragraph{Package.} This package is available on
% CTAN\footnote{\CTANpkg{pdfcolparcolumns}}:

%该软件包可从 CTAN\footnote{\CTANpkg{pdfcolparcolumns}} 下载:
% \begin{description}
% \item[\CTAN{macros/latex/contrib/oberdiek/pdfcolparcolumns.dtx}] The source file.
% \item[\CTAN{macros/latex/contrib/oberdiek/pdfcolparcolumns.pdf}] Documentation.
% \end{description}
%
%
% \paragraph{Bundle.} All the packages of the bundle `oberdiek'
% are also available in a TDS compliant ZIP archive. There
% the packages are already unpacked and the documentation files
% are generated. The files and directories obey the TDS standard.
%
%捆绑包“oberdiek”中的所有软件包也可在符合 TDS 标准的 ZIP 存档中获取。在该存档中,软件包已经被解包,文档文件已经生成,文件和目录符合 TDS 标准。
% \begin{description}
% \item[\CTANinstall{install/macros/latex/contrib/oberdiek.tds.zip}]
% \end{description}
% \emph{TDS} refers to the standard ``A Directory Structure
% for \TeX\ Files'' (\CTANpkg{tds}). Directories
% with \xfile{texmf} in their name are usually organized this way.
%
%\emph{TDS} 指的是标准“用于 \TeX\ 文件的目录结构”(\CTANpkg{tds})。名字中包含\xfile{texmf}的目录通常都是按这种方式组织的。
% \subsection{Bundle installation\\捆绑包安装}
%
% \paragraph{Unpacking.} Unpack the \xfile{oberdiek.tds.zip} in the
% TDS tree (also known as \xfile{texmf} tree) of your choice.
% Example (linux):
%
%\paragraph{解包。}在您选择的 TDS 树(也称为\xfile{texmf}树)中解压\xfile{oberdiek.tds.zip}。例如(在Linux中):
% \begin{quote}
%   |unzip oberdiek.tds.zip -d ~/texmf|
% \end{quote}
%
% \subsection{Package installation\\软件包安装}
%
% \paragraph{Unpacking.} The \xfile{.dtx} file is a self-extracting
% \docstrip\ archive. The files are extracted by running the
% \xfile{.dtx} through \plainTeX:
%
%\paragraph{解包。} \xfile{.dtx} 文件是一个自解压的 \docstrip\ 存档。运行\xfile{.dtx}通过\plainTeX\ 来提取文件:
% \begin{quote}
%   \verb|tex pdfcolparcolumns.dtx|
% \end{quote}
%
% \paragraph{TDS.} Now the different files must be moved into
% the different directories in your installation TDS tree
% (also known as \xfile{texmf} tree):
%
%\paragraph{TDS。}现在,不同的文件必须移动到安装 TDS 树(也称为\xfile{texmf}树)中的不同目录中:
% \begin{quote}
% \def\t{^^A
% \begin{tabular}{@{}>{\ttfamily}l@{ $\rightarrow$ }>{\ttfamily}l@{}}
%   pdfcolparcolumns.sty & tex/latex/oberdiek/pdfcolparcolumns.sty\\
%   pdfcolparcolumns.pdf & doc/latex/oberdiek/pdfcolparcolumns.pdf\\
%   pdfcolparcolumns.dtx & source/latex/oberdiek/pdfcolparcolumns.dtx\\
% \end{tabular}^^A
% }^^A
% \sbox0{\t}^^A
% \ifdim\wd0>\linewidth
%   \begingroup
%     \advance\linewidth by\leftmargin
%     \advance\linewidth by\rightmargin
%   \edef\x{\endgroup
%     \def\noexpand\lw{\the\linewidth}^^A
%   }\x
%   \def\lwbox{^^A
%     \leavevmode
%     \hbox to \linewidth{^^A
%       \kern-\leftmargin\relax
%       \hss
%       \usebox0
%       \hss
%       \kern-\rightmargin\relax
%     }^^A
%   }^^A
%   \ifdim\wd0>\lw
%     \sbox0{\small\t}^^A
%     \ifdim\wd0>\linewidth
%       \ifdim\wd0>\lw
%         \sbox0{\footnotesize\t}^^A
%         \ifdim\wd0>\linewidth
%           \ifdim\wd0>\lw
%             \sbox0{\scriptsize\t}^^A
%             \ifdim\wd0>\linewidth
%               \ifdim\wd0>\lw
%                 \sbox0{\tiny\t}^^A
%                 \ifdim\wd0>\linewidth
%                   \lwbox
%                 \else
%                   \usebox0
%                 \fi
%               \else
%                 \lwbox
%               \fi
%             \else
%               \usebox0
%             \fi
%           \else
%             \lwbox
%           \fi
%         \else
%           \usebox0
%         \fi
%       \else
%         \lwbox
%       \fi
%     \else
%       \usebox0
%     \fi
%   \else
%     \lwbox
%   \fi
% \else
%   \usebox0
% \fi
% \end{quote}
% If you have a \xfile{docstrip.cfg} that configures and enables \docstrip's
% TDS installing feature, then some files can already be in the right
% place, see the documentation of \docstrip.
%
%如果你有一个\xfile{docstrip.cfg}文件配置和启用了\docstrip 的TDS安装功能,那么一些文件可能已经位于正确的位置,具体请参见\docstrip 的文档。
% \subsection{Refresh file name databases\\刷新文件名数据库}
%
% If your \TeX~distribution
% (\TeX\,Live, \mikTeX, \dots) relies on file name databases, you must refresh
% these. For example, \TeX\,Live\ users run \verb|texhash| or
% \verb|mktexlsr|.
%
%如果你的\TeX~发行版(\TeX,Live、\mikTeX,等等)依赖于文件名数据库,你必须刷新它们。例如,\TeX,Live用户运行\verb|texhash|或\verb|mktexlsr|。
% \subsection{Some details for the interested\\一些细节供感兴趣的人使用}
%
% \paragraph{Unpacking with \LaTeX.}
% The \xfile{.dtx} chooses its action depending on the format:

%\paragraph{用\LaTeX 进行解包。} \xfile{.dtx}文件根据格式选择操作:
% \begin{description}
% \item[\plainTeX:] Run \docstrip\ and extract the files.
% \item[\LaTeX:] Generate the documentation.
% \end{description}
% If you insist on using \LaTeX\ for \docstrip\ (really,
% \docstrip\ does not need \LaTeX), then inform the autodetect routine
% about your intention:

%如果你坚持要用\LaTeX 进行\docstrip (其实\docstrip 不需要\LaTeX ),那么请告知自动检测例程你的意图:

% \begin{quote}
%   \verb|latex \let\install=y\input{pdfcolparcolumns.dtx}|
% \end{quote}
% Do not forget to quote the argument according to the demands
% of your shell.

%不要忘记根据你的shell的要求引用参数。
%
% \paragraph{Generating the documentation.}
% You can use both the \xfile{.dtx} or the \xfile{.drv} to generate
% the documentation. The process can be configured by the
% configuration file \xfile{ltxdoc.cfg}. For instance, put this
% line into this file, if you want to have A4 as paper format:
%
%\paragraph{生成文档。} 你可以使用\xfile{.dtx}或\xfile{.drv}生成文档。这个过程可以由配置文件\xfile{ltxdoc.cfg}配置。例如,如果你想要A4作为纸张格式,将此行放入文件中:

% \begin{quote}
%   \verb|\PassOptionsToClass{a4paper}{article}|
% \end{quote}
% An example follows how to generate the
% documentation with pdf\LaTeX:

%以下是使用pdf\LaTeX 生成文档的示例:
% \begin{quote}
%\begin{verbatim}
%pdflatex pdfcolparcolumns.dtx
%makeindex -s gind.ist pdfcolparcolumns.idx
%pdflatex pdfcolparcolumns.dtx
%makeindex -s gind.ist pdfcolparcolumns.idx
%pdflatex pdfcolparcolumns.dtx
%\end{verbatim}
% \end{quote}
%
% \begin{thebibliography}{9}
%
% \bibitem{parcolumns}
%   Jonathan Sauer: \textit{The \xpackage{parcolumns} package};
%   2004/11/25;\\
%   \CTANpkg{parcolumns}.
%
% \bibitem{pdfcol}
%   Heiko Oberdiek: \textit{The \xpackage{pdfcol} package};
%   2007/09/09;\\
%   \CTANpkg{pdfcol}.
%
% \end{thebibliography}
%
% \begin{History}
%   \begin{Version}{2007/07/26 v1.0}
%   \item
%     First version, published in the newsgroup \xnewsgroup{comp.text.tex}
%     with the name \xpackage{parcolumns-colorstacks}: ^^A no line break
%     \URL{``\link{Re: \xpackage{xcolor} glitches}''}^^A
%     {https://groups.google.com/group/comp.text.tex/msg/56bd897b11bca414}

%第一个版本,以 \xnewsgroup{comp.text.tex} 新闻组发布,名称为 \xpackage{parcolumns-colorstacks}:^^A 没有换行符 
%\URL{``\link{Re: \xpackage{xcolor} glitches}''}^^A
%{https://groups.google.com/group/comp.text.tex/msg/56bd897b11bca414}
%   \end{Version}
%   \begin{Version}{2007/09/09 v1.1}
%   \item
%     CTAN version, package name renamed to \xpackage{pdfcolparcolumns}.
%
%CTAN 版本,将包名改为 \xpackage{pdfcolparcolumns}。
%   \item
%     Uses package \xpackage{pdfcol}.
%
%使用 \xpackage{pdfcol} 包。
%   \item
%     Documentation added.
%
%添加文档。
%   \item
%     Test file added.
%
%添加测试文件。
%   \end{Version}
%   \begin{Version}{2008/08/11 v1.2}
%   \item
%     Code is not changed.
%
%代码未改动。
%   \item
%     URLs updated.
%
%更新 URL。
%   \end{Version}
%   \begin{Version}{2010/01/11 v1.3}
%   \item
%     Fix for rule color.
%
%修复分隔线的颜色。
%   \item
%     New option \xoption{rulebetweencolor} for environment |parcolumns|.
%
%为环境 |parcolumns| 添加新选项 \xoption{rulebetweencolor}。
%   \end{Version}
%   \begin{Version}{2016/05/16 v1.4}
%   \item
%     Documentation updates.
%
%更新文档。
%   \end{Version}
%   \begin{Version}{2019/12/29 v1.5}
%   \item
%     \cs{PassOptionsToPackage} not \cs{PassoptionsToPackage}
%
%\cs{PassOptionsToPackage} 改为不区分大小写的 \cs{PassoptionsToPackage}。
%   \end{Version}
% \end{History}
%
% \PrintIndex
%
% \Finale
\endinput

%        (quote the arguments according to the demands of your shell)
%
% Documentation:
%    (a) If pdfcolparcolumns.drv is present:
%           latex pdfcolparcolumns.drv
%    (b) Without pdfcolparcolumns.drv:
%           latex pdfcolparcolumns.dtx; ...
%    The class ltxdoc loads the configuration file ltxdoc.cfg
%    if available. Here you can specify further options, e.g.
%    use A4 as paper format:
%       \PassOptionsToClass{a4paper}{article}
%
%    Programm calls to get the documentation (example):
%       pdflatex pdfcolparcolumns.dtx
%       makeindex -s gind.ist pdfcolparcolumns.idx
%       pdflatex pdfcolparcolumns.dtx
%       makeindex -s gind.ist pdfcolparcolumns.idx
%       pdflatex pdfcolparcolumns.dtx
%
% Installation:
%    TDS:tex/latex/oberdiek/pdfcolparcolumns.sty
%    TDS:doc/latex/oberdiek/pdfcolparcolumns.pdf
%    TDS:source/latex/oberdiek/pdfcolparcolumns.dtx
%
%<*ignore>
\begingroup
  \catcode123=1 %
  \catcode125=2 %
  \def\x{LaTeX2e}%
\expandafter\endgroup
\ifcase 0\ifx\install y1\fi\expandafter
         \ifx\csname processbatchFile\endcsname\relax\else1\fi
         \ifx\fmtname\x\else 1\fi\relax
\else\csname fi\endcsname
%</ignore>
%<*install>
\input docstrip.tex
\Msg{************************************************************************}
\Msg{* Installation}
\Msg{* Package: pdfcolparcolumns 2019/12/29 v1.5 Color stacks for parcolumns (HO)}
\Msg{************************************************************************}

\keepsilent
\askforoverwritefalse

\let\MetaPrefix\relax
\preamble

This is a generated file.

Project: pdfcolparcolumns
Version: 2019/12/29 v1.5

Copyright (C)
   2007, 2008, 2010 Heiko Oberdiek
   2016-2019 Oberdiek Package Support Group

This work may be distributed and/or modified under the
conditions of the LaTeX Project Public License, either
version 1.3c of this license or (at your option) any later
version. This version of this license is in
   https://www.latex-project.org/lppl/lppl-1-3c.txt
and the latest version of this license is in
   https://www.latex-project.org/lppl.txt
and version 1.3 or later is part of all distributions of
LaTeX version 2005/12/01 or later.

This work has the LPPL maintenance status "maintained".

The Current Maintainers of this work are
Heiko Oberdiek and the Oberdiek Package Support Group
https://github.com/ho-tex/oberdiek/issues


This work consists of the main source file pdfcolparcolumns.dtx
and the derived files
   pdfcolparcolumns.sty, pdfcolparcolumns.pdf, pdfcolparcolumns.ins,
   pdfcolparcolumns.drv, pdfcolparcolumns-test1.tex.

\endpreamble
\let\MetaPrefix\DoubleperCent

\generate{%
  \file{pdfcolparcolumns.ins}{\from{pdfcolparcolumns.dtx}{install}}%
  \file{pdfcolparcolumns.drv}{\from{pdfcolparcolumns.dtx}{driver}}%
  \usedir{tex/latex/oberdiek}%
  \file{pdfcolparcolumns.sty}{\from{pdfcolparcolumns.dtx}{package}}%
%  \usedir{doc/latex/oberdiek/test}%
%  \file{pdfcolparcolumns-test1.tex}{\from{pdfcolparcolumns.dtx}{test1}}%
}

\catcode32=13\relax% active space
\let =\space%
\Msg{************************************************************************}
\Msg{*}
\Msg{* To finish the installation you have to move the following}
\Msg{* file into a directory searched by TeX:}
\Msg{*}
\Msg{*     pdfcolparcolumns.sty}
\Msg{*}
\Msg{* To produce the documentation run the file `pdfcolparcolumns.drv'}
\Msg{* through LaTeX.}
\Msg{*}
\Msg{* Happy TeXing!}
\Msg{*}
\Msg{************************************************************************}

\endbatchfile
%</install>
%<*ignore>
\fi
%</ignore>
%<*driver>
\NeedsTeXFormat{LaTeX2e}
\ProvidesFile{pdfcolparcolumns.drv}%
  [2019/12/29 v1.5 Color stacks for parcolumns (HO)]%
\documentclass{ltxdoc}
\usepackage{holtxdoc}[2011/11/22]
\usepackage[scheme=plain]{ctex}
\setCJKmainfont{方正书宋_GBK}%方正书宋_GBK.TTF  设置文本的中文有衬线字体为“方正书宋_GBK”
\setCJKsansfont{方正黑体简体}%方正黑体_GBK.TTF  设置文本的中文无衬线字体为“方正黑体简体”
\setCJKmonofont{方正书宋简体}%方正仿宋_GBK.TTF  设置文本的中文等宽字体为“方正书宋简体”
\begin{document}
  \DocInput{pdfcolparcolumns.dtx}%
\end{document}
%</driver>
% \fi
%
%
%
% \GetFileInfo{pdfcolparcolumns.drv}
%
% \title{The \xpackage{pdfcolparcolumns} package}
% \date{2019/12/29 v1.5}
% \author{Heiko Oberdiek\thanks
% {Please report any issues at \url{https://github.com/ho-tex/oberdiek/issues}}\and 翻译\\{\tt virhuiai@qq.com}}
%
% \maketitle
%
% \begin{abstract}
%\makebox[0pt]{}
% \begin{parcolumns}[nofirstindent]{2}
%\colchunk{Since version 1.40 \pdfTeX\ supports several color stacks.
%This package uses them to fix color problems in
%package \xpackage{parcolumns}.}
%\colchunk{自从版本1.40起,\pdfTeX 支持多个颜色栈。
%此包使用它们来解决 \xpackage{parcolumns} 包中的颜色问题。}
%\end{parcolumns}



% \end{abstract}
%
% \tableofcontents
%
% \section{Usage\\用法}
%
% \begin{quote}
% |\usepackage{pdfcolparcolumns}|
% \end{quote}
% The package \xpackage{pdfcolparcolumns} loads package \xpackage{parcolums}
% \cite{parcolumns}. If color stacks are available then the
% macros of \xpackage{parcolumns} are patched to add support
% for color stacks.

%包 \xpackage{pdfcolparcolumns} 载入了 \xpackage{parcolumns} \cite{parcolumns} 包。如果有可用的颜色栈,则会对 \xpackage{parcolumns} 的宏进行修补,以添加对颜色栈的支持。

%
% \subsection{Option \xoption{rulebetweencolor}\\选项 \xoption{rulebetweencolor}}
%
% Package \xpackage{pdfcolparcolumns} also fixes the color for the
% rule between columns (if \xoption{rulebetween} is set).
% Default color is \cs{normalcolor}. But this can be changed by using
% option \xoption{rulebetweencolor}. It takes a color specification
% as value. If the value is empty, then the default (\cs{normalcolor})
% is used.
% Examples:

% 包 \xpackage{pdfcolparcolumns} 还修复了列之间分隔线的颜色(如果设置了 \xoption{rulebetween})。默认颜色为 \cs{normalcolor}。但是,可以使用选项 \xoption{rulebetweencolor} 来更改它。它接受一个颜色规范作为值。如果该值为空,则使用默认值(\cs{normalcolor})。例如:

% \begin{quote}
%   |rulebetweencolor=blue|,\\
%   |rulebetweencolor={red}|,\\
%   |rulebetweencolor={}|, \textit{\% \cs{normalcolor} is used}\\
%   |rulebetweencolor=[rgb]{1,0,.5}| \textit{\% see below}
% \end{quote}
% If used inside the optional argument of environment \textsf{parcolumns}
% and the value contains an optional argument, then whole value
% must be put in curly braces:

% 如果在 \textsf{parcolumns} 环境的可选参数中使用,并且值包含可选参数,则整个值必须用花括号括起来:
%\begin{quote}
%\begin{verbatim}
%\begin{parcolumns}[
%  rulebetween,
%  rulebetweencolor={[rgb]{1,0,.5}},
%]{2}
%  ...
%\end{parcolumns}
%\end{verbatim}
%\end{quote}
% This option \xoption{rulebetweencolor} can also be set using
% \cs{setkeys}:

%也可以使用 \cs{setkeys} 来设置选项 \xoption{rulebetweencolor}:
%\begin{quote}
%\begin{verbatim}
%\setkeys{parcolumns}{rulebetweencolor=green}
%\end{verbatim}
%\end{quote}
%
% \subsection{Future\\未来展望}
%
% Currently package \xpackage{parcolumns} does not seem to be
% maintained. Nevertheless if there will be a new version that
% adds support for color stacks, then this package may become
% obsolete.
%
%目前,\xpackage{parcolumns} 包似乎没有维护了。尽管如此,如果出现了新版本并支持颜色栈,那么这个包可能会变得过时。
% \StopEventually{
% }
%
% \section{Implementation\\实现}
%
% \subsection{Identification\\标识}
%
%    \begin{macrocode}
%<*package>
\NeedsTeXFormat{LaTeX2e}
\ProvidesPackage{pdfcolparcolumns}%
  [2019/12/29 v1.5 Color stacks for parcolumns (HO)]%
%    \end{macrocode}
%
% \subsection{Load packages\\加载宏包}
%
% \subsubsection{Package \xpackage{parcolumns}\\宏包 \xpackage{parcolumns}}
%
%    Currently package \xpackage{parcolumns} does not define options.
%    Thus it is just a precaution that the options of
%    package \xpackage{pdfcolparcolumns} are passed to
%    package \xpackage{parcolumns}.

%目前,宏包 \xpackage{parcolumns} 没有定义选项。因此,它只是一个预防措施,以确保将宏包 \xpackage{pdfcolparcolumns} 的选项传递给宏包 \xpackage{parcolumns}。
%    \begin{macrocode}
\DeclareOption*{%
  \PassOptionsToPackage{\CurrentOption}{parcolumns}%
}
\ProcessOptions\relax
\RequirePackage{parcolumns}[2004/11/25]
%    \end{macrocode}
%
% \subsubsection{Package \xpackage{pdfcol}\\宏包 \xpackage{pdfcol}}
%
%    \begin{macrocode}
\RequirePackage{pdfcol}[2007/09/09]
\ifpdfcolAvailable
\else
  \PackageInfo{pdfcolparcolumns}{%
    Loading aborted, because color stacks are not available%
  }%
  \expandafter\endinput
\fi
%    \end{macrocode}
%
% \subsubsection{Package \xpackage{infwarerr}\\宏包 \xpackage{infwarerr}}
%
%    \begin{macrocode}
\RequirePackage{infwarerr}[2007/09/09]
%    \end{macrocode}
%
% \subsection{Color stack macros\\颜色堆栈宏}
%
%    \begin{macro}{\pcpc@MaxStack}
%    Macro \cs{pcpc@MaxStack} holds the highest number of
%    allocated stacks.

%宏 \cs{pcpc@MaxStack} 存储已分配的堆栈中最大的编号。
%    \begin{macrocode}
\global\chardef\pcpc@MaxStack=\z@
%    \end{macrocode}
%    \end{macro}
%    \begin{macro}{\pcpc@InitStacks}
%    Macro \cs{pcpc@InitStacks} takes the number of columns
%    as argument and ensures that there are enough color
%    stacks for all columns.
%
%宏 \cs{pcpc@InitStacks} 接受列数作为参数,并确保为所有列分配了足够的颜色堆栈。
%    \begin{macrocode}
\def\pcpc@InitStacks#1{%
  \ifnum#1>\pcpc@MaxStack
    \begingroup
      \count@\pcpc@MaxStack
      \loop
        \advance\count@\@ne
        \pdfcolInitStack{pcpc@\the\count@}%
      \ifnum#1>\count@
      \repeat
      \global\chardef\pcpc@MaxStack=\count@
    \endgroup
  \fi
}
%    \end{macrocode}
%    \end{macro}
%
%    \begin{macro}{\pcpc@SwitchStack}
%    \begin{macrocode}
\def\pcpc@SwitchStack#1{%
  \pdfcolSwitchStack{pcpc@\number#1}%
}
%    \end{macrocode}
%    \end{macro}
%
%    \begin{macro}{\pcpc@SetCurrent}
%    \begin{macrocode}
\def\pcpc@SetCurrent#1{%
  \pdfcolSetCurrent{pcpc@\number#1}%
}
%    \end{macrocode}
%    \end{macro}
%
% \subsection{Patches\\补丁}
%
%     Now the color stack macros are patched into the macros
%     of package \xpackage{parcolumns}.
%
%现在颜色堆栈宏已经被打补丁到了 \xpackage{parcolumns} 宏包的宏中。
% \subsubsection{Init stacks\\初始化堆栈}
%
%    \cs{pcpc@InitStacks} should go into the definition of
%    environment |parcolumns|. \cs{pc@alloccolumns} is executed
%    there and nowhere else, thus we hook into it.

%\cs{pcpc@InitStacks} 应该被放在环境 |parcolumns| 的定义中。\cs{pc@alloccolumns} 仅在那里执行,因此我们将其连接起来。
%    \begin{macrocode}
\g@addto@macro\pc@alloccolumns{%
  \pcpc@InitStacks\pc@columncount
}
%    \end{macrocode}
%
% \subsubsection{Switch stack\\切换堆栈}
%
%    \cs{pcpc@SwitchStack} should be called by marco \cs{colchunk@}.
%    However it is easier to patch \cs{pc@setcolumnwidth} that
%    is executed in \cs{colchunk@} only.
%
%\cs{pcpc@SwitchStack} 应该被宏 \cs{colchunk@} 调用。不过,我们更容易修补仅在 \cs{colchunk@} 中执行的 \cs{pc@setcolumnwidth}。
%    \begin{macrocode}
\g@addto@macro\pc@setcolumnwidth{%
  \pcpc@SwitchStack\pc@columnctr
}
%    \end{macrocode}
%
% \subsubsection{Set current stack color\\设置当前堆栈颜色}
%
%    \cs{pcpc@SetCurrent} is set at the begin of each line.
%    It must be inserted into \cs{pc@placeboxes}. Unhappily
%    there is no easy way. Therefore we check and
%    redefine \cs{pc@placeboxes}.
%
%\cs{pcpc@SetCurrent} 在每行开始时设置。它必须插入到 \cs{pc@placeboxes} 中。不幸的是,没有简单的方法。因此,我们检查并重新定义 \cs{pc@placeboxes}。
%    \begin{macrocode}
\begingroup
  \def\x{%
    \global\let\@tempa\relax
    \count@\z@
    \hb@xt@\linewidth{%
      \vfuzz30ex %
      \vbadness\@M
      \splittopskip\z@skip
      \loop
      \ifnum\count@<\pc@columncount
        \advance\count@\@ne
        \expandafter\ifvoid\csname pc@column@\number\count@\endcsname
          \hskip\csname pc@column@width@\number\count@\endcsname
        \else
          \expandafter\setbox\expandafter\@tempboxa\expandafter
          \vsplit\csname pc@column@\number\count@\endcsname
              to \dp\strutbox
          \vbox{%
            \unvbox\@tempboxa
          }%
        \fi
        \expandafter\ifvoid\csname pc@column@\number\count@\endcsname
        \else
          \global\let\@tempa\pc@placeboxes
        \fi
        \ifnum\count@<\pc@columncount
          \strut
          \hfill
          \ifpc@rulebetween
            \vrule
            \hfill
          \fi
        \fi
      \repeat
    }%
    \@tempa
  }%
  \ifx\x\pc@placeboxes
  \else
    \@PackageWarningNoLine{pdfcolparcolumns}{%
      Command \string\pc@placeboxes\space has changed.\MessageBreak
      Supported versions of package `parcolumns':\MessageBreak
      \space\space 2004/08/05.\MessageBreak
      The redefinition of \string\pc@placeboxes\space may not%
      \MessageBreak
      behave correctly depending on the changes%
    }%
  \fi
\endgroup
%    \end{macrocode}
%    \begin{macro}{\pc@placeboxes}
%    \begin{macrocode}
\renewcommand*{\pc@placeboxes}{%
  \global\let\@tempa\relax
  \count@\z@
  \hb@xt@\linewidth{%
    \vfuzz30ex %
    \vbadness\@M
    \splittopskip\z@skip
    \loop
    \ifnum\count@<\pc@columncount
      \advance\count@\@ne
      \expandafter\ifvoid\csname pc@column@\number\count@\endcsname
        \hskip\csname pc@column@width@\number\count@\endcsname
      \else
        \expandafter\setbox\expandafter\@tempboxa\expandafter
        \vsplit\csname pc@column@\number\count@\endcsname
            to \dp\strutbox
        \vbox{%
          \pcpc@SetCurrent\count@
          \unvbox\@tempboxa
        }%
      \fi
      \expandafter\ifvoid\csname pc@column@\number\count@\endcsname
      \else
        \global\let\@tempa\pc@placeboxes
      \fi
      \ifnum\count@<\pc@columncount
        \strut
        \hfill
        \ifpc@rulebetween
          \begingroup
            \pcpc@RuleBetweenColor
            \vrule
          \endgroup
          \hfill
        \fi
      \fi
    \repeat
  }%
  \@tempa
}
%    \end{macrocode}
%    \end{macro}
%    \begin{macro}{\pcpc@RuleBetweenColorDefault}
%    \begin{macrocode}
\def\pcpc@RuleBetweenColorDefault{%
  \normalcolor
}
%    \end{macrocode}
%    \end{macro}
%    \begin{macro}{\pcpc@RuleBetweenColor}
%    \begin{macrocode}
\let\pcpc@RuleBetweenColor\pcpc@RuleBetweenColorDefault
%    \end{macrocode}
%    \end{macro}
%    \begin{macrocode}
\define@key{parcolumns}{rulebetweencolor}{%
  \edef\pcpc@temp{#1}%
  \ifx\pcpc@temp\@empty
    \let\pcpc@RuleBetweenColor\pcpc@RuleBetweenColorDefault
  \else
    \edef\pcpc@temp{%
      \noexpand\@ifnextchar[{%
        \def\noexpand\pcpc@RuleBetweenColor{%
          \noexpand\color\pcpc@temp
        }%
        \noexpand\pcpc@GobbleNil
      }{%
        \def\noexpand\pcpc@RuleBetweenColor{%
          \noexpand\color{\pcpc@temp}%
        }%
        \noexpand\pcpc@GobbleNil
      }%
      \pcpc@temp\noexpand\@nil
    }%
    \pcpc@temp
  \fi
}
%    \end{macrocode}
%    \begin{macro}{\pcpc@GobbleNil}
%    \begin{macrocode}
\long\def\pcpc@GobbleNil#1\@nil{}
%    \end{macrocode}
%    \end{macro}
%
%    \begin{macrocode}
%</package>
%    \end{macrocode}
%% \section{Installation\\安装}
%
% \subsection{Download\\下载}
%
% \paragraph{Package.} This package is available on
% CTAN\footnote{\CTANpkg{pdfcolparcolumns}}:

%该软件包可从 CTAN\footnote{\CTANpkg{pdfcolparcolumns}} 下载:
% \begin{description}
% \item[\CTAN{macros/latex/contrib/oberdiek/pdfcolparcolumns.dtx}] The source file.
% \item[\CTAN{macros/latex/contrib/oberdiek/pdfcolparcolumns.pdf}] Documentation.
% \end{description}
%
%
% \paragraph{Bundle.} All the packages of the bundle `oberdiek'
% are also available in a TDS compliant ZIP archive. There
% the packages are already unpacked and the documentation files
% are generated. The files and directories obey the TDS standard.
%
%捆绑包“oberdiek”中的所有软件包也可在符合 TDS 标准的 ZIP 存档中获取。在该存档中,软件包已经被解包,文档文件已经生成,文件和目录符合 TDS 标准。
% \begin{description}
% \item[\CTANinstall{install/macros/latex/contrib/oberdiek.tds.zip}]
% \end{description}
% \emph{TDS} refers to the standard ``A Directory Structure
% for \TeX\ Files'' (\CTANpkg{tds}). Directories
% with \xfile{texmf} in their name are usually organized this way.
%
%\emph{TDS} 指的是标准“用于 \TeX\ 文件的目录结构”(\CTANpkg{tds})。名字中包含\xfile{texmf}的目录通常都是按这种方式组织的。
% \subsection{Bundle installation\\捆绑包安装}
%
% \paragraph{Unpacking.} Unpack the \xfile{oberdiek.tds.zip} in the
% TDS tree (also known as \xfile{texmf} tree) of your choice.
% Example (linux):
%
%\paragraph{解包。}在您选择的 TDS 树(也称为\xfile{texmf}树)中解压\xfile{oberdiek.tds.zip}。例如(在Linux中):
% \begin{quote}
%   |unzip oberdiek.tds.zip -d ~/texmf|
% \end{quote}
%
% \subsection{Package installation\\软件包安装}
%
% \paragraph{Unpacking.} The \xfile{.dtx} file is a self-extracting
% \docstrip\ archive. The files are extracted by running the
% \xfile{.dtx} through \plainTeX:
%
%\paragraph{解包。} \xfile{.dtx} 文件是一个自解压的 \docstrip\ 存档。运行\xfile{.dtx}通过\plainTeX\ 来提取文件:
% \begin{quote}
%   \verb|tex pdfcolparcolumns.dtx|
% \end{quote}
%
% \paragraph{TDS.} Now the different files must be moved into
% the different directories in your installation TDS tree
% (also known as \xfile{texmf} tree):
%
%\paragraph{TDS。}现在,不同的文件必须移动到安装 TDS 树(也称为\xfile{texmf}树)中的不同目录中:
% \begin{quote}
% \def\t{^^A
% \begin{tabular}{@{}>{\ttfamily}l@{ $\rightarrow$ }>{\ttfamily}l@{}}
%   pdfcolparcolumns.sty & tex/latex/oberdiek/pdfcolparcolumns.sty\\
%   pdfcolparcolumns.pdf & doc/latex/oberdiek/pdfcolparcolumns.pdf\\
%   pdfcolparcolumns.dtx & source/latex/oberdiek/pdfcolparcolumns.dtx\\
% \end{tabular}^^A
% }^^A
% \sbox0{\t}^^A
% \ifdim\wd0>\linewidth
%   \begingroup
%     \advance\linewidth by\leftmargin
%     \advance\linewidth by\rightmargin
%   \edef\x{\endgroup
%     \def\noexpand\lw{\the\linewidth}^^A
%   }\x
%   \def\lwbox{^^A
%     \leavevmode
%     \hbox to \linewidth{^^A
%       \kern-\leftmargin\relax
%       \hss
%       \usebox0
%       \hss
%       \kern-\rightmargin\relax
%     }^^A
%   }^^A
%   \ifdim\wd0>\lw
%     \sbox0{\small\t}^^A
%     \ifdim\wd0>\linewidth
%       \ifdim\wd0>\lw
%         \sbox0{\footnotesize\t}^^A
%         \ifdim\wd0>\linewidth
%           \ifdim\wd0>\lw
%             \sbox0{\scriptsize\t}^^A
%             \ifdim\wd0>\linewidth
%               \ifdim\wd0>\lw
%                 \sbox0{\tiny\t}^^A
%                 \ifdim\wd0>\linewidth
%                   \lwbox
%                 \else
%                   \usebox0
%                 \fi
%               \else
%                 \lwbox
%               \fi
%             \else
%               \usebox0
%             \fi
%           \else
%             \lwbox
%           \fi
%         \else
%           \usebox0
%         \fi
%       \else
%         \lwbox
%       \fi
%     \else
%       \usebox0
%     \fi
%   \else
%     \lwbox
%   \fi
% \else
%   \usebox0
% \fi
% \end{quote}
% If you have a \xfile{docstrip.cfg} that configures and enables \docstrip's
% TDS installing feature, then some files can already be in the right
% place, see the documentation of \docstrip.
%
%如果你有一个\xfile{docstrip.cfg}文件配置和启用了\docstrip 的TDS安装功能,那么一些文件可能已经位于正确的位置,具体请参见\docstrip 的文档。
% \subsection{Refresh file name databases\\刷新文件名数据库}
%
% If your \TeX~distribution
% (\TeX\,Live, \mikTeX, \dots) relies on file name databases, you must refresh
% these. For example, \TeX\,Live\ users run \verb|texhash| or
% \verb|mktexlsr|.
%
%如果你的\TeX~发行版(\TeX,Live、\mikTeX,等等)依赖于文件名数据库,你必须刷新它们。例如,\TeX,Live用户运行\verb|texhash|或\verb|mktexlsr|。
% \subsection{Some details for the interested\\一些细节供感兴趣的人使用}
%
% \paragraph{Unpacking with \LaTeX.}
% The \xfile{.dtx} chooses its action depending on the format:

%\paragraph{用\LaTeX 进行解包。} \xfile{.dtx}文件根据格式选择操作:
% \begin{description}
% \item[\plainTeX:] Run \docstrip\ and extract the files.
% \item[\LaTeX:] Generate the documentation.
% \end{description}
% If you insist on using \LaTeX\ for \docstrip\ (really,
% \docstrip\ does not need \LaTeX), then inform the autodetect routine
% about your intention:

%如果你坚持要用\LaTeX 进行\docstrip (其实\docstrip 不需要\LaTeX ),那么请告知自动检测例程你的意图:

% \begin{quote}
%   \verb|latex \let\install=y% \iffalse meta-comment
%
% File: pdfcolparcolumns.dtx
% Version: 2019/12/29 v1.5
% Info: Color stacks for parcolumns
%
% Copyright (C)
%    2007, 2008, 2010 Heiko Oberdiek
%    2016-2019 Oberdiek Package Support Group
%    https://github.com/ho-tex/oberdiek/issues
%
% This work may be distributed and/or modified under the
% conditions of the LaTeX Project Public License, either
% version 1.3c of this license or (at your option) any later
% version. This version of this license is in
%    https://www.latex-project.org/lppl/lppl-1-3c.txt
% and the latest version of this license is in
%    https://www.latex-project.org/lppl.txt
% and version 1.3 or later is part of all distributions of
% LaTeX version 2005/12/01 or later.
%
% This work has the LPPL maintenance status "maintained".
%
% The Current Maintainers of this work are
% Heiko Oberdiek and the Oberdiek Package Support Group
% https://github.com/ho-tex/oberdiek/issues
%
% This work consists of the main source file pdfcolparcolumns.dtx
% and the derived files
%    pdfcolparcolumns.sty, pdfcolparcolumns.pdf, pdfcolparcolumns.ins,
%    pdfcolparcolumns.drv, pdfcolparcolumns-test1.tex.
%
% Distribution:
%    CTAN:macros/latex/contrib/oberdiek/pdfcolparcolumns.dtx
%    CTAN:macros/latex/contrib/oberdiek/pdfcolparcolumns.pdf
%
% Unpacking:
%    (a) If pdfcolparcolumns.ins is present:
%           tex pdfcolparcolumns.ins
%    (b) Without pdfcolparcolumns.ins:
%           tex pdfcolparcolumns.dtx
%    (c) If you insist on using LaTeX
%           latex \let\install=y\input{pdfcolparcolumns.dtx}
%        (quote the arguments according to the demands of your shell)
%
% Documentation:
%    (a) If pdfcolparcolumns.drv is present:
%           latex pdfcolparcolumns.drv
%    (b) Without pdfcolparcolumns.drv:
%           latex pdfcolparcolumns.dtx; ...
%    The class ltxdoc loads the configuration file ltxdoc.cfg
%    if available. Here you can specify further options, e.g.
%    use A4 as paper format:
%       \PassOptionsToClass{a4paper}{article}
%
%    Programm calls to get the documentation (example):
%       pdflatex pdfcolparcolumns.dtx
%       makeindex -s gind.ist pdfcolparcolumns.idx
%       pdflatex pdfcolparcolumns.dtx
%       makeindex -s gind.ist pdfcolparcolumns.idx
%       pdflatex pdfcolparcolumns.dtx
%
% Installation:
%    TDS:tex/latex/oberdiek/pdfcolparcolumns.sty
%    TDS:doc/latex/oberdiek/pdfcolparcolumns.pdf
%    TDS:source/latex/oberdiek/pdfcolparcolumns.dtx
%
%<*ignore>
\begingroup
  \catcode123=1 %
  \catcode125=2 %
  \def\x{LaTeX2e}%
\expandafter\endgroup
\ifcase 0\ifx\install y1\fi\expandafter
         \ifx\csname processbatchFile\endcsname\relax\else1\fi
         \ifx\fmtname\x\else 1\fi\relax
\else\csname fi\endcsname
%</ignore>
%<*install>
\input docstrip.tex
\Msg{************************************************************************}
\Msg{* Installation}
\Msg{* Package: pdfcolparcolumns 2019/12/29 v1.5 Color stacks for parcolumns (HO)}
\Msg{************************************************************************}

\keepsilent
\askforoverwritefalse

\let\MetaPrefix\relax
\preamble

This is a generated file.

Project: pdfcolparcolumns
Version: 2019/12/29 v1.5

Copyright (C)
   2007, 2008, 2010 Heiko Oberdiek
   2016-2019 Oberdiek Package Support Group

This work may be distributed and/or modified under the
conditions of the LaTeX Project Public License, either
version 1.3c of this license or (at your option) any later
version. This version of this license is in
   https://www.latex-project.org/lppl/lppl-1-3c.txt
and the latest version of this license is in
   https://www.latex-project.org/lppl.txt
and version 1.3 or later is part of all distributions of
LaTeX version 2005/12/01 or later.

This work has the LPPL maintenance status "maintained".

The Current Maintainers of this work are
Heiko Oberdiek and the Oberdiek Package Support Group
https://github.com/ho-tex/oberdiek/issues


This work consists of the main source file pdfcolparcolumns.dtx
and the derived files
   pdfcolparcolumns.sty, pdfcolparcolumns.pdf, pdfcolparcolumns.ins,
   pdfcolparcolumns.drv, pdfcolparcolumns-test1.tex.

\endpreamble
\let\MetaPrefix\DoubleperCent

\generate{%
  \file{pdfcolparcolumns.ins}{\from{pdfcolparcolumns.dtx}{install}}%
  \file{pdfcolparcolumns.drv}{\from{pdfcolparcolumns.dtx}{driver}}%
  \usedir{tex/latex/oberdiek}%
  \file{pdfcolparcolumns.sty}{\from{pdfcolparcolumns.dtx}{package}}%
%  \usedir{doc/latex/oberdiek/test}%
%  \file{pdfcolparcolumns-test1.tex}{\from{pdfcolparcolumns.dtx}{test1}}%
}

\catcode32=13\relax% active space
\let =\space%
\Msg{************************************************************************}
\Msg{*}
\Msg{* To finish the installation you have to move the following}
\Msg{* file into a directory searched by TeX:}
\Msg{*}
\Msg{*     pdfcolparcolumns.sty}
\Msg{*}
\Msg{* To produce the documentation run the file `pdfcolparcolumns.drv'}
\Msg{* through LaTeX.}
\Msg{*}
\Msg{* Happy TeXing!}
\Msg{*}
\Msg{************************************************************************}

\endbatchfile
%</install>
%<*ignore>
\fi
%</ignore>
%<*driver>
\NeedsTeXFormat{LaTeX2e}
\ProvidesFile{pdfcolparcolumns.drv}%
  [2019/12/29 v1.5 Color stacks for parcolumns (HO)]%
\documentclass{ltxdoc}
\usepackage{holtxdoc}[2011/11/22]
\usepackage[scheme=plain]{ctex}
\setCJKmainfont{方正书宋_GBK}%方正书宋_GBK.TTF  设置文本的中文有衬线字体为“方正书宋_GBK”
\setCJKsansfont{方正黑体简体}%方正黑体_GBK.TTF  设置文本的中文无衬线字体为“方正黑体简体”
\setCJKmonofont{方正书宋简体}%方正仿宋_GBK.TTF  设置文本的中文等宽字体为“方正书宋简体”
\begin{document}
  \DocInput{pdfcolparcolumns.dtx}%
\end{document}
%</driver>
% \fi
%
%
%
% \GetFileInfo{pdfcolparcolumns.drv}
%
% \title{The \xpackage{pdfcolparcolumns} package}
% \date{2019/12/29 v1.5}
% \author{Heiko Oberdiek\thanks
% {Please report any issues at \url{https://github.com/ho-tex/oberdiek/issues}}\and 翻译\\{\tt virhuiai@qq.com}}
%
% \maketitle
%
% \begin{abstract}
%\makebox[0pt]{}
% \begin{parcolumns}[nofirstindent]{2}
%\colchunk{Since version 1.40 \pdfTeX\ supports several color stacks.
%This package uses them to fix color problems in
%package \xpackage{parcolumns}.}
%\colchunk{自从版本1.40起,\pdfTeX 支持多个颜色栈。
%此包使用它们来解决 \xpackage{parcolumns} 包中的颜色问题。}
%\end{parcolumns}



% \end{abstract}
%
% \tableofcontents
%
% \section{Usage\\用法}
%
% \begin{quote}
% |\usepackage{pdfcolparcolumns}|
% \end{quote}
% The package \xpackage{pdfcolparcolumns} loads package \xpackage{parcolums}
% \cite{parcolumns}. If color stacks are available then the
% macros of \xpackage{parcolumns} are patched to add support
% for color stacks.

%包 \xpackage{pdfcolparcolumns} 载入了 \xpackage{parcolumns} \cite{parcolumns} 包。如果有可用的颜色栈,则会对 \xpackage{parcolumns} 的宏进行修补,以添加对颜色栈的支持。

%
% \subsection{Option \xoption{rulebetweencolor}\\选项 \xoption{rulebetweencolor}}
%
% Package \xpackage{pdfcolparcolumns} also fixes the color for the
% rule between columns (if \xoption{rulebetween} is set).
% Default color is \cs{normalcolor}. But this can be changed by using
% option \xoption{rulebetweencolor}. It takes a color specification
% as value. If the value is empty, then the default (\cs{normalcolor})
% is used.
% Examples:

% 包 \xpackage{pdfcolparcolumns} 还修复了列之间分隔线的颜色(如果设置了 \xoption{rulebetween})。默认颜色为 \cs{normalcolor}。但是,可以使用选项 \xoption{rulebetweencolor} 来更改它。它接受一个颜色规范作为值。如果该值为空,则使用默认值(\cs{normalcolor})。例如:

% \begin{quote}
%   |rulebetweencolor=blue|,\\
%   |rulebetweencolor={red}|,\\
%   |rulebetweencolor={}|, \textit{\% \cs{normalcolor} is used}\\
%   |rulebetweencolor=[rgb]{1,0,.5}| \textit{\% see below}
% \end{quote}
% If used inside the optional argument of environment \textsf{parcolumns}
% and the value contains an optional argument, then whole value
% must be put in curly braces:

% 如果在 \textsf{parcolumns} 环境的可选参数中使用,并且值包含可选参数,则整个值必须用花括号括起来:
%\begin{quote}
%\begin{verbatim}
%\begin{parcolumns}[
%  rulebetween,
%  rulebetweencolor={[rgb]{1,0,.5}},
%]{2}
%  ...
%\end{parcolumns}
%\end{verbatim}
%\end{quote}
% This option \xoption{rulebetweencolor} can also be set using
% \cs{setkeys}:

%也可以使用 \cs{setkeys} 来设置选项 \xoption{rulebetweencolor}:
%\begin{quote}
%\begin{verbatim}
%\setkeys{parcolumns}{rulebetweencolor=green}
%\end{verbatim}
%\end{quote}
%
% \subsection{Future\\未来展望}
%
% Currently package \xpackage{parcolumns} does not seem to be
% maintained. Nevertheless if there will be a new version that
% adds support for color stacks, then this package may become
% obsolete.
%
%目前,\xpackage{parcolumns} 包似乎没有维护了。尽管如此,如果出现了新版本并支持颜色栈,那么这个包可能会变得过时。
% \StopEventually{
% }
%
% \section{Implementation\\实现}
%
% \subsection{Identification\\标识}
%
%    \begin{macrocode}
%<*package>
\NeedsTeXFormat{LaTeX2e}
\ProvidesPackage{pdfcolparcolumns}%
  [2019/12/29 v1.5 Color stacks for parcolumns (HO)]%
%    \end{macrocode}
%
% \subsection{Load packages\\加载宏包}
%
% \subsubsection{Package \xpackage{parcolumns}\\宏包 \xpackage{parcolumns}}
%
%    Currently package \xpackage{parcolumns} does not define options.
%    Thus it is just a precaution that the options of
%    package \xpackage{pdfcolparcolumns} are passed to
%    package \xpackage{parcolumns}.

%目前,宏包 \xpackage{parcolumns} 没有定义选项。因此,它只是一个预防措施,以确保将宏包 \xpackage{pdfcolparcolumns} 的选项传递给宏包 \xpackage{parcolumns}。
%    \begin{macrocode}
\DeclareOption*{%
  \PassOptionsToPackage{\CurrentOption}{parcolumns}%
}
\ProcessOptions\relax
\RequirePackage{parcolumns}[2004/11/25]
%    \end{macrocode}
%
% \subsubsection{Package \xpackage{pdfcol}\\宏包 \xpackage{pdfcol}}
%
%    \begin{macrocode}
\RequirePackage{pdfcol}[2007/09/09]
\ifpdfcolAvailable
\else
  \PackageInfo{pdfcolparcolumns}{%
    Loading aborted, because color stacks are not available%
  }%
  \expandafter\endinput
\fi
%    \end{macrocode}
%
% \subsubsection{Package \xpackage{infwarerr}\\宏包 \xpackage{infwarerr}}
%
%    \begin{macrocode}
\RequirePackage{infwarerr}[2007/09/09]
%    \end{macrocode}
%
% \subsection{Color stack macros\\颜色堆栈宏}
%
%    \begin{macro}{\pcpc@MaxStack}
%    Macro \cs{pcpc@MaxStack} holds the highest number of
%    allocated stacks.

%宏 \cs{pcpc@MaxStack} 存储已分配的堆栈中最大的编号。
%    \begin{macrocode}
\global\chardef\pcpc@MaxStack=\z@
%    \end{macrocode}
%    \end{macro}
%    \begin{macro}{\pcpc@InitStacks}
%    Macro \cs{pcpc@InitStacks} takes the number of columns
%    as argument and ensures that there are enough color
%    stacks for all columns.
%
%宏 \cs{pcpc@InitStacks} 接受列数作为参数,并确保为所有列分配了足够的颜色堆栈。
%    \begin{macrocode}
\def\pcpc@InitStacks#1{%
  \ifnum#1>\pcpc@MaxStack
    \begingroup
      \count@\pcpc@MaxStack
      \loop
        \advance\count@\@ne
        \pdfcolInitStack{pcpc@\the\count@}%
      \ifnum#1>\count@
      \repeat
      \global\chardef\pcpc@MaxStack=\count@
    \endgroup
  \fi
}
%    \end{macrocode}
%    \end{macro}
%
%    \begin{macro}{\pcpc@SwitchStack}
%    \begin{macrocode}
\def\pcpc@SwitchStack#1{%
  \pdfcolSwitchStack{pcpc@\number#1}%
}
%    \end{macrocode}
%    \end{macro}
%
%    \begin{macro}{\pcpc@SetCurrent}
%    \begin{macrocode}
\def\pcpc@SetCurrent#1{%
  \pdfcolSetCurrent{pcpc@\number#1}%
}
%    \end{macrocode}
%    \end{macro}
%
% \subsection{Patches\\补丁}
%
%     Now the color stack macros are patched into the macros
%     of package \xpackage{parcolumns}.
%
%现在颜色堆栈宏已经被打补丁到了 \xpackage{parcolumns} 宏包的宏中。
% \subsubsection{Init stacks\\初始化堆栈}
%
%    \cs{pcpc@InitStacks} should go into the definition of
%    environment |parcolumns|. \cs{pc@alloccolumns} is executed
%    there and nowhere else, thus we hook into it.

%\cs{pcpc@InitStacks} 应该被放在环境 |parcolumns| 的定义中。\cs{pc@alloccolumns} 仅在那里执行,因此我们将其连接起来。
%    \begin{macrocode}
\g@addto@macro\pc@alloccolumns{%
  \pcpc@InitStacks\pc@columncount
}
%    \end{macrocode}
%
% \subsubsection{Switch stack\\切换堆栈}
%
%    \cs{pcpc@SwitchStack} should be called by marco \cs{colchunk@}.
%    However it is easier to patch \cs{pc@setcolumnwidth} that
%    is executed in \cs{colchunk@} only.
%
%\cs{pcpc@SwitchStack} 应该被宏 \cs{colchunk@} 调用。不过,我们更容易修补仅在 \cs{colchunk@} 中执行的 \cs{pc@setcolumnwidth}。
%    \begin{macrocode}
\g@addto@macro\pc@setcolumnwidth{%
  \pcpc@SwitchStack\pc@columnctr
}
%    \end{macrocode}
%
% \subsubsection{Set current stack color\\设置当前堆栈颜色}
%
%    \cs{pcpc@SetCurrent} is set at the begin of each line.
%    It must be inserted into \cs{pc@placeboxes}. Unhappily
%    there is no easy way. Therefore we check and
%    redefine \cs{pc@placeboxes}.
%
%\cs{pcpc@SetCurrent} 在每行开始时设置。它必须插入到 \cs{pc@placeboxes} 中。不幸的是,没有简单的方法。因此,我们检查并重新定义 \cs{pc@placeboxes}。
%    \begin{macrocode}
\begingroup
  \def\x{%
    \global\let\@tempa\relax
    \count@\z@
    \hb@xt@\linewidth{%
      \vfuzz30ex %
      \vbadness\@M
      \splittopskip\z@skip
      \loop
      \ifnum\count@<\pc@columncount
        \advance\count@\@ne
        \expandafter\ifvoid\csname pc@column@\number\count@\endcsname
          \hskip\csname pc@column@width@\number\count@\endcsname
        \else
          \expandafter\setbox\expandafter\@tempboxa\expandafter
          \vsplit\csname pc@column@\number\count@\endcsname
              to \dp\strutbox
          \vbox{%
            \unvbox\@tempboxa
          }%
        \fi
        \expandafter\ifvoid\csname pc@column@\number\count@\endcsname
        \else
          \global\let\@tempa\pc@placeboxes
        \fi
        \ifnum\count@<\pc@columncount
          \strut
          \hfill
          \ifpc@rulebetween
            \vrule
            \hfill
          \fi
        \fi
      \repeat
    }%
    \@tempa
  }%
  \ifx\x\pc@placeboxes
  \else
    \@PackageWarningNoLine{pdfcolparcolumns}{%
      Command \string\pc@placeboxes\space has changed.\MessageBreak
      Supported versions of package `parcolumns':\MessageBreak
      \space\space 2004/08/05.\MessageBreak
      The redefinition of \string\pc@placeboxes\space may not%
      \MessageBreak
      behave correctly depending on the changes%
    }%
  \fi
\endgroup
%    \end{macrocode}
%    \begin{macro}{\pc@placeboxes}
%    \begin{macrocode}
\renewcommand*{\pc@placeboxes}{%
  \global\let\@tempa\relax
  \count@\z@
  \hb@xt@\linewidth{%
    \vfuzz30ex %
    \vbadness\@M
    \splittopskip\z@skip
    \loop
    \ifnum\count@<\pc@columncount
      \advance\count@\@ne
      \expandafter\ifvoid\csname pc@column@\number\count@\endcsname
        \hskip\csname pc@column@width@\number\count@\endcsname
      \else
        \expandafter\setbox\expandafter\@tempboxa\expandafter
        \vsplit\csname pc@column@\number\count@\endcsname
            to \dp\strutbox
        \vbox{%
          \pcpc@SetCurrent\count@
          \unvbox\@tempboxa
        }%
      \fi
      \expandafter\ifvoid\csname pc@column@\number\count@\endcsname
      \else
        \global\let\@tempa\pc@placeboxes
      \fi
      \ifnum\count@<\pc@columncount
        \strut
        \hfill
        \ifpc@rulebetween
          \begingroup
            \pcpc@RuleBetweenColor
            \vrule
          \endgroup
          \hfill
        \fi
      \fi
    \repeat
  }%
  \@tempa
}
%    \end{macrocode}
%    \end{macro}
%    \begin{macro}{\pcpc@RuleBetweenColorDefault}
%    \begin{macrocode}
\def\pcpc@RuleBetweenColorDefault{%
  \normalcolor
}
%    \end{macrocode}
%    \end{macro}
%    \begin{macro}{\pcpc@RuleBetweenColor}
%    \begin{macrocode}
\let\pcpc@RuleBetweenColor\pcpc@RuleBetweenColorDefault
%    \end{macrocode}
%    \end{macro}
%    \begin{macrocode}
\define@key{parcolumns}{rulebetweencolor}{%
  \edef\pcpc@temp{#1}%
  \ifx\pcpc@temp\@empty
    \let\pcpc@RuleBetweenColor\pcpc@RuleBetweenColorDefault
  \else
    \edef\pcpc@temp{%
      \noexpand\@ifnextchar[{%
        \def\noexpand\pcpc@RuleBetweenColor{%
          \noexpand\color\pcpc@temp
        }%
        \noexpand\pcpc@GobbleNil
      }{%
        \def\noexpand\pcpc@RuleBetweenColor{%
          \noexpand\color{\pcpc@temp}%
        }%
        \noexpand\pcpc@GobbleNil
      }%
      \pcpc@temp\noexpand\@nil
    }%
    \pcpc@temp
  \fi
}
%    \end{macrocode}
%    \begin{macro}{\pcpc@GobbleNil}
%    \begin{macrocode}
\long\def\pcpc@GobbleNil#1\@nil{}
%    \end{macrocode}
%    \end{macro}
%
%    \begin{macrocode}
%</package>
%    \end{macrocode}
%% \section{Installation\\安装}
%
% \subsection{Download\\下载}
%
% \paragraph{Package.} This package is available on
% CTAN\footnote{\CTANpkg{pdfcolparcolumns}}:

%该软件包可从 CTAN\footnote{\CTANpkg{pdfcolparcolumns}} 下载:
% \begin{description}
% \item[\CTAN{macros/latex/contrib/oberdiek/pdfcolparcolumns.dtx}] The source file.
% \item[\CTAN{macros/latex/contrib/oberdiek/pdfcolparcolumns.pdf}] Documentation.
% \end{description}
%
%
% \paragraph{Bundle.} All the packages of the bundle `oberdiek'
% are also available in a TDS compliant ZIP archive. There
% the packages are already unpacked and the documentation files
% are generated. The files and directories obey the TDS standard.
%
%捆绑包“oberdiek”中的所有软件包也可在符合 TDS 标准的 ZIP 存档中获取。在该存档中,软件包已经被解包,文档文件已经生成,文件和目录符合 TDS 标准。
% \begin{description}
% \item[\CTANinstall{install/macros/latex/contrib/oberdiek.tds.zip}]
% \end{description}
% \emph{TDS} refers to the standard ``A Directory Structure
% for \TeX\ Files'' (\CTANpkg{tds}). Directories
% with \xfile{texmf} in their name are usually organized this way.
%
%\emph{TDS} 指的是标准“用于 \TeX\ 文件的目录结构”(\CTANpkg{tds})。名字中包含\xfile{texmf}的目录通常都是按这种方式组织的。
% \subsection{Bundle installation\\捆绑包安装}
%
% \paragraph{Unpacking.} Unpack the \xfile{oberdiek.tds.zip} in the
% TDS tree (also known as \xfile{texmf} tree) of your choice.
% Example (linux):
%
%\paragraph{解包。}在您选择的 TDS 树(也称为\xfile{texmf}树)中解压\xfile{oberdiek.tds.zip}。例如(在Linux中):
% \begin{quote}
%   |unzip oberdiek.tds.zip -d ~/texmf|
% \end{quote}
%
% \subsection{Package installation\\软件包安装}
%
% \paragraph{Unpacking.} The \xfile{.dtx} file is a self-extracting
% \docstrip\ archive. The files are extracted by running the
% \xfile{.dtx} through \plainTeX:
%
%\paragraph{解包。} \xfile{.dtx} 文件是一个自解压的 \docstrip\ 存档。运行\xfile{.dtx}通过\plainTeX\ 来提取文件:
% \begin{quote}
%   \verb|tex pdfcolparcolumns.dtx|
% \end{quote}
%
% \paragraph{TDS.} Now the different files must be moved into
% the different directories in your installation TDS tree
% (also known as \xfile{texmf} tree):
%
%\paragraph{TDS。}现在,不同的文件必须移动到安装 TDS 树(也称为\xfile{texmf}树)中的不同目录中:
% \begin{quote}
% \def\t{^^A
% \begin{tabular}{@{}>{\ttfamily}l@{ $\rightarrow$ }>{\ttfamily}l@{}}
%   pdfcolparcolumns.sty & tex/latex/oberdiek/pdfcolparcolumns.sty\\
%   pdfcolparcolumns.pdf & doc/latex/oberdiek/pdfcolparcolumns.pdf\\
%   pdfcolparcolumns.dtx & source/latex/oberdiek/pdfcolparcolumns.dtx\\
% \end{tabular}^^A
% }^^A
% \sbox0{\t}^^A
% \ifdim\wd0>\linewidth
%   \begingroup
%     \advance\linewidth by\leftmargin
%     \advance\linewidth by\rightmargin
%   \edef\x{\endgroup
%     \def\noexpand\lw{\the\linewidth}^^A
%   }\x
%   \def\lwbox{^^A
%     \leavevmode
%     \hbox to \linewidth{^^A
%       \kern-\leftmargin\relax
%       \hss
%       \usebox0
%       \hss
%       \kern-\rightmargin\relax
%     }^^A
%   }^^A
%   \ifdim\wd0>\lw
%     \sbox0{\small\t}^^A
%     \ifdim\wd0>\linewidth
%       \ifdim\wd0>\lw
%         \sbox0{\footnotesize\t}^^A
%         \ifdim\wd0>\linewidth
%           \ifdim\wd0>\lw
%             \sbox0{\scriptsize\t}^^A
%             \ifdim\wd0>\linewidth
%               \ifdim\wd0>\lw
%                 \sbox0{\tiny\t}^^A
%                 \ifdim\wd0>\linewidth
%                   \lwbox
%                 \else
%                   \usebox0
%                 \fi
%               \else
%                 \lwbox
%               \fi
%             \else
%               \usebox0
%             \fi
%           \else
%             \lwbox
%           \fi
%         \else
%           \usebox0
%         \fi
%       \else
%         \lwbox
%       \fi
%     \else
%       \usebox0
%     \fi
%   \else
%     \lwbox
%   \fi
% \else
%   \usebox0
% \fi
% \end{quote}
% If you have a \xfile{docstrip.cfg} that configures and enables \docstrip's
% TDS installing feature, then some files can already be in the right
% place, see the documentation of \docstrip.
%
%如果你有一个\xfile{docstrip.cfg}文件配置和启用了\docstrip 的TDS安装功能,那么一些文件可能已经位于正确的位置,具体请参见\docstrip 的文档。
% \subsection{Refresh file name databases\\刷新文件名数据库}
%
% If your \TeX~distribution
% (\TeX\,Live, \mikTeX, \dots) relies on file name databases, you must refresh
% these. For example, \TeX\,Live\ users run \verb|texhash| or
% \verb|mktexlsr|.
%
%如果你的\TeX~发行版(\TeX,Live、\mikTeX,等等)依赖于文件名数据库,你必须刷新它们。例如,\TeX,Live用户运行\verb|texhash|或\verb|mktexlsr|。
% \subsection{Some details for the interested\\一些细节供感兴趣的人使用}
%
% \paragraph{Unpacking with \LaTeX.}
% The \xfile{.dtx} chooses its action depending on the format:

%\paragraph{用\LaTeX 进行解包。} \xfile{.dtx}文件根据格式选择操作:
% \begin{description}
% \item[\plainTeX:] Run \docstrip\ and extract the files.
% \item[\LaTeX:] Generate the documentation.
% \end{description}
% If you insist on using \LaTeX\ for \docstrip\ (really,
% \docstrip\ does not need \LaTeX), then inform the autodetect routine
% about your intention:

%如果你坚持要用\LaTeX 进行\docstrip (其实\docstrip 不需要\LaTeX ),那么请告知自动检测例程你的意图:

% \begin{quote}
%   \verb|latex \let\install=y\input{pdfcolparcolumns.dtx}|
% \end{quote}
% Do not forget to quote the argument according to the demands
% of your shell.

%不要忘记根据你的shell的要求引用参数。
%
% \paragraph{Generating the documentation.}
% You can use both the \xfile{.dtx} or the \xfile{.drv} to generate
% the documentation. The process can be configured by the
% configuration file \xfile{ltxdoc.cfg}. For instance, put this
% line into this file, if you want to have A4 as paper format:
%
%\paragraph{生成文档。} 你可以使用\xfile{.dtx}或\xfile{.drv}生成文档。这个过程可以由配置文件\xfile{ltxdoc.cfg}配置。例如,如果你想要A4作为纸张格式,将此行放入文件中:

% \begin{quote}
%   \verb|\PassOptionsToClass{a4paper}{article}|
% \end{quote}
% An example follows how to generate the
% documentation with pdf\LaTeX:

%以下是使用pdf\LaTeX 生成文档的示例:
% \begin{quote}
%\begin{verbatim}
%pdflatex pdfcolparcolumns.dtx
%makeindex -s gind.ist pdfcolparcolumns.idx
%pdflatex pdfcolparcolumns.dtx
%makeindex -s gind.ist pdfcolparcolumns.idx
%pdflatex pdfcolparcolumns.dtx
%\end{verbatim}
% \end{quote}
%
% \begin{thebibliography}{9}
%
% \bibitem{parcolumns}
%   Jonathan Sauer: \textit{The \xpackage{parcolumns} package};
%   2004/11/25;\\
%   \CTANpkg{parcolumns}.
%
% \bibitem{pdfcol}
%   Heiko Oberdiek: \textit{The \xpackage{pdfcol} package};
%   2007/09/09;\\
%   \CTANpkg{pdfcol}.
%
% \end{thebibliography}
%
% \begin{History}
%   \begin{Version}{2007/07/26 v1.0}
%   \item
%     First version, published in the newsgroup \xnewsgroup{comp.text.tex}
%     with the name \xpackage{parcolumns-colorstacks}: ^^A no line break
%     \URL{``\link{Re: \xpackage{xcolor} glitches}''}^^A
%     {https://groups.google.com/group/comp.text.tex/msg/56bd897b11bca414}

%第一个版本,以 \xnewsgroup{comp.text.tex} 新闻组发布,名称为 \xpackage{parcolumns-colorstacks}:^^A 没有换行符 
%\URL{``\link{Re: \xpackage{xcolor} glitches}''}^^A
%{https://groups.google.com/group/comp.text.tex/msg/56bd897b11bca414}
%   \end{Version}
%   \begin{Version}{2007/09/09 v1.1}
%   \item
%     CTAN version, package name renamed to \xpackage{pdfcolparcolumns}.
%
%CTAN 版本,将包名改为 \xpackage{pdfcolparcolumns}。
%   \item
%     Uses package \xpackage{pdfcol}.
%
%使用 \xpackage{pdfcol} 包。
%   \item
%     Documentation added.
%
%添加文档。
%   \item
%     Test file added.
%
%添加测试文件。
%   \end{Version}
%   \begin{Version}{2008/08/11 v1.2}
%   \item
%     Code is not changed.
%
%代码未改动。
%   \item
%     URLs updated.
%
%更新 URL。
%   \end{Version}
%   \begin{Version}{2010/01/11 v1.3}
%   \item
%     Fix for rule color.
%
%修复分隔线的颜色。
%   \item
%     New option \xoption{rulebetweencolor} for environment |parcolumns|.
%
%为环境 |parcolumns| 添加新选项 \xoption{rulebetweencolor}。
%   \end{Version}
%   \begin{Version}{2016/05/16 v1.4}
%   \item
%     Documentation updates.
%
%更新文档。
%   \end{Version}
%   \begin{Version}{2019/12/29 v1.5}
%   \item
%     \cs{PassOptionsToPackage} not \cs{PassoptionsToPackage}
%
%\cs{PassOptionsToPackage} 改为不区分大小写的 \cs{PassoptionsToPackage}。
%   \end{Version}
% \end{History}
%
% \PrintIndex
%
% \Finale
\endinput
|
% \end{quote}
% Do not forget to quote the argument according to the demands
% of your shell.

%不要忘记根据你的shell的要求引用参数。
%
% \paragraph{Generating the documentation.}
% You can use both the \xfile{.dtx} or the \xfile{.drv} to generate
% the documentation. The process can be configured by the
% configuration file \xfile{ltxdoc.cfg}. For instance, put this
% line into this file, if you want to have A4 as paper format:
%
%\paragraph{生成文档。} 你可以使用\xfile{.dtx}或\xfile{.drv}生成文档。这个过程可以由配置文件\xfile{ltxdoc.cfg}配置。例如,如果你想要A4作为纸张格式,将此行放入文件中:

% \begin{quote}
%   \verb|\PassOptionsToClass{a4paper}{article}|
% \end{quote}
% An example follows how to generate the
% documentation with pdf\LaTeX:

%以下是使用pdf\LaTeX 生成文档的示例:
% \begin{quote}
%\begin{verbatim}
%pdflatex pdfcolparcolumns.dtx
%makeindex -s gind.ist pdfcolparcolumns.idx
%pdflatex pdfcolparcolumns.dtx
%makeindex -s gind.ist pdfcolparcolumns.idx
%pdflatex pdfcolparcolumns.dtx
%\end{verbatim}
% \end{quote}
%
% \begin{thebibliography}{9}
%
% \bibitem{parcolumns}
%   Jonathan Sauer: \textit{The \xpackage{parcolumns} package};
%   2004/11/25;\\
%   \CTANpkg{parcolumns}.
%
% \bibitem{pdfcol}
%   Heiko Oberdiek: \textit{The \xpackage{pdfcol} package};
%   2007/09/09;\\
%   \CTANpkg{pdfcol}.
%
% \end{thebibliography}
%
% \begin{History}
%   \begin{Version}{2007/07/26 v1.0}
%   \item
%     First version, published in the newsgroup \xnewsgroup{comp.text.tex}
%     with the name \xpackage{parcolumns-colorstacks}: ^^A no line break
%     \URL{``\link{Re: \xpackage{xcolor} glitches}''}^^A
%     {https://groups.google.com/group/comp.text.tex/msg/56bd897b11bca414}

%第一个版本,以 \xnewsgroup{comp.text.tex} 新闻组发布,名称为 \xpackage{parcolumns-colorstacks}:^^A 没有换行符 
%\URL{``\link{Re: \xpackage{xcolor} glitches}''}^^A
%{https://groups.google.com/group/comp.text.tex/msg/56bd897b11bca414}
%   \end{Version}
%   \begin{Version}{2007/09/09 v1.1}
%   \item
%     CTAN version, package name renamed to \xpackage{pdfcolparcolumns}.
%
%CTAN 版本,将包名改为 \xpackage{pdfcolparcolumns}。
%   \item
%     Uses package \xpackage{pdfcol}.
%
%使用 \xpackage{pdfcol} 包。
%   \item
%     Documentation added.
%
%添加文档。
%   \item
%     Test file added.
%
%添加测试文件。
%   \end{Version}
%   \begin{Version}{2008/08/11 v1.2}
%   \item
%     Code is not changed.
%
%代码未改动。
%   \item
%     URLs updated.
%
%更新 URL。
%   \end{Version}
%   \begin{Version}{2010/01/11 v1.3}
%   \item
%     Fix for rule color.
%
%修复分隔线的颜色。
%   \item
%     New option \xoption{rulebetweencolor} for environment |parcolumns|.
%
%为环境 |parcolumns| 添加新选项 \xoption{rulebetweencolor}。
%   \end{Version}
%   \begin{Version}{2016/05/16 v1.4}
%   \item
%     Documentation updates.
%
%更新文档。
%   \end{Version}
%   \begin{Version}{2019/12/29 v1.5}
%   \item
%     \cs{PassOptionsToPackage} not \cs{PassoptionsToPackage}
%
%\cs{PassOptionsToPackage} 改为不区分大小写的 \cs{PassoptionsToPackage}。
%   \end{Version}
% \end{History}
%
% \PrintIndex
%
% \Finale
\endinput

%        (quote the arguments according to the demands of your shell)
%
% Documentation:
%    (a) If pdfcolparcolumns.drv is present:
%           latex pdfcolparcolumns.drv
%    (b) Without pdfcolparcolumns.drv:
%           latex pdfcolparcolumns.dtx; ...
%    The class ltxdoc loads the configuration file ltxdoc.cfg
%    if available. Here you can specify further options, e.g.
%    use A4 as paper format:
%       \PassOptionsToClass{a4paper}{article}
%
%    Programm calls to get the documentation (example):
%       pdflatex pdfcolparcolumns.dtx
%       makeindex -s gind.ist pdfcolparcolumns.idx
%       pdflatex pdfcolparcolumns.dtx
%       makeindex -s gind.ist pdfcolparcolumns.idx
%       pdflatex pdfcolparcolumns.dtx
%
% Installation:
%    TDS:tex/latex/oberdiek/pdfcolparcolumns.sty
%    TDS:doc/latex/oberdiek/pdfcolparcolumns.pdf
%    TDS:source/latex/oberdiek/pdfcolparcolumns.dtx
%
%<*ignore>
\begingroup
  \catcode123=1 %
  \catcode125=2 %
  \def\x{LaTeX2e}%
\expandafter\endgroup
\ifcase 0\ifx\install y1\fi\expandafter
         \ifx\csname processbatchFile\endcsname\relax\else1\fi
         \ifx\fmtname\x\else 1\fi\relax
\else\csname fi\endcsname
%</ignore>
%<*install>
\input docstrip.tex
\Msg{************************************************************************}
\Msg{* Installation}
\Msg{* Package: pdfcolparcolumns 2019/12/29 v1.5 Color stacks for parcolumns (HO)}
\Msg{************************************************************************}

\keepsilent
\askforoverwritefalse

\let\MetaPrefix\relax
\preamble

This is a generated file.

Project: pdfcolparcolumns
Version: 2019/12/29 v1.5

Copyright (C)
   2007, 2008, 2010 Heiko Oberdiek
   2016-2019 Oberdiek Package Support Group

This work may be distributed and/or modified under the
conditions of the LaTeX Project Public License, either
version 1.3c of this license or (at your option) any later
version. This version of this license is in
   https://www.latex-project.org/lppl/lppl-1-3c.txt
and the latest version of this license is in
   https://www.latex-project.org/lppl.txt
and version 1.3 or later is part of all distributions of
LaTeX version 2005/12/01 or later.

This work has the LPPL maintenance status "maintained".

The Current Maintainers of this work are
Heiko Oberdiek and the Oberdiek Package Support Group
https://github.com/ho-tex/oberdiek/issues


This work consists of the main source file pdfcolparcolumns.dtx
and the derived files
   pdfcolparcolumns.sty, pdfcolparcolumns.pdf, pdfcolparcolumns.ins,
   pdfcolparcolumns.drv, pdfcolparcolumns-test1.tex.

\endpreamble
\let\MetaPrefix\DoubleperCent

\generate{%
  \file{pdfcolparcolumns.ins}{\from{pdfcolparcolumns.dtx}{install}}%
  \file{pdfcolparcolumns.drv}{\from{pdfcolparcolumns.dtx}{driver}}%
  \usedir{tex/latex/oberdiek}%
  \file{pdfcolparcolumns.sty}{\from{pdfcolparcolumns.dtx}{package}}%
%  \usedir{doc/latex/oberdiek/test}%
%  \file{pdfcolparcolumns-test1.tex}{\from{pdfcolparcolumns.dtx}{test1}}%
}

\catcode32=13\relax% active space
\let =\space%
\Msg{************************************************************************}
\Msg{*}
\Msg{* To finish the installation you have to move the following}
\Msg{* file into a directory searched by TeX:}
\Msg{*}
\Msg{*     pdfcolparcolumns.sty}
\Msg{*}
\Msg{* To produce the documentation run the file `pdfcolparcolumns.drv'}
\Msg{* through LaTeX.}
\Msg{*}
\Msg{* Happy TeXing!}
\Msg{*}
\Msg{************************************************************************}

\endbatchfile
%</install>
%<*ignore>
\fi
%</ignore>
%<*driver>
\NeedsTeXFormat{LaTeX2e}
\ProvidesFile{pdfcolparcolumns.drv}%
  [2019/12/29 v1.5 Color stacks for parcolumns (HO)]%
\documentclass{ltxdoc}
\usepackage{holtxdoc}[2011/11/22]
\usepackage[scheme=plain]{ctex}
\setCJKmainfont{方正书宋_GBK}%方正书宋_GBK.TTF  设置文本的中文有衬线字体为“方正书宋_GBK”
\setCJKsansfont{方正黑体简体}%方正黑体_GBK.TTF  设置文本的中文无衬线字体为“方正黑体简体”
\setCJKmonofont{方正书宋简体}%方正仿宋_GBK.TTF  设置文本的中文等宽字体为“方正书宋简体”
\begin{document}
  \DocInput{pdfcolparcolumns.dtx}%
\end{document}
%</driver>
% \fi
%
%
%
% \GetFileInfo{pdfcolparcolumns.drv}
%
% \title{The \xpackage{pdfcolparcolumns} package}
% \date{2019/12/29 v1.5}
% \author{Heiko Oberdiek\thanks
% {Please report any issues at \url{https://github.com/ho-tex/oberdiek/issues}}\and 翻译\\{\tt virhuiai@qq.com}}
%
% \maketitle
%
% \begin{abstract}
%\makebox[0pt]{}
% \begin{parcolumns}[nofirstindent]{2}
%\colchunk{Since version 1.40 \pdfTeX\ supports several color stacks.
%This package uses them to fix color problems in
%package \xpackage{parcolumns}.}
%\colchunk{自从版本1.40起,\pdfTeX 支持多个颜色栈。
%此包使用它们来解决 \xpackage{parcolumns} 包中的颜色问题。}
%\end{parcolumns}



% \end{abstract}
%
% \tableofcontents
%
% \section{Usage\\用法}
%
% \begin{quote}
% |\usepackage{pdfcolparcolumns}|
% \end{quote}
% The package \xpackage{pdfcolparcolumns} loads package \xpackage{parcolums}
% \cite{parcolumns}. If color stacks are available then the
% macros of \xpackage{parcolumns} are patched to add support
% for color stacks.

%包 \xpackage{pdfcolparcolumns} 载入了 \xpackage{parcolumns} \cite{parcolumns} 包。如果有可用的颜色栈,则会对 \xpackage{parcolumns} 的宏进行修补,以添加对颜色栈的支持。

%
% \subsection{Option \xoption{rulebetweencolor}\\选项 \xoption{rulebetweencolor}}
%
% Package \xpackage{pdfcolparcolumns} also fixes the color for the
% rule between columns (if \xoption{rulebetween} is set).
% Default color is \cs{normalcolor}. But this can be changed by using
% option \xoption{rulebetweencolor}. It takes a color specification
% as value. If the value is empty, then the default (\cs{normalcolor})
% is used.
% Examples:

% 包 \xpackage{pdfcolparcolumns} 还修复了列之间分隔线的颜色(如果设置了 \xoption{rulebetween})。默认颜色为 \cs{normalcolor}。但是,可以使用选项 \xoption{rulebetweencolor} 来更改它。它接受一个颜色规范作为值。如果该值为空,则使用默认值(\cs{normalcolor})。例如:

% \begin{quote}
%   |rulebetweencolor=blue|,\\
%   |rulebetweencolor={red}|,\\
%   |rulebetweencolor={}|, \textit{\% \cs{normalcolor} is used}\\
%   |rulebetweencolor=[rgb]{1,0,.5}| \textit{\% see below}
% \end{quote}
% If used inside the optional argument of environment \textsf{parcolumns}
% and the value contains an optional argument, then whole value
% must be put in curly braces:

% 如果在 \textsf{parcolumns} 环境的可选参数中使用,并且值包含可选参数,则整个值必须用花括号括起来:
%\begin{quote}
%\begin{verbatim}
%\begin{parcolumns}[
%  rulebetween,
%  rulebetweencolor={[rgb]{1,0,.5}},
%]{2}
%  ...
%\end{parcolumns}
%\end{verbatim}
%\end{quote}
% This option \xoption{rulebetweencolor} can also be set using
% \cs{setkeys}:

%也可以使用 \cs{setkeys} 来设置选项 \xoption{rulebetweencolor}:
%\begin{quote}
%\begin{verbatim}
%\setkeys{parcolumns}{rulebetweencolor=green}
%\end{verbatim}
%\end{quote}
%
% \subsection{Future\\未来展望}
%
% Currently package \xpackage{parcolumns} does not seem to be
% maintained. Nevertheless if there will be a new version that
% adds support for color stacks, then this package may become
% obsolete.
%
%目前,\xpackage{parcolumns} 包似乎没有维护了。尽管如此,如果出现了新版本并支持颜色栈,那么这个包可能会变得过时。
% \StopEventually{
% }
%
% \section{Implementation\\实现}
%
% \subsection{Identification\\标识}
%
%    \begin{macrocode}
%<*package>
\NeedsTeXFormat{LaTeX2e}
\ProvidesPackage{pdfcolparcolumns}%
  [2019/12/29 v1.5 Color stacks for parcolumns (HO)]%
%    \end{macrocode}
%
% \subsection{Load packages\\加载宏包}
%
% \subsubsection{Package \xpackage{parcolumns}\\宏包 \xpackage{parcolumns}}
%
%    Currently package \xpackage{parcolumns} does not define options.
%    Thus it is just a precaution that the options of
%    package \xpackage{pdfcolparcolumns} are passed to
%    package \xpackage{parcolumns}.

%目前,宏包 \xpackage{parcolumns} 没有定义选项。因此,它只是一个预防措施,以确保将宏包 \xpackage{pdfcolparcolumns} 的选项传递给宏包 \xpackage{parcolumns}。
%    \begin{macrocode}
\DeclareOption*{%
  \PassOptionsToPackage{\CurrentOption}{parcolumns}%
}
\ProcessOptions\relax
\RequirePackage{parcolumns}[2004/11/25]
%    \end{macrocode}
%
% \subsubsection{Package \xpackage{pdfcol}\\宏包 \xpackage{pdfcol}}
%
%    \begin{macrocode}
\RequirePackage{pdfcol}[2007/09/09]
\ifpdfcolAvailable
\else
  \PackageInfo{pdfcolparcolumns}{%
    Loading aborted, because color stacks are not available%
  }%
  \expandafter\endinput
\fi
%    \end{macrocode}
%
% \subsubsection{Package \xpackage{infwarerr}\\宏包 \xpackage{infwarerr}}
%
%    \begin{macrocode}
\RequirePackage{infwarerr}[2007/09/09]
%    \end{macrocode}
%
% \subsection{Color stack macros\\颜色堆栈宏}
%
%    \begin{macro}{\pcpc@MaxStack}
%    Macro \cs{pcpc@MaxStack} holds the highest number of
%    allocated stacks.

%宏 \cs{pcpc@MaxStack} 存储已分配的堆栈中最大的编号。
%    \begin{macrocode}
\global\chardef\pcpc@MaxStack=\z@
%    \end{macrocode}
%    \end{macro}
%    \begin{macro}{\pcpc@InitStacks}
%    Macro \cs{pcpc@InitStacks} takes the number of columns
%    as argument and ensures that there are enough color
%    stacks for all columns.
%
%宏 \cs{pcpc@InitStacks} 接受列数作为参数,并确保为所有列分配了足够的颜色堆栈。
%    \begin{macrocode}
\def\pcpc@InitStacks#1{%
  \ifnum#1>\pcpc@MaxStack
    \begingroup
      \count@\pcpc@MaxStack
      \loop
        \advance\count@\@ne
        \pdfcolInitStack{pcpc@\the\count@}%
      \ifnum#1>\count@
      \repeat
      \global\chardef\pcpc@MaxStack=\count@
    \endgroup
  \fi
}
%    \end{macrocode}
%    \end{macro}
%
%    \begin{macro}{\pcpc@SwitchStack}
%    \begin{macrocode}
\def\pcpc@SwitchStack#1{%
  \pdfcolSwitchStack{pcpc@\number#1}%
}
%    \end{macrocode}
%    \end{macro}
%
%    \begin{macro}{\pcpc@SetCurrent}
%    \begin{macrocode}
\def\pcpc@SetCurrent#1{%
  \pdfcolSetCurrent{pcpc@\number#1}%
}
%    \end{macrocode}
%    \end{macro}
%
% \subsection{Patches\\补丁}
%
%     Now the color stack macros are patched into the macros
%     of package \xpackage{parcolumns}.
%
%现在颜色堆栈宏已经被打补丁到了 \xpackage{parcolumns} 宏包的宏中。
% \subsubsection{Init stacks\\初始化堆栈}
%
%    \cs{pcpc@InitStacks} should go into the definition of
%    environment |parcolumns|. \cs{pc@alloccolumns} is executed
%    there and nowhere else, thus we hook into it.

%\cs{pcpc@InitStacks} 应该被放在环境 |parcolumns| 的定义中。\cs{pc@alloccolumns} 仅在那里执行,因此我们将其连接起来。
%    \begin{macrocode}
\g@addto@macro\pc@alloccolumns{%
  \pcpc@InitStacks\pc@columncount
}
%    \end{macrocode}
%
% \subsubsection{Switch stack\\切换堆栈}
%
%    \cs{pcpc@SwitchStack} should be called by marco \cs{colchunk@}.
%    However it is easier to patch \cs{pc@setcolumnwidth} that
%    is executed in \cs{colchunk@} only.
%
%\cs{pcpc@SwitchStack} 应该被宏 \cs{colchunk@} 调用。不过,我们更容易修补仅在 \cs{colchunk@} 中执行的 \cs{pc@setcolumnwidth}。
%    \begin{macrocode}
\g@addto@macro\pc@setcolumnwidth{%
  \pcpc@SwitchStack\pc@columnctr
}
%    \end{macrocode}
%
% \subsubsection{Set current stack color\\设置当前堆栈颜色}
%
%    \cs{pcpc@SetCurrent} is set at the begin of each line.
%    It must be inserted into \cs{pc@placeboxes}. Unhappily
%    there is no easy way. Therefore we check and
%    redefine \cs{pc@placeboxes}.
%
%\cs{pcpc@SetCurrent} 在每行开始时设置。它必须插入到 \cs{pc@placeboxes} 中。不幸的是,没有简单的方法。因此,我们检查并重新定义 \cs{pc@placeboxes}。
%    \begin{macrocode}
\begingroup
  \def\x{%
    \global\let\@tempa\relax
    \count@\z@
    \hb@xt@\linewidth{%
      \vfuzz30ex %
      \vbadness\@M
      \splittopskip\z@skip
      \loop
      \ifnum\count@<\pc@columncount
        \advance\count@\@ne
        \expandafter\ifvoid\csname pc@column@\number\count@\endcsname
          \hskip\csname pc@column@width@\number\count@\endcsname
        \else
          \expandafter\setbox\expandafter\@tempboxa\expandafter
          \vsplit\csname pc@column@\number\count@\endcsname
              to \dp\strutbox
          \vbox{%
            \unvbox\@tempboxa
          }%
        \fi
        \expandafter\ifvoid\csname pc@column@\number\count@\endcsname
        \else
          \global\let\@tempa\pc@placeboxes
        \fi
        \ifnum\count@<\pc@columncount
          \strut
          \hfill
          \ifpc@rulebetween
            \vrule
            \hfill
          \fi
        \fi
      \repeat
    }%
    \@tempa
  }%
  \ifx\x\pc@placeboxes
  \else
    \@PackageWarningNoLine{pdfcolparcolumns}{%
      Command \string\pc@placeboxes\space has changed.\MessageBreak
      Supported versions of package `parcolumns':\MessageBreak
      \space\space 2004/08/05.\MessageBreak
      The redefinition of \string\pc@placeboxes\space may not%
      \MessageBreak
      behave correctly depending on the changes%
    }%
  \fi
\endgroup
%    \end{macrocode}
%    \begin{macro}{\pc@placeboxes}
%    \begin{macrocode}
\renewcommand*{\pc@placeboxes}{%
  \global\let\@tempa\relax
  \count@\z@
  \hb@xt@\linewidth{%
    \vfuzz30ex %
    \vbadness\@M
    \splittopskip\z@skip
    \loop
    \ifnum\count@<\pc@columncount
      \advance\count@\@ne
      \expandafter\ifvoid\csname pc@column@\number\count@\endcsname
        \hskip\csname pc@column@width@\number\count@\endcsname
      \else
        \expandafter\setbox\expandafter\@tempboxa\expandafter
        \vsplit\csname pc@column@\number\count@\endcsname
            to \dp\strutbox
        \vbox{%
          \pcpc@SetCurrent\count@
          \unvbox\@tempboxa
        }%
      \fi
      \expandafter\ifvoid\csname pc@column@\number\count@\endcsname
      \else
        \global\let\@tempa\pc@placeboxes
      \fi
      \ifnum\count@<\pc@columncount
        \strut
        \hfill
        \ifpc@rulebetween
          \begingroup
            \pcpc@RuleBetweenColor
            \vrule
          \endgroup
          \hfill
        \fi
      \fi
    \repeat
  }%
  \@tempa
}
%    \end{macrocode}
%    \end{macro}
%    \begin{macro}{\pcpc@RuleBetweenColorDefault}
%    \begin{macrocode}
\def\pcpc@RuleBetweenColorDefault{%
  \normalcolor
}
%    \end{macrocode}
%    \end{macro}
%    \begin{macro}{\pcpc@RuleBetweenColor}
%    \begin{macrocode}
\let\pcpc@RuleBetweenColor\pcpc@RuleBetweenColorDefault
%    \end{macrocode}
%    \end{macro}
%    \begin{macrocode}
\define@key{parcolumns}{rulebetweencolor}{%
  \edef\pcpc@temp{#1}%
  \ifx\pcpc@temp\@empty
    \let\pcpc@RuleBetweenColor\pcpc@RuleBetweenColorDefault
  \else
    \edef\pcpc@temp{%
      \noexpand\@ifnextchar[{%
        \def\noexpand\pcpc@RuleBetweenColor{%
          \noexpand\color\pcpc@temp
        }%
        \noexpand\pcpc@GobbleNil
      }{%
        \def\noexpand\pcpc@RuleBetweenColor{%
          \noexpand\color{\pcpc@temp}%
        }%
        \noexpand\pcpc@GobbleNil
      }%
      \pcpc@temp\noexpand\@nil
    }%
    \pcpc@temp
  \fi
}
%    \end{macrocode}
%    \begin{macro}{\pcpc@GobbleNil}
%    \begin{macrocode}
\long\def\pcpc@GobbleNil#1\@nil{}
%    \end{macrocode}
%    \end{macro}
%
%    \begin{macrocode}
%</package>
%    \end{macrocode}
%% \section{Installation\\安装}
%
% \subsection{Download\\下载}
%
% \paragraph{Package.} This package is available on
% CTAN\footnote{\CTANpkg{pdfcolparcolumns}}:

%该软件包可从 CTAN\footnote{\CTANpkg{pdfcolparcolumns}} 下载:
% \begin{description}
% \item[\CTAN{macros/latex/contrib/oberdiek/pdfcolparcolumns.dtx}] The source file.
% \item[\CTAN{macros/latex/contrib/oberdiek/pdfcolparcolumns.pdf}] Documentation.
% \end{description}
%
%
% \paragraph{Bundle.} All the packages of the bundle `oberdiek'
% are also available in a TDS compliant ZIP archive. There
% the packages are already unpacked and the documentation files
% are generated. The files and directories obey the TDS standard.
%
%捆绑包“oberdiek”中的所有软件包也可在符合 TDS 标准的 ZIP 存档中获取。在该存档中,软件包已经被解包,文档文件已经生成,文件和目录符合 TDS 标准。
% \begin{description}
% \item[\CTANinstall{install/macros/latex/contrib/oberdiek.tds.zip}]
% \end{description}
% \emph{TDS} refers to the standard ``A Directory Structure
% for \TeX\ Files'' (\CTANpkg{tds}). Directories
% with \xfile{texmf} in their name are usually organized this way.
%
%\emph{TDS} 指的是标准“用于 \TeX\ 文件的目录结构”(\CTANpkg{tds})。名字中包含\xfile{texmf}的目录通常都是按这种方式组织的。
% \subsection{Bundle installation\\捆绑包安装}
%
% \paragraph{Unpacking.} Unpack the \xfile{oberdiek.tds.zip} in the
% TDS tree (also known as \xfile{texmf} tree) of your choice.
% Example (linux):
%
%\paragraph{解包。}在您选择的 TDS 树(也称为\xfile{texmf}树)中解压\xfile{oberdiek.tds.zip}。例如(在Linux中):
% \begin{quote}
%   |unzip oberdiek.tds.zip -d ~/texmf|
% \end{quote}
%
% \subsection{Package installation\\软件包安装}
%
% \paragraph{Unpacking.} The \xfile{.dtx} file is a self-extracting
% \docstrip\ archive. The files are extracted by running the
% \xfile{.dtx} through \plainTeX:
%
%\paragraph{解包。} \xfile{.dtx} 文件是一个自解压的 \docstrip\ 存档。运行\xfile{.dtx}通过\plainTeX\ 来提取文件:
% \begin{quote}
%   \verb|tex pdfcolparcolumns.dtx|
% \end{quote}
%
% \paragraph{TDS.} Now the different files must be moved into
% the different directories in your installation TDS tree
% (also known as \xfile{texmf} tree):
%
%\paragraph{TDS。}现在,不同的文件必须移动到安装 TDS 树(也称为\xfile{texmf}树)中的不同目录中:
% \begin{quote}
% \def\t{^^A
% \begin{tabular}{@{}>{\ttfamily}l@{ $\rightarrow$ }>{\ttfamily}l@{}}
%   pdfcolparcolumns.sty & tex/latex/oberdiek/pdfcolparcolumns.sty\\
%   pdfcolparcolumns.pdf & doc/latex/oberdiek/pdfcolparcolumns.pdf\\
%   pdfcolparcolumns.dtx & source/latex/oberdiek/pdfcolparcolumns.dtx\\
% \end{tabular}^^A
% }^^A
% \sbox0{\t}^^A
% \ifdim\wd0>\linewidth
%   \begingroup
%     \advance\linewidth by\leftmargin
%     \advance\linewidth by\rightmargin
%   \edef\x{\endgroup
%     \def\noexpand\lw{\the\linewidth}^^A
%   }\x
%   \def\lwbox{^^A
%     \leavevmode
%     \hbox to \linewidth{^^A
%       \kern-\leftmargin\relax
%       \hss
%       \usebox0
%       \hss
%       \kern-\rightmargin\relax
%     }^^A
%   }^^A
%   \ifdim\wd0>\lw
%     \sbox0{\small\t}^^A
%     \ifdim\wd0>\linewidth
%       \ifdim\wd0>\lw
%         \sbox0{\footnotesize\t}^^A
%         \ifdim\wd0>\linewidth
%           \ifdim\wd0>\lw
%             \sbox0{\scriptsize\t}^^A
%             \ifdim\wd0>\linewidth
%               \ifdim\wd0>\lw
%                 \sbox0{\tiny\t}^^A
%                 \ifdim\wd0>\linewidth
%                   \lwbox
%                 \else
%                   \usebox0
%                 \fi
%               \else
%                 \lwbox
%               \fi
%             \else
%               \usebox0
%             \fi
%           \else
%             \lwbox
%           \fi
%         \else
%           \usebox0
%         \fi
%       \else
%         \lwbox
%       \fi
%     \else
%       \usebox0
%     \fi
%   \else
%     \lwbox
%   \fi
% \else
%   \usebox0
% \fi
% \end{quote}
% If you have a \xfile{docstrip.cfg} that configures and enables \docstrip's
% TDS installing feature, then some files can already be in the right
% place, see the documentation of \docstrip.
%
%如果你有一个\xfile{docstrip.cfg}文件配置和启用了\docstrip 的TDS安装功能,那么一些文件可能已经位于正确的位置,具体请参见\docstrip 的文档。
% \subsection{Refresh file name databases\\刷新文件名数据库}
%
% If your \TeX~distribution
% (\TeX\,Live, \mikTeX, \dots) relies on file name databases, you must refresh
% these. For example, \TeX\,Live\ users run \verb|texhash| or
% \verb|mktexlsr|.
%
%如果你的\TeX~发行版(\TeX,Live、\mikTeX,等等)依赖于文件名数据库,你必须刷新它们。例如,\TeX,Live用户运行\verb|texhash|或\verb|mktexlsr|。
% \subsection{Some details for the interested\\一些细节供感兴趣的人使用}
%
% \paragraph{Unpacking with \LaTeX.}
% The \xfile{.dtx} chooses its action depending on the format:

%\paragraph{用\LaTeX 进行解包。} \xfile{.dtx}文件根据格式选择操作:
% \begin{description}
% \item[\plainTeX:] Run \docstrip\ and extract the files.
% \item[\LaTeX:] Generate the documentation.
% \end{description}
% If you insist on using \LaTeX\ for \docstrip\ (really,
% \docstrip\ does not need \LaTeX), then inform the autodetect routine
% about your intention:

%如果你坚持要用\LaTeX 进行\docstrip (其实\docstrip 不需要\LaTeX ),那么请告知自动检测例程你的意图:

% \begin{quote}
%   \verb|latex \let\install=y% \iffalse meta-comment
%
% File: pdfcolparcolumns.dtx
% Version: 2019/12/29 v1.5
% Info: Color stacks for parcolumns
%
% Copyright (C)
%    2007, 2008, 2010 Heiko Oberdiek
%    2016-2019 Oberdiek Package Support Group
%    https://github.com/ho-tex/oberdiek/issues
%
% This work may be distributed and/or modified under the
% conditions of the LaTeX Project Public License, either
% version 1.3c of this license or (at your option) any later
% version. This version of this license is in
%    https://www.latex-project.org/lppl/lppl-1-3c.txt
% and the latest version of this license is in
%    https://www.latex-project.org/lppl.txt
% and version 1.3 or later is part of all distributions of
% LaTeX version 2005/12/01 or later.
%
% This work has the LPPL maintenance status "maintained".
%
% The Current Maintainers of this work are
% Heiko Oberdiek and the Oberdiek Package Support Group
% https://github.com/ho-tex/oberdiek/issues
%
% This work consists of the main source file pdfcolparcolumns.dtx
% and the derived files
%    pdfcolparcolumns.sty, pdfcolparcolumns.pdf, pdfcolparcolumns.ins,
%    pdfcolparcolumns.drv, pdfcolparcolumns-test1.tex.
%
% Distribution:
%    CTAN:macros/latex/contrib/oberdiek/pdfcolparcolumns.dtx
%    CTAN:macros/latex/contrib/oberdiek/pdfcolparcolumns.pdf
%
% Unpacking:
%    (a) If pdfcolparcolumns.ins is present:
%           tex pdfcolparcolumns.ins
%    (b) Without pdfcolparcolumns.ins:
%           tex pdfcolparcolumns.dtx
%    (c) If you insist on using LaTeX
%           latex \let\install=y% \iffalse meta-comment
%
% File: pdfcolparcolumns.dtx
% Version: 2019/12/29 v1.5
% Info: Color stacks for parcolumns
%
% Copyright (C)
%    2007, 2008, 2010 Heiko Oberdiek
%    2016-2019 Oberdiek Package Support Group
%    https://github.com/ho-tex/oberdiek/issues
%
% This work may be distributed and/or modified under the
% conditions of the LaTeX Project Public License, either
% version 1.3c of this license or (at your option) any later
% version. This version of this license is in
%    https://www.latex-project.org/lppl/lppl-1-3c.txt
% and the latest version of this license is in
%    https://www.latex-project.org/lppl.txt
% and version 1.3 or later is part of all distributions of
% LaTeX version 2005/12/01 or later.
%
% This work has the LPPL maintenance status "maintained".
%
% The Current Maintainers of this work are
% Heiko Oberdiek and the Oberdiek Package Support Group
% https://github.com/ho-tex/oberdiek/issues
%
% This work consists of the main source file pdfcolparcolumns.dtx
% and the derived files
%    pdfcolparcolumns.sty, pdfcolparcolumns.pdf, pdfcolparcolumns.ins,
%    pdfcolparcolumns.drv, pdfcolparcolumns-test1.tex.
%
% Distribution:
%    CTAN:macros/latex/contrib/oberdiek/pdfcolparcolumns.dtx
%    CTAN:macros/latex/contrib/oberdiek/pdfcolparcolumns.pdf
%
% Unpacking:
%    (a) If pdfcolparcolumns.ins is present:
%           tex pdfcolparcolumns.ins
%    (b) Without pdfcolparcolumns.ins:
%           tex pdfcolparcolumns.dtx
%    (c) If you insist on using LaTeX
%           latex \let\install=y\input{pdfcolparcolumns.dtx}
%        (quote the arguments according to the demands of your shell)
%
% Documentation:
%    (a) If pdfcolparcolumns.drv is present:
%           latex pdfcolparcolumns.drv
%    (b) Without pdfcolparcolumns.drv:
%           latex pdfcolparcolumns.dtx; ...
%    The class ltxdoc loads the configuration file ltxdoc.cfg
%    if available. Here you can specify further options, e.g.
%    use A4 as paper format:
%       \PassOptionsToClass{a4paper}{article}
%
%    Programm calls to get the documentation (example):
%       pdflatex pdfcolparcolumns.dtx
%       makeindex -s gind.ist pdfcolparcolumns.idx
%       pdflatex pdfcolparcolumns.dtx
%       makeindex -s gind.ist pdfcolparcolumns.idx
%       pdflatex pdfcolparcolumns.dtx
%
% Installation:
%    TDS:tex/latex/oberdiek/pdfcolparcolumns.sty
%    TDS:doc/latex/oberdiek/pdfcolparcolumns.pdf
%    TDS:source/latex/oberdiek/pdfcolparcolumns.dtx
%
%<*ignore>
\begingroup
  \catcode123=1 %
  \catcode125=2 %
  \def\x{LaTeX2e}%
\expandafter\endgroup
\ifcase 0\ifx\install y1\fi\expandafter
         \ifx\csname processbatchFile\endcsname\relax\else1\fi
         \ifx\fmtname\x\else 1\fi\relax
\else\csname fi\endcsname
%</ignore>
%<*install>
\input docstrip.tex
\Msg{************************************************************************}
\Msg{* Installation}
\Msg{* Package: pdfcolparcolumns 2019/12/29 v1.5 Color stacks for parcolumns (HO)}
\Msg{************************************************************************}

\keepsilent
\askforoverwritefalse

\let\MetaPrefix\relax
\preamble

This is a generated file.

Project: pdfcolparcolumns
Version: 2019/12/29 v1.5

Copyright (C)
   2007, 2008, 2010 Heiko Oberdiek
   2016-2019 Oberdiek Package Support Group

This work may be distributed and/or modified under the
conditions of the LaTeX Project Public License, either
version 1.3c of this license or (at your option) any later
version. This version of this license is in
   https://www.latex-project.org/lppl/lppl-1-3c.txt
and the latest version of this license is in
   https://www.latex-project.org/lppl.txt
and version 1.3 or later is part of all distributions of
LaTeX version 2005/12/01 or later.

This work has the LPPL maintenance status "maintained".

The Current Maintainers of this work are
Heiko Oberdiek and the Oberdiek Package Support Group
https://github.com/ho-tex/oberdiek/issues


This work consists of the main source file pdfcolparcolumns.dtx
and the derived files
   pdfcolparcolumns.sty, pdfcolparcolumns.pdf, pdfcolparcolumns.ins,
   pdfcolparcolumns.drv, pdfcolparcolumns-test1.tex.

\endpreamble
\let\MetaPrefix\DoubleperCent

\generate{%
  \file{pdfcolparcolumns.ins}{\from{pdfcolparcolumns.dtx}{install}}%
  \file{pdfcolparcolumns.drv}{\from{pdfcolparcolumns.dtx}{driver}}%
  \usedir{tex/latex/oberdiek}%
  \file{pdfcolparcolumns.sty}{\from{pdfcolparcolumns.dtx}{package}}%
%  \usedir{doc/latex/oberdiek/test}%
%  \file{pdfcolparcolumns-test1.tex}{\from{pdfcolparcolumns.dtx}{test1}}%
}

\catcode32=13\relax% active space
\let =\space%
\Msg{************************************************************************}
\Msg{*}
\Msg{* To finish the installation you have to move the following}
\Msg{* file into a directory searched by TeX:}
\Msg{*}
\Msg{*     pdfcolparcolumns.sty}
\Msg{*}
\Msg{* To produce the documentation run the file `pdfcolparcolumns.drv'}
\Msg{* through LaTeX.}
\Msg{*}
\Msg{* Happy TeXing!}
\Msg{*}
\Msg{************************************************************************}

\endbatchfile
%</install>
%<*ignore>
\fi
%</ignore>
%<*driver>
\NeedsTeXFormat{LaTeX2e}
\ProvidesFile{pdfcolparcolumns.drv}%
  [2019/12/29 v1.5 Color stacks for parcolumns (HO)]%
\documentclass{ltxdoc}
\usepackage{holtxdoc}[2011/11/22]
\usepackage[scheme=plain]{ctex}
\setCJKmainfont{方正书宋_GBK}%方正书宋_GBK.TTF  设置文本的中文有衬线字体为“方正书宋_GBK”
\setCJKsansfont{方正黑体简体}%方正黑体_GBK.TTF  设置文本的中文无衬线字体为“方正黑体简体”
\setCJKmonofont{方正书宋简体}%方正仿宋_GBK.TTF  设置文本的中文等宽字体为“方正书宋简体”
\begin{document}
  \DocInput{pdfcolparcolumns.dtx}%
\end{document}
%</driver>
% \fi
%
%
%
% \GetFileInfo{pdfcolparcolumns.drv}
%
% \title{The \xpackage{pdfcolparcolumns} package}
% \date{2019/12/29 v1.5}
% \author{Heiko Oberdiek\thanks
% {Please report any issues at \url{https://github.com/ho-tex/oberdiek/issues}}\and 翻译\\{\tt virhuiai@qq.com}}
%
% \maketitle
%
% \begin{abstract}
%\makebox[0pt]{}
% \begin{parcolumns}[nofirstindent]{2}
%\colchunk{Since version 1.40 \pdfTeX\ supports several color stacks.
%This package uses them to fix color problems in
%package \xpackage{parcolumns}.}
%\colchunk{自从版本1.40起,\pdfTeX 支持多个颜色栈。
%此包使用它们来解决 \xpackage{parcolumns} 包中的颜色问题。}
%\end{parcolumns}



% \end{abstract}
%
% \tableofcontents
%
% \section{Usage\\用法}
%
% \begin{quote}
% |\usepackage{pdfcolparcolumns}|
% \end{quote}
% The package \xpackage{pdfcolparcolumns} loads package \xpackage{parcolums}
% \cite{parcolumns}. If color stacks are available then the
% macros of \xpackage{parcolumns} are patched to add support
% for color stacks.

%包 \xpackage{pdfcolparcolumns} 载入了 \xpackage{parcolumns} \cite{parcolumns} 包。如果有可用的颜色栈,则会对 \xpackage{parcolumns} 的宏进行修补,以添加对颜色栈的支持。

%
% \subsection{Option \xoption{rulebetweencolor}\\选项 \xoption{rulebetweencolor}}
%
% Package \xpackage{pdfcolparcolumns} also fixes the color for the
% rule between columns (if \xoption{rulebetween} is set).
% Default color is \cs{normalcolor}. But this can be changed by using
% option \xoption{rulebetweencolor}. It takes a color specification
% as value. If the value is empty, then the default (\cs{normalcolor})
% is used.
% Examples:

% 包 \xpackage{pdfcolparcolumns} 还修复了列之间分隔线的颜色(如果设置了 \xoption{rulebetween})。默认颜色为 \cs{normalcolor}。但是,可以使用选项 \xoption{rulebetweencolor} 来更改它。它接受一个颜色规范作为值。如果该值为空,则使用默认值(\cs{normalcolor})。例如:

% \begin{quote}
%   |rulebetweencolor=blue|,\\
%   |rulebetweencolor={red}|,\\
%   |rulebetweencolor={}|, \textit{\% \cs{normalcolor} is used}\\
%   |rulebetweencolor=[rgb]{1,0,.5}| \textit{\% see below}
% \end{quote}
% If used inside the optional argument of environment \textsf{parcolumns}
% and the value contains an optional argument, then whole value
% must be put in curly braces:

% 如果在 \textsf{parcolumns} 环境的可选参数中使用,并且值包含可选参数,则整个值必须用花括号括起来:
%\begin{quote}
%\begin{verbatim}
%\begin{parcolumns}[
%  rulebetween,
%  rulebetweencolor={[rgb]{1,0,.5}},
%]{2}
%  ...
%\end{parcolumns}
%\end{verbatim}
%\end{quote}
% This option \xoption{rulebetweencolor} can also be set using
% \cs{setkeys}:

%也可以使用 \cs{setkeys} 来设置选项 \xoption{rulebetweencolor}:
%\begin{quote}
%\begin{verbatim}
%\setkeys{parcolumns}{rulebetweencolor=green}
%\end{verbatim}
%\end{quote}
%
% \subsection{Future\\未来展望}
%
% Currently package \xpackage{parcolumns} does not seem to be
% maintained. Nevertheless if there will be a new version that
% adds support for color stacks, then this package may become
% obsolete.
%
%目前,\xpackage{parcolumns} 包似乎没有维护了。尽管如此,如果出现了新版本并支持颜色栈,那么这个包可能会变得过时。
% \StopEventually{
% }
%
% \section{Implementation\\实现}
%
% \subsection{Identification\\标识}
%
%    \begin{macrocode}
%<*package>
\NeedsTeXFormat{LaTeX2e}
\ProvidesPackage{pdfcolparcolumns}%
  [2019/12/29 v1.5 Color stacks for parcolumns (HO)]%
%    \end{macrocode}
%
% \subsection{Load packages\\加载宏包}
%
% \subsubsection{Package \xpackage{parcolumns}\\宏包 \xpackage{parcolumns}}
%
%    Currently package \xpackage{parcolumns} does not define options.
%    Thus it is just a precaution that the options of
%    package \xpackage{pdfcolparcolumns} are passed to
%    package \xpackage{parcolumns}.

%目前,宏包 \xpackage{parcolumns} 没有定义选项。因此,它只是一个预防措施,以确保将宏包 \xpackage{pdfcolparcolumns} 的选项传递给宏包 \xpackage{parcolumns}。
%    \begin{macrocode}
\DeclareOption*{%
  \PassOptionsToPackage{\CurrentOption}{parcolumns}%
}
\ProcessOptions\relax
\RequirePackage{parcolumns}[2004/11/25]
%    \end{macrocode}
%
% \subsubsection{Package \xpackage{pdfcol}\\宏包 \xpackage{pdfcol}}
%
%    \begin{macrocode}
\RequirePackage{pdfcol}[2007/09/09]
\ifpdfcolAvailable
\else
  \PackageInfo{pdfcolparcolumns}{%
    Loading aborted, because color stacks are not available%
  }%
  \expandafter\endinput
\fi
%    \end{macrocode}
%
% \subsubsection{Package \xpackage{infwarerr}\\宏包 \xpackage{infwarerr}}
%
%    \begin{macrocode}
\RequirePackage{infwarerr}[2007/09/09]
%    \end{macrocode}
%
% \subsection{Color stack macros\\颜色堆栈宏}
%
%    \begin{macro}{\pcpc@MaxStack}
%    Macro \cs{pcpc@MaxStack} holds the highest number of
%    allocated stacks.

%宏 \cs{pcpc@MaxStack} 存储已分配的堆栈中最大的编号。
%    \begin{macrocode}
\global\chardef\pcpc@MaxStack=\z@
%    \end{macrocode}
%    \end{macro}
%    \begin{macro}{\pcpc@InitStacks}
%    Macro \cs{pcpc@InitStacks} takes the number of columns
%    as argument and ensures that there are enough color
%    stacks for all columns.
%
%宏 \cs{pcpc@InitStacks} 接受列数作为参数,并确保为所有列分配了足够的颜色堆栈。
%    \begin{macrocode}
\def\pcpc@InitStacks#1{%
  \ifnum#1>\pcpc@MaxStack
    \begingroup
      \count@\pcpc@MaxStack
      \loop
        \advance\count@\@ne
        \pdfcolInitStack{pcpc@\the\count@}%
      \ifnum#1>\count@
      \repeat
      \global\chardef\pcpc@MaxStack=\count@
    \endgroup
  \fi
}
%    \end{macrocode}
%    \end{macro}
%
%    \begin{macro}{\pcpc@SwitchStack}
%    \begin{macrocode}
\def\pcpc@SwitchStack#1{%
  \pdfcolSwitchStack{pcpc@\number#1}%
}
%    \end{macrocode}
%    \end{macro}
%
%    \begin{macro}{\pcpc@SetCurrent}
%    \begin{macrocode}
\def\pcpc@SetCurrent#1{%
  \pdfcolSetCurrent{pcpc@\number#1}%
}
%    \end{macrocode}
%    \end{macro}
%
% \subsection{Patches\\补丁}
%
%     Now the color stack macros are patched into the macros
%     of package \xpackage{parcolumns}.
%
%现在颜色堆栈宏已经被打补丁到了 \xpackage{parcolumns} 宏包的宏中。
% \subsubsection{Init stacks\\初始化堆栈}
%
%    \cs{pcpc@InitStacks} should go into the definition of
%    environment |parcolumns|. \cs{pc@alloccolumns} is executed
%    there and nowhere else, thus we hook into it.

%\cs{pcpc@InitStacks} 应该被放在环境 |parcolumns| 的定义中。\cs{pc@alloccolumns} 仅在那里执行,因此我们将其连接起来。
%    \begin{macrocode}
\g@addto@macro\pc@alloccolumns{%
  \pcpc@InitStacks\pc@columncount
}
%    \end{macrocode}
%
% \subsubsection{Switch stack\\切换堆栈}
%
%    \cs{pcpc@SwitchStack} should be called by marco \cs{colchunk@}.
%    However it is easier to patch \cs{pc@setcolumnwidth} that
%    is executed in \cs{colchunk@} only.
%
%\cs{pcpc@SwitchStack} 应该被宏 \cs{colchunk@} 调用。不过,我们更容易修补仅在 \cs{colchunk@} 中执行的 \cs{pc@setcolumnwidth}。
%    \begin{macrocode}
\g@addto@macro\pc@setcolumnwidth{%
  \pcpc@SwitchStack\pc@columnctr
}
%    \end{macrocode}
%
% \subsubsection{Set current stack color\\设置当前堆栈颜色}
%
%    \cs{pcpc@SetCurrent} is set at the begin of each line.
%    It must be inserted into \cs{pc@placeboxes}. Unhappily
%    there is no easy way. Therefore we check and
%    redefine \cs{pc@placeboxes}.
%
%\cs{pcpc@SetCurrent} 在每行开始时设置。它必须插入到 \cs{pc@placeboxes} 中。不幸的是,没有简单的方法。因此,我们检查并重新定义 \cs{pc@placeboxes}。
%    \begin{macrocode}
\begingroup
  \def\x{%
    \global\let\@tempa\relax
    \count@\z@
    \hb@xt@\linewidth{%
      \vfuzz30ex %
      \vbadness\@M
      \splittopskip\z@skip
      \loop
      \ifnum\count@<\pc@columncount
        \advance\count@\@ne
        \expandafter\ifvoid\csname pc@column@\number\count@\endcsname
          \hskip\csname pc@column@width@\number\count@\endcsname
        \else
          \expandafter\setbox\expandafter\@tempboxa\expandafter
          \vsplit\csname pc@column@\number\count@\endcsname
              to \dp\strutbox
          \vbox{%
            \unvbox\@tempboxa
          }%
        \fi
        \expandafter\ifvoid\csname pc@column@\number\count@\endcsname
        \else
          \global\let\@tempa\pc@placeboxes
        \fi
        \ifnum\count@<\pc@columncount
          \strut
          \hfill
          \ifpc@rulebetween
            \vrule
            \hfill
          \fi
        \fi
      \repeat
    }%
    \@tempa
  }%
  \ifx\x\pc@placeboxes
  \else
    \@PackageWarningNoLine{pdfcolparcolumns}{%
      Command \string\pc@placeboxes\space has changed.\MessageBreak
      Supported versions of package `parcolumns':\MessageBreak
      \space\space 2004/08/05.\MessageBreak
      The redefinition of \string\pc@placeboxes\space may not%
      \MessageBreak
      behave correctly depending on the changes%
    }%
  \fi
\endgroup
%    \end{macrocode}
%    \begin{macro}{\pc@placeboxes}
%    \begin{macrocode}
\renewcommand*{\pc@placeboxes}{%
  \global\let\@tempa\relax
  \count@\z@
  \hb@xt@\linewidth{%
    \vfuzz30ex %
    \vbadness\@M
    \splittopskip\z@skip
    \loop
    \ifnum\count@<\pc@columncount
      \advance\count@\@ne
      \expandafter\ifvoid\csname pc@column@\number\count@\endcsname
        \hskip\csname pc@column@width@\number\count@\endcsname
      \else
        \expandafter\setbox\expandafter\@tempboxa\expandafter
        \vsplit\csname pc@column@\number\count@\endcsname
            to \dp\strutbox
        \vbox{%
          \pcpc@SetCurrent\count@
          \unvbox\@tempboxa
        }%
      \fi
      \expandafter\ifvoid\csname pc@column@\number\count@\endcsname
      \else
        \global\let\@tempa\pc@placeboxes
      \fi
      \ifnum\count@<\pc@columncount
        \strut
        \hfill
        \ifpc@rulebetween
          \begingroup
            \pcpc@RuleBetweenColor
            \vrule
          \endgroup
          \hfill
        \fi
      \fi
    \repeat
  }%
  \@tempa
}
%    \end{macrocode}
%    \end{macro}
%    \begin{macro}{\pcpc@RuleBetweenColorDefault}
%    \begin{macrocode}
\def\pcpc@RuleBetweenColorDefault{%
  \normalcolor
}
%    \end{macrocode}
%    \end{macro}
%    \begin{macro}{\pcpc@RuleBetweenColor}
%    \begin{macrocode}
\let\pcpc@RuleBetweenColor\pcpc@RuleBetweenColorDefault
%    \end{macrocode}
%    \end{macro}
%    \begin{macrocode}
\define@key{parcolumns}{rulebetweencolor}{%
  \edef\pcpc@temp{#1}%
  \ifx\pcpc@temp\@empty
    \let\pcpc@RuleBetweenColor\pcpc@RuleBetweenColorDefault
  \else
    \edef\pcpc@temp{%
      \noexpand\@ifnextchar[{%
        \def\noexpand\pcpc@RuleBetweenColor{%
          \noexpand\color\pcpc@temp
        }%
        \noexpand\pcpc@GobbleNil
      }{%
        \def\noexpand\pcpc@RuleBetweenColor{%
          \noexpand\color{\pcpc@temp}%
        }%
        \noexpand\pcpc@GobbleNil
      }%
      \pcpc@temp\noexpand\@nil
    }%
    \pcpc@temp
  \fi
}
%    \end{macrocode}
%    \begin{macro}{\pcpc@GobbleNil}
%    \begin{macrocode}
\long\def\pcpc@GobbleNil#1\@nil{}
%    \end{macrocode}
%    \end{macro}
%
%    \begin{macrocode}
%</package>
%    \end{macrocode}
%% \section{Installation\\安装}
%
% \subsection{Download\\下载}
%
% \paragraph{Package.} This package is available on
% CTAN\footnote{\CTANpkg{pdfcolparcolumns}}:

%该软件包可从 CTAN\footnote{\CTANpkg{pdfcolparcolumns}} 下载:
% \begin{description}
% \item[\CTAN{macros/latex/contrib/oberdiek/pdfcolparcolumns.dtx}] The source file.
% \item[\CTAN{macros/latex/contrib/oberdiek/pdfcolparcolumns.pdf}] Documentation.
% \end{description}
%
%
% \paragraph{Bundle.} All the packages of the bundle `oberdiek'
% are also available in a TDS compliant ZIP archive. There
% the packages are already unpacked and the documentation files
% are generated. The files and directories obey the TDS standard.
%
%捆绑包“oberdiek”中的所有软件包也可在符合 TDS 标准的 ZIP 存档中获取。在该存档中,软件包已经被解包,文档文件已经生成,文件和目录符合 TDS 标准。
% \begin{description}
% \item[\CTANinstall{install/macros/latex/contrib/oberdiek.tds.zip}]
% \end{description}
% \emph{TDS} refers to the standard ``A Directory Structure
% for \TeX\ Files'' (\CTANpkg{tds}). Directories
% with \xfile{texmf} in their name are usually organized this way.
%
%\emph{TDS} 指的是标准“用于 \TeX\ 文件的目录结构”(\CTANpkg{tds})。名字中包含\xfile{texmf}的目录通常都是按这种方式组织的。
% \subsection{Bundle installation\\捆绑包安装}
%
% \paragraph{Unpacking.} Unpack the \xfile{oberdiek.tds.zip} in the
% TDS tree (also known as \xfile{texmf} tree) of your choice.
% Example (linux):
%
%\paragraph{解包。}在您选择的 TDS 树(也称为\xfile{texmf}树)中解压\xfile{oberdiek.tds.zip}。例如(在Linux中):
% \begin{quote}
%   |unzip oberdiek.tds.zip -d ~/texmf|
% \end{quote}
%
% \subsection{Package installation\\软件包安装}
%
% \paragraph{Unpacking.} The \xfile{.dtx} file is a self-extracting
% \docstrip\ archive. The files are extracted by running the
% \xfile{.dtx} through \plainTeX:
%
%\paragraph{解包。} \xfile{.dtx} 文件是一个自解压的 \docstrip\ 存档。运行\xfile{.dtx}通过\plainTeX\ 来提取文件:
% \begin{quote}
%   \verb|tex pdfcolparcolumns.dtx|
% \end{quote}
%
% \paragraph{TDS.} Now the different files must be moved into
% the different directories in your installation TDS tree
% (also known as \xfile{texmf} tree):
%
%\paragraph{TDS。}现在,不同的文件必须移动到安装 TDS 树(也称为\xfile{texmf}树)中的不同目录中:
% \begin{quote}
% \def\t{^^A
% \begin{tabular}{@{}>{\ttfamily}l@{ $\rightarrow$ }>{\ttfamily}l@{}}
%   pdfcolparcolumns.sty & tex/latex/oberdiek/pdfcolparcolumns.sty\\
%   pdfcolparcolumns.pdf & doc/latex/oberdiek/pdfcolparcolumns.pdf\\
%   pdfcolparcolumns.dtx & source/latex/oberdiek/pdfcolparcolumns.dtx\\
% \end{tabular}^^A
% }^^A
% \sbox0{\t}^^A
% \ifdim\wd0>\linewidth
%   \begingroup
%     \advance\linewidth by\leftmargin
%     \advance\linewidth by\rightmargin
%   \edef\x{\endgroup
%     \def\noexpand\lw{\the\linewidth}^^A
%   }\x
%   \def\lwbox{^^A
%     \leavevmode
%     \hbox to \linewidth{^^A
%       \kern-\leftmargin\relax
%       \hss
%       \usebox0
%       \hss
%       \kern-\rightmargin\relax
%     }^^A
%   }^^A
%   \ifdim\wd0>\lw
%     \sbox0{\small\t}^^A
%     \ifdim\wd0>\linewidth
%       \ifdim\wd0>\lw
%         \sbox0{\footnotesize\t}^^A
%         \ifdim\wd0>\linewidth
%           \ifdim\wd0>\lw
%             \sbox0{\scriptsize\t}^^A
%             \ifdim\wd0>\linewidth
%               \ifdim\wd0>\lw
%                 \sbox0{\tiny\t}^^A
%                 \ifdim\wd0>\linewidth
%                   \lwbox
%                 \else
%                   \usebox0
%                 \fi
%               \else
%                 \lwbox
%               \fi
%             \else
%               \usebox0
%             \fi
%           \else
%             \lwbox
%           \fi
%         \else
%           \usebox0
%         \fi
%       \else
%         \lwbox
%       \fi
%     \else
%       \usebox0
%     \fi
%   \else
%     \lwbox
%   \fi
% \else
%   \usebox0
% \fi
% \end{quote}
% If you have a \xfile{docstrip.cfg} that configures and enables \docstrip's
% TDS installing feature, then some files can already be in the right
% place, see the documentation of \docstrip.
%
%如果你有一个\xfile{docstrip.cfg}文件配置和启用了\docstrip 的TDS安装功能,那么一些文件可能已经位于正确的位置,具体请参见\docstrip 的文档。
% \subsection{Refresh file name databases\\刷新文件名数据库}
%
% If your \TeX~distribution
% (\TeX\,Live, \mikTeX, \dots) relies on file name databases, you must refresh
% these. For example, \TeX\,Live\ users run \verb|texhash| or
% \verb|mktexlsr|.
%
%如果你的\TeX~发行版(\TeX,Live、\mikTeX,等等)依赖于文件名数据库,你必须刷新它们。例如,\TeX,Live用户运行\verb|texhash|或\verb|mktexlsr|。
% \subsection{Some details for the interested\\一些细节供感兴趣的人使用}
%
% \paragraph{Unpacking with \LaTeX.}
% The \xfile{.dtx} chooses its action depending on the format:

%\paragraph{用\LaTeX 进行解包。} \xfile{.dtx}文件根据格式选择操作:
% \begin{description}
% \item[\plainTeX:] Run \docstrip\ and extract the files.
% \item[\LaTeX:] Generate the documentation.
% \end{description}
% If you insist on using \LaTeX\ for \docstrip\ (really,
% \docstrip\ does not need \LaTeX), then inform the autodetect routine
% about your intention:

%如果你坚持要用\LaTeX 进行\docstrip (其实\docstrip 不需要\LaTeX ),那么请告知自动检测例程你的意图:

% \begin{quote}
%   \verb|latex \let\install=y\input{pdfcolparcolumns.dtx}|
% \end{quote}
% Do not forget to quote the argument according to the demands
% of your shell.

%不要忘记根据你的shell的要求引用参数。
%
% \paragraph{Generating the documentation.}
% You can use both the \xfile{.dtx} or the \xfile{.drv} to generate
% the documentation. The process can be configured by the
% configuration file \xfile{ltxdoc.cfg}. For instance, put this
% line into this file, if you want to have A4 as paper format:
%
%\paragraph{生成文档。} 你可以使用\xfile{.dtx}或\xfile{.drv}生成文档。这个过程可以由配置文件\xfile{ltxdoc.cfg}配置。例如,如果你想要A4作为纸张格式,将此行放入文件中:

% \begin{quote}
%   \verb|\PassOptionsToClass{a4paper}{article}|
% \end{quote}
% An example follows how to generate the
% documentation with pdf\LaTeX:

%以下是使用pdf\LaTeX 生成文档的示例:
% \begin{quote}
%\begin{verbatim}
%pdflatex pdfcolparcolumns.dtx
%makeindex -s gind.ist pdfcolparcolumns.idx
%pdflatex pdfcolparcolumns.dtx
%makeindex -s gind.ist pdfcolparcolumns.idx
%pdflatex pdfcolparcolumns.dtx
%\end{verbatim}
% \end{quote}
%
% \begin{thebibliography}{9}
%
% \bibitem{parcolumns}
%   Jonathan Sauer: \textit{The \xpackage{parcolumns} package};
%   2004/11/25;\\
%   \CTANpkg{parcolumns}.
%
% \bibitem{pdfcol}
%   Heiko Oberdiek: \textit{The \xpackage{pdfcol} package};
%   2007/09/09;\\
%   \CTANpkg{pdfcol}.
%
% \end{thebibliography}
%
% \begin{History}
%   \begin{Version}{2007/07/26 v1.0}
%   \item
%     First version, published in the newsgroup \xnewsgroup{comp.text.tex}
%     with the name \xpackage{parcolumns-colorstacks}: ^^A no line break
%     \URL{``\link{Re: \xpackage{xcolor} glitches}''}^^A
%     {https://groups.google.com/group/comp.text.tex/msg/56bd897b11bca414}

%第一个版本,以 \xnewsgroup{comp.text.tex} 新闻组发布,名称为 \xpackage{parcolumns-colorstacks}:^^A 没有换行符 
%\URL{``\link{Re: \xpackage{xcolor} glitches}''}^^A
%{https://groups.google.com/group/comp.text.tex/msg/56bd897b11bca414}
%   \end{Version}
%   \begin{Version}{2007/09/09 v1.1}
%   \item
%     CTAN version, package name renamed to \xpackage{pdfcolparcolumns}.
%
%CTAN 版本,将包名改为 \xpackage{pdfcolparcolumns}。
%   \item
%     Uses package \xpackage{pdfcol}.
%
%使用 \xpackage{pdfcol} 包。
%   \item
%     Documentation added.
%
%添加文档。
%   \item
%     Test file added.
%
%添加测试文件。
%   \end{Version}
%   \begin{Version}{2008/08/11 v1.2}
%   \item
%     Code is not changed.
%
%代码未改动。
%   \item
%     URLs updated.
%
%更新 URL。
%   \end{Version}
%   \begin{Version}{2010/01/11 v1.3}
%   \item
%     Fix for rule color.
%
%修复分隔线的颜色。
%   \item
%     New option \xoption{rulebetweencolor} for environment |parcolumns|.
%
%为环境 |parcolumns| 添加新选项 \xoption{rulebetweencolor}。
%   \end{Version}
%   \begin{Version}{2016/05/16 v1.4}
%   \item
%     Documentation updates.
%
%更新文档。
%   \end{Version}
%   \begin{Version}{2019/12/29 v1.5}
%   \item
%     \cs{PassOptionsToPackage} not \cs{PassoptionsToPackage}
%
%\cs{PassOptionsToPackage} 改为不区分大小写的 \cs{PassoptionsToPackage}。
%   \end{Version}
% \end{History}
%
% \PrintIndex
%
% \Finale
\endinput

%        (quote the arguments according to the demands of your shell)
%
% Documentation:
%    (a) If pdfcolparcolumns.drv is present:
%           latex pdfcolparcolumns.drv
%    (b) Without pdfcolparcolumns.drv:
%           latex pdfcolparcolumns.dtx; ...
%    The class ltxdoc loads the configuration file ltxdoc.cfg
%    if available. Here you can specify further options, e.g.
%    use A4 as paper format:
%       \PassOptionsToClass{a4paper}{article}
%
%    Programm calls to get the documentation (example):
%       pdflatex pdfcolparcolumns.dtx
%       makeindex -s gind.ist pdfcolparcolumns.idx
%       pdflatex pdfcolparcolumns.dtx
%       makeindex -s gind.ist pdfcolparcolumns.idx
%       pdflatex pdfcolparcolumns.dtx
%
% Installation:
%    TDS:tex/latex/oberdiek/pdfcolparcolumns.sty
%    TDS:doc/latex/oberdiek/pdfcolparcolumns.pdf
%    TDS:source/latex/oberdiek/pdfcolparcolumns.dtx
%
%<*ignore>
\begingroup
  \catcode123=1 %
  \catcode125=2 %
  \def\x{LaTeX2e}%
\expandafter\endgroup
\ifcase 0\ifx\install y1\fi\expandafter
         \ifx\csname processbatchFile\endcsname\relax\else1\fi
         \ifx\fmtname\x\else 1\fi\relax
\else\csname fi\endcsname
%</ignore>
%<*install>
\input docstrip.tex
\Msg{************************************************************************}
\Msg{* Installation}
\Msg{* Package: pdfcolparcolumns 2019/12/29 v1.5 Color stacks for parcolumns (HO)}
\Msg{************************************************************************}

\keepsilent
\askforoverwritefalse

\let\MetaPrefix\relax
\preamble

This is a generated file.

Project: pdfcolparcolumns
Version: 2019/12/29 v1.5

Copyright (C)
   2007, 2008, 2010 Heiko Oberdiek
   2016-2019 Oberdiek Package Support Group

This work may be distributed and/or modified under the
conditions of the LaTeX Project Public License, either
version 1.3c of this license or (at your option) any later
version. This version of this license is in
   https://www.latex-project.org/lppl/lppl-1-3c.txt
and the latest version of this license is in
   https://www.latex-project.org/lppl.txt
and version 1.3 or later is part of all distributions of
LaTeX version 2005/12/01 or later.

This work has the LPPL maintenance status "maintained".

The Current Maintainers of this work are
Heiko Oberdiek and the Oberdiek Package Support Group
https://github.com/ho-tex/oberdiek/issues


This work consists of the main source file pdfcolparcolumns.dtx
and the derived files
   pdfcolparcolumns.sty, pdfcolparcolumns.pdf, pdfcolparcolumns.ins,
   pdfcolparcolumns.drv, pdfcolparcolumns-test1.tex.

\endpreamble
\let\MetaPrefix\DoubleperCent

\generate{%
  \file{pdfcolparcolumns.ins}{\from{pdfcolparcolumns.dtx}{install}}%
  \file{pdfcolparcolumns.drv}{\from{pdfcolparcolumns.dtx}{driver}}%
  \usedir{tex/latex/oberdiek}%
  \file{pdfcolparcolumns.sty}{\from{pdfcolparcolumns.dtx}{package}}%
%  \usedir{doc/latex/oberdiek/test}%
%  \file{pdfcolparcolumns-test1.tex}{\from{pdfcolparcolumns.dtx}{test1}}%
}

\catcode32=13\relax% active space
\let =\space%
\Msg{************************************************************************}
\Msg{*}
\Msg{* To finish the installation you have to move the following}
\Msg{* file into a directory searched by TeX:}
\Msg{*}
\Msg{*     pdfcolparcolumns.sty}
\Msg{*}
\Msg{* To produce the documentation run the file `pdfcolparcolumns.drv'}
\Msg{* through LaTeX.}
\Msg{*}
\Msg{* Happy TeXing!}
\Msg{*}
\Msg{************************************************************************}

\endbatchfile
%</install>
%<*ignore>
\fi
%</ignore>
%<*driver>
\NeedsTeXFormat{LaTeX2e}
\ProvidesFile{pdfcolparcolumns.drv}%
  [2019/12/29 v1.5 Color stacks for parcolumns (HO)]%
\documentclass{ltxdoc}
\usepackage{holtxdoc}[2011/11/22]
\usepackage[scheme=plain]{ctex}
\setCJKmainfont{方正书宋_GBK}%方正书宋_GBK.TTF  设置文本的中文有衬线字体为“方正书宋_GBK”
\setCJKsansfont{方正黑体简体}%方正黑体_GBK.TTF  设置文本的中文无衬线字体为“方正黑体简体”
\setCJKmonofont{方正书宋简体}%方正仿宋_GBK.TTF  设置文本的中文等宽字体为“方正书宋简体”
\begin{document}
  \DocInput{pdfcolparcolumns.dtx}%
\end{document}
%</driver>
% \fi
%
%
%
% \GetFileInfo{pdfcolparcolumns.drv}
%
% \title{The \xpackage{pdfcolparcolumns} package}
% \date{2019/12/29 v1.5}
% \author{Heiko Oberdiek\thanks
% {Please report any issues at \url{https://github.com/ho-tex/oberdiek/issues}}\and 翻译\\{\tt virhuiai@qq.com}}
%
% \maketitle
%
% \begin{abstract}
%\makebox[0pt]{}
% \begin{parcolumns}[nofirstindent]{2}
%\colchunk{Since version 1.40 \pdfTeX\ supports several color stacks.
%This package uses them to fix color problems in
%package \xpackage{parcolumns}.}
%\colchunk{自从版本1.40起,\pdfTeX 支持多个颜色栈。
%此包使用它们来解决 \xpackage{parcolumns} 包中的颜色问题。}
%\end{parcolumns}



% \end{abstract}
%
% \tableofcontents
%
% \section{Usage\\用法}
%
% \begin{quote}
% |\usepackage{pdfcolparcolumns}|
% \end{quote}
% The package \xpackage{pdfcolparcolumns} loads package \xpackage{parcolums}
% \cite{parcolumns}. If color stacks are available then the
% macros of \xpackage{parcolumns} are patched to add support
% for color stacks.

%包 \xpackage{pdfcolparcolumns} 载入了 \xpackage{parcolumns} \cite{parcolumns} 包。如果有可用的颜色栈,则会对 \xpackage{parcolumns} 的宏进行修补,以添加对颜色栈的支持。

%
% \subsection{Option \xoption{rulebetweencolor}\\选项 \xoption{rulebetweencolor}}
%
% Package \xpackage{pdfcolparcolumns} also fixes the color for the
% rule between columns (if \xoption{rulebetween} is set).
% Default color is \cs{normalcolor}. But this can be changed by using
% option \xoption{rulebetweencolor}. It takes a color specification
% as value. If the value is empty, then the default (\cs{normalcolor})
% is used.
% Examples:

% 包 \xpackage{pdfcolparcolumns} 还修复了列之间分隔线的颜色(如果设置了 \xoption{rulebetween})。默认颜色为 \cs{normalcolor}。但是,可以使用选项 \xoption{rulebetweencolor} 来更改它。它接受一个颜色规范作为值。如果该值为空,则使用默认值(\cs{normalcolor})。例如:

% \begin{quote}
%   |rulebetweencolor=blue|,\\
%   |rulebetweencolor={red}|,\\
%   |rulebetweencolor={}|, \textit{\% \cs{normalcolor} is used}\\
%   |rulebetweencolor=[rgb]{1,0,.5}| \textit{\% see below}
% \end{quote}
% If used inside the optional argument of environment \textsf{parcolumns}
% and the value contains an optional argument, then whole value
% must be put in curly braces:

% 如果在 \textsf{parcolumns} 环境的可选参数中使用,并且值包含可选参数,则整个值必须用花括号括起来:
%\begin{quote}
%\begin{verbatim}
%\begin{parcolumns}[
%  rulebetween,
%  rulebetweencolor={[rgb]{1,0,.5}},
%]{2}
%  ...
%\end{parcolumns}
%\end{verbatim}
%\end{quote}
% This option \xoption{rulebetweencolor} can also be set using
% \cs{setkeys}:

%也可以使用 \cs{setkeys} 来设置选项 \xoption{rulebetweencolor}:
%\begin{quote}
%\begin{verbatim}
%\setkeys{parcolumns}{rulebetweencolor=green}
%\end{verbatim}
%\end{quote}
%
% \subsection{Future\\未来展望}
%
% Currently package \xpackage{parcolumns} does not seem to be
% maintained. Nevertheless if there will be a new version that
% adds support for color stacks, then this package may become
% obsolete.
%
%目前,\xpackage{parcolumns} 包似乎没有维护了。尽管如此,如果出现了新版本并支持颜色栈,那么这个包可能会变得过时。
% \StopEventually{
% }
%
% \section{Implementation\\实现}
%
% \subsection{Identification\\标识}
%
%    \begin{macrocode}
%<*package>
\NeedsTeXFormat{LaTeX2e}
\ProvidesPackage{pdfcolparcolumns}%
  [2019/12/29 v1.5 Color stacks for parcolumns (HO)]%
%    \end{macrocode}
%
% \subsection{Load packages\\加载宏包}
%
% \subsubsection{Package \xpackage{parcolumns}\\宏包 \xpackage{parcolumns}}
%
%    Currently package \xpackage{parcolumns} does not define options.
%    Thus it is just a precaution that the options of
%    package \xpackage{pdfcolparcolumns} are passed to
%    package \xpackage{parcolumns}.

%目前,宏包 \xpackage{parcolumns} 没有定义选项。因此,它只是一个预防措施,以确保将宏包 \xpackage{pdfcolparcolumns} 的选项传递给宏包 \xpackage{parcolumns}。
%    \begin{macrocode}
\DeclareOption*{%
  \PassOptionsToPackage{\CurrentOption}{parcolumns}%
}
\ProcessOptions\relax
\RequirePackage{parcolumns}[2004/11/25]
%    \end{macrocode}
%
% \subsubsection{Package \xpackage{pdfcol}\\宏包 \xpackage{pdfcol}}
%
%    \begin{macrocode}
\RequirePackage{pdfcol}[2007/09/09]
\ifpdfcolAvailable
\else
  \PackageInfo{pdfcolparcolumns}{%
    Loading aborted, because color stacks are not available%
  }%
  \expandafter\endinput
\fi
%    \end{macrocode}
%
% \subsubsection{Package \xpackage{infwarerr}\\宏包 \xpackage{infwarerr}}
%
%    \begin{macrocode}
\RequirePackage{infwarerr}[2007/09/09]
%    \end{macrocode}
%
% \subsection{Color stack macros\\颜色堆栈宏}
%
%    \begin{macro}{\pcpc@MaxStack}
%    Macro \cs{pcpc@MaxStack} holds the highest number of
%    allocated stacks.

%宏 \cs{pcpc@MaxStack} 存储已分配的堆栈中最大的编号。
%    \begin{macrocode}
\global\chardef\pcpc@MaxStack=\z@
%    \end{macrocode}
%    \end{macro}
%    \begin{macro}{\pcpc@InitStacks}
%    Macro \cs{pcpc@InitStacks} takes the number of columns
%    as argument and ensures that there are enough color
%    stacks for all columns.
%
%宏 \cs{pcpc@InitStacks} 接受列数作为参数,并确保为所有列分配了足够的颜色堆栈。
%    \begin{macrocode}
\def\pcpc@InitStacks#1{%
  \ifnum#1>\pcpc@MaxStack
    \begingroup
      \count@\pcpc@MaxStack
      \loop
        \advance\count@\@ne
        \pdfcolInitStack{pcpc@\the\count@}%
      \ifnum#1>\count@
      \repeat
      \global\chardef\pcpc@MaxStack=\count@
    \endgroup
  \fi
}
%    \end{macrocode}
%    \end{macro}
%
%    \begin{macro}{\pcpc@SwitchStack}
%    \begin{macrocode}
\def\pcpc@SwitchStack#1{%
  \pdfcolSwitchStack{pcpc@\number#1}%
}
%    \end{macrocode}
%    \end{macro}
%
%    \begin{macro}{\pcpc@SetCurrent}
%    \begin{macrocode}
\def\pcpc@SetCurrent#1{%
  \pdfcolSetCurrent{pcpc@\number#1}%
}
%    \end{macrocode}
%    \end{macro}
%
% \subsection{Patches\\补丁}
%
%     Now the color stack macros are patched into the macros
%     of package \xpackage{parcolumns}.
%
%现在颜色堆栈宏已经被打补丁到了 \xpackage{parcolumns} 宏包的宏中。
% \subsubsection{Init stacks\\初始化堆栈}
%
%    \cs{pcpc@InitStacks} should go into the definition of
%    environment |parcolumns|. \cs{pc@alloccolumns} is executed
%    there and nowhere else, thus we hook into it.

%\cs{pcpc@InitStacks} 应该被放在环境 |parcolumns| 的定义中。\cs{pc@alloccolumns} 仅在那里执行,因此我们将其连接起来。
%    \begin{macrocode}
\g@addto@macro\pc@alloccolumns{%
  \pcpc@InitStacks\pc@columncount
}
%    \end{macrocode}
%
% \subsubsection{Switch stack\\切换堆栈}
%
%    \cs{pcpc@SwitchStack} should be called by marco \cs{colchunk@}.
%    However it is easier to patch \cs{pc@setcolumnwidth} that
%    is executed in \cs{colchunk@} only.
%
%\cs{pcpc@SwitchStack} 应该被宏 \cs{colchunk@} 调用。不过,我们更容易修补仅在 \cs{colchunk@} 中执行的 \cs{pc@setcolumnwidth}。
%    \begin{macrocode}
\g@addto@macro\pc@setcolumnwidth{%
  \pcpc@SwitchStack\pc@columnctr
}
%    \end{macrocode}
%
% \subsubsection{Set current stack color\\设置当前堆栈颜色}
%
%    \cs{pcpc@SetCurrent} is set at the begin of each line.
%    It must be inserted into \cs{pc@placeboxes}. Unhappily
%    there is no easy way. Therefore we check and
%    redefine \cs{pc@placeboxes}.
%
%\cs{pcpc@SetCurrent} 在每行开始时设置。它必须插入到 \cs{pc@placeboxes} 中。不幸的是,没有简单的方法。因此,我们检查并重新定义 \cs{pc@placeboxes}。
%    \begin{macrocode}
\begingroup
  \def\x{%
    \global\let\@tempa\relax
    \count@\z@
    \hb@xt@\linewidth{%
      \vfuzz30ex %
      \vbadness\@M
      \splittopskip\z@skip
      \loop
      \ifnum\count@<\pc@columncount
        \advance\count@\@ne
        \expandafter\ifvoid\csname pc@column@\number\count@\endcsname
          \hskip\csname pc@column@width@\number\count@\endcsname
        \else
          \expandafter\setbox\expandafter\@tempboxa\expandafter
          \vsplit\csname pc@column@\number\count@\endcsname
              to \dp\strutbox
          \vbox{%
            \unvbox\@tempboxa
          }%
        \fi
        \expandafter\ifvoid\csname pc@column@\number\count@\endcsname
        \else
          \global\let\@tempa\pc@placeboxes
        \fi
        \ifnum\count@<\pc@columncount
          \strut
          \hfill
          \ifpc@rulebetween
            \vrule
            \hfill
          \fi
        \fi
      \repeat
    }%
    \@tempa
  }%
  \ifx\x\pc@placeboxes
  \else
    \@PackageWarningNoLine{pdfcolparcolumns}{%
      Command \string\pc@placeboxes\space has changed.\MessageBreak
      Supported versions of package `parcolumns':\MessageBreak
      \space\space 2004/08/05.\MessageBreak
      The redefinition of \string\pc@placeboxes\space may not%
      \MessageBreak
      behave correctly depending on the changes%
    }%
  \fi
\endgroup
%    \end{macrocode}
%    \begin{macro}{\pc@placeboxes}
%    \begin{macrocode}
\renewcommand*{\pc@placeboxes}{%
  \global\let\@tempa\relax
  \count@\z@
  \hb@xt@\linewidth{%
    \vfuzz30ex %
    \vbadness\@M
    \splittopskip\z@skip
    \loop
    \ifnum\count@<\pc@columncount
      \advance\count@\@ne
      \expandafter\ifvoid\csname pc@column@\number\count@\endcsname
        \hskip\csname pc@column@width@\number\count@\endcsname
      \else
        \expandafter\setbox\expandafter\@tempboxa\expandafter
        \vsplit\csname pc@column@\number\count@\endcsname
            to \dp\strutbox
        \vbox{%
          \pcpc@SetCurrent\count@
          \unvbox\@tempboxa
        }%
      \fi
      \expandafter\ifvoid\csname pc@column@\number\count@\endcsname
      \else
        \global\let\@tempa\pc@placeboxes
      \fi
      \ifnum\count@<\pc@columncount
        \strut
        \hfill
        \ifpc@rulebetween
          \begingroup
            \pcpc@RuleBetweenColor
            \vrule
          \endgroup
          \hfill
        \fi
      \fi
    \repeat
  }%
  \@tempa
}
%    \end{macrocode}
%    \end{macro}
%    \begin{macro}{\pcpc@RuleBetweenColorDefault}
%    \begin{macrocode}
\def\pcpc@RuleBetweenColorDefault{%
  \normalcolor
}
%    \end{macrocode}
%    \end{macro}
%    \begin{macro}{\pcpc@RuleBetweenColor}
%    \begin{macrocode}
\let\pcpc@RuleBetweenColor\pcpc@RuleBetweenColorDefault
%    \end{macrocode}
%    \end{macro}
%    \begin{macrocode}
\define@key{parcolumns}{rulebetweencolor}{%
  \edef\pcpc@temp{#1}%
  \ifx\pcpc@temp\@empty
    \let\pcpc@RuleBetweenColor\pcpc@RuleBetweenColorDefault
  \else
    \edef\pcpc@temp{%
      \noexpand\@ifnextchar[{%
        \def\noexpand\pcpc@RuleBetweenColor{%
          \noexpand\color\pcpc@temp
        }%
        \noexpand\pcpc@GobbleNil
      }{%
        \def\noexpand\pcpc@RuleBetweenColor{%
          \noexpand\color{\pcpc@temp}%
        }%
        \noexpand\pcpc@GobbleNil
      }%
      \pcpc@temp\noexpand\@nil
    }%
    \pcpc@temp
  \fi
}
%    \end{macrocode}
%    \begin{macro}{\pcpc@GobbleNil}
%    \begin{macrocode}
\long\def\pcpc@GobbleNil#1\@nil{}
%    \end{macrocode}
%    \end{macro}
%
%    \begin{macrocode}
%</package>
%    \end{macrocode}
%% \section{Installation\\安装}
%
% \subsection{Download\\下载}
%
% \paragraph{Package.} This package is available on
% CTAN\footnote{\CTANpkg{pdfcolparcolumns}}:

%该软件包可从 CTAN\footnote{\CTANpkg{pdfcolparcolumns}} 下载:
% \begin{description}
% \item[\CTAN{macros/latex/contrib/oberdiek/pdfcolparcolumns.dtx}] The source file.
% \item[\CTAN{macros/latex/contrib/oberdiek/pdfcolparcolumns.pdf}] Documentation.
% \end{description}
%
%
% \paragraph{Bundle.} All the packages of the bundle `oberdiek'
% are also available in a TDS compliant ZIP archive. There
% the packages are already unpacked and the documentation files
% are generated. The files and directories obey the TDS standard.
%
%捆绑包“oberdiek”中的所有软件包也可在符合 TDS 标准的 ZIP 存档中获取。在该存档中,软件包已经被解包,文档文件已经生成,文件和目录符合 TDS 标准。
% \begin{description}
% \item[\CTANinstall{install/macros/latex/contrib/oberdiek.tds.zip}]
% \end{description}
% \emph{TDS} refers to the standard ``A Directory Structure
% for \TeX\ Files'' (\CTANpkg{tds}). Directories
% with \xfile{texmf} in their name are usually organized this way.
%
%\emph{TDS} 指的是标准“用于 \TeX\ 文件的目录结构”(\CTANpkg{tds})。名字中包含\xfile{texmf}的目录通常都是按这种方式组织的。
% \subsection{Bundle installation\\捆绑包安装}
%
% \paragraph{Unpacking.} Unpack the \xfile{oberdiek.tds.zip} in the
% TDS tree (also known as \xfile{texmf} tree) of your choice.
% Example (linux):
%
%\paragraph{解包。}在您选择的 TDS 树(也称为\xfile{texmf}树)中解压\xfile{oberdiek.tds.zip}。例如(在Linux中):
% \begin{quote}
%   |unzip oberdiek.tds.zip -d ~/texmf|
% \end{quote}
%
% \subsection{Package installation\\软件包安装}
%
% \paragraph{Unpacking.} The \xfile{.dtx} file is a self-extracting
% \docstrip\ archive. The files are extracted by running the
% \xfile{.dtx} through \plainTeX:
%
%\paragraph{解包。} \xfile{.dtx} 文件是一个自解压的 \docstrip\ 存档。运行\xfile{.dtx}通过\plainTeX\ 来提取文件:
% \begin{quote}
%   \verb|tex pdfcolparcolumns.dtx|
% \end{quote}
%
% \paragraph{TDS.} Now the different files must be moved into
% the different directories in your installation TDS tree
% (also known as \xfile{texmf} tree):
%
%\paragraph{TDS。}现在,不同的文件必须移动到安装 TDS 树(也称为\xfile{texmf}树)中的不同目录中:
% \begin{quote}
% \def\t{^^A
% \begin{tabular}{@{}>{\ttfamily}l@{ $\rightarrow$ }>{\ttfamily}l@{}}
%   pdfcolparcolumns.sty & tex/latex/oberdiek/pdfcolparcolumns.sty\\
%   pdfcolparcolumns.pdf & doc/latex/oberdiek/pdfcolparcolumns.pdf\\
%   pdfcolparcolumns.dtx & source/latex/oberdiek/pdfcolparcolumns.dtx\\
% \end{tabular}^^A
% }^^A
% \sbox0{\t}^^A
% \ifdim\wd0>\linewidth
%   \begingroup
%     \advance\linewidth by\leftmargin
%     \advance\linewidth by\rightmargin
%   \edef\x{\endgroup
%     \def\noexpand\lw{\the\linewidth}^^A
%   }\x
%   \def\lwbox{^^A
%     \leavevmode
%     \hbox to \linewidth{^^A
%       \kern-\leftmargin\relax
%       \hss
%       \usebox0
%       \hss
%       \kern-\rightmargin\relax
%     }^^A
%   }^^A
%   \ifdim\wd0>\lw
%     \sbox0{\small\t}^^A
%     \ifdim\wd0>\linewidth
%       \ifdim\wd0>\lw
%         \sbox0{\footnotesize\t}^^A
%         \ifdim\wd0>\linewidth
%           \ifdim\wd0>\lw
%             \sbox0{\scriptsize\t}^^A
%             \ifdim\wd0>\linewidth
%               \ifdim\wd0>\lw
%                 \sbox0{\tiny\t}^^A
%                 \ifdim\wd0>\linewidth
%                   \lwbox
%                 \else
%                   \usebox0
%                 \fi
%               \else
%                 \lwbox
%               \fi
%             \else
%               \usebox0
%             \fi
%           \else
%             \lwbox
%           \fi
%         \else
%           \usebox0
%         \fi
%       \else
%         \lwbox
%       \fi
%     \else
%       \usebox0
%     \fi
%   \else
%     \lwbox
%   \fi
% \else
%   \usebox0
% \fi
% \end{quote}
% If you have a \xfile{docstrip.cfg} that configures and enables \docstrip's
% TDS installing feature, then some files can already be in the right
% place, see the documentation of \docstrip.
%
%如果你有一个\xfile{docstrip.cfg}文件配置和启用了\docstrip 的TDS安装功能,那么一些文件可能已经位于正确的位置,具体请参见\docstrip 的文档。
% \subsection{Refresh file name databases\\刷新文件名数据库}
%
% If your \TeX~distribution
% (\TeX\,Live, \mikTeX, \dots) relies on file name databases, you must refresh
% these. For example, \TeX\,Live\ users run \verb|texhash| or
% \verb|mktexlsr|.
%
%如果你的\TeX~发行版(\TeX,Live、\mikTeX,等等)依赖于文件名数据库,你必须刷新它们。例如,\TeX,Live用户运行\verb|texhash|或\verb|mktexlsr|。
% \subsection{Some details for the interested\\一些细节供感兴趣的人使用}
%
% \paragraph{Unpacking with \LaTeX.}
% The \xfile{.dtx} chooses its action depending on the format:

%\paragraph{用\LaTeX 进行解包。} \xfile{.dtx}文件根据格式选择操作:
% \begin{description}
% \item[\plainTeX:] Run \docstrip\ and extract the files.
% \item[\LaTeX:] Generate the documentation.
% \end{description}
% If you insist on using \LaTeX\ for \docstrip\ (really,
% \docstrip\ does not need \LaTeX), then inform the autodetect routine
% about your intention:

%如果你坚持要用\LaTeX 进行\docstrip (其实\docstrip 不需要\LaTeX ),那么请告知自动检测例程你的意图:

% \begin{quote}
%   \verb|latex \let\install=y% \iffalse meta-comment
%
% File: pdfcolparcolumns.dtx
% Version: 2019/12/29 v1.5
% Info: Color stacks for parcolumns
%
% Copyright (C)
%    2007, 2008, 2010 Heiko Oberdiek
%    2016-2019 Oberdiek Package Support Group
%    https://github.com/ho-tex/oberdiek/issues
%
% This work may be distributed and/or modified under the
% conditions of the LaTeX Project Public License, either
% version 1.3c of this license or (at your option) any later
% version. This version of this license is in
%    https://www.latex-project.org/lppl/lppl-1-3c.txt
% and the latest version of this license is in
%    https://www.latex-project.org/lppl.txt
% and version 1.3 or later is part of all distributions of
% LaTeX version 2005/12/01 or later.
%
% This work has the LPPL maintenance status "maintained".
%
% The Current Maintainers of this work are
% Heiko Oberdiek and the Oberdiek Package Support Group
% https://github.com/ho-tex/oberdiek/issues
%
% This work consists of the main source file pdfcolparcolumns.dtx
% and the derived files
%    pdfcolparcolumns.sty, pdfcolparcolumns.pdf, pdfcolparcolumns.ins,
%    pdfcolparcolumns.drv, pdfcolparcolumns-test1.tex.
%
% Distribution:
%    CTAN:macros/latex/contrib/oberdiek/pdfcolparcolumns.dtx
%    CTAN:macros/latex/contrib/oberdiek/pdfcolparcolumns.pdf
%
% Unpacking:
%    (a) If pdfcolparcolumns.ins is present:
%           tex pdfcolparcolumns.ins
%    (b) Without pdfcolparcolumns.ins:
%           tex pdfcolparcolumns.dtx
%    (c) If you insist on using LaTeX
%           latex \let\install=y\input{pdfcolparcolumns.dtx}
%        (quote the arguments according to the demands of your shell)
%
% Documentation:
%    (a) If pdfcolparcolumns.drv is present:
%           latex pdfcolparcolumns.drv
%    (b) Without pdfcolparcolumns.drv:
%           latex pdfcolparcolumns.dtx; ...
%    The class ltxdoc loads the configuration file ltxdoc.cfg
%    if available. Here you can specify further options, e.g.
%    use A4 as paper format:
%       \PassOptionsToClass{a4paper}{article}
%
%    Programm calls to get the documentation (example):
%       pdflatex pdfcolparcolumns.dtx
%       makeindex -s gind.ist pdfcolparcolumns.idx
%       pdflatex pdfcolparcolumns.dtx
%       makeindex -s gind.ist pdfcolparcolumns.idx
%       pdflatex pdfcolparcolumns.dtx
%
% Installation:
%    TDS:tex/latex/oberdiek/pdfcolparcolumns.sty
%    TDS:doc/latex/oberdiek/pdfcolparcolumns.pdf
%    TDS:source/latex/oberdiek/pdfcolparcolumns.dtx
%
%<*ignore>
\begingroup
  \catcode123=1 %
  \catcode125=2 %
  \def\x{LaTeX2e}%
\expandafter\endgroup
\ifcase 0\ifx\install y1\fi\expandafter
         \ifx\csname processbatchFile\endcsname\relax\else1\fi
         \ifx\fmtname\x\else 1\fi\relax
\else\csname fi\endcsname
%</ignore>
%<*install>
\input docstrip.tex
\Msg{************************************************************************}
\Msg{* Installation}
\Msg{* Package: pdfcolparcolumns 2019/12/29 v1.5 Color stacks for parcolumns (HO)}
\Msg{************************************************************************}

\keepsilent
\askforoverwritefalse

\let\MetaPrefix\relax
\preamble

This is a generated file.

Project: pdfcolparcolumns
Version: 2019/12/29 v1.5

Copyright (C)
   2007, 2008, 2010 Heiko Oberdiek
   2016-2019 Oberdiek Package Support Group

This work may be distributed and/or modified under the
conditions of the LaTeX Project Public License, either
version 1.3c of this license or (at your option) any later
version. This version of this license is in
   https://www.latex-project.org/lppl/lppl-1-3c.txt
and the latest version of this license is in
   https://www.latex-project.org/lppl.txt
and version 1.3 or later is part of all distributions of
LaTeX version 2005/12/01 or later.

This work has the LPPL maintenance status "maintained".

The Current Maintainers of this work are
Heiko Oberdiek and the Oberdiek Package Support Group
https://github.com/ho-tex/oberdiek/issues


This work consists of the main source file pdfcolparcolumns.dtx
and the derived files
   pdfcolparcolumns.sty, pdfcolparcolumns.pdf, pdfcolparcolumns.ins,
   pdfcolparcolumns.drv, pdfcolparcolumns-test1.tex.

\endpreamble
\let\MetaPrefix\DoubleperCent

\generate{%
  \file{pdfcolparcolumns.ins}{\from{pdfcolparcolumns.dtx}{install}}%
  \file{pdfcolparcolumns.drv}{\from{pdfcolparcolumns.dtx}{driver}}%
  \usedir{tex/latex/oberdiek}%
  \file{pdfcolparcolumns.sty}{\from{pdfcolparcolumns.dtx}{package}}%
%  \usedir{doc/latex/oberdiek/test}%
%  \file{pdfcolparcolumns-test1.tex}{\from{pdfcolparcolumns.dtx}{test1}}%
}

\catcode32=13\relax% active space
\let =\space%
\Msg{************************************************************************}
\Msg{*}
\Msg{* To finish the installation you have to move the following}
\Msg{* file into a directory searched by TeX:}
\Msg{*}
\Msg{*     pdfcolparcolumns.sty}
\Msg{*}
\Msg{* To produce the documentation run the file `pdfcolparcolumns.drv'}
\Msg{* through LaTeX.}
\Msg{*}
\Msg{* Happy TeXing!}
\Msg{*}
\Msg{************************************************************************}

\endbatchfile
%</install>
%<*ignore>
\fi
%</ignore>
%<*driver>
\NeedsTeXFormat{LaTeX2e}
\ProvidesFile{pdfcolparcolumns.drv}%
  [2019/12/29 v1.5 Color stacks for parcolumns (HO)]%
\documentclass{ltxdoc}
\usepackage{holtxdoc}[2011/11/22]
\usepackage[scheme=plain]{ctex}
\setCJKmainfont{方正书宋_GBK}%方正书宋_GBK.TTF  设置文本的中文有衬线字体为“方正书宋_GBK”
\setCJKsansfont{方正黑体简体}%方正黑体_GBK.TTF  设置文本的中文无衬线字体为“方正黑体简体”
\setCJKmonofont{方正书宋简体}%方正仿宋_GBK.TTF  设置文本的中文等宽字体为“方正书宋简体”
\begin{document}
  \DocInput{pdfcolparcolumns.dtx}%
\end{document}
%</driver>
% \fi
%
%
%
% \GetFileInfo{pdfcolparcolumns.drv}
%
% \title{The \xpackage{pdfcolparcolumns} package}
% \date{2019/12/29 v1.5}
% \author{Heiko Oberdiek\thanks
% {Please report any issues at \url{https://github.com/ho-tex/oberdiek/issues}}\and 翻译\\{\tt virhuiai@qq.com}}
%
% \maketitle
%
% \begin{abstract}
%\makebox[0pt]{}
% \begin{parcolumns}[nofirstindent]{2}
%\colchunk{Since version 1.40 \pdfTeX\ supports several color stacks.
%This package uses them to fix color problems in
%package \xpackage{parcolumns}.}
%\colchunk{自从版本1.40起,\pdfTeX 支持多个颜色栈。
%此包使用它们来解决 \xpackage{parcolumns} 包中的颜色问题。}
%\end{parcolumns}



% \end{abstract}
%
% \tableofcontents
%
% \section{Usage\\用法}
%
% \begin{quote}
% |\usepackage{pdfcolparcolumns}|
% \end{quote}
% The package \xpackage{pdfcolparcolumns} loads package \xpackage{parcolums}
% \cite{parcolumns}. If color stacks are available then the
% macros of \xpackage{parcolumns} are patched to add support
% for color stacks.

%包 \xpackage{pdfcolparcolumns} 载入了 \xpackage{parcolumns} \cite{parcolumns} 包。如果有可用的颜色栈,则会对 \xpackage{parcolumns} 的宏进行修补,以添加对颜色栈的支持。

%
% \subsection{Option \xoption{rulebetweencolor}\\选项 \xoption{rulebetweencolor}}
%
% Package \xpackage{pdfcolparcolumns} also fixes the color for the
% rule between columns (if \xoption{rulebetween} is set).
% Default color is \cs{normalcolor}. But this can be changed by using
% option \xoption{rulebetweencolor}. It takes a color specification
% as value. If the value is empty, then the default (\cs{normalcolor})
% is used.
% Examples:

% 包 \xpackage{pdfcolparcolumns} 还修复了列之间分隔线的颜色(如果设置了 \xoption{rulebetween})。默认颜色为 \cs{normalcolor}。但是,可以使用选项 \xoption{rulebetweencolor} 来更改它。它接受一个颜色规范作为值。如果该值为空,则使用默认值(\cs{normalcolor})。例如:

% \begin{quote}
%   |rulebetweencolor=blue|,\\
%   |rulebetweencolor={red}|,\\
%   |rulebetweencolor={}|, \textit{\% \cs{normalcolor} is used}\\
%   |rulebetweencolor=[rgb]{1,0,.5}| \textit{\% see below}
% \end{quote}
% If used inside the optional argument of environment \textsf{parcolumns}
% and the value contains an optional argument, then whole value
% must be put in curly braces:

% 如果在 \textsf{parcolumns} 环境的可选参数中使用,并且值包含可选参数,则整个值必须用花括号括起来:
%\begin{quote}
%\begin{verbatim}
%\begin{parcolumns}[
%  rulebetween,
%  rulebetweencolor={[rgb]{1,0,.5}},
%]{2}
%  ...
%\end{parcolumns}
%\end{verbatim}
%\end{quote}
% This option \xoption{rulebetweencolor} can also be set using
% \cs{setkeys}:

%也可以使用 \cs{setkeys} 来设置选项 \xoption{rulebetweencolor}:
%\begin{quote}
%\begin{verbatim}
%\setkeys{parcolumns}{rulebetweencolor=green}
%\end{verbatim}
%\end{quote}
%
% \subsection{Future\\未来展望}
%
% Currently package \xpackage{parcolumns} does not seem to be
% maintained. Nevertheless if there will be a new version that
% adds support for color stacks, then this package may become
% obsolete.
%
%目前,\xpackage{parcolumns} 包似乎没有维护了。尽管如此,如果出现了新版本并支持颜色栈,那么这个包可能会变得过时。
% \StopEventually{
% }
%
% \section{Implementation\\实现}
%
% \subsection{Identification\\标识}
%
%    \begin{macrocode}
%<*package>
\NeedsTeXFormat{LaTeX2e}
\ProvidesPackage{pdfcolparcolumns}%
  [2019/12/29 v1.5 Color stacks for parcolumns (HO)]%
%    \end{macrocode}
%
% \subsection{Load packages\\加载宏包}
%
% \subsubsection{Package \xpackage{parcolumns}\\宏包 \xpackage{parcolumns}}
%
%    Currently package \xpackage{parcolumns} does not define options.
%    Thus it is just a precaution that the options of
%    package \xpackage{pdfcolparcolumns} are passed to
%    package \xpackage{parcolumns}.

%目前,宏包 \xpackage{parcolumns} 没有定义选项。因此,它只是一个预防措施,以确保将宏包 \xpackage{pdfcolparcolumns} 的选项传递给宏包 \xpackage{parcolumns}。
%    \begin{macrocode}
\DeclareOption*{%
  \PassOptionsToPackage{\CurrentOption}{parcolumns}%
}
\ProcessOptions\relax
\RequirePackage{parcolumns}[2004/11/25]
%    \end{macrocode}
%
% \subsubsection{Package \xpackage{pdfcol}\\宏包 \xpackage{pdfcol}}
%
%    \begin{macrocode}
\RequirePackage{pdfcol}[2007/09/09]
\ifpdfcolAvailable
\else
  \PackageInfo{pdfcolparcolumns}{%
    Loading aborted, because color stacks are not available%
  }%
  \expandafter\endinput
\fi
%    \end{macrocode}
%
% \subsubsection{Package \xpackage{infwarerr}\\宏包 \xpackage{infwarerr}}
%
%    \begin{macrocode}
\RequirePackage{infwarerr}[2007/09/09]
%    \end{macrocode}
%
% \subsection{Color stack macros\\颜色堆栈宏}
%
%    \begin{macro}{\pcpc@MaxStack}
%    Macro \cs{pcpc@MaxStack} holds the highest number of
%    allocated stacks.

%宏 \cs{pcpc@MaxStack} 存储已分配的堆栈中最大的编号。
%    \begin{macrocode}
\global\chardef\pcpc@MaxStack=\z@
%    \end{macrocode}
%    \end{macro}
%    \begin{macro}{\pcpc@InitStacks}
%    Macro \cs{pcpc@InitStacks} takes the number of columns
%    as argument and ensures that there are enough color
%    stacks for all columns.
%
%宏 \cs{pcpc@InitStacks} 接受列数作为参数,并确保为所有列分配了足够的颜色堆栈。
%    \begin{macrocode}
\def\pcpc@InitStacks#1{%
  \ifnum#1>\pcpc@MaxStack
    \begingroup
      \count@\pcpc@MaxStack
      \loop
        \advance\count@\@ne
        \pdfcolInitStack{pcpc@\the\count@}%
      \ifnum#1>\count@
      \repeat
      \global\chardef\pcpc@MaxStack=\count@
    \endgroup
  \fi
}
%    \end{macrocode}
%    \end{macro}
%
%    \begin{macro}{\pcpc@SwitchStack}
%    \begin{macrocode}
\def\pcpc@SwitchStack#1{%
  \pdfcolSwitchStack{pcpc@\number#1}%
}
%    \end{macrocode}
%    \end{macro}
%
%    \begin{macro}{\pcpc@SetCurrent}
%    \begin{macrocode}
\def\pcpc@SetCurrent#1{%
  \pdfcolSetCurrent{pcpc@\number#1}%
}
%    \end{macrocode}
%    \end{macro}
%
% \subsection{Patches\\补丁}
%
%     Now the color stack macros are patched into the macros
%     of package \xpackage{parcolumns}.
%
%现在颜色堆栈宏已经被打补丁到了 \xpackage{parcolumns} 宏包的宏中。
% \subsubsection{Init stacks\\初始化堆栈}
%
%    \cs{pcpc@InitStacks} should go into the definition of
%    environment |parcolumns|. \cs{pc@alloccolumns} is executed
%    there and nowhere else, thus we hook into it.

%\cs{pcpc@InitStacks} 应该被放在环境 |parcolumns| 的定义中。\cs{pc@alloccolumns} 仅在那里执行,因此我们将其连接起来。
%    \begin{macrocode}
\g@addto@macro\pc@alloccolumns{%
  \pcpc@InitStacks\pc@columncount
}
%    \end{macrocode}
%
% \subsubsection{Switch stack\\切换堆栈}
%
%    \cs{pcpc@SwitchStack} should be called by marco \cs{colchunk@}.
%    However it is easier to patch \cs{pc@setcolumnwidth} that
%    is executed in \cs{colchunk@} only.
%
%\cs{pcpc@SwitchStack} 应该被宏 \cs{colchunk@} 调用。不过,我们更容易修补仅在 \cs{colchunk@} 中执行的 \cs{pc@setcolumnwidth}。
%    \begin{macrocode}
\g@addto@macro\pc@setcolumnwidth{%
  \pcpc@SwitchStack\pc@columnctr
}
%    \end{macrocode}
%
% \subsubsection{Set current stack color\\设置当前堆栈颜色}
%
%    \cs{pcpc@SetCurrent} is set at the begin of each line.
%    It must be inserted into \cs{pc@placeboxes}. Unhappily
%    there is no easy way. Therefore we check and
%    redefine \cs{pc@placeboxes}.
%
%\cs{pcpc@SetCurrent} 在每行开始时设置。它必须插入到 \cs{pc@placeboxes} 中。不幸的是,没有简单的方法。因此,我们检查并重新定义 \cs{pc@placeboxes}。
%    \begin{macrocode}
\begingroup
  \def\x{%
    \global\let\@tempa\relax
    \count@\z@
    \hb@xt@\linewidth{%
      \vfuzz30ex %
      \vbadness\@M
      \splittopskip\z@skip
      \loop
      \ifnum\count@<\pc@columncount
        \advance\count@\@ne
        \expandafter\ifvoid\csname pc@column@\number\count@\endcsname
          \hskip\csname pc@column@width@\number\count@\endcsname
        \else
          \expandafter\setbox\expandafter\@tempboxa\expandafter
          \vsplit\csname pc@column@\number\count@\endcsname
              to \dp\strutbox
          \vbox{%
            \unvbox\@tempboxa
          }%
        \fi
        \expandafter\ifvoid\csname pc@column@\number\count@\endcsname
        \else
          \global\let\@tempa\pc@placeboxes
        \fi
        \ifnum\count@<\pc@columncount
          \strut
          \hfill
          \ifpc@rulebetween
            \vrule
            \hfill
          \fi
        \fi
      \repeat
    }%
    \@tempa
  }%
  \ifx\x\pc@placeboxes
  \else
    \@PackageWarningNoLine{pdfcolparcolumns}{%
      Command \string\pc@placeboxes\space has changed.\MessageBreak
      Supported versions of package `parcolumns':\MessageBreak
      \space\space 2004/08/05.\MessageBreak
      The redefinition of \string\pc@placeboxes\space may not%
      \MessageBreak
      behave correctly depending on the changes%
    }%
  \fi
\endgroup
%    \end{macrocode}
%    \begin{macro}{\pc@placeboxes}
%    \begin{macrocode}
\renewcommand*{\pc@placeboxes}{%
  \global\let\@tempa\relax
  \count@\z@
  \hb@xt@\linewidth{%
    \vfuzz30ex %
    \vbadness\@M
    \splittopskip\z@skip
    \loop
    \ifnum\count@<\pc@columncount
      \advance\count@\@ne
      \expandafter\ifvoid\csname pc@column@\number\count@\endcsname
        \hskip\csname pc@column@width@\number\count@\endcsname
      \else
        \expandafter\setbox\expandafter\@tempboxa\expandafter
        \vsplit\csname pc@column@\number\count@\endcsname
            to \dp\strutbox
        \vbox{%
          \pcpc@SetCurrent\count@
          \unvbox\@tempboxa
        }%
      \fi
      \expandafter\ifvoid\csname pc@column@\number\count@\endcsname
      \else
        \global\let\@tempa\pc@placeboxes
      \fi
      \ifnum\count@<\pc@columncount
        \strut
        \hfill
        \ifpc@rulebetween
          \begingroup
            \pcpc@RuleBetweenColor
            \vrule
          \endgroup
          \hfill
        \fi
      \fi
    \repeat
  }%
  \@tempa
}
%    \end{macrocode}
%    \end{macro}
%    \begin{macro}{\pcpc@RuleBetweenColorDefault}
%    \begin{macrocode}
\def\pcpc@RuleBetweenColorDefault{%
  \normalcolor
}
%    \end{macrocode}
%    \end{macro}
%    \begin{macro}{\pcpc@RuleBetweenColor}
%    \begin{macrocode}
\let\pcpc@RuleBetweenColor\pcpc@RuleBetweenColorDefault
%    \end{macrocode}
%    \end{macro}
%    \begin{macrocode}
\define@key{parcolumns}{rulebetweencolor}{%
  \edef\pcpc@temp{#1}%
  \ifx\pcpc@temp\@empty
    \let\pcpc@RuleBetweenColor\pcpc@RuleBetweenColorDefault
  \else
    \edef\pcpc@temp{%
      \noexpand\@ifnextchar[{%
        \def\noexpand\pcpc@RuleBetweenColor{%
          \noexpand\color\pcpc@temp
        }%
        \noexpand\pcpc@GobbleNil
      }{%
        \def\noexpand\pcpc@RuleBetweenColor{%
          \noexpand\color{\pcpc@temp}%
        }%
        \noexpand\pcpc@GobbleNil
      }%
      \pcpc@temp\noexpand\@nil
    }%
    \pcpc@temp
  \fi
}
%    \end{macrocode}
%    \begin{macro}{\pcpc@GobbleNil}
%    \begin{macrocode}
\long\def\pcpc@GobbleNil#1\@nil{}
%    \end{macrocode}
%    \end{macro}
%
%    \begin{macrocode}
%</package>
%    \end{macrocode}
%% \section{Installation\\安装}
%
% \subsection{Download\\下载}
%
% \paragraph{Package.} This package is available on
% CTAN\footnote{\CTANpkg{pdfcolparcolumns}}:

%该软件包可从 CTAN\footnote{\CTANpkg{pdfcolparcolumns}} 下载:
% \begin{description}
% \item[\CTAN{macros/latex/contrib/oberdiek/pdfcolparcolumns.dtx}] The source file.
% \item[\CTAN{macros/latex/contrib/oberdiek/pdfcolparcolumns.pdf}] Documentation.
% \end{description}
%
%
% \paragraph{Bundle.} All the packages of the bundle `oberdiek'
% are also available in a TDS compliant ZIP archive. There
% the packages are already unpacked and the documentation files
% are generated. The files and directories obey the TDS standard.
%
%捆绑包“oberdiek”中的所有软件包也可在符合 TDS 标准的 ZIP 存档中获取。在该存档中,软件包已经被解包,文档文件已经生成,文件和目录符合 TDS 标准。
% \begin{description}
% \item[\CTANinstall{install/macros/latex/contrib/oberdiek.tds.zip}]
% \end{description}
% \emph{TDS} refers to the standard ``A Directory Structure
% for \TeX\ Files'' (\CTANpkg{tds}). Directories
% with \xfile{texmf} in their name are usually organized this way.
%
%\emph{TDS} 指的是标准“用于 \TeX\ 文件的目录结构”(\CTANpkg{tds})。名字中包含\xfile{texmf}的目录通常都是按这种方式组织的。
% \subsection{Bundle installation\\捆绑包安装}
%
% \paragraph{Unpacking.} Unpack the \xfile{oberdiek.tds.zip} in the
% TDS tree (also known as \xfile{texmf} tree) of your choice.
% Example (linux):
%
%\paragraph{解包。}在您选择的 TDS 树(也称为\xfile{texmf}树)中解压\xfile{oberdiek.tds.zip}。例如(在Linux中):
% \begin{quote}
%   |unzip oberdiek.tds.zip -d ~/texmf|
% \end{quote}
%
% \subsection{Package installation\\软件包安装}
%
% \paragraph{Unpacking.} The \xfile{.dtx} file is a self-extracting
% \docstrip\ archive. The files are extracted by running the
% \xfile{.dtx} through \plainTeX:
%
%\paragraph{解包。} \xfile{.dtx} 文件是一个自解压的 \docstrip\ 存档。运行\xfile{.dtx}通过\plainTeX\ 来提取文件:
% \begin{quote}
%   \verb|tex pdfcolparcolumns.dtx|
% \end{quote}
%
% \paragraph{TDS.} Now the different files must be moved into
% the different directories in your installation TDS tree
% (also known as \xfile{texmf} tree):
%
%\paragraph{TDS。}现在,不同的文件必须移动到安装 TDS 树(也称为\xfile{texmf}树)中的不同目录中:
% \begin{quote}
% \def\t{^^A
% \begin{tabular}{@{}>{\ttfamily}l@{ $\rightarrow$ }>{\ttfamily}l@{}}
%   pdfcolparcolumns.sty & tex/latex/oberdiek/pdfcolparcolumns.sty\\
%   pdfcolparcolumns.pdf & doc/latex/oberdiek/pdfcolparcolumns.pdf\\
%   pdfcolparcolumns.dtx & source/latex/oberdiek/pdfcolparcolumns.dtx\\
% \end{tabular}^^A
% }^^A
% \sbox0{\t}^^A
% \ifdim\wd0>\linewidth
%   \begingroup
%     \advance\linewidth by\leftmargin
%     \advance\linewidth by\rightmargin
%   \edef\x{\endgroup
%     \def\noexpand\lw{\the\linewidth}^^A
%   }\x
%   \def\lwbox{^^A
%     \leavevmode
%     \hbox to \linewidth{^^A
%       \kern-\leftmargin\relax
%       \hss
%       \usebox0
%       \hss
%       \kern-\rightmargin\relax
%     }^^A
%   }^^A
%   \ifdim\wd0>\lw
%     \sbox0{\small\t}^^A
%     \ifdim\wd0>\linewidth
%       \ifdim\wd0>\lw
%         \sbox0{\footnotesize\t}^^A
%         \ifdim\wd0>\linewidth
%           \ifdim\wd0>\lw
%             \sbox0{\scriptsize\t}^^A
%             \ifdim\wd0>\linewidth
%               \ifdim\wd0>\lw
%                 \sbox0{\tiny\t}^^A
%                 \ifdim\wd0>\linewidth
%                   \lwbox
%                 \else
%                   \usebox0
%                 \fi
%               \else
%                 \lwbox
%               \fi
%             \else
%               \usebox0
%             \fi
%           \else
%             \lwbox
%           \fi
%         \else
%           \usebox0
%         \fi
%       \else
%         \lwbox
%       \fi
%     \else
%       \usebox0
%     \fi
%   \else
%     \lwbox
%   \fi
% \else
%   \usebox0
% \fi
% \end{quote}
% If you have a \xfile{docstrip.cfg} that configures and enables \docstrip's
% TDS installing feature, then some files can already be in the right
% place, see the documentation of \docstrip.
%
%如果你有一个\xfile{docstrip.cfg}文件配置和启用了\docstrip 的TDS安装功能,那么一些文件可能已经位于正确的位置,具体请参见\docstrip 的文档。
% \subsection{Refresh file name databases\\刷新文件名数据库}
%
% If your \TeX~distribution
% (\TeX\,Live, \mikTeX, \dots) relies on file name databases, you must refresh
% these. For example, \TeX\,Live\ users run \verb|texhash| or
% \verb|mktexlsr|.
%
%如果你的\TeX~发行版(\TeX,Live、\mikTeX,等等)依赖于文件名数据库,你必须刷新它们。例如,\TeX,Live用户运行\verb|texhash|或\verb|mktexlsr|。
% \subsection{Some details for the interested\\一些细节供感兴趣的人使用}
%
% \paragraph{Unpacking with \LaTeX.}
% The \xfile{.dtx} chooses its action depending on the format:

%\paragraph{用\LaTeX 进行解包。} \xfile{.dtx}文件根据格式选择操作:
% \begin{description}
% \item[\plainTeX:] Run \docstrip\ and extract the files.
% \item[\LaTeX:] Generate the documentation.
% \end{description}
% If you insist on using \LaTeX\ for \docstrip\ (really,
% \docstrip\ does not need \LaTeX), then inform the autodetect routine
% about your intention:

%如果你坚持要用\LaTeX 进行\docstrip (其实\docstrip 不需要\LaTeX ),那么请告知自动检测例程你的意图:

% \begin{quote}
%   \verb|latex \let\install=y\input{pdfcolparcolumns.dtx}|
% \end{quote}
% Do not forget to quote the argument according to the demands
% of your shell.

%不要忘记根据你的shell的要求引用参数。
%
% \paragraph{Generating the documentation.}
% You can use both the \xfile{.dtx} or the \xfile{.drv} to generate
% the documentation. The process can be configured by the
% configuration file \xfile{ltxdoc.cfg}. For instance, put this
% line into this file, if you want to have A4 as paper format:
%
%\paragraph{生成文档。} 你可以使用\xfile{.dtx}或\xfile{.drv}生成文档。这个过程可以由配置文件\xfile{ltxdoc.cfg}配置。例如,如果你想要A4作为纸张格式,将此行放入文件中:

% \begin{quote}
%   \verb|\PassOptionsToClass{a4paper}{article}|
% \end{quote}
% An example follows how to generate the
% documentation with pdf\LaTeX:

%以下是使用pdf\LaTeX 生成文档的示例:
% \begin{quote}
%\begin{verbatim}
%pdflatex pdfcolparcolumns.dtx
%makeindex -s gind.ist pdfcolparcolumns.idx
%pdflatex pdfcolparcolumns.dtx
%makeindex -s gind.ist pdfcolparcolumns.idx
%pdflatex pdfcolparcolumns.dtx
%\end{verbatim}
% \end{quote}
%
% \begin{thebibliography}{9}
%
% \bibitem{parcolumns}
%   Jonathan Sauer: \textit{The \xpackage{parcolumns} package};
%   2004/11/25;\\
%   \CTANpkg{parcolumns}.
%
% \bibitem{pdfcol}
%   Heiko Oberdiek: \textit{The \xpackage{pdfcol} package};
%   2007/09/09;\\
%   \CTANpkg{pdfcol}.
%
% \end{thebibliography}
%
% \begin{History}
%   \begin{Version}{2007/07/26 v1.0}
%   \item
%     First version, published in the newsgroup \xnewsgroup{comp.text.tex}
%     with the name \xpackage{parcolumns-colorstacks}: ^^A no line break
%     \URL{``\link{Re: \xpackage{xcolor} glitches}''}^^A
%     {https://groups.google.com/group/comp.text.tex/msg/56bd897b11bca414}

%第一个版本,以 \xnewsgroup{comp.text.tex} 新闻组发布,名称为 \xpackage{parcolumns-colorstacks}:^^A 没有换行符 
%\URL{``\link{Re: \xpackage{xcolor} glitches}''}^^A
%{https://groups.google.com/group/comp.text.tex/msg/56bd897b11bca414}
%   \end{Version}
%   \begin{Version}{2007/09/09 v1.1}
%   \item
%     CTAN version, package name renamed to \xpackage{pdfcolparcolumns}.
%
%CTAN 版本,将包名改为 \xpackage{pdfcolparcolumns}。
%   \item
%     Uses package \xpackage{pdfcol}.
%
%使用 \xpackage{pdfcol} 包。
%   \item
%     Documentation added.
%
%添加文档。
%   \item
%     Test file added.
%
%添加测试文件。
%   \end{Version}
%   \begin{Version}{2008/08/11 v1.2}
%   \item
%     Code is not changed.
%
%代码未改动。
%   \item
%     URLs updated.
%
%更新 URL。
%   \end{Version}
%   \begin{Version}{2010/01/11 v1.3}
%   \item
%     Fix for rule color.
%
%修复分隔线的颜色。
%   \item
%     New option \xoption{rulebetweencolor} for environment |parcolumns|.
%
%为环境 |parcolumns| 添加新选项 \xoption{rulebetweencolor}。
%   \end{Version}
%   \begin{Version}{2016/05/16 v1.4}
%   \item
%     Documentation updates.
%
%更新文档。
%   \end{Version}
%   \begin{Version}{2019/12/29 v1.5}
%   \item
%     \cs{PassOptionsToPackage} not \cs{PassoptionsToPackage}
%
%\cs{PassOptionsToPackage} 改为不区分大小写的 \cs{PassoptionsToPackage}。
%   \end{Version}
% \end{History}
%
% \PrintIndex
%
% \Finale
\endinput
|
% \end{quote}
% Do not forget to quote the argument according to the demands
% of your shell.

%不要忘记根据你的shell的要求引用参数。
%
% \paragraph{Generating the documentation.}
% You can use both the \xfile{.dtx} or the \xfile{.drv} to generate
% the documentation. The process can be configured by the
% configuration file \xfile{ltxdoc.cfg}. For instance, put this
% line into this file, if you want to have A4 as paper format:
%
%\paragraph{生成文档。} 你可以使用\xfile{.dtx}或\xfile{.drv}生成文档。这个过程可以由配置文件\xfile{ltxdoc.cfg}配置。例如,如果你想要A4作为纸张格式,将此行放入文件中:

% \begin{quote}
%   \verb|\PassOptionsToClass{a4paper}{article}|
% \end{quote}
% An example follows how to generate the
% documentation with pdf\LaTeX:

%以下是使用pdf\LaTeX 生成文档的示例:
% \begin{quote}
%\begin{verbatim}
%pdflatex pdfcolparcolumns.dtx
%makeindex -s gind.ist pdfcolparcolumns.idx
%pdflatex pdfcolparcolumns.dtx
%makeindex -s gind.ist pdfcolparcolumns.idx
%pdflatex pdfcolparcolumns.dtx
%\end{verbatim}
% \end{quote}
%
% \begin{thebibliography}{9}
%
% \bibitem{parcolumns}
%   Jonathan Sauer: \textit{The \xpackage{parcolumns} package};
%   2004/11/25;\\
%   \CTANpkg{parcolumns}.
%
% \bibitem{pdfcol}
%   Heiko Oberdiek: \textit{The \xpackage{pdfcol} package};
%   2007/09/09;\\
%   \CTANpkg{pdfcol}.
%
% \end{thebibliography}
%
% \begin{History}
%   \begin{Version}{2007/07/26 v1.0}
%   \item
%     First version, published in the newsgroup \xnewsgroup{comp.text.tex}
%     with the name \xpackage{parcolumns-colorstacks}: ^^A no line break
%     \URL{``\link{Re: \xpackage{xcolor} glitches}''}^^A
%     {https://groups.google.com/group/comp.text.tex/msg/56bd897b11bca414}

%第一个版本,以 \xnewsgroup{comp.text.tex} 新闻组发布,名称为 \xpackage{parcolumns-colorstacks}:^^A 没有换行符 
%\URL{``\link{Re: \xpackage{xcolor} glitches}''}^^A
%{https://groups.google.com/group/comp.text.tex/msg/56bd897b11bca414}
%   \end{Version}
%   \begin{Version}{2007/09/09 v1.1}
%   \item
%     CTAN version, package name renamed to \xpackage{pdfcolparcolumns}.
%
%CTAN 版本,将包名改为 \xpackage{pdfcolparcolumns}。
%   \item
%     Uses package \xpackage{pdfcol}.
%
%使用 \xpackage{pdfcol} 包。
%   \item
%     Documentation added.
%
%添加文档。
%   \item
%     Test file added.
%
%添加测试文件。
%   \end{Version}
%   \begin{Version}{2008/08/11 v1.2}
%   \item
%     Code is not changed.
%
%代码未改动。
%   \item
%     URLs updated.
%
%更新 URL。
%   \end{Version}
%   \begin{Version}{2010/01/11 v1.3}
%   \item
%     Fix for rule color.
%
%修复分隔线的颜色。
%   \item
%     New option \xoption{rulebetweencolor} for environment |parcolumns|.
%
%为环境 |parcolumns| 添加新选项 \xoption{rulebetweencolor}。
%   \end{Version}
%   \begin{Version}{2016/05/16 v1.4}
%   \item
%     Documentation updates.
%
%更新文档。
%   \end{Version}
%   \begin{Version}{2019/12/29 v1.5}
%   \item
%     \cs{PassOptionsToPackage} not \cs{PassoptionsToPackage}
%
%\cs{PassOptionsToPackage} 改为不区分大小写的 \cs{PassoptionsToPackage}。
%   \end{Version}
% \end{History}
%
% \PrintIndex
%
% \Finale
\endinput
|
% \end{quote}
% Do not forget to quote the argument according to the demands
% of your shell.

%不要忘记根据你的shell的要求引用参数。
%
% \paragraph{Generating the documentation.}
% You can use both the \xfile{.dtx} or the \xfile{.drv} to generate
% the documentation. The process can be configured by the
% configuration file \xfile{ltxdoc.cfg}. For instance, put this
% line into this file, if you want to have A4 as paper format:
%
%\paragraph{生成文档。} 你可以使用\xfile{.dtx}或\xfile{.drv}生成文档。这个过程可以由配置文件\xfile{ltxdoc.cfg}配置。例如,如果你想要A4作为纸张格式,将此行放入文件中:

% \begin{quote}
%   \verb|\PassOptionsToClass{a4paper}{article}|
% \end{quote}
% An example follows how to generate the
% documentation with pdf\LaTeX:

%以下是使用pdf\LaTeX 生成文档的示例:
% \begin{quote}
%\begin{verbatim}
%pdflatex pdfcolparcolumns.dtx
%makeindex -s gind.ist pdfcolparcolumns.idx
%pdflatex pdfcolparcolumns.dtx
%makeindex -s gind.ist pdfcolparcolumns.idx
%pdflatex pdfcolparcolumns.dtx
%\end{verbatim}
% \end{quote}
%
% \begin{thebibliography}{9}
%
% \bibitem{parcolumns}
%   Jonathan Sauer: \textit{The \xpackage{parcolumns} package};
%   2004/11/25;\\
%   \CTANpkg{parcolumns}.
%
% \bibitem{pdfcol}
%   Heiko Oberdiek: \textit{The \xpackage{pdfcol} package};
%   2007/09/09;\\
%   \CTANpkg{pdfcol}.
%
% \end{thebibliography}
%
% \begin{History}
%   \begin{Version}{2007/07/26 v1.0}
%   \item
%     First version, published in the newsgroup \xnewsgroup{comp.text.tex}
%     with the name \xpackage{parcolumns-colorstacks}: ^^A no line break
%     \URL{``\link{Re: \xpackage{xcolor} glitches}''}^^A
%     {https://groups.google.com/group/comp.text.tex/msg/56bd897b11bca414}

%第一个版本,以 \xnewsgroup{comp.text.tex} 新闻组发布,名称为 \xpackage{parcolumns-colorstacks}:^^A 没有换行符 
%\URL{``\link{Re: \xpackage{xcolor} glitches}''}^^A
%{https://groups.google.com/group/comp.text.tex/msg/56bd897b11bca414}
%   \end{Version}
%   \begin{Version}{2007/09/09 v1.1}
%   \item
%     CTAN version, package name renamed to \xpackage{pdfcolparcolumns}.
%
%CTAN 版本,将包名改为 \xpackage{pdfcolparcolumns}。
%   \item
%     Uses package \xpackage{pdfcol}.
%
%使用 \xpackage{pdfcol} 包。
%   \item
%     Documentation added.
%
%添加文档。
%   \item
%     Test file added.
%
%添加测试文件。
%   \end{Version}
%   \begin{Version}{2008/08/11 v1.2}
%   \item
%     Code is not changed.
%
%代码未改动。
%   \item
%     URLs updated.
%
%更新 URL。
%   \end{Version}
%   \begin{Version}{2010/01/11 v1.3}
%   \item
%     Fix for rule color.
%
%修复分隔线的颜色。
%   \item
%     New option \xoption{rulebetweencolor} for environment |parcolumns|.
%
%为环境 |parcolumns| 添加新选项 \xoption{rulebetweencolor}。
%   \end{Version}
%   \begin{Version}{2016/05/16 v1.4}
%   \item
%     Documentation updates.
%
%更新文档。
%   \end{Version}
%   \begin{Version}{2019/12/29 v1.5}
%   \item
%     \cs{PassOptionsToPackage} not \cs{PassoptionsToPackage}
%
%\cs{PassOptionsToPackage} 改为不区分大小写的 \cs{PassoptionsToPackage}。
%   \end{Version}
% \end{History}
%
% \PrintIndex
%
% \Finale
\endinput

%        (quote the arguments according to the demands of your shell)
%
% Documentation:
%    (a) If pdfcolparcolumns.drv is present:
%           latex pdfcolparcolumns.drv
%    (b) Without pdfcolparcolumns.drv:
%           latex pdfcolparcolumns.dtx; ...
%    The class ltxdoc loads the configuration file ltxdoc.cfg
%    if available. Here you can specify further options, e.g.
%    use A4 as paper format:
%       \PassOptionsToClass{a4paper}{article}
%
%    Programm calls to get the documentation (example):
%       pdflatex pdfcolparcolumns.dtx
%       makeindex -s gind.ist pdfcolparcolumns.idx
%       pdflatex pdfcolparcolumns.dtx
%       makeindex -s gind.ist pdfcolparcolumns.idx
%       pdflatex pdfcolparcolumns.dtx
%
% Installation:
%    TDS:tex/latex/oberdiek/pdfcolparcolumns.sty
%    TDS:doc/latex/oberdiek/pdfcolparcolumns.pdf
%    TDS:source/latex/oberdiek/pdfcolparcolumns.dtx
%
%<*ignore>
\begingroup
  \catcode123=1 %
  \catcode125=2 %
  \def\x{LaTeX2e}%
\expandafter\endgroup
\ifcase 0\ifx\install y1\fi\expandafter
         \ifx\csname processbatchFile\endcsname\relax\else1\fi
         \ifx\fmtname\x\else 1\fi\relax
\else\csname fi\endcsname
%</ignore>
%<*install>
\input docstrip.tex
\Msg{************************************************************************}
\Msg{* Installation}
\Msg{* Package: pdfcolparcolumns 2019/12/29 v1.5 Color stacks for parcolumns (HO)}
\Msg{************************************************************************}

\keepsilent
\askforoverwritefalse

\let\MetaPrefix\relax
\preamble

This is a generated file.

Project: pdfcolparcolumns
Version: 2019/12/29 v1.5

Copyright (C)
   2007, 2008, 2010 Heiko Oberdiek
   2016-2019 Oberdiek Package Support Group

This work may be distributed and/or modified under the
conditions of the LaTeX Project Public License, either
version 1.3c of this license or (at your option) any later
version. This version of this license is in
   https://www.latex-project.org/lppl/lppl-1-3c.txt
and the latest version of this license is in
   https://www.latex-project.org/lppl.txt
and version 1.3 or later is part of all distributions of
LaTeX version 2005/12/01 or later.

This work has the LPPL maintenance status "maintained".

The Current Maintainers of this work are
Heiko Oberdiek and the Oberdiek Package Support Group
https://github.com/ho-tex/oberdiek/issues


This work consists of the main source file pdfcolparcolumns.dtx
and the derived files
   pdfcolparcolumns.sty, pdfcolparcolumns.pdf, pdfcolparcolumns.ins,
   pdfcolparcolumns.drv, pdfcolparcolumns-test1.tex.

\endpreamble
\let\MetaPrefix\DoubleperCent

\generate{%
  \file{pdfcolparcolumns.ins}{\from{pdfcolparcolumns.dtx}{install}}%
  \file{pdfcolparcolumns.drv}{\from{pdfcolparcolumns.dtx}{driver}}%
  \usedir{tex/latex/oberdiek}%
  \file{pdfcolparcolumns.sty}{\from{pdfcolparcolumns.dtx}{package}}%
%  \usedir{doc/latex/oberdiek/test}%
%  \file{pdfcolparcolumns-test1.tex}{\from{pdfcolparcolumns.dtx}{test1}}%
}

\catcode32=13\relax% active space
\let =\space%
\Msg{************************************************************************}
\Msg{*}
\Msg{* To finish the installation you have to move the following}
\Msg{* file into a directory searched by TeX:}
\Msg{*}
\Msg{*     pdfcolparcolumns.sty}
\Msg{*}
\Msg{* To produce the documentation run the file `pdfcolparcolumns.drv'}
\Msg{* through LaTeX.}
\Msg{*}
\Msg{* Happy TeXing!}
\Msg{*}
\Msg{************************************************************************}

\endbatchfile
%</install>
%<*ignore>
\fi
%</ignore>
%<*driver>
\NeedsTeXFormat{LaTeX2e}
\ProvidesFile{pdfcolparcolumns.drv}%
  [2019/12/29 v1.5 Color stacks for parcolumns (HO)]%
\documentclass{ltxdoc}
\usepackage{holtxdoc}[2011/11/22]
\usepackage[scheme=plain]{ctex}
\setCJKmainfont{方正书宋_GBK}%方正书宋_GBK.TTF  设置文本的中文有衬线字体为“方正书宋_GBK”
\setCJKsansfont{方正黑体简体}%方正黑体_GBK.TTF  设置文本的中文无衬线字体为“方正黑体简体”
\setCJKmonofont{方正书宋简体}%方正仿宋_GBK.TTF  设置文本的中文等宽字体为“方正书宋简体”
\begin{document}
  \DocInput{pdfcolparcolumns.dtx}%
\end{document}
%</driver>
% \fi
%
%
%
% \GetFileInfo{pdfcolparcolumns.drv}
%
% \title{The \xpackage{pdfcolparcolumns} package}
% \date{2019/12/29 v1.5}
% \author{Heiko Oberdiek\thanks
% {Please report any issues at \url{https://github.com/ho-tex/oberdiek/issues}}\and 翻译\\{\tt virhuiai@qq.com}}
%
% \maketitle
%
% \begin{abstract}
%\makebox[0pt]{}
% \begin{parcolumns}[nofirstindent]{2}
%\colchunk{Since version 1.40 \pdfTeX\ supports several color stacks.
%This package uses them to fix color problems in
%package \xpackage{parcolumns}.}
%\colchunk{自从版本1.40起,\pdfTeX 支持多个颜色栈。
%此包使用它们来解决 \xpackage{parcolumns} 包中的颜色问题。}
%\end{parcolumns}



% \end{abstract}
%
% \tableofcontents
%
% \section{Usage\\用法}
%
% \begin{quote}
% |\usepackage{pdfcolparcolumns}|
% \end{quote}
% The package \xpackage{pdfcolparcolumns} loads package \xpackage{parcolums}
% \cite{parcolumns}. If color stacks are available then the
% macros of \xpackage{parcolumns} are patched to add support
% for color stacks.

%包 \xpackage{pdfcolparcolumns} 载入了 \xpackage{parcolumns} \cite{parcolumns} 包。如果有可用的颜色栈,则会对 \xpackage{parcolumns} 的宏进行修补,以添加对颜色栈的支持。

%
% \subsection{Option \xoption{rulebetweencolor}\\选项 \xoption{rulebetweencolor}}
%
% Package \xpackage{pdfcolparcolumns} also fixes the color for the
% rule between columns (if \xoption{rulebetween} is set).
% Default color is \cs{normalcolor}. But this can be changed by using
% option \xoption{rulebetweencolor}. It takes a color specification
% as value. If the value is empty, then the default (\cs{normalcolor})
% is used.
% Examples:

% 包 \xpackage{pdfcolparcolumns} 还修复了列之间分隔线的颜色(如果设置了 \xoption{rulebetween})。默认颜色为 \cs{normalcolor}。但是,可以使用选项 \xoption{rulebetweencolor} 来更改它。它接受一个颜色规范作为值。如果该值为空,则使用默认值(\cs{normalcolor})。例如:

% \begin{quote}
%   |rulebetweencolor=blue|,\\
%   |rulebetweencolor={red}|,\\
%   |rulebetweencolor={}|, \textit{\% \cs{normalcolor} is used}\\
%   |rulebetweencolor=[rgb]{1,0,.5}| \textit{\% see below}
% \end{quote}
% If used inside the optional argument of environment \textsf{parcolumns}
% and the value contains an optional argument, then whole value
% must be put in curly braces:

% 如果在 \textsf{parcolumns} 环境的可选参数中使用,并且值包含可选参数,则整个值必须用花括号括起来:
%\begin{quote}
%\begin{verbatim}
%\begin{parcolumns}[
%  rulebetween,
%  rulebetweencolor={[rgb]{1,0,.5}},
%]{2}
%  ...
%\end{parcolumns}
%\end{verbatim}
%\end{quote}
% This option \xoption{rulebetweencolor} can also be set using
% \cs{setkeys}:

%也可以使用 \cs{setkeys} 来设置选项 \xoption{rulebetweencolor}:
%\begin{quote}
%\begin{verbatim}
%\setkeys{parcolumns}{rulebetweencolor=green}
%\end{verbatim}
%\end{quote}
%
% \subsection{Future\\未来展望}
%
% Currently package \xpackage{parcolumns} does not seem to be
% maintained. Nevertheless if there will be a new version that
% adds support for color stacks, then this package may become
% obsolete.
%
%目前,\xpackage{parcolumns} 包似乎没有维护了。尽管如此,如果出现了新版本并支持颜色栈,那么这个包可能会变得过时。
% \StopEventually{
% }
%
% \section{Implementation\\实现}
%
% \subsection{Identification\\标识}
%
%    \begin{macrocode}
%<*package>
\NeedsTeXFormat{LaTeX2e}
\ProvidesPackage{pdfcolparcolumns}%
  [2019/12/29 v1.5 Color stacks for parcolumns (HO)]%
%    \end{macrocode}
%
% \subsection{Load packages\\加载宏包}
%
% \subsubsection{Package \xpackage{parcolumns}\\宏包 \xpackage{parcolumns}}
%
%    Currently package \xpackage{parcolumns} does not define options.
%    Thus it is just a precaution that the options of
%    package \xpackage{pdfcolparcolumns} are passed to
%    package \xpackage{parcolumns}.

%目前,宏包 \xpackage{parcolumns} 没有定义选项。因此,它只是一个预防措施,以确保将宏包 \xpackage{pdfcolparcolumns} 的选项传递给宏包 \xpackage{parcolumns}。
%    \begin{macrocode}
\DeclareOption*{%
  \PassOptionsToPackage{\CurrentOption}{parcolumns}%
}
\ProcessOptions\relax
\RequirePackage{parcolumns}[2004/11/25]
%    \end{macrocode}
%
% \subsubsection{Package \xpackage{pdfcol}\\宏包 \xpackage{pdfcol}}
%
%    \begin{macrocode}
\RequirePackage{pdfcol}[2007/09/09]
\ifpdfcolAvailable
\else
  \PackageInfo{pdfcolparcolumns}{%
    Loading aborted, because color stacks are not available%
  }%
  \expandafter\endinput
\fi
%    \end{macrocode}
%
% \subsubsection{Package \xpackage{infwarerr}\\宏包 \xpackage{infwarerr}}
%
%    \begin{macrocode}
\RequirePackage{infwarerr}[2007/09/09]
%    \end{macrocode}
%
% \subsection{Color stack macros\\颜色堆栈宏}
%
%    \begin{macro}{\pcpc@MaxStack}
%    Macro \cs{pcpc@MaxStack} holds the highest number of
%    allocated stacks.

%宏 \cs{pcpc@MaxStack} 存储已分配的堆栈中最大的编号。
%    \begin{macrocode}
\global\chardef\pcpc@MaxStack=\z@
%    \end{macrocode}
%    \end{macro}
%    \begin{macro}{\pcpc@InitStacks}
%    Macro \cs{pcpc@InitStacks} takes the number of columns
%    as argument and ensures that there are enough color
%    stacks for all columns.
%
%宏 \cs{pcpc@InitStacks} 接受列数作为参数,并确保为所有列分配了足够的颜色堆栈。
%    \begin{macrocode}
\def\pcpc@InitStacks#1{%
  \ifnum#1>\pcpc@MaxStack
    \begingroup
      \count@\pcpc@MaxStack
      \loop
        \advance\count@\@ne
        \pdfcolInitStack{pcpc@\the\count@}%
      \ifnum#1>\count@
      \repeat
      \global\chardef\pcpc@MaxStack=\count@
    \endgroup
  \fi
}
%    \end{macrocode}
%    \end{macro}
%
%    \begin{macro}{\pcpc@SwitchStack}
%    \begin{macrocode}
\def\pcpc@SwitchStack#1{%
  \pdfcolSwitchStack{pcpc@\number#1}%
}
%    \end{macrocode}
%    \end{macro}
%
%    \begin{macro}{\pcpc@SetCurrent}
%    \begin{macrocode}
\def\pcpc@SetCurrent#1{%
  \pdfcolSetCurrent{pcpc@\number#1}%
}
%    \end{macrocode}
%    \end{macro}
%
% \subsection{Patches\\补丁}
%
%     Now the color stack macros are patched into the macros
%     of package \xpackage{parcolumns}.
%
%现在颜色堆栈宏已经被打补丁到了 \xpackage{parcolumns} 宏包的宏中。
% \subsubsection{Init stacks\\初始化堆栈}
%
%    \cs{pcpc@InitStacks} should go into the definition of
%    environment |parcolumns|. \cs{pc@alloccolumns} is executed
%    there and nowhere else, thus we hook into it.

%\cs{pcpc@InitStacks} 应该被放在环境 |parcolumns| 的定义中。\cs{pc@alloccolumns} 仅在那里执行,因此我们将其连接起来。
%    \begin{macrocode}
\g@addto@macro\pc@alloccolumns{%
  \pcpc@InitStacks\pc@columncount
}
%    \end{macrocode}
%
% \subsubsection{Switch stack\\切换堆栈}
%
%    \cs{pcpc@SwitchStack} should be called by marco \cs{colchunk@}.
%    However it is easier to patch \cs{pc@setcolumnwidth} that
%    is executed in \cs{colchunk@} only.
%
%\cs{pcpc@SwitchStack} 应该被宏 \cs{colchunk@} 调用。不过,我们更容易修补仅在 \cs{colchunk@} 中执行的 \cs{pc@setcolumnwidth}。
%    \begin{macrocode}
\g@addto@macro\pc@setcolumnwidth{%
  \pcpc@SwitchStack\pc@columnctr
}
%    \end{macrocode}
%
% \subsubsection{Set current stack color\\设置当前堆栈颜色}
%
%    \cs{pcpc@SetCurrent} is set at the begin of each line.
%    It must be inserted into \cs{pc@placeboxes}. Unhappily
%    there is no easy way. Therefore we check and
%    redefine \cs{pc@placeboxes}.
%
%\cs{pcpc@SetCurrent} 在每行开始时设置。它必须插入到 \cs{pc@placeboxes} 中。不幸的是,没有简单的方法。因此,我们检查并重新定义 \cs{pc@placeboxes}。
%    \begin{macrocode}
\begingroup
  \def\x{%
    \global\let\@tempa\relax
    \count@\z@
    \hb@xt@\linewidth{%
      \vfuzz30ex %
      \vbadness\@M
      \splittopskip\z@skip
      \loop
      \ifnum\count@<\pc@columncount
        \advance\count@\@ne
        \expandafter\ifvoid\csname pc@column@\number\count@\endcsname
          \hskip\csname pc@column@width@\number\count@\endcsname
        \else
          \expandafter\setbox\expandafter\@tempboxa\expandafter
          \vsplit\csname pc@column@\number\count@\endcsname
              to \dp\strutbox
          \vbox{%
            \unvbox\@tempboxa
          }%
        \fi
        \expandafter\ifvoid\csname pc@column@\number\count@\endcsname
        \else
          \global\let\@tempa\pc@placeboxes
        \fi
        \ifnum\count@<\pc@columncount
          \strut
          \hfill
          \ifpc@rulebetween
            \vrule
            \hfill
          \fi
        \fi
      \repeat
    }%
    \@tempa
  }%
  \ifx\x\pc@placeboxes
  \else
    \@PackageWarningNoLine{pdfcolparcolumns}{%
      Command \string\pc@placeboxes\space has changed.\MessageBreak
      Supported versions of package `parcolumns':\MessageBreak
      \space\space 2004/08/05.\MessageBreak
      The redefinition of \string\pc@placeboxes\space may not%
      \MessageBreak
      behave correctly depending on the changes%
    }%
  \fi
\endgroup
%    \end{macrocode}
%    \begin{macro}{\pc@placeboxes}
%    \begin{macrocode}
\renewcommand*{\pc@placeboxes}{%
  \global\let\@tempa\relax
  \count@\z@
  \hb@xt@\linewidth{%
    \vfuzz30ex %
    \vbadness\@M
    \splittopskip\z@skip
    \loop
    \ifnum\count@<\pc@columncount
      \advance\count@\@ne
      \expandafter\ifvoid\csname pc@column@\number\count@\endcsname
        \hskip\csname pc@column@width@\number\count@\endcsname
      \else
        \expandafter\setbox\expandafter\@tempboxa\expandafter
        \vsplit\csname pc@column@\number\count@\endcsname
            to \dp\strutbox
        \vbox{%
          \pcpc@SetCurrent\count@
          \unvbox\@tempboxa
        }%
      \fi
      \expandafter\ifvoid\csname pc@column@\number\count@\endcsname
      \else
        \global\let\@tempa\pc@placeboxes
      \fi
      \ifnum\count@<\pc@columncount
        \strut
        \hfill
        \ifpc@rulebetween
          \begingroup
            \pcpc@RuleBetweenColor
            \vrule
          \endgroup
          \hfill
        \fi
      \fi
    \repeat
  }%
  \@tempa
}
%    \end{macrocode}
%    \end{macro}
%    \begin{macro}{\pcpc@RuleBetweenColorDefault}
%    \begin{macrocode}
\def\pcpc@RuleBetweenColorDefault{%
  \normalcolor
}
%    \end{macrocode}
%    \end{macro}
%    \begin{macro}{\pcpc@RuleBetweenColor}
%    \begin{macrocode}
\let\pcpc@RuleBetweenColor\pcpc@RuleBetweenColorDefault
%    \end{macrocode}
%    \end{macro}
%    \begin{macrocode}
\define@key{parcolumns}{rulebetweencolor}{%
  \edef\pcpc@temp{#1}%
  \ifx\pcpc@temp\@empty
    \let\pcpc@RuleBetweenColor\pcpc@RuleBetweenColorDefault
  \else
    \edef\pcpc@temp{%
      \noexpand\@ifnextchar[{%
        \def\noexpand\pcpc@RuleBetweenColor{%
          \noexpand\color\pcpc@temp
        }%
        \noexpand\pcpc@GobbleNil
      }{%
        \def\noexpand\pcpc@RuleBetweenColor{%
          \noexpand\color{\pcpc@temp}%
        }%
        \noexpand\pcpc@GobbleNil
      }%
      \pcpc@temp\noexpand\@nil
    }%
    \pcpc@temp
  \fi
}
%    \end{macrocode}
%    \begin{macro}{\pcpc@GobbleNil}
%    \begin{macrocode}
\long\def\pcpc@GobbleNil#1\@nil{}
%    \end{macrocode}
%    \end{macro}
%
%    \begin{macrocode}
%</package>
%    \end{macrocode}
%% \section{Installation\\安装}
%
% \subsection{Download\\下载}
%
% \paragraph{Package.} This package is available on
% CTAN\footnote{\CTANpkg{pdfcolparcolumns}}:

%该软件包可从 CTAN\footnote{\CTANpkg{pdfcolparcolumns}} 下载:
% \begin{description}
% \item[\CTAN{macros/latex/contrib/oberdiek/pdfcolparcolumns.dtx}] The source file.
% \item[\CTAN{macros/latex/contrib/oberdiek/pdfcolparcolumns.pdf}] Documentation.
% \end{description}
%
%
% \paragraph{Bundle.} All the packages of the bundle `oberdiek'
% are also available in a TDS compliant ZIP archive. There
% the packages are already unpacked and the documentation files
% are generated. The files and directories obey the TDS standard.
%
%捆绑包“oberdiek”中的所有软件包也可在符合 TDS 标准的 ZIP 存档中获取。在该存档中,软件包已经被解包,文档文件已经生成,文件和目录符合 TDS 标准。
% \begin{description}
% \item[\CTANinstall{install/macros/latex/contrib/oberdiek.tds.zip}]
% \end{description}
% \emph{TDS} refers to the standard ``A Directory Structure
% for \TeX\ Files'' (\CTANpkg{tds}). Directories
% with \xfile{texmf} in their name are usually organized this way.
%
%\emph{TDS} 指的是标准“用于 \TeX\ 文件的目录结构”(\CTANpkg{tds})。名字中包含\xfile{texmf}的目录通常都是按这种方式组织的。
% \subsection{Bundle installation\\捆绑包安装}
%
% \paragraph{Unpacking.} Unpack the \xfile{oberdiek.tds.zip} in the
% TDS tree (also known as \xfile{texmf} tree) of your choice.
% Example (linux):
%
%\paragraph{解包。}在您选择的 TDS 树(也称为\xfile{texmf}树)中解压\xfile{oberdiek.tds.zip}。例如(在Linux中):
% \begin{quote}
%   |unzip oberdiek.tds.zip -d ~/texmf|
% \end{quote}
%
% \subsection{Package installation\\软件包安装}
%
% \paragraph{Unpacking.} The \xfile{.dtx} file is a self-extracting
% \docstrip\ archive. The files are extracted by running the
% \xfile{.dtx} through \plainTeX:
%
%\paragraph{解包。} \xfile{.dtx} 文件是一个自解压的 \docstrip\ 存档。运行\xfile{.dtx}通过\plainTeX\ 来提取文件:
% \begin{quote}
%   \verb|tex pdfcolparcolumns.dtx|
% \end{quote}
%
% \paragraph{TDS.} Now the different files must be moved into
% the different directories in your installation TDS tree
% (also known as \xfile{texmf} tree):
%
%\paragraph{TDS。}现在,不同的文件必须移动到安装 TDS 树(也称为\xfile{texmf}树)中的不同目录中:
% \begin{quote}
% \def\t{^^A
% \begin{tabular}{@{}>{\ttfamily}l@{ $\rightarrow$ }>{\ttfamily}l@{}}
%   pdfcolparcolumns.sty & tex/latex/oberdiek/pdfcolparcolumns.sty\\
%   pdfcolparcolumns.pdf & doc/latex/oberdiek/pdfcolparcolumns.pdf\\
%   pdfcolparcolumns.dtx & source/latex/oberdiek/pdfcolparcolumns.dtx\\
% \end{tabular}^^A
% }^^A
% \sbox0{\t}^^A
% \ifdim\wd0>\linewidth
%   \begingroup
%     \advance\linewidth by\leftmargin
%     \advance\linewidth by\rightmargin
%   \edef\x{\endgroup
%     \def\noexpand\lw{\the\linewidth}^^A
%   }\x
%   \def\lwbox{^^A
%     \leavevmode
%     \hbox to \linewidth{^^A
%       \kern-\leftmargin\relax
%       \hss
%       \usebox0
%       \hss
%       \kern-\rightmargin\relax
%     }^^A
%   }^^A
%   \ifdim\wd0>\lw
%     \sbox0{\small\t}^^A
%     \ifdim\wd0>\linewidth
%       \ifdim\wd0>\lw
%         \sbox0{\footnotesize\t}^^A
%         \ifdim\wd0>\linewidth
%           \ifdim\wd0>\lw
%             \sbox0{\scriptsize\t}^^A
%             \ifdim\wd0>\linewidth
%               \ifdim\wd0>\lw
%                 \sbox0{\tiny\t}^^A
%                 \ifdim\wd0>\linewidth
%                   \lwbox
%                 \else
%                   \usebox0
%                 \fi
%               \else
%                 \lwbox
%               \fi
%             \else
%               \usebox0
%             \fi
%           \else
%             \lwbox
%           \fi
%         \else
%           \usebox0
%         \fi
%       \else
%         \lwbox
%       \fi
%     \else
%       \usebox0
%     \fi
%   \else
%     \lwbox
%   \fi
% \else
%   \usebox0
% \fi
% \end{quote}
% If you have a \xfile{docstrip.cfg} that configures and enables \docstrip's
% TDS installing feature, then some files can already be in the right
% place, see the documentation of \docstrip.
%
%如果你有一个\xfile{docstrip.cfg}文件配置和启用了\docstrip 的TDS安装功能,那么一些文件可能已经位于正确的位置,具体请参见\docstrip 的文档。
% \subsection{Refresh file name databases\\刷新文件名数据库}
%
% If your \TeX~distribution
% (\TeX\,Live, \mikTeX, \dots) relies on file name databases, you must refresh
% these. For example, \TeX\,Live\ users run \verb|texhash| or
% \verb|mktexlsr|.
%
%如果你的\TeX~发行版(\TeX,Live、\mikTeX,等等)依赖于文件名数据库,你必须刷新它们。例如,\TeX,Live用户运行\verb|texhash|或\verb|mktexlsr|。
% \subsection{Some details for the interested\\一些细节供感兴趣的人使用}
%
% \paragraph{Unpacking with \LaTeX.}
% The \xfile{.dtx} chooses its action depending on the format:

%\paragraph{用\LaTeX 进行解包。} \xfile{.dtx}文件根据格式选择操作:
% \begin{description}
% \item[\plainTeX:] Run \docstrip\ and extract the files.
% \item[\LaTeX:] Generate the documentation.
% \end{description}
% If you insist on using \LaTeX\ for \docstrip\ (really,
% \docstrip\ does not need \LaTeX), then inform the autodetect routine
% about your intention:

%如果你坚持要用\LaTeX 进行\docstrip (其实\docstrip 不需要\LaTeX ),那么请告知自动检测例程你的意图:

% \begin{quote}
%   \verb|latex \let\install=y% \iffalse meta-comment
%
% File: pdfcolparcolumns.dtx
% Version: 2019/12/29 v1.5
% Info: Color stacks for parcolumns
%
% Copyright (C)
%    2007, 2008, 2010 Heiko Oberdiek
%    2016-2019 Oberdiek Package Support Group
%    https://github.com/ho-tex/oberdiek/issues
%
% This work may be distributed and/or modified under the
% conditions of the LaTeX Project Public License, either
% version 1.3c of this license or (at your option) any later
% version. This version of this license is in
%    https://www.latex-project.org/lppl/lppl-1-3c.txt
% and the latest version of this license is in
%    https://www.latex-project.org/lppl.txt
% and version 1.3 or later is part of all distributions of
% LaTeX version 2005/12/01 or later.
%
% This work has the LPPL maintenance status "maintained".
%
% The Current Maintainers of this work are
% Heiko Oberdiek and the Oberdiek Package Support Group
% https://github.com/ho-tex/oberdiek/issues
%
% This work consists of the main source file pdfcolparcolumns.dtx
% and the derived files
%    pdfcolparcolumns.sty, pdfcolparcolumns.pdf, pdfcolparcolumns.ins,
%    pdfcolparcolumns.drv, pdfcolparcolumns-test1.tex.
%
% Distribution:
%    CTAN:macros/latex/contrib/oberdiek/pdfcolparcolumns.dtx
%    CTAN:macros/latex/contrib/oberdiek/pdfcolparcolumns.pdf
%
% Unpacking:
%    (a) If pdfcolparcolumns.ins is present:
%           tex pdfcolparcolumns.ins
%    (b) Without pdfcolparcolumns.ins:
%           tex pdfcolparcolumns.dtx
%    (c) If you insist on using LaTeX
%           latex \let\install=y% \iffalse meta-comment
%
% File: pdfcolparcolumns.dtx
% Version: 2019/12/29 v1.5
% Info: Color stacks for parcolumns
%
% Copyright (C)
%    2007, 2008, 2010 Heiko Oberdiek
%    2016-2019 Oberdiek Package Support Group
%    https://github.com/ho-tex/oberdiek/issues
%
% This work may be distributed and/or modified under the
% conditions of the LaTeX Project Public License, either
% version 1.3c of this license or (at your option) any later
% version. This version of this license is in
%    https://www.latex-project.org/lppl/lppl-1-3c.txt
% and the latest version of this license is in
%    https://www.latex-project.org/lppl.txt
% and version 1.3 or later is part of all distributions of
% LaTeX version 2005/12/01 or later.
%
% This work has the LPPL maintenance status "maintained".
%
% The Current Maintainers of this work are
% Heiko Oberdiek and the Oberdiek Package Support Group
% https://github.com/ho-tex/oberdiek/issues
%
% This work consists of the main source file pdfcolparcolumns.dtx
% and the derived files
%    pdfcolparcolumns.sty, pdfcolparcolumns.pdf, pdfcolparcolumns.ins,
%    pdfcolparcolumns.drv, pdfcolparcolumns-test1.tex.
%
% Distribution:
%    CTAN:macros/latex/contrib/oberdiek/pdfcolparcolumns.dtx
%    CTAN:macros/latex/contrib/oberdiek/pdfcolparcolumns.pdf
%
% Unpacking:
%    (a) If pdfcolparcolumns.ins is present:
%           tex pdfcolparcolumns.ins
%    (b) Without pdfcolparcolumns.ins:
%           tex pdfcolparcolumns.dtx
%    (c) If you insist on using LaTeX
%           latex \let\install=y% \iffalse meta-comment
%
% File: pdfcolparcolumns.dtx
% Version: 2019/12/29 v1.5
% Info: Color stacks for parcolumns
%
% Copyright (C)
%    2007, 2008, 2010 Heiko Oberdiek
%    2016-2019 Oberdiek Package Support Group
%    https://github.com/ho-tex/oberdiek/issues
%
% This work may be distributed and/or modified under the
% conditions of the LaTeX Project Public License, either
% version 1.3c of this license or (at your option) any later
% version. This version of this license is in
%    https://www.latex-project.org/lppl/lppl-1-3c.txt
% and the latest version of this license is in
%    https://www.latex-project.org/lppl.txt
% and version 1.3 or later is part of all distributions of
% LaTeX version 2005/12/01 or later.
%
% This work has the LPPL maintenance status "maintained".
%
% The Current Maintainers of this work are
% Heiko Oberdiek and the Oberdiek Package Support Group
% https://github.com/ho-tex/oberdiek/issues
%
% This work consists of the main source file pdfcolparcolumns.dtx
% and the derived files
%    pdfcolparcolumns.sty, pdfcolparcolumns.pdf, pdfcolparcolumns.ins,
%    pdfcolparcolumns.drv, pdfcolparcolumns-test1.tex.
%
% Distribution:
%    CTAN:macros/latex/contrib/oberdiek/pdfcolparcolumns.dtx
%    CTAN:macros/latex/contrib/oberdiek/pdfcolparcolumns.pdf
%
% Unpacking:
%    (a) If pdfcolparcolumns.ins is present:
%           tex pdfcolparcolumns.ins
%    (b) Without pdfcolparcolumns.ins:
%           tex pdfcolparcolumns.dtx
%    (c) If you insist on using LaTeX
%           latex \let\install=y\input{pdfcolparcolumns.dtx}
%        (quote the arguments according to the demands of your shell)
%
% Documentation:
%    (a) If pdfcolparcolumns.drv is present:
%           latex pdfcolparcolumns.drv
%    (b) Without pdfcolparcolumns.drv:
%           latex pdfcolparcolumns.dtx; ...
%    The class ltxdoc loads the configuration file ltxdoc.cfg
%    if available. Here you can specify further options, e.g.
%    use A4 as paper format:
%       \PassOptionsToClass{a4paper}{article}
%
%    Programm calls to get the documentation (example):
%       pdflatex pdfcolparcolumns.dtx
%       makeindex -s gind.ist pdfcolparcolumns.idx
%       pdflatex pdfcolparcolumns.dtx
%       makeindex -s gind.ist pdfcolparcolumns.idx
%       pdflatex pdfcolparcolumns.dtx
%
% Installation:
%    TDS:tex/latex/oberdiek/pdfcolparcolumns.sty
%    TDS:doc/latex/oberdiek/pdfcolparcolumns.pdf
%    TDS:source/latex/oberdiek/pdfcolparcolumns.dtx
%
%<*ignore>
\begingroup
  \catcode123=1 %
  \catcode125=2 %
  \def\x{LaTeX2e}%
\expandafter\endgroup
\ifcase 0\ifx\install y1\fi\expandafter
         \ifx\csname processbatchFile\endcsname\relax\else1\fi
         \ifx\fmtname\x\else 1\fi\relax
\else\csname fi\endcsname
%</ignore>
%<*install>
\input docstrip.tex
\Msg{************************************************************************}
\Msg{* Installation}
\Msg{* Package: pdfcolparcolumns 2019/12/29 v1.5 Color stacks for parcolumns (HO)}
\Msg{************************************************************************}

\keepsilent
\askforoverwritefalse

\let\MetaPrefix\relax
\preamble

This is a generated file.

Project: pdfcolparcolumns
Version: 2019/12/29 v1.5

Copyright (C)
   2007, 2008, 2010 Heiko Oberdiek
   2016-2019 Oberdiek Package Support Group

This work may be distributed and/or modified under the
conditions of the LaTeX Project Public License, either
version 1.3c of this license or (at your option) any later
version. This version of this license is in
   https://www.latex-project.org/lppl/lppl-1-3c.txt
and the latest version of this license is in
   https://www.latex-project.org/lppl.txt
and version 1.3 or later is part of all distributions of
LaTeX version 2005/12/01 or later.

This work has the LPPL maintenance status "maintained".

The Current Maintainers of this work are
Heiko Oberdiek and the Oberdiek Package Support Group
https://github.com/ho-tex/oberdiek/issues


This work consists of the main source file pdfcolparcolumns.dtx
and the derived files
   pdfcolparcolumns.sty, pdfcolparcolumns.pdf, pdfcolparcolumns.ins,
   pdfcolparcolumns.drv, pdfcolparcolumns-test1.tex.

\endpreamble
\let\MetaPrefix\DoubleperCent

\generate{%
  \file{pdfcolparcolumns.ins}{\from{pdfcolparcolumns.dtx}{install}}%
  \file{pdfcolparcolumns.drv}{\from{pdfcolparcolumns.dtx}{driver}}%
  \usedir{tex/latex/oberdiek}%
  \file{pdfcolparcolumns.sty}{\from{pdfcolparcolumns.dtx}{package}}%
%  \usedir{doc/latex/oberdiek/test}%
%  \file{pdfcolparcolumns-test1.tex}{\from{pdfcolparcolumns.dtx}{test1}}%
}

\catcode32=13\relax% active space
\let =\space%
\Msg{************************************************************************}
\Msg{*}
\Msg{* To finish the installation you have to move the following}
\Msg{* file into a directory searched by TeX:}
\Msg{*}
\Msg{*     pdfcolparcolumns.sty}
\Msg{*}
\Msg{* To produce the documentation run the file `pdfcolparcolumns.drv'}
\Msg{* through LaTeX.}
\Msg{*}
\Msg{* Happy TeXing!}
\Msg{*}
\Msg{************************************************************************}

\endbatchfile
%</install>
%<*ignore>
\fi
%</ignore>
%<*driver>
\NeedsTeXFormat{LaTeX2e}
\ProvidesFile{pdfcolparcolumns.drv}%
  [2019/12/29 v1.5 Color stacks for parcolumns (HO)]%
\documentclass{ltxdoc}
\usepackage{holtxdoc}[2011/11/22]
\usepackage[scheme=plain]{ctex}
\setCJKmainfont{方正书宋_GBK}%方正书宋_GBK.TTF  设置文本的中文有衬线字体为“方正书宋_GBK”
\setCJKsansfont{方正黑体简体}%方正黑体_GBK.TTF  设置文本的中文无衬线字体为“方正黑体简体”
\setCJKmonofont{方正书宋简体}%方正仿宋_GBK.TTF  设置文本的中文等宽字体为“方正书宋简体”
\begin{document}
  \DocInput{pdfcolparcolumns.dtx}%
\end{document}
%</driver>
% \fi
%
%
%
% \GetFileInfo{pdfcolparcolumns.drv}
%
% \title{The \xpackage{pdfcolparcolumns} package}
% \date{2019/12/29 v1.5}
% \author{Heiko Oberdiek\thanks
% {Please report any issues at \url{https://github.com/ho-tex/oberdiek/issues}}\and 翻译\\{\tt virhuiai@qq.com}}
%
% \maketitle
%
% \begin{abstract}
%\makebox[0pt]{}
% \begin{parcolumns}[nofirstindent]{2}
%\colchunk{Since version 1.40 \pdfTeX\ supports several color stacks.
%This package uses them to fix color problems in
%package \xpackage{parcolumns}.}
%\colchunk{自从版本1.40起,\pdfTeX 支持多个颜色栈。
%此包使用它们来解决 \xpackage{parcolumns} 包中的颜色问题。}
%\end{parcolumns}



% \end{abstract}
%
% \tableofcontents
%
% \section{Usage\\用法}
%
% \begin{quote}
% |\usepackage{pdfcolparcolumns}|
% \end{quote}
% The package \xpackage{pdfcolparcolumns} loads package \xpackage{parcolums}
% \cite{parcolumns}. If color stacks are available then the
% macros of \xpackage{parcolumns} are patched to add support
% for color stacks.

%包 \xpackage{pdfcolparcolumns} 载入了 \xpackage{parcolumns} \cite{parcolumns} 包。如果有可用的颜色栈,则会对 \xpackage{parcolumns} 的宏进行修补,以添加对颜色栈的支持。

%
% \subsection{Option \xoption{rulebetweencolor}\\选项 \xoption{rulebetweencolor}}
%
% Package \xpackage{pdfcolparcolumns} also fixes the color for the
% rule between columns (if \xoption{rulebetween} is set).
% Default color is \cs{normalcolor}. But this can be changed by using
% option \xoption{rulebetweencolor}. It takes a color specification
% as value. If the value is empty, then the default (\cs{normalcolor})
% is used.
% Examples:

% 包 \xpackage{pdfcolparcolumns} 还修复了列之间分隔线的颜色(如果设置了 \xoption{rulebetween})。默认颜色为 \cs{normalcolor}。但是,可以使用选项 \xoption{rulebetweencolor} 来更改它。它接受一个颜色规范作为值。如果该值为空,则使用默认值(\cs{normalcolor})。例如:

% \begin{quote}
%   |rulebetweencolor=blue|,\\
%   |rulebetweencolor={red}|,\\
%   |rulebetweencolor={}|, \textit{\% \cs{normalcolor} is used}\\
%   |rulebetweencolor=[rgb]{1,0,.5}| \textit{\% see below}
% \end{quote}
% If used inside the optional argument of environment \textsf{parcolumns}
% and the value contains an optional argument, then whole value
% must be put in curly braces:

% 如果在 \textsf{parcolumns} 环境的可选参数中使用,并且值包含可选参数,则整个值必须用花括号括起来:
%\begin{quote}
%\begin{verbatim}
%\begin{parcolumns}[
%  rulebetween,
%  rulebetweencolor={[rgb]{1,0,.5}},
%]{2}
%  ...
%\end{parcolumns}
%\end{verbatim}
%\end{quote}
% This option \xoption{rulebetweencolor} can also be set using
% \cs{setkeys}:

%也可以使用 \cs{setkeys} 来设置选项 \xoption{rulebetweencolor}:
%\begin{quote}
%\begin{verbatim}
%\setkeys{parcolumns}{rulebetweencolor=green}
%\end{verbatim}
%\end{quote}
%
% \subsection{Future\\未来展望}
%
% Currently package \xpackage{parcolumns} does not seem to be
% maintained. Nevertheless if there will be a new version that
% adds support for color stacks, then this package may become
% obsolete.
%
%目前,\xpackage{parcolumns} 包似乎没有维护了。尽管如此,如果出现了新版本并支持颜色栈,那么这个包可能会变得过时。
% \StopEventually{
% }
%
% \section{Implementation\\实现}
%
% \subsection{Identification\\标识}
%
%    \begin{macrocode}
%<*package>
\NeedsTeXFormat{LaTeX2e}
\ProvidesPackage{pdfcolparcolumns}%
  [2019/12/29 v1.5 Color stacks for parcolumns (HO)]%
%    \end{macrocode}
%
% \subsection{Load packages\\加载宏包}
%
% \subsubsection{Package \xpackage{parcolumns}\\宏包 \xpackage{parcolumns}}
%
%    Currently package \xpackage{parcolumns} does not define options.
%    Thus it is just a precaution that the options of
%    package \xpackage{pdfcolparcolumns} are passed to
%    package \xpackage{parcolumns}.

%目前,宏包 \xpackage{parcolumns} 没有定义选项。因此,它只是一个预防措施,以确保将宏包 \xpackage{pdfcolparcolumns} 的选项传递给宏包 \xpackage{parcolumns}。
%    \begin{macrocode}
\DeclareOption*{%
  \PassOptionsToPackage{\CurrentOption}{parcolumns}%
}
\ProcessOptions\relax
\RequirePackage{parcolumns}[2004/11/25]
%    \end{macrocode}
%
% \subsubsection{Package \xpackage{pdfcol}\\宏包 \xpackage{pdfcol}}
%
%    \begin{macrocode}
\RequirePackage{pdfcol}[2007/09/09]
\ifpdfcolAvailable
\else
  \PackageInfo{pdfcolparcolumns}{%
    Loading aborted, because color stacks are not available%
  }%
  \expandafter\endinput
\fi
%    \end{macrocode}
%
% \subsubsection{Package \xpackage{infwarerr}\\宏包 \xpackage{infwarerr}}
%
%    \begin{macrocode}
\RequirePackage{infwarerr}[2007/09/09]
%    \end{macrocode}
%
% \subsection{Color stack macros\\颜色堆栈宏}
%
%    \begin{macro}{\pcpc@MaxStack}
%    Macro \cs{pcpc@MaxStack} holds the highest number of
%    allocated stacks.

%宏 \cs{pcpc@MaxStack} 存储已分配的堆栈中最大的编号。
%    \begin{macrocode}
\global\chardef\pcpc@MaxStack=\z@
%    \end{macrocode}
%    \end{macro}
%    \begin{macro}{\pcpc@InitStacks}
%    Macro \cs{pcpc@InitStacks} takes the number of columns
%    as argument and ensures that there are enough color
%    stacks for all columns.
%
%宏 \cs{pcpc@InitStacks} 接受列数作为参数,并确保为所有列分配了足够的颜色堆栈。
%    \begin{macrocode}
\def\pcpc@InitStacks#1{%
  \ifnum#1>\pcpc@MaxStack
    \begingroup
      \count@\pcpc@MaxStack
      \loop
        \advance\count@\@ne
        \pdfcolInitStack{pcpc@\the\count@}%
      \ifnum#1>\count@
      \repeat
      \global\chardef\pcpc@MaxStack=\count@
    \endgroup
  \fi
}
%    \end{macrocode}
%    \end{macro}
%
%    \begin{macro}{\pcpc@SwitchStack}
%    \begin{macrocode}
\def\pcpc@SwitchStack#1{%
  \pdfcolSwitchStack{pcpc@\number#1}%
}
%    \end{macrocode}
%    \end{macro}
%
%    \begin{macro}{\pcpc@SetCurrent}
%    \begin{macrocode}
\def\pcpc@SetCurrent#1{%
  \pdfcolSetCurrent{pcpc@\number#1}%
}
%    \end{macrocode}
%    \end{macro}
%
% \subsection{Patches\\补丁}
%
%     Now the color stack macros are patched into the macros
%     of package \xpackage{parcolumns}.
%
%现在颜色堆栈宏已经被打补丁到了 \xpackage{parcolumns} 宏包的宏中。
% \subsubsection{Init stacks\\初始化堆栈}
%
%    \cs{pcpc@InitStacks} should go into the definition of
%    environment |parcolumns|. \cs{pc@alloccolumns} is executed
%    there and nowhere else, thus we hook into it.

%\cs{pcpc@InitStacks} 应该被放在环境 |parcolumns| 的定义中。\cs{pc@alloccolumns} 仅在那里执行,因此我们将其连接起来。
%    \begin{macrocode}
\g@addto@macro\pc@alloccolumns{%
  \pcpc@InitStacks\pc@columncount
}
%    \end{macrocode}
%
% \subsubsection{Switch stack\\切换堆栈}
%
%    \cs{pcpc@SwitchStack} should be called by marco \cs{colchunk@}.
%    However it is easier to patch \cs{pc@setcolumnwidth} that
%    is executed in \cs{colchunk@} only.
%
%\cs{pcpc@SwitchStack} 应该被宏 \cs{colchunk@} 调用。不过,我们更容易修补仅在 \cs{colchunk@} 中执行的 \cs{pc@setcolumnwidth}。
%    \begin{macrocode}
\g@addto@macro\pc@setcolumnwidth{%
  \pcpc@SwitchStack\pc@columnctr
}
%    \end{macrocode}
%
% \subsubsection{Set current stack color\\设置当前堆栈颜色}
%
%    \cs{pcpc@SetCurrent} is set at the begin of each line.
%    It must be inserted into \cs{pc@placeboxes}. Unhappily
%    there is no easy way. Therefore we check and
%    redefine \cs{pc@placeboxes}.
%
%\cs{pcpc@SetCurrent} 在每行开始时设置。它必须插入到 \cs{pc@placeboxes} 中。不幸的是,没有简单的方法。因此,我们检查并重新定义 \cs{pc@placeboxes}。
%    \begin{macrocode}
\begingroup
  \def\x{%
    \global\let\@tempa\relax
    \count@\z@
    \hb@xt@\linewidth{%
      \vfuzz30ex %
      \vbadness\@M
      \splittopskip\z@skip
      \loop
      \ifnum\count@<\pc@columncount
        \advance\count@\@ne
        \expandafter\ifvoid\csname pc@column@\number\count@\endcsname
          \hskip\csname pc@column@width@\number\count@\endcsname
        \else
          \expandafter\setbox\expandafter\@tempboxa\expandafter
          \vsplit\csname pc@column@\number\count@\endcsname
              to \dp\strutbox
          \vbox{%
            \unvbox\@tempboxa
          }%
        \fi
        \expandafter\ifvoid\csname pc@column@\number\count@\endcsname
        \else
          \global\let\@tempa\pc@placeboxes
        \fi
        \ifnum\count@<\pc@columncount
          \strut
          \hfill
          \ifpc@rulebetween
            \vrule
            \hfill
          \fi
        \fi
      \repeat
    }%
    \@tempa
  }%
  \ifx\x\pc@placeboxes
  \else
    \@PackageWarningNoLine{pdfcolparcolumns}{%
      Command \string\pc@placeboxes\space has changed.\MessageBreak
      Supported versions of package `parcolumns':\MessageBreak
      \space\space 2004/08/05.\MessageBreak
      The redefinition of \string\pc@placeboxes\space may not%
      \MessageBreak
      behave correctly depending on the changes%
    }%
  \fi
\endgroup
%    \end{macrocode}
%    \begin{macro}{\pc@placeboxes}
%    \begin{macrocode}
\renewcommand*{\pc@placeboxes}{%
  \global\let\@tempa\relax
  \count@\z@
  \hb@xt@\linewidth{%
    \vfuzz30ex %
    \vbadness\@M
    \splittopskip\z@skip
    \loop
    \ifnum\count@<\pc@columncount
      \advance\count@\@ne
      \expandafter\ifvoid\csname pc@column@\number\count@\endcsname
        \hskip\csname pc@column@width@\number\count@\endcsname
      \else
        \expandafter\setbox\expandafter\@tempboxa\expandafter
        \vsplit\csname pc@column@\number\count@\endcsname
            to \dp\strutbox
        \vbox{%
          \pcpc@SetCurrent\count@
          \unvbox\@tempboxa
        }%
      \fi
      \expandafter\ifvoid\csname pc@column@\number\count@\endcsname
      \else
        \global\let\@tempa\pc@placeboxes
      \fi
      \ifnum\count@<\pc@columncount
        \strut
        \hfill
        \ifpc@rulebetween
          \begingroup
            \pcpc@RuleBetweenColor
            \vrule
          \endgroup
          \hfill
        \fi
      \fi
    \repeat
  }%
  \@tempa
}
%    \end{macrocode}
%    \end{macro}
%    \begin{macro}{\pcpc@RuleBetweenColorDefault}
%    \begin{macrocode}
\def\pcpc@RuleBetweenColorDefault{%
  \normalcolor
}
%    \end{macrocode}
%    \end{macro}
%    \begin{macro}{\pcpc@RuleBetweenColor}
%    \begin{macrocode}
\let\pcpc@RuleBetweenColor\pcpc@RuleBetweenColorDefault
%    \end{macrocode}
%    \end{macro}
%    \begin{macrocode}
\define@key{parcolumns}{rulebetweencolor}{%
  \edef\pcpc@temp{#1}%
  \ifx\pcpc@temp\@empty
    \let\pcpc@RuleBetweenColor\pcpc@RuleBetweenColorDefault
  \else
    \edef\pcpc@temp{%
      \noexpand\@ifnextchar[{%
        \def\noexpand\pcpc@RuleBetweenColor{%
          \noexpand\color\pcpc@temp
        }%
        \noexpand\pcpc@GobbleNil
      }{%
        \def\noexpand\pcpc@RuleBetweenColor{%
          \noexpand\color{\pcpc@temp}%
        }%
        \noexpand\pcpc@GobbleNil
      }%
      \pcpc@temp\noexpand\@nil
    }%
    \pcpc@temp
  \fi
}
%    \end{macrocode}
%    \begin{macro}{\pcpc@GobbleNil}
%    \begin{macrocode}
\long\def\pcpc@GobbleNil#1\@nil{}
%    \end{macrocode}
%    \end{macro}
%
%    \begin{macrocode}
%</package>
%    \end{macrocode}
%% \section{Installation\\安装}
%
% \subsection{Download\\下载}
%
% \paragraph{Package.} This package is available on
% CTAN\footnote{\CTANpkg{pdfcolparcolumns}}:

%该软件包可从 CTAN\footnote{\CTANpkg{pdfcolparcolumns}} 下载:
% \begin{description}
% \item[\CTAN{macros/latex/contrib/oberdiek/pdfcolparcolumns.dtx}] The source file.
% \item[\CTAN{macros/latex/contrib/oberdiek/pdfcolparcolumns.pdf}] Documentation.
% \end{description}
%
%
% \paragraph{Bundle.} All the packages of the bundle `oberdiek'
% are also available in a TDS compliant ZIP archive. There
% the packages are already unpacked and the documentation files
% are generated. The files and directories obey the TDS standard.
%
%捆绑包“oberdiek”中的所有软件包也可在符合 TDS 标准的 ZIP 存档中获取。在该存档中,软件包已经被解包,文档文件已经生成,文件和目录符合 TDS 标准。
% \begin{description}
% \item[\CTANinstall{install/macros/latex/contrib/oberdiek.tds.zip}]
% \end{description}
% \emph{TDS} refers to the standard ``A Directory Structure
% for \TeX\ Files'' (\CTANpkg{tds}). Directories
% with \xfile{texmf} in their name are usually organized this way.
%
%\emph{TDS} 指的是标准“用于 \TeX\ 文件的目录结构”(\CTANpkg{tds})。名字中包含\xfile{texmf}的目录通常都是按这种方式组织的。
% \subsection{Bundle installation\\捆绑包安装}
%
% \paragraph{Unpacking.} Unpack the \xfile{oberdiek.tds.zip} in the
% TDS tree (also known as \xfile{texmf} tree) of your choice.
% Example (linux):
%
%\paragraph{解包。}在您选择的 TDS 树(也称为\xfile{texmf}树)中解压\xfile{oberdiek.tds.zip}。例如(在Linux中):
% \begin{quote}
%   |unzip oberdiek.tds.zip -d ~/texmf|
% \end{quote}
%
% \subsection{Package installation\\软件包安装}
%
% \paragraph{Unpacking.} The \xfile{.dtx} file is a self-extracting
% \docstrip\ archive. The files are extracted by running the
% \xfile{.dtx} through \plainTeX:
%
%\paragraph{解包。} \xfile{.dtx} 文件是一个自解压的 \docstrip\ 存档。运行\xfile{.dtx}通过\plainTeX\ 来提取文件:
% \begin{quote}
%   \verb|tex pdfcolparcolumns.dtx|
% \end{quote}
%
% \paragraph{TDS.} Now the different files must be moved into
% the different directories in your installation TDS tree
% (also known as \xfile{texmf} tree):
%
%\paragraph{TDS。}现在,不同的文件必须移动到安装 TDS 树(也称为\xfile{texmf}树)中的不同目录中:
% \begin{quote}
% \def\t{^^A
% \begin{tabular}{@{}>{\ttfamily}l@{ $\rightarrow$ }>{\ttfamily}l@{}}
%   pdfcolparcolumns.sty & tex/latex/oberdiek/pdfcolparcolumns.sty\\
%   pdfcolparcolumns.pdf & doc/latex/oberdiek/pdfcolparcolumns.pdf\\
%   pdfcolparcolumns.dtx & source/latex/oberdiek/pdfcolparcolumns.dtx\\
% \end{tabular}^^A
% }^^A
% \sbox0{\t}^^A
% \ifdim\wd0>\linewidth
%   \begingroup
%     \advance\linewidth by\leftmargin
%     \advance\linewidth by\rightmargin
%   \edef\x{\endgroup
%     \def\noexpand\lw{\the\linewidth}^^A
%   }\x
%   \def\lwbox{^^A
%     \leavevmode
%     \hbox to \linewidth{^^A
%       \kern-\leftmargin\relax
%       \hss
%       \usebox0
%       \hss
%       \kern-\rightmargin\relax
%     }^^A
%   }^^A
%   \ifdim\wd0>\lw
%     \sbox0{\small\t}^^A
%     \ifdim\wd0>\linewidth
%       \ifdim\wd0>\lw
%         \sbox0{\footnotesize\t}^^A
%         \ifdim\wd0>\linewidth
%           \ifdim\wd0>\lw
%             \sbox0{\scriptsize\t}^^A
%             \ifdim\wd0>\linewidth
%               \ifdim\wd0>\lw
%                 \sbox0{\tiny\t}^^A
%                 \ifdim\wd0>\linewidth
%                   \lwbox
%                 \else
%                   \usebox0
%                 \fi
%               \else
%                 \lwbox
%               \fi
%             \else
%               \usebox0
%             \fi
%           \else
%             \lwbox
%           \fi
%         \else
%           \usebox0
%         \fi
%       \else
%         \lwbox
%       \fi
%     \else
%       \usebox0
%     \fi
%   \else
%     \lwbox
%   \fi
% \else
%   \usebox0
% \fi
% \end{quote}
% If you have a \xfile{docstrip.cfg} that configures and enables \docstrip's
% TDS installing feature, then some files can already be in the right
% place, see the documentation of \docstrip.
%
%如果你有一个\xfile{docstrip.cfg}文件配置和启用了\docstrip 的TDS安装功能,那么一些文件可能已经位于正确的位置,具体请参见\docstrip 的文档。
% \subsection{Refresh file name databases\\刷新文件名数据库}
%
% If your \TeX~distribution
% (\TeX\,Live, \mikTeX, \dots) relies on file name databases, you must refresh
% these. For example, \TeX\,Live\ users run \verb|texhash| or
% \verb|mktexlsr|.
%
%如果你的\TeX~发行版(\TeX,Live、\mikTeX,等等)依赖于文件名数据库,你必须刷新它们。例如,\TeX,Live用户运行\verb|texhash|或\verb|mktexlsr|。
% \subsection{Some details for the interested\\一些细节供感兴趣的人使用}
%
% \paragraph{Unpacking with \LaTeX.}
% The \xfile{.dtx} chooses its action depending on the format:

%\paragraph{用\LaTeX 进行解包。} \xfile{.dtx}文件根据格式选择操作:
% \begin{description}
% \item[\plainTeX:] Run \docstrip\ and extract the files.
% \item[\LaTeX:] Generate the documentation.
% \end{description}
% If you insist on using \LaTeX\ for \docstrip\ (really,
% \docstrip\ does not need \LaTeX), then inform the autodetect routine
% about your intention:

%如果你坚持要用\LaTeX 进行\docstrip (其实\docstrip 不需要\LaTeX ),那么请告知自动检测例程你的意图:

% \begin{quote}
%   \verb|latex \let\install=y\input{pdfcolparcolumns.dtx}|
% \end{quote}
% Do not forget to quote the argument according to the demands
% of your shell.

%不要忘记根据你的shell的要求引用参数。
%
% \paragraph{Generating the documentation.}
% You can use both the \xfile{.dtx} or the \xfile{.drv} to generate
% the documentation. The process can be configured by the
% configuration file \xfile{ltxdoc.cfg}. For instance, put this
% line into this file, if you want to have A4 as paper format:
%
%\paragraph{生成文档。} 你可以使用\xfile{.dtx}或\xfile{.drv}生成文档。这个过程可以由配置文件\xfile{ltxdoc.cfg}配置。例如,如果你想要A4作为纸张格式,将此行放入文件中:

% \begin{quote}
%   \verb|\PassOptionsToClass{a4paper}{article}|
% \end{quote}
% An example follows how to generate the
% documentation with pdf\LaTeX:

%以下是使用pdf\LaTeX 生成文档的示例:
% \begin{quote}
%\begin{verbatim}
%pdflatex pdfcolparcolumns.dtx
%makeindex -s gind.ist pdfcolparcolumns.idx
%pdflatex pdfcolparcolumns.dtx
%makeindex -s gind.ist pdfcolparcolumns.idx
%pdflatex pdfcolparcolumns.dtx
%\end{verbatim}
% \end{quote}
%
% \begin{thebibliography}{9}
%
% \bibitem{parcolumns}
%   Jonathan Sauer: \textit{The \xpackage{parcolumns} package};
%   2004/11/25;\\
%   \CTANpkg{parcolumns}.
%
% \bibitem{pdfcol}
%   Heiko Oberdiek: \textit{The \xpackage{pdfcol} package};
%   2007/09/09;\\
%   \CTANpkg{pdfcol}.
%
% \end{thebibliography}
%
% \begin{History}
%   \begin{Version}{2007/07/26 v1.0}
%   \item
%     First version, published in the newsgroup \xnewsgroup{comp.text.tex}
%     with the name \xpackage{parcolumns-colorstacks}: ^^A no line break
%     \URL{``\link{Re: \xpackage{xcolor} glitches}''}^^A
%     {https://groups.google.com/group/comp.text.tex/msg/56bd897b11bca414}

%第一个版本,以 \xnewsgroup{comp.text.tex} 新闻组发布,名称为 \xpackage{parcolumns-colorstacks}:^^A 没有换行符 
%\URL{``\link{Re: \xpackage{xcolor} glitches}''}^^A
%{https://groups.google.com/group/comp.text.tex/msg/56bd897b11bca414}
%   \end{Version}
%   \begin{Version}{2007/09/09 v1.1}
%   \item
%     CTAN version, package name renamed to \xpackage{pdfcolparcolumns}.
%
%CTAN 版本,将包名改为 \xpackage{pdfcolparcolumns}。
%   \item
%     Uses package \xpackage{pdfcol}.
%
%使用 \xpackage{pdfcol} 包。
%   \item
%     Documentation added.
%
%添加文档。
%   \item
%     Test file added.
%
%添加测试文件。
%   \end{Version}
%   \begin{Version}{2008/08/11 v1.2}
%   \item
%     Code is not changed.
%
%代码未改动。
%   \item
%     URLs updated.
%
%更新 URL。
%   \end{Version}
%   \begin{Version}{2010/01/11 v1.3}
%   \item
%     Fix for rule color.
%
%修复分隔线的颜色。
%   \item
%     New option \xoption{rulebetweencolor} for environment |parcolumns|.
%
%为环境 |parcolumns| 添加新选项 \xoption{rulebetweencolor}。
%   \end{Version}
%   \begin{Version}{2016/05/16 v1.4}
%   \item
%     Documentation updates.
%
%更新文档。
%   \end{Version}
%   \begin{Version}{2019/12/29 v1.5}
%   \item
%     \cs{PassOptionsToPackage} not \cs{PassoptionsToPackage}
%
%\cs{PassOptionsToPackage} 改为不区分大小写的 \cs{PassoptionsToPackage}。
%   \end{Version}
% \end{History}
%
% \PrintIndex
%
% \Finale
\endinput

%        (quote the arguments according to the demands of your shell)
%
% Documentation:
%    (a) If pdfcolparcolumns.drv is present:
%           latex pdfcolparcolumns.drv
%    (b) Without pdfcolparcolumns.drv:
%           latex pdfcolparcolumns.dtx; ...
%    The class ltxdoc loads the configuration file ltxdoc.cfg
%    if available. Here you can specify further options, e.g.
%    use A4 as paper format:
%       \PassOptionsToClass{a4paper}{article}
%
%    Programm calls to get the documentation (example):
%       pdflatex pdfcolparcolumns.dtx
%       makeindex -s gind.ist pdfcolparcolumns.idx
%       pdflatex pdfcolparcolumns.dtx
%       makeindex -s gind.ist pdfcolparcolumns.idx
%       pdflatex pdfcolparcolumns.dtx
%
% Installation:
%    TDS:tex/latex/oberdiek/pdfcolparcolumns.sty
%    TDS:doc/latex/oberdiek/pdfcolparcolumns.pdf
%    TDS:source/latex/oberdiek/pdfcolparcolumns.dtx
%
%<*ignore>
\begingroup
  \catcode123=1 %
  \catcode125=2 %
  \def\x{LaTeX2e}%
\expandafter\endgroup
\ifcase 0\ifx\install y1\fi\expandafter
         \ifx\csname processbatchFile\endcsname\relax\else1\fi
         \ifx\fmtname\x\else 1\fi\relax
\else\csname fi\endcsname
%</ignore>
%<*install>
\input docstrip.tex
\Msg{************************************************************************}
\Msg{* Installation}
\Msg{* Package: pdfcolparcolumns 2019/12/29 v1.5 Color stacks for parcolumns (HO)}
\Msg{************************************************************************}

\keepsilent
\askforoverwritefalse

\let\MetaPrefix\relax
\preamble

This is a generated file.

Project: pdfcolparcolumns
Version: 2019/12/29 v1.5

Copyright (C)
   2007, 2008, 2010 Heiko Oberdiek
   2016-2019 Oberdiek Package Support Group

This work may be distributed and/or modified under the
conditions of the LaTeX Project Public License, either
version 1.3c of this license or (at your option) any later
version. This version of this license is in
   https://www.latex-project.org/lppl/lppl-1-3c.txt
and the latest version of this license is in
   https://www.latex-project.org/lppl.txt
and version 1.3 or later is part of all distributions of
LaTeX version 2005/12/01 or later.

This work has the LPPL maintenance status "maintained".

The Current Maintainers of this work are
Heiko Oberdiek and the Oberdiek Package Support Group
https://github.com/ho-tex/oberdiek/issues


This work consists of the main source file pdfcolparcolumns.dtx
and the derived files
   pdfcolparcolumns.sty, pdfcolparcolumns.pdf, pdfcolparcolumns.ins,
   pdfcolparcolumns.drv, pdfcolparcolumns-test1.tex.

\endpreamble
\let\MetaPrefix\DoubleperCent

\generate{%
  \file{pdfcolparcolumns.ins}{\from{pdfcolparcolumns.dtx}{install}}%
  \file{pdfcolparcolumns.drv}{\from{pdfcolparcolumns.dtx}{driver}}%
  \usedir{tex/latex/oberdiek}%
  \file{pdfcolparcolumns.sty}{\from{pdfcolparcolumns.dtx}{package}}%
%  \usedir{doc/latex/oberdiek/test}%
%  \file{pdfcolparcolumns-test1.tex}{\from{pdfcolparcolumns.dtx}{test1}}%
}

\catcode32=13\relax% active space
\let =\space%
\Msg{************************************************************************}
\Msg{*}
\Msg{* To finish the installation you have to move the following}
\Msg{* file into a directory searched by TeX:}
\Msg{*}
\Msg{*     pdfcolparcolumns.sty}
\Msg{*}
\Msg{* To produce the documentation run the file `pdfcolparcolumns.drv'}
\Msg{* through LaTeX.}
\Msg{*}
\Msg{* Happy TeXing!}
\Msg{*}
\Msg{************************************************************************}

\endbatchfile
%</install>
%<*ignore>
\fi
%</ignore>
%<*driver>
\NeedsTeXFormat{LaTeX2e}
\ProvidesFile{pdfcolparcolumns.drv}%
  [2019/12/29 v1.5 Color stacks for parcolumns (HO)]%
\documentclass{ltxdoc}
\usepackage{holtxdoc}[2011/11/22]
\usepackage[scheme=plain]{ctex}
\setCJKmainfont{方正书宋_GBK}%方正书宋_GBK.TTF  设置文本的中文有衬线字体为“方正书宋_GBK”
\setCJKsansfont{方正黑体简体}%方正黑体_GBK.TTF  设置文本的中文无衬线字体为“方正黑体简体”
\setCJKmonofont{方正书宋简体}%方正仿宋_GBK.TTF  设置文本的中文等宽字体为“方正书宋简体”
\begin{document}
  \DocInput{pdfcolparcolumns.dtx}%
\end{document}
%</driver>
% \fi
%
%
%
% \GetFileInfo{pdfcolparcolumns.drv}
%
% \title{The \xpackage{pdfcolparcolumns} package}
% \date{2019/12/29 v1.5}
% \author{Heiko Oberdiek\thanks
% {Please report any issues at \url{https://github.com/ho-tex/oberdiek/issues}}\and 翻译\\{\tt virhuiai@qq.com}}
%
% \maketitle
%
% \begin{abstract}
%\makebox[0pt]{}
% \begin{parcolumns}[nofirstindent]{2}
%\colchunk{Since version 1.40 \pdfTeX\ supports several color stacks.
%This package uses them to fix color problems in
%package \xpackage{parcolumns}.}
%\colchunk{自从版本1.40起,\pdfTeX 支持多个颜色栈。
%此包使用它们来解决 \xpackage{parcolumns} 包中的颜色问题。}
%\end{parcolumns}



% \end{abstract}
%
% \tableofcontents
%
% \section{Usage\\用法}
%
% \begin{quote}
% |\usepackage{pdfcolparcolumns}|
% \end{quote}
% The package \xpackage{pdfcolparcolumns} loads package \xpackage{parcolums}
% \cite{parcolumns}. If color stacks are available then the
% macros of \xpackage{parcolumns} are patched to add support
% for color stacks.

%包 \xpackage{pdfcolparcolumns} 载入了 \xpackage{parcolumns} \cite{parcolumns} 包。如果有可用的颜色栈,则会对 \xpackage{parcolumns} 的宏进行修补,以添加对颜色栈的支持。

%
% \subsection{Option \xoption{rulebetweencolor}\\选项 \xoption{rulebetweencolor}}
%
% Package \xpackage{pdfcolparcolumns} also fixes the color for the
% rule between columns (if \xoption{rulebetween} is set).
% Default color is \cs{normalcolor}. But this can be changed by using
% option \xoption{rulebetweencolor}. It takes a color specification
% as value. If the value is empty, then the default (\cs{normalcolor})
% is used.
% Examples:

% 包 \xpackage{pdfcolparcolumns} 还修复了列之间分隔线的颜色(如果设置了 \xoption{rulebetween})。默认颜色为 \cs{normalcolor}。但是,可以使用选项 \xoption{rulebetweencolor} 来更改它。它接受一个颜色规范作为值。如果该值为空,则使用默认值(\cs{normalcolor})。例如:

% \begin{quote}
%   |rulebetweencolor=blue|,\\
%   |rulebetweencolor={red}|,\\
%   |rulebetweencolor={}|, \textit{\% \cs{normalcolor} is used}\\
%   |rulebetweencolor=[rgb]{1,0,.5}| \textit{\% see below}
% \end{quote}
% If used inside the optional argument of environment \textsf{parcolumns}
% and the value contains an optional argument, then whole value
% must be put in curly braces:

% 如果在 \textsf{parcolumns} 环境的可选参数中使用,并且值包含可选参数,则整个值必须用花括号括起来:
%\begin{quote}
%\begin{verbatim}
%\begin{parcolumns}[
%  rulebetween,
%  rulebetweencolor={[rgb]{1,0,.5}},
%]{2}
%  ...
%\end{parcolumns}
%\end{verbatim}
%\end{quote}
% This option \xoption{rulebetweencolor} can also be set using
% \cs{setkeys}:

%也可以使用 \cs{setkeys} 来设置选项 \xoption{rulebetweencolor}:
%\begin{quote}
%\begin{verbatim}
%\setkeys{parcolumns}{rulebetweencolor=green}
%\end{verbatim}
%\end{quote}
%
% \subsection{Future\\未来展望}
%
% Currently package \xpackage{parcolumns} does not seem to be
% maintained. Nevertheless if there will be a new version that
% adds support for color stacks, then this package may become
% obsolete.
%
%目前,\xpackage{parcolumns} 包似乎没有维护了。尽管如此,如果出现了新版本并支持颜色栈,那么这个包可能会变得过时。
% \StopEventually{
% }
%
% \section{Implementation\\实现}
%
% \subsection{Identification\\标识}
%
%    \begin{macrocode}
%<*package>
\NeedsTeXFormat{LaTeX2e}
\ProvidesPackage{pdfcolparcolumns}%
  [2019/12/29 v1.5 Color stacks for parcolumns (HO)]%
%    \end{macrocode}
%
% \subsection{Load packages\\加载宏包}
%
% \subsubsection{Package \xpackage{parcolumns}\\宏包 \xpackage{parcolumns}}
%
%    Currently package \xpackage{parcolumns} does not define options.
%    Thus it is just a precaution that the options of
%    package \xpackage{pdfcolparcolumns} are passed to
%    package \xpackage{parcolumns}.

%目前,宏包 \xpackage{parcolumns} 没有定义选项。因此,它只是一个预防措施,以确保将宏包 \xpackage{pdfcolparcolumns} 的选项传递给宏包 \xpackage{parcolumns}。
%    \begin{macrocode}
\DeclareOption*{%
  \PassOptionsToPackage{\CurrentOption}{parcolumns}%
}
\ProcessOptions\relax
\RequirePackage{parcolumns}[2004/11/25]
%    \end{macrocode}
%
% \subsubsection{Package \xpackage{pdfcol}\\宏包 \xpackage{pdfcol}}
%
%    \begin{macrocode}
\RequirePackage{pdfcol}[2007/09/09]
\ifpdfcolAvailable
\else
  \PackageInfo{pdfcolparcolumns}{%
    Loading aborted, because color stacks are not available%
  }%
  \expandafter\endinput
\fi
%    \end{macrocode}
%
% \subsubsection{Package \xpackage{infwarerr}\\宏包 \xpackage{infwarerr}}
%
%    \begin{macrocode}
\RequirePackage{infwarerr}[2007/09/09]
%    \end{macrocode}
%
% \subsection{Color stack macros\\颜色堆栈宏}
%
%    \begin{macro}{\pcpc@MaxStack}
%    Macro \cs{pcpc@MaxStack} holds the highest number of
%    allocated stacks.

%宏 \cs{pcpc@MaxStack} 存储已分配的堆栈中最大的编号。
%    \begin{macrocode}
\global\chardef\pcpc@MaxStack=\z@
%    \end{macrocode}
%    \end{macro}
%    \begin{macro}{\pcpc@InitStacks}
%    Macro \cs{pcpc@InitStacks} takes the number of columns
%    as argument and ensures that there are enough color
%    stacks for all columns.
%
%宏 \cs{pcpc@InitStacks} 接受列数作为参数,并确保为所有列分配了足够的颜色堆栈。
%    \begin{macrocode}
\def\pcpc@InitStacks#1{%
  \ifnum#1>\pcpc@MaxStack
    \begingroup
      \count@\pcpc@MaxStack
      \loop
        \advance\count@\@ne
        \pdfcolInitStack{pcpc@\the\count@}%
      \ifnum#1>\count@
      \repeat
      \global\chardef\pcpc@MaxStack=\count@
    \endgroup
  \fi
}
%    \end{macrocode}
%    \end{macro}
%
%    \begin{macro}{\pcpc@SwitchStack}
%    \begin{macrocode}
\def\pcpc@SwitchStack#1{%
  \pdfcolSwitchStack{pcpc@\number#1}%
}
%    \end{macrocode}
%    \end{macro}
%
%    \begin{macro}{\pcpc@SetCurrent}
%    \begin{macrocode}
\def\pcpc@SetCurrent#1{%
  \pdfcolSetCurrent{pcpc@\number#1}%
}
%    \end{macrocode}
%    \end{macro}
%
% \subsection{Patches\\补丁}
%
%     Now the color stack macros are patched into the macros
%     of package \xpackage{parcolumns}.
%
%现在颜色堆栈宏已经被打补丁到了 \xpackage{parcolumns} 宏包的宏中。
% \subsubsection{Init stacks\\初始化堆栈}
%
%    \cs{pcpc@InitStacks} should go into the definition of
%    environment |parcolumns|. \cs{pc@alloccolumns} is executed
%    there and nowhere else, thus we hook into it.

%\cs{pcpc@InitStacks} 应该被放在环境 |parcolumns| 的定义中。\cs{pc@alloccolumns} 仅在那里执行,因此我们将其连接起来。
%    \begin{macrocode}
\g@addto@macro\pc@alloccolumns{%
  \pcpc@InitStacks\pc@columncount
}
%    \end{macrocode}
%
% \subsubsection{Switch stack\\切换堆栈}
%
%    \cs{pcpc@SwitchStack} should be called by marco \cs{colchunk@}.
%    However it is easier to patch \cs{pc@setcolumnwidth} that
%    is executed in \cs{colchunk@} only.
%
%\cs{pcpc@SwitchStack} 应该被宏 \cs{colchunk@} 调用。不过,我们更容易修补仅在 \cs{colchunk@} 中执行的 \cs{pc@setcolumnwidth}。
%    \begin{macrocode}
\g@addto@macro\pc@setcolumnwidth{%
  \pcpc@SwitchStack\pc@columnctr
}
%    \end{macrocode}
%
% \subsubsection{Set current stack color\\设置当前堆栈颜色}
%
%    \cs{pcpc@SetCurrent} is set at the begin of each line.
%    It must be inserted into \cs{pc@placeboxes}. Unhappily
%    there is no easy way. Therefore we check and
%    redefine \cs{pc@placeboxes}.
%
%\cs{pcpc@SetCurrent} 在每行开始时设置。它必须插入到 \cs{pc@placeboxes} 中。不幸的是,没有简单的方法。因此,我们检查并重新定义 \cs{pc@placeboxes}。
%    \begin{macrocode}
\begingroup
  \def\x{%
    \global\let\@tempa\relax
    \count@\z@
    \hb@xt@\linewidth{%
      \vfuzz30ex %
      \vbadness\@M
      \splittopskip\z@skip
      \loop
      \ifnum\count@<\pc@columncount
        \advance\count@\@ne
        \expandafter\ifvoid\csname pc@column@\number\count@\endcsname
          \hskip\csname pc@column@width@\number\count@\endcsname
        \else
          \expandafter\setbox\expandafter\@tempboxa\expandafter
          \vsplit\csname pc@column@\number\count@\endcsname
              to \dp\strutbox
          \vbox{%
            \unvbox\@tempboxa
          }%
        \fi
        \expandafter\ifvoid\csname pc@column@\number\count@\endcsname
        \else
          \global\let\@tempa\pc@placeboxes
        \fi
        \ifnum\count@<\pc@columncount
          \strut
          \hfill
          \ifpc@rulebetween
            \vrule
            \hfill
          \fi
        \fi
      \repeat
    }%
    \@tempa
  }%
  \ifx\x\pc@placeboxes
  \else
    \@PackageWarningNoLine{pdfcolparcolumns}{%
      Command \string\pc@placeboxes\space has changed.\MessageBreak
      Supported versions of package `parcolumns':\MessageBreak
      \space\space 2004/08/05.\MessageBreak
      The redefinition of \string\pc@placeboxes\space may not%
      \MessageBreak
      behave correctly depending on the changes%
    }%
  \fi
\endgroup
%    \end{macrocode}
%    \begin{macro}{\pc@placeboxes}
%    \begin{macrocode}
\renewcommand*{\pc@placeboxes}{%
  \global\let\@tempa\relax
  \count@\z@
  \hb@xt@\linewidth{%
    \vfuzz30ex %
    \vbadness\@M
    \splittopskip\z@skip
    \loop
    \ifnum\count@<\pc@columncount
      \advance\count@\@ne
      \expandafter\ifvoid\csname pc@column@\number\count@\endcsname
        \hskip\csname pc@column@width@\number\count@\endcsname
      \else
        \expandafter\setbox\expandafter\@tempboxa\expandafter
        \vsplit\csname pc@column@\number\count@\endcsname
            to \dp\strutbox
        \vbox{%
          \pcpc@SetCurrent\count@
          \unvbox\@tempboxa
        }%
      \fi
      \expandafter\ifvoid\csname pc@column@\number\count@\endcsname
      \else
        \global\let\@tempa\pc@placeboxes
      \fi
      \ifnum\count@<\pc@columncount
        \strut
        \hfill
        \ifpc@rulebetween
          \begingroup
            \pcpc@RuleBetweenColor
            \vrule
          \endgroup
          \hfill
        \fi
      \fi
    \repeat
  }%
  \@tempa
}
%    \end{macrocode}
%    \end{macro}
%    \begin{macro}{\pcpc@RuleBetweenColorDefault}
%    \begin{macrocode}
\def\pcpc@RuleBetweenColorDefault{%
  \normalcolor
}
%    \end{macrocode}
%    \end{macro}
%    \begin{macro}{\pcpc@RuleBetweenColor}
%    \begin{macrocode}
\let\pcpc@RuleBetweenColor\pcpc@RuleBetweenColorDefault
%    \end{macrocode}
%    \end{macro}
%    \begin{macrocode}
\define@key{parcolumns}{rulebetweencolor}{%
  \edef\pcpc@temp{#1}%
  \ifx\pcpc@temp\@empty
    \let\pcpc@RuleBetweenColor\pcpc@RuleBetweenColorDefault
  \else
    \edef\pcpc@temp{%
      \noexpand\@ifnextchar[{%
        \def\noexpand\pcpc@RuleBetweenColor{%
          \noexpand\color\pcpc@temp
        }%
        \noexpand\pcpc@GobbleNil
      }{%
        \def\noexpand\pcpc@RuleBetweenColor{%
          \noexpand\color{\pcpc@temp}%
        }%
        \noexpand\pcpc@GobbleNil
      }%
      \pcpc@temp\noexpand\@nil
    }%
    \pcpc@temp
  \fi
}
%    \end{macrocode}
%    \begin{macro}{\pcpc@GobbleNil}
%    \begin{macrocode}
\long\def\pcpc@GobbleNil#1\@nil{}
%    \end{macrocode}
%    \end{macro}
%
%    \begin{macrocode}
%</package>
%    \end{macrocode}
%% \section{Installation\\安装}
%
% \subsection{Download\\下载}
%
% \paragraph{Package.} This package is available on
% CTAN\footnote{\CTANpkg{pdfcolparcolumns}}:

%该软件包可从 CTAN\footnote{\CTANpkg{pdfcolparcolumns}} 下载:
% \begin{description}
% \item[\CTAN{macros/latex/contrib/oberdiek/pdfcolparcolumns.dtx}] The source file.
% \item[\CTAN{macros/latex/contrib/oberdiek/pdfcolparcolumns.pdf}] Documentation.
% \end{description}
%
%
% \paragraph{Bundle.} All the packages of the bundle `oberdiek'
% are also available in a TDS compliant ZIP archive. There
% the packages are already unpacked and the documentation files
% are generated. The files and directories obey the TDS standard.
%
%捆绑包“oberdiek”中的所有软件包也可在符合 TDS 标准的 ZIP 存档中获取。在该存档中,软件包已经被解包,文档文件已经生成,文件和目录符合 TDS 标准。
% \begin{description}
% \item[\CTANinstall{install/macros/latex/contrib/oberdiek.tds.zip}]
% \end{description}
% \emph{TDS} refers to the standard ``A Directory Structure
% for \TeX\ Files'' (\CTANpkg{tds}). Directories
% with \xfile{texmf} in their name are usually organized this way.
%
%\emph{TDS} 指的是标准“用于 \TeX\ 文件的目录结构”(\CTANpkg{tds})。名字中包含\xfile{texmf}的目录通常都是按这种方式组织的。
% \subsection{Bundle installation\\捆绑包安装}
%
% \paragraph{Unpacking.} Unpack the \xfile{oberdiek.tds.zip} in the
% TDS tree (also known as \xfile{texmf} tree) of your choice.
% Example (linux):
%
%\paragraph{解包。}在您选择的 TDS 树(也称为\xfile{texmf}树)中解压\xfile{oberdiek.tds.zip}。例如(在Linux中):
% \begin{quote}
%   |unzip oberdiek.tds.zip -d ~/texmf|
% \end{quote}
%
% \subsection{Package installation\\软件包安装}
%
% \paragraph{Unpacking.} The \xfile{.dtx} file is a self-extracting
% \docstrip\ archive. The files are extracted by running the
% \xfile{.dtx} through \plainTeX:
%
%\paragraph{解包。} \xfile{.dtx} 文件是一个自解压的 \docstrip\ 存档。运行\xfile{.dtx}通过\plainTeX\ 来提取文件:
% \begin{quote}
%   \verb|tex pdfcolparcolumns.dtx|
% \end{quote}
%
% \paragraph{TDS.} Now the different files must be moved into
% the different directories in your installation TDS tree
% (also known as \xfile{texmf} tree):
%
%\paragraph{TDS。}现在,不同的文件必须移动到安装 TDS 树(也称为\xfile{texmf}树)中的不同目录中:
% \begin{quote}
% \def\t{^^A
% \begin{tabular}{@{}>{\ttfamily}l@{ $\rightarrow$ }>{\ttfamily}l@{}}
%   pdfcolparcolumns.sty & tex/latex/oberdiek/pdfcolparcolumns.sty\\
%   pdfcolparcolumns.pdf & doc/latex/oberdiek/pdfcolparcolumns.pdf\\
%   pdfcolparcolumns.dtx & source/latex/oberdiek/pdfcolparcolumns.dtx\\
% \end{tabular}^^A
% }^^A
% \sbox0{\t}^^A
% \ifdim\wd0>\linewidth
%   \begingroup
%     \advance\linewidth by\leftmargin
%     \advance\linewidth by\rightmargin
%   \edef\x{\endgroup
%     \def\noexpand\lw{\the\linewidth}^^A
%   }\x
%   \def\lwbox{^^A
%     \leavevmode
%     \hbox to \linewidth{^^A
%       \kern-\leftmargin\relax
%       \hss
%       \usebox0
%       \hss
%       \kern-\rightmargin\relax
%     }^^A
%   }^^A
%   \ifdim\wd0>\lw
%     \sbox0{\small\t}^^A
%     \ifdim\wd0>\linewidth
%       \ifdim\wd0>\lw
%         \sbox0{\footnotesize\t}^^A
%         \ifdim\wd0>\linewidth
%           \ifdim\wd0>\lw
%             \sbox0{\scriptsize\t}^^A
%             \ifdim\wd0>\linewidth
%               \ifdim\wd0>\lw
%                 \sbox0{\tiny\t}^^A
%                 \ifdim\wd0>\linewidth
%                   \lwbox
%                 \else
%                   \usebox0
%                 \fi
%               \else
%                 \lwbox
%               \fi
%             \else
%               \usebox0
%             \fi
%           \else
%             \lwbox
%           \fi
%         \else
%           \usebox0
%         \fi
%       \else
%         \lwbox
%       \fi
%     \else
%       \usebox0
%     \fi
%   \else
%     \lwbox
%   \fi
% \else
%   \usebox0
% \fi
% \end{quote}
% If you have a \xfile{docstrip.cfg} that configures and enables \docstrip's
% TDS installing feature, then some files can already be in the right
% place, see the documentation of \docstrip.
%
%如果你有一个\xfile{docstrip.cfg}文件配置和启用了\docstrip 的TDS安装功能,那么一些文件可能已经位于正确的位置,具体请参见\docstrip 的文档。
% \subsection{Refresh file name databases\\刷新文件名数据库}
%
% If your \TeX~distribution
% (\TeX\,Live, \mikTeX, \dots) relies on file name databases, you must refresh
% these. For example, \TeX\,Live\ users run \verb|texhash| or
% \verb|mktexlsr|.
%
%如果你的\TeX~发行版(\TeX,Live、\mikTeX,等等)依赖于文件名数据库,你必须刷新它们。例如,\TeX,Live用户运行\verb|texhash|或\verb|mktexlsr|。
% \subsection{Some details for the interested\\一些细节供感兴趣的人使用}
%
% \paragraph{Unpacking with \LaTeX.}
% The \xfile{.dtx} chooses its action depending on the format:

%\paragraph{用\LaTeX 进行解包。} \xfile{.dtx}文件根据格式选择操作:
% \begin{description}
% \item[\plainTeX:] Run \docstrip\ and extract the files.
% \item[\LaTeX:] Generate the documentation.
% \end{description}
% If you insist on using \LaTeX\ for \docstrip\ (really,
% \docstrip\ does not need \LaTeX), then inform the autodetect routine
% about your intention:

%如果你坚持要用\LaTeX 进行\docstrip (其实\docstrip 不需要\LaTeX ),那么请告知自动检测例程你的意图:

% \begin{quote}
%   \verb|latex \let\install=y% \iffalse meta-comment
%
% File: pdfcolparcolumns.dtx
% Version: 2019/12/29 v1.5
% Info: Color stacks for parcolumns
%
% Copyright (C)
%    2007, 2008, 2010 Heiko Oberdiek
%    2016-2019 Oberdiek Package Support Group
%    https://github.com/ho-tex/oberdiek/issues
%
% This work may be distributed and/or modified under the
% conditions of the LaTeX Project Public License, either
% version 1.3c of this license or (at your option) any later
% version. This version of this license is in
%    https://www.latex-project.org/lppl/lppl-1-3c.txt
% and the latest version of this license is in
%    https://www.latex-project.org/lppl.txt
% and version 1.3 or later is part of all distributions of
% LaTeX version 2005/12/01 or later.
%
% This work has the LPPL maintenance status "maintained".
%
% The Current Maintainers of this work are
% Heiko Oberdiek and the Oberdiek Package Support Group
% https://github.com/ho-tex/oberdiek/issues
%
% This work consists of the main source file pdfcolparcolumns.dtx
% and the derived files
%    pdfcolparcolumns.sty, pdfcolparcolumns.pdf, pdfcolparcolumns.ins,
%    pdfcolparcolumns.drv, pdfcolparcolumns-test1.tex.
%
% Distribution:
%    CTAN:macros/latex/contrib/oberdiek/pdfcolparcolumns.dtx
%    CTAN:macros/latex/contrib/oberdiek/pdfcolparcolumns.pdf
%
% Unpacking:
%    (a) If pdfcolparcolumns.ins is present:
%           tex pdfcolparcolumns.ins
%    (b) Without pdfcolparcolumns.ins:
%           tex pdfcolparcolumns.dtx
%    (c) If you insist on using LaTeX
%           latex \let\install=y\input{pdfcolparcolumns.dtx}
%        (quote the arguments according to the demands of your shell)
%
% Documentation:
%    (a) If pdfcolparcolumns.drv is present:
%           latex pdfcolparcolumns.drv
%    (b) Without pdfcolparcolumns.drv:
%           latex pdfcolparcolumns.dtx; ...
%    The class ltxdoc loads the configuration file ltxdoc.cfg
%    if available. Here you can specify further options, e.g.
%    use A4 as paper format:
%       \PassOptionsToClass{a4paper}{article}
%
%    Programm calls to get the documentation (example):
%       pdflatex pdfcolparcolumns.dtx
%       makeindex -s gind.ist pdfcolparcolumns.idx
%       pdflatex pdfcolparcolumns.dtx
%       makeindex -s gind.ist pdfcolparcolumns.idx
%       pdflatex pdfcolparcolumns.dtx
%
% Installation:
%    TDS:tex/latex/oberdiek/pdfcolparcolumns.sty
%    TDS:doc/latex/oberdiek/pdfcolparcolumns.pdf
%    TDS:source/latex/oberdiek/pdfcolparcolumns.dtx
%
%<*ignore>
\begingroup
  \catcode123=1 %
  \catcode125=2 %
  \def\x{LaTeX2e}%
\expandafter\endgroup
\ifcase 0\ifx\install y1\fi\expandafter
         \ifx\csname processbatchFile\endcsname\relax\else1\fi
         \ifx\fmtname\x\else 1\fi\relax
\else\csname fi\endcsname
%</ignore>
%<*install>
\input docstrip.tex
\Msg{************************************************************************}
\Msg{* Installation}
\Msg{* Package: pdfcolparcolumns 2019/12/29 v1.5 Color stacks for parcolumns (HO)}
\Msg{************************************************************************}

\keepsilent
\askforoverwritefalse

\let\MetaPrefix\relax
\preamble

This is a generated file.

Project: pdfcolparcolumns
Version: 2019/12/29 v1.5

Copyright (C)
   2007, 2008, 2010 Heiko Oberdiek
   2016-2019 Oberdiek Package Support Group

This work may be distributed and/or modified under the
conditions of the LaTeX Project Public License, either
version 1.3c of this license or (at your option) any later
version. This version of this license is in
   https://www.latex-project.org/lppl/lppl-1-3c.txt
and the latest version of this license is in
   https://www.latex-project.org/lppl.txt
and version 1.3 or later is part of all distributions of
LaTeX version 2005/12/01 or later.

This work has the LPPL maintenance status "maintained".

The Current Maintainers of this work are
Heiko Oberdiek and the Oberdiek Package Support Group
https://github.com/ho-tex/oberdiek/issues


This work consists of the main source file pdfcolparcolumns.dtx
and the derived files
   pdfcolparcolumns.sty, pdfcolparcolumns.pdf, pdfcolparcolumns.ins,
   pdfcolparcolumns.drv, pdfcolparcolumns-test1.tex.

\endpreamble
\let\MetaPrefix\DoubleperCent

\generate{%
  \file{pdfcolparcolumns.ins}{\from{pdfcolparcolumns.dtx}{install}}%
  \file{pdfcolparcolumns.drv}{\from{pdfcolparcolumns.dtx}{driver}}%
  \usedir{tex/latex/oberdiek}%
  \file{pdfcolparcolumns.sty}{\from{pdfcolparcolumns.dtx}{package}}%
%  \usedir{doc/latex/oberdiek/test}%
%  \file{pdfcolparcolumns-test1.tex}{\from{pdfcolparcolumns.dtx}{test1}}%
}

\catcode32=13\relax% active space
\let =\space%
\Msg{************************************************************************}
\Msg{*}
\Msg{* To finish the installation you have to move the following}
\Msg{* file into a directory searched by TeX:}
\Msg{*}
\Msg{*     pdfcolparcolumns.sty}
\Msg{*}
\Msg{* To produce the documentation run the file `pdfcolparcolumns.drv'}
\Msg{* through LaTeX.}
\Msg{*}
\Msg{* Happy TeXing!}
\Msg{*}
\Msg{************************************************************************}

\endbatchfile
%</install>
%<*ignore>
\fi
%</ignore>
%<*driver>
\NeedsTeXFormat{LaTeX2e}
\ProvidesFile{pdfcolparcolumns.drv}%
  [2019/12/29 v1.5 Color stacks for parcolumns (HO)]%
\documentclass{ltxdoc}
\usepackage{holtxdoc}[2011/11/22]
\usepackage[scheme=plain]{ctex}
\setCJKmainfont{方正书宋_GBK}%方正书宋_GBK.TTF  设置文本的中文有衬线字体为“方正书宋_GBK”
\setCJKsansfont{方正黑体简体}%方正黑体_GBK.TTF  设置文本的中文无衬线字体为“方正黑体简体”
\setCJKmonofont{方正书宋简体}%方正仿宋_GBK.TTF  设置文本的中文等宽字体为“方正书宋简体”
\begin{document}
  \DocInput{pdfcolparcolumns.dtx}%
\end{document}
%</driver>
% \fi
%
%
%
% \GetFileInfo{pdfcolparcolumns.drv}
%
% \title{The \xpackage{pdfcolparcolumns} package}
% \date{2019/12/29 v1.5}
% \author{Heiko Oberdiek\thanks
% {Please report any issues at \url{https://github.com/ho-tex/oberdiek/issues}}\and 翻译\\{\tt virhuiai@qq.com}}
%
% \maketitle
%
% \begin{abstract}
%\makebox[0pt]{}
% \begin{parcolumns}[nofirstindent]{2}
%\colchunk{Since version 1.40 \pdfTeX\ supports several color stacks.
%This package uses them to fix color problems in
%package \xpackage{parcolumns}.}
%\colchunk{自从版本1.40起,\pdfTeX 支持多个颜色栈。
%此包使用它们来解决 \xpackage{parcolumns} 包中的颜色问题。}
%\end{parcolumns}



% \end{abstract}
%
% \tableofcontents
%
% \section{Usage\\用法}
%
% \begin{quote}
% |\usepackage{pdfcolparcolumns}|
% \end{quote}
% The package \xpackage{pdfcolparcolumns} loads package \xpackage{parcolums}
% \cite{parcolumns}. If color stacks are available then the
% macros of \xpackage{parcolumns} are patched to add support
% for color stacks.

%包 \xpackage{pdfcolparcolumns} 载入了 \xpackage{parcolumns} \cite{parcolumns} 包。如果有可用的颜色栈,则会对 \xpackage{parcolumns} 的宏进行修补,以添加对颜色栈的支持。

%
% \subsection{Option \xoption{rulebetweencolor}\\选项 \xoption{rulebetweencolor}}
%
% Package \xpackage{pdfcolparcolumns} also fixes the color for the
% rule between columns (if \xoption{rulebetween} is set).
% Default color is \cs{normalcolor}. But this can be changed by using
% option \xoption{rulebetweencolor}. It takes a color specification
% as value. If the value is empty, then the default (\cs{normalcolor})
% is used.
% Examples:

% 包 \xpackage{pdfcolparcolumns} 还修复了列之间分隔线的颜色(如果设置了 \xoption{rulebetween})。默认颜色为 \cs{normalcolor}。但是,可以使用选项 \xoption{rulebetweencolor} 来更改它。它接受一个颜色规范作为值。如果该值为空,则使用默认值(\cs{normalcolor})。例如:

% \begin{quote}
%   |rulebetweencolor=blue|,\\
%   |rulebetweencolor={red}|,\\
%   |rulebetweencolor={}|, \textit{\% \cs{normalcolor} is used}\\
%   |rulebetweencolor=[rgb]{1,0,.5}| \textit{\% see below}
% \end{quote}
% If used inside the optional argument of environment \textsf{parcolumns}
% and the value contains an optional argument, then whole value
% must be put in curly braces:

% 如果在 \textsf{parcolumns} 环境的可选参数中使用,并且值包含可选参数,则整个值必须用花括号括起来:
%\begin{quote}
%\begin{verbatim}
%\begin{parcolumns}[
%  rulebetween,
%  rulebetweencolor={[rgb]{1,0,.5}},
%]{2}
%  ...
%\end{parcolumns}
%\end{verbatim}
%\end{quote}
% This option \xoption{rulebetweencolor} can also be set using
% \cs{setkeys}:

%也可以使用 \cs{setkeys} 来设置选项 \xoption{rulebetweencolor}:
%\begin{quote}
%\begin{verbatim}
%\setkeys{parcolumns}{rulebetweencolor=green}
%\end{verbatim}
%\end{quote}
%
% \subsection{Future\\未来展望}
%
% Currently package \xpackage{parcolumns} does not seem to be
% maintained. Nevertheless if there will be a new version that
% adds support for color stacks, then this package may become
% obsolete.
%
%目前,\xpackage{parcolumns} 包似乎没有维护了。尽管如此,如果出现了新版本并支持颜色栈,那么这个包可能会变得过时。
% \StopEventually{
% }
%
% \section{Implementation\\实现}
%
% \subsection{Identification\\标识}
%
%    \begin{macrocode}
%<*package>
\NeedsTeXFormat{LaTeX2e}
\ProvidesPackage{pdfcolparcolumns}%
  [2019/12/29 v1.5 Color stacks for parcolumns (HO)]%
%    \end{macrocode}
%
% \subsection{Load packages\\加载宏包}
%
% \subsubsection{Package \xpackage{parcolumns}\\宏包 \xpackage{parcolumns}}
%
%    Currently package \xpackage{parcolumns} does not define options.
%    Thus it is just a precaution that the options of
%    package \xpackage{pdfcolparcolumns} are passed to
%    package \xpackage{parcolumns}.

%目前,宏包 \xpackage{parcolumns} 没有定义选项。因此,它只是一个预防措施,以确保将宏包 \xpackage{pdfcolparcolumns} 的选项传递给宏包 \xpackage{parcolumns}。
%    \begin{macrocode}
\DeclareOption*{%
  \PassOptionsToPackage{\CurrentOption}{parcolumns}%
}
\ProcessOptions\relax
\RequirePackage{parcolumns}[2004/11/25]
%    \end{macrocode}
%
% \subsubsection{Package \xpackage{pdfcol}\\宏包 \xpackage{pdfcol}}
%
%    \begin{macrocode}
\RequirePackage{pdfcol}[2007/09/09]
\ifpdfcolAvailable
\else
  \PackageInfo{pdfcolparcolumns}{%
    Loading aborted, because color stacks are not available%
  }%
  \expandafter\endinput
\fi
%    \end{macrocode}
%
% \subsubsection{Package \xpackage{infwarerr}\\宏包 \xpackage{infwarerr}}
%
%    \begin{macrocode}
\RequirePackage{infwarerr}[2007/09/09]
%    \end{macrocode}
%
% \subsection{Color stack macros\\颜色堆栈宏}
%
%    \begin{macro}{\pcpc@MaxStack}
%    Macro \cs{pcpc@MaxStack} holds the highest number of
%    allocated stacks.

%宏 \cs{pcpc@MaxStack} 存储已分配的堆栈中最大的编号。
%    \begin{macrocode}
\global\chardef\pcpc@MaxStack=\z@
%    \end{macrocode}
%    \end{macro}
%    \begin{macro}{\pcpc@InitStacks}
%    Macro \cs{pcpc@InitStacks} takes the number of columns
%    as argument and ensures that there are enough color
%    stacks for all columns.
%
%宏 \cs{pcpc@InitStacks} 接受列数作为参数,并确保为所有列分配了足够的颜色堆栈。
%    \begin{macrocode}
\def\pcpc@InitStacks#1{%
  \ifnum#1>\pcpc@MaxStack
    \begingroup
      \count@\pcpc@MaxStack
      \loop
        \advance\count@\@ne
        \pdfcolInitStack{pcpc@\the\count@}%
      \ifnum#1>\count@
      \repeat
      \global\chardef\pcpc@MaxStack=\count@
    \endgroup
  \fi
}
%    \end{macrocode}
%    \end{macro}
%
%    \begin{macro}{\pcpc@SwitchStack}
%    \begin{macrocode}
\def\pcpc@SwitchStack#1{%
  \pdfcolSwitchStack{pcpc@\number#1}%
}
%    \end{macrocode}
%    \end{macro}
%
%    \begin{macro}{\pcpc@SetCurrent}
%    \begin{macrocode}
\def\pcpc@SetCurrent#1{%
  \pdfcolSetCurrent{pcpc@\number#1}%
}
%    \end{macrocode}
%    \end{macro}
%
% \subsection{Patches\\补丁}
%
%     Now the color stack macros are patched into the macros
%     of package \xpackage{parcolumns}.
%
%现在颜色堆栈宏已经被打补丁到了 \xpackage{parcolumns} 宏包的宏中。
% \subsubsection{Init stacks\\初始化堆栈}
%
%    \cs{pcpc@InitStacks} should go into the definition of
%    environment |parcolumns|. \cs{pc@alloccolumns} is executed
%    there and nowhere else, thus we hook into it.

%\cs{pcpc@InitStacks} 应该被放在环境 |parcolumns| 的定义中。\cs{pc@alloccolumns} 仅在那里执行,因此我们将其连接起来。
%    \begin{macrocode}
\g@addto@macro\pc@alloccolumns{%
  \pcpc@InitStacks\pc@columncount
}
%    \end{macrocode}
%
% \subsubsection{Switch stack\\切换堆栈}
%
%    \cs{pcpc@SwitchStack} should be called by marco \cs{colchunk@}.
%    However it is easier to patch \cs{pc@setcolumnwidth} that
%    is executed in \cs{colchunk@} only.
%
%\cs{pcpc@SwitchStack} 应该被宏 \cs{colchunk@} 调用。不过,我们更容易修补仅在 \cs{colchunk@} 中执行的 \cs{pc@setcolumnwidth}。
%    \begin{macrocode}
\g@addto@macro\pc@setcolumnwidth{%
  \pcpc@SwitchStack\pc@columnctr
}
%    \end{macrocode}
%
% \subsubsection{Set current stack color\\设置当前堆栈颜色}
%
%    \cs{pcpc@SetCurrent} is set at the begin of each line.
%    It must be inserted into \cs{pc@placeboxes}. Unhappily
%    there is no easy way. Therefore we check and
%    redefine \cs{pc@placeboxes}.
%
%\cs{pcpc@SetCurrent} 在每行开始时设置。它必须插入到 \cs{pc@placeboxes} 中。不幸的是,没有简单的方法。因此,我们检查并重新定义 \cs{pc@placeboxes}。
%    \begin{macrocode}
\begingroup
  \def\x{%
    \global\let\@tempa\relax
    \count@\z@
    \hb@xt@\linewidth{%
      \vfuzz30ex %
      \vbadness\@M
      \splittopskip\z@skip
      \loop
      \ifnum\count@<\pc@columncount
        \advance\count@\@ne
        \expandafter\ifvoid\csname pc@column@\number\count@\endcsname
          \hskip\csname pc@column@width@\number\count@\endcsname
        \else
          \expandafter\setbox\expandafter\@tempboxa\expandafter
          \vsplit\csname pc@column@\number\count@\endcsname
              to \dp\strutbox
          \vbox{%
            \unvbox\@tempboxa
          }%
        \fi
        \expandafter\ifvoid\csname pc@column@\number\count@\endcsname
        \else
          \global\let\@tempa\pc@placeboxes
        \fi
        \ifnum\count@<\pc@columncount
          \strut
          \hfill
          \ifpc@rulebetween
            \vrule
            \hfill
          \fi
        \fi
      \repeat
    }%
    \@tempa
  }%
  \ifx\x\pc@placeboxes
  \else
    \@PackageWarningNoLine{pdfcolparcolumns}{%
      Command \string\pc@placeboxes\space has changed.\MessageBreak
      Supported versions of package `parcolumns':\MessageBreak
      \space\space 2004/08/05.\MessageBreak
      The redefinition of \string\pc@placeboxes\space may not%
      \MessageBreak
      behave correctly depending on the changes%
    }%
  \fi
\endgroup
%    \end{macrocode}
%    \begin{macro}{\pc@placeboxes}
%    \begin{macrocode}
\renewcommand*{\pc@placeboxes}{%
  \global\let\@tempa\relax
  \count@\z@
  \hb@xt@\linewidth{%
    \vfuzz30ex %
    \vbadness\@M
    \splittopskip\z@skip
    \loop
    \ifnum\count@<\pc@columncount
      \advance\count@\@ne
      \expandafter\ifvoid\csname pc@column@\number\count@\endcsname
        \hskip\csname pc@column@width@\number\count@\endcsname
      \else
        \expandafter\setbox\expandafter\@tempboxa\expandafter
        \vsplit\csname pc@column@\number\count@\endcsname
            to \dp\strutbox
        \vbox{%
          \pcpc@SetCurrent\count@
          \unvbox\@tempboxa
        }%
      \fi
      \expandafter\ifvoid\csname pc@column@\number\count@\endcsname
      \else
        \global\let\@tempa\pc@placeboxes
      \fi
      \ifnum\count@<\pc@columncount
        \strut
        \hfill
        \ifpc@rulebetween
          \begingroup
            \pcpc@RuleBetweenColor
            \vrule
          \endgroup
          \hfill
        \fi
      \fi
    \repeat
  }%
  \@tempa
}
%    \end{macrocode}
%    \end{macro}
%    \begin{macro}{\pcpc@RuleBetweenColorDefault}
%    \begin{macrocode}
\def\pcpc@RuleBetweenColorDefault{%
  \normalcolor
}
%    \end{macrocode}
%    \end{macro}
%    \begin{macro}{\pcpc@RuleBetweenColor}
%    \begin{macrocode}
\let\pcpc@RuleBetweenColor\pcpc@RuleBetweenColorDefault
%    \end{macrocode}
%    \end{macro}
%    \begin{macrocode}
\define@key{parcolumns}{rulebetweencolor}{%
  \edef\pcpc@temp{#1}%
  \ifx\pcpc@temp\@empty
    \let\pcpc@RuleBetweenColor\pcpc@RuleBetweenColorDefault
  \else
    \edef\pcpc@temp{%
      \noexpand\@ifnextchar[{%
        \def\noexpand\pcpc@RuleBetweenColor{%
          \noexpand\color\pcpc@temp
        }%
        \noexpand\pcpc@GobbleNil
      }{%
        \def\noexpand\pcpc@RuleBetweenColor{%
          \noexpand\color{\pcpc@temp}%
        }%
        \noexpand\pcpc@GobbleNil
      }%
      \pcpc@temp\noexpand\@nil
    }%
    \pcpc@temp
  \fi
}
%    \end{macrocode}
%    \begin{macro}{\pcpc@GobbleNil}
%    \begin{macrocode}
\long\def\pcpc@GobbleNil#1\@nil{}
%    \end{macrocode}
%    \end{macro}
%
%    \begin{macrocode}
%</package>
%    \end{macrocode}
%% \section{Installation\\安装}
%
% \subsection{Download\\下载}
%
% \paragraph{Package.} This package is available on
% CTAN\footnote{\CTANpkg{pdfcolparcolumns}}:

%该软件包可从 CTAN\footnote{\CTANpkg{pdfcolparcolumns}} 下载:
% \begin{description}
% \item[\CTAN{macros/latex/contrib/oberdiek/pdfcolparcolumns.dtx}] The source file.
% \item[\CTAN{macros/latex/contrib/oberdiek/pdfcolparcolumns.pdf}] Documentation.
% \end{description}
%
%
% \paragraph{Bundle.} All the packages of the bundle `oberdiek'
% are also available in a TDS compliant ZIP archive. There
% the packages are already unpacked and the documentation files
% are generated. The files and directories obey the TDS standard.
%
%捆绑包“oberdiek”中的所有软件包也可在符合 TDS 标准的 ZIP 存档中获取。在该存档中,软件包已经被解包,文档文件已经生成,文件和目录符合 TDS 标准。
% \begin{description}
% \item[\CTANinstall{install/macros/latex/contrib/oberdiek.tds.zip}]
% \end{description}
% \emph{TDS} refers to the standard ``A Directory Structure
% for \TeX\ Files'' (\CTANpkg{tds}). Directories
% with \xfile{texmf} in their name are usually organized this way.
%
%\emph{TDS} 指的是标准“用于 \TeX\ 文件的目录结构”(\CTANpkg{tds})。名字中包含\xfile{texmf}的目录通常都是按这种方式组织的。
% \subsection{Bundle installation\\捆绑包安装}
%
% \paragraph{Unpacking.} Unpack the \xfile{oberdiek.tds.zip} in the
% TDS tree (also known as \xfile{texmf} tree) of your choice.
% Example (linux):
%
%\paragraph{解包。}在您选择的 TDS 树(也称为\xfile{texmf}树)中解压\xfile{oberdiek.tds.zip}。例如(在Linux中):
% \begin{quote}
%   |unzip oberdiek.tds.zip -d ~/texmf|
% \end{quote}
%
% \subsection{Package installation\\软件包安装}
%
% \paragraph{Unpacking.} The \xfile{.dtx} file is a self-extracting
% \docstrip\ archive. The files are extracted by running the
% \xfile{.dtx} through \plainTeX:
%
%\paragraph{解包。} \xfile{.dtx} 文件是一个自解压的 \docstrip\ 存档。运行\xfile{.dtx}通过\plainTeX\ 来提取文件:
% \begin{quote}
%   \verb|tex pdfcolparcolumns.dtx|
% \end{quote}
%
% \paragraph{TDS.} Now the different files must be moved into
% the different directories in your installation TDS tree
% (also known as \xfile{texmf} tree):
%
%\paragraph{TDS。}现在,不同的文件必须移动到安装 TDS 树(也称为\xfile{texmf}树)中的不同目录中:
% \begin{quote}
% \def\t{^^A
% \begin{tabular}{@{}>{\ttfamily}l@{ $\rightarrow$ }>{\ttfamily}l@{}}
%   pdfcolparcolumns.sty & tex/latex/oberdiek/pdfcolparcolumns.sty\\
%   pdfcolparcolumns.pdf & doc/latex/oberdiek/pdfcolparcolumns.pdf\\
%   pdfcolparcolumns.dtx & source/latex/oberdiek/pdfcolparcolumns.dtx\\
% \end{tabular}^^A
% }^^A
% \sbox0{\t}^^A
% \ifdim\wd0>\linewidth
%   \begingroup
%     \advance\linewidth by\leftmargin
%     \advance\linewidth by\rightmargin
%   \edef\x{\endgroup
%     \def\noexpand\lw{\the\linewidth}^^A
%   }\x
%   \def\lwbox{^^A
%     \leavevmode
%     \hbox to \linewidth{^^A
%       \kern-\leftmargin\relax
%       \hss
%       \usebox0
%       \hss
%       \kern-\rightmargin\relax
%     }^^A
%   }^^A
%   \ifdim\wd0>\lw
%     \sbox0{\small\t}^^A
%     \ifdim\wd0>\linewidth
%       \ifdim\wd0>\lw
%         \sbox0{\footnotesize\t}^^A
%         \ifdim\wd0>\linewidth
%           \ifdim\wd0>\lw
%             \sbox0{\scriptsize\t}^^A
%             \ifdim\wd0>\linewidth
%               \ifdim\wd0>\lw
%                 \sbox0{\tiny\t}^^A
%                 \ifdim\wd0>\linewidth
%                   \lwbox
%                 \else
%                   \usebox0
%                 \fi
%               \else
%                 \lwbox
%               \fi
%             \else
%               \usebox0
%             \fi
%           \else
%             \lwbox
%           \fi
%         \else
%           \usebox0
%         \fi
%       \else
%         \lwbox
%       \fi
%     \else
%       \usebox0
%     \fi
%   \else
%     \lwbox
%   \fi
% \else
%   \usebox0
% \fi
% \end{quote}
% If you have a \xfile{docstrip.cfg} that configures and enables \docstrip's
% TDS installing feature, then some files can already be in the right
% place, see the documentation of \docstrip.
%
%如果你有一个\xfile{docstrip.cfg}文件配置和启用了\docstrip 的TDS安装功能,那么一些文件可能已经位于正确的位置,具体请参见\docstrip 的文档。
% \subsection{Refresh file name databases\\刷新文件名数据库}
%
% If your \TeX~distribution
% (\TeX\,Live, \mikTeX, \dots) relies on file name databases, you must refresh
% these. For example, \TeX\,Live\ users run \verb|texhash| or
% \verb|mktexlsr|.
%
%如果你的\TeX~发行版(\TeX,Live、\mikTeX,等等)依赖于文件名数据库,你必须刷新它们。例如,\TeX,Live用户运行\verb|texhash|或\verb|mktexlsr|。
% \subsection{Some details for the interested\\一些细节供感兴趣的人使用}
%
% \paragraph{Unpacking with \LaTeX.}
% The \xfile{.dtx} chooses its action depending on the format:

%\paragraph{用\LaTeX 进行解包。} \xfile{.dtx}文件根据格式选择操作:
% \begin{description}
% \item[\plainTeX:] Run \docstrip\ and extract the files.
% \item[\LaTeX:] Generate the documentation.
% \end{description}
% If you insist on using \LaTeX\ for \docstrip\ (really,
% \docstrip\ does not need \LaTeX), then inform the autodetect routine
% about your intention:

%如果你坚持要用\LaTeX 进行\docstrip (其实\docstrip 不需要\LaTeX ),那么请告知自动检测例程你的意图:

% \begin{quote}
%   \verb|latex \let\install=y\input{pdfcolparcolumns.dtx}|
% \end{quote}
% Do not forget to quote the argument according to the demands
% of your shell.

%不要忘记根据你的shell的要求引用参数。
%
% \paragraph{Generating the documentation.}
% You can use both the \xfile{.dtx} or the \xfile{.drv} to generate
% the documentation. The process can be configured by the
% configuration file \xfile{ltxdoc.cfg}. For instance, put this
% line into this file, if you want to have A4 as paper format:
%
%\paragraph{生成文档。} 你可以使用\xfile{.dtx}或\xfile{.drv}生成文档。这个过程可以由配置文件\xfile{ltxdoc.cfg}配置。例如,如果你想要A4作为纸张格式,将此行放入文件中:

% \begin{quote}
%   \verb|\PassOptionsToClass{a4paper}{article}|
% \end{quote}
% An example follows how to generate the
% documentation with pdf\LaTeX:

%以下是使用pdf\LaTeX 生成文档的示例:
% \begin{quote}
%\begin{verbatim}
%pdflatex pdfcolparcolumns.dtx
%makeindex -s gind.ist pdfcolparcolumns.idx
%pdflatex pdfcolparcolumns.dtx
%makeindex -s gind.ist pdfcolparcolumns.idx
%pdflatex pdfcolparcolumns.dtx
%\end{verbatim}
% \end{quote}
%
% \begin{thebibliography}{9}
%
% \bibitem{parcolumns}
%   Jonathan Sauer: \textit{The \xpackage{parcolumns} package};
%   2004/11/25;\\
%   \CTANpkg{parcolumns}.
%
% \bibitem{pdfcol}
%   Heiko Oberdiek: \textit{The \xpackage{pdfcol} package};
%   2007/09/09;\\
%   \CTANpkg{pdfcol}.
%
% \end{thebibliography}
%
% \begin{History}
%   \begin{Version}{2007/07/26 v1.0}
%   \item
%     First version, published in the newsgroup \xnewsgroup{comp.text.tex}
%     with the name \xpackage{parcolumns-colorstacks}: ^^A no line break
%     \URL{``\link{Re: \xpackage{xcolor} glitches}''}^^A
%     {https://groups.google.com/group/comp.text.tex/msg/56bd897b11bca414}

%第一个版本,以 \xnewsgroup{comp.text.tex} 新闻组发布,名称为 \xpackage{parcolumns-colorstacks}:^^A 没有换行符 
%\URL{``\link{Re: \xpackage{xcolor} glitches}''}^^A
%{https://groups.google.com/group/comp.text.tex/msg/56bd897b11bca414}
%   \end{Version}
%   \begin{Version}{2007/09/09 v1.1}
%   \item
%     CTAN version, package name renamed to \xpackage{pdfcolparcolumns}.
%
%CTAN 版本,将包名改为 \xpackage{pdfcolparcolumns}。
%   \item
%     Uses package \xpackage{pdfcol}.
%
%使用 \xpackage{pdfcol} 包。
%   \item
%     Documentation added.
%
%添加文档。
%   \item
%     Test file added.
%
%添加测试文件。
%   \end{Version}
%   \begin{Version}{2008/08/11 v1.2}
%   \item
%     Code is not changed.
%
%代码未改动。
%   \item
%     URLs updated.
%
%更新 URL。
%   \end{Version}
%   \begin{Version}{2010/01/11 v1.3}
%   \item
%     Fix for rule color.
%
%修复分隔线的颜色。
%   \item
%     New option \xoption{rulebetweencolor} for environment |parcolumns|.
%
%为环境 |parcolumns| 添加新选项 \xoption{rulebetweencolor}。
%   \end{Version}
%   \begin{Version}{2016/05/16 v1.4}
%   \item
%     Documentation updates.
%
%更新文档。
%   \end{Version}
%   \begin{Version}{2019/12/29 v1.5}
%   \item
%     \cs{PassOptionsToPackage} not \cs{PassoptionsToPackage}
%
%\cs{PassOptionsToPackage} 改为不区分大小写的 \cs{PassoptionsToPackage}。
%   \end{Version}
% \end{History}
%
% \PrintIndex
%
% \Finale
\endinput
|
% \end{quote}
% Do not forget to quote the argument according to the demands
% of your shell.

%不要忘记根据你的shell的要求引用参数。
%
% \paragraph{Generating the documentation.}
% You can use both the \xfile{.dtx} or the \xfile{.drv} to generate
% the documentation. The process can be configured by the
% configuration file \xfile{ltxdoc.cfg}. For instance, put this
% line into this file, if you want to have A4 as paper format:
%
%\paragraph{生成文档。} 你可以使用\xfile{.dtx}或\xfile{.drv}生成文档。这个过程可以由配置文件\xfile{ltxdoc.cfg}配置。例如,如果你想要A4作为纸张格式,将此行放入文件中:

% \begin{quote}
%   \verb|\PassOptionsToClass{a4paper}{article}|
% \end{quote}
% An example follows how to generate the
% documentation with pdf\LaTeX:

%以下是使用pdf\LaTeX 生成文档的示例:
% \begin{quote}
%\begin{verbatim}
%pdflatex pdfcolparcolumns.dtx
%makeindex -s gind.ist pdfcolparcolumns.idx
%pdflatex pdfcolparcolumns.dtx
%makeindex -s gind.ist pdfcolparcolumns.idx
%pdflatex pdfcolparcolumns.dtx
%\end{verbatim}
% \end{quote}
%
% \begin{thebibliography}{9}
%
% \bibitem{parcolumns}
%   Jonathan Sauer: \textit{The \xpackage{parcolumns} package};
%   2004/11/25;\\
%   \CTANpkg{parcolumns}.
%
% \bibitem{pdfcol}
%   Heiko Oberdiek: \textit{The \xpackage{pdfcol} package};
%   2007/09/09;\\
%   \CTANpkg{pdfcol}.
%
% \end{thebibliography}
%
% \begin{History}
%   \begin{Version}{2007/07/26 v1.0}
%   \item
%     First version, published in the newsgroup \xnewsgroup{comp.text.tex}
%     with the name \xpackage{parcolumns-colorstacks}: ^^A no line break
%     \URL{``\link{Re: \xpackage{xcolor} glitches}''}^^A
%     {https://groups.google.com/group/comp.text.tex/msg/56bd897b11bca414}

%第一个版本,以 \xnewsgroup{comp.text.tex} 新闻组发布,名称为 \xpackage{parcolumns-colorstacks}:^^A 没有换行符 
%\URL{``\link{Re: \xpackage{xcolor} glitches}''}^^A
%{https://groups.google.com/group/comp.text.tex/msg/56bd897b11bca414}
%   \end{Version}
%   \begin{Version}{2007/09/09 v1.1}
%   \item
%     CTAN version, package name renamed to \xpackage{pdfcolparcolumns}.
%
%CTAN 版本,将包名改为 \xpackage{pdfcolparcolumns}。
%   \item
%     Uses package \xpackage{pdfcol}.
%
%使用 \xpackage{pdfcol} 包。
%   \item
%     Documentation added.
%
%添加文档。
%   \item
%     Test file added.
%
%添加测试文件。
%   \end{Version}
%   \begin{Version}{2008/08/11 v1.2}
%   \item
%     Code is not changed.
%
%代码未改动。
%   \item
%     URLs updated.
%
%更新 URL。
%   \end{Version}
%   \begin{Version}{2010/01/11 v1.3}
%   \item
%     Fix for rule color.
%
%修复分隔线的颜色。
%   \item
%     New option \xoption{rulebetweencolor} for environment |parcolumns|.
%
%为环境 |parcolumns| 添加新选项 \xoption{rulebetweencolor}。
%   \end{Version}
%   \begin{Version}{2016/05/16 v1.4}
%   \item
%     Documentation updates.
%
%更新文档。
%   \end{Version}
%   \begin{Version}{2019/12/29 v1.5}
%   \item
%     \cs{PassOptionsToPackage} not \cs{PassoptionsToPackage}
%
%\cs{PassOptionsToPackage} 改为不区分大小写的 \cs{PassoptionsToPackage}。
%   \end{Version}
% \end{History}
%
% \PrintIndex
%
% \Finale
\endinput

%        (quote the arguments according to the demands of your shell)
%
% Documentation:
%    (a) If pdfcolparcolumns.drv is present:
%           latex pdfcolparcolumns.drv
%    (b) Without pdfcolparcolumns.drv:
%           latex pdfcolparcolumns.dtx; ...
%    The class ltxdoc loads the configuration file ltxdoc.cfg
%    if available. Here you can specify further options, e.g.
%    use A4 as paper format:
%       \PassOptionsToClass{a4paper}{article}
%
%    Programm calls to get the documentation (example):
%       pdflatex pdfcolparcolumns.dtx
%       makeindex -s gind.ist pdfcolparcolumns.idx
%       pdflatex pdfcolparcolumns.dtx
%       makeindex -s gind.ist pdfcolparcolumns.idx
%       pdflatex pdfcolparcolumns.dtx
%
% Installation:
%    TDS:tex/latex/oberdiek/pdfcolparcolumns.sty
%    TDS:doc/latex/oberdiek/pdfcolparcolumns.pdf
%    TDS:source/latex/oberdiek/pdfcolparcolumns.dtx
%
%<*ignore>
\begingroup
  \catcode123=1 %
  \catcode125=2 %
  \def\x{LaTeX2e}%
\expandafter\endgroup
\ifcase 0\ifx\install y1\fi\expandafter
         \ifx\csname processbatchFile\endcsname\relax\else1\fi
         \ifx\fmtname\x\else 1\fi\relax
\else\csname fi\endcsname
%</ignore>
%<*install>
\input docstrip.tex
\Msg{************************************************************************}
\Msg{* Installation}
\Msg{* Package: pdfcolparcolumns 2019/12/29 v1.5 Color stacks for parcolumns (HO)}
\Msg{************************************************************************}

\keepsilent
\askforoverwritefalse

\let\MetaPrefix\relax
\preamble

This is a generated file.

Project: pdfcolparcolumns
Version: 2019/12/29 v1.5

Copyright (C)
   2007, 2008, 2010 Heiko Oberdiek
   2016-2019 Oberdiek Package Support Group

This work may be distributed and/or modified under the
conditions of the LaTeX Project Public License, either
version 1.3c of this license or (at your option) any later
version. This version of this license is in
   https://www.latex-project.org/lppl/lppl-1-3c.txt
and the latest version of this license is in
   https://www.latex-project.org/lppl.txt
and version 1.3 or later is part of all distributions of
LaTeX version 2005/12/01 or later.

This work has the LPPL maintenance status "maintained".

The Current Maintainers of this work are
Heiko Oberdiek and the Oberdiek Package Support Group
https://github.com/ho-tex/oberdiek/issues


This work consists of the main source file pdfcolparcolumns.dtx
and the derived files
   pdfcolparcolumns.sty, pdfcolparcolumns.pdf, pdfcolparcolumns.ins,
   pdfcolparcolumns.drv, pdfcolparcolumns-test1.tex.

\endpreamble
\let\MetaPrefix\DoubleperCent

\generate{%
  \file{pdfcolparcolumns.ins}{\from{pdfcolparcolumns.dtx}{install}}%
  \file{pdfcolparcolumns.drv}{\from{pdfcolparcolumns.dtx}{driver}}%
  \usedir{tex/latex/oberdiek}%
  \file{pdfcolparcolumns.sty}{\from{pdfcolparcolumns.dtx}{package}}%
%  \usedir{doc/latex/oberdiek/test}%
%  \file{pdfcolparcolumns-test1.tex}{\from{pdfcolparcolumns.dtx}{test1}}%
}

\catcode32=13\relax% active space
\let =\space%
\Msg{************************************************************************}
\Msg{*}
\Msg{* To finish the installation you have to move the following}
\Msg{* file into a directory searched by TeX:}
\Msg{*}
\Msg{*     pdfcolparcolumns.sty}
\Msg{*}
\Msg{* To produce the documentation run the file `pdfcolparcolumns.drv'}
\Msg{* through LaTeX.}
\Msg{*}
\Msg{* Happy TeXing!}
\Msg{*}
\Msg{************************************************************************}

\endbatchfile
%</install>
%<*ignore>
\fi
%</ignore>
%<*driver>
\NeedsTeXFormat{LaTeX2e}
\ProvidesFile{pdfcolparcolumns.drv}%
  [2019/12/29 v1.5 Color stacks for parcolumns (HO)]%
\documentclass{ltxdoc}
\usepackage{holtxdoc}[2011/11/22]
\usepackage[scheme=plain]{ctex}
\setCJKmainfont{方正书宋_GBK}%方正书宋_GBK.TTF  设置文本的中文有衬线字体为“方正书宋_GBK”
\setCJKsansfont{方正黑体简体}%方正黑体_GBK.TTF  设置文本的中文无衬线字体为“方正黑体简体”
\setCJKmonofont{方正书宋简体}%方正仿宋_GBK.TTF  设置文本的中文等宽字体为“方正书宋简体”
\begin{document}
  \DocInput{pdfcolparcolumns.dtx}%
\end{document}
%</driver>
% \fi
%
%
%
% \GetFileInfo{pdfcolparcolumns.drv}
%
% \title{The \xpackage{pdfcolparcolumns} package}
% \date{2019/12/29 v1.5}
% \author{Heiko Oberdiek\thanks
% {Please report any issues at \url{https://github.com/ho-tex/oberdiek/issues}}\and 翻译\\{\tt virhuiai@qq.com}}
%
% \maketitle
%
% \begin{abstract}
%\makebox[0pt]{}
% \begin{parcolumns}[nofirstindent]{2}
%\colchunk{Since version 1.40 \pdfTeX\ supports several color stacks.
%This package uses them to fix color problems in
%package \xpackage{parcolumns}.}
%\colchunk{自从版本1.40起,\pdfTeX 支持多个颜色栈。
%此包使用它们来解决 \xpackage{parcolumns} 包中的颜色问题。}
%\end{parcolumns}



% \end{abstract}
%
% \tableofcontents
%
% \section{Usage\\用法}
%
% \begin{quote}
% |\usepackage{pdfcolparcolumns}|
% \end{quote}
% The package \xpackage{pdfcolparcolumns} loads package \xpackage{parcolums}
% \cite{parcolumns}. If color stacks are available then the
% macros of \xpackage{parcolumns} are patched to add support
% for color stacks.

%包 \xpackage{pdfcolparcolumns} 载入了 \xpackage{parcolumns} \cite{parcolumns} 包。如果有可用的颜色栈,则会对 \xpackage{parcolumns} 的宏进行修补,以添加对颜色栈的支持。

%
% \subsection{Option \xoption{rulebetweencolor}\\选项 \xoption{rulebetweencolor}}
%
% Package \xpackage{pdfcolparcolumns} also fixes the color for the
% rule between columns (if \xoption{rulebetween} is set).
% Default color is \cs{normalcolor}. But this can be changed by using
% option \xoption{rulebetweencolor}. It takes a color specification
% as value. If the value is empty, then the default (\cs{normalcolor})
% is used.
% Examples:

% 包 \xpackage{pdfcolparcolumns} 还修复了列之间分隔线的颜色(如果设置了 \xoption{rulebetween})。默认颜色为 \cs{normalcolor}。但是,可以使用选项 \xoption{rulebetweencolor} 来更改它。它接受一个颜色规范作为值。如果该值为空,则使用默认值(\cs{normalcolor})。例如:

% \begin{quote}
%   |rulebetweencolor=blue|,\\
%   |rulebetweencolor={red}|,\\
%   |rulebetweencolor={}|, \textit{\% \cs{normalcolor} is used}\\
%   |rulebetweencolor=[rgb]{1,0,.5}| \textit{\% see below}
% \end{quote}
% If used inside the optional argument of environment \textsf{parcolumns}
% and the value contains an optional argument, then whole value
% must be put in curly braces:

% 如果在 \textsf{parcolumns} 环境的可选参数中使用,并且值包含可选参数,则整个值必须用花括号括起来:
%\begin{quote}
%\begin{verbatim}
%\begin{parcolumns}[
%  rulebetween,
%  rulebetweencolor={[rgb]{1,0,.5}},
%]{2}
%  ...
%\end{parcolumns}
%\end{verbatim}
%\end{quote}
% This option \xoption{rulebetweencolor} can also be set using
% \cs{setkeys}:

%也可以使用 \cs{setkeys} 来设置选项 \xoption{rulebetweencolor}:
%\begin{quote}
%\begin{verbatim}
%\setkeys{parcolumns}{rulebetweencolor=green}
%\end{verbatim}
%\end{quote}
%
% \subsection{Future\\未来展望}
%
% Currently package \xpackage{parcolumns} does not seem to be
% maintained. Nevertheless if there will be a new version that
% adds support for color stacks, then this package may become
% obsolete.
%
%目前,\xpackage{parcolumns} 包似乎没有维护了。尽管如此,如果出现了新版本并支持颜色栈,那么这个包可能会变得过时。
% \StopEventually{
% }
%
% \section{Implementation\\实现}
%
% \subsection{Identification\\标识}
%
%    \begin{macrocode}
%<*package>
\NeedsTeXFormat{LaTeX2e}
\ProvidesPackage{pdfcolparcolumns}%
  [2019/12/29 v1.5 Color stacks for parcolumns (HO)]%
%    \end{macrocode}
%
% \subsection{Load packages\\加载宏包}
%
% \subsubsection{Package \xpackage{parcolumns}\\宏包 \xpackage{parcolumns}}
%
%    Currently package \xpackage{parcolumns} does not define options.
%    Thus it is just a precaution that the options of
%    package \xpackage{pdfcolparcolumns} are passed to
%    package \xpackage{parcolumns}.

%目前,宏包 \xpackage{parcolumns} 没有定义选项。因此,它只是一个预防措施,以确保将宏包 \xpackage{pdfcolparcolumns} 的选项传递给宏包 \xpackage{parcolumns}。
%    \begin{macrocode}
\DeclareOption*{%
  \PassOptionsToPackage{\CurrentOption}{parcolumns}%
}
\ProcessOptions\relax
\RequirePackage{parcolumns}[2004/11/25]
%    \end{macrocode}
%
% \subsubsection{Package \xpackage{pdfcol}\\宏包 \xpackage{pdfcol}}
%
%    \begin{macrocode}
\RequirePackage{pdfcol}[2007/09/09]
\ifpdfcolAvailable
\else
  \PackageInfo{pdfcolparcolumns}{%
    Loading aborted, because color stacks are not available%
  }%
  \expandafter\endinput
\fi
%    \end{macrocode}
%
% \subsubsection{Package \xpackage{infwarerr}\\宏包 \xpackage{infwarerr}}
%
%    \begin{macrocode}
\RequirePackage{infwarerr}[2007/09/09]
%    \end{macrocode}
%
% \subsection{Color stack macros\\颜色堆栈宏}
%
%    \begin{macro}{\pcpc@MaxStack}
%    Macro \cs{pcpc@MaxStack} holds the highest number of
%    allocated stacks.

%宏 \cs{pcpc@MaxStack} 存储已分配的堆栈中最大的编号。
%    \begin{macrocode}
\global\chardef\pcpc@MaxStack=\z@
%    \end{macrocode}
%    \end{macro}
%    \begin{macro}{\pcpc@InitStacks}
%    Macro \cs{pcpc@InitStacks} takes the number of columns
%    as argument and ensures that there are enough color
%    stacks for all columns.
%
%宏 \cs{pcpc@InitStacks} 接受列数作为参数,并确保为所有列分配了足够的颜色堆栈。
%    \begin{macrocode}
\def\pcpc@InitStacks#1{%
  \ifnum#1>\pcpc@MaxStack
    \begingroup
      \count@\pcpc@MaxStack
      \loop
        \advance\count@\@ne
        \pdfcolInitStack{pcpc@\the\count@}%
      \ifnum#1>\count@
      \repeat
      \global\chardef\pcpc@MaxStack=\count@
    \endgroup
  \fi
}
%    \end{macrocode}
%    \end{macro}
%
%    \begin{macro}{\pcpc@SwitchStack}
%    \begin{macrocode}
\def\pcpc@SwitchStack#1{%
  \pdfcolSwitchStack{pcpc@\number#1}%
}
%    \end{macrocode}
%    \end{macro}
%
%    \begin{macro}{\pcpc@SetCurrent}
%    \begin{macrocode}
\def\pcpc@SetCurrent#1{%
  \pdfcolSetCurrent{pcpc@\number#1}%
}
%    \end{macrocode}
%    \end{macro}
%
% \subsection{Patches\\补丁}
%
%     Now the color stack macros are patched into the macros
%     of package \xpackage{parcolumns}.
%
%现在颜色堆栈宏已经被打补丁到了 \xpackage{parcolumns} 宏包的宏中。
% \subsubsection{Init stacks\\初始化堆栈}
%
%    \cs{pcpc@InitStacks} should go into the definition of
%    environment |parcolumns|. \cs{pc@alloccolumns} is executed
%    there and nowhere else, thus we hook into it.

%\cs{pcpc@InitStacks} 应该被放在环境 |parcolumns| 的定义中。\cs{pc@alloccolumns} 仅在那里执行,因此我们将其连接起来。
%    \begin{macrocode}
\g@addto@macro\pc@alloccolumns{%
  \pcpc@InitStacks\pc@columncount
}
%    \end{macrocode}
%
% \subsubsection{Switch stack\\切换堆栈}
%
%    \cs{pcpc@SwitchStack} should be called by marco \cs{colchunk@}.
%    However it is easier to patch \cs{pc@setcolumnwidth} that
%    is executed in \cs{colchunk@} only.
%
%\cs{pcpc@SwitchStack} 应该被宏 \cs{colchunk@} 调用。不过,我们更容易修补仅在 \cs{colchunk@} 中执行的 \cs{pc@setcolumnwidth}。
%    \begin{macrocode}
\g@addto@macro\pc@setcolumnwidth{%
  \pcpc@SwitchStack\pc@columnctr
}
%    \end{macrocode}
%
% \subsubsection{Set current stack color\\设置当前堆栈颜色}
%
%    \cs{pcpc@SetCurrent} is set at the begin of each line.
%    It must be inserted into \cs{pc@placeboxes}. Unhappily
%    there is no easy way. Therefore we check and
%    redefine \cs{pc@placeboxes}.
%
%\cs{pcpc@SetCurrent} 在每行开始时设置。它必须插入到 \cs{pc@placeboxes} 中。不幸的是,没有简单的方法。因此,我们检查并重新定义 \cs{pc@placeboxes}。
%    \begin{macrocode}
\begingroup
  \def\x{%
    \global\let\@tempa\relax
    \count@\z@
    \hb@xt@\linewidth{%
      \vfuzz30ex %
      \vbadness\@M
      \splittopskip\z@skip
      \loop
      \ifnum\count@<\pc@columncount
        \advance\count@\@ne
        \expandafter\ifvoid\csname pc@column@\number\count@\endcsname
          \hskip\csname pc@column@width@\number\count@\endcsname
        \else
          \expandafter\setbox\expandafter\@tempboxa\expandafter
          \vsplit\csname pc@column@\number\count@\endcsname
              to \dp\strutbox
          \vbox{%
            \unvbox\@tempboxa
          }%
        \fi
        \expandafter\ifvoid\csname pc@column@\number\count@\endcsname
        \else
          \global\let\@tempa\pc@placeboxes
        \fi
        \ifnum\count@<\pc@columncount
          \strut
          \hfill
          \ifpc@rulebetween
            \vrule
            \hfill
          \fi
        \fi
      \repeat
    }%
    \@tempa
  }%
  \ifx\x\pc@placeboxes
  \else
    \@PackageWarningNoLine{pdfcolparcolumns}{%
      Command \string\pc@placeboxes\space has changed.\MessageBreak
      Supported versions of package `parcolumns':\MessageBreak
      \space\space 2004/08/05.\MessageBreak
      The redefinition of \string\pc@placeboxes\space may not%
      \MessageBreak
      behave correctly depending on the changes%
    }%
  \fi
\endgroup
%    \end{macrocode}
%    \begin{macro}{\pc@placeboxes}
%    \begin{macrocode}
\renewcommand*{\pc@placeboxes}{%
  \global\let\@tempa\relax
  \count@\z@
  \hb@xt@\linewidth{%
    \vfuzz30ex %
    \vbadness\@M
    \splittopskip\z@skip
    \loop
    \ifnum\count@<\pc@columncount
      \advance\count@\@ne
      \expandafter\ifvoid\csname pc@column@\number\count@\endcsname
        \hskip\csname pc@column@width@\number\count@\endcsname
      \else
        \expandafter\setbox\expandafter\@tempboxa\expandafter
        \vsplit\csname pc@column@\number\count@\endcsname
            to \dp\strutbox
        \vbox{%
          \pcpc@SetCurrent\count@
          \unvbox\@tempboxa
        }%
      \fi
      \expandafter\ifvoid\csname pc@column@\number\count@\endcsname
      \else
        \global\let\@tempa\pc@placeboxes
      \fi
      \ifnum\count@<\pc@columncount
        \strut
        \hfill
        \ifpc@rulebetween
          \begingroup
            \pcpc@RuleBetweenColor
            \vrule
          \endgroup
          \hfill
        \fi
      \fi
    \repeat
  }%
  \@tempa
}
%    \end{macrocode}
%    \end{macro}
%    \begin{macro}{\pcpc@RuleBetweenColorDefault}
%    \begin{macrocode}
\def\pcpc@RuleBetweenColorDefault{%
  \normalcolor
}
%    \end{macrocode}
%    \end{macro}
%    \begin{macro}{\pcpc@RuleBetweenColor}
%    \begin{macrocode}
\let\pcpc@RuleBetweenColor\pcpc@RuleBetweenColorDefault
%    \end{macrocode}
%    \end{macro}
%    \begin{macrocode}
\define@key{parcolumns}{rulebetweencolor}{%
  \edef\pcpc@temp{#1}%
  \ifx\pcpc@temp\@empty
    \let\pcpc@RuleBetweenColor\pcpc@RuleBetweenColorDefault
  \else
    \edef\pcpc@temp{%
      \noexpand\@ifnextchar[{%
        \def\noexpand\pcpc@RuleBetweenColor{%
          \noexpand\color\pcpc@temp
        }%
        \noexpand\pcpc@GobbleNil
      }{%
        \def\noexpand\pcpc@RuleBetweenColor{%
          \noexpand\color{\pcpc@temp}%
        }%
        \noexpand\pcpc@GobbleNil
      }%
      \pcpc@temp\noexpand\@nil
    }%
    \pcpc@temp
  \fi
}
%    \end{macrocode}
%    \begin{macro}{\pcpc@GobbleNil}
%    \begin{macrocode}
\long\def\pcpc@GobbleNil#1\@nil{}
%    \end{macrocode}
%    \end{macro}
%
%    \begin{macrocode}
%</package>
%    \end{macrocode}
%% \section{Installation\\安装}
%
% \subsection{Download\\下载}
%
% \paragraph{Package.} This package is available on
% CTAN\footnote{\CTANpkg{pdfcolparcolumns}}:

%该软件包可从 CTAN\footnote{\CTANpkg{pdfcolparcolumns}} 下载:
% \begin{description}
% \item[\CTAN{macros/latex/contrib/oberdiek/pdfcolparcolumns.dtx}] The source file.
% \item[\CTAN{macros/latex/contrib/oberdiek/pdfcolparcolumns.pdf}] Documentation.
% \end{description}
%
%
% \paragraph{Bundle.} All the packages of the bundle `oberdiek'
% are also available in a TDS compliant ZIP archive. There
% the packages are already unpacked and the documentation files
% are generated. The files and directories obey the TDS standard.
%
%捆绑包“oberdiek”中的所有软件包也可在符合 TDS 标准的 ZIP 存档中获取。在该存档中,软件包已经被解包,文档文件已经生成,文件和目录符合 TDS 标准。
% \begin{description}
% \item[\CTANinstall{install/macros/latex/contrib/oberdiek.tds.zip}]
% \end{description}
% \emph{TDS} refers to the standard ``A Directory Structure
% for \TeX\ Files'' (\CTANpkg{tds}). Directories
% with \xfile{texmf} in their name are usually organized this way.
%
%\emph{TDS} 指的是标准“用于 \TeX\ 文件的目录结构”(\CTANpkg{tds})。名字中包含\xfile{texmf}的目录通常都是按这种方式组织的。
% \subsection{Bundle installation\\捆绑包安装}
%
% \paragraph{Unpacking.} Unpack the \xfile{oberdiek.tds.zip} in the
% TDS tree (also known as \xfile{texmf} tree) of your choice.
% Example (linux):
%
%\paragraph{解包。}在您选择的 TDS 树(也称为\xfile{texmf}树)中解压\xfile{oberdiek.tds.zip}。例如(在Linux中):
% \begin{quote}
%   |unzip oberdiek.tds.zip -d ~/texmf|
% \end{quote}
%
% \subsection{Package installation\\软件包安装}
%
% \paragraph{Unpacking.} The \xfile{.dtx} file is a self-extracting
% \docstrip\ archive. The files are extracted by running the
% \xfile{.dtx} through \plainTeX:
%
%\paragraph{解包。} \xfile{.dtx} 文件是一个自解压的 \docstrip\ 存档。运行\xfile{.dtx}通过\plainTeX\ 来提取文件:
% \begin{quote}
%   \verb|tex pdfcolparcolumns.dtx|
% \end{quote}
%
% \paragraph{TDS.} Now the different files must be moved into
% the different directories in your installation TDS tree
% (also known as \xfile{texmf} tree):
%
%\paragraph{TDS。}现在,不同的文件必须移动到安装 TDS 树(也称为\xfile{texmf}树)中的不同目录中:
% \begin{quote}
% \def\t{^^A
% \begin{tabular}{@{}>{\ttfamily}l@{ $\rightarrow$ }>{\ttfamily}l@{}}
%   pdfcolparcolumns.sty & tex/latex/oberdiek/pdfcolparcolumns.sty\\
%   pdfcolparcolumns.pdf & doc/latex/oberdiek/pdfcolparcolumns.pdf\\
%   pdfcolparcolumns.dtx & source/latex/oberdiek/pdfcolparcolumns.dtx\\
% \end{tabular}^^A
% }^^A
% \sbox0{\t}^^A
% \ifdim\wd0>\linewidth
%   \begingroup
%     \advance\linewidth by\leftmargin
%     \advance\linewidth by\rightmargin
%   \edef\x{\endgroup
%     \def\noexpand\lw{\the\linewidth}^^A
%   }\x
%   \def\lwbox{^^A
%     \leavevmode
%     \hbox to \linewidth{^^A
%       \kern-\leftmargin\relax
%       \hss
%       \usebox0
%       \hss
%       \kern-\rightmargin\relax
%     }^^A
%   }^^A
%   \ifdim\wd0>\lw
%     \sbox0{\small\t}^^A
%     \ifdim\wd0>\linewidth
%       \ifdim\wd0>\lw
%         \sbox0{\footnotesize\t}^^A
%         \ifdim\wd0>\linewidth
%           \ifdim\wd0>\lw
%             \sbox0{\scriptsize\t}^^A
%             \ifdim\wd0>\linewidth
%               \ifdim\wd0>\lw
%                 \sbox0{\tiny\t}^^A
%                 \ifdim\wd0>\linewidth
%                   \lwbox
%                 \else
%                   \usebox0
%                 \fi
%               \else
%                 \lwbox
%               \fi
%             \else
%               \usebox0
%             \fi
%           \else
%             \lwbox
%           \fi
%         \else
%           \usebox0
%         \fi
%       \else
%         \lwbox
%       \fi
%     \else
%       \usebox0
%     \fi
%   \else
%     \lwbox
%   \fi
% \else
%   \usebox0
% \fi
% \end{quote}
% If you have a \xfile{docstrip.cfg} that configures and enables \docstrip's
% TDS installing feature, then some files can already be in the right
% place, see the documentation of \docstrip.
%
%如果你有一个\xfile{docstrip.cfg}文件配置和启用了\docstrip 的TDS安装功能,那么一些文件可能已经位于正确的位置,具体请参见\docstrip 的文档。
% \subsection{Refresh file name databases\\刷新文件名数据库}
%
% If your \TeX~distribution
% (\TeX\,Live, \mikTeX, \dots) relies on file name databases, you must refresh
% these. For example, \TeX\,Live\ users run \verb|texhash| or
% \verb|mktexlsr|.
%
%如果你的\TeX~发行版(\TeX,Live、\mikTeX,等等)依赖于文件名数据库,你必须刷新它们。例如,\TeX,Live用户运行\verb|texhash|或\verb|mktexlsr|。
% \subsection{Some details for the interested\\一些细节供感兴趣的人使用}
%
% \paragraph{Unpacking with \LaTeX.}
% The \xfile{.dtx} chooses its action depending on the format:

%\paragraph{用\LaTeX 进行解包。} \xfile{.dtx}文件根据格式选择操作:
% \begin{description}
% \item[\plainTeX:] Run \docstrip\ and extract the files.
% \item[\LaTeX:] Generate the documentation.
% \end{description}
% If you insist on using \LaTeX\ for \docstrip\ (really,
% \docstrip\ does not need \LaTeX), then inform the autodetect routine
% about your intention:

%如果你坚持要用\LaTeX 进行\docstrip (其实\docstrip 不需要\LaTeX ),那么请告知自动检测例程你的意图:

% \begin{quote}
%   \verb|latex \let\install=y% \iffalse meta-comment
%
% File: pdfcolparcolumns.dtx
% Version: 2019/12/29 v1.5
% Info: Color stacks for parcolumns
%
% Copyright (C)
%    2007, 2008, 2010 Heiko Oberdiek
%    2016-2019 Oberdiek Package Support Group
%    https://github.com/ho-tex/oberdiek/issues
%
% This work may be distributed and/or modified under the
% conditions of the LaTeX Project Public License, either
% version 1.3c of this license or (at your option) any later
% version. This version of this license is in
%    https://www.latex-project.org/lppl/lppl-1-3c.txt
% and the latest version of this license is in
%    https://www.latex-project.org/lppl.txt
% and version 1.3 or later is part of all distributions of
% LaTeX version 2005/12/01 or later.
%
% This work has the LPPL maintenance status "maintained".
%
% The Current Maintainers of this work are
% Heiko Oberdiek and the Oberdiek Package Support Group
% https://github.com/ho-tex/oberdiek/issues
%
% This work consists of the main source file pdfcolparcolumns.dtx
% and the derived files
%    pdfcolparcolumns.sty, pdfcolparcolumns.pdf, pdfcolparcolumns.ins,
%    pdfcolparcolumns.drv, pdfcolparcolumns-test1.tex.
%
% Distribution:
%    CTAN:macros/latex/contrib/oberdiek/pdfcolparcolumns.dtx
%    CTAN:macros/latex/contrib/oberdiek/pdfcolparcolumns.pdf
%
% Unpacking:
%    (a) If pdfcolparcolumns.ins is present:
%           tex pdfcolparcolumns.ins
%    (b) Without pdfcolparcolumns.ins:
%           tex pdfcolparcolumns.dtx
%    (c) If you insist on using LaTeX
%           latex \let\install=y% \iffalse meta-comment
%
% File: pdfcolparcolumns.dtx
% Version: 2019/12/29 v1.5
% Info: Color stacks for parcolumns
%
% Copyright (C)
%    2007, 2008, 2010 Heiko Oberdiek
%    2016-2019 Oberdiek Package Support Group
%    https://github.com/ho-tex/oberdiek/issues
%
% This work may be distributed and/or modified under the
% conditions of the LaTeX Project Public License, either
% version 1.3c of this license or (at your option) any later
% version. This version of this license is in
%    https://www.latex-project.org/lppl/lppl-1-3c.txt
% and the latest version of this license is in
%    https://www.latex-project.org/lppl.txt
% and version 1.3 or later is part of all distributions of
% LaTeX version 2005/12/01 or later.
%
% This work has the LPPL maintenance status "maintained".
%
% The Current Maintainers of this work are
% Heiko Oberdiek and the Oberdiek Package Support Group
% https://github.com/ho-tex/oberdiek/issues
%
% This work consists of the main source file pdfcolparcolumns.dtx
% and the derived files
%    pdfcolparcolumns.sty, pdfcolparcolumns.pdf, pdfcolparcolumns.ins,
%    pdfcolparcolumns.drv, pdfcolparcolumns-test1.tex.
%
% Distribution:
%    CTAN:macros/latex/contrib/oberdiek/pdfcolparcolumns.dtx
%    CTAN:macros/latex/contrib/oberdiek/pdfcolparcolumns.pdf
%
% Unpacking:
%    (a) If pdfcolparcolumns.ins is present:
%           tex pdfcolparcolumns.ins
%    (b) Without pdfcolparcolumns.ins:
%           tex pdfcolparcolumns.dtx
%    (c) If you insist on using LaTeX
%           latex \let\install=y\input{pdfcolparcolumns.dtx}
%        (quote the arguments according to the demands of your shell)
%
% Documentation:
%    (a) If pdfcolparcolumns.drv is present:
%           latex pdfcolparcolumns.drv
%    (b) Without pdfcolparcolumns.drv:
%           latex pdfcolparcolumns.dtx; ...
%    The class ltxdoc loads the configuration file ltxdoc.cfg
%    if available. Here you can specify further options, e.g.
%    use A4 as paper format:
%       \PassOptionsToClass{a4paper}{article}
%
%    Programm calls to get the documentation (example):
%       pdflatex pdfcolparcolumns.dtx
%       makeindex -s gind.ist pdfcolparcolumns.idx
%       pdflatex pdfcolparcolumns.dtx
%       makeindex -s gind.ist pdfcolparcolumns.idx
%       pdflatex pdfcolparcolumns.dtx
%
% Installation:
%    TDS:tex/latex/oberdiek/pdfcolparcolumns.sty
%    TDS:doc/latex/oberdiek/pdfcolparcolumns.pdf
%    TDS:source/latex/oberdiek/pdfcolparcolumns.dtx
%
%<*ignore>
\begingroup
  \catcode123=1 %
  \catcode125=2 %
  \def\x{LaTeX2e}%
\expandafter\endgroup
\ifcase 0\ifx\install y1\fi\expandafter
         \ifx\csname processbatchFile\endcsname\relax\else1\fi
         \ifx\fmtname\x\else 1\fi\relax
\else\csname fi\endcsname
%</ignore>
%<*install>
\input docstrip.tex
\Msg{************************************************************************}
\Msg{* Installation}
\Msg{* Package: pdfcolparcolumns 2019/12/29 v1.5 Color stacks for parcolumns (HO)}
\Msg{************************************************************************}

\keepsilent
\askforoverwritefalse

\let\MetaPrefix\relax
\preamble

This is a generated file.

Project: pdfcolparcolumns
Version: 2019/12/29 v1.5

Copyright (C)
   2007, 2008, 2010 Heiko Oberdiek
   2016-2019 Oberdiek Package Support Group

This work may be distributed and/or modified under the
conditions of the LaTeX Project Public License, either
version 1.3c of this license or (at your option) any later
version. This version of this license is in
   https://www.latex-project.org/lppl/lppl-1-3c.txt
and the latest version of this license is in
   https://www.latex-project.org/lppl.txt
and version 1.3 or later is part of all distributions of
LaTeX version 2005/12/01 or later.

This work has the LPPL maintenance status "maintained".

The Current Maintainers of this work are
Heiko Oberdiek and the Oberdiek Package Support Group
https://github.com/ho-tex/oberdiek/issues


This work consists of the main source file pdfcolparcolumns.dtx
and the derived files
   pdfcolparcolumns.sty, pdfcolparcolumns.pdf, pdfcolparcolumns.ins,
   pdfcolparcolumns.drv, pdfcolparcolumns-test1.tex.

\endpreamble
\let\MetaPrefix\DoubleperCent

\generate{%
  \file{pdfcolparcolumns.ins}{\from{pdfcolparcolumns.dtx}{install}}%
  \file{pdfcolparcolumns.drv}{\from{pdfcolparcolumns.dtx}{driver}}%
  \usedir{tex/latex/oberdiek}%
  \file{pdfcolparcolumns.sty}{\from{pdfcolparcolumns.dtx}{package}}%
%  \usedir{doc/latex/oberdiek/test}%
%  \file{pdfcolparcolumns-test1.tex}{\from{pdfcolparcolumns.dtx}{test1}}%
}

\catcode32=13\relax% active space
\let =\space%
\Msg{************************************************************************}
\Msg{*}
\Msg{* To finish the installation you have to move the following}
\Msg{* file into a directory searched by TeX:}
\Msg{*}
\Msg{*     pdfcolparcolumns.sty}
\Msg{*}
\Msg{* To produce the documentation run the file `pdfcolparcolumns.drv'}
\Msg{* through LaTeX.}
\Msg{*}
\Msg{* Happy TeXing!}
\Msg{*}
\Msg{************************************************************************}

\endbatchfile
%</install>
%<*ignore>
\fi
%</ignore>
%<*driver>
\NeedsTeXFormat{LaTeX2e}
\ProvidesFile{pdfcolparcolumns.drv}%
  [2019/12/29 v1.5 Color stacks for parcolumns (HO)]%
\documentclass{ltxdoc}
\usepackage{holtxdoc}[2011/11/22]
\usepackage[scheme=plain]{ctex}
\setCJKmainfont{方正书宋_GBK}%方正书宋_GBK.TTF  设置文本的中文有衬线字体为“方正书宋_GBK”
\setCJKsansfont{方正黑体简体}%方正黑体_GBK.TTF  设置文本的中文无衬线字体为“方正黑体简体”
\setCJKmonofont{方正书宋简体}%方正仿宋_GBK.TTF  设置文本的中文等宽字体为“方正书宋简体”
\begin{document}
  \DocInput{pdfcolparcolumns.dtx}%
\end{document}
%</driver>
% \fi
%
%
%
% \GetFileInfo{pdfcolparcolumns.drv}
%
% \title{The \xpackage{pdfcolparcolumns} package}
% \date{2019/12/29 v1.5}
% \author{Heiko Oberdiek\thanks
% {Please report any issues at \url{https://github.com/ho-tex/oberdiek/issues}}\and 翻译\\{\tt virhuiai@qq.com}}
%
% \maketitle
%
% \begin{abstract}
%\makebox[0pt]{}
% \begin{parcolumns}[nofirstindent]{2}
%\colchunk{Since version 1.40 \pdfTeX\ supports several color stacks.
%This package uses them to fix color problems in
%package \xpackage{parcolumns}.}
%\colchunk{自从版本1.40起,\pdfTeX 支持多个颜色栈。
%此包使用它们来解决 \xpackage{parcolumns} 包中的颜色问题。}
%\end{parcolumns}



% \end{abstract}
%
% \tableofcontents
%
% \section{Usage\\用法}
%
% \begin{quote}
% |\usepackage{pdfcolparcolumns}|
% \end{quote}
% The package \xpackage{pdfcolparcolumns} loads package \xpackage{parcolums}
% \cite{parcolumns}. If color stacks are available then the
% macros of \xpackage{parcolumns} are patched to add support
% for color stacks.

%包 \xpackage{pdfcolparcolumns} 载入了 \xpackage{parcolumns} \cite{parcolumns} 包。如果有可用的颜色栈,则会对 \xpackage{parcolumns} 的宏进行修补,以添加对颜色栈的支持。

%
% \subsection{Option \xoption{rulebetweencolor}\\选项 \xoption{rulebetweencolor}}
%
% Package \xpackage{pdfcolparcolumns} also fixes the color for the
% rule between columns (if \xoption{rulebetween} is set).
% Default color is \cs{normalcolor}. But this can be changed by using
% option \xoption{rulebetweencolor}. It takes a color specification
% as value. If the value is empty, then the default (\cs{normalcolor})
% is used.
% Examples:

% 包 \xpackage{pdfcolparcolumns} 还修复了列之间分隔线的颜色(如果设置了 \xoption{rulebetween})。默认颜色为 \cs{normalcolor}。但是,可以使用选项 \xoption{rulebetweencolor} 来更改它。它接受一个颜色规范作为值。如果该值为空,则使用默认值(\cs{normalcolor})。例如:

% \begin{quote}
%   |rulebetweencolor=blue|,\\
%   |rulebetweencolor={red}|,\\
%   |rulebetweencolor={}|, \textit{\% \cs{normalcolor} is used}\\
%   |rulebetweencolor=[rgb]{1,0,.5}| \textit{\% see below}
% \end{quote}
% If used inside the optional argument of environment \textsf{parcolumns}
% and the value contains an optional argument, then whole value
% must be put in curly braces:

% 如果在 \textsf{parcolumns} 环境的可选参数中使用,并且值包含可选参数,则整个值必须用花括号括起来:
%\begin{quote}
%\begin{verbatim}
%\begin{parcolumns}[
%  rulebetween,
%  rulebetweencolor={[rgb]{1,0,.5}},
%]{2}
%  ...
%\end{parcolumns}
%\end{verbatim}
%\end{quote}
% This option \xoption{rulebetweencolor} can also be set using
% \cs{setkeys}:

%也可以使用 \cs{setkeys} 来设置选项 \xoption{rulebetweencolor}:
%\begin{quote}
%\begin{verbatim}
%\setkeys{parcolumns}{rulebetweencolor=green}
%\end{verbatim}
%\end{quote}
%
% \subsection{Future\\未来展望}
%
% Currently package \xpackage{parcolumns} does not seem to be
% maintained. Nevertheless if there will be a new version that
% adds support for color stacks, then this package may become
% obsolete.
%
%目前,\xpackage{parcolumns} 包似乎没有维护了。尽管如此,如果出现了新版本并支持颜色栈,那么这个包可能会变得过时。
% \StopEventually{
% }
%
% \section{Implementation\\实现}
%
% \subsection{Identification\\标识}
%
%    \begin{macrocode}
%<*package>
\NeedsTeXFormat{LaTeX2e}
\ProvidesPackage{pdfcolparcolumns}%
  [2019/12/29 v1.5 Color stacks for parcolumns (HO)]%
%    \end{macrocode}
%
% \subsection{Load packages\\加载宏包}
%
% \subsubsection{Package \xpackage{parcolumns}\\宏包 \xpackage{parcolumns}}
%
%    Currently package \xpackage{parcolumns} does not define options.
%    Thus it is just a precaution that the options of
%    package \xpackage{pdfcolparcolumns} are passed to
%    package \xpackage{parcolumns}.

%目前,宏包 \xpackage{parcolumns} 没有定义选项。因此,它只是一个预防措施,以确保将宏包 \xpackage{pdfcolparcolumns} 的选项传递给宏包 \xpackage{parcolumns}。
%    \begin{macrocode}
\DeclareOption*{%
  \PassOptionsToPackage{\CurrentOption}{parcolumns}%
}
\ProcessOptions\relax
\RequirePackage{parcolumns}[2004/11/25]
%    \end{macrocode}
%
% \subsubsection{Package \xpackage{pdfcol}\\宏包 \xpackage{pdfcol}}
%
%    \begin{macrocode}
\RequirePackage{pdfcol}[2007/09/09]
\ifpdfcolAvailable
\else
  \PackageInfo{pdfcolparcolumns}{%
    Loading aborted, because color stacks are not available%
  }%
  \expandafter\endinput
\fi
%    \end{macrocode}
%
% \subsubsection{Package \xpackage{infwarerr}\\宏包 \xpackage{infwarerr}}
%
%    \begin{macrocode}
\RequirePackage{infwarerr}[2007/09/09]
%    \end{macrocode}
%
% \subsection{Color stack macros\\颜色堆栈宏}
%
%    \begin{macro}{\pcpc@MaxStack}
%    Macro \cs{pcpc@MaxStack} holds the highest number of
%    allocated stacks.

%宏 \cs{pcpc@MaxStack} 存储已分配的堆栈中最大的编号。
%    \begin{macrocode}
\global\chardef\pcpc@MaxStack=\z@
%    \end{macrocode}
%    \end{macro}
%    \begin{macro}{\pcpc@InitStacks}
%    Macro \cs{pcpc@InitStacks} takes the number of columns
%    as argument and ensures that there are enough color
%    stacks for all columns.
%
%宏 \cs{pcpc@InitStacks} 接受列数作为参数,并确保为所有列分配了足够的颜色堆栈。
%    \begin{macrocode}
\def\pcpc@InitStacks#1{%
  \ifnum#1>\pcpc@MaxStack
    \begingroup
      \count@\pcpc@MaxStack
      \loop
        \advance\count@\@ne
        \pdfcolInitStack{pcpc@\the\count@}%
      \ifnum#1>\count@
      \repeat
      \global\chardef\pcpc@MaxStack=\count@
    \endgroup
  \fi
}
%    \end{macrocode}
%    \end{macro}
%
%    \begin{macro}{\pcpc@SwitchStack}
%    \begin{macrocode}
\def\pcpc@SwitchStack#1{%
  \pdfcolSwitchStack{pcpc@\number#1}%
}
%    \end{macrocode}
%    \end{macro}
%
%    \begin{macro}{\pcpc@SetCurrent}
%    \begin{macrocode}
\def\pcpc@SetCurrent#1{%
  \pdfcolSetCurrent{pcpc@\number#1}%
}
%    \end{macrocode}
%    \end{macro}
%
% \subsection{Patches\\补丁}
%
%     Now the color stack macros are patched into the macros
%     of package \xpackage{parcolumns}.
%
%现在颜色堆栈宏已经被打补丁到了 \xpackage{parcolumns} 宏包的宏中。
% \subsubsection{Init stacks\\初始化堆栈}
%
%    \cs{pcpc@InitStacks} should go into the definition of
%    environment |parcolumns|. \cs{pc@alloccolumns} is executed
%    there and nowhere else, thus we hook into it.

%\cs{pcpc@InitStacks} 应该被放在环境 |parcolumns| 的定义中。\cs{pc@alloccolumns} 仅在那里执行,因此我们将其连接起来。
%    \begin{macrocode}
\g@addto@macro\pc@alloccolumns{%
  \pcpc@InitStacks\pc@columncount
}
%    \end{macrocode}
%
% \subsubsection{Switch stack\\切换堆栈}
%
%    \cs{pcpc@SwitchStack} should be called by marco \cs{colchunk@}.
%    However it is easier to patch \cs{pc@setcolumnwidth} that
%    is executed in \cs{colchunk@} only.
%
%\cs{pcpc@SwitchStack} 应该被宏 \cs{colchunk@} 调用。不过,我们更容易修补仅在 \cs{colchunk@} 中执行的 \cs{pc@setcolumnwidth}。
%    \begin{macrocode}
\g@addto@macro\pc@setcolumnwidth{%
  \pcpc@SwitchStack\pc@columnctr
}
%    \end{macrocode}
%
% \subsubsection{Set current stack color\\设置当前堆栈颜色}
%
%    \cs{pcpc@SetCurrent} is set at the begin of each line.
%    It must be inserted into \cs{pc@placeboxes}. Unhappily
%    there is no easy way. Therefore we check and
%    redefine \cs{pc@placeboxes}.
%
%\cs{pcpc@SetCurrent} 在每行开始时设置。它必须插入到 \cs{pc@placeboxes} 中。不幸的是,没有简单的方法。因此,我们检查并重新定义 \cs{pc@placeboxes}。
%    \begin{macrocode}
\begingroup
  \def\x{%
    \global\let\@tempa\relax
    \count@\z@
    \hb@xt@\linewidth{%
      \vfuzz30ex %
      \vbadness\@M
      \splittopskip\z@skip
      \loop
      \ifnum\count@<\pc@columncount
        \advance\count@\@ne
        \expandafter\ifvoid\csname pc@column@\number\count@\endcsname
          \hskip\csname pc@column@width@\number\count@\endcsname
        \else
          \expandafter\setbox\expandafter\@tempboxa\expandafter
          \vsplit\csname pc@column@\number\count@\endcsname
              to \dp\strutbox
          \vbox{%
            \unvbox\@tempboxa
          }%
        \fi
        \expandafter\ifvoid\csname pc@column@\number\count@\endcsname
        \else
          \global\let\@tempa\pc@placeboxes
        \fi
        \ifnum\count@<\pc@columncount
          \strut
          \hfill
          \ifpc@rulebetween
            \vrule
            \hfill
          \fi
        \fi
      \repeat
    }%
    \@tempa
  }%
  \ifx\x\pc@placeboxes
  \else
    \@PackageWarningNoLine{pdfcolparcolumns}{%
      Command \string\pc@placeboxes\space has changed.\MessageBreak
      Supported versions of package `parcolumns':\MessageBreak
      \space\space 2004/08/05.\MessageBreak
      The redefinition of \string\pc@placeboxes\space may not%
      \MessageBreak
      behave correctly depending on the changes%
    }%
  \fi
\endgroup
%    \end{macrocode}
%    \begin{macro}{\pc@placeboxes}
%    \begin{macrocode}
\renewcommand*{\pc@placeboxes}{%
  \global\let\@tempa\relax
  \count@\z@
  \hb@xt@\linewidth{%
    \vfuzz30ex %
    \vbadness\@M
    \splittopskip\z@skip
    \loop
    \ifnum\count@<\pc@columncount
      \advance\count@\@ne
      \expandafter\ifvoid\csname pc@column@\number\count@\endcsname
        \hskip\csname pc@column@width@\number\count@\endcsname
      \else
        \expandafter\setbox\expandafter\@tempboxa\expandafter
        \vsplit\csname pc@column@\number\count@\endcsname
            to \dp\strutbox
        \vbox{%
          \pcpc@SetCurrent\count@
          \unvbox\@tempboxa
        }%
      \fi
      \expandafter\ifvoid\csname pc@column@\number\count@\endcsname
      \else
        \global\let\@tempa\pc@placeboxes
      \fi
      \ifnum\count@<\pc@columncount
        \strut
        \hfill
        \ifpc@rulebetween
          \begingroup
            \pcpc@RuleBetweenColor
            \vrule
          \endgroup
          \hfill
        \fi
      \fi
    \repeat
  }%
  \@tempa
}
%    \end{macrocode}
%    \end{macro}
%    \begin{macro}{\pcpc@RuleBetweenColorDefault}
%    \begin{macrocode}
\def\pcpc@RuleBetweenColorDefault{%
  \normalcolor
}
%    \end{macrocode}
%    \end{macro}
%    \begin{macro}{\pcpc@RuleBetweenColor}
%    \begin{macrocode}
\let\pcpc@RuleBetweenColor\pcpc@RuleBetweenColorDefault
%    \end{macrocode}
%    \end{macro}
%    \begin{macrocode}
\define@key{parcolumns}{rulebetweencolor}{%
  \edef\pcpc@temp{#1}%
  \ifx\pcpc@temp\@empty
    \let\pcpc@RuleBetweenColor\pcpc@RuleBetweenColorDefault
  \else
    \edef\pcpc@temp{%
      \noexpand\@ifnextchar[{%
        \def\noexpand\pcpc@RuleBetweenColor{%
          \noexpand\color\pcpc@temp
        }%
        \noexpand\pcpc@GobbleNil
      }{%
        \def\noexpand\pcpc@RuleBetweenColor{%
          \noexpand\color{\pcpc@temp}%
        }%
        \noexpand\pcpc@GobbleNil
      }%
      \pcpc@temp\noexpand\@nil
    }%
    \pcpc@temp
  \fi
}
%    \end{macrocode}
%    \begin{macro}{\pcpc@GobbleNil}
%    \begin{macrocode}
\long\def\pcpc@GobbleNil#1\@nil{}
%    \end{macrocode}
%    \end{macro}
%
%    \begin{macrocode}
%</package>
%    \end{macrocode}
%% \section{Installation\\安装}
%
% \subsection{Download\\下载}
%
% \paragraph{Package.} This package is available on
% CTAN\footnote{\CTANpkg{pdfcolparcolumns}}:

%该软件包可从 CTAN\footnote{\CTANpkg{pdfcolparcolumns}} 下载:
% \begin{description}
% \item[\CTAN{macros/latex/contrib/oberdiek/pdfcolparcolumns.dtx}] The source file.
% \item[\CTAN{macros/latex/contrib/oberdiek/pdfcolparcolumns.pdf}] Documentation.
% \end{description}
%
%
% \paragraph{Bundle.} All the packages of the bundle `oberdiek'
% are also available in a TDS compliant ZIP archive. There
% the packages are already unpacked and the documentation files
% are generated. The files and directories obey the TDS standard.
%
%捆绑包“oberdiek”中的所有软件包也可在符合 TDS 标准的 ZIP 存档中获取。在该存档中,软件包已经被解包,文档文件已经生成,文件和目录符合 TDS 标准。
% \begin{description}
% \item[\CTANinstall{install/macros/latex/contrib/oberdiek.tds.zip}]
% \end{description}
% \emph{TDS} refers to the standard ``A Directory Structure
% for \TeX\ Files'' (\CTANpkg{tds}). Directories
% with \xfile{texmf} in their name are usually organized this way.
%
%\emph{TDS} 指的是标准“用于 \TeX\ 文件的目录结构”(\CTANpkg{tds})。名字中包含\xfile{texmf}的目录通常都是按这种方式组织的。
% \subsection{Bundle installation\\捆绑包安装}
%
% \paragraph{Unpacking.} Unpack the \xfile{oberdiek.tds.zip} in the
% TDS tree (also known as \xfile{texmf} tree) of your choice.
% Example (linux):
%
%\paragraph{解包。}在您选择的 TDS 树(也称为\xfile{texmf}树)中解压\xfile{oberdiek.tds.zip}。例如(在Linux中):
% \begin{quote}
%   |unzip oberdiek.tds.zip -d ~/texmf|
% \end{quote}
%
% \subsection{Package installation\\软件包安装}
%
% \paragraph{Unpacking.} The \xfile{.dtx} file is a self-extracting
% \docstrip\ archive. The files are extracted by running the
% \xfile{.dtx} through \plainTeX:
%
%\paragraph{解包。} \xfile{.dtx} 文件是一个自解压的 \docstrip\ 存档。运行\xfile{.dtx}通过\plainTeX\ 来提取文件:
% \begin{quote}
%   \verb|tex pdfcolparcolumns.dtx|
% \end{quote}
%
% \paragraph{TDS.} Now the different files must be moved into
% the different directories in your installation TDS tree
% (also known as \xfile{texmf} tree):
%
%\paragraph{TDS。}现在,不同的文件必须移动到安装 TDS 树(也称为\xfile{texmf}树)中的不同目录中:
% \begin{quote}
% \def\t{^^A
% \begin{tabular}{@{}>{\ttfamily}l@{ $\rightarrow$ }>{\ttfamily}l@{}}
%   pdfcolparcolumns.sty & tex/latex/oberdiek/pdfcolparcolumns.sty\\
%   pdfcolparcolumns.pdf & doc/latex/oberdiek/pdfcolparcolumns.pdf\\
%   pdfcolparcolumns.dtx & source/latex/oberdiek/pdfcolparcolumns.dtx\\
% \end{tabular}^^A
% }^^A
% \sbox0{\t}^^A
% \ifdim\wd0>\linewidth
%   \begingroup
%     \advance\linewidth by\leftmargin
%     \advance\linewidth by\rightmargin
%   \edef\x{\endgroup
%     \def\noexpand\lw{\the\linewidth}^^A
%   }\x
%   \def\lwbox{^^A
%     \leavevmode
%     \hbox to \linewidth{^^A
%       \kern-\leftmargin\relax
%       \hss
%       \usebox0
%       \hss
%       \kern-\rightmargin\relax
%     }^^A
%   }^^A
%   \ifdim\wd0>\lw
%     \sbox0{\small\t}^^A
%     \ifdim\wd0>\linewidth
%       \ifdim\wd0>\lw
%         \sbox0{\footnotesize\t}^^A
%         \ifdim\wd0>\linewidth
%           \ifdim\wd0>\lw
%             \sbox0{\scriptsize\t}^^A
%             \ifdim\wd0>\linewidth
%               \ifdim\wd0>\lw
%                 \sbox0{\tiny\t}^^A
%                 \ifdim\wd0>\linewidth
%                   \lwbox
%                 \else
%                   \usebox0
%                 \fi
%               \else
%                 \lwbox
%               \fi
%             \else
%               \usebox0
%             \fi
%           \else
%             \lwbox
%           \fi
%         \else
%           \usebox0
%         \fi
%       \else
%         \lwbox
%       \fi
%     \else
%       \usebox0
%     \fi
%   \else
%     \lwbox
%   \fi
% \else
%   \usebox0
% \fi
% \end{quote}
% If you have a \xfile{docstrip.cfg} that configures and enables \docstrip's
% TDS installing feature, then some files can already be in the right
% place, see the documentation of \docstrip.
%
%如果你有一个\xfile{docstrip.cfg}文件配置和启用了\docstrip 的TDS安装功能,那么一些文件可能已经位于正确的位置,具体请参见\docstrip 的文档。
% \subsection{Refresh file name databases\\刷新文件名数据库}
%
% If your \TeX~distribution
% (\TeX\,Live, \mikTeX, \dots) relies on file name databases, you must refresh
% these. For example, \TeX\,Live\ users run \verb|texhash| or
% \verb|mktexlsr|.
%
%如果你的\TeX~发行版(\TeX,Live、\mikTeX,等等)依赖于文件名数据库,你必须刷新它们。例如,\TeX,Live用户运行\verb|texhash|或\verb|mktexlsr|。
% \subsection{Some details for the interested\\一些细节供感兴趣的人使用}
%
% \paragraph{Unpacking with \LaTeX.}
% The \xfile{.dtx} chooses its action depending on the format:

%\paragraph{用\LaTeX 进行解包。} \xfile{.dtx}文件根据格式选择操作:
% \begin{description}
% \item[\plainTeX:] Run \docstrip\ and extract the files.
% \item[\LaTeX:] Generate the documentation.
% \end{description}
% If you insist on using \LaTeX\ for \docstrip\ (really,
% \docstrip\ does not need \LaTeX), then inform the autodetect routine
% about your intention:

%如果你坚持要用\LaTeX 进行\docstrip (其实\docstrip 不需要\LaTeX ),那么请告知自动检测例程你的意图:

% \begin{quote}
%   \verb|latex \let\install=y\input{pdfcolparcolumns.dtx}|
% \end{quote}
% Do not forget to quote the argument according to the demands
% of your shell.

%不要忘记根据你的shell的要求引用参数。
%
% \paragraph{Generating the documentation.}
% You can use both the \xfile{.dtx} or the \xfile{.drv} to generate
% the documentation. The process can be configured by the
% configuration file \xfile{ltxdoc.cfg}. For instance, put this
% line into this file, if you want to have A4 as paper format:
%
%\paragraph{生成文档。} 你可以使用\xfile{.dtx}或\xfile{.drv}生成文档。这个过程可以由配置文件\xfile{ltxdoc.cfg}配置。例如,如果你想要A4作为纸张格式,将此行放入文件中:

% \begin{quote}
%   \verb|\PassOptionsToClass{a4paper}{article}|
% \end{quote}
% An example follows how to generate the
% documentation with pdf\LaTeX:

%以下是使用pdf\LaTeX 生成文档的示例:
% \begin{quote}
%\begin{verbatim}
%pdflatex pdfcolparcolumns.dtx
%makeindex -s gind.ist pdfcolparcolumns.idx
%pdflatex pdfcolparcolumns.dtx
%makeindex -s gind.ist pdfcolparcolumns.idx
%pdflatex pdfcolparcolumns.dtx
%\end{verbatim}
% \end{quote}
%
% \begin{thebibliography}{9}
%
% \bibitem{parcolumns}
%   Jonathan Sauer: \textit{The \xpackage{parcolumns} package};
%   2004/11/25;\\
%   \CTANpkg{parcolumns}.
%
% \bibitem{pdfcol}
%   Heiko Oberdiek: \textit{The \xpackage{pdfcol} package};
%   2007/09/09;\\
%   \CTANpkg{pdfcol}.
%
% \end{thebibliography}
%
% \begin{History}
%   \begin{Version}{2007/07/26 v1.0}
%   \item
%     First version, published in the newsgroup \xnewsgroup{comp.text.tex}
%     with the name \xpackage{parcolumns-colorstacks}: ^^A no line break
%     \URL{``\link{Re: \xpackage{xcolor} glitches}''}^^A
%     {https://groups.google.com/group/comp.text.tex/msg/56bd897b11bca414}

%第一个版本,以 \xnewsgroup{comp.text.tex} 新闻组发布,名称为 \xpackage{parcolumns-colorstacks}:^^A 没有换行符 
%\URL{``\link{Re: \xpackage{xcolor} glitches}''}^^A
%{https://groups.google.com/group/comp.text.tex/msg/56bd897b11bca414}
%   \end{Version}
%   \begin{Version}{2007/09/09 v1.1}
%   \item
%     CTAN version, package name renamed to \xpackage{pdfcolparcolumns}.
%
%CTAN 版本,将包名改为 \xpackage{pdfcolparcolumns}。
%   \item
%     Uses package \xpackage{pdfcol}.
%
%使用 \xpackage{pdfcol} 包。
%   \item
%     Documentation added.
%
%添加文档。
%   \item
%     Test file added.
%
%添加测试文件。
%   \end{Version}
%   \begin{Version}{2008/08/11 v1.2}
%   \item
%     Code is not changed.
%
%代码未改动。
%   \item
%     URLs updated.
%
%更新 URL。
%   \end{Version}
%   \begin{Version}{2010/01/11 v1.3}
%   \item
%     Fix for rule color.
%
%修复分隔线的颜色。
%   \item
%     New option \xoption{rulebetweencolor} for environment |parcolumns|.
%
%为环境 |parcolumns| 添加新选项 \xoption{rulebetweencolor}。
%   \end{Version}
%   \begin{Version}{2016/05/16 v1.4}
%   \item
%     Documentation updates.
%
%更新文档。
%   \end{Version}
%   \begin{Version}{2019/12/29 v1.5}
%   \item
%     \cs{PassOptionsToPackage} not \cs{PassoptionsToPackage}
%
%\cs{PassOptionsToPackage} 改为不区分大小写的 \cs{PassoptionsToPackage}。
%   \end{Version}
% \end{History}
%
% \PrintIndex
%
% \Finale
\endinput

%        (quote the arguments according to the demands of your shell)
%
% Documentation:
%    (a) If pdfcolparcolumns.drv is present:
%           latex pdfcolparcolumns.drv
%    (b) Without pdfcolparcolumns.drv:
%           latex pdfcolparcolumns.dtx; ...
%    The class ltxdoc loads the configuration file ltxdoc.cfg
%    if available. Here you can specify further options, e.g.
%    use A4 as paper format:
%       \PassOptionsToClass{a4paper}{article}
%
%    Programm calls to get the documentation (example):
%       pdflatex pdfcolparcolumns.dtx
%       makeindex -s gind.ist pdfcolparcolumns.idx
%       pdflatex pdfcolparcolumns.dtx
%       makeindex -s gind.ist pdfcolparcolumns.idx
%       pdflatex pdfcolparcolumns.dtx
%
% Installation:
%    TDS:tex/latex/oberdiek/pdfcolparcolumns.sty
%    TDS:doc/latex/oberdiek/pdfcolparcolumns.pdf
%    TDS:source/latex/oberdiek/pdfcolparcolumns.dtx
%
%<*ignore>
\begingroup
  \catcode123=1 %
  \catcode125=2 %
  \def\x{LaTeX2e}%
\expandafter\endgroup
\ifcase 0\ifx\install y1\fi\expandafter
         \ifx\csname processbatchFile\endcsname\relax\else1\fi
         \ifx\fmtname\x\else 1\fi\relax
\else\csname fi\endcsname
%</ignore>
%<*install>
\input docstrip.tex
\Msg{************************************************************************}
\Msg{* Installation}
\Msg{* Package: pdfcolparcolumns 2019/12/29 v1.5 Color stacks for parcolumns (HO)}
\Msg{************************************************************************}

\keepsilent
\askforoverwritefalse

\let\MetaPrefix\relax
\preamble

This is a generated file.

Project: pdfcolparcolumns
Version: 2019/12/29 v1.5

Copyright (C)
   2007, 2008, 2010 Heiko Oberdiek
   2016-2019 Oberdiek Package Support Group

This work may be distributed and/or modified under the
conditions of the LaTeX Project Public License, either
version 1.3c of this license or (at your option) any later
version. This version of this license is in
   https://www.latex-project.org/lppl/lppl-1-3c.txt
and the latest version of this license is in
   https://www.latex-project.org/lppl.txt
and version 1.3 or later is part of all distributions of
LaTeX version 2005/12/01 or later.

This work has the LPPL maintenance status "maintained".

The Current Maintainers of this work are
Heiko Oberdiek and the Oberdiek Package Support Group
https://github.com/ho-tex/oberdiek/issues


This work consists of the main source file pdfcolparcolumns.dtx
and the derived files
   pdfcolparcolumns.sty, pdfcolparcolumns.pdf, pdfcolparcolumns.ins,
   pdfcolparcolumns.drv, pdfcolparcolumns-test1.tex.

\endpreamble
\let\MetaPrefix\DoubleperCent

\generate{%
  \file{pdfcolparcolumns.ins}{\from{pdfcolparcolumns.dtx}{install}}%
  \file{pdfcolparcolumns.drv}{\from{pdfcolparcolumns.dtx}{driver}}%
  \usedir{tex/latex/oberdiek}%
  \file{pdfcolparcolumns.sty}{\from{pdfcolparcolumns.dtx}{package}}%
%  \usedir{doc/latex/oberdiek/test}%
%  \file{pdfcolparcolumns-test1.tex}{\from{pdfcolparcolumns.dtx}{test1}}%
}

\catcode32=13\relax% active space
\let =\space%
\Msg{************************************************************************}
\Msg{*}
\Msg{* To finish the installation you have to move the following}
\Msg{* file into a directory searched by TeX:}
\Msg{*}
\Msg{*     pdfcolparcolumns.sty}
\Msg{*}
\Msg{* To produce the documentation run the file `pdfcolparcolumns.drv'}
\Msg{* through LaTeX.}
\Msg{*}
\Msg{* Happy TeXing!}
\Msg{*}
\Msg{************************************************************************}

\endbatchfile
%</install>
%<*ignore>
\fi
%</ignore>
%<*driver>
\NeedsTeXFormat{LaTeX2e}
\ProvidesFile{pdfcolparcolumns.drv}%
  [2019/12/29 v1.5 Color stacks for parcolumns (HO)]%
\documentclass{ltxdoc}
\usepackage{holtxdoc}[2011/11/22]
\usepackage[scheme=plain]{ctex}
\setCJKmainfont{方正书宋_GBK}%方正书宋_GBK.TTF  设置文本的中文有衬线字体为“方正书宋_GBK”
\setCJKsansfont{方正黑体简体}%方正黑体_GBK.TTF  设置文本的中文无衬线字体为“方正黑体简体”
\setCJKmonofont{方正书宋简体}%方正仿宋_GBK.TTF  设置文本的中文等宽字体为“方正书宋简体”
\begin{document}
  \DocInput{pdfcolparcolumns.dtx}%
\end{document}
%</driver>
% \fi
%
%
%
% \GetFileInfo{pdfcolparcolumns.drv}
%
% \title{The \xpackage{pdfcolparcolumns} package}
% \date{2019/12/29 v1.5}
% \author{Heiko Oberdiek\thanks
% {Please report any issues at \url{https://github.com/ho-tex/oberdiek/issues}}\and 翻译\\{\tt virhuiai@qq.com}}
%
% \maketitle
%
% \begin{abstract}
%\makebox[0pt]{}
% \begin{parcolumns}[nofirstindent]{2}
%\colchunk{Since version 1.40 \pdfTeX\ supports several color stacks.
%This package uses them to fix color problems in
%package \xpackage{parcolumns}.}
%\colchunk{自从版本1.40起,\pdfTeX 支持多个颜色栈。
%此包使用它们来解决 \xpackage{parcolumns} 包中的颜色问题。}
%\end{parcolumns}



% \end{abstract}
%
% \tableofcontents
%
% \section{Usage\\用法}
%
% \begin{quote}
% |\usepackage{pdfcolparcolumns}|
% \end{quote}
% The package \xpackage{pdfcolparcolumns} loads package \xpackage{parcolums}
% \cite{parcolumns}. If color stacks are available then the
% macros of \xpackage{parcolumns} are patched to add support
% for color stacks.

%包 \xpackage{pdfcolparcolumns} 载入了 \xpackage{parcolumns} \cite{parcolumns} 包。如果有可用的颜色栈,则会对 \xpackage{parcolumns} 的宏进行修补,以添加对颜色栈的支持。

%
% \subsection{Option \xoption{rulebetweencolor}\\选项 \xoption{rulebetweencolor}}
%
% Package \xpackage{pdfcolparcolumns} also fixes the color for the
% rule between columns (if \xoption{rulebetween} is set).
% Default color is \cs{normalcolor}. But this can be changed by using
% option \xoption{rulebetweencolor}. It takes a color specification
% as value. If the value is empty, then the default (\cs{normalcolor})
% is used.
% Examples:

% 包 \xpackage{pdfcolparcolumns} 还修复了列之间分隔线的颜色(如果设置了 \xoption{rulebetween})。默认颜色为 \cs{normalcolor}。但是,可以使用选项 \xoption{rulebetweencolor} 来更改它。它接受一个颜色规范作为值。如果该值为空,则使用默认值(\cs{normalcolor})。例如:

% \begin{quote}
%   |rulebetweencolor=blue|,\\
%   |rulebetweencolor={red}|,\\
%   |rulebetweencolor={}|, \textit{\% \cs{normalcolor} is used}\\
%   |rulebetweencolor=[rgb]{1,0,.5}| \textit{\% see below}
% \end{quote}
% If used inside the optional argument of environment \textsf{parcolumns}
% and the value contains an optional argument, then whole value
% must be put in curly braces:

% 如果在 \textsf{parcolumns} 环境的可选参数中使用,并且值包含可选参数,则整个值必须用花括号括起来:
%\begin{quote}
%\begin{verbatim}
%\begin{parcolumns}[
%  rulebetween,
%  rulebetweencolor={[rgb]{1,0,.5}},
%]{2}
%  ...
%\end{parcolumns}
%\end{verbatim}
%\end{quote}
% This option \xoption{rulebetweencolor} can also be set using
% \cs{setkeys}:

%也可以使用 \cs{setkeys} 来设置选项 \xoption{rulebetweencolor}:
%\begin{quote}
%\begin{verbatim}
%\setkeys{parcolumns}{rulebetweencolor=green}
%\end{verbatim}
%\end{quote}
%
% \subsection{Future\\未来展望}
%
% Currently package \xpackage{parcolumns} does not seem to be
% maintained. Nevertheless if there will be a new version that
% adds support for color stacks, then this package may become
% obsolete.
%
%目前,\xpackage{parcolumns} 包似乎没有维护了。尽管如此,如果出现了新版本并支持颜色栈,那么这个包可能会变得过时。
% \StopEventually{
% }
%
% \section{Implementation\\实现}
%
% \subsection{Identification\\标识}
%
%    \begin{macrocode}
%<*package>
\NeedsTeXFormat{LaTeX2e}
\ProvidesPackage{pdfcolparcolumns}%
  [2019/12/29 v1.5 Color stacks for parcolumns (HO)]%
%    \end{macrocode}
%
% \subsection{Load packages\\加载宏包}
%
% \subsubsection{Package \xpackage{parcolumns}\\宏包 \xpackage{parcolumns}}
%
%    Currently package \xpackage{parcolumns} does not define options.
%    Thus it is just a precaution that the options of
%    package \xpackage{pdfcolparcolumns} are passed to
%    package \xpackage{parcolumns}.

%目前,宏包 \xpackage{parcolumns} 没有定义选项。因此,它只是一个预防措施,以确保将宏包 \xpackage{pdfcolparcolumns} 的选项传递给宏包 \xpackage{parcolumns}。
%    \begin{macrocode}
\DeclareOption*{%
  \PassOptionsToPackage{\CurrentOption}{parcolumns}%
}
\ProcessOptions\relax
\RequirePackage{parcolumns}[2004/11/25]
%    \end{macrocode}
%
% \subsubsection{Package \xpackage{pdfcol}\\宏包 \xpackage{pdfcol}}
%
%    \begin{macrocode}
\RequirePackage{pdfcol}[2007/09/09]
\ifpdfcolAvailable
\else
  \PackageInfo{pdfcolparcolumns}{%
    Loading aborted, because color stacks are not available%
  }%
  \expandafter\endinput
\fi
%    \end{macrocode}
%
% \subsubsection{Package \xpackage{infwarerr}\\宏包 \xpackage{infwarerr}}
%
%    \begin{macrocode}
\RequirePackage{infwarerr}[2007/09/09]
%    \end{macrocode}
%
% \subsection{Color stack macros\\颜色堆栈宏}
%
%    \begin{macro}{\pcpc@MaxStack}
%    Macro \cs{pcpc@MaxStack} holds the highest number of
%    allocated stacks.

%宏 \cs{pcpc@MaxStack} 存储已分配的堆栈中最大的编号。
%    \begin{macrocode}
\global\chardef\pcpc@MaxStack=\z@
%    \end{macrocode}
%    \end{macro}
%    \begin{macro}{\pcpc@InitStacks}
%    Macro \cs{pcpc@InitStacks} takes the number of columns
%    as argument and ensures that there are enough color
%    stacks for all columns.
%
%宏 \cs{pcpc@InitStacks} 接受列数作为参数,并确保为所有列分配了足够的颜色堆栈。
%    \begin{macrocode}
\def\pcpc@InitStacks#1{%
  \ifnum#1>\pcpc@MaxStack
    \begingroup
      \count@\pcpc@MaxStack
      \loop
        \advance\count@\@ne
        \pdfcolInitStack{pcpc@\the\count@}%
      \ifnum#1>\count@
      \repeat
      \global\chardef\pcpc@MaxStack=\count@
    \endgroup
  \fi
}
%    \end{macrocode}
%    \end{macro}
%
%    \begin{macro}{\pcpc@SwitchStack}
%    \begin{macrocode}
\def\pcpc@SwitchStack#1{%
  \pdfcolSwitchStack{pcpc@\number#1}%
}
%    \end{macrocode}
%    \end{macro}
%
%    \begin{macro}{\pcpc@SetCurrent}
%    \begin{macrocode}
\def\pcpc@SetCurrent#1{%
  \pdfcolSetCurrent{pcpc@\number#1}%
}
%    \end{macrocode}
%    \end{macro}
%
% \subsection{Patches\\补丁}
%
%     Now the color stack macros are patched into the macros
%     of package \xpackage{parcolumns}.
%
%现在颜色堆栈宏已经被打补丁到了 \xpackage{parcolumns} 宏包的宏中。
% \subsubsection{Init stacks\\初始化堆栈}
%
%    \cs{pcpc@InitStacks} should go into the definition of
%    environment |parcolumns|. \cs{pc@alloccolumns} is executed
%    there and nowhere else, thus we hook into it.

%\cs{pcpc@InitStacks} 应该被放在环境 |parcolumns| 的定义中。\cs{pc@alloccolumns} 仅在那里执行,因此我们将其连接起来。
%    \begin{macrocode}
\g@addto@macro\pc@alloccolumns{%
  \pcpc@InitStacks\pc@columncount
}
%    \end{macrocode}
%
% \subsubsection{Switch stack\\切换堆栈}
%
%    \cs{pcpc@SwitchStack} should be called by marco \cs{colchunk@}.
%    However it is easier to patch \cs{pc@setcolumnwidth} that
%    is executed in \cs{colchunk@} only.
%
%\cs{pcpc@SwitchStack} 应该被宏 \cs{colchunk@} 调用。不过,我们更容易修补仅在 \cs{colchunk@} 中执行的 \cs{pc@setcolumnwidth}。
%    \begin{macrocode}
\g@addto@macro\pc@setcolumnwidth{%
  \pcpc@SwitchStack\pc@columnctr
}
%    \end{macrocode}
%
% \subsubsection{Set current stack color\\设置当前堆栈颜色}
%
%    \cs{pcpc@SetCurrent} is set at the begin of each line.
%    It must be inserted into \cs{pc@placeboxes}. Unhappily
%    there is no easy way. Therefore we check and
%    redefine \cs{pc@placeboxes}.
%
%\cs{pcpc@SetCurrent} 在每行开始时设置。它必须插入到 \cs{pc@placeboxes} 中。不幸的是,没有简单的方法。因此,我们检查并重新定义 \cs{pc@placeboxes}。
%    \begin{macrocode}
\begingroup
  \def\x{%
    \global\let\@tempa\relax
    \count@\z@
    \hb@xt@\linewidth{%
      \vfuzz30ex %
      \vbadness\@M
      \splittopskip\z@skip
      \loop
      \ifnum\count@<\pc@columncount
        \advance\count@\@ne
        \expandafter\ifvoid\csname pc@column@\number\count@\endcsname
          \hskip\csname pc@column@width@\number\count@\endcsname
        \else
          \expandafter\setbox\expandafter\@tempboxa\expandafter
          \vsplit\csname pc@column@\number\count@\endcsname
              to \dp\strutbox
          \vbox{%
            \unvbox\@tempboxa
          }%
        \fi
        \expandafter\ifvoid\csname pc@column@\number\count@\endcsname
        \else
          \global\let\@tempa\pc@placeboxes
        \fi
        \ifnum\count@<\pc@columncount
          \strut
          \hfill
          \ifpc@rulebetween
            \vrule
            \hfill
          \fi
        \fi
      \repeat
    }%
    \@tempa
  }%
  \ifx\x\pc@placeboxes
  \else
    \@PackageWarningNoLine{pdfcolparcolumns}{%
      Command \string\pc@placeboxes\space has changed.\MessageBreak
      Supported versions of package `parcolumns':\MessageBreak
      \space\space 2004/08/05.\MessageBreak
      The redefinition of \string\pc@placeboxes\space may not%
      \MessageBreak
      behave correctly depending on the changes%
    }%
  \fi
\endgroup
%    \end{macrocode}
%    \begin{macro}{\pc@placeboxes}
%    \begin{macrocode}
\renewcommand*{\pc@placeboxes}{%
  \global\let\@tempa\relax
  \count@\z@
  \hb@xt@\linewidth{%
    \vfuzz30ex %
    \vbadness\@M
    \splittopskip\z@skip
    \loop
    \ifnum\count@<\pc@columncount
      \advance\count@\@ne
      \expandafter\ifvoid\csname pc@column@\number\count@\endcsname
        \hskip\csname pc@column@width@\number\count@\endcsname
      \else
        \expandafter\setbox\expandafter\@tempboxa\expandafter
        \vsplit\csname pc@column@\number\count@\endcsname
            to \dp\strutbox
        \vbox{%
          \pcpc@SetCurrent\count@
          \unvbox\@tempboxa
        }%
      \fi
      \expandafter\ifvoid\csname pc@column@\number\count@\endcsname
      \else
        \global\let\@tempa\pc@placeboxes
      \fi
      \ifnum\count@<\pc@columncount
        \strut
        \hfill
        \ifpc@rulebetween
          \begingroup
            \pcpc@RuleBetweenColor
            \vrule
          \endgroup
          \hfill
        \fi
      \fi
    \repeat
  }%
  \@tempa
}
%    \end{macrocode}
%    \end{macro}
%    \begin{macro}{\pcpc@RuleBetweenColorDefault}
%    \begin{macrocode}
\def\pcpc@RuleBetweenColorDefault{%
  \normalcolor
}
%    \end{macrocode}
%    \end{macro}
%    \begin{macro}{\pcpc@RuleBetweenColor}
%    \begin{macrocode}
\let\pcpc@RuleBetweenColor\pcpc@RuleBetweenColorDefault
%    \end{macrocode}
%    \end{macro}
%    \begin{macrocode}
\define@key{parcolumns}{rulebetweencolor}{%
  \edef\pcpc@temp{#1}%
  \ifx\pcpc@temp\@empty
    \let\pcpc@RuleBetweenColor\pcpc@RuleBetweenColorDefault
  \else
    \edef\pcpc@temp{%
      \noexpand\@ifnextchar[{%
        \def\noexpand\pcpc@RuleBetweenColor{%
          \noexpand\color\pcpc@temp
        }%
        \noexpand\pcpc@GobbleNil
      }{%
        \def\noexpand\pcpc@RuleBetweenColor{%
          \noexpand\color{\pcpc@temp}%
        }%
        \noexpand\pcpc@GobbleNil
      }%
      \pcpc@temp\noexpand\@nil
    }%
    \pcpc@temp
  \fi
}
%    \end{macrocode}
%    \begin{macro}{\pcpc@GobbleNil}
%    \begin{macrocode}
\long\def\pcpc@GobbleNil#1\@nil{}
%    \end{macrocode}
%    \end{macro}
%
%    \begin{macrocode}
%</package>
%    \end{macrocode}
%% \section{Installation\\安装}
%
% \subsection{Download\\下载}
%
% \paragraph{Package.} This package is available on
% CTAN\footnote{\CTANpkg{pdfcolparcolumns}}:

%该软件包可从 CTAN\footnote{\CTANpkg{pdfcolparcolumns}} 下载:
% \begin{description}
% \item[\CTAN{macros/latex/contrib/oberdiek/pdfcolparcolumns.dtx}] The source file.
% \item[\CTAN{macros/latex/contrib/oberdiek/pdfcolparcolumns.pdf}] Documentation.
% \end{description}
%
%
% \paragraph{Bundle.} All the packages of the bundle `oberdiek'
% are also available in a TDS compliant ZIP archive. There
% the packages are already unpacked and the documentation files
% are generated. The files and directories obey the TDS standard.
%
%捆绑包“oberdiek”中的所有软件包也可在符合 TDS 标准的 ZIP 存档中获取。在该存档中,软件包已经被解包,文档文件已经生成,文件和目录符合 TDS 标准。
% \begin{description}
% \item[\CTANinstall{install/macros/latex/contrib/oberdiek.tds.zip}]
% \end{description}
% \emph{TDS} refers to the standard ``A Directory Structure
% for \TeX\ Files'' (\CTANpkg{tds}). Directories
% with \xfile{texmf} in their name are usually organized this way.
%
%\emph{TDS} 指的是标准“用于 \TeX\ 文件的目录结构”(\CTANpkg{tds})。名字中包含\xfile{texmf}的目录通常都是按这种方式组织的。
% \subsection{Bundle installation\\捆绑包安装}
%
% \paragraph{Unpacking.} Unpack the \xfile{oberdiek.tds.zip} in the
% TDS tree (also known as \xfile{texmf} tree) of your choice.
% Example (linux):
%
%\paragraph{解包。}在您选择的 TDS 树(也称为\xfile{texmf}树)中解压\xfile{oberdiek.tds.zip}。例如(在Linux中):
% \begin{quote}
%   |unzip oberdiek.tds.zip -d ~/texmf|
% \end{quote}
%
% \subsection{Package installation\\软件包安装}
%
% \paragraph{Unpacking.} The \xfile{.dtx} file is a self-extracting
% \docstrip\ archive. The files are extracted by running the
% \xfile{.dtx} through \plainTeX:
%
%\paragraph{解包。} \xfile{.dtx} 文件是一个自解压的 \docstrip\ 存档。运行\xfile{.dtx}通过\plainTeX\ 来提取文件:
% \begin{quote}
%   \verb|tex pdfcolparcolumns.dtx|
% \end{quote}
%
% \paragraph{TDS.} Now the different files must be moved into
% the different directories in your installation TDS tree
% (also known as \xfile{texmf} tree):
%
%\paragraph{TDS。}现在,不同的文件必须移动到安装 TDS 树(也称为\xfile{texmf}树)中的不同目录中:
% \begin{quote}
% \def\t{^^A
% \begin{tabular}{@{}>{\ttfamily}l@{ $\rightarrow$ }>{\ttfamily}l@{}}
%   pdfcolparcolumns.sty & tex/latex/oberdiek/pdfcolparcolumns.sty\\
%   pdfcolparcolumns.pdf & doc/latex/oberdiek/pdfcolparcolumns.pdf\\
%   pdfcolparcolumns.dtx & source/latex/oberdiek/pdfcolparcolumns.dtx\\
% \end{tabular}^^A
% }^^A
% \sbox0{\t}^^A
% \ifdim\wd0>\linewidth
%   \begingroup
%     \advance\linewidth by\leftmargin
%     \advance\linewidth by\rightmargin
%   \edef\x{\endgroup
%     \def\noexpand\lw{\the\linewidth}^^A
%   }\x
%   \def\lwbox{^^A
%     \leavevmode
%     \hbox to \linewidth{^^A
%       \kern-\leftmargin\relax
%       \hss
%       \usebox0
%       \hss
%       \kern-\rightmargin\relax
%     }^^A
%   }^^A
%   \ifdim\wd0>\lw
%     \sbox0{\small\t}^^A
%     \ifdim\wd0>\linewidth
%       \ifdim\wd0>\lw
%         \sbox0{\footnotesize\t}^^A
%         \ifdim\wd0>\linewidth
%           \ifdim\wd0>\lw
%             \sbox0{\scriptsize\t}^^A
%             \ifdim\wd0>\linewidth
%               \ifdim\wd0>\lw
%                 \sbox0{\tiny\t}^^A
%                 \ifdim\wd0>\linewidth
%                   \lwbox
%                 \else
%                   \usebox0
%                 \fi
%               \else
%                 \lwbox
%               \fi
%             \else
%               \usebox0
%             \fi
%           \else
%             \lwbox
%           \fi
%         \else
%           \usebox0
%         \fi
%       \else
%         \lwbox
%       \fi
%     \else
%       \usebox0
%     \fi
%   \else
%     \lwbox
%   \fi
% \else
%   \usebox0
% \fi
% \end{quote}
% If you have a \xfile{docstrip.cfg} that configures and enables \docstrip's
% TDS installing feature, then some files can already be in the right
% place, see the documentation of \docstrip.
%
%如果你有一个\xfile{docstrip.cfg}文件配置和启用了\docstrip 的TDS安装功能,那么一些文件可能已经位于正确的位置,具体请参见\docstrip 的文档。
% \subsection{Refresh file name databases\\刷新文件名数据库}
%
% If your \TeX~distribution
% (\TeX\,Live, \mikTeX, \dots) relies on file name databases, you must refresh
% these. For example, \TeX\,Live\ users run \verb|texhash| or
% \verb|mktexlsr|.
%
%如果你的\TeX~发行版(\TeX,Live、\mikTeX,等等)依赖于文件名数据库,你必须刷新它们。例如,\TeX,Live用户运行\verb|texhash|或\verb|mktexlsr|。
% \subsection{Some details for the interested\\一些细节供感兴趣的人使用}
%
% \paragraph{Unpacking with \LaTeX.}
% The \xfile{.dtx} chooses its action depending on the format:

%\paragraph{用\LaTeX 进行解包。} \xfile{.dtx}文件根据格式选择操作:
% \begin{description}
% \item[\plainTeX:] Run \docstrip\ and extract the files.
% \item[\LaTeX:] Generate the documentation.
% \end{description}
% If you insist on using \LaTeX\ for \docstrip\ (really,
% \docstrip\ does not need \LaTeX), then inform the autodetect routine
% about your intention:

%如果你坚持要用\LaTeX 进行\docstrip (其实\docstrip 不需要\LaTeX ),那么请告知自动检测例程你的意图:

% \begin{quote}
%   \verb|latex \let\install=y% \iffalse meta-comment
%
% File: pdfcolparcolumns.dtx
% Version: 2019/12/29 v1.5
% Info: Color stacks for parcolumns
%
% Copyright (C)
%    2007, 2008, 2010 Heiko Oberdiek
%    2016-2019 Oberdiek Package Support Group
%    https://github.com/ho-tex/oberdiek/issues
%
% This work may be distributed and/or modified under the
% conditions of the LaTeX Project Public License, either
% version 1.3c of this license or (at your option) any later
% version. This version of this license is in
%    https://www.latex-project.org/lppl/lppl-1-3c.txt
% and the latest version of this license is in
%    https://www.latex-project.org/lppl.txt
% and version 1.3 or later is part of all distributions of
% LaTeX version 2005/12/01 or later.
%
% This work has the LPPL maintenance status "maintained".
%
% The Current Maintainers of this work are
% Heiko Oberdiek and the Oberdiek Package Support Group
% https://github.com/ho-tex/oberdiek/issues
%
% This work consists of the main source file pdfcolparcolumns.dtx
% and the derived files
%    pdfcolparcolumns.sty, pdfcolparcolumns.pdf, pdfcolparcolumns.ins,
%    pdfcolparcolumns.drv, pdfcolparcolumns-test1.tex.
%
% Distribution:
%    CTAN:macros/latex/contrib/oberdiek/pdfcolparcolumns.dtx
%    CTAN:macros/latex/contrib/oberdiek/pdfcolparcolumns.pdf
%
% Unpacking:
%    (a) If pdfcolparcolumns.ins is present:
%           tex pdfcolparcolumns.ins
%    (b) Without pdfcolparcolumns.ins:
%           tex pdfcolparcolumns.dtx
%    (c) If you insist on using LaTeX
%           latex \let\install=y\input{pdfcolparcolumns.dtx}
%        (quote the arguments according to the demands of your shell)
%
% Documentation:
%    (a) If pdfcolparcolumns.drv is present:
%           latex pdfcolparcolumns.drv
%    (b) Without pdfcolparcolumns.drv:
%           latex pdfcolparcolumns.dtx; ...
%    The class ltxdoc loads the configuration file ltxdoc.cfg
%    if available. Here you can specify further options, e.g.
%    use A4 as paper format:
%       \PassOptionsToClass{a4paper}{article}
%
%    Programm calls to get the documentation (example):
%       pdflatex pdfcolparcolumns.dtx
%       makeindex -s gind.ist pdfcolparcolumns.idx
%       pdflatex pdfcolparcolumns.dtx
%       makeindex -s gind.ist pdfcolparcolumns.idx
%       pdflatex pdfcolparcolumns.dtx
%
% Installation:
%    TDS:tex/latex/oberdiek/pdfcolparcolumns.sty
%    TDS:doc/latex/oberdiek/pdfcolparcolumns.pdf
%    TDS:source/latex/oberdiek/pdfcolparcolumns.dtx
%
%<*ignore>
\begingroup
  \catcode123=1 %
  \catcode125=2 %
  \def\x{LaTeX2e}%
\expandafter\endgroup
\ifcase 0\ifx\install y1\fi\expandafter
         \ifx\csname processbatchFile\endcsname\relax\else1\fi
         \ifx\fmtname\x\else 1\fi\relax
\else\csname fi\endcsname
%</ignore>
%<*install>
\input docstrip.tex
\Msg{************************************************************************}
\Msg{* Installation}
\Msg{* Package: pdfcolparcolumns 2019/12/29 v1.5 Color stacks for parcolumns (HO)}
\Msg{************************************************************************}

\keepsilent
\askforoverwritefalse

\let\MetaPrefix\relax
\preamble

This is a generated file.

Project: pdfcolparcolumns
Version: 2019/12/29 v1.5

Copyright (C)
   2007, 2008, 2010 Heiko Oberdiek
   2016-2019 Oberdiek Package Support Group

This work may be distributed and/or modified under the
conditions of the LaTeX Project Public License, either
version 1.3c of this license or (at your option) any later
version. This version of this license is in
   https://www.latex-project.org/lppl/lppl-1-3c.txt
and the latest version of this license is in
   https://www.latex-project.org/lppl.txt
and version 1.3 or later is part of all distributions of
LaTeX version 2005/12/01 or later.

This work has the LPPL maintenance status "maintained".

The Current Maintainers of this work are
Heiko Oberdiek and the Oberdiek Package Support Group
https://github.com/ho-tex/oberdiek/issues


This work consists of the main source file pdfcolparcolumns.dtx
and the derived files
   pdfcolparcolumns.sty, pdfcolparcolumns.pdf, pdfcolparcolumns.ins,
   pdfcolparcolumns.drv, pdfcolparcolumns-test1.tex.

\endpreamble
\let\MetaPrefix\DoubleperCent

\generate{%
  \file{pdfcolparcolumns.ins}{\from{pdfcolparcolumns.dtx}{install}}%
  \file{pdfcolparcolumns.drv}{\from{pdfcolparcolumns.dtx}{driver}}%
  \usedir{tex/latex/oberdiek}%
  \file{pdfcolparcolumns.sty}{\from{pdfcolparcolumns.dtx}{package}}%
%  \usedir{doc/latex/oberdiek/test}%
%  \file{pdfcolparcolumns-test1.tex}{\from{pdfcolparcolumns.dtx}{test1}}%
}

\catcode32=13\relax% active space
\let =\space%
\Msg{************************************************************************}
\Msg{*}
\Msg{* To finish the installation you have to move the following}
\Msg{* file into a directory searched by TeX:}
\Msg{*}
\Msg{*     pdfcolparcolumns.sty}
\Msg{*}
\Msg{* To produce the documentation run the file `pdfcolparcolumns.drv'}
\Msg{* through LaTeX.}
\Msg{*}
\Msg{* Happy TeXing!}
\Msg{*}
\Msg{************************************************************************}

\endbatchfile
%</install>
%<*ignore>
\fi
%</ignore>
%<*driver>
\NeedsTeXFormat{LaTeX2e}
\ProvidesFile{pdfcolparcolumns.drv}%
  [2019/12/29 v1.5 Color stacks for parcolumns (HO)]%
\documentclass{ltxdoc}
\usepackage{holtxdoc}[2011/11/22]
\usepackage[scheme=plain]{ctex}
\setCJKmainfont{方正书宋_GBK}%方正书宋_GBK.TTF  设置文本的中文有衬线字体为“方正书宋_GBK”
\setCJKsansfont{方正黑体简体}%方正黑体_GBK.TTF  设置文本的中文无衬线字体为“方正黑体简体”
\setCJKmonofont{方正书宋简体}%方正仿宋_GBK.TTF  设置文本的中文等宽字体为“方正书宋简体”
\begin{document}
  \DocInput{pdfcolparcolumns.dtx}%
\end{document}
%</driver>
% \fi
%
%
%
% \GetFileInfo{pdfcolparcolumns.drv}
%
% \title{The \xpackage{pdfcolparcolumns} package}
% \date{2019/12/29 v1.5}
% \author{Heiko Oberdiek\thanks
% {Please report any issues at \url{https://github.com/ho-tex/oberdiek/issues}}\and 翻译\\{\tt virhuiai@qq.com}}
%
% \maketitle
%
% \begin{abstract}
%\makebox[0pt]{}
% \begin{parcolumns}[nofirstindent]{2}
%\colchunk{Since version 1.40 \pdfTeX\ supports several color stacks.
%This package uses them to fix color problems in
%package \xpackage{parcolumns}.}
%\colchunk{自从版本1.40起,\pdfTeX 支持多个颜色栈。
%此包使用它们来解决 \xpackage{parcolumns} 包中的颜色问题。}
%\end{parcolumns}



% \end{abstract}
%
% \tableofcontents
%
% \section{Usage\\用法}
%
% \begin{quote}
% |\usepackage{pdfcolparcolumns}|
% \end{quote}
% The package \xpackage{pdfcolparcolumns} loads package \xpackage{parcolums}
% \cite{parcolumns}. If color stacks are available then the
% macros of \xpackage{parcolumns} are patched to add support
% for color stacks.

%包 \xpackage{pdfcolparcolumns} 载入了 \xpackage{parcolumns} \cite{parcolumns} 包。如果有可用的颜色栈,则会对 \xpackage{parcolumns} 的宏进行修补,以添加对颜色栈的支持。

%
% \subsection{Option \xoption{rulebetweencolor}\\选项 \xoption{rulebetweencolor}}
%
% Package \xpackage{pdfcolparcolumns} also fixes the color for the
% rule between columns (if \xoption{rulebetween} is set).
% Default color is \cs{normalcolor}. But this can be changed by using
% option \xoption{rulebetweencolor}. It takes a color specification
% as value. If the value is empty, then the default (\cs{normalcolor})
% is used.
% Examples:

% 包 \xpackage{pdfcolparcolumns} 还修复了列之间分隔线的颜色(如果设置了 \xoption{rulebetween})。默认颜色为 \cs{normalcolor}。但是,可以使用选项 \xoption{rulebetweencolor} 来更改它。它接受一个颜色规范作为值。如果该值为空,则使用默认值(\cs{normalcolor})。例如:

% \begin{quote}
%   |rulebetweencolor=blue|,\\
%   |rulebetweencolor={red}|,\\
%   |rulebetweencolor={}|, \textit{\% \cs{normalcolor} is used}\\
%   |rulebetweencolor=[rgb]{1,0,.5}| \textit{\% see below}
% \end{quote}
% If used inside the optional argument of environment \textsf{parcolumns}
% and the value contains an optional argument, then whole value
% must be put in curly braces:

% 如果在 \textsf{parcolumns} 环境的可选参数中使用,并且值包含可选参数,则整个值必须用花括号括起来:
%\begin{quote}
%\begin{verbatim}
%\begin{parcolumns}[
%  rulebetween,
%  rulebetweencolor={[rgb]{1,0,.5}},
%]{2}
%  ...
%\end{parcolumns}
%\end{verbatim}
%\end{quote}
% This option \xoption{rulebetweencolor} can also be set using
% \cs{setkeys}:

%也可以使用 \cs{setkeys} 来设置选项 \xoption{rulebetweencolor}:
%\begin{quote}
%\begin{verbatim}
%\setkeys{parcolumns}{rulebetweencolor=green}
%\end{verbatim}
%\end{quote}
%
% \subsection{Future\\未来展望}
%
% Currently package \xpackage{parcolumns} does not seem to be
% maintained. Nevertheless if there will be a new version that
% adds support for color stacks, then this package may become
% obsolete.
%
%目前,\xpackage{parcolumns} 包似乎没有维护了。尽管如此,如果出现了新版本并支持颜色栈,那么这个包可能会变得过时。
% \StopEventually{
% }
%
% \section{Implementation\\实现}
%
% \subsection{Identification\\标识}
%
%    \begin{macrocode}
%<*package>
\NeedsTeXFormat{LaTeX2e}
\ProvidesPackage{pdfcolparcolumns}%
  [2019/12/29 v1.5 Color stacks for parcolumns (HO)]%
%    \end{macrocode}
%
% \subsection{Load packages\\加载宏包}
%
% \subsubsection{Package \xpackage{parcolumns}\\宏包 \xpackage{parcolumns}}
%
%    Currently package \xpackage{parcolumns} does not define options.
%    Thus it is just a precaution that the options of
%    package \xpackage{pdfcolparcolumns} are passed to
%    package \xpackage{parcolumns}.

%目前,宏包 \xpackage{parcolumns} 没有定义选项。因此,它只是一个预防措施,以确保将宏包 \xpackage{pdfcolparcolumns} 的选项传递给宏包 \xpackage{parcolumns}。
%    \begin{macrocode}
\DeclareOption*{%
  \PassOptionsToPackage{\CurrentOption}{parcolumns}%
}
\ProcessOptions\relax
\RequirePackage{parcolumns}[2004/11/25]
%    \end{macrocode}
%
% \subsubsection{Package \xpackage{pdfcol}\\宏包 \xpackage{pdfcol}}
%
%    \begin{macrocode}
\RequirePackage{pdfcol}[2007/09/09]
\ifpdfcolAvailable
\else
  \PackageInfo{pdfcolparcolumns}{%
    Loading aborted, because color stacks are not available%
  }%
  \expandafter\endinput
\fi
%    \end{macrocode}
%
% \subsubsection{Package \xpackage{infwarerr}\\宏包 \xpackage{infwarerr}}
%
%    \begin{macrocode}
\RequirePackage{infwarerr}[2007/09/09]
%    \end{macrocode}
%
% \subsection{Color stack macros\\颜色堆栈宏}
%
%    \begin{macro}{\pcpc@MaxStack}
%    Macro \cs{pcpc@MaxStack} holds the highest number of
%    allocated stacks.

%宏 \cs{pcpc@MaxStack} 存储已分配的堆栈中最大的编号。
%    \begin{macrocode}
\global\chardef\pcpc@MaxStack=\z@
%    \end{macrocode}
%    \end{macro}
%    \begin{macro}{\pcpc@InitStacks}
%    Macro \cs{pcpc@InitStacks} takes the number of columns
%    as argument and ensures that there are enough color
%    stacks for all columns.
%
%宏 \cs{pcpc@InitStacks} 接受列数作为参数,并确保为所有列分配了足够的颜色堆栈。
%    \begin{macrocode}
\def\pcpc@InitStacks#1{%
  \ifnum#1>\pcpc@MaxStack
    \begingroup
      \count@\pcpc@MaxStack
      \loop
        \advance\count@\@ne
        \pdfcolInitStack{pcpc@\the\count@}%
      \ifnum#1>\count@
      \repeat
      \global\chardef\pcpc@MaxStack=\count@
    \endgroup
  \fi
}
%    \end{macrocode}
%    \end{macro}
%
%    \begin{macro}{\pcpc@SwitchStack}
%    \begin{macrocode}
\def\pcpc@SwitchStack#1{%
  \pdfcolSwitchStack{pcpc@\number#1}%
}
%    \end{macrocode}
%    \end{macro}
%
%    \begin{macro}{\pcpc@SetCurrent}
%    \begin{macrocode}
\def\pcpc@SetCurrent#1{%
  \pdfcolSetCurrent{pcpc@\number#1}%
}
%    \end{macrocode}
%    \end{macro}
%
% \subsection{Patches\\补丁}
%
%     Now the color stack macros are patched into the macros
%     of package \xpackage{parcolumns}.
%
%现在颜色堆栈宏已经被打补丁到了 \xpackage{parcolumns} 宏包的宏中。
% \subsubsection{Init stacks\\初始化堆栈}
%
%    \cs{pcpc@InitStacks} should go into the definition of
%    environment |parcolumns|. \cs{pc@alloccolumns} is executed
%    there and nowhere else, thus we hook into it.

%\cs{pcpc@InitStacks} 应该被放在环境 |parcolumns| 的定义中。\cs{pc@alloccolumns} 仅在那里执行,因此我们将其连接起来。
%    \begin{macrocode}
\g@addto@macro\pc@alloccolumns{%
  \pcpc@InitStacks\pc@columncount
}
%    \end{macrocode}
%
% \subsubsection{Switch stack\\切换堆栈}
%
%    \cs{pcpc@SwitchStack} should be called by marco \cs{colchunk@}.
%    However it is easier to patch \cs{pc@setcolumnwidth} that
%    is executed in \cs{colchunk@} only.
%
%\cs{pcpc@SwitchStack} 应该被宏 \cs{colchunk@} 调用。不过,我们更容易修补仅在 \cs{colchunk@} 中执行的 \cs{pc@setcolumnwidth}。
%    \begin{macrocode}
\g@addto@macro\pc@setcolumnwidth{%
  \pcpc@SwitchStack\pc@columnctr
}
%    \end{macrocode}
%
% \subsubsection{Set current stack color\\设置当前堆栈颜色}
%
%    \cs{pcpc@SetCurrent} is set at the begin of each line.
%    It must be inserted into \cs{pc@placeboxes}. Unhappily
%    there is no easy way. Therefore we check and
%    redefine \cs{pc@placeboxes}.
%
%\cs{pcpc@SetCurrent} 在每行开始时设置。它必须插入到 \cs{pc@placeboxes} 中。不幸的是,没有简单的方法。因此,我们检查并重新定义 \cs{pc@placeboxes}。
%    \begin{macrocode}
\begingroup
  \def\x{%
    \global\let\@tempa\relax
    \count@\z@
    \hb@xt@\linewidth{%
      \vfuzz30ex %
      \vbadness\@M
      \splittopskip\z@skip
      \loop
      \ifnum\count@<\pc@columncount
        \advance\count@\@ne
        \expandafter\ifvoid\csname pc@column@\number\count@\endcsname
          \hskip\csname pc@column@width@\number\count@\endcsname
        \else
          \expandafter\setbox\expandafter\@tempboxa\expandafter
          \vsplit\csname pc@column@\number\count@\endcsname
              to \dp\strutbox
          \vbox{%
            \unvbox\@tempboxa
          }%
        \fi
        \expandafter\ifvoid\csname pc@column@\number\count@\endcsname
        \else
          \global\let\@tempa\pc@placeboxes
        \fi
        \ifnum\count@<\pc@columncount
          \strut
          \hfill
          \ifpc@rulebetween
            \vrule
            \hfill
          \fi
        \fi
      \repeat
    }%
    \@tempa
  }%
  \ifx\x\pc@placeboxes
  \else
    \@PackageWarningNoLine{pdfcolparcolumns}{%
      Command \string\pc@placeboxes\space has changed.\MessageBreak
      Supported versions of package `parcolumns':\MessageBreak
      \space\space 2004/08/05.\MessageBreak
      The redefinition of \string\pc@placeboxes\space may not%
      \MessageBreak
      behave correctly depending on the changes%
    }%
  \fi
\endgroup
%    \end{macrocode}
%    \begin{macro}{\pc@placeboxes}
%    \begin{macrocode}
\renewcommand*{\pc@placeboxes}{%
  \global\let\@tempa\relax
  \count@\z@
  \hb@xt@\linewidth{%
    \vfuzz30ex %
    \vbadness\@M
    \splittopskip\z@skip
    \loop
    \ifnum\count@<\pc@columncount
      \advance\count@\@ne
      \expandafter\ifvoid\csname pc@column@\number\count@\endcsname
        \hskip\csname pc@column@width@\number\count@\endcsname
      \else
        \expandafter\setbox\expandafter\@tempboxa\expandafter
        \vsplit\csname pc@column@\number\count@\endcsname
            to \dp\strutbox
        \vbox{%
          \pcpc@SetCurrent\count@
          \unvbox\@tempboxa
        }%
      \fi
      \expandafter\ifvoid\csname pc@column@\number\count@\endcsname
      \else
        \global\let\@tempa\pc@placeboxes
      \fi
      \ifnum\count@<\pc@columncount
        \strut
        \hfill
        \ifpc@rulebetween
          \begingroup
            \pcpc@RuleBetweenColor
            \vrule
          \endgroup
          \hfill
        \fi
      \fi
    \repeat
  }%
  \@tempa
}
%    \end{macrocode}
%    \end{macro}
%    \begin{macro}{\pcpc@RuleBetweenColorDefault}
%    \begin{macrocode}
\def\pcpc@RuleBetweenColorDefault{%
  \normalcolor
}
%    \end{macrocode}
%    \end{macro}
%    \begin{macro}{\pcpc@RuleBetweenColor}
%    \begin{macrocode}
\let\pcpc@RuleBetweenColor\pcpc@RuleBetweenColorDefault
%    \end{macrocode}
%    \end{macro}
%    \begin{macrocode}
\define@key{parcolumns}{rulebetweencolor}{%
  \edef\pcpc@temp{#1}%
  \ifx\pcpc@temp\@empty
    \let\pcpc@RuleBetweenColor\pcpc@RuleBetweenColorDefault
  \else
    \edef\pcpc@temp{%
      \noexpand\@ifnextchar[{%
        \def\noexpand\pcpc@RuleBetweenColor{%
          \noexpand\color\pcpc@temp
        }%
        \noexpand\pcpc@GobbleNil
      }{%
        \def\noexpand\pcpc@RuleBetweenColor{%
          \noexpand\color{\pcpc@temp}%
        }%
        \noexpand\pcpc@GobbleNil
      }%
      \pcpc@temp\noexpand\@nil
    }%
    \pcpc@temp
  \fi
}
%    \end{macrocode}
%    \begin{macro}{\pcpc@GobbleNil}
%    \begin{macrocode}
\long\def\pcpc@GobbleNil#1\@nil{}
%    \end{macrocode}
%    \end{macro}
%
%    \begin{macrocode}
%</package>
%    \end{macrocode}
%% \section{Installation\\安装}
%
% \subsection{Download\\下载}
%
% \paragraph{Package.} This package is available on
% CTAN\footnote{\CTANpkg{pdfcolparcolumns}}:

%该软件包可从 CTAN\footnote{\CTANpkg{pdfcolparcolumns}} 下载:
% \begin{description}
% \item[\CTAN{macros/latex/contrib/oberdiek/pdfcolparcolumns.dtx}] The source file.
% \item[\CTAN{macros/latex/contrib/oberdiek/pdfcolparcolumns.pdf}] Documentation.
% \end{description}
%
%
% \paragraph{Bundle.} All the packages of the bundle `oberdiek'
% are also available in a TDS compliant ZIP archive. There
% the packages are already unpacked and the documentation files
% are generated. The files and directories obey the TDS standard.
%
%捆绑包“oberdiek”中的所有软件包也可在符合 TDS 标准的 ZIP 存档中获取。在该存档中,软件包已经被解包,文档文件已经生成,文件和目录符合 TDS 标准。
% \begin{description}
% \item[\CTANinstall{install/macros/latex/contrib/oberdiek.tds.zip}]
% \end{description}
% \emph{TDS} refers to the standard ``A Directory Structure
% for \TeX\ Files'' (\CTANpkg{tds}). Directories
% with \xfile{texmf} in their name are usually organized this way.
%
%\emph{TDS} 指的是标准“用于 \TeX\ 文件的目录结构”(\CTANpkg{tds})。名字中包含\xfile{texmf}的目录通常都是按这种方式组织的。
% \subsection{Bundle installation\\捆绑包安装}
%
% \paragraph{Unpacking.} Unpack the \xfile{oberdiek.tds.zip} in the
% TDS tree (also known as \xfile{texmf} tree) of your choice.
% Example (linux):
%
%\paragraph{解包。}在您选择的 TDS 树(也称为\xfile{texmf}树)中解压\xfile{oberdiek.tds.zip}。例如(在Linux中):
% \begin{quote}
%   |unzip oberdiek.tds.zip -d ~/texmf|
% \end{quote}
%
% \subsection{Package installation\\软件包安装}
%
% \paragraph{Unpacking.} The \xfile{.dtx} file is a self-extracting
% \docstrip\ archive. The files are extracted by running the
% \xfile{.dtx} through \plainTeX:
%
%\paragraph{解包。} \xfile{.dtx} 文件是一个自解压的 \docstrip\ 存档。运行\xfile{.dtx}通过\plainTeX\ 来提取文件:
% \begin{quote}
%   \verb|tex pdfcolparcolumns.dtx|
% \end{quote}
%
% \paragraph{TDS.} Now the different files must be moved into
% the different directories in your installation TDS tree
% (also known as \xfile{texmf} tree):
%
%\paragraph{TDS。}现在,不同的文件必须移动到安装 TDS 树(也称为\xfile{texmf}树)中的不同目录中:
% \begin{quote}
% \def\t{^^A
% \begin{tabular}{@{}>{\ttfamily}l@{ $\rightarrow$ }>{\ttfamily}l@{}}
%   pdfcolparcolumns.sty & tex/latex/oberdiek/pdfcolparcolumns.sty\\
%   pdfcolparcolumns.pdf & doc/latex/oberdiek/pdfcolparcolumns.pdf\\
%   pdfcolparcolumns.dtx & source/latex/oberdiek/pdfcolparcolumns.dtx\\
% \end{tabular}^^A
% }^^A
% \sbox0{\t}^^A
% \ifdim\wd0>\linewidth
%   \begingroup
%     \advance\linewidth by\leftmargin
%     \advance\linewidth by\rightmargin
%   \edef\x{\endgroup
%     \def\noexpand\lw{\the\linewidth}^^A
%   }\x
%   \def\lwbox{^^A
%     \leavevmode
%     \hbox to \linewidth{^^A
%       \kern-\leftmargin\relax
%       \hss
%       \usebox0
%       \hss
%       \kern-\rightmargin\relax
%     }^^A
%   }^^A
%   \ifdim\wd0>\lw
%     \sbox0{\small\t}^^A
%     \ifdim\wd0>\linewidth
%       \ifdim\wd0>\lw
%         \sbox0{\footnotesize\t}^^A
%         \ifdim\wd0>\linewidth
%           \ifdim\wd0>\lw
%             \sbox0{\scriptsize\t}^^A
%             \ifdim\wd0>\linewidth
%               \ifdim\wd0>\lw
%                 \sbox0{\tiny\t}^^A
%                 \ifdim\wd0>\linewidth
%                   \lwbox
%                 \else
%                   \usebox0
%                 \fi
%               \else
%                 \lwbox
%               \fi
%             \else
%               \usebox0
%             \fi
%           \else
%             \lwbox
%           \fi
%         \else
%           \usebox0
%         \fi
%       \else
%         \lwbox
%       \fi
%     \else
%       \usebox0
%     \fi
%   \else
%     \lwbox
%   \fi
% \else
%   \usebox0
% \fi
% \end{quote}
% If you have a \xfile{docstrip.cfg} that configures and enables \docstrip's
% TDS installing feature, then some files can already be in the right
% place, see the documentation of \docstrip.
%
%如果你有一个\xfile{docstrip.cfg}文件配置和启用了\docstrip 的TDS安装功能,那么一些文件可能已经位于正确的位置,具体请参见\docstrip 的文档。
% \subsection{Refresh file name databases\\刷新文件名数据库}
%
% If your \TeX~distribution
% (\TeX\,Live, \mikTeX, \dots) relies on file name databases, you must refresh
% these. For example, \TeX\,Live\ users run \verb|texhash| or
% \verb|mktexlsr|.
%
%如果你的\TeX~发行版(\TeX,Live、\mikTeX,等等)依赖于文件名数据库,你必须刷新它们。例如,\TeX,Live用户运行\verb|texhash|或\verb|mktexlsr|。
% \subsection{Some details for the interested\\一些细节供感兴趣的人使用}
%
% \paragraph{Unpacking with \LaTeX.}
% The \xfile{.dtx} chooses its action depending on the format:

%\paragraph{用\LaTeX 进行解包。} \xfile{.dtx}文件根据格式选择操作:
% \begin{description}
% \item[\plainTeX:] Run \docstrip\ and extract the files.
% \item[\LaTeX:] Generate the documentation.
% \end{description}
% If you insist on using \LaTeX\ for \docstrip\ (really,
% \docstrip\ does not need \LaTeX), then inform the autodetect routine
% about your intention:

%如果你坚持要用\LaTeX 进行\docstrip (其实\docstrip 不需要\LaTeX ),那么请告知自动检测例程你的意图:

% \begin{quote}
%   \verb|latex \let\install=y\input{pdfcolparcolumns.dtx}|
% \end{quote}
% Do not forget to quote the argument according to the demands
% of your shell.

%不要忘记根据你的shell的要求引用参数。
%
% \paragraph{Generating the documentation.}
% You can use both the \xfile{.dtx} or the \xfile{.drv} to generate
% the documentation. The process can be configured by the
% configuration file \xfile{ltxdoc.cfg}. For instance, put this
% line into this file, if you want to have A4 as paper format:
%
%\paragraph{生成文档。} 你可以使用\xfile{.dtx}或\xfile{.drv}生成文档。这个过程可以由配置文件\xfile{ltxdoc.cfg}配置。例如,如果你想要A4作为纸张格式,将此行放入文件中:

% \begin{quote}
%   \verb|\PassOptionsToClass{a4paper}{article}|
% \end{quote}
% An example follows how to generate the
% documentation with pdf\LaTeX:

%以下是使用pdf\LaTeX 生成文档的示例:
% \begin{quote}
%\begin{verbatim}
%pdflatex pdfcolparcolumns.dtx
%makeindex -s gind.ist pdfcolparcolumns.idx
%pdflatex pdfcolparcolumns.dtx
%makeindex -s gind.ist pdfcolparcolumns.idx
%pdflatex pdfcolparcolumns.dtx
%\end{verbatim}
% \end{quote}
%
% \begin{thebibliography}{9}
%
% \bibitem{parcolumns}
%   Jonathan Sauer: \textit{The \xpackage{parcolumns} package};
%   2004/11/25;\\
%   \CTANpkg{parcolumns}.
%
% \bibitem{pdfcol}
%   Heiko Oberdiek: \textit{The \xpackage{pdfcol} package};
%   2007/09/09;\\
%   \CTANpkg{pdfcol}.
%
% \end{thebibliography}
%
% \begin{History}
%   \begin{Version}{2007/07/26 v1.0}
%   \item
%     First version, published in the newsgroup \xnewsgroup{comp.text.tex}
%     with the name \xpackage{parcolumns-colorstacks}: ^^A no line break
%     \URL{``\link{Re: \xpackage{xcolor} glitches}''}^^A
%     {https://groups.google.com/group/comp.text.tex/msg/56bd897b11bca414}

%第一个版本,以 \xnewsgroup{comp.text.tex} 新闻组发布,名称为 \xpackage{parcolumns-colorstacks}:^^A 没有换行符 
%\URL{``\link{Re: \xpackage{xcolor} glitches}''}^^A
%{https://groups.google.com/group/comp.text.tex/msg/56bd897b11bca414}
%   \end{Version}
%   \begin{Version}{2007/09/09 v1.1}
%   \item
%     CTAN version, package name renamed to \xpackage{pdfcolparcolumns}.
%
%CTAN 版本,将包名改为 \xpackage{pdfcolparcolumns}。
%   \item
%     Uses package \xpackage{pdfcol}.
%
%使用 \xpackage{pdfcol} 包。
%   \item
%     Documentation added.
%
%添加文档。
%   \item
%     Test file added.
%
%添加测试文件。
%   \end{Version}
%   \begin{Version}{2008/08/11 v1.2}
%   \item
%     Code is not changed.
%
%代码未改动。
%   \item
%     URLs updated.
%
%更新 URL。
%   \end{Version}
%   \begin{Version}{2010/01/11 v1.3}
%   \item
%     Fix for rule color.
%
%修复分隔线的颜色。
%   \item
%     New option \xoption{rulebetweencolor} for environment |parcolumns|.
%
%为环境 |parcolumns| 添加新选项 \xoption{rulebetweencolor}。
%   \end{Version}
%   \begin{Version}{2016/05/16 v1.4}
%   \item
%     Documentation updates.
%
%更新文档。
%   \end{Version}
%   \begin{Version}{2019/12/29 v1.5}
%   \item
%     \cs{PassOptionsToPackage} not \cs{PassoptionsToPackage}
%
%\cs{PassOptionsToPackage} 改为不区分大小写的 \cs{PassoptionsToPackage}。
%   \end{Version}
% \end{History}
%
% \PrintIndex
%
% \Finale
\endinput
|
% \end{quote}
% Do not forget to quote the argument according to the demands
% of your shell.

%不要忘记根据你的shell的要求引用参数。
%
% \paragraph{Generating the documentation.}
% You can use both the \xfile{.dtx} or the \xfile{.drv} to generate
% the documentation. The process can be configured by the
% configuration file \xfile{ltxdoc.cfg}. For instance, put this
% line into this file, if you want to have A4 as paper format:
%
%\paragraph{生成文档。} 你可以使用\xfile{.dtx}或\xfile{.drv}生成文档。这个过程可以由配置文件\xfile{ltxdoc.cfg}配置。例如,如果你想要A4作为纸张格式,将此行放入文件中:

% \begin{quote}
%   \verb|\PassOptionsToClass{a4paper}{article}|
% \end{quote}
% An example follows how to generate the
% documentation with pdf\LaTeX:

%以下是使用pdf\LaTeX 生成文档的示例:
% \begin{quote}
%\begin{verbatim}
%pdflatex pdfcolparcolumns.dtx
%makeindex -s gind.ist pdfcolparcolumns.idx
%pdflatex pdfcolparcolumns.dtx
%makeindex -s gind.ist pdfcolparcolumns.idx
%pdflatex pdfcolparcolumns.dtx
%\end{verbatim}
% \end{quote}
%
% \begin{thebibliography}{9}
%
% \bibitem{parcolumns}
%   Jonathan Sauer: \textit{The \xpackage{parcolumns} package};
%   2004/11/25;\\
%   \CTANpkg{parcolumns}.
%
% \bibitem{pdfcol}
%   Heiko Oberdiek: \textit{The \xpackage{pdfcol} package};
%   2007/09/09;\\
%   \CTANpkg{pdfcol}.
%
% \end{thebibliography}
%
% \begin{History}
%   \begin{Version}{2007/07/26 v1.0}
%   \item
%     First version, published in the newsgroup \xnewsgroup{comp.text.tex}
%     with the name \xpackage{parcolumns-colorstacks}: ^^A no line break
%     \URL{``\link{Re: \xpackage{xcolor} glitches}''}^^A
%     {https://groups.google.com/group/comp.text.tex/msg/56bd897b11bca414}

%第一个版本,以 \xnewsgroup{comp.text.tex} 新闻组发布,名称为 \xpackage{parcolumns-colorstacks}:^^A 没有换行符 
%\URL{``\link{Re: \xpackage{xcolor} glitches}''}^^A
%{https://groups.google.com/group/comp.text.tex/msg/56bd897b11bca414}
%   \end{Version}
%   \begin{Version}{2007/09/09 v1.1}
%   \item
%     CTAN version, package name renamed to \xpackage{pdfcolparcolumns}.
%
%CTAN 版本,将包名改为 \xpackage{pdfcolparcolumns}。
%   \item
%     Uses package \xpackage{pdfcol}.
%
%使用 \xpackage{pdfcol} 包。
%   \item
%     Documentation added.
%
%添加文档。
%   \item
%     Test file added.
%
%添加测试文件。
%   \end{Version}
%   \begin{Version}{2008/08/11 v1.2}
%   \item
%     Code is not changed.
%
%代码未改动。
%   \item
%     URLs updated.
%
%更新 URL。
%   \end{Version}
%   \begin{Version}{2010/01/11 v1.3}
%   \item
%     Fix for rule color.
%
%修复分隔线的颜色。
%   \item
%     New option \xoption{rulebetweencolor} for environment |parcolumns|.
%
%为环境 |parcolumns| 添加新选项 \xoption{rulebetweencolor}。
%   \end{Version}
%   \begin{Version}{2016/05/16 v1.4}
%   \item
%     Documentation updates.
%
%更新文档。
%   \end{Version}
%   \begin{Version}{2019/12/29 v1.5}
%   \item
%     \cs{PassOptionsToPackage} not \cs{PassoptionsToPackage}
%
%\cs{PassOptionsToPackage} 改为不区分大小写的 \cs{PassoptionsToPackage}。
%   \end{Version}
% \end{History}
%
% \PrintIndex
%
% \Finale
\endinput
|
% \end{quote}
% Do not forget to quote the argument according to the demands
% of your shell.

%不要忘记根据你的shell的要求引用参数。
%
% \paragraph{Generating the documentation.}
% You can use both the \xfile{.dtx} or the \xfile{.drv} to generate
% the documentation. The process can be configured by the
% configuration file \xfile{ltxdoc.cfg}. For instance, put this
% line into this file, if you want to have A4 as paper format:
%
%\paragraph{生成文档。} 你可以使用\xfile{.dtx}或\xfile{.drv}生成文档。这个过程可以由配置文件\xfile{ltxdoc.cfg}配置。例如,如果你想要A4作为纸张格式,将此行放入文件中:

% \begin{quote}
%   \verb|\PassOptionsToClass{a4paper}{article}|
% \end{quote}
% An example follows how to generate the
% documentation with pdf\LaTeX:

%以下是使用pdf\LaTeX 生成文档的示例:
% \begin{quote}
%\begin{verbatim}
%pdflatex pdfcolparcolumns.dtx
%makeindex -s gind.ist pdfcolparcolumns.idx
%pdflatex pdfcolparcolumns.dtx
%makeindex -s gind.ist pdfcolparcolumns.idx
%pdflatex pdfcolparcolumns.dtx
%\end{verbatim}
% \end{quote}
%
% \begin{thebibliography}{9}
%
% \bibitem{parcolumns}
%   Jonathan Sauer: \textit{The \xpackage{parcolumns} package};
%   2004/11/25;\\
%   \CTANpkg{parcolumns}.
%
% \bibitem{pdfcol}
%   Heiko Oberdiek: \textit{The \xpackage{pdfcol} package};
%   2007/09/09;\\
%   \CTANpkg{pdfcol}.
%
% \end{thebibliography}
%
% \begin{History}
%   \begin{Version}{2007/07/26 v1.0}
%   \item
%     First version, published in the newsgroup \xnewsgroup{comp.text.tex}
%     with the name \xpackage{parcolumns-colorstacks}: ^^A no line break
%     \URL{``\link{Re: \xpackage{xcolor} glitches}''}^^A
%     {https://groups.google.com/group/comp.text.tex/msg/56bd897b11bca414}

%第一个版本,以 \xnewsgroup{comp.text.tex} 新闻组发布,名称为 \xpackage{parcolumns-colorstacks}:^^A 没有换行符 
%\URL{``\link{Re: \xpackage{xcolor} glitches}''}^^A
%{https://groups.google.com/group/comp.text.tex/msg/56bd897b11bca414}
%   \end{Version}
%   \begin{Version}{2007/09/09 v1.1}
%   \item
%     CTAN version, package name renamed to \xpackage{pdfcolparcolumns}.
%
%CTAN 版本,将包名改为 \xpackage{pdfcolparcolumns}。
%   \item
%     Uses package \xpackage{pdfcol}.
%
%使用 \xpackage{pdfcol} 包。
%   \item
%     Documentation added.
%
%添加文档。
%   \item
%     Test file added.
%
%添加测试文件。
%   \end{Version}
%   \begin{Version}{2008/08/11 v1.2}
%   \item
%     Code is not changed.
%
%代码未改动。
%   \item
%     URLs updated.
%
%更新 URL。
%   \end{Version}
%   \begin{Version}{2010/01/11 v1.3}
%   \item
%     Fix for rule color.
%
%修复分隔线的颜色。
%   \item
%     New option \xoption{rulebetweencolor} for environment |parcolumns|.
%
%为环境 |parcolumns| 添加新选项 \xoption{rulebetweencolor}。
%   \end{Version}
%   \begin{Version}{2016/05/16 v1.4}
%   \item
%     Documentation updates.
%
%更新文档。
%   \end{Version}
%   \begin{Version}{2019/12/29 v1.5}
%   \item
%     \cs{PassOptionsToPackage} not \cs{PassoptionsToPackage}
%
%\cs{PassOptionsToPackage} 改为不区分大小写的 \cs{PassoptionsToPackage}。
%   \end{Version}
% \end{History}
%
% \PrintIndex
%
% \Finale
\endinput
|
% \end{quote}
% Do not forget to quote the argument according to the demands
% of your shell.

%不要忘记根据你的shell的要求引用参数。
%
% \paragraph{Generating the documentation.}
% You can use both the \xfile{.dtx} or the \xfile{.drv} to generate
% the documentation. The process can be configured by the
% configuration file \xfile{ltxdoc.cfg}. For instance, put this
% line into this file, if you want to have A4 as paper format:
%
%\paragraph{生成文档。} 你可以使用\xfile{.dtx}或\xfile{.drv}生成文档。这个过程可以由配置文件\xfile{ltxdoc.cfg}配置。例如,如果你想要A4作为纸张格式,将此行放入文件中:

% \begin{quote}
%   \verb|\PassOptionsToClass{a4paper}{article}|
% \end{quote}
% An example follows how to generate the
% documentation with pdf\LaTeX:

%以下是使用pdf\LaTeX 生成文档的示例:
% \begin{quote}
%\begin{verbatim}
%pdflatex pdfcolparcolumns.dtx
%makeindex -s gind.ist pdfcolparcolumns.idx
%pdflatex pdfcolparcolumns.dtx
%makeindex -s gind.ist pdfcolparcolumns.idx
%pdflatex pdfcolparcolumns.dtx
%\end{verbatim}
% \end{quote}
%
% \begin{thebibliography}{9}
%
% \bibitem{parcolumns}
%   Jonathan Sauer: \textit{The \xpackage{parcolumns} package};
%   2004/11/25;\\
%   \CTANpkg{parcolumns}.
%
% \bibitem{pdfcol}
%   Heiko Oberdiek: \textit{The \xpackage{pdfcol} package};
%   2007/09/09;\\
%   \CTANpkg{pdfcol}.
%
% \end{thebibliography}
%
% \begin{History}
%   \begin{Version}{2007/07/26 v1.0}
%   \item
%     First version, published in the newsgroup \xnewsgroup{comp.text.tex}
%     with the name \xpackage{parcolumns-colorstacks}: ^^A no line break
%     \URL{``\link{Re: \xpackage{xcolor} glitches}''}^^A
%     {https://groups.google.com/group/comp.text.tex/msg/56bd897b11bca414}

%第一个版本,以 \xnewsgroup{comp.text.tex} 新闻组发布,名称为 \xpackage{parcolumns-colorstacks}:^^A 没有换行符 
%\URL{``\link{Re: \xpackage{xcolor} glitches}''}^^A
%{https://groups.google.com/group/comp.text.tex/msg/56bd897b11bca414}
%   \end{Version}
%   \begin{Version}{2007/09/09 v1.1}
%   \item
%     CTAN version, package name renamed to \xpackage{pdfcolparcolumns}.
%
%CTAN 版本,将包名改为 \xpackage{pdfcolparcolumns}。
%   \item
%     Uses package \xpackage{pdfcol}.
%
%使用 \xpackage{pdfcol} 包。
%   \item
%     Documentation added.
%
%添加文档。
%   \item
%     Test file added.
%
%添加测试文件。
%   \end{Version}
%   \begin{Version}{2008/08/11 v1.2}
%   \item
%     Code is not changed.
%
%代码未改动。
%   \item
%     URLs updated.
%
%更新 URL。
%   \end{Version}
%   \begin{Version}{2010/01/11 v1.3}
%   \item
%     Fix for rule color.
%
%修复分隔线的颜色。
%   \item
%     New option \xoption{rulebetweencolor} for environment |parcolumns|.
%
%为环境 |parcolumns| 添加新选项 \xoption{rulebetweencolor}。
%   \end{Version}
%   \begin{Version}{2016/05/16 v1.4}
%   \item
%     Documentation updates.
%
%更新文档。
%   \end{Version}
%   \begin{Version}{2019/12/29 v1.5}
%   \item
%     \cs{PassOptionsToPackage} not \cs{PassoptionsToPackage}
%
%\cs{PassOptionsToPackage} 改为不区分大小写的 \cs{PassoptionsToPackage}。
%   \end{Version}
% \end{History}
%
% \PrintIndex
%
% \Finale
\endinput
