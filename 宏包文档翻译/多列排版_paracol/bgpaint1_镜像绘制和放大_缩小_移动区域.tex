
% 
% \newpage
% \backgroundcolor{t(0pt,0pt)(0pt,-4pt)}[rgb]{0.7,0,0}
% \backgroundcolor{b(0pt,-4pt)(0pt,0pt)}[rgb]{0.8,0.6,0}
% \backgroundcolor{l(0pt,4pt)(-4pt,4pt)}[rgb]{0,0,0.7}
% \backgroundcolor{r(-4pt,4pt)(0pt,4pt)}[rgb]{0,0.7,0}
% \backgroundcolor{c[0](4pt,4pt)}[rgb]{1,0.8,1}
% \backgroundcolor{c[1](4pt,4pt)}[rgb]{1,1,0.8}
% \backgroundcolor{g(-4pt,4pt)}[rgb]{0.8,1,1}
% \backgroundcolor{f(4pt,4pt)(4pt,-4pt)}[rgb]{0.8,0,1}
% \backgroundcolor{n(4pt,-4pt)(4pt,4pt)}[rgb]{0.8,0.6,1}
% \backgroundcolor{p(4pt,4pt)}[rgb]{0.8,1,0.6}
% \backgroundcolor{s(4pt,-4pt)}[rgb]{0.8,0.8,0.8}
% 
% \subsection{Mirrored Painting and Enlarging/Shrinking/Shifting Regions\\ 镜像绘制和放大/缩小/移动区域}
% \label{sec:bgpaint-me}
% \twosided
% 
% At a glance, this and the next three pages look painted similarly to
% previous four pages, but by a careful examination you should notice
% two important differences.  The first one is found in the colors
% of left and right margins.  As the author enabled all features of
% \Uidx{\!\twosided!} including |b| for \mirror{}ing and we are now in an
% even-numbered page \pageref{sec:bgpaint-me}, the left and outside margin
% is painted by dark green for the region |r|(ight margin), while the right
% and inside one is painted by dark blue for |l|(eft margin).

% 乍一看,这页和接下来的三页看起来与前面的四页的绘画类似,但是仔细观察你应该会注意到两个重要的区别。第一个区别在于左右边距的颜色。由于作者启用了 \Uidx{\!\twosided!} 的所有特性,包括\mirror{}ing的|b|特性,并且我们现在处于一个偶数页\pageref{sec:bgpaint-me},左边和外部边距由深绿色绘制,表示|r|(ight margin),而右边和内部边距由深蓝色绘制,表示|l|(eft margin)。
% 
% The other is that regions are enlarged, shrunk or shifted by 4\,|pt| by
% the following \!\backgroundcolor! commands with extensions.

另一个是通过以下带有扩展的 \!\backgroundcolor! 命令,通过4,|pt|来扩大、缩小或移动区域。
% \begin{itemize}\item[]
% |\backgroundcolor{t(0pt,0pt)(0pt,-4pt)}[rgb]{0.7,0,0}   |
%     |% B up|\\
% |\backgroundcolor{b(0pt,-4pt)(0pt,0pt)}[rgb]{0.8,0.6,0} |
%     |% T down|\\
% |\backgroundcolor{l(0pt,4pt)(-4pt,4pt)}[rgb]{0,0,0.7}   |
%     |% R left T/B outside|\\
% |\backgroundcolor{r(-4pt,4pt)(0pt,4pt)}[rgb]{0,0.7,0}   |
%     |% L right T/B outside|\\
% |\backgroundcolor{c[0](4pt,4pt)}[rgb]{1,0.8,1}          |
%     |% all edges outside|\\
% |\backgroundcolor{c[1](4pt,4pt)}[rgb]{1,1,0.8}          |
%     |% all edges outside|\\
% |\backgroundcolor{g(-4pt,4pt)}[rgb]{0.8,1,1}            |
%     |% L/R inside & T/B outside|\\
% |\backgroundcolor{f(4pt,4pt)(4pt,-4pt)}[rgb]{0.8,0,1}   |
%     |% L/R outside & T/B up|\\
% |\backgroundcolor{n(4pt,-4pt)(4pt,4pt)}[rgb]{0.8,0.6,1} |
%     |% L/R outside & T/B down|\\
% |\backgroundcolor{p(4pt,4pt)}[rgb]{0.8,1,0.6}           |
%     |% all edges outside|\\
% |\backgroundcolor{s(4pt,-4pt)}[rgb]{0.8,0.8,0.8}        |
%     |% L/R outside & T/B inside|
% \end{itemize}
% 
% \SpecialUsageIndex{\backgroundcolor}

% In the comments above, |L|(eft), |R|(ight), |T|(op) and |B|(ottom) mean
% edges moved by a given extension.  Therefore, for example,
% ``|L/R outside & T/B up|'' for |f|(loat) region means it is enlarged
% horizontally and shifted up vertically by the asymmetric extension
% |(4pt,4pt)(4pt,-4pt)|.  These a little bit complicated setting of
% extensions are to solve the problems in the fundamental example shown in
% previous four pages, namely too strict definition of the regions to be
% painted.  That is, both vertical edges of regions having texts, e.g.,
% |c|(olumn) regions, should look too close to the first and last letters.
% Similarly both horizontal edges of those regions seem too close especially
% when the first line is tall (e.g., the section title in
% p.\Tie\pageref{sec:bgpaint} and the page-wise figure in
% p.\Tie\pageref{page:bgpaint2}) and the last line of a column is followed by
% \mctext{} or \postenv.  Therefore, the author made fine tuning moving
% inside edges of margins outside, and so on.  We will come back this issue
% after exemplifying the effect of the tuning.

在上面的注释中,|L|(eft)、|R|(ight)、|T|(op)和|B|(ottom)表示给定扩展移动的边缘。因此,例如,对于|f|(loat)区域的``|L/R outside & T/B up|''意味着它在水平方向上扩大,在垂直方向上通过不对称扩展|(4pt,4pt)(4pt,-4pt)|向上移动。这些稍微复杂的扩展设置是为了解决前面四页中所示的基本示例中的问题,即对要绘制的区域的定义过于严格。也就是说,具有文本的区域的两个垂直边缘,例如|c|(olumn)区域,看起来离第一个和最后一个字母太近了。同样,当第一行很高时(例如,在p.\Tie\pageref{sec:bgpaint}中的节标题和p.\Tie\pageref{page:bgpaint2}中的每页图)以及一列的最后一行后面跟着\mctext{}或\postenv 时,这些区域的两个水平边缘看起来也太近了。因此,作者对外部边缘内部移动进行了微调等。在示例效果之后,我们将回到这个问题。
% \par\bigskip
% 
% \advance\skip\footins4pt\relax
% \begin{paracol}{2}
% By the tuning to enlarge this |c|(olumn) region, this paragraph has
% comfortable spaces above and below it, as well as at the both side edges.
% 
% \switchcolumn
% \begingroup\it
% This paragraph is surrounded by spaces of a small but comfortable amount as
% well.\footnote{
% 
% Shifting this (foot)|n|(ote) region down a little bit, the space below this
% footnote and above the top edge of the |b|(ottom margin) region is enlarged.}.
% 
% \par\endgroup
% \switchcolumn*[\subsection*{The background of this |s|(panning text)
% region is painted by light gray and enlarged horizontally but shrunk
% vertically}\par\medskip]
% 
% \begin{figure*}\nosv
% \def\arraystretch{0.8}
% \centerline{\begin{tabular}[b]{|c|}\hline
%     \hbox to.9\textwidth{}\\
%     shifting up this \texttt{f}(loat) region gives us a small space above
%     the top edge of the rectangle\\
%     \\\hline
%     \end{tabular}}
% \caption{A Page-Wise Figure}
% \end{figure*}
% 
% This paragraph is to show how well the first line of a paragraph just below a
% \mctext{} is separated from the boundary of two painted regions.
% \par\vfill
% 
% \switchcolumn
% \begingroup\it
% See the right column for the reason why this paragraph is here.
% \par\vfill
% 
% See the right column for what we are now doing.
% \par\endgroup
% \switchcolumn
% 
% By enlarging this |c|(olumn) region and shift the (foot)|n|(ote) region
% down, this paragraph has a comfortable amount of space below it.
% \flushpage
% 
% Similarly to other paragraphs below \pwstuff, this paragraph is well
% separated from the bottom edge of the |f|(loat) region above.
% 
% \par\vfill\label{page:bgpaint-me2}
% 
% As in the case of the line above \Scfnote{}s, the last line of this
% paragraph has a sufficient space separating it from the top edge of the
% |b|(ottom margin) region.
% \par\newpage
% 
% This page is to show how the page without any \pwstuff{} looks like.  As
% you are seeing, the space above this paragraph is sufficient and
% comfortable.
% \par\vfill
% 
% Shortly we will close this \env{paracol} environment in the next page.
% \par\newpage
% 
% Now we are closing this \env{paracol} environment to show how this
% paragraph is separated from the boundary of |c|(olumn) and
% |p|(ost-environment) regions.
% 
% \switchcolumn
% \begingroup\it
% See the comment in the left column for the intention of placing this
% paragraph here.
% \par\vfill
% 
% See the comment in the left column, too.
% \par\newpage
% 
% See the right column for the reason why we have this almost blank page.
% \par\vfill
% 
% See the right column for what will happen shortly.
% \par\newpage
% 
% See the left column for the reason why we are now closing the environment.
% \endgroup
% \end{paracol}
% \bigskip

% The \bground{} of this paragraph in |p|(ost-environment) region is
% painted by pale green as done in p.\Tie\pageref{page:bgpaint4}, but its top
% and bottom edges \emph{look} shifted down and up to give spaces below and
% above the last and first paragraphs in \env{paracol} environments,
% respectively.

这个段落在|p|(ost-environment)区域的\bground{}被涂成了淡绿色,就像在第\pageref{page:bgpaint4}页上所做的那样,但它的顶部和底部边缘\emph{看起来}向下和向上移动了,以在\env{paracol}环境的最后一个段落和第一个段落之上和之下留出空间。
% \par\bigskip
% 
% \begin{paracol}{2}
% This short \env{paracol} environment illustrates how the \preenv{} of this
% environment, or the \postenv{} of the last environment in other words, is
% painted.
% 
% \switchcolumn
% \begingroup\it
% Therefore, the author does not have much to say in this column, except for
% giving a footnote here\footnote{
% 
% As the footnote \ref{fn:bgpaint1} in p.\Tie\pageref{fn:bgpaint1}, this
% \Mgfnote{} is a part of \postenv{} and thus painted by pale green rather
% than light purple.\label{fn:bgpaint-me1}}.
% \endgroup
% \end{paracol}
% \bigskip
% 
% In the setting with \!\backgroundcolor! commands in
% p.\Tie\pageref{sec:bgpaint-me}, the author carefully moved contacting edges
% of regions.  For example, to enlarge |c|(olumn) regions, the inside edges
% of |l|(eft margin) and |r|(ight margin) regions are moved outside, and both
% vertical edges of the |g|(ap) region shifted toward its inside.  As for
% the horizontal edges, the bottom edges of |t|(op margin) and |f|(loat)
% regions are moved up, the top edges of |b|(ottom margin) and
% (foot)|n|(ote) regions are moved down, and both top and bottom edges of
% the |s|(panning text) region are shifted toward its inside.
% 
在设置中,通过在第\Tie\pageref{sec:bgpaint-me}页中使用 \!\backgroundcolor! 命令,作者仔细移动了区域的接触边缘。例如,为了扩大|c|(olumn)区域,将|l|(eft margin)和|r|(ight margin)区域的内部边缘移到外部,并将|g|(ap)区域的两个垂直边缘向内移动。至于水平边缘,将|t|(op margin)和|f|(loat)区域的底部边缘向上移动,将|b|(ottom margin)和(foot)|n|(ote)区域的顶部边缘向下移动,将|s|(panning text)区域的顶部和底部边缘都向内移动。

% These edge shifting could make a region too narrow or too much shifted
% resulting in a material in it overreaching its boundary, especially in
% vertical shifting of horizontal edges.  However we can exploit some large
% space automatically or manually inserted above and/or below the material
% to avoid overreaching.  That is the author exploited the following spaces;
% \!\headsep! below the page head (though it is empty in this document);
% \!\dbltextfloatsep! below the bottom-most page-wise float; spaces that
% \!\subsection!|*| inserts above and below it together with manually
% inserted \!\medskip! below it; \!\skip!\!\footins!\footnote{%
% This is a kind of ``length command'' maybe not widely known.}

这些边缘移动可能会使区域过窄或过多移动,导致其中的内容超出其边界,特别是在水平边缘的垂直移动中。然而,我们可以利用自动或手动插入在材料上方和/或下方的一些较大空间来避免超出。也就是说,作者利用了以下空间:页面头部下方的 \!\headsep!(尽管在本文档中为空);最底部的页面浮动下方的 \!\dbltextfloatsep!;\!\subsection!|*|插入的空间以及其上下的手动插入的 \!\medskip!;在第一个脚注上方的\!\skip!\!\footins!\footnote{这是一种可能不太常见的``长度命令''。},作者临时将其放大了4,|pt|,用于本节和下一节;以及从文本区域的底边到页码的底边的 \!\footskip!。

% above the first footnote which the author enlarged by 4\,|pt| temporarily
% for this and the next subsections; and \!\footskip! from the bottom edge
% of text area to that of the page number.

在第一个脚注之上,作者通过临时将其放大4\,|pt|,为本节和下一节预留了一些空间。此外,\!\footskip! 的高度是从文本区域的底边到页码的底边。


% Now you might notice that the explanation above does not mention the |p|
% region for \Preenv{} and \postenv.  As you should find in the settings,
% this region is enlarged horizontally \emph{and vertically} so that its top
% and bottom edges are moved up and down when the region is at the top or
% bottom of a page, as you are seeing now and find in
% p.\Tie\pageref{sec:bgpaint-me}.  However, this enlargement of course has a
% side effect that the region collides against |c|(olumn) and |g|(ap) regions
% also enlarged vertically making them overlapped.  This overlap will be
% invisible with most of \emph{printers} because, as shown in
% Section\Tie\ref{sec:ref-bgpaint}, |p| region is painted \emph{before} |c|
% and |g| regions are painted.  In addition, since relatively large spaces
% of \!\bigskip! are manually inserted before each \beginparacol{} and after
% each \Endparacol{}, texts in \Preenv{} and \postenv{} are well separated
% from region boundaries.
% 
现在您可能会注意到上面的解释没有提到\Preenv{}和\postenv{}的|p|区域。正如您在设置中找到的那样,这个区域在水平上\emph{和垂直上}被放大,所以当该区域在页面的顶部或底部时,其顶部和底部边缘会向上和向下移动,就像您现在看到的并且在第\pageref{sec:bgpaint-me}页中找到的那样。然而,这种放大当然会产生一个副作用,即该区域与垂直放大的|c|(olumn)和|g|(ap)区域发生碰撞,使它们重叠在一起。这种重叠对于大多数\emph{打印机}来说是看不见的,因为如第\ref{sec:ref-bgpaint}节所示,|p|区域是在|c|和|g|区域之前绘制的。此外,由于在每个\beginparacol{}之前和每个\Endparacol{}之后手动插入了相对较大的 \!\bigskip! 空间,\Preenv{}和\postenv{}中的文本与区域边界之间有很好的分隔。

% This overlay painting |c| and |g| over |p|, however, might produce an
% unexpected result with some printer with which, for example, two colors
% are \emph{blended} in the thin overlapped strip\footnote{%
% For example, a dvi previewer |dviout| produces such a blended result with
% the default setting of coloring.}.

然而,这种|c|和|g|覆盖|p|的叠加绘制可能会在某些打印机上产生意外的结果,例如,在细小的重叠条带中\emph{混合}了两种颜色\footnote{例如,一个dvi预览器|dviout|在默认的着色设置下会产生这样的混合结果。}。

% Unfortunately, this overlay painting is inevitable in the current version
% 1.3, but in a future version, hopefully 1.4, more sophisticated
% \emph{position-dependent} region definition, for example, to shift the top
% edge of |p| region only when the region is at the top of page, could be
% introduced.

不幸的是,在当前的1.3版本中,这种叠加绘制是不可避免的,但在将来的版本中,希望是1.4版本,可以引入更复杂的\emph{位置依赖}区域定义,例如,仅当区域位于页面顶部时才移动|p|区域的顶部边缘。
% 
% Another remark is that the \mirror{}ing specified by the |b| feature of
% \!\twosided! works not only on the colors of side margins but also on
% their asymmetric shrinkage.  That is, the asymmetric shifts of vertical
% edges of |l| and |r| regions correctly performed irrespective of their
% physical positions, i.e., even when the |l| (resp.\ |r|) region is at
% the right (resp.\ left) margin and the edge to be shift is the left
% (resp.\ right) one rather than right (resp.\ left).
% 
另一个要注意的是,\!\twosided!的|b|特性所指定的\mirror{}不仅适用于侧边栏的颜色,也适用于它们的非对称收缩。也就是说,无论左侧(resp.\ 右侧)的|l|(resp.\ |r|)区域是否位于右侧(resp.\ 左侧)边缘,以及待移动的边缘是左侧(resp.\ 右侧)边缘还是右侧(resp.\ 左侧)边缘,|l|和|r|区域的垂直边缘的非对称移动都可以正确进行。
% 