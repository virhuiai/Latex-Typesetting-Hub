
% \newpage \suppressfloats
% \nobackgroundcolor{t}
% \nobackgroundcolor{b}
% \nobackgroundcolor{l}
% \nobackgroundcolor{r}
% \nobackgroundcolor{g}
% \backgroundcolor{c[0](4pt,4pt)(0.5\columnsep,4pt)}[rgb]{1,0.8,1}
% \backgroundcolor{c[1](0.5\columnsep,4pt)(4pt,4pt)}[rgb]{1,1,0.8}
% \backgroundcolor{C[0](10000pt,10000pt)(0.5\columnsep,10000pt)}[rgb]{1,0.8,1}
% \backgroundcolor{C[1](0.5\columnsep,10000pt)(10000pt,10000pt)}[rgb]{1,1,0.8}
% 
% \subsection{Regions with Infinite Extensions}
% \label{sec:bgpaint-inf}
% 
% You are now seeing another \bgpaint{} much different from previous two
% examples.  That is, after disabling painting of |t|, |b|, |l|, |r| and |g|
% regions by \Uidx{\!\nobackgroundcolor!}, the author gave the followings
% for painting this and the next pages.
% 
% \begin{itemize}\item[]
% |\backgroundcolor|
%     |{c[0](4pt,4pt)(0.5\columnsep,4pt)}[rgb]{1,0.8,1}|\\
% |\backgroundcolor|
%     |{c[1](0.5\columnsep,4pt)(4pt,4pt)}[rgb]{1,1,0.8}|\\
% |\backgroundcolor|
%     |{C[0](10000pt,10000pt)(0.5\columnsep,10000pt)}[rgb]{1,0.8,1}|\\
% |\backgroundcolor|
%     |{C[1](0.5\columnsep,10000pt)(10000pt,10000pt)}[rgb]{1,1,0.8}|
% \end{itemize}
% 
% \SpecialUsageIndex{\backgroundcolor}
% 
% The first two lines above is different from the previous declaration
% because inside edges of |c[0]| and |c[1]| regions are shifted toward
% outside of them and thus inside of unpainted |g| region so that the edges
% are contacted.  On the other hand, the last two lines are for
% \emph{under-painting} of columns and has \emph{\bginfext} to make top,
% bottom and outside edges of |C| regions reaching to the corresponding
% paper edges.  Since this under-painting is done with colors same as those
% of over-painting of |c| regions, you will have an impression that the
% paper is two-toned and \pwstuff{} are pasted on the paper\footnote{
% 
% This footnote is given outside \env{paracol} environment but its
% \bground{} is painted by light purple because it is merged with the
% footnote \ref{fn:bgpaint-inf2}.\label{fn:bgpaint-inf1}}.
% 
% \par\bigskip
% 
% \begin{figure}\nosv
% \def\arraystretch{0.8}
% \centerline{\begin{tabular}[b]{|c|}\hline
%     \hbox to.9\textwidth{}\\
%     \parbox{.8\textwidth}{
% 	This \texttt{f}(loat) region could be extended to both side edges
%	and the top edge of the paper if its extension were
%	\texttt{(10000pt,10000pt)(10000pt,-4pt)}.}\\
%     \\\hline
%     \end{tabular}}
% \caption{A Page-Wise Figure \emph{Imported} from Pre-Environment}
% \label{fig:bgpaint-inf}
% \end{figure}
% 
% \begin{paracol}{2}
% Though you cannot see, the right edge of this over-painted |c[0]| region
% is shifted right by 4\,|pt| to hide the small patch at the right bottom
% corner of the |p| region above by overlaying.
% 
% \switchcolumn
% \begingroup\it
% As explained in the right column, this {\rm|c[1]|} region also has an
% invisible left edge shifted left by {\rm4\,|pt|}\footnote{
% 
% This (foot)|n|(ote) region could be extended to both side edges and the
% bottom edge of the paper if its extension were
% \texttt{(10000pt,-4pt)(10000pt,10000pt)}.\label{fn:bgpaint-inf2}}.
% \endgroup
% 
% \switchcolumn*[\subsection*{This \texttt{s}(panning text) region could be
% extended to both side edges of the paper if its extension were
% \texttt{(10000pt,-4pt)}.}\par\medskip]
% 
% The author does not have much to say now for this column chunk.
% \par\vfill
% 
% Still nothing to say particular to the page break we will have shortly.
% \par\newpage
% 
% This paragraph is just for keeping the \env{paracol} environment alive in
% this page.
% \switchcolumn
% 
% \begingroup\it
% Little to say as well.
% \par\vfill
% 
% Nothing to say as well.
% \par\newpage
% 
% This paragraph is not necessary for keeping alive the environment but is
% given for consistent view.
% \endgroup
% 
% \begin{figure*}\nosv
% \def\arraystretch{0.8}
% \centerline{\begin{tabular}[b]{|c|}\hline
%     \hbox to.9\textwidth{}\\
%     \parbox{.8\textwidth}{
% 	This figure is given in the \env{paracol} environment closed in the
%	previous page but its background is not painted.}\\
%     \\\hline
%     \end{tabular}}
% \caption{A Page-Wise Figure \emph{Exported} to Post-Environment}
% \label{fig:bgpaint-inf2}
% \end{figure*}
% \end{paracol}
% \bigskip
% 
% Note that overlay painting is inevitable for two-toned page painting, as
% far as you want to paint \bground{} of \pwstuff.
% 
% The last issue of \bgpaint{} is about painting materials given outside
% \env{paracol}.  As you have seen, \Preenv{} and \postenv{} are painted but
% it is done only when they reside in a page having a portion of a
% \env{paracol} environment (maybe) of course.  Therefore, the next page is
% \emph{not} painted because the page does not have any parallel-columned
% stuff.  Therefore, even if you wish to paint the whole of your document
% including pages without \env{paracol} stuff, you cannot do it just with
% \Paracol{} package, at least so far.
% 
% On the other hand, some materials given outside \env{paracol} environments
% are painted as if they are given in the environment when they are
% \emph{imported} into the environment.  One category has footnotes given in
% \preenv{} when \!\footnotelayout!|{m}| is specified for merging, as
% exemplified by the footnote \ref{fn:bgpaint-inf1} in the previous page.
% Note that such a footnote is painted by the color for |n| region rather
% than |p| region even when there are no footnotes in the \env{paracol}
% environment.  The other category has ordinary floats given by \env{figure}
% and/or \env{table}
% (i.e., neither \env{figure*} nor \env{table*}) environments outside
% \env{paracol} and then \emph{deferred} to a page having (a portion of)
% stuff produced by \env{paracol}.  Since such a float, e.g.,
% Figure\Tie\ref{fig:bgpaint-inf} in this page, is considered as a page-wise
% float given in the \env{paracol} environment in this section, its
% background is painted by the color for the |f| region, rather than that
% for the |p| region which would be used if the float were is placed in the
% previous page.  Note that such a deferred float import could occur not
% only from the page having \beginparacol{} but also from pages preceding
% it.  For example, if you have three \env{figure} environments in a page
% $p-1$ just preceding the page $p$ in which you start a \env{paracol}
% environment, it could happen that first one is placed in $p-1$ without
% painting, the second is placed in $p$ and painted by the color for |p|,
% and the third is placed in $p+1$ and painted by the color for |f|.
% 
% Finally some materials \emph{exported} from a \env{paracol} environment
% are painted as if they are in \postenv.  In previous two subsections, we
% saw \Mgfnote{}s (e.g., \ref{fn:bgpaint1} in p.\Tie\pageref{fn:bgpaint1}
% and \ref{fn:bgpaint-me1} in p.\Tie\pageref{fn:bgpaint-me1}) are painted by
% the color of |p| rather than |n|.  The other kind of exportation is of
% page-wise floats given in a \env{paracol} environment but deferred to the
% page next to the page having \Endparacol, or further.  For example,
% Figure~\ref{fig:bgpaint-inf2} is given in the \env{paracol} environment
% above in this page, but its \bground{} is not painted because the next page
% in which the figure is placed does not have any parallel-columned
% stuff\footnote{
% 
% If it has, the background is painted by the color for |p|.}.
% 
% \newpage\vspace*{\fill}
% \centerline{(intentionally blanked page to show this page is \emph{not}
% painted)}
% \vfill
% \advance\skip\footins-4pt\relax
% \endinput
