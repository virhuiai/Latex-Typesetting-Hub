%% LaTeX file 'paracol-man'.
%% Copyright (C) 2005-2018
%%   Hiroshi Nakashima <h.nakashima@DOMAIN;  DOMAIN=media.kyoto-u.ac.jp>
%%   (Kyoto University)
%% This program can be redistributed and/or modified under the terms
%% of the LaTeX Project Public License distributed from CTAN
%% archives in directory macros/latex/base/lppl.txt; either
%% version 1 of the License, or any later version.

\ProvidesFile{paracol-man.tex}
[2018/12/31 v1.35 ]
\documentclass{ltxdoc}\normalmarginpar
\usepackage[heading=true
,scheme=chinese%中文方案
,fontset=none%不使用默认的字体设置
,space=auto%自动调整中英文间距
]{ctex}
\setCJKmainfont{FangZhengShuSong-GBK-1.ttf}[Path=/Users/virhuiai/hlProjects/Latex-Typesetting-Hub/font/方正/]%设置文本的中文有衬线字体
\setCJKsansfont{FangZhengHeiTi-GBK-1.ttf}[Path=/Users/virhuiai/hlProjects/Latex-Typesetting-Hub/font/方正/]%设置文本的中文无衬线字体为
\setCJKmonofont{FangZhengFangSong-GBK-1.ttf}[Path=/Users/virhuiai/hlProjects/Latex-Typesetting-Hub/font/方正/] %设置文本的中文等宽字体 
% \setCJKfamilyfont{fontKai}{LXGWWenKai-Regular.ttf}[Path=/Users/virhuiai/hlProjects/Latex-Typesetting-Hub/font/霞鹜文楷/]
\setCJKfamilyfont{fontKai}{FangZhengKaiTi-GBK-1.ttf}[Path=/Users/virhuiai/hlProjects/Latex-Typesetting-Hub/font/方正/]
\newcommand\fontKai{\CJKfamily{fontKai}}

\usepackage{color}
\usepackage{paracol}
\usepackage{newvbtm}
\DisableCrossrefs
\PageIndex
\CodelineNumbered
\RecordChanges
\OnlyDescription
\def\ONLYDESCRIPTION{}
\textwidth210mm
\advance\textwidth-40mm \oddsidemargin20mm \advance\oddsidemargin-1in
\columnsep10mm
\marginparwidth20mm \advance\marginparwidth-\marginparsep
\marginparsep.5\marginparsep
\raggedbottom
\OnlyDescription
\begin{document}
\DocInput{paracol.dtx}
\end{document}


% \columnratio{0.3,0.42,0.28}
\begin{paracol}{3}[\section{Introduction\hfill 介绍}]
\begin{VerbatimII}
\columnratio{0.3,0.42,0.28}
\begin{paracol}{3}[\section{Introduction\hfill 介绍}]
\begin{Verbatim}
左侧源码
\end{Verbatim}
\switchcolumn
This document..
\switchcolumn
本文档..
\switchcolumn[1]
Suppose ...
\switchcolumn
假设...
\end{paracol}    
\end{VerbatimII}
%%%%%%%%%%%%%%%%%%%%%%%%%%%%%%%%%%%%%%%%%%%%%%%%%%%%%%%
\switchcolumn
This document describes the usage of yet another multi-column package named
\textsf{paracol}.  The unique feature of the package is that columns are
typeset {\em in parallel.}
\switchcolumn
本文档介绍了另一个名为 \textsf{paracol} 的多栏排版宏包的使用方法。该宏包的独特特点是可以将栏以{\em 并行}的方式排版。 

\switchcolumn[1]
Suppose you are writing a bilingual document whose left column is written in
a language, say English, and right column has the translation of the left
column in another language, e.g., Japanese.  With the \textsf{paracol}
package you may write an English part of arbitrarily length and then {\em
switch} to its Japanese counterpart to place both parts side by side.  Of
course you may return to the English writing similarly.
\switchcolumn
假设你正在撰写一份双语文档,左栏使用一种语言(如英语),右栏则是左栏的另一种语言(如日语)的翻译。使用 \textsf{paracol} 宏包,你可以先写任意长度的英文部分,然后{\em 切换}到对应的日文部分,将两部分并排放置在一起。当然,你也可以类似地返回到英文撰写。

\switchcolumn[1]
The {\em\Uidx\cswitch} is always allowed when you complete an outermost
level paragraph.  You may be unaware whether a column is broken into
multiple pages before switching because the package automatically goes
back and forward to the correct page and vertical position when you switch
the column.  Moreover, you may {\em\Uidx\sync{}e} columns so that the tops
of the first paragraphs after switching in all columns are vertically
aligned.  At a \sync{}ation point, you may give a single-column text,
for example a common section header, optionally.  You may also switch
single-column and multi-column in a page arbitrary.
\switchcolumn
在外层段落完成后,总是允许使用 {\em\Uidx\cswitch} 命令。在切换之前,你可能不知道栏是否被分成多个页面,因为当你切换栏时,宏包会自动回到正确的页面和垂直位置。此外,你可以通过 {\em\Uidx\sync{}e} 命令来使列对齐,这样在切换后,所有列中第一个段落的顶部会垂直对齐。在 \sync{}ation 点,你可以选择给出单栏文本,例如一个公共的章节标题。你还可以随意在页面上切换单栏和多栏排版。

\switchcolumn[1]*
This manual itself is an example of two-column documents typeset by
\textsf{paracol}.  Since the author is not familiar with languages other
than English and Japanese and the latter should be hardly understood by
most of readers, the right column is the translation of the left English
column into a computational language.  That is, the right column is the
\LaTeX{} source code of the left column\footnote{%
Not really but its essence
is shown.\label{fn:first}}.
\switchcolumn
该手册本身是使用 \textsf{paracol} 排版的双栏文档的一个示例。由于作者对英语和日语以外的语言不熟悉,并且后者可能很难被大多数读者理解,所以右栏是左侧英文栏的计算语言翻译。也就是说,右栏是左栏的 \LaTeX{} 源代码\footnote{%
虽然不完全准确,但其本质得以展示。\label{fn:first}virhuiai在翻译时,做成了三栏,第一栏是源代码,第二栏是英文,第三栏是中文。}。
\end{paracol}


\end{document}

\begin{paracol}{2}

\switchcolumn



\switchcolumn*
This manual itself is an example of two-column documents typeset by
\textsf{paracol}.  
\switchcolumn
本手册本身就是使用 \textsf{paracol} 宏包排版的两栏文档的一个示例。

\end{paracol}
\begin{Verbatim}
\begin{paracol}{2}[\section{Introduction}]
\hbadness5000
en.....
\switchcolumn
中文....

\switchcolumn*
en.....
\switchcolumn
中文....
\switchcolumn*[\section{Basic Usage}]....
\end{paracol}
\end{Verbatim}
% 
\section{Basic Usage\hfill 基本用法}
\columnratio{0.2,0.48}
\begin{paracol}{3}
\begin{VerbatimII}
\section{Basic Usage\hfill 基本用法}
\columnratio{0.2,0.48}
\begin{paracol}{3}
\begin{Verbatim}
左侧源码
\end{Verbatim}
\switchcolumn
Loading..
\switchcolumn
加载...
\switchcolumn[1]*
The fundamental...
\switchcolumn
并列...
\end{paracol}
\end{VerbatimII}

\switchcolumn%%%%%%%%%%%%%%%%%%%%%%%%%%%%%%
Loading the package is very simple.  What you have to do is
\!\usepackage!|{|\Uidx{\env{paracol}}|}| in the preamble.  Note that
\textsf{paracol} can be used with \LaTeXe{} and does not work with
\LaTeX{} 2.09.
\switchcolumn
加载该宏包非常简单。在导言区使用 \!\usepackage!|{|\Uidx{\env{paracol}}|}| 命令即可。请注意,\textsf{paracol} 可以与 \LaTeXe{} 一起使用,不支持 \LaTeX{} 2.09。

\switchcolumn[1]
The fundamental means of parallel-column typesetting are the environment
\env{paracol} and the command \Uidx{\!\switchcolumn!}.  The \env{paracol}
environment needs an argument to specify the number of columns.  Thus the
following is the basic construct for two-parallel-column documents.
\switchcolumn
并列栏排版的基本手段是使用 \env{paracol} 环境和命令 \Uidx{\!\switchcolumn!}。\env{paracol} 环境需要一个参数来指定栏的数量。因此,以下是两栏并列文档的基本结构。

\switchcolumn[1]
\begin{quote}
\!\begin!|{|\env{paracol}|}{2}|\\
\textit{left column text}\\
\!\switchcolumn!\\
\textit{right column text}\\
\!\switchcolumn!\\
\textit{left column text}\\
\!\switchcolumn!\\
\textit{right column text}\\
\!\switchcolumn!\\
\mbox{\hspace{4em}}$\vdots$\\
\!\end!|{|\env{paracol}|}|
\end{quote}
\switchcolumn
\begin{quote}
\!\begin!|{|\env{paracol}|}{2}|\\
\textit{左栏文本}\\
\!\switchcolumn!\\
\textit{右栏文本}\\
\!\switchcolumn!\\
\textit{左栏文本}\\
\!\switchcolumn!\\
\textit{右栏文本}\\
\!\switchcolumn!\\
\mbox{\hspace{4em}}$\vdots$\\
\!\end!|{|\env{paracol}|}|
\end{quote}

\switchcolumn[1]*
The \!\switchcolumn! command may have an optional argument to specify the
column number (zero origin) to start.  That is, \!\switchcolumn!|[0]|
means to switch to the leftmost column, |\switchcolumn[1]| is to start the
second column and so on.  Thus the |\switchcolumn| without the optional
argument may be considered as \!\switchcolumn!|[|$i+1\bmod{n}$|]| where
$i$ is the ordinal of the column you are leaving from and $n$ is the
number of columns given to \env{paracol} environment.
\switchcolumn
\!\switchcolumn! 命令可以带有可选参数来指定从第几栏(从零开始计数)开始切换。也就是说,\!\switchcolumn!|[0]| 表示切换到最左边的栏,|\switchcolumn[1]| 表示从第二栏开始,依此类推。因此,不带可选参数的 |\switchcolumn| 可以视为 \!\switchcolumn!|[|$i+1\bmod{n}$|]|,其中 $i$ 是你离开的栏的序号,$n$ 是给定给 \env{paracol} 环境的栏数。

\end{paracol}
% \section{Column Synchronization\\栏同步}\label{sec:sync}

\columnratio{0.3,0.42,0.28}
\begin{paracol}{3}
%%%%%%%%%%%%%%%%%%%%%%%%%%%%%%%%%%%%%%%%%%%%%%%%%%%%%%%
\begin{VerbatimII}
\columnratio{0.3,0.42,0.28}
\begin{paracol}{3}
第1栏
\switchcolumn
第2栏
\switchcolumn
第3栏

\switchcolumn[0]*
同步 ...
\switchcolumn
...
\end{paracol}  
\end{VerbatimII}
%%%%%%%%%%%%%%%%%%%%%%%%%%%%%%%%%%%%%%%%%%%%%%%%%%%%%%%

\switchcolumn
The \!\switchcolumn! command may also be followed by a `|*|' to
{\em\Uidx\sync{}e} columns.  After you switch from a column to another by
\!\switchcolumn!|*| (or \!\switchcolumn!|[|$i$|]*|), all the columns are
vertically aligned at the bottom of the {\em deepest} one preceding the
command.  For example, the previous section has three \!\switchcolumn!|*|
commands at which left and right columns are vertically aligned.
\switchcolumn
\!\switchcolumn! 命令后面可以加上 `|*|',用来{\em 同步}栏。当你使用 \!\switchcolumn!|*|(或 \!\switchcolumn!|[|$i$|]*|)从一栏切换到另一栏时,所有栏都会垂直对齐在该命令之前最{\em 深}的栏的底部。例如,前一节使用了三个 \!\switchcolumn!|*| 命令,使左右两栏垂直对齐。

\switchcolumn[1]
The {\em starred} version of \!\switchcolumn! may have an optional
argument to specify a single-column {\em\Uidx\mctext} whose bottom is the
vertical alignment point of columns.  For example, \!\section!
commands in this manual are given as optional arguments
of \!\switchcolumn!|*| like;
\switchcolumn
{\em 带星号}版本的 \!\switchcolumn! 命令可以带有可选参数,用来指定一个单栏的{\em 同步文本},其底部作为栏的垂直对齐点。例如,本手册中的 \!\section! 命令作为 \!\switchcolumn!|*| 的可选参数给出,如下所示:

\switchcolumn[1]
\begin{Verbatim}
\switchcolumn*[\section{Basic Usage}]
\end{Verbatim}
\switchcolumn
\begin{Verbatim}
\switchcolumn*[\section{基础用法}]
\end{Verbatim}
\switchcolumn[1]
The \env{paracol} environment may also start with a \mctext{} by
specifying it as the optional argument of \!\begin!|{|\env{paracol}|}|.
For example, at the beginning of this document, the author put;
\switchcolumn
\env{paracol} 环境也可以以一个 \mctext{} 开始,将其指定为 \!\begin!|{|\env{paracol}|}| 的可选参数。例如,在本文档的开头,作者使用了以下代码:

\switchcolumn[1]*
\begin{Verbatim}
\begin{paracol}{2}[\section{Introduction}]
\end{Verbatim}

\switchcolumn
\begin{Verbatim}
\begin{paracol}{2}[\section{介绍}]
\end{Verbatim}

\end{paracol}    
\end{document}%%%%%%
% \section{Environments for Columns\hfill 栏环境}\label{sec:env}

\columnratio{0.3,0.42,0.28}
\begin{paracol}{3}
\begin{column}
\begin{VerbatimII}
...
\begin{column*}[\section{Environments for Columns}]
...
\end{column*}
\begin{column}
...
\end{column}
\end{VerbatimII}
%%%%%%%%%%%%%%%%%%%%%%%%%%%%%%%%%%%%%%%%%%%%%%%%%%%%%%%
\end{column}

\begin{column}
\Uidx{\Index{column-switching environment}}
\subsection{Environment \texttt{column}}
The \!\switchcolumn! is simple but you may prefer to pack the contents of a
column in an environment.  The \Uidx{\env{column}} environment is
available for this well-structuralization of \LaTeX{} sources for
parallel-columned documents. A construct;
\end{column}

\begin{column}
\subsection{\ttfamily column环境}
\!\switchcolumn! 命令很简单,但你可能更喜欢将一个栏的内容封装在一个环境中。\Uidx{\env{column}} 环境可以用于在 \LaTeX{} 文档中良好地组织并列栏的内容。以下结构:
\end{column}

\switchcolumn[1]
\begin{quote}
\!\begin!|{|\env{column}|}|\\
\textit{text for a column}\\
\!\end!|{|\env{column}|}|
\end{quote}
\noindent is (almost) equivalent to;
\begin{quote}
\!\switchcolumn!\\
\textit{text for a column}
\end{quote}
\switchcolumn
\begin{quote}
\!\begin!|{|\env{column}|}|\\
\textit{栏中文字}\\
\!\end!|{|\env{column}|}|
\end{quote}
(几乎)等同于:
\begin{quote}
\!\switchcolumn!\\
\textit{栏中文字}
\end{quote}



\end{paracol}


\end{document}%%%%%%





\begin{paracol}{2}
\switchcolumn
\switchcolumn*
The \Uidx{\env{column*}} environment is also available for the column
\sync{}ation and may have an optional argument for \mctext.
\switchcolumn
\Uidx{\env{column*}} 环境也可用于栏的同步,并且可以有一个可选参数用于 \mctext。
\end{paracol}

\begin{paracol}{2}
\begin{nthcolumn}{0}
\subsection{Environment \texttt{nthcolumn}}
The \!\switchcolumn! can start an arbitrarily specified column with the
column number given through its optional argument, but the \env{column}
environment cannot do it.  If you want to start $i$-th column, you have to
do \!\begin!|{|\Uidx{\env{nthcolumn}}|}{|$i$|}| (or
\Uidx{\env{nthcolumn*}} with an optional argument to \sync{}e).
\end{nthcolumn}

\begin{nthcolumn}{1}
\subsection{\texttt{nthcolumn}环境}
\!\switchcolumn! 可以通过可选参数指定要开始的任意列的列号,但 \env{column} 环境不能这样做。如果你想要开始第 $i$ 列,你需要使用 \!\begin!|{|\Uidx{\env{nthcolumn}}|}{|$i$|}|(或带有可选参数的 \Uidx{\env{nthcolumn*}} 来进行同步)。
\end{nthcolumn}
\end{paracol}
\begin{Verbatim}
\begin{paracol}{2}
\begin{nthcolumn*}{1}
\subsection{...}
...
\end{nthcolumn*}

\begin{nthcolumn}{0}
\subsection{...}
...
\end{nthcolumn}
\end{paracol}
\end{Verbatim}


\begin{paracol}{2}
\begin{leftcolumn*}
\subsection[Environments \texttt{leftcolumn} and \texttt{rightcolumn}]
    {Environments \texttt{leftcolumn} and\\\texttt{rightcolumn}}
The environments \Uidx{\env{leftcolumn}} and \Uidx{\env{rightcolumn}} (and
their starred versions with an optional argument) are available as more
convenient means than saying \!\begin!|{|\env{nthcolumn}|}{0}| to switch
to the left(most) column and
\!\begin!|{|\env{nthcolumn}|}{1}| to the right (but may not be rightmost)
one.

\Uidx{\EnvIndex{leftcolumn*}}\Uidx{\EnvIndex{rightcolumn*}}

\end{leftcolumn*}

\begin{rightcolumn}
\subsection{\ttfamily leftcolumn 和 rightcolumn \\环境}
环境 \Uidx{\env{leftcolumn}} 和 \Uidx{\env{rightcolumn}}(以及带有可选参数的星号版本)可作为比使用 \!\begin!|{|\env{nthcolumn}|}{0}| 切换到最左栏 和 \!\begin!|{|\env{nthcolumn}|}{1}| 切换到右栏(可能不是最右)更方便的方法。
\end{rightcolumn}

\end{paracol}
%\section{Floats, Footnotes and Counters}

\columnratio{0.3,0.42,0.28}
\begin{paracol}{3}

\begin{VerbatimII}
\switchcolumn[0]*
\begin{figure*}\nosv
\def\arraystretch{0.8}
\centerline{\begin{tabular}[b]{|c|}\hline
    \hbox to.9\textwidth{}\\
    three-column figure \#1\\
    \\\hline
    \end{tabular}}
\caption{A Three-Column Figure}
\end{figure*}

\switchcolumn
\begin{figure}[t]\nosv
\def\arraystretch{0.8}
\centerline{\begin{tabular}[b]{|c|}\hline
    \hbox to.9\columnwidth{}\\\\
    single-column figure \#1\\
    \\\\\hline
    \end{tabular}}
\caption{A Single-Column Figure}
\end{figure}

\switchcolumn
\begin{figure}[t]\nosv
\def\arraystretch{0.8}
\centerline{\begin{tabular}[b]{|c|}\hline
    \hbox to.9\columnwidth{}\\
    \ttfamily single-column figure \#2\\
    \\\hline
    \end{tabular}}
\caption{\ttfamily Another Single-Column Figure}
\end{figure}
\end{VerbatimII}
%%%%%%%%%%%%%%%%%%%%%%%%%%%%%%%%%%%%%%%%%%%%%%%%%%%%%%%
\switchcolumn
\subsection{Figures and Tables}
Double-column figures\slash tables (or those
spanned multiple columns if you have three or more) may be placed by
\env{figure*} and \env{table*} environments as usual\footnote{
See Section~\ref{sec:problem} for the appearance order issue of
double-column floats.}.

\switchcolumn
\subsection{图表}
双栏图表(如果有三栏或更多栏,则为跨多栏的图表)可以像往常一样使用 \env{figure*} 和 \env{table*} 环境来放置\footnote{请参见第~\ref{sec:problem} 节有关双栏浮动体出现顺序问题的内容。}。

\switchcolumn[1]
A single-column figure\slash table will be placed in the column in which
you put \env{figure} and \env{table}.  For example, the body of a
\env{figure} environment in a \env{leftcolumn} environment is
\emph{always} placed in a left column.  That is, even if the column of the
\emph{current} page does not have enough room to place the figure, it will
not be thrown to the right column but will be placed in the left column of
the next page\footnote{Or some farther page if \LaTeX{} cannot solve the placement problem wisely.}.
\switchcolumn
单栏图表将放置在你放置 \env{figure} 和 \env{table} 环境的栏中。例如,在 \env{leftcolumn} 环境中的 \env{figure} 环境中的内容将始终放置在左栏中。也就是说,即使当前页面的栏没有足够的空间放置图表,它也不会被放置在右栏,而是会放置在下一页的左栏\footnote{如果 \LaTeX{} 无法明智地解决放置问题,则可能放置在更远的页面上。}。

\switchcolumn[1]
\begin{table}[b]\nosv
\caption{A Single-Column Table}
\centerline{\begin{tabular}[t]{|l|c|r|}\hline
An&example&of\\\hline
single&column&table\\\hline
\end{tabular}}
\end{table}
\switchcolumn
\begin{table}[b]\nosv
\caption{\ttfamily Another Single-Column Table}
\label{tab:right}
\centerline{\ttfamily \begin{tabular}[t]{|l|r|}\hline
  Another&example\\\hline
  of&single\\\hline
  column&table\\\hline
  \end{tabular}}
\end{table}

\switchcolumn[1]
Another caution about float placement is that you have to be careful when
you try to put a top-float explicitly with |t|-option or implicitly without
placement option (i.e., |tbp| in most classes) and to \sync{}e columns.
The rule is as follows; after you \sync{}e columns in a page, the page
cannot have top-floats any more.  When you \sync{}e columns,
\textsf{paracol} fixes a virtual horizontal line in the page as the
\sync{}ation barrier.  Thus no top-floats cannot be added above the
line\footnote{Even if you have enough space above, sorry.}.
\switchcolumn
关于浮动位置的另一个警告是,当你试图使用 |t| 选项显式地放置一个顶部浮动,或者不使用放置选项隐式地放置(即,在大多数类中的 |tbp|),并且要同步列时,你必须小心。规则如下:在你在一个页面中同步列后,该页面不能再有顶部浮动。当你同步列时,\textsf{paracol} 在页面中固定一个虚拟的水平线作为同步屏障。因此,不能在该线以上添加顶部浮动\footnote{即使你在上方有足够的空间,抱歉。}。

\switchcolumn[1]
Therefore, the author put two \env{figure} environments for the figures
shown in this page into the \env{leftcolumn*} and \env{rightcolumn}
environment for the previous section.
\switchcolumn
因此,作者将在上一节的 \env{leftcolumn*} 和 \env{rightcolumn} 环境中放入本页显示的两个 \env{figure} 环境。

\switchcolumn[1]
\subsection{Footnotes and Marginal Notes}
Footnotes are also put at the bottom of the column in which \!\footnote!
commands and their references reside (like this\footnote{%
Unless you specify to make footnotes {\em page-wise} as explained in
Section \ref{sec:ref-scfnote} and \ref{sec:fnnp}.}),
\switchcolumn
\subsection{脚注和边注}
脚注也会放置在包含 \!\footnote! 命令及其引用的栏的底部(如本页所示\footnote{除非你在第 \ref{sec:ref-scfnote} 节和 \ref{sec:fnnp} 节中指定将脚注{\em 按页}处理。}),

\switchcolumn[1]
as shown in page~\pageref{fn:first} and this page.  Marginal
notes behave similarly like what you are seeing in the left margin of this
sentence\marginpar{\raggedright An example of marginal note.}
\switchcolumn
如第~\pageref{fn:first}页和本页所示。边注表现类似于你看到的这句话左 margin 中的样式\marginpar{\raggedright 一个边注示例。}

\switchcolumn[1]  
and the right marginal note in this page\footnote{%
If you have three or more columns, marginal notes of the second or
succeeding columns are placed in the right margin in default setting.  The
\textsf{paracol} package solves the placement problem of marginal notes
from two or more columns sharing a side margin by moving some of them down
if they conflict over the space with each other.}.
\switchcolumn
以及本页中的右边距注释\footnote{如果你有三列或更多列,第二列或后续列的边距注释在默认设置中放置在右边距。 \textsf{paracol}包处理来自两个或更多共享侧边距的列的边距注释的放置问题,如果它们在空间上彼此冲突,将其中一些向下移动。}。
\end{paracol}


\columnratio{0.3,0.42,0.28}
\begin{paracol}{3}

\begin{VerbatimII}
\end{VerbatimII}

\switchcolumn[1]
\subsection{Local and Global Counters}
\UsageIndex{local counter}
\UsageIndex{global counter}
You probably found that the numbering of figures and tables is \emph{global}
while that of footnotes are \emph{local}.  That is, the figure in the right
column of the previous page has number~3 following its left-column
counterpart Figure~2.  The tables in the page are also numbered as 1 and 2
crossing the column boundary.  However, the footnotes in each column have
their own numbering sequence.  Moreover, the footnote numbers in left
columns are typeset in roman font while those in right columns have italic
shapes.  Similarly, subsection numbering is local and the headings in right
columns have typewriter-face numbers.
\switchcolumn
\subsection{局部和全局计数器}
你可能发现,图表的编号是\emph{全局}的,而脚注的编号是\emph{局部}的。也就是说,上一页右栏的图表在其左栏对应的图表之后编号为3,而页面上的表格也是以1和2为编号跨越栏边界。然而,每栏中的脚注有自己的编号序列。此外,左栏中的脚注号码以罗马字体排版,而右栏中的脚注号码以斜体形式排版。类似地,小节编号是局部的,右栏标题的编号使用打字机字体。

\switchcolumn[0]*
\begin{itemize}\item[]
\Uidx{\!\globalcounter!}|{figure}|\\
\!\globalcounter!|{table}|
\end{itemize}

\switchcolumn[1]
This happens because the author declared the counters \counter{figure} and
\counter{table} are \emph{global} in the preamble of this document by
saying;
\switchcolumn
这是因为作者在文档的导言部分中声明了计数器 \counter{figure} 和 \counter{table} 是\emph{全局}的,声明首栏所示。

\switchcolumn[1]
and do nothing about \counter{footnote} and \counter{subsection} counters.
By default, all the counters except for |page| are local to columns.  The
value of a \lcounter{} of a column is saved somewhere when you leave the
column, and it is restored when you revisit the column.  The initial values
of the \lcounter{}s are the values they have at
\!\begin!|{|\env{paracol}|}|.  After you close the \env{paracol}
environment, the values of the leftmost column are used for the rest of
your document until you start new \env{paracol} environment.  On a
restart, \lcounter{}s in a column have the values they had at the last
\Endparacol, except for those which have been modified outside the
environment because the modifications are \emph{broadcasted} to
\lcounter{}s in all columns.  You will see the effect of this
inter-environment counter value conservation in the footnote numbers in
the right column in page~\pageref{fn:right3} and \pageref{fn:right4}.
\switchcolumn
但对于计数器 \counter{footnote} 和 \counter{subsection} 未进行任何处理。默认情况下,除了 |page| 计数器外,所有的计数器都是局部的。当你离开栏目时,栏目的局部计数器值会被保存住,当你再次访问该栏目时,该值会被恢复。在 \env{paracol} 环境的初始值为局部计数器的值。当你关闭 \env{paracol} 环境后,剩余部分的文档将使用最左边栏的值,直到你开始新的 \env{paracol} 环境。重新开始时,栏目中的局部计数器具有最后一个 \Endparacol 时的值,除非在环境外进行了修改,因为这些修改会被\emph{广播}到所有栏的局部计数器中。你将在第\pageref{fn:right3}页和第\pageref{fn:right4}页中看到这种跨环境计数值保存的效果,表现在右栏的脚注号码上。

\switchcolumn[1]
This broadcasting of a \lcounter{} value can be done explicitly in
\env{paracol} environments by a command $\Uidx{\!\synccounter!}\Arg{ctr}$.
This command makes $\mathit{ctr}$ in all columns have the value of that in
the column in which the command appears.  In addition, another command
\Uidx{\!\syncallcounters!} performs this broadcasting for all \lcounter{}s.
\switchcolumn
可以在\env{paracol}环境中通过命令$\Uidx{\!\synccounter!}\Arg{ctr}$来显式地进行局部计数器值的广播。所有栏中的$\mathit{ctr}$都将同步为命令调用所在栏中的值。此外,另一个命令 \Uidx{\!\syncallcounters!} 可以对所有局部计数器进行这种广播操作。

\switchcolumn[1]
If you make a counter global by the command \!\globalcounter!, the
save/restore operations are not performed to the counter and thus it is
globally incremented by \verb|\[ref]|\AB|stepcounter|
\SpecialIndex{\refstepcounter}\SpecialIndex{\stepcounter}

\switchcolumn
若用 \!\globalcounter! 将某计数器声明为全局的,则不会对其执行保存/恢复操作,它会通过 \verb|\[ref]|\AB|stepcounter| 全局递增。


\end{paracol}


\end{document}
%%%%%%%%%%%%%%%%%%%%%%%%%%%%%%%%%%%%%%%%%%%%%%%%%%%%%%%
%%%%%%%%%%%%%%%%%%%%%%%%%%%%%%%%%%%%%%%%%%%%%%%%%%%%%%%










\switchcolumn*
or commands such as \!\caption! and \!\section!.  Note that the value of a
\gcounter{} depends on the place where it is incremented (or set) in
the \emph{source code} rather than where it appears in the output.  Thus
if the author put a \env{table} environment here to increment \env{table}
counter, the right-column table at the bottom of page~\pageref{tab:right}
would be Table~3 because its \env{table} environment does not appear yet
in the source code.  Note that, however, though the counter \counter{page}
is global as expected, its numbering is consistent among all columns as
far as you refer to the value by $\!\pageref!\Arg{label}$ and/or see the
values in table of contents, etc.
\switchcolumn
或者诸如 \!\caption! 和 \!\section! 等命令。请注意,一个\gcounter{}的值取决于它在\emph{源代码}中递增(或设置)的位置,而不是它在输出中出现的位置。因此,如果作者在这里放置了一个\env{table}环境来递增\env{table}计数器,那么在第\pageref{tab:right}页底部的右栏表格将被标记为表格3,因为它的\env{table}环境在源代码中尚未出现。请注意,尽管计数器\counter{page}是全局的,但只要通过 $\!\pageref!\Arg{label}$ 引用该值,或者在目录中查看值等,其编号在所有栏目中是一致的。

\switchcolumn*
Another counter which the author made global in this document is
\counter{section}.  As explained in Section~\ref{sec:sync}, an optional
\mctext{} of \cswitch{} is considered as in the leftmost column.  Since
\!\section! commands in this document are always given in \mctext{}s, so
far, it seems unnecessary to make \counter{section} global because it is
incremented correctly in the leftmost column.  However, the stepping
\counter{section} has a side effect to reset its descendent counter
\counter{subsection} and referred to from \!\thesubsection! command.  Thus
if \counter{section} were local, the right-column subsections in
Section~\ref{sec:env} would be numbered as ``0.1'', ``0.2'' and ``0.3''
because the local value of \counter{section} would be zero.  Moreover, the
right-column subsections of this section would be ``0.4'', ``0.5'' and
``0.6'' because stepping \counter{section} local to the left column would
not reset \counter{subsection} local to the right column.
\switchcolumn
在本文档中,作者还将\counter{section}计数器声明为全局的。如第~\ref{sec:sync}节所述,\cswitch{}的可选\mctext{}被视为最左边的栏目。由于本文档中的 \!\section! 命令总是在\mctext{}中给出,因此目前似乎没有必要将\counter{section}设置为全局,因为它在最左边的栏目中递增是正确的。然而,递增\counter{section}会对其子计数器\counter{subsection}产生副作用,并且从 \!\thesubsection! 命令中引用。因此,如果\counter{section}是局部的,那么在第~\ref{sec:env}节中右栏的子章节将被编号为“0.1”、“0.2”和“0.3”,因为\counter{section}的局部值将为零。此外,本节的右栏子章节将被编号为“0.4”、“0.5”和“0.6”,因为局部递增的\counter{section}不会重置右栏局部的\counter{subsection}。


\switchcolumn*
You may give a local appearance to a counter \textit{ctr} for the $i$-th
column (zero origin) by a command;
\begin{itemize}\item[]
\Uidx{\!\definethecounter!}|{|\textit{ctr}|}{|$i$|}{|\textit{def}|}|
\end{itemize}
\switchcolumn
你可以通过命令给第$i$栏目(从零开始计数)的计数器\textit{ctr}赋予局部的外观;

\switchcolumn*
where \textit{def} is to be the body of the local definition of
|\the|\textit{ctr}.  For example, the preamble of this document has the
following to give non-default defitions to \!\thefootnote! and
\!\thesubsection! for right columns.

\begin{Verbatim}
\definethecounter{footnote}{1}{%
\textit{\arabic{footnote}}}
\definethecounter{subsection}{1}{%
\texttt{%
    \arabic{section}.\arabic{subsection}}}
\end{Verbatim}
\switchcolumn
其中\textit{def}是局部定义|\the|\textit{ctr}的内容。例如,本文档的导言部分具有以下内容,为右栏的\!\thefootnote!和\!\thesubsection!赋予非默认的定义。

\end{paracol}


% \section{Closing \texttt{paracol} Environment and Page Flushing\hfill 关闭 \texttt{paracol} 环境和页面刷新}
\label{sec:man-close}

The final example shown here is this single-column text which the author put
after the \env{paracol} environment above is closed.  As you are seeing, a
\env{paracol} environment can be finished at any vertical position in a
page and can be followed by ordinary single column texts.


这里展示的最后一个例子是在上面关闭的\env{paracol} 环境之后,作者放置的这个单栏文本。正如你所见,\env{paracol} 环境可以在页面的任何垂直位置结束,并且可以跟随普通的单栏文本。

\columnratio{0.3,0.42,0.28}
\begin{paracol}{3}
\begin{VerbatimII}
\begin{paracol}{2}
\begin{leftcolumn}
The enviro ... 
\end{leftcolumn}
\begin{rightcolumn}
source
\end{rightcolumn}
\end{paracol}
Now the aurthor will do ...
\end{VerbatimII}
%%%%%%%%%%%%%%%%%%%%%%%%%%%%%%%%%%%%%%%%%%%%%%%%%%%%%%%
\switchcolumn

\switchcolumn[1]
The environment may also be restarted anywhere you like as shown here.
\switchcolumn
此处展示了环境可以在任何位置重新开始。

\switchcolumn[1]
The last issue is to flush a page.  The ordinary \!\newpage! command works
as you expect.  If you say \!\newpage! in the left column in a page, the
contents following it will appear in the left column in the next page.  Note
that this does not affect the layout of the right column.
\switchcolumn
最后一个问题是如何换页。\!\newpage! 命令按照你的期望工作。如果你在页面的左栏使用 \!\newpage! 命令,在它之后的内容将出现在下一页的左栏中。请注意,这不会影响右栏的布局。

\switchcolumn[1]
To flush all columns in a page, a command \Uidx{\!\flushpage!} is
available.  This command in $i$-th column is almost equivalent to;
\begin{itemize}\item[]
\!\switchcolumn!|[|$i$|]*[|\!\newpage!|]|
\end{itemize}
\switchcolumn
要在页面中刷新所有栏目,可以使用命令 \Uidx{\!\flushpage!}。这个命令在第$i$栏中几乎等同于:
\begin{itemize}\item[]
\!\switchcolumn!|[|$i$|]*[|\!\newpage!|]|
\end{itemize}

\switchcolumn[1]
but more robust\footnotemark\label{fn:flush}.
The ordinary page breaking command \Uidx{\!\clearpage!} may also be used
to flush all columns and to start a fresh page, but it has a side effect
to put all figures and tables which are not yet output.
\switchcolumn
但更加健壮 \footnotemark\label{fn:flush} 。普通的换页命令 \Uidx{\!\clearpage!} 也可以用于刷新所有栏目并开始新的一页,但它会导致尚未输出的所有图表被放置在同一页中。
\end{paracol}

Now the author will do |\flushpage| shortly to start a real binlingual
example from the next page, after showing another example of closing 
\env{paracol} environments in this sentence and of restarting in the next
one, in which {\em unbalanced column width} is demonstrated using
\Uidx{\!\columnratio!} command shown in Section~\ref{sec:ref-colwidth}.

现在作者将很快使用|\flushpage|命令,在下一页开始一个真正的双语示例,此前在本句中展示了另一个关闭\env{paracol}环境的例子,并在下一句中重新开始,在其中使用了在第~\ref{sec:ref-colwidth}节中展示的 \Uidx{\!\columnratio!} 命令演示了{\em 不平衡的列宽}。

\columnratio{0.3,0.42,0.28}
\begin{paracol}{3}
\begin{VerbatimII}
\columnratio{0.6} 
\begin{paracol}{2} 
\begin{leftcolumn}
O.K., ... 
\end{leftcolumn} 
\begin{rightcolumn}
source 
\end{rightcolumn}
\end{VerbatimII}
%%%%%%%%%%%%%%%%%%%%%%%%%%%%%%%%%%%%%%%%%%%%%%%%%%%%%%%
\switchcolumn

O.K., we have restarted \env{paracol} environment and we will see the
effect of \!\flushpage! now!!\footnotetext{
For example \texttt{\string\switchcolumn*} may flush a page for the
\sync{}ation and thus \texttt{\string\newpage} may leave an empty page.}
\switchcolumn
好的,我们已经重新开始了\env{paracol}环境,现在我们将看到 \!\flushpage! 命令的效果!!\footnotetext{例如,\texttt{\string\switchcolumn*}可能会为同步而刷新页面,因此\texttt{\string\newpage}可能会留下一个空白页。}

\newenvironment{Gverse}{\ensurevspace{2\baselineskip}
\begin{leftcolumn*}
\begin{myverse}}
{\end{myverse}\end{leftcolumn*}}
\newenvironment{Everse}{%
\begin{rightcolumn}
\begin{myverse}}
{\end{myverse}\end{rightcolumn}}
\newenvironment{Cverse}{%
\begin{nthcolumn}{2}
\begin{myverse}}
{\end{myverse}\end{nthcolumn}}
\makeatletter
\newenvironment{myverse}{\leftmargini0pt\partopsep0pt\verse}{\endverse}


\begin{leftcolumn*}[
\centerline{\Large An Die Freude/To Joy}\label{page:bfreude}\smallskip
\centerline{\large Friedrich Schiller 弗里德里希·席勒}\smallskip
The following is the libretto of the fourth movement of Beethoven's Ninth
Symphony, his adaptation of Schiller's ode ``An Die Freude'' (or ``To Joy'' in
English). Beethoven's additions and revisions are indicated in italics.

以下是贝多芬第九交响曲第四乐章的歌剧剧本,他改编自席勒的颂歌《致欢乐》(或英文版的《To Joy》)。贝多芬的添加和修订以斜体显示。]
\end{leftcolumn*}

\begin{Gverse}
\itshape O Freunde, nicht diese T\"one! \\
Sondern la{\ss}t uns angenehmere anstimmen und freu\-denvollere
\footnote{If I had been a good student in my German class, I could find
the German translation of the right column footnote \ref{fn:right4} is
``Dieser Teil wurde van Beethoven hinzugef\"ugt'' by myself without
the kind help from a user.}.
\end{Gverse}
\begin{Everse}
\itshape Oh friends, no more of these sad tones!\\
Let us rather raise our voices together\\
In more pleasant and joyful tones
\footnote{This part was added by Beethoven.\label{fn:right4}}.
\end{Everse}
\begin{Cverse}
    a
\end{Cverse}


\end{paracol}


\end{document}
%%%%%%%%%%%%%%%%%%%%%%%%%%%%%%%%%%%%%%%%%%%%%%%%%%%%%%%
%%%%%%%%%%%%%%%%%%%%%%%%%%%%%%%%%%%%%%%%%%%%%%%%%%%%%%%


\begin{leftcolumn}




% \subsection{Environment \texttt{paracol}\hfill \texttt{paracol}环境}
\label{sec:ref-paracol}

\begin{description}%值得学习
\item[\ENV{paracol}{\marg{num}\oarg{text}}]\mbox{}\par
\columnratio{0.6}
\begin{paracol}{2}
The environment \env{paracol} contains \meta{body} typeset in \meta{num}
columns in parallel.  The optional \meta{text} is put spanning all columns
prior to the multi-columned \meta{body}.
\switchcolumn
环境\env{paracol}中包含以 \meta{num} 栏并列排列的 \meta{body}。可选的 \meta{text} 将跨越所有栏之前放置在多栏的 \meta{body} 之前。
\end{paracol}

\begin{itemize}
\columnratio{0.6}
\begin{paracol}{2}
\item
The environment may start from \emph{any} vertical position in a page,
i.e., not necessary at the top of a page.  The single-column
{\em\Uidx\preenv} of the {\em\Uidx\spage} in which \beginparacol{} lies
are naturally connected to the beginning part of \meta{body} in each
column, unless the page has footnotes\footnote{%
With \Mgfnote{} layout shown in Section~\ref{sec:ref-scfnote}, the
footnotes in the single-column contents are merged with those in
\env{paracol} environment and are put at the bottom of the \spage{}
together as shown in this page.}

or bottom floats.  If these kinds of bottom stuff exist, they are
put above the multi-columned \meta{body}, or the spanning \meta{text}

\UsageIndex{spanning text}

if provided, with a vertical skip of \!\textfloatsep! separating them if
bottom floats exist, or of \!\belowfootnoteskip! described in
Section~\ref{sec:ref-scfnote} if only footnotes exist.  The
\emph{deferred} floats which have not yet appeared in the starting page
and thus will appear in the next or succeeding pages are considered as
\pwise{} floats given in the environment.
\switchcolumn
\item
此环境可以从页面的\emph{任何}垂直位置开始,
即不一定在页面顶部。位于 \beginparacol{} 所在的 {\em\Uidx\spage} 中的单栏 {\em\Uidx\preenv} 自然与每个栏的\meta{body}的开头部分连接在一起,除非页面有脚注\footnote{使用在第~\ref{sec:ref-scfnote}节中展示的\Mgfnote{}布局,单栏内容中的脚注与\env{paracol}环境中的脚注合并在一起,并一起放置在\spage{}的底部,就像本页所示。}
,或底部浮动体。如果存在这些底部内容,则它们将位于多栏的\meta{body}之上,或者位于跨越的\meta{text}之上(如果提供了),并使用垂直间距 \!\textfloatsep! 将它们分隔开(如果存在底部浮动体),或者使用在第~\ref{sec:ref-scfnote}节中描述的 \!\belowfootnoteskip! (仅当存在脚注时)。尚未出现在起始页面中的\emph{延迟}浮动体将被视为在环境中给出的\pwise{}浮动体,它们将出现在下一页或后续页面中。
\switchcolumn[0]*
\item
The environment can be enclosed in a \env{list}{\em-like environment} such
as \env{enumerate}, \env{itemize} and \env{description}.  If so, \!\item!s
in each column are typeset using the parameters of the surrounding
environment such as \!\leftmargin! and \!\rightmargin!.  %For example, the
following short \env{paracol} environment is included in an \env{itemize}
for this and other \!\item!s in this page.
\switchcolumn
\item
该环境可以被封装在类似于 \env{enumerate}、\env{itemize} 和 \env{description} 的\emph{类似列表}环境中。如果这样做,每个栏中的 \!\item! 将使用周围环境的参数进行排版,如 \!\leftmargin! 和 \!\rightmargin!。%例如,以下简短的\env{paracol}环境被包含在一个\env{itemize}中,用于本页和其他 \!\item!。

\end{paracol}

You are now seeing the switching to/from multi-columned and \env{itemize}d
texts are naturally connected with the last and this single-columned
sentences.  You may feel the space between two columns above is too large
but it simply results from the large total \!\leftmargin!s of the outer
\env{description} and this \env{itemize}, which make the right column
shifted right.  A simple remedy for this large space is to make
\!\columnsep! narrower, for example 0\,pt as shown below.

您现在看到的切换到/从多栏和\env{itemize}文本与上一个和本个单栏句子自然连接在一起。您可能会觉得上面两栏之间的空间太大,但这只是由于外部\env{description}和此\env{itemize}的总 \!\leftmargin! 较大,使得右栏向右偏移。修复这个大空间的简单方法是使 \!\columnsep!变窄,例如像下面显示的0\,pt。

\begin{Verbatim}
\columnsep0pt
\end{Verbatim}

\columnsep0pt
\begin{paracol}{2}
\item
This \!\item! is wider than the last \!\item! above because
\!\columnsep! is 0\,pt.

这个 \!\item! 比上面的最后一个 \!\item! 更宽,因为 \!\columnsep! 是0\,pt。
\switchcolumn

\item
Therefore, this \!\item! is shifted left a little bit to make
inter-column spece narrower.

因此,为了使栏间距更窄,这个 \!\item! 向左移动了一点。
\end{paracol}

\columnratio{0.55}
\begin{paracol}{2}
\item
All \Uidx\lcounter{}s in all columns are initialized to have the values at
\beginparacol{} on its first occurrence.  On the second and succeeding
occurrences of \beginparacol, the \lcounter{}s in each column have the
value at the last \Endparacol, unless they are modified after the
\Endparacol.  If a counter is modified (or declared by \!\newcounter!)
after the \Endparacol, the local versions of the counter in all columns
commonly have the value at \beginparacol.
\switchcolumn
\item
所有栏中的局部计数器都被设为\beginparacol{}首次出现时的值。在\beginparacol{}的第二次及后续出现中,每个栏中的局部计数器都具有上一个\Endparacol{}处的值,除非在\Endparacol{}之后对其进行了修改。如果在\Endparacol{}之后修改了计数器(或通过 \!\newcounter! 声明了计数器),所有栏中的局部计数器都通常具有\beginparacol{}处的值。
\switchcolumn[0]*
\item
The environment may end at \emph{any} vertical position in a page, i.e.,
the {\em\Uidx\postenv} being the single-column texts and others
following \Endparacol{} in the {\em\Uidx\lpage} of the environment may not
start from the top of a page.  If any columns don't have deferred
\cwise{} floats and the most advanced {\em\Uidx\lcolumn} at
\Endparacol{} has neither of footnotes\footnote{

With \Mgfnote{} layout shown in Section~\ref{sec:ref-scfnote}, the
footnotes in the closing \env{paracol} environment are merged with those
in \postenv{} and are put at the bottom of the page{} together as shown in
this page.}

nor bottom floats, its bottom is naturally connected to the \postenv{}.
If the \lcolumn{} has these kinds of bottom stuff, they are put above the
\postenv{}, with a vertical skip of \!\textfloatsep! separating them if
bottom floats exist.  All deferred \cwise{} floats given in the
environment are flushed before the \postenv{} appears, possibly creating
{\em\Uidx\fcolumn{}s} only with floats.  On the other hand, deferred
\pwise{} floats given in the environment are considered as deferred
(single-) \cwise{} floats given just after \Endparacol.
\switchcolumn
该环境可以在页面的\emph{任何}垂直位置结束,即\emph{\Uidx\postenv}是单栏文本,而在环境的\emph{\Uidx\lpage}中的\Endparacol{}之后的其他内容可能不会从页面顶部开始。如果任何栏没有延迟的\cwise{}浮动体,并且最后一个\Endparacol{}处的\emph{\Uidx\lcolumn}既没有脚注\footnote{使用在第~\ref{sec:ref-scfnote}节中展示的\Mgfnote{}布局,\env{paracol}环境中的脚注与\postenv{}中的脚注合并在一起,并一起放置在页面底部,就像本页所示。},也没有底部浮动体,则其底部自然与\postenv{}连接在一起。如果\lcolumn{}具有这些类型的底部内容,则它们将位于\postenv{}之上,如果存在底部浮动体,则它们之间使用垂直间距 \!\textfloatsep! 分隔开。在\postenv{}出现之前,环境中给出的所有延迟\cwise{}浮动体都会被清除,可能只留下具有浮动体的{\em\Uidx\fcolumn{}s} 。另一方面,环境中给出的延迟\pwise{}浮动体被视为在\Endparacol{}之后立即给出的延迟(单个)\cwise{}浮动体。
\switchcolumn[0]*
\item
The values of all \lcounter{}s in the leftmost column are used as the
initial values of them in the \postenv.
\switchcolumn
\item
左侧栏中所有\lcounter{}的值被用作\postenv{}中对应\lcounter{}的初始值。
\switchcolumn[0]*
\item
The \env{paracol} environment cannot be nested, or you will have an error
message of illegal nesting.
\switchcolumn
\item 不能嵌套使用\env{paracol}环境,否则会出现非法嵌套的错误消息。 
\switchcolumn[0]*
\item
The commands \!\switchcolumn!, \!\synccounter!, \!\syncallcounters! and
\!\flushpage!, and environments \env{column}(|*|), \env{nthcolumn}(|*|),
\env{leftcolumn}(|*|) and \env{rightcolumn}(|*|) are {\em local} to
\env{paracol} environment and thus undefined outside the
environment\footnote{

Unless you dare to define them.}.

The command \!\clearpage! is of course usable outside and inside the
environment but its function inside is a little bit different from outside.
\switchcolumn
\item 命令 \!\switchcolumn!、\!\synccounter!、\!\syncallcounters! 和 \!\flushpage!,以及环境\env{column}(|*|)、\env{nthcolumn}(|*|)、\env{leftcolumn}(|*|)和\env{rightcolumn}(|*|)是\env{paracol}环境中的{\em 局部}命令和环境,因此在环境外部是未定义的\footnote{除非你敢于定义它们。}。

命令 \!\clearpage! 当然可以在环境内外使用,但在环境内部的功能与外部略有不同。
\end{paracol}
\end{itemize}



\item[\ENV{paracol}{\oarg{numleft}\marg{num}\oarg{text}}]\mbox{}
\Item[\ENV{paracol}{\oarg{numleft}\texttt{*}\marg{num}\oarg{text}}]
\mbox{}\par
\changes{v1.3-2}{2013/09/17}
{Add description of parallel-paging.}

If a \beginparacol{} has the optional \meta{numleft} argument to specify
the number of leading columns $n_l$ together with the total $n$ given by
\meta{num}, columns in the environment are laid out across two adjacent
pages.  In this {\em\Uidx\parapag{}e} typesetting, the first $n_l$ columns
are placed in the {\em left} page while remaining $n_r=n-n_l$ columns go to
the next {\em right} page.  The pair of left and right pages is
considered as comprising a virtual {\em\Uidx\paired} page and thus shares
a common page number, unless {\em\Uidx\npaired} typesetting is specified
by the optional `|*|' following the optional \meta{numleft} argument.  In
the \npaired{} \parapag{}ing, when the leading $n_l$ columns are put in a
page $p$, the trailing $n_r$ columns are in the page $p+1$.

如果\beginparacol{}的可选参数\meta{numleft}用于指定前导列的数量$n_l$,同时总列数由\meta{num}给出,那么环境中的列会跨两个相邻的页面进行布局。在这种{\em\Uidx\parapag{}e}排版中,前$n_l$列放置在{\em 左侧}页面,而剩下的$n_r=n-n_l$列放置在下一个{\em 右侧}页面。左侧和右侧页面的配对被认为是组成一个虚拟的{\em\Uidx\paired}页面,因此它们共享一个相同的页码,除非通过在可选的\meta{numleft}参数后面添加`|*|'来指定{\em\Uidx\npaired}排版。在\npaired{} \parapag{}ing中,当前导的$n_l$列放置在页面$p$上时,后续的$n_r$列会在页面$p+1$上。

\begin{itemize}
\item
All {\em\Uidx\pwstuff}, i.e., \Preenv{} and \postenv, \pwise{} floats,
\mctext{} and (\mgfnote{} or non-merged) \Scfnote{}s, are placed only in
left \parapag{}es leaving corresponding regions in right \parapag{}es
blank\footnote{

Someday the author could devise an advanced mechanism to exploit the space
in right \parapag{}es.}.

所有的{\em\Uidx\pwstuff},即\Preenv{}和\postenv,\pwise{}浮动体,\mctext{}和(\mgfnote{}或非合并的)\Scfnote{},只会放置在左侧\parapag{}es中,让右侧\parapag{}es中相应的区域保持空白\footnote{将来作者可能会设计一个高级机制来利用右侧\parapag{}es中的空间。}。
\item
A \npaired{} left \parapag{}e is not necessary to be even-numbered, though
the printing tradition requires so if you naturally want to have a
\parapag{}e pair in a double spread.  The page number given to the first
left \parapag{}e is simply the number of the page $p_1$ in which
\beginparacol{} reside, and that for the $k$-th left \parapag{}e is
$p_1+2(k-1)$\footnote{

Unless you make some change to \counter{page} counter.}.

Therefore, to make it sure $p_1$ is even, you might need to have an
ordinary page of blank, a title, etc., or to let \counter{page} counter have
an even number by \!\setcounter!, etc., before starting a \env{paracol}
environment.

一个没有成对出现的左页不一定是偶数页,尽管印刷传统要求如果你自然地希望在双页中有一个成对的页面。第一个左页的页码只是在\beginparacol{}所在的页$p_1$的页码,而第$k$个左页的页码是$p_1+2(k-1)$\footnote{除非你对\counter{page}计数器进行了一些更改。}。

因此,为了确保$p_1$是偶数,你可能需要在开始\env{paracol}环境之前有一个普通的空白页、一个标题等,或者通过 \!\setcounter! 等方法使\counter{page}计数器的值成为一个偶数。

\item
Section~\ref{sec:ppts} shows examples of \parapag{}ing together with
related issues on two-sided typesetting.

第~\ref{sec:ppts} 节展示了 \parapag{} 的示例,以及双面排版相关问题。
\end{itemize}
\end{description}
 
% \subsection{切换栏的命令和环境 \hfill Column-Switching Command and Environments}
\label{sec:ref-switchcolumn}

\begin{description}
\item[\Midx{\!\switchcolumn!}\oarg{col}]\mbox{}
\Item[\Midx{\!\switchcolumn!}\oarg{col}\texttt{*}\oarg{text}]\mbox{}\par
\columnratio{0.6}
\begin{paracol}{2}
The command switches columns from $i$ to $j$ where $i$ and $j$ is the
zero-origin ordinals of the columns from/to which we are leaving\slash
visiting respectively.  Without the optional \meta{col}, $j=i+1\bmod n$
where $n$ is the number of columns given to \beginparacol, while
$j=\meta{col}$ with the optional argument.  If the command (or
\oarg{col} if specified) is followed by a |*|, the \cswitch{} takes
place after \sync{}ation and, if specified, the optional spanning
\meta{text} is put.
\switchcolumn
命令从第$i$列切换到第$j$列,其中$i$和$j$是我们离开/访问的列的零起始序号。如果没有可选参数\meta{col},则$j=i+1\bmod n$,其中$n$是给定给\beginparacol{}的列数,而如果有可选参数,则$j=\meta{col}$。如果命令(或如果指定了\oarg{col})后面跟着一个|*|,则\cswitch{}将在\sync{}ation之后进行,并且如果指定了可选的跨列\meta{text},则会放置它。
\Index{spanning text}
\end{paracol}

\begin{itemize}
\columnratio{0.6}
\begin{paracol}{2}
\item
Using \!\switchcolumn! in a \env{list}-like environment \emph{included} in
a \env{paracol} environment causes an ugly result without any error\slash
warning messages.  This caution is effectual for all \csenv{}s too.
\switchcolumn\item
在\env{paracol}环境中使用 \!\switchcolumn! 命令来切换到包含在\env{list}-like环境中会导致一个不美观的结果,而且没有任何错误或警告信息。同样的注意事项也适用于所有的\csenv{}。
\switchcolumn[0]*
\item
If $\meta{col}\notin\LBRP0n$, an error is reported and, if you dare to
continue, you will switch to the leftmost column 0.
\switchcolumn\item
如果 $\meta{col}\notin\LBRP0n$,将报告错误,并且如果你敢继续,将切换到最左边的列0。    
\switchcolumn[0]*
\item
The \sync{}ation point is set just below the last line of the \lcolumn{}
in a page $p$, partly taking deferred floats into account.  That is, all
deferred floats are put in the pages up to $p-1$ and at the top of $p$ if
possible.  Then, if a non-\lcolumn{} has footnotes and/or bottom floats
and they cannot be pushed down below the \sync{}ation point, the point is
moved to the next page top\footnote{

Or below top floats deferred to the page.}.
\switchcolumn\item
\sync{}ation点设置在页$p$的\lcolumn{}的最后一行的下方,部分考虑了延迟浮动。也就是说,所有延迟浮动都放在前$p-1$页和$p$页的顶部(如果可能的话)。然后,如果非\lcolumn{}有脚注和/或底部浮动,并且它们不能被推到\sync{}ation点以下,那么点就会被移动到下一页的顶部\footnote{

或下推到页面的延迟顶部浮动下方。}。
\switchcolumn[0]*
\item
In a page having one or more \sync{}ation points, stretch and shrink
factors of all vertical spaces, such as those surrounding sectionning
commands, are ignored.  Therefore, even if you specify \!\flushbottom!,
the page is typeset as if \!\raggedbottom! were specified.
\switchcolumn\item
在一个或多个\sync{}ation点的页面中,所有垂直空间的拉伸和收缩因子都被忽略,例如围绕节标题命令的空间。因此,即使您指定了 \!\flushbottom!,页面的排版也会像指定了 \!\raggedbottom! 一样进行。
\switchcolumn[0]*
\item
After a \sync{}ation point is set, no top floats will be inserted in the page
having the point, thus they will be deferred to the next page or further one.
\switchcolumn\item
在设置了同步点之后,不会在具有该点的页面中插入顶部浮动对象,因此它们将被推迟到下一页或更远的页面。
\end{paracol}
\end{itemize}


\item[\ENV{column}{}]\mbox{}
\Item[\ENV{column*}{\oarg{text}}]\mbox{}\par
\columnratio{0.55}
\begin{paracol}{2}
The environment \env{column} contains \meta{body} for the column next to
what we are in just before \!\begin!|{|\env{column}|}|.  The starred
version \env{column*} does the same after \sync{}ation and, if specified,
the optional spanning \meta{text} is put.
\switchcolumn
环境\env{column} 包含了在 \!\begin!|{|\env{column}|}| 之前我们所在的列旁边的\meta{body}。星号版本 \env{column*} 在{\fontKai 同眇}之后执行相同的操作,并且如果指定了可选的跨列\meta{text},则会放置它。

\Index{spanning text}
\end{paracol}

\begin{itemize}
\columnratio{0.55}
\begin{paracol}{2}
\item
The environments are almost equivalent to;
\begin{quote}
|{|\!\switchcolumn!\quad\meta{body}\quad\CSIndex{par}|}|\\
|{|\!\switchcolumn!|*|\oarg{text}\quad\meta{body}\quad\CSIndex{par}|}|
\end{quote}
except for their first occurrences which don't switch to the column 1
(i.e., right column if two-columned) but stay in the leftmost column 0.
More precisely, \!\begin!|{|\env{column}(|*|)|}| does not make \cswitch{}
if it is not preceded by \!\switchcolumn! nor other \csenv{}s.
\switchcolumn\item
这些环境几乎等同于:
\begin{quote}
|{|\!\switchcolumn!\quad\meta{body}\quad\CSIndex{par}|}|\\
|{|\!\switchcolumn!|*|\oarg{text}\quad\meta{body}\quad\CSIndex{par}|}|
\end{quote}
除了第一次出现的情况外,它们不会切换到列1(即双栏时的右栏),而是保持在最左边的列0。更准确地说,如果 \!\begin!|{|\env{column}(|*|)|}| 没有在 \!\switchcolumn! 或其他 \csenv{} 之前出现,就不会进行 \cswitch{}。
\switchcolumn[0]*
\item
The \meta{body} of the environments cannot have \!\switchcolumn! nor
\csenv{}s including \env{column}(|*|) themselves, or you will have an
error message of illegal use of command\slash environment.
\switchcolumn
\item 环境的\meta{body}不能包含 \!\switchcolumn! 或包含\env{column}(|*|)本身的\csenv{},否则会出现非法使用命令\slash 环境的错误消息。
\switchcolumn[0]*
\item
Column-switching\index{column-switching} does not take place at
\!\end!|{|\env{column}(|*|)|}|.  Therefore, texts following the
environments are put in the column in which \meta{body} resides until a
\cswitch{} command\slash environment is given.
\switchcolumn\item
在 \!\end!|{|\env{column}(|*|)|}| 处不会发生列切换\index{column-switching}。因此,在环境后面的文本会放置在\meta{body}所在的列中,直到出现\cswitch{}命令\slash 环境。
\end{paracol}
\end{itemize}



\item[\ENV{nthcolumn}{\marg{col}}]\mbox{}
\Item[\ENV{nthcolumn*}{\marg{col}\oarg{text}}]\mbox{}\par
\columnratio{0.55}
\begin{paracol}{2}
The environment \env{nthcolumn} contains \meta{body} for the column
\meta{col}.  The starred version \env{nthcolumn*} does the same after
\sync{}ation and, if specified, the optional spanning \meta{text} is put.
\switchcolumn
环境\env{nthcolumn}包含了第\meta{col}列的\meta{body}。星号版本\env{nthcolumn*}在\sync{}ation之后执行相同的操作,并且如果指定了可选的跨列\meta{text},则会放置它。
\Index{spanning text}
\end{paracol}

\begin{itemize}
\columnratio{0.55}
\begin{paracol}{2}
\item
The environments are equivalent to;
\begin{quote}
|{|\!\switchcolumn!\oarg{col}\quad\meta{body}\quad\CSIndex{par}|}|\\
|{|\!\switchcolumn!\oarg{col}|*|\oarg{text}\quad
    \meta{body}\quad\CSIndex{par}|}|
\end{quote}
\switchcolumn\item 
这些环境等同于:
\begin{quote}
|{|\!\switchcolumn!\oarg{col}\quad\meta{body}\quad\CSIndex{par}|}|\\
|{|\!\switchcolumn!\oarg{col}|*|\oarg{text}\quad
    \meta{body}\quad\CSIndex{par}|}|
\end{quote}
\switchcolumn[0]*
\item
The \meta{body} of the environments cannot have \!\switchcolumn! nor
\csenv{}s including \env{nthcolumn}(|*|) themselves, or you will have an
error message of illegal use of command\slash environment.
\switchcolumn
\item
环境的 \meta{body} 不能包含 \!\switchcolumn! 或包括 \env{nthcolumn}(|*|) 在内的 \csenv{},否则会出现非法使用命令\slash 环境的错误消息。
\switchcolumn[0]*
\item
Column-switching\index{column-switching} does not take place at
\!\end!|{|\env{nthcolumn}(|*|)|}|.  Therefore, texts following the
environments are put in the column in which \meta{body} resides until a
\cswitch{} command\slash environment is given.
\switchcolumn\item
列切换\index{column-switching}不会在 \!\end!|{|\env{nthcolumn}(|*|)|}| 处发生。因此,环境后的文本会被放在\meta{body}所在的列中,直到出现\cswitch{}命令\slash 环境为止。
\end{paracol}
\end{itemize}

\KeepSpace{4}
\item[\ENV{leftcolumn}{}]\mbox{}
\Item[\ENV{leftcolumn*}{\oarg{text}}]\mbox{}
\Item[\ENV{rightcolumn}{}]\mbox{}
\Item[\ENV{rightcolumn*}{\oarg{text}}]\mbox{}\par\nobreak
\columnratio{0.55}
\begin{paracol}{2}
The environment \env{leftcolumn} contains \meta{body} for the leftmost
column 0, while \env{rightcolumn} for the column 1 being the right column
in two-column typesetting.  The starred versions \env{leftcolumn*} and
\env{rightcolumn*} do the same after \sync{}ation and, if specified, the
optional spanning \meta{text} is put.
\switchcolumn
环境\env{leftcolumn}包含了最左侧列0的\meta{body},而\env{rightcolumn} 包含了在双栏排版中作为右侧列的列1的\meta{body}。星号版本\env{leftcolumn*}和\env{rightcolumn*}在\sync{}ation之后执行相同的操作,并且如果指定了可选的跨列\meta{text},则会放置它。
\Index{spanning text}
\end{paracol}


\begin{itemize}
\columnratio{0.55}
\begin{paracol}{2}
\item
The environments \env{leftcolumn}(|*|) are equivalent to;
\begin{quote}
\Env{nthcolumn}{\Arg{\texttt{0}}}\\
\Env{nthcolumn*}{\Arg{\texttt{0}}\oarg{text}}
\end{quote}
while \env{rightcolumn}(|*|) are equivalent to;
\begin{quote}
\Env{nthcolumn}{\Arg{\texttt{1}}}\\
\Env{nthcolumn*}{\Arg{\texttt{1}}\oarg{text}}
\end{quote}
\switchcolumn
\item
环境\env{leftcolumn}(|*|)等同于:
\begin{quote}
\Env{nthcolumn}{\Arg{\texttt{0}}}\\
\Env{nthcolumn*}{\Arg{\texttt{0}}\oarg{text}}
\end{quote}
而\env{rightcolumn}(|*|)等价于:
\begin{quote}
\Env{nthcolumn}{\Arg{\texttt{1}}}\\
\Env{nthcolumn*}{\Arg{\texttt{1}}\oarg{text}}
\end{quote}
\end{paracol}
\end{itemize}


\item[\Midx{\!\thecolumn!}]\mbox{}\par
\columnratio{0.55}
\begin{paracol}{2}
The command gives you the zero-origin ordinal of the column in which this
command appears.  Therefore, the following code snip;
\switchcolumn
该命令给出了此命令出现的列的零起始序号。因此,以下代码片段:
\end{paracol}

\begin{itemize}\item[]
|\begin{paracol}{3}|\\
|Column-\thecolumn.\switchcolumn|
|Column-\thecolumn.\switchcolumn|
|Column-\thecolumn.|\\
|\end{paracol}|
\end{itemize}
\columnratio{0.55}
\begin{paracol}{2}
gives us the followings.
\switchcolumn
我们得到了以下结果。
\end{paracol}
% \par\medskip
\begin{paracol}{3}
Column-\thecolumn.\switchcolumn
Column-\thecolumn.\switchcolumn
Column-\thecolumn.
\end{paracol}

\begin{itemize}
\columnratio{0.55}
\begin{paracol}{2}
\item
The command is {\em neither} a \LaTeX's counter nor \!\count! register of
native \TeX{}, and thus the value it keeps cannot be modified.  However,
it can be used wherever an integer number is required or appropriate.
Therefore for example, \!\setcounter!|{mycounter}{|\!\thecolumn!|}| works
well to give the column ordinal to the counter |mycounter|.
\switchcolumn\item
该命令既不是 \LaTeX 的计数器,也不是原生 \TeX 的 \!\count! 寄存器,因此它所保存的值无法修改。然而,它可以在需要或适当的地方使用整数值。因此,例如,将列序数赋给计数器 |mycounter|只要:\!\setcounter!|{mycounter}{|\!\thecolumn!|}|。
\end{paracol}
\end{itemize}

\item[\Midx{\!\definecolumnpreamble!}\marg{col}\marg{pream}]\mbox{}\par
\columnratio{0.55}
\begin{paracol}{2}
The command is to define the {\Uidx\colpream} \meta{pream} for the column
\meta{col}, which is inserted at every \cswitch{} to the column.  More
specifically, the command let \!\switchcolumn! to \meta{col} act as if you
sepcify;
\begin{itemize}\item[]
\!\switchcolumn! $\arg{pream\ for\ col}$
\end{itemize}
\switchcolumn
该命令用于为列 \meta{col} 定义{\Uidx\colpream} \meta{pream},该 \meta{pream} 在每次切换到该列时插入。更具体地说,该命令使得 \!\switchcolumn! 到 \meta{col} 的行为与您指定的一样。
\begin{itemize}\item[]
\!\switchcolumn! $\arg{pream\ for\ col}$
\end{itemize}
\end{paracol}



and \csenv{}s such as \env{nthcolumn} act as if you specify;

而\env{nthcolumn}等\csenv{}则会表现得好像你指定了:

\begin{itemize}\item[]
|\begin{nthcolumn}{|\meta{col}|}| $\arg{pream\ for\ col}$

\end{itemize}

\begin{itemize}
\item
\begingroup\hfuzz1.5pt
The optional \sptext{} of \!\switchcolumn!, \csenv{}s and \beginparacol{}
is considered to be in a virtual column $-1$, and thus if you need a
\Colpream{} for \sptext{}s do \!\definecolumnpreamble!|{-1}|\marg{pream}.

\!\switchcolumn!命令、\csenv{}和\beginparacol{}的可选参数\sptext{}被视为虚拟列$-1$中的内容,因此如果你需要为\sptext{}添加\Colpream{},请使用 \!\definecolumnpreamble!|{-1}|\marg{pream}。
\par\endgroup

\item
The command may appear in a \env{paracol} environment and, if so,
\meta{pream} is effective from the succeeding \cswitch{} to \meta{col}.

该命令可以出现在 \env{paracol} 环境中,如果是这样的话,\meta{pream} 从后续的 \cswitch{} 到 \meta{col} 是有效的。
\item
The definition of \meta{pream} is made globally.

\meta{pream} 的定义是全局的。
\end{itemize}



\item[\Midx{\!\ensurevspace!}\marg{len}]\mbox{}\par
\changes{v1.3-5}{2013/09/17}
{Add description of \cs{ensurevspace}.}

The command tells the first \sync{}ing \cswitch{} command (i.e.,
\!\switchcolumn!\oarg{col}|*|) or environment (i.e., \env{column*}, etc.\@)
following this command that the page must be broken before \sync{}ation
unless the \sync{}ation point has the space of \meta{len} or more below it
in the page.  If a \sync{}ation does not have the command after the
previous \sync{}ation, it is assumed that
\!\ensurevspace!|{|\!\baselineskip!|}| is given.

该命令告诉紧随该命令之后的第一个\sync{}ing \cswitch{}命令(即 \!\switchcolumn!\oarg{col}||)或环境(即\env{column}等),除非页面中\sync{}ation点下方有\meta{len}或更多的空间,否则页面必须在\sync{}ation前被分页。如果前一个\sync{}ation之后没有该命令,则假定已给出\!\ensurevspace!|{|\!\baselineskip!|}|。

\begin{itemize}
\item
This command is to be used when a \sync{}ation point would be placed near
the bottom of a page $p$ and the space below it is not sufficient for a column
$c$ to put anything in the page, while another column $c'$ can have a few
lines in the page.  If this happens, the first line after the \sync{}ation
should start at the top of the page $p+1$ in the column $c$, while that of
$c'$ is still in the page $p$, giving you an impression that the
\sync{}ation fails to align the top of all columns below it.  The fact is,
however, the \sync{}ation point is properly established near at the bottom
of the page but the first line of $c$ needs some large space due to, for
example, the followings.

当\sync{}ation点位于页面$p$的底部附近,并且其下方的空间不足以容纳列$c$中的内容,而另一列$c'$可以在页面$p$中有几行时,应使用此命令。如果发生这种情况,则\sync{}ation后的第一行应从页面$p+1$的列$c$顶部开始,而$c'$的第一行仍在页面$p$中,给您一种印象,即\sync{}ation无法使所有列的顶部对齐。然而,事实是,\sync{}ation点确实正确地建立在页面底部附近,但由于某些原因,例如以下原因,列$c$的第一行需要一些较大的空间。

\begin{itemize}
\item
The line has unusually tall stuff including larger font letters.

该行包含异常高的内容,包括较大字号的字母。
\item
The line has a footnote reference which is hardly apart from the
footnote, and thus the line and the footnote go to the next page together.

该行有一个脚注引用,与脚注之间几乎没有间隔,因此该行和脚注一起跳转到下一页。
\item
The parameter \!\clubpenalty! is too large (e.g., 10000) to break the
first and second lines into separate pages.

参数 \!\clubpenalty! 太大(例如10000),导致第一行和第二行无法分开分页。
\item
The first line follows a vertical space.

第一行后面有一个垂直间距。
\end{itemize}

\item
This manual itself has some instances of \!\ensurevspace! command in the
page \pageref{page:bfreude} and \pageref{page:efreude} in which each German
stanza is enclosed in \env{verse} and then \env{leftcolumn*} environments
and has \!\ensurevspace!|{2|\!\baselineskip!|}| before the \!\begin!ing of
the outer \env{leftcolumn*} because the first line of the stanza is
preceded by a vertical space inserted by \!\begin!|{|\env{verse}|}|.  In
fact without \!\ensurevspace!, the first two lines of the sixth English
stanza would be in the page \pageref{page:bfreude}, while corresponding
German stanza go to the next page \pageref{page:efreude} as a whole, due
to the difference of the height of footnotes in each column, i.e., German
ones are taller than English ones to narrow the space for the German
column.

本手册本身在第 \pageref{page:bfreude} 页和第\pageref{page:efreude}页有一些 \!\ensurevspace! 命令的实例,在这些页中,每个德语诗节都被包含在\env{verse}环境和\env{leftcolumn*}环境中,并且在外部\env{leftcolumn*}的 \!\begin!之前有一个 \!\ensurevspace!|{2|\!\baselineskip!|}| ,因为诗节的第一行前面有一个由 \!\begin!|{|\env{verse}|}| 插入的垂直间距。实际上,如果没有 \!\ensurevspace!,第六首英文诗节的前两行将在第\pageref{page:bfreude}页,而相应的德文诗节将作为整体移到下一页\pageref{page:efreude},这是因为每列脚注的高度不同,即德文脚注比英文脚注更高,以缩小德文列的空间。

\item
As the author does in the ``An die Freude/To Joy'' example, it is a good
tactics to have an \!\ensurevspace! with some vertical space larger than the
default \!\baselineskip! if it is sure that a column has a feature shown
above regardless of the position of the \sync{}ation point in question,
because the point goes up or down with revisions of your document and
using an \!\ensurevspace! for a \sync{}ation far above the page bottom is
perfectly harmless.  Similarly, if you find a problem in a \sync{}ation
and add an \!\ensurevspace! to solve it, keeping the command attached is
recommended even when the \sync{}ation point moves up or down to make the
command unnecessary.

正如作者在“An die Freude/To Joy”示例中所做的那样,如果确定某一列具有上述特征,无论问题点的\sync{}ation点位置如何变化,使用比默认
\!\baselineskip! 更大的一些垂直间距的 \!\ensurevspace! 是一个好策略,因为该点随着文档的修订而上下移动,并且在页面底部上方使用
\!\ensurevspace! 是完全无害的。同样,如果在\sync{}ation中发现问题并添加了 \!\ensurevspace! 来解决问题,则建议保留该命令,即使\sync{}ation点上下移动以使命令不再需要。

\end{itemize}
\end{description}

 
%  
 \subsection{用于列和间隔宽度的命令\hfill Commands for Column and Gap Width}
 \label{sec:ref-colwidth}
 
 \begin{description}
 \item[\Midx{\!\columnratio!}\Arg{$r_0,r_1,\cdots,r_k$}
                              {|[|$r'_0,r'_1,\cdots,r'_{k'}$|]|}]\mbox{}\par
\columnratio{0.55}
\begin{paracol}{2}
The command defines the width of each column by the fraction $r_i$ to
specify the portion which $i$-th ($i=0$ for the leftmost) column
occupies.  More specifically, the width $\Midx{\w}_i$ of the $i$-th column
is defined as follows, where $W$ is \!\textwidth!, $S$ is \!\columnsep!,
and $n$ is the number of columns given to \beginparacol.
\begin{eqnarray*}
W'&=&W-(n-1)S\\
w_i&=&\cases{
r_iW'
    \vrule height1.5\ht\strutbox depth1.5\dp\strutbox width0pt&$i\leq k$\cr
\displaystyle{(1-\sum_{j=0}^k r_j)W'\over n-(k+1)}&$i>k$}
\end{eqnarray*}
\switchcolumn
该命令通过分数$r_i$来定义每列的宽度,
以指定第$i$列($i=0$表示最左边的列)所占的比例。
具体而言,第$i$列的宽度$\Midx{\w}_i$定义如下,
其中$W$是 \!\textwidth!,
$S$是 \!\columnsep!,
$n$是传递给 \beginparacol 的列数。
\begin{eqnarray*}
W'&=&W-(n-1)S\\
w_i&=&\cases{
r_iW'
\vrule height1.5\ht\strutbox depth1.5\dp\strutbox width0pt&$i\leq k$\cr
\displaystyle{(1-\sum_{j=0}^k r_j)W'\over n-(k+1)}&$i>k$}
\end{eqnarray*}
\switchcolumn[0]*
 For a \env{paracol} environment with \parapag{}ing, $n$ is replaced with
 $n_l$ for the columns in left \parapag{}es, while $n$ and $w_i$ are
 replaced with $n_r$ and $w_{n_r+i}$ for those in right \parapag{}es.
 Moreover, if the optional argument having $r'_0,r'_1,\cdots,r'_{k'}$ is
 provided, $w_{n_r+i}$ for a column in right \parapag{}es is determined
 by $r'_i$ and $k'$ instead of $r_i$ and $k$.
 \switchcolumn
 对于具有\parapag{} 分页的\env{paracol}环境,对于左侧\parapag{}的列,将$n$替换为$n_l$,而对于右侧\parapag{}的列,将$n$和$w_i$替换为$n_r$和$w_{n_r+i}$。此外,如果提供了具有$r'0,r'1,\cdots,r'{k'}$的可选参数,则右侧\parapag{}的列中的$w{n_r+i}$由$r'_i$和$k'$确定,而不是由$r_i$和$k$确定。
\end{paracol}


\begin{itemize}
\columnratio{0.55}
\begin{paracol}{2}
\item
The equations above imply that $k<n-1$, $r_i>0$ and $\sum_{j=0}^k
r_j<1$.  If $k\geq n-1$, $k$ is assumed to be $n-2$ and all $r_i$ such
that $i\geq n-1$ are ignored.  If $r_i$ or its sum does not satisfy the
conditions, you will have an ugly result with ``Overfull'' messages.
\switchcolumn\item
上述方程表明$k<n-1$,$r_i>0$且$\sum_{j=0}^k r_j<1$。如果$k\geq n-1$,则假设$k$为$n-2$,并忽略所有满足$i\geq n-1$的$r_i$。如果$r_i$或其总和不满足条件,你将得到一个带有“Overfull”消息的不美观的结果。
\switchcolumn[0]*
\item
The argument $r_0,r_1,\cdots,r_k$ can be empty to mean $k=-1$ to let all
column widths be $W'/n$ as default.
\switchcolumn\item
参数$r_0,r_1,\cdots,r_k$可以为空,表示$k=-1$,使得所有列宽默认为$W'/n$。
\switchcolumn[0]*
\item
The setting of column width by the command takes effect in the |paracol|
environments following the command\footnote{%
If the command is in a \texttt{paracol} environment, the command does not
affect the column widths of the environment but does the next ones, though
such usage is very unusual.}.
\switchcolumn\item
该命令设置的列宽度在命令后的 |paracol| 环境中生效\footnote{如果该命令在 \texttt{paracol} 环境中,该命令不会影响环境的列宽度,而是影响后续的列宽度,尽管这种用法非常不常见。}。
\switchcolumn[0]*
Therefore, though placing the command in the preamble is the most natural
way\footnote{%
Or second most to not using it at all, of course.},
\switchcolumn
因此,将该命令放在导言区是最自然的方式\footnote{%
当然,第二自然的方式是不使用它。}。
\switchcolumn[0]*
you may place this command between two |paracol| environments to change
the column layout for the second one even when they appear in a page as
shown in Section~\ref{sec:man-close}.
\switchcolumn
在两个|paracol|环境之间放置此命令,即可更改第二个环境的列布局,即使它们在页面中出现,如第~\ref{sec:man-close}节所示。
%%%%%%%%
\switchcolumn[0]*
 \item
 In the $i$-th column, \!\columnwidth! has $w_i$ and, for outermost
 paragraphs in the column, \!\hsize! has $w_i$ as well.  As for
 \!\linewidth!, it has $w_i-(\!\textwidth!-l)$ where $l$ is what
 \!\linewidth! had at \beginparacol{}, i.e., the \!\linewidth! for the
 \env{list}-like environment surrounding \env{paracol} if any, or
 \!\textwidth! otherwise.
 \switchcolumn\item
 在第$i$列中,\!\columnwidth! 的值为$w_i$,对于列中的最外层段落,\!\hsize!的值也为$w_i$。至于 \!\linewidth!,它的值为 $w_i-(\!\textwidth!-l)$,其中$l$是在\beginparacol{}中 \!\linewidth! 所具有的值,即如果有的话,是包围\env{paracol}的\env{list}-like环境的 \!\linewidth!,否则是 \!\textwidth!。
 \switchcolumn[0]*
 \item
 You can specify width of each column and that of each {\em gap} between
 two columns more detailedly by \!\setcolumnwidth! shown below.  If your
 document has both of \!\columnratio! and \!\setcolumnwidth! prior to a
 \env{paracol} environment, the command given later is effective for the
 environment.
 \switchcolumn\item
 您可以通过下面的 \!\setcolumnwidth! 更详细地指定每列的宽度和每两列之间的{\em 间隙}的宽度。如果在\env{paracol}环境之前的文档中同时存在 \!\columnratio! 和 \!\setcolumnwidth!,则后给出的命令对该环境有效。
\end{paracol}
\end{itemize}
 
 
\item[\Midx{\!\setcolumnwidth!}\Arg{$s_0,s_1,\cdots,s_k$}
                              {|[|$s'_0,s'_1,\cdots,s'_{k'}$|]|}]\mbox{}\par
\columnratio{0.55}
\begin{paracol}{2}
The command defines the width of each column and that of each {\em gap}
between two columns by the column/gap specification $s_i$ for the $i$-th
column and the gap between it and the $(i{+}1)$-th column.  More
specifically, $s_i$ has the form of $\hat w_i$ or $\hat w_i\,|/|\,\hat g_i$
where each of $\hat w_i$ and $\hat g_i$ is a proper glue including a
proper dimension, or an empty string to mean $\hat w_i=\!\fill!$ and $\hat
g_i=\!\columnsep!$, to determine the width of $i$-th column $\w_i$ and that
of $i$-th gap $\Midx{\gap}_i$ as follows, where $\mathit{nat}(x)$ is the
natural width of the glue $x$, $\mathit{str}(x)$ is the infinite stretch
factor of $x$, $W$ is \!\textwidth!, and $n$ is the number of columns
given to \beginparacol.

\begin{eqnarray*}
W'&=&\sum_{i=0}^{n-2}\big(\mathit{nat}(\hat w_i)+\mathit{nat}(\hat g_i)\big)+
\mathit{nat}(\hat w_{n-1})\\
F&=&\sum_{i=0}^{n-2}\big(\mathit{str}(\hat g_i)+\mathit{str}(\hat g_i)\big)+
\mathit{str}(\hat w_{n-1})\\
x_i&=&\cases{(W/W')\mathit{nat}(\hat x_i)&$W'\geq W\;\lor\;F\leq0$\cr
\mathit{nat}(\hat x_i)+(\mathit{str}(\hat x_i)/F)(W-W')&
$W'< W\;\land\;F>0$}
\qquad(x\in\{w,g\})
\end{eqnarray*}
\switchcolumn
该命令通过列/间隔规范$s_i$定义每个列和每个{\em 间隔}的宽度,其中$s_i$是第$i$列和它与$(i{+}1)$列之间的间隔。
具体来说,$s_i$的形式为$\hat w_i$或$\hat w_i,|/|,\hat g_i$,其中$\hat w_i$和$\hat g_i$都是包含适当尺寸的适当粘连,
或者是一个空字符串来表示 $\hat w_i=\!\fill!$ 和$\hat g_i=\!\columnsep!$,以确定第$i$列$\w_i$和第$i$个间隔$\Midx{\gap}_i$的宽度,
其中$\mathit{nat}(x)$是粘连$x$的自然宽度,$\mathit{str}(x)$是$x$的无限伸展因子,$W$是 \!\textwidth!,$n$是传递给\beginparacol 的列数。

\begin{eqnarray*}
W'&=&\sum_{i=0}^{n-2}\big(\mathit{nat}(\hat w_i)+\mathit{nat}(\hat g_i)\big)+
\mathit{nat}(\hat w_{n-1})\\
F&=&\sum_{i=0}^{n-2}\big(\mathit{str}(\hat g_i)+\mathit{str}(\hat g_i)\big)+
\mathit{str}(\hat w_{n-1})\\
x_i&=&\cases{(W/W')\mathit{nat}(\hat x_i)&$W'\geq W\;\lor\;F\leq0$\cr
\mathit{nat}(\hat x_i)+(\mathit{str}(\hat x_i)/F)(W-W')&
$W'< W\;\land\;F>0$}
\qquad(x\in\{w,g\})
\end{eqnarray*}
\end{paracol}

 That is, if the total of natural widths $W'$ is larger than \!\textwidth!
 $W$ or there are no infinite stretch factors in the specification, given
 widths are scaled down or up so that the scaled total is equal to $W$.
 Otherwise, each width with an infinite stretch factor is extended
 according to its ratio in the total stretch so that the stretched total is
 equal to $W$.

 也就是说,如果自然宽度的总和$W'$大于 \!\textwidth! $W$,或者规范中没有无限伸展因子,给定的宽度将被缩小或放大,使得缩放后的总和等于$W$。否则,每个具有无限伸展因子的宽度将根据其在总伸展中的比例进行扩展,以使伸展后的总和等于$W$。
 
 For a \env{paracol} environment with \parapag{}ing, $n$ is replaced with
 $n_l$ for the columns in left \parapag{}es, while $n$, $w_i$ and $g_i$ are
 replaced with $n_r$, $w_{n_r+i}$ and $g_{n_r+i}$ for those in right
 \parapag{}es.  Moreover, if the optional argument having
 $s'_0,s'_1,\cdots,s'_{k'}$ is provided, $w_{n_r+i}$ and $g_{n_r+i}$ for a
 column in right \parapag{}es are determined by $s'_i$ instead of $s_i$.

 对于具有\parapag{}分页的\env{paracol}环境,对于左侧\parapag{}的列,将$n$替换为$n_l$,而对于右侧\parapag{}的列,将$n$,$w_i$和$g_i$分别替换为$n_r$,$w_{n_r+i}$和$g_{n_r+i}$。此外,如果提供了具有$s'0,s'1,\cdots,s'{k'}$的可选参数,则右侧\parapag{}的列中的$w{n_r+i}$和$g_{n_r+i}$由$s'_i$确定,而不是由$s_i$确定。
 \begin{itemize}
 \item
 In \env{paracol} environments having $n$ columns, $s_i$ s.t.\ $i\geq n$
 and $\hat g_{n-1}$ are ignored.  On the other hand if $k<n-1$, it is
 assumed $s_i$ is an empty string for all $i>k$.

 在具有$n$列的\env{paracol}环境中,忽略满足$i\geq n$和$\hat g_{n-1}$的$s_i$。另一方面,如果$k<n-1$,则假设对于所有$i>k$,$s_i$都是一个空字符串。
 
 \item
 Finite stretch factors and finite or infinite shrink factors in $\hat w_i$
 and $\hat g_i$ are ignored.
 
 在$\hat w_i$和$\hat g_i$中,有限的拉伸因子和有限或无限的收缩因子被忽略。
 \item
 Unlike \TeX's genuine glue addition, all infinite unit |fil|, |fill| and
 |filll| are not distinguished in the summation for $F$.  Also unlike
 \TeX's genuine scaling of a glue primitive, 
 $f\!\fill!$ means $0\,|pt|\ |plus|\ f\,|fill|$ for convenience\footnote{
 
 In \TeX's grammar, $f\!\fill!$ means a dimension rather than a glue and is
 $0\,|pt|$ because the natural component of \!\fill! is 0.}.

 与 \TeX 的真正粘连添加不同,所有无限单位的 |fil|、|fill| 和 |filll| 在 $F$ 的求和中没有区别。另外,与 \TeX 的真正粘连原语的缩放不同, $f\!\fill!$ 表示为 $0,|pt|\ |plus|\ f,|fill|$,以方便使用\footnote{在 \TeX 的语法中,$f\!\fill!$ 表示的是一个尺寸而不是粘连,并且是 $0,|pt|$,因为 \!\fill! 的自然分量为 0。}。
 
 \item
 The division $W/W'$ and $\mathit{str}(\hat x_i)/F$ can have some
 arithmetic errors and thus the total of $w_i$ and $g_i$ may not be equal to
 $W$ exactly but can be a little bit less than $W$.  This small error is,
 however, equally distributed to $g_i$ in typesetting of a page to make the
 total width of columns and gaps is exactly $W$\footnote{
 
 If we may ignore the arithmetic error inherent in \TeX.}.

 除法$W/W'$和$\mathit{str}(\hat x_i)/F$可能存在一些算术误差,因此$w_i$和$g_i$的总和可能不完全等于$W$,而可能略小于$W$。然而,在页面排版中,这个小的误差被等分给$g_i$,以确保列和间隙的总宽度恰好为$W$\footnote{
 
 如果我们可以忽略\TeX 中固有的算术误差。}。
 
 \item
 All the specifications shown in the table below give us same results for a
 \env{paracol} environment having three columns, providing
 $\!\textwidth!=360\,|pt|$ and $\!\columnsep!=S=20\,|pt|$.

 下表中显示的所有规格都可以得到相同的结果,适用于具有三列的\env{paracol}环境,其中 $\!\textwidth!=360,|pt|$和$\!\columnsep!=S=20,|pt|$。
 
 \par\hbox to\textwidth\bgroup\hfil
 \nosv \def\|{\verb|}\small\arraycolsep0pt\def\arraystretch{1.1}
 $\begin{array}[b]{l|ccccc}
 s_0,s_1,s_2&w_0&g_0&w_1&g_1&w_2\rlap{ (in \texttt{pt})}\\\hline
 \|50pt/20pt,100pt/40pt,150pt|&50&20&100&40&150\\
 \|50pt,100pt/2\columnsep,150pt|&50&S&
                                100&2S&150\\
 \|50pt/\fill,100pt/2\fill,150pt|&50&(1/3)\cdot60&100&(2/3)\cdot60&150\\
 \|,2\fill/2\columnsep,3\fill|&\ (1/6)\cdot300&S&
                              (2/6)\cdot300&2S&
                              (3/6)\cdot300\\
 \|50pt/20,50pt plus 1fil/40pt,50pt plus 2fil |&
                              50&20&50+(1/3)\cdot150&40&
                              50+(2/3)\cdot150\\
 \|5pt/2pt,10pt/4pt,15pt|&10\cdot5&10\cdot2&10\cdot10&10\cdot4&
                         10\cdot15\\
 \|100pt/40pt,200pt/80pt,300pt|&0.5\cdot100&0.5\cdot40&
                               0.5\cdot200&0.5\cdot80&
                               0.5\cdot300
 \end{array}$\hfil\egroup
 
 \item
 If your document has both of \!\columnratio! and \!\setcolumnwidth! prior
 to a \env{paracol} environment, the command given later is effective for
 the environment.

如果在\env{paracol}环境之前的文档中同时存在 \!\columnratio! 和 \!\setcolumnwidth!,则后面给出的命令对该环境有效。
 \end{itemize}
 \end{description}
 
 
% 
\subsection{用于双面排版和边注的放置的命令\hfill Commands for Two-Sided Typesetting and Marginal Note Placement}
\label{sec:ref-twoside}

\begin{description}
\item[\Midx{\!\twosided!}{$|[|t_1t_2\cdots t_k|]|$}]\mbox{}\par
\columnratio{0.55}
\begin{paracol}{2}
The command enables a set of two-sided typesetting features
$\Set{t_i}{t_i\in\{|p|,|c|,|m|,|b|\},\ 1\leq i\leq k}$ explicitly by the
optional argument, or all of the following four features as a whole
without the argument, in even-numbered pages.
\switchcolumn
该命令通过可选参数显式地启用一组双面排版功能$\Set{t_i}{t_i\in{|p|,|c|,|m|,|b|},\ 1\leq i\leq k}$,或者在偶数页上作为一个整体启用以下四个功能,而无需参数。    
\end{paracol}
\begin{description}
\columnratio{0.55}
\begin{paracol}{2}
\item[|p|\rm(\textit{age})]
for ordinary two-sided paging, letting the left side margin be
\!\evensidemargin!, page headers be different from those in odd-numbered
pages with |headings| or |myheadings| page style, and \!\cleardoublepage!
leave an even-numbered page blank if it is used in an odd-numbered page.
\switchcolumn
\item[|p|\rm(\textit{age})]
对于普通的双面分页,左侧边距为 \!\evensidemargin!,页面页眉与奇数页中的 |headings| 或 |myheadings| 页面样式不同,并且 \!\cleardoublepage! 在奇数页中使用时会使偶数页保持空白。
\switchcolumn[0]*
\item[|c|\rm(\textit{olumn})]
for {\em\Uidx\cswap} to \emph{print} columns in even-numbered pages in
reverse order.  This feature is sometimes preferable in typesetting
especially with unbalanced parallel columns to make, for example, a wider
columns are always \emph{inside} while narrower ones are \emph{outside}.
\switchcolumn
\item[|c|\rm(\textit{olumn})]
对于{\em\Uidx\cswap}来在偶数页上以相反的顺序\emph{打印}列。这个功能在排版中有时是可取的,特别是在不平衡的并列列中,可以使较宽的列始终位于\emph{内部},而较窄的列位于\emph{外部}。
\switchcolumn[0]*
\item[|m|\rm(\textit{arginal text})]
to place marginal notes in the side margin opposite to that specified by
the command \!\marginparthreshold! discussed shortly.
\switchcolumn
\item[|m|\rm(\textit{arginal text})]
将边注放置在与命令 \!\marginparthreshold! 指定的相反侧边缘中(稍后会讨论)。
\switchcolumn[0]*
\item[|b|\rm(\textit{ackground painting})]
to make \bgpaint, shown in Section~\ref{sec:ref-bgpaint},
\emph{\mirror{}ed} so that, for example, a color specified for the left
margin is used to paint the right margin instead.
\switchcolumn\item[|b|\rm(\textit{ackground painting})]
为了使\bgpaint(参见第~\ref{sec:ref-bgpaint}节)是\emph{\mirror{}ed}的,例如,为左边距指定的颜色将用于绘制右边距。
\end{paracol}
\end{description}

\begin{itemize}
\columnratio{0.55}
\begin{paracol}{2}
\item
The feature |p| is also enabled by the |twoside| option of
\!\documentclass! with almost all classes including |article|, |book|,
|report|, etc.  Though it is strongly recommended to make both settings by
\!\documentclass! and this command consistent, they can be inconsistent
resulting in lack of some expected functions.  For example, enabling |p|
feature by \!\twosided! without |twoside| option in \!\documentclass!
makes the format of headers and footers in all pages same even with
\!\pagestyle!|{headings}|.
\switchcolumn\item
|p|特性也可以通过 \!\documentclass! 的 |twoside| 选项启用,几乎适用于包括 |article|、|book|、|report| 等在内的所有类。虽然强烈建议通过 \!\documentclass! 和此命令使两个设置保持一致,但它们可能不一致,导致缺少某些期望的功能。例如,通过在 \!\documentclass! 中启用 |twoside| 选项而不使用 \!\twosided!,会使所有页面上的页眉和页脚的格式相同,即使使用了 \!\pagestyle!|{headings}|。
\switchcolumn[0]*
\item
The \cswap{} enabled by the feature |c| is ineffective in \npaired{}
\parapag{}ing because it is meaningless\footnote{%
Unless somebody tells the author it is meaningful.},
and thus silently ignored.
\switchcolumn\item
在\npaired{}\parapag{}ing中,由特性 |c| 启用的\cswap{}是无效的,因为它是没有意义的\footnote{除非有人告诉作者它是有意义的。},因此会被悄悄地忽略。
\switchcolumn[0]*
\item
In ordinary single-column typesetting, marginal note swapping in
even-numbered pages is enabled by the |twoside| option, while it never takes
place in ordinary two-column typesetting.  For marginal notes given in
\env{paracol} environments, however, swapping of them in
even-numbered pages is enabled by giving the feature |m| to \!\twosided!.
\switchcolumn\item
在普通的单栏排版中,通过|twoside|选项启用了在偶数页中交换边注的功能,而在普通的双栏排版中则不会出现这种情况。然而,对于在\env{paracol} 环境中给出的边注,可以通过给予 \!\twosided! 功能特性 |m| 来在偶数页中启用它们的交换。
\switchcolumn[0]*
\item\label{page:cswap}
The command has to be outside of \env{paracol} environments to decide the
action in the environments following them.  If it appears in a
\env{paracol} environment, you will have a warning message saying it is
ignored.
\switchcolumn\item
该命令必须位于\env{paracol} 环境之外,以决定其后环境中的操作。如果它出现在\env{paracol} 环境中,您将收到一个警告消息,指示它被忽略。
\end{paracol}    


\twosided[c]\columnratio{0.55}\columnsep0pt
\begin{Verbatim}
\twosided[c]\columnratio{0.55}\columnsep0pt
\end{Verbatim}
\begin{paracol}{2}
\hfuzz2pt
\item
Here is an example of column swapping.  Since this page
\pageref{page:cswap} is odd, this wider column-0 with roman font is placed
in left side and thus inside at the begining, but now we are in an even
page in which this column is in right side.
\switchcolumn
\item
这是一个列交换的示例。由于此页\pageref{page:cswap}是奇数页,因此带有罗马字体的较宽的列-0被放置在左侧,因此在开始时位于内部,但现在我们处于一个偶数页,此列位于右侧。
\switchcolumn
\item\it
This narrower, outside and italicized column-1 is at first in right
side but the page break has changed the position to the left.

\switchcolumn\item
这个较窄、位于外侧并且斜体的列1最初在右侧,但页面断页导致其位置改变到左侧。
\switchcolumn
\item
\changes{v1.2-4}{2013/05/11}
{Add description of \cs{[no]swapcolumninevenpages}.}
\changes{v1.3-5}{2013/09/17}
{Remove description of \cs{[no]swapcolumninevenpages} but mention
    they are still available.}

In old versions of \Paracol, namely 1.2 and its minor revisions 1.2x,
\cswap{} was controlled by lengthy commmands
\Midx{\!\swapcolumninevenpages!} and \Midx{\!\noswapcolumninevenpages!}.
Though they are still available and will be so forever for backward
compatibility, it is recommended to use \!\twosided! with or without the
feature |c|.  The old versions also have a problem that \spanning{}
crossing a page boundary is placed incorrectly after the page break in it,
but this problem is solved by a fix incorporated in version 1.3.
\switchcolumn\item
在旧版本的 \Paracol 中,即1.2版本及其小的修订版本1.2x中,\cswap{}通过冗长的命令\Midx{\!\swapcolumninevenpages!}和\Midx{\!\noswapcolumninevenpages!}进行控制。尽管它们仍然可用,并且将永远用于向后兼容性,但建议使用带有或不带有特性|c|的 \!\twosided!。旧版本还存在一个问题,即跨页的\spanning{}在页面断页后放置不正确,但这个问题在1.3版本中通过修复得到解决。
\switchcolumn
\item
It must be $t_i\in\{|p|,|c|,|m|,|b|\}$, or you will have an error message
of illegal two-siding feature.
\switchcolumn\item
必须是$t_i\in{|p|,|c|,|m|,|b|}$,否则会出现非法双面特性的错误消息。
\switchcolumn
\item
Section~\ref{sec:ppts} shows examples of two-sided typesetting together
with related issues on \parapag{}ing.
\switchcolumn\item
第~\ref{sec:ppts}节展示了双面排版的示例,以及与\parapag{}分页相关的问题。
\end{paracol}
\end{itemize}


\item[\Midx{\!\marginparthreshold!}$\Arg{k}{|[|k'|]|}$]\mbox{}\par
\columnratio{0.55}
\begin{paracol}{2}
The command specifies the minimum ordinal $k$ of columns whose marginal
notes are placed in right margin.  That is, marginal notes given in a
column-$i$ go to left margin if $i<k$, while they go to right if $i\geq
k$.  The optional argument $k'$, if given, is for columns in right
\parapag{}es to decide the margin where their marginal notes are placed.
In default, $k=1$ is assumed to let marginal notes from the leftmost
column-0 go to left margin while those from other columns go to right.
\switchcolumn
该命令指定了边注放置在右边页边距中的最小列序数$k$。也就是说,在列$i$中给出的边注如果$i<k$,则放置在左边页边距中,而如果$i\geq k$,则放置在右边页边距中。如果给定可选参数$k'$,则用于决定右边\parapag{}es中的列的边注放置在哪个页边距。默认情况下,假设$k=1$,左边最左列-0的边注放置在左边页边距中,而其他列的边注放置在右边页边距中。
\end{paracol}

\begin{itemize}
\columnratio{0.55}
\begin{paracol}{2}
\item
You may specify $k=0$ to let all marginal notes go to right margin, or may
give the command a large number, say 100, to place all of them in left
margin.
\switchcolumn\item
您可以将$k$指定为0,使所有边注都放在右侧边距,或者可以给命令一个较大的数,比如100,将它们全部放在左侧边距。
\switchcolumn[0]*
\item
The setting $k=0$ or $k=100$ above makes a side margin \emph{shared} by
marginal notes from different columns, and sharing is inevitable when a
(parallel-) page has three or more columns.  When a margin is shared by
marginal notes from two or more columns, it can happen that two marginal
notes from different columns conflict over the space to be occupied by each
of them.  This conflict is solved by \Paracol{} to push down the note
given later in your source |.tex| until an available space for it is
found.  Note that the marginal note to be pushed down is determined by the
position in the source rather than that in the printed result.  Also note
that \Paracol{} exploits space between two marginal notes having been
already placed in the placement of other note coming later to place it at
the natural position if possible or to minimize the amount of pushing down
otherwise.
\switchcolumn\item
上述设置$k=0$或$k=100$使得边注从不同的列共享一个侧边距,当一个(并列)页面有三个或更多列时,共享是不可避免的。当一个侧边距被来自两个或更多列的边注共享时,可能会发生两个来自不同列的边注在它们各自要占据的空间上发生冲突的情况。这个冲突通过\Paracol{}来解决,它会将后面给出的边注推到更低的位置,直到找到一个可用的空间为止。请注意,要被推到下方的边注是由源代码中的位置决定的,而不是打印结果中的位置。同时,请注意\Paracol{}利用已经放置的两个边注之间的空间,在后面的边注放置时尽可能地在自然位置上放置,或者尽量减少推下的量。
\switchcolumn[0]*
\item
In the decision of the real margin in which a marginal note is placed,
other two factors are involved;  |m| feature of \!\twosided! command and
the parity of the page; and \LaTeX's genuine command \!\reversemarginpar!.
More specifically, after the first preliminary decision is made according
to the threshold given to \!\marginparthreshold!, we have the following
two steps to modify the decision;  if |m| feature has been specified in
\!\twosided! command and the marginal note belongs to an even-numbered
page, the decision is reversed to have the second preliminary result;  and
then if \!\reversemarginpar! has been specified, the second result is
reversed (again) to have the final result.
\switchcolumn\item
在确定边注放置的实际边距时,还涉及其他两个因素:\!\twosided! 命令的 |m| 特性和页面的奇偶性;以及\LaTeX 的原始命令 \!\reversemarginpar!。具体而言,在根据 \!\marginparthreshold! 给定的阈值做出第一次初步决策后,我们有以下两个步骤来修改决策;如果 \!\twosided! 命令中指定了 |m| 特性,并且边注属于偶数页,决策将被反转为得到第二次初步结果;然后,如果指定了 \!\reversemarginpar!,第二个结果将被(再次)反转为得到最终结果。
\switchcolumn[0]*
\item
In old versions of \Paracol, namely older than 1.3, marginal note
placement was not only uncontrollable but also gave ugly results when your
document has three or more columns because the marginal notes from a column
not being leftmost or rightmost were placed in the gap following the
column rather than a margin.  This miserable {\em gap note} placement does
not happen any more, or in other words this is no more available because
the author believes nobody loves it.
\switchcolumn\item
在旧版本的\Paracol 中(即1.3之前的版本),边注的放置不仅无法控制,而且在文档具有三列或更多列时会产生丑陋的结果,因为不在最左侧或最右侧的列的边注会放置在列后的间隙中,而不是边距中。这种痛苦的{\em 间隙边注}放置不再发生,换句话说,不再可用,因为作者认为没有人喜欢它。
\switchcolumn[0]*
\item
Section~\ref{sec:ppts} shows examples of marginal note placement together
with related issues on \parapag{}ing and two-sided typesetting.
\switchcolumn\item
第~\ref{sec:ppts}节展示了边注放置的示例,以及与\parapag{}ing和双面排版相关的问题。
\end{paracol}
\end{itemize}


\item[\Midx{\!\marginnote!}\oarg{left}\marg{right}\oarg{voffset}]\mbox{}\par
\columnratio{0.55}
\begin{paracol}{2}
You may use the package \textsf{marginnote} and its command \!\marginnote!
in \env{paracol} environments as a replacement of \!\marginpar!.  However,
the command is \emph{emulated} with \!\marginpar! and \textsf{paracol}'s
own mechanism of marginal note placement.  Therefore, some of
\textsf{marginnote}'s functionality are not effective in \env{paracol}
environment except for the following features.
\switchcolumn
您可以在 \env{paracol} 环境中使用 \textsf{marginnote} 宏包及其命令 \!\marginnote! 作为 \!\marginpar! 的替代。然而,该命令是通过 \!\marginpar! 和 \textsf{paracol} 自身的边注放置机制进行\emph{模拟}的。因此,在 \env{paracol} 环境中,除了以下功能外,一些 \textsf{marginnote} 的功能是不起作用的。
\end{paracol}

\begin{itemize}
\columnratio{0.55}
\begin{paracol}{2}
\item
Shifting up/down a marginal note by the optional \meta{voffset}.
\switchcolumn\item
通过可选参数\meta{voffset}将边注上下移动。
\switchcolumn[0]*
\item
Defining fonts (and others) for marginal notes by \!\marginfont!.
\switchcolumn\item
通过 \!\marginfont! 为边注定义字体(和其他样式)。
\switchcolumn[0]*\item
Controlling the holizontal paragraph alignment by \!\raggedleftmarginnote!
and |\raggedright|\~|marginnote|\SpecialIndex{\raggedrightmarginnote}.
\switchcolumn\item
通过 \!\raggedleftmarginnote! 和|\raggedright|~|marginnote| 控制水平段落对齐方式。
\end{paracol}
\end{itemize}

\columnratio{0.55}
\begin{paracol}{2}
Note that you will see a warning message ``|\margninnote| is emulated by
|\marginpar|'' at the first in-\env{paracol} occurrence of the command to
let you know the imperfection.
\switchcolumn
请注意,在第一次使用该命令的\env{paracol}环境中,您将看到一个警告消息“|\margninnote| is emulated by |\marginpar|”,以便让您知道这种不完美的情况。
\end{paracol}

\end{description}

 
% 
 \subsection{计数器的命令\hfill Commands for Counters}
 \label{sec:ref-counter}
 
\begin{description}
\item[\Midx{\!\globalcounter!}\marg{ctr}]\mbox{}
\Item[\Midx{\!\globalcounter!}\texttt{*}]\mbox{}\par
\columnratio{0.55}
\begin{paracol}{2}
The command \!\globalcounter!\marg{ctr} declares that the counter
\meta{ctr} is global to all columns, while \!\globalcounter!|*| does so
for all counters.  An update of a \Uidx\gcounter{} in a column is seen by
any other columns.
\switchcolumn
命令 \!\globalcounter!\marg{ctr} 声明计数器\meta{ctr}在所有列中是全局的,而 \!\globalcounter!|*| 则对所有计数器都是如此。在某列中更新全局计数器会被其他列看到。 
\end{paracol}

\begin{itemize}
\columnratio{0.55}
\begin{paracol}{2}

\item
All column-local values of a descendant \lcounter{} of a \gcounter{} are
zero-cleared when the \gcounter{} is explicitly stepped by \!\stepcounter!
or \!\refstepcounter!, or implicitly by a sectioning command and so on.
\switchcolumn
当一个全局计数器被 \!\stepcounter! 或 \!\refstepcounter! 显式步进,或者通过节标题命令等隐式步进时,其子孙局部计数器的所有列局部值都会被清零。
\switchcolumn[0]*
\item
The counter \counter{page} is always global but an explicit update of it
by e.g., \!\setcounter! in a non-leftmost column is not seen by other
columns and is canceled even for the column itself after a \cswitch{} or a
page break in the column.  Therefore, if you want to make a \emph{jump} of
\counter{page}, it must be done in the leftmost column 0.  Note that a
jump from a page $p$ to $q$ can be seen in other columns even if they have
gone beyond $p$ \emph{before} the column 0 makes the jump, as far as
\counter{page} having $q$ (or its successor) is referred to by \!\pageref!
or through \emph{contents} files such as |.toc|\footnote{
Direct reference to \counter{page} may give an inconsistent result, as you
might have in ordinary \LaTeX{} documents.}.
\switchcolumn\item
计数器\counter{page}始终是全局的,但是在非最左列中通过 \!\setcounter! 进行的显式更新在其他列中是不可见的,并且在该列进行\cswitch{}或页面断页后,甚至对于该列本身也会被取消。因此,如果要进行\emph{jump}(即跳转)\counter{page},必须在最左列0中进行。请注意,即使其他列在列0进行跳转\emph{之前}已经超过了页面$p$,只要\counter{page}具有$q$(或其后继者)的值,并且通过 \!\pageref! 或通过\emph{contents}文件(如|.toc|)进行引用,其他列仍然可以看到从页面$p$跳转到$q$。\footnote{直接引用 \counter{page} 可能会导致不一致的结果,就像在普通的 \LaTeX{}文档中可能遇到的那样。}
\switchcolumn[0]*
\item
All counters except for \counter{page} are local by default.  This feature
may cause a problem with some packages including \textsf{marginnote} and
\textsf{(auto-)pst-pdf} having their own counters which must be global.
Since it is tough to find the name of such counters from package sources,
if you have something wrong with these (or other) packages, try to put
\!\globalcounter!|*| in your preamble and use \!\localcounter! shown below
to localize specific counters which you need to be local.
\switchcolumn\item
除了\counter{page}计数器外,默认情况下所有计数器都是局部的。这一特性可能会导致一些包(包括\textsf{marginnote}和\textsf{(auto-)pst-pdf})出现问题,这些包具有必须是全局的计数器。由于很难从包的源代码中找到这些计数器的名称,如果您在使用这些(或其他)包时遇到问题,请尝试在导言区中使用 \!\globalcounter!|*| 命令,并使用下面显示的 \!\localcounter! 命令将需要局部化的特定计数器局部化。
\switchcolumn[0]*
\item
Globalizing a \meta{ctr} being already global is just ignored without any
complaints.
\switchcolumn\item
如果一个已经是全局的\meta{ctr}被再次全局化,它会被静默地忽略,而不会有任何警告。
\end{paracol}
 \end{itemize}
 
 
 
 \item[\Midx{\!\localcounter!}\marg{ctr}]\mbox{}\par
 The command declares that the counter \meta{ctr} is local for each column.

这个命令声明计数器\meta{ctr}在每个栏目中都是局部的。
 \begin{itemize}
 \item
 Though this command is intended for localizing a \meta{ctr} which is once
 globalized, localizing a local counter does not causes any error but is
 just ignored.  Localizing the permanently global \counter{page} is also
 just ignored without any complaints.

 尽管该命令旨在将一次全局化的\meta{ctr}局部化,但将局部计数器局部化不会引起任何错误,只是被忽略。将永久全局\counter{page}局部化也只是被忽略,没有任何警告。
 \end{itemize}
 
 \item[\Midx{\!\definethecounter!}\marg{ctr}\marg{col}\marg{rep}]\mbox{}\par
 The command defines |\the|\meta{ctr} being \marg{rep} for the local use in
 the column \meta{col}.  That is, |\the|\meta{ctr} in the column \meta{col}
 acts as if it is defined by
 \!\renewcommand!\Arg{\cs{the}\meta{ctr}}\Arg{\meta{rep}}.

 该命令定义 |\the|\meta{ctr} 作为在列 \meta{col} 中的局部使用,其值为 \marg{rep}。也就是说,在列 \meta{col} 中,|\the|\meta{ctr} 的行为就像是通过 \!\renewcommand!\Arg{\cs{the}\meta{ctr}}\Arg{\meta{rep}} 定义的一样。
 
 
 
 \item[\Midx{\!\synccounter!}\marg{ctr}]\mbox{}\par
 The command \emph{broadcasts} the value of the \lcounter{} \meta{ctr} in
 the column in which the command appears to the values in all other columns.
 
 该命令将出现在的列中的\lcounter{} \meta{ctr}的值向所有其他列中的值进行\emph{broadcasts}(即广播)。
 \item[\Midx{\!\syncallcounters!}]\mbox{}\par
 The command broadcasts the values of all \lcounter{}s in the column in
 which the command appears to the values in all other columns.

该命令将出现在其中的列中的所有\lcounter{}的值广播到所有其他列中的相应值。
 \end{description}
 
% 

