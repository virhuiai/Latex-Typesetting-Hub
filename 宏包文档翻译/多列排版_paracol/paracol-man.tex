%% LaTeX file 'paracol-man'.
%% Copyright (C) 2005-2018
%%   Hiroshi Nakashima <h.nakashima@DOMAIN;  DOMAIN=media.kyoto-u.ac.jp>
%%   (Kyoto University)
%% This program can be redistributed and/or modified under the terms
%% of the LaTeX Project Public License distributed from CTAN
%% archives in directory macros/latex/base/lppl.txt; either
%% version 1 of the License, or any later version.

\ProvidesFile{paracol-man.tex}
[2018/12/31 v1.35 ]
\documentclass{ltxdoc}\normalmarginpar
\usepackage[heading=true
,scheme=chinese%中文方案
,fontset=none%不使用默认的字体设置
,space=auto%自动调整中英文间距
]{ctex}
\setCJKmainfont{FangZhengShuSong-GBK-1.ttf}[Path=/Users/virhuiai/hlProjects/Latex-Typesetting-Hub/font/方正/]%设置文本的中文有衬线字体
\setCJKsansfont{FangZhengHeiTi-GBK-1.ttf}[Path=/Users/virhuiai/hlProjects/Latex-Typesetting-Hub/font/方正/]%设置文本的中文无衬线字体为
\setCJKmonofont{FangZhengFangSong-GBK-1.ttf}[Path=/Users/virhuiai/hlProjects/Latex-Typesetting-Hub/font/方正/] %设置文本的中文等宽字体 
% \setCJKfamilyfont{fontKai}{LXGWWenKai-Regular.ttf}[Path=/Users/virhuiai/hlProjects/Latex-Typesetting-Hub/font/霞鹜文楷/]
\setCJKfamilyfont{fontKai}{FangZhengKaiTi-GBK-1.ttf}[Path=/Users/virhuiai/hlProjects/Latex-Typesetting-Hub/font/方正/]
\newcommand\fontKai{\CJKfamily{fontKai}}

\usepackage{color}
\usepackage{paracol}
\usepackage{newvbtm}
\DisableCrossrefs
\PageIndex
\CodelineNumbered
\RecordChanges
\OnlyDescription
\def\ONLYDESCRIPTION{}
\textwidth210mm
\advance\textwidth-40mm \oddsidemargin20mm \advance\oddsidemargin-1in
\columnsep10mm
\marginparwidth20mm \advance\marginparwidth-\marginparsep
\marginparsep.5\marginparsep
\raggedbottom
\OnlyDescription
\begin{document}
\DocInput{paracol.dtx}
\end{document}


% \columnratio{0.3,0.42,0.28}
\begin{paracol}{3}[\section{Introduction\hfill 介绍}]
\begin{VerbatimII}
\columnratio{0.3,0.42,0.28}
\begin{paracol}{3}[\section{Introduction\hfill 介绍}]
\begin{Verbatim}
左侧源码
\end{Verbatim}
\switchcolumn
This document..
\switchcolumn
本文档..
\switchcolumn[1]
Suppose ...
\switchcolumn
假设...
\end{paracol}    
\end{VerbatimII}
%%%%%%%%%%%%%%%%%%%%%%%%%%%%%%%%%%%%%%%%%%%%%%%%%%%%%%%
\switchcolumn
This document describes the usage of yet another multi-column package named
\textsf{paracol}.  The unique feature of the package is that columns are
typeset {\em in parallel.}
\switchcolumn
本文档介绍了另一个名为 \textsf{paracol} 的多栏排版宏包的使用方法。该宏包的独特特点是可以将栏以{\em 并行}的方式排版。 

\switchcolumn[1]
Suppose you are writing a bilingual document whose left column is written in
a language, say English, and right column has the translation of the left
column in another language, e.g., Japanese.  With the \textsf{paracol}
package you may write an English part of arbitrarily length and then {\em
switch} to its Japanese counterpart to place both parts side by side.  Of
course you may return to the English writing similarly.
\switchcolumn
假设你正在撰写一份双语文档,左栏使用一种语言(如英语),右栏则是左栏的另一种语言(如日语)的翻译。使用 \textsf{paracol} 宏包,你可以先写任意长度的英文部分,然后{\em 切换}到对应的日文部分,将两部分并排放置在一起。当然,你也可以类似地返回到英文撰写。

\switchcolumn[1]
The {\em\Uidx\cswitch} is always allowed when you complete an outermost
level paragraph.  You may be unaware whether a column is broken into
multiple pages before switching because the package automatically goes
back and forward to the correct page and vertical position when you switch
the column.  Moreover, you may {\em\Uidx\sync{}e} columns so that the tops
of the first paragraphs after switching in all columns are vertically
aligned.  At a \sync{}ation point, you may give a single-column text,
for example a common section header, optionally.  You may also switch
single-column and multi-column in a page arbitrary.
\switchcolumn
在外层段落完成后,总是允许使用 {\em\Uidx\cswitch} 命令。在切换之前,你可能不知道栏是否被分成多个页面,因为当你切换栏时,宏包会自动回到正确的页面和垂直位置。此外,你可以通过 {\em\Uidx\sync{}e} 命令来使列对齐,这样在切换后,所有列中第一个段落的顶部会垂直对齐。在 \sync{}ation 点,你可以选择给出单栏文本,例如一个公共的章节标题。你还可以随意在页面上切换单栏和多栏排版。

\switchcolumn[1]*
This manual itself is an example of two-column documents typeset by
\textsf{paracol}.  Since the author is not familiar with languages other
than English and Japanese and the latter should be hardly understood by
most of readers, the right column is the translation of the left English
column into a computational language.  That is, the right column is the
\LaTeX{} source code of the left column\footnote{%
Not really but its essence
is shown.\label{fn:first}}.
\switchcolumn
该手册本身是使用 \textsf{paracol} 排版的双栏文档的一个示例。由于作者对英语和日语以外的语言不熟悉,并且后者可能很难被大多数读者理解,所以右栏是左侧英文栏的计算语言翻译。也就是说,右栏是左栏的 \LaTeX{} 源代码\footnote{%
虽然不完全准确,但其本质得以展示。\label{fn:first}virhuiai在翻译时,做成了三栏,第一栏是源代码,第二栏是英文,第三栏是中文。}。
\end{paracol}


\end{document}

\begin{paracol}{2}

\switchcolumn



\switchcolumn*
This manual itself is an example of two-column documents typeset by
\textsf{paracol}.  
\switchcolumn
本手册本身就是使用 \textsf{paracol} 宏包排版的两栏文档的一个示例。

\end{paracol}
\begin{Verbatim}
\begin{paracol}{2}[\section{Introduction}]
\hbadness5000
en.....
\switchcolumn
中文....

\switchcolumn*
en.....
\switchcolumn
中文....
\switchcolumn*[\section{Basic Usage}]....
\end{paracol}
\end{Verbatim}
% 
\section{Basic Usage\hfill 基本用法}
\columnratio{0.2,0.48}
\begin{paracol}{3}
\begin{VerbatimII}
\section{Basic Usage\hfill 基本用法}
\columnratio{0.2,0.48}
\begin{paracol}{3}
\begin{Verbatim}
左侧源码
\end{Verbatim}
\switchcolumn
Loading..
\switchcolumn
加载...
\switchcolumn[1]*
The fundamental...
\switchcolumn
并列...
\end{paracol}
\end{VerbatimII}

\switchcolumn%%%%%%%%%%%%%%%%%%%%%%%%%%%%%%
Loading the package is very simple.  What you have to do is
\!\usepackage!|{|\Uidx{\env{paracol}}|}| in the preamble.  Note that
\textsf{paracol} can be used with \LaTeXe{} and does not work with
\LaTeX{} 2.09.
\switchcolumn
加载该宏包非常简单。在导言区使用 \!\usepackage!|{|\Uidx{\env{paracol}}|}| 命令即可。请注意,\textsf{paracol} 可以与 \LaTeXe{} 一起使用,不支持 \LaTeX{} 2.09。

\switchcolumn[1]
The fundamental means of parallel-column typesetting are the environment
\env{paracol} and the command \Uidx{\!\switchcolumn!}.  The \env{paracol}
environment needs an argument to specify the number of columns.  Thus the
following is the basic construct for two-parallel-column documents.
\switchcolumn
并列栏排版的基本手段是使用 \env{paracol} 环境和命令 \Uidx{\!\switchcolumn!}。\env{paracol} 环境需要一个参数来指定栏的数量。因此,以下是两栏并列文档的基本结构。

\switchcolumn[1]
\begin{quote}
\!\begin!|{|\env{paracol}|}{2}|\\
\textit{left column text}\\
\!\switchcolumn!\\
\textit{right column text}\\
\!\switchcolumn!\\
\textit{left column text}\\
\!\switchcolumn!\\
\textit{right column text}\\
\!\switchcolumn!\\
\mbox{\hspace{4em}}$\vdots$\\
\!\end!|{|\env{paracol}|}|
\end{quote}
\switchcolumn
\begin{quote}
\!\begin!|{|\env{paracol}|}{2}|\\
\textit{左栏文本}\\
\!\switchcolumn!\\
\textit{右栏文本}\\
\!\switchcolumn!\\
\textit{左栏文本}\\
\!\switchcolumn!\\
\textit{右栏文本}\\
\!\switchcolumn!\\
\mbox{\hspace{4em}}$\vdots$\\
\!\end!|{|\env{paracol}|}|
\end{quote}

\switchcolumn[1]*
The \!\switchcolumn! command may have an optional argument to specify the
column number (zero origin) to start.  That is, \!\switchcolumn!|[0]|
means to switch to the leftmost column, |\switchcolumn[1]| is to start the
second column and so on.  Thus the |\switchcolumn| without the optional
argument may be considered as \!\switchcolumn!|[|$i+1\bmod{n}$|]| where
$i$ is the ordinal of the column you are leaving from and $n$ is the
number of columns given to \env{paracol} environment.
\switchcolumn
\!\switchcolumn! 命令可以带有可选参数来指定从第几栏(从零开始计数)开始切换。也就是说,\!\switchcolumn!|[0]| 表示切换到最左边的栏,|\switchcolumn[1]| 表示从第二栏开始,依此类推。因此,不带可选参数的 |\switchcolumn| 可以视为 \!\switchcolumn!|[|$i+1\bmod{n}$|]|,其中 $i$ 是你离开的栏的序号,$n$ 是给定给 \env{paracol} 环境的栏数。

\end{paracol}
% \section{Column Synchronization\\栏同步}\label{sec:sync}

\columnratio{0.3,0.42,0.28}
\begin{paracol}{3}
%%%%%%%%%%%%%%%%%%%%%%%%%%%%%%%%%%%%%%%%%%%%%%%%%%%%%%%
\begin{VerbatimII}
\columnratio{0.3,0.42,0.28}
\begin{paracol}{3}
第1栏
\switchcolumn
第2栏
\switchcolumn
第3栏

\switchcolumn[0]*
同步 ...
\switchcolumn
...
\end{paracol}  
\end{VerbatimII}
%%%%%%%%%%%%%%%%%%%%%%%%%%%%%%%%%%%%%%%%%%%%%%%%%%%%%%%

\switchcolumn
The \!\switchcolumn! command may also be followed by a `|*|' to
{\em\Uidx\sync{}e} columns.  After you switch from a column to another by
\!\switchcolumn!|*| (or \!\switchcolumn!|[|$i$|]*|), all the columns are
vertically aligned at the bottom of the {\em deepest} one preceding the
command.  For example, the previous section has three \!\switchcolumn!|*|
commands at which left and right columns are vertically aligned.
\switchcolumn
\!\switchcolumn! 命令后面可以加上 `|*|',用来{\em 同步}栏。当你使用 \!\switchcolumn!|*|(或 \!\switchcolumn!|[|$i$|]*|)从一栏切换到另一栏时,所有栏都会垂直对齐在该命令之前最{\em 深}的栏的底部。例如,前一节使用了三个 \!\switchcolumn!|*| 命令,使左右两栏垂直对齐。

\switchcolumn[1]
The {\em starred} version of \!\switchcolumn! may have an optional
argument to specify a single-column {\em\Uidx\mctext} whose bottom is the
vertical alignment point of columns.  For example, \!\section!
commands in this manual are given as optional arguments
of \!\switchcolumn!|*| like;
\switchcolumn
{\em 带星号}版本的 \!\switchcolumn! 命令可以带有可选参数,用来指定一个单栏的{\em 同步文本},其底部作为栏的垂直对齐点。例如,本手册中的 \!\section! 命令作为 \!\switchcolumn!|*| 的可选参数给出,如下所示:

\switchcolumn[1]
\begin{Verbatim}
\switchcolumn*[\section{Basic Usage}]
\end{Verbatim}
\switchcolumn
\begin{Verbatim}
\switchcolumn*[\section{基础用法}]
\end{Verbatim}
\switchcolumn[1]
The \env{paracol} environment may also start with a \mctext{} by
specifying it as the optional argument of \!\begin!|{|\env{paracol}|}|.
For example, at the beginning of this document, the author put;
\switchcolumn
\env{paracol} 环境也可以以一个 \mctext{} 开始,将其指定为 \!\begin!|{|\env{paracol}|}| 的可选参数。例如,在本文档的开头,作者使用了以下代码:

\switchcolumn[1]*
\begin{Verbatim}
\begin{paracol}{2}[\section{Introduction}]
\end{Verbatim}

\switchcolumn
\begin{Verbatim}
\begin{paracol}{2}[\section{介绍}]
\end{Verbatim}

\end{paracol}    
\end{document}%%%%%%
% \section{Environments for Columns\hfill 栏环境}\label{sec:env}

\columnratio{0.3,0.42,0.28}
\begin{paracol}{3}
\begin{column}
\begin{VerbatimII}
...
\begin{column*}[\section{Environments for Columns}]
...
\end{column*}
\begin{column}
...
\end{column}
\end{VerbatimII}
%%%%%%%%%%%%%%%%%%%%%%%%%%%%%%%%%%%%%%%%%%%%%%%%%%%%%%%
\end{column}

\begin{column}
\Uidx{\Index{column-switching environment}}
\subsection{Environment \texttt{column}}
The \!\switchcolumn! is simple but you may prefer to pack the contents of a
column in an environment.  The \Uidx{\env{column}} environment is
available for this well-structuralization of \LaTeX{} sources for
parallel-columned documents. A construct;
\end{column}

\begin{column}
\subsection{\ttfamily column环境}
\!\switchcolumn! 命令很简单,但你可能更喜欢将一个栏的内容封装在一个环境中。\Uidx{\env{column}} 环境可以用于在 \LaTeX{} 文档中良好地组织并列栏的内容。以下结构:
\end{column}

\switchcolumn[1]
\begin{quote}
\!\begin!|{|\env{column}|}|\\
\textit{text for a column}\\
\!\end!|{|\env{column}|}|
\end{quote}
\noindent is (almost) equivalent to;
\begin{quote}
\!\switchcolumn!\\
\textit{text for a column}
\end{quote}
\switchcolumn
\begin{quote}
\!\begin!|{|\env{column}|}|\\
\textit{栏中文字}\\
\!\end!|{|\env{column}|}|
\end{quote}
(几乎)等同于:
\begin{quote}
\!\switchcolumn!\\
\textit{栏中文字}
\end{quote}



\end{paracol}


\end{document}%%%%%%





\begin{paracol}{2}
\switchcolumn
\switchcolumn*
The \Uidx{\env{column*}} environment is also available for the column
\sync{}ation and may have an optional argument for \mctext.
\switchcolumn
\Uidx{\env{column*}} 环境也可用于栏的同步,并且可以有一个可选参数用于 \mctext。
\end{paracol}

\begin{paracol}{2}
\begin{nthcolumn}{0}
\subsection{Environment \texttt{nthcolumn}}
The \!\switchcolumn! can start an arbitrarily specified column with the
column number given through its optional argument, but the \env{column}
environment cannot do it.  If you want to start $i$-th column, you have to
do \!\begin!|{|\Uidx{\env{nthcolumn}}|}{|$i$|}| (or
\Uidx{\env{nthcolumn*}} with an optional argument to \sync{}e).
\end{nthcolumn}

\begin{nthcolumn}{1}
\subsection{\texttt{nthcolumn}环境}
\!\switchcolumn! 可以通过可选参数指定要开始的任意列的列号,但 \env{column} 环境不能这样做。如果你想要开始第 $i$ 列,你需要使用 \!\begin!|{|\Uidx{\env{nthcolumn}}|}{|$i$|}|(或带有可选参数的 \Uidx{\env{nthcolumn*}} 来进行同步)。
\end{nthcolumn}
\end{paracol}
\begin{Verbatim}
\begin{paracol}{2}
\begin{nthcolumn*}{1}
\subsection{...}
...
\end{nthcolumn*}

\begin{nthcolumn}{0}
\subsection{...}
...
\end{nthcolumn}
\end{paracol}
\end{Verbatim}


\begin{paracol}{2}
\begin{leftcolumn*}
\subsection[Environments \texttt{leftcolumn} and \texttt{rightcolumn}]
    {Environments \texttt{leftcolumn} and\\\texttt{rightcolumn}}
The environments \Uidx{\env{leftcolumn}} and \Uidx{\env{rightcolumn}} (and
their starred versions with an optional argument) are available as more
convenient means than saying \!\begin!|{|\env{nthcolumn}|}{0}| to switch
to the left(most) column and
\!\begin!|{|\env{nthcolumn}|}{1}| to the right (but may not be rightmost)
one.

\Uidx{\EnvIndex{leftcolumn*}}\Uidx{\EnvIndex{rightcolumn*}}

\end{leftcolumn*}

\begin{rightcolumn}
\subsection{\ttfamily leftcolumn 和 rightcolumn \\环境}
环境 \Uidx{\env{leftcolumn}} 和 \Uidx{\env{rightcolumn}}(以及带有可选参数的星号版本)可作为比使用 \!\begin!|{|\env{nthcolumn}|}{0}| 切换到最左栏 和 \!\begin!|{|\env{nthcolumn}|}{1}| 切换到右栏(可能不是最右)更方便的方法。
\end{rightcolumn}

\end{paracol}
%\section{Floats, Footnotes and Counters}

\columnratio{0.3,0.42,0.28}
\begin{paracol}{3}

\begin{VerbatimII}
\switchcolumn[0]*
\begin{figure*}\nosv
\def\arraystretch{0.8}
\centerline{\begin{tabular}[b]{|c|}\hline
    \hbox to.9\textwidth{}\\
    three-column figure \#1\\
    \\\hline
    \end{tabular}}
\caption{A Three-Column Figure}
\end{figure*}

\switchcolumn
\begin{figure}[t]\nosv
\def\arraystretch{0.8}
\centerline{\begin{tabular}[b]{|c|}\hline
    \hbox to.9\columnwidth{}\\\\
    single-column figure \#1\\
    \\\\\hline
    \end{tabular}}
\caption{A Single-Column Figure}
\end{figure}

\switchcolumn
\begin{figure}[t]\nosv
\def\arraystretch{0.8}
\centerline{\begin{tabular}[b]{|c|}\hline
    \hbox to.9\columnwidth{}\\
    \ttfamily single-column figure \#2\\
    \\\hline
    \end{tabular}}
\caption{\ttfamily Another Single-Column Figure}
\end{figure}
\end{VerbatimII}
%%%%%%%%%%%%%%%%%%%%%%%%%%%%%%%%%%%%%%%%%%%%%%%%%%%%%%%
\switchcolumn
\subsection{Figures and Tables}
Double-column figures\slash tables (or those
spanned multiple columns if you have three or more) may be placed by
\env{figure*} and \env{table*} environments as usual\footnote{
See Section~\ref{sec:problem} for the appearance order issue of
double-column floats.}.

\switchcolumn
\subsection{图表}
双栏图表(如果有三栏或更多栏,则为跨多栏的图表)可以像往常一样使用 \env{figure*} 和 \env{table*} 环境来放置\footnote{请参见第~\ref{sec:problem} 节有关双栏浮动体出现顺序问题的内容。}。

\switchcolumn[1]
A single-column figure\slash table will be placed in the column in which
you put \env{figure} and \env{table}.  For example, the body of a
\env{figure} environment in a \env{leftcolumn} environment is
\emph{always} placed in a left column.  That is, even if the column of the
\emph{current} page does not have enough room to place the figure, it will
not be thrown to the right column but will be placed in the left column of
the next page\footnote{Or some farther page if \LaTeX{} cannot solve the placement problem wisely.}.
\switchcolumn
单栏图表将放置在你放置 \env{figure} 和 \env{table} 环境的栏中。例如,在 \env{leftcolumn} 环境中的 \env{figure} 环境中的内容将始终放置在左栏中。也就是说,即使当前页面的栏没有足够的空间放置图表,它也不会被放置在右栏,而是会放置在下一页的左栏\footnote{如果 \LaTeX{} 无法明智地解决放置问题,则可能放置在更远的页面上。}。

\switchcolumn[1]
\begin{table}[b]\nosv
\caption{A Single-Column Table}
\centerline{\begin{tabular}[t]{|l|c|r|}\hline
An&example&of\\\hline
single&column&table\\\hline
\end{tabular}}
\end{table}
\switchcolumn
\begin{table}[b]\nosv
\caption{\ttfamily Another Single-Column Table}
\label{tab:right}
\centerline{\ttfamily \begin{tabular}[t]{|l|r|}\hline
  Another&example\\\hline
  of&single\\\hline
  column&table\\\hline
  \end{tabular}}
\end{table}

\switchcolumn[1]
Another caution about float placement is that you have to be careful when
you try to put a top-float explicitly with |t|-option or implicitly without
placement option (i.e., |tbp| in most classes) and to \sync{}e columns.
The rule is as follows; after you \sync{}e columns in a page, the page
cannot have top-floats any more.  When you \sync{}e columns,
\textsf{paracol} fixes a virtual horizontal line in the page as the
\sync{}ation barrier.  Thus no top-floats cannot be added above the
line\footnote{Even if you have enough space above, sorry.}.
\switchcolumn
关于浮动位置的另一个警告是,当你试图使用 |t| 选项显式地放置一个顶部浮动,或者不使用放置选项隐式地放置(即,在大多数类中的 |tbp|),并且要同步列时,你必须小心。规则如下:在你在一个页面中同步列后,该页面不能再有顶部浮动。当你同步列时,\textsf{paracol} 在页面中固定一个虚拟的水平线作为同步屏障。因此,不能在该线以上添加顶部浮动\footnote{即使你在上方有足够的空间,抱歉。}。

\switchcolumn[1]
Therefore, the author put two \env{figure} environments for the figures
shown in this page into the \env{leftcolumn*} and \env{rightcolumn}
environment for the previous section.
\switchcolumn
因此,作者将在上一节的 \env{leftcolumn*} 和 \env{rightcolumn} 环境中放入本页显示的两个 \env{figure} 环境。

\switchcolumn[1]
\subsection{Footnotes and Marginal Notes}
Footnotes are also put at the bottom of the column in which \!\footnote!
commands and their references reside (like this\footnote{%
Unless you specify to make footnotes {\em page-wise} as explained in
Section \ref{sec:ref-scfnote} and \ref{sec:fnnp}.}),
\switchcolumn
\subsection{脚注和边注}
脚注也会放置在包含 \!\footnote! 命令及其引用的栏的底部(如本页所示\footnote{除非你在第 \ref{sec:ref-scfnote} 节和 \ref{sec:fnnp} 节中指定将脚注{\em 按页}处理。}),

\switchcolumn[1]
as shown in page~\pageref{fn:first} and this page.  Marginal
notes behave similarly like what you are seeing in the left margin of this
sentence\marginpar{\raggedright An example of marginal note.}
\switchcolumn
如第~\pageref{fn:first}页和本页所示。边注表现类似于你看到的这句话左 margin 中的样式\marginpar{\raggedright 一个边注示例。}

\switchcolumn[1]  
and the right marginal note in this page\footnote{%
If you have three or more columns, marginal notes of the second or
succeeding columns are placed in the right margin in default setting.  The
\textsf{paracol} package solves the placement problem of marginal notes
from two or more columns sharing a side margin by moving some of them down
if they conflict over the space with each other.}.
\switchcolumn
以及本页中的右边距注释\footnote{如果你有三列或更多列,第二列或后续列的边距注释在默认设置中放置在右边距。 \textsf{paracol}包处理来自两个或更多共享侧边距的列的边距注释的放置问题,如果它们在空间上彼此冲突,将其中一些向下移动。}。
\end{paracol}


\columnratio{0.3,0.42,0.28}
\begin{paracol}{3}

\begin{VerbatimII}
\end{VerbatimII}

\switchcolumn[1]
\subsection{Local and Global Counters}
\UsageIndex{local counter}
\UsageIndex{global counter}
You probably found that the numbering of figures and tables is \emph{global}
while that of footnotes are \emph{local}.  That is, the figure in the right
column of the previous page has number~3 following its left-column
counterpart Figure~2.  The tables in the page are also numbered as 1 and 2
crossing the column boundary.  However, the footnotes in each column have
their own numbering sequence.  Moreover, the footnote numbers in left
columns are typeset in roman font while those in right columns have italic
shapes.  Similarly, subsection numbering is local and the headings in right
columns have typewriter-face numbers.
\switchcolumn
\subsection{局部和全局计数器}
你可能发现,图表的编号是\emph{全局}的,而脚注的编号是\emph{局部}的。也就是说,上一页右栏的图表在其左栏对应的图表之后编号为3,而页面上的表格也是以1和2为编号跨越栏边界。然而,每栏中的脚注有自己的编号序列。此外,左栏中的脚注号码以罗马字体排版,而右栏中的脚注号码以斜体形式排版。类似地,小节编号是局部的,右栏标题的编号使用打字机字体。

\switchcolumn[0]*
\begin{itemize}\item[]
\Uidx{\!\globalcounter!}|{figure}|\\
\!\globalcounter!|{table}|
\end{itemize}

\switchcolumn[1]
This happens because the author declared the counters \counter{figure} and
\counter{table} are \emph{global} in the preamble of this document by
saying;
\switchcolumn
这是因为作者在文档的导言部分中声明了计数器 \counter{figure} 和 \counter{table} 是\emph{全局}的,声明首栏所示。

\switchcolumn[1]
and do nothing about \counter{footnote} and \counter{subsection} counters.
By default, all the counters except for |page| are local to columns.  The
value of a \lcounter{} of a column is saved somewhere when you leave the
column, and it is restored when you revisit the column.  The initial values
of the \lcounter{}s are the values they have at
\!\begin!|{|\env{paracol}|}|.  After you close the \env{paracol}
environment, the values of the leftmost column are used for the rest of
your document until you start new \env{paracol} environment.  On a
restart, \lcounter{}s in a column have the values they had at the last
\Endparacol, except for those which have been modified outside the
environment because the modifications are \emph{broadcasted} to
\lcounter{}s in all columns.  You will see the effect of this
inter-environment counter value conservation in the footnote numbers in
the right column in page~\pageref{fn:right3} and \pageref{fn:right4}.
\switchcolumn
但对于计数器 \counter{footnote} 和 \counter{subsection} 未进行任何处理。默认情况下,除了 |page| 计数器外,所有的计数器都是局部的。当你离开栏目时,栏目的局部计数器值会被保存住,当你再次访问该栏目时,该值会被恢复。在 \env{paracol} 环境的初始值为局部计数器的值。当你关闭 \env{paracol} 环境后,剩余部分的文档将使用最左边栏的值,直到你开始新的 \env{paracol} 环境。重新开始时,栏目中的局部计数器具有最后一个 \Endparacol 时的值,除非在环境外进行了修改,因为这些修改会被\emph{广播}到所有栏的局部计数器中。你将在第\pageref{fn:right3}页和第\pageref{fn:right4}页中看到这种跨环境计数值保存的效果,表现在右栏的脚注号码上。

\switchcolumn[1]
This broadcasting of a \lcounter{} value can be done explicitly in
\env{paracol} environments by a command $\Uidx{\!\synccounter!}\Arg{ctr}$.
This command makes $\mathit{ctr}$ in all columns have the value of that in
the column in which the command appears.  In addition, another command
\Uidx{\!\syncallcounters!} performs this broadcasting for all \lcounter{}s.
\switchcolumn
可以在\env{paracol}环境中通过命令$\Uidx{\!\synccounter!}\Arg{ctr}$来显式地进行局部计数器值的广播。所有栏中的$\mathit{ctr}$都将同步为命令调用所在栏中的值。此外,另一个命令 \Uidx{\!\syncallcounters!} 可以对所有局部计数器进行这种广播操作。

\switchcolumn[1]
If you make a counter global by the command \!\globalcounter!, the
save/restore operations are not performed to the counter and thus it is
globally incremented by \verb|\[ref]|\AB|stepcounter|
\SpecialIndex{\refstepcounter}\SpecialIndex{\stepcounter}

\switchcolumn
若用 \!\globalcounter! 将某计数器声明为全局的,则不会对其执行保存/恢复操作,它会通过 \verb|\[ref]|\AB|stepcounter| 全局递增。


\end{paracol}


\end{document}
%%%%%%%%%%%%%%%%%%%%%%%%%%%%%%%%%%%%%%%%%%%%%%%%%%%%%%%
%%%%%%%%%%%%%%%%%%%%%%%%%%%%%%%%%%%%%%%%%%%%%%%%%%%%%%%










\switchcolumn*
or commands such as \!\caption! and \!\section!.  Note that the value of a
\gcounter{} depends on the place where it is incremented (or set) in
the \emph{source code} rather than where it appears in the output.  Thus
if the author put a \env{table} environment here to increment \env{table}
counter, the right-column table at the bottom of page~\pageref{tab:right}
would be Table~3 because its \env{table} environment does not appear yet
in the source code.  Note that, however, though the counter \counter{page}
is global as expected, its numbering is consistent among all columns as
far as you refer to the value by $\!\pageref!\Arg{label}$ and/or see the
values in table of contents, etc.
\switchcolumn
或者诸如 \!\caption! 和 \!\section! 等命令。请注意,一个\gcounter{}的值取决于它在\emph{源代码}中递增(或设置)的位置,而不是它在输出中出现的位置。因此,如果作者在这里放置了一个\env{table}环境来递增\env{table}计数器,那么在第\pageref{tab:right}页底部的右栏表格将被标记为表格3,因为它的\env{table}环境在源代码中尚未出现。请注意,尽管计数器\counter{page}是全局的,但只要通过 $\!\pageref!\Arg{label}$ 引用该值,或者在目录中查看值等,其编号在所有栏目中是一致的。

\switchcolumn*
Another counter which the author made global in this document is
\counter{section}.  As explained in Section~\ref{sec:sync}, an optional
\mctext{} of \cswitch{} is considered as in the leftmost column.  Since
\!\section! commands in this document are always given in \mctext{}s, so
far, it seems unnecessary to make \counter{section} global because it is
incremented correctly in the leftmost column.  However, the stepping
\counter{section} has a side effect to reset its descendent counter
\counter{subsection} and referred to from \!\thesubsection! command.  Thus
if \counter{section} were local, the right-column subsections in
Section~\ref{sec:env} would be numbered as ``0.1'', ``0.2'' and ``0.3''
because the local value of \counter{section} would be zero.  Moreover, the
right-column subsections of this section would be ``0.4'', ``0.5'' and
``0.6'' because stepping \counter{section} local to the left column would
not reset \counter{subsection} local to the right column.
\switchcolumn
在本文档中,作者还将\counter{section}计数器声明为全局的。如第~\ref{sec:sync}节所述,\cswitch{}的可选\mctext{}被视为最左边的栏目。由于本文档中的 \!\section! 命令总是在\mctext{}中给出,因此目前似乎没有必要将\counter{section}设置为全局,因为它在最左边的栏目中递增是正确的。然而,递增\counter{section}会对其子计数器\counter{subsection}产生副作用,并且从 \!\thesubsection! 命令中引用。因此,如果\counter{section}是局部的,那么在第~\ref{sec:env}节中右栏的子章节将被编号为“0.1”、“0.2”和“0.3”,因为\counter{section}的局部值将为零。此外,本节的右栏子章节将被编号为“0.4”、“0.5”和“0.6”,因为局部递增的\counter{section}不会重置右栏局部的\counter{subsection}。


\switchcolumn*
You may give a local appearance to a counter \textit{ctr} for the $i$-th
column (zero origin) by a command;
\begin{itemize}\item[]
\Uidx{\!\definethecounter!}|{|\textit{ctr}|}{|$i$|}{|\textit{def}|}|
\end{itemize}
\switchcolumn
你可以通过命令给第$i$栏目(从零开始计数)的计数器\textit{ctr}赋予局部的外观;

\switchcolumn*
where \textit{def} is to be the body of the local definition of
|\the|\textit{ctr}.  For example, the preamble of this document has the
following to give non-default defitions to \!\thefootnote! and
\!\thesubsection! for right columns.

\begin{Verbatim}
\definethecounter{footnote}{1}{%
\textit{\arabic{footnote}}}
\definethecounter{subsection}{1}{%
\texttt{%
    \arabic{section}.\arabic{subsection}}}
\end{Verbatim}
\switchcolumn
其中\textit{def}是局部定义|\the|\textit{ctr}的内容。例如,本文档的导言部分具有以下内容,为右栏的\!\thefootnote!和\!\thesubsection!赋予非默认的定义。

\end{paracol}


% \section{Closing \texttt{paracol} Environment and Page Flushing\hfill 关闭 \texttt{paracol} 环境和页面刷新}
\label{sec:man-close}

The final example shown here is this single-column text which the author put
after the \env{paracol} environment above is closed.  As you are seeing, a
\env{paracol} environment can be finished at any vertical position in a
page and can be followed by ordinary single column texts.


这里展示的最后一个例子是在上面关闭的\env{paracol} 环境之后,作者放置的这个单栏文本。正如你所见,\env{paracol} 环境可以在页面的任何垂直位置结束,并且可以跟随普通的单栏文本。

\columnratio{0.3,0.42,0.28}
\begin{paracol}{3}
\begin{VerbatimII}
\begin{paracol}{2}
\begin{leftcolumn}
The enviro ... 
\end{leftcolumn}
\begin{rightcolumn}
source
\end{rightcolumn}
\end{paracol}
Now the aurthor will do ...
\end{VerbatimII}
%%%%%%%%%%%%%%%%%%%%%%%%%%%%%%%%%%%%%%%%%%%%%%%%%%%%%%%
\switchcolumn

\switchcolumn[1]
The environment may also be restarted anywhere you like as shown here.
\switchcolumn
此处展示了环境可以在任何位置重新开始。

\switchcolumn[1]
The last issue is to flush a page.  The ordinary \!\newpage! command works
as you expect.  If you say \!\newpage! in the left column in a page, the
contents following it will appear in the left column in the next page.  Note
that this does not affect the layout of the right column.
\switchcolumn
最后一个问题是如何换页。\!\newpage! 命令按照你的期望工作。如果你在页面的左栏使用 \!\newpage! 命令,在它之后的内容将出现在下一页的左栏中。请注意,这不会影响右栏的布局。

\switchcolumn[1]
To flush all columns in a page, a command \Uidx{\!\flushpage!} is
available.  This command in $i$-th column is almost equivalent to;
\begin{itemize}\item[]
\!\switchcolumn!|[|$i$|]*[|\!\newpage!|]|
\end{itemize}
\switchcolumn
要在页面中刷新所有栏目,可以使用命令 \Uidx{\!\flushpage!}。这个命令在第$i$栏中几乎等同于:
\begin{itemize}\item[]
\!\switchcolumn!|[|$i$|]*[|\!\newpage!|]|
\end{itemize}

\switchcolumn[1]
but more robust\footnotemark\label{fn:flush}.
The ordinary page breaking command \Uidx{\!\clearpage!} may also be used
to flush all columns and to start a fresh page, but it has a side effect
to put all figures and tables which are not yet output.
\switchcolumn
但更加健壮 \footnotemark\label{fn:flush} 。普通的换页命令 \Uidx{\!\clearpage!} 也可以用于刷新所有栏目并开始新的一页,但它会导致尚未输出的所有图表被放置在同一页中。
\end{paracol}

Now the author will do |\flushpage| shortly to start a real binlingual
example from the next page, after showing another example of closing 
\env{paracol} environments in this sentence and of restarting in the next
one, in which {\em unbalanced column width} is demonstrated using
\Uidx{\!\columnratio!} command shown in Section~\ref{sec:ref-colwidth}.

现在作者将很快使用|\flushpage|命令,在下一页开始一个真正的双语示例,此前在本句中展示了另一个关闭\env{paracol}环境的例子,并在下一句中重新开始,在其中使用了在第~\ref{sec:ref-colwidth}节中展示的 \Uidx{\!\columnratio!} 命令演示了{\em 不平衡的列宽}。

\columnratio{0.3,0.42,0.28}
\begin{paracol}{3}
\begin{VerbatimII}
\columnratio{0.6} 
\begin{paracol}{2} 
\begin{leftcolumn}
O.K., ... 
\end{leftcolumn} 
\begin{rightcolumn}
source 
\end{rightcolumn}
\end{VerbatimII}
%%%%%%%%%%%%%%%%%%%%%%%%%%%%%%%%%%%%%%%%%%%%%%%%%%%%%%%
\switchcolumn

O.K., we have restarted \env{paracol} environment and we will see the
effect of \!\flushpage! now!!\footnotetext{
For example \texttt{\string\switchcolumn*} may flush a page for the
\sync{}ation and thus \texttt{\string\newpage} may leave an empty page.}
\switchcolumn
好的,我们已经重新开始了\env{paracol}环境,现在我们将看到 \!\flushpage! 命令的效果!!\footnotetext{例如,\texttt{\string\switchcolumn*}可能会为同步而刷新页面,因此\texttt{\string\newpage}可能会留下一个空白页。}

\newenvironment{Gverse}{\ensurevspace{2\baselineskip}
\begin{leftcolumn*}
\begin{myverse}}
{\end{myverse}\end{leftcolumn*}}
\newenvironment{Everse}{%
\begin{rightcolumn}
\begin{myverse}}
{\end{myverse}\end{rightcolumn}}
\newenvironment{Cverse}{%
\begin{nthcolumn}{2}
\begin{myverse}}
{\end{myverse}\end{nthcolumn}}
\makeatletter
\newenvironment{myverse}{\leftmargini0pt\partopsep0pt\verse}{\endverse}


\begin{leftcolumn*}[
\centerline{\Large An Die Freude/To Joy}\label{page:bfreude}\smallskip
\centerline{\large Friedrich Schiller 弗里德里希·席勒}\smallskip
The following is the libretto of the fourth movement of Beethoven's Ninth
Symphony, his adaptation of Schiller's ode ``An Die Freude'' (or ``To Joy'' in
English). Beethoven's additions and revisions are indicated in italics.

以下是贝多芬第九交响曲第四乐章的歌剧剧本,他改编自席勒的颂歌《致欢乐》(或英文版的《To Joy》)。贝多芬的添加和修订以斜体显示。]
\end{leftcolumn*}

\begin{Gverse}
\itshape O Freunde, nicht diese T\"one! \\
Sondern la{\ss}t uns angenehmere anstimmen und freu\-denvollere
\footnote{If I had been a good student in my German class, I could find
the German translation of the right column footnote \ref{fn:right4} is
``Dieser Teil wurde van Beethoven hinzugef\"ugt'' by myself without
the kind help from a user.}.
\end{Gverse}
\begin{Everse}
\itshape Oh friends, no more of these sad tones!\\
Let us rather raise our voices together\\
In more pleasant and joyful tones
\footnote{This part was added by Beethoven.\label{fn:right4}}.
\end{Everse}
\begin{Cverse}
    a
\end{Cverse}


\end{paracol}


\end{document}
%%%%%%%%%%%%%%%%%%%%%%%%%%%%%%%%%%%%%%%%%%%%%%%%%%%%%%%
%%%%%%%%%%%%%%%%%%%%%%%%%%%%%%%%%%%%%%%%%%%%%%%%%%%%%%%


\begin{leftcolumn}




% \subsection{Environment \texttt{paracol}\hfill \texttt{paracol}环境}
\label{sec:ref-paracol}

\begin{description}%值得学习
\item[\ENV{paracol}{\marg{num}\oarg{text}}]\mbox{}\par
\columnratio{0.6}
\begin{paracol}{2}
The environment \env{paracol} contains \meta{body} typeset in \meta{num}
columns in parallel.  The optional \meta{text} is put spanning all columns
prior to the multi-columned \meta{body}.
\switchcolumn
环境\env{paracol}中包含以 \meta{num} 栏并列排列的 \meta{body}。可选的 \meta{text} 将跨越所有栏之前放置在多栏的 \meta{body} 之前。
\end{paracol}

\begin{itemize}
\columnratio{0.6}
\begin{paracol}{2}
\item
The environment may start from \emph{any} vertical position in a page,
i.e., not necessary at the top of a page.  The single-column
{\em\Uidx\preenv} of the {\em\Uidx\spage} in which \beginparacol{} lies
are naturally connected to the beginning part of \meta{body} in each
column, unless the page has footnotes\footnote{%
With \Mgfnote{} layout shown in Section~\ref{sec:ref-scfnote}, the
footnotes in the single-column contents are merged with those in
\env{paracol} environment and are put at the bottom of the \spage{}
together as shown in this page.}

or bottom floats.  If these kinds of bottom stuff exist, they are
put above the multi-columned \meta{body}, or the spanning \meta{text}

\UsageIndex{spanning text}

if provided, with a vertical skip of \!\textfloatsep! separating them if
bottom floats exist, or of \!\belowfootnoteskip! described in
Section~\ref{sec:ref-scfnote} if only footnotes exist.  The
\emph{deferred} floats which have not yet appeared in the starting page
and thus will appear in the next or succeeding pages are considered as
\pwise{} floats given in the environment.
\switchcolumn
\item
此环境可以从页面的\emph{任何}垂直位置开始,
即不一定在页面顶部。位于 \beginparacol{} 所在的 {\em\Uidx\spage} 中的单栏 {\em\Uidx\preenv} 自然与每个栏的\meta{body}的开头部分连接在一起,除非页面有脚注\footnote{使用在第~\ref{sec:ref-scfnote}节中展示的\Mgfnote{}布局,单栏内容中的脚注与\env{paracol}环境中的脚注合并在一起,并一起放置在\spage{}的底部,就像本页所示。}
,或底部浮动体。如果存在这些底部内容,则它们将位于多栏的\meta{body}之上,或者位于跨越的\meta{text}之上(如果提供了),并使用垂直间距 \!\textfloatsep! 将它们分隔开(如果存在底部浮动体),或者使用在第~\ref{sec:ref-scfnote}节中描述的 \!\belowfootnoteskip! (仅当存在脚注时)。尚未出现在起始页面中的\emph{延迟}浮动体将被视为在环境中给出的\pwise{}浮动体,它们将出现在下一页或后续页面中。
\switchcolumn[0]*
\item
The environment can be enclosed in a \env{list}{\em-like environment} such
as \env{enumerate}, \env{itemize} and \env{description}.  If so, \!\item!s
in each column are typeset using the parameters of the surrounding
environment such as \!\leftmargin! and \!\rightmargin!.  %For example, the
following short \env{paracol} environment is included in an \env{itemize}
for this and other \!\item!s in this page.
\switchcolumn
\item
该环境可以被封装在类似于 \env{enumerate}、\env{itemize} 和 \env{description} 的\emph{类似列表}环境中。如果这样做,每个栏中的 \!\item! 将使用周围环境的参数进行排版,如 \!\leftmargin! 和 \!\rightmargin!。%例如,以下简短的\env{paracol}环境被包含在一个\env{itemize}中,用于本页和其他 \!\item!。

\end{paracol}

You are now seeing the switching to/from multi-columned and \env{itemize}d
texts are naturally connected with the last and this single-columned
sentences.  You may feel the space between two columns above is too large
but it simply results from the large total \!\leftmargin!s of the outer
\env{description} and this \env{itemize}, which make the right column
shifted right.  A simple remedy for this large space is to make
\!\columnsep! narrower, for example 0\,pt as shown below.

您现在看到的切换到/从多栏和\env{itemize}文本与上一个和本个单栏句子自然连接在一起。您可能会觉得上面两栏之间的空间太大,但这只是由于外部\env{description}和此\env{itemize}的总 \!\leftmargin! 较大,使得右栏向右偏移。修复这个大空间的简单方法是使 \!\columnsep!变窄,例如像下面显示的0\,pt。

\begin{Verbatim}
\columnsep0pt
\end{Verbatim}

\columnsep0pt
\begin{paracol}{2}
\item
This \!\item! is wider than the last \!\item! above because
\!\columnsep! is 0\,pt.

这个 \!\item! 比上面的最后一个 \!\item! 更宽,因为 \!\columnsep! 是0\,pt。
\switchcolumn

\item
Therefore, this \!\item! is shifted left a little bit to make
inter-column spece narrower.

因此,为了使栏间距更窄,这个 \!\item! 向左移动了一点。
\end{paracol}

\columnratio{0.55}
\begin{paracol}{2}
\item
All \Uidx\lcounter{}s in all columns are initialized to have the values at
\beginparacol{} on its first occurrence.  On the second and succeeding
occurrences of \beginparacol, the \lcounter{}s in each column have the
value at the last \Endparacol, unless they are modified after the
\Endparacol.  If a counter is modified (or declared by \!\newcounter!)
after the \Endparacol, the local versions of the counter in all columns
commonly have the value at \beginparacol.
\switchcolumn
\item
所有栏中的局部计数器都被设为\beginparacol{}首次出现时的值。在\beginparacol{}的第二次及后续出现中,每个栏中的局部计数器都具有上一个\Endparacol{}处的值,除非在\Endparacol{}之后对其进行了修改。如果在\Endparacol{}之后修改了计数器(或通过 \!\newcounter! 声明了计数器),所有栏中的局部计数器都通常具有\beginparacol{}处的值。
\switchcolumn[0]*
\item
The environment may end at \emph{any} vertical position in a page, i.e.,
the {\em\Uidx\postenv} being the single-column texts and others
following \Endparacol{} in the {\em\Uidx\lpage} of the environment may not
start from the top of a page.  If any columns don't have deferred
\cwise{} floats and the most advanced {\em\Uidx\lcolumn} at
\Endparacol{} has neither of footnotes\footnote{

With \Mgfnote{} layout shown in Section~\ref{sec:ref-scfnote}, the
footnotes in the closing \env{paracol} environment are merged with those
in \postenv{} and are put at the bottom of the page{} together as shown in
this page.}

nor bottom floats, its bottom is naturally connected to the \postenv{}.
If the \lcolumn{} has these kinds of bottom stuff, they are put above the
\postenv{}, with a vertical skip of \!\textfloatsep! separating them if
bottom floats exist.  All deferred \cwise{} floats given in the
environment are flushed before the \postenv{} appears, possibly creating
{\em\Uidx\fcolumn{}s} only with floats.  On the other hand, deferred
\pwise{} floats given in the environment are considered as deferred
(single-) \cwise{} floats given just after \Endparacol.
\switchcolumn
该环境可以在页面的\emph{任何}垂直位置结束,即\emph{\Uidx\postenv}是单栏文本,而在环境的\emph{\Uidx\lpage}中的\Endparacol{}之后的其他内容可能不会从页面顶部开始。如果任何栏没有延迟的\cwise{}浮动体,并且最后一个\Endparacol{}处的\emph{\Uidx\lcolumn}既没有脚注\footnote{使用在第~\ref{sec:ref-scfnote}节中展示的\Mgfnote{}布局,\env{paracol}环境中的脚注与\postenv{}中的脚注合并在一起,并一起放置在页面底部,就像本页所示。},也没有底部浮动体,则其底部自然与\postenv{}连接在一起。如果\lcolumn{}具有这些类型的底部内容,则它们将位于\postenv{}之上,如果存在底部浮动体,则它们之间使用垂直间距 \!\textfloatsep! 分隔开。在\postenv{}出现之前,环境中给出的所有延迟\cwise{}浮动体都会被清除,可能只留下具有浮动体的{\em\Uidx\fcolumn{}s} 。另一方面,环境中给出的延迟\pwise{}浮动体被视为在\Endparacol{}之后立即给出的延迟(单个)\cwise{}浮动体。
\switchcolumn[0]*
\item
The values of all \lcounter{}s in the leftmost column are used as the
initial values of them in the \postenv.
\switchcolumn
\item
左侧栏中所有\lcounter{}的值被用作\postenv{}中对应\lcounter{}的初始值。
\switchcolumn[0]*
\item
The \env{paracol} environment cannot be nested, or you will have an error
message of illegal nesting.
\switchcolumn
\item 不能嵌套使用\env{paracol}环境,否则会出现非法嵌套的错误消息。 
\switchcolumn[0]*
\item
The commands \!\switchcolumn!, \!\synccounter!, \!\syncallcounters! and
\!\flushpage!, and environments \env{column}(|*|), \env{nthcolumn}(|*|),
\env{leftcolumn}(|*|) and \env{rightcolumn}(|*|) are {\em local} to
\env{paracol} environment and thus undefined outside the
environment\footnote{

Unless you dare to define them.}.

The command \!\clearpage! is of course usable outside and inside the
environment but its function inside is a little bit different from outside.
\switchcolumn
\item 命令 \!\switchcolumn!、\!\synccounter!、\!\syncallcounters! 和 \!\flushpage!,以及环境\env{column}(|*|)、\env{nthcolumn}(|*|)、\env{leftcolumn}(|*|)和\env{rightcolumn}(|*|)是\env{paracol}环境中的{\em 局部}命令和环境,因此在环境外部是未定义的\footnote{除非你敢于定义它们。}。

命令 \!\clearpage! 当然可以在环境内外使用,但在环境内部的功能与外部略有不同。
\end{paracol}
\end{itemize}



\item[\ENV{paracol}{\oarg{numleft}\marg{num}\oarg{text}}]\mbox{}
\Item[\ENV{paracol}{\oarg{numleft}\texttt{*}\marg{num}\oarg{text}}]
\mbox{}\par
\changes{v1.3-2}{2013/09/17}
{Add description of parallel-paging.}

If a \beginparacol{} has the optional \meta{numleft} argument to specify
the number of leading columns $n_l$ together with the total $n$ given by
\meta{num}, columns in the environment are laid out across two adjacent
pages.  In this {\em\Uidx\parapag{}e} typesetting, the first $n_l$ columns
are placed in the {\em left} page while remaining $n_r=n-n_l$ columns go to
the next {\em right} page.  The pair of left and right pages is
considered as comprising a virtual {\em\Uidx\paired} page and thus shares
a common page number, unless {\em\Uidx\npaired} typesetting is specified
by the optional `|*|' following the optional \meta{numleft} argument.  In
the \npaired{} \parapag{}ing, when the leading $n_l$ columns are put in a
page $p$, the trailing $n_r$ columns are in the page $p+1$.

如果\beginparacol{}的可选参数\meta{numleft}用于指定前导列的数量$n_l$,同时总列数由\meta{num}给出,那么环境中的列会跨两个相邻的页面进行布局。在这种{\em\Uidx\parapag{}e}排版中,前$n_l$列放置在{\em 左侧}页面,而剩下的$n_r=n-n_l$列放置在下一个{\em 右侧}页面。左侧和右侧页面的配对被认为是组成一个虚拟的{\em\Uidx\paired}页面,因此它们共享一个相同的页码,除非通过在可选的\meta{numleft}参数后面添加`|*|'来指定{\em\Uidx\npaired}排版。在\npaired{} \parapag{}ing中,当前导的$n_l$列放置在页面$p$上时,后续的$n_r$列会在页面$p+1$上。

\begin{itemize}
\item
All {\em\Uidx\pwstuff}, i.e., \Preenv{} and \postenv, \pwise{} floats,
\mctext{} and (\mgfnote{} or non-merged) \Scfnote{}s, are placed only in
left \parapag{}es leaving corresponding regions in right \parapag{}es
blank\footnote{

Someday the author could devise an advanced mechanism to exploit the space
in right \parapag{}es.}.

所有的{\em\Uidx\pwstuff},即\Preenv{}和\postenv,\pwise{}浮动体,\mctext{}和(\mgfnote{}或非合并的)\Scfnote{},只会放置在左侧\parapag{}es中,让右侧\parapag{}es中相应的区域保持空白\footnote{将来作者可能会设计一个高级机制来利用右侧\parapag{}es中的空间。}。
\item
A \npaired{} left \parapag{}e is not necessary to be even-numbered, though
the printing tradition requires so if you naturally want to have a
\parapag{}e pair in a double spread.  The page number given to the first
left \parapag{}e is simply the number of the page $p_1$ in which
\beginparacol{} reside, and that for the $k$-th left \parapag{}e is
$p_1+2(k-1)$\footnote{

Unless you make some change to \counter{page} counter.}.

Therefore, to make it sure $p_1$ is even, you might need to have an
ordinary page of blank, a title, etc., or to let \counter{page} counter have
an even number by \!\setcounter!, etc., before starting a \env{paracol}
environment.

一个没有成对出现的左页不一定是偶数页,尽管印刷传统要求如果你自然地希望在双页中有一个成对的页面。第一个左页的页码只是在\beginparacol{}所在的页$p_1$的页码,而第$k$个左页的页码是$p_1+2(k-1)$\footnote{除非你对\counter{page}计数器进行了一些更改。}。

因此,为了确保$p_1$是偶数,你可能需要在开始\env{paracol}环境之前有一个普通的空白页、一个标题等,或者通过 \!\setcounter! 等方法使\counter{page}计数器的值成为一个偶数。

\item
Section~\ref{sec:ppts} shows examples of \parapag{}ing together with
related issues on two-sided typesetting.

第~\ref{sec:ppts} 节展示了 \parapag{} 的示例,以及双面排版相关问题。
\end{itemize}
\end{description}
 
