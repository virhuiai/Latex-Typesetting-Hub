%% LaTeX file 'paracol-man'.
%% Copyright (C) 2005-2018
%%   Hiroshi Nakashima <h.nakashima@DOMAIN;  DOMAIN=media.kyoto-u.ac.jp>
%%   (Kyoto University)
%% This program can be redistributed and/or modified under the terms
%% of the LaTeX Project Public License distributed from CTAN
%% archives in directory macros/latex/base/lppl.txt; either
%% version 1 of the License, or any later version.

\ProvidesFile{paracol-man.tex}
[2018/12/31 v1.35 ]
\documentclass{ltxdoc}\normalmarginpar
\usepackage[heading=true
,scheme=chinese%中文方案
,fontset=none%不使用默认的字体设置
,space=auto%自动调整中英文间距
]{ctex}
\setCJKmainfont{FangZhengShuSong-GBK-1.ttf}[Path=/Users/virhuiai/hlProjects/Latex-Typesetting-Hub/font/方正/]%设置文本的中文有衬线字体
\setCJKsansfont{FangZhengHeiTi-GBK-1.ttf}[Path=/Users/virhuiai/hlProjects/Latex-Typesetting-Hub/font/方正/]%设置文本的中文无衬线字体为
\setCJKmonofont{FangZhengFangSong-GBK-1.ttf}[Path=/Users/virhuiai/hlProjects/Latex-Typesetting-Hub/font/方正/] %设置文本的中文等宽字体 
% \setCJKfamilyfont{fontKai}{LXGWWenKai-Regular.ttf}[Path=/Users/virhuiai/hlProjects/Latex-Typesetting-Hub/font/霞鹜文楷/]
\setCJKfamilyfont{fontKai}{FangZhengKaiTi-GBK-1.ttf}[Path=/Users/virhuiai/hlProjects/Latex-Typesetting-Hub/font/方正/]
\newcommand\fontKai{\CJKfamily{fontKai}}

\usepackage{color}
\usepackage{paracol}
\usepackage{newvbtm}
\DisableCrossrefs
\PageIndex
\CodelineNumbered
\RecordChanges
\OnlyDescription
\def\ONLYDESCRIPTION{}
\textwidth210mm
\advance\textwidth-40mm \oddsidemargin20mm \advance\oddsidemargin-1in
\columnsep10mm
\marginparwidth20mm \advance\marginparwidth-\marginparsep
\marginparsep.5\marginparsep
\raggedbottom
\OnlyDescription
\begin{document}
\DocInput{paracol.dtx}
\end{document}


% \columnratio{0.3,0.42,0.28}
\begin{paracol}{3}[\section{Introduction\hfill 介绍}]
\begin{VerbatimII}
\columnratio{0.3,0.42,0.28}
\begin{paracol}{3}[\section{Introduction\hfill 介绍}]
\begin{Verbatim}
左侧源码
\end{Verbatim}
\switchcolumn
This document..
\switchcolumn
本文档..
\switchcolumn[1]
Suppose ...
\switchcolumn
假设...
\end{paracol}    
\end{VerbatimII}
%%%%%%%%%%%%%%%%%%%%%%%%%%%%%%%%%%%%%%%%%%%%%%%%%%%%%%%
\switchcolumn
This document describes the usage of yet another multi-column package named
\textsf{paracol}.  The unique feature of the package is that columns are
typeset {\em in parallel.}
\switchcolumn
本文档介绍了另一个名为 \textsf{paracol} 的多栏排版宏包的使用方法。该宏包的独特特点是可以将栏以{\em 并行}的方式排版。 

\switchcolumn[1]
Suppose you are writing a bilingual document whose left column is written in
a language, say English, and right column has the translation of the left
column in another language, e.g., Japanese.  With the \textsf{paracol}
package you may write an English part of arbitrarily length and then {\em
switch} to its Japanese counterpart to place both parts side by side.  Of
course you may return to the English writing similarly.
\switchcolumn
假设你正在撰写一份双语文档,左栏使用一种语言(如英语),右栏则是左栏的另一种语言(如日语)的翻译。使用 \textsf{paracol} 宏包,你可以先写任意长度的英文部分,然后{\em 切换}到对应的日文部分,将两部分并排放置在一起。当然,你也可以类似地返回到英文撰写。

\switchcolumn[1]
The {\em\Uidx\cswitch} is always allowed when you complete an outermost
level paragraph.  You may be unaware whether a column is broken into
multiple pages before switching because the package automatically goes
back and forward to the correct page and vertical position when you switch
the column.  Moreover, you may {\em\Uidx\sync{}e} columns so that the tops
of the first paragraphs after switching in all columns are vertically
aligned.  At a \sync{}ation point, you may give a single-column text,
for example a common section header, optionally.  You may also switch
single-column and multi-column in a page arbitrary.
\switchcolumn
在外层段落完成后,总是允许使用 {\em\Uidx\cswitch} 命令。在切换之前,你可能不知道栏是否被分成多个页面,因为当你切换栏时,宏包会自动回到正确的页面和垂直位置。此外,你可以通过 {\em\Uidx\sync{}e} 命令来使列对齐,这样在切换后,所有列中第一个段落的顶部会垂直对齐。在 \sync{}ation 点,你可以选择给出单栏文本,例如一个公共的章节标题。你还可以随意在页面上切换单栏和多栏排版。

\switchcolumn[1]*
This manual itself is an example of two-column documents typeset by
\textsf{paracol}.  Since the author is not familiar with languages other
than English and Japanese and the latter should be hardly understood by
most of readers, the right column is the translation of the left English
column into a computational language.  That is, the right column is the
\LaTeX{} source code of the left column\footnote{%
Not really but its essence
is shown.\label{fn:first}}.
\switchcolumn
该手册本身是使用 \textsf{paracol} 排版的双栏文档的一个示例。由于作者对英语和日语以外的语言不熟悉,并且后者可能很难被大多数读者理解,所以右栏是左侧英文栏的计算语言翻译。也就是说,右栏是左栏的 \LaTeX{} 源代码\footnote{%
虽然不完全准确,但其本质得以展示。\label{fn:first}virhuiai在翻译时,做成了三栏,第一栏是源代码,第二栏是英文,第三栏是中文。}。
\end{paracol}


\end{document}

\begin{paracol}{2}

\switchcolumn



\switchcolumn*
This manual itself is an example of two-column documents typeset by
\textsf{paracol}.  
\switchcolumn
本手册本身就是使用 \textsf{paracol} 宏包排版的两栏文档的一个示例。

\end{paracol}
\begin{Verbatim}
\begin{paracol}{2}[\section{Introduction}]
\hbadness5000
en.....
\switchcolumn
中文....

\switchcolumn*
en.....
\switchcolumn
中文....
\switchcolumn*[\section{Basic Usage}]....
\end{paracol}
\end{Verbatim}
% 
\section{Basic Usage\hfill 基本用法}
\columnratio{0.2,0.48}
\begin{paracol}{3}
\begin{VerbatimII}
\section{Basic Usage\hfill 基本用法}
\columnratio{0.2,0.48}
\begin{paracol}{3}
\begin{Verbatim}
左侧源码
\end{Verbatim}
\switchcolumn
Loading..
\switchcolumn
加载...
\switchcolumn[1]*
The fundamental...
\switchcolumn
并列...
\end{paracol}
\end{VerbatimII}

\switchcolumn%%%%%%%%%%%%%%%%%%%%%%%%%%%%%%
Loading the package is very simple.  What you have to do is
\!\usepackage!|{|\Uidx{\env{paracol}}|}| in the preamble.  Note that
\textsf{paracol} can be used with \LaTeXe{} and does not work with
\LaTeX{} 2.09.
\switchcolumn
加载该宏包非常简单。在导言区使用 \!\usepackage!|{|\Uidx{\env{paracol}}|}| 命令即可。请注意,\textsf{paracol} 可以与 \LaTeXe{} 一起使用,不支持 \LaTeX{} 2.09。

\switchcolumn[1]
The fundamental means of parallel-column typesetting are the environment
\env{paracol} and the command \Uidx{\!\switchcolumn!}.  The \env{paracol}
environment needs an argument to specify the number of columns.  Thus the
following is the basic construct for two-parallel-column documents.
\switchcolumn
并列栏排版的基本手段是使用 \env{paracol} 环境和命令 \Uidx{\!\switchcolumn!}。\env{paracol} 环境需要一个参数来指定栏的数量。因此,以下是两栏并列文档的基本结构。

\switchcolumn[1]
\begin{quote}
\!\begin!|{|\env{paracol}|}{2}|\\
\textit{left column text}\\
\!\switchcolumn!\\
\textit{right column text}\\
\!\switchcolumn!\\
\textit{left column text}\\
\!\switchcolumn!\\
\textit{right column text}\\
\!\switchcolumn!\\
\mbox{\hspace{4em}}$\vdots$\\
\!\end!|{|\env{paracol}|}|
\end{quote}
\switchcolumn
\begin{quote}
\!\begin!|{|\env{paracol}|}{2}|\\
\textit{左栏文本}\\
\!\switchcolumn!\\
\textit{右栏文本}\\
\!\switchcolumn!\\
\textit{左栏文本}\\
\!\switchcolumn!\\
\textit{右栏文本}\\
\!\switchcolumn!\\
\mbox{\hspace{4em}}$\vdots$\\
\!\end!|{|\env{paracol}|}|
\end{quote}

\switchcolumn[1]*
The \!\switchcolumn! command may have an optional argument to specify the
column number (zero origin) to start.  That is, \!\switchcolumn!|[0]|
means to switch to the leftmost column, |\switchcolumn[1]| is to start the
second column and so on.  Thus the |\switchcolumn| without the optional
argument may be considered as \!\switchcolumn!|[|$i+1\bmod{n}$|]| where
$i$ is the ordinal of the column you are leaving from and $n$ is the
number of columns given to \env{paracol} environment.
\switchcolumn
\!\switchcolumn! 命令可以带有可选参数来指定从第几栏(从零开始计数)开始切换。也就是说,\!\switchcolumn!|[0]| 表示切换到最左边的栏,|\switchcolumn[1]| 表示从第二栏开始,依此类推。因此,不带可选参数的 |\switchcolumn| 可以视为 \!\switchcolumn!|[|$i+1\bmod{n}$|]|,其中 $i$ 是你离开的栏的序号,$n$ 是给定给 \env{paracol} 环境的栏数。

\end{paracol}
% \section{Column Synchronization\\栏同步}\label{sec:sync}

\columnratio{0.3,0.42,0.28}
\begin{paracol}{3}
%%%%%%%%%%%%%%%%%%%%%%%%%%%%%%%%%%%%%%%%%%%%%%%%%%%%%%%
\begin{VerbatimII}
\columnratio{0.3,0.42,0.28}
\begin{paracol}{3}
第1栏
\switchcolumn
第2栏
\switchcolumn
第3栏

\switchcolumn[0]*
同步 ...
\switchcolumn
...
\end{paracol}  
\end{VerbatimII}
%%%%%%%%%%%%%%%%%%%%%%%%%%%%%%%%%%%%%%%%%%%%%%%%%%%%%%%

\switchcolumn
The \!\switchcolumn! command may also be followed by a `|*|' to
{\em\Uidx\sync{}e} columns.  After you switch from a column to another by
\!\switchcolumn!|*| (or \!\switchcolumn!|[|$i$|]*|), all the columns are
vertically aligned at the bottom of the {\em deepest} one preceding the
command.  For example, the previous section has three \!\switchcolumn!|*|
commands at which left and right columns are vertically aligned.
\switchcolumn
\!\switchcolumn! 命令后面可以加上 `|*|',用来{\em 同步}栏。当你使用 \!\switchcolumn!|*|(或 \!\switchcolumn!|[|$i$|]*|)从一栏切换到另一栏时,所有栏都会垂直对齐在该命令之前最{\em 深}的栏的底部。例如,前一节使用了三个 \!\switchcolumn!|*| 命令,使左右两栏垂直对齐。

\switchcolumn[1]
The {\em starred} version of \!\switchcolumn! may have an optional
argument to specify a single-column {\em\Uidx\mctext} whose bottom is the
vertical alignment point of columns.  For example, \!\section!
commands in this manual are given as optional arguments
of \!\switchcolumn!|*| like;
\switchcolumn
{\em 带星号}版本的 \!\switchcolumn! 命令可以带有可选参数,用来指定一个单栏的{\em 同步文本},其底部作为栏的垂直对齐点。例如,本手册中的 \!\section! 命令作为 \!\switchcolumn!|*| 的可选参数给出,如下所示:

\switchcolumn[1]
\begin{Verbatim}
\switchcolumn*[\section{Basic Usage}]
\end{Verbatim}
\switchcolumn
\begin{Verbatim}
\switchcolumn*[\section{基础用法}]
\end{Verbatim}
\switchcolumn[1]
The \env{paracol} environment may also start with a \mctext{} by
specifying it as the optional argument of \!\begin!|{|\env{paracol}|}|.
For example, at the beginning of this document, the author put;
\switchcolumn
\env{paracol} 环境也可以以一个 \mctext{} 开始,将其指定为 \!\begin!|{|\env{paracol}|}| 的可选参数。例如,在本文档的开头,作者使用了以下代码:

\switchcolumn[1]*
\begin{Verbatim}
\begin{paracol}{2}[\section{Introduction}]
\end{Verbatim}

\switchcolumn
\begin{Verbatim}
\begin{paracol}{2}[\section{介绍}]
\end{Verbatim}

\end{paracol}    
\end{document}%%%%%%