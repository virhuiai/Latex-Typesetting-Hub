%%%%%%%%%%%%%%%{fontspec}%%%%%%%%%%%%%%%
%将no-math选项传递给fontspec宏包,该选项禁用了使用fontspec宏包中的数学字体功能。
\PassOptionsToPackage{no-math}{fontspec}
%%%%%%%%%%%%%%%{xeCJK}%%%%%%%%%%%%%%%
%将AutoFakeBold和AutoFakeSlant选项传递给xeCJK宏包,这两个选项让xeCJK宏包自动产生伪粗体和伪斜体效果。
\PassOptionsToPackage{AutoFakeBold=true,AutoFakeSlant=true}{xeCJK}

\documentclass{article}
\usepackage[
fontset=none%不使用默认的字体设置
,space=auto%自动调整中英文间距
]{ctex}
\setCJKmainfont{FangZhengShuSong-GBK-1.ttf}[Path=/Users/virhuiai/hlProjects/Latex-Typesetting-Hub/font/方正/]%设置文本的中文有衬线字体
\setCJKsansfont{FangZhengHeiTi-GBK-1.ttf}[Path=/Users/virhuiai/hlProjects/Latex-Typesetting-Hub/font/方正/]%设置文本的中文无衬线字体为
\setCJKmonofont{FangZhengFangSong-GBK-1.ttf}[Path=/Users/virhuiai/hlProjects/Latex-Typesetting-Hub/font/方正/] %设置文本的中文等宽字体 
\usepackage{xpinyin}

\begin{document}


\begin{pinyinscope}
    \section*{好吃的橘子}

    

    缤纷的水果世界,有很多好吃的水果,西瓜、火龙果、桃子、%
    木瓜。但是我最喜欢的还是橘子。
    
    现在正是吃橘子的季节。橘子圆圆的,差不多乒乓球大小,金黄%
    金黄的,仿佛穿上了一件金黄的衣服,从远处看,还以为是一个个%
    小灯笼呢。
    
    记得有一次,爸爸带回了一大箱冰糖橘,我迫不及待地拿出一%
    个,撕开皮,皮很薄,软软的,凉凉的,刚一撕开,一股浓香马上%
    弥漫开来。
    
    橘子有大约十瓣,``果肉大家庭''紧紧挤在一起,好像小兄弟们围%
    在一起说着什么悄悄话,我掰下一瓣,弯弯的,像半轮月亮,太美%
    了!
    
    我轻轻地咬一口,汁水竟然溅了我一手,尝一尝,果然像冰糖一%
    样甜!我连着吃了十几个冰糖橘子!爸爸说:``不可以再吃了!冰%
    糖橘子吃多了,会上火的。''我只得停下来,无奈地看着。
    
    橘子富含多种维生素,如B2,维C,可以美容养颜,预防癌症。它%
    还含有钙、磷、铁等微量元素,有益于健康。
    
    吃冰糖橘子,让我感受到了生活的甜蜜!

    \section*{五感法写桔子}    

    \begin{description}
        \item[视觉]妈妈买回一箱砂糖橘,一个个扁圆的桔子好像一盏盏橙色的%
        小灯笼,颜色新鲜美丽。
        \item[触觉]我迫不及待伸手拿起一个仔细观察,橘皮表面有许多小小的
        凸起,用手一摸疙疙瘩瘩的,非常粗糙。
        \item[嗅觉]我掰开橘皮,顿时一股清香扑鼻而来。
        \item[触觉+味觉]橘皮里面,十个软软的小橘瓣像十个可爱的小月亮整%
        齐得围坐一圈。我剥下几瓣放到嘴里,香甜的汁水溢满口腔。
        \item[语言描写]``妈妈这次买的桔子超好吃,大家快来品尝!''我连忙%
        招呼爸妈和妹妹。``哇,桔瓣上白色的丝线是什么?''妹妹好奇地%
        问。``这是橘络,吃了可以止咳顺气的,放心吃吧!''听,还是老%
        爸最有学问。大家一边吃一边议论,欢声笑语充满了这个温馨的%
        家。
    \end{description}
\end{pinyinscope}






\end{document}