% \section{Two-Sided Typesetting and Parallel-Paging\hfill 双面排版和并列分页}
% \label{sec:ppts}
% \changes{v1.3-2}{2013/09/17}
%	{Add the section ``Two-Sided Typesetting and Parallel-Paging''.}
% \changes{v1.3-4}{2013/09/17}
%	{Add the section ``Two-Sided Typesetting and Parallel-Paging''.}
% \changes{v1.3-5}{2013/09/17}
%	{Add the section ``Two-Sided Typesetting and Parallel-Paging''.}
% 
% This and the next section are typeset with \Uidx{\!\twosided!} enabling
% features |p|, |c| and |m| and also |b| for a part of the next section.
% The effect of |p| feature can be seen by the \oddeven{left}{right}, or in
% other word inside, margin of this \oddeven{odd}{even}-numbered page is
% narrower than that of the previous pages because the author reduced the
% effective \oddeven{left}{right} side margin being calculated from
% \oddeven{\cs{oddsidemargin}}{\cs{evesidemargin}}

% 这一节和下一节使用 \Uidx{\!\twosided!} 启用特性|p|、|c|和|m|,以及部分下一节的特性|b|进行排版。通过查看此\oddeven{奇数}{偶数}页的内边距,可以看到|p|特性的效果,即比前面的页面的内边距更窄,因为作者减小了从 \oddeven{\cs{oddsidemargin}}{\cs{evesidemargin}} 计算出的有效 \oddeven{left}{right} 边距。

% 
% \SpecialIndex{\oddsidemargin}
% \SpecialIndex{\evensidemargin}
% 
% by 75\,\%\footnote{
% This document itself does not have |twoside| option in its
% \!\documentclass! but the inconsistency between the option and
% \!\twosided! is not visible because \!\pagestyle! is |plain|.}.
% This setting makes the \oddeven{right}{left} side or outside margin of
% this page enlarged by 125\,\%, as well as the \oddeven{left}{right} side
% and outside margin of the next \oddeven{even}{odd}-numbered page specified
% by \oddeven{\cs{evensidemargin}}{\cs{oddsidemargin}}.

% 由于75,%的设置\footnote{%
% 此文档本身在其 \!\documentclass! 中没有 |twoside| 选项,但是选项与 \!\twosided! 之间的不一致之处并不可见,因为 \!\pagestyle! 是 |plain|。}。
% 这个设置使得本页的\oddeven{右}{左}边缘或外侧边缘增大了125,%,以及下一页的\oddeven{左}{右}边缘和外侧边缘,由 \oddeven{\cs{evensidemargin}}{\cs{oddsidemargin}} 指定的\oddeven{偶数}{奇数}页。

% Next, we see the effects of |c| and |m| features by the \env{paracol}
% environment below for which \Uidx{\!\columnratio!}|{0.6}| and
% \Uidx{\!\marginparthreshold!}|{0}| are declared to make the \emph{inside}
% columns (\oddeven{left}{right} ones in \oddeven{odd}{even}-numbered pages)
% are wider than the \emph{outside} ones and all marginal notes go to
% outside (\oddeven{right}{left} in \oddeven{odd}{even}-numbered pages)
% margins.

% 接下来,我们看到以下 \env{paracol} 环境通过 \Uidx{\!\columnratio!}|{0.6}| 和  \Uidx{\!\marginparthreshold!}|{0}| 来实现 |c| 和 |m| 特性的效果,使得\emph{内部}列(在\oddeven{奇数}{偶数}页中的\oddeven{左}{右}列)比\emph{外部}列更宽,并且所有的边注都放在外侧边缘(在\oddeven{奇数}{偶数}页中的\oddeven{右}{左}边缘)。

% \columnratio{0.6}\marginparthreshold{0}
% 
% \par\Hrule
% \begin{paracol}{2}
% \switchcolumn
% \footnotetext*{Since the author is temporarily disabling the warning from
% marginal note placement mechanism of \LaTeX, pushing down the second
% marginal note from column-1 is silently performed when you process this
% document.}
% \switchcolumn
% This line\Marginpar{First marginal note from column-0.} of the first
% paragraph of the inside column-0 has a marginal note.  Now the author puts
% a few dummy lines to keep a space below the marginal note.\\
% \Dotfill\\ \Dotfill\\ \Dotfill\\ \Dotfill\\ \Dotfill\\ \Dotfill\\
% \Dotfill\\ \Dotfill\\ \Dotfill\\ \Dotfill\\ \Dotfill\\ \Dotfill\par
% 
% This line\Marginpar{Second marginal note from column-0.} of the second
% paragraph of the inside column-0 also has a marginal note.  Now the author
% puts a few dummy lines again but this time to go down to the bottom of the
% page.\\
% \Dotfill\\ \Dotfill\\ \Dotfill\\ \Dotfill\\ \Dotfill\\ \Dotfill\\
% \Dotfill\\ \Dotfill\\ \Dotfill\\ \Dotfill\\ \Dotfill\\ \Dotfill\\
% \Dotfill\\ \Dotfill\par
% 
% This is the third paragraph of the inside column-0 having a page break in
% it.  Since shortly we will be in an \oddeven{even}{odd}-numbered page
% \pageref{page:ppts2} (now), this wider column\Marginpar{Third marginal
% note from column-0} is now \oddeven{right}{left} one keeping it
% inside, while the marginal note given in the first line of this page goes
% to \oddeven{left}{right} and outside.  Now we will have a \!\switchcolumn!
% below this paragraph to go to the column-1 and back to the previous page
% \pageref{sec:ppts}.\label{page:ppts2}
% \switchcolumn
% \it
% This is the first paragraph in the narrower, italicized and outside
% column-1.  In this paragraph, we shortly have a marginal note, italicized
% too, which goes to the outside margin shared by all marginal notes from
% both columns.\Marginpar{\it First marginal note from column-1.}  The
% marginal note given here is placed its natural position and its first line
% is aligned to the first line of the second sentence of this paragraph by
% exploitation of the space between two marginal notes from the column-0,
% though we already have had three notes from the column.
% 
% Now\Marginpar{\it Second marginal note from column-1.} the author puts
% another marginal note whose first line would be aligned to that of this
% paragraph, but it is pushed down below the second marginal note from the
% column-0 because two notes conflict with each other over the
% space\footnotemark*[+0].  Note that since the note from this column is given
% \emph{after} that from the column-0 was given, the conflict is solved
% pushing the note from this column down rather than that from the
% column-0.  Now the author puts a few dummy lines to go to the second last
% line of this page.\\
% \Dotfill\\ \Dotfill\\ \Dotfill\\ \Dotfill\\ \Dotfill\\
% \Dotfill\par
% 
% This is the third paragraph of the outside column-1, which becomes
% \oddeven{left}{right} shortly by the page break.\Marginpar{\it Third
% marginal note from column-1.}  The third marginal note is given in the
% first line of this page, but it is pushed down again due to the conflict
% with the note from the column-0.
% \end{paracol}
% \Hrule
% 
% Note that the position of the last marginal note in the \env{paracol}
% \Marginpar{Marginal note given after \env{paracol} environment is closed.}
% environment which we just have closed affects the marginal note placement
% in \postenv.  For example, the marginal note given in the first line of
% this paragraph is pushed down.

% 请注意,在我们刚刚关闭的\env{paracol}环境中,最后一个边注的位置会影响\postenv 中的边注位置。例如,给出在本段落第一行的边注会被推下去。

% \ifodd\value{page}
% We will see a few examples of \parapag{}ing shortly, but before that we
% will have an intentional black page to make the first page of the example
% odd-numbered to avoid you have an impression that its layout is
% incorrect\footnote{%
% At least the author himself had such impression without the blank page.}
% because if it were in an even page you would see the {\em outside\/} third
% and fourth supplementary {\em columns\/} at first.

% \ifodd\value{page}
% 不久我们将看到一些\parapag{}的例子,但在此之前,我们将有一个有意留白的页面,使示例的第一页成为奇数页,以避免给您一种布局错误的印象\footnote{至少在没有空白页面的情况下,作者本人也有这样的印象。},因为如果它在偶数页,您将首先看到第三和第四个辅助{\em 列}的{\em 外侧}。

% \newpage\vspace*{\fill}\centerline{(intentionally blanked page)}\vfill
% 
% \else
% From the next page, we will see a few examples of \parapag{}ing.

从下一页开始,我们将看到一些\parapag{}的例子。
% \fi
% 
