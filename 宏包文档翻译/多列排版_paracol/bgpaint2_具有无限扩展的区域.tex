% \newpage \suppressfloats
% \nobackgroundcolor{t}
% \nobackgroundcolor{b}
% \nobackgroundcolor{l}
% \nobackgroundcolor{r}
% \nobackgroundcolor{g}
% \backgroundcolor{c[0](4pt,4pt)(0.5\columnsep,4pt)}[rgb]{1,0.8,1}
% \backgroundcolor{c[1](0.5\columnsep,4pt)(4pt,4pt)}[rgb]{1,1,0.8}
% \backgroundcolor{C[0](10000pt,10000pt)(0.5\columnsep,10000pt)}[rgb]{1,0.8,1}
% \backgroundcolor{C[1](0.5\columnsep,10000pt)(10000pt,10000pt)}[rgb]{1,1,0.8}
% 
% \subsection{Regions with Infinite Extensions\hfill 具有无限扩展的区域}
% \label{sec:bgpaint-inf}
% 
% You are now seeing another \bgpaint{} much different from previous two
% examples.  That is, after disabling painting of |t|, |b|, |l|, |r| and |g|
% regions by \Uidx{\!\nobackgroundcolor!}, the author gave the followings
% for painting this and the next pages.

% 现在你看到了另一个与前两个示例非常不同的\bgpaint{}。也就是说,在通过 \Uidx{\!\nobackgroundcolor!} 禁用|t|、|b|、|l|、|r|和|g|区域的绘制之后,作者为绘制本页和下一页给出了以下设置。

% \begin{itemize}\item[]
% |\backgroundcolor|
%     |{c[0](4pt,4pt)(0.5\columnsep,4pt)}[rgb]{1,0.8,1}|\\
% |\backgroundcolor|
%     |{c[1](0.5\columnsep,4pt)(4pt,4pt)}[rgb]{1,1,0.8}|\\
% |\backgroundcolor|
%     |{C[0](10000pt,10000pt)(0.5\columnsep,10000pt)}[rgb]{1,0.8,1}|\\
% |\backgroundcolor|
%     |{C[1](0.5\columnsep,10000pt)(10000pt,10000pt)}[rgb]{1,1,0.8}|
% \end{itemize}
% 
% \SpecialUsageIndex{\backgroundcolor}

% The first two lines above is different from the previous declaration
% because inside edges of |c[0]| and |c[1]| regions are shifted toward
% outside of them and thus inside of unpainted |g| region so that the edges
% are contacted.  On the other hand, the last two lines are for
% \emph{under-painting} of columns and has \emph{\bginfext} to make top,
% bottom and outside edges of |C| regions reaching to the corresponding
% paper edges.  Since this under-painting is done with colors same as those
% of over-painting of |c| regions, you will have an impression that the
% paper is two-toned and \pwstuff{} are pasted on the paper\footnote{%
% This footnote is given outside \env{paracol} environment but its
% \bground{} is painted by light purple because it is merged with the
% footnote \ref{fn:bgpaint-inf2}.\label{fn:bgpaint-inf1}}.

% 上面的前两行与之前的声明不同,因为|c[0]|和|c[1]|区域的内侧边缘向外移动,进入未绘制的|g|区域,使边缘相接触。另一方面,最后两行是用于对列进行\emph{下层绘制},并且具有\emph{\bginfext},使|C|区域的顶部、底部和外部边缘达到相应的纸张边缘。由于此下层绘制使用的颜色与|c|区域的上层绘制相同,所以你会有一种纸张是双色的,并且\pwstuff{}被粘贴在纸张上的印象\footnote{%
% 这个脚注是在\env{paracol}环境之外给出的,但是它的\bground{}被浅紫色绘制,因为它与脚注\ref{fn:bgpaint-inf2}合并了。\label{fn:bgpaint-inf1}}。
% \par\bigskip
% 
% \begin{figure}\nosv
% \def\arraystretch{0.8}
% \centerline{\begin{tabular}[b]{|c|}\hline
%     \hbox to.9\textwidth{}\\
%     \parbox{.8\textwidth}{
% 	This \texttt{f}(loat) region could be extended to both side edges
%	and the top edge of the paper if its extension were
%	\texttt{(10000pt,10000pt)(10000pt,-4pt)}.}\\
%     \\\hline
%     \end{tabular}}
% \caption{A Page-Wise Figure \emph{Imported} from Pre-Environment}
% \label{fig:bgpaint-inf}
% \end{figure}

% \begin{paracol}{2}
% Though you cannot see, the right edge of this over-painted |c[0]| region
% is shifted right by 4\,|pt| to hide the small patch at the right bottom
% corner of the |p| region above by overlaying.
% 
% \switchcolumn
% \begingroup\it
% As explained in the right column, this {\rm|c[1]|} region also has an
% invisible left edge shifted left by {\rm4\,|pt|}\footnote{
% 
% This (foot)|n|(ote) region could be extended to both side edges and the
% bottom edge of the paper if its extension were
% \texttt{(10000pt,-4pt)(10000pt,10000pt)}.\label{fn:bgpaint-inf2}}.
% \endgroup
% 
% \switchcolumn*[\subsection*{This \texttt{s}(panning text) region could be
% extended to both side edges of the paper if its extension were
% \texttt{(10000pt,-4pt)}.}\par\medskip]
% 
% The author does not have much to say now for this column chunk.
% \par\vfill
% 
% Still nothing to say particular to the page break we will have shortly.
% \par\newpage
% 
% This paragraph is just for keeping the \env{paracol} environment alive in
% this page.
% \switchcolumn
% 
% \begingroup\it
% Little to say as well.
% \par\vfill
% 
% Nothing to say as well.
% \par\newpage
% 
% This paragraph is not necessary for keeping alive the environment but is
% given for consistent view.
% \endgroup
% 
% \begin{figure*}\nosv
% \def\arraystretch{0.8}
% \centerline{\begin{tabular}[b]{|c|}\hline
%     \hbox to.9\textwidth{}\\
%     \parbox{.8\textwidth}{
% 	This figure is given in the \env{paracol} environment closed in the
%	previous page but its background is not painted.}\\
%     \\\hline
%     \end{tabular}}
% \caption{A Page-Wise Figure \emph{Exported} to Post-Environment}
% \label{fig:bgpaint-inf2}
% \end{figure*}
% \end{paracol}
% \bigskip
% 
% Note that overlay painting is inevitable for two-toned page painting, as
% far as you want to paint \bground{} of \pwstuff.
%
请注意,如果您希望绘制\pwstuff{}的\bground{},那么对于双色页面绘制,覆盖绘制是不可避免的。

% The last issue of \bgpaint{} is about painting materials given outside
% \env{paracol}.  As you have seen, \Preenv{} and \postenv{} are painted but
% it is done only when they reside in a page having a portion of a
% \env{paracol} environment (maybe) of course.  Therefore, the next page is
% \emph{not} painted because the page does not have any parallel-columned
% stuff.  Therefore, even if you wish to paint the whole of your document
% including pages without \env{paracol} stuff, you cannot do it just with
% \Paracol{} package, at least so far.

\bgpaint{}的最后一个问题是关于在\env{paracol}之外给出的材料的绘制。正如您所见,\Preenv{}和\postenv{}是被绘制的,但只有当它们位于具有\env{paracol}环境(可能)的页面中时才进行绘制。因此,下一页\emph{不会}被绘制,因为该页没有任何平行列的内容。因此,即使您希望绘制整个文档,包括没有\env{paracol}内容的页面,至少目前您无法仅使用\Paracol{}宏包来实现。
% 
% On the other hand, some materials given outside \env{paracol} environments
% are painted as if they are given in the environment when they are
% \emph{imported} into the environment.  One category has footnotes given in
% \preenv{} when \!\footnotelayout!|{m}| is specified for merging, as
% exemplified by the footnote \ref{fn:bgpaint-inf1} in the previous page.
% Note that such a footnote is painted by the color for |n| region rather
% than |p| region even when there are no footnotes in the \env{paracol}
% environment.  The other category has ordinary floats given by \env{figure}
% and/or \env{table}
% (i.e., neither \env{figure*} nor \env{table*}) environments outside
% \env{paracol} and then \emph{deferred} to a page having (a portion of)
% stuff produced by \env{paracol}.  Since such a float, e.g.,
% Figure\Tie\ref{fig:bgpaint-inf} in this page, is considered as a page-wise
% float given in the \env{paracol} environment in this section, its
% background is painted by the color for the |f| region, rather than that
% for the |p| region which would be used if the float were is placed in the
% previous page.  Note that such a deferred float import could occur not
% only from the page having \beginparacol{} but also from pages preceding
% it.  For example, if you have three \env{figure} environments in a page
% $p-1$ just preceding the page $p$ in which you start a \env{paracol}
% environment, it could happen that first one is placed in $p-1$ without
% painting, the second is placed in $p$ and painted by the color for |p|,
% and the third is placed in $p+1$ and painted by the color for |f|.

另一方面,一些在\env{paracol}环境之外给出的材料在被\emph{导入}到环境中时会被绘制,就好像它们是在环境中给出的一样。一个类别是在\preenv{}中给出的,在指定 \!\footnotelayout!|{m}|进行合并时的脚注,例如前一页的脚注\ref{fn:bgpaint-inf1}。请注意,即使在\env{paracol}环境中没有脚注,这样的脚注也会使用|n|区域的颜色而不是|p|区域的颜色进行绘制。另一类是由\env{figure}和/或\env{table}(即既不是\env{figure*}也不是\env{table*})环境给出的普通浮动体,然后被\emph{延迟}到由\env{paracol}产生的(部分)内容的页面。因为这样的浮动体,例如本页的Figure\Tie\ref{fig:bgpaint-inf},被认为是在本节的\env{paracol}环境中给出的整页浮动体,所以它的背景会使用|f|区域的颜色进行绘制,而不是如果该浮动体放在前一页上时将使用|p|区域的颜色。请注意,这样的延迟浮动体导入不仅可能来自具有\beginparacol{}的页面,也可能来自之前的页面。例如,如果在您开始一个\env{paracol}环境的页面$p$的前一页$p-1$中有三个\env{figure}环境,可能发生以下情况:第一个放置在$p-1$中而没有绘制,第二个放置在$p$中并使用|p|的颜色进行绘制,第三个放置在$p+1$中并使用|f|的颜色进行绘制。

% 
% Finally some materials \emph{exported} from a \env{paracol} environment
% are painted as if they are in \postenv.  In previous two subsections, we
% saw \Mgfnote{}s (e.g., \ref{fn:bgpaint1} in p.\Tie\pageref{fn:bgpaint1}
% and \ref{fn:bgpaint-me1} in p.\Tie\pageref{fn:bgpaint-me1}) are painted by
% the color of |p| rather than |n|.  The other kind of exportation is of
% page-wise floats given in a \env{paracol} environment but deferred to the
% page next to the page having \Endparacol, or further.  For example,
% Figure~\ref{fig:bgpaint-inf2} is given in the \env{paracol} environment
% above in this page, but its \bground{} is not painted because the next page
% in which the figure is placed does not have any parallel-columned
% stuff\footnote{%
% If it has, the background is painted by the color for |p|.}.

最后,一些从\env{paracol}环境中\emph{导出}的材料被绘制,就好像它们在\postenv{}中一样。在前两个小节中,我们看到\Mgfnote{}(例如,\pageref{fn:bgpaint1}页上的\ref{fn:bgpaint1}和\pageref{fn:bgpaint-me1}页上的\ref{fn:bgpaint-me1})被绘制为|p|区域的颜色,而不是|n|。另一种导出的类型是在\env{paracol}环境中给出的整页浮动体,但是延迟到\Endparacol{}所在页面的下一页或更后面的页面。例如,本页上方的\env{paracol}环境中给出了Figure~\ref{fig:bgpaint-inf2},但是它的\bground{}没有被绘制,因为放置该图的下一页没有任何平行列的内容\footnote{如果有,背景将使用|p|区域的颜色进行绘制。}。

% \newpage\vspace*{\fill}
% \centerline{(intentionally blanked page to show this page is \emph{not}
% painted)}
% \vfill
% \advance\skip\footins-4pt\relax