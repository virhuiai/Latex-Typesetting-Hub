% \def\Dotquad{\leavevmode\cleaders\hbox to.44em{\hss.\hss}%
%   \hskip\parindent\kern0pt}
% \def\fnpar#1#2{#1 paragraph\Dotfill\\\Dotfill with a footnote#2\Dotfill
%   in it.\par}
% \def\Fnpar#1#2#3{#1 paragraph\Dotfill\\\Dotfill with a footnote#2\Dotfill
%   in it.\\#3\par}
% 



% 
% \subsection{Multiple \cs{switchcolumn} in a Page\hfill 页面中的多个\cs{switchcolumn}}
% \label{sec:fnnp-multsc}
% 
% Here we have an example with three \!\switchcolumn! commands in a page
% having six footnotes.  Hereafter, footnotes are typeset with
% \Uidx{\!\footnotelayout!}|{m}|\footnote{And thus this footnote is merged with those in the \env{paracol}
% environment.}.

下面是一个在页面中使用了三个 \!\switchcolumn! 命令的示例,其中包含六个脚注。在此之后,使用 \Uidx{\!\footnotelayout!}|{m}|\footnote{这个脚注与\env{paracol}环境中的脚注合并了。}设置脚注样式。
% 
% \footnotelayout{m}
% \Hrule
% \begin{paracol}{2}
% \fnpar{First left-column 左列的第一个}{\footnote{First left-column
% footnote.左列的第一个脚注。 \label{fn:2L1}}}
% \Fnpar{Second left-column 左列的第二个}{\footnote{Second left-column
% footnote. 左列的第二个脚注。\label{fn:2L2}}}{
% It is followed by a \cs{switchcolumn}. 它后面跟着一个\cs{switchcolumn}。}
% \switchcolumn
% \Fnpar{First right-column 右列的第一个}{\footnote{First right-column footnote but
% following the second left-column one. 右列的第一个脚注,但是在第二个左列脚注之后。\label{fn:2R1}}}{It is followed by a
% \cs{switchcolumn*}. 它后面跟着一个\cs{switchcolumn*}。}
% \switchcolumn*
% \Fnpar{Third and synchronized left-column 左列的第三个(同步的)}{\footnote{Third left-column
% footnote but following the first right-column one. 左列的第三个脚注,但是在第一个右列脚注之后。\label{fn:2L3}}}{It is
% followed by a \cs{switchcolumn}. 它后面跟着一个\cs{switchcolumn}。}
% \switchcolumn
% \fnpar{Second and synchronized right-column 右列的第二个(同步的)}{\footnote{Second right-column
% footnote but following the third left-column one. 右列的第二个脚注,但是在第三个左列脚注之后。\label{fn:2R2}}}
% \fnpar{Third right-column 右列的第三个}{\footnote{Third right-column
% footnote.\label{fn:2R3} 右列的第三个脚注。}}
% \end{paracol}
% \newpage
% 
% The example in the previous page should look weird because the order of
% the third footnote in the left column \ref{fn:2L3} and the first in the
% right \ref{fn:2R1} are reversed in their numbers and in the stack at the
% page bottom.  However, the result is \emph{natural} because they are
% numbered and stacked in the order of occurrence in the source |.tex| as
% always done in any documents without \textsf{paracol} and with it but
% \mcfnote{} footnote typesetting.  Since the \textsf{paracol} cannot
% maintain the order automatically\footnote{
% 
% So far, because the maintenance is extremely tough.  But since it is not
% impossible, some day you could have an improved version of
% \textsf{paracol} with the automatic ordering.},

在上一页的示例中,看起来有些奇怪,因为左列中的第三个脚注 \ref{fn:2L3} 和右列中的第一个脚注 \ref{fn:2R1} 在它们的编号和页面底部的堆栈中的顺序上是颠倒的。然而,这个结果是“自然”的,因为它们按照在源 |.tex| 中出现的顺序进行编号和堆叠,这是任何没有使用 \textsf{paracol} 或使用了 \mcfnote{} 脚注排版的文档中都会这样做的。由于 \textsf{paracol} 无法自动维护顺序\footnote{至今为止,因为维护顺序非常困难。但是,既然不是不可能,总有一天您可能会有一个改进版的 \textsf{paracol},具有自动排序功能。},

% you have to maintain it by yourself.

% 你需要自己维护这个问题。

% The problem is partly solved by using \!\footnote! with its optional
% argument \oarg{num} to number the first right-column and the third
% left-column footnotes explicitly, i.e., to give
% $\mathit{num}=\ref{fn:2L3}$ to the former and $\mathit{num}=\ref{fn:2R1}$
% to the latter.  One caution is that you have to remember that \!\footnote!
% with the optional \meta{num} does not update \counter{footnote} counter
% and thus you have to do
% \!\setcounter!|{footnote}{|\texttt{\ref{fn:2L3}}|}| or
% \!\addtocounter!|{footnote}{2}| after the third left-column footnote.

% 部分解决这个问题的方法是使用带有可选参数\oarg{num}的 \!\footnote! 命令,来显式地对第一个右列和第三个左列的脚注进行编号,即给前者赋值$\mathit{num}=\ref{fn:2L3}$,给后者赋值$\mathit{num}=\ref{fn:2R1}$。需要注意的是,你必须记住,带有可选参数\meta{num}的 \!\footnote! 命令不会更新\counter{footnote}计数器,因此你需要在第三个左列脚注之后使用 \!\setcounter!|{footnote}{|\texttt{\ref{fn:2L3}}|}| 或 \!\addtocounter!|{footnote}{2}|。
% 
% This remedy, however, cannot change the stacking order of these two
% footnotes of course.  Therefore, you need another trick with
% \!\footnotemark! and \!\footnotetext! to stack the third left-column
% footnote above the first right-column one.  More specifically, you can
% solve the problem inserting

然而,这种方法当然无法改变这两个脚注的堆叠顺序。因此,您需要使用 \!\footnotemark! 和  \!\footnotetext! 来将第三个左列脚注堆叠在第一个右列脚注上面。具体来说,您可以通过插入以下内容来解决这个问题:
% \begin{quote}
% \!\footnotetext!\texttt{[\ref{fn:2R1}]}
%     |{|\textit{text for the third left footnote}|}|
% \end{quote}

% somewhere between \!\footnote! commands for the second left-column and the
% first right-column ones, e.g., at the end of the second left-column
% paragraph, and attaching its mark to the appropriate word for the footnote
% by \!\footnotemark!\texttt{[\ref{fn:2R1}]}, to have the following.
% 
在第二个左列的 \!\footnote! 命令和第一个右列的 \!\footnote! 命令之间的某个位置,例如在第二个左列段落的末尾,并通过  \!\footnotemark!\texttt{[\ref{fn:2R1}]} 将其标记附加到脚注所对应的单词上,可以得到以下效果。
% \Hrule
% \begin{paracol}{2}
% \fnpar{First left-column}{\footnote{First left-column
% footnote.\label{fn:3L1}}}
% \Fnpar{Second left-column}{\footnote{Second left-column
% footnote.\label{fn:3L2}}}{
% It is followed by \cs{footnotetext}\texttt{[\ref{fn:3L3}]}\marg{text}
% and a \cs{switchcolumn}.}
% \addtocounter{footnote}{1}
% \footnotetext[\arabic{footnote}]{Third left-column footnote given by
% \cs{footnotetext}\texttt{[\ref{fn:3L3}]}\marg{text} placed at the end of
% the second left-column paragraph.\label{fn:3L3}}
% \switchcolumn
% \addtocounter{footnote}{1}
% \Fnpar{First right-column}{\footnote[\arabic{footnote}]{First right-column
% footnote whose number \ref{fn:3R1} is explicitly given by
% \cs{footnote}\texttt{[\ref{fn:3R1}]}\marg{text}.\label{fn:3R1}}}{It is
% followed by a \cs{switchcolumn*}.}
% \addtocounter{footnote}{-1}
% \switchcolumn*
% Third and synchroized left-column paragraph\Dotfill\\
% \Dotfill with a footnote whose mark
% here\footnotemark[\arabic{footnote}]\Dotfill\\
% \Dotfill is given by \!\footnotemark!\texttt{[\ref{fn:3L3}]}\Dotfill in
% it.\\
% It is followed by \!\addtocounter!|{footnote}{2}| and a \!\switchcolumn!.
% \addtocounter{footnote}{1}
% \switchcolumn
% \fnpar{Second and synchronized right-column}{\footnote{Second right-column
% footnote correctly following the first right-column one.\label{fn:3R2}}}
% \fnpar{Third right-column}{\footnote{Third right-column
% footnote.\label{fn:3R3}}}
% \end{paracol}
% \Hrule
% 
% Though this solution gives a good result, however, it has the following
% two problems.  First, you have to explicitly specify the footnote number
% through the optional arguments \oarg{num} of \!\footnote!,
% \!\footnotetext! and \!\footnotemark!.  This problem is quite severe
% because, for example, if you add a footnote somewhere preceding the
% \env{paracol} environment in question, you have to modify all
% \oarg{num} arguments of footnote-related commands in the environment.
% This means that when the footnote addition is done in the first page of a
% 100-page document having \env{paracol} environments with explicitly numbered
% footnotes in every page, you have to make the corrections for environments
% in 99 pages.  The other a little bit less severe problem is that you have
% to keep \counter{footnote} counter having correct value by
% \!\setcounter!, \!\addtocounter! or \!\stepcounter! for footnotes following
% those with explicit numbering so that their numbers are given by the
% default action of \!\footnote!.
% 

虽然这种解决方案可以得到一个很好的结果,但它存在以下两个问题。首先,您必须通过\!\footnote!、\!\footnotetext! 和 \!\footnotemark! 命令的可选参数\oarg{num}显式地指定脚注编号。这个问题非常严重,因为例如,如果您在所讨论的\env{paracol}环境之前的某个地方添加了一个脚注,您必须修改环境中所有脚注相关命令的\oarg{num}参数。这意味着当在一个具有每页都有显式编号脚注的\env{paracol}环境的100页文档的第一页中进行脚注添加时,您必须对99页中的环境进行更正。另一个稍微不那么严重的问题是,您必须通过 \!\setcounter!、\!\addtocounter! 或 \!\stepcounter! 保持\counter{footnote}计数器具有正确的值,以便对那些具有显式编号的脚注之后的脚注进行默认的编号。

% To cope with these two problems, \textsf{paracol} provides you with the
% \emph{starred} versions of \!\footnote! and its relatives as introduced in
% Section~\ref{sec:ref-scfnote} and detailedly explained in the next
% Section~\ref{sec:fnnp-starred}.
% 
为了解决这两个问题,\textsf{paracol}为您提供了\emph{带星号}的 \!\footnote! 及其相关命令,如在第~\ref{sec:ref-scfnote} 节中介绍的,并在下一节~\ref{sec:fnnp-starred} 中详细解释。
% 
% \subsection{Commands \cs{footnote*} and Relatives}
% \label{sec:fnnp-starred}
% 
% \begin{description}
% \item[\Midx{\!\footnote!}\texttt{*}\oarg{|+|disp}\marg{text}]\mbox{}
% \Item[\Midx{\!\footnote!}\texttt{*}\oarg{|-|disp}\marg{text}]\mbox{}
% \Item[\Midx{\!\footnote!}\texttt{*}\oarg{disp}\marg{text}]\mbox{}\par
% The command is similar to its non-starred counterpart but the explicit
% numbering with the optional argument is done in \emph{self-relative} or
% \emph{base-displacement} style.  That is, if the optional argument has a
% leading `|+|' or `|-|',  the number given to the footnote is
% $f+\meta{disp}$ or $f-\meta{disp}$ respectively where $f$ is the value of
% \counter{footnote} counter, or in other words the number given to the last
% footnote\footnote{
% 
% If it is put by the ordinary \cs{footnote}.}.
% 
% Otherwise, i.e., the optional argument is a number without |+|/|-| sign,
% the number given to the footnote is $b+\meta{disp}$ where $b$ is the base
% value of \counter{footnote} counter at \beginparacol{} for the environment
% in which the command appears, or in other words the number given to the
% last \Preenv{} footnote\footnote{
% 
% Or the last footnote in the previous \env{paracol} environment,
% etc.\label{fn:4L0}}.
% 
% In addition, unlike the non-starred version, this command updates
% \counter{footnote} counter with the number given to the footnote, i.e.,
% $f\gets f+\meta{disp}$, $f\gets f-\meta{disp}$ or $f\gets b+\meta{disp}$
% is performed, so that following \!\footnote! without explicit numbering
% option have numbers $f+1$, $f+2$ and so on with new $f$.
% 
% \begin{itemize}
% \item
% If the optional argument is not provided, it is assumed that |[+1]| is
% given and thus \!\footnote!|*|\marg{text} acts as \!\footnote!\marg{text}.
% \end{itemize}
% 
% \item[\Midx{\!\footnotemark!}\rm|*[|{[|+-|]}\meta{disp}{|]|}]\mbox{}\par
% This command is a mixture of its non-starred counterpart and
% \!\footnote!|*|.  That is the number for the footnote mark is calculated
% in the way of \!\footnote!|*| and \counter{footnote} counter is updated.
% 
% \item[\Midx{\!\footnotetext!}\rm|*[|{[|+-|]}\meta{disp}{|]|}\marg{text}]
% \mbox{}\par
% Without the optional argument |[|[|+-|]\meta{disp}|]|, this command does what
% \!\footnotetext!\marg{text} does but in addition increments
% \counter{footnote} counter before that.  With the optional argument, on
% the other hand, the number given to the footnote \meta{text} is calculated
% as done in \!\footnote!, but the \counter{footnote} counter is not
% updated.
% \end{description}
% 
% With these starred commands, you can produce the following using the
% base-displacement mechanism without worrying about the absolute value of
% \!\footnote! counter and its change.
% 
% \Hrule
% \begin{paracol}{2}
% \tolerance5000\hbadness5000
% \fnpar{First left-column}{\footnote{First left-column
% footnote.\label{fn:4L1}}}
% \Fnpar{Second left-column}{\footnote{Second left-column
% footnote.\label{fn:4L2}}}{
% It is followed by \cs{footnotetext}|*[3]|\marg{text} and a
% \cs{switchcolumn}.}
% \footnotetext*[3]{Third left-column footnote given by
% \cs{footnotetext}|*[3]|\marg{text} placed at the end of
% the second left-column paragraph to have
% $\ref{fn:4L3}=\ref{fn:4L0}+3$.\label{fn:4L3}}
% \switchcolumn
% \Fnpar{First right-column}{\footnote*[4]{First right-column
% footnote whose number \ref{fn:4R1} is given by
% \cs{footnote}|*[4]|\marg{text} because
% $\ref{fn:4R1}=\ref{fn:4L0}+4$.\label{fn:4R1}}}{It is followed by a
% \cs{switchcolumn*}.}
% \switchcolumn*
% Third and synchronized left-column paragraph\Dotfill\\
% \Dotfill with a footnote whose mark
% here\footnotemark*[3]\Dotfill\\
% is given by \!\footnotemark!|*[3]| because $\ref{fn:4L3}=\ref{fn:4L0}+3$.
% It is followed by a \!\switchcolumn!.
% \switchcolumn
% \fnpar{Second and synchronized right-column}{\footnote*[5]{Second right-column
% footnote produced by \cs{footnote}|*[5]|\marg{text} because
% $\ref{fn:4R2}=\ref{fn:4L0}+5$.\label{fn:4R2}}}
% \fnpar{Third right-column}{\footnote{Third right-column
% footnote produced by \cs{footnote}\marg{text} because
% $\ref{fn:4R3}=\ref{fn:4R2}+1$.\label{fn:4R3}}}
% \end{paracol}
% \newpage
% 
% The other way to produce the same result except for the absolute footnote
% numbers is to use the self-relative mechanism and to exploit the progress
% of \counter{footnote} counter as follows.
% 
% \Hrule
% \begin{paracol}{2}
% \tolerance5000\hbadness5000
% \fnpar{First left-column}{\footnote{First left-column
% footnote.\label{fn:5L1}}}
% \Fnpar{Second left-column}{\footnote{Second left-column
% footnote.\label{fn:5L2}}}{
% It is followed by \cs{footnotetext}|*|\marg{text} and a
% \cs{switchcolumn}.}
% \footnotetext*{Third left-column footnote given by
% \cs{footnotetext}|*|\marg{text} placed at the end of
% the second left-column paragraph because it follows the second footnote
% \ref{fn:5L2}.\label{fn:5L3}}
% \switchcolumn
% \Fnpar{First right-column}{\footnote{First right-column
% footnote whose number \ref{fn:5R1} is given by
% \cs{footnote}\marg{text} because
% $\ref{fn:5R1}=\ref{fn:5L3}+1$ and \cs{footnotetext*} for \ref{fn:5L3} lets
% \counter{footnote} have the value.\label{fn:5R1}}}{It is followed by a
% \cs{switchcolumn*}.}
% \switchcolumn*
% Third and synchronized left-column paragraph\Dotfill\\
% \Dotfill with a footnote whose mark
% here\footnotemark*[-1]\Dotfill\\
% is given by \!\footnotemark!|*[-1]| because $\ref{fn:5L3}=\ref{fn:5R1}-1$.
% It is followed by a \!\switchcolumn!.
% \switchcolumn
% \fnpar{Second and synchronized right-column}{\footnote*[+2]{Second
% right-column footnote produced by \cs{footnote}|*[+2]|\marg{text} because
% $\ref{fn:5R2}=\ref{fn:5L3}+2$.\label{fn:5R2}}}
% \fnpar{Third right-column}{\footnote{Third right-column
% footnote produced by \cs{footnote}\marg{text} because
% $\ref{fn:5R3}=\ref{fn:5R2}+1$.\label{fn:5R3}}}
% \end{paracol}
% \Hrule
% 
% It depends on the structure of your document which of the
% base-displacement and self-relative is better.  If your document has
% frequent switching between single- and multi-column text typesetting and
% thus the contents of a \env{paracol} environment is relatively small, the
% base-displacement is a good choice because you may concentrate on one
% base value of \counter{footnote} counter.  Otherwise, especially when your
% document consists of one single and large \env{paracol} environment, the
% base-displacement is almost equivalent to maintaining absolute values and
% thus the self-relative should be preferred.
% 
% Note that if the last \!\footnote! or \!\footnotemark! in a \env{paracol}
% environment is starred, the command lets \counter{footnote} counter have
% some value smaller than that for the last stacked footnote.  For example, 
% if the second and third right-column footnotes \ref{fn:5R2} and
% \ref{fn:5R3} are omitted from the example above, the last footnote-related
% command will be \!\footnotemark!|*[-1]| which makes the counter at
% \Endparacol{} \ref{fn:5L3} rather than \ref{fn:5R1}.  You may not worry
% about this problem, however, because \Endparacol{} automatically maintains
% the counter letting it have $b+n$ where $n$ is the number of \!\footnote!
% and \!\footnotemark! in the environment, if the maintenance is ordered by
% the command \!\fncounteradjustment! which is automatically executed by
% \!\footnotelayout! with the argument |p| or |m|.
% 
% 
% 
% \subsection{Page Break}
% \label{sec:fnnp-pbreak}
% 
% When a \env{paracol} environment with footnotes lays across a page boundary,
% you could have some weird result even if the environment have just one
% \!\switchcolumn! as shown below.
% 
% \Hrule
% \begin{paracol}{2}
% First left-column paragraph \Dotfill\\
% \Dotfill with a footnote\footnote{First left-column
% footnote.\label{fn:6L1}}\Dotfill\\
% \Dotfill\\ \Dotfill\\ \Dotfill\\ \Dotfill\\ \Dotfill\\
% \Dotfill in it.
% \par
% Second left-column paragraph \Dotfill\\
% \Dotfill with a footnote\footnote{Second left-column
% footnote.\label{fn:6L2}}
% \Dotfill in it.
% \switchcolumn
% First right-column paragraph \Dotfill\\
% \Dotfill with a footnote\footnote{First right-column
% footnote weirdly placed here while the footnoted main text is in the
% previous page.\label{fn:6R1}}\Dotfill\\
% \Dotfill\\ \Dotfill\\ \Dotfill\\ \Dotfill\\ \Dotfill\\
% \Dotfill in it.
% \par
% Second right-column paragraph \Dotfill\\
% \Dotfill with a footnote\footnote{Second right-column
% footnote whose mark in the main text gives impression that footnote
% numbering jumps from \ref{fn:6L2} to \ref{fn:6R2}.\label{fn:6R2}}
% \Dotfill in it.
% \end{paracol}
% \Hrule
% 
% Since the part of the source |.tex| for this example above is
% fundamentally same as that in p.~\pageref{sec:fnnp} at the beginning of
% this Section~\ref{sec:fnnp}, footnotes are simply numbered in
% left-column-first manner without any tricks.  However it results in
% giving an impression that two paragraphs in each of both columns at the
% bottom of the last page have footnote marks of inconsecutive numbers
% \ref{fn:6L1} and \ref{fn:6R1} due to the second left-column paragraph and
% the footnote \ref{fn:6L2} in it.  More weirdly, the first right-column
% footnote \ref{fn:6R1} is not put in the last page where its mark is shown
% but is stacked below \ref{fn:6L2} in this page.
% 
% The reason why this happens is that a footnote is not immediately put to
% the bottom of the page where its mark resides but to the page constructing
% at the time when the footnote is processed at the end of the paragraph in
% which the corresponding \!\footnote! (or \!\footnotetext!)
% occurs\footnote{
% 
% More accurately, the footnote is kept in a place in \TeX{} together with
% other preceding but still unprocessed footnotes and then \TeX{} examines
% them at the end of a paragraph in which a page break is found to decide
% whether each of them is included in the page just being completed.}.
% 
% Therefore, it may happen even in an ordinary single-column document or a
% \env{paracol}ed multi-column one with \Mcfnote{}s that a
% footnote is thrown to the page $p+1$ next to the page $p$ in which its
% mark is left, when the mark is placed around the bottom of the page
% $p$.
% 
% This footnote placement mechanism becomes clearly visible in the example
% above in which the footnote \ref{fn:6R1} is processed {\em after} the
% second left-column paragraph is processed to complete the last page giving
% no chance to the footnote placed in the page\footnote{%
% 
% In fact, even \cs{footnote} for the footnote is processed after the page
% break in this case.}.
% 
% Therefore, the solution of this placement problem is to let the first
% right-column footnote processed {\em before} the page is broken by the
% progress of the left-column.  That is, in the solution shown below the
% author inserted \!\switchcolumn! after the first left-column paragraph to
% let the first right-column paragraph and its footnote are processed, and
% then did \!\switchcolumn! again after the right-column paragraph to go
% back to the left-column.
% 
% \Hrule
% \begin{paracol}{2}
% First left-column paragraph \Dotfill\\
% \Dotfill with a footnote\footnote{First left-column
% footnote.\label{fn:7L1}}\Dotfill\\
% \Dotfill\\ \Dotfill\\ \Dotfill\\ \Dotfill\\ \Dotfill\\ \Dotfill\\ \Dotfill\\
% \Dotfill\\ \Dotfill\\ \Dotfill\\ \Dotfill\\ \Dotfill\\ \Dotfill\\ \Dotfill\\
% \Dotfill\\
% \Dotfill in it.\\
% It is followed by a \!\switchcolumn!.
% \par\switchcolumn
% First right-column paragraph \Dotfill\\
% \Dotfill with a footnote\footnote{First right-column
% footnote which is now placed in this page where its mark \ref{fn:7R1}
% resides.\label{fn:7R1}}\Dotfill\\
% \Dotfill\\ \Dotfill\\ \Dotfill\\ \Dotfill\\ \Dotfill\\ \Dotfill\\ \Dotfill\\
% \Dotfill\\ \Dotfill\\ \Dotfill\\ \Dotfill\\ \Dotfill\\ \Dotfill\\ \Dotfill\\
% \Dotfill in it.\\
% It is followed by a \!\switchcolumn! to go back to the left column.
% \par\newpage\switchcolumn
% Second left-column paragraph \Dotfill\\
% \Dotfill with a footnote\footnote{Second left-column
% footnote whose number \ref{fn:7L2} follows the right-column footnote
% \ref{fn:7R1} in the last page.\label{fn:7L2}}
% \Dotfill in it.\\
% It is also followed by a \!\switchcolumn!.
% \switchcolumn
% Second right-column paragraph \Dotfill\\
% \Dotfill with a footnote\footnote{Second right-column
% footnote whose number \ref{fn:7R2} follows the left-column footnote
% \ref{fn:7L2}.\label{fn:7R2}}
% \Dotfill in it.
% \end{paracol}
% \Hrule
% 
% Unfortunately, this tactics does not always solve the problem.  If a
% left-column paragraph has a page break in it and a footnote before the
% break, doing \!\switchcolumn! after the paragraph is too late to let
% right-column footnotes reside in the page just having been broken, while
% inserting \!\switchcolumn! before the paragraph should cause incorrect
% stacking order.
% 
% The remedy for this problem is similar to that shown in
% Section~\ref{sec:fnnp-multsc} to cope with multiple \!\switchcolumn! in a
% \env{paracol} environment.  Here it is shown a little bit more formally.
% Suppose we have a page in a \env{paracol} environment in which a page
% break occurs in $p_l$-th and $p_r$-th paragraphs in the left and right
% columns respectively.  Thus we have $p_l-1$ and $p_r-1$ completed
% paragraphs in each of both columns.  Let $n_l$ (resp.\ $n_r$) be the
% number of footnotes in the pre-break left-column (resp.\ right-column)
% paragraphs, and $m_l$ (resp.\ $m_r$) be the number of pre-break footnotes
% in the $p_l$-th (resp.\ $p_r$-th) paragraph.  Thus we have $n_l+m_l$
% (resp.\ $n_r+m_r$) footnotes in the left (resp.\ right) column of the page
% before the break.  The following construct assures that those footnotes
% are properly numbered and stacked at the bottom of the page.
% 
% \begin{list}{}{\rightmargin\leftmargin \itemindent-.5\leftmargin
% \listparindent\itemindent \leftmargin1.5\leftmargin \parsep0pt}\it\item
% First to $(p_l-1)$-th paragraphs with $n_l$ footnotes in total given by
% {\rm\!\footnote!\marg{text}}.\par
% {\rm\!\footnotetext!|*{|{\it 1st footnote in $p_l$-th paragraph}|}|}\par
% \mbox{\qquad}\ldots\par
% {\rm\!\footnotetext!|*{|{\it$m_l$-th footnote in $p_l$-th paragraph}|}|}\par
% \!\switchcolumn!\par
% First to $(p_r-1)$-th paragraphs with $n_r$ footnotes in total given by
% {\rm\!\footnote!\marg{text}}.\par
% {\rm\!\footnotetext!|*{|{\it 1st footnote in $p_r$-th paragraph}|}|}\par
% \mbox{\qquad}\ldots\par
% {\rm\!\footnotetext!|*{|{\it$m_r$-th footnote in $p_r$-th paragraph}|}|}\par
% \!\switchcolumn!\par
% $p_l$-th paragraph whose first footnote mark is given by
% {\rm\!\footnotemark!|*[-|$(m_l{+}n_r{+}m_r{-1})$|]|}, while second to
% $m_l$-th ones are given by \!\footnotemark! without {\rm|*|} nor optional
% {\rm\oarg{num}}.  The first subsequent footnotes beyond the page break, if
% any, is given by {\rm\!\footnote!|*[+|$(n_r{+}m_r{+1})$|]|\marg{text}}
% while further subsequent ones are given by
% {\rm\!\footnote!\marg{text}}.\par
% \!\switchcolumn!\par
% $p_r$-th paragraph whose first footnote mark is given by
% {\rm\!\footnotemark!|*[-|$(m_r{+}k_l{-1})$|]|} where $k_l$ is the number
% of left-column footnotes beyond the break, while second to $m_r$-th ones
% are given by \!\footnotemark!.  The first subsequent footnotes beyond the
% page break, if any, is given by
% {\rm\!\footnote!|*[+|$(k_l{+1})$|]|\marg{text}}, while further subsequent
% ones are given by {\rm\!\footnote!\marg{text}}.
% \end{list}
% %
% The example shown in the next two pages is for the case of
% $p_l=p_r=n_l=n_r=m_l=m_r=k_l=2$.
% 
% \newpage
% \Hrule
% \begin{paracol}{2}
% First left-column paragraph with two footnotes\break
% \mbox{}\Dotquad here\footnote{First left-column footnote given by
% \cs{footnote}\marg{text}.\label{fn;8L1}} by
% \!\footnote!\marg{text}\Dotfill\\
% \Dotquad and here\footnote{Second left-column footnote also given by
% \cs{footnote}\marg{text}.\label{fn:8L2}} also by
% \!\footnote!\marg{text}\Dotfill\\
% \Dotfill\\\Dotfill\\\Dotfill\\\Dotfill\\\Dotfill\\
% \Dotfill\\\Dotfill\\\Dotfill\\\Dotfill\\\Dotfill\\
% \Dotfill\\\Dotfill\\\Dotfill\\\Dotfill\\\Dotfill\\
% followed by a series of \!\footnotetext!|*|\marg{text} and then a
% \!\switchcolumn!.
% \footnotetext*{Third left-column footnote given by
% \cs{footnotetext*}\marg{text}.\label{fn:8L3}}
% \footnotetext*{Fourth left-column footnote given by
% \cs{footnotetext*}\marg{text}.\label{fn:8L4}}
% 
% \switchcolumn
% First right-column paragraph with two footnotes\break
% \mbox{}\Dotquad here\footnote{First right-column footnote given by
% \cs{footnote}\marg{text}.\label{fn;8R1}} by
% \!\footnote!\marg{text}\Dotfill\\
% \Dotquad and here\footnote{Second right-column footnote also given by
% \cs{footnote}\marg{text}.\label{fn:8R2}} also by
% \!\footnote!\marg{text}\Dotfill\\
% \Dotfill\\\Dotfill\\\Dotfill\\\Dotfill\\\Dotfill\\
% \Dotfill\\\Dotfill\\\Dotfill\\\Dotfill\\\Dotfill\\
% \Dotfill\\\Dotfill\\\Dotfill\\\Dotfill\\\Dotfill\\
% followed by a series of \!\footnotetext!|*|\marg{text} and then a
% \!\switchcolumn!.
% \footnotetext*{Third right-column footnote given by
% \cs{footnotetext*}\marg{text}.\label{fn:8R3}}
% \footnotetext*{Fourth right-column footnote given by
% \cs{footnotetext*}\marg{text}.\label{fn:8R4}}
% 
% \switchcolumn
% Second left-column paragraph across two pages\break
% \mbox{}\Dotquad with two pre-break footnotes\Dotfill\\
% \Dotquad here\footnotemark*[-5] by \!\footnotemark!|*[-5]|
% because $m_l+n_r+m_r-1=2+2+2-1=5$ and thus
% $\ref{fn:8L3}=\ref{fn:8R4}-5$\Dotfill\\
% \Dotquad and here\footnotemark{} by \!\footnotemark!\Dotfill\\
% \Dotfill\\\Dotfill\\\Dotfill\\\Dotfill\\\Dotfill\\
% \Dotfill\\\Dotfill\\\Dotfill\\\Dotfill\\\Dotfill\\
% \Dotfill\\\Dotfill\\\Dotfill\\\Dotfill\\\Dotfill\\
% \Dotfill\\\Dotfill\\
% \Dotquad and two post-break footnotes\Dotfill\\
% \Dotquad here\footnote*[+5]{Fifth left-column footnote given by
% \cs{footnote}|*[+5]| because $n_r+m_r+1=2+2+1=5$ and thus
% $\ref{fn:8L5}=\ref{fn:8L4}+5$.\label{fn:8L5}} by
% \!\footnote!|*[+5]|\marg{text}\Dotfill\\
% \Dotquad and here\footnote{Sixth left-column foootnote given by
% \cs{footnote}\marg{text}.\label{fn:8L6}} by
% \!\footnote!\marg{text}\Dotfill\\
% followed by a \!\switchcolumn!.
% 
% \switchcolumn
% Second right-column paragraph across two pages\break
% \mbox{}\Dotquad with two pre-break footnotes\Dotfill\\
% \Dotquad here\footnotemark*[-3] by \!\footnotemark!|*[-3]|
% because $m_r+k_l-1=2+2-1=3$ and thus
% $\ref{fn:8R3}=\ref{fn:8L6}-3$\Dotfill\\
% \Dotquad and here\footnotemark{} by \!\footnotemark!\Dotfill\\
% \Dotfill\\\Dotfill\\\Dotfill\\\Dotfill\\\Dotfill\\
% \Dotfill\\\Dotfill\\\Dotfill\\\Dotfill\\\Dotfill\\
% \Dotfill\\\Dotfill\\\Dotfill\\\Dotfill\\\Dotfill\\
% \Dotfill\\\Dotfill\\
% \Dotquad and two post-break footnotes\Dotfill\\
% \Dotquad here\footnote*[+3]{Fifth right-column footnote given by
% \cs{footnote}|*[+3]| because $k_l+1=3$ and thus
% $\ref{fn:8R5}=\ref{fn:8R4}+3$.\label{fn:8R5}} by
% \!\footnote!|*[+3]|\marg{text}\Dotfill\\
% \Dotquad and here\footnote{Sixth right-column foootnote given by
% \cs{footnote}\marg{text}.\label{fn:8R6}} by
% \!\footnote!\marg{text}\Dotfill.
% \end{paracol}
% \Hrule
% 
% Note that though the remedy works well as shown above, it is not a good
% idea to do that when you are writing draft versions of your document
% because page break points go up and down by your modifications to the
% document.  Therefore, it is recommended to put all footnotes by
% non-starred \!\footnote! until your document becomes perfect except for
% footnote numbering and placement and then to adjust them by the techique
% described in this section.
% 
% \endinput
