 
% \begin{paracol}{2}
% 
% \switchcolumn[0]*[\section{Column Synchronization\\栏同步}\label{sec:sync}]
% The \!\switchcolumn! command may also be followed by a `|*|' to
% {\em\Uidx\sync{}e} columns.  After you switch from a column to another by
% \!\switchcolumn!|*| (or \!\switchcolumn!|[|$i$|]*|), all the columns are
% vertically aligned at the bottom of the {\em deepest} one preceding the
% command.  For example, the previous section has three \!\switchcolumn!|*|
% commands at which left and right columns are vertically aligned.
% \switchcolumn
% \!\switchcolumn! 命令后面可以加上 `|*|',用来{\em 同步}栏。当你使用 \!\switchcolumn!|*|(或 \!\switchcolumn!|[|$i$|]*|)从一栏切换到另一栏时,所有栏都会垂直对齐在该命令之前最{\em 深}的栏的底部。例如,前一节使用了三个 \!\switchcolumn!|*| 命令,使左右两栏垂直对齐。
% \switchcolumn*
% The {\em starred} version of \!\switchcolumn! may have an optional
% argument to specify a single-column {\em\Uidx\mctext} whose bottom is the
% vertical alignment point of columns.  For example, \!\section!
% commands in this manual are given as optional arguments
% of \!\switchcolumn!|*| like;
% \switchcolumn
% {\em 带星号}版本的 \!\switchcolumn! 命令可以带有可选参数,用来指定一个单栏的{\em 同步文本},其底部作为栏的垂直对齐点。例如,本手册中的 \!\section! 命令作为 \!\switchcolumn!|*| 的可选参数给出,如下所示:

% \switchcolumn*
%\begin{Verbatim}
%  \switchcolumn*[\section{Basic Usage}]
%\end{Verbatim}
% \switchcolumn[0]*
% The \env{paracol} environment may also start with a \mctext{} by
% specifying it as the optional argument of \!\begin!|{|\env{paracol}|}|.
% For example, at the beginning of this document, the author put;
% \switchcolumn
% \env{paracol} 环境也可以以一个 \mctext{} 开始,将其指定为 \!\begin!|{|\env{paracol}|}| 的可选参数。例如,在本文档的开头,作者使用了以下代码:
%\switchcolumn[0]*
%\begin{Verbatim}
%  \begin{paracol}{2}[\section{Introduction}]
%\end{Verbatim}

% \end{paracol}