%%%%%%%%%%%%%%%{fontspec}%%%%%%%%%%%%%%%
%将no-math选项传递给fontspec宏包,该选项禁用了使用fontspec宏包中的数学字体功能。
\PassOptionsToPackage{no-math}{fontspec}
%%%%%%%%%%%%%%%{xeCJK}%%%%%%%%%%%%%%%
%将AutoFakeBold和AutoFakeSlant选项传递给xeCJK宏包,这两个选项让xeCJK宏包自动产生伪粗体和伪斜体效果。
\PassOptionsToPackage{AutoFakeBold=true,AutoFakeSlant=true}{xeCJK}

\documentclass{article}
\usepackage[
fontset=none%不使用默认的字体设置
,space=auto%自动调整中英文间距
]{ctex}
\setCJKmainfont{FangZhengShuSong-GBK-1.ttf}[Path=/Users/virhuiai/hlProjects/Latex-Typesetting-Hub/font/方正/]%设置文本的中文有衬线字体
\setCJKsansfont{FangZhengHeiTi-GBK-1.ttf}[Path=/Users/virhuiai/hlProjects/Latex-Typesetting-Hub/font/方正/]%设置文本的中文无衬线字体为
\setCJKmonofont{FangZhengFangSong-GBK-1.ttf}[Path=/Users/virhuiai/hlProjects/Latex-Typesetting-Hub/font/方正/] %设置文本的中文等宽字体 
\usepackage[left=15mm,right=15mm,top=15mm,bottom=15mm]{geometry}
\usepackage{zitie}


\begin{document}

\section*{好吃的橘子}


\zitiesetup{charcolor=blue,charstroke=solid,fillcolor={},frametype=咪,width=2em,resize=square,}

\hspace{2em}\framezi[frametype=咪]{缤纷的水果世界,有很多好吃的水果,西瓜、火龙果、桃子、木瓜。但是我最喜欢的还是橘子。}

\hspace{2em}\framezi[frametype=咪]{现在正是吃橘子的季节。橘子圆圆的,差不多乒乓球大小,金黄金黄的,仿佛穿上了一件金黄的衣服,从远处看,还以为是一个个小灯笼呢。}

\hspace{2em}\framezi[frametype=咪]{记得有一次,爸爸带回了一大箱冰糖橘,我迫不及待地拿出一个,撕开皮,皮很薄,软软的,凉凉的,刚一撕开,一股浓香马上弥漫开来。}

\hspace{2em}\framezi[frametype=咪]{橘子有大约十瓣,``果肉大家庭''紧紧挤在一起,好像小兄弟们围在一起说着什么悄悄话,我掰下一瓣,弯弯的,像半轮月亮,太美了!}

\hspace{2em}\framezi[frametype=咪]{我轻轻地咬一口,汁水竟然溅了我一手,尝一尝,果然像冰糖一样甜!我连着吃了十几个冰糖橘子!爸爸说:``不可以再吃了!冰糖橘子吃多了,会上火的。''我只得停下来,无奈地看着。}

\hspace{2em}\framezi[frametype=咪]{橘子富含多种维生素,如B2,维C,可以美容养颜,预防癌症。它还含有钙、磷、铁等微量元素,有益于健康。}

\hspace{2em}\framezi[frametype=咪]{吃冰糖橘子,让我感受到了生活的甜蜜!}


\newpage
\section*{五感法写桔子}

\hspace{2em}\framezi[frametype=咪]{视觉}\hspace{2em} \framezi[frametype=咪]{妈妈买回一箱砂糖橘,一个个扁圆的桔子好像一盏盏橙色的小灯笼,颜色新鲜美丽。}

\hspace{2em}\framezi[frametype=咪]{触觉}\hspace{2em} \framezi[frametype=咪]{我迫不及待伸手拿起一个仔细观察,橘皮表面有许多小小的凸起,用手一摸疙疙瘩瘩的,非常粗糙。}

\hspace{2em}\framezi[frametype=咪]{嗅觉}\hspace{2em} \framezi[frametype=咪]{我掰开橘皮,顿时一股清香扑鼻而来。}

\hspace{2em}\framezi[frametype=咪]{触觉+味觉}\hspace{2em} \framezi[frametype=咪]{橘皮里面,十个软软的小橘瓣像十个可爱的小月亮整齐得围坐一圈。我剥下几瓣放到嘴里,香甜的汁水溢满口腔。}

\hspace{2em}\framezi[frametype=咪]{语言描写}\hspace{2em} \framezi[frametype=咪]{ ``妈妈这次买的桔子超好吃,大家快来品尝!''我连忙招呼爸妈和妹妹。``哇,桔瓣上白色的丝线是什么?''妹妹好奇地问。``这是橘络,吃了可以止咳顺气的,放心吃吧!''听,还是老爸最有学问。大家一边吃一边议论,欢声笑语充满了这个温馨的家。}


% 请将每一段文字替换为如下效果:

% \hspace{2em}\framezi[frametype=咪]{替换为文字}
\end{document}


% \begin{zitieframe}[charcolor=blue,charstroke=solid,fillcolor={},frametype=咪,width=1cm,resize=square,]

    
% \end{zitieframe}

\newpage
\section*{五感法写桔子}
\begin{zitieframe}[charcolor=blue,charstroke=solid,fillcolor={},frametype=咪,width=1cm,resize=square,]
有以下文字:    
视觉 妈妈买回一箱砂糖橘,一个个扁圆的桔子好像一盏盏橙色的%
小灯笼,颜色新鲜美丽。

触觉我迫不及待伸手拿起一个仔细观察,橘皮表面有许多小小的
凸起,用手一摸疙疙瘩瘩的,非常粗糙。

嗅觉我掰开橘皮,顿时一股清香扑鼻而来。
触觉+味觉橘皮里面,十个软软的小橘瓣像十个可爱的小月亮整%
齐得围坐一圈。我剥下几瓣放到嘴里,香甜的汁水溢满口腔。

语言描写 ``妈妈这次买的桔子超好吃,大家快来品尝!''我连忙%
招呼爸妈和妹妹。``哇,桔瓣上白色的丝线是什么?''妹妹好奇地%
问。``这是橘络,吃了可以止咳顺气的,放心吃吧!''听,还是老%
爸最有学问。大家一边吃一边议论,欢声笑语充满了这个温馨的%
家。


请将每一段文字替换为如下效果:

\hspace{2em}\framezi[frametype=咪]{替换为文字}


\end{zitieframe}


% {\zitiesetup{font=楷体,width=1cm,resize=square,framecolor=black,linewidth=.7pt}
% \framezi[frametype=none]{无}
% \framezi[frametype=口]{口}
% \framezi[frametype=十]{十}
% \framezi[frametype=田]{田}
% \framezi[frametype=×]{叉}
% \framezi[frametype=米]{米}
% \framezi[frametype=咪]{咪}}

% \framezi[width=2\ccwd,charcolor=blue,charstroke=solid,fillcolor={},frametype=咪,width=1cm,resize=square,]{好}


\end{document}
