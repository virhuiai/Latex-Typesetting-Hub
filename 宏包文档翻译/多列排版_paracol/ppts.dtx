% \def\oddeven#1#2{\ifodd\value{page}#1\else#2\fi}
% \marginparpush5pt \@mparswitchtrue

% 
% \newpage
% \subsection{Example of Paired Parallel-Paging\hfill 并列分页的示例}
% \label{sec:ppts-paired}
% 
% Shortly we will start a \env{paracol} environment by \beginparacol|[2]{4}|
% having four columns but two for each of left and right \paired{}
% \parapag{}es.  Since the author declares \!\columnratio!|{0.6}[0.5]|, the
% columns in left pages are made unbalanced while those in right pages are
% balanced.


% 不久我们将通过 \beginparacol|[2]{4}|开始一个\env{paracol}环境,该环境有四列,但是每个左右\paired{}\parapag{}中有两列。由于作者声明了 \!\columnratio!|{0.6}[0.5]|,左页中的列是不平衡的,而右页中的列是平衡的。

% \columnratio{0.6}[0.5]
% \par\Hrule
% \begin{paracol}[2]{4}
% This is the first paragraph of the leftmost column-0,
% \Marginpar{Marginal note from column-0.}
% whose first line has a marginal note placed in the right margin because
% the setting of \!\marginparthreshold! being 0 is still effective and we
% are in the odd-numbered page \pageref{sec:ppts-paired}.  Now we
% have a \!\switchcolumn! to the next column-1.
% 
% \switchcolumn
% \begin{Hfuzz}{1.1pt}\it
% This is the first paragraph of the second and right column-1 in the left
% \parapag{}e.  We shortly give an italicized mar\-gin\-al note carefully, so
% that it does not conflict with the marginal note from the column-0.
% \Marginpar{\it Marginal note from column-1.}
% That is, now the author puts the note.  Now we
% have a \!\switchcolumn! to the next column-2.
% \end{Hfuzz}
% \footnotetext*{This footnote is put in the left \parapag{}e together with
% another footnote below given in the column-2 in the right \parapag{}e.
% \label{fn:ppts-paired1}}
% 
% \switchcolumn
% \begingroup\sf
% This is the first paragraph of the column-2 being the left column of the
% right \parapag{}e.  Though we are in a page different from that column-0
% and 1 reside in, this page is still numbered \pageref{sec:ppts-paired}
% because the left and right page is \paired.  Therefore, the left margin of
% this page is narrower than the right margin because the page number is
% odd.
% 
% \footnotetext*{This footnote is \emph{not} put in the right \parapag{}e
% though it is given in the column-2 in the right \parapag{}e and thus its
% reference is in the column, of course.\label{fn:ppts-paired2}}
% 
% You have to notice
% \Marginpar{\sf Marginal note from column-2.}
% the first paragraph does not start from the page top
% but above it we have some space of exactly same size as the \preenv{}
% shown in the left \parapag{}e.  Therefore, the top of the first paragraphs
% in all columns are aligned.  The marginal note given in the first line of
% this paragraph goes to the right margin of this page because of the
% \!\marginparthreshold! setting and the parity of this page.  Now we have a
% \!\switchcolumn! to the next column-3.
% \par\endgroup
% \begin{figure*}\nosv
% \def\arraystretch{0.8}
% \centerline{\begin{tabular}[b]{|c|}\hline
%     \hbox to.9\textwidth{}\\
%     \sf page-wise figure given in column-2\\
%     \\\hline
%     \end{tabular}}
% \caption{A Page-Wise Figure}
% \end{figure*}
% 
% \switchcolumn
% \begingroup\sl
% This is the first paragraph
% \Marginpar{\sl Marginal note from column-3.}
% in the last rightmost column-3 whose width is equal to that of the column-2.
% The marginal note given in the first line goes to right and does not
% conflict with that from the column-2.  We are now going back to the
% column-0 by a {\rm\!\switchcolumn!|*|} with a \mctext.
% \endgroup
% 
% \switchcolumn*[\subsection*{A Spanning Text: though this is wider than the
% page width, this text does not span the boundary between the left and
% right parallel-pages.}]
% 
% We have come back to this column-0.  The space above the \mctext{} is due
% to the \sync{}ation because two paragraphs in the column-2 are
% significantly taller in total than the paragraphs in other columns.  As
% the spanning text itself says, it cannot extend to the right \parapag{}e.
% The author puts dummy lines to go to the page bottom.\\
% \Dotfill\\ \Dotfill\\ \Dotfill\\ \Dotfill\\ \Dotfill\\ \Dotfill\\
% \Dotfill\\ \Dotfill\\ \Dotfill\\ \Dotfill\\ \Dotfill\par
% 
% Now we will have a page break shortly.  You could be surprised by seeing
% this column is not in the left \parapag{}e after the break but in the
% right one.  This is because the feature |c| is enabled to swap not only
% columns in a page but also the left and right \paired{} \parapag{}es when
% they are even-numbered.  The other feature |p| makes the left outside
% margins of this right and the previous left pages wider than the right
% inside margins.\label{page:ppts-paired2}
% 
% \switchcolumn
% \begingroup\it
% We have restarted this column-1.  This paragraph has a
% footnote\footnotemark*[-1] as shown below.\\
% \Dotfill\\ \Dotfill\\ \Dotfill\\ \Dotfill\\ \Dotfill\\ \Dotfill\\
% \Dotfill\\ \Dotfill\\ \Dotfill\\ \Dotfill\\ \Dotfill\\ \Dotfill\\
% \Dotfill\\ \Dotfill\par
% 
% After the page break below, this column also goes to the right page
% together with the column-0
% \Marginpar{\it Another marginal note from column-1.}
% and is placed outside (left) in the page, as well as the marginal note
% in this right page but in the outside margin.
% \par\endgroup
% 
% \switchcolumn
% \begingroup\sf
% We have a few other materials not shown in right \parapag{}es.  The space
% above this paragraph is for the \mctext{} placed in the left \parapag{}e.
% The \Scfnote{} given here\footnotemark{} is also not in this page but in
% the left.  Finally, the author has put a page-wise figure spanning columns
% just before \!\switchcolumn! by which we left this column, but it will be
% in the right page \pageref{page:ppts-paired2} together with column-0 and
% 1.\\
% \Dotfill\\ \Dotfill\\ \Dotfill\\ \Dotfill\\ \Dotfill\\ \Dotfill\\
% \Dotfill\\ \Dotfill\\ \Dotfill\par
% 
% Though the footnote numbered \ref{fn:ppts-paired2} goes to the left page,
% its space and that of \ref{fn:ppts-paired1} make this and the next columns
% shorter in the previous page.  Similarly, we have a space above for the
% page-wise figure shown in the right page.
% \par\endgroup
% 
% \switchcolumn
% \begingroup\sl
% As expected, this line is aligned to the first line of the paragraph in
% the column-2 as well as those in column-0 and 1.  It is also consistent
% the first lines including that of this paragraph are not indented because
% the \mctext{} is given by {\rm\!\subsection!|*|} which makes first
% paragraphs unindented.\\
% \Dotfill\\ \Dotfill\\ \Dotfill\\ \Dotfill\\ \Dotfill\\ \Dotfill\\
% \Dotfill\\ \Dotfill\\ \Dotfill\\ \Dotfill\\ \Dotfill\par
%
% After the page break we will have shortly, this column becomes the
% leftmost in the left \parapag{}e, as you are seeing now,
% \Marginpar{\sl Another marginal note from column-3.}
% but still outermost as well as the marginal note in the outside left
% margin.
% \endgroup
% \end{paracol}
% \Hrule
% 
% Now you are seeing yet another material placed only in the page in which
% the column-0 resides and thus being the right page now, i.e., this
% paragraph and the next one in the \postenv.  You might be disappointed by
% the fact the \emph{outside} pages, i.e., left in this page
% \pageref{page:ppts-paired2} and right in the previous page
% \pageref{sec:ppts-paired}, cannot have \pwstuff{} but it is what the
% author can do now for the version 1.3 and thus you have to wait some
% future versions in which the author could devise a mechanism to exploit
% the corresponding space in the pages\footnote{%
% You might complain the immaturity of \parapag{}ing and might claim that it
% should be included in \Paracol{} after the author implements the
% mechanism.  In fact the author himself is frustrated current features of
% \parapag{}ing but he dared to release the version 1.3 knowing that there
% are people who happily typeset their \parapag{}ed documents with the
% current limited features.}.

现在您正在看到的是仅放置在列-0所在页面中的另一个材料,因此现在是右侧页面,即本段和下一个段落在 \postenv 中。您可能会对这样一个事实感到失望,即\emph{外部}页面,即本页的左侧(\pageref{page:ppts-paired2}页)和前一页的右侧(\pageref{sec:ppts-paired}页),无法使用 \pwstuff{},但这是作者目前版本1.3能做的,因此您必须等待未来的版本,届时作者可能会设计一种机制来利用页面上的相应空间\footnote{%
您可能会对 \parapag{} 的不成熟感到不满,并声称作者应该在实现该机制后将其包含在 \Paracol{} 中。实际上,作者自己对当前 \parapag{} 的功能感到沮丧,但他还是敢于发布版本1.3,因为他知道有人愉快地使用当前有限的功能来排版他们的 \parapag{} 文档。}。

% In addition, you might think it is weird that the |c| feature of
% \!\twosided! swaps columns \emph{and} paired pages.  However this swapping
% is a natural consequence of the combination of \cswap{} and \paired{}
% \parapag{}ing.  Therefore, you can simply disable the |c| feature (maybe
% together with other features) to have more intuitive results.

此外,您可能会觉得奇怪的是,\!\twosided! 命令的|c|功能交换了列\emph{和}配对的页面。然而,这种交换是\cswap{}和\paired{}\parapag{}的组合的自然结果。因此,您可以简单地禁用|c|功能(可能与其他功能一起禁用),以获得更直观的结果。

% In the next Section~\ref{sec:ppts-npaired}, you will see another kind of
% \parapag{}ing namely \npaired{} one.  Before that, we need a blank page to
% let the \npaired{} \parapag{}ing start from an even-numbered page so that
% a left and right page pair comprises a double spread.  A short remark on
% the blank next page is that it does not have a right counterpart
% \parapag{}e because the page is outside \env{paracol} environments and does
% not have any portion from the environments\footnote{%
% To illustrate this fact, the author dares to put a real blank page rather
% than stepping the \counter{page} counter.}.

在接下来的第~\ref{sec:ppts-npaired}节中,你将看到另一种\parapag{}分页方式,即\npaired{}分页。在此之前,我们需要一个空白页,以便让\npaired{} \parapag{}从偶数页开始,这样左右的页面对就构成一个双页展开。关于空白的下一页的一个简短说明是,它没有右侧对应的\parapag{},因为该页位于\env{paracol}环境之外,并且不包含来自这些环境的任何部分\footnote{为了说明这个事实,作者敢于放置一个真正的空白页,而不是增加\counter{page}计数器的值。}。

% \newpage\vspace*{\fill}\centerline{(intentionally blanked page)}\vfill
% 
% 
% 
% \newpage
% \subsection{Example of Non-Paired Parallel-Paging}
% \label{sec:ppts-npaired}
% 
% This and following three pages are to show an example of \npaired{}
% \parapag{}ing, in which the author keeps the setting of \!\twosided!,
% \!\columnratio! and \!\marginparthreshold! unchanged.
% The arguments of \beginparacol{} for column population are also unchanged
% to have $2+2$ configuration, but the first argument is followed by |*| for
% \npaired{} typesetting.  That is, the environment below starts by
% \beginparacol|[2]*{4}|.  The contents of the environment is also almost
% same as the previous Section~\ref{sec:ppts-paired}, while
% \Emph{bold-faced} words show the difference from the \paired{}
% typesetting.
% 
% \columnratio{0.6}[0.5]
% \par\Hrule
% \begin{paracol}[2]*{4}
% This is the first paragraph of the leftmost column-0,
% \Marginpar{Marginal note from column-0.}
% whose first line has a marginal note placed in the \Emph{left} margin
% because the setting of \!\marginparthreshold! being 0 is still effective
% and we are in the \Emph{even}-numbered page
% \Emph{\pageref{sec:ppts-npaired}}.  Now we have a \!\switchcolumn! to the
% next column-1.
% 
% \switchcolumn
% \begingroup\it
% This is the first paragraph of the second and right column-1 in the left
% \parapag{}e.  We shortly give an italicized mar\-gin\-al note carefully, so
% that it does not conflict with the marginal note from the column-0.
% \Marginpar{\it Marginal note from column-1.}
% That is, now the author puts the note.  Now we
% have a \!\switchcolumn! to the next column-2.
% \par\endgroup
% \footnotetext*{This footnote is put in the left \parapag{}e together with
% another footnote below given in the column-2 in the right \parapag{}e.
% \label{fn:ppts-npaired1}}
% 
% \switchcolumn
% \begingroup\sf\label{page:ppts-npaired1r}
% This is the first paragraph of the column-2 being the left column of the
% right \parapag{}e.  \Emph{Since we are in the page next to} that column-0
% and 1 reside in, this page is numbered \Emph{\pageref{page:ppts-npaired1r}}
% because the left and right page is \Emph{\npaired}.  Therefore, the left
% margin of this page is narrower than the right margin because the page
% number is odd.
% 
% \footnotetext*{This footnote is \emph{not} put in the right \parapag{}e
% though it is given in the column-2 in the right \parapag{}e and thus its
% reference is in the column, of course.\label{fn:ppts-npaired2}}
% 
% You have to notice
% \Marginpar{\sf Marginal note from column-2.}
% the first paragraph does not start from the page top
% but above it we have some space of exactly same size as the \preenv{}
% shown in the left \parapag{}e.  Therefore, the top of the first paragraphs
% in all columns are aligned.  The marginal note given in the first line of
% this paragraph goes to the right margin of this page because of the
% \!\marginparthreshold! setting and the parity of this page.  Now we have a
% \!\switchcolumn! to the next column-3.
% \par\endgroup
% \begin{figure*}\nosv
% \def\arraystretch{0.8}
% \centerline{\begin{tabular}[b]{|c|}\hline
%     \hbox to.9\textwidth{}\\
%     \sf page-wise figure given in column-2\\
%     \\\hline
%     \end{tabular}}
% \caption{A Page-Wise Figure}
% \end{figure*}
% 
% \switchcolumn
% \begingroup\sl
% This is the first paragraph
% \Marginpar{\sl Marginal note from column-3.}
% in the last rightmost column-3 whose width is equal to that of the column-2.
% The marginal note given in the first line goes to right and does not
% conflict with that from the column-2.  We are now going back to the
% column-0 by a {\rm\!\switchcolumn!|*|} with a \mctext.
% \endgroup
% 
% \switchcolumn*[\subsection*{A Spanning Text: though this is wider than the
% page width, this text does not span the boundary between the left and
% right parallel-pages.}]
% 
% We have come back to this column-0.  The space above the \mctext{} is due
% to the \sync{}ation because two paragraphs in the column-2 are
% significantly taller in total than the paragraphs in other columns.  As
% the spanning text itself says, it cannot extend to the right \parapag{}e.
% The author puts dummy lines to go to the page bottom.\\
% \Dotfill\\ \Dotfill\\ \Dotfill\\ \Dotfill\\ \Dotfill\\ \Dotfill\\
% \Dotfill\\ \Dotfill\par
% 
% Now we will have a page break shortly.  You \Emph{will not} be surprised
% by seeing this column \Emph{is still in the left \parapag{}e after the
% break.}  This is because the feature |c| is \Emph{not effective in
% \npaired{} \parapag{}ing.}  The other feature |p| \Emph{consistently makes
% the left outside margins of this and the previous page in which this
% column resides} wider than the right inside margins.
% \label{page:ppts-npaired2}
% 
% \switchcolumn
% \begingroup\it
% We have restarted this column-1.  This paragraph has a
% footnote\footnotemark*[-1] as shown below.\\
% \Dotfill\\ \Dotfill\\ \Dotfill\\ \Dotfill\\ \Dotfill\\ \Dotfill\\
% \Dotfill\\ \Dotfill\\ \Dotfill\\ \Dotfill\\ \Dotfill\par
% 
% After the page break below, this column also \Emph{stays in the left page}
% together with the column-0
% \Marginpar{\it Another marginal note from column-1.}
% and is placed \Emph{inside (right)} in the page, as well as the marginal
% note in this \Emph{left} page \Emph{still} in the outside margin.
% \par\endgroup
% 
% \switchcolumn
% \begingroup\sf
% We have a few other materials not shown in right \parapag{}es.  The space
% above this paragraph is for the \mctext{} placed in the left \parapag{}e.
% The \Scfnote{} given here\footnotemark{} is also not in this page but in
% the left.  Finally, the author has put a page-wise figure spanning columns
% just before \!\switchcolumn! by which we left this column, but it will be
% in the \Emph{left} page \Emph{\pageref{page:ppts-npaired2}} together with
% column-0 and 1.\\
% \Dotfill\\ \Dotfill\\ \Dotfill\\ \Dotfill\\ \Dotfill\\ \Dotfill\par
% 
% Though the footnote numbered \Emph{\ref{fn:ppts-npaired2}} goes to the
% left page, its space and that of \Emph{\ref{fn:ppts-npaired1}} make this
% and the next columns shorter in the previous page.  Similarly, we have a
% space above for the page-wise figure shown in the \Emph{left} page.
% \par\endgroup
% 
% \switchcolumn
% \begingroup\sl
% As expected, this line is aligned to the first line of the paragraph in
% the column-2 as well as those in column-0 and 1.  It is also consistent
% the first lines including that of this paragraph are not indented because
% the \mctext{} is given by {\rm\!\subsection!|*|} which makes first
% paragraphs unindented.\\
% \Dotfill\\ \Dotfill\\ \Dotfill\\ \Dotfill\\ \Dotfill\\ \Dotfill\\
% \Dotfill\\ \Dotfill\par
%
% After the page break we will have shortly, this column \Emph{is kept being
% the rightmost in the right \parapag{}e}, as you are seeing now,
% \Marginpar{\sl Another marginal note from column-3.}
% \Emph{and} still outermost as well as the marginal note in the outside
% \Emph{right} margin.
% \endgroup
% \end{paracol}
% \Hrule
% 
% As the \postenv{} in Section~\ref{sec:ppts-paired} is, this paragraph
% being the \postenv{} of the \npaired{} \parapag{}es appears only in the
% \parapag{}e in which the column-0 belongs to, and thus in the left
% \parapag{}e in this case.
% \endinput
