% \subsection{Prerequisites\\前提条件}
%
^^A % \addvspace{\parskip}
% ^^A\begin{parcolumns}[rulebetween=true,colwidths={1=.6\linewidth}]{2}
{\app{Pygments} is written in Python, so make sure that you have Python 2.6 or later installed on your system.  This may be easily checked from the command line:}
% \par{\app{Pygments} 是用 Python 写的,所以请确保您的系统上安装了 Python 2.6 或更高版本\footnotemark:}
^^A \app{Pygments} 是用 Python 编写的,所以请确保你的系统安装了 Python 2.6 或更高版本。 这可以很容易地从命令行中检查。你可以很容易地从命令行查看版本号\footnote{译注:最新的使用pip3安装,2版本的安装不了了!}:
^^A% \end{parcolumns}
%\footnotetext{译注:最新的使用pip3安装,2版本的安装不了了!}
% \begin{Verbatim}[gobble=3,commandchars=\\\{\}]
%   \$ python --version
%   Python 2.7.5
% \end{Verbatim}
%
^^A % \addvspace{\parskip}
% ^^A\begin{parcolumns}[rulebetween=true,colwidths={1=.6\linewidth}]{2}
{If you don't have Python installed, you can download it from the \href{http://www.python.org/download/}{Python website} or use your operating system's package manager.}
% \par{如果您没有安装 Python,则可以从 \href{http://www.python.org/download/}{Python 官网}下载,或使用操作系统的软件包管理器。}
^^A 如果您没有安装 Python,可以去\href{http://www.python.org/download/}{Python 网站} 或使用你的操作系统的软件包管理器下载。
^^A% \end{parcolumns}
%
^^A % \addvspace{\parskip}
% ^^A\begin{parcolumns}[rulebetween=true,colwidths={1=.6\linewidth}]{2}
{Some Python distributions include \pkg{Pygments} (see some of the options under ``Alternative Implementations'' on the Python site).  %
Otherwise, you will need to install \pkg{Pygments} manually. %
This may  be done by installing \href{http://pypi.python.org/pypi/setuptools}{\app{setuptools}}, which facilitates the distribution of Python applications.  You can then install \app{Pygments} using the following command:}
% \par{一些 Python 发行版包括 \pkg{Pygments}(请参阅 Python 站点下“替代实现”选项中的一些选项)。否则,您需要手动安装 \pkg{Pygments}。这可以通过安装 \href{http://pypi.python.org/pypi/setuptools}{\app{setuptools}} 完成,它可以促进 Python 应用程序的分发。然后,您可以使用以下命令安装 \app{Pygments}:}
^^A 一些 Python 发行版包含了\pkg{Pygments}(请参阅 Python 站点上 ``Alternative Implementations'' 下的一些选项)。%
^^A 否则,你将需要手动安装 \pkg{Pygments}。
^^A 可以通过 \href{http://pypi.python.org/pypi/setuptools}{\app{setuptools}} 来安装,它可以帮助我们更简单的创建和分发Python包。可以使用以下命令安装 \app{Pygments}:
^^A% \end{parcolumns}
% \begin{Verbatim}[gobble=3,commandchars=\\\{\}]
%   \$ sudo easy_install Pygments
% \end{Verbatim}
^^A % \addvspace{\parskip}
% ^^A\begin{parcolumns}[rulebetween=true,colwidths={1=.6\linewidth}]{2}
% \par{Under Windows, you will not need the |sudo|, but may need to run the command prompt as administrator.  \pkg{Pygments} may also be installed with |pip|:}
% \par{在 Windows 下,您不需要使用 |sudo|,但可能需要以管理员身份运行命令提示符。也可以使用 |pip| 安装 \pkg{Pygments}:}
^^A 在 Windows 下,你不需要 |sudo|,但可能需要以管理员身份运行命令提示符。 \pkg{Pygments} 也可以用 |pip| 安装。
^^A% \end{parcolumns}
% \begin{Verbatim}[gobble=3,commandchars=\\\{\}]
%   \$ pip install Pygments
% \end{Verbatim}
%
^^A % \addvspace{\parskip}
% ^^A\begin{parcolumns}[rulebetween=true,colwidths={1=.6\linewidth}]{2}
{If you already have \app{Pygments} installed, be aware that the latest version is recommended (at least 1.4 or later).  Some features, such as |escapeinside|, will only work with 2.0+.  \pkg{minted} may work with versions as early as 1.2, but there are no guarantees.}
% \par{如果您已经安装了 \app{Pygments},请注意推荐使用最新版本(至少为 1.4 或更高版本)。一些功能,如 |escapeinside|,仅适用于 2.0+。 \pkg{minted} 可能与早至 1.2 的版本一起使用,但没有保证。}
^^A 如果你已经安装了 \app{Pygments},请注意,建议使用最新版本(至少1.4或更高版本)。有些功能,如 |escapeinside|,只能在2.0+
^^A %以上
^^A 版本中使用。 \pkg{minted}可能会在1.2的版本下工作,但不能保证。
^^A% \end{parcolumns}
%
%
% \subsection{Required packages\\必要的依赖包}
%
^^A % \addvspace{\parskip}
% ^^A\begin{parcolumns}[rulebetween=true,colwidths={1=.6\linewidth}]{2}
{\pkg{minted} requires that the following packages be available and reasonably up to date on your system.  All of these ship with recent \TeX\ distributions.}
% \par{\pkg{minted} 要求您的系统上可用并且相对最新的以下包。所有这些都随最近的 \TeX\ 发行版一起发布。}
^^A \pkg{minted}要求你的系统上有以下宏包,并且是合理的最新版本。 所有这些都是在最近的 \TeX\ 发行版中提供的。
^^A% \end{parcolumns}
%
% \begin{multicols}{3}
% \begingroup
% \setlength\parskip{0pt}
% \setlength\topsep{0pt}
% \begin{list}{\textrm{\labelitemi}}{\ttfamily}
%   \item keyval
%   \item kvoptions
%   \item fancyvrb
%   \item fvextra
%   \item upquote
%   \item float
%   \item ifthen
%   \item calc
%   \item ifplatform
%   \item pdftexcmds
%   \item etoolbox
%   \item xstring
%   \item xcolor
%   \item lineno
%   \item framed
%   \item shellesc (for luatex 0.87+)
%   \item catchfile
%
%   ~
% \end{list}
% \endgroup
% \end{multicols}
%
%
% \subsection{Installing \pkg{minted}\\安装\pkg{minted}}
% \label{sec:installing:installing}
%
^^A % \addvspace{\parskip}
% ^^A\begin{parcolumns}[rulebetween=true,colwidths={1=.6\linewidth}]{2}
{You can probably install \pkg{minted} with your \TeX\ distribution's package manager.  Otherwise, or if you want the absolute latest version, you can install it manually by following the directions below.}
% \par{你可以使用你的\TeX 发行版的软件包管理器安装\pkg{minted}。否则,如果你想要最新版本,你可以按照以下指示手动安装。}
^^A 你可以用你的\TeX\ 发行版的包管理器安装 \pkg{minted}。 或者,如果你想要最新的版本,你可以按照下面的说明手动安装。
^^A% \end{parcolumns}
%
^^A % \addvspace{\parskip}
% ^^A\begin{parcolumns}[rulebetween=true,colwidths={1=.6\linewidth}]{2}
{You may download |minted.sty| from the
\href{https://github.com/gpoore/minted}{project's homepage}.  %
We have to install the file so that \TeX\ is able to find it.%
In order to do that, please refer to the
\href{http://www.tex.ac.uk/cgi-bin/texfaq2html?label=inst-wlcf}{\TeX{} FAQ}.%
If you just want to experiment with the latest version, you could locate your current |minted.sty| in your \TeX\ installation and replace it with the latest version.  Or you could just put the latest |minted.sty| in the same directory as the file you wish to use it with.}
% \par{你可以从\href{https://github.com/gpoore/minted}{项目主页}下载|minted.sty|。我们需要安装该文件以便\TeX 能够找到它。^^A为了做到这一点,
请参考\href{http://www.tex.ac.uk/cgi-bin/texfaq2html?label=inst-wlcf}{\TeX{}常见问题}来做到这一点。如果你只想尝试最新版本,你可以在\TeX 安装中找到当前的|minted.sty|并将其替换为最新版本。或者你也可以将最新的|minted.sty|放在与你要使用它的文件相同的目录中。}
^^A TODO 链接在左右栏都出现...
^^A 你可以从 \href{https://github.com/gpoore/minted}{minted项目的主页} 下载 |minted.sty|。
^^A 我们必须安装该文件,以便让 \TeX\ 能够找到它。
^^A 为了做到这一点,请参考
^^A \href{http://www.tex.ac.uk/cgi-bin/texfaq2html?label=inst-wlcf}{\TeX{} FAQ}。
^^A 如果你只是想试验一下最新的版本,你可以在你的 \TeX\ 安装路径中找到你当前的 |minted.sty| ,然后用最新的版本替换它。 或者你可以把最新的|minted.sty|放在与你想使用的文件相同的目录中。
^^A% \end{parcolumns}