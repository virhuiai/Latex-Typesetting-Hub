% \iffalse meta-comment
%
% Copyright (C) 2013-2021 by Geoffrey M. Poore <gpoore@gmail.com>
% Copyright (C) 2010-2011 by Konrad Rudolph <konrad_rudolph@madrat.net>
% ---------------------------------------------------------------------------
% This work may be distributed and/or modified under the
% conditions of the LaTeX Project Public License, either version 1.3
% of this license or (at your option) any later version.
% The latest version of this license is in
%   http://www.latex-project.org/lppl.txt
% and version 1.3 or later is part of all distributions of LaTeX
% version 2005/12/01 or later.
%
% Additionally, the project may be distributed under the terms of the 3-Clause
% ("New") BSD license:  http://opensource.org/licenses/BSD-3-Clause.
%
% This work has the LPPL maintenance status `maintained'.
%
% The Current Maintainer of this work is Geoffrey M. Poore.
%
% This work consists of the files minted.dtx and minted.ins
% and the derived file minted.sty.
%
%
% \fi
%
% \iffalse
%<*driver>
\ProvidesFile{minted.dtx}
%</driver>
%<package>\NeedsTeXFormat{LaTeX2e}
%<package>\ProvidesPackage{minted}
%<*package>
    [2021/12/24 v2.6 Yet another Pygments shim for LaTeX]
%</package>
%<*driver>
%%%%%%%%%%%%%%%{fontspec}%%%%%%%%%%%%%%%
%将no-math选项传递给fontspec宏包,该选项禁用了使用fontspec宏包中的数学字体功能。
\PassOptionsToPackage{no-math}{fontspec}
%%%%%%%%%%%%%%%{xeCJK}%%%%%%%%%%%%%%%
%将AutoFakeBold和AutoFakeSlant选项传递给xeCJK宏包,这两个选项让xeCJK宏包自动产生伪粗体和伪斜体效果。
\PassOptionsToPackage{AutoFakeBold=true,AutoFakeSlant=true}{xeCJK}
\documentclass{ltxdoc}
%^^A \usepackage[landscape,
%^^A top=1.5cm,
%^^A bottom=1.5cm,
%^^A left=5.5cm, right=0.5cm
%^^A ]{geometry}
\usepackage[scheme=plain,fontset=none
,scheme=chinese^^A 中文方案
,fontset=none^^A 不使用默认的字体设置
,space=auto^^A 自动调整中英文间距
]{ctex}
\setCJKmainfont{方正书宋_GBK}^^A 方正书宋_GBK.TTF  设置文本的中文有衬线字体为“方正书宋_GBK”
\setCJKsansfont{方正黑体简体}^^A 方正黑体_GBK.TTF  设置文本的中文无衬线字体为“方正黑体简体”
\setCJKmonofont{方正书宋简体}^^A 方正仿宋_GBK.TTF  设置文本的中文等宽字体为“方正书宋简体”    
\usepackage{pdfcolparcolumns}^^A <-- improved version of parcolumns
\EnableCrossrefs
\CodelineIndex
%\DisableCrossrefs
%\RecordChanges
%\OnlyDescription
\usepackage[utf8]{inputenc}
\usepackage[T1]{fontenc}
\usepackage{lmodern}
\usepackage{dingbat}
\usepackage{microtype}

\usepackage[svgnames]{xcolor}
\usepackage{textcomp}
^^A cache, 
\usepackage[langlinenos
,outputdir=/Volumes/RamDisk]{minted}

\usepackage{environ}
\usepackage{graphicx}
\usepackage{multicol}
\usepackage{hyperref}


\makeatletter
% The typesetting for macrocode doesn't use \@noligs, which upquote modifies.
% So apply the upquote fix to \verbatim@nolig@list as well, which is in macrocode.
\begingroup
\catcode`'=\active
\catcode``=\active
\g@addto@macro\verbatim@nolig@list{%
  \let'\textquotesingle
  \let`\textasciigrave
  \ifx\encodingdefault\upquote@OTone
  \ifx\ttdefault\upquote@cmtt
    \def'{\char13 }%
    \def`{\char18 }%
  \fi\fi}
\endgroup

% Create a short verbatim pipe that handles quotation marks properly
\begingroup
\catcode`\|=\active
\gdef\pipe@active@verbatim{%
  \begingroup
  \let\do\@makeother\dospecials
  \catcode`\|=\active
  \catcode`\`=\active
  \catcode`\'=\active
  \catcode`\<=\active
  \catcode`\>=\active
  \catcode`\-=\active
  \catcode`\,=\active
  \catcode`\ =\active
  \pipe@active@verbatim@i}
\gdef\pipe@active@verbatim@i#1|{%
  \endgroup
  \begingroup
  \def\FV@SV@pipe@active@verbatim{%
    \FV@Gobble
    \expandafter\FV@ProcessLine\expandafter{#1}}%
  %\let\FV@BeginVBox\relax
  %\let\FV@EndVBox\relax
  %\def\FV@BProcessLine##1{\FancyVerbFormatLine{##1}}%
  \BUseVerbatim{pipe@active@verbatim}%
  \endgroup}
\AtBeginDocument{\let|\pipe@active@verbatim}
\endgroup

\newcommand{\changestext}{}
\NewEnviron{changelog}[2]{%
    \g@addto@macro\changestext{\item[#1] (#2) \begin{itemize}}%
    \expandafter\g@addto@macro\expandafter\changestext\expandafter{\BODY}%
    \g@addto@macro\changestext{\end{itemize}}%
}
\newcommand{\PrintChangelog}{%
    \section*{Version History}%
    \addcontentsline{toc}{section}{Version History}
    \label{sec:version-history}
    \begin{description}%
    \changestext
    \end{description}%
}

\def\MacroFont{%
  \fontencoding\encodingdefault%
  \fontfamily\ttdefault%
  \fontseries\mddefault%
  \fontshape\updefault%
  \small}

\definecolor{minted@mint}{HTML}{0B610B}
\definecolor{minted@samplebg}{HTML}{F0F0E0}
\colorlet{minted@linkcolor}{minted@mint}
%
\def\PrintDescribeMacro#1{\strut \MacroFont\textcolor{minted@linkcolor}{\string #1\ }}
\let\PrintDescribeEnv\PrintDescribeMacro
\let\PrintMacroName\PrintDescribeMacro
\let\PrintEnvName\PrintDescribeEnv
\def\theCodelineNo{\textcolor{minted@linkcolor}{\sffamily\footnotesize\oldstylenums{\arabic{CodelineNo}}}}
%
\hypersetup{
  pdftitle=The minted package: Highlighted source code in LaTeX,
  pdfauthor=Geoffrey M. Poore,
  pdfsubject={Minted LaTeX package manual},
  colorlinks=true,
  allcolors=minted@linkcolor,
}
%
% Shamelessly stolen from http://blog.karssen.org/2009/11/15/a-latex-example-environment/
\newenvironment{example}
  {\VerbatimEnvironment
   \begin{VerbatimOut}[gobble=3]{example.out}}
  {\end{VerbatimOut}%
   \vspace{1ex}%
   \setlength{\parindent}{0pt}%
   \fbox{\begin{minipage}{0.5\linewidth}%
     \inputminted[resetmargins]{latex}{example.out}%
   \end{minipage}%
   \hspace{0.05\linewidth}%
   \begin{minipage}{0.45\linewidth-2\fboxsep}%
     \input{example.out}%
   \end{minipage}%
   \vspace{1ex}}}

\newenvironment{longexample}
  {\VerbatimEnvironment
   \begin{VerbatimOut}[gobble=3]{example.out}}
  {\end{VerbatimOut}%
   \vspace{1ex}%
   \setlength{\parindent}{0pt}%
   \fbox{\begin{minipage}{\linewidth-2\fboxsep}%
     \inputminted[resetmargins]{latex}{example.out}%
     ~\hrulefill~
     \input{example.out}%
   \end{minipage}%
   \vspace{1ex}}}

\def\minted@printopt#1(#2) (#3){%
  \vspace{0.1in}%
  \leavevmode%
  \marginpar{\raggedleft\texttt{\textcolor{minted@linkcolor}{\small #1}}\ }%
  \kern-\parindent\textsf{(#2)}\hfill(default: \texttt{#3})\\}

\newenvironment{optionlist}%
 {%
  ~\par\vspace{-14pt}%
  \def\pipechar{|}
  \let\|\pipechar
  \newcommand*\mintednext{}%
  \renewcommand*\item[1][]{%
    \mintednext%
    \renewcommand*\mintednext{\par}%
    \minted@printopt##1%
    \ignorespaces}}
 {%
  \par}

\edef\hashchar{\string#}
\makeatother

\newcommand{\hide}[1]{}

\setlength{\parindent}{0pt}
\addtolength{\parskip}{0.5\baselineskip}

\begin{document}
^^A %%%%%%%%%%%%%%%%%%%%%%%%%%%%%%%%%%%%%%%%%%%%%%%%%%%%%%%%%%%%%%%%
^^A %%%%%%指定 include 的路径
^^A %%%%%%    设置
^^A %%%%%%%%%%%%%%%%%%%%%%%%%%%%%%%%%%%%%%%%%%%%%%%%%%%%%%%%%%%%%%%%
\makeatletter
\providecommand*\input@path{}
\newcommand*\addinputpath[1]{\expandafter\def\expandafter\input@path\expandafter{\input@path#1}}
\makeatother

\addinputpath{
{/Volumes/RamDisk/}
{/Volumes/RamDisk/_minted-minted-cn}
{/Users/virhuiai/hlProjects/Latex-Typesetting-Hub/宏包文档翻译/minted}
}
^^A xelatex --output-directory=/Volumes/RamDisk/ -synctex=1 -shell-escape   /Users/virhuiai/hlProjects/Latex-Typesetting-Hub/宏包文档翻译/minted/minted-cn.dtx

\DocInput{minted-cn.dtx}
\PrintIndex
\end{document}
%</driver>
% \fi
%
% \CheckSum{2593}
%
% \CharacterTable
%  {Upper-case    \A\B\C\D\E\F\G\H\I\J\K\L\M\N\O\P\Q\R\S\T\U\V\W\X\Y\Z
%   Lower-case    \a\b\c\d\e\f\g\h\i\j\k\l\m\n\o\p\q\r\s\t\u\v\w\x\y\z
%   Digits        \0\1\2\3\4\5\6\7\8\9
%   Exclamation   \!     Double quote  \"     Hash (number) \#
%   Dollar        \$     Percent       \%     Ampersand     \&
%   Acute accent  \'     Left paren    \(     Right paren   \)
%   Asterisk      \*     Plus          \+     Comma         \,
%   Minus         \-     Point         \.     Solidus       \/
%   Colon         \:     Semicolon     \;     Less than     \<
%   Equals        \=     Greater than  \>     Question mark \?
%   Commercial at \@     Left bracket  \[     Backslash     \\
%   Right bracket \]     Circumflex    \^     Underscore    \_
%   Grave accent  \`     Left brace    \{     Vertical bar  \|
%   Right brace   \}     Tilde         \~}
%
%
%
%
% \begin{changelog}{v2.6}{2021/12/24}
% \item \texttt{autogobble} automatically uses \texttt{python} or \texttt{python3} executables, depending on availability, instead of requiring \texttt{python}.  A custom executable can be specified by redefining \texttt{\string\MintedPython} (\#277, \#287).
% \item Fixed \texttt{autogobble} compatibility with \texttt{fancyvrb} 4.0+ (\#315, \#316).
% \item Pygments style names may now contain arbitrary non-whitespace characters.  Previously, style names containing digits and some punctuation characters were incompatible (\#210, \#294, \#299, \#317).  Pygments macros are now only defined just before use locally within \texttt{minted} commands and environments, rather than globally.  Pygments macros now always use a \texttt{\string\PYG} prefix regardless of style, rather than a prefix of the form \texttt{\string\PYG<style>} (for example, what was previously \texttt{\string\PYGdefault} is now simply \texttt{\string\PYG}).
% \item Removed Python-based MD5 hashing for XeTeX, which was necessary before XeTeX added \texttt{\string\mdfivesum} in 2017.
% \item The default for \texttt{stripnl} is now \texttt{false}, so that original code is preserved exactly by default (\#198).
% \item Added support for \texttt{fontencoding} option from \texttt{fvextra} (\#208).
% \item Added note to FAQ about getting \texttt{texi2pdf} to work with \texttt{minted} given \texttt{texi2pdf}'s assumptions about temp files (\#186).
% \item Reimplemented \texttt{bgcolor} option to be compatible with \texttt{color} package.
% \end{changelog}
%
%
% \begin{changelog}{v2.5}{2017/07/19}
% \item The default placement for the \texttt{listing} float is now \texttt{tbp} instead of \texttt{h}, to parallel \texttt{figure} and \texttt{table} and also avoid warnings caused by \texttt{h} (\#165).  The documentation now contains information about changing default placement.  The \texttt{float} package is no longer loaded when the \texttt{newfloat} package option is used.
% \item Added support for \texttt{*nchars} options from \texttt{fvextra} v1.3 that allow setting \texttt{breaklines}-related indentation in terms of a number of characters, rather than as a fixed dimension.
% \item Fixed incompatibility with \texttt{babel magyar} (\#158).
% \item Added support for \texttt{beamer} overlays with \texttt{beameroverlays} option (\#155).
% \item Comments in the Pygments \LaTeX\ style files no longer appear as literal text when \texttt{minted} is used in \texttt{.dtx} files (\#161).
% \item \texttt{autogobble} now works with package option \texttt{kpsewhich} (\#151).  Under Windows, the \texttt{kpsewhich} option no longer requires PowerShell.
% \item Fixed a bug that prevented \texttt{finalizecache} from working with \texttt{outputdir} (\#149).
% \item Fixed a bug with \texttt{firstline} and \texttt{lastline} that prevented them from working with the \texttt{minted} environment (\#145).
% \item Added note on \texttt{breqn} conflicts to FAQ (\#163).
% \end{changelog}
%
%
% \begin{changelog}{v2.4.1}{2016/10/31}
% \item Single quotation marks in \texttt{\string\jobname} are now replaced with underscores in \texttt{\string\minted@jobname} to prevent quoting errors (\#137).
% \item The \texttt{autogobble} option now takes \texttt{firstline} and \texttt{lastline} into account (\#130).
% \item Fixed \texttt{numberblanklines}, which had been lost in the transition to v2.0+ (\#135).
% \end{changelog}
%
%
% \begin{changelog}{v2.4}{2016/07/20}
% \item Line breaking and all associated options are now completely delegated to \texttt{fvextra}.
% \item Fixed a bug from v2.2 that could cause the first command or environment to vanish when \texttt{cache=false} (related to work on \texttt{\string\MintedPygmentize}).
% \end{changelog}
%
%
% \begin{changelog}{v2.3}{2016/07/14}
% \item The \texttt{fvextra} package is now required.  \texttt{fvextra} extends and patches \texttt{fancyvrb}, and includes improved versions of \texttt{fancyvrb} extensions that were formerly in \texttt{minted}.
% \item As part of \texttt{fvextra}, the \texttt{upquote} package is always loaded.  \texttt{fvextra} brings the new option \texttt{curlyquotes}, which allows curly single quotation marks instead of the literal backtick and typewriter single quotation mark produced by \texttt{upquote}.  This allows the default \texttt{upquote} behavior to be disabled when desired.
% \item Thanks to \texttt{fvextra}, the options \texttt{breakbefore}, \texttt{breakafter}, and \texttt{breakanywhere} are now compatible with non-ASCII characters under pdfTeX (\#123).
% \item Thanks to \texttt{fvextra}, \texttt{obeytabs} no longer causes lines in multi-line comments or strings to vanish (\#88), and is now compatible with \texttt{breaklines} (\#99).  \texttt{obeytabs} will now always give correct results with tabs used for indentation.  However, tab stops are not guaranteed to be correct for tabs in the midst of text.
% \item \texttt{fvextra} brings the new options \texttt{space}, \texttt{spacecolor}, \texttt{tab}, and \texttt{tabcolor} that allow these characters and their colors to be redefined (\#98).  The tab may now be redefined to a flexible-width character such as \texttt{\string\rightarrowfill}.  The visible tab will now always be black by default, instead of changing colors depending on whether it is part of indentation for a multiline string or comment.
% \item \texttt{fvextra} brings the new options \texttt{highlightcolor} and \texttt{highlightlines}, which allow single lines or ranges of lines to be highlighted based on line number (\#124).
% \item \texttt{fvextra} brings the new options \texttt{numberfirstline}, \texttt{stepnumberfromfirst}, and \texttt{stepnumberoffsetvalues} that provide better control over line numbering when \texttt{stepnumber} is not 1.
% \item Fixed a bug from v2.2.2 that prevented \texttt{upquote} from working.
% \end{changelog} 
%
%
% \begin{changelog}{v2.2.2}{2016/06/21}
% \item Fixed a bug introduced in v2.2 that prevented setting the Pygments style in the preamble.  Style definitions are now more compatible with using \texttt{\string\MintedPygmentize} to call a custom \texttt{pygmentize}.
% \end{changelog}
%
%
% \begin{changelog}{v2.2.1}{2016/06/15}
% \item The \texttt{shellesc} package is loaded before \texttt{ifplatform} and other packages that might invoke \texttt{\string\write18} (\#112).
% \item When caching is enabled, XeTeX uses the new \texttt{\string\mdfivesum} macro from TeX Live 2016 to hash cache content, rather than using \texttt{\string\ShellEscape} with Python to perform hashing.
% \end{changelog}
%
%
% \begin{changelog}{v2.2}{2016/06/08}
% \item All uses of \texttt{\string\ShellEscape} (\texttt{\string\write18}) no longer wrap file names and paths with double quotes.  This allows a cache directory to be specified relative to a user's home directory, for example, \texttt{\string~/minted\_cache}.  \texttt{cachedir} and \texttt{outputdir} paths containing spaces will now require explicit quoting of the parts of the paths that contain spaces, since \texttt{minted} no longer supplies quoting.  See the updated documentation for examples (\#89).
% \item Added \texttt{breakbefore}, \texttt{breakbeforegroup}, \texttt{breakbeforesymbolpre}, and \texttt{breakbeforesymbolpost}.  These parallel \texttt{breakafter*}.  It is possible to use \texttt{breakbefore} and \texttt{breakafter} for the same character, so long as \texttt{breakbeforegroup} and \texttt{breakaftergroup} have the same setting (\#117).
% \item Added package options \texttt{finalizecache} and \texttt{frozencache}.  These allow the cache to be prepared for (\texttt{finalizecache}) and then used (\texttt{frozencache}) in an environment in which \texttt{-shell-escape}, Python, and/or Pygments are not available.  Note that this only works if \texttt{minted} content does not need to be modified, and if no settings that depend on Pygments or Python need to be changed (\#113).
% \item Style names containing hyphens and underscores (\texttt{paraiso-light}, \texttt{paraiso-dark}, \texttt{algol\_nu}) now work (\#111).
% \item The \texttt{shellesc} package is now loaded, when available, for compatibility with LuaTeX 0.87+ (TeX Live 2016+, etc.).  \texttt{\string\ShellEscape} is now used everywhere instead of \texttt{\string\immediate\string\write18}.  If \texttt{shellesc} is not available, then a \texttt{\string\ShellEscape} macro is created.  When \texttt{shellesc} is loaded, there is a check for versions before v0.01c to patch a bug in v0.01b (present in TeX Live 2015) (\#112).
% \item The \texttt{bgcolor} option now uses the \texttt{snugshade*} environment from the \texttt{framed} package, so \texttt{bgcolor} is now compatible with page breaks.  When \texttt{bgcolor} is in use, immediately preceding text will no longer push the \texttt{minted} environment into the margin, and there is now adequate spacing from surrounding text (\#121).
% \item Added missing support for \texttt{fancyvrb}'s \texttt{labelposition} (\#102).
% \item Improved fix for TikZ externalization, thanks to Patrick Vogt (\#73).
% \item Fixed \texttt{breakautoindent}; it was disabled in version 2.1 due to a bug in \texttt{breakanywhere}.
% \item Properly fixed handling of \texttt{\string\MintedPygmentize} (\#62).
% \item Added note on incompatibility of \texttt{breaklines} and \texttt{obeytabs} options.  Trying to use these together will now result in a package error (\#99).  Added note on issues with \texttt{obeytabs} and multiline comments (\#88).  Due to the various \texttt{obeytabs} issues, the docs now discourage using \texttt{obeytabs}.
% \item Added note to FAQ on \texttt{fancybox} and \texttt{fancyvrb} conflict (\#87).
% \item Added note to docs on the need for \texttt{\string\VerbatimEnvironment} in environment definitions based on \texttt{minted} environments.
% \end{changelog}
%
%
% \begin{changelog}{v2.1}{2015/09/09}
% \item Changing the highlighting style now no longer involves re-highlighing code.  Style may be changed with almost no overhead.
% \item Improved control of automatic line breaks.  New option \texttt{breakanywhere} allows line breaks anywhere when \texttt{breaklines=true}.  The pre-break and post-break symbols for these types of breaks may be set with \texttt{breakanywheresymbolpre} and \texttt{breakanywheresymbolpost} (\#79).  New option \texttt{breakafter} allows specifying characters after which line breaks are allowed.  Breaks between adjacent, identical characters may be controlled with \texttt{breakaftergroup}.  The pre-break and post-break symbols for these types of breaks may be set with \texttt{breakaftersymbolpre} and \texttt{breakaftersymbolpost}.
% \item \texttt{breakbytoken} now only breaks lines between tokens that are separated by spaces, matching the documentation.  The new option \texttt{breakbytokenanywhere} allows for breaking between tokens that are immediately adjacent.  Fixed a bug in \texttt{\string\mintinline} that produced a following linebreak when \texttt{\string\mintinline} was the first thing in a paragraph and \texttt{breakbytoken} was true (\#77).
% \item Fixed a bug in draft mode option handling for \texttt{\string\inputminted} (\#75).
% \item Fixed a bug with \texttt{\string\MintedPygmentize} when a custom \texttt{pygmentize} was specified and there was no \texttt{pygmentize} on the default path (\#62).
% \item Added note to docs on caching large numbers of code blocks under OS~X (\#78).
% \item Added discussion of current limitations of \texttt{texcomments} (\#66) and \texttt{escapeinside} (\#70).
% \item PGF/Ti\textit{k}Z externalization is automatically detected and supported (\#73).
% \item The package is now compatible with \LaTeX\ files whose names contain spaces (\#85).
% \end{changelog}
%
%
% \begin{changelog}{v2.0}{2015/01/31}
% \item Added the compatibility package \texttt{minted1}, which provides the \pkg{minted} 1.7 code.  This may be loaded when 1.7 compatibility is required.  This package works with other packages that \texttt{\string\RequirePackage\{minted\}}, so long as it is loaded first.
% \item Moved all old \texttt{\string\changes} into \texttt{changelog}.
% \end{changelog}
%
%
% \begin{changelog}{Development releases for 2.0}{2014--January 2015}
% \item Caching is now on by default.
% \item Fixed a bug that prevented compiling under Windows when file names contained commas.
% \item Added \texttt{breaksymbolleft}, \texttt{breaksymbolsepleft}, \texttt{breaksymbolindentleft}, \texttt{breaksymbolright}, \texttt{breaksymbolsepright}, and \texttt{breaksymbolindentright} options.  \texttt{breaksymbol}, \texttt{breaksymbolsep}, and \texttt{breaksymbolindent} are now aliases for the correspondent \texttt{*left} options.
% \item Added \texttt{kpsewhich} package option. This uses \texttt{kpsewhich} to locate the files that are to be highlighted. This provides compatibility with build tools like \texttt{texi2pdf} that function by modifying \texttt{TEXINPUTS} (\#25).
% \item Fixed a bug that prevented \texttt{\string\inputminted} from working with \texttt{outputdir}.
% \item Added informative error messages when Pygments output is missing.
% \item Added \texttt{final} package option (opposite of \texttt{draft}).
% \item Renamed the default cache directory to \texttt{\_minted-<jobname>} (replaced leading period with underscore).  The leading period caused the cache directory to be hidden on many systems, which was a potential source of confusion.
% \item \texttt{breaklines} and \texttt{breakbytoken} now work with \texttt{\string\mintinline} (\#31).
% \item \texttt{bgcolor} may now be set through \texttt{\string\setminted} and \texttt{\string\setmintedinline}.
% \item When math is enabled via \texttt{texcomments}, \texttt{mathescape}, or \texttt{escapeinside}, space characters now behave as in normal math by vanishing, instead of appearing as literal spaces.  Math need no longer be specially formatted to avoid undesired spaces.
% \item In default value of \texttt{\string\listoflistingscaption}, capitalized ``Listings'' so that capitalization is consistent with default values for other lists (figures, tables, algorithms, etc.).
% \item Added \texttt{newfloat} package option that creates the \texttt{listing} environment using \texttt{newfloat} rather than \texttt{float}, thus providing better compatibility with the \texttt{caption} package (\#12).
% \item Added support for Pygments option \texttt{stripall}.
% \item Added \texttt{breakbytoken} option that prevents \texttt{breaklines} from breaking lines within Pygments tokens.
% \item \texttt{\string\mintinline} uses a \texttt{\string\colorbox} when \texttt{bgcolor} is set, to give more reasonable behavior (\#57).
% \item For PHP, \texttt{\string\mintinline} automatically begins with \texttt{startinline=true} (\#23).
% \item Fixed a bug that threw off line numbering in \texttt{minted} when \texttt{langlinenos=false} and \texttt{firstnumber=last}.  Fixed a bug in \texttt{\string\mintinline} that threw off subsequent line numbering when \texttt{langlinenos=false} and \texttt{firstnumber=last}.
% \item Improved behavior of \texttt{\string\mint} and \texttt{\string\mintinline} in \texttt{draft} mode.
% \item The \texttt{\string\mint} command now has the additional capability to take code delimited by paired curly braces \texttt{\{\}}.
% \item It is now possible to set options only for \texttt{\string\mintinline} using the new \texttt{\string\setmintedinline} command.  Inline options override options specified via \texttt{\string\setminted}.
% \item Completely rewrote option handling.  \pkg{fancyvrb} options are now handled on the \LaTeX\ side directly, rather than being passed to Pygments and then returned.  This makes caching more efficient, since code is no longer rehighlighted just because \pkg{fancyvrb} options changed.
% \item Fixed buffer size error caused by using \texttt{cache} with a very large number of files (\#61).
% \item Fixed \texttt{autogobble} bug that caused failure under some operating systems.
% \item Added support for \texttt{escapeinside} (requires Pygments 2.0+; \#38).
% \item Fixed issues with XeTeX and caching (\#40).
% \item The \texttt{upquote} package now works correctly with single quotes when using Pygments 1.6+ (\#34).
% \item Fixed caching incompatibility with Linux and OS X under xelatex (\#18 and \#42).
% \item Fixed \texttt{autogobble} incompatibility with Linux and OS X.
% \item \texttt{\string\mintinline} and derived commands are now robust, via \texttt{\string\newrobustcmd} from \texttt{etoolbox}.
% \item Unused styles are now cleaned up when caching.
% \item Fixed a bug that could interfere with caching (\#24).
% \item Added \texttt{draft} package option (\#39).  This typesets all code using \texttt{fancyvrb}; Pygments is not used.  This trades syntax highlighting for maximum speed in compiling.
% \item Added automatic line breaking with \texttt{breaklines} and related options (\#1).
% \item Fixed a bug with boolean options that needed a False argument to cooperate with \texttt{\string\setminted} (\#48).
% \end{changelog}
%
% \begin{changelog}{v2.0-alpha3}{2013/12/21}
% \item Added \texttt{autogobble} option.  This sends code through Python's \texttt{textwrap.dedent()} to remove common leading whitespace.
% \item Added package option \texttt{cachedir}.  This allows the directory in which cached content is saved to be specified.
% \item Added package option \texttt{outputdir}.  This allows an output directory for temporary files to be specified, so that the package can work with LaTeX's \texttt{-output-directory} command-line option.
% \item The \texttt{kvoptions} package is now required.  It is needed to process key-value package options, such as the new \texttt{cachedir} option.
% \item Many small improvements, including better handling of paths under Windows and improved key system.
% \end{changelog}
%
% \begin{changelog}{v2.0-alpha2}{2013/08/21}
% \item \texttt{\string\DeleteFile} now only deletes files if they do indeed exist.  This eliminates warning messages due to missing files.
% \item Fixed a bug in the definition of \texttt{\string\DeleteFile} for non-Windows systems.
% \item Added support for Pygments option \texttt{stripnl}.
% \item Settings macros that were previously defined globally are now defined locally, so that \texttt{\string\setminted} may be confined by \texttt{\string\begingroup...\string\endgroup} as expected.
% \item Macro definitions for a given style are now loaded only once per document, rather than once per command/environment.  This works even without caching.
% \item A custom script/executable may now be substituted for \texttt{pygmentize} by redefining \texttt{\string\MintedPygmentize}.
% \end{changelog}
%
%
% \begin{changelog}{v2.0alpha}{2013/07/30}
% \item Added the package option \texttt{cache}.  This significantly increases compilation speed by caching old output.  For example, compiling the documentation is around 5x faster.
% \item New inline command \texttt{\string\mintinline}.  Custom versions can be created via \texttt{\string\newmintinline}.  The command works inside other commands (for example, footnotes) in most situations, so long as the percent and hash characters are avoided.
% \item The new \texttt{\string\setminted} command allows options to be specified at the document and language levels.
% \item All extended characters (Unicode, etc.) supported by \texttt{inputenc} now work under the pdfTeX engine.  This involved using \texttt{\string\detokenize} on everything prior to saving.
% \item New package option \texttt{langlinenos} allows line numbering to pick up where it left off for a given language when \texttt{firstnumber=last}.
% \item New options, including \texttt{style}, \texttt{encoding}, \texttt{outencoding}, \texttt{codetagify}, \texttt{keywordcase}, \texttt{texcomments} (same as \texttt{texcl}), \texttt{python3} (for the \texttt{PythonConsoleLexer}), and \texttt{numbers}.
% \item \texttt{\string\usemintedstyle} now takes an optional argument to specify the style for a particular language, and works anywhere in the document.
% \item \texttt{xcolor} is only loaded if \texttt{color} isn't, preventing potential package clashes.
% \end{changelog}
%
%
% \begin{changelog}{1.7}{2011/09/17}
% \item Options for float placement added [2011/09/12]
% \item Fixed \texttt{tabsize} option [2011/08/30]
% \item More robust detection of the \texttt{-shell-escape} option [2011/01/21]
% \item Added the \texttt{label} option [2011/01/04]
% \item Installation instructions added [2010/03/16]
% \item Minimal working example added [2010/03/16]
% \item Added PHP-specific options [2010/03/14]
% \item Removed unportable flag from Unix shell command [2010/02/16]
% \end{changelog}
%
%
% \begin{changelog}{1.6}{2010/01/31}
% \item Added font-related options [2010/01/27]
% \item Windows support added [2010/01/27]
% \item Added command shortcuts [2010/01/22]
% \item Simpler versioning scheme [2010/01/22]
% \end{changelog}
%
%
% \begin{changelog}{0.1.5}{2010/01/13}
% \item Added \texttt{fillcolor} option [2010/01/10]
% \item Added float support [2010/01/10]
% \item Fixed \texttt{firstnumber} option [2010/01/10]
% \item Removed \texttt{caption} option [2010/01/10]
% \end{changelog}
%
%
% \begin{changelog}{0.0.4}{2010/01/08}
% \item Initial version [2010/01/08]
% \end{changelog}
%
%
% %\DoNotIndex{\newcommand,\newenvironment}
% %\DoNotIndex{\#,\$,\%,\&,\@,\\,\{,\},\^,\_,\~,\ }
% %\DoNotIndex{\@ne}
% %\DoNotIndex{\advance,\begingroup,\catcode,\closein}
% %\DoNotIndex{\closeout,\day,\def,\edef,\else,\empty,\endgroup}
% %\DoNotIndex{\begin,\end,\bgroup,\egroup}
% %\DoNotIndex{\@namedef,\@nameuse,=,\csname,\endcsname}
%
%
% \GetFileInfo{minted.sty}
%
% \newcommand\pkg[1]{\textsf{#1}}
% \newcommand\app[1]{\textsf{#1}}
% \setlength{\parindent}{0pt}
% \title{The \textcolor{minted@mint}{\pkg{minted}} package:\\Highlighted source code in \LaTeX}
% \author{Geoffrey M.\ Poore \\ \url{gpoore@gmail.com} \\ \href{https://github.com/gpoore/minted}{\texttt{github.com/gpoore/minted}} \\ ~\\ Originally created and maintained (2009--2013) by \\ Konrad Rudolph\and 翻译by\\{\tt virhuiai@qq.com}}
% \date{\fileversion~from \filedate}
%
% \maketitle
%
%\begin{abstract}
%\makebox[0pt]{}^^A 奇怪,不加点东西,编译不过,先这样吧!todo
%\begin{parcolumns}[rulebetween=true,colwidths={1=.6\linewidth}]{2}
%\colchunk{\noindent\pkg{minted} is a package that facilitates expressive syntax highlighting
using the powerful \app{Pygments} library. The package also provides options to
customize the highlighted source code output.}
%\colchunk{\noindent\pkg{minted} 是一个利用强大的 \app{Pygments} 库实现语法高亮的包。该包还提供了选项来自定义高亮的源代码输出。}
^^A \colchunk{\noindent\pkg{minted}是一个设置代码高亮的宏包,它使用了强大的Pygments库。此包有提供了高亮源代码输出的自定义选项。}
% \end{parcolumns}
% \end{abstract}
%
% \vspace{2in}
%
% \subsection*{License\\许可证}
% \begin{parcolumns}[rulebetween=true,colwidths={1=.6\linewidth}]{2}
% \colchunk{\href{http://www.latex-project.org/lppl.txt}{LaTeX Project Public License (LPPL)} version 1.3.}
% \colchunk{\href{http://www.latex-project.org/lppl.txt}{LaTeX 项目公共许可证(LPPL)}版本1.3。}
% \end{parcolumns}
%
% \begin{parcolumns}[rulebetween=true,colwidths={1=.6\linewidth}]{2}
% \colchunk{Additionally, the project may be distributed under the terms of the 3-Clause
% (``New'') BSD license:  \url{http://opensource.org/licenses/BSD-3-Clause}.}
% \colchunk{此外,该项目可以按照3条款(“新”)BSD许可证的条款分发:\url{http://opensource.org/licenses/BSD-3-Clause}。}
% \end{parcolumns}
%
% \pagebreak
%
%   \tableofcontents
%
% \fvset{
%   codes={\catcode`\%=9}, ^^A Ignore initial |%|
%   numbersep=5pt,
% }
% \setlength{\fboxsep}{1ex}
%
% \mbox{}\newpage
^^A % \section{Introduction\\介绍}
^^A %%
^^A % \addvspace{\parskip}
^^A% \begin{parcolumns}[rulebetween=true,colwidths={1=.6\linewidth}]{2}
{\pkg{minted} is a package that allows formatting source code in \LaTeX.
% For example:}

\pkg{minted} 是一个允许在 \LaTeX 中格式化源代码的包。例如:

^^A% \end{parcolumns}
%
^^A TODO 这是在哪里生成的 minted.doc.out
% \begin{VerbatimOut}[gobble=1]{minted.doc.out}
%   \begin{minted}{<language>}
%     <code>
%   \end{minted}
% \end{VerbatimOut}
% \inputminted[gobble=2,frame=lines]{latex}{minted.doc.out}
%
^^A % \addvspace{\parskip}
^^A% \begin{parcolumns}[rulebetween=true,colwidths={1=.6\linewidth}]{2}
{will highlight a piece of code in a chosen language.
% The appearance can be customized with a number of options and color schemes.}
% \par{将会高亮显示所选语言的一段代码。外观可以使用许多选项和颜色方案进行自定义。}
^^A 将会以选定的语言高亮显示一段代码。
^^A 外观可以通过许多选项和配色方案进行定制。
^^A% \end{parcolumns}
%
%
^^A % \addvspace{\parskip}
^^A% \begin{parcolumns}[rulebetween=true,colwidths={1=.6\linewidth}]{2}
{Unlike some other packages, most notably \pkg{listings}, \pkg{minted} requires
the installation of additional software, \app{Pygments}.
This may seem like a disadvantage, but there are also significant advantages.}
% \par{与其他一些包(特别是 \pkg{listings})不同,\pkg{minted} 需要安装额外的软件 \app{Pygments}。这可能看起来像是一个劣势,但也有显著的优势。}
^^A% \end{parcolumns}
^^A 与其他一些软件包不同(最著名的是\pkg{listings}),\pkg{minted}需要额外安装Pygments。%
^^A 这似乎是一个缺点,但也有显著的优点。
^^A % todo pip3安装,在minted.sty中也有体现
^^A % \addvspace{\parskip}
^^A% \begin{parcolumns}[rulebetween=true,colwidths={1=.6\linewidth}]{2}
% \par{\app{Pygments} provides superior syntax highlighting 
compared to conventional packages.
For example, \pkg{listings} basically only highlights strings, comments and keywords.
\app{Pygments}, on the other hand, 
can be completely customized to highlight any kind of token 
the source language might support.
This might include special formatting sequences inside strings, numbers, different kinds of
identifiers and exotic constructs such as HTML tags.}
% \par{相比于传统的包,\app{Pygments} 提供了更好的语法高亮。例如,\pkg{listings} 基本上只高亮字符串、注释和关键字。另一方面,\app{Pygments} 可以完全定制以高亮源语言可能支持的任何类型的标记。这可能包括字符串内的特殊格式序列、数字、不同类型的标识符和像 HTML 标记这样的奇特构造。}
^^A 与常规的代码高亮宏包相比,\app{Pygments}提供了卓越的语法高亮功能。%
^^A 例如,\pkg{listings}基本上只突出显示字符串、注释和关键词。%
^^A 而\app{Pygments}, 则可以完全定制要突出显示的,源语言可能支持的任何类型的token。
^^A 这其中包括字符串中的特殊格式化序列、数字、不同种类的标识符和特殊的结构,如HTML标签。
^^A% \end{parcolumns}
%
^^A % \addvspace{\parskip}
^^A% \begin{parcolumns}[rulebetween=true,colwidths={1=.6\linewidth}]{2}
{Some languages make this especially desirable.
Consider the following Ruby code as an extreme, but at the same time typical, example:}
% \par{对于某些语言,这特别有用。考虑以下 Ruby 代码作为一个极端但同时也是典型的例子:}
^^A 有些语言特别需要这样做。%
^^A 考虑一下以下Ruby代码,作为一个极端、同时也是典型的的例子:
^^A% \end{parcolumns}
%
% \begin{minted}[gobble=3]{ruby}
%   class Foo
%     def init
%       pi = Math::PI
%       @var = "Pi is approx. #{pi}"
%     end
%   end
% \end{minted}
%
^^A % \addvspace{\parskip}
^^A% \begin{parcolumns}[rulebetween=true,colwidths={1=.6\linewidth}]{2}
{Here we have four different colors for identifiers (five, if you count keywords) and escapes from
inside strings, none of which pose a problem for \app{Pygments}.}
% \par{这里有四种不同的标识符颜色(如果您关键字也算上则为五种),以及来自字符串内部的转义,它们都对 \app{Pygments} 没有问题。}
^^A 在这里,我们有四种不同颜色的标识符(如果算上关键字,则是五种),以及字符串内部的转义,
^^A %完成这些需求,\app{Pygments} 都没问题。
^^A \app{Pygments} 都可以完成这些需求。
^^A% \end{parcolumns}
%
^^A % \addvspace{\parskip}
^^A% \begin{parcolumns}[rulebetween=true,colwidths={1=.6\linewidth}]{2}
{Additionally, installing \app{Pygments} is actually incredibly easy (see the next section).}
% \par{此外,安装 \app{Pygments} 实际上非常简单(见下一节)。}
^^A 此外,安装\app{Pygments}%实际上
^^A 是非常容易的(请参阅下一节)。
^^A% \end{parcolumns}
%
^^A %
^^A % \section{Installation\\安装}
^^A %% \subsection{Prerequisites\\前提条件}
%
^^A % \addvspace{\parskip}
% ^^A\begin{parcolumns}[rulebetween=true,colwidths={1=.6\linewidth}]{2}
{\app{Pygments} is written in Python, so make sure that you have Python 2.6 or later installed on your system.  This may be easily checked from the command line:}
% \par{\app{Pygments} 是用 Python 写的,所以请确保您的系统上安装了 Python 2.6 或更高版本\footnotemark:}
^^A \app{Pygments} 是用 Python 编写的,所以请确保你的系统安装了 Python 2.6 或更高版本。 这可以很容易地从命令行中检查。你可以很容易地从命令行查看版本号\footnote{译注:最新的使用pip3安装,2版本的安装不了了!}:
^^A% \end{parcolumns}
%\footnotetext{译注:最新的使用pip3安装,2版本的安装不了了!}
% \begin{Verbatim}[gobble=3,commandchars=\\\{\}]
%   \$ python --version
%   Python 2.7.5
% \end{Verbatim}
%
^^A % \addvspace{\parskip}
% ^^A\begin{parcolumns}[rulebetween=true,colwidths={1=.6\linewidth}]{2}
{If you don't have Python installed, you can download it from the \href{http://www.python.org/download/}{Python website} or use your operating system's package manager.}
% \par{如果您没有安装 Python,则可以从 \href{http://www.python.org/download/}{Python 官网}下载,或使用操作系统的软件包管理器。}
^^A 如果您没有安装 Python,可以去\href{http://www.python.org/download/}{Python 网站} 或使用你的操作系统的软件包管理器下载。
^^A% \end{parcolumns}
%
^^A % \addvspace{\parskip}
% ^^A\begin{parcolumns}[rulebetween=true,colwidths={1=.6\linewidth}]{2}
{Some Python distributions include \pkg{Pygments} (see some of the options under ``Alternative Implementations'' on the Python site).  %
Otherwise, you will need to install \pkg{Pygments} manually. %
This may  be done by installing \href{http://pypi.python.org/pypi/setuptools}{\app{setuptools}}, which facilitates the distribution of Python applications.  You can then install \app{Pygments} using the following command:}
% \par{一些 Python 发行版包括 \pkg{Pygments}(请参阅 Python 站点下“替代实现”选项中的一些选项)。否则,您需要手动安装 \pkg{Pygments}。这可以通过安装 \href{http://pypi.python.org/pypi/setuptools}{\app{setuptools}} 完成,它可以促进 Python 应用程序的分发。然后,您可以使用以下命令安装 \app{Pygments}:}
^^A 一些 Python 发行版包含了\pkg{Pygments}(请参阅 Python 站点上 ``Alternative Implementations'' 下的一些选项)。%
^^A 否则,你将需要手动安装 \pkg{Pygments}。
^^A 可以通过 \href{http://pypi.python.org/pypi/setuptools}{\app{setuptools}} 来安装,它可以帮助我们更简单的创建和分发Python包。可以使用以下命令安装 \app{Pygments}:
^^A% \end{parcolumns}
% \begin{Verbatim}[gobble=3,commandchars=\\\{\}]
%   \$ sudo easy_install Pygments
% \end{Verbatim}
^^A % \addvspace{\parskip}
% ^^A\begin{parcolumns}[rulebetween=true,colwidths={1=.6\linewidth}]{2}
% \par{Under Windows, you will not need the |sudo|, but may need to run the command prompt as administrator.  \pkg{Pygments} may also be installed with |pip|:}
% \par{在 Windows 下,您不需要使用 |sudo|,但可能需要以管理员身份运行命令提示符。也可以使用 |pip| 安装 \pkg{Pygments}:}
^^A 在 Windows 下,你不需要 |sudo|,但可能需要以管理员身份运行命令提示符。 \pkg{Pygments} 也可以用 |pip| 安装。
^^A% \end{parcolumns}
% \begin{Verbatim}[gobble=3,commandchars=\\\{\}]
%   \$ pip install Pygments
% \end{Verbatim}
%
^^A % \addvspace{\parskip}
% ^^A\begin{parcolumns}[rulebetween=true,colwidths={1=.6\linewidth}]{2}
{If you already have \app{Pygments} installed, be aware that the latest version is recommended (at least 1.4 or later).  Some features, such as |escapeinside|, will only work with 2.0+.  \pkg{minted} may work with versions as early as 1.2, but there are no guarantees.}
% \par{如果您已经安装了 \app{Pygments},请注意推荐使用最新版本(至少为 1.4 或更高版本)。一些功能,如 |escapeinside|,仅适用于 2.0+。 \pkg{minted} 可能与早至 1.2 的版本一起使用,但没有保证。}
^^A 如果你已经安装了 \app{Pygments},请注意,建议使用最新版本(至少1.4或更高版本)。有些功能,如 |escapeinside|,只能在2.0+
^^A %以上
^^A 版本中使用。 \pkg{minted}可能会在1.2的版本下工作,但不能保证。
^^A% \end{parcolumns}
%
%
% \subsection{Required packages\\必要的依赖包}
%
^^A % \addvspace{\parskip}
% ^^A\begin{parcolumns}[rulebetween=true,colwidths={1=.6\linewidth}]{2}
{\pkg{minted} requires that the following packages be available and reasonably up to date on your system.  All of these ship with recent \TeX\ distributions.}
% \par{\pkg{minted} 要求您的系统上可用并且相对最新的以下包。所有这些都随最近的 \TeX\ 发行版一起发布。}
^^A \pkg{minted}要求你的系统上有以下宏包,并且是合理的最新版本。 所有这些都是在最近的 \TeX\ 发行版中提供的。
^^A% \end{parcolumns}
%
% \begin{multicols}{3}
% \begingroup
% \setlength\parskip{0pt}
% \setlength\topsep{0pt}
% \begin{list}{\textrm{\labelitemi}}{\ttfamily}
%   \item keyval
%   \item kvoptions
%   \item fancyvrb
%   \item fvextra
%   \item upquote
%   \item float
%   \item ifthen
%   \item calc
%   \item ifplatform
%   \item pdftexcmds
%   \item etoolbox
%   \item xstring
%   \item xcolor
%   \item lineno
%   \item framed
%   \item shellesc (for luatex 0.87+)
%   \item catchfile
%
%   ~
% \end{list}
% \endgroup
% \end{multicols}
%
%
% \subsection{Installing \pkg{minted}\\安装\pkg{minted}}
% \label{sec:installing:installing}
%
^^A % \addvspace{\parskip}
% ^^A\begin{parcolumns}[rulebetween=true,colwidths={1=.6\linewidth}]{2}
{You can probably install \pkg{minted} with your \TeX\ distribution's package manager.  Otherwise, or if you want the absolute latest version, you can install it manually by following the directions below.}
% \par{你可以使用你的\TeX 发行版的软件包管理器安装\pkg{minted}。否则,如果你想要最新版本,你可以按照以下指示手动安装。}
^^A 你可以用你的\TeX\ 发行版的包管理器安装 \pkg{minted}。 或者,如果你想要最新的版本,你可以按照下面的说明手动安装。
^^A% \end{parcolumns}
%
^^A % \addvspace{\parskip}
% ^^A\begin{parcolumns}[rulebetween=true,colwidths={1=.6\linewidth}]{2}
{You may download |minted.sty| from the
\href{https://github.com/gpoore/minted}{project's homepage}.  %
We have to install the file so that \TeX\ is able to find it.%
In order to do that, please refer to the
\href{http://www.tex.ac.uk/cgi-bin/texfaq2html?label=inst-wlcf}{\TeX{} FAQ}.%
If you just want to experiment with the latest version, you could locate your current |minted.sty| in your \TeX\ installation and replace it with the latest version.  Or you could just put the latest |minted.sty| in the same directory as the file you wish to use it with.}
% \par{你可以从\href{https://github.com/gpoore/minted}{项目主页}下载|minted.sty|。我们需要安装该文件以便\TeX 能够找到它。^^A为了做到这一点,
请参考\href{http://www.tex.ac.uk/cgi-bin/texfaq2html?label=inst-wlcf}{\TeX{}常见问题}来做到这一点。如果你只想尝试最新版本,你可以在\TeX 安装中找到当前的|minted.sty|并将其替换为最新版本。或者你也可以将最新的|minted.sty|放在与你要使用它的文件相同的目录中。}
^^A TODO 链接在左右栏都出现...
^^A 你可以从 \href{https://github.com/gpoore/minted}{minted项目的主页} 下载 |minted.sty|。
^^A 我们必须安装该文件,以便让 \TeX\ 能够找到它。
^^A 为了做到这一点,请参考
^^A \href{http://www.tex.ac.uk/cgi-bin/texfaq2html?label=inst-wlcf}{\TeX{} FAQ}。
^^A 如果你只是想试验一下最新的版本,你可以在你的 \TeX\ 安装路径中找到你当前的 |minted.sty| ,然后用最新的版本替换它。 或者你可以把最新的|minted.sty|放在与你想使用的文件相同的目录中。
^^A% \end{parcolumns}
% \section{Basic usage\\基本用法}
%% \subsection{Preliminary\\预备知识}
% \label{sec:basic:preliminary}
%
^^A % \addvspace{\parskip}
^^A % \begin{parcolumns}[rulebetween=true,colwidths={1=.6\linewidth}]{2}
% \par{Since \pkg{minted} makes calls to the outside world (that is, \app{Pygments}), you need to tell the
\LaTeX{} processor about this by passing it the |-shell-escape| option or it won't allow such calls.
In effect, instead of calling the processor like this:}
% \par{由于\pkg{minted}需要调用外部工具(即\app{Pygments}),你需要使用|-shell-escape|选项告诉\LaTeX 排版引擎。否则它不会允许这样的调用。实际上,不要像这样调用排版引擎:}
^^A 由于 \pkg{minted} 会调用外部应用(即 \app{Pygments}),你需要告诉
^^A \LaTeX{}排版引擎,将 |-shell-escape| 选项传递给它,否则它就不允许这样的调用。
^^A 实际上,不要像这样:
^^A % \end{parcolumns}
% \begin{Verbatim}[gobble=3,commandchars=\\\{\}]
%   \$ latex input
% \end{Verbatim}
%
^^A % \addvspace{\parskip}
% \begin{parcolumns}[rulebetween=true,colwidths={1=.6\linewidth}]{2}
% \colchunk{you need to call it like this:}
% \colchunk{你需要像这样调用它:}
% \end{parcolumns}
%
% \begin{Verbatim}[gobble=3,commandchars=\\\{\}]
%   \$ latex -shell-escape input
% \end{Verbatim}
%
% \addvspace{\parskip}
^^A % \begin{parcolumns}[rulebetween=true,colwidths={1=.6\linewidth}]{2}
% \par{The same holds for other processors, such as |pdflatex| or |xelatex|.}
% \par{对于其他排版引擎(例如|pdflatex|或|xelatex|)也是同样的情况。}
^^A 其他的排版引擎也是如此,如|pdflatex|或|xelatex|。
^^A % \end{parcolumns}

% \addvspace{\parskip}
^^A % \begin{parcolumns}[rulebetween=true,colwidths={1=.6\linewidth}]{2}
% \par{You should be aware that using |-shell-escape| allows \LaTeX\ to run potentially arbitrary commands on your system.  It is probably best to use |-shell-escape| only when you need it, and to use it only with documents from trusted sources.}
% \par{你应该知道,使用|-shell-escape|允许\LaTeX 在你的系统上运行潜在的任意命令。最好仅在需要时使用|-shell-escape|,并且仅与来自可信源的文档一起使用。}
^^A 你应该知道,使用|-shell-escape|允许 \LaTeX\ 在你的系统上运行潜在的任意命令。最好是在需要时才使用|-shell-escape|,并且只对可信来源的文档使用。
^^A % \end{parcolumns}

%
%
% \subsubsection*{Working with OS X\\在 OS X 上使用}
%
% \addvspace{\parskip}
^^A % \begin{parcolumns}[rulebetween=true,colwidths={1=.6\linewidth}]{2}
% \par{If you are using \pkg{minted} with some versions/configurations of OS X, and are using caching with a large number of code blocks ($>256$), you may receive an error like}
% \par{如果你在某些版本/配置的 OS X 上使用\pkg{minted},并且使用具有大量代码块($>256$)的缓存,你可能会收到以下错误:}
^^A 如果你在OS X的某些版本/配置下使用\pkg{minted},并且在使用大量代码块的缓存时($>256$),你可能会收到一个错误,如 
^^A % \end{parcolumns}
%\begin{Verbatim}
%OSError: [Errno 24] Too many open files:
%\end{Verbatim}
^^A % \addvspace{\parskip}


This is due to the way files are handled by the operating system, combined with the way that caching works.  To resolve this, you may use the OS X commands to increase the number of files that may be used:

这是由于操作系统处理文件的方式与缓存工作方式的结合引起的。为了解决这个问题,你可以使用 OS X 命令增加可使用的文件数量:

|launchctl limit maxfiles| or |ulimit -n| 


^^A 这是由于操作系统处理文件的方式,再加上缓存的工作方式。 为了解决这个问题,你可以使用OS X的命令|launchctl limit maxfiles|或|ulimit -n|来增加可使用的文件数量。
^^A % \end{parcolumns}
%
%
%
% \subsection{A minimal complete example\\一个最小的完整示例}
%
% \addvspace{\parskip}
% \begin{parcolumns}[rulebetween=true,colwidths={1=.6\linewidth}]{2}
% \colchunk{The following file |minimal.tex| shows the basic usage of \pkg{minted}.}
% \colchunk{下面的文件|minimal.tex|展示了\pkg{minted}的基本用法。}
^^A 下面的文件|minimal.tex|显示了 \pkg{minted}的基本用法。
% \end{parcolumns}
%
% \begin{VerbatimOut}[gobble=1]{minted.doc.out}
%   \documentclass{article}
%
%   \usepackage{minted}
%
%   \begin{document}
%   \begin{minted}{c}
%   int main() {
%       printf("hello, world");
%       return 0;
%   }
%   \end{minted}
%   \end{document}
% \end{VerbatimOut}
% \inputminted[gobble=2,frame=lines]{latex}{minted.doc.out}
%
^^A % \addvspace{\parskip}
% \begin{parcolumns}[rulebetween=true,colwidths={1=.6\linewidth}]{2}
% \colchunk{By compiling the source file like this:}
% \colchunk{通过这样编译源文件:}
% \end{parcolumns}
%
% \begin{Verbatim}[gobble=3,commandchars=\\\{\}]
%   \$ pdflatex -shell-escape minimal
% \end{Verbatim}
%
% \addvspace{\parskip}
^^A % \begin{parcolumns}[rulebetween=true,colwidths={1=.6\linewidth}]{2}
% \par{we end up with the following output in |minimal.pdf|:}
% \par{运行上述命令后,会生成 |minimal.pdf| 文件,其中包含如下代码的输出:}
^^A % \end{parcolumns}
%
% \hfill
% \colorbox{minted@samplebg}{\begin{minipage}{0.6\textwidth}
%   \inputminted[firstline=7,lastline=10]{c}{minted.doc.out}
% \end{minipage}}
% \hfill\hfill
%
%
% \subsection{Formatting source code\\格式化源代码}
%
% \DescribeEnv{minted}
^^A % \addvspace{\parskip}
^^A % \begin{parcolumns}[rulebetween=true,colwidths={1=.6\linewidth}]{2}
%{Using \pkg{minted} is straightforward. For example, to highlight some Python source code we might use the following code snippet (result on the right):}
% \par{使用 \pkg{minted} 包很简单。例如,若要高亮显示一些 Python 代码,可以使用以下代码片段(右侧为输出结果):}
^^A % \end{parcolumns}
%
% \begin{example}
%   \begin{minted}{python}
%   def boring(args = None):
%       pass
%   \end{minted}
% \end{example}
%
% \addvspace{\parskip}
^^A % \begin{parcolumns}[rulebetween=true,colwidths={1=.6\linewidth}]{2}
% \par{Optionally, the environment accepts a number of options in |key=value| notation, which are described
% in more detail below.}
% \par{此外,该环境还接受一些以 |key=value| 形式的选项,详见下文。}
^^A 可选的是,环境接受一些以|key=value|符号表示的选项,这些选项将在下面详细描述。
^^A % \end{parcolumns}
%
%
% \DescribeMacro{\mint}
^^A % \addvspace{\parskip}
^^A % \begin{parcolumns}[rulebetween=true,colwidths={1=.6\linewidth}]{2}
% {For a single line of source code, you can alternatively use a shorthand notation:}
^^A % \par{对于单行代码,也可以使用简写:}

对于单行源代码,您可以使用简写形式:
^^A % \end{parcolumns}
%
% \begin{example}
%   \mint{python}|import this|
% \end{example}
%
% \addvspace{\parskip}
^^A % \begin{parcolumns}[rulebetween=true,colwidths={1=.6\linewidth}]{2}
% \par{% This typesets a single line of code using a command rather than an environment, so it saves a little typing, but its output is equivalent to that of the |minted| environment.}
% \par{% 这样可以使用命令而不是环境来排版一行代码,因此节省了一些打字量,但输出结果与 |minted| 环境的结果等效。}
^^A 这是用一个命令而不是环境来排版一行代码,所以它节省了一点打字量,但它的输出与|minted|环境的输出是相等的。 
^^A % \end{parcolumns}


The code is delimited by a pair of identical characters, similar to how  |\verb| works.  The complete syntax is 

^^A 代码由一对相同的字符进行定界,与 |\verb| 命令类似。完整语法为
代码由一对相同的字符分隔,类似于 |\verb| 的工作方式。完整的语法是

\cmd\mint\oarg{options}\marg{language}\meta{delim}\meta{code}\meta{delim}

% where the code delimiter can be almost any punctuation character.
% The \meta{code} may also be delimited with matched curly braces |{}|, so long as \meta{code} itself does not contain unmatched curly braces.
% Again, this command supports a number of options described below.

其中代码定界符可以是几乎任意标点符号。代码片段也可以用匹配的花括号 |{}| 进行定界,只要代码本身不包含未匹配的花括号。该命令也支持一些选项,详见下文。
^^A % \end{parcolumns}

%
% \addvspace{\parskip}
^^A % \begin{parcolumns}[rulebetween=true,colwidths={1=.6\linewidth}]{2}
% \par{% Note that the |\mint| command \textbf{is not for inline use}.  Rather, it is a shortcut for |minted| when only a single line of code is present.  The |\mintinline| command is provided for inline use.}
% \par{% 注意,|\mint| 命令\textbf{不能用于行内排版}。相反,它是当仅有一行代码时使用 |minted| 的一种快捷方式。对于行内排版,应使用 |\mintinline| 命令。}
^^A % \end{parcolumns}
%
%
% \DescribeMacro{\mintinline}
^^A % \addvspace{\parskip}
^^A % \begin{parcolumns}[rulebetween=true,colwidths={1=.6\linewidth}]{2}
% {% Code can be typeset inline:}
% \par{% 可以行内排版代码:}
^^A % \end{parcolumns}
% 
% \begin{example}
%   X\mintinline{python}{print(x**2)}X
% \end{example}
%
% \addvspace{\parskip}
^^A % \begin{parcolumns}[rulebetween=true,colwidths={1=.6\linewidth}]{2}
% \par{The syntax is}
% \par{语法为}
^^A % \end{parcolumns}

\cmd\mintinline\oarg{options}\marg{language}\meta{delim}\meta{code}\meta{delim}

^^A % \begin{parcolumns}[rulebetween=true,colwidths={1=.6\linewidth}]{2}
% \par{The delimiters can be a pair of characters, as for \cmd\mint.  They can also be a matched pair of curly braces, |{}|.}
% \par{定界符可以是一对字符,如 \cmd\mint。它们也可以是匹配的花括号 |{}|。}
^^A % \end{parcolumns}
%
% \addvspace{\parskip}
^^A % \begin{parcolumns}[rulebetween=true,colwidths={1=.6\linewidth}]{2}
% \par{The command has been carefully crafted so that in most cases it will function correctly when used inside other commands.\footnote{For example, \mintinline{latex}{\mintinline} works in footnotes!  The main exception is when the code contains the percent \texttt{\%} or hash \texttt{\#} characters, or unmatched curly braces.}}
% \par{该命令被精心制作,以便在大多数情况下,在其他命令内使用时它会正确地工作。\footnote{例如,\mintinline{latex}{\mintinline}在脚注中也可以使用!主要例外是当代码包含百分号\texttt{\%}或井号\texttt{\#}字符,或未匹配的花括号时。}}
^^A % \end{parcolumns}
%
% \DescribeMacro{\inputminted}
^^A % \begin{parcolumns}[rulebetween=true,colwidths={1=.6\linewidth}]{2}
% Finally, there's the |\inputminted| command to read and format whole files.
% Its syntax is 

最后,有|\inputminted|命令来读取和格式化整个文件。它的语法为

\cmd\inputminted\oarg{options}\marg{language}\marg{filename}

^^A % \end{parcolumns}
%
%
% \subsection{Using different styles\\使用不同的样式}
%
% \DescribeMacro{\usemintedstyle}
^^A % \begin{parcolumns}[rulebetween=true,colwidths={1=.6\linewidth}]{2}
% {Instead of using the default style you may choose another stylesheet provided by \app{Pygments}.  This may be done via the following:}
% \par{你可以选择由\app{Pygments}提供的另一个样式表,而不是使用默认样式表。这可以通过以下方式实现:}
^^A % \end{parcolumns}
%
% \mint[frame=lines]{latex}/\usemintedstyle{name}/
%
^^A % \begin{parcolumns}[rulebetween=true,colwidths={1=.6\linewidth}]{2}
% {The full syntax is \cmd\usemintedstyle\oarg{language}\marg{style}.  The style may be set for the document as a whole (no language specified), or only for a particular language.  Note that the style may also be set via \cmd\setminted\ and via the optional argument for each command and environment.\footnote{Version 2.0 added the optional language argument and removed the restriction that the command be used in the preamble.}}
% \par{完整的语法是\cmd\usemintedstyle\oarg{language}\marg{style}。可以为整个文档设置样式(未指定语言),或仅为特定语言设置样式。请注意,样式也可以通过\cmd\setminted和每个命令和环境的可选参数进行设置。\footnote{版本2.0添加了可选语言参数,并删除了必须在导言中使用该命令的限制。}}
^^A % \end{parcolumns}
%
^^A % \begin{parcolumns}[rulebetween=true,colwidths={1=.6\linewidth}]{2}
% \par{To get a list of all available stylesheets, see the online demo at the \href{http://pygments.org/demo/}{Pygments website} or execute the following command on the command line:}
% \par{要获取所有可用样式表的列表,请参见\href{http://pygments.org/demo/}{Pygments网站}上的在线演示,或在命令行上执行以下命令:}
^^A % \end{parcolumns}
%
% \begin{Verbatim}[gobble=3,commandchars=\\\{\}]
%   \$ pygmentize -L styles
% \end{Verbatim}
%
^^A % \begin{parcolumns}[rulebetween=true,colwidths={1=.6\linewidth}]{2}
% \par{Creating your own styles is also easy. Just follow the instructions provided on the
\href{http://pygments.org/docs/styles/#creating-own-styles}{\pkg{Pygments} website}.}
% \par{创建自己的样式也很容易。只需按照\href{http://pygments.org/docs/styles/#creating-own-styles}{\pkg{Pygments}网站}提供的说明即可。}
^^A % \end{parcolumns}
%
%
% \subsection{Supported languages\\支持的语言}
%
^^A % \begin{parcolumns}[rulebetween=true,colwidths={1=.6\linewidth}]{2}
% \par{\app{Pygments} supports over 300 different programming languages, template languages, and other markup languages.  To see an exhaustive list of the currently supported languages, use the command}
% \par{\app{Pygments}支持300多种不同的编程语言、模板语言和其他标记语言。要查看当前支持的语言的详细列表,请使用以下命令:}
^^A % \end{parcolumns}
%
% \begin{Verbatim}[gobble=3,commandchars=\\\{\}]
%   \$ pygmentize -L lexers
% \end{Verbatim}
^^A % \section{Floating listings\\浮的代码块}\label{sec:float}
^^A %% \DescribeEnv{listing}
% \begin{parcolumns}[rulebetween=true,colwidths={1=.6\linewidth}]{2}
% \colchunk{\pkg{minted} provides the |listing| environment to wrap around a source code block.  This puts the code into a floating box, with the default placement |tbp| like figures and tables.  You can also provide a |\caption| and a |\label| for such a listing in the usual way (that is, as for the |figure| and |table| environments):}
% \colchunk{\pkg{minted}提供了|listing|环境来包装源代码块。这将把代码放入一个浮动框中,其默认的放置方式为|tbp|,与图表相同。您还可以提供|\caption|和|\label|来定义此类代码清单(像使用|figure| 和 |table|环境一样):}
% \end{parcolumns}

% \begin{VerbatimOut}[gobble=1]{minted.doc.out}
%   \begin{listing}[H]
%     \mint{cl}/(car (cons 1 '(2)))/
%     \caption{Example of a listing.}
%     \label{lst:example}
%   \end{listing}
%
%   Listing \ref{lst:example} contains an example of a listing.
% \end{VerbatimOut}
% \inputminted[gobble=2,frame=lines]{latex}{minted.doc.out}
%
% will yield:
%
% \hfill
% \colorbox{minted@samplebg}{\begin{minipage}{0.6\textwidth}
%   \input{minted.doc.out}
% \end{minipage}}
% \hfill\hfill
%
% \begin{parcolumns}[rulebetween=true,colwidths={1=.6\linewidth}]{2}
% \colchunk{The default |listing| placement can be modified easily.  When the package option |newfloat=false| (default), the \pkg{float} package is used to create the |listing| environment.  Placement can be modified by redefining |\fps@listing|.  For example,}
% \colchunk{默认的清单放置方式可以轻松修改。当使用包选项|newfloat=false|(默认值)时,\pkg{minted}使用\pkg{float}包来创建|listing|环境。通过重新定义|\fps@listing|可以修改清单的放置方式。例如,}
% \end{parcolumns}
%\begin{verbatim}
%\makeatletter
%\renewcommand{\fps@listing}{htp}
%\makeatother
%\end{verbatim}

% \begin{parcolumns}[rulebetween=true,colwidths={1=.6\linewidth}]{2}
% \colchunk{When |newfloat=true|, the more powerful \pkg{newfloat} package is used to create the |listing| environment.  In that case, \pkg{newfloat} commands are available to customize |listing|:}
% \colchunk{当使用选项|newfloat=true|时,将使用更强大的\pkg{newfloat}包来创建|listing|环境。在这种情况下,可使用\pkg{newfloat}命令来自定义|listing|:}
% \end{parcolumns}
%\begin{verbatim}
%\SetupFloatingEnvironment{listing}{placement=htp}
%\end{verbatim}
%
%
% \DescribeMacro{\listoflistings}
% \begin{parcolumns}[rulebetween=true,colwidths={1=.6\linewidth}]{2}
% \colchunk{The |\listoflistings| macro will insert a list of all (floated) listings in the document:}
% \colchunk{|\listoflistings|宏将插入文档中所有(浮动的)清单的列表:}
% \end{parcolumns}
%
% \begin{example}
%   \listoflistings
% \end{example}
%
% \subsection*{Customizing the \texttt{listing} environment\\自定义 \texttt{listing} 环境}
% \begin{parcolumns}[rulebetween=true,colwidths={1=.6\linewidth}]{2}
% \colchunk{By default, the |listing| environment is created using the \pkg{float} package.  In that case, the |\listingscaption| and |\listoflistingscaption| macros described below may be used to customize the caption and list of listings.  If \pkg{minted} is loaded with the |newfloat| option, then the |listing| environment will be created with the more powerful \href{http://www.ctan.org/pkg/newfloat}{\pkg{newfloat}} package instead.  \pkg{newfloat} is part of \href{http://www.ctan.org/pkg/caption}{\pkg{caption}}, which provides many options for customizing captions.}
% \colchunk{默认情况下,|listing| 环境是使用 \pkg{float} 宏包创建的。在这种情况下,可以使用下面描述的 |\listingscaption| 和 |\listoflistingscaption| 宏来自定义标题和列表。如果使用 |newfloat| 选项加载了 \pkg{minted},则 |listing| 环境将使用更强大的 \href{http://www.ctan.org/pkg/newfloat}{\pkg{newfloat}} 宏包创建。\pkg{newfloat} 是 \href{http://www.ctan.org/pkg/caption}{\pkg{caption}} 的一部分,它提供了许多自定义标题的选项。}
% \end{parcolumns}

% \begin{parcolumns}[rulebetween=true,colwidths={1=.6\linewidth}]{2}
% \colchunk{When \pkg{newfloat} is used to create the |listing| environment, customization should be achieved using \pkg{newfloat}'s |\SetupFloatingEnvironment| command.  For example, the string ``Listing'' in the caption could be changed to ``Program code'' using }
% \colchunk{当使用 \pkg{newfloat} 创建 |listing| 环境时,应使用 \pkg{newfloat} 的 |\SetupFloatingEnvironment| 命令进行自定义。例如,可以使用以下命令将标题中的字符串“Listing”更改为“程序代码”:}
% \end{parcolumns}

%\begin{verbatim}
%\SetupFloatingEnvironment{listing}{name=Program code}
%\end{verbatim}
% \begin{parcolumns}[rulebetween=true,colwidths={1=.6\linewidth}]{2}
% \colchunk{And ``List of Listings'' could be changed to ``List of Program Code'' with}
% \colchunk{并且可以使用以下命令将“List of Listings”更改为“程序代码列表”:}
% \end{parcolumns} %\begin{verbatim}
%\SetupFloatingEnvironment{listing}{listname=List of Program Code}
%\end{verbatim}
% \begin{parcolumns}[rulebetween=true,colwidths={1=.6\linewidth}]{2}
% \colchunk{Refer to the \pkg{newfloat} and \pkg{caption} documentation for additional information.}
% \colchunk{有关详细信息,请参阅 \pkg{newfloat} 和 \pkg{caption} 文档。}
% \end{parcolumns}
%
% \DescribeMacro{\listingscaption}
% \begin{parcolumns}[rulebetween=true,colwidths={1=.6\linewidth}]{2}
% \colchunk{(Only applies when package option |newfloat| is not used.) The string ``Listing'' in a listing's caption can be changed. To do this, simply redefine the macro |\listingscaption|, for example:}
% \colchunk{(仅适用于未使用包选项 |newfloat| 的情况。)可以更改列表标题中的字符串“Listing”。要做到这一点,只需重新定义宏 |\listingscaption|,例如:}
% \end{parcolumns}
%
% \mint[frame=lines]{latex}/\renewcommand{\listingscaption}{Program code}/
%
% \DescribeMacro{\listoflistingscaption}
% \begin{parcolumns}[rulebetween=true,colwidths={1=.6\linewidth}]{2}
% \colchunk{(Only applies when package option |newfloat| is not used.) Likewise, the caption of the listings list, ``List of Listings,'' can be changed by redefining
% |\listoflistingscaption|:}
% \colchunk{(仅适用于未使用包选项 |newfloat| 的情况。)同样,可以通过重新定义 |\listoflistingscaption| 来更改列表的标题“List of Listings”:}
% \end{parcolumns}
%
% \mint[frame=lines]{latex}/\renewcommand{\listoflistingscaption}{List of Program Code}/^^A v2
^^A % \section{Options\\选项}
^^A %
\subsection{Package options\\宏包选项}

\DescribeMacro{chapter}
\vspace*{-2\parskip}
\begin{parcolumns}[rulebetween=false,colwidths={2=.55\linewidth}]{2}
\colchunk{可以在加载 \pkg{minted} 宏包时传递 |section| 或 |chapter| 选项,以控制 \LaTeX{} 如何计数 |listing| 浮动体。例如,以下命令将导致列表按章进行计数:}
\colchunk{To control how \LaTeX{} counts the |listing| floats, you can pass either the
|section| or |chapter| option when loading the \pkg{minted} package.
For example, the following will cause listings to be counted by chapter:}
\end{parcolumns}

\mint[frame=lines]{latex}/\usepackage[chapter]{minted}/

\DescribeMacro{cache=\meta{boolean} (default:~true)}
\vspace*{-2\parskip}
\begin{parcolumns}[rulebetween=false,colwidths={2=.55\linewidth}]{2}
\colchunk{\pkg{minted} 通过将代码保存到临时文件,通过 \app{Pygments} 对代码进行高亮处理并将输出保存到另一个临时文件,然后将输出输入到 \LaTeX\ 文档中来工作。如果需要突出显示多个代码块,则这个过程可能会变得非常缓慢。为避免这种情况,该宏包提供了一个 |cache| 选项。默认情况下,此选项已启用。}
\colchunk{\pkg{minted} works by saving code to a temporary file, highlighting the code via \app{Pygments} and saving the output to another temporary file, and inputting the output into the \LaTeX\ document.  This process can become quite slow if there are several chunks of code to highlight.  To avoid this, the package provides a |cache| option.  This is on by default.}
\end{parcolumns}

\begin{parcolumns}[rulebetween=false,colwidths={2=.55\linewidth}]{2}
\colchunk{选项 |cache| 在文档的根目录下创建一个名为}
\colchunk{The |cache| option creates a directory}
\end{parcolumns}

|_minted-|\meta{jobname}

\begin{parcolumns}[rulebetween=false,colwidths={2=.55\linewidth}]{2}
\colchunk{的目录(可以使用选项 |cachedir| 自定义此目录名)。\footnotemark 高亮代码文件存储在此目录中,以便将来无需再次高亮代码。在大多数情况下,缓存将显着加快文档的编译速度。}
\colchunk{in the document's root directory (this may be customized with the |cachedir| option).\footnotemark
Files of highlighted code are stored in this directory, so that the code will not have to be highlighted again in the future.  In most cases, caching will significantly speed up document compilation.}
\end{parcolumns}

\footnotetext{实际上,目录的名称使用“已消毒”的 \meta{jobname} 副本命名,其中空格和星号已替换为下划线,双引号已被剥离。如果文件名包含空格,则 \texttt{\string\jobname} 将包含一个带引号的名称,除了使用星号替换空格的旧版 MiKTeX 外。使用“已消毒”的 \meta{jobname} 比适应各种转义约定更简单。}
\footnotetext{The directory is actually named using a ``sanitized'' copy of \meta{jobname}, in which spaces and asterisks have been replaced by underscores, and double quotation marks have been stripped.  If the file name contains spaces, \texttt{\string\jobname} will contain a quote-wrapped name, except under older versions of MiKTeX which used the name with spaces replaced by asterisks.  Using a ``sanitized'' \meta{jobname} is simpler than accomodating the various escaping conventions.}


\begin{parcolumns}[rulebetween=false,colwidths={2=.55\linewidth}]{2}
\colchunk{不再使用的缓存文件将被自动删除。\footnotemark}
\colchunk{Cached files that are no longer in use are automatically deleted.\footnotemark}
\end{parcolumns}
\footnotetext{这取决于主辅助文件未被删除或损坏。如果发生这种情况,可以简单地删除缓存目录并重新开始。}
\footnotetext{This depends on the main auxiliary file not being deleted or becoming corrupted.  If that happens, you could simply delete the cache directory and start over.}


\DescribeMacro{cachedir=\meta{directory} (def:~\_minted-\meta{jobname})}
\vspace*{-2\parskip}
\begin{parcolumns}[rulebetween=false,colwidths={2=.55\linewidth}]{2}
\colchunk{这允许指定存储缓存文件的目录。路径应使用正斜杠,即使在 Windows 下也是如此。}
\colchunk{This allows the directory in which cached files are stored to be specified.  Paths should use forward slashes, even under Windows.}
\end{parcolumns}

\begin{parcolumns}[rulebetween=false,colwidths={2=.55\linewidth}]{2}
\colchunk{必须转义特殊字符。例如,|cachedir=/mintedcache| 将无法工作,因为波浪号 || 将转换为非换行空格的 LaTeX 命令,而不是被直接处理。可以使用}
\colchunk{Special characters must be escaped.  For example, |cachedir=~/mintedcache| would not work because the tilde |~| would be converted into the \LaTeX\ commands for a non-breaking space, rather than being treated literally.  Instead, use }
\end{parcolumns}

|\string~/mintedcache|,|\detokenize{~/mintedcache}|

\begin{parcolumns}[rulebetween=false,colwidths={2=.55\linewidth}]{2}
\colchunk{或等效的解决方案。}
\colchunk{or an equivalent solution.}
\end{parcolumns}

\begin{parcolumns}[rulebetween=false,colwidths={2=.55\linewidth}]{2}
\colchunk{路径可以包含空格,但仅在整个 \meta{directory} 被大括号 |{}| 包裹并且空格被引用时。例如,}
\colchunk{Paths may contain spaces, but only if the entire \meta{directory} is wrapped in curly braces |{}|, and only if the spaces are quoted.  For example,}

\end{parcolumns}

\begin{Verbatim}
cachedir = {\detokenize{~/"minted cache"/"with spaces"}}
\end{Verbatim}

\begin{parcolumns}[rulebetween=false,colwidths={2=.6\linewidth}]{2}
\colchunk{请注意,如果指定了 |outputdir|,则缓存目录相对于其而言。}
\colchunk{Note that the cache directory is relative to the |outputdir|, if an |outputdir| is specified.}
\end{parcolumns}


\DescribeMacro{finalizecache=\meta{boolean} (default:~false)}
\vspace*{-2\parskip}
\begin{parcolumns}[rulebetween=false,colwidths={2=.55\linewidth}]{2}
\colchunk{在某些情况下,可能希望在不允许 |-shell-escape| 的环境中使用 \pkg{minted}。文档可能会被提交给出版商或预印本服务器,或者与不支持 |-shell-escape| 的在线服务一起使用。只要不需要修改 \pkg{minted} 的内容,就可以这样做。}
\colchunk{In some cases, it may be desirable to use \pkg{minted} in an environment in which |-shell-escape| is not allowed.  A document might be submitted to a publisher or preprint server or used with an online service that does not support |-shell-escape|.  This is possible as long as \pkg{minted} content does not need to be modified.}
\end{parcolumns}


\begin{parcolumns}[rulebetween=false,colwidths={2=.55\linewidth}]{2}
\colchunk{使用 |finalizecache| 选项进行编译会为在没有 |-shell-escape| 的环境中使用准备缓存。\footnotemark
完成后,可以将 |finalizecache| 选项替换为 |frozencache| 选项,以后可以在不需要 |-shell-escape| 的情况下使用冻结(静态)缓存。}
\colchunk{Compiling with the |finalizecache| option prepares the cache for use in an environment without |-shell-escape|.\footnotemark
Once this has been done, the |finalizecache| option may be swapped for the |frozencache| option, which will then use the frozen (static) cache in the future, without needing |-shell-escape|.}
\end{parcolumns}
\footnotetext{通常,缓存文件使用突出显示设置和突出显示文本的 MD5 哈希命名。 \texttt{finalizecache} 使用 \texttt{listing<number>.pygtex} 方案重命名缓存文件。这使得匹配文档内容和缓存文件更加简单,也是必要的,因为在 TeX Live 2016 之前,XeTeX 引擎缺乏 pdfTeX 和 LuaTeX 具有的内置 MD5 能力。}
\footnotetext{Ordinarily, cache files are named using an MD5 hash of highlighting settings and highlighted text.  \texttt{finalizecache} renames cache files using a \texttt{listing<number>.pygtex} scheme.  This makes it simpler to match up document content and cache files, and is also necessary for the XeTeX engine since prior to TeX Live 2016 it lacked the built-in MD5 capabilities that pdfTeX and LuaTeX have.}



\DescribeMacro{fontencoding=\meta{encoding} (default:~\meta{doc~encoding})}
\vspace*{-2\parskip}
\begin{parcolumns}[rulebetween=false,colwidths={2=.55\linewidth}]{2}
\colchunk{设置用于排版代码的字体编码。}
\colchunk{Set font encoding used for typesetting code.}

\colplacechunks
\colchunk{例如,|fontencoding=T1|。}
\colchunk{For example, |fontencoding=T1|.}
\end{parcolumns}

 \DescribeMacro{frozencache=\meta{boolean} (default:~false)}
\vspace*{-2\parskip}
\begin{parcolumns}[rulebetween=false,colwidths={2=.55\linewidth}]{2}
\colchunk{使用使用 |finalizecache| 选项创建的冻结(静态)缓存。打开 |frozencache| 时,不需要 |-shell-escape|,也不需要 Python 和 Pygments。此外,通过 |\inputminted| 访问的任何外部文件都不再需要。}
\colchunk{Use a frozen (static) cache created with the |finalizecache| option.  When |frozencache| is on, |-shell-escape| is not needed, and Python and Pygments are not required.  In addition, any external files accessed through |\inputminted| are no longer necessary.}
\end{parcolumns}

\begin{parcolumns}[rulebetween=false,colwidths={2=.5\linewidth}]{2}
\colchunk{\textbf{必须谨慎使用此选项。文档在 \pkg{minted} 考虑的范围内必须处于最终形式,并且在打开 \texttt{frozencache} 之前必须使用 \texttt{finalizecache} 进行编译。打开此选项后,除非直接编辑缓存文件,否则无法修改 \pkg{minted} 内容。无法更改需要 Pygments 或 Python 的任何 \pkg{minted} 设置。如果在打开 \texttt{frozencache} 后不正确地修改了 \pkg{minted} 内容,则 \pkg{minted} 无法检测到修改。}}
\colchunk{\textbf{This option must be used with care.  A document \emph{must} be in final form, as far as \pkg{minted} is concerned, \emph{before} \texttt{frozencache} is turned on, and the document \emph{must} have been compiled with \texttt{finalizecache}.   When this option is on, \pkg{minted} content cannot be modified, except by editing the cache files directly.  Changing any \pkg{minted} settings that require Pygments or Python is not possible.  If \pkg{minted} content is incorrectly modified after \texttt{frozencache} is turned on, \pkg{minted} \emph{cannot} detect the modification.}}
\end{parcolumns}

\begin{parcolumns}[rulebetween=false,colwidths={2=.55\linewidth}]{2}
\colchunk{如果正在使用 |frozencache|,并且要验证 \pkg{minted} 设置或内容是否未被以无效的方式修改,则可以使用以下过程测试缓存。}
\colchunk{If you are using |frozencache|, and want to verify that \pkg{minted} settings or content have not been modified in an invalid fashion, you can test the cache using the following procedure.}
\end{parcolumns}

\begin{parcolumns}[rulebetween=false,colwidths={2=.55\linewidth}]{2}
\colchunk{%
\begin{enumerate}
\item 获取使用 |frozencache| 的缓存的副本。
\item 在支持 |-shell-escape| 的环境中使用 |finalizecache=true| 和 |frozencache=false| 编译文档。这本质上重新生成了冻结(静态)缓存。
\item 将原始缓存与新生成的缓存进行比较。在 Linux 和 OS X 下,可以使用 |diff|;在 Windows 下,可能需要使用 |fc|。如果 \pkg{minted} 内容和设置未以无效的方式进行修改,则所有文件都将相同(假设两个缓存都使用兼容的Pygments版本)。
\end{enumerate}
}
\colchunk{%
\begin{enumerate}
\item Obtain a copy of the cache used with |frozencache|.
\item Compile the document in an environment that supports |-shell-escape|, with |finalizecache=true| and |frozencache=false|.  This essentially regenerates the frozen (static) cache.
\item Compare the original cache with the newly generated cache.  Under Linux and OS X, you could use |diff|; under Windows, you probably want |fc|.  If \pkg{minted} content and settings have not been modified in an invalid fashion, all files will be identical (assuming that compatible versions of Pygments are used for both caches).
\end{enumerate}
}
\end{parcolumns}

\DescribeMacro{draft=\meta{boolean} (default:~false)}
\vspace*{-2\parskip}
\begin{parcolumns}[rulebetween=false,colwidths={2=.55\linewidth}]{2}
\colchunk{这仅使用\pkg{fancyvrb}进行所有排版;不使用\app{Pygments}。这是以交换语法高亮和一些其他\app{minted}功能为代价的,但编译速度更快。性能应与直接使用\pkg{fancyvrb}时基本相同;不使用外部临时文件。请注意,如果您在编译之间没有更改太多的代码,则缓存模式和草稿模式之间的性能差异可能很小。还请注意,草稿设置通常从文档类继承。}
\colchunk{This uses \pkg{fancyvrb} alone for all typesetting; \app{Pygments} is not used.  This trades syntax highlighting and some other \app{minted} features for faster compiling.  Performance should be essentially the same as using \pkg{fancyvrb} directly; no external temporary files are used.  Note that if you are not changing much code between compiles, the difference in performance between caching and draft mode may be minimal.  Also note that |draft| settings are typically inherited from the document class.}
\end{parcolumns}

\begin{parcolumns}[rulebetween=false,colwidths={2=.55\linewidth}]{2}
\colchunk{草稿模式不支持|autogobble|。在草稿模式下,常规的|gobble|,|linenos|和大多数与语法高亮无关的选项仍将正常运行。}
\colchunk{Draft mode does not support |autogobble|.  Regular |gobble|, |linenos|, and most other options not related to syntax highlighting will still function in draft mode.}
\end{parcolumns}

\begin{parcolumns}[rulebetween=false,colwidths={2=.55\linewidth}]{2}
\colchunk{通常可以在草稿模式下编译文档而无需使用shell escape。 \pkg{ifplatform}包可能会发出关于由于禁用了shell escape而导致功能有限的警告,但在几乎所有情况下都可以忽略此警告。 (仅当您的系统配置不寻常,使得|\ifwindows|宏必须借助shell escape来确定系统时,才真正需要shell escape。有关详细信息,请参见\pkg{ifplatform}文档。)}
\colchunk{Documents can usually be compiled without shell escape in draft mode.  The \pkg{ifplatform} package may issue a warning about limited functionality due to shell escape being disabled, but this may be ignored in almost all cases.  (Shell escape is only really required if you have an unusual system configuration such that the |\ifwindows| macro must fall back to using shell escape to determine the system.  See the \pkg{ifplatform} documentation for more details:  \url{http://www.ctan.org/pkg/ifplatform}.)}
\end{parcolumns}


\begin{parcolumns}[rulebetween=false,colwidths={2=.55\linewidth}]{2}
\colchunk{如果设置了|cache|选项,则在草稿模式下保留所有现有缓存文件。这允许在不需要每次完全重建缓存的情况下间歇性地使用缓存模式和草稿模式。一旦关闭草稿模式,未使用的缓存文件的自动清理将恢复。 (这假设辅助文件在此期间未被删除;它包含缓存历史记录,并允许自动清理未使用的文件。)}
\colchunk{If the |cache| option is set, then all existing cache files will be kept while draft mode is on.  This allows caching to be used intermitently with draft mode without requiring that the cache be completely recreated each time.  Automatic cleanup of cached files will resume as soon as draft mode is turned off.  (This assumes that the auxiliary file has not been deleted in the meantime; it contains the cache history and allows automatic cleanup of unused files.)}
\end{parcolumns}



\DescribeMacro{final=\meta{boolean} (default:~true)}
\vspace*{-2\parskip}
\begin{parcolumns}[rulebetween=false,colwidths={2=.55\linewidth}]{2}
\colchunk{这是|draft|的相反;它等效于|draft=false|。\\同样,请注意,|draft|和|final|设置通常从文档类继承。}
\colchunk{This is the opposite of |draft|; it is equivalent to |draft=false|.  Again, note that |draft| and |final| settings are typically inherited from the document class.}
\end{parcolumns}


\DescribeMacro{kpsewhich=\meta{boolean} (default:~false)}
\vspace*{-2\parskip}
\begin{parcolumns}[rulebetween=false,colwidths={2=.55\linewidth}]{2}
\colchunk{这个选项使用 |kpsewhich| 来定位需要突出显示的文件。一些构建工具(如 |texi2pdf|)通过修改 |TEXINPUTS| 来实现功能;在某些情况下,用户也可以自定义 |TEXINPUTS|。|kpsewhich| 选项允许 \pkg{minted} 与这些配置一起工作。}
\colchunk{This option uses |kpsewhich| to locate files that are to be highlighted.  Some build tools such as |texi2pdf| function by modifying |TEXINPUTS|; in some cases, users may customize |TEXINPUTS| as well.  The |kpsewhich| option allows \pkg{minted} to work with such configurations.}
\end{parcolumns}

\begin{parcolumns}[rulebetween=false,colwidths={2=.55\linewidth}]{2}
\colchunk{该选项可能会在某些系统或某些系统配置下增加明显的开销。}
\colchunk{This option may add a noticeable amount of overhead on some systems, or with some system configurations.}
\end{parcolumns}


\begin{parcolumns}[rulebetween=false,colwidths={2=.55\linewidth}]{2}
\colchunk{该选项\emph{不}使得 \pkg{minted} 与 \LaTeX 的 |-output-directory| 和 |-aux-directory| 命令行选项一起工作。对于这些选项,请参见 |outputdir| 包选项。}
\colchunk{This option does \emph{not} make \pkg{minted} work with the |-output-directory| and |-aux-directory| command-line options for \LaTeX.  For those, see the |outputdir| package option.}
\end{parcolumns}



 \DescribeMacro{langlinenos=\meta{boolean} (default:~false)}
 \vspace*{-2\parskip}
\begin{parcolumns}[rulebetween=false,colwidths={2=.55\linewidth}]{2}
\colchunk{\pkg{minted} 在后台使用 \pkg{fancyvrb} 宏包进行代码排版。 \pkg{fancyvrb} 提供了一个选项 |firstnumber|,允许指定环境的起始行号。为了方便起见,还有一个选项 |firstnumber=last|,允许行号从上一个环境结束的地方继续。|langlinenos| 选项使得 |firstnumber| 可以对每个语言进行单独设置,对于所有 |minted| 和 |\mint| 的使用情况。例如,考虑以下代码和输出。}
\colchunk{\pkg{minted} uses the \pkg{fancyvrb} package behind the scenes for the code typesetting.  \pkg{fancyvrb} provides an option |firstnumber| that allows the starting line number of an environment to be specified.  For convenience, there is an option |firstnumber=last| that allows line numbering to pick up where it left off.  The |langlinenos| option makes |firstnumber| work for each language individually with all |minted| and |\mint| usages.  For example, consider the code and output below.}
\end{parcolumns}


 \begin{VerbatimOut}[gobble=1]{minted.doc.out}
   \begin{minted}[linenos]{python}
   def f(x):
       return x**2
   \end{minted}

   \begin{minted}[linenos]{ruby}
   def func
       puts "message"
   end
   \end{minted}

   \begin{minted}[linenos, firstnumber=last]{python}
   def g(x):
       return 2*x
   \end{minted}
 \end{VerbatimOut}
 \inputminted[frame=single, rulecolor=minted@linkcolor]{latex}{minted.doc.out}
 \hfill
 \colorbox{minted@samplebg}{\begin{minipage}{0.6\textwidth}
   \input{minted.doc.out}
 \end{minipage}}
 \hfill\hfill

\begin{parcolumns}[rulebetween=false,colwidths={2=.55\linewidth}]{2}
\colchunk{如果没有使用 |langlinenos| 选项,则第二个 Python 环境中的行号将不会从第一个 Python 环境结束的地方开始编号,而是从 Ruby 的行号开始。}
\colchunk{Without the |langlinenos| option, the line numbering in the second Python environment would not pick up where the first Python environment left off.  Rather, it would pick up with the Ruby line numbering.}
\end{parcolumns}



\DescribeMacro{newfloat=\meta{boolean} (default:~false)}
\vspace*{-2\parskip}
\begin{parcolumns}[rulebetween=false,colwidths={2=.55\linewidth}]{2}
\colchunk{默认情况下,|listing| 环境使用 \pkg{float} 宏包创建。|newfloat| 选项使用 \pkg{newfloat} 来创建该环境,这可以更好地与 \pkg{caption} 宏包集成。}
\colchunk{By default, the |listing| environment is created using the \pkg{float} package.  The |newfloat| option creates the environment using \pkg{newfloat} instead.  This provides better integration with the \pkg{caption} package.}
\end{parcolumns}



\DescribeMacro{outputdir=\meta{directory} (default:~\meta{none})}
\vspace*{-2\parskip}
\begin{parcolumns}[rulebetween=false,colwidths={2=.55\linewidth}]{2}
\colchunk{对于\LaTeX{}中的 |-output-directory| 和 |-aux-directory|(MiKTeX)命令行选项,对于\pkg{minted}会造成问题,因为\pkg{minted}的临时文件保存在 |<outputdir>| 中,但\pkg{minted}仍会在文档根目录中查找它们。无法访问命令行选项的值,以便\pkg{minted}可以自动查找正确的位置。但是,可以将输出目录手动指定为包选项。}
\colchunk{The |-output-directory| and |-aux-directory| (MiKTeX) command-line options for \LaTeX\ cause problems for \pkg{minted}, because the \pkg{minted} temporary files are saved in |<outputdir>|, but \pkg{minted} still looks for them in the document root directory.  There is no way to access the value of the command-line option so that \pkg{minted} can automatically look in the right place.  But it is possible to allow the output directory to be specified manually as a package option.}
\end{parcolumns}


\begin{parcolumns}[rulebetween=false,colwidths={2=.55\linewidth}]{2}
\colchunk{应使用绝对路径或相对于文档根目录的路径来指定输出目录。路径应使用正斜杠,即使在Windows下也是如此。必须转义特殊字符,而空格则需要用引号引起来,并需要用大括号|{}|将整个\meta{directory}包装起来。有关转义和引用的示例,请参见上面的|cachedir|。}
\colchunk{The output directory should be specified using an absolute path or a path relative to the document root directory.  Paths should use forward slashes, even under Windows.  Special characters must be escaped, while spaces require quoting and need the entire \meta{directory} to be wrapped in curly braces |{}|.  See |cachedir| above for examples of escaping and quoting.}
\end{parcolumns}



\DescribeMacro{section}
\vspace*{-2\parskip}
\begin{parcolumns}[rulebetween=false,colwidths={2=.55\linewidth}]{2}
\colchunk{要控制\LaTeX{}如何计数|listing|浮动对象,可以在加载\pkg{minted}包时传递|section|或|chapter|选项。}
\colchunk{To control how \LaTeX{} counts the |listing| floats, you can pass either the
|section| or |chapter| option when loading the \pkg{minted} package.}
\end{parcolumns}

\subsection{Macro option usage\\宏选项用法}

\begin{parcolumns}[rulebetween=false,colwidths={2=.55\linewidth}]{2}
\colchunk{所有\pkg{minted}高亮命令都接受相同的选项集。选项指定为逗号分隔的|key=value|对的列表。例如,我们可以指定行应编号:}
\colchunk{All \pkg{minted} highlighting commands accept the same set of options.
Options are specified as a comma-separated list of |key=value| pairs.
For example, we can specify that the lines should be numbered:}
\end{parcolumns}


% \begin{example}
%   \begin{minted}[linenos=true]{c++}
%   #include <iostream>
%   int main() {
%       std::cout << "Hello "
%                 << "world"
%                 << std::endl;
%   }
%   \end{minted}
% \end{example}

\begin{parcolumns}[rulebetween=false,colwidths={2=.55\linewidth}]{2}
\colchunk{选项值|true|也可以完全省略(包括“|=|”)。要进一步自定义行号的显示,请覆盖|\theFancyVerbLine|命令。有关详细信息,请参阅\pkg{fancyvrb}文档。}
\colchunk{An option value of |true| may also be omitted entirely (including the ``|=|'').
To customize the display of the line numbers further, override the |\theFancyVerbLine| command.
Consult the \pkg{fancyvrb} documentation for details.}
\end{parcolumns}


\begin{parcolumns}[rulebetween=false,colwidths={2=.55\linewidth}]{2}
\colchunk{|\mint|接受相同的选项:}
\colchunk{|\mint| accepts the same options:}
\end{parcolumns}
 

\begin{example}
\mint[linenos]{perl}|$x=~/foo/|
\end{example}

\begin{parcolumns}[rulebetween=false,colwidths={2=.55\linewidth}]{2}
\colchunk{这里是另一个示例:我们想在注释中使用\LaTeX{}数学模式:}
\colchunk{Here's another example: we want to use the \LaTeX{} math mode inside comments:}
\end{parcolumns}

% \begin{example}
%   \begin{minted}[mathescape]{python}
%   # Returns $\sum_{i=1}^{n}i$
%   def sum_from_one_to(n):
%       r = range(1, n + 1)
%       return sum(r)
%   \end{minted}
% \end{example}

\begin{parcolumns}[rulebetween=false,colwidths={2=.55\linewidth}]{2}
\colchunk{为了使您的\LaTeX{}代码更易于阅读,您可能希望在|minted|环境中缩进代码。选项|gobble|从输出中删除这些不必要的空白字符。还有一个|autogobble|选项,它会自动检测这个空格的长度。}
\colchunk{To make your \LaTeX{} code more readable you might want to indent the code inside a |minted|
environment.
The option |gobble| removes these unnecessary whitespace characters from the output.  There is also an |autogobble| option that detects the length of this whitespace automatically.}
\end{parcolumns}


 \begin{example}
   \begin{minted}[gobble=2,
     showspaces]{python}
     def boring(args = None):
         pass
   \end{minted}

   versus

   \begin{minted}[showspaces]{python}
     def boring(args = None):
         pass
   \end{minted}
 \end{example}

 \DescribeMacro{\setminted}
\vspace*{-\parskip}
\begin{parcolumns}[rulebetween=false,colwidths={2=.55\linewidth}]{2}
\colchunk{您可能希望为整个文档或整个语言设置选项。这可以通过使用命令\cmd\setminted\oarg{language}\marg{key=value,...}实现。特定于语言的选项会覆盖文档范围内的选项。个别命令和环境选项会覆盖特定于语言的选项。}
\colchunk{You may wish to set options for the document as a whole, or for an entire language.  This is possible via \cmd\setminted\oarg{language}\marg{key=value,...}.  Language-specific options override document-wide options.  Individual command and environment options override language-specific options.}
\end{parcolumns}

 

 \DescribeMacro{\setmintedinline}
\vspace*{-2\parskip}
\begin{parcolumns}[rulebetween=false,colwidths={2=.55\linewidth}]{2}
\colchunk{您可能希望为\cmd\mintinline 单独设置选项,无论是为整个文档还是为特定语言。这可以通过使用命令\cmd\setmintedinline 实现。语法如下:

\cmd\setmintedinline\oarg{language}\marg{key=value,...}}
\colchunk{You may wish to set separate options for \cmd\mintinline, either for the document as a whole or for a specific language.  This is possible via \cmd\setmintedinline.  The syntax is}
\end{parcolumns}

\begin{parcolumns}[rulebetween=false,colwidths={2=.55\linewidth}]{2}
\colchunk{特定于语言的选项会覆盖文档范围内的选项。个别命令选项会覆盖特定于语言的选项。所有使用\cmd\setmintedinline\ 指定的设置都会覆盖使用\cmd\setminted\ 指定的设置。也就是说,内联设置的优先级始终高于一般设置。}
\colchunk{Language-specific options override document-wide options. Individual command options override language-specific options.  All settings specified with \cmd\setmintedinline\ override those set with \cmd\setminted.  That is, inline settings always have a higher precedence than general settings.}
\end{parcolumns}


 \subsection{Available options\\可用选项}

 \newcommand\appliesto[1]{\textsf{[For #1 only]}}

\begin{parcolumns}[rulebetween=false,colwidths={2=.55\linewidth}]{2}
\colchunk{以下是可用选项的完整列表。有关更详细的选项描述,请参阅\pkg{fancyvrb}和\app{Pygments}文档。}
\colchunk{Following is a full list of available options.
For more detailed option descriptions please refer to the \pkg{fancyvrb} and \app{Pygments} documentation.}
\end{parcolumns}


 
 \begin{optionlist}
   \item[autogobble (boolean) (false)]
\begin{parcolumns}[rulebetween=false,colwidths={2=.55\linewidth}]{2}
\vspace*{-2\parskip}
\colchunk{从代码中删除(gobble)所有常见的前导空格。基本上是自动确定要删除什么的 |gobble| 的版本。适用于最初未缩进但在粘贴到\LaTeX\ 文档后手动缩进的代码。}
\colchunk{Remove (gobble) all common leading whitespace from code.  Essentially a version of |gobble| that automatically determines what should be removed.  Good for code that originally is not indented, but is manually indented after being pasted into a \LaTeX\ document.}
\end{parcolumns}

 \begin{example}
   ...text.
   \begin{minted}[autogobble]{python}
       def f(x):
           return x**2
   \end{minted}
 \end{example}

   \item[baselinestretch (dimension) (\meta{document default})]
\vspace*{-2\parskip}
\begin{parcolumns}[rulebetween=false,colwidths={2=.55\linewidth}]{2}
\colchunk{用于代码列表内部的行距的值。}
\colchunk{Value to use as for baselinestretch inside the listing.}
\end{parcolumns}
     


 \item[beameroverlays (boolean) (false)]
\vspace*{-2\parskip}
\begin{parcolumns}[rulebetween=false,colwidths={2=.55\linewidth}]{2}
\colchunk{在使用 |escapeinside| 和 |texcomments| 时,为 |<| 和 |>| 字符提供其正常的文本含义,以便形式为 |\only<1>{...}| 的\pkg{beamer}叠加效果正常工作。}
\colchunk{Give the |<| and |>| characters their normal text meanings when used with |escapeinside| and |texcomments|, so that \pkg{beamer} overlays of the form |\only<1>{...}| will work.}
\end{parcolumns}



\item[breakafter (string) (\meta{none})]
\vspace*{-2\parskip}
\begin{parcolumns}[rulebetween=false,colwidths={2=.55\linewidth}]{2}
\colchunk{(当\texttt{breaklines=true}时)在指定的字符后换行,而不仅仅在空格处换行。不适用于 |\mintinline|。}
\colchunk{Break lines after specified characters, not just at spaces, when \texttt{breaklines=true}.  Does not apply to |\mintinline|.}
\end{parcolumns}


\begin{parcolumns}[rulebetween=false,colwidths={2=.55\linewidth}]{2}
\colchunk{例如,\texttt{breakafter=-/} 允许在连字符或斜杠后面断行。 \texttt{breakafter} 中给定的特殊字符应进行反斜杠转义(通常为 \texttt{\hashchar}, \texttt{\{}, \texttt{\}}, \texttt{\%}, \texttt{[}, \texttt{]};通过 \texttt{\textbackslash\textbackslash} 可以获得反斜杠 \texttt{\textbackslash})。}
\colchunk{For example, \texttt{breakafter=-/} would allow breaks after any hyphens or slashes.  Special characters given to \texttt{breakafter} should be backslash-escaped (usually \texttt{\hashchar}, \texttt{\{}, \texttt{\}}, \texttt{\%}, \texttt{[}, \texttt{]}; the backslash \texttt{\textbackslash} may be obtained via \texttt{\textbackslash\textbackslash}).}
\end{parcolumns}


\begin{parcolumns}[rulebetween=false,colwidths={2=.5\linewidth}]{2}
\colchunk{另一种选择,请参见 \texttt{breakbefore}。当为相同字符使用 \texttt{breakbefore} 和 \texttt{breakafter} 时,\texttt{breakbeforegroup} 和 \texttt{breakaftergroup} 必须具有相同的设置。}
\colchunk{For an alternative, see \texttt{breakbefore}.  When \texttt{breakbefore} and \texttt{breakafter} are used for the same character, \texttt{breakbeforegroup} and \texttt{breakaftergroup} must both have the same setting.}
\end{parcolumns}


 \begin{longexample}
   \begin{minted}[breaklines, breakafter=d]{python}
   some_string = 'SomeTextThatGoesOnAndOnForSoLongThatItCouldNeverFitOnOneLine'
   \end{minted}
 \end{longexample}

   \item[breakaftergroup] (boolean) (true)
\vspace*{-2\parskip}
\begin{parcolumns}[rulebetween=false,colwidths={2=.5\linewidth}]{2}
\colchunk{当使用 \texttt{breakafter} 时,将所有相邻的相同字符分组在一起,并仅在最后一个字符后允许断行。当为相同字符使用 \texttt{breakbefore} 和 \texttt{breakafter} 时,\texttt{breakbeforegroup} 和 \texttt{breakaftergroup} 必须具有相同的设置。}
\colchunk{When \texttt{breakafter} is used, group all adjacent identical characters together, and only allow a break after the last character.  When \texttt{breakbefore} and \texttt{breakafter} are used for the same character, \texttt{breakbeforegroup} and \texttt{breakaftergroup} must both have the same setting.}
\end{parcolumns}


   \item[breakaftersymbolpre (string) (\string\,\string\footnotesize\string\ensuremath\{\_\string\rfloor\}, \,\footnotesize\ensuremath{_\rfloor})]
\vspace*{-2\parskip}
\begin{parcolumns}[rulebetween=false,colwidths={2=.55\linewidth}]{2}
\colchunk{在由 \texttt{breakafter} 插入的换行符前插入的符号。}
\colchunk{The symbol inserted pre-break for breaks inserted by \texttt{breakafter}.}
\end{parcolumns}
     

   \item[breakaftersymbolpost (string) (\meta{none})]
\vspace*{-2\parskip}
\begin{parcolumns}[rulebetween=false,colwidths={2=.55\linewidth}]{2}
\colchunk{在由 \texttt{breakafter} 插入的换行符后插入的符号。}
\colchunk{The symbol inserted post-break for breaks inserted by \texttt{breakafter}.}
\end{parcolumns}
     

   \item[breakanywhere (boolean) (false)]
\vspace*{-2\parskip}
\begin{parcolumns}[rulebetween=false,colwidths={2=.55\linewidth}]{2}
\colchunk{在 \texttt{breaklines = true} 时,不仅在空格处,而是在任何位置断行。不适用于 |\mintinline|。}
\colchunk{Break lines anywhere, not just at spaces, when \texttt{breaklines=true}.  Does not apply to |\mintinline|.}
\end{parcolumns}


 \begingroup
 \fvset{xrightmargin=1em}
 \begin{longexample}
   \begin{minted}[breaklines, breakanywhere]{python}
   some_string = 'SomeTextThatGoesOnAndOnForSoLongThatItCouldNeverFitOnOneLine'
   \end{minted}
 \end{longexample}
 \endgroup

   \item[breakanywheresymbolpre (string) (\string\,\string\footnotesize\string\ensuremath\{\_\string\rfloor\}, \,\footnotesize\ensuremath{_\rfloor})]
\vspace*{-2\parskip}
\begin{parcolumns}[rulebetween=false,colwidths={2=.55\linewidth}]{2}
\colchunk{在由 \texttt{breakanywhere} 插入的换行符前插入的符号。}
\colchunk{The symbol inserted pre-break for breaks inserted by \texttt{breakanywhere}.}
\end{parcolumns}   


   \item[breakanywheresymbolpost (string) (\meta{none})]
\vspace*{-2\parskip}
\begin{parcolumns}[rulebetween=false,colwidths={2=.55\linewidth}]{2}
\colchunk{在由 \texttt{breakanywhere} 插入的换行符后插入的符号。}
\colchunk{The symbol inserted post-break for breaks inserted by \texttt{breakanywhere}.}
\end{parcolumns}


   \item[breakautoindent (boolean) (true)]
\vspace*{-2\parskip}
\begin{parcolumns}[rulebetween=false,colwidths={2=.55\linewidth}]{2}
\colchunk{自动缩进延续行,使其缩进等级与第一行相同。当同时使用 \texttt{breakautoindent} 和 \texttt{breakindent} 时,缩进会加和。此缩进与 \texttt{breaksymbolindentleft} 组合以给出总实际左缩进。不适用于 \texttt{\string\mintinline}。}
\colchunk{When a line is broken, automatically indent the continuation lines to the indentation level of the first line.  When \texttt{breakautoindent} and \texttt{breakindent} are used together, the indentations add.  This  indentation is combined with \texttt{breaksymbolindentleft} to give the total actual left indentation.  Does not apply to \texttt{\string\mintinline}.}
\end{parcolumns}


   \item[breakbefore (string) (\meta{none})]
\vspace*{-2\parskip}
\begin{parcolumns}[rulebetween=false,colwidths={2=.55\linewidth}]{2}
\colchunk{当 \texttt{breaklines=true} 时,在指定字符之前而非仅在空格处断行。不适用于 |\mintinline|。}
\colchunk{Break lines before specified characters, not just at spaces, when \texttt{breaklines=true}.  Does not apply to |\mintinline|.}
\end{parcolumns}


\begin{parcolumns}[rulebetween=false,colwidths={2=.55\linewidth}]{2}
\colchunk{例如,\texttt{breakbefore=A} 允许在大写字母 A 之前断行。应对特殊字符应使用反斜杠转义 (通常为 \texttt{\hashchar}, \texttt{\{}, \texttt{\}}, \texttt{\%}, \texttt{[}, \texttt{]};反斜杠 \texttt{\textbackslash} 可通过 \texttt{\textbackslash\textbackslash} 获得)。}
\colchunk{For example, \texttt{breakbefore=A} would allow breaks before capital A's.  Special characters given to \texttt{breakbefore} should be backslash-escaped (usually \texttt{\hashchar}, \texttt{\{}, \texttt{\}}, \texttt{\%}, \texttt{[}, \texttt{]}; the backslash \texttt{\textbackslash} may be obtained via \texttt{\textbackslash\textbackslash}).}
\end{parcolumns}


\begin{parcolumns}[rulebetween=false,colwidths={2=.55\linewidth}]{2}
\colchunk{另请参见 \texttt{breakafter}。当 \texttt{breakbefore} 和 \texttt{breakafter} 用于相同字符时,\texttt{breakbeforegroup} 和 \texttt{breakaftergroup} 必须具有相同的设置。}
\colchunk{For an alternative, see \texttt{breakafter}.  When \texttt{breakbefore} and \texttt{breakafter} are used for the same character, \texttt{breakbeforegroup} and \texttt{breakaftergroup} must both have the same setting.}
\end{parcolumns}


 \begin{longexample}
   \begin{minted}[breaklines, breakbefore=A]{python}
   some_string = 'SomeTextThatGoesOnAndOnForSoLongThatItCouldNeverFitOnOneLine'
   \end{minted}
 \end{longexample}

\item[breakbeforegroup] (boolean) (true)
\vspace*{-2\parskip}
\begin{parcolumns}[rulebetween=false,colwidths={2=.55\linewidth}]{2}
\colchunk{当使用 \texttt{breakbefore} 时,将所有相邻的相同字符分组在一起,并仅在第一个字符之前允许断行。当 \texttt{breakbefore} 和 \texttt{breakafter} 用于相同字符时,\texttt{breakbeforegroup} 和 \texttt{breakaftergroup} 必须具有相同的设置。}
\colchunk{When \texttt{breakbefore} is used, group all adjacent identical characters together, and only allow a break before the first character.  When \texttt{breakbefore} and \texttt{breakafter} are used for the same character, \texttt{breakbeforegroup} and \texttt{breakaftergroup} must both have the same setting.}
\end{parcolumns}


\item[breakbeforesymbolpre (string) (\string\,\string\footnotesize\string\ensuremath\{\_\string\rfloor\}, \,\footnotesize\ensuremath{_\rfloor})]
\vspace*{-2\parskip}
\begin{parcolumns}[rulebetween=false,colwidths={2=.55\linewidth}]{2}
\colchunk{用于由 \texttt{breakbefore} 插入断行的前置符号。}
\colchunk{The symbol inserted pre-break for breaks inserted by \texttt{breakbefore}.}
\end{parcolumns}


\item[breakbeforesymbolpost (string) (\meta{none})]
\vspace*{-2\parskip}
\begin{parcolumns}[rulebetween=false,colwidths={2=.55\linewidth}]{2}
\colchunk{用于由 \texttt{breakbefore} 插入断行的后置符号。}
\colchunk{The symbol inserted post-break for breaks inserted by \texttt{breakbefore}.}
\end{parcolumns}


\item[breakbytoken (boolean) (false)]
\vspace*{-2\parskip}
\begin{parcolumns}[rulebetween=false,colwidths={2=.55\linewidth}]{2}
\colchunk{仅在不在标记中的位置处断行;防止标记被行中断开。默认情况下,\texttt{breaklines} 在最靠近边缘的空格处断行。虽然这可以使必需的换行数最小化,但如果在字符串或类似标记的中间出现换行,则可能会不方便。}
\colchunk{Only break lines at locations that are not within tokens; prevent tokens from being split by line breaks.  By default, \texttt{breaklines} causes line breaking at the space nearest the margin.  While this minimizes the number of line breaks that are necessary, it can be inconvenient if a break occurs in the middle of a string or similar token.}
\end{parcolumns}


\begin{parcolumns}[rulebetween=false,colwidths={2=.55\linewidth}]{2}
\colchunk{这与\texttt{draft}模式不兼容。Pygments标记的完整列表可在\url{http://pygments.org/docs/tokens/}上找到。如果\texttt{breakbytoken}提供的断点出现在意外的位置,则可能表明该语言的Pygments词法分析器存在缺陷或短板。}
\colchunk{This is not compatible with \texttt{draft} mode.  A complete list of Pygments tokens is available at \url{http://pygments.org/docs/tokens/}.  If the breaks provided by \texttt{breakbytoken} occur in unexpected locations, it may indicate a bug or shortcoming in the Pygments lexer for the language.}
\end{parcolumns}


\item[breakbytokenanywhere (boolean) (false)] 
\vspace*{-2\parskip}
\begin{parcolumns}[rulebetween=false,colwidths={2=.55\linewidth}]{2}
\colchunk{类似于\texttt{breakbytoken},但还允许在紧邻的标记之间,而不仅仅是在由空格分隔的标记之间换行。与\texttt{breakanywhere}一起使用\texttt{breakbytokenanywhere}是多余的。}
\colchunk{Like \texttt{breakbytoken}, but also allows line breaks between immediately adjacent tokens, not just between tokens that are separated by spaces.  Using \texttt{breakbytokenanywhere} with \texttt{breakanywhere} is redundant.}
\end{parcolumns}


\item[breakindent (dimension) (\meta{breakindentnchars})]
\vspace*{-2\parskip}
\begin{parcolumns}[rulebetween=false,colwidths={2=.55\linewidth}]{2}
\colchunk{当断行时,将续行缩进这么多。当同时使用|breakautoindent|和|breakindent|时,缩进会相加。此缩进与|breaksymbolindentleft|结合使用,以给出总实际左缩进量。}
\colchunk{When a line is broken, indent the continuation lines by this amount.  When |breakautoindent| and |breakindent| are used together, the indentations add.  This  indentation is combined with |breaksymbolindentleft| to give the total actual left indentation.}
\end{parcolumns}

\begin{parcolumns}[rulebetween=false,colwidths={2=.55\linewidth}]{2}
\colchunk{不适用于\texttt{\string\mintinline}。}
\colchunk{Does not apply to \texttt{\string\mintinline}.}
\end{parcolumns}


\item[breakindentnchars (integer) (0)]
\vspace*{-2\parskip}
\begin{parcolumns}[rulebetween=false,colwidths={2=.55\linewidth}]{2}
\colchunk{这允许将|breakindent|指定为字符数的整数,而不是作为维度(假定为固定宽度字体)。}
\colchunk{This allows |breakindent| to be specified as an integer number of characters rather than as a dimension (assumes a fixed-width font).}
\end{parcolumns}


\item[breaklines (boolean) (false)] 
\vspace*{-2\parskip}
\begin{parcolumns}[rulebetween=false,colwidths={2=.55\linewidth}]{2}
\colchunk{在\texttt{minted}环境和\texttt{\string\mint}命令中自动换行长行,并在\texttt{\string\mintinline}中换行较长的行。}
\colchunk{Automatically break long lines in \texttt{minted} environments and \texttt{\string\mint} commands, and wrap longer lines in \texttt{\string\mintinline}.}
\end{parcolumns}



\begin{parcolumns}[rulebetween=false,colwidths={2=.55\linewidth}]{2}
\colchunk{默认情况下,在空格字符处自动换行。使用\texttt{breakanywhere}启用任意位置的换行;使用\texttt{breakbytoken}、\texttt{breakbytokenanywhere}、\texttt{breakbefore}和\texttt{breakafter}进行更精细的换行。目前,仅\texttt{breakbytoken}和\texttt{breakbytokenanywhere}与|\mintinline|一起使用。使用\texttt{escapeinside}转义到\LaTeX 并插入手动换行也是一种选择。例如,使用\texttt{escapeinside=||},然后在适当的位置插入\texttt{|\textbackslash\textbackslash|}。(请注意,\texttt{escapeinside}不适用于字符串内部。)}
\colchunk{By default, automatic breaks occur at space characters.  Use \texttt{breakanywhere} to enable breaking anywhere; use \texttt{breakbytoken}, \texttt{breakbytokenanywhere}, \texttt{breakbefore}, and \texttt{breakafter} for more fine-tuned breaking.  Currently, only \texttt{breakbytoken} and \texttt{breakbytokenanywhere} work with |\mintinline|.  Using \texttt{escapeinside} to escape to \LaTeX\ and then insert a manual break is also an option.  For example, use \texttt{escapeinside=||}, and then insert \texttt{|\textbackslash\textbackslash|} at the appropriate point.  (Note that \texttt{escapeinside} does not work within strings.)}
\end{parcolumns}


 \begin{example}
   ...text.
   \begin{minted}[breaklines]{python}
   def f(x):
       return 'Some text ' + str(x)
   \end{minted}
 \end{example}


\begin{parcolumns}[rulebetween=false,colwidths={2=.55\linewidth}]{2}
\colchunk{\texttt{minted}和\texttt{\string\mint}中的断行可以通过多种方式自定义。要自定义断行行的缩进,请参见\texttt{breakindent}和\texttt{breakautoindent}。要自定义行延续符号,请使用\texttt{breaksymbolleft}和\texttt{breaksymbolright}。要自定义延续符号与代码之间的间隔,请使用\texttt{breaksymbolsepleft}和\texttt{breaksymbolsepright}。要自定义为断裂符号提供的额外缩进,请使用\texttt{breaksymbolindentleft}和\texttt{breaksymbolindentright}。由于默认情况下仅使用左手符号,因此还可以使用别名选项\texttt{breaksymbol},\texttt{breaksymbolsep}和\texttt{breaksymbolindent}进行修改。请注意,由于在行内上下文中不相关,因此这些选项均不适用于\texttt{\string\mintinline}。}
\colchunk{Breaking in \texttt{minted} and \texttt{\string\mint} may be customized in several ways.  To customize the indentation of broken lines, see \texttt{breakindent} and \texttt{breakautoindent}.  To customize the line continuation symbols, use \texttt{breaksymbolleft} and \texttt{breaksymbolright}.  To customize the separation between the continuation symbols and the code, use \texttt{breaksymbolsepleft} and \texttt{breaksymbolsepright}.  To customize the extra indentation that is supplied to make room for the break symbols, use \texttt{breaksymbolindentleft} and \texttt{breaksymbolindentright}.  Since only the left-hand symbol is used by default, it may also be modified using the alias options \texttt{breaksymbol}, \texttt{breaksymbolsep}, and \texttt{breaksymbolindent}.  Note than none of these options applies to \texttt{\string\mintinline}, since they are not relevant in the inline context.}
\end{parcolumns}

^^A \vspace*{-2\parskip}
\begin{parcolumns}[rulebetween=false,colwidths={2=.55\linewidth}]{2}
\colchunk{下面是一个使用这些选项自定义\texttt{minted}环境的示例。它使用\pkg{dingbat}包中的\texttt{\string\carriagereturn}符号。}
\colchunk{An example using these options to customize the \texttt{minted} environment is shown below.  This uses the \texttt{\string\carriagereturn} symbol from the \pkg{dingbat} package.}
\end{parcolumns}
 

 \begingroup
 \fvset{breaklines, xleftmargin=2em, xrightmargin=2em}
 \begin{longexample}
   \begin{minted}[breaklines,
                  breakautoindent=false,
                  breaksymbolleft=\raisebox{0.8ex}{
                    \small\reflectbox{\carriagereturn}},
                  breaksymbolindentleft=0pt,
                  breaksymbolsepleft=0pt,
                  breaksymbolright=\small\carriagereturn,
                  breaksymbolindentright=0pt,
                  breaksymbolsepright=0pt]{python}
   def f(x):
       return 'Some text ' + str(x) + ' some more text ' + str(x) + ' even more text that goes on for a while'
   \end{minted}
 \end{longexample}
 \endgroup

 ^^A \vspace*{-2\parskip}
\begin{parcolumns}[rulebetween=false,colwidths={2=.55\linewidth}]{2}
\colchunk{使用带有大量文本背景色的\app{Pygments}样式的自动换行受到限制。这种着色是通过\texttt{\string\colorbox}实现的,它无法跨行换行。可能可以创建一个支持换行的\texttt{\string\colorbox}替代方案,例如使用\pkg{TikZ},但作者不知道有什么令人满意的解决方案。唯一的当前替代方案是重新定义\texttt{\string\colorbox},使其不起作用。例如,}
\colchunk{Automatic line breaks are limited with \app{Pygments} styles that use a colored background behind large chunks of text.  This coloring is accomplished with \texttt{\string\colorbox}, which cannot break across lines.  It may be possible to create an alternative to \texttt{\string\colorbox} that supports line breaks, perhaps with \pkg{TikZ}, but the author is unaware of a satisfactory solution.  The only current alternative is to redefine \texttt{\string\colorbox} so that it does nothing.  For example,}
\end{parcolumns}
     
\begin{Verbatim}
\AtBeginEnvironment{minted}{\renewcommand{\colorbox}[3][]{#3}}
\end{Verbatim}
^^A \vspace*{-2\parskip}
\begin{parcolumns}[rulebetween=false,colwidths={2=.55\linewidth}]{2}
\colchunk{使用\pkg{etoolbox}包在所有\texttt{minted}环境中重新定义\texttt{\string\colorbox}。}
\colchunk{uses the \pkg{etoolbox} package to redefine \texttt{\string\colorbox} within all \texttt{minted} environments.}
\end{parcolumns}
     

    ^^A \vspace*{-2\parskip}
\begin{parcolumns}[rulebetween=false,colwidths={2=.55\linewidth}]{2}
\colchunk{如果您使用\texttt{showspaces=true},则自动换行将不起作用,除非您使用\texttt{breakanywhere}或\texttt{breakafter=\string\space}。}
\colchunk{Automatic line breaks will not work with \texttt{showspaces=true} unless you use \texttt{breakanywhere} or \texttt{breakafter=\string\space}.}
\end{parcolumns}
     

   \item[breaksymbol (string) (breaksymbolleft)]
\vspace*{-2\parskip}
\begin{parcolumns}[rulebetween=false,colwidths={2=.55\linewidth}]{2}
\colchunk{\texttt{breaksymbolleft} 的别名。}
\colchunk{Alias for \texttt{breaksymbolleft}.}
\end{parcolumns}
     

   \item[breaksymbolleft (string) (\string\tiny\string\ensuremath\{\string\hookrightarrow\}, {\tiny\ensuremath{\hookrightarrow}})]
\vspace*{-2\parskip}
\begin{parcolumns}[rulebetween=false,colwidths={2=.55\linewidth}]{2}
\colchunk{在\texttt{breaklines=true}时,用于表示续行的符号。若要没有符号,则将 \texttt{breaksymbolleft} 设置为空字符串(“\texttt{=,}”或“\texttt{={}}”)即可。当使用时,该符号被包装在花括号 \texttt{{}} 中,因此不会出现格式命令(如\texttt{\string\tiny})的“逃逸”。}
\colchunk{The symbol used at the beginning (left) of continuation lines when \texttt{breaklines=true}.  To have no symbol, simply set \texttt{breaksymbolleft} to an empty string (``\texttt{=,}'' or ``\texttt{=\{\}}'').  The symbol is wrapped within curly braces \texttt{\{\}} when used, so there is no danger of formatting commands such as \texttt{\string\tiny} ``escaping.''}
\end{parcolumns}
     
\begin{parcolumns}[rulebetween=false,colwidths={2=.55\linewidth}]{2}
\colchunk{\texttt{\string\hookrightarrow} 和 \texttt{\string\hookleftarrow} 可通过 \pkg{graphicx} 提供的 \texttt{\string\rotatebox} 命令进行进一步定制。其他可能有用的箭头类型符号可在 \pkg{dingbat}(\texttt{\string\carriagereturn})和 \pkg{mnsymbol}(钩和曲线箭头)等包中获得。}
\colchunk{The \texttt{\string\hookrightarrow} and \texttt{\string\hookleftarrow} may be further customized by the use of the \texttt{\string\rotatebox} command provided by \pkg{graphicx}.  Additional arrow-type symbols that may be useful are available in the \pkg{dingbat} (\texttt{\string\carriagereturn}) and \pkg{mnsymbol} (hook and curve arrows) packages, among others.}
\end{parcolumns}
     

     \begin{parcolumns}[rulebetween=false,colwidths={2=.55\linewidth}]{2}
\colchunk{不适用于 \texttt{\string\mintinline}。}
\colchunk{Does not apply to \texttt{\string\mintinline}.}
\end{parcolumns}
     

   \item[breaksymbolright (string) (\meta{none})]
\vspace*{-2\parskip}
\begin{parcolumns}[rulebetween=false,colwidths={2=.55\linewidth}]{2}
\colchunk{在 \texttt{breaklines=true} 时,在断点(右侧)使用的符号。不会出现在断行的最后一个片段的末尾。}
\colchunk{The symbol used at breaks (right) when \texttt{breaklines=true}. Does not appear at the end of the very last segment of a broken line.}
\end{parcolumns}
     


 \item[breaksymbolindent (dimension) (\meta{breaksymbolindentleftnchars})]
\vspace*{-2\parskip}
\begin{parcolumns}[rulebetween=false,colwidths={2=.55\linewidth}]{2}
\colchunk{\texttt{breaksymbolindentleft} 的别名。}
\colchunk{Alias for |breaksymbolindentleft|.}
\end{parcolumns}
 

 \item[breaksymbolindentnchars (integer) (\meta{breaksymbolindentleftnchars})]
 \vspace*{-2\parskip}
\begin{parcolumns}[rulebetween=false,colwidths={2=.55\linewidth}]{2}
\colchunk{\texttt{breaksymbolindentleftnchars} 的别名。}
\colchunk{Alias for |breaksymbolindentleftnchars|.}
\end{parcolumns}
 


 \item[breaksymbolindentleft (dimension) (\meta{breaksymbolindentleftnchars})]
 \vspace*{-2\parskip}
\begin{parcolumns}[rulebetween=false,colwidths={2=.55\linewidth}]{2}
\colchunk{为给 |breaksymbolleft| 留出空间而提供的额外左缩进。仅当存在 |breaksymbolleft| 时才应用此缩进。}
\colchunk{The extra left indentation that is provided to make room for |breaksymbolleft|.  This indentation is only applied when there is a |breaksymbolleft|.}

\colplacechunks
\colchunk{不适用于 \texttt{\string\mintinline}。}
\colchunk{Does not apply to \texttt{\string\mintinline}.}
\end{parcolumns}
 

     

 \item[breaksymbolindentleftnchars (integer) (4)]
 \vspace*{-2\parskip}
\begin{parcolumns}[rulebetween=false,colwidths={2=.55\linewidth}]{2}
\colchunk{这允许将 |breaksymbolindentleft| 指定为字符的整数数量,而不是维度(假定为等宽字体)。}
\colchunk{This allows |breaksymbolindentleft| to be specified as an integer number of characters rather than as a dimension (assumes a fixed-width font).}
\end{parcolumns}
 

 \item[breaksymbolindentright (dimension) (\meta{breaksymbolindentrightnchars})]
 \vspace*{-2\parskip}
\begin{parcolumns}[rulebetween=false,colwidths={2=.55\linewidth}]{2}
\colchunk{为给 |breaksymbolright| 留出空间而提供的额外右缩进。仅当存在 |breaksymbolright| 时才应用此缩进。}
\colchunk{The extra right indentation that is provided to make room for |breaksymbolright|.  This indentation is only applied when there is a |breaksymbolright|.}
\end{parcolumns}
 

 \item[breaksymbolindentrightnchars (integer) (4)]
 \vspace*{-2\parskip}
\begin{parcolumns}[rulebetween=false,colwidths={2=.55\linewidth}]{2}
\colchunk{这允许将 |breaksymbolindentright| 指定为字符的整数数量,而不是维度(假定为等宽字体)。}
\colchunk{This allows |breaksymbolindentright| to be specified as an integer number of characters rather than as a dimension (assumes a fixed-width font).}
\end{parcolumns}
 


 \item[breaksymbolsep (dimension) (\meta{breaksymbolsepleftnchars})]
 \vspace*{-2\parskip}
\begin{parcolumns}[rulebetween=false,colwidths={2=.55\linewidth}]{2}
\colchunk{\texttt{breaksymbolsepleft} 的别名。}
\colchunk{Alias for |breaksymbolsepleft|.}
\end{parcolumns}
 

 \item[breaksymbolsepnchars (integer) (\meta{breaksymbolsepleftnchars})]
 ^^A \vspace*{-2\parskip}
^^A \begin{parcolumns}[rulebetween=false,colwidths={2=.55\linewidth}]{2}
^^A \colchunk{}
^^A \colchunk{}
^^A \end{parcolumns}
 Alias for |breaksymbolsepleftnchars|.


 \item[breaksymbolsepleft (dimension) (\meta{breaksymbolsepleftnchars})]
 ^^A \vspace*{-2\parskip}
^^A \begin{parcolumns}[rulebetween=false,colwidths={2=.55\linewidth}]{2}
^^A \colchunk{}
^^A \colchunk{}
^^A \end{parcolumns}
 The separation between the |breaksymbolleft| and the adjacent text. 

 \item[breaksymbolsepleftnchars (integer) (2)]
 ^^A \vspace*{-2\parskip}
^^A \begin{parcolumns}[rulebetween=false,colwidths={2=.55\linewidth}]{2}
^^A \colchunk{}
^^A \colchunk{}
^^A \end{parcolumns}
 Allows |breaksymbolsepleft| to be specified as an integer number of characters rather than as a dimension (assumes a fixed-width font).


 \item[breaksymbolsepright (dimension) (\meta{breaksymbolseprightnchars})]
 ^^A \vspace*{-2\parskip}
^^A \begin{parcolumns}[rulebetween=false,colwidths={2=.55\linewidth}]{2}
^^A \colchunk{}
^^A \colchunk{}
^^A \end{parcolumns}
 The \emph{minimum} separation between the |breaksymbolright| and the adjacent text.  This is the separation between |breaksymbolright| and the furthest extent to which adjacent text could reach.  In practice, |\linewidth| will typically not be an exact integer multiple of the character width (assuming a fixed-width font), so the actual separation between the |breaksymbolright| and adjacent text will generally be larger than |breaksymbolsepright|.  This ensures that break symbols have the same spacing from the margins on both left and right.  If the same spacing from text is desired instead, |breaksymbolsepright| may be adjusted.  (See the definition of |\FV@makeLineNumber| in \pkg{fvextra} for implementation details.)

 \item[breaksymbolseprightnchars (integer) (2)]
 ^^A \vspace*{-2\parskip}
^^A \begin{parcolumns}[rulebetween=false,colwidths={2=.55\linewidth}]{2}
^^A \colchunk{}
^^A \colchunk{}
^^A \end{parcolumns}
 Allows |breaksymbolsepright| to be specified as an integer number of characters rather than as a dimension (assumes a fixed-width font).


   \item[bgcolor (string) (\meta{none})]
   ^^A \vspace*{-2\parskip}
^^A \begin{parcolumns}[rulebetween=false,colwidths={2=.55\linewidth}]{2}
^^A \colchunk{}
^^A \colchunk{}
^^A \end{parcolumns}
     Background color of the listing.  Be aware that this option has several limitations (described below); see ``Framing alternatives'' below for more powerful alternatives.

     ^^A \vspace*{-2\parskip}
^^A \begin{parcolumns}[rulebetween=false,colwidths={2=.55\linewidth}]{2}
^^A \colchunk{}
^^A \colchunk{}
^^A \end{parcolumns}
     The value of this option must \emph{not} be a color command. Instead, it must be a color \emph{name}, given as a string, of a previously-defined color:

 \begin{example}
   \definecolor{bg}{rgb}{0.95,0.95,0.95}
   \begin{minted}[bgcolor=bg]{php}
   <?php
     echo "Hello, $x";
   ?>
   \end{minted}
 \end{example}

 ^^A \vspace*{-2\parskip}
^^A \begin{parcolumns}[rulebetween=false,colwidths={2=.55\linewidth}]{2}
^^A \colchunk{}
^^A \colchunk{}
^^A \end{parcolumns}
 This option puts \texttt{minted} environments and \texttt{\string\mint} commands in a \texttt{snugshade*} environment from the \pkg{framed} package, which supports breaks across pages.  (Prior to \pkg{minted} 2.2, a \texttt{minipage} was used, which prevented page breaks and gave undesirable spacing from surrounding text.)  Be aware that if \texttt{bgcolor} is used with \texttt{breaklines=true}, and a line break occurs just before a page break, then text may extend below the colored background in some instances.  It is best to use a more advanced framing package in those cases; see ``Framing alternatives'' below.

 ^^A \vspace*{-2\parskip}
^^A \begin{parcolumns}[rulebetween=false,colwidths={2=.55\linewidth}]{2}
^^A \colchunk{}
^^A \colchunk{}
^^A \end{parcolumns}
 This option puts \texttt{\string\mintinline} inside a \texttt{\string\colorbox}, which \textbf{does not allow line breaks}.  If you want to use \texttt{\string\setminted} to set background colors, and only want background colors on \texttt{minted} and \texttt{\string\mint}, you may use \texttt{\string\setmintedinline\{bgcolor=\{\}\}} to turn off the coloring for inline commands.

 ~

 \textbf{Framing alternatives}

 ^^A \vspace*{-2\parskip}
^^A \begin{parcolumns}[rulebetween=false,colwidths={2=.55\linewidth}]{2}
^^A \colchunk{}
^^A \colchunk{}
^^A \end{parcolumns}
 If you want more reliable and advanced options for background colors and framing, you should consider a more advanced framing package such as \pkg{mdframed} or \pkg{tcolorbox}.  It is easy to add framing to \pkg{minted} commands and environments using the \pkg{etoolbox} package, which is automatically loaded by \pkg{minted}.  For example, using \pkg{mdframed}:
\begin{Verbatim}
\BeforeBeginEnvironment{minted}{\begin{mdframed}}
\AfterEndEnvironment{minted}{\end{mdframed}}
\end{Verbatim}
^^A \vspace*{-2\parskip}
^^A \begin{parcolumns}[rulebetween=false,colwidths={2=.55\linewidth}]{2}
^^A \colchunk{}
^^A \colchunk{}
^^A \end{parcolumns}
 Some framing packages also provide built-in commands for such purposes.  For example, \pkg{mdframed} provides a \texttt{\string\surroundwithmdframed} command, which could be used to add a frame to all \texttt{minted} environments:
\begin{Verbatim}
\surroundwithmdframed{minted}
\end{Verbatim}
^^A \vspace*{-2\parskip}
^^A \begin{parcolumns}[rulebetween=false,colwidths={2=.55\linewidth}]{2}
^^A \colchunk{}
^^A \colchunk{}
^^A \end{parcolumns}
 \pkg{tcolorbox} even provides a built-in framing environment with \pkg{minted} support.  Simply use \texttt{\string\tcbuselibrary\{minted\}} in the preamble, and then put code within a \texttt{tcblisting} environment:
\begin{Verbatim}
\begin{tcblisting}{<tcb options>,
                   minted language=<language>,
                   minted style=<style>,
                   minted options={<option list>} }
<code>
\end{tcblisting}
\end{Verbatim}
^^A \vspace*{-2\parskip}
^^A \begin{parcolumns}[rulebetween=false,colwidths={2=.55\linewidth}]{2}
^^A \colchunk{}
^^A \colchunk{}
^^A \end{parcolumns}
 \pkg{tcolorbox} provides other commands and environments for fine-tuning listing appearance and for working with external code files.

   \item[codetagify (list of strings) (\textrm{highlight \texttt{XXX}, \texttt{TODO}, \texttt{BUG}, and \texttt{NOTE}})]
   ^^A \vspace*{-2\parskip}
^^A \begin{parcolumns}[rulebetween=false,colwidths={2=.55\linewidth}]{2}
^^A \colchunk{}
^^A \colchunk{}
^^A \end{parcolumns}
     Highlight special code tags in comments and docstrings.

 \item[curlyquotes (boolean) (false)]
 ^^A \vspace*{-2\parskip}
^^A \begin{parcolumns}[rulebetween=false,colwidths={2=.55\linewidth}]{2}
^^A \colchunk{}
^^A \colchunk{}
^^A \end{parcolumns}
 By default, the backtick \texttt{\textasciigrave} and typewriter single quotation mark \texttt{\textquotesingle} always appear literally, instead of becoming the left and right curly single quotation marks \texttt{`'}.  This option allows these characters to be replaced by the curly quotation marks when that is desirable.

   \item[encoding (string) (\meta{system-specific})]
   ^^A \vspace*{-2\parskip}
^^A \begin{parcolumns}[rulebetween=false,colwidths={2=.55\linewidth}]{2}
^^A \colchunk{}
^^A \colchunk{}
^^A \end{parcolumns}
     Sets the file encoding that \app{Pygments} expects.  See also |outencoding|.
   \item[escapeinside (string) (\meta{none})]
   ^^A \vspace*{-2\parskip}
^^A \begin{parcolumns}[rulebetween=false,colwidths={2=.55\linewidth}]{2}
^^A \colchunk{}
^^A \colchunk{}
^^A \end{parcolumns}
   Escape to \LaTeX\ between the two characters specified in \texttt{\string(string\string)}.  All code between the two characters will be interpreted as \LaTeX\ and typeset accordingly.  This allows for additional formatting.  The escape characters need not be identical.  Special \LaTeX\ characters must be escaped when they are used as the escape characters (for example, \texttt{escapeinside=\textbackslash\#\textbackslash\%}).  Requires \app{Pygments} 2.0+.

   ^^A \vspace*{-2\parskip}
^^A \begin{parcolumns}[rulebetween=false,colwidths={2=.55\linewidth}]{2}
^^A \colchunk{}
^^A \colchunk{}
^^A \end{parcolumns}
 \textbf{Escaping does not work inside strings and comments (for comments, there is \texttt{texcomments}).  As of Pygments 2.0.2, this means that escaping is ``fragile'' with some lexers.}  Due to the way that Pygments implements \texttt{escapeinside}, any ``escaped'' \LaTeX\ code that resembles a string or comment for the current lexer may break \texttt{escapeinside}.  There is a \href{https://bitbucket.org/birkenfeld/pygments-main/issue/1118}{Pygments issue} for this case.  Additional details and a limited workaround for some scenarios are available on the \href{https://github.com/gpoore/minted/issues/70#issuecomment-111729930}{\pkg{minted} GitHub site}.

 \begingroup  ^^A Need to prevent active "|" from causing problems
 \catcode`\|=11
 \begin{example}
   \begin{minted}[escapeinside=||]{py}
   def f(x):
       y = x|\colorbox{green}{**}|2
       return y
   \end{minted}
 \end{example}
 \endgroup

 ^^A \vspace*{-2\parskip}
^^A \begin{parcolumns}[rulebetween=false,colwidths={2=.55\linewidth}]{2}
^^A \colchunk{}
^^A \colchunk{}
^^A \end{parcolumns}
 \textbf{Note that when math is used inside escapes, any active characters beyond those that are normally active in verbatim can cause problems.}  Any package that relies on special active characters in math mode (for example, \pkg{icomma}) will produce errors along the lines of \texttt{TeX capacity exceeded} and \texttt{\string\leavevmode \string\kern \string\z@}.  This may be fixed by modifying \texttt{\string\@noligs}, as described at \url{http://tex.stackexchange.com/questions/223876}.
   \item[firstline (integer) (1)]
     The first line to be shown.
     All lines before that line are ignored and do not appear in the output.
   \item[firstnumber (auto \| last \| integer) (auto = 1)]
     Line number of the first line.
   \item[fontfamily (family name) (tt)]
     The font family to use.
     |tt|, |courier| and |helvetica| are pre-defined.
   \item[fontseries (series name) (auto \textrm{-- the same as the current font})]
     The font series to use.
   \item[fontsize (font size) (auto \textrm{-- the same as the current font})]
     The size of the font to use, as a size command, e.g. |\footnotesize|.
   \item[fontshape (font shape) (auto \textrm{-- the same as the current font})]
     The font shape to use.
   \item[formatcom (command) (\meta{none})]
     A format to execute before printing verbatim text.
   \item[frame (none \| leftline \| topline \| bottomline \| lines \| single) (none)]
     The type of frame to put around the source code listing.
   \item[framerule (dimension) (0.4pt)]
     Width of the frame.
   \item[framesep (dimension) (\cmd\fboxsep)]
     Distance between frame and content.
   \item[funcnamehighlighting (boolean) (true)] \appliesto{PHP}
     If |true|, highlights built-in function names.
   \item[gobble (integer) (0)]
     Remove the first $n$ characters from each input line.

 \item[highlightcolor (string) (LightCyan)]
 Set the color used for |highlightlines|, using a predefined color name from \pkg{color} or \pkg{xcolor}, or a color defined via |\definecolor|.


 \item[highlightlines (string) (\meta{none})]
 This highlights a single line or a range of lines based on line numbers.  For example, |highlightlines={1, 3-4}|.  The line numbers refer to the line numbers that would appear if |linenos=true|, etc.  They do not refer to original or actual line numbers before adjustment by |firstnumber|.

 The highlighting color can be customized with |highlightcolor|.


   \item[keywordcase] (string) (lower)
     Changes capitalization of keywords.  Takes |lower|, |upper|, or |capitalize|.
   \item[label (string) (\emph{empty})]
     Add a label to the top, the bottom or both of the frames around the code.
     See the \pkg{fancyvrb} documentation for more information and examples.
     \emph{Note:} This does \emph{not} add a |\label| to the current listing.
     To achieve that, use a floating environment (section \ref{sec:float}) instead.
   \item[labelposition (none \| topline \| bottomline \| all) (topline, all,  \textrm{or} none)]
     Position where to print the label (see above;  default: |topline| if one label is defined, |all| if two are defined, |none| else).
     See the \pkg{fancyvrb} documentation for more information.
   \item[lastline (integer) (\meta{last line of input})]
     The last line to be shown.
   \item[linenos (boolean) (false)]
     Enables line numbers.
     In order to customize the display style of line numbers, you need to redefine the |\theFancyVerbLine| macro:

 \begin{example}
   \renewcommand{\theFancyVerbLine}{\sffamily
     \textcolor[rgb]{0.5,0.5,1.0}{\scriptsize
     \oldstylenums{\arabic{FancyVerbLine}}}}

   \begin{minted}[linenos,
     firstnumber=11]{python}
   def all(iterable):
       for i in iterable:
           if not i: 
               return False
       return True
   \end{minted}
 \end{example}

 \item[numberfirstline (boolean) (false)]
 Always number the first line, regardless of |stepnumber|.


   \item[numbers (left \| right \| both \| none) (none)]
     Essentially the same as |linenos|, except the side on which the numbers appear may be specified.
   \item[mathescape (boolean) (false)]
     Enable \LaTeX{} math mode inside comments.
     Usage as in package \pkg{listings}.
     See the note under \texttt{escapeinside} regarding math and ligatures.
   \item[numberblanklines (boolean) (true)]
     Enables or disables numbering of blank lines.
   \item[numbersep (dimension) (12pt)]
     Gap between numbers and start of line.
   \item[obeytabs (boolean) (false)]
     Treat tabs as tabs instead of converting them to spaces---that is, expand them to tab stops determined by |tabsize|.  \textcolor{DarkRed}{\textbf{While this will correctly expand tabs within leading indentation, usually it will not correctly expand tabs that are preceded by anything other than spaces or other tabs.  It should be avoided in those case.}}

   \item[outencoding (string) (\meta{system-specific})]
     Sets the file encoding that \app{Pygments} uses for highlighted output.  Overrides any encoding previously set via |encoding|.
   \item[python3 (boolean) (false)] \appliesto{PythonConsoleLexer}
     Specifies whether Python 3 highlighting is applied.
   \item[resetmargins (boolean) (false)]
     Resets the left margin inside other environments.
   \item[rulecolor (color command) (black)]
     The color of the frame.
   \item[samepage (boolean) (false)]
     Forces the whole listing to appear on the same page, even if it doesn't fit.
   \item[showspaces (boolean) (false)]
     Enables visible spaces: \verb*/visible spaces/.
   \item[showtabs (boolean) (false)]
     Enables visible tabs---only works in combination with |obeytabs|.


 \item[space (macro) (\string\textvisiblespace, \textvisiblespace)]
 Redefine the visible space character.  Note that this is only used if |showspaces=true|.


 \item[spacecolor (string) (none)]
 Set the color of visible spaces.  By default (|none|), they take the color of their surroundings.


   \item[startinline (boolean) (false)] \appliesto{PHP}
     Specifies that the code starts in PHP mode, i.e., leading |<?php| is omitted.
   \item[style] (string) (\meta{default})
     Sets the stylesheet used by \app{Pygments}.
   \item[stepnumber (integer) (1)]
     Interval at which line numbers appear.


 \item[stepnumberfromfirst (boolean) (false)]
 By default, when line numbering is used with |stepnumber| $\ne 1$, only line numbers that are a multiple of |stepnumber| are included.  This offsets the line numbering from the first line, so that the first line, and all lines separated from it by a multiple of |stepnumber|, are numbered.


 \item[stepnumberoffsetvalues (boolean) (false)]
 By default, when line numbering is used with |stepnumber| $\ne 1$, only line numbers that are a multiple of |stepnumber| are included.  Using |firstnumber| to offset the numbering will change which lines are numbered and which line gets which number, but will not change which \emph{numbers} appear.  This option causes |firstnumber| to be ignored in determining which line numbers are a multiple of |stepnumber|.  |firstnumber| is still used in calculating the actual numbers that appear.  As a result, the line numbers that appear will be a multiple of |stepnumber|, plus |firstnumber| minus 1.


   \item[stripall (boolean) (false)]
     Strip all leading and trailing whitespace from the input.
   \item[stripnl (boolean) (false)]
     Strip leading and trailing newlines from the input.


 \item[tab (macro) ({\rmfamily\pkg{fancyvrb}'s} \string\FancyVerbTab, \FancyVerbTab)]
 Redefine the visible tab character.  Note that this is only used if |showtabs=true|.  |\rightarrowfill|, \hbox to 2em{\rightarrowfill}, may be a nice alternative.


 \item[tabcolor (string) (black)]
 Set the color of visible tabs.  If |tabcolor=none|, tabs take the color of their surroundings.  This is typically undesirable for tabs that indent multiline comments or strings.


   \item[tabsize (integer) (8)]
     The number of spaces a tab is equivalent to.  If |obeytabs| is \emph{not} active, tabs will be converted into this number of spaces.  If |obeytabs| is active, tab stops will be set this number of space characters apart.
   \item[texcl (boolean) (false)]
     Enables \LaTeX{} code inside comments.
     Usage as in package \pkg{listings}.  See the note under \texttt{escapeinside} regarding math and ligatures.
   \item[texcomments (boolean) (false)]
     Enables \LaTeX{} code inside comments.  The newer name for |texcl|.  See the note under |escapeinside| regarding math and ligatures.

     As of Pygments 2.0.2, \texttt{texcomments} fails with multiline C/C++ preprocessor directives, and may fail in some other circumstances.  This is because preprocessor directives are \href{http://pygments.org/docs/tokens/}{tokenized as \texttt{Comment.Preproc}}, so \texttt{texcomments} causes preprocessor directives to be treated as literal \LaTeX\ code.  \href{https://bitbucket.org/birkenfeld/pygments-main/issue/1086/wrong-processing-of-in-c-c-macros-if-is}{An issue has been opened} at the Pygments site; additional details are also available on the \href{https://github.com/gpoore/minted/issues/66}{\pkg{minted} GitHub site}.
   \item[xleftmargin (dimension) (0)]
     Indentation to add before the listing.
   \item[xrightmargin (dimension) (0)]
     Indentation to add after the listing.

 \end{optionlist}
^^A v1 ok
^^A % \section{Defining shortcuts\\定义快捷方式}
^^A %% Large documents with a lot of listings will nonetheless use the same source language and the
% same set of options for most listings.
% Always specifying all options is redundant, a lot to type and makes performing changes hard.

^^A % 有大量列表的大型文件仍然会使用相同的源语言和
^^A % 大多数列表使用相同的选项集。
^^A % 总是指定所有的选项是多余的,要打很多字,而且很难进行修改。

%对于有大量源码列表的大型文档,尽管使用相同的源语言和相同的选项集,但仍然需要为大多数源码列表指定所有选项,这是冗余的,需要输入很多内容,并且使更改变得困难。 

% One option is to use \cmd\setminted, but even then you must still specify the language each time.

%一种选项是使用\cmd\setminted ,但即使如此,您仍然必须每次指定语言。
%
% \pkg{minted} therefore defines a set of commands that lets you define shortcuts for the highlighting commands.
% Each shortcut is specific for one programming language.
%
%因此,\pkg{minted}定义了一组命令,让您为高亮命令定义快捷方式。每个快捷方式都特定于一种编程语言。

% \DescribeMacro{\newminted}
% |\newminted| defines a new alias for the |minted| environment:
%
|\newminted|定义了一个|minted|环境的新别名:

% \begin{example}
%   \newminted{cpp}{gobble=2,linenos}
%
%   \begin{cppcode}
%     template <typename T>
%     T id(T value) {
%         return value;
%     }
%   \end{cppcode}
% \end{example}
%
% If you want to provide extra options on the fly, or override existing default options, you can do that, too:
%
%如果您想要在操作过程中提供额外的选项,或覆盖默认的选项设置,您也可以这样做:

% \begin{example}
%   \newminted{cpp}{gobble=2,linenos}
%
%   \begin{cppcode*}{linenos=false,
%                    frame=single}
%     int const answer = 42;
%   \end{cppcode*}
% \end{example}
%
% Notice the star ``|*|'' behind the environment name---due to restrictions in \pkg{fancyvrb}'s handling
% of options, it is necessary to provide a \emph{separate} environment that accepts options, and the options
% are \emph{not} optional on the starred version of the environment.
%
% 请注意环境名称后面的星号“|*|”,由于\pkg{fancyvrb}处理选项的限制,需要提供一个\emph{单独的}接受选项的环境,而这些选项在星号版本的环境中\emph{不是}可选的。

% The default name of the environment is \meta{language}|code|.
% If this name clashes with another environment or if you want to choose an own name for another reason, you may
% do so by specifying it as the first argument:

%环境的默认名称为\meta{language}|code|。如果此名称与另一个环境冲突,或者出于其他原因想要选择自己的名称,可以将其指定为第一个参数:

\cmd\newminted\oarg{environment name}\marg{language}\marg{options}。

%
% Like normal \pkg{minted} environments, environments created with |\newminted| may be used within other environment definitions.  Since the \pkg{minted} environments use \pkg{fancyvrb} internally, any environment based on them must include the \pkg{fancyvrb} command |\VerbatimEnvironment|.  This allows \pkg{fancyvrb} to determine the name of the environment that is being defined, and correctly find its end.  It is best to include this command at the beginning of the definition.  For example,

与普通的\pkg{minted}环境一样,使用|\newminted|创建的环境可以在其他环境定义中使用。由于\pkg{minted}环境在内部使用\pkg{fancyvrb},因此基于它们的任何环境都必须包括\pkg{fancyvrb}命令|\VerbatimEnvironment|。这使得\pkg{fancyvrb}能够确定正在定义的环境的名称,并正确找到其结尾。最好在定义的开头包含此命令。例如,

%\begin{Verbatim}
%\newminted{cpp}{gobble=2,linenos}
%\newenvironment{env}{\VerbatimEnvironment\begin{cppcode}}{\end{cppcode}}
%\end{Verbatim}
%
% \DescribeMacro{\newmint}
% The above macro only defines shortcuts for the |minted| environment.
% The main reason is that the short command form |\mint| often needs different options---at the very least, it
% will generally not use the |gobble| option.
% A shortcut for |\mint| is defined using \cmd\newmint\oarg{macro name}\marg{language}\marg{options}.
% The arguments and usage are identical to |\newminted|.
% If no \meta{macro name} is specified, \meta{language} is used.
%
%上面的宏仅为|minted|环境定义了快捷方式。 主要原因是|mint|的短命令形式通常需要不同的选项——至少,它通常不使用|gobble|选项。 使用\cmd\newmint\oarg{宏名}\marg{语言}\marg{选项}定义了一个|\mint|的快捷方式。 参数和用法与|\newminted|相同。 如果没有指定\meta{macro name},则使用\meta{language}。

% \begin{example}
%   \newmint{perl}{showspaces}
%
%   \perl/my $foo = $bar;/
% \end{example}
%
%
% \DescribeMacro{\newmintinline}
% This creates custom versions of \cmd\mintinline.  The syntax is the same as that for \cmd\newmint:

%这将创建\cmd\mintinline的自定义版本。其语法与\cmd\newmint的语法相同:

\cmd\newmintinline\oarg{macro~name}\marg{language}\marg{options}.  

If a \meta{macro~name} is not specified, then the created macro is called |\|\meta{language}|inline|.
%
如果未指定\meta{macro~name},则创建的宏将被称为|\|\meta{language}|inline|。
 
% \begin{example}
%   \newmintinline{perl}{showspaces}
%
%   X\perlinline/my $foo = $bar;/X
% \end{example}
%
% \DescribeMacro{\newmintedfile}
% This creates custom versions of \cmd\inputminted.  The syntax is

%这将创建自定义版本的\cmd\inputminted。语法为

% \begin{center}
% \cmd\newmintedfile\oarg{macro~name}\marg{language}\marg{options}
% \end{center}
% If no \meta{macro name} is given, then the macro is called |\|\meta{language}|file|.
%
 如果没有给出\meta{macro name},则该宏被称为||\meta{language}|file|。^^A % v1
^^A % \section{FAQ and Troubleshooting\\常见问题与故障排除}^^A v1 ok
^^A 在某些情况下,\pkg{minted} 可能由于其他文档设置而无法达到所期望的效果,这些设置是它无法控制的。下面列出了常见问题及解决方法或解决方案。如果您正在使用 \pkg{minted} 进行非典型上下文的工作,您也可以搜索 \href{http://tex.stackexchange.com/}{tex.stackexchange.com} 或在那里提问。

% In some cases, \pkg{minted} may not give the desired result due to other document settings that it cannot control.  Common issues are described below, with workarounds or solutions.  You may also wish to search \href{http://tex.stackexchange.com/}{tex.stackexchange.com} or ask a question there, if you are working with \pkg{minted} in a non-typical context.
%
% \begin{itemize}
\item \textbf{有时会出现“无法写入文件”的错误。} 这可能是由于在与 Dropbox 或类似的文件同步程序同步的目录中使用 \pkg{minted} 引起的。这些程序可能在它仍需要修改它们的时候尝试同步 \pkg{minted} 的临时文件。解决方案是关闭文件同步或使用未同步的目录。
\\\textbf{There are intermittent ``I can't write on file'' errors.}  This can be caused by using \pkg{minted} in a directory that is synchronized with Dropbox or a similar file syncing program.  These programs can try to sync \pkg{minted}'s temporary files while it still needs to be able to modify them.  The solution is to turn off file syncing or use a directory that is not synced.

\item \textbf{我收到“Font Warning: Some font shapes were not available”消息,或者粗体或斜体似乎丢失了。} 这是由于当前用于排版代码的字体的限制。在某些情况下,\LaTeX\ 默认替换的字体形状是完全足够的,可以忽略警告。在其他情况下,字体替换可能不清楚地指示粗体或斜体文本,您需要切换到不同的字体。请参见 \LaTeX\ 字体目录中关于\href{http://www.tug.dk/FontCatalogue/typewriterfonts.html}{打字机字体}的部分,了解替代方案。如果您喜欢默认的 \LaTeX\ 字体,则 \pkg{lmodern} 包是一个很好的起点。\pkg{beramono} 和 \pkg{courier} 包也是不错的选择。
%\\ \textbf{I receive a ``Font Warning:  Some font shapes were not available'' message, or bold or italic seem to be missing.}  This is due to a limitation in the font that is currently in use for typesetting code.  In some cases, the default font shapes that \LaTeX\ substitutes are perfectly adequate, and the warning may be ignored.  In other cases, the font substitutions may not clearly indicate bold or italic text, and you will want to switch to a different font.  See The \LaTeX\ Font Catalogue's section on \href{http://www.tug.dk/FontCatalogue/typewriterfonts.html}{Typewriter Fonts} for alternatives.  If you like the default \LaTeX\ fonts, the \pkg{lmodern} package is a good place to start.  The \pkg{beramono} and \pkg{courier} packages may also be good options.

\item \textbf{在使用缓存时,在 OS X 下出现“Too many open files”错误。} 请参见第~\ref{sec:basic:preliminary} 节中关于 OS X 的说明。
% \\ \textbf{I receive a ``Too many open files'' error under OS X when using caching.}  See the note on OS X under Section~\ref{sec:basic:preliminary}.

\item \textbf{TeXShop 找不到 \texttt{pygmentize}。} 您可能需要创建一个符号链接。请参见 \url{https://tex.stackexchange.com/questions/279214}。
% \\ \textbf{TeXShop can't find \texttt{pygmentize}.}  You may need to create a symlink.  See \url{https://tex.stackexchange.com/questions/279214}.

\item \textbf{当我使用 \pkg{fancybox} 宏包时,会出现一些奇怪的事情。} \pkg{fancybox} 与 \pkg{minted} 内部使用的 \pkg{fancyvrb} 冲突。在使用 \pkg{fancybox} 时,请确保在加载 \pkg{minted}(或在未由 \pkg{minted} 加载 \pkg{fancyvrb} 的情况下,在加载 \pkg{fancyvrb} 之前)之前加载。
% \\ \textbf{Weird things happen when I use the \pkg{fancybox} package.}  \pkg{fancybox} conflicts with \pkg{fancyvrb}, which \pkg{minted} uses internally.  When using \pkg{fancybox}, make sure that it is loaded before \pkg{minted} (or before \pkg{fancyvrb}, if \pkg{fancyvrb} is not loaded by \pkg{minted}).

\item \textbf{当我在使用KOMA-Script文档类时,使用\pkg{minted}会提示关于\texttt{\string\float@addtolists}的警告。} \pkg{minted}使用\pkg{float}宏包生成浮动列表,但这与KOMA-Script生成浮动对象的方式冲突。解决方法是加载\pkg{scrhack}宏包或使用\pkg{minted}的|newfloat|选项。
% \\ \textbf{When I use \pkg{minted} with KOMA-Script document classes, I get warnings about \texttt{\string\float@addtolists}.}  \pkg{minted} uses the \pkg{float} package to produce floated listings, but this conflicts with the way KOMA-Script does floats.  Load the package \pkg{scrhack} to resolve the conflict.  Or use \pkg{minted}'s |newfloat| package option.

\item \textbf{Tilde字符(\texttt{\string~})被抬高,几乎像上标。} 这是字体的问题。你需要使用不同的字体编码,可能是使用不同的字体。尝试使用|\usepackage[T1]{fontenc}|,也许搭配|\usepackage{lmodern}|或类似的宏包。
% \\ \textbf{Tilde characters \texttt{\string~} are raised, almost like superscripts.}
% This is a font issue.  You need a different font encoding, possibly with a different font.  Try |\usepackage[T1]{fontenc}|, perhaps with |\usepackage{lmodern}|, or something similar.
%
\item \textbf{使用数学公式时出现错误,比如出现\texttt{TeX capacity exceeded}和\texttt{\string\leavevmode \string\kern \string\z@}。} 这是由于公式内容中禁用了连字号造成的。请查看|escapeinside|下的注意事项。
% \\ \textbf{I'm getting errors with math, something like \texttt{TeX capacity exceeded} and \texttt{\string\leavevmode \string\kern \string\z@}.}  This is due to ligatures being disabled within verbatim content.  See the note under |escapeinside|.
%
\item \textbf{在使用\pkg{breqn}等特殊数学宏包时,使用\texttt{mathescape}会导致文档无法完成编译,或出现其他意外结果。} 一些数学宏包(如\pkg{breqn})会给逗号等字符赋予数学模式下的特殊含义,这可能与\pkg{minted}发生冲突。在\pkg{breqn}和逗号的情况下,可以通过在|minted|环境中重新定义逗号来解决问题:
\\ \textbf{With \texttt{mathescape} and the \pkg{breqn} package (or another special math package), the document never finishes compiling or there are other unexpected results.}  Some math packages like \pkg{breqn} give certain characters like the comma special meanings in math mode.  These can conflict with \pkg{minted}.  In the \pkg{breqn} and comma case, this can be fixed by redefining the comma within |minted| environments:
%\begin{verbatim}
%\AtBeginEnvironment{minted}{\catcode`\,=12\mathcode`\,="613B}
%\end{verbatim}
其他宏包或特殊字符可能需要自己的修改。

%Other packages/special characters may need their own modifications.
%

\item \textbf{在使用Beamer时出现错误。} 由于Beamer的处理方式,你可能需要在包含\pkg{minted}命令和环境的幻灯片上使用|fragile|或|fragile=singleslide|选项。|fragile=singleslide|效果最好,但它禁用了叠加。|fragile|通过将每个幻灯片的内容保存到临时文件中再重复使用,这种方法允许使用叠加,但如果你在一行的开头有字符串|\end{frame}|(例如在|minted|环境中),它将会出错。为解决这个问题,可以将环境的内容缩进(以便|\end{frame}|前面有一个或多个空格),然后使用|gobble|或|autogobble|选项来删除缩进。
\\ \textbf{I'm getting errors with Beamer.}  Due to how Beamer treats verbatim content, you may need to use either the |fragile| or |fragile=singleslide| options for frames that contain \pkg{minted} commands and environments.  |fragile=singleslide| works best, but it disables overlays.  |fragile| works by saving the contents of each frame to a temp file and then reusing them.  This approach allows overlays, but will break if you have the string |\end{frame}| at the beginning of a line (for example, in a |minted| environment).  To work around that, you can indent the content of the environment (so that the |\end{frame}| is preceded by one or more spaces) and then use the |gobble| or |autogobble| options to remove the indentation.
%

\item \textbf{Beamer会吞掉制表符。} 这是由于\href{https://bitbucket.org/rivanvx/beamer/issue/310/tab-characters-in-listings-lost-when-using}{Beamer在处理抄录内容时存在错误}。请升级Beamer或使用链接的补丁。否则,如果您不需要覆盖图层,请尝试使用|fragile=singleslide|,或考虑使用\cmd\inputminted,或将制表符转换为空格。
\\ \textbf{Tabs are eaten by Beamer.}  This is due to \href{https://bitbucket.org/rivanvx/beamer/issue/310/tab-characters-in-listings-lost-when-using}{a bug in Beamer's treatment of verbatim content}.  Upgrade Beamer or use the linked patch.  Otherwise, try |fragile=singleslide| if you don't need overlays, or consider using \cmd\inputminted\ or converting the tabs into spaces.
%
\item \textbf{我正在尝试创建几个新的\pkg{minted}命令/环境,并希望它们都具有相同的设置。我将设置保存在一个宏中,然后在定义命令/环境时使用该宏。但它失败了。}
这是由于\pkg{keyval}的工作方式(\pkg{minted}使用它来管理选项)所致。参数不会被扩展。有关更多信息,请参见\href{http://tex.stackexchange.com/questions/13563/building-keyval-arguments-using-a-macro/13564#13564}{此处}和\href{http://tex.stackexchange.com/questions/145363/why-does-includegraphics-varone-vartwo-not-compile/145366#145366}{此处}。仍然可以实现您想要的功能;您只需在传递给创建新命令/环境的命令之前扩展选项宏即可。下面是一个示例。|\expandafter|是至关重要的部分。
\\ \textbf{I'm trying to create several new \pkg{minted} commands/environments, and want them all to have the same settings.  I'm saving the settings in a macro and then using the macro when defining the commands/environments.  But it's failing.}
% This is due to the way that \pkg{keyval} works (\pkg{minted} uses it to manage options). Arguments are not expanded. See \href{http://tex.stackexchange.com/questions/13563/building-keyval-arguments-using-a-macro/13564#13564}{this} and \href{http://tex.stackexchange.com/questions/145363/why-does-includegraphics-varone-vartwo-not-compile/145366#145366}{this} for more information.  It is still possible to do what you want; you just need to expand the options macro before passing it to the commands that create the new commands/environments.  An example is shown below.  The |\expandafter| is the vital part.
%\begin{verbatim}
%\def\args{linenos,frame=single,fontsize=\footnotesize,style=bw}
%
%\newcommand{\makenewmintedfiles}[1]{%
%  \newmintedfile[inputlatex]{latex}{#1}%
%  \newmintedfile[inputc]{c}{#1}%
%}
%
%\expandafter\makenewmintedfiles\expandafter{\args}
%\end{verbatim}
%

\item \textbf{我想在通常不允许使用抄录内容的上下文中使用\texttt{\string\mintinline}。}
|\mintinline|命令已经可以在许多不允许使用普通抄录命令(如|\verb|)的地方使用,因此请先尝试。如果不起作用,则最简单的替代方法之一是将代码保存在一个盒子中,然后稍后使用它。例如,
\\ \textbf{I want to use \texttt{\string\mintinline} in a context that normally doesn't allow verbatim content.}
% The |\mintinline| command will already work in many places that do not allow normal verbatim commands like |\verb|, so make sure to try it first.  If it doesn't work, one of the simplest alternatives is to save your code in a box, and then use it later.  For example,
%\begin{verbatim}
%\newsavebox\mybox
%\begin{lrbox}{\mybox}
%\mintinline{cpp}{std::cout}
%\end{lrbox}
%
%\commandthatdoesnotlikeverbatim{Text \usebox{\mybox}}
%\end{verbatim}
%

\item \textbf{在\pkg{minted}命令和环境内无法使用扩展字符,即使使用了\pkg{inputenc}宏包也不行。} 版本2.0增加了对pdfTeX引擎下扩展字符的支持。但是,如果需要使用\pkg{inputenc}不支持的字符,则应使用XeTeX或LuaTeX引擎。
\\ \textbf{Extended characters do not work inside \pkg{minted} commands and environments, even when the \pkg{inputenc} package is used.}
% Version 2.0 adds support for extended characters under the pdfTeX engine.  But if you need characters that are not supported by \pkg{inputenc}, you should use the XeTeX or LuaTeX engines instead.
%

\item \textbf{\pkg{polyglossia}宏包对代码进行了一些不必要的更改。(例如,在法语中在冒号周围添加额外的空格。)} 您可能需要将代码放在|\begin{english}...\end{english}|中。这可以在导言区使用\pkg{etoolbox}为所有|minted|环境完成:
\\ \textbf{The \pkg{polyglossia} package is doing undesirable things to code. (For example, adding extra space around colons in French.)}  You may need to put your code within |\begin{english}...\end{english}|.  This may done for all |minted| environments using \pkg{etoolbox} in the preamble:
% \begin{verbatim}
%\usepackage{etoolbox}
%\BeforeBeginEnvironment{minted}{\begin{english}}
%\AfterEndEnvironment{minted}{\end{english}}
% \end{verbatim}
%
\item \textbf{制表符被转换为字符序列\texttt{\string^\string^I}}。这发生在使用XeLaTeX时。您需要使用|-8bit|命令行选项,以便制表符可以正确地写入临时文件。有关XeLaTeX处理制表符的更多信息,请参见\url{http://tex.stackexchange.com/questions/58732/how-to-output-a-tabulation-into-a-file}。
\\ \textbf{Tabs are being turned into the character sequence \texttt{\string^\string^I}}.
% This happens when you use XeLaTeX.  You need to use the |-8bit| command-line option so that tabs may be written correctly to temporary files.  See \url{http://tex.stackexchange.com/questions/58732/how-to-output-a-tabulation-into-a-file} for more on XeLaTeX's handling of tab characters. 
%
\item \textbf{\pkg{caption}宏包在与\pkg{minted}一起使用\texttt{\string\captionof}和其他命令时会产生错误。} 使用选项|compatibility=false|加载\pkg{caption}宏包。或者更好的方法是使用\pkg{minted}的|newfloat|包选项,这提供了更好的\pkg{caption}兼容性。
\\ \textbf{The \pkg{caption} package produces an error when \texttt{\string\captionof} and other commands are used in combination with \pkg{minted}.}
% Load the \pkg{caption} package with the option |compatibility=false|.  Or better yet, use \pkg{minted}'s |newfloat| package option, which provides better \pkg{caption} compatibility.
%
\item \textbf{我需要一个支持分页的列表环境。} 内置的列表环境是标准浮动体,不支持分页。您可能需要为长浮动体定义一个新环境。例如,
\\ \textbf{I need a listing environment that supports page breaks.}  The built-in listing environment is a standard float; it doesn't support page breaks.  You will probably want to define a new environment for long floats.  For example, 
%\begin{verbatim}
%\usepackage{caption}
%\newenvironment{longlisting}{\captionsetup{type=listing}}{}
%\end{verbatim}
使用\pkg{caption}宏包时,最好使用\pkg{minted}的|newfloat|包选项。有关带有分页的|listing|环境的更多信息,请参见\url{http://tex.stackexchange.com/a/53540/10742}。



% With the \pkg{caption} package, it is best to use \pkg{minted}'s |newfloat| package option.  See \url{http://tex.stackexchange.com/a/53540/10742} for more on |listing| environments with page breaks.
%

\item \textbf{我想使用自定义脚本/可执行文件来访问Pygments,而不是|pygmentize|。} 重新定义|\MintedPygmentize|:
\\ \textbf{I want to use a custom script/executable to access Pygments, rather than |pygmentize|.}  Redefine |\MintedPygmentize|:
%\begin{verbatim}
%\renewcommand{\MintedPygmentize}{...}
%\end{verbatim}
%

\item \textbf{我想使用命令行选项\texttt{-output-directory}或MiKTeX的\texttt{-aux-directory},但是出现了错误。}使用包选项|outputdir|来指定输出目录的位置。不幸的是,\pkg{minted}无法自动检测输出目录的位置。
\\ \textbf{I want to use the command-line option \texttt{-output-directory}, or MiKTeX's \texttt{-aux-directory}, but am getting errors.}  Use the package option |outputdir| to specify the location of the output directory.  Unfortunately, there is no way for \pkg{minted} to detect the output directory automatically.
%
\item \textbf{我想在帧标签中使用扩展字符,但是出现了错误。} 这可能是由于\pkg{minted}<2.0和Python 2.7中的\href{https://bitbucket.org/birkenfeld/pygments-main/issue/801/python-2-fails-to-detect-terminal-encoding}{Pygments终端编码问题}导致的。它应该可以与任何版本的Python与\pkg{minted} 2.0+一起使用,因为它会在内部处理标签并将其发送到Python。
\\ \textbf{I want extended characters in frame labels, but am getting errors.}  This can happen with \pkg{minted} <2.0 and Python 2.7, due to a \href{https://bitbucket.org/birkenfeld/pygments-main/issue/801/python-2-fails-to-detect-terminal-encoding}{terminal encoding issue with Pygments}.  It should work with any version of Python with \pkg{minted} 2.0+, which processes labels internally and does not send them to Python.

\item \textbf{\texttt{minted}环境在\texttt{tabular}内部具有额外的垂直空间。}可以\href{https://github.com/gpoore/minted/issues/82}{创建自定义环境}来消除额外的空间。但是,在存在相邻文本的情况下表现符合预期的一般解决方案仍有待发现。
\\ \textbf{\texttt{minted} environments have extra vertical space inside \texttt{tabular}.}  It is possible to \href{https://github.com/gpoore/minted/issues/82}{create a custom environment} that eliminates the extra space.  However, a general solution that behaves as expected in the presence of adjacent text remains to be found.

\item \textbf{我收到来自\texttt{lineno.sty}的警告,提示“Command \texttt{\string@parboxrestore} has changed.”} 当在\pkg{csquotes}之后加载\pkg{minted}时,可能会发生这种情况。尝试先加载\pkg{minted}。如果您在不使用\pkg{csquotes}时收到此消息,则可能需要尝试加载包的顺序,并可能还要打开问题。
\\ \textbf{I'm receiving a warning from \texttt{lineno.sty} that ``Command \texttt{\string\@parboxrestore} has changed.''}  This can happen when \pkg{minted} is loaded after \pkg{csquotes}.  Try loading \pkg{minted} first.  If you receive this message when you are not using \pkg{csquotes}, you may want to experiment with the order of loading packages and might also open an issue.

\item \textbf{我正在使用\app{texi2pdf},并且从\app{tar}收到“Cannot stat”错误:}这是由于\app{texi2pdf}处理临时文件的方式。 \pkg{minted}会自动清理其临时文件,但是\app{texi2pdf}会认为在运行结束时任何曾经创建的临时文件仍然存在,因此尝试访问\pkg{minted}已删除的文件。可以通过在|\usepackage{minted}|之后添加|\renewcommand{\DeleteFile}[2][]{}|来禁用\pkg{minted}的临时文件清理。
\\ \textbf{I'm using \app{texi2pdf}, and getting ``Cannot stat'' errors from \app{tar}}:  This is due to the way that \app{texi2pdf} handles temporary files.  \pkg{minted} automatically cleans up its temporary files, but \app{texi2pdf} assumes that any temporary file that is ever created will still exist at the end of the run, so it tries to access the files that \pkg{minted} has deleted. It's possible to disable minted's temp file cleanup by adding |\renewcommand{\DeleteFile}[2][]{}| after the |\usepackage{minted}|.
% \end{itemize}
%
%
%
%
% \section*{Acknowledgements\\致谢}
% \addcontentsline{toc}{section}{Acknowledgements}
%
\textbf{Konrad Rudolph:}特别感谢Philipp Stephani和其他来自\texttt{comp.text.tex}和\texttt{tex.stackexchange.com}的人员。

% \textbf{Konrad Rudolph:}  Special thanks to Philipp Stephani and the rest of the guys from \texttt{comp.text.tex} and \texttt{tex.stackexchange.com}.
%
% \textbf{Geoffrey Poore:}
% \begin{itemize}
% \item Thanks to Marco Daniel for the code on \url{tex.stackexchange.com} that inspired automatic line breaking.
% \item Thanks to Patrick Vogt for improving TikZ externalization compatibility.
% \item Thanks to \textsf{@muzimuzhi} for assistance with GitHub issues.
% \item Thanks to \textsf{@jfbu} for suggestions and discussion regarding support for arbitrary Pygments style names (\#210, \#294, \#299, \#317), and for debugging assistance.
% \end{itemize}
%
% \PrintChangelog
%
% \StopEventually{} %\StopEventually{\PrintIndex}
%
%
%
^^A % \section{Implementation}
^^A %\input{8_Implementation}



^^A \vspace*{-2\parskip}
^^A \begin{parcolumns}[rulebetween=false,colwidths={2=.55\linewidth}]{2}
^^A \colchunk{}
^^A \colchunk{}
^^A \end{parcolumns}


^^A \begin{parcolumns}[rulebetween=false,colwidths={2=.55\linewidth}]{2}
^^A \colchunk{}
^^A \colchunk{}
^^A \end{parcolumns}

^^A \colplacechunks
^^A \colchunk{}
^^A \colchunk{}

^^A xelatex  -synctex=1  -shell-escape  minted-cn.dtx