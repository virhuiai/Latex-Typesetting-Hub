\columnratio{0.55}
\begin{paracol}{2}
\subsection{Loading Style Files}
\label{sec:usage-load}
\switchcolumn
\subsection{加载样式文件}

\switchcolumn[0]*
Both style files are usable to both \LaTeXe{} and \LaTeX-2.09
users with their standard package loading declaration.  If you use
\LaTeXe{} and wish to load, for example, \textsf{newvbtm}, simply do the
following.

\SpecialUsageIndex{\usepackage}
\begin{iverbatim}
\usepackage{newvbtm}
\end{iverbatim}
\switchcolumn
这两个样式文件都适用于\LaTeXe{}和\LaTeX-2.09用户,并且可以使用它们的标准包加载声明。如果你使用\LaTeXe{}并希望加载,例如\textsf{newvbtm},只需执行以下操作。
\begin{minted}{latex}
\usepackage{newvbtm}
\end{minted}

\switchcolumn[0]*
If you still love \LaTeX-2.09, the following is what you have to do.
\begin{quote}
\SpecialUsageIndex{\documentstyle}
|\documentstyle[..,newvbtm,...]|\Meta{main-style}
\end{quote}
\switchcolumn
如果你仍然在使用\LaTeX-2.09,你需要执行以下操作。
\begin{quote}
|\documentstyle[..,newvbtm,...]|\Meta{main-style}
\end{quote}

\switchcolumn[0]*
Note that loading \textsf{varvbtm} automatically loads \textsf{newvbtm}
too.  Thus you may not load both though doing so is safe.
\switchcolumn
请注意,加载\textsf{varvbtm}会自动加载\textsf{newvbtm}。因此,尽管可以安全地这样做,但不要同时加载两者。

\switchcolumn[0]*
\subsection{\textsf{newvbtm}: Define \texttt{verbatim}-like Environments}
\label{sec:usage-new}
\switchcolumn
\subsection{\textsf{newvbtm}:定义类似\texttt{verbatim}的环境}

\switchcolumn[0]*

\SpecialOptIndex{newvbtm}{style file}
\DescribeMacro{\newverbatim}
The command;
\begin{quote}
|\newverbatim|\Meta{env}\opt{n-args}
	\Meta{beg-def-outer}\Meta{beg-def-inner}|%|\\
\phantom{\texttt{+newverbatim}\Meta{env}\opt{n-args}}
	\Meta{end-def-inner}\Meta{end-def-outer}
\end{quote}
\switchcolumn
命令
\begin{quote}
|\newverbatim|\Meta{env}\opt{n-args}
    \Meta{beg-def-outer}\Meta{beg-def-inner}|%|\\
\phantom{\texttt{+newverbatim}\Meta{env}\opt{n-args}}
    \Meta{end-def-inner}\Meta{end-def-outer}
\end{quote}

\switchcolumn[0]*
defines an environment named \meta{env} with \meta{n-args} arguments
(optionally), and acting conceptually as follows:
\begin{quote}
\meta{beg-def-outer}|\begin{verbatim}|\meta{beg-def-inner}\\
\meta{body-of-environment}\\
\meta{end-def-inner}|\end{verbatim}|\meta{end-def-outer}
\end{quote}
\switchcolumn
定义了一个名为\meta{env}的环境,有\meta{n-args}个参数(可选),并且在概念上的作用如下:
\begin{quote}
\meta{beg-def-outer}|\begin{verbatim}|\meta{beg-def-inner}\\
\meta{body-of-environment}\\
\meta{end-def-inner}|\end{verbatim}|\meta{end-def-outer}
\end{quote}

\switchcolumn[0]*
Thus to have indented-footnotesize-|verbatim| named, say |indfnsverbatim|, 
you may simply do the following.
\begin{iverbatim}
\newverbatim{indfnsverbatim}{\begin{itemize}\item[]\footnotesize}{}{}%
                           {\end{itemize}}
\end{iverbatim}
\switchcolumn
因此,要定义一个名为\texttt{indfnsverbatim}的缩进脚注大小的\texttt{verbatim}环境,只需执行以下操作。
\begin{iverbatim}
\newverbatim{indfnsverbatim}{\begin{itemize}\item[]\footnotesize}{}{}%
                        {\end{itemize}}
\end{iverbatim}

\switchcolumn[0]*
Since |\newverbatim| defines not only \meta{env} but also its starred 
counterpart \meta{env}\texttt{*} that acts like |verbatim*|, the
definition above also defines |indfnsverbatim*| environment.
\switchcolumn
由于|\newverbatim|不仅定义了\meta{env},还定义了和|verbatim*|类似的星号版本\meta{env}\texttt{*},上述定义也定义了|indfnsverbatim*|环境。

\switchcolumn[0]*
If you use \LaTeXe{}, you may make \meta{env} have an optional argument
whose default value is \meta{default} by;

\begin{itemize}\item[]
|\newverbatim|\Meta{env}\opt{n-args}\opt{default}
	\Meta{beg-def-outer}\Meta{beg-def-inner}|%|\\
\phantom{\texttt{+newverbatim}\Meta{env}\opt{n-args}\opt{default}}
	\Meta{end-def-inner}\Meta{end-def-outer}
\end{itemize}

\switchcolumn
如果你使用\LaTeXe{},你可以通过以下方式使\meta{env}具有可选参数,其默认值为\meta{default}。
\begin{itemize}\item[]
    |\newverbatim|\Meta{env}\opt{n-args}\opt{default}
        \Meta{beg-def-outer}\Meta{beg-def-inner}|%|\\
    \phantom{\texttt{+newverbatim}\Meta{env}\opt{n-args}\opt{default}}
        \Meta{end-def-inner}\Meta{end-def-outer}
    \end{itemize}

\switchcolumn[0]*
For example, our |indfnsverbatim| environment can have an optional
argument to specify a font size other than |\footnotesize| by the
following definition.

\begin{iverbatim}
\newverbatim{indfnsverbatim}[1][\footnotesize]%
       {\begin{itemize}\item[]#1}{}{}{\end{itemize}}
\end{iverbatim}
\switchcolumn
例如,我们的|indfnsverbatim|环境可以通过以下定义具有可选参数,以指定除|\footnotesize|之外的字体大小。
\begin{iverbatim}
    \newverbatim{indfnsverbatim}[1][\footnotesize]%
           {\begin{itemize}\item[]#1}{}{}{\end{itemize}}
    \end{iverbatim}

\switchcolumn[0]*
The argument \meta{beg-def-inner} is for \TeX{}perts who wish to do
something overriding what \LaTeX's |\verbatim| does.  Even if you don't
have much confidence in your \TeX{}pertise, however, you can do some
useful thing with this argument.  For example, the following is obtained
by itself.

\begin{itemize}\item[]
\newverbatim{slverbatim}{\ttfamily}{\slshape}{}{}
\begin{slverbatim}
\newverbatim{slverbatim}{}{\slshape}{}{}
\end{slverbatim}
\end{itemize}

\switchcolumn
\meta{beg-def-inner}参数是给希望覆盖\LaTeX 的|\verbatim|命令的\TeX 专家使用的。然而,即使你对自己的\TeX 专业知识没有太多信心,你也可以使用这个参数做一些有用的事情。例如,以下内容是通过以下命令得到的。
\begin{itemize}\item[]
\newverbatim{slverbatim}{\ttfamily}{\slshape}{}{}
\begin{slverbatim}
\newverbatim{slverbatim}{}{\slshape}{}{}
\end{slverbatim}
\end{itemize}

\switchcolumn[0]*
Also you will find a few commands for this argument in
\S\ref{sec:usage-var}.
\switchcolumn 此外,你还可以在第\ref{sec:usage-var}节中找到一些用于此参数的命令。

\switchcolumn[0]*
The needs of \meta{end-def-inner} is much more limited.  One example is to
check if |\end{verbatim}| is at the beginning of a line.  This examination
is done by;

\begin{quote}
|\newverbatim{myverbatim}{...}{...}%|\\
|        {\ifvmode |\meta{at-bol}| \else |\meta{not-at-bol}| \fi}{...}|
\end{quote}
\switchcolumn
\meta{end-def-inner}参数的需求要少得多。一个例子是检查|\end{verbatim}|是否位于一行的开头。可以通过以下方式进行检查:
\begin{quote}
|\newverbatim{myverbatim}{...}{...}%|\\
|        {\ifvmode |\meta{at-bol}| \else |\meta{not-at-bol}| \fi}{...}|
\end{quote}

\switchcolumn[0]*
\DescribeMacro{\renewverbatim}
You may redefine your own |verbatim|-like environment, or even |verbatim|
itself, by |\renewverbatim| whose arguments are same as those of
|\newenvironment|.

\end{paracol}