

\columnratio{0.55}
\begin{paracol}{2}

\subsubsection{Verbatim Input}
\label{sec:usage-var-input}
\switchcolumn
\subsubsection{抄录输入}

\switchcolumn[0]*
The last thing \textsf{varvbtm} gives you is;

\begin{itemize}\item[]
|\(re)newverbatiminput|\Meta{command}\opt{n-args}\opt{default}|%|\\
\phantom{\texttt{+(re)newverbatiminput}}
	\Meta{beg-def-outer}\Meta{beg-def-inner}|%|\\
\phantom{\texttt{+(re)newverbatiminput}}
	\Meta{end-def-inner}\Meta{end-def-outer}
\end{itemize}
\switchcolumn
最后一件事情是\textsf{varvbtm}给你提供的:
\begin{itemize}\item[]
|\(re)newverbatiminput|\Meta{command}\opt{n-args}\opt{default}|%|\\
\phantom{\texttt{+(re)newverbatiminput}}
    \Meta{beg-def-outer}\Meta{beg-def-inner}|%|\\
\phantom{\texttt{+(re)newverbatiminput}}
    \Meta{end-def-inner}\Meta{end-def-outer}
\end{itemize}
\switchcolumn[0]*
to define a \meta{command} to |\input| a file.  Since this define a
\meta{command} instead of an environment, \meta{command} should have `|\|'
as its prefix.  The \meta{command} has at least one mandatory argument,
\meta{file} to be input, which can be referred as first argument if
\opt{default} is not supplied, or as second otherwise.  Note that,
however, if the \meta{command} does not have any other arguments, you can
omit \opt{n-arg}.

For example;

\begin{iverbatim}
\newverbatiminput{\vinput}{}{}{}{}
\end{iverbatim}
\switchcolumn
这样可以定义一个\meta{command}来用于|\input|文件。由于定义的是\meta{command}而不是环境,所以\meta{command}应该以'\'作为前缀。至少有一个必需的参数\meta{file}要被输入,如果没有提供\opt{default},则可以将其作为第一个参数引用,否则作为第二个参数引用。然而,请注意,如果\meta{command}没有其他参数,可以省略\opt{n-arg}。

例如:
\begin{iverbatim}
\newverbatiminput{\vinput}{}{}{}{}
\end{iverbatim}
    
\switchcolumn[0]*
defines |\vinput|\Meta{file} (and |\vinput*|) that |\input| a \meta{file}
as if the \meta{file} has |\begin|\slash|\end{verbatim}| at its first and
last lines.  A little bit more complicated example;

\begin{iverbatim}
\newverbatiminput{\indfnsvinput}[2][\footnotesize]%
       {\begin{itemize}\item[]#1}{}{}{\end{itemize}}
\end{iverbatim}

\switchcolumn
定义了|\vinput|\Meta{file}(以及|\vinput*|),它们以|\begin|\slash|\end{verbatim}|作为\meta{file}的第一行和最后一行进行|\input|。再举一个略微复杂的例子:
\begin{iverbatim}
\newverbatiminput{\indfnsvinput}[2][\footnotesize]%
        {\begin{itemize}\item[]#1}{}{}{\end{itemize}}
\end{iverbatim}

\switchcolumn[0]*
defines a indented-footnotesize-by-default version of |\vinput|.
\switchcolumn
定义了一个默认为缩进的 |\vinput| 的 \texttt{footnotesize} 版本。

\switchcolumn[0]*
\IndexPrologue{\newpage\section*{Index}
Underlined number refers to the page where the specification of
corresponding entry is described.}
\StopEventually{
\section*{Acknowledgments\hfill 致谢}

\switchcolumn[0]*
The author thanks to Noboru Matsuda and Carlos Puchol whose posts to
news groups triggered writing very first version of macros in
\textsf{newvbtm} and \textsf{varvbtm}.

\switchcolumn
作者感谢 Noboru Matsuda 和 Carlos Puchol,他们在新闻组中的帖子触发了在 \textsf{newvbtm} 和 \textsf{varvbtm} 中编写宏的第一个版本。

\switchcolumn[0]*
For the implementation of these style files, the author refers the base
implementations of the macros for \texttt{verbatim} environment.
These macros are written by Leslie Lamport as a part of
\LaTeX-2.09 and \LaTeXe{} (1997/12/01) to which Johannes Braams and other
authors also contributed.
\switchcolumn
对于这些样式文件的实现,作者参考了 \texttt{verbatim} 环境的宏的基本实现。这些宏是由 Leslie Lamport 编写的,作为 \LaTeX-2.09 和 \LaTeXe{}(1997/12/01)的一部分,Johannes Braams 和其他作者也做出了贡献。

\end{paracol}

\PrintIndex}



\newpage