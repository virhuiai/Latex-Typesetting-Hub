%%
%% This is file `newvbtm-man.tex',
%% generated with the docstrip utility.
%%
%% The original source files were:
%%
%% newvbtm.dtx  (with options: `driver')
%% 
%% IMPORTANT NOTICE:
%% 
%% For the copyright see the source file.
%% 
%% Any modified versions of this file must be renamed
%% with new filenames distinct from newvbtm-man.tex.
%% 
%% For distribution of the original source see the terms
%% for copying and modification in the file newvbtm.dtx.
%% 
%% This generated file may be distributed as long as the
%% original source files, as listed above, are part of the
%% same distribution. (The sources need not necessarily be
%% in the same archive or directory.)
%% Copyright (C) 1999-2002  Hiroshi Nakashima <nakasima@tutics.tut.ac.jp>
%%     (Toyohashi Univ. of Tech.)
%%
%% This program can be redistributed and/or modified under the terms
%% of the LaTeX Project Public License distributed from CTAN
%% archives in directory macros/latex/base/lppl.txt; either
%% version 1 of the License, or any later version.
%%
%% \CharacterTable
%%  {Upper-case    \A\B\C\D\E\F\G\H\I\J\K\L\M\N\O\P\Q\R\S\T\U\V\W\X\Y\Z
%%   Lower-case    \a\b\c\d\e\f\g\h\i\j\k\l\m\n\o\p\q\r\s\t\u\v\w\x\y\z
%%   Digits        \0\1\2\3\4\5\6\7\8\9
%%   Exclamation   \!     Double quote  \"     Hash (number) \#
%%   Dollar        \$     Percent       \%     Ampersand     \&
%%   Acute accent  \'     Left paren    \(     Right paren   \)
%%   Asterisk      \*     Plus          \+     Comma         \,
%%   Minus         \-     Point         \.     Solidus       \/
%%   Colon         \:     Semicolon     \;     Less than     \<
%%   Equals        \=     Greater than  \>     Question mark \?
%%   Commercial at \@     Left bracket  \[     Backslash     \\
%%   Right bracket \]     Circumflex    \^     Underscore    \_
%%   Grave accent  \`     Left brace    \{     Vertical bar  \|
%%   Right brace   \}     Tilde         \~}
%%
%%
\ProvidesFile{newvbtm-man.tex}
[2002/04/08 v1.1 ]
\documentclass{ltxdoc}
\usepackage{varvbtm}
\DisableCrossrefs
\PageIndex
\CodelineNumbered
\RecordChanges
\OnlyDescription
\def\ONLYDESCRIPTION{}
\advance\textwidth4em

\usepackage[heading=true
,scheme=chinese%中文方案
,fontset=none%不使用默认的字体设置
,space=auto%自动调整中英文间距
]{ctex}
\setCJKmainfont{FangZhengShuSong-GBK-1.ttf}[Path=/Users/virhuiai/hlProjects/Latex-Typesetting-Hub/font/方正/]%设置文本的中文有衬线字体
\setCJKsansfont{FangZhengHeiTi-GBK-1.ttf}[Path=/Users/virhuiai/hlProjects/Latex-Typesetting-Hub/font/方正/]%设置文本的中文无衬线字体为
\setCJKmonofont{FangZhengFangSong-GBK-1.ttf}[Path=/Users/virhuiai/hlProjects/Latex-Typesetting-Hub/font/方正/] %设置文本的中文等宽字体 
% \setCJKfamilyfont{fontKai}{LXGWWenKai-Regular.ttf}[Path=/Users/virhuiai/hlProjects/Latex-Typesetting-Hub/font/霞鹜文楷/]
\setCJKfamilyfont{fontKai}{FangZhengKaiTi-GBK-1.ttf}[Path=/Users/virhuiai/hlProjects/Latex-Typesetting-Hub/font/方正/]
\newcommand\fontKai{\CJKfamily{fontKai}}
\usepackage[a3paper%, hmargin=2.5cm, vmargin=1cm, includeheadfoot
,landscape]{geometry}
\usepackage{parskip}
\usepackage{paracol}
\usepackage{minted}
\begin{document}
\parindent=0pt
 
\DocInput{newvbtm_1.dtx}

\globalcounter{section}
% \title{\textsf{newvbtm} and \textsf{varvbtm}\\
	Packages for Variants of \texttt{verbatim} Environment\thanks{
	This file has version number \fileversion, last revised \filedate.}}
\author{Hiroshi Nakashima\\(Toyohashi Univ. of Tech.)\and 翻译:virhuiai\\福建师范大学}
\date{\filedate}
\maketitle

\begin{abstract}
\columnratio{0.55}
\begin{paracol}{2}
\noindent
This file provides two style files; \textsf{newvbtm} to define
|verbatim|-like environments; \textsf{varvbtm} to provide set of macros
for variants of |verbatim|, e.g.\ in which \texttt{\char`\^I} acts as a tab.    
\switchcolumn
\noindent
本文件提供了两个样式文件:\textsf{newvbtm} 用于定义类似于 |verbatim| 的环境;\textsf{varvbtm} 提供一组宏,用于处理 |verbatim| 的变体,例如其中 \texttt{\char`^I} 作为制表符。
\end{paracol}    
\end{abstract}

\tableofcontents
\newpage

% \columnratio{0.55}
\begin{paracol}{2}
\section{Introduction}
\label{sec:intro}
\switchcolumn
\section*{介绍}

\switchcolumn[0]*
\SpecialOptIndex{verbatim}{environment}
\LaTeX{} users often have trouble when they wish to have their own
customized |verbatim|-like environment.  Probably you once wished to have
an indented-footnotesize-|verbatim| instead of always typing;
\switchcolumn
当 \LaTeX{} 用户希望拥有自定义的类似于 |verbatim| 的环境时,通常会遇到困难。也许您曾经希望有一个缩进的脚注大小的 |verbatim|,而不是每次都要输入以下内容:
\switchcolumn[0]*
\begin{iverbatim}
\begin{itemize}\item[]\footnotesize
\begin{verbatim}
...
\end{verbatim}
\end{itemize}
\end{iverbatim}
\switchcolumn

\begin{minted}{latex}
\begin{itemize}\item[]\footnotesize
\begin{verbatim}
...
\end{verbatim}
\end{itemize}
\end{minted}

\switchcolumn[0]*
and tried the following just to know it does not work.
\begin{iverbatim}
\newenvironment{myverbatim}{\begin{itemize}\item[]\footnotesize
                           \begin{verbatim}}%
                          {\end{verbatim}\end{itemize}}
\end{iverbatim}
\switchcolumn 并尝试了以下代码,只是发现它无法正常工作:
\begin{minted}{latex}
\newenvironment{myverbatim}{\begin{itemize}\item[]\footnotesize
    \begin{verbatim}}%
    {\end{verbatim}\end{itemize}}
\end{minted}

\switchcolumn[0]*
Another trouble you probablly have had is that what you see in |verbatim|
text with |<TAB>| is not what you get because |<TAB>| does not acts as an
tab but a space.
\switchcolumn
您可能遇到的另一个问题是,使用 |<TAB>| 在 |verbatim| 文本中所看到的不是所获得的,因为 |<TAB>| 并不像制表符一样起作用,而是一个空格。
\switchcolumn[0]*
Of course it is possible to define your own |verbatim|-like environments
if you have enough knowledge of the implementation of |verbatim| including 
dirty tricks with |\catcode|.  However, even a \TeX{}pert should be bored
with typing a dirty code like;
\switchcolumn 
当然,如果您对包括使用 |\catcode| 进行一些“脏技巧”在内的 |verbatim| 的实现有足够的了解,那么您可以自定义自己的 |verbatim| 环境。然而,即使是 \TeX{} 专家,也会对输入以下这种“脏代码”感到厌烦:

\switchcolumn[0]*
\begin{iverbatim}
\begingroup \catcode`\|=0 \catcode`\[=1 \catcode`\]=2
\catcode`\{=12 \catcode`\}=12 \catcode`\\=12
|long|def|@myxverbatim##1\end{myverbatim}[##1|end[myverbatim]]
|endgroup
\end{iverbatim}
\switchcolumn
\begin{minted}{latex}
\begingroup \catcode`\|=0 \catcode`\[=1 \catcode`\]=2
\catcode`\{=12 \catcode`\}=12 \catcode`\\=12
|long|def|@myxverbatim##1\end{myverbatim}[##1|end[myverbatim]]
|endgroup
\end{minted}

\switchcolumn[0]*
\DescribeOpt{newvbtm}{style file}
\DescribeOpt{varvbtm}{style file}
The style files distributed with this document will solve these problems.
You will have two style files, \textsf{newvbtm.sty} and
\textsf{varvbtm.sty}, by processing \textsf{newvbtm.dtx} with
\textsf{docstrip}, or simply doing the following.
\begin{iverbatim}
% tex newvbtm.ins
\end{iverbatim}
\switchcolumn
本文档附带的样式文件将解决这些问题。通过处理 \textsf{newvbtm.dtx} 并使用 \textsf{docstrip},您可以得到两个样式文件:\textsf{newvbtm.sty} 和 \textsf{varvbtm.sty},或者只需要执行以下命令:
\begin{minted}{shell}
% tex newvbtm.ins
\end{minted}

\switchcolumn[0]*
The former style provides you |\(re)newverbatim| command to (re)define
your own |verbatim|-like environment easily.  The latter gives you a set
of various macros for tab-emulation, page break control, etc.
\switchcolumn
前者提供了 |\(re)newverbatim| 命令,可以轻松(重新)定义自己的 |verbatim| 类似环境。后者为您提供了一组用于模拟制表符、控制分页等的各种宏。
\end{paracol}
% 

\section{Usage\hfill 使用方法}
\label{sec:usage}



%%%%%


%%%%%%%%%%%%%%%%%%%%%%%%


% \columnratio{0.55}
\begin{paracol}{2}
\subsection{Loading Style Files}
\label{sec:usage-load}
\switchcolumn
\subsection{加载样式文件}

\switchcolumn[0]*
Both style files are usable to both \LaTeXe{} and \LaTeX-2.09
users with their standard package loading declaration.  If you use
\LaTeXe{} and wish to load, for example, \textsf{newvbtm}, simply do the
following.

\SpecialUsageIndex{\usepackage}
\begin{iverbatim}
\usepackage{newvbtm}
\end{iverbatim}
\switchcolumn
这两个样式文件都适用于\LaTeXe{}和\LaTeX-2.09用户,并且可以使用它们的标准包加载声明。如果你使用\LaTeXe{}并希望加载,例如\textsf{newvbtm},只需执行以下操作。
\begin{minted}{latex}
\usepackage{newvbtm}
\end{minted}

\switchcolumn[0]*
If you still love \LaTeX-2.09, the following is what you have to do.
\begin{quote}
\SpecialUsageIndex{\documentstyle}
|\documentstyle[..,newvbtm,...]|\Meta{main-style}
\end{quote}
\switchcolumn
如果你仍然在使用\LaTeX-2.09,你需要执行以下操作。
\begin{quote}
|\documentstyle[..,newvbtm,...]|\Meta{main-style}
\end{quote}

\switchcolumn[0]*
Note that loading \textsf{varvbtm} automatically loads \textsf{newvbtm}
too.  Thus you may not load both though doing so is safe.
\switchcolumn
请注意,加载\textsf{varvbtm}会自动加载\textsf{newvbtm}。因此,尽管可以安全地这样做,但不要同时加载两者。

\switchcolumn[0]*
\subsection{\textsf{newvbtm}: Define \texttt{verbatim}-like Environments}
\label{sec:usage-new}
\switchcolumn
\subsection{\textsf{newvbtm}:定义类似\texttt{verbatim}的环境}

\switchcolumn[0]*

\SpecialOptIndex{newvbtm}{style file}
\DescribeMacro{\newverbatim}
The command;
\begin{quote}
|\newverbatim|\Meta{env}\opt{n-args}
	\Meta{beg-def-outer}\Meta{beg-def-inner}|%|\\
\phantom{\texttt{+newverbatim}\Meta{env}\opt{n-args}}
	\Meta{end-def-inner}\Meta{end-def-outer}
\end{quote}
\switchcolumn
命令
\begin{quote}
|\newverbatim|\Meta{env}\opt{n-args}
    \Meta{beg-def-outer}\Meta{beg-def-inner}|%|\\
\phantom{\texttt{+newverbatim}\Meta{env}\opt{n-args}}
    \Meta{end-def-inner}\Meta{end-def-outer}
\end{quote}

\switchcolumn[0]*
defines an environment named \meta{env} with \meta{n-args} arguments
(optionally), and acting conceptually as follows:
\begin{quote}
\meta{beg-def-outer}|\begin{verbatim}|\meta{beg-def-inner}\\
\meta{body-of-environment}\\
\meta{end-def-inner}|\end{verbatim}|\meta{end-def-outer}
\end{quote}
\switchcolumn
定义了一个名为\meta{env}的环境,有\meta{n-args}个参数(可选),并且在概念上的作用如下:
\begin{quote}
\meta{beg-def-outer}|\begin{verbatim}|\meta{beg-def-inner}\\
\meta{body-of-environment}\\
\meta{end-def-inner}|\end{verbatim}|\meta{end-def-outer}
\end{quote}

\switchcolumn[0]*
Thus to have indented-footnotesize-|verbatim| named, say |indfnsverbatim|, 
you may simply do the following.
\begin{iverbatim}
\newverbatim{indfnsverbatim}{\begin{itemize}\item[]\footnotesize}{}{}%
                           {\end{itemize}}
\end{iverbatim}
\switchcolumn
因此,要定义一个名为\texttt{indfnsverbatim}的缩进脚注大小的\texttt{verbatim}环境,只需执行以下操作。
\begin{iverbatim}
\newverbatim{indfnsverbatim}{\begin{itemize}\item[]\footnotesize}{}{}%
                        {\end{itemize}}
\end{iverbatim}

\switchcolumn[0]*
Since |\newverbatim| defines not only \meta{env} but also its starred 
counterpart \meta{env}\texttt{*} that acts like |verbatim*|, the
definition above also defines |indfnsverbatim*| environment.
\switchcolumn
由于|\newverbatim|不仅定义了\meta{env},还定义了和|verbatim*|类似的星号版本\meta{env}\texttt{*},上述定义也定义了|indfnsverbatim*|环境。

\switchcolumn[0]*
If you use \LaTeXe{}, you may make \meta{env} have an optional argument
whose default value is \meta{default} by;

\begin{itemize}\item[]
|\newverbatim|\Meta{env}\opt{n-args}\opt{default}
	\Meta{beg-def-outer}\Meta{beg-def-inner}|%|\\
\phantom{\texttt{+newverbatim}\Meta{env}\opt{n-args}\opt{default}}
	\Meta{end-def-inner}\Meta{end-def-outer}
\end{itemize}

\switchcolumn
如果你使用\LaTeXe{},你可以通过以下方式使\meta{env}具有可选参数,其默认值为\meta{default}。
\begin{itemize}\item[]
    |\newverbatim|\Meta{env}\opt{n-args}\opt{default}
        \Meta{beg-def-outer}\Meta{beg-def-inner}|%|\\
    \phantom{\texttt{+newverbatim}\Meta{env}\opt{n-args}\opt{default}}
        \Meta{end-def-inner}\Meta{end-def-outer}
    \end{itemize}

\switchcolumn[0]*
For example, our |indfnsverbatim| environment can have an optional
argument to specify a font size other than |\footnotesize| by the
following definition.

\begin{iverbatim}
\newverbatim{indfnsverbatim}[1][\footnotesize]%
       {\begin{itemize}\item[]#1}{}{}{\end{itemize}}
\end{iverbatim}
\switchcolumn
例如,我们的|indfnsverbatim|环境可以通过以下定义具有可选参数,以指定除|\footnotesize|之外的字体大小。
\begin{iverbatim}
    \newverbatim{indfnsverbatim}[1][\footnotesize]%
           {\begin{itemize}\item[]#1}{}{}{\end{itemize}}
    \end{iverbatim}

\switchcolumn[0]*
The argument \meta{beg-def-inner} is for \TeX{}perts who wish to do
something overriding what \LaTeX's |\verbatim| does.  Even if you don't
have much confidence in your \TeX{}pertise, however, you can do some
useful thing with this argument.  For example, the following is obtained
by itself.

\begin{itemize}\item[]
\newverbatim{slverbatim}{\ttfamily}{\slshape}{}{}
\begin{slverbatim}
\newverbatim{slverbatim}{}{\slshape}{}{}
\end{slverbatim}
\end{itemize}

\switchcolumn
\meta{beg-def-inner}参数是给希望覆盖\LaTeX 的|\verbatim|命令的\TeX 专家使用的。然而,即使你对自己的\TeX 专业知识没有太多信心,你也可以使用这个参数做一些有用的事情。例如,以下内容是通过以下命令得到的。
\begin{itemize}\item[]
\newverbatim{slverbatim}{\ttfamily}{\slshape}{}{}
\begin{slverbatim}
\newverbatim{slverbatim}{}{\slshape}{}{}
\end{slverbatim}
\end{itemize}

\switchcolumn[0]*
Also you will find a few commands for this argument in
\S\ref{sec:usage-var}.
\switchcolumn 此外,你还可以在第\ref{sec:usage-var}节中找到一些用于此参数的命令。

\switchcolumn[0]*
The needs of \meta{end-def-inner} is much more limited.  One example is to
check if |\end{verbatim}| is at the beginning of a line.  This examination
is done by;

\begin{quote}
|\newverbatim{myverbatim}{...}{...}%|\\
|        {\ifvmode |\meta{at-bol}| \else |\meta{not-at-bol}| \fi}{...}|
\end{quote}
\switchcolumn
\meta{end-def-inner}参数的需求要少得多。一个例子是检查|\end{verbatim}|是否位于一行的开头。可以通过以下方式进行检查:
\begin{quote}
|\newverbatim{myverbatim}{...}{...}%|\\
|        {\ifvmode |\meta{at-bol}| \else |\meta{not-at-bol}| \fi}{...}|
\end{quote}

\switchcolumn[0]*
\DescribeMacro{\renewverbatim}
You may redefine your own |verbatim|-like environment, or even |verbatim|
itself, by |\renewverbatim| whose arguments are same as those of
|\newenvironment|.

\end{paracol}
% \columnratio{0.55}
\begin{paracol}{2}

\subsection{\textsf{varvbtm}: To Make Variants of \texttt{verbatim}}
\label{sec:usage-var}

\subsubsection{Tab Emulation}
\label{sec:usage-var-tab}
\switchcolumn
\subsection{\textsf{varvbtm}: 生成\texttt{verbatim}的变体}
\subsubsection{模拟制表符}

\switchcolumn[0]*
\DescribeMacro{\newtabverbatim}
\DescribeMacro{\renewtabverbatim}
The commands |\(re)newtabverbatim| is to (re)define a |verbatim|-like
environment in which |<TAB>| acts as a tab.  The syntax of the command is
same as that of |\(re)newverbatim|, and its operation is equivalent to;

\begin{itemize}\item[]
|\(re)newverbatim|\Meta{env}\opt{n-args}\opt{default}\\
\mbox{}\qquad\qquad\Meta{beg-def-outer}|%|\\
\mbox{}\qquad\qquad|{|\meta{beg-def-inner}\meta{beg-def-for-tab}|}%|\\
\mbox{}\qquad\qquad|{|\meta{end-def-for-tab}\meta{end-def-inner}|}%|\\
\mbox{}\qquad\qquad\Meta{end-def-outer}
\end{itemize}
\switchcolumn
命令|\(re)newtabverbatim|用于(重新)定义一个类似于|verbatim|的环境,其中|<TAB>|被视为制表符。该命令的语法与|\(re)newverbatim|相同,其操作与其等效;
\begin{itemize}\item[]
|\(re)newverbatim|\Meta{env}\opt{n-args}\opt{default}\\
\mbox{}\qquad\qquad\Meta{beg-def-outer}|%|\\
\mbox{}\qquad\qquad|{|\meta{beg-def-inner}\meta{beg-def-for-tab}|}%|\\
\mbox{}\qquad\qquad|{|\meta{end-def-for-tab}\meta{end-def-inner}|}%|\\
\mbox{}\qquad\qquad\Meta{end-def-outer}
\end{itemize}

\switchcolumn[0]*
For example;

\begin{iverbatim}
\newtabverbatim{tabverbatim}{}{}{}{}
\end{iverbatim}
\switchcolumn
例如
\begin{iverbatim}
\newtabverbatim{tabverbatim}{}{}{}{}
\end{iverbatim}

\switchcolumn[0]*
defines |tabverbatim| environment just to make |<TAB>| act as a tab.
Another example to have tab emulation version of |indfnsverbatim| with
optional argument, say |indfnstabverbatim| is;

\begin{iverbatim}
\newtabverbatim{indfnstabverbatim}[1][\footnotesize]%
       {\begin{itemize}\item[]#1}{}{}{\end{itemize}}
\end{iverbatim}

\switchcolumn
定义了|tabverbatim|环境,使得|<TAB>|被视为制表符。还可以通过以下示例定义具有可选参数的|indfnsverbatim|的模拟制表符版本,例如|indfnstabverbatim|;
\begin{iverbatim}
\newtabverbatim{indfnstabverbatim}[1][\footnotesize]%
        {\begin{itemize}\item[]#1}{}{}{\end{itemize}}
\end{iverbatim}

\switchcolumn[0]*
Note that in the starred version, e.g. |tabverbatim*|, a |<TAB>| is
translated into a sequence of \verb*! !.
\switchcolumn
请注意,在星号版本中,例如|tabverbatim*|,|<TAB>|被转换为一个空格序列\verb*! !。

\switchcolumn[0]*
\DescribeOpt{VVBtabwidth}{counter}
The distance between tab stops is the width of eight characters of the
font used in the environment, i.e. typewriter font usually.  If you want
to change this default value, set the counter |VVBtabwidth| to the number
of characters of the distance.
\switchcolumn
制表符之间的距离是环境中使用的字体的八个字符的宽度,即通常的等宽字体。如果要更改此默认值,请将计数器|VVBtabwidth|设置为距离的字符数。

\switchcolumn[0]*
\DescribeMacro{\VVBbegintab}
\DescribeMacro{\VVBendtab}
The magical stuff for \meta{beg-def-for-tab} and \meta{end-def-for-tab}
is also accessible through commands |\VVBbegintab| and |\VVBendtab| for
\TeX{}perts who wish to do something with |\(re)newverbatim| rather than
|\(re)newtabverbatim|.
\switchcolumn
对于希望对|\(re)newverbatim|进行操作而不是对|\(re)newtabverbatim|进行操作的\TeX{}专家,
可以通过命令 |\VVBbegintab| 和 |\VVBendtab| 访问 \meta{beg-def-for-tab} 和 \meta{end-def-for-tab} 的神奇内容。
\end{paracol}
% 
\columnratio{0.55}
\begin{paracol}{2}
\subsubsection{Form Feed Character}
\label{sec:usage-var-ff}
\switchcolumn
\subsubsection{换页符}

\switchcolumn[0]*
\DescribeMacro{\VVBprintFF}
\DescribeMacro{\VVBprintFFas}
You might have found that |<FF>| (or |^L|) in |verbatim| caused a
mysterious error;
\begin{iverbatim}
! Forbidden control sequence found while scanning use of \@xverbatim.
\end{iverbatim}
\switchcolumn
您可能发现在|verbatim|中使用的|<FF>|(或|^L|)会导致一个神秘的错误;
\begin{iverbatim}
! Forbidden control sequence found while scanning use of \@xverbatim.
\end{iverbatim}

\switchcolumn[0]*
This is because |<FF>| is not {\em verbatimized}.  Giving the command
|\VVBprintFF| to $\langle$\textit{beg-def-}\allowbreak
\textit{outer}$\rangle$ (or \meta{beg-def-inner}) of |\newverbatim| does
it for you and makes |<FF>| printed as |^L| in default.  You may change
this default print image by;

\begin{quote}
|\VVBprintFFas|\Meta{str}
\end{quote}
\switchcolumn
这是因为|<FF>|没有被{\em verbatimized}。给出|\VVBprintFF|命令到$\langle$\textit{beg-def-}\allowbreak
\textit{outer}$\rangle$(或\meta{beg-def-inner})中的|\newverbatim|可以为您完成这个操作,并且默认情况下将|<FF>|打印为|^{L}|。您可以通过以下方式更改此默认打印图像;
\begin{quote}
|\VVBprintFFas|\Meta{str}
\end{quote}

\switchcolumn[0]*
where \meta{str} is a sequence of any printable characters other than |{|
and |}|.  Note that this command is very {\em fragile} as |\verb| and
|\index|, and thus should not be used in an argument of other commands
including |\(re)newverbatim|.
\switchcolumn
其中\meta{str}是除了|{|和|}|之外的任何可打印字符的序列。请注意,此命令非常{\em 脆弱},就像|\verb|和|\index|一样,因此不应在其他命令的参数中使用,包括|\(re)newverbatim|。

\switchcolumn[0]*
\DescribeMacro{\VVBbreakatFF}
\DescribeMacro{\VVBbreakatFFonly}
The other way to make |<FF>| acceptable is to give it a useful and natural
job, i.e.\ page breaking.  This is done by giving |\VVBbreakatFF| to
\meta{beg-def-inner} (not {\em outer}).  Its more powerful relative,
|\VVBbreakatFFonly|, is also available to allow page breaking at |<FF>|
only.  Unfortunately, these two commands are incompatible with
|\(re)newtabverbatim| and thus you have to use |\(re)newverbatim| with
|\VVBbegintab| {\em followed by} them.
\switchcolumn
另一种使 |<FF>| 可接受的方法是给它一个有用和自然的任务,即分页。这是通过将 |\VVBbreakatFF| 给予 \meta{beg-def-inner}(而不是 {\em outer})来实现的。它的更强大的形式 |\VVBbreakatFFonly| 也可用于仅在 |<FF>| 处分页。不幸的是,这两个命令与 |(re)newtabverbatim| 不兼容,因此您必须使用 |(re)newverbatim| 与 |\VVBbegintab| {\em followed by} 它们。
\end{paracol}
% 

\columnratio{0.55}
\begin{paracol}{2}
\subsubsection{Non-Verbatim Stuff in \texttt{verbatim}-like Environment}
\label{sec:usage-var-nonverb}
\switchcolumn
\subsubsection{非抄录环境中的非抄录内容}

\switchcolumn[0]*
\DescribeMacro{\VVBnonverb}
You might have once wished to insert a few non-verbatim stuff, for example 
math stuff.  The command, to be given to \meta{beg-def-outer};

\begin{quote}
\newverbatim{verbatimwithnv}{\VVBnonverb{\!}\ttfamily}{}{}{}
\begin{verbatimwithnv}
\VVBnonverb{\!$\langle\mbox{\textit{char}}\rangle$!}
\end{verbatimwithnv}
\end{quote}
\switchcolumn
你可能曾经希望插入一些非抄录内容,比如数学内容。可以通过给予\meta{beg-def-outer}的命令来实现;
\begin{quote}
\newverbatim{verbatimwithnv}{\VVBnonverb{\!}\ttfamily}{}{}{}
\begin{verbatimwithnv}
\VVBnonverb{\!$\langle\mbox{\textit{char}}\rangle$!}
\end{verbatimwithnv}
\end{quote}


\switchcolumn[0]*
makes it possible.  For example, the author just did the following to
produce the result shown above.

\begin{iverbatim}
\newverbatim{verbatimwithnv}{\VVBnonverb{\!}}{}{}{}
\begin{verbatimwithnv}
\VVBnonverb{\!$\langle\mbox{\textit{char}}\rangle$!}
\end{verbatimwithnv}
\end{iverbatim}
\switchcolumn
这样就可以了。例如,作者就是通过以下方式生成了上面显示的结果。
\begin{iverbatim}
\newverbatim{verbatimwithnv}{\VVBnonverb{\!}}{}{}{}
\begin{verbatimwithnv}
\VVBnonverb{\!$\langle\mbox{\textit{char}}\rangle$!}
\end{verbatimwithnv}
\end{iverbatim}


\switchcolumn[0]*
As shown in the example above, the non-verbatim staff is surrounded by
a pair of \meta{char}, the letter `|!|' in this case.  Note that
\meta{char} has to be preceded by `|\|' when it is given as the argument of
|\VVBnonbverb|, and \meta{char} should not be `|\|'.  Also note that the
default font for the non-verbatim part % is not that for verbatim part,
but the font used outside the environment

\footnote{Strictly speaking, the font used when \cs{VVBnonverb} is
invoked.  Thus if \cs{VVBnonverb} is preceded by a font changing command,
the fond chosen by the command will be used.}.
\switchcolumn
如上面的示例所示,非抄录部分被一对\meta{char}包围,这里是'!'字符。注意,当\meta{char}被作为|\VVBnonbverb|的参数给出时,应该在其前面加上`\',而且\meta{char}不能是'\'。另外,请注意非抄录部分的默认字体不是抄录部分的字体,而是环境外部使用的字体。

\switchcolumn[0]*
\DescribeMacro{\VVBnonverbmath}
As mentioned above, math stuffs will be most desirable to be
non-verbatim.  Thus the macro;
\begin{quote}
\newverbatim{verbatimwithnv}{\VVBnonverbmath\ttfamily}{}{}{}
\begin{verbatimwithnv}
\VVBnonverbmath[\$\langle\mbox{\textit{char}}\rangle$]
\end{verbatimwithnv}
\end{quote}
\switchcolumn
如上所述,非抄录的数学内容是最理想的。因此,可以使用以下宏;
\begin{quote}
\newverbatim{verbatimwithnv}{\VVBnonverbmath\ttfamily}{}{}{}
\begin{verbatimwithnv}
\VVBnonverbmath[\$\langle\mbox{\textit{char}}\rangle$]
\end{verbatimwithnv}
\end{quote}

\switchcolumn[0]*
gives you a shorthand to typeset the stuff surrounded by a pair of
\meta{char} in math mode.  Since the default of \meta{char} is |$| as
expected, the example above may be;
\begin{iverbatim}
\newverbatim{verbatimwithnv}{\VVBnonverbmath}{}{}{}
\begin{verbatimwithnv}
\VVBnonverb{\$\langle\mbox{\textit{char}}\rangle$}
\end{verbatimwithnv}
\end{iverbatim}
\switchcolumn
它让你可以很方便地在数学模式下生成一对\meta{char}包围的内容。由于\meta{char}的默认值是期望的 |$|,所以上面的示例也可以写成:
\begin{iverbatim}
\newverbatim{verbatimwithnv}{\VVBnonverbmath}{}{}{}
\begin{verbatimwithnv}
\VVBnonverb{\$\langle\mbox{\textit{char}}\rangle$}
\end{verbatimwithnv}
\end{iverbatim}

\end{paracol}
% 

\columnratio{0.55}
\begin{paracol}{2}

\subsubsection{Verbatim Input}
\label{sec:usage-var-input}
\switchcolumn
\subsubsection{抄录输入}

\switchcolumn[0]*
The last thing \textsf{varvbtm} gives you is;

\begin{itemize}\item[]
|\(re)newverbatiminput|\Meta{command}\opt{n-args}\opt{default}|%|\\
\phantom{\texttt{+(re)newverbatiminput}}
	\Meta{beg-def-outer}\Meta{beg-def-inner}|%|\\
\phantom{\texttt{+(re)newverbatiminput}}
	\Meta{end-def-inner}\Meta{end-def-outer}
\end{itemize}
\switchcolumn
最后一件事情是\textsf{varvbtm}给你提供的:
\begin{itemize}\item[]
|\(re)newverbatiminput|\Meta{command}\opt{n-args}\opt{default}|%|\\
\phantom{\texttt{+(re)newverbatiminput}}
    \Meta{beg-def-outer}\Meta{beg-def-inner}|%|\\
\phantom{\texttt{+(re)newverbatiminput}}
    \Meta{end-def-inner}\Meta{end-def-outer}
\end{itemize}
\switchcolumn[0]*
to define a \meta{command} to |\input| a file.  Since this define a
\meta{command} instead of an environment, \meta{command} should have `|\|'
as its prefix.  The \meta{command} has at least one mandatory argument,
\meta{file} to be input, which can be referred as first argument if
\opt{default} is not supplied, or as second otherwise.  Note that,
however, if the \meta{command} does not have any other arguments, you can
omit \opt{n-arg}.

For example;

\begin{iverbatim}
\newverbatiminput{\vinput}{}{}{}{}
\end{iverbatim}
\switchcolumn
这样可以定义一个\meta{command}来用于|\input|文件。由于定义的是\meta{command}而不是环境,所以\meta{command}应该以'\'作为前缀。至少有一个必需的参数\meta{file}要被输入,如果没有提供\opt{default},则可以将其作为第一个参数引用,否则作为第二个参数引用。然而,请注意,如果\meta{command}没有其他参数,可以省略\opt{n-arg}。

例如:
\begin{iverbatim}
\newverbatiminput{\vinput}{}{}{}{}
\end{iverbatim}
    
\switchcolumn[0]*
defines |\vinput|\Meta{file} (and |\vinput*|) that |\input| a \meta{file}
as if the \meta{file} has |\begin|\slash|\end{verbatim}| at its first and
last lines.  A little bit more complicated example;

\begin{iverbatim}
\newverbatiminput{\indfnsvinput}[2][\footnotesize]%
       {\begin{itemize}\item[]#1}{}{}{\end{itemize}}
\end{iverbatim}

\switchcolumn
定义了|\vinput|\Meta{file}(以及|\vinput*|),它们以|\begin|\slash|\end{verbatim}|作为\meta{file}的第一行和最后一行进行|\input|。再举一个略微复杂的例子:
\begin{iverbatim}
\newverbatiminput{\indfnsvinput}[2][\footnotesize]%
        {\begin{itemize}\item[]#1}{}{}{\end{itemize}}
\end{iverbatim}

\switchcolumn[0]*
defines a indented-footnotesize-by-default version of |\vinput|.
\switchcolumn
定义了一个默认为缩进的 |\vinput| 的 \texttt{footnotesize} 版本。

\switchcolumn[0]*
\IndexPrologue{\newpage\section*{Index}
Underlined number refers to the page where the specification of
corresponding entry is described.}
\StopEventually{
\section*{Acknowledgments\hfill 致谢}

\switchcolumn[0]*
The author thanks to Noboru Matsuda and Carlos Puchol whose posts to
news groups triggered writing very first version of macros in
\textsf{newvbtm} and \textsf{varvbtm}.

\switchcolumn
作者感谢 Noboru Matsuda 和 Carlos Puchol,他们在新闻组中的帖子触发了在 \textsf{newvbtm} 和 \textsf{varvbtm} 中编写宏的第一个版本。

\switchcolumn[0]*
For the implementation of these style files, the author refers the base
implementations of the macros for \texttt{verbatim} environment.
These macros are written by Leslie Lamport as a part of
\LaTeX-2.09 and \LaTeXe{} (1997/12/01) to which Johannes Braams and other
authors also contributed.
\switchcolumn
对于这些样式文件的实现,作者参考了 \texttt{verbatim} 环境的宏的基本实现。这些宏是由 Leslie Lamport 编写的,作为 \LaTeX-2.09 和 \LaTeXe{}(1997/12/01)的一部分,Johannes Braams 和其他作者也做出了贡献。

\end{paracol}

\PrintIndex}



\newpage 



\DocInput{newvbtm.dtx} 


\end{document}
%% tex newvbtm.ins
\endinput
%%
%% End of file `newvbtm-man.tex'.
