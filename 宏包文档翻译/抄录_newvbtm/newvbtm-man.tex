%%
%% This is file `newvbtm-man.tex',
%% generated with the docstrip utility.
%%
%% The original source files were:
%%
%% newvbtm.dtx  (with options: `driver')
%% 
%% IMPORTANT NOTICE:
%% 
%% For the copyright see the source file.
%% 
%% Any modified versions of this file must be renamed
%% with new filenames distinct from newvbtm-man.tex.
%% 
%% For distribution of the original source see the terms
%% for copying and modification in the file newvbtm.dtx.
%% 
%% This generated file may be distributed as long as the
%% original source files, as listed above, are part of the
%% same distribution. (The sources need not necessarily be
%% in the same archive or directory.)
%% Copyright (C) 1999-2002  Hiroshi Nakashima <nakasima@tutics.tut.ac.jp>
%%     (Toyohashi Univ. of Tech.)
%%
%% This program can be redistributed and/or modified under the terms
%% of the LaTeX Project Public License distributed from CTAN
%% archives in directory macros/latex/base/lppl.txt; either
%% version 1 of the License, or any later version.
%%
%% \CharacterTable
%%  {Upper-case    \A\B\C\D\E\F\G\H\I\J\K\L\M\N\O\P\Q\R\S\T\U\V\W\X\Y\Z
%%   Lower-case    \a\b\c\d\e\f\g\h\i\j\k\l\m\n\o\p\q\r\s\t\u\v\w\x\y\z
%%   Digits        \0\1\2\3\4\5\6\7\8\9
%%   Exclamation   \!     Double quote  \"     Hash (number) \#
%%   Dollar        \$     Percent       \%     Ampersand     \&
%%   Acute accent  \'     Left paren    \(     Right paren   \)
%%   Asterisk      \*     Plus          \+     Comma         \,
%%   Minus         \-     Point         \.     Solidus       \/
%%   Colon         \:     Semicolon     \;     Less than     \<
%%   Equals        \=     Greater than  \>     Question mark \?
%%   Commercial at \@     Left bracket  \[     Backslash     \\
%%   Right bracket \]     Circumflex    \^     Underscore    \_
%%   Grave accent  \`     Left brace    \{     Vertical bar  \|
%%   Right brace   \}     Tilde         \~}
%%
%%
\ProvidesFile{newvbtm-man.tex}
[2002/04/08 v1.1 ]
\documentclass{ltxdoc}
\usepackage{varvbtm}
\DisableCrossrefs
\PageIndex
\CodelineNumbered
\RecordChanges
\OnlyDescription
\def\ONLYDESCRIPTION{}
\advance\textwidth4em

\usepackage[heading=true
,scheme=chinese%中文方案
,fontset=none%不使用默认的字体设置
,space=auto%自动调整中英文间距
]{ctex}
\setCJKmainfont{FangZhengShuSong-GBK-1.ttf}[Path=/Users/virhuiai/hlProjects/Latex-Typesetting-Hub/font/方正/]%设置文本的中文有衬线字体
\setCJKsansfont{FangZhengHeiTi-GBK-1.ttf}[Path=/Users/virhuiai/hlProjects/Latex-Typesetting-Hub/font/方正/]%设置文本的中文无衬线字体为
\setCJKmonofont{FangZhengFangSong-GBK-1.ttf}[Path=/Users/virhuiai/hlProjects/Latex-Typesetting-Hub/font/方正/] %设置文本的中文等宽字体 
% \setCJKfamilyfont{fontKai}{LXGWWenKai-Regular.ttf}[Path=/Users/virhuiai/hlProjects/Latex-Typesetting-Hub/font/霞鹜文楷/]
\setCJKfamilyfont{fontKai}{FangZhengKaiTi-GBK-1.ttf}[Path=/Users/virhuiai/hlProjects/Latex-Typesetting-Hub/font/方正/]
\newcommand\fontKai{\CJKfamily{fontKai}}
\usepackage[a3paper%, hmargin=2.5cm, vmargin=1cm, includeheadfoot
,landscape]{geometry}
\usepackage{parskip}
\usepackage{paracol}
\usepackage{minted}
\begin{document}
\parindent=0pt
 
\DocInput{newvbtm_1.dtx}

\globalcounter{section}
% \title{\textsf{newvbtm} and \textsf{varvbtm}\\
	Packages for Variants of \texttt{verbatim} Environment\thanks{
	This file has version number \fileversion, last revised \filedate.}}
\author{Hiroshi Nakashima\\(Toyohashi Univ. of Tech.)\and 翻译:virhuiai\\福建师范大学}
\date{\filedate}
\maketitle

\begin{abstract}
\columnratio{0.55}
\begin{paracol}{2}
\noindent
This file provides two style files; \textsf{newvbtm} to define
|verbatim|-like environments; \textsf{varvbtm} to provide set of macros
for variants of |verbatim|, e.g.\ in which \texttt{\char`\^I} acts as a tab.    
\switchcolumn
\noindent
本文件提供了两个样式文件:\textsf{newvbtm} 用于定义类似于 |verbatim| 的环境;\textsf{varvbtm} 提供一组宏,用于处理 |verbatim| 的变体,例如其中 \texttt{\char`^I} 作为制表符。
\end{paracol}    
\end{abstract}

\tableofcontents
\newpage

% \columnratio{0.55}
\begin{paracol}{2}
\section{Introduction}
\label{sec:intro}
\switchcolumn
\section*{介绍}

\switchcolumn[0]*
\SpecialOptIndex{verbatim}{environment}
\LaTeX{} users often have trouble when they wish to have their own
customized |verbatim|-like environment.  Probably you once wished to have
an indented-footnotesize-|verbatim| instead of always typing;
\switchcolumn
当 \LaTeX{} 用户希望拥有自定义的类似于 |verbatim| 的环境时,通常会遇到困难。也许您曾经希望有一个缩进的脚注大小的 |verbatim|,而不是每次都要输入以下内容:
\switchcolumn[0]*
\begin{iverbatim}
\begin{itemize}\item[]\footnotesize
\begin{verbatim}
...
\end{verbatim}
\end{itemize}
\end{iverbatim}
\switchcolumn

\begin{minted}{latex}
\begin{itemize}\item[]\footnotesize
\begin{verbatim}
...
\end{verbatim}
\end{itemize}
\end{minted}

\switchcolumn[0]*
and tried the following just to know it does not work.
\begin{iverbatim}
\newenvironment{myverbatim}{\begin{itemize}\item[]\footnotesize
                           \begin{verbatim}}%
                          {\end{verbatim}\end{itemize}}
\end{iverbatim}
\switchcolumn 并尝试了以下代码,只是发现它无法正常工作:
\begin{minted}{latex}
\newenvironment{myverbatim}{\begin{itemize}\item[]\footnotesize
    \begin{verbatim}}%
    {\end{verbatim}\end{itemize}}
\end{minted}

\switchcolumn[0]*
Another trouble you probablly have had is that what you see in |verbatim|
text with |<TAB>| is not what you get because |<TAB>| does not acts as an
tab but a space.
\switchcolumn
您可能遇到的另一个问题是,使用 |<TAB>| 在 |verbatim| 文本中所看到的不是所获得的,因为 |<TAB>| 并不像制表符一样起作用,而是一个空格。
\switchcolumn[0]*
Of course it is possible to define your own |verbatim|-like environments
if you have enough knowledge of the implementation of |verbatim| including 
dirty tricks with |\catcode|.  However, even a \TeX{}pert should be bored
with typing a dirty code like;
\switchcolumn 
当然,如果您对包括使用 |\catcode| 进行一些“脏技巧”在内的 |verbatim| 的实现有足够的了解,那么您可以自定义自己的 |verbatim| 环境。然而,即使是 \TeX{} 专家,也会对输入以下这种“脏代码”感到厌烦:

\switchcolumn[0]*
\begin{iverbatim}
\begingroup \catcode`\|=0 \catcode`\[=1 \catcode`\]=2
\catcode`\{=12 \catcode`\}=12 \catcode`\\=12
|long|def|@myxverbatim##1\end{myverbatim}[##1|end[myverbatim]]
|endgroup
\end{iverbatim}
\switchcolumn
\begin{minted}{latex}
\begingroup \catcode`\|=0 \catcode`\[=1 \catcode`\]=2
\catcode`\{=12 \catcode`\}=12 \catcode`\\=12
|long|def|@myxverbatim##1\end{myverbatim}[##1|end[myverbatim]]
|endgroup
\end{minted}

\switchcolumn[0]*
\DescribeOpt{newvbtm}{style file}
\DescribeOpt{varvbtm}{style file}
The style files distributed with this document will solve these problems.
You will have two style files, \textsf{newvbtm.sty} and
\textsf{varvbtm.sty}, by processing \textsf{newvbtm.dtx} with
\textsf{docstrip}, or simply doing the following.
\begin{iverbatim}
% tex newvbtm.ins
\end{iverbatim}
\switchcolumn
本文档附带的样式文件将解决这些问题。通过处理 \textsf{newvbtm.dtx} 并使用 \textsf{docstrip},您可以得到两个样式文件:\textsf{newvbtm.sty} 和 \textsf{varvbtm.sty},或者只需要执行以下命令:
\begin{minted}{shell}
% tex newvbtm.ins
\end{minted}

\switchcolumn[0]*
The former style provides you |\(re)newverbatim| command to (re)define
your own |verbatim|-like environment easily.  The latter gives you a set
of various macros for tab-emulation, page break control, etc.
\switchcolumn
前者提供了 |\(re)newverbatim| 命令,可以轻松(重新)定义自己的 |verbatim| 类似环境。后者为您提供了一组用于模拟制表符、控制分页等的各种宏。
\end{paracol}

\DocInput{newvbtm.dtx}


\end{document}
%% tex newvbtm.ins
\endinput
%%
%% End of file `newvbtm-man.tex'.
