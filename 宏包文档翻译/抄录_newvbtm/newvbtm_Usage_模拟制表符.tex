\columnratio{0.55}
\begin{paracol}{2}

\subsection{\textsf{varvbtm}: To Make Variants of \texttt{verbatim}}
\label{sec:usage-var}

\subsubsection{Tab Emulation}
\label{sec:usage-var-tab}
\switchcolumn
\subsection{\textsf{varvbtm}: 生成\texttt{verbatim}的变体}
\subsubsection{模拟制表符}

\switchcolumn[0]*
\DescribeMacro{\newtabverbatim}
\DescribeMacro{\renewtabverbatim}
The commands |\(re)newtabverbatim| is to (re)define a |verbatim|-like
environment in which |<TAB>| acts as a tab.  The syntax of the command is
same as that of |\(re)newverbatim|, and its operation is equivalent to;

\begin{itemize}\item[]
|\(re)newverbatim|\Meta{env}\opt{n-args}\opt{default}\\
\mbox{}\qquad\qquad\Meta{beg-def-outer}|%|\\
\mbox{}\qquad\qquad|{|\meta{beg-def-inner}\meta{beg-def-for-tab}|}%|\\
\mbox{}\qquad\qquad|{|\meta{end-def-for-tab}\meta{end-def-inner}|}%|\\
\mbox{}\qquad\qquad\Meta{end-def-outer}
\end{itemize}
\switchcolumn
命令|\(re)newtabverbatim|用于(重新)定义一个类似于|verbatim|的环境,其中|<TAB>|被视为制表符。该命令的语法与|\(re)newverbatim|相同,其操作与其等效;
\begin{itemize}\item[]
|\(re)newverbatim|\Meta{env}\opt{n-args}\opt{default}\\
\mbox{}\qquad\qquad\Meta{beg-def-outer}|%|\\
\mbox{}\qquad\qquad|{|\meta{beg-def-inner}\meta{beg-def-for-tab}|}%|\\
\mbox{}\qquad\qquad|{|\meta{end-def-for-tab}\meta{end-def-inner}|}%|\\
\mbox{}\qquad\qquad\Meta{end-def-outer}
\end{itemize}

\switchcolumn[0]*
For example;

\begin{iverbatim}
\newtabverbatim{tabverbatim}{}{}{}{}
\end{iverbatim}
\switchcolumn
例如
\begin{iverbatim}
\newtabverbatim{tabverbatim}{}{}{}{}
\end{iverbatim}

\switchcolumn[0]*
defines |tabverbatim| environment just to make |<TAB>| act as a tab.
Another example to have tab emulation version of |indfnsverbatim| with
optional argument, say |indfnstabverbatim| is;

\begin{iverbatim}
\newtabverbatim{indfnstabverbatim}[1][\footnotesize]%
       {\begin{itemize}\item[]#1}{}{}{\end{itemize}}
\end{iverbatim}

\switchcolumn
定义了|tabverbatim|环境,使得|<TAB>|被视为制表符。还可以通过以下示例定义具有可选参数的|indfnsverbatim|的模拟制表符版本,例如|indfnstabverbatim|;
\begin{iverbatim}
\newtabverbatim{indfnstabverbatim}[1][\footnotesize]%
        {\begin{itemize}\item[]#1}{}{}{\end{itemize}}
\end{iverbatim}

\switchcolumn[0]*
Note that in the starred version, e.g. |tabverbatim*|, a |<TAB>| is
translated into a sequence of \verb*! !.
\switchcolumn
请注意,在星号版本中,例如|tabverbatim*|,|<TAB>|被转换为一个空格序列\verb*! !。

\switchcolumn[0]*
\DescribeOpt{VVBtabwidth}{counter}
The distance between tab stops is the width of eight characters of the
font used in the environment, i.e. typewriter font usually.  If you want
to change this default value, set the counter |VVBtabwidth| to the number
of characters of the distance.
\switchcolumn
制表符之间的距离是环境中使用的字体的八个字符的宽度,即通常的等宽字体。如果要更改此默认值,请将计数器|VVBtabwidth|设置为距离的字符数。

\switchcolumn[0]*
\DescribeMacro{\VVBbegintab}
\DescribeMacro{\VVBendtab}
The magical stuff for \meta{beg-def-for-tab} and \meta{end-def-for-tab}
is also accessible through commands |\VVBbegintab| and |\VVBendtab| for
\TeX{}perts who wish to do something with |\(re)newverbatim| rather than
|\(re)newtabverbatim|.
\switchcolumn
对于希望对|\(re)newverbatim|进行操作而不是对|\(re)newtabverbatim|进行操作的\TeX{}专家,
可以通过命令 |\VVBbegintab| 和 |\VVBendtab| 访问 \meta{beg-def-for-tab} 和 \meta{end-def-for-tab} 的神奇内容。
\end{paracol}