\columnratio{0.55}
\begin{paracol}{2}
\section{Introduction}
\label{sec:intro}
\switchcolumn
\section*{介绍}

\switchcolumn[0]*
\SpecialOptIndex{verbatim}{environment}
\LaTeX{} users often have trouble when they wish to have their own
customized |verbatim|-like environment.  Probably you once wished to have
an indented-footnotesize-|verbatim| instead of always typing;
\switchcolumn
当 \LaTeX{} 用户希望拥有自定义的类似于 |verbatim| 的环境时,通常会遇到困难。也许您曾经希望有一个缩进的脚注大小的 |verbatim|,而不是每次都要输入以下内容:
\switchcolumn[0]*
\begin{iverbatim}
\begin{itemize}\item[]\footnotesize
\begin{verbatim}
...
\end{verbatim}
\end{itemize}
\end{iverbatim}
\switchcolumn

\begin{minted}{latex}
\begin{itemize}\item[]\footnotesize
\begin{verbatim}
...
\end{verbatim}
\end{itemize}
\end{minted}

\switchcolumn[0]*
and tried the following just to know it does not work.
\begin{iverbatim}
\newenvironment{myverbatim}{\begin{itemize}\item[]\footnotesize
                           \begin{verbatim}}%
                          {\end{verbatim}\end{itemize}}
\end{iverbatim}
\switchcolumn 并尝试了以下代码,只是发现它无法正常工作:
\begin{minted}{latex}
\newenvironment{myverbatim}{\begin{itemize}\item[]\footnotesize
    \begin{verbatim}}%
    {\end{verbatim}\end{itemize}}
\end{minted}

\switchcolumn[0]*
Another trouble you probablly have had is that what you see in |verbatim|
text with |<TAB>| is not what you get because |<TAB>| does not acts as an
tab but a space.
\switchcolumn
您可能遇到的另一个问题是,使用 |<TAB>| 在 |verbatim| 文本中所看到的不是所获得的,因为 |<TAB>| 并不像制表符一样起作用,而是一个空格。
\switchcolumn[0]*
Of course it is possible to define your own |verbatim|-like environments
if you have enough knowledge of the implementation of |verbatim| including 
dirty tricks with |\catcode|.  However, even a \TeX{}pert should be bored
with typing a dirty code like;
\switchcolumn 
当然,如果您对包括使用 |\catcode| 进行一些“脏技巧”在内的 |verbatim| 的实现有足够的了解,那么您可以自定义自己的 |verbatim| 环境。然而,即使是 \TeX{} 专家,也会对输入以下这种“脏代码”感到厌烦:

\switchcolumn[0]*
\begin{iverbatim}
\begingroup \catcode`\|=0 \catcode`\[=1 \catcode`\]=2
\catcode`\{=12 \catcode`\}=12 \catcode`\\=12
|long|def|@myxverbatim##1\end{myverbatim}[##1|end[myverbatim]]
|endgroup
\end{iverbatim}
\switchcolumn
\begin{minted}{latex}
\begingroup \catcode`\|=0 \catcode`\[=1 \catcode`\]=2
\catcode`\{=12 \catcode`\}=12 \catcode`\\=12
|long|def|@myxverbatim##1\end{myverbatim}[##1|end[myverbatim]]
|endgroup
\end{minted}

\switchcolumn[0]*
\DescribeOpt{newvbtm}{style file}
\DescribeOpt{varvbtm}{style file}
The style files distributed with this document will solve these problems.
You will have two style files, \textsf{newvbtm.sty} and
\textsf{varvbtm.sty}, by processing \textsf{newvbtm.dtx} with
\textsf{docstrip}, or simply doing the following.
\begin{iverbatim}
% tex newvbtm.ins
\end{iverbatim}
\switchcolumn
本文档附带的样式文件将解决这些问题。通过处理 \textsf{newvbtm.dtx} 并使用 \textsf{docstrip},您可以得到两个样式文件:\textsf{newvbtm.sty} 和 \textsf{varvbtm.sty},或者只需要执行以下命令:
\begin{minted}{shell}
% tex newvbtm.ins
\end{minted}

\switchcolumn[0]*
The former style provides you |\(re)newverbatim| command to (re)define
your own |verbatim|-like environment easily.  The latter gives you a set
of various macros for tab-emulation, page break control, etc.
\switchcolumn
前者提供了 |\(re)newverbatim| 命令,可以轻松(重新)定义自己的 |verbatim| 类似环境。后者为您提供了一组用于模拟制表符、控制分页等的各种宏。
\end{paracol}