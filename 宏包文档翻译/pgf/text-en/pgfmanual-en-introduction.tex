


\subsection{How to Read This Manual}

This manual describes both the design of \tikzname\ and its usage. The
organization is very roughly according to ``user-friendliness''. The commands
and subpackages that are easiest and most frequently used are described first,
more low-level and esoteric features are discussed later.

If you have not yet installed \tikzname, please read the installation first.
Second, it might be a good idea to read the tutorial. Finally, you might wish
to skim through the description of \tikzname. Typically, you will not need to
read the sections on the basic layer. You will only need to read the part on
the system layer if you intend to write your own frontend or if you wish to
port \pgfname\ to a new driver.

The ``public'' commands and environments provided by the system are described
throughout the text. In each such description, the described command,
environment or option is printed in red. Text shown in green is optional and
can be left out.


\subsection{Authors and Acknowledgements}
\label{section-authors}

The bulk of the \pgfname\ system and its documentation was written by Till
Tantau. A further member of the main team is Mark Wibrow, who is responsible,
for example, for the \pgfname\ mathematical engine, many shapes, the decoration
engine, and matrices. The third member is Christian Feuers\"anger who
contributed the floating point library, image externalization, extended key
processing, and automatic hyperlinks in the manual.

Furthermore, occasional contributions have been made by Christophe Jorssen,
Jin-Hwan Cho, Olivier Binda, Matthias Schulz, Ren\'ee Ahrens, Stephan Schuster,
and Thomas Neumann.

Additionally, numerous people have contributed to the \pgfname\ system by
writing emails, spotting bugs, or sending libraries and patches. Many thanks to
all these people, who are too numerous to name them all!


\subsection{Getting Help}

When you need help with \pgfname\ and \tikzname, please do the following:

\begin{enumerate}
    \item Read the manual, at least the part that has to do with your
        problem.
    \item If that does not solve the problem, try having a look at the
        GitHub development page for \pgfname\ and \tikzname\ (see the
        title of this document). Perhaps someone has already reported a
        similar problem and someone has found a solution.
    \item On the website you will find numerous forums for getting help.
        There, you can write to help forums, file bug reports, join mailing
        lists, and so on.
    \item Before you file a bug report, especially a bug report concerning
        the installation, make sure that this is really a bug. In particular,
        have a look at the |.log| file that results when you \TeX\ your
        files. This |.log| file should show that all the right files are
        loaded from the right directories. Nearly all installation problems
        can be resolved by looking at the |.log| file.
    \item \emph{As a last resort} you can try to email me (Till Tantau) or,
        if the problem concerns the mathematical engine, Mark Wibrow. I do
        not mind getting emails, I simply get way too many of them. Because
        of this, I cannot guarantee that your emails will be answered in a 
        timely fashion or even at all. Your chances that your problem will
        be fixed are somewhat higher if you mail to the \pgfname\ mailing
        list (naturally, I read this list and answer questions when I have
        the time).
\end{enumerate}
我2 3
\end{document}
