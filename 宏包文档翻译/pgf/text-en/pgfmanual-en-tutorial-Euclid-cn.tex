% % Copyright 2019 by Till Tantau
% %
% % This file may be distributed and/or modified
% %
% % 1. under the LaTeX Project Public License and/or
% % 2. under the GNU Free Documentation License.
% %
% % See the file doc/generic/pgf/licenses/LICENSE for more details.


% \section{Tutorial: Euclid's Amber Version of the \emph{Elements}\\教程:欧几里德的《几何原本》琥珀版}

% In this third tutorial we have a look at how \tikzname\ can be used to draw
% geometric constructions.

% 在这个第三个教程中,我们将看看如何使用\tikzname\ 绘制几何构造。

% Euclid is currently quite busy writing his new book series, whose working title
% is ``Elements'' (Euclid is not quite sure whether this title will convey the
% message of the series to future generations correctly, but he intends to change
% the title before it goes to the publisher). Up to now, he wrote down his text
% and graphics on papyrus, but his publisher suddenly insists that he must submit
% in electronic form. Euclid tries to argue with the publisher that electronics
% will only be discovered thousands of years later, but the publisher informs him
% that the use of papyrus is no longer cutting edge technology and Euclid will
% just have to keep up with modern tools.

% 欧几里德目前正在忙于撰写他的新书系列,暂定名为《几何原本》(欧几里德不确定这个标题是否能正确传达系列的信息给后代,但他打算在提交给出版商之前更改标题)。到目前为止,他把自己的文本和图形写在纸莎草上,但他的出版商突然坚持他必须以电子形式提交。欧几里德试图向出版商辩解,电子技术只会在几千年后被发现,但出版商告诉他,使用纸莎草已经不再是最前沿的技术,欧几里德必须跟上现代工具的步伐。

% Slightly disgruntled, Euclid starts converting his papyrus entitled ``Book I,
% Proposition I'' to an amber version.

% 欧几里德有些不满,开始将自己的纸莎草《第一卷,命题一》转换成琥珀版本。


% \subsection{Book I, Proposition I\\第一卷,命题一}

% The drawing on his papyrus looks like this:\footnote{The text is taken from the
% wonderful interactive version of Euclid's Elements by David E. Joyce, to be
% found on his website at Clark University.}

% 他的纸莎草上的绘图如下所示:\footnote{文本摘自大卫·E·乔伊斯(David E. Joyce)出色的《欧几里德几何原本》互动版本,在克拉克大学的网站上可以找到。}

% \bigskip
% \noindent
% \begin{tikzpicture}[thick,help lines/.style={thin,draw=black!50}]
%   \pgfmathsetseed{1}
%   \def\A{\textcolor{input}{$A$}}
%   \def\B{\textcolor{input}{$B$}}
%   \def\C{\textcolor{output}{$C$}}
%   \def\D{$D$}
%   \def\E{$E$}

%   \colorlet{input}{blue!80!black}
%   \colorlet{output}{red!70!black}
%   \colorlet{triangle}{orange}

%   \coordinate [label=left:\A]
%     (A) at ($ (0,0) + .1*(rand,rand) $);
%   \coordinate [label=right:\B]
%     (B) at ($ (1.25,0.25) + .1*(rand,rand) $);

%   \draw [input] (A) -- (B);

%   \node [name path=D,help lines,draw,label=left:\D] (D) at (A) [circle through=(B)] {};
%   \node [name path=E,help lines,draw,label=right:\E] (E) at (B) [circle through=(A)] {};

%   \path [name intersections={of=D and E,by={[label=above:\C]C}}];

%   \draw [output] (A) -- (C);
%   \draw [output] (B) -- (C);

%   \foreach \point in {A,B,C}
%     \fill [black,opacity=.5] (\point) circle (2pt);

%   \begin{pgfonlayer}{background}
%     \fill[triangle!80] (A) -- (C) -- (B) -- cycle;
%   \end{pgfonlayer}

%   \node [below right,text width=10cm,align=justify] at (4,3)
%   {
%     \small
%     \textbf{Proposition I\hfill 命题 I}\par
%     \emph{To construct an \textcolor{triangle}{equilateral triangle}
%       on a given \textcolor{input}{finite straight line}.}
    
%       \emph{在给定的\textcolor{input}{有限直线}上构造一个\textcolor{triangle}{等边三角形}。}
%       \par
%     \vskip1em
%     Let \A\B\ be the given \textcolor{input}{finite straight line}. It
%     is required to construct an \textcolor{triangle}{equilateral
%       triangle} on the \textcolor{input}{straight line}~\A\B.

%       设\A\B\ 是给定的\textcolor{input}{有限直线}。要在直线\A\B 上构造一个\textcolor{triangle}{等边三角形}。

%       Describe the circle \B\C\D\ with center~\A\ and radius \A\B. Again
%     describe the circle \A\C\E\ with center~\B\ and radius \B\A. Join the
%     \textcolor{output}{straight lines} \C\A\ and \C\B\ from the
%     point~\C\ at which the circles cut one another to the points~\A\ and~\B.

%     以\A\ 为圆心,\A\B\ 为半径描述圆\B\C\D\ 。再以\B\ 为圆心,\B\A\ 为半径描述圆\A\C\E\ 。连接圆相交处\C\ 到点\A\ 和\B\ 的\textcolor{output}{直线}\C\A\ 和\C\B\ 。


%     Now, since the point~\A\ is the center of the circle \C\D\B,
%     therefore \A\C\ equals \A\B. Again, since the point \B\ is the
%     center of the circle \C\A\E, therefore \B\C\ equals \B\A. But
%     \A\C\ was proved equal to \A\B, therefore each of the straight
%     lines \A\C\ and \B\C\ equals \A\B. And
%     things which equal the same thing also equal one another,
%     therefore \A\C\ also equals \B\C. Therefore the three straight
%     lines \A\C, \A\B, and \B\C\ equal one another.
%     Therefore the \textcolor{triangle}{triangle} \A\B\C\ is
%     equilateral, and it has been  constructed on the given finite
%     \textcolor{input}{straight line}~\A\B.

%     由于点\A\ 是圆\C\D\B\ 的圆心,因此\A\C\ 等于\A\B。再由于点\B\ 是圆\C\A\E\ 的圆心,因此\B\C\ 等于\B\A。但是已经证明\A\C\ 等于\A\B,因此直线\A\C\ 和\B\C\ 都等于\A\B。而且相等于同一物体的物体也相等,因此\A\C\ 也等于\B\C。因此直线\A\C、\A\B\ 和\B\C\ 三者相等。所以\textcolor{triangle}{三角形}\A\B\C\ 是等边的,它已经在给定的有限\textcolor{input}{直线}\A\B\ 上被构造出来了。

%   };
% \end{tikzpicture}
% \bigskip

% Let us have a look at how Euclid can turn this into \tikzname\ code.

% 现在让我们看看欧几里德如何将其转换为\tikzname\ 代码。


% \subsubsection{Setting up the Environment\\设置环境}

% As in the previous tutorials, Euclid needs to load \tikzname, together with
% some libraries. These libraries are |calc|, |intersections|, |through|, and
% |backgrounds|. Depending on which format he uses, Euclid would use one of the
% following in the preamble:

% 与之前的教程一样,欧几里德需要加载\tikzname\ ,以及一些库。这些库是|calc|、|intersections|、|through|和|backgrounds|。根据他使用的格式,欧几里德会在导言区使用以下之一:
% \begin{codeexample}[code only]
% % For LaTeX:
% \usepackage{tikz}
% \usetikzlibrary{calc,intersections,through,backgrounds}
% \end{codeexample}

% \begin{codeexample}[code only]
% % For plain TeX:
% \input tikz.tex
% \usetikzlibrary{calc,intersections,through,backgrounds}
% \end{codeexample}

% \begin{codeexample}[code only]
% % For ConTeXt:
% \usemodule[tikz]
% \usetikzlibrary[calc,intersections,through,backgrounds]
% \end{codeexample}


% \subsubsection{The Line \emph{AB}\\线段\emph{AB}}

% The first part of the picture that Euclid wishes to draw is the line $AB$. That
% is easy enough, something like |\draw (0,0) -- (2,1);| might do. However,
% Euclid does not wish to reference the two points $A$ and $B$ as $(0,0)$ and
% $(2,1)$ subsequently. Rather, he wishes to just write |A| and |B|. Indeed, the
% whole point of his book is that the points $A$ and $B$ can be arbitrary and all
% other points (like $C$) are constructed in terms of their positions. It would
% not do if Euclid were to write down the coordinates of $C$ explicitly.

% 欧几里德希望绘制的图片的第一部分是线段$AB$。这很容易,像|\draw (0,0) -- (2,1);|这样的代码可能就可以了。然而,欧几里德不希望在后续引用点$A$和$B$时使用$(0,0)$和$(2,1)$。相反,他希望只写|A|和|B|。实际上,他的整个观点是点$A$和$B$可以是任意的,而所有其他点(如$C$)都是基于它们的位置构造的。如果欧几里德明确写下$C$的坐标,那就不太好了。

% So, Euclid starts with defining two coordinates using the |\coordinate|
% command:


% 因此,欧几里德开始使用|\coordinate|命令定义两个坐标:
% %
% \begin{codeexample}[]
% \begin{tikzpicture}
%   \coordinate (A) at (0,0);
%   \coordinate (B) at (1.25,0.25);

%   \draw[blue] (A) -- (B);
% \end{tikzpicture}
% \end{codeexample}

% That was easy enough. What is missing at this point are the labels for the
% coordinates. Euclid does not want them \emph{on} the points, but next to them.
% He decides to use the |label| option:

% 这很简单。这时还缺少坐标的标签。欧几里德不希望它们在点上,而是在点旁边。他决定使用|label|选项:
% %
% \begin{codeexample}[]
% \begin{tikzpicture}
%   \coordinate [label=left:\textcolor{blue}{$A$}]  (A) at (0,0);
%   \coordinate [label=right:\textcolor{blue}{$B$}] (B) at (1.25,0.25);

%   \draw[blue] (A) -- (B);
% \end{tikzpicture}
% \end{codeexample}

% At this point, Euclid decides that it would be even nicer if the points $A$ and
% $B$ were in some sense ``random''. Then, neither Euclid nor the reader can make
% the mistake of taking ``anything for granted'' concerning these position of
% these points. Euclid is pleased to learn that there is a |rand| function in
% \tikzname\ that does exactly what he needs: It produces a number between $-1$
% and $1$. Since \tikzname\ can do a bit of math, Euclid can change the
% coordinates of the points as follows:
% %

% 此时,欧几里德认为如果点$A$和$B$在某种意义上是“随机的”会更好。这样,欧几里德和读者都不会错误地将这些点的位置视为“理所当然”。欧几里德很高兴得知\tikzname\ 中有一个|rand|函数正好可以满足他的需求:它会生成一个介于$-1$和$1$之间的数字。由于\tikzname\ 可以进行一些数学计算,欧几里德可以按如下方式更改点的坐标:
% \begin{codeexample}[code only]
% \coordinate [...] (A) at (0+0.1*rand,0+0.1*rand);
% \coordinate [...] (B) at (1.25+0.1*rand,0.25+0.1*rand);
% \end{codeexample}

% This works fine. However, Euclid is not quite satisfied since he would prefer
% that the ``main coordinates'' $(0,0)$ and $(1.25,0.25)$ are ``kept separate''
% from the perturbation $0.1(\mathit{rand},\mathit{rand})$. This means, he would
% like to specify that coordinate $A$ as ``the point that is at $(0,0)$ plus one
% tenth of the vector  $(\mathit{rand},\mathit{rand})$''.


% 这样做很好。然而,欧几里德并不完全满意,因为他更希望“主要坐标”$(0,0)$和$(1.25,0.25)$与扰动$0.1(\mathit{rand},\mathit{rand})$“保持分离”。这意味着,他希望将坐标$A$指定为“位于$(0,0)$加上向量$(\mathit{rand},\mathit{rand})$的十分之一”的点。

% It turns out that the |calc| library allows him to do exactly this kind of
% computation. When this library is loaded, you can use special coordinates that
% start with |($| and end with |$)| rather than just |(| and~|)|. Inside these
% special coordinates you can give a linear combination of coordinates. (Note
% that the dollar signs are only intended to signal that a ``computation'' is
% going on; no mathematical typesetting is done.)

% 事实证明,|calc|库允许他进行这种计算。当加载了该库后,您可以使用以|($|开头,以|$)|结尾的特殊坐标,而不是只使用|(|和~|)|。在这些特殊坐标内,您可以给出坐标的线性组合。(请注意,美元符号仅用于表示正在进行“计算”,而不进行数学排版。)

% The new code for the coordinates is the following:

% 坐标的新代码如下:
% %
% \begin{codeexample}[code only]
% \coordinate [...] (A) at ($ (0,0) + .1*(rand,rand) $);
% \coordinate [...] (B) at ($ (1.25,0.25) + .1*(rand,rand) $);
% \end{codeexample}

% Note that if a coordinate in such a computation has a factor (like |.1|), you
% must place a |*| directly before the opening parenthesis of the coordinate. You
% can nest such computations.

% 请注意,如果计算中的坐标具有因子(如|.1|),则必须在坐标的开括号之前直接放置|*|。您可以嵌套这样的计算。


