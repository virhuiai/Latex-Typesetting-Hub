% % Copyright 2019 by Till Tantau
% %
% % This file may be distributed and/or modified
% %
% % 1. under the LaTeX Project Public License and/or
% % 2. under the GNU Free Documentation License.
% %
% % See the file doc/generic/pgf/licenses/LICENSE for more details.


% \section{Tutorial: Euclid's Amber Version of the \emph{Elements}\\教程:欧几里德的《几何原本》琥珀版}

% In this third tutorial we have a look at how \tikzname\ can be used to draw
% geometric constructions.

% 在这个第三个教程中,我们将看看如何使用\tikzname\ 绘制几何构造。

% Euclid is currently quite busy writing his new book series, whose working title
% is ``Elements'' (Euclid is not quite sure whether this title will convey the
% message of the series to future generations correctly, but he intends to change
% the title before it goes to the publisher). Up to now, he wrote down his text
% and graphics on papyrus, but his publisher suddenly insists that he must submit
% in electronic form. Euclid tries to argue with the publisher that electronics
% will only be discovered thousands of years later, but the publisher informs him
% that the use of papyrus is no longer cutting edge technology and Euclid will
% just have to keep up with modern tools.

% 欧几里德目前正在忙于撰写他的新书系列,暂定名为《几何原本》(欧几里德不确定这个标题是否能正确传达系列的信息给后代,但他打算在提交给出版商之前更改标题)。到目前为止,他把自己的文本和图形写在纸莎草上,但他的出版商突然坚持他必须以电子形式提交。欧几里德试图向出版商辩解,电子技术只会在几千年后被发现,但出版商告诉他,使用纸莎草已经不再是最前沿的技术,欧几里德必须跟上现代工具的步伐。

% Slightly disgruntled, Euclid starts converting his papyrus entitled ``Book I,
% Proposition I'' to an amber version.

% 欧几里德有些不满,开始将自己的纸莎草《第一卷,命题一》转换成琥珀版本。

