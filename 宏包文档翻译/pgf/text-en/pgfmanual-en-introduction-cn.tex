% % Copyright 2019 by Till Tantau
% %
% % This file may be distributed and/or modified
% %
% % 1. under the LaTeX Project Public License and/or
% % 2. under the GNU Free Documentation License.
% %
% % See the file doc/generic/pgf/licenses/LICENSE for more details.


% \section{Introduction\\简介}

% Welcome to the documentation of \tikzname\ and the underlying \pgfname\ system.
% What began as a small \LaTeX\ style for creating the graphics in my (Till
% Tantau's) PhD thesis directly with pdf\LaTeX\ has now grown to become a
% full-blown graphics language with a manual of over a thousand pages. The wealth
% of options offered by \tikzname\ is often daunting to beginners; but
% fortunately this documentation comes with a number slowly-paced tutorials that
% will teach you almost all you should know about \tikzname\ without your having
% to read the rest.

% 欢迎阅读 \tikzname\ 和底层的 \pgfname\ 系统的文档。最初只是一个为我(Till Tantau)的博士论文创建图形的小型 \LaTeX\ 样式,现在已经成长为一种拥有超过一千页手册的全功能图形语言。 \tikzname\ 提供的丰富选项%经常让初学者感到困惑;%
% % 提供的大量选项
% 往往会让初学者望而生畏;%
% 但幸运的是,这份文档附带了一系列循序渐进的教程,它们将教你几乎所有你应该了解的 \tikzname\ 知识,无需你阅读其余部分。



% I wish to start with the questions ``What is \tikzname?'' Basically, it just
% defines a number of \TeX\ commands that draw graphics. %
% For example, the code
% |\tikz \draw (0pt,0pt) -- (20pt,6pt);| yields the line \tikz \draw (0pt,0pt) --
% (20pt,6pt); %
% and the code |\tikz \fill[orange] (1ex,1ex) circle (1ex);| yields
% \tikz \fill[orange] (1ex,1ex) circle (1ex);. %
% In a sense, when you use
% \tikzname\ you ``program'' your graphics, just as you ``program'' your document
% when you use \TeX. %
% This also explains the name: \tikzname\ is a recursive
% acronym in the tradition of ``\textsc{gnu}'s Not Unix'' and means ``\tikzname\
% ist \emph{kein} Zeichenprogramm'', which translates to ``\tikzname\ is not a
% drawing program'', cautioning the reader as to what to expect. With \tikzname\
% you get all the advantages of the ``\TeX-approach to typesetting'' for your
% graphics: quick creation of simple graphics, precise positioning, the use of
% macros, often superior typography. You also inherit all the disadvantages:
% steep learning curve, no \textsc{wysiwyg}, small changes require a long
% recompilation time, and the code does not really ``show'' how things will look
% like.

% 我希望从“什么是 \tikzname?”这个问题开始。基本上,它只定义了一些绘制图形的 \TeX\ 命令。%
% 例如,\\代码 |\tikz \draw (0pt,0pt) -- (20pt,6pt);| 生成的线条是 \tikz \draw (0pt,0pt) --
% (20pt,6pt);%
% ,\\代码 |\tikz \fill[orange] (1ex,1ex) circle (1ex);| 生成的是 \tikz \fill[orange] (1ex,1ex) circle (1ex);。
% 在某种意义上,当你使用 \tikzname\ 时,你是在“编程”你的图形,就像你使用 \TeX\ “编程”你的文档一样。%
% 这也解释了名称:\tikzname\ 是一种递归的缩写,在 ``\textsc{gnu}'s Not Unix'' 的传统中意思是``\tikzname\
% ist \emph{kein} Zeichenprogramm'',翻译成汉语是“\tikzname\ 不是一个绘图程序”,
% 警告读者不要期待太多。使用 \tikzname\ ,你可以得到所有的“\TeX-排版方法”的优点:快速创建简单图形,精确定位,使用宏,通常提供优越的排版。你也会继承所有的缺点:陡峭的学习曲线,没有\textsc{wysiwyg},小的变动需要长时间的重新编译,代码并不能真正“显示”事物的样子。


% Now that we know what \tikzname\ is, what about ``\pgfname''? As mentioned
% earlier, \tikzname\ started out as a project to implement \TeX\ graphics macros
% that can be used both with pdf\LaTeX\ and also with the classical
% (PostScript-based) \LaTeX. In other words, I wanted to implement a ``portable
% graphics format'' for \TeX\ -- hence the name \pgfname. These early macros are
% still around and they form the ``basic layer'' of the system described in this
% manual, but most of the interaction an author has theses days is with
% \tikzname\ -- which has become a whole language of its own.

% 既然我们知道了什么是 \tikzname\ ,那么“\pgfname”又是什么呢?如前所述,\tikzname\ 开始是一个项目,用于实现可以与 pdf\LaTeX\ 和传统的(基于 PostScript 的)\LaTeX\ 一起使用的 \TeX\ 图形宏。换句话说,我想要实现一个 \TeX\ 的“便携式图形格式” —— 因此得名 \pgfname。这些早期的宏仍然存在,它们构成了本手册中描述的系统的“基础层”,但如今作者的大部分交互都是与 \tikzname\  进行的,它已经成为一种独立的语言。
