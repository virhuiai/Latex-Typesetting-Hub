% Copyright 2019 by Till Tantau
%
% This file may be distributed and/or modified
%
% 1. under the LaTeX Project Public License and/or
% 2. under the GNU Free Documentation License.
%
% See the file doc/generic/pgf/licenses/LICENSE for more details.


\section{Introduction\\简介}

Welcome to the documentation of \tikzname\ and the underlying \pgfname\ system.
What began as a small \LaTeX\ style for creating the graphics in my (Till
Tantau's) PhD thesis directly with pdf\LaTeX\ has now grown to become a
full-blown graphics language with a manual of over a thousand pages. The wealth
of options offered by \tikzname\ is often daunting to beginners; but
fortunately this documentation comes with a number slowly-paced tutorials that
will teach you almost all you should know about \tikzname\ without your having
to read the rest.

欢迎阅读 \tikzname\ 和底层的 \pgfname\ 系统的文档。最初只是一个为我(Till Tantau)的博士论文创建图形的小型 \LaTeX\ 样式,现在已经成长为一种拥有超过一千页手册的全功能图形语言。 \tikzname\ 提供的丰富选项%经常让初学者感到困惑;%
% 提供的大量选项
往往会让初学者望而生畏;%
但幸运的是,这份文档附带了一系列循序渐进的教程,它们将教你几乎所有你应该了解的 \tikzname\ 知识,无需你阅读其余部分。



I wish to start with the questions ``What is \tikzname?'' Basically, it just
defines a number of \TeX\ commands that draw graphics. %
For example, the code
|\tikz \draw (0pt,0pt) -- (20pt,6pt);| yields the line \tikz \draw (0pt,0pt) --
(20pt,6pt); %
and the code |\tikz \fill[orange] (1ex,1ex) circle (1ex);| yields
\tikz \fill[orange] (1ex,1ex) circle (1ex);. %
In a sense, when you use
\tikzname\ you ``program'' your graphics, just as you ``program'' your document
when you use \TeX. %
This also explains the name: \tikzname\ is a recursive
acronym in the tradition of ``\textsc{gnu}'s Not Unix'' and means ``\tikzname\
ist \emph{kein} Zeichenprogramm'', which translates to ``\tikzname\ is not a
drawing program'', cautioning the reader as to what to expect. With \tikzname\
you get all the advantages of the ``\TeX-approach to typesetting'' for your
graphics: quick creation of simple graphics, precise positioning, the use of
macros, often superior typography. You also inherit all the disadvantages:
steep learning curve, no \textsc{wysiwyg}, small changes require a long
recompilation time, and the code does not really ``show'' how things will look
like.

我希望从“什么是 \tikzname?”这个问题开始。基本上,它只定义了一些绘制图形的 \TeX\ 命令。%
例如,\\代码 |\tikz \draw (0pt,0pt) -- (20pt,6pt);| 生成的线条是 \tikz \draw (0pt,0pt) --
(20pt,6pt);%
,\\代码 |\tikz \fill[orange] (1ex,1ex) circle (1ex);| 生成的是 \tikz \fill[orange] (1ex,1ex) circle (1ex);。
在某种意义上,当你使用 \tikzname\ 时,你是在“编程”你的图形,就像你使用 \TeX\ “编程”你的文档一样。%
这也解释了名称:\tikzname\ 是一种递归的缩写,在 ``\textsc{gnu}'s Not Unix'' 的传统中意思是``\tikzname\
ist \emph{kein} Zeichenprogramm'',翻译成汉语是“\tikzname\ 不是一个绘图程序”,
警告读者不要期待太多。使用 \tikzname\ ,你可以得到所有的“\TeX-排版方法”的优点:快速创建简单图形,精确定位,使用宏,通常提供优越的排版。你也会继承所有的缺点:陡峭的学习曲线,没有\textsc{wysiwyg},小的变动需要长时间的重新编译,代码并不能真正“显示”事物的样子。


Now that we know what \tikzname\ is, what about ``\pgfname''? As mentioned
earlier, \tikzname\ started out as a project to implement \TeX\ graphics macros
that can be used both with pdf\LaTeX\ and also with the classical
(PostScript-based) \LaTeX. In other words, I wanted to implement a ``portable
graphics format'' for \TeX\ -- hence the name \pgfname. These early macros are
still around and they form the ``basic layer'' of the system described in this
manual, but most of the interaction an author has theses days is with
\tikzname\ -- which has become a whole language of its own.

既然我们知道了什么是 \tikzname\ ,那么“\pgfname”又是什么呢?如前所述,\tikzname\ 开始是一个项目,用于实现可以与 pdf\LaTeX\ 和传统的(基于 PostScript 的)\LaTeX\ 一起使用的 \TeX\ 图形宏。换句话说,我想要实现一个 \TeX\ 的“便携式图形格式” —— 因此得名 \pgfname。这些早期的宏仍然存在,它们构成了本手册中描述的系统的“基础层”,但如今作者的大部分交互都是与 \tikzname\  进行的,它已经成为一种独立的语言。



\subsection{The Layers Below \tikzname\\在\tikzname{}下面的层}

It turns out that there are actually \emph{two} layers below \tikzname:

% 事实证明,
实际上在 \tikzname\ 下面有\emph{两}层:
%
\begin{description}
    \item[System layer 系统层:] This layer provides a complete abstraction of what
        is going on ``in the driver''. The driver is a program like |dvips|
        or |dvipdfm| that takes a |.dvi| file as input and generates a |.ps|
        or a |.pdf| file. (The |pdftex| program also counts as a driver, even
        though it does not take a |.dvi| file as input. Never mind.) Each
        driver has its own syntax for the generation of graphics, causing
        headaches to everyone who wants to create graphics in a portable way.
        \pgfname's system layer ``abstracts away'' these differences. For
        example, the system command |\pgfsys@lineto{10pt}{10pt}| extends the
        current path  to the coordinate $(10\mathrm{pt},10\mathrm{pt})$ of
        the current |{pgfpicture}|. Depending on whether |dvips|, |dvipdfm|,
        or |pdftex| is used to process the document, the system command will
        be converted to different |\special| commands. The system layer is as
        ``minimalistic'' as possible since each additional command makes it
        more work to port \pgfname\ to a new driver.

        这一层完全抽象了驱动器中正在发生的事情。驱动器是像 |dvips| 或 |dvipdfm| 这样的程序,它以 |.dvi| 文件为输入并生成 |.ps| 或 |.pdf| 文件。(|pdftex| 程序也算作驱动器,尽管它不以 |.dvi| 文件为输入。不用在意。)每个驱动器都有自己生成图形的语法,给想要以便携式方式创建图形的人带来了头痛。 \pgfname 的系统层“抽象化”了这些差异。例如,系统命令 |\pgfsys@lineto{10pt}{10pt}| 将当前路径扩展到当前 |{pgfpicture}| 的坐标 $(10\mathrm{pt},10\mathrm{pt})$。取决于是否使用 |dvips|,|dvipdfm|,或者 |pdftex| 来处理文档,系统命令将被转换为不同的 |\special| 命令。系统层尽可能“简化”,因为每一个额外的命令都会增加将 \pgfname\ 移植到新驱动器的工作。

        As a user, you will not use the system layer directly.

        作为用户,你不会直接使用系统层。
    \item[Basic layer 基础层:] The basic layer provides a set of basic commands that
        allow you to produce complex graphics in a much easier manner than by
        using the system layer directly. For example, the system layer provides
        no commands for creating circles since circles can be composed from the
        more basic Bézier curves (well, almost). However, as a user you will
        want to have a simple command to create circles (at least I do) instead
        of having to write down half a page of Bézier curve support
        coordinates. Thus, the basic layer provides a command |\pgfpathcircle|
        that generates the necessary curve coordinates for you.

        基础层提供了一组基本命令,使你能够比直接使用系统层更容易地制作复杂的图形。例如,系统层没有提供创建圆的命令,因为圆可以由更基本的贝塞尔曲线组成(好吧,差不多)。然而,作为用户,你会希望有一个简单的命令来创建圆(至少我希望如此),而不是需要写下半页的贝塞尔曲线支持坐标。因此,基础层提供了一个命令 |\pgfpathcircle|,它为你生成必要的曲线坐标。

        The basic layer consists of a \emph{core}, which consists of several
        interdependent packages that can only be loaded \emph{en bloc}, and
        additional \emph{modules} that extend the core by more
        special-purpose commands like node management or a plotting
        interface. For instance, the \textsc{beamer} package uses only the
        core and not, say, the |shapes| modules.

        基础层由一个\emph{core}组成,这是由几个相互依赖的包组成的,它们只能被\emph{en bloc}加载,以及额外的\emph{modules},它们通过更多的特殊目的命令来扩展 core,如节点管理或绘图接口。例如,\textsc{beamer} 包只使用 core,而不使用,比如说,|shapes| 模块。
\end{description}

In theory, \tikzname\ itself is just one of several possible ``frontends''.
which are sets of commands or a special syntax that makes using the basic layer
easier. A problem with directly using the basic layer is that code written for
this layer is often too ``verbose''. For example, to draw a simple triangle,
you may need as many as five commands when using the basic layer: One for
beginning a path at the first corner of the triangle, one for extending the
path to the second corner, one for going to the third, one for closing the
path, and one for actually painting the triangle (as opposed to filling it).
With the \tikzname\ frontend all this boils down to a single simple
\textsc{metafont}-like command:

理论上,\tikzname\ 本身只是几种可能的“前端”之一,这些前端是一组命令或一种特殊语法,使得使用基础层更加容易。直接使用基础层的问题在于,为这一层编写的代码通常过于“冗长”。例如,要画一个简单的三角形,你可能需要使用基础层的五个命令:一个用于在三角形的第一个角开始一条路径,一个用于将路径延伸到第二个角,一个用于去到第三个角,一个用于关闭路径,以及一个用于实际绘制三角形(而不是填充它)。使用 \tikzname\ 前端,所有这些都可以简化为一个简单的 \textsc{metafont} 类似的命令:
%
\begin{verbatim}
\draw (0,0) -- (1,0) -- (1,1) -- cycle;
\end{verbatim}

In practice, \tikzname\ is the only ``serious'' frontend for \pgfname. It gives
you access to all features of \pgfname, but it is intended to be easy to use.
The syntax is a mixture of \textsc{metafont} and \textsc{pstricks} and some
ideas of myself. There are other frontends besides \tikzname, but they are intended
more as ``technology studies'' and less as serious alternatives to
\tikzname. In particular, the |pgfpict2e| frontend   reimplements the standard
\LaTeX\ |{picture}|  environment and commands like |\line| or |\vector| using
the \pgfname\ basic layer. This layer is not really ``necessary'' since the
|pict2e.sty| package does at least as good a job at reimplementing the
|{picture}| environment. Rather, the idea behind this package is to have a
simple demonstration of how a frontend can be implemented.

在实践中,\tikzname\ 是\pgfname 的唯一“严肃的”前端。它为您提供了\pgfname 的所有功能,但旨在易于使用。其语法是\textsc{metafont}和\textsc{pstricks}以及我自己的一些想法的混合。除了\tikzname\ 之外还有其他前端,但它们的目的更多是作为“技术研究”,而不是作为\tikzname 的真正替代品。特别是,|pgfpict2e|前端使用\pgfname 基本层重新实现了标准\LaTeX|{picture}|环境和|\line|或|\vector|等命令。这个层次并不是真的“必要的”,因为|pict2e.sty|包至少和重新实现|{picture}|环境一样出色。相反,这个包背后的想法是展示一个前端如何实现的简单演示。

% 实际上,\tikzname\ 是 \pgfname\ 唯一的“严肃”前端。它使你能够访问 \pgfname\ 的所有功能,但它旨在易于使用。其语法是 \textsc{metafont} 和 \textsc{pstricks} 的混合,以及一些我自己的想法。除了 \tikzname\ 之外还有其他前端,但它们更像是“技术研究”而不是 \tikzname\ 的严肃替代品。特别是,|pgfpict2e| 前端使用 \pgfname\ 基础层重新实现了标准的 \LaTeX\ |{picture}| 环境以及像 |\line| 或 |\vector| 这样的命令。这个层次并不真正“必要”,因为 |pict2e.sty| 包至少能做得和重新实现 |{picture}| 环境一样好。相反,这个包背后的理念是展示如何可以实现一个前端。



Since most users will only use \tikzname\ and almost no one will use the system
layer directly, this manual is mainly about \tikzname\ in the first parts; the
basic layer and the system layer are explained at the end.

由于大多数用户只会使用\tikzname ,几乎没有人会直接使用系统层,所以在前面部分这本手册主要关于\tikzname ;基本层和系统层在后面解释。

% 由于大多数用户只会使用 \tikzname\,几乎没有人会直接使用系统层,所以这个手册的主要内容是关于 \tikzname\ 的第一部分;基础层和系统层在最后解释。




\subsection{Comparison with Other Graphics Packages\\与其他绘图包的比较}

\tikzname\ is not the only graphics package for \TeX. In the following, I try
to give a reasonably fair comparison of \tikzname\ and other packages.
%

\tikzname 并不是 \TeX 的唯一绘图包。接下来,我将尽量公正地比较 \tikzname 和其他包的特点。

\begin{enumerate}
    \item The standard \LaTeX\ |{picture}| environment allows you to create
        simple graphics, but little more. This is certainly not due to a lack
        of knowledge or imagination on the part of \LaTeX's designer(s).
        Rather, this is the price paid for the |{picture}| environment's
        portability: It works together with all backend drivers.

        标准的 \LaTeX\ |{picture}| 环境允许你创建简单的图形,但是不太可能做更多。这肯定不是因为 \LaTeX 的设计者缺乏知识或想象力。反而,这是为了 |{picture}| 环境的可移植性而付出的代价:它可以与所有后端驱动一起工作。
    \item The |pstricks| package is certainly powerful enough to create any
        conceivable kind of graphic, but it is not really portable. Most
        importantly, it does not work with |pdftex| nor with any other driver
        that produces anything but PostScript code.

        |pstricks| 包肯定足够强大,能够创建任何可想象的图形,但它实际上并不真正可移植。最重要的是,它不与 |pdftex| 或任何其他产生除 PostScript 代码之外的东西的驱动一起工作。

        Compared to \tikzname, |pstricks| has a similar support base. There
        are many nice extra packages for special purpose situations that have
        been contributed by users over the last decade. The \tikzname\ syntax
        is more consistent than the |pstricks| syntax as \tikzname\ was
        developed ``in a more centralized manner'' and also ``with the
        shortcomings on |pstricks| in mind''.

        与 \tikzname 相比,|pstricks| 有着相似的支持基础。过去十年里,用户贡献了许多用于特殊目的情况的漂亮的额外包。与 |pstricks| 语法相比,\tikzname\ 的语法更加一致,因为 \tikzname\ 是“以更集中的方式”开发的,也是“考虑到 |pstricks| 的不足”开发的。
    \item The |xypic| package is an older package for creating graphics.
        However, it is more difficult to use and to learn because the syntax
        and the documentation are a bit cryptic.

        |xypic| 包是一个用于创建图形的旧包。然而,由于语法和文档都有些晦涩,所以使用它并学习它更加困难。
    \item The |dratex| package is a small graphic package for creating a
        graphics. Compared to the other package, including \tikzname, it is
        very small, which may or may not be an advantage.

        |dratex| 包是一个用于创建图形的小型图形包。与包括 \tikzname 在内的其他包相比,它非常小,这可能是优点也可能是缺点。
    \item The |metapost| program is a powerful alternative to \tikzname. It
        used to be an external program, which entailed a bunch of problems,
        but in Lua\TeX\ it is now built in. An obstacle with |metapost| is
        the inclusion of labels. This is \emph{much} easier to achieve using
        \pgfname.

        |metapost| 程序是 \tikzname 的一个强大的替代品。它过去是一个外部程序,这带来了一堆问题,但在 Lua\TeX 中,它现在是内置的。使用 |metapost| 的一个障碍是标签的包含。使用 \pgfname 这个任务就\emph{容易得多}。
    \item The |xfig| program is an important alternative to \tikzname\ for
        users who do not wish to ``program'' their graphics as is necessary
        with \tikzname\ and the other packages above. There is a conversion
        program that will convert |xfig| graphics to \tikzname.

        |xfig| 程序是对那些不希望像 \tikzname 和上述其他包那样“编程”其图形的用户来说,是 \tikzname 的一个重要替代品。有一个转换程序可以将 |xfig| 图形转换为 \tikzname。
\end{enumerate}





\subsection{Utility Packages\\实用工具包}

The \pgfname\ package comes along with a number of utility package that are not
really about creating graphics and which can be used independently of \pgfname.
However, they are bundled with \pgfname, partly out of convenience, partly
because their functionality is closely intertwined with \pgfname. These utility
packages are:

\pgfname\ 包附带了一些不真正用于创建图形的实用工具包,这些工具包可以独立于 \pgfname 使用。然而,它们被捆绑在 \pgfname 中,部分是出于方便,部分是因为它们的功能与 \pgfname 密切相关。这些实用工具包包括:

%
\begin{enumerate}
    \item The |pgfkeys| package defines a powerful key management facility.
        It can be used completely independently of \pgfname.

        |pgfkeys| 包定义了一个强大的键管理设施。它可以完全独立于 \pgfname 使用。

        % pgfkeys 包定义了一个强大的键管理功能。它可以完全独立于 PGF 使用。
    \item The |pgffor| package defines a useful |\foreach| statement.

        |pgffor| 包定义了一个有用的 |\foreach| 语句。
    \item The |pgfcalendar| package defines macros for creating calendars.
        Typically, these calendars will be rendered using \pgfname's graphic
        engine, but you can use |pgfcalendar| also typeset calendars using
        normal text. The package also defines commands for ``working'' with
        dates.

        |pgfcalendar| 包定义了用于创建日历的宏。通常,这些日历将使用 \pgfname 的图形引擎进行渲染,但您也可以使用 |pgfcalendar| 使用普通文本进行日历的排版。该包还定义了用于“处理”日期的命令。
    \item The |pgfpages| package is used to assemble several pages into a
        single page. It provides commands for assembling several ``virtual
        pages'' into a single ``physical page''. The idea is that whenever
        \TeX\ has a page ready for ``shipout'', |pgfpages| interrupts this
        shipout and instead stores the page to be shipped out in a special
        box. When enough ``virtual pages'' have been accumulated in this way,
        they are scaled down and arranged on a ``physical page'', which then
        \emph{really} shipped out. This mechanism allows you to create ``two
        page on one page'' versions of a document directly inside \LaTeX\
        without the use of any external programs. However, |pgfpages| can do
        quite a lot more than that. You can use it to put logos and watermark
        on pages, print up to 16 pages on one page, add borders to pages, and
        more.

        |pgfpages| 包用于将多个页面组合成一个页面。它提供了将多个“虚拟页面”组合成一个“物理页面”的命令。其想法是,每当 \TeX\ 有一个页面准备好进行“输出”时,|pgfpages| 会中断这个输出过程,而是将准备输出的页面存储在一个特殊的盒子中。当以这种方式积累了足够的“虚拟页面”时,它们被缩小并排列在一个“物理页面”上,然后\emph{真正地}输出出去。这种机制让您可以直接在 \LaTeX\ 中创建文档的“一页上的两页”版本,而无需使用任何外部程序。然而,|pgfpages| 可以做的远不止这些。您可以用它在页面上放置徽标和水印,一次打印多达16页的页面,为页面添加边框,等等。
\end{enumerate}



\subsection{How to Read This Manual\\如何阅读本手册}

This manual describes both the design of \tikzname\ and its usage. The
organization is very roughly according to ``user-friendliness''. The commands
and subpackages that are easiest and most frequently used are described first,
more low-level and esoteric features are discussed later.



此手册描述了 \tikzname\ 的设计和使用。其组织结构大致遵循“用户友好性”的原则。最简单、最常用的命令和子包被描述在前面,之后是更低级和更深奥的特性。


If you have not yet installed \tikzname, please read the installation first.
Second, it might be a good idea to read the tutorial. Finally, you might wish
to skim through the description of \tikzname. Typically, you will not need to
read the sections on the basic layer. You will only need to read the part on
the system layer if you intend to write your own frontend or if you wish to
port \pgfname\ to a new driver.

如果您尚未安装 \tikzname,请首先阅读安装部分。其次,阅读教程可能是个好主意。最后,您可能希望浏览一下 \tikzname 的描述。通常,您不需要阅读关于基础层的部分。只有当您打算编写自己的前端或希望将 \pgfname\ 移植到新驱动程序时,才需要阅读关于系统层的部分。

The ``public'' commands and environments provided by the system are described
throughout the text. In each such description, the described command,
environment or option is printed in red. Text shown in green is optional and
can be left out.

系统提供的 "公开" 命令和环境在文本中有所描述。在每个此类描述中,描述的命令、环境或选项以红色打印。绿色显示的文本是可选的,可以省略。




\subsection{Authors and Acknowledgements\\作者和致谢}
\label{section-authors}

The bulk of the \pgfname\ system and its documentation was written by Till
Tantau. A further member of the main team is Mark Wibrow, who is responsible,
for example, for the \pgfname\ mathematical engine, many shapes, the decoration
engine, and matrices. The third member is Christian Feuers\"anger who
contributed the floating point library, image externalization, extended key
processing, and automatic hyperlinks in the manual.

大部分 \pgfname\ 系统和其文档是由 Till Tantau 编写的。主要团队的另一位成员是 Mark Wibrow,他负责 \pgfname\ 数学引擎、许多形状、装饰引擎和矩阵等。第三位成员是 Christian Feuers"anger,他贡献了浮点库、图像外部化、扩展键处理和手册中的自动超链接。

Furthermore, occasional contributions have been made by Christophe Jorssen,
Jin-Hwan Cho, Olivier Binda, Matthias Schulz, Ren\'ee Ahrens, Stephan Schuster,
and Thomas Neumann.

此外,Christophe Jorssen、Jin-Hwan Cho、Olivier Binda、Matthias Schulz、Ren'ee Ahrens、Stephan Schuster 和 Thomas Neumann 也偶尔有所贡献。


Additionally, numerous people have contributed to the \pgfname\ system by
writing emails, spotting bugs, or sending libraries and patches. Many thanks to
all these people, who are too numerous to name them all!

此外,许多人通过写电子邮件、发现错误或发送库和补丁,对 \pgfname\ 系统做出了贡献。非常感谢所有这些人,他们太多,无法一一列举名字!


\subsection{Getting Help\\获取帮助}

When you need help with \pgfname\ and \tikzname, please do the following:

当您需要 \pgfname\ 和 \tikzname 的帮助时,请按照以下步骤操作:


\begin{enumerate}
    \item Read the manual, at least the part that has to do with your
        problem.

        阅读手册,至少阅读与您的问题有关的部分。
    \item If that does not solve the problem, try having a look at the
        GitHub development page for \pgfname\ and \tikzname\ (see the
        title of this document). Perhaps someone has already reported a
        similar problem and someone has found a solution.

        如果这不能解决问题,尝试查看 \pgfname\ 和 \tikzname 的 GitHub 开发页面(请参见本文档的标题)。也许有人已经报告了类似的问题,有人找到了解决方案。
    \item On the website you will find numerous forums for getting help.
        There, you can write to help forums, file bug reports, join mailing
        lists, and so on.

        在网站上,您会找到许多获取帮助的论坛。在那里,您可以向帮助论坛写信,提交错误报告,加入邮件列表等。
    \item Before you file a bug report, especially a bug report concerning
        the installation, make sure that this is really a bug. In particular,
        have a look at the |.log| file that results when you \TeX\ your
        files. This |.log| file should show that all the right files are
        loaded from the right directories. Nearly all installation problems
        can be resolved by looking at the |.log| file.

        在您提交错误报告之前,特别是关于安装的错误报告,请确保这确实是一个错误。特别是,看看当您 \TeX\ 您的文件时产生的 |.log| 文件。这个 |.log| 文件应该显示所有正确的文件都从正确的目录加载。几乎所有的安装问题都可以通过查看 |.log| 文件来解决。
    \item \emph{As a last resort} you can try to email me (Till Tantau) or,
        if the problem concerns the mathematical engine, Mark Wibrow. I do
        not mind getting emails, I simply get way too many of them. Because
        of this, I cannot guarantee that your emails will be answered in a 
        timely fashion or even at all. Your chances that your problem will
        be fixed are somewhat higher if you mail to the \pgfname\ mailing
        list (naturally, I read this list and answer questions when I have
        the time).

        \emph{最后的手段},你可以尝试给我(Till Tantau)发邮件,或者,如果问题涉及到数学引擎,可以给 Mark Wibrow 发邮件。我并不介意收到电子邮件,我只是收到的太多了。因此,我不能保证您的邮件会及时回复,甚至根本不会回复。如果您将问题发送到 \pgfname\ 邮件列表,您的问题被解决的几率会稍高一些(当然,我会阅读这个列表,并在我有时间的时候回答问题)。
\end{enumerate}