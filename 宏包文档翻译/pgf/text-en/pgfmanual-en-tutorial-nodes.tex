

\subsection{Placing Nodes Using the At Syntax\\使用At语法放置节点}

Hagen now understands how the |node| operation adds nodes to the path, but it
seems a bit silly to create a path using the |\path| operation, consisting of
numerous superfluous move-to operations, only to place nodes. He is pleased to
learn that there are ways to add nodes in a more sensible manner.

Hagen现在明白了|node|操作是如何将节点添加到路径中的,但是只使用|\path|操作创建一个包含许多多余的移动操作的路径来放置节点似乎有点愚蠢。他很高兴地得知有更合理的方法来添加节点。

First, the |node| operation allows one to add |at (|\meta{coordinate}|)| in
order to directly specify where the node should be placed, sidestepping the
rule that nodes are placed on the last coordinate. Hagen can then write the
following:

首先,|node|操作允许在|at (|\meta{coordinate}|)|中直接指定节点应该放置的位置,绕过了节点放置在最后一个坐标上的规则。Hagen可以这样写:
%
\begin{codeexample}[]
\begin{tikzpicture}
  \path node at ( 0,2) [shape=circle,draw] {}
        node at ( 0,1) [shape=circle,draw] {}
        node at ( 0,0) [shape=circle,draw] {}
        node at ( 1,1) [shape=rectangle,draw] {}
        node at (-1,1) [shape=rectangle,draw] {};
\end{tikzpicture}
\end{codeexample}

Now Hagen is still left with a single empty path, but at least the path no
longer contains strange move-to's. It turns out that this can be improved
further: The |\node| command is an abbreviation for |\path node|, which allows
Hagen to write:

现在Hagen还是得到了一个空路径,但至少路径不再包含奇怪的移动操作了。事实证明,这还可以进一步改进:|\node|命令是|\path node|的缩写形式,这使得Hagen可以这样写:
%
\begin{codeexample}[]
\begin{tikzpicture}
  \node at ( 0,2) [circle,draw] {};
  \node at ( 0,1) [circle,draw] {};
  \node at ( 0,0) [circle,draw] {};
  \node at ( 1,1) [rectangle,draw] {};
  \node at (-1,1) [rectangle,draw] {};
\end{tikzpicture}
\end{codeexample}

Hagen likes this syntax much better than the previous one. Note that Hagen has
also omitted the |shape=| since, like |color=|, \tikzname\ allows you to omit
the |shape=| if there is no confusion.

% Hagen比起之前的语法更喜欢这种语法。请注意,Hagen还省略了|shape=|,因为像|color=|一样,如果没有混淆,\tikzname\ 允许省略|shape=|。

Hagen非常喜欢这种语法,比起之前的语法要好得多。请注意,Hagen还省略了|shape=|,因为像|color=|一样,如果没有混淆,\tikzname 允许省略|shape=|。



\subsection{Using Styles\\使用样式}

Feeling adventurous, Hagen tries to make the nodes look nicer. In the final
picture, the circles and rectangle should be filled with different colors,
resulting in the following code:

Hagen感到充满冒险精神,试图让节点看起来更漂亮。在最终的图片中,圆圈和矩形应该用不同的颜色填充,代码如下所示:

%
\begin{codeexample}[]
\begin{tikzpicture}[thick]
  \node at ( 0,2) [circle,draw=blue!50,fill=blue!20] {};
  \node at ( 0,1) [circle,draw=blue!50,fill=blue!20] {};
  \node at ( 0,0) [circle,draw=blue!50,fill=blue!20] {};
  \node at ( 1,1) [rectangle,draw=black!50,fill=black!20] {};
  \node at (-1,1) [rectangle,draw=black!50,fill=black!20] {};
\end{tikzpicture}
\end{codeexample}

While this looks nicer in the picture, the code starts to get a bit ugly.
Ideally, we would like our code to transport the message ``there are three
places and two transitions'' and not so much which filling colors should be
used.

虽然这在图片中看起来更漂亮,但代码变得有点丑陋。理想情况下,我们希望我们的代码传达的是"有三个位置和两个转换"这一信息,而不是使用哪些填充颜色。

To solve this problem, Hagen uses styles. He defines a style for places and
another style for transitions:

为了解决这个问题,Hagen使用样式。他为位置定义了一个样式,为转换定义了另一个样式:

%
\begin{codeexample}[]
\begin{tikzpicture}
  [place/.style={circle,draw=blue!50,fill=blue!20,thick},
   transition/.style={rectangle,draw=black!50,fill=black!20,thick}]
  \node at ( 0,2) [place] {};
  \node at ( 0,1) [place] {};
  \node at ( 0,0) [place] {};
  \node at ( 1,1) [transition] {};
  \node at (-1,1) [transition] {};
\end{tikzpicture}
\end{codeexample}


\subsection{Node Size\\节点大小}

Before Hagen starts naming and connecting the nodes, let us first make sure
that the nodes get their final appearance. They are still too small. Indeed,
Hagen wonders why they have any size at all, after all, the text is empty. The
reason is that \tikzname\ automatically adds some space around the text. The
amount is set using the option |inner sep|. So, to increase the size of the
nodes, Hagen could write:

在Hagen开始为节点命名和连接它们之前,我们先确保节点具有最终的外观。它们仍然太小了。实际上,Hagen想知道为什么它们有任何大小,毕竟文本是空的。原因是\tikzname\ 自动在文本周围添加一些空间。这个空间的大小由选项|inner sep|设置。因此,为了增加节点的大小,Hagen可以这样写:

%
\begin{codeexample}[]
\begin{tikzpicture}
  [inner sep=2mm,
   place/.style={circle,draw=blue!50,fill=blue!20,thick},
   transition/.style={rectangle,draw=black!50,fill=black!20,thick}]
  \node at ( 0,2) [place] {};
  \node at ( 0,1) [place] {};
  \node at ( 0,0) [place] {};
  \node at ( 1,1) [transition] {};
  \node at (-1,1) [transition] {};
\end{tikzpicture}
\end{codeexample}

However, this is not really the best way to achieve the desired effect. It is
much better to use the |minimum size| option instead. This option allows Hagen
to specify a minimum size that the node should have. If the node actually needs
to be bigger because of a longer text, it will be larger, but if the text is
empty, then the node will have |minimum size|. This option is also useful to
ensure that several nodes containing different amounts of text have the same
size. The options |minimum height| and |minimum width| allow you to specify the
minimum height and width independently.

So, what Hagen needs to do is to provide |minimum size| for the nodes. To be on
the safe side, he also sets |inner sep=0pt|. This ensures that the nodes will
really have size |minimum size| and not, for very small minimum sizes, the
minimal size necessary to encompass the automatically added space.
%
\begin{codeexample}[]
\begin{tikzpicture}
  [place/.style={circle,draw=blue!50,fill=blue!20,thick,
                 inner sep=0pt,minimum size=6mm},
   transition/.style={rectangle,draw=black!50,fill=black!20,thick,
                      inner sep=0pt,minimum size=4mm}]
  \node at ( 0,2) [place] {};
  \node at ( 0,1) [place] {};
  \node at ( 0,0) [place] {};
  \node at ( 1,1) [transition] {};
  \node at (-1,1) [transition] {};
\end{tikzpicture}
\end{codeexample}

\end{document}
\subsection{Naming Nodes}

Hagen's next aim is to connect the nodes using arrows. This seems like a tricky
business since the arrows should not start in the middle of the nodes, but
somewhere on the border and Hagen would very much like to avoid computing these
positions by hand.

Fortunately, \pgfname\ will perform all the necessary calculations for him.
However, he first has to assign names to the nodes so that he can reference
them later on.

There are two ways to name a node. The first is to use the |name=| option. The
second method is to write the desired name in parentheses after the |node|
operation. Hagen thinks that this second method seems strange, but he will soon
change his opinion.
%
\begin{codeexample}[setup code,hidden]
\tikzset{
    place/.style={circle,draw=blue!50,fill=blue!20,thick,
                  inner sep=0pt,minimum size=6mm},
    transition/.style={rectangle,draw=black!50,fill=black!20,thick,
                       inner sep=0pt,minimum size=4mm}
}
\end{codeexample}
%
\begin{codeexample}[]
% ... set up styles
\begin{tikzpicture}
  \node (waiting 1)      at ( 0,2) [place] {};
  \node (critical 1)     at ( 0,1) [place] {};
  \node (semaphore)      at ( 0,0) [place] {};
  \node (leave critical) at ( 1,1) [transition] {};
  \node (enter critical) at (-1,1) [transition] {};
\end{tikzpicture}
\end{codeexample}

Hagen is pleased to note that the names help in understanding the code. Names
for nodes can be pretty arbitrary, but they should not contain commas, periods,
parentheses, colons, and some other special characters. However, they can
contain underscores and hyphens.

The syntax for the |node| operation is quite liberal with respect to the order
in which node names, the |at| specifier, and the options must come. Indeed, you
can even have multiple option blocks between the |node| and the text in curly
braces, they accumulate. You can rearrange them arbitrarily and perhaps the
following might be preferable:
%
\begin{codeexample}[]
\begin{tikzpicture}
  \node[place]      (waiting 1)      at ( 0,2) {};
  \node[place]      (critical 1)     at ( 0,1) {};
  \node[place]      (semaphore)      at ( 0,0) {};
  \node[transition] (leave critical) at ( 1,1) {};
  \node[transition] (enter critical) at (-1,1) {};
\end{tikzpicture}
\end{codeexample}


\subsection{Placing Nodes Using Relative Placement}

Although Hagen still wishes to connect the nodes, he first wishes to address
another problem again: The placement of the nodes. Although he likes the |at|
syntax, in this particular case he would prefer placing the nodes ``relative to
each other''. So, Hagen would like to say that the |critical 1| node should be
below the |waiting 1| node, wherever the |waiting 1| node might be. There are
different ways of achieving this, but the nicest one in Hagen's case is the
|below| option:
%
\begin{codeexample}[preamble={\usetikzlibrary{positioning}}]
\begin{tikzpicture}
  \node[place]      (waiting)                            {};
  \node[place]      (critical)       [below=of waiting]  {};
  \node[place]      (semaphore)      [below=of critical] {};
  \node[transition] (leave critical) [right=of critical] {};
  \node[transition] (enter critical) [left=of critical]  {};
\end{tikzpicture}
\end{codeexample}

With the |positioning| library loaded, when an option like |below| is followed
by |of|, then the position of the node is shifted in such a manner that it is
placed at the distance |node distance| in the specified direction of the given
direction. The |node distance| is either the distance between the centers of
the nodes (when the |on grid| option is set to true) or the distance between
the borders (when the |on grid| option is set to false, which is the default).

Even though the above code has the same effect as the earlier code, Hagen can
pass it to his colleagues who will be able to just read and understand it,
perhaps without even having to see the picture.


\subsection{Adding Labels Next to Nodes}

Before we have a look at how Hagen can connect the nodes, let us add the
capacity ``$s \le 3$'' to the bottom node. For this, two approaches are
possible:
%
\begin{enumerate}
    \item Hagen can just add a new node above the |north| anchor of the
        |semaphore| node.
        %
\begin{codeexample}[preamble={\usetikzlibrary{positioning}}]
\begin{tikzpicture}
  \node[place]      (waiting)                            {};
  \node[place]      (critical)       [below=of waiting]  {};
  \node[place]      (semaphore)      [below=of critical] {};
  \node[transition] (leave critical) [right=of critical] {};
  \node[transition] (enter critical) [left=of critical]  {};

  \node [red,above] at (semaphore.north) {$s\le 3$};
\end{tikzpicture}
\end{codeexample}
        %
        This is a general approach that will ``always work''.

    \item Hagen can use the special |label| option. This option is given to a
        |node| and it causes \emph{another} node to be added next to the node
        where the option is given. Here is the idea: When we construct the
        |semaphore| node, we wish to indicate that we want another node with
        the capacity above it. For this, we use the option
        |label=above:$s\le 3$|. This option is interpreted as follows: We want
        a node above the |semaphore| node and this node should read ``$s \le
        3$''. Instead of |above| we could also use things like |below left|
        before the colon or a number like |60|.
        %
\begin{codeexample}[preamble={\usetikzlibrary{positioning}}]
\begin{tikzpicture}
  \node[place]      (waiting)                            {};
  \node[place]      (critical)       [below=of waiting]  {};
  \node[place]      (semaphore)      [below=of critical,
                                      label=above:$s\le3$] {};
  \node[transition] (leave critical) [right=of critical] {};
  \node[transition] (enter critical) [left=of critical]  {};
\end{tikzpicture}
\end{codeexample}
        %
        It is also possible to give multiple |label| options, this causes
        multiple labels to be drawn.
        %
\begin{codeexample}[]
\tikz
  \node [circle,draw,label=60:$60^\circ$,label=below:$-90^\circ$] {my circle};
\end{codeexample}
        %
        Hagen is not fully satisfied with the |label| option since the label
        is not red. To achieve this, he has two options: First, he can
        redefine the |every label| style. Second, he can add options to the
        label's node. These options are given following the |label=|, so he
        would write |label=[red]above:$s\le3$|. However, this does not quite
        work since \TeX\ thinks that the |]| closes the whole option list of
        the |semaphore| node. So, Hagen has to add braces and writes
        |label={[red]above:$s\le3$}|. Since this looks a bit ugly, Hagen
        decides to redefine the |every label| style.
        %
\begin{codeexample}[preamble={\usetikzlibrary{positioning}}]
\begin{tikzpicture}[every label/.style={red}]
  \node[place]      (waiting)                            {};
  \node[place]      (critical)       [below=of waiting]  {};
  \node[place]      (semaphore)      [below=of critical,
                                      label=above:$s\le3$] {};
  \node[transition] (leave critical) [right=of critical] {};
  \node[transition] (enter critical) [left=of critical]  {};
\end{tikzpicture}
\end{codeexample}
\end{enumerate}


\subsection{Connecting Nodes}

It is now high time to connect the nodes. Let us start with something simple,
namely with the straight line from |enter critical| to |critical|. We want this
line to start at the right side of |enter critical| and to end at the left side
of |critical|. For this, we can use the \emph{anchors} of the nodes. Every node
defines a whole bunch of anchors that lie on its border or inside it. For
example, the |center| anchor is at the center of the node, the |west| anchor is
on the left of the node, and so on. To access the coordinate of a node, we use
a coordinate that contains the node's name followed by a dot, followed by the
anchor's name:
%
\begin{codeexample}[preamble={\usetikzlibrary{positioning}}]
\begin{tikzpicture}
  \node[place]      (waiting)                            {};
  \node[place]      (critical)       [below=of waiting]  {};
  \node[place]      (semaphore)      [below=of critical] {};
  \node[transition] (leave critical) [right=of critical] {};
  \node[transition] (enter critical) [left=of critical]  {};
  \draw [->] (enter critical.east) -- (critical.west);
\end{tikzpicture}
\end{codeexample}

Next, let us tackle the curve from |waiting| to |enter critical|. This can be
specified using curves and controls:
%
\begin{codeexample}[preamble={\usetikzlibrary{positioning}}]
\begin{tikzpicture}
  \node[place]      (waiting)                            {};
  \node[place]      (critical)       [below=of waiting]  {};
  \node[place]      (semaphore)      [below=of critical] {};
  \node[transition] (leave critical) [right=of critical] {};
  \node[transition] (enter critical) [left=of critical]  {};
  \draw [->] (enter critical.east) -- (critical.west);
  \draw [->] (waiting.west) .. controls +(left:5mm) and +(up:5mm)
                            .. (enter critical.north);
\end{tikzpicture}
\end{codeexample}

Hagen sees how he can now add all his edges, but the whole process seems a but
awkward and not very flexible. Again, the code seems to obscure the structure
of the graphic rather than showing it.

So, let us start improving the code for the edges. First, Hagen can leave out
the anchors:
%
\begin{codeexample}[preamble={\usetikzlibrary{positioning}}]
\begin{tikzpicture}
  \node[place]      (waiting)                            {};
  \node[place]      (critical)       [below=of waiting]  {};
  \node[place]      (semaphore)      [below=of critical] {};
  \node[transition] (leave critical) [right=of critical] {};
  \node[transition] (enter critical) [left=of critical]  {};
  \draw [->] (enter critical) -- (critical);
  \draw [->] (waiting) .. controls +(left:8mm) and +(up:8mm)
                       .. (enter critical);
\end{tikzpicture}
\end{codeexample}

Hagen is a bit surprised that this works. After all, how did \tikzname\ know
that the line from |enter critical| to |critical| should actually start on the
borders? Whenever \tikzname\ encounters a whole node name as a ``coordinate'',
it tries to ``be smart'' about the anchor that it should choose for this node.
Depending on what happens next, \tikzname\ will choose an anchor that lies on
the border of the node on a line to the next coordinate or control point. The
exact rules are a bit complex, but the chosen point will usually be correct --
and when it is not, Hagen can still specify the desired anchor by hand.

Hagen would now like to simplify the curve operation somehow. It turns out that
this can be accomplished using a special path operation: the |to| operation.
This operation takes many options (you can even define new ones yourself). One
pair of options is useful for Hagen: The pair |in| and |out|. These options
take angles at which a curve should leave or reach the start or target
coordinates. Without these options, a straight line is drawn:
%
\begin{codeexample}[preamble={\usetikzlibrary{positioning}}]
\begin{tikzpicture}
  \node[place]      (waiting)                            {};
  \node[place]      (critical)       [below=of waiting]  {};
  \node[place]      (semaphore)      [below=of critical] {};
  \node[transition] (leave critical) [right=of critical] {};
  \node[transition] (enter critical) [left=of critical]  {};
  \draw [->] (enter critical) to                 (critical);
  \draw [->] (waiting)        to [out=180,in=90] (enter critical);
\end{tikzpicture}
\end{codeexample}

There is another option for the |to| operation, that is even better suited to
Hagen's problem: The |bend right| option. This option also takes an angle, but
this angle only specifies the angle by which the curve is bent to the right:
%
\begin{codeexample}[preamble={\usetikzlibrary{positioning}}]
\begin{tikzpicture}
  \node[place]      (waiting)                            {};
  \node[place]      (critical)       [below=of waiting]  {};
  \node[place]      (semaphore)      [below=of critical] {};
  \node[transition] (leave critical) [right=of critical] {};
  \node[transition] (enter critical) [left=of critical]  {};
  \draw [->] (enter critical) to                 (critical);
  \draw [->] (waiting)        to [bend right=45] (enter critical);
  \draw [->] (enter critical) to [bend right=45] (semaphore);
\end{tikzpicture}
\end{codeexample}

It is now time for Hagen to learn about yet another way of specifying edges:
Using the |edge| path operation. This operation is very similar to the |to|
operation, but there is one important difference: Like a node the edge
generated by the |edge| operation is not part of the main path, but is added
only later. This may not seem very important, but it has some nice
consequences. For example, every edge can have its own arrow tips and its own
color and so on and, still, all the edges can be given on the same path. This
allows Hagen to write the following:
%
\begin{codeexample}[preamble={\usetikzlibrary{positioning}}]
\begin{tikzpicture}
  \node[place]      (waiting)                            {};
  \node[place]      (critical)       [below=of waiting]  {};
  \node[place]      (semaphore)      [below=of critical] {};
  \node[transition] (leave critical) [right=of critical] {};
  \node[transition] (enter critical) [left=of critical]  {}
    edge [->]               (critical)
    edge [<-,bend left=45]  (waiting)
    edge [->,bend right=45] (semaphore);
\end{tikzpicture}
\end{codeexample}

Each |edge| caused a new path to be constructed, consisting of a |to| between
the node |enter critical| and the node following the |edge| command.

The finishing touch is to introduce two styles |pre| and |post| and to use the
|bend angle=45| option to set the bend angle once and for all:
%
\begin{codeexample}[preamble={\usetikzlibrary{arrows.meta,positioning}}]
% Styles place and transition as before
\begin{tikzpicture}
  [bend angle=45,
   pre/.style={<-,shorten <=1pt,>={Stealth[round]},semithick},
   post/.style={->,shorten >=1pt,>={Stealth[round]},semithick}]

  \node[place]      (waiting)                            {};
  \node[place]      (critical)       [below=of waiting]  {};
  \node[place]      (semaphore)      [below=of critical] {};

  \node[transition] (leave critical) [right=of critical] {}
    edge [pre]             (critical)
    edge [post,bend right] (waiting)
    edge [pre, bend left]  (semaphore);
  \node[transition] (enter critical) [left=of critical]  {}
    edge [post]            (critical)
    edge [pre, bend left]  (waiting)
    edge [post,bend right] (semaphore);
\end{tikzpicture}
\end{codeexample}


\subsection{Adding Labels Next to Lines}

The next thing that Hagen needs to add is the ``$2$'' at the arcs. For this
Hagen can use \tikzname's automatic node placement: By adding the option
|auto|, \tikzname\ will position nodes on curves and lines in such a way that
they are not on the curve but next to it. Adding |swap| will mirror the label
with respect to the line. Here is a general example:
%
% TODOsp: codeexamples: styles not needed here
\begin{codeexample}[]
\begin{tikzpicture}[auto,bend right]
  \node (a) at (0:1) {$0^\circ$};
  \node (b) at (120:1) {$120^\circ$};
  \node (c) at (240:1) {$240^\circ$};

  \draw (a) to node {1} node [swap] {1'} (b)
        (b) to node {2} node [swap] {2'} (c)
        (c) to node {3} node [swap] {3'} (a);
\end{tikzpicture}
\end{codeexample}

What is happening here? The nodes are given somehow inside the |to| operation!
When this is done, the node is placed on the middle of the curve or line
created by the |to| operation. The |auto| option then causes the node to be
moved in such a way that it does not lie on the curve, but next to it. In the
example we provide even two nodes on each |to| operation.

For Hagen that |auto| option is not really necessary since the two ``2'' labels
could also easily be placed ``by hand''. However, in a complicated plot with
numerous edges automatic placement can be a blessing.
%
\begin{codeexample}[
    preamble={\usetikzlibrary{arrows.meta,positioning}},
    pre={\tikzset{
    pre/.style={<-,shorten <=1pt,>={Stealth[round]},semithick},
    post/.style={->,shorten >=1pt,>={Stealth[round]},semithick},
}},
]
% Styles as before
\begin{tikzpicture}[bend angle=45]
  \node[place]      (waiting)                            {};
  \node[place]      (critical)       [below=of waiting]  {};
  \node[place]      (semaphore)      [below=of critical] {};

  \node[transition] (leave critical) [right=of critical] {}
    edge [pre]                                 (critical)
    edge [post,bend right] node[auto,swap] {2} (waiting)
    edge [pre, bend left]                      (semaphore);
  \node[transition] (enter critical) [left=of critical]  {}
    edge [post]                                (critical)
    edge [pre, bend left]                      (waiting)
    edge [post,bend right]                     (semaphore);
\end{tikzpicture}
\end{codeexample}
% TODOsp: codeexamples: styles and `positioning` are needed up to here


\subsection{Adding the Snaked Line and Multi-Line Text}

With the node mechanism Hagen can now easily create the two Petri nets. What he
is unsure of is how he can create the snaked line between the nets.

For this he can use a \emph{decoration}. To draw the snaked line, Hagen only
needs to set the two options |decoration=snake| and |decorate| on the path.
This causes all lines of the path to be replaced by snakes. It is also possible
to use snakes only in certain parts of a path, but Hagen will not need this.
%
\begin{codeexample}[preamble={\usetikzlibrary{decorations.pathmorphing}}]
\begin{tikzpicture}
  \draw [->,decorate,decoration=snake] (0,0) -- (2,0);
\end{tikzpicture}
\end{codeexample}

Well, that does not look quite right, yet. The problem is that the snake
happens to end exactly at the position where the arrow begins. Fortunately,
there is an option that helps here. Also, the snake should be a bit smaller,
which can be influenced by even more options.
%
\begin{codeexample}[preamble={\usetikzlibrary{decorations.pathmorphing}}]
\begin{tikzpicture}
  \draw [->,decorate,
     decoration={snake,amplitude=.4mm,segment length=2mm,post length=1mm}]
    (0,0) -- (3,0);
\end{tikzpicture}
\end{codeexample}

Now Hagen needs to add the text above the snake. This text is a bit challenging
since it is a multi-line text. Hagen has two options for this: First, he can
specify an |align=center| and then use the |\\| command to enforce the line
breaks at the desired positions.
%
\begin{codeexample}[preamble={\usetikzlibrary{decorations.pathmorphing}}]
\begin{tikzpicture}
  \draw [->,decorate,
      decoration={snake,amplitude=.4mm,segment length=2mm,post length=1mm}]
    (0,0) -- (3,0)
    node [above,align=center,midway]
    {
      replacement of\\
      the \textcolor{red}{capacity}\\
      by \textcolor{red}{two places}
    };
\end{tikzpicture}
\end{codeexample}

Instead of specifying the line breaks ``by hand'', Hagen can also specify a
width for the text and let \TeX\ perform the line breaking for him:
%
\begin{codeexample}[preamble={\usetikzlibrary{decorations.pathmorphing}}]
\begin{tikzpicture}
  \draw [->,decorate,
      decoration={snake,amplitude=.4mm,segment length=2mm,post length=1mm}]
    (0,0) -- (3,0)
    node [above,text width=3cm,align=center,midway]
    {
      replacement of the \textcolor{red}{capacity} by
      \textcolor{red}{two places}
    };
\end{tikzpicture}
\end{codeexample}


\subsection{Using Layers: The Background Rectangles}

Hagen still needs to add the background rectangles. These are a bit tricky:
Hagen would like to draw the rectangles \emph{after} the Petri nets are
finished. The reason is that only then can he conveniently refer to the
coordinates that make up the corners of the rectangle. If Hagen draws the
rectangle first, then he needs to know the exact size of the Petri net -- which
he does not.

The solution is to use \emph{layers}. When the |backgrounds| library is loaded,
Hagen can put parts of his picture inside a scope with the
|on background layer| option. Then this part of the picture becomes part of the
layer that is given as an argument to this environment. When the
|{tikzpicture}| environment ends, the layers are put on top of each other,
starting with the background layer. This causes everything drawn on the
background layer to be behind the main text.

The next tricky question is, how big should the rectangle be? Naturally, Hagen
can compute the size ``by hand'' or using some clever observations concerning
the $x$- and $y$-coordinates of the nodes, but it would be nicer to just have
\tikzname\ compute a rectangle into which all the nodes ``fit''. For this, the
|fit| library can be used. It defines the |fit| options, which, when given to a
node, causes the node to be resized and shifted such that it exactly covers all
the nodes and coordinates given as parameters to the |fit| option.
%
% TODOsp: codeexamples: redo/add styles starting from here
\begin{codeexample}[
    preamble={\usetikzlibrary{arrows.meta,backgrounds,fit,positioning}},
    pre={\tikzset{
    pre/.style={<-,shorten <=1pt,>={Stealth[round]},semithick},
    post/.style={->,shorten >=1pt,>={Stealth[round]},semithick},
}},
]
% Styles as before
\begin{tikzpicture}[bend angle=45]
  \node[place]      (waiting)                            {};
  \node[place]      (critical)       [below=of waiting]  {};
  \node[place]      (semaphore)      [below=of critical] {};

  \node[transition] (leave critical) [right=of critical] {}
    edge [pre]                                 (critical)
    edge [post,bend right] node[auto,swap] {2} (waiting)
    edge [pre, bend left]                      (semaphore);
  \node[transition] (enter critical) [left=of critical]  {}
    edge [post]                                (critical)
    edge [pre, bend left]                      (waiting)
    edge [post,bend right]                     (semaphore);

  \begin{scope}[on background layer]
    \node [fill=black!30,fit=(waiting) (critical) (semaphore)
             (leave critical) (enter critical)] {};
  \end{scope}
\end{tikzpicture}
\end{codeexample}


\subsection{The Complete Code}

Hagen has now finally put everything together. Only then does he learn that
there is already a library for drawing Petri nets! It turns out that this
library mainly provides the same definitions as Hagen did. For example, it
defines a |place| style in a similar way as Hagen did. Adjusting the code so
that it uses the library shortens Hagen code a bit, as shown in the following.

First, Hagen needs less style definitions, but he still needs to specify the
colors of places and transitions.
%
\begin{codeexample}[code only]
\begin{tikzpicture}
  [node distance=1.3cm,on grid,>={Stealth[round]},bend angle=45,auto,
   every place/.style=     {minimum size=6mm,thick,draw=blue!75,fill=blue!20},
   every transition/.style={thick,draw=black!75,fill=black!20},
   red place/.style=       {place,draw=red!75,fill=red!20},
   every label/.style=     {red}]
\end{codeexample}

Now comes the code for the nets:
%
\ifpgfmanualexternalize\tikzexternaldisable\fi
\begin{codeexample}[
    preamble={\usetikzlibrary{arrows.meta,petri,positioning}},
    pre={\tikzset{
    every place/.style={minimum size=6mm,thick,draw=blue!75,fill=blue!20},
    every transition/.style={thick,draw=black!75,fill=black!20},
    every label/.style={red},
    every picture/.style={on grid,node distance=1.3cm,>={Stealth[round]},bend angle=45,auto},
}%
\begin{tikzpicture}},
    post={\end{tikzpicture}},
]
   \node [place,tokens=1] (w1)                                    {};
   \node [place]          (c1) [below=of w1]                      {};
   \node [place]          (s)  [below=of c1,label=above:$s\le 3$] {};
   \node [place]          (c2) [below=of s]                       {};
   \node [place,tokens=1] (w2) [below=of c2]                      {};

   \node [transition] (e1) [left=of c1] {}
     edge [pre,bend left]                  (w1)
     edge [post,bend right]                (s)
     edge [post]                           (c1);
   \node [transition] (e2) [left=of c2] {}
     edge [pre,bend right]                 (w2)
     edge [post,bend left]                 (s)
     edge [post]                           (c2);
   \node [transition] (l1) [right=of c1] {}
     edge [pre]                            (c1)
     edge [pre,bend left]                  (s)
     edge [post,bend right] node[swap] {2} (w1);
   \node [transition] (l2) [right=of c2] {}
     edge [pre]                            (c2)
     edge [pre,bend right]                 (s)
     edge [post,bend left]  node {2}       (w2);
\end{codeexample}

\ifpgfmanualexternalize\tikzexternaldisable\fi
\begin{codeexample}[
    preamble={\usetikzlibrary{arrows.meta,petri,positioning}},
    pre={\tikzset{
    every place/.style={minimum size=6mm,thick,draw=blue!75,fill=blue!20},
    every transition/.style={thick,draw=black!75,fill=black!20},
    red place/.style=  {place,draw=red!75,fill=red!20},
    every label/.style={red},
    every picture/.style={on grid,node distance=1.3cm,>={Stealth[round]},bend angle=45,auto},
}%
\begin{tikzpicture}},
    post={\end{tikzpicture}},
]
  \begin{scope}[xshift=6cm]
    \node [place,tokens=1]     (w1')                            {};
    \node [place]              (c1') [below=of w1']             {};
    \node [red place]          (s1') [below=of c1',xshift=-5mm]
            [label=left:$s$]                                    {};
    \node [red place,tokens=3] (s2') [below=of c1',xshift=5mm]
            [label=right:$\bar s$]                              {};
    \node [place]              (c2') [below=of s1',xshift=5mm]  {};
    \node [place,tokens=1]     (w2') [below=of c2']             {};

    \node [transition] (e1') [left=of c1'] {}
      edge [pre,bend left]                  (w1')
      edge [post]                           (s1')
      edge [pre]                            (s2')
      edge [post]                           (c1');
    \node [transition] (e2') [left=of c2'] {}
      edge [pre,bend right]                 (w2')
      edge [post]                           (s1')
      edge [pre]                            (s2')
      edge [post]                           (c2');
    \node [transition] (l1') [right=of c1'] {}
      edge [pre]                            (c1')
      edge [pre]                            (s1')
      edge [post]                           (s2')
      edge [post,bend right] node[swap] {2} (w1');
    \node [transition] (l2') [right=of c2'] {}
      edge [pre]                            (c2')
      edge [pre]                            (s1')
      edge [post]                           (s2')
      edge [post,bend left]  node {2}       (w2');
  \end{scope}
\end{codeexample}

The code for the background and the snake is the following:
%
\begin{codeexample}[code only]
  \begin{scope}[on background layer]
    \node (r1) [fill=black!10,rounded corners,fit=(w1)(w2)(e1)(e2)(l1)(l2)] {};
    \node (r2) [fill=black!10,rounded corners,fit=(w1')(w2')(e1')(e2')(l1')(l2')] {};
  \end{scope}

  \draw [shorten >=1mm,->,thick,decorate,
         decoration={snake,amplitude=.4mm,segment length=2mm,
                     pre=moveto,pre length=1mm,post length=2mm}]
    (r1) -- (r2) node [above=1mm,midway,text width=3cm,align=center]
      {replacement of the \textcolor{red}{capacity} by \textcolor{red}{two places}};
\end{tikzpicture}
\end{codeexample}

% -----------------------------------------------------------------------------
% TODOsp: codeexamples: This is needed because -- unlike I thought --
%         `setup code is remembered also outside this file. Thus the changed
%         style of `place` and `transition` are "remembered" in
%         <pgfmanual-en-library-petri.tex>
\begin{codeexample}[setup code,hidden]
% from <tikzlibrarypetri.code.tex>
\tikzset{
    place/.style={circle,draw,inner sep=0pt,minimum size=5ex,every place},
    transition/.style={rectangle,draw,inner sep=0pt,minimum size=4mm,every transition},
}
\end{codeexample}
% -----------------------------------------------------------------------------
