

\subsection{Using Layers: The Background Rectangles\\使用图层:背景矩形}

Hagen still needs to add the background rectangles. These are a bit tricky:
Hagen would like to draw the rectangles \emph{after} the Petri nets are
finished. The reason is that only then can he conveniently refer to the
coordinates that make up the corners of the rectangle. If Hagen draws the
rectangle first, then he needs to know the exact size of the Petri net -- which
he does not.

Hagen 还需要添加背景矩形。这有点棘手:Hagen 希望在 Petri 网完成后绘制矩形。原因是只有在那时他才能方便地引用组成矩形角落的坐标。如果 Hagen 先绘制矩形,那么他需要知道 Petri 网的确切大小,而这是他不知道的。

The solution is to use \emph{layers}. When the |backgrounds| library is loaded,
Hagen can put parts of his picture inside a scope with the
|on background layer| option. Then this part of the picture becomes part of the
layer that is given as an argument to this environment. When the
|{tikzpicture}| environment ends, the layers are put on top of each other,
starting with the background layer. This causes everything drawn on the
background layer to be behind the main text.

解决方案是使用\emph{图层}。当加载 |backgrounds| 库时,Hagen 可以将图片的部分放在带有 |on background layer| 选项的作用域内。然后,这部分图片成为该环境的参数所给出的图层的一部分。当 |{tikzpicture}| 环境结束时,图层会叠加在一起,从背景图层开始。这使得绘制在背景图层上的所有内容都位于主文本之后。


The next tricky question is, how big should the rectangle be? Naturally, Hagen
can compute the size ``by hand'' or using some clever observations concerning
the $x$- and $y$-coordinates of the nodes, but it would be nicer to just have
\tikzname\ compute a rectangle into which all the nodes ``fit''. For this, the
|fit| library can be used. It defines the |fit| options, which, when given to a
node, causes the node to be resized and shifted such that it exactly covers all
the nodes and coordinates given as parameters to the |fit| option.

下一个棘手的问题是,矩形应该有多大?当然,Hagen 可以手动计算大小,或者利用关于节点的 $x$ 和 $y$ 坐标的一些巧妙观察来计算,但更好的方法是让 \tikzname\ 计算一个完全包含所有节点的矩形。为此,可以使用 |fit| 库。它定义了 |fit| 选项,当给定给节点时,会导致节点被调整大小和移动,以确切覆盖所有给定给 |fit| 选项的参数的节点和坐标。

%
% TODOsp: codeexamples: redo/add styles starting from here
\begin{codeexample}[
    preamble={\usetikzlibrary{arrows.meta,backgrounds,fit,positioning}},
    pre={\tikzset{
    pre/.style={<-,shorten <=1pt,>={Stealth[round]},semithick},
    post/.style={->,shorten >=1pt,>={Stealth[round]},semithick},
}},
]
% Styles as before
\begin{tikzpicture}[bend angle=45]
  \node[place]      (waiting)                            {};
  \node[place]      (critical)       [below=of waiting]  {};
  \node[place]      (semaphore)      [below=of critical] {};

  \node[transition] (leave critical) [right=of critical] {}
    edge [pre]                                 (critical)
    edge [post,bend right] node[auto,swap] {2} (waiting)
    edge [pre, bend left]                      (semaphore);
  \node[transition] (enter critical) [left=of critical]  {}
    edge [post]                                (critical)
    edge [pre, bend left]                      (waiting)
    edge [post,bend right]                     (semaphore);

  \begin{scope}[on background layer]
    \node [fill=black!30,fit=(waiting) (critical) (semaphore)
             (leave critical) (enter critical)] {};
  \end{scope}
\end{tikzpicture}
\end{codeexample}


\subsection{The Complete Code\\完整代码}

Hagen has now finally put everything together. Only then does he learn that
there is already a library for drawing Petri nets! It turns out that this
library mainly provides the same definitions as Hagen did. For example, it
defines a |place| style in a similar way as Hagen did. Adjusting the code so
that it uses the library shortens Hagen code a bit, as shown in the following.

Hagen 现在终于把所有东西都放在一起。然后他才知道已经有一个用于绘制 Petri 网的库!事实证明,该库主要提供与 Hagen 相同的定义。例如,它以类似于 Hagen 的方式定义了 |place| 样式。调整代码以使用该库可以稍微缩短 Hagen 的代码,如下所示。


First, Hagen needs less style definitions, but he still needs to specify the
colors of places and transitions.

首先,Hagen 需要较少的样式定义,但仍然需要指定位置和过渡的颜色。

%
\begin{codeexample}[code only]
\begin{tikzpicture}
  [node distance=1.3cm,on grid,>={Stealth[round]},bend angle=45,auto,
   every place/.style=     {minimum size=6mm,thick,draw=blue!75,fill=blue!20},
   every transition/.style={thick,draw=black!75,fill=black!20},
   red place/.style=       {place,draw=red!75,fill=red!20},
   every label/.style=     {red}]
\end{codeexample}

Now comes the code for the nets:

现在是网的代码:

%
\ifpgfmanualexternalize\tikzexternaldisable\fi
\begin{codeexample}[
    preamble={\usetikzlibrary{arrows.meta,petri,positioning}},
    pre={\tikzset{
    every place/.style={minimum size=6mm,thick,draw=blue!75,fill=blue!20},
    every transition/.style={thick,draw=black!75,fill=black!20},
    every label/.style={red},
    every picture/.style={on grid,node distance=1.3cm,>={Stealth[round]},bend angle=45,auto},
}%
\begin{tikzpicture}},
    post={\end{tikzpicture}},
]
   \node [place,tokens=1] (w1)                                    {};
   \node [place]          (c1) [below=of w1]                      {};
   \node [place]          (s)  [below=of c1,label=above:$s\le 3$] {};
   \node [place]          (c2) [below=of s]                       {};
   \node [place,tokens=1] (w2) [below=of c2]                      {};

   \node [transition] (e1) [left=of c1] {}
     edge [pre,bend left]                  (w1)
     edge [post,bend right]                (s)
     edge [post]                           (c1);
   \node [transition] (e2) [left=of c2] {}
     edge [pre,bend right]                 (w2)
     edge [post,bend left]                 (s)
     edge [post]                           (c2);
   \node [transition] (l1) [right=of c1] {}
     edge [pre]                            (c1)
     edge [pre,bend left]                  (s)
     edge [post,bend right] node[swap] {2} (w1);
   \node [transition] (l2) [right=of c2] {}
     edge [pre]                            (c2)
     edge [pre,bend right]                 (s)
     edge [post,bend left]  node {2}       (w2);
\end{codeexample}

\ifpgfmanualexternalize\tikzexternaldisable\fi
\begin{codeexample}[
    preamble={\usetikzlibrary{arrows.meta,petri,positioning}},
    pre={\tikzset{
    every place/.style={minimum size=6mm,thick,draw=blue!75,fill=blue!20},
    every transition/.style={thick,draw=black!75,fill=black!20},
    red place/.style=  {place,draw=red!75,fill=red!20},
    every label/.style={red},
    every picture/.style={on grid,node distance=1.3cm,>={Stealth[round]},bend angle=45,auto},
}%
\begin{tikzpicture}},
    post={\end{tikzpicture}},
]
  \begin{scope}[xshift=6cm]
    \node [place,tokens=1]     (w1')                            {};
    \node [place]              (c1') [below=of w1']             {};
    \node [red place]          (s1') [below=of c1',xshift=-5mm]
            [label=left:$s$]                                    {};
    \node [red place,tokens=3] (s2') [below=of c1',xshift=5mm]
            [label=right:$\bar s$]                              {};
    \node [place]              (c2') [below=of s1',xshift=5mm]  {};
    \node [place,tokens=1]     (w2') [below=of c2']             {};

    \node [transition] (e1') [left=of c1'] {}
      edge [pre,bend left]                  (w1')
      edge [post]                           (s1')
      edge [pre]                            (s2')
      edge [post]                           (c1');
    \node [transition] (e2') [left=of c2'] {}
      edge [pre,bend right]                 (w2')
      edge [post]                           (s1')
      edge [pre]                            (s2')
      edge [post]                           (c2');
    \node [transition] (l1') [right=of c1'] {}
      edge [pre]                            (c1')
      edge [pre]                            (s1')
      edge [post]                           (s2')
      edge [post,bend right] node[swap] {2} (w1');
    \node [transition] (l2') [right=of c2'] {}
      edge [pre]                            (c2')
      edge [pre]                            (s1')
      edge [post]                           (s2')
      edge [post,bend left]  node {2}       (w2');
  \end{scope}
\end{codeexample}

The code for the background and the snake is the following:

背景和蛇形线的代码如下:

%
\begin{codeexample}[code only]
  \begin{scope}[on background layer]
    \node (r1) [fill=black!10,rounded corners,fit=(w1)(w2)(e1)(e2)(l1)(l2)] {};
    \node (r2) [fill=black!10,rounded corners,fit=(w1')(w2')(e1')(e2')(l1')(l2')] {};
  \end{scope}

  \draw [shorten >=1mm,->,thick,decorate,
         decoration={snake,amplitude=.4mm,segment length=2mm,
                     pre=moveto,pre length=1mm,post length=2mm}]
    (r1) -- (r2) node [above=1mm,midway,text width=3cm,align=center]
      {replacement of the \textcolor{red}{capacity} by \textcolor{red}{two places}};
\end{tikzpicture}
\end{codeexample}

% -----------------------------------------------------------------------------
% TODOsp: codeexamples: This is needed because -- unlike I thought --
%         `setup code is remembered also outside this file. Thus the changed
%         style of `place` and `transition` are "remembered" in
%         <pgfmanual-en-library-petri.tex>
\begin{codeexample}[setup code,hidden]
% from <tikzlibrarypetri.code.tex>
\tikzset{
    place/.style={circle,draw,inner sep=0pt,minimum size=5ex,every place},
    transition/.style={rectangle,draw,inner sep=0pt,minimum size=4mm,every transition},
}
\end{codeexample}
% -----------------------------------------------------------------------------
