% % Copyright 2019 by Till Tantau
% %
% % This file may be distributed and/or modified
% %
% % 1. under the LaTeX Project Public License and/or
% % 2. under the GNU Free Documentation License.
% %
% % See the file doc/generic/pgf/licenses/LICENSE for more details.


% \section{Tutorial: A Picture for Karl's Students\\教程:为卡尔的学生绘制图片}

% This tutorial is intended for new users of \tikzname. It does not give an
% exhaustive account of all the features of \tikzname, just of those that you are
% likely to use right away.

% 本教程适用于初次使用 \tikzname 的用户。它并不详尽地介绍 \tikzname 的所有功能,只介绍了你可能会立即使用的部分。

% Karl is a math and chemistry high-school teacher. He used to create the
% graphics in his worksheets and exams using \LaTeX's |{picture}| environment.
% While the results were acceptable, creating the graphics often turned out to be
% a lengthy process. Also, there tended to be problems with lines having slightly
% wrong angles and circles also seemed to be hard to get right. Naturally, his
% students could not care less whether the lines had the exact right angles and
% they find Karl's exams too difficult no matter how nicely they were drawn. But
% Karl was never entirely satisfied with the result.

% 卡尔是一位数学和化学高中教师。他过去使用 \LaTeX 的 |{picture}| 环境在他的教学材料和考试中创建图形。虽然结果还可以接受,但创建图形往往是一个耗时的过程。而且,直线的角度可能稍有偏差,画圆也似乎很难做到完美。自然地,他的学生并不关心线条是否具有准确的角度,无论图形画得多好看,他们都觉得卡尔的考试太难了。但是卡尔对结果从未完全满意。

% Karl's son, who was even less satisfied with the results (he did not have to
% take the exams, after all), told Karl that he might wish to try out a new
% package for creating graphics. A bit confusingly, this package seems to have
% two names: First, Karl had to download and install a package called \pgfname.
% Then it turns out that inside this package there is another package called
% \tikzname, which is supposed to stand for ``\tikzname\ ist \emph{kein}
% Zeichenprogramm''. Karl finds this all a bit strange and \tikzname\ seems to
% indicate that the package does not do what he needs. However, having used
% \textsc{gnu} software for quite some time and ``\textsc{gnu} not being Unix'',
% there seems to be hope yet. His son assures him that \tikzname's name is
% intended to warn people that \tikzname\ is not a program that you can use to
% draw graphics with your mouse or tablet. Rather, it is more like a ``graphics
% language''.

% 卡尔的儿子对结果更不满意(毕竟他不必参加考试),他告诉卡尔可以尝试一个新的用于创建图形的软件包。有点令人困惑的是,这个软件包似乎有两个名称:首先,卡尔需要下载和安装一个名为 \pgfname 的软件包。然后发现,在这个软件包中有另一个名为 \tikzname 的软件包,它的缩写意味着\tikzname\ 不是绘图程序''。卡尔觉得这一切有点奇怪,而且 \tikzname\ 似乎表明这个软件包并不能满足他的需求。然而,他使用 \textsc{gnu} 软件已经有一段时间了,而 ``\textsc{gnu} 不是 Unix'',这似乎还有希望。他的儿子向他保证,\tikzname 的名字旨在告诫人们,\tikzname 不是一个可以用鼠标或平板电脑绘制图形的程序。相反,它更像是一种``图形语言''。



% \subsection{Problem Statement\\问题陈述}

% Karl wants to put a graphic on the next worksheet for his students. He is
% currently teaching his students about sine and cosine. What he would like to
% have is something that looks like this (ideally):

% Karl想在下一张工作表上为他的学生放置一个图形。他目前正在教学生正弦和余弦。他希望有一个看起来像这样的东西(理想情况下):

% % %
% % \noindent
% % \begin{tikzpicture}
% %   [scale=3,line cap=round,
% %    % Styles
% %    axes/.style=,
% %    important line/.style={very thick},
% %    information text/.style={rounded corners,fill=red!10,inner sep=1ex}]

% %   % Local definitions
% %   \def\costhirty{0.8660256}

% %   % Colors
% %   \colorlet{anglecolor}{green!50!black}
% %   \colorlet{sincolor}{red}
% %   \colorlet{tancolor}{orange!80!black}
% %   \colorlet{coscolor}{blue}

% %   % The graphic
% %   \draw[help lines,step=0.5cm] (-1.4,-1.4) grid (1.4,1.4);

% %   \draw (0,0) circle [radius=1cm];

% %   \begin{scope}[axes]
% %     \draw[->] (-1.5,0) -- (1.5,0) node[right] {$x$};
% %     \draw[->] (0,-1.5) -- (0,1.5) node[above] {$y$};

% %     \foreach \x/\xtext in {-1, -.5/-\frac{1}{2}, 1}
% %       \draw[xshift=\x cm] (0pt,1pt) -- (0pt,-1pt) node[below,fill=white] {$\xtext$};

% %     \foreach \y/\ytext in {-1, -.5/-\frac{1}{2}, .5/\frac{1}{2}, 1}
% %       \draw[yshift=\y cm] (1pt,0pt) -- (-1pt,0pt) node[left,fill=white] {$\ytext$};
% %   \end{scope}

% %   \filldraw[fill=green!20,draw=anglecolor] (0,0) -- (3mm,0pt) arc(0:30:3mm);
% %   \draw (15:2mm) node[anglecolor] {$\alpha$};

% %   \draw[important line,sincolor]
% %     (30:1cm) -- node[left=1pt,fill=white] {$\sin \alpha$} +(0,-.5);

% %   \draw[important line,coscolor]
% %     (0,0) -- node[below=2pt,fill=white] {$\cos \alpha$} (\costhirty,0);

% %   \draw[important line,tancolor] (1,0) --
% %     node [right=1pt,fill=white]
% %     {
% %       $\displaystyle \tan \alpha \color{black}=
% %       \frac{{\color{sincolor}\sin \alpha}}{\color{coscolor}\cos \alpha}$
% %     } (intersection of 0,0--30:1cm and 1,0--1,1) coordinate (t);

% %   \draw (0,0) -- (t);

% %   \draw[xshift=1.85cm] node [right,text width=6cm,information text]
% %     {
% %       The {\color{anglecolor} angle $\alpha$} is $30^\circ$ in the
% %       example ($\pi/6$ in radians). The {\color{sincolor}sine of
% %         $\alpha$}, which is the height of the red line, is
% %       \[
% %       {\color{sincolor} \sin \alpha} = 1/2.
% %       \]
% %       By the Theorem of Pythagoras we have ${\color{coscolor}\cos^2 \alpha} +
% %       {\color{sincolor}\sin^2\alpha} =1$. Thus the length of the blue
% %       line, which is the {\color{coscolor}cosine of $\alpha$}, must be
% %       \[
% %       {\color{coscolor}\cos\alpha} = \sqrt{1 - 1/4} = \textstyle
% %       \frac{1}{2} \sqrt 3.
% %       \]%
% %       This shows that {\color{tancolor}$\tan \alpha$}, which is the
% %       height of the orange line, is
% %       \[
% %       {\color{tancolor}\tan\alpha} = \frac{{\color{sincolor}\sin
% %           \alpha}}{\color{coscolor}\cos \alpha} = 1/\sqrt 3.
% %       \]%
% %     };
% % \end{tikzpicture}

% %
% \noindent
% \begin{tikzpicture}%用于创建TikZ图形的代码。
%   [scale=3,line cap=round,
%    % Styles 定义了一些自定义样式,可以在整个图形中使用。
%    axes/.style=,
%    important line/.style={very thick},
%    information text/.style={rounded corners,fill=red!10,inner sep=1ex}]

%   % Local definitions 
%   \def\costhirty{0.8660256}

%   % Colors
%   \colorlet{anglecolor}{green!50!black}
%   \colorlet{sincolor}{red}
%   \colorlet{tancolor}{orange!80!black}
%   \colorlet{coscolor}{blue}

%   % The graphic 绘制了一个网格,使用了帮助线样式,并且每0.5cm有一条线。
%   \draw[help lines,step=0.5cm] (-1.4,-1.4) grid (1.4,1.4);

%   \draw (0,0) circle [radius=1cm];%绘制了一个以原点(0,0)为中心,半径为1cm的圆。

%   \begin{scope}[axes]
%     \draw[->] (-1.5,0) -- (1.5,0) node[right] {$x$};
%     \draw[->] (0,-1.5) -- (0,1.5) node[above] {$y$};%绘制了x和y坐标轴。

%     \foreach \x/\xtext in {-1, -.5/-\frac{1}{2}, 1}
%       \draw[xshift=\x cm] (0pt,1pt) -- (0pt,-1pt) node[below,fill=white] {$\xtext$};

%     \foreach \y/\ytext in {-1, -.5/-\frac{1}{2}, .5/\frac{1}{2}, 1}
%       \draw[yshift=\y cm] (1pt,0pt) -- (-1pt,0pt) node[left,fill=white] {$\ytext$};
%   \end{scope}

%   \filldraw[fill=green!20,draw=anglecolor] (0,0) -- (3mm,0pt) arc(0:30:3mm);%绘制了一个表示角度α的弧,并用绿色填充了它。
%   \draw (15:2mm) node[anglecolor] {$\alpha$};%绘制一个点,并在该点的位置上标记角度符号α。

%   \draw[important line,sincolor]
%     (30:1cm) -- node[left=1pt,fill=white] {$\sin \alpha$} +(0,-.5);%绘制一条从原点开始的线段,表示正弦函数的值。这条线段位于角度α的位置,长度为1cm,并且标有$\sin \alpha$的标签。

%   \draw[important line,coscolor]
%     (0,0) -- node[below=2pt,fill=white] {$\cos \alpha$} (\costhirty,0);%绘制一条水平线段,表示余弦函数的值。这条线段的长度为$\cos 30^\circ$,并且标有$\cos \alpha$的标签。

%   %绘制一条从点(1,0)开始的线段,表示正切函数的值。这条线段的斜率等于$\frac{{\sin \alpha}}{{\cos \alpha}}$,并且标有$\tan \alpha$的标签。该线段与前面绘制的两条线段相交,交点的坐标被命名为(t)。
%   \draw[important line,tancolor] (1,0) --
%     node [right=1pt,fill=white]
%     {
%       $\displaystyle \tan \alpha \color{black}=
%       \frac{{\color{sincolor}\sin \alpha}}{\color{coscolor}\cos \alpha}$
%     } (intersection of 0,0--30:1cm and 1,0--1,1) coordinate (t);

%   \draw (0,0) -- (t);%绘制一条连接原点和点(t)的线段。

%   %在指定的位置绘制一个文本框,用于提供关于三角函数图形的说明。文本框右侧有一条竖线,并包含了一些文本和数学公式。
%   \draw[xshift=1.85cm] node [right,text width=6cm,information text]
%     {
%       The {\color{anglecolor} angle $\alpha$} is $30^\circ$ in the
%       example ($\pi/6$ in radians). The {\color{sincolor}sine of
%         $\alpha$}, which is the height of the red line, is
      
%       这个{\color{anglecolor} 角度 $\alpha$}在例子中是 $30^\circ$(弧度为 $\pi/6$)。{\color{sincolor} $\alpha$ 的正弦},也就是红线的高度,是

%       \[
%       {\color{sincolor} \sin \alpha} = 1/2.
%       \]
%       By the Theorem of Pythagoras we have ${\color{coscolor}\cos^2 \alpha} +
%       {\color{sincolor}\sin^2\alpha} =1$. Thus the length of the blue
%       line, which is the {\color{coscolor}cosine of $\alpha$}, must be

%       根据勾股定理,我们有 ${\color{coscolor}\cos^2 \alpha} +
%       {\color{sincolor}\sin^2\alpha} =1$。因此,蓝线的长度,也就是{\color{coscolor} $\alpha$ 的余弦},必须是

%       \[
%       {\color{coscolor}\cos\alpha} = \sqrt{1 - 1/4} = \textstyle
%       \frac{1}{2} \sqrt 3.
%       \]%
%       This shows that {\color{tancolor}$\tan \alpha$}, which is the
%       height of the orange line, is

%       这表明{\color{tancolor} $\alpha$ 的正切},也就是橙线的高度,是
%       \[
%       {\color{tancolor}\tan\alpha} = \frac{{\color{sincolor}\sin
%           \alpha}}{\color{coscolor}\cos \alpha} = 1/\sqrt 3.
%       \]%
%     };
% \end{tikzpicture}


% \subsection{Setting up the Environment\\设置环境}

% In \tikzname, to draw a picture, at the start of the picture you need to tell
% \TeX\ or \LaTeX\ that you want to start a picture. In \LaTeX\ this is done
% using the environment |{tikzpicture}|, in plain \TeX\ you just use
% |\tikzpicture| to start the picture and |\endtikzpicture| to end it.

% 在\tikzname 中,要绘制一个图形,你需要在图形开始时告诉\TeX\ 或\LaTeX\ 你想要开始一个图形。在\LaTeX\ 中,使用环境|{tikzpicture}|来完成这个操作,在plain \TeX\ 中,你只需使用|\tikzpicture|来开始图形,使用|\endtikzpicture|来结束图形。

% \subsubsection{Setting up the Environment in \LaTeX\\在 \LaTeX 中设置环境}

% Karl, being a \LaTeX\ user, thus sets up his file as follows:
% %

% 作为一个 \LaTeX 用户,Karl 设置他的文件如下:

% \begin{codeexample}[code only]
% \documentclass{article} % say
% \usepackage{tikz}
% \begin{document}
% We are working on
% \begin{tikzpicture}
%   \draw (-1.5,0) -- (1.5,0);
%   \draw (0,-1.5) -- (0,1.5);
% \end{tikzpicture}.
% \end{document}
% \end{codeexample}

% When executed, that is, run via |pdflatex| or via |latex| followed by |dvips|,
% the resulting will contain something that looks like this:
% %

% 当执行该代码,即通过 |pdflatex| 或 |latex| 后跟 |dvips| 运行时,结果将会包含以下内容:

% \begin{codeexample}[width=7cm]
% We are working on
% \begin{tikzpicture}
%   \draw (-1.5,0) -- (1.5,0);
%   \draw (0,-1.5) -- (0,1.5);
% \end{tikzpicture}.
% \end{codeexample}

% Admittedly, not quite the whole picture, yet, but we do have the axes
% established. Well, not quite, but we have the lines that make up the axes
% drawn. Karl suddenly has a sinking feeling that the picture is still some way
% off.

% 诚然,这还不是完整的图像,但我们已经建立了坐标轴。嗯,并不完全准确,但我们已经绘制了构成坐标轴的线段。Karl 突然有一种不安的感觉,认为图像还有一段路要走。



% Let's have a more detailed look at the code. First, the package |tikz| is
% loaded. This package is a so-called ``frontend'' to the basic \pgfname\ system.
% The basic layer, which is also described in this manual, is somewhat more,
% well, basic and thus harder to use. The frontend makes things easier by
% providing a simpler syntax.

% 让我们更详细地看一下代码。首先,加载了 |tikz| 宏包。该宏包是基本 \pgfname\ 系统的所谓“前端”。基本层在本手册中也有所描述,它更基础一些,因此使用起来更困难。前端通过提供更简单的语法使事情更容易。



% Inside the environment there are two |\draw| commands. They mean: ``The path,
% which is specified following the command up to the semicolon, should be
% drawn.'' The first path is specified as |(-1.5,0) -- (0,1.5)|, which means ``a
% straight line from the point at position $(-1.5,0)$ to the point at position
% $(0,1.5)$''. Here, the positions are specified within a special coordinate
% system in which, initially, one unit is 1cm.

% 在环境内部有两个 |\draw| 命令。它们的意思是:“应该绘制路径,路径在命令后面到分号之前指定。”第一个路径被指定为 |(-1.5,0) -- (0,1.5)|,意思是“从位置 $(-1.5,0)$ 到位置 $(0,1.5)$ 绘制一条直线”。这里,位置在一个特殊的坐标系中指定,初始情况下,一个单位等于 1cm。



% Karl is quite pleased to note that the environment automatically reserves
% enough space to encompass the picture.

% Karl 很高兴地注意到环境会自动保留足够的空间来容纳图片。



% \subsubsection{Setting up the Environment in Plain \TeX%
% \\在 Plain \TeX 中设置环境
% }

% Karl's wife Gerda, who also happens to be a math teacher, is not a \LaTeX\
% user, but uses plain \TeX\ since she prefers to do things ``the old way''. She
% can also use \tikzname. Instead of |\usepackage{tikz}| she has to write
% |\input tikz.tex| and instead of |\begin{tikzpicture}| she writes
% |\tikzpicture| and instead of |\end{tikzpicture}| she writes |\endtikzpicture|.

% Karl 的妻子 Gerda 也是一名数学教师,她不使用 \LaTeX,而是使用 Plain \TeX,因为她更喜欢“老派”的方式。她也可以使用 \tikzname。她需要将 |\usepackage{tikz}| 替换为 |\input tikz.tex|,将 |\begin{tikzpicture}| 替换为 |\tikzpicture|,将 |\end{tikzpicture}| 替换为 |\endtikzpicture|。



% Thus, she would use:

% 因此,她将使用以下代码:

% %
% \begin{codeexample}[code only]
% %% Plain TeX file
% \input tikz.tex
% \baselineskip=12pt
% \hsize=6.3truein
% \vsize=8.7truein
% We are working on
% \tikzpicture
%   \draw (-1.5,0) -- (1.5,0);
%   \draw (0,-1.5) -- (0,1.5);
% \endtikzpicture.
% \bye
% \end{codeexample}

% Gerda can typeset this file using either |pdftex| or |tex| together with
% |dvips|. \tikzname\ will automatically discern which driver she is using. If
% she wishes to use |dvipdfm| together with |tex|, she either needs to modify the
% file |pgf.cfg| or can write |\def\pgfsysdriver{pgfsys-dvipdfm.def}| somewhere
% \emph{before} she inputs |tikz.tex| or |pgf.tex|.

% Gerda 可以使用 |pdftex| 或 |tex| 与 |dvips| 来排版此文件。\tikzname 会自动判断她使用的是哪个驱动程序。如果她希望使用 |dvipdfm| 与 |tex| 一起使用,她需要修改文件 |pgf.cfg|,或者可以在输入 |tikz.tex| 或 |pgf.tex| 之前的某个地方写上 |\def\pgfsysdriver{pgfsys-dvipdfm.def}|。



% \subsubsection{Setting up the Environment in Con\TeX t\\在 Con\TeX t 中设置环境}

% Karl's uncle Hans uses Con\TeX t. Like Gerda, Hans can also use \tikzname.
% Instead of |\usepackage{tikz}| he says |\usemodule[tikz]|. Instead of
% |\begin{tikzpicture}| he writes |\starttikzpicture| and  instead of
% |\end{tikzpicture}| he writes |\stoptikzpicture|.

% % Karl 的叔叔 Hans 使用 Con\subsubsection{在 Con\TeX t 中设置环境}

% Karl 的叔叔 Hans 使用 Con\TeX t。和 Gerda 一样,Hans 也可以使用 \tikzname。他需要将 |\usepackage{tikz}| 替换为 |\usemodule[tikz]|。将 |\begin{tikzpicture}| 替换为 |\starttikzpicture|,将 |\end{tikzpicture}| 替换为 |\stoptikzpicture|。



% His version of the example looks like this:
% %

% 他的示例代码如下:

% \begin{codeexample}[code only]
% %% ConTeXt file
% \usemodule[tikz]

% \starttext
%   We are working on
%   \starttikzpicture
%     \draw (-1.5,0) -- (1.5,0);
%     \draw (0,-1.5) -- (0,1.5);
%   \stoptikzpicture.
% \stoptext
% \end{codeexample}

% Hans will now typeset this file in the usual way using |texexec| or |context|.


% Hans 可以使用 |texexec| 或 |context| 来排版这个文件。



% \subsection{Straight Path Construction\\直线路径构建}

% The basic building block of all pictures in \tikzname\ is the path. A
% \emph{path} is a series of straight lines and curves that are connected (that
% is not the whole picture, but let us ignore the complications for the moment).
% You start a path by specifying the coordinates of the start position as a point
% in round brackets, as in |(0,0)|. This is followed by a series of ``path
% extension operations''. The simplest is |--|, which we used already. It must be
% followed by another coordinate and it extends the path in a straight line to
% this new position. For example, if we were to turn the two paths of the axes
% into one path, the following would result:
% %

% 在\tikzname 中,所有图片的基本构建块是路径。一个\emph{路径}是由连接的直线和曲线组成的序列(这并不构成整个图片,但我们暂且忽略这些复杂性)。你可以通过将起始位置的坐标作为圆括号中的点来开始一个路径,例如 |(0,0)|。接下来是一系列的“路径扩展操作”。最简单的操作是 |--|,我们已经使用过了。它必须后跟另一个坐标,并将路径以直线延伸到新的位置。例如,如果我们将坐标轴的两条路径合并为一条路径,结果如下:


% \begin{codeexample}[]
% \tikz \draw (-1.5,0) -- (1.5,0) -- (0,-1.5) -- (0,1.5);
% \end{codeexample}

% Karl is a bit confused by the fact that there is no |{tikzpicture}|
% environment, here. Instead, the little command |\tikz| is used. This command
% either takes one argument (starting with an opening brace as in
% |\tikz{\draw (0,0) -- (1.5,0)}|, which yields \tikz{\draw (0,0) --(1.5,0);}) or
% collects everything up to the next semicolon and puts it inside a
% |{tikzpicture}| environment. As a rule of thumb, all \tikzname\ graphic drawing
% commands must occur as an argument of |\tikz| or inside a |{tikzpicture}|
% environment. Fortunately, the command |\draw| will only be defined inside this
% environment, so there is little chance that you will accidentally do something
% wrong here.

% 卡尔有点困惑,因为这里没有 |{tikzpicture}| 环境。取而代之,使用了命令 |\tikz|。这个命令可以接受一个参数(以打开的大括号开始,比如 |\tikz{\draw (0,0) -- (1.5,0)}|,它会生成 \tikz{\draw (0,0) --(1.5,0);})或者收集直到下一个分号的所有内容,并将其放入 |{tikzpicture}| 环境中。作为经验法则,所有的 \tikzname 图形绘制命令必须作为 |\tikz| 的参数或者在 |{tikzpicture}| 环境内部。幸运的是,命令 |\draw| 只会在此环境内部定义,所以你很少会在这里意外地做错事。




% \subsection{Curved Path Construction\\曲线路径构建}

% The next thing Karl wants to do is to draw the circle. For this, straight lines
% obviously will not do. Instead, we need some way to draw curves. For this,
% \tikzname\ provides a special syntax. One or two ``control points'' are needed.
% The math behind them is not quite trivial, but here is the basic idea: Suppose
% you are at point $x$ and the first control point is $y$. Then the curve will
% start ``going in the direction of~$y$ at~$x$'', that is, the tangent of the
% curve at $x$ will point toward~$y$. Next, suppose the curve should end at $z$
% and the second support point is $w$. Then the curve will, indeed, end at $z$
% and the tangent of the curve at point $z$ will go through $w$.

% 卡尔接下来想要做的是绘制圆。显然,直线无法实现这一点。相反,我们需要一种方法来绘制曲线。为此,\tikzname 提供了一种特殊的语法。需要一个或两个“控制点”。它们背后的数学并不是非常复杂,但基本思想是:假设你在点 $x$,第一个控制点是 $y$。那么曲线将在“以 $x$ 为中心向 $y$ 的方向开始”,也就是说,曲线在点 $x$ 的切线将指向 $y$。接下来,假设曲线应该在 $z$ 结束,第二个支撑点是 $w$。那么曲线确实会在 $z$ 结束,并且曲线在点 $z$ 的切线将经过 $w$。

% Here is an example (the control points have been added for clarity):

% 这是一个示例(为了清晰起见,添加了控制点):
% %
% \begin{codeexample}[]
% \begin{tikzpicture}
%   \filldraw [gray] (0,0) circle [radius=2pt]
%                    (1,1) circle [radius=2pt]
%                    (2,1) circle [radius=2pt]
%                    (2,0) circle [radius=2pt];
%   \draw (0,0) .. controls (1,1) and (2,1) .. (2,0);
% \end{tikzpicture}
% \end{codeexample}

% The general syntax for extending a path in a ``curved'' way is |.. controls|
% \meta{first control point} |and| \meta{second control point} |..|
% \meta{end point}. You can leave out the |and| \meta{second control point},
% which causes the first one to be used twice.

% 通过使用 |.. controls| \meta{第一个控制点} |and| \meta{第二个控制点} |..| \meta{结束点} 的一般语法,可以以“曲线”的方式扩展路径。可以省略 |and| \meta{第二个控制点},这样将使用第一个控制点两次。

% % 在LaTeX中,这个命令是用于在TikZ库中绘制一条贝塞尔曲线。TikZ是一个非常强大的工具,用于创建矢量图形。

% % \draw (0,0) .. controls (1,1) and (2,1) .. (2,0); 这行代码的意思是从点 (0,0) 到点 (2,0) 绘制一条贝塞尔曲线,其中 (1,1) 和 (2,1) 是控制点。

% % 下面是对这个命令的详细解读:

% % \draw: TikZ中的一个命令,它指示LaTeX开始绘制一条线或曲线。

% % (0,0): 这是你要开始绘制的曲线的起始点,坐标是 (0,0)。

% % .. controls (1,1) and (2,1) ..: 这告诉LaTeX你要绘制的是一条贝塞尔曲线,而不是直线。关键词 controls 后面的两个点 (1,1) 和 (2,1) 是控制曲线形状的控制点。

% % (2,0): 这是你要结束绘制的曲线的终点,坐标是 (2,0)。

% % -----

% % 贝塞尔曲线是一种数学曲线,用于描述平滑的曲线路径。它由数个控制点确定,通过这些控制点的位置和权重,可以定义曲线的形状。

% % 贝塞尔曲线最常见的形式是二次贝塞尔曲线和三次贝塞尔曲线。二次贝塞尔曲线由三个控制点确定,分别为起始点、终点和一个中间点。三次贝塞尔曲线由四个控制点确定,分别为起始点、终点和两个中间点。

% % 贝塞尔曲线的形状由控制点之间的插值算法确定。在二次贝塞尔曲线中,起始点和终点是曲线的端点,而中间点则会影响曲线的弯曲程度和方向。在三次贝塞尔曲线中,除了起始点和终点外,两个中间点的位置和权重也会影响曲线的形状。

% % 贝塞尔曲线具有平滑性和局部控制性的特点。通过调整控制点的位置和权重,可以创建各种形状的曲线,包括弧线、圆弧、S形曲线等。贝塞尔曲线在计算机图形学、计算机辅助设计(CAD)、动画和游戏开发等领域得到广泛应用。

% % 在绘图软件和绘图库中,如Adobe Illustrator、Inkscape和TikZ等,贝塞尔曲线通常由控制点和曲线类型(二次或三次)来定义。通过调整控制点的位置,可以实时预览和修改曲线的形状,从而满足具体的设计需求。

% % ---
% % 贝塞尔曲线(Bézier curve)是计算机图形学中常用的一种参数曲线。其主要思想是使用控制点来approximate一个曲线。

% % 贝塞尔曲线的一般形式为:

% % B(t) = ∑_{i=0}^n B_i^n(t) p_i

% % 其中p_i是控制点,B_i^n(t)是伯恩斯坦多项式(Bernstein polynomial)。

% % 贝塞尔曲线具有以下几个特点:

% % 通过移动控制点,可以方便地改变曲线的形状。

% % 曲线必然通过第一个和最后一个控制点。

% % 曲线在控制点附近的形状主要由相邻的几个控制点决定。

% % 随着控制点数量的增加,贝塞尔曲线可以逼近任意形状。

% % 贝塞尔曲线是可微的并且连续。

% % 贝塞尔曲线计算简单,适合在计算机上实现。

% % 贝塞尔曲线可以方便地进行仿射变换。

% % 贝塞尔曲线在许多图形软件中都有应用,如Adobe Illustrator和Inkscape等。它为设计师提供了一个简单直观的方法来绘制平滑曲线。

% % --- bard

% % 贝塞尔曲线是数学中一种用于描述平面曲线的曲线。它由一系列控制点定义,控制点是曲线沿其路径的特定位置。贝塞尔曲线可以用来生成各种各样的曲线,包括直线、圆弧、抛物线等。

% % 贝塞尔曲线的定义如下:

% % ```
% % B(t) = (1 - t)^3 * P_0 + 3t(1 - t)^2 * P_1 + 3t^2(1 - t) * P_2 + t^3 * P_3
% % ```

% % 其中:

% % * $t$ 是参数,取值范围为 $0 \le t \le 1$。
% % * $P_0$、$P_1$、$P_2$、$P_3$ 是控制点,分别位于曲线的起点、第一个控制点、第二个控制点和终点。

% % 贝塞尔曲线的性质如下:

% % * 贝塞尔曲线是光滑的,即曲线在任何点的切线都是连续的。
% % * 贝塞尔曲线可以通过控制点进行任意形状的变形。
% % * 贝塞尔曲线可以用来生成任意阶的多项式曲线。

% % 贝塞尔曲线在计算机图形学中有着广泛的应用,例如用于绘制图形、生成动画等。

% % **贝塞尔曲线的应用**

% % 贝塞尔曲线在计算机图形学中有着广泛的应用,例如用于绘制图形、生成动画等。

% % * 绘制图形:贝塞尔曲线可以用于绘制各种各样的图形,例如线条、曲线、图形等。
% % * 生成动画:贝塞尔曲线可以用于生成动画,例如人物动画、物体动画等。
% % * 图像处理:贝塞尔曲线可以用于图像处理,例如图像变形、图像合成等。

% % **贝塞尔曲线的优点**

% % 贝塞尔曲线具有以下优点:

% % * 光滑:贝塞尔曲线是光滑的,即曲线在任何点的切线都是连续的。
% % * 可变形:贝塞尔曲线可以通过控制点进行任意形状的变形。
% % * 通用性:贝塞尔曲线可以用来生成任意阶的多项式曲线。

% % **贝塞尔曲线的缺点**

% % 贝塞尔曲线具有以下缺点:

% % * 计算复杂:贝塞尔曲线的计算比较复杂。
% % * 存储空间大:贝塞尔曲线的存储空间比较大。

% % **总结**

% % 贝塞尔曲线是一种重要的数学曲线,在计算机图形学中有着广泛的应用。它具有光滑、可变形和通用性的优点,但计算复杂和存储空间大的缺点。

% So, Karl can now add the first half circle to the picture:

% 因此,卡尔现在可以将第一个半圆添加到图片中:


% %
% \begin{codeexample}[]
% \begin{tikzpicture}
%   \draw (-1.5,0) -- (1.5,0);
%   \draw (0,-1.5) -- (0,1.5);
%   \draw (-1,0) .. controls (-1,0.555) and (-0.555,1) .. (0,1)
%                .. controls (0.555,1) and (1,0.555) .. (1,0);
% \end{tikzpicture}
% \end{codeexample}

% Karl is happy with the result, but finds specifying circles in this way to be
% extremely awkward. Fortunately, there is a much simpler way.

% Karl对结果感到满意,但他发现用这种方式指定圆非常麻烦。幸运的是,有一种更简单的方法。


% \subsection{Circle Path Construction\\绘制圆形路径}

% In order to draw a circle, the path construction operation |circle| can be
% used. This operation is followed by a radius in brackets as in the following
% example: (Note that the previous position is used as the \emph{center} of the
% circle.)
% %

% 为了绘制一个圆,可以使用路径构造操作|circle|。该操作后跟在括号内的半径,如下面的示例所示:(请注意,前一个位置被用作圆的\emph{中心}。)

% \begin{codeexample}[]
% \tikz \draw (0,0) circle [radius=10pt];
% \end{codeexample}

% You can also append an ellipse to the path using the |ellipse| operation.
% Instead of a single radius you can specify two of them:
% %

% 您还可以使用|ellipse|操作将椭圆附加到路径上。您可以指定两个半径,而不是一个:

% \begin{codeexample}[]
% \tikz \draw (0,0) ellipse [x radius=20pt, y radius=10pt];
% \end{codeexample}

% To draw an ellipse whose axes are not horizontal and vertical, but point in an
% arbitrary direction (a ``turned ellipse'' like \tikz \draw[rotate=30] (0,0)
% ellipse [x radius=6pt, y radius=3pt];) you can use transformations, which are
% explained later. The code for the little ellipse is
% |\tikz \draw[rotate=30] (0,0) ellipse [x radius=6pt, y radius=3pt];|, by the
% way.

% 要绘制一个轴不是水平和垂直的椭圆,而是指向任意方向的椭圆(一个``旋转椭圆'',如\tikz \draw[rotate=30] (0,0) ellipse [x radius=6pt, y radius=3pt];),您可以使用后面将解释的变换。顺便说一下,小椭圆的代码是\\|\tikz \draw[rotate=30] (0,0) ellipse [x radius=6pt, y radius=3pt];|。

% So, returning to Karl's problem, he can write
% |\draw (0,0) circle [radius=1cm];| to draw the circle:
% %

% 因此,回到Karl的问题,他可以写|\draw (0,0) circle [radius=1cm];|来绘制圆:

% \begin{codeexample}[]
% \begin{tikzpicture}
%   \draw (-1.5,0) -- (1.5,0);
%   \draw (0,-1.5) -- (0,1.5);
%   \draw (0,0) circle [radius=1cm];
% \end{tikzpicture}
% \end{codeexample}

% At this point, Karl is a bit alarmed that the circle is so small when he wants
% the final picture to be much bigger. He is pleased to learn that \tikzname\ has
% powerful transformation options and scaling everything by a factor of three is
% very easy. But let us leave the size as it is for the moment to save some
% space.

% 此时,Karl有点担心当他希望最终的图片更大时,圆太小了。他很高兴地知道\tikzname\ 有强大的变换选项,将所有内容放大三倍非常容易。但为了节省空间,让我们暂时保持尺寸不变。

% \subsection{Rectangle Path Construction\\矩形路径构造}

% The next things we would like to have is the grid in the background. There are
% several ways to produce it. For example, one might draw lots of rectangles.
% Since rectangles are so common, there is a special syntax for them: To add a
% rectangle to the current path, use the |rectangle| path construction operation.
% This operation should be followed by another coordinate and will append a
% rectangle to the path such that the previous coordinate and the next
% coordinates are corners of the rectangle. So, let us add two rectangles to the
% picture:
% %

% 接下来我们想要的是背景中的网格。有几种方法可以实现。例如,可以绘制许多矩形。由于矩形非常常见,因此有一种特殊的语法可以用于绘制矩形:要将矩形添加到当前路径中,请使用|rectangle|路径构造操作。此操作后应跟另一个坐标,并将矩形附加到路径中,以使前一个坐标和下一个坐标成为矩形的角。让我们在图片中添加两个矩形:

% \begin{codeexample}[]
% \begin{tikzpicture}
%   \draw (-1.5,0) -- (1.5,0);
%   \draw (0,-1.5) -- (0,1.5);
%   \draw (0,0) circle [radius=1cm];
%   \draw (0,0) rectangle (0.5,0.5);
%   \draw (-0.5,-0.5) rectangle (-1,-1);
% \end{tikzpicture}
% \end{codeexample}

% While this may be nice in other situations, this is not really leading anywhere
% with Karl's problem: First, we would need an awful lot of these rectangles and
% then there is the border that is not ``closed''.

% 虽然在其他情况下这可能很好,但对于Karl的问题来说,这并没有真正解决问题:首先,我们需要大量这些矩形,其次边界并没有被“闭合”。


% So, Karl is about to resort to simply drawing four vertical and four horizontal
% lines using the nice |\draw| command, when he learns that there is a |grid|
% path construction operation.

% 所以,当Karl打算使用简单的|\draw|命令绘制四条垂直线和四条水平线时,他得知有一个|grid|路径构造操作。


% \subsection{Grid Path Construction\\网格路径构造}

% The |grid| path operation adds a grid to the current path. It will add lines
% making up a grid that fills the rectangle whose one corner is the current point
% and whose other corner is the point following the |grid| operation. For
% example, the code |\tikz \draw[step=2pt] (0,0) grid (10pt,10pt);| produces
% \tikz \draw[step=2pt] (0,0) grid (10pt,10pt);. Note how the optional argument
% for |\draw| can be used to specify a grid width (there are also |xstep| and
% |ystep| to define the steppings independently). As Karl will learn soon, there
% are \emph{lots} of things that can be influenced using such options.

% |grid|路径操作将网格添加到当前路径中。它将添加构成填充由当前点和|grid|操作后的点确定的矩形的网格线。例如,代码|\tikz \draw[step=2pt] (0,0) grid (10pt,10pt);|将产生\tikz \draw[step=2pt] (0,0) grid (10pt,10pt);。请注意,|\draw|的可选参数可用于指定网格宽度(也有|xstep|和|ystep|可单独定义步进)。正如Karl很快会了解到的,有很多事情可以通过这些选项来控制。

% For Karl, the following code could be used:
% %

% 对于Karl来说,可以使用以下代码:

% \begin{codeexample}[]
% \begin{tikzpicture}
%   \draw (-1.5,0) -- (1.5,0);
%   \draw (0,-1.5) -- (0,1.5);
%   \draw (0,0) circle [radius=1cm];
%   \draw[step=.5cm] (-1.4,-1.4) grid (1.4,1.4);
% \end{tikzpicture}
% \end{codeexample}

% Having another look at the desired picture, Karl notices that it would be nice
% for the grid to be more subdued. (His son told him that grids tend to be
% distracting if they are not subdued.) To subdue the grid, Karl adds two more
% options to the |\draw| command that draws the grid. First, he uses the color
% |gray| for the grid lines. Second, he reduces the line width to |very thin|.
% Finally, he swaps the ordering of the commands so that the grid is drawn first
% and everything else on top.
% %

% 再次查看所需的图片,Karl注意到网格更加柔和会更好(他的儿子告诉他,如果网格没有柔和处理,会分散注意力)。为了使网格柔和,Karl向绘制网格的|\draw|命令添加了两个选项。首先,他使用|gray|颜色绘制网格线。其次,他将线宽减小为|very thin|。最后,他交换了命令的顺序,使网格先绘制,其他内容在其上绘制。

% \begin{codeexample}[]
% \begin{tikzpicture}
%   \draw[step=.5cm,gray,very thin] (-1.4,-1.4) grid (1.4,1.4);
%   \draw (-1.5,0) -- (1.5,0);
%   \draw (0,-1.5) -- (0,1.5);
%   \draw (0,0) circle [radius=1cm];
% \end{tikzpicture}
% \end{codeexample}



% \subsection{Adding a Touch of  Style\\添加一点样式}

% Instead of the options |gray,very thin| Karl could also have said |help lines|.
% \emph{Styles} are predefined sets of options that can be used to organize how a
% graphic is drawn. By saying |help lines| you say ``use the style that I (or
% someone else) has set for drawing help lines''. If Karl decides, at some later
% point, that grids should be drawn, say, using the color |blue!50| instead of
% |gray|, he could provide the following option somewhere:
% %

% 卡尔可以选择使用选项|gray,very thin|,也可以选择|help lines|。\emph{样式}是预定义的选项集,用于组织绘制图形的方式。通过使用|help lines|,你在实际上是在说“使用我(或其他人)为绘制辅助线设置的样式”。如果卡尔在以后的某个时间决定,网格应该使用颜色|blue!50|而不是|gray|来绘制,他可以在某个地方提供以下选项:


% \begin{codeexample}[code only]
% help lines/.style={color=blue!50,very thin}
% \end{codeexample}
% %
% The effect of this ``style setter'' is that in the current scope or environment
% the |help lines| option has the same effect as |color=blue!50,very thin|.

% 这个“样式设置器”的效果是,在当前的作用域或环境中,|help lines|选项具有与|color=blue!50,very thin|相同的效果。



% Using styles makes your graphics code more flexible. You can change the way
% things look easily in a consistent manner. Normally, styles are defined at the
% beginning of a picture. However, you may sometimes wish to define a style
% globally, so that all pictures of your document can use this style. Then you
% can easily change the way all graphics look by changing this one style. In this
% situation you can use the |\tikzset| command at the beginning of the document
% as in
% %

% 使用样式可以使你的图形代码更加灵活。你可以轻松以一致的方式改变事物的外观。通常,样式在图片的开头定义。然而,有时你可能希望全局定义一个样式,这样文档中的所有图片都可以使用这个样式。然后,你可以在文档开头使用|\tikzset|命令,如下所示:


% \begin{codeexample}[code only]
% \tikzset{help lines/.style=very thin}
% \end{codeexample}

% To build a hierarchy of styles you can have one style use another. So in order
% to define a style |Karl's grid| that is based on the |grid| style Karl could
% say
% %

% 为了构建样式的层次结构,你可以让一个样式使用另一个样式。因此,为了定义一个基于样式|grid|的样式|Karl's grid|,卡尔可以这样说:

% \begin{codeexample}[code only]
% \tikzset{Karl's grid/.style={help lines,color=blue!50}}
% ...
% \draw[Karl's grid] (0,0) grid (5,5);
% \end{codeexample}

% Styles are made even more powerful by parametrization. This means that, like
% other options, styles can also be used with a parameter. For instance, Karl
% could parameterize his grid so that, by default, it is blue, but he could also
% use another color.
% %

% 通过参数化,样式的功能更加强大。这意味着,像其他选项一样,样式也可以带有参数。例如,卡尔可以使他的网格参数化,以便默认情况下为蓝色,但他也可以使用其他颜色。

% \begin{codeexample}[code only]
% \begin{tikzpicture}
%   [Karl's grid/.style  ={help lines,color=#1!50},
%    Karl's grid/.default=blue]

%   \draw[Karl's grid]     (0,0) grid (1.5,2);
%   \draw[Karl's grid=red] (2,0) grid (3.5,2);
% \end{tikzpicture}
% \end{codeexample}

%  In this example, the definition of the style |Karl's grid| is given as an
%  optional argument to the |{tikzpicture}| environment. Additional styles for other
%  elements would follow after a comma. With many styles in effect, the optional
%  argument of the environment may easily happen to be longer than the actual
%  contents.

%  在这个例子中,样式|Karl's grid|的定义作为可选参数给出,放在|{tikzpicture}|环境中。其他元素的附加样式将在逗号后面跟随。当许多样式生效时,环境的可选参数可能比实际内容还要长。

% \subsection{Drawing Options\\绘制选项}

% Karl wonders what other options there are that influence how a path is drawn.
% He saw already that the |color=|\meta{color} option can be used to set the
% line's color. The option |draw=|\meta{color} does nearly the same, only it sets
% the color for the lines only and a different color can be used for filling
% (Karl will need this when he fills the arc for the angle).

% 卡尔想知道还有哪些选项会影响路径的绘制方式。他已经看到了|color=|\meta{color}选项可以用于设置线条的颜色。选项|draw=|\meta{color}几乎相同,只是它只设置线条的颜色,可以使用不同的颜色进行填充(当卡尔填充角度的弧时会用到这个)。

% He saw that the style |very thin| yields very thin lines. Karl is not really
% surprised by this and neither is he surprised to learn that |thin| yields thin
% lines,  |thick| yields thick lines, |very thick| yields very thick lines,
% |ultra thick| yields really, really thick lines and |ultra thin| yields lines
% that are so thin that low-resolution printers and displays will have trouble
% showing them. He wonders what gives lines of ``normal'' thickness. It turns out
% that |thin| is the correct choice, since it gives the same thickness as \TeX's
% |\hrule| command. Nevertheless, Karl would like to know whether there is
% anything ``in the middle'' between |thin| and |thick|. There is: |semithick|.


% 他已经看到 |very thin| 样式可以绘制非常细的线条。卡尔对此并不感到惊讶,他也不惊讶地了解到 |thin| 样式可以绘制细线条,|thick| 样式可以绘制粗线条,|very thick| 样式可以绘制非常粗的线条,|ultra thick| 样式可以绘制非常非常粗的线条,而 |ultra thin| 样式可以绘制非常细的线条,低分辨率的打印机和显示器可能无法显示它们。他想知道是否有一种线条厚度介于 |thin| 和 |thick| 之间的选项。答案是有的,这就是 |semithick|。



% Another useful thing one can do with lines is to dash or dot them. For this,
% the two styles |dashed| and |dotted| can be used, yielding \tikz[baseline]
% \draw[dashed] (0,.5ex) -- ++(2em,0pt); and \tikz[baseline] \draw[dotted]
% (0,.5ex) -- ++(2em,0pt);. Both options also exist in a loose and a dense
% version, called |loosely dashed|, |densely dashed|, |loosely dotted|, and
% |densely dotted|. If he really, really  needs to, Karl can also define much
% more complex dashing patterns with the |dash pattern| option, but his son
% insists that dashing is to be used with utmost care and mostly distracts.
% Karl's son claims that complicated dashing patterns are evil. Karl's students
% do not care about dashing patterns.

% 对于线条,还有一个有用的功能是虚线和点线。可以使用两个样式 |dashed| 和 |dotted|,分别绘制虚线和点线,得到的效果分别如下:\tikz[baseline] \draw[dashed] (0,.5ex) -- ++(2em,0pt); 和 \tikz[baseline] \draw[dotted] (0,.5ex) -- ++(2em,0pt);。这两个选项还有一种稀疏和稠密的版本,分别称为 |loosely dashed|、|densely dashed|、|loosely dotted| 和 |densely dotted|。如果卡尔确实需要的话,还可以使用 |dash pattern| 选项定义更复杂的虚线样式,但他的儿子坚持认为虚线应该小心使用,因为它们会分散注意力。卡尔的儿子声称复杂的虚线样式是有害的。卡尔的学生们对虚线样式不太关心。


% \subsection{Arc Path Construction\\弧路径构建}

% Our next obstacle is to draw the arc for the angle. For this, the |arc| path
% construction operation is useful, which draws part of a circle or ellipse. This
% |arc| operation is followed by options in brackets that specify the arc. An
% example would be \texttt{arc[start angle=10, end angle=80, radius=10pt]}, which
% means exactly what it says. Karl obviously needs an arc from $0^\circ$ to
% $30^\circ$. The radius should be something relatively small, perhaps around one
% third of the circle's radius. When one uses the arc path construction
% operation, the specified arc will be added with its starting point at the
% current position. So, we first have to ``get there''.

% 下一个障碍是绘制角度的弧。为此,\texttt{arc}路径构建操作非常有用,它绘制了圆或椭圆的一部分。这个\texttt{arc}操作后面跟着方括号中指定弧的选项。一个例子是\texttt{arc[start angle=10, end angle=80, radius=10pt]},它的含义如字面意思所述。卡尔显然需要从$0^\circ$到$30^\circ$的弧。半径应该相对较小,可能是整个圆半径的三分之一左右。当使用弧路径构建操作时,指定的弧将从当前位置开始添加。因此,我们首先要“到达那里”。


% %
% \begin{codeexample}[]
% \begin{tikzpicture}
%   \draw[step=.5cm,gray,very thin] (-1.4,-1.4) grid (1.4,1.4);
%   \draw (-1.5,0) -- (1.5,0);
%   \draw (0,-1.5) -- (0,1.5);
%   \draw (0,0) circle [radius=1cm];
%   \draw (3mm,0mm) arc [start angle=0, end angle=30, radius=3mm];
% \end{tikzpicture}
% \end{codeexample}

% Karl thinks this is really a bit small and he cannot continue unless he learns
% how to do scaling. For this, he can add the |[scale=3]| option. He could add
% this option to each |\draw| command, but that would be awkward. Instead, he
% adds it to the whole environment, which causes this option to apply to
% everything within.

% 卡尔认为这个弧太小了,除非他学会如何进行缩放,否则无法继续。为此,他可以添加\texttt{[scale=3]}选项。他可以将此选项添加到每个\texttt{\textbackslash draw}命令,但那样会很麻烦。相反,他将其添加到整个环境中,这使得此选项适用于其中的所有内容。


% %
% \begin{codeexample}[]
% \begin{tikzpicture}[scale=3]
%   \draw[step=.5cm,gray,very thin] (-1.4,-1.4) grid (1.4,1.4);
%   \draw (-1.5,0) -- (1.5,0);
%   \draw (0,-1.5) -- (0,1.5);
%   \draw (0,0) circle [radius=1cm];
%   \draw (3mm,0mm) arc [start angle=0, end angle=30, radius=3mm];
% \end{tikzpicture}
% \end{codeexample}

% As for circles, you can specify ``two'' radii in order to get an elliptical
% arc.
% %

% 对于圆,您可以指定“两个”半径以得到椭圆弧。


% \begin{codeexample}[]
%   \tikz \draw (0,0)
%     arc [start angle=0, end angle=315,
%          x radius=1.75cm, y radius=1cm];
% \end{codeexample}



% \subsection{Clipping a Path\\裁剪路径}

% In order to save space in this manual, it would be nice to clip Karl's graphics
% a bit so that we can focus on the ``interesting'' parts. Clipping is pretty
% easy in \tikzname. You can use the |\clip| command to clip all subsequent
% drawing. It works like |\draw|, only it does not draw anything, but uses the
% given path to clip everything subsequently.
% %

% 为了在本手册中节省空间,我们希望将卡尔的图形裁剪一下,以便我们可以专注于“有趣”的部分。在\tikzname 中,裁剪非常简单。您可以使用|\clip|命令来裁剪所有后续的绘图。它的使用方式类似于|\draw|,只是它不绘制任何东西,而是使用给定的路径来裁剪后续的一切。


% \begin{codeexample}[]
% \begin{tikzpicture}[scale=3]
%   \clip (-0.1,-0.2) rectangle (1.1,0.75);
%   \draw[step=.5cm,gray,very thin] (-1.4,-1.4) grid (1.4,1.4);
%   \draw (-1.5,0) -- (1.5,0);
%   \draw (0,-1.5) -- (0,1.5);
%   \draw (0,0) circle [radius=1cm];
%   \draw (3mm,0mm) arc [start angle=0, end angle=30, radius=3mm];
% \end{tikzpicture}
% \end{codeexample}

% You can also do both at the same time: Draw \emph{and} clip a path. For this,
% use the |\draw| command and add the |clip| option. (This is not the whole
% picture: You can also use the |\clip| command and add the |draw| option. Well,
% that is also not the whole picture: In reality, |\draw| is just a shorthand for
% |\path[draw]| and |\clip| is a shorthand for |\path[clip]| and you could also
% say |\path[draw,clip]|.) Here is an example:
% %

% % 您还可以同时进行绘制和裁剪路径。为此,使用|\draw|命令并以|\clip|命令开头;这样,第一个命令是裁剪命令,而随后的命令将用于绘制。
% 您还可以同时绘制路径和剪裁路径。为此,请使用 |\draw| 命令并添加 |clip| 选项。(这还不是全部内容:您还可以使用 |\clip| 命令并添加 |draw| 选项。嗯,这还不是全部内容:实际上,|\draw| 只是 |\path[draw]| 的简写,|\clip| 是 |\path[clip]| 的简写,您也可以使用 |\path[draw,clip]|。)以下是一个示例:



% \begin{codeexample}[]
% \begin{tikzpicture}[scale=3]
%   \clip[draw] (0.5,0.5) circle (.6cm);
%   \draw[step=.5cm,gray,very thin] (-1.4,-1.4) grid (1.4,1.4);
%   \draw (-1.5,0) -- (1.5,0);
%   \draw (0,-1.5) -- (0,1.5);
%   \draw (0,0) circle [radius=1cm];
%   \draw (3mm,0mm) arc [start angle=0, end angle=30, radius=3mm];
% \end{tikzpicture}
% \end{codeexample}



% \subsection{Parabola and Sine Path Construction\\抛物线和正弦路径构造}

% Although Karl does not need them for his picture, he is pleased to learn that
% there are |parabola| and |sin| and |cos| path operations for adding parabolas
% and sine and cosine curves to the current path. For the |parabola| operation,
% the current point will lie on the parabola as well as the point given after the
% parabola operation. Consider the following example:
% %

% 虽然 Karl 并不需要它们来绘制他的图片,但他很高兴地了解到存在 |parabola|、|sin| 和 |cos| 路径操作,用于在当前路径中添加抛物线和正弦曲线以及余弦曲线。对于 |parabola| 操作,当前点将位于抛物线上,以及在抛物线操作之后给出的点上。考虑以下示例:


% \begin{codeexample}[]
% \tikz \draw (0,0) rectangle (1,1)  (0,0) parabola (1,1);
% \end{codeexample}
% % (0,0) parabola (1,1) 这部分代码绘制了一条抛物线,起始点为 (0,0),终点为 (1,1)。抛物线经过起始点和终点,并且曲线形状由这两个点决定。

% It is also possible to place the bend somewhere else:
% %

% 也可以将曲线弯曲到其他位置:


% \begin{codeexample}[]
% \tikz \draw[x=1pt,y=1pt] (0,0) parabola bend (4,16) (6,12);
% \end{codeexample}
% %|(0,0) parabola bend (4,16) (6,12)|:这部分代码绘制了一条抛物线。起始点为 (0,0),通过 parabola 命令绘制的抛物线将起始点和终点 (4,16) 连接起来。然后,使用 bend 命令将曲线弯曲到另一个坐标点 (6,12)。

% The operations |sin| and |cos| add a sine or cosine curve in the interval
% $[0,\pi/2]$ such that the previous current point is at the start of the curve
% and the curve ends at the given end point. Here are two examples:
% %

% |sin| 和 |cos| 操作在区间 $[0,\pi/2]$ 中添加正弦曲线或余弦曲线,使得前一个当前点位于曲线的起点,曲线在给定的终点结束。以下是两个示例:


% \begin{codeexample}[]
% A sine \tikz \draw[x=1ex,y=1ex] (0,0) sin (1.57,1); curve.
% \end{codeexample}
% % |x=1ex,y=1ex|:这部分代码设置了坐标系的缩放,将 x 和 y 方向的单位长度设置为 1ex。这样可以控制图形的尺寸。

% % |(0,0) sin (1.57,1)|:这部分代码绘制了一条正弦曲线。起始点为 (0,0),通过 sin 命令绘制的曲线连接起始点和终点 (1.57,1)。

% \begin{codeexample}[]
% \tikz \draw[x=1.57ex,y=1ex] (0,0) sin (1,1) cos (2,0) sin (3,-1) cos (4,0)
%                             (0,1) cos (1,0) sin (2,-1) cos (3,0) sin (4,1);
% \end{codeexample}
% % |(0,0) sin (1,1) cos (2,0) sin (3,-1) cos (4,0) (0,1) cos (1,0) sin (2,-1) cos (3,0) sin (4,1)|:这部分代码绘制了一组相互连接的正弦曲线和余弦曲线。每个命令表示一个曲线段,其中包括起始点和终点。通过使用 sin 命令绘制正弦曲线,使用 cos 命令绘制余弦曲线。

% \subsection{Filling and Drawing\\填充和绘制}

% Returning to the picture, Karl now wants the angle to be ``filled'' with a very
% light green. For this he uses |\fill| instead of |\draw|. Here is what Karl
% does:

% 回到图形,Karl现在想要“填充”角度,使用非常浅的绿色。为此,他使用|\fill|而不是|\draw|。以下是Karl的操作:

% \begin{codeexample}[]
% \begin{tikzpicture}[scale=3]
%   \clip (-0.1,-0.2) rectangle (1.1,0.75);
%   \draw[step=.5cm,gray,very thin] (-1.4,-1.4) grid (1.4,1.4);
%   \draw (-1.5,0) -- (1.5,0);
%   \draw (0,-1.5) -- (0,1.5);
%   \draw (0,0) circle [radius=1cm];
%   \fill[green!20!white] (0,0) -- (3mm,0mm)
%     arc [start angle=0, end angle=30, radius=3mm] -- (0,0);
% \end{tikzpicture}
% \end{codeexample}
% % [green!20!white] 指定填充的颜色,是绿色和白色的混合,绿色占80%。
% % (0,0) -- (3mm,0mm) 绘制一条线段,从(0,0)到(3mm,0mm)。
% % arc 指定绘制一段圆弧。
% % -- (0,0) 将终点和起点相连,形成封闭的路径。





% The color |green!20!white| means 20\% green and 80\% white mixed together. Such
% color expression are possible since \tikzname\ uses Uwe Kern's |xcolor|
% package, see the documentation of that package for details on color
% expressions.

% 颜色|green!20!white|表示20\%的绿色和80\%的白色混合在一起。由于\tikzname 使用了Uwe Kern的|xcolor|宏包,因此可以使用这种颜色表达式,请参阅该宏包的文档以了解有关颜色表达式的详细信息。


% What would have happened, if Karl had not ``closed'' the path using |--(0,0)|
% at the end? In this case, the path is closed automatically, so this could have
% been omitted. Indeed, it would even have been better to write the following,
% instead:

% 如果Karl在最后没有使用|--(0,0)|“闭合”路径会发生什么?在这种情况下,路径会自动闭合,因此可以省略这一部分。事实上,最好写成以下形式:

% \begin{codeexample}[code only]
%   \fill[green!20!white] (0,0) -- (3mm,0mm)
%     arc [start angle=0, end angle=30, radius=3mm] -- cycle;
% \end{codeexample}
% %
% The |--cycle| causes the current path to be closed (actually the current part
% of the current path) by smoothly joining the first and last point. To
% appreciate the difference, consider the following example:


% |--cycle|通过平滑连接第一个和最后一个点来关闭当前路径(实际上是当前路径的当前部分)。为了体会到差异,请考虑以下示例:

% \begin{codeexample}[]
% \begin{tikzpicture}[line width=5pt]
%   \draw (0,0) -- (1,0) -- (1,1) -- (0,0);
%   \draw (2,0) -- (3,0) -- (3,1) -- cycle;
%   \useasboundingbox (0,1.5); % make bounding box higher
% \end{tikzpicture}
% \end{codeexample}

% You can also fill and draw a path at the same time using the |\filldraw|
% command. This will first draw the path, then fill it. This may not seem too
% useful, but you can specify different colors to be used for filling and for
% stroking. These are specified as optional arguments like this:
% %

% 你还可以同时填充和绘制路径,使用|\filldraw|命令。这将首先绘制路径,然后填充它。这可能看起来并不太有用,但是你可以指定用于填充和描边的不同颜色。这些颜色可以作为可选参数指定,如下所示:

% \begin{codeexample}[]
% \begin{tikzpicture}[scale=3]
%   \clip (-0.1,-0.2) rectangle (1.1,0.75);
%   \draw[step=.5cm,gray,very thin] (-1.4,-1.4) grid (1.4,1.4);
%   \draw (-1.5,0) -- (1.5,0);
%   \draw (0,-1.5) -- (0,1.5);
%   \draw (0,0) circle [radius=1cm];
%   \filldraw[fill=green!20!white, draw=green!50!black] (0,0) -- (3mm,0mm)
%     arc [start angle=0, end angle=30, radius=3mm] -- cycle;
% \end{tikzpicture}
% \end{codeexample}



% \subsection{Shading\\渐变}

% Karl briefly considers the possibility of making the angle ``more fancy'' by
% \emph{shading} it. Instead of filling the area with a uniform color, a smooth
% transition between different colors is used. For this, |\shade| and
% |\shadedraw|, for shading and drawing at the same time, can be used:

% Karl简要考虑了通过\emph{渐变}使角度“更加花哨”的可能性。与使用均匀颜色填充区域不同,使用不同颜色之间的平滑过渡。为此,可以使用|\shade|和|\shadedraw|进行渐变和同时绘制:

% \begin{codeexample}[]
%   \tikz \shade (0,0) rectangle (2,1)  (3,0.5) circle (.5cm);
% \end{codeexample}
% %
% The default shading is a smooth transition from gray to white. To specify
% different colors, you can use options:


% 默认的渐变是从灰色到白色的平滑过渡。要指定不同的颜色,可以使用选项:

% \begin{codeexample}[]
% \begin{tikzpicture}[rounded corners,ultra thick]
%   \shade[top color=yellow,bottom color=black] (0,0) rectangle +(2,1);
%   \shade[left color=yellow,right color=black] (3,0) rectangle +(2,1);
%   \shadedraw[inner color=yellow,outer color=black,draw=yellow] (6,0) rectangle +(2,1);
%   \shade[ball color=green] (9,.5) circle (.5cm);
% \end{tikzpicture}
% \end{codeexample}

% For Karl, the following might be appropriate:

% 对于Karl来说,可能适合的是:

% \begin{codeexample}[]
% \begin{tikzpicture}[scale=3]
%   \clip (-0.1,-0.2) rectangle (1.1,0.75);
%   \draw[step=.5cm,gray,very thin] (-1.4,-1.4) grid (1.4,1.4);
%   \draw (-1.5,0) -- (1.5,0);
%   \draw (0,-1.5) -- (0,1.5);
%   \draw (0,0) circle [radius=1cm];
%   \shadedraw[left color=gray,right color=green, draw=green!50!black]
%     (0,0) -- (3mm,0mm)
%     arc [start angle=0, end angle=30, radius=3mm] -- cycle;
% \end{tikzpicture}
% \end{codeexample}

% However, he wisely decides that shadings usually only distract without adding
% anything to the picture.

% 然而,他明智地决定,渐变通常只会分散注意力,而不会为图形增添任何东西。



% \subsection{Specifying Coordinates\\指定坐标}

% Karl now wants to add the sine and cosine lines. He knows already that he can
% use the |color=| option to set the lines' colors. So, what is the best way to
% specify the coordinates?

% 现在Karl想要添加正弦和余弦线。他已经知道可以使用|color=|选项来设置线条的颜色。那么,最好的方法是如何指定坐标呢?



% There are different ways of specifying coordinates. The easiest way is to say
% something like |(10pt,2cm)|. This means 10pt in $x$-direction and 2cm in
% $y$-directions. Alternatively, you can also leave out the units as in |(1,2)|,
% which means ``one times the current $x$-vector plus twice the current
% $y$-vector''. These vectors default to 1cm in the $x$-direction and 1cm in the
% $y$-direction, respectively.

% 有不同的指定坐标的方法。最简单的方法是使用类似于|(10pt,2cm)|的表示方式。这表示在$x$方向上为10pt,在$y$方向上为2cm。另外,你也可以省略单位,例如|(1,2)|,它表示``当前$x$向量的一倍加上当前$y$向量的两倍''。这些向量默认分别为1cm和1cm。

% In order to specify points in polar coordinates, use the notation |(30:1cm)|,
% which means 1cm in direction 30 degree. This is obviously quite useful to ``get
% to the point $(\cos 30^\circ,\sin 30^\circ)$ on the circle''.

% 为了指定极坐标中的点,可以使用|(30:1cm)|的表示方式,它表示方向为30度的1cm。这显然非常有用,可以``到达圆上的点$(\cos 30^\circ,\sin 30^\circ)$''。

% % 极坐标是表示平面上的点的一种坐标系统。

% % 极坐标将平面上的每个点表示成(r,θ)的形式,其中:

% % r是该点到坐标原点的距离,叫做极径。

% % θ是该点在平面上的角度,叫做极角。

% % 极角θ的取值范围通常是[0, 2π),也可以是[0, 360°)。θ=0对应正右方。

% % 极坐标与笛卡尔坐标的转换关系为:

% % x = rcosθ

% % y = rsinθ

% % 极坐标的性质和应用包括:

% % 极坐标通过距离原点和角度唯一确定一个点。

% % 极坐标对于表示圆、螺线等曲线方便。

% % 极坐标可用于表示向心力和离心力问题。

% % 极坐标可简化表达微分方程的解析解。

% % 极坐标在计算机图形学中也有应用。

% % 极坐标结合复数的指数形式,可以表示复平面上的点。

% % 总之,极坐标是除了笛卡尔坐标以外描述平面点的重要方法。它在某些问题中表达更简洁和直观。但笛卡尔坐标对于表示直线方程更方便。两者都有各自的优点。

% You can add a single |+| sign in front of a coordinate or two of them as in
% |+(0cm,1cm)| or |++(2cm,0cm)|. Such coordinates are interpreted differently:
% The first form means ``1cm upwards from the previous specified position'' and
% the second means ``2cm to the right of the previous specified position, making
% this the new specified position''. For example, we can draw the sine line as
% follows:
% %

% 你可以在坐标前面加一个单独的|+|符号,或者两个|+|符号,例如|+(0cm,1cm)|或|++(2cm,0cm)|。这些坐标的解释方式不同:第一种形式表示相对于之前指定的位置向上移动1cm'',而第二种表示相对于之前指定的位置向右移动2cm,并将其作为新指定的位置''。例如,我们可以如下绘制正弦线:

% \begin{codeexample}[]
% \begin{tikzpicture}[scale=3]
%   \clip (-0.1,-0.2) rectangle (1.1,0.75);
%   \draw[step=.5cm,gray,very thin] (-1.4,-1.4) grid (1.4,1.4);
%   \draw (-1.5,0) -- (1.5,0);
%   \draw (0,-1.5) -- (0,1.5);
%   \draw (0,0) circle [radius=1cm];
%   \filldraw[fill=green!20,draw=green!50!black] (0,0) -- (3mm,0mm)
%       arc [start angle=0, end angle=30, radius=3mm] -- cycle;
%   \draw[red,very thick] (30:1cm) -- +(0,-0.5);
% \end{tikzpicture}
% \end{codeexample}

% Karl used the fact $\sin 30^\circ = 1/2$. However, he very much doubts that his
% students know this, so it would be nice to have a way of specifying ``the point
% straight down from |(30:1cm)| that lies on the $x$-axis''. This is, indeed,
% possible using a special syntax: Karl can write \verb!(30:1cm |- 0,0)!. In
% general, the meaning of |(|\meta{p}\verb! |- !\meta{q}|)| is ``the intersection
% of a vertical line through $p$ and a horizontal line through $q$''.

% Karl事实上使用了$\sin 30^\circ = 1/2$。然而,他非常怀疑他的学生们是否知道这一点,所以最好有一种方法来指定从|(30:1cm)|指向$x$轴下方的点''。事实上,使用一种特殊的语法是可以实现的:Karl可以写成\verb!(30:1cm |- 0,0)!。一般来说,|(|\meta{p}\verb! |- !\meta{q}|)|的意思是通过$p$的垂直线与$q$的水平线的交点''。
% %%% todo 这段需要再看看

% Next, let us draw the cosine line. One way would be to say
% \verb!(30:1cm |- 0,0) -- (0,0)!. Another way is the following: we ``continue''
% from where the sine ends:
% %

% 接下来,让我们绘制余弦线。一种方式是\verb!(30:1cm |- 0,0) -- (0,0)!。另一种方式是:我们``延续''正弦线的结束位置:

% \begin{codeexample}[]
% \begin{tikzpicture}[scale=3]
%   \clip (-0.1,-0.2) rectangle (1.1,0.75);
%   \draw[step=.5cm,gray,very thin] (-1.4,-1.4) grid (1.4,1.4);
%   \draw (-1.5,0) -- (1.5,0);
%   \draw (0,-1.5) -- (0,1.5);
%   \draw (0,0) circle [radius=1cm];
%   \filldraw[fill=green!20,draw=green!50!black] (0,0) -- (3mm,0mm)
%       arc [start angle=0, end angle=30, radius=3mm] -- cycle;
%   \draw[red,very thick]  (30:1cm) -- +(0,-0.5);
%   \draw[blue,very thick] (30:1cm) ++(0,-0.5) -- (0,0);
% \end{tikzpicture}
% \end{codeexample}

% Note that there is no |--| between |(30:1cm)| and |++(0,-0.5)|. In detail, this
% path is interpreted as follows: ``First, the |(30:1cm)| tells me to move my pen
% to $(\cos 30^\circ,1/2)$. Next, there comes another coordinate specification,
% so I move my pen there without drawing anything. This new point is half a unit
% down from the last position, thus it is at $(\cos 30^\circ,0)$. Finally, I move
% the pen to the origin, but this time drawing something (because of the |--|).''

% 请注意,在 |(30:1cm)| 和 |++(0,-0.5)| 之间没有 |--|。具体来说,该路径的解释如下:"首先,|(30:1cm)| 告诉我将笔移动到 $(\cos 30^\circ,1/2)$。接下来,出现了另一个坐标指定,所以我将笔移动到那个位置,但不画任何东西。这个新点相对于上一个位置向下移动了半个单位,因此它位于 $(\cos 30^\circ,0)$。最后,我将笔移动到原点,但这次画了一些东西(因为有 |--|)"。



% To appreciate the difference between |+| and |++| consider the following
% example:


% 为了理解 |+| 和 |++| 之间的区别,请考虑以下示例:%todo 后面再看看...
% %
% \begin{codeexample}[]
% \begin{tikzpicture}
%   \def\rectanglepath{-- ++(1cm,0cm)  -- ++(0cm,1cm)  -- ++(-1cm,0cm) -- cycle}
%   \draw (0,0) \rectanglepath;
%   \draw (1.5,0) \rectanglepath;
% \end{tikzpicture}
% \end{codeexample}

% By comparison, when using a single |+|, the coordinates are different:
% %

% 相比之下,当使用单个 |+| 时,坐标是不同的:

% \begin{codeexample}[]
% \begin{tikzpicture}
%   \def\rectanglepath{-- +(1cm,0cm)  -- +(1cm,1cm)  -- +(0cm,1cm) -- cycle}
%   \draw (0,0) \rectanglepath;
%   \draw (1.5,0) \rectanglepath;
% \end{tikzpicture}
% \end{codeexample}


% Naturally, all of this could have been written more clearly and more
% economically like this (either with a single or a double |+|):

% 当然,所有这些都可以更清晰、更简洁地写成这样(无论是使用单个还是双个 |+|):

% \begin{codeexample}[]
% \tikz \draw (0,0) rectangle +(1,1)  (1.5,0) rectangle +(1,1);
% \end{codeexample}

