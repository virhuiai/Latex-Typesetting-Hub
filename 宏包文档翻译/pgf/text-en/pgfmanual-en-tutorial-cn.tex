% % Copyright 2019 by Till Tantau
% %
% % This file may be distributed and/or modified
% %
% % 1. under the LaTeX Project Public License and/or
% % 2. under the GNU Free Documentation License.
% %
% % See the file doc/generic/pgf/licenses/LICENSE for more details.


% \section{Tutorial: A Picture for Karl's Students\\教程:为卡尔的学生绘制图片}

% This tutorial is intended for new users of \tikzname. It does not give an
% exhaustive account of all the features of \tikzname, just of those that you are
% likely to use right away.

% 本教程适用于初次使用 \tikzname 的用户。它并不详尽地介绍 \tikzname 的所有功能,只介绍了你可能会立即使用的部分。

% Karl is a math and chemistry high-school teacher. He used to create the
% graphics in his worksheets and exams using \LaTeX's |{picture}| environment.
% While the results were acceptable, creating the graphics often turned out to be
% a lengthy process. Also, there tended to be problems with lines having slightly
% wrong angles and circles also seemed to be hard to get right. Naturally, his
% students could not care less whether the lines had the exact right angles and
% they find Karl's exams too difficult no matter how nicely they were drawn. But
% Karl was never entirely satisfied with the result.

% 卡尔是一位数学和化学高中教师。他过去使用 \LaTeX 的 |{picture}| 环境在他的教学材料和考试中创建图形。虽然结果还可以接受,但创建图形往往是一个耗时的过程。而且,直线的角度可能稍有偏差,画圆也似乎很难做到完美。自然地,他的学生并不关心线条是否具有准确的角度,无论图形画得多好看,他们都觉得卡尔的考试太难了。但是卡尔对结果从未完全满意。

% Karl's son, who was even less satisfied with the results (he did not have to
% take the exams, after all), told Karl that he might wish to try out a new
% package for creating graphics. A bit confusingly, this package seems to have
% two names: First, Karl had to download and install a package called \pgfname.
% Then it turns out that inside this package there is another package called
% \tikzname, which is supposed to stand for ``\tikzname\ ist \emph{kein}
% Zeichenprogramm''. Karl finds this all a bit strange and \tikzname\ seems to
% indicate that the package does not do what he needs. However, having used
% \textsc{gnu} software for quite some time and ``\textsc{gnu} not being Unix'',
% there seems to be hope yet. His son assures him that \tikzname's name is
% intended to warn people that \tikzname\ is not a program that you can use to
% draw graphics with your mouse or tablet. Rather, it is more like a ``graphics
% language''.

% 卡尔的儿子对结果更不满意(毕竟他不必参加考试),他告诉卡尔可以尝试一个新的用于创建图形的软件包。有点令人困惑的是,这个软件包似乎有两个名称:首先,卡尔需要下载和安装一个名为 \pgfname 的软件包。然后发现,在这个软件包中有另一个名为 \tikzname 的软件包,它的缩写意味着\tikzname\ 不是绘图程序''。卡尔觉得这一切有点奇怪,而且 \tikzname\ 似乎表明这个软件包并不能满足他的需求。然而,他使用 \textsc{gnu} 软件已经有一段时间了,而 ``\textsc{gnu} 不是 Unix'',这似乎还有希望。他的儿子向他保证,\tikzname 的名字旨在告诫人们,\tikzname 不是一个可以用鼠标或平板电脑绘制图形的程序。相反,它更像是一种``图形语言''。



% \subsection{Problem Statement\\问题陈述}

% Karl wants to put a graphic on the next worksheet for his students. He is
% currently teaching his students about sine and cosine. What he would like to
% have is something that looks like this (ideally):

% Karl想在下一张工作表上为他的学生放置一个图形。他目前正在教学生正弦和余弦。他希望有一个看起来像这样的东西(理想情况下):

% % %
% % \noindent
% % \begin{tikzpicture}
% %   [scale=3,line cap=round,
% %    % Styles
% %    axes/.style=,
% %    important line/.style={very thick},
% %    information text/.style={rounded corners,fill=red!10,inner sep=1ex}]

% %   % Local definitions
% %   \def\costhirty{0.8660256}

% %   % Colors
% %   \colorlet{anglecolor}{green!50!black}
% %   \colorlet{sincolor}{red}
% %   \colorlet{tancolor}{orange!80!black}
% %   \colorlet{coscolor}{blue}

% %   % The graphic
% %   \draw[help lines,step=0.5cm] (-1.4,-1.4) grid (1.4,1.4);

% %   \draw (0,0) circle [radius=1cm];

% %   \begin{scope}[axes]
% %     \draw[->] (-1.5,0) -- (1.5,0) node[right] {$x$};
% %     \draw[->] (0,-1.5) -- (0,1.5) node[above] {$y$};

% %     \foreach \x/\xtext in {-1, -.5/-\frac{1}{2}, 1}
% %       \draw[xshift=\x cm] (0pt,1pt) -- (0pt,-1pt) node[below,fill=white] {$\xtext$};

% %     \foreach \y/\ytext in {-1, -.5/-\frac{1}{2}, .5/\frac{1}{2}, 1}
% %       \draw[yshift=\y cm] (1pt,0pt) -- (-1pt,0pt) node[left,fill=white] {$\ytext$};
% %   \end{scope}

% %   \filldraw[fill=green!20,draw=anglecolor] (0,0) -- (3mm,0pt) arc(0:30:3mm);
% %   \draw (15:2mm) node[anglecolor] {$\alpha$};

% %   \draw[important line,sincolor]
% %     (30:1cm) -- node[left=1pt,fill=white] {$\sin \alpha$} +(0,-.5);

% %   \draw[important line,coscolor]
% %     (0,0) -- node[below=2pt,fill=white] {$\cos \alpha$} (\costhirty,0);

% %   \draw[important line,tancolor] (1,0) --
% %     node [right=1pt,fill=white]
% %     {
% %       $\displaystyle \tan \alpha \color{black}=
% %       \frac{{\color{sincolor}\sin \alpha}}{\color{coscolor}\cos \alpha}$
% %     } (intersection of 0,0--30:1cm and 1,0--1,1) coordinate (t);

% %   \draw (0,0) -- (t);

% %   \draw[xshift=1.85cm] node [right,text width=6cm,information text]
% %     {
% %       The {\color{anglecolor} angle $\alpha$} is $30^\circ$ in the
% %       example ($\pi/6$ in radians). The {\color{sincolor}sine of
% %         $\alpha$}, which is the height of the red line, is
% %       \[
% %       {\color{sincolor} \sin \alpha} = 1/2.
% %       \]
% %       By the Theorem of Pythagoras we have ${\color{coscolor}\cos^2 \alpha} +
% %       {\color{sincolor}\sin^2\alpha} =1$. Thus the length of the blue
% %       line, which is the {\color{coscolor}cosine of $\alpha$}, must be
% %       \[
% %       {\color{coscolor}\cos\alpha} = \sqrt{1 - 1/4} = \textstyle
% %       \frac{1}{2} \sqrt 3.
% %       \]%
% %       This shows that {\color{tancolor}$\tan \alpha$}, which is the
% %       height of the orange line, is
% %       \[
% %       {\color{tancolor}\tan\alpha} = \frac{{\color{sincolor}\sin
% %           \alpha}}{\color{coscolor}\cos \alpha} = 1/\sqrt 3.
% %       \]%
% %     };
% % \end{tikzpicture}

% %
% \noindent
% \begin{tikzpicture}%用于创建TikZ图形的代码。
%   [scale=3,line cap=round,
%    % Styles 定义了一些自定义样式,可以在整个图形中使用。
%    axes/.style=,
%    important line/.style={very thick},
%    information text/.style={rounded corners,fill=red!10,inner sep=1ex}]

%   % Local definitions 
%   \def\costhirty{0.8660256}

%   % Colors
%   \colorlet{anglecolor}{green!50!black}
%   \colorlet{sincolor}{red}
%   \colorlet{tancolor}{orange!80!black}
%   \colorlet{coscolor}{blue}

%   % The graphic 绘制了一个网格,使用了帮助线样式,并且每0.5cm有一条线。
%   \draw[help lines,step=0.5cm] (-1.4,-1.4) grid (1.4,1.4);

%   \draw (0,0) circle [radius=1cm];%绘制了一个以原点(0,0)为中心,半径为1cm的圆。

%   \begin{scope}[axes]
%     \draw[->] (-1.5,0) -- (1.5,0) node[right] {$x$};
%     \draw[->] (0,-1.5) -- (0,1.5) node[above] {$y$};%绘制了x和y坐标轴。

%     \foreach \x/\xtext in {-1, -.5/-\frac{1}{2}, 1}
%       \draw[xshift=\x cm] (0pt,1pt) -- (0pt,-1pt) node[below,fill=white] {$\xtext$};

%     \foreach \y/\ytext in {-1, -.5/-\frac{1}{2}, .5/\frac{1}{2}, 1}
%       \draw[yshift=\y cm] (1pt,0pt) -- (-1pt,0pt) node[left,fill=white] {$\ytext$};
%   \end{scope}

%   \filldraw[fill=green!20,draw=anglecolor] (0,0) -- (3mm,0pt) arc(0:30:3mm);%绘制了一个表示角度α的弧,并用绿色填充了它。
%   \draw (15:2mm) node[anglecolor] {$\alpha$};%绘制一个点,并在该点的位置上标记角度符号α。

%   \draw[important line,sincolor]
%     (30:1cm) -- node[left=1pt,fill=white] {$\sin \alpha$} +(0,-.5);%绘制一条从原点开始的线段,表示正弦函数的值。这条线段位于角度α的位置,长度为1cm,并且标有$\sin \alpha$的标签。

%   \draw[important line,coscolor]
%     (0,0) -- node[below=2pt,fill=white] {$\cos \alpha$} (\costhirty,0);%绘制一条水平线段,表示余弦函数的值。这条线段的长度为$\cos 30^\circ$,并且标有$\cos \alpha$的标签。

%   %绘制一条从点(1,0)开始的线段,表示正切函数的值。这条线段的斜率等于$\frac{{\sin \alpha}}{{\cos \alpha}}$,并且标有$\tan \alpha$的标签。该线段与前面绘制的两条线段相交,交点的坐标被命名为(t)。
%   \draw[important line,tancolor] (1,0) --
%     node [right=1pt,fill=white]
%     {
%       $\displaystyle \tan \alpha \color{black}=
%       \frac{{\color{sincolor}\sin \alpha}}{\color{coscolor}\cos \alpha}$
%     } (intersection of 0,0--30:1cm and 1,0--1,1) coordinate (t);

%   \draw (0,0) -- (t);%绘制一条连接原点和点(t)的线段。

%   %在指定的位置绘制一个文本框,用于提供关于三角函数图形的说明。文本框右侧有一条竖线,并包含了一些文本和数学公式。
%   \draw[xshift=1.85cm] node [right,text width=6cm,information text]
%     {
%       The {\color{anglecolor} angle $\alpha$} is $30^\circ$ in the
%       example ($\pi/6$ in radians). The {\color{sincolor}sine of
%         $\alpha$}, which is the height of the red line, is
      
%       这个{\color{anglecolor} 角度 $\alpha$}在例子中是 $30^\circ$(弧度为 $\pi/6$)。{\color{sincolor} $\alpha$ 的正弦},也就是红线的高度,是

%       \[
%       {\color{sincolor} \sin \alpha} = 1/2.
%       \]
%       By the Theorem of Pythagoras we have ${\color{coscolor}\cos^2 \alpha} +
%       {\color{sincolor}\sin^2\alpha} =1$. Thus the length of the blue
%       line, which is the {\color{coscolor}cosine of $\alpha$}, must be

%       根据勾股定理,我们有 ${\color{coscolor}\cos^2 \alpha} +
%       {\color{sincolor}\sin^2\alpha} =1$。因此,蓝线的长度,也就是{\color{coscolor} $\alpha$ 的余弦},必须是

%       \[
%       {\color{coscolor}\cos\alpha} = \sqrt{1 - 1/4} = \textstyle
%       \frac{1}{2} \sqrt 3.
%       \]%
%       This shows that {\color{tancolor}$\tan \alpha$}, which is the
%       height of the orange line, is

%       这表明{\color{tancolor} $\alpha$ 的正切},也就是橙线的高度,是
%       \[
%       {\color{tancolor}\tan\alpha} = \frac{{\color{sincolor}\sin
%           \alpha}}{\color{coscolor}\cos \alpha} = 1/\sqrt 3.
%       \]%
%     };
% \end{tikzpicture}


% \subsection{Setting up the Environment\\设置环境}

% In \tikzname, to draw a picture, at the start of the picture you need to tell
% \TeX\ or \LaTeX\ that you want to start a picture. In \LaTeX\ this is done
% using the environment |{tikzpicture}|, in plain \TeX\ you just use
% |\tikzpicture| to start the picture and |\endtikzpicture| to end it.

% 在\tikzname 中,要绘制一个图形,你需要在图形开始时告诉\TeX\ 或\LaTeX\ 你想要开始一个图形。在\LaTeX\ 中,使用环境|{tikzpicture}|来完成这个操作,在plain \TeX\ 中,你只需使用|\tikzpicture|来开始图形,使用|\endtikzpicture|来结束图形。

% \subsubsection{Setting up the Environment in \LaTeX\\在 \LaTeX 中设置环境}

% Karl, being a \LaTeX\ user, thus sets up his file as follows:
% %

% 作为一个 \LaTeX 用户,Karl 设置他的文件如下:

% \begin{codeexample}[code only]
% \documentclass{article} % say
% \usepackage{tikz}
% \begin{document}
% We are working on
% \begin{tikzpicture}
%   \draw (-1.5,0) -- (1.5,0);
%   \draw (0,-1.5) -- (0,1.5);
% \end{tikzpicture}.
% \end{document}
% \end{codeexample}

% When executed, that is, run via |pdflatex| or via |latex| followed by |dvips|,
% the resulting will contain something that looks like this:
% %

% 当执行该代码,即通过 |pdflatex| 或 |latex| 后跟 |dvips| 运行时,结果将会包含以下内容:

% \begin{codeexample}[width=7cm]
% We are working on
% \begin{tikzpicture}
%   \draw (-1.5,0) -- (1.5,0);
%   \draw (0,-1.5) -- (0,1.5);
% \end{tikzpicture}.
% \end{codeexample}

% Admittedly, not quite the whole picture, yet, but we do have the axes
% established. Well, not quite, but we have the lines that make up the axes
% drawn. Karl suddenly has a sinking feeling that the picture is still some way
% off.

% 诚然,这还不是完整的图像,但我们已经建立了坐标轴。嗯,并不完全准确,但我们已经绘制了构成坐标轴的线段。Karl 突然有一种不安的感觉,认为图像还有一段路要走。



% Let's have a more detailed look at the code. First, the package |tikz| is
% loaded. This package is a so-called ``frontend'' to the basic \pgfname\ system.
% The basic layer, which is also described in this manual, is somewhat more,
% well, basic and thus harder to use. The frontend makes things easier by
% providing a simpler syntax.

% 让我们更详细地看一下代码。首先,加载了 |tikz| 宏包。该宏包是基本 \pgfname\ 系统的所谓“前端”。基本层在本手册中也有所描述,它更基础一些,因此使用起来更困难。前端通过提供更简单的语法使事情更容易。



% Inside the environment there are two |\draw| commands. They mean: ``The path,
% which is specified following the command up to the semicolon, should be
% drawn.'' The first path is specified as |(-1.5,0) -- (0,1.5)|, which means ``a
% straight line from the point at position $(-1.5,0)$ to the point at position
% $(0,1.5)$''. Here, the positions are specified within a special coordinate
% system in which, initially, one unit is 1cm.

% 在环境内部有两个 |\draw| 命令。它们的意思是:“应该绘制路径,路径在命令后面到分号之前指定。”第一个路径被指定为 |(-1.5,0) -- (0,1.5)|,意思是“从位置 $(-1.5,0)$ 到位置 $(0,1.5)$ 绘制一条直线”。这里,位置在一个特殊的坐标系中指定,初始情况下,一个单位等于 1cm。



% Karl is quite pleased to note that the environment automatically reserves
% enough space to encompass the picture.

% Karl 很高兴地注意到环境会自动保留足够的空间来容纳图片。



% \subsubsection{Setting up the Environment in Plain \TeX%
% \\在 Plain \TeX 中设置环境
% }

% Karl's wife Gerda, who also happens to be a math teacher, is not a \LaTeX\
% user, but uses plain \TeX\ since she prefers to do things ``the old way''. She
% can also use \tikzname. Instead of |\usepackage{tikz}| she has to write
% |\input tikz.tex| and instead of |\begin{tikzpicture}| she writes
% |\tikzpicture| and instead of |\end{tikzpicture}| she writes |\endtikzpicture|.

% Karl 的妻子 Gerda 也是一名数学教师,她不使用 \LaTeX,而是使用 Plain \TeX,因为她更喜欢“老派”的方式。她也可以使用 \tikzname。她需要将 |\usepackage{tikz}| 替换为 |\input tikz.tex|,将 |\begin{tikzpicture}| 替换为 |\tikzpicture|,将 |\end{tikzpicture}| 替换为 |\endtikzpicture|。



% Thus, she would use:

% 因此,她将使用以下代码:

% %
% \begin{codeexample}[code only]
% %% Plain TeX file
% \input tikz.tex
% \baselineskip=12pt
% \hsize=6.3truein
% \vsize=8.7truein
% We are working on
% \tikzpicture
%   \draw (-1.5,0) -- (1.5,0);
%   \draw (0,-1.5) -- (0,1.5);
% \endtikzpicture.
% \bye
% \end{codeexample}

% Gerda can typeset this file using either |pdftex| or |tex| together with
% |dvips|. \tikzname\ will automatically discern which driver she is using. If
% she wishes to use |dvipdfm| together with |tex|, she either needs to modify the
% file |pgf.cfg| or can write |\def\pgfsysdriver{pgfsys-dvipdfm.def}| somewhere
% \emph{before} she inputs |tikz.tex| or |pgf.tex|.

% Gerda 可以使用 |pdftex| 或 |tex| 与 |dvips| 来排版此文件。\tikzname 会自动判断她使用的是哪个驱动程序。如果她希望使用 |dvipdfm| 与 |tex| 一起使用,她需要修改文件 |pgf.cfg|,或者可以在输入 |tikz.tex| 或 |pgf.tex| 之前的某个地方写上 |\def\pgfsysdriver{pgfsys-dvipdfm.def}|。



% \subsubsection{Setting up the Environment in Con\TeX t\\在 Con\TeX t 中设置环境}

% Karl's uncle Hans uses Con\TeX t. Like Gerda, Hans can also use \tikzname.
% Instead of |\usepackage{tikz}| he says |\usemodule[tikz]|. Instead of
% |\begin{tikzpicture}| he writes |\starttikzpicture| and  instead of
% |\end{tikzpicture}| he writes |\stoptikzpicture|.

% % Karl 的叔叔 Hans 使用 Con\subsubsection{在 Con\TeX t 中设置环境}

% Karl 的叔叔 Hans 使用 Con\TeX t。和 Gerda 一样,Hans 也可以使用 \tikzname。他需要将 |\usepackage{tikz}| 替换为 |\usemodule[tikz]|。将 |\begin{tikzpicture}| 替换为 |\starttikzpicture|,将 |\end{tikzpicture}| 替换为 |\stoptikzpicture|。



% His version of the example looks like this:
% %

% 他的示例代码如下:

% \begin{codeexample}[code only]
% %% ConTeXt file
% \usemodule[tikz]

% \starttext
%   We are working on
%   \starttikzpicture
%     \draw (-1.5,0) -- (1.5,0);
%     \draw (0,-1.5) -- (0,1.5);
%   \stoptikzpicture.
% \stoptext
% \end{codeexample}

% Hans will now typeset this file in the usual way using |texexec| or |context|.


% Hans 可以使用 |texexec| 或 |context| 来排版这个文件。

