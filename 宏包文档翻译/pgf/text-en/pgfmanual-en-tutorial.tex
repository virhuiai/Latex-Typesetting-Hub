

\subsection{Pics: The Angle Revisited\\图像:角度再探}

Karl expects that the code of certain parts of the picture he created might be
so useful that he might wish to reuse them in the future. A natural thing to do
is to create \TeX\ macros that store the code he wishes to reuse. However,
\tikzname\ offers another way that is integrated directly into its parser:
pics!

Karl预计他创建的图片的某些部分的代码可能非常有用,他未来可能希望重用它们。一个自然的做法是创建\TeX\ 宏来存储他希望重用的代码。然而,\tikzname\ 提供了另一种直接集成到其解析器中的方法:pics(图片)!


A ``pic'' is ``not quite a full picture'', hence the short name. The idea is
that a pic is simply some code that you can add to a picture at different
places using the |pic| command whose syntax is almost identical to the |node|
command. The main difference is that instead of specifying some text in curly
braces that should be shown, you specify the name of a predefined picture that
should be shown.

pic``(图片)是不完整的图片'',因此名字很短。这个想法是,pic只是一些代码,你可以使用与|node|命令几乎相同的语法,将它们添加到不同的位置的图片中。主要区别在于,你不是在花括号中指定要显示的某些文本,而是指定应该显示的预定义图片的名称。


Defining new pics is easy enough, see Section~\ref{section-pics}, but right now
we just want to use one such predefined pic: the |angle| pic. As the name
suggests, it is a small drawing of an angle consisting of a little wedge and an
arc together with some text (Karl needs to load the |angles| library and the
|quotes| for the following examples). What makes this pic useful is the fact
that the size of the wedge will be computed automatically.

定义新的pic非常简单,请参见第~\ref{section-pics}节,但现在我们只想使用一个这样的预定义pic:|angle|(角度)pic。顾名思义,它是一个小角度的绘图,由一个小楔和一段弧线以及一些文本组成(Karl需要加载|angles|库和|quotes|用于以下示例)。使得这个pic有用的是楔形的尺寸将自动计算。


The |angle| pic draws an angle between the two lines $BA$ and $BC$, where $A$,
$B$, and $C$ are three coordinates. In our case, $B$ is the origin, $A$ is
somewhere on the $x$-axis and $C$ is somewhere on a line at $30^\circ$.
%

|angle|(角度)pic在两条线$BA$和$BC$之间绘制一个角度,其中$A$,$B$和$C$是三个坐标。在我们的例子中,$B$是原点,$A$在$x$轴上的某处,$C$在$30^\circ$处的一条线上。

\begin{codeexample}[preamble={\usetikzlibrary{angles,quotes}}]
\begin{tikzpicture}[scale=3]
  \coordinate (A) at (1,0);
  \coordinate (B) at (0,0);
  \coordinate (C) at (30:1cm);

  \draw (A) -- (B) -- (C)
        pic [draw=green!50!black, fill=green!20, angle radius=9mm,
             "$\alpha$"] {angle = A--B--C};
\end{tikzpicture}
\end{codeexample}

Let us see, what is happening here. First we have specified three
\emph{coordinates} using the |\coordinate| command. It allows us to name a
specific coordinate in the picture. Then comes something that starts as a
normal |\draw|, but then comes the |pic| command. This command gets lots of
options and, in curly braces, comes the most important point: We specify that
we want to add an |angle| pic and this angle should be between the points we
named |A|, |B|, and |C| (we could use other names). Note that the text that we
want to be shown in the pic is specified in quotes inside the options of the
|pic|, not inside the curly braces.

让我们看看这里发生了什么。首先,我们使用|\coordinate|命令指定了三个\emph{坐标}。它允许我们为图片中的特定坐标命名。然后就是一个看起来像普通的|\draw|命令,但接着是|pic|命令。这个命令有很多选项,在花括号中最重要的是:我们指定要添加一个|angle|(角度)pic,并且这个角度应该在我们命名为|A|、|B|和|C|的点之间(我们可以使用其他名称)。请注意,我们希望在pic中显示的文本是在|pic|的选项中用引号指定的,而不是在花括号中。



To learn more about pics, please see Section~\ref{section-pics}.

要了解有关pics的更多信息,请参见第~\ref{section-pics}节。

