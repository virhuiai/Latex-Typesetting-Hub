

\subsection{Curved Path Construction\\曲线路径构建}

The next thing Karl wants to do is to draw the circle. For this, straight lines
obviously will not do. Instead, we need some way to draw curves. For this,
\tikzname\ provides a special syntax. One or two ``control points'' are needed.
The math behind them is not quite trivial, but here is the basic idea: Suppose
you are at point $x$ and the first control point is $y$. Then the curve will
start ``going in the direction of~$y$ at~$x$'', that is, the tangent of the
curve at $x$ will point toward~$y$. Next, suppose the curve should end at $z$
and the second support point is $w$. Then the curve will, indeed, end at $z$
and the tangent of the curve at point $z$ will go through $w$.

卡尔接下来想要做的是绘制圆。显然,直线无法实现这一点。相反,我们需要一种方法来绘制曲线。为此,\tikzname 提供了一种特殊的语法。需要一个或两个“控制点”。它们背后的数学并不是非常复杂,但基本思想是:假设你在点 $x$,第一个控制点是 $y$。那么曲线将在“以 $x$ 为中心向 $y$ 的方向开始”,也就是说,曲线在点 $x$ 的切线将指向 $y$。接下来,假设曲线应该在 $z$ 结束,第二个支撑点是 $w$。那么曲线确实会在 $z$ 结束,并且曲线在点 $z$ 的切线将经过 $w$。

Here is an example (the control points have been added for clarity):

这是一个示例(为了清晰起见,添加了控制点):
%
\begin{codeexample}[]
\begin{tikzpicture}
  \filldraw [gray] (0,0) circle [radius=2pt]
                   (1,1) circle [radius=2pt]
                   (2,1) circle [radius=2pt]
                   (2,0) circle [radius=2pt];
  \draw (0,0) .. controls (1,1) and (2,1) .. (2,0);
\end{tikzpicture}
\end{codeexample}

The general syntax for extending a path in a ``curved'' way is |.. controls|
\meta{first control point} |and| \meta{second control point} |..|
\meta{end point}. You can leave out the |and| \meta{second control point},
which causes the first one to be used twice.

通过使用 |.. controls| \meta{第一个控制点} |and| \meta{第二个控制点} |..| \meta{结束点} 的一般语法,可以以“曲线”的方式扩展路径。可以省略 |and| \meta{第二个控制点},这样将使用第一个控制点两次。

% 在LaTeX中,这个命令是用于在TikZ库中绘制一条贝塞尔曲线。TikZ是一个非常强大的工具,用于创建矢量图形。

% \draw (0,0) .. controls (1,1) and (2,1) .. (2,0); 这行代码的意思是从点 (0,0) 到点 (2,0) 绘制一条贝塞尔曲线,其中 (1,1) 和 (2,1) 是控制点。

% 下面是对这个命令的详细解读:

% \draw: TikZ中的一个命令,它指示LaTeX开始绘制一条线或曲线。

% (0,0): 这是你要开始绘制的曲线的起始点,坐标是 (0,0)。

% .. controls (1,1) and (2,1) ..: 这告诉LaTeX你要绘制的是一条贝塞尔曲线,而不是直线。关键词 controls 后面的两个点 (1,1) 和 (2,1) 是控制曲线形状的控制点。

% (2,0): 这是你要结束绘制的曲线的终点,坐标是 (2,0)。

% -----

% 贝塞尔曲线是一种数学曲线,用于描述平滑的曲线路径。它由数个控制点确定,通过这些控制点的位置和权重,可以定义曲线的形状。

% 贝塞尔曲线最常见的形式是二次贝塞尔曲线和三次贝塞尔曲线。二次贝塞尔曲线由三个控制点确定,分别为起始点、终点和一个中间点。三次贝塞尔曲线由四个控制点确定,分别为起始点、终点和两个中间点。

% 贝塞尔曲线的形状由控制点之间的插值算法确定。在二次贝塞尔曲线中,起始点和终点是曲线的端点,而中间点则会影响曲线的弯曲程度和方向。在三次贝塞尔曲线中,除了起始点和终点外,两个中间点的位置和权重也会影响曲线的形状。

% 贝塞尔曲线具有平滑性和局部控制性的特点。通过调整控制点的位置和权重,可以创建各种形状的曲线,包括弧线、圆弧、S形曲线等。贝塞尔曲线在计算机图形学、计算机辅助设计(CAD)、动画和游戏开发等领域得到广泛应用。

% 在绘图软件和绘图库中,如Adobe Illustrator、Inkscape和TikZ等,贝塞尔曲线通常由控制点和曲线类型(二次或三次)来定义。通过调整控制点的位置,可以实时预览和修改曲线的形状,从而满足具体的设计需求。

% ---
% 贝塞尔曲线(Bézier curve)是计算机图形学中常用的一种参数曲线。其主要思想是使用控制点来approximate一个曲线。

% 贝塞尔曲线的一般形式为:

% B(t) = ∑_{i=0}^n B_i^n(t) p_i

% 其中p_i是控制点,B_i^n(t)是伯恩斯坦多项式(Bernstein polynomial)。

% 贝塞尔曲线具有以下几个特点:

% 通过移动控制点,可以方便地改变曲线的形状。

% 曲线必然通过第一个和最后一个控制点。

% 曲线在控制点附近的形状主要由相邻的几个控制点决定。

% 随着控制点数量的增加,贝塞尔曲线可以逼近任意形状。

% 贝塞尔曲线是可微的并且连续。

% 贝塞尔曲线计算简单,适合在计算机上实现。

% 贝塞尔曲线可以方便地进行仿射变换。

% 贝塞尔曲线在许多图形软件中都有应用,如Adobe Illustrator和Inkscape等。它为设计师提供了一个简单直观的方法来绘制平滑曲线。

% --- bard

% 贝塞尔曲线是数学中一种用于描述平面曲线的曲线。它由一系列控制点定义,控制点是曲线沿其路径的特定位置。贝塞尔曲线可以用来生成各种各样的曲线,包括直线、圆弧、抛物线等。

% 贝塞尔曲线的定义如下:

% ```
% B(t) = (1 - t)^3 * P_0 + 3t(1 - t)^2 * P_1 + 3t^2(1 - t) * P_2 + t^3 * P_3
% ```

% 其中:

% * $t$ 是参数,取值范围为 $0 \le t \le 1$。
% * $P_0$、$P_1$、$P_2$、$P_3$ 是控制点,分别位于曲线的起点、第一个控制点、第二个控制点和终点。

% 贝塞尔曲线的性质如下:

% * 贝塞尔曲线是光滑的,即曲线在任何点的切线都是连续的。
% * 贝塞尔曲线可以通过控制点进行任意形状的变形。
% * 贝塞尔曲线可以用来生成任意阶的多项式曲线。

% 贝塞尔曲线在计算机图形学中有着广泛的应用,例如用于绘制图形、生成动画等。

% **贝塞尔曲线的应用**

% 贝塞尔曲线在计算机图形学中有着广泛的应用,例如用于绘制图形、生成动画等。

% * 绘制图形:贝塞尔曲线可以用于绘制各种各样的图形,例如线条、曲线、图形等。
% * 生成动画:贝塞尔曲线可以用于生成动画,例如人物动画、物体动画等。
% * 图像处理:贝塞尔曲线可以用于图像处理,例如图像变形、图像合成等。

% **贝塞尔曲线的优点**

% 贝塞尔曲线具有以下优点:

% * 光滑:贝塞尔曲线是光滑的,即曲线在任何点的切线都是连续的。
% * 可变形:贝塞尔曲线可以通过控制点进行任意形状的变形。
% * 通用性:贝塞尔曲线可以用来生成任意阶的多项式曲线。

% **贝塞尔曲线的缺点**

% 贝塞尔曲线具有以下缺点:

% * 计算复杂:贝塞尔曲线的计算比较复杂。
% * 存储空间大:贝塞尔曲线的存储空间比较大。

% **总结**

% 贝塞尔曲线是一种重要的数学曲线,在计算机图形学中有着广泛的应用。它具有光滑、可变形和通用性的优点,但计算复杂和存储空间大的缺点。

So, Karl can now add the first half circle to the picture:

因此,卡尔现在可以将第一个半圆添加到图片中:


%
\begin{codeexample}[]
\begin{tikzpicture}
  \draw (-1.5,0) -- (1.5,0);
  \draw (0,-1.5) -- (0,1.5);
  \draw (-1,0) .. controls (-1,0.555) and (-0.555,1) .. (0,1)
               .. controls (0.555,1) and (1,0.555) .. (1,0);
\end{tikzpicture}
\end{codeexample}

Karl is happy with the result, but finds specifying circles in this way to be
extremely awkward. Fortunately, there is a much simpler way.

Karl对结果感到满意,但他发现用这种方式指定圆非常麻烦。幸运的是,有一种更简单的方法。


\subsection{Circle Path Construction\\绘制圆形路径}

In order to draw a circle, the path construction operation |circle| can be
used. This operation is followed by a radius in brackets as in the following
example: (Note that the previous position is used as the \emph{center} of the
circle.)
%

为了绘制一个圆,可以使用路径构造操作|circle|。该操作后跟在括号内的半径,如下面的示例所示:(请注意,前一个位置被用作圆的\emph{中心}。)

\begin{codeexample}[]
\tikz \draw (0,0) circle [radius=10pt];
\end{codeexample}

You can also append an ellipse to the path using the |ellipse| operation.
Instead of a single radius you can specify two of them:
%

您还可以使用|ellipse|操作将椭圆附加到路径上。您可以指定两个半径,而不是一个:

\begin{codeexample}[]
\tikz \draw (0,0) ellipse [x radius=20pt, y radius=10pt];
\end{codeexample}

To draw an ellipse whose axes are not horizontal and vertical, but point in an
arbitrary direction (a ``turned ellipse'' like \tikz \draw[rotate=30] (0,0)
ellipse [x radius=6pt, y radius=3pt];) you can use transformations, which are
explained later. The code for the little ellipse is
|\tikz \draw[rotate=30] (0,0) ellipse [x radius=6pt, y radius=3pt];|, by the
way.

要绘制一个轴不是水平和垂直的椭圆,而是指向任意方向的椭圆(一个``旋转椭圆'',如\tikz \draw[rotate=30] (0,0) ellipse [x radius=6pt, y radius=3pt];),您可以使用后面将解释的变换。顺便说一下,小椭圆的代码是|\tikz \draw[rotate=30] (0,0) ellipse [x radius=6pt, y radius=3pt];|。

So, returning to Karl's problem, he can write
|\draw (0,0) circle [radius=1cm];| to draw the circle:
%

因此,回到Karl的问题,他可以写|\draw (0,0) circle [radius=1cm];|来绘制圆:

\begin{codeexample}[]
\begin{tikzpicture}
  \draw (-1.5,0) -- (1.5,0);
  \draw (0,-1.5) -- (0,1.5);
  \draw (0,0) circle [radius=1cm];
\end{tikzpicture}
\end{codeexample}

At this point, Karl is a bit alarmed that the circle is so small when he wants
the final picture to be much bigger. He is pleased to learn that \tikzname\ has
powerful transformation options and scaling everything by a factor of three is
very easy. But let us leave the size as it is for the moment to save some
space.

此时,Karl有点担心当他希望最终的图片更大时,圆太小了。他很高兴地知道\tikzname\ 有强大的变换选项,将所有内容放大三倍非常容易。但为了节省空间,让我们暂时保持尺寸不变。


\mshowc{section}
\mshowc{subsection}
\mshowc{subsubsection}
\end{document}


\subsection{Rectangle Path Construction}

The next things we would like to have is the grid in the background. There are
several ways to produce it. For example, one might draw lots of rectangles.
Since rectangles are so common, there is a special syntax for them: To add a
rectangle to the current path, use the |rectangle| path construction operation.
This operation should be followed by another coordinate and will append a
rectangle to the path such that the previous coordinate and the next
coordinates are corners of the rectangle. So, let us add two rectangles to the
picture:
%
\begin{codeexample}[]
\begin{tikzpicture}
  \draw (-1.5,0) -- (1.5,0);
  \draw (0,-1.5) -- (0,1.5);
  \draw (0,0) circle [radius=1cm];
  \draw (0,0) rectangle (0.5,0.5);
  \draw (-0.5,-0.5) rectangle (-1,-1);
\end{tikzpicture}
\end{codeexample}

While this may be nice in other situations, this is not really leading anywhere
with Karl's problem: First, we would need an awful lot of these rectangles and
then there is the border that is not ``closed''.

So, Karl is about to resort to simply drawing four vertical and four horizontal
lines using the nice |\draw| command, when he learns that there is a |grid|
path construction operation.


\subsection{Grid Path Construction}

The |grid| path operation adds a grid to the current path. It will add lines
making up a grid that fills the rectangle whose one corner is the current point
and whose other corner is the point following the |grid| operation. For
example, the code |\tikz \draw[step=2pt] (0,0) grid (10pt,10pt);| produces
\tikz \draw[step=2pt] (0,0) grid (10pt,10pt);. Note how the optional argument
for |\draw| can be used to specify a grid width (there are also |xstep| and
|ystep| to define the steppings independently). As Karl will learn soon, there
are \emph{lots} of things that can be influenced using such options.

For Karl, the following code could be used:
%
\begin{codeexample}[]
\begin{tikzpicture}
  \draw (-1.5,0) -- (1.5,0);
  \draw (0,-1.5) -- (0,1.5);
  \draw (0,0) circle [radius=1cm];
  \draw[step=.5cm] (-1.4,-1.4) grid (1.4,1.4);
\end{tikzpicture}
\end{codeexample}

Having another look at the desired picture, Karl notices that it would be nice
for the grid to be more subdued. (His son told him that grids tend to be
distracting if they are not subdued.) To subdue the grid, Karl adds two more
options to the |\draw| command that draws the grid. First, he uses the color
|gray| for the grid lines. Second, he reduces the line width to |very thin|.
Finally, he swaps the ordering of the commands so that the grid is drawn first
and everything else on top.
%
\begin{codeexample}[]
\begin{tikzpicture}
  \draw[step=.5cm,gray,very thin] (-1.4,-1.4) grid (1.4,1.4);
  \draw (-1.5,0) -- (1.5,0);
  \draw (0,-1.5) -- (0,1.5);
  \draw (0,0) circle [radius=1cm];
\end{tikzpicture}
\end{codeexample}


\subsection{Adding a Touch of  Style}

Instead of the options |gray,very thin| Karl could also have said |help lines|.
\emph{Styles} are predefined sets of options that can be used to organize how a
graphic is drawn. By saying |help lines| you say ``use the style that I (or
someone else) has set for drawing help lines''. If Karl decides, at some later
point, that grids should be drawn, say, using the color |blue!50| instead of
|gray|, he could provide the following option somewhere:
%
\begin{codeexample}[code only]
help lines/.style={color=blue!50,very thin}
\end{codeexample}
%
The effect of this ``style setter'' is that in the current scope or environment
the |help lines| option has the same effect as |color=blue!50,very thin|.

Using styles makes your graphics code more flexible. You can change the way
things look easily in a consistent manner. Normally, styles are defined at the
beginning of a picture. However, you may sometimes wish to define a style
globally, so that all pictures of your document can use this style. Then you
can easily change the way all graphics look by changing this one style. In this
situation you can use the |\tikzset| command at the beginning of the document
as in
%
\begin{codeexample}[code only]
\tikzset{help lines/.style=very thin}
\end{codeexample}

To build a hierarchy of styles you can have one style use another. So in order
to define a style |Karl's grid| that is based on the |grid| style Karl could
say
%
\begin{codeexample}[code only]
\tikzset{Karl's grid/.style={help lines,color=blue!50}}
...
\draw[Karl's grid] (0,0) grid (5,5);
\end{codeexample}

Styles are made even more powerful by parametrization. This means that, like
other options, styles can also be used with a parameter. For instance, Karl
could parameterize his grid so that, by default, it is blue, but he could also
use another color.
%
\begin{codeexample}[code only]
\begin{tikzpicture}
  [Karl's grid/.style  ={help lines,color=#1!50},
   Karl's grid/.default=blue]

  \draw[Karl's grid]     (0,0) grid (1.5,2);
  \draw[Karl's grid=red] (2,0) grid (3.5,2);
\end{tikzpicture}
\end{codeexample}

 In this example, the definition of the style |Karl's grid| is given as an
 optional argument to the |{tikzpicture}| environment. Additional styles for other
 elements would follow after a comma. With many styles in effect, the optional
 argument of the environment may easily happen to be longer than the actual
 contents.

\subsection{Drawing Options}

Karl wonders what other options there are that influence how a path is drawn.
He saw already that the |color=|\meta{color} option can be used to set the
line's color. The option |draw=|\meta{color} does nearly the same, only it sets
the color for the lines only and a different color can be used for filling
(Karl will need this when he fills the arc for the angle).

He saw that the style |very thin| yields very thin lines. Karl is not really
surprised by this and neither is he surprised to learn that |thin| yields thin
lines,  |thick| yields thick lines, |very thick| yields very thick lines,
|ultra thick| yields really, really thick lines and |ultra thin| yields lines
that are so thin that low-resolution printers and displays will have trouble
showing them. He wonders what gives lines of ``normal'' thickness. It turns out
that |thin| is the correct choice, since it gives the same thickness as \TeX's
|\hrule| command. Nevertheless, Karl would like to know whether there is
anything ``in the middle'' between |thin| and |thick|. There is: |semithick|.

Another useful thing one can do with lines is to dash or dot them. For this,
the two styles |dashed| and |dotted| can be used, yielding \tikz[baseline]
\draw[dashed] (0,.5ex) -- ++(2em,0pt); and \tikz[baseline] \draw[dotted]
(0,.5ex) -- ++(2em,0pt);. Both options also exist in a loose and a dense
version, called |loosely dashed|, |densely dashed|, |loosely dotted|, and
|densely dotted|. If he really, really  needs to, Karl can also define much
more complex dashing patterns with the |dash pattern| option, but his son
insists that dashing is to be used with utmost care and mostly distracts.
Karl's son claims that complicated dashing patterns are evil. Karl's students
do not care about dashing patterns.


\subsection{Arc Path Construction}

Our next obstacle is to draw the arc for the angle. For this, the |arc| path
construction operation is useful, which draws part of a circle or ellipse. This
|arc| operation is followed by options in brackets that specify the arc. An
example would be \texttt{arc[start angle=10, end angle=80, radius=10pt]}, which
means exactly what it says. Karl obviously needs an arc from $0^\circ$ to
$30^\circ$. The radius should be something relatively small, perhaps around one
third of the circle's radius. When one uses the arc path construction
operation, the specified arc will be added with its starting point at the
current position. So, we first have to ``get there''.
%
\begin{codeexample}[]
\begin{tikzpicture}
  \draw[step=.5cm,gray,very thin] (-1.4,-1.4) grid (1.4,1.4);
  \draw (-1.5,0) -- (1.5,0);
  \draw (0,-1.5) -- (0,1.5);
  \draw (0,0) circle [radius=1cm];
  \draw (3mm,0mm) arc [start angle=0, end angle=30, radius=3mm];
\end{tikzpicture}
\end{codeexample}

Karl thinks this is really a bit small and he cannot continue unless he learns
how to do scaling. For this, he can add the |[scale=3]| option. He could add
this option to each |\draw| command, but that would be awkward. Instead, he
adds it to the whole environment, which causes this option to apply to
everything within.
%
\begin{codeexample}[]
\begin{tikzpicture}[scale=3]
  \draw[step=.5cm,gray,very thin] (-1.4,-1.4) grid (1.4,1.4);
  \draw (-1.5,0) -- (1.5,0);
  \draw (0,-1.5) -- (0,1.5);
  \draw (0,0) circle [radius=1cm];
  \draw (3mm,0mm) arc [start angle=0, end angle=30, radius=3mm];
\end{tikzpicture}
\end{codeexample}

As for circles, you can specify ``two'' radii in order to get an elliptical
arc.
%
\begin{codeexample}[]
  \tikz \draw (0,0)
    arc [start angle=0, end angle=315,
         x radius=1.75cm, y radius=1cm];
\end{codeexample}


\subsection{Clipping a Path}

In order to save space in this manual, it would be nice to clip Karl's graphics
a bit so that we can focus on the ``interesting'' parts. Clipping is pretty
easy in \tikzname. You can use the |\clip| command to clip all subsequent
drawing. It works like |\draw|, only it does not draw anything, but uses the
given path to clip everything subsequently.
%
\begin{codeexample}[]
\begin{tikzpicture}[scale=3]
  \clip (-0.1,-0.2) rectangle (1.1,0.75);
  \draw[step=.5cm,gray,very thin] (-1.4,-1.4) grid (1.4,1.4);
  \draw (-1.5,0) -- (1.5,0);
  \draw (0,-1.5) -- (0,1.5);
  \draw (0,0) circle [radius=1cm];
  \draw (3mm,0mm) arc [start angle=0, end angle=30, radius=3mm];
\end{tikzpicture}
\end{codeexample}

You can also do both at the same time: Draw \emph{and} clip a path. For this,
use the |\draw| command and add the |clip| option. (This is not the whole
picture: You can also use the |\clip| command and add the |draw| option. Well,
that is also not the whole picture: In reality, |\draw| is just a shorthand for
|\path[draw]| and |\clip| is a shorthand for |\path[clip]| and you could also
say |\path[draw,clip]|.) Here is an example:
%
\begin{codeexample}[]
\begin{tikzpicture}[scale=3]
  \clip[draw] (0.5,0.5) circle (.6cm);
  \draw[step=.5cm,gray,very thin] (-1.4,-1.4) grid (1.4,1.4);
  \draw (-1.5,0) -- (1.5,0);
  \draw (0,-1.5) -- (0,1.5);
  \draw (0,0) circle [radius=1cm];
  \draw (3mm,0mm) arc [start angle=0, end angle=30, radius=3mm];
\end{tikzpicture}
\end{codeexample}


\subsection{Parabola and Sine Path Construction}

Although Karl does not need them for his picture, he is pleased to learn that
there are |parabola| and |sin| and |cos| path operations for adding parabolas
and sine and cosine curves to the current path. For the |parabola| operation,
the current point will lie on the parabola as well as the point given after the
parabola operation. Consider the following example:
%
\begin{codeexample}[]
\tikz \draw (0,0) rectangle (1,1)  (0,0) parabola (1,1);
\end{codeexample}

It is also possible to place the bend somewhere else:
%
\begin{codeexample}[]
\tikz \draw[x=1pt,y=1pt] (0,0) parabola bend (4,16) (6,12);
\end{codeexample}

The operations |sin| and |cos| add a sine or cosine curve in the interval
$[0,\pi/2]$ such that the previous current point is at the start of the curve
and the curve ends at the given end point. Here are two examples:
%
\begin{codeexample}[]
A sine \tikz \draw[x=1ex,y=1ex] (0,0) sin (1.57,1); curve.
\end{codeexample}

\begin{codeexample}[]
\tikz \draw[x=1.57ex,y=1ex] (0,0) sin (1,1) cos (2,0) sin (3,-1) cos (4,0)
                            (0,1) cos (1,0) sin (2,-1) cos (3,0) sin (4,1);
\end{codeexample}


\subsection{Filling and Drawing}

Returning to the picture, Karl now wants the angle to be ``filled'' with a very
light green. For this he uses |\fill| instead of |\draw|. Here is what Karl
does:
%
\begin{codeexample}[]
\begin{tikzpicture}[scale=3]
  \clip (-0.1,-0.2) rectangle (1.1,0.75);
  \draw[step=.5cm,gray,very thin] (-1.4,-1.4) grid (1.4,1.4);
  \draw (-1.5,0) -- (1.5,0);
  \draw (0,-1.5) -- (0,1.5);
  \draw (0,0) circle [radius=1cm];
  \fill[green!20!white] (0,0) -- (3mm,0mm)
    arc [start angle=0, end angle=30, radius=3mm] -- (0,0);
\end{tikzpicture}
\end{codeexample}

The color |green!20!white| means 20\% green and 80\% white mixed together. Such
color expression are possible since \tikzname\ uses Uwe Kern's |xcolor|
package, see the documentation of that package for details on color
expressions.

What would have happened, if Karl had not ``closed'' the path using |--(0,0)|
at the end? In this case, the path is closed automatically, so this could have
been omitted. Indeed, it would even have been better to write the following,
instead:
%
\begin{codeexample}[code only]
  \fill[green!20!white] (0,0) -- (3mm,0mm)
    arc [start angle=0, end angle=30, radius=3mm] -- cycle;
\end{codeexample}
%
The |--cycle| causes the current path to be closed (actually the current part
of the current path) by smoothly joining the first and last point. To
appreciate the difference, consider the following example:
%
\begin{codeexample}[]
\begin{tikzpicture}[line width=5pt]
  \draw (0,0) -- (1,0) -- (1,1) -- (0,0);
  \draw (2,0) -- (3,0) -- (3,1) -- cycle;
  \useasboundingbox (0,1.5); % make bounding box higher
\end{tikzpicture}
\end{codeexample}

You can also fill and draw a path at the same time using the |\filldraw|
command. This will first draw the path, then fill it. This may not seem too
useful, but you can specify different colors to be used for filling and for
stroking. These are specified as optional arguments like this:
%
\begin{codeexample}[]
\begin{tikzpicture}[scale=3]
  \clip (-0.1,-0.2) rectangle (1.1,0.75);
  \draw[step=.5cm,gray,very thin] (-1.4,-1.4) grid (1.4,1.4);
  \draw (-1.5,0) -- (1.5,0);
  \draw (0,-1.5) -- (0,1.5);
  \draw (0,0) circle [radius=1cm];
  \filldraw[fill=green!20!white, draw=green!50!black] (0,0) -- (3mm,0mm)
    arc [start angle=0, end angle=30, radius=3mm] -- cycle;
\end{tikzpicture}
\end{codeexample}


\subsection{Shading}

Karl briefly considers the possibility of making the angle ``more fancy'' by
\emph{shading} it. Instead of filling the area with a uniform color, a smooth
transition between different colors is used. For this, |\shade| and
|\shadedraw|, for shading and drawing at the same time, can be used:
%
\begin{codeexample}[]
  \tikz \shade (0,0) rectangle (2,1)  (3,0.5) circle (.5cm);
\end{codeexample}
%
The default shading is a smooth transition from gray to white. To specify
different colors, you can use options:
%
\begin{codeexample}[]
\begin{tikzpicture}[rounded corners,ultra thick]
  \shade[top color=yellow,bottom color=black] (0,0) rectangle +(2,1);
  \shade[left color=yellow,right color=black] (3,0) rectangle +(2,1);
  \shadedraw[inner color=yellow,outer color=black,draw=yellow] (6,0) rectangle +(2,1);
  \shade[ball color=green] (9,.5) circle (.5cm);
\end{tikzpicture}
\end{codeexample}

For Karl, the following might be appropriate:
%
\begin{codeexample}[]
\begin{tikzpicture}[scale=3]
  \clip (-0.1,-0.2) rectangle (1.1,0.75);
  \draw[step=.5cm,gray,very thin] (-1.4,-1.4) grid (1.4,1.4);
  \draw (-1.5,0) -- (1.5,0);
  \draw (0,-1.5) -- (0,1.5);
  \draw (0,0) circle [radius=1cm];
  \shadedraw[left color=gray,right color=green, draw=green!50!black]
    (0,0) -- (3mm,0mm)
    arc [start angle=0, end angle=30, radius=3mm] -- cycle;
\end{tikzpicture}
\end{codeexample}

However, he wisely decides that shadings usually only distract without adding
anything to the picture.


\subsection{Specifying Coordinates}

Karl now wants to add the sine and cosine lines. He knows already that he can
use the |color=| option to set the lines' colors. So, what is the best way to
specify the coordinates?

There are different ways of specifying coordinates. The easiest way is to say
something like |(10pt,2cm)|. This means 10pt in $x$-direction and 2cm in
$y$-directions. Alternatively, you can also leave out the units as in |(1,2)|,
which means ``one times the current $x$-vector plus twice the current
$y$-vector''. These vectors default to 1cm in the $x$-direction and 1cm in the
$y$-direction, respectively.

In order to specify points in polar coordinates, use the notation |(30:1cm)|,
which means 1cm in direction 30 degree. This is obviously quite useful to ``get
to the point $(\cos 30^\circ,\sin 30^\circ)$ on the circle''.

You can add a single |+| sign in front of a coordinate or two of them as in
|+(0cm,1cm)| or |++(2cm,0cm)|. Such coordinates are interpreted differently:
The first form means ``1cm upwards from the previous specified position'' and
the second means ``2cm to the right of the previous specified position, making
this the new specified position''. For example, we can draw the sine line as
follows:
%
\begin{codeexample}[]
\begin{tikzpicture}[scale=3]
  \clip (-0.1,-0.2) rectangle (1.1,0.75);
  \draw[step=.5cm,gray,very thin] (-1.4,-1.4) grid (1.4,1.4);
  \draw (-1.5,0) -- (1.5,0);
  \draw (0,-1.5) -- (0,1.5);
  \draw (0,0) circle [radius=1cm];
  \filldraw[fill=green!20,draw=green!50!black] (0,0) -- (3mm,0mm)
      arc [start angle=0, end angle=30, radius=3mm] -- cycle;
  \draw[red,very thick] (30:1cm) -- +(0,-0.5);
\end{tikzpicture}
\end{codeexample}

Karl used the fact $\sin 30^\circ = 1/2$. However, he very much doubts that his
students know this, so it would be nice to have a way of specifying ``the point
straight down from |(30:1cm)| that lies on the $x$-axis''. This is, indeed,
possible using a special syntax: Karl can write \verb!(30:1cm |- 0,0)!. In
general, the meaning of |(|\meta{p}\verb! |- !\meta{q}|)| is ``the intersection
of a vertical line through $p$ and a horizontal line through $q$''.

Next, let us draw the cosine line. One way would be to say
\verb!(30:1cm |- 0,0) -- (0,0)!. Another way is the following: we ``continue''
from where the sine ends:
%
\begin{codeexample}[]
\begin{tikzpicture}[scale=3]
  \clip (-0.1,-0.2) rectangle (1.1,0.75);
  \draw[step=.5cm,gray,very thin] (-1.4,-1.4) grid (1.4,1.4);
  \draw (-1.5,0) -- (1.5,0);
  \draw (0,-1.5) -- (0,1.5);
  \draw (0,0) circle [radius=1cm];
  \filldraw[fill=green!20,draw=green!50!black] (0,0) -- (3mm,0mm)
      arc [start angle=0, end angle=30, radius=3mm] -- cycle;
  \draw[red,very thick]  (30:1cm) -- +(0,-0.5);
  \draw[blue,very thick] (30:1cm) ++(0,-0.5) -- (0,0);
\end{tikzpicture}
\end{codeexample}

Note that there is no |--| between |(30:1cm)| and |++(0,-0.5)|. In detail, this
path is interpreted as follows: ``First, the |(30:1cm)| tells me to move my pen
to $(\cos 30^\circ,1/2)$. Next, there comes another coordinate specification,
so I move my pen there without drawing anything. This new point is half a unit
down from the last position, thus it is at $(\cos 30^\circ,0)$. Finally, I move
the pen to the origin, but this time drawing something (because of the |--|).''

To appreciate the difference between |+| and |++| consider the following
example:
%
\begin{codeexample}[]
\begin{tikzpicture}
  \def\rectanglepath{-- ++(1cm,0cm)  -- ++(0cm,1cm)  -- ++(-1cm,0cm) -- cycle}
  \draw (0,0) \rectanglepath;
  \draw (1.5,0) \rectanglepath;
\end{tikzpicture}
\end{codeexample}

By comparison, when using a single |+|, the coordinates are different:
%
\begin{codeexample}[]
\begin{tikzpicture}
  \def\rectanglepath{-- +(1cm,0cm)  -- +(1cm,1cm)  -- +(0cm,1cm) -- cycle}
  \draw (0,0) \rectanglepath;
  \draw (1.5,0) \rectanglepath;
\end{tikzpicture}
\end{codeexample}


Naturally, all of this could have been written more clearly and more
economically like this (either with a single or a double |+|):
%
\begin{codeexample}[]
\tikz \draw (0,0) rectangle +(1,1)  (1.5,0) rectangle +(1,1);
\end{codeexample}


\subsection{Intersecting Paths}

Karl is left with the line for $\tan \alpha$, which seems difficult to specify
using transformations and polar coordinates. The first -- and easiest -- thing
he can do is so simply use the coordinate |(1,{tan(30)})| since \tikzname's
math engine knows how to compute things like |tan(30)|. Note the added braces
since, otherwise, \tikzname's parser would think that the first closing
parenthesis ends the coordinate (in general, you need to add braces around
components of coordinates when these components contain parentheses).

Karl can, however, also use a more elaborate, but also more ``geometric'' way
of computing the length of the orange line: He can specify intersections of
paths as coordinates. The line for $\tan \alpha$ starts at $(1,0)$ and goes
upward to a point that is at the intersection of a line going ``up'' and a line
going from the origin through |(30:1cm)|. Such computations are made available
by the |intersections| library.

What Karl must do is to create two ``invisible'' paths that intersect at the
position of interest. Creating paths that are not otherwise seen can be done
using the |\path| command without any options like |draw| or |fill|. Then, Karl
can add the |name path| option to the path for later reference. Once the paths
have been constructed, Karl can use the |name intersections| to assign names to
the coordinate for later reference.
%
\begin{codeexample}[code only]
\path [name path=upward line] (1,0) -- (1,1);
\path [name path=sloped line] (0,0) -- (30:1.5cm); % a bit longer, so that there is an intersection

% (add `\usetikzlibrary{intersections}' after loading tikz in the preamble)
\draw [name intersections={of=upward line and sloped line, by=x}]
  [very thick,orange] (1,0) -- (x);
\end{codeexample}


\subsection{Adding Arrow Tips}

Karl now wants to add the little arrow tips at the end of the axes. He has
noticed that in many plots, even in scientific journals, these arrow tips seem
to be missing, presumably because the generating programs cannot produce them.
Karl thinks arrow tips belong at the end of axes. His son agrees. His students
do not care about arrow tips.

It turns out that adding arrow tips is pretty easy: Karl adds the option |->|
to the drawing commands for the axes:
%
\begin{codeexample}[preamble={\usetikzlibrary{intersections}}]
\begin{tikzpicture}[scale=3]
  \clip (-0.1,-0.2) rectangle (1.1,1.51);
  \draw[step=.5cm,gray,very thin] (-1.4,-1.4) grid (1.4,1.4);
  \draw[->] (-1.5,0) -- (1.5,0);
  \draw[->] (0,-1.5) -- (0,1.5);
  \draw (0,0) circle [radius=1cm];
  \filldraw[fill=green!20,draw=green!50!black] (0,0) -- (3mm,0mm)
        arc [start angle=0, end angle=30, radius=3mm] -- cycle;
  \draw[red,very thick]    (30:1cm) -- +(0,-0.5);
  \draw[blue,very thick]   (30:1cm) ++(0,-0.5) -- (0,0);

  \path [name path=upward line] (1,0) -- (1,1);
  \path [name path=sloped line] (0,0) -- (30:1.5cm);
  \draw [name intersections={of=upward line and sloped line, by=x}]
        [very thick,orange] (1,0) -- (x);
\end{tikzpicture}
\end{codeexample}

If Karl had used the option |<-| instead of |->|, arrow tips would have been
put at the beginning of the path. The option |<->| puts arrow tips at both ends
of the path.

There are certain restrictions to the kind of paths to which arrow tips can be
added. As a rule of thumb, you can add arrow tips only to a single open
``line''. For example, you cannot add tips to, say, a rectangle or a circle.
However, you can add arrow tips to curved paths and to paths that have several
segments, as in the following examples:
%
\begin{codeexample}[]
\begin{tikzpicture}
  \draw [<->] (0,0) arc [start angle=180, end angle=30, radius=10pt];
  \draw [<->] (1,0) -- (1.5cm,10pt) -- (2cm,0pt) -- (2.5cm,10pt);
\end{tikzpicture}
\end{codeexample}

Karl has a more detailed look at the arrow that \tikzname\ puts at the end. It
looks like this when he zooms it: \tikz[baseline] \draw[->,line width=1pt]
(0pt,.5ex) -- ++(10pt,0pt);. The shape seems vaguely familiar and, indeed, this
is exactly the end of \TeX's standard arrow used in something like $f\colon A
\to B$.

Karl likes the arrow, especially since it is not ``as thick'' as the arrows
offered by many other packages. However, he expects that, sometimes, he might
need to use some other kinds of arrow. To do so, Karl can say |>=|\meta{kind of
end arrow tip}, where \meta{kind of end arrow tip} is a special arrow tip
specification. For example, if Karl says |>=Stealth|, then he tells \tikzname\
that he would like  ``stealth-fighter-like'' arrow tips:
\todosp{remaining instance of bug \#473}
%
\begin{codeexample}[preamble={\usetikzlibrary{arrows.meta}}]
\begin{tikzpicture}[>=Stealth]
  \draw [->] (0,0) arc [start angle=180, end angle=30, radius=10pt];
  \draw [<<-,very thick] (1,0) -- (1.5cm,10pt) -- (2cm,0pt) -- (2.5cm,10pt);
\end{tikzpicture}
\end{codeexample}

Karl wonders whether such a military name for the arrow type is really
necessary. He is not really mollified when his son tells him that Microsoft's
PowerPoint uses the same name. He decides to have his students discuss this at
some point.

In addition to |Stealth| there are several other predefined kinds of arrow tips
Karl can choose from, see Section~\ref{section-arrows}. Furthermore, he can
define arrows types himself, if he needs new ones.


\subsection{Scoping}

Karl saw already that there are numerous graphic options that affect how paths
are rendered. Often, he would like to apply certain options to a whole set of
graphic commands. For example, Karl might wish to draw three paths using a
|thick| pen, but would like everything else to be drawn ``normally''.

If Karl wishes to set a certain graphic option for the whole picture, he can
simply pass this option to the |\tikz| command or to the |{tikzpicture}|
environment (Gerda would pass the options to |\tikzpicture| and Hans passes
them to |\starttikzpicture|). However, if Karl wants to apply graphic options
to a local group, he put these commands inside a |{scope}| environment (Gerda
uses |\scope| and |\endscope|, Hans uses |\startscope| and |\stopscope|). This
environment takes graphic options as an optional argument and these options
apply to everything inside the scope, but not to anything outside.

Here is an example:
%
\begin{codeexample}[]
\begin{tikzpicture}[ultra thick]
  \draw (0,0) -- (0,1);
  \begin{scope}[thin]
    \draw (1,0) -- (1,1);
    \draw (2,0) -- (2,1);
  \end{scope}
  \draw (3,0) -- (3,1);
\end{tikzpicture}
\end{codeexample}

Scoping has another interesting effect: Any changes to the clipping area are
local to the scope. Thus, if you say |\clip| somewhere inside a scope, the
effect of the |\clip| command ends at the end of the scope. This is useful
since there is no other way of ``enlarging'' the clipping area.

Karl has also already seen that giving options to commands like |\draw| apply
only to that command. It turns out that the situation is slightly more complex.
First, options to a command like |\draw| are not really options to the command,
but they are ``path options'' and can be given anywhere on the path. So,
instead of |\draw[thin] (0,0) -- (1,0);| one can also write
|\draw (0,0) [thin] -- (1,0);| or |\draw (0,0) -- (1,0) [thin];|; all of these
have the same effect. This might seem strange since in the last case, it would
appear that the |thin| should take effect only ``after'' the line from $(0,0)$
to $(1,0)$ has been drawn. However, most graphic options only apply to the
whole path. Indeed, if you say both |thin| and |thick| on the same path, the
last option given will ``win''.

When reading the above, Karl notices that only ``most'' graphic options apply
to the whole path. Indeed, all transformation options do \emph{not} apply to
the whole path, but only to ``everything following them on the path''. We will
have a more detailed look at this in a moment. Nevertheless, all options given
during a path construction apply only to this path.


\subsection{Transformations}

When you specify a  coordinate like |(1cm,1cm)|, where is that coordinate
placed on the page? To determine the position, \tikzname, \TeX, and
\textsc{pdf} or PostScript all apply certain transformations to the given
coordinate in order to determine the final position on the page.

\tikzname\ provides numerous options that allow you to transform coordinates in
\tikzname's private coordinate system. For example, the |xshift| option allows
you to shift all subsequent points by a certain amount:

\begin{codeexample}[]
\tikz \draw (0,0) -- (0,0.5) [xshift=2pt] (0,0) -- (0,0.5);
\end{codeexample}

It is important to note that you can change transformation ``in the middle of a
path'', a feature that is not supported by \pdf\ or PostScript. The reason is
that \tikzname\ keeps track of its own transformation matrix.

Here is a more complicated example:
%
\begin{codeexample}[]
\begin{tikzpicture}[even odd rule,rounded corners=2pt,x=10pt,y=10pt]
  \filldraw[fill=yellow!80!black] (0,0)   rectangle (1,1)
        [xshift=5pt,yshift=5pt]   (0,0)   rectangle (1,1)
                    [rotate=30]   (-1,-1) rectangle (2,2);
\end{tikzpicture}
\end{codeexample}

The most useful transformations are |xshift| and |yshift| for shifting, |shift|
for shifting to a given point as in |shift={(1,0)}| or |shift={+(0,0)}| (the
braces are necessary so that \TeX\ does not mistake the comma for separating
options), |rotate| for rotating by a certain angle (there is also a
|rotate around| for rotating around a given point), |scale| for scaling by a
certain factor, |xscale| and |yscale| for scaling only in the $x$- or
$y$-direction (|xscale=-1| is a flip), and |xslant| and |yslant| for slanting.
If these transformation and those that I have not mentioned are not sufficient,
the |cm| option allows you to apply an arbitrary transformation matrix. Karl's
students, by the way, do not know what a transformation matrix is.


\subsection{Repeating Things: For-Loops}

Karl's next aim is to add little ticks on the axes at positions $-1$, $-1/2$,
$1/2$, and $1$. For this, it would be nice to use some kind of ``loop'',
especially since he wishes to do the same thing at each of these positions.
There are different packages for doing this. \LaTeX\ has its own internal
command for this, |pstricks| comes along with the powerful |\multido| command.
All of these can be used together with \tikzname, so if you are familiar with
them, feel free to use them. \tikzname\ introduces yet another command, called
|\foreach|, which I introduced since I could never remember the syntax of the
other packages. |\foreach| is defined in the package |pgffor| and can be used
independently of \tikzname, but \tikzname\ includes it automatically.

In its basic form, the |\foreach| command is easy to use:
%
\begin{codeexample}[]
\foreach \x in {1,2,3} {$x =\x$, }
\end{codeexample}

The general syntax is
|\foreach| \meta{variable}| in {|\meta{list of values}|} |\meta{commands}.
Inside the \meta{commands}, the \meta{variable} will be assigned to the
different values. If the \meta{commands} do not start with a brace, everything
up to the next semicolon is used as \meta{commands}.

For Karl and the ticks on the axes, he could use the following code:
%
\begin{codeexample}[]
\begin{tikzpicture}[scale=3]
  \clip (-0.1,-0.2) rectangle (1.1,1.51);
  \draw[step=.5cm,gray,very thin] (-1.4,-1.4) grid (1.4,1.4);
  \filldraw[fill=green!20,draw=green!50!black] (0,0) -- (3mm,0mm)
      arc [start angle=0, end angle=30, radius=3mm] -- cycle;
  \draw[->] (-1.5,0) -- (1.5,0);
  \draw[->] (0,-1.5) -- (0,1.5);
  \draw (0,0) circle [radius=1cm];

  \foreach \x in {-1cm,-0.5cm,1cm}
    \draw (\x,-1pt) -- (\x,1pt);
  \foreach \y in {-1cm,-0.5cm,0.5cm,1cm}
    \draw (-1pt,\y) -- (1pt,\y);
\end{tikzpicture}
\end{codeexample}

As a matter of fact, there are many different ways of creating the ticks. For
example, Karl could have put the |\draw ...;| inside curly braces. He could
also have used, say,
%
\begin{codeexample}[code only]
\foreach \x in {-1,-0.5,1}
  \draw[xshift=\x cm] (0pt,-1pt) -- (0pt,1pt);
\end{codeexample}

Karl is curious what would happen in a more complicated situation where there
are, say, 20 ticks. It seems bothersome to explicitly mention all these numbers
in the set for |\foreach|. Indeed, it is possible to use |...| inside the
|\foreach| statement to iterate over a large number of values (which must,
however, be dimensionless real numbers) as in the following example:
%
\begin{codeexample}[]
\tikz \foreach \x in {1,...,10}
        \draw (\x,0) circle (0.4cm);
\end{codeexample}

If you provide \emph{two} numbers before the |...|, the |\foreach| statement
will use their difference for the stepping:
%
\begin{codeexample}[]
\tikz \foreach \x in {-1,-0.5,...,1}
       \draw (\x cm,-1pt) -- (\x cm,1pt);
\end{codeexample}

We can also nest loops to create interesting effects:
%
\begin{codeexample}[]
\begin{tikzpicture}
  \foreach \x in {1,2,...,5,7,8,...,12}
    \foreach \y in {1,...,5}
    {
      \draw (\x,\y) +(-.5,-.5) rectangle ++(.5,.5);
      \draw (\x,\y) node{\x,\y};
    }
\end{tikzpicture}
\end{codeexample}

The |\foreach| statement can do even trickier stuff, but the above gives the
idea.


\subsection{Adding Text}

Karl is, by now, quite satisfied with the picture. However, the most important
parts, namely the labels, are still missing!

\tikzname\ offers an easy-to-use and powerful system for adding text and, more
generally, complex shapes to a picture at specific positions. The basic idea is
the following: When \tikzname\ is constructing a path and encounters the
keyword |node| in the middle of a path, it reads a \emph{node specification}.
The keyword |node| is typically followed by some options and then some text
between curly braces. This text is put inside a normal \TeX\ box (if the node
specification directly follows a coordinate, which is usually the case,
\tikzname\ is able to perform some magic so that it is even possible to use
verbatim text inside the boxes) and then placed at the current position, that
is, at the last specified position (possibly shifted a bit, according to the
given options). However, all nodes are drawn only after the path has been
completely drawn/filled/shaded/clipped/whatever.
%
\begin{codeexample}[]
\begin{tikzpicture}
  \draw (0,0) rectangle (2,2);
  \draw (0.5,0.5) node [fill=yellow!80!black]
                       {Text at \verb!node 1!}
     -- (1.5,1.5) node {Text at \verb!node 2!};
\end{tikzpicture}
\end{codeexample}

Obviously, Karl would not only like to place nodes \emph{on} the last specified
position, but also to the left or the right of these positions. For this, every
node object that you put in your picture is equipped with several
\emph{anchors}. For example, the |north| anchor is in the middle at the upper
end of the shape, the |south| anchor is at the bottom and the |north east|
anchor is in the upper right corner. When you give the option |anchor=north|,
the text will be placed such that this northern anchor will lie on the current
position and the text is, thus, below the current position. Karl uses this to
draw the ticks as follows:
%
\begin{codeexample}[]
\begin{tikzpicture}[scale=3]
  \clip (-0.6,-0.2) rectangle (0.6,1.51);
  \draw[step=.5cm,help lines] (-1.4,-1.4) grid (1.4,1.4);
  \filldraw[fill=green!20,draw=green!50!black] (0,0) -- (3mm,0mm)
    arc [start angle=0, end angle=30, radius=3mm] -- cycle;
  \draw[->] (-1.5,0) -- (1.5,0);   \draw[->] (0,-1.5) -- (0,1.5);
  \draw (0,0) circle [radius=1cm];

  \foreach \x in {-1,-0.5,1}
    \draw (\x cm,1pt) -- (\x cm,-1pt) node[anchor=north] {$\x$};
  \foreach \y in {-1,-0.5,0.5,1}
    \draw (1pt,\y cm) -- (-1pt,\y cm) node[anchor=east] {$\y$};
\end{tikzpicture}
\end{codeexample}

This is quite nice, already. Using these anchors, Karl can now add most of the
other text elements. However, Karl thinks that, though ``correct'', it is quite
counter-intuitive that in order to place something \emph{below} a given point,
he has to use the \emph{north} anchor. For this reason, there is an option
called |below|, which does the same as |anchor=north|. Similarly, |above right|
does the same as |anchor=south west|. In addition, |below| takes an optional
dimension argument. If given, the shape will additionally be shifted downwards
by the given amount. So, |below=1pt| can be used to put a text label below some
point and, additionally shift it  1pt downwards.

Karl is not quite satisfied with the ticks. He would like to have $1/2$ or
$\frac{1}{2}$ shown instead of $0.5$, partly to show off the nice capabilities
of \TeX\ and \tikzname, partly because for positions like $1/3$ or $\pi$ it is
certainly very much preferable to have the ``mathematical'' tick there instead
of just the ``numeric'' tick. His students, on the other hand, prefer $0.5$
over $1/2$ since they are not too fond of fractions in general.

Karl now faces a problem: For the |\foreach| statement, the position |\x|
should still be given as |0.5| since \tikzname\ will not know where
|\frac{1}{2}| is supposed to be. On the other hand, the typeset text should
really be  |\frac{1}{2}|. To solve this problem, |\foreach| offers a special
syntax: Instead of having one variable |\x|, Karl can specify two (or even
more) variables separated by a slash as in |\x / \xtext|. Then, the elements in
the set over which |\foreach| iterates must also be of the form
\meta{first}|/|\meta{second}. In each iteration, |\x| will be set to
\meta{first} and |\xtext| will be set to \meta{second}. If no \meta{second} is
given, the \meta{first} will be used again. So, here is the new code for the
ticks:
%
\begin{codeexample}[]
\begin{tikzpicture}[scale=3]
  \clip (-0.6,-0.2) rectangle (0.6,1.51);
  \draw[step=.5cm,help lines] (-1.4,-1.4) grid (1.4,1.4);
  \filldraw[fill=green!20,draw=green!50!black] (0,0) -- (3mm,0mm)
      arc [start angle=0, end angle=30, radius=3mm] -- cycle;
  \draw[->] (-1.5,0) -- (1.5,0); \draw[->] (0,-1.5) -- (0,1.5);
  \draw (0,0) circle [radius=1cm];

  \foreach \x/\xtext in {-1, -0.5/-\frac{1}{2}, 1}
    \draw (\x cm,1pt) -- (\x cm,-1pt) node[anchor=north] {$\xtext$};
  \foreach \y/\ytext in {-1, -0.5/-\frac{1}{2}, 0.5/\frac{1}{2}, 1}
    \draw (1pt,\y cm) -- (-1pt,\y cm) node[anchor=east] {$\ytext$};
\end{tikzpicture}
\end{codeexample}

Karl is quite pleased with the result, but his son points out that this is
still not perfectly satisfactory: The grid and the circle interfere with the
numbers and decrease their legibility. Karl is not very concerned by this (his
students do not even notice), but his son insists that there is an easy
solution: Karl can add the |[fill=white]| option to fill out the background of
the text shape with a white color.

The next thing Karl wants to do is to add the labels like $\sin \alpha$. For
this, he would like to place a label ``in the middle of the line''. To do so,
instead of specifying the label |node {$\sin\alpha$}|  directly after one of
the endpoints of the line (which would place the label at that endpoint), Karl
can give the label directly after the |--|, before the coordinate. By default,
this places the label in the middle of the line, but the |pos=| options can be
used to modify this. Also, options like |near start| and |near end| can be used
to modify this position:
%
\begin{codeexample}[preamble={\usetikzlibrary{intersections}}]
\begin{tikzpicture}[scale=3]
  \clip (-2,-0.2) rectangle (2,0.8);
  \draw[step=.5cm,gray,very thin] (-1.4,-1.4) grid (1.4,1.4);
  \filldraw[fill=green!20,draw=green!50!black] (0,0) -- (3mm,0mm)
    arc [start angle=0, end angle=30, radius=3mm] -- cycle;
  \draw[->] (-1.5,0) -- (1.5,0) coordinate (x axis);
  \draw[->] (0,-1.5) -- (0,1.5) coordinate (y axis);
  \draw (0,0) circle [radius=1cm];

  \draw[very thick,red]
    (30:1cm) -- node[left=1pt,fill=white] {$\sin \alpha$} (30:1cm |- x axis);
  \draw[very thick,blue]
    (30:1cm |- x axis) -- node[below=2pt,fill=white] {$\cos \alpha$} (0,0);
  \path [name path=upward line] (1,0) -- (1,1);
  \path [name path=sloped line] (0,0) -- (30:1.5cm);
  \draw [name intersections={of=upward line and sloped line, by=t}]
    [very thick,orange] (1,0) -- node [right=1pt,fill=white]
    {$\displaystyle \tan \alpha \color{black}=
      \frac{{\color{red}\sin \alpha}}{\color{blue}\cos \alpha}$} (t);

  \draw (0,0) -- (t);

  \foreach \x/\xtext in {-1, -0.5/-\frac{1}{2}, 1}
    \draw (\x cm,1pt) -- (\x cm,-1pt) node[anchor=north,fill=white] {$\xtext$};
  \foreach \y/\ytext in {-1, -0.5/-\frac{1}{2}, 0.5/\frac{1}{2}, 1}
    \draw (1pt,\y cm) -- (-1pt,\y cm) node[anchor=east,fill=white] {$\ytext$};
\end{tikzpicture}
\end{codeexample}

You can also position labels on curves and, by adding the |sloped| option, have
them rotated such that they match the line's slope. Here is an example:
%
\begin{codeexample}[]
\begin{tikzpicture}
  \draw (0,0) .. controls (6,1) and (9,1) ..
    node[near start,sloped,above] {near start}
    node {midway}
    node[very near end,sloped,below] {very near end} (12,0);
\end{tikzpicture}
\end{codeexample}

It remains to draw the explanatory text at the right of the picture. The main
difficulty here lies in limiting the width of the text ``label'', which is
quite long, so that line breaking is used. Fortunately, Karl can use the option
|text width=6cm| to get the desired effect. So, here is the full code:
%
\begin{codeexample}[code only]
\begin{tikzpicture}
  [scale=3,line cap=round,
  % Styles
  axes/.style=,
  important line/.style={very thick},
  information text/.style={rounded corners,fill=red!10,inner sep=1ex}]

  % Colors
  \colorlet{anglecolor}{green!50!black}
  \colorlet{sincolor}{red}
  \colorlet{tancolor}{orange!80!black}
  \colorlet{coscolor}{blue}

  % The graphic
  \draw[help lines,step=0.5cm] (-1.4,-1.4) grid (1.4,1.4);

  \draw (0,0) circle [radius=1cm];

  \begin{scope}[axes]
    \draw[->] (-1.5,0) -- (1.5,0) node[right] {$x$} coordinate(x axis);
    \draw[->] (0,-1.5) -- (0,1.5) node[above] {$y$} coordinate(y axis);

    \foreach \x/\xtext in {-1, -.5/-\frac{1}{2}, 1}
      \draw[xshift=\x cm] (0pt,1pt) -- (0pt,-1pt) node[below,fill=white] {$\xtext$};

    \foreach \y/\ytext in {-1, -.5/-\frac{1}{2}, .5/\frac{1}{2}, 1}
      \draw[yshift=\y cm] (1pt,0pt) -- (-1pt,0pt) node[left,fill=white] {$\ytext$};
  \end{scope}

  \filldraw[fill=green!20,draw=anglecolor] (0,0) -- (3mm,0pt)
    arc [start angle=0, end angle=30, radius=3mm];
  \draw (15:2mm) node[anglecolor] {$\alpha$};

  \draw[important line,sincolor]
    (30:1cm) -- node[left=1pt,fill=white] {$\sin \alpha$} (30:1cm |- x axis);

  \draw[important line,coscolor]
    (30:1cm |- x axis) -- node[below=2pt,fill=white] {$\cos \alpha$} (0,0);

  \path [name path=upward line] (1,0) -- (1,1);
  \path [name path=sloped line] (0,0) -- (30:1.5cm);
  \draw [name intersections={of=upward line and sloped line, by=t}]
    [very thick,orange] (1,0) -- node [right=1pt,fill=white]
    {$\displaystyle \tan \alpha \color{black}=
      \frac{{\color{red}\sin \alpha}}{\color{blue}\cos \alpha}$} (t);

  \draw (0,0) -- (t);

  \draw[xshift=1.85cm]
    node[right,text width=6cm,information text]
    {
      The {\color{anglecolor} angle $\alpha$} is $30^\circ$ in the
      example ($\pi/6$ in radians). The {\color{sincolor}sine of
        $\alpha$}, which is the height of the red line, is
      \[
      {\color{sincolor} \sin \alpha} = 1/2.
      \]
      By the Theorem of Pythagoras ...
    };
\end{tikzpicture}
\end{codeexample}


\subsection{Pics: The Angle Revisited}

Karl expects that the code of certain parts of the picture he created might be
so useful that he might wish to reuse them in the future. A natural thing to do
is to create \TeX\ macros that store the code he wishes to reuse. However,
\tikzname\ offers another way that is integrated directly into its parser:
pics!

A ``pic'' is ``not quite a full picture'', hence the short name. The idea is
that a pic is simply some code that you can add to a picture at different
places using the |pic| command whose syntax is almost identical to the |node|
command. The main difference is that instead of specifying some text in curly
braces that should be shown, you specify the name of a predefined picture that
should be shown.

Defining new pics is easy enough, see Section~\ref{section-pics}, but right now
we just want to use one such predefined pic: the |angle| pic. As the name
suggests, it is a small drawing of an angle consisting of a little wedge and an
arc together with some text (Karl needs to load the |angles| library and the
|quotes| for the following examples). What makes this pic useful is the fact
that the size of the wedge will be computed automatically.

The |angle| pic draws an angle between the two lines $BA$ and $BC$, where $A$,
$B$, and $C$ are three coordinates. In our case, $B$ is the origin, $A$ is
somewhere on the $x$-axis and $C$ is somewhere on a line at $30^\circ$.
%
\begin{codeexample}[preamble={\usetikzlibrary{angles,quotes}}]
\begin{tikzpicture}[scale=3]
  \coordinate (A) at (1,0);
  \coordinate (B) at (0,0);
  \coordinate (C) at (30:1cm);

  \draw (A) -- (B) -- (C)
        pic [draw=green!50!black, fill=green!20, angle radius=9mm,
             "$\alpha$"] {angle = A--B--C};
\end{tikzpicture}
\end{codeexample}

Let us see, what is happening here. First we have specified three
\emph{coordinates} using the |\coordinate| command. It allows us to name a
specific coordinate in the picture. Then comes something that starts as a
normal |\draw|, but then comes the |pic| command. This command gets lots of
options and, in curly braces, comes the most important point: We specify that
we want to add an |angle| pic and this angle should be between the points we
named |A|, |B|, and |C| (we could use other names). Note that the text that we
want to be shown in the pic is specified in quotes inside the options of the
|pic|, not inside the curly braces.

To learn more about pics, please see Section~\ref{section-pics}.
