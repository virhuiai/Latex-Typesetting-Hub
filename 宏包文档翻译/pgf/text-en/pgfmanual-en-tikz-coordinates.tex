
\subsubsection{Tangent Coordinate Systems\\切线坐标系统}

\begin{coordinatesystem}{tangent}
    This coordinate system, which is available only when the \tikzname\ library
    |calc| is loaded, allows you to compute the point that lies tangent to a
    shape. In detail, consider a \meta{node} and a \meta{point}. Now, draw a
    straight line from the \meta{point} so that it ``touches'' the \meta{node}
    (more formally, so that it is \emph{tangent} to this \meta{node}). The
    point where the line touches the shape is the point referred to by the
    |tangent| coordinate system.

    这个坐标系统仅在加载了 \tikzname\ 库 |calc| 后才可用,它允许你计算与形状相切的点。具体而言,考虑一个 \meta{节点} 和一个 \meta{点}。现在,从 \meta{点} 画一条直线,使其“接触” \meta{节点}(更形式化地说,使其与该 \meta{节点} \emph{切线})。线与形状接触的点就是由 |tangent| 坐标系统引用的点。



    The following options may be given:

    可以给出以下选项:

    %
    \begin{key}{/tikz/cs/node=\meta{node}}
        This key specifies the node on whose border the tangent should lie.

        指定切线应位于其边界上的节点。

    \end{key}
    %
    \begin{key}{/tikz/cs/point=\meta{point}}
        This key specifies the point through which the tangent should go.

        指定切线应通过的点。
    \end{key}
    %
    \begin{key}{/tikz/cs/solution=\meta{number}}
        Specifies which solution should be used if there are more than one.

        如果有多个解决方案,则指定要使用的解决方案。
    \end{key}

    A special algorithm is needed in order to compute the tangent for a given
    shape. Currently, tangents can be computed for nodes whose shape is one of
    the following:
    %

    为了计算给定形状的切线,需要一个特殊的算法。目前,可以计算以下形状的节点的切线:
    \begin{itemize}
        \item |coordinate|
        \item |circle|
    \end{itemize}
    %
\begin{codeexample}[preamble={\usetikzlibrary{calc}}]
\begin{tikzpicture}
  \draw[help lines] (0,0) grid (3,2);

  \coordinate (a) at (3,2);

  \node [circle,draw] (c) at (1,1) [minimum size=40pt] {$c$};

  \draw[red] (a)  -- (tangent cs:node=c,point={(a)},solution=1) --
       (c.center) -- (tangent cs:node=c,point={(a)},solution=2) -- cycle;
\end{tikzpicture}
\end{codeexample}

    There is no implicit syntax for this coordinate system.

    这个坐标系统没有隐式语法。

\end{coordinatesystem}


\subsubsection{Defining New Coordinate Systems\\定义新的坐标系统}

While the set of coordinate systems that \tikzname\ can parse via their special
syntax is fixed, it is possible and quite easy to define new explicitly named
coordinate systems. For this, the following commands are used:

尽管 \tikzname\ 可以通过其特殊语法解析一组坐标系统,但是定义新的显式命名的坐标系统是可能且非常容易的。为此,使用以下命令:

\begin{command}{\tikzdeclarecoordinatesystem\marg{name}\marg{code}}
    This command declares a new coordinate system named \meta{name} that can
    later on be used by writing |(|\meta{name}| cs:|\meta{arguments}|)|. When
    \tikzname\ encounters a coordinate specified in this way, the
    \meta{arguments} are passed to \meta{code} as argument |#1|.

    这个命令声明了一个名为 \meta{name} 的新坐标系统,可以通过写 |(|\meta{name}| cs:|\meta{参数}|)| 来使用。当 \tikzname\ 遇到以这种方式指定的坐标时,将把 \meta{参数} 作为参数 |#1| 传递给 \meta{code}。


    It is now the job of \meta{code} to make sense of the \meta{arguments}. At
    the end of \meta{code}, the two \TeX\ dimensions |\pgf@x| and |\pgf@y|
    should be have the $x$- and $y$-canvas coordinate of the coordinate.

    现在,\meta{code} 的任务是理解 \meta{参数}。在 \meta{code} 的末尾,两个 \TeX\ 尺寸 |\pgf@x| 和 |\pgf@y| 应该具有坐标的 $x$- 和 $y$-画布坐标。


    It is not necessary, but customary, to parse \meta{arguments} using the
    key--value syntax. However, you can also parse it in any way you like.

    不必要,但习惯上,使用键-值语法解析 \meta{参数}。但你也可以以任何你喜欢的方式解析它。


    In the following example, a coordinate system |cylindrical| is defined.
    
    在下面的示例中,定义了一个坐标系统 |cylindrical|。
%
\begin{codeexample}[]
\makeatletter
\define@key{cylindricalkeys}{angle}{\def\myangle{#1}}
\define@key{cylindricalkeys}{radius}{\def\myradius{#1}}
\define@key{cylindricalkeys}{z}{\def\myz{#1}}
\tikzdeclarecoordinatesystem{cylindrical}%
{%
  \setkeys{cylindricalkeys}{#1}%
  \pgfpointadd{\pgfpointxyz{0}{0}{\myz}}{\pgfpointpolarxy{\myangle}{\myradius}}
}
\begin{tikzpicture}[z=0.2pt]
  \draw [->] (0,0,0) -- (0,0,350);
  \foreach \num in {0,10,...,350}
    \fill (cylindrical cs:angle=\num,radius=1,z=\num) circle (1pt);
\end{tikzpicture}
\end{codeexample}
    %
\end{command}

\begin{command}{\tikzaliascoordinatesystem\marg{new name}\marg{old name}}
    Creates an alias of \meta{old name}.
\end{command}


\subsection{Coordinates at Intersections\\交点坐标}
\label{section-intersection-coordinates}

You will wish to compute the intersection of two paths. For the special and
frequent case of two perpendicular lines, a special coordinate system called
|perpendicular| is available. For more general cases, the |intersection|
library can be used.

您可能希望计算两条路径的交点。对于两条垂直线的特殊且频繁的情况,有一种称为 |perpendicular| 的特殊坐标系可用。对于更一般的情况,可以使用 |intersection| 库。


\subsubsection{Intersections of Perpendicular Lines\\垂直线的交点}

A frequent special case of path intersections is the intersection of a vertical
line going through a point $p$ and a horizontal line going through some other
point $q$. For this situation there is a useful coordinate system.

路径交点的频繁特殊情况是通过点 $p$ 经过的垂直线与通过另一点 $q$ 的水平线的交点。对于这种情况,有一个有用的坐标系。


\begin{coordinatesystem}{perpendicular}
    You can specify the two lines using the following keys:

    您可以使用以下键来指定这两条线:



    \begin{key}{/tikz/cs/horizontal line through={\ttfamily\char`\{}|(|\meta{coordinate}|)|{\ttfamily\char`\}}}
        Specifies that one line is a horizontal line that goes through the
        given coordinate.

        指定一个经过给定坐标的水平线。

    \end{key}
    %
    \begin{key}{/tikz/cs/vertical line through={\ttfamily\char`\{}|(|\meta{coordinate}|)|{\ttfamily\char`\}}}
        Specifies that the other line is vertical and goes through the given
        coordinate.

        指定另一条是垂直线,经过给定坐标。

    \end{key}

    However, in almost all cases you should, instead, use the implicit syntax.
    Here, you write \declare{|(|\meta{p}\verb! |- !\meta{q}|)|} or
    \declare{|(|\meta{q}\verb! -| !\meta{p}|)|}.

    然而,在几乎所有情况下,您应该使用隐式语法。在这里,您可以写成 \declare{|(|\meta{p}\verb! |- !\meta{q}|)|} 或 \declare{|(|\meta{q}\verb! -| !\meta{p}|)|}。


    For example, \verb!(2,1 |- 3,4)! and  \verb!(3,4 -| 2,1)! both yield the
    same as \verb!(2,4)! (provided the $xy$-co\-or\-di\-nate system has not
    been modified).

    例如,\verb!(2,1 |- 3,4)! 和 \verb!(3,4 -| 2,1)! 都可以得到与 \verb!(2,4)! 相同的结果(前提是 $xy$ 坐标系没有被修改)。


    The most useful application of the syntax is to draw a line up to some
    point on a vertical or horizontal line. Here is an example:

    该语法最有用的应用是在垂直或水平线上绘制一条线到达某个点。以下是一个示例:

    %
\begin{codeexample}[]
\begin{tikzpicture}
  \path (30:1cm) node(p1) {$p_1$}   (75:1cm) node(p2) {$p_2$};

  \draw (-0.2,0) -- (1.2,0) node(xline)[right] {$q_1$};
  \draw (2,-0.2) -- (2,1.2) node(yline)[above] {$q_2$};

  \draw[->] (p1) -- (p1 |- xline);
  \draw[->] (p2) -- (p2 |- xline);
  \draw[->] (p1) -- (p1 -| yline);
  \draw[->] (p2) -- (p2 -| yline);
\end{tikzpicture}
\end{codeexample}

    Note that in \declare{|(|\meta{c}\verb! |- !\meta{d}|)|} the coordinates
    \meta{c} and \meta{d} are \emph{not} surrounded by parentheses. If they
    need to be complicated expressions (like a computation using the
    |$|-syntax), you must surround them with braces; parentheses will then be   %$
    added around them.

    请注意,在 \declare{|(|\meta{c}\verb! |- !\meta{d}|)|} 中,坐标 \meta{c} 和 \meta{d} \emph{不}用括号括起来。如果它们需要复杂的表达式(如使用 |$|-syntax 进行计算),则必须用花括号括起来;括号将被添加在它们周围。


    As an example, let us specify a point that lies horizontally at the middle
    of the line from $A$ to~$B$ and vertically at the middle of the line from
    $C$ to~$D$:

    例如,让我们指定一个点,它水平地位于从 $A$ 到 $B$ 的线的中间,垂直地位于从 $C$ 到 $D$ 的线的中间:

    %
\begin{codeexample}[preamble={\usetikzlibrary{calc}}]
\begin{tikzpicture}
  \node (A) at (0,1)    {A};
  \node (B) at (1,1.5)  {B};
  \node (C) at (2,0)    {C};
  \node (D) at (2.5,-2) {D};

  \draw (A) -- (B) node [midway] {x};
  \draw (C) -- (D) node [midway] {x};

  \node at ({$(A)!.5!(B)$} -| {$(C)!.5!(D)$}) {X};
\end{tikzpicture}
\end{codeexample}
    %
\end{coordinatesystem}


\subsubsection{Intersections of Arbitrary Paths\\任意路径的交点}

\begin{tikzlibrary}{intersections}
    This library enables the calculation of intersections of two arbitrary
    paths. However, due to the low accuracy of \TeX, the paths should not be
    ``too complicated''. In particular, you should not try to intersect paths
    consisting of lots of very small segments such as plots or decorated paths.

    此库使得能够计算两条任意路径的交点。然而,由于 \TeX 的低精度,路径不应该太过复杂。特别是,您不应该尝试计算由许多非常小的线段组成的路径的交点,比如曲线或装饰路径。
\end{tikzlibrary}

To find the intersections of two paths in \tikzname, they must be ``named''. A
``named path'' is, quite simply, a path that has been named using the following
key (note that this is a \emph{different} key from the |name| key, which only
attaches a hyperlink target to a path, but does not store the path in a way the
is useful for the intersection computation):

要在 \tikzname 中找到两条路径的交点,它们必须被“命名”。“命名路径”就是简单地使用以下键给路径命名(请注意,这是与 |name| 键不同的键,|name| 键仅为路径附加超链接目标,但不以对交点计算有用的方式存储路径):

\begin{keylist}{%
    /tikz/name path=\meta{name},
    /tikz/name path global=\meta{name}%
}
    The effect of this key is that, after the path has been constructed, just
    before it is used, it is associated with \meta{name}. For |name path|, this
    association survives beyond the final semi-colon of the path but not the
    end of the surrounding scope. For |name path global|, the association will
    survive beyond any scope as well. Handle with care.

    此键的效果是,在构建路径后,在使用路径之前,将其与 \meta{name} 关联。对于 |name path|,此关联将在路径的最后一个分号之后仍然存在,但不会超出周围作用域的末尾。对于 |name path global|,此关联将在任何作用域之外仍然存在。请谨慎使用。

    Any paths created by nodes on the (main) path are ignored, unless this key
    is explicitly used. If the same \meta{name} is used for the main path and
    the node path(s), then the paths will be added together and then associated
    with \meta{name}.

    除非显式使用此键,否则将忽略路径上的节点创建的任何路径。如果主路径和节点路径使用相同的 \meta{name},则路径将被合并,然后与 \meta{name} 关联。
\end{keylist}

To find the intersection of named paths, the following key is used:

要找到命名路径的交点,使用以下键:

\begin{key}{/tikz/name intersections=\marg{options}}
    This key changes the key path to |/tikz/intersection| and processes
    \meta{options}. These options determine, among other things, which paths to
    use for the intersection. Having processed the options, any intersections
    are then found. A coordinate is created at each intersection, which by
    default, will be named |intersection-1|, |intersection-2|, and so on.
    Optionally, the prefix |intersection| can be changed, and the total number
    of intersections stored in a \TeX-macro.

    此键将键路径更改为 |/tikz/intersection| 并处理 \meta{options}。这些选项确定交点要使用的路径。在处理完选项后,将找到任何交点。在每个交点处创建一个坐标,默认情况下,它们将被命名为 |intersection-1|、|intersection-2| 等。还可以选择更改前缀 |intersection|,以及将交点总数存储在 \TeX 宏中。


    %
\begin{codeexample}[preamble={\usetikzlibrary{intersections}}]
\begin{tikzpicture}[every node/.style={opacity=1, black, above left}]
  \draw [help lines] grid (3,2);
  \draw [name path=ellipse] (2,0.5) ellipse (0.75cm and 1cm);
  \draw [name path=rectangle, rotate=10] (0.5,0.5) rectangle +(2,1);
  \fill [red, opacity=0.5, name intersections={of=ellipse and rectangle}]
    (intersection-1) circle (2pt) node {1}
    (intersection-2) circle (2pt) node {2};
\end{tikzpicture}
\end{codeexample}

    The following keys can be used in \meta{options}:

    以下键可以在\meta{选项}中使用:


    \begin{key}{/tikz/intersection/of=\meta{name path 1}| and |\meta{name path 2}}
        This key is used to specify the names of the paths to use for the
        intersection.

        此键用于指定用于求交的路径的名称。
    \end{key}

    \begin{key}{/tikz/intersection/name=\meta{prefix} (initially intersection)}
        This key specifies the prefix name for the coordinate nodes placed at
        each intersection.

        此键指定放置在每个交点处的坐标节点的前缀名称。
    \end{key}

    \begin{key}{/tikz/intersection/total=\meta{macro}}
        This key means that the total number of intersections found will be
        stored in \meta{macro}.

        此键表示找到的交点总数将存储在\meta{macro}中。
    \end{key}

\begin{codeexample}[preamble={\usetikzlibrary{intersections}}]
\begin{tikzpicture}
  \clip (-2,-2) rectangle (2,2);
  \draw [name path=curve 1] (-2,-1) .. controls (8,-1) and (-8,1) .. (2,1);
  \draw [name path=curve 2] (-1,-2) .. controls (-1,8) and (1,-8) .. (1,2);

  \fill [name intersections={of=curve 1 and curve 2, name=i, total=\t}]
        [red, opacity=0.5, every node/.style={above left, black, opacity=1}]
        \foreach \s in {1,...,\t}{(i-\s) circle (2pt) node {\footnotesize\s}};
\end{tikzpicture}
\end{codeexample}

    \begin{key}{/tikz/intersection/by=\meta{comma-separated list}}
        This key allows you to specify a list of names for the intersection
        coordinates. The intersection coordinates will still be named
        \meta{prefix}|-|\meta{number}, but additionally the first coordinate
        will also be named by the first element of the \meta{comma-separated
        list}. What happens is that the \meta{comma-separated list} is passed
        to the |\foreach| statement and for \meta{list member} a coordinate is
        created at the already-named intersection.
        
        此键允许您指定交点坐标的名称列表。交点坐标仍然被命名为\meta{prefix}|-|\meta{number},但是还会使用\meta{逗号分隔的列表}的第一个元素为第一个坐标命名。所发生的情况是\meta{逗号分隔的列表}被传递给|\foreach|语句,并且对于\meta{列表成员}在已命名的交点处创建一个坐标。

\begin{codeexample}[preamble={\usetikzlibrary{intersections}}]
\begin{tikzpicture}
  \clip (-2,-2) rectangle (2,2);
  \draw [name path=curve 1] (-2,-1) .. controls (8,-1) and (-8,1) .. (2,1);
  \draw [name path=curve 2] (-1,-2) .. controls (-1,8) and (1,-8) .. (1,2);

  \fill [name intersections={of=curve 1 and curve 2, by={a,b}}]
        (a) circle (2pt)
        (b) circle (2pt);
\end{tikzpicture}
\end{codeexample}

        You can also use the |...| notation of the |\foreach| statement inside
        the \meta{comma-separated list}.

        您还可以在\meta{逗号分隔的列表}中使用|\foreach|语句的|...|表示法。

        In case an element of the \meta{comma-separated list} starts with
        options in square brackets, these options are used when the coordinate
        is created. A coordinate name can still, but need not, follow the
        options. This makes it easy to add labels to intersections:

        如果\meta{逗号分隔的列表}的元素以方括号中的选项开头,则在创建坐标时使用这些选项。坐标名称仍然可以,但不必遵循这些选项。这使得向交点添加标签变得很容易:

        %
\begin{codeexample}[preamble={\usetikzlibrary{intersections}}]
\begin{tikzpicture}
  \clip (-2,-2) rectangle (2,2);
  \draw [name path=curve 1] (-2,-1) .. controls (8,-1) and (-8,1) .. (2,1);
  \draw [name path=curve 2] (-1,-2) .. controls (-1,8) and (1,-8) .. (1,2);

  \fill [name intersections={
          of=curve 1 and curve 2,
          by={[label=center:a],[label=center:...],[label=center:i]}}];
\end{tikzpicture}
\end{codeexample}
    \end{key}

    \begin{key}{/tikz/intersection/sort by=\meta{path name}}
        By default, the intersections are simply returned in the order that the
        intersection algorithm finds them. Unfortunately, this is not
        necessarily a ``helpful'' ordering. This key can be used to sort the
        intersections along the path specified by \meta{path name}, which
        should be one of the paths mentioned in the |/tikz/intersection/of|
        key.
        %

        默认情况下,交点按照交点算法发现它们的顺序简单返回。不幸的是,这不一定是一个“有用”的排序方式。此键可用于沿着由\meta{路径名称}指定的路径对交点进行排序,该路径应为|/tikz/intersection/of|键中提到的路径之一。
\begin{codeexample}[preamble={\usetikzlibrary{intersections}}]
\begin{tikzpicture}
\clip (-0.5,-0.75) rectangle (3.25,2.25);
\foreach \pathname/\shift in {line/0cm, curve/2cm}{
  \tikzset{xshift=\shift}
  \draw [->, name path=curve] (1,1.5) .. controls (-1,1) and (2,0.5) .. (0,0);
  \draw [->, name path=line]  (0,-.5) -- (1,2) ;
  \fill [name intersections={of=line and curve,sort by=\pathname, name=i}]
    [red, opacity=0.5, every node/.style={left=.25cm, black, opacity=1}]
    \foreach \s in {1,2,3}{(i-\s) circle (2pt) node {\footnotesize\s}};
}
\end{tikzpicture}
\end{codeexample}
    \end{key}
\end{key}


\subsection{Relative and Incremental Coordinates\\相对和增量坐标}

\subsubsection{Specifying Relative Coordinates\\指定相对坐标}

You can prefix coordinates by |++| to make them ``relative''. A coordinate such
as |++(1cm,0pt)| means ``1cm to the right of the previous position, making this
the new current position''. Relative coordinates are often useful in ``local''
contexts:

您可以在坐标前面加上|++|使其成为“相对”坐标。例如,像|++(1cm,0pt)|这样的坐标意味着“相对于前一个位置向右1cm,这成为新的当前位置”。在“局部”环境中,相对坐标通常很有用:

%
\begin{codeexample}[]
\begin{tikzpicture}
  \draw (0,0)     -- ++(1,0) -- ++(0,1) -- ++(-1,0) -- cycle;
  \draw (2,0)     -- ++(1,0) -- ++(0,1) -- ++(-1,0) -- cycle;
  \draw (1.5,1.5) -- ++(1,0) -- ++(0,1) -- ++(-1,0) -- cycle;
\end{tikzpicture}
\end{codeexample}

Instead of |++| you can also use a single |+|. This also specifies a relative
coordinate, but it does not ``update'' the current point for subsequent usages
of relative coordinates. Thus, you can use this notation to specify numerous
points, all relative to the same ``initial'' point:

您也可以使用单个|+|代替|++|。这也指定了一个相对坐标,但它不会“更新”当前点以供后续使用的相对坐标。因此,您可以使用此记法指定多个点,所有这些点都相对于同一个“初始”点:


\begin{codeexample}[]
\begin{tikzpicture}
  \draw (0,0)     -- +(1,0) -- +(1,1) -- +(0,1) -- cycle;
  \draw (2,0)     -- +(1,0) -- +(1,1) -- +(0,1) -- cycle;
  \draw (1.5,1.5) -- +(1,0) -- +(1,1) -- +(0,1) -- cycle;
\end{tikzpicture}
\end{codeexample}

There is a special situation, where relative coordinates are interpreted
differently. If you use a relative coordinate as a control point of a Bézier
curve, the following rule applies: First, a relative first control point is
taken relative to the beginning of the curve. Second, a relative second control
point is taken relative to the end of the curve. Third, a relative end point of
a curve is taken relative to the start of the curve.

有一种特殊情况,其中相对坐标的解释方式不同。如果你将相对坐标用作贝塞尔曲线的控制点,则应遵循以下规则:首先,相对于曲线的起点取相对第一个控制点;其次,相对于曲线的终点取相对第二个控制点;第三,相对于曲线的起点取相对终点。


This special behavior makes it easy to specify that a curve should ``leave or
arrive from a certain direction'' at the start or end. In the following
example, the curve ``leaves'' at $30^\circ$ and ``arrives'' at $60^\circ$:


这种特殊行为使得在起点或终点处指定曲线应该“从某个方向离开或到达”变得容易。在下面的示例中,曲线“离开”时角度为$30^\circ$,而“到达”时角度为$60^\circ$:

%
\begin{codeexample}[]
\begin{tikzpicture}
  \draw (1,0) .. controls +(30:1cm) and +(60:1cm) .. (3,-1);
  \draw[gray,->] (1,0) -- +(30:1cm);
  \draw[gray,<-] (3,-1) -- +(60:1cm);
\end{tikzpicture}
\end{codeexample}


\subsubsection{Rotational Relative Coordinates\\旋转相对坐标}

You may sometimes wish to specify points relative not only to the previous
point, but additionally relative to the tangent entering the previous point.
For this, the following key is useful:


有时,你可能希望指定相对于前一个点以及相对于进入前一个点的切线的点。为此,以下关键字很有用:

\begin{key}{/tikz/turn}
    This key can be given as an option to a \meta{coordinate} as in the
    following example:
    
    你可以将该关键字作为选项给出,如下例所示:%
\begin{codeexample}[]
\tikz \draw (0,0) -- (1,1) -- ([turn]-45:1cm) -- ([turn]-30:1cm);
\end{codeexample}
    %
    The effect of this key is to locally shift the coordinate system so that
    the last point reached is at the origin and the coordinate system is
    ``turned'' so that the $x$-axis points in the direction of a tangent
    entering the last point. This means, in effect, that when you use polar
    coordinates of the form \meta{relative angle}|:|\meta{distance} together
    with the |turn| option, you specify a point that lies at \meta{distance}
    from the last point in the direction of the last tangent entering the last
    point, but with a rotation of \meta{relative angle}.

    此关键字的效果是局部移动坐标系,使得到达的最后一个点位于原点,并且坐标系“转动”,使得$x$轴指向进入最后一个点的切线方向。这实际上意味着,当你使用形式为\meta{relative angle}|:|\meta{distance}的极坐标与|turn|选项一起使用时,你指定了一个位于距离最后一个点\meta{distance}、方向为进入最后一个点的切线方向、旋转角为\meta{relative angle}的点。


    This key also works with curves \dots

    该关键字还适用于曲线……

    %
\begin{codeexample}[]
\tikz [delta angle=30, radius=1cm]
  \draw (0,0) arc [start angle=0]  -- ([turn]0:1cm)
              arc [start angle=30] -- ([turn]0:1cm)
              arc [start angle=60] -- ([turn]30:1cm);
\end{codeexample}
\begin{codeexample}[]
\tikz \draw (0,0) to [bend left] (2,1) -- ([turn]0:1cm);
\end{codeexample}
    %
    \dots and with plots \dots

    ……以及绘图……
    %
\begin{codeexample}[]
\tikz \draw plot coordinates {(0,0) (1,1) (2,0) (3,0) } -- ([turn]30:1cm);
\end{codeexample}

    Although the above examples use polar coordinates with |turn|, you can also
    use any normal coordinate. For instance, |([turn]1,1)| will append a line
    of length $\sqrt 2$ that is turns by $45^\circ$ relative to the tangent to
    the last point.
    %

    虽然上面的示例使用了带有|turn|的极坐标,但你也可以使用任何普通坐标。例如,|([turn]1,1)| 将附加一条长度为 $\sqrt 2$ 的线,相对于最后一个点的切线旋转 $45^\circ$。
\begin{codeexample}[]
\tikz \draw (0.5,0.5) -| (2,1) -- ([turn]1,1)
         .. controls ([turn]0:1cm) .. ([turn]-90:1cm);
\end{codeexample}
    %
\end{key}


\subsubsection{Relative Coordinates and Scopes\\相对坐标和作用域}
\label{section-scopes-relative}

An interesting question is, how do relative coordinates behave in the presence
of scopes? That is, suppose we use curly braces in a path to make part of it
``local'', how does that affect the current position? On the one hand, the
current position certainly changes since the scope only affects options, not
the path itself. On the other hand, it may be useful to ``temporarily escape''
from the updating of the current point.

一个有趣的问题是,在作用域存在的情况下,相对坐标的行为如何?也就是说,假设我们在路径中使用花括号使其的一部分“局部化”,那么这将如何影响当前位置呢?一方面,当前位置肯定会发生变化,因为作用域只影响选项,而不影响路径本身。另一方面,暂时“逃离”当前点的更新可能是有用的。


Since both interpretations of how the current point and scopes should
``interact'' are useful, there is a (local!) option that allows you to decide
which you need.

由于当前点和作用域应如何“交互”的这两种解释都是有用的,因此有一个(局部的!)选项可以让你决定你需要哪种方式。


\begin{key}{/tikz/current point is local=\opt{\meta{boolean}} (initially false)}
    Normally, the scope path operation has no effect on the current point. That
    is, curly braces on a path have no effect on the current position:
    
    通常来说,作用域路径操作对当前点没有影响。也就是说,路径中的花括号对当前位置没有影响:
%
\begin{codeexample}[]
\begin{tikzpicture}
  \draw      (0,0) -- ++(1,0)   -- ++(0,1)   -- ++(-1,0);
  \draw[red] (2,0) -- ++(1,0) { -- ++(0,1) } -- ++(-1,0);
\end{tikzpicture}
\end{codeexample}
    %
    If you set this key to |true|, this behavior changes. In this case, at the
    end of a group created on a path, the last current position reverts to
    whatever value it had at the beginning of the scope. More precisely, when
    \tikzname\ encounters |}| on a path, it checks whether at this particular
    moment the key is set to |true|. If so, the current position reverts to the
    value it had when the matching |{| was read.
    
    如果将此关键字设置为 |true|,则此行为会发生变化。在这种情况下,在路径上的一个组的末尾,当前位置将恢复为与读取相匹配的 |{| 时的值。更准确地说,当 \tikzname\ 在路径上遇到 |}| 时,它会检查此时该关键字是否设置为 |true|。如果是,则当前位置将恢复为读取匹配的 |{| 时的值。

%
\begin{codeexample}[]
\begin{tikzpicture}
  \draw      (0,0) -- ++(1,0)   -- ++(0,1)   -- ++(-1,0);
  \draw[red] (2,0) -- ++(1,0)
     { [current point is local] -- ++(0,1) } -- ++(-1,0);
\end{tikzpicture}
\end{codeexample}
    %
    In the above example, we could also have given the option outside the
    scope, for instance as a parameter to the whole scope.

    在上面的示例中,我们也可以在作用域之外给出选项,例如作为整个作用域的参数。
\end{key}


\subsection{Coordinate Calculations\\坐标计算}
\label{tikz-lib-calc}

\begin{tikzlibrary}{calc}
    You need to load this library in order to use the coordinate calculation
    functions described in the present section.

    在使用本节中描述的坐标计算函数之前,您需要加载此库。
\end{tikzlibrary}

It is possible to do some basic calculations that involve coordinates. In
essence, you can add and subtract coordinates, scale them, compute midpoints,
and do projections. For instance, |($(a) + 1/3*(1cm,0)$)| is the coordinate
that is $1/3 \text{cm}$ to the right of the point |a|:


可以进行涉及坐标的基本计算。基本上,可以添加和减去坐标,对其进行缩放,计算中点和投影。例如,|($(a) + 1/3*(1cm,0)$)| 是点 |a| 右侧 $1/3 \text{cm}$ 处的坐标:
%
\begin{codeexample}[preamble={\usetikzlibrary{calc}}]
\begin{tikzpicture}
  \draw [help lines] (0,0) grid (3,2);

  \node (a) at (1,1) {A};
  \fill [red] ($(a) + 1/3*(1cm,0)$) circle (2pt);
\end{tikzpicture}
\end{codeexample}


\subsubsection{The General Syntax\\通用语法}

The general syntax is the following:

通用语法如下:
%
\begin{quote}
    \declare{|(|\opt{|[|\meta{options}|]|}|$|\meta{coordinate computation}|$)|}.
\end{quote}

As you can see, the syntax uses the \TeX\ math symbol |$| to %$
indicate that a ``mathematical computation'' is involved. However, the |$| %$
has no other effect, in particular, no mathematical text is typeset.

如您所见,语法使用 \TeX\ 数学符号 |$| 表示涉及``数学计算''。但是,|$| 没有其他作用,特别是没有进行数学文本排版。


The \meta{coordinate computation} has the following structure:

\meta{坐标计算} 具有以下结构:
%
\begin{enumerate}
    \item It starts with
        %

        以如下形式开始:
        \begin{quote}
            \opt{\meta{factor}|*|}\meta{coordinate}\opt{\meta{modifiers}}
        \end{quote}
    \item This is optionally followed by |+| or |-| and then another

    可选地后跟 |+| 或 |-|,然后是另一个
        %
        \begin{quote}
            \opt{\meta{factor}|*|}\meta{coordinate}\opt{\meta{modifiers}}
        \end{quote}
    \item This is once more followed by |+| or |-| and another of the above
        modified coordinate; and so on.

        再次后跟 |+| 或 |-| 和上述修饰过的坐标;以此类推。
\end{enumerate}

In the following, the syntax of factors and of the different modifiers
is explained in detail.

接下来,详细解释因子和不同修饰符的语法。


\subsubsection{The Syntax of Factors\\因子的语法}

The \meta{factor}s are optional and detected by checking whether the
\meta{coordinate computation} starts with a |(|. Also, after each $\pm$ a
\meta{factor} is present if, and only if, the |+| or |-| sign is not directly
followed by~|(|.

\meta{因子}是可选的,通过检查 \meta{坐标计算} 是否以 |(| 开头来确定。此外,在每个 $\pm$ 后,只有在 |+| 或 |-| 符号直接后面没有跟随 |(| 时才出现 \meta{因子}。

If a \meta{factor} is present, it is evaluated using the |\pgfmathparse| macro.
This means that you can use pretty complicated computations inside a factor. A
\meta{factor} may even contain opening parentheses, which creates a
complication: How does \tikzname\ know where a \meta{factor} ends and where a
coordinate starts? For instance, if the beginning of a \meta{coordinate
computation} is |2*(3+4|\dots, it is not clear whether |3+4| is part of a
\meta{coordinate} or part of a \meta{factor}. Because of this, the following
rule is used: Once it has been determined, that a \meta{factor} is present, in
principle, the \meta{factor} contains everything up to the next occurrence of
|*(|. Note that there is no space between the asterisk and the parenthesis.

如果存在 \meta{因子},则使用 |\pgfmathparse| 宏对其进行求值。这意味着您可以在因子内使用相当复杂的计算。一个 \meta{因子} 甚至可以包含括号,这会导致一个问题:\tikzname\ 如何知道 \meta{因子} 在哪里结束,坐标从哪里开始?例如,如果 \meta{坐标计算} 的开头是 |2*(3+4|\dots,不清楚 |3+4| 是 \meta{坐标} 的一部分还是 \meta{因子} 的一部分。因此,采用以下规则:一旦确定存在 \meta{因子},原则上,\meta{因子} 包含直到下一个出现的 |*(| 为止。请注意,星号和括号之间没有空格。

It is permissible to put the \meta{factor} in curly braces. This can be used
whenever it is unclear where the \meta{factor} would end.

可以将 \meta{因子} 放在花括号中。这可以在不清楚 \meta{因子} 结束的地方使用。

Here are some examples of coordinate specifications that consist of exactly one
\meta{factor} and one \meta{coordinate}:

以下是由一个 \meta{因子} 和一个 \meta{坐标} 组成的坐标规范的示例:

%
\begin{codeexample}[preamble={\usetikzlibrary{calc}}]
\begin{tikzpicture}
  \draw [help lines] (0,0) grid (3,2);

  \fill [red] ($2*(1,1)$) circle (2pt);
  \fill [green] (${1+1}*(1,.5)$) circle (2pt);
  \fill [blue] ($cos(0)*sin(90)*(1,1)$) circle (2pt);
  \fill [black] (${3*(4-3)}*(1,0.5)$) circle (2pt);
\end{tikzpicture}
\end{codeexample}


\subsubsection{The Syntax of Partway Modifiers\\部分修饰符的语法}

A \meta{coordinate} can be followed by different \meta{modifiers}. The first
kind of modifier is the \emph{partway modifier}. The syntax (which is loosely
inspired by Uwe Kern's |xcolor| package) is the following:

\meta{坐标}可以后跟不同的\meta{修饰符}。第一种修饰符是\emph{部分修饰符}。其语法(受Uwe Kern的|xcolor|宏包的启发,但略有不同)如下:
%
\begin{quote}
    \meta{coordinate}\declare{|!|\meta{number}|!|\opt{\meta{angle}|:|}\meta{second coordinate}}
\end{quote}
%
One could write for instance

例如,可以写作:
%
\begin{codeexample}[code only]
(1,2)!.75!(3,4)
\end{codeexample}
%
The meaning of this is: ``Use the coordinate that is three quarters on the way
from |(1,2)| to |(3,4)|.'' In general, \meta{coordinate
x}|!|\meta{number}|!|\meta{coordinate y} yields the coordinate $(1 -
\meta{number})\meta{coordinate x} + \meta{number} \meta{coordinate y}$. Note
that this is a bit different from the way the \meta{number} is interpreted in
the |xcolor| package: First, you use a factor between $0$ and $1$, not a
percentage, and, second, as the \meta{number} approaches $1$, we approach the
second coordinate, not the first. It is permissible to use a \meta{number} that
is smaller than $0$ or larger than $1$. The \meta{number} is evaluated using
the |\pgfmathparse| command and, thus, it can involve complicated computations.


它的含义是:“使用从|(1,2)|到|(3,4)|的路径的四分之三的位置上的坐标。” 一般而言,\meta{坐标x}|!|\meta{数值}|!|\meta{坐标y}将生成坐标$(1-\meta{数值})\meta{坐标x} + \meta{数值}\meta{坐标y}$。需要注意的是,这与|xcolor|宏包中对\meta{数值}的解释有所不同:首先,使用的是0到1之间的因子,而不是百分比;其次,当\meta{数值}趋近于1时,接近第二个坐标,而不是第一个坐标。可以使用小于0或大于1的\meta{数值}。\meta{数值}通过|\pgfmathparse|命令计算,因此可以涉及复杂的计算。
%
\begin{codeexample}[preamble={\usetikzlibrary{calc}}]
\begin{tikzpicture}
  \draw [help lines] (0,0) grid (3,2);

  \draw (1,0) -- (3,2);

  \foreach \i in {0,0.2,0.5,0.9,1}
    \node at ($(1,0)!\i!(3,2)$) {\i};
\end{tikzpicture}
\end{codeexample}

The \meta{second coordinate} may be prefixed by an \meta{angle}, separated with
a colon, as in |(1,1)!.5!60:(2,2)|. The general meaning of
\meta{a}|!|\meta{factor}|!|\meta{angle}|:|\meta{b} is: ``First, consider the
line from \meta{a} to \meta{b}. Then rotate this line by \meta{angle}
\emph{around the point \meta{a}}. Then the two endpoints of this line will be
\meta{a} and some point \meta{c}. Use this point \meta{c} for the subsequent
computation, namely the partway computation.''

\meta{第二个坐标}可以带有前缀\meta{角度},用冒号分隔,例如 |(1,1)!.5!60:(2,2)|。\meta{a}|!|\meta{因子}|!|\meta{角度}|:|\meta{b}的一般含义是:“首先,考虑从\meta{a}到\meta{b}的直线。然后以\meta{a}为中心,将这条直线旋转\meta{角度}。这条直线的两个端点将是\meta{a}和某个点\meta{c}。使用该点\meta{c}进行后续计算,即部分计算。”

Here are two examples:

以下是两个示例:
%
\begin{codeexample}[preamble={\usetikzlibrary{calc}}]
\begin{tikzpicture}
  \draw [help lines] (0,0) grid (3,3);

  \coordinate (a) at (1,0);
  \coordinate (b) at (3,2);

  \draw[->] (a) -- (b);

  \coordinate (c) at ($ (a)!1! 10:(b) $);

  \draw[->,red] (a) -- (c);

  \fill ($ (a)!.5! 10:(b) $) circle (2pt);
\end{tikzpicture}
\end{codeexample}

\begin{codeexample}[preamble={\usetikzlibrary{calc}}]
\begin{tikzpicture}
  \draw [help lines] (0,0) grid (4,4);

  \foreach \i in {0,0.1,...,2}
    \fill ($(2,2) !\i! \i*180:(3,2)$) circle (2pt);
\end{tikzpicture}
\end{codeexample}

You can repeatedly apply modifiers. That is, after any modifier you can add
another (possibly different) modifier.

可以重复应用修饰符。也就是说,在任何修饰符之后,可以添加另一个(可能不同的)修饰符。
%
\begin{codeexample}[preamble={\usetikzlibrary{calc}}]
\begin{tikzpicture}
  \draw [help lines] (0,0) grid (3,2);

  \draw (0,0) -- (3,2);
  \draw[red] ($(0,0)!.3!(3,2)$) -- (3,0);
  \fill[red] ($(0,0)!.3!(3,2)!.7!(3,0)$) circle (2pt);
\end{tikzpicture}
\end{codeexample}


\subsubsection{The Syntax of Distance Modifiers\\部分修饰符的语法}

A \emph{distance modifier} has nearly the same syntax as a partway modifier,
only you use a \meta{dimension} (something like |1cm|) instead of a
\meta{factor} (something like |0.5|):

一个\emph{距离修饰符}的语法几乎与\emph{部分修饰符}相同,只是你使用一个\meta{dimension}(类似于|1cm|的东西)而不是\meta{factor}(类似于|0.5|):


%
\begin{quote}
    \meta{coordinate}\declare{|!|\meta{dimension}|!|\opt{\meta{angle}|:|}\meta{second coordinate}}
\end{quote}

When you write \meta{a}|!|\meta{dimension}|!|\meta{b}, this means the
following: Use the point that is distanced \meta{dimension} from \meta{a} on
the straight line from \meta{a} to \meta{b}. Here is an example:

当你写下\meta{a}|!|\meta{dimension}|!|\meta{b}时,意味着以下内容:在从\meta{a}到\meta{b}的直线上,距离\meta{a} \meta{dimension}的点。以下是一个示例:

%
\begin{codeexample}[preamble={\usetikzlibrary{calc}}]
\begin{tikzpicture}
  \draw [help lines] (0,0) grid (3,2);

  \draw (1,0) -- (3,2);

  \foreach \i in {0cm,1cm,15mm}
    \node at ($(1,0)!\i!(3,2)$) {\i};
\end{tikzpicture}
\end{codeexample}

As before, if you use a \meta{angle}, the \meta{second coordinate} is rotated
by this much around the \meta{coordinate} before it is used.

与之前一样,如果使用\meta{angle},\meta{second coordinate}在使用之前会旋转这么多。

The combination of an \meta{angle} of |90| degrees with a distance can be used
to ``offset'' a point relative to a line. Suppose, for instance, that you have
computed a point |(c)| that lies somewhere on a line from |(a)| to~|(b)| and
you now wish to offset this point by |1cm| so that the distance from this
offset point to the line is |1cm|. This can be achieved as follows:

将\meta{angle}为|90|度与一个距离结合起来,可以用于相对于一条直线进行“偏移”一个点。例如,假设你计算出一个点|(c)|位于从|(a)|到|(b)|的某条线上,现在你希望将此点偏移|1cm|,使得该偏移点到该线的距离为|1cm|。可以按如下方式实现:

%
\begin{codeexample}[preamble={\usetikzlibrary{calc}}]
\begin{tikzpicture}
  \draw [help lines] (0,0) grid (3,2);

  \coordinate (a) at (1,0);
  \coordinate (b) at (3,1);

  \draw (a) -- (b);

  \coordinate (c) at ($ (a)!.25!(b) $);
  \coordinate (d) at ($ (c)!1cm!90:(b) $);

  \draw [<->] (c) -- (d) node [sloped,midway,above] {1cm};
\end{tikzpicture}
\end{codeexample}


\subsubsection{The Syntax of Projection Modifiers\\投影修饰符的语法}

The projection modifier is also similar to the above modifiers: It also gives a
point on a line from the \meta{coordinate} to the \meta{second coordinate}.
However, the \meta{number} or \meta{dimension} is replaced by a
\meta{projection coordinate}:


投影修饰符与上述修饰符类似:它也会给出从\meta{坐标}到\meta{第二个坐标}的线上的一个点。然而,\meta{数字}或\meta{尺寸}被替换为\meta{投影坐标}:
%
\begin{quote}
    \meta{coordinate}\declare{|!|\meta{projection coordinate}|!|\opt{\meta{angle}|:|}\meta{second coordinate}}
\end{quote}

Here is an example:

下面是一个示例:
%
\begin{codeexample}[code only]
(1,2)!(0,5)!(3,4)
\end{codeexample}

The effect is the following: We project the \meta{projection coordinate}
orthogonally onto the line from \meta{coordinate} to \meta{second coordinate}.
This makes it easy to compute projected points:

效果如下:我们将\meta{投影坐标}在垂直于从\meta{坐标}到\meta{第二个坐标}的直线上进行投影。这使得计算投影点变得容易:
%
\begin{codeexample}[preamble={\usetikzlibrary{calc}}]
\begin{tikzpicture}
  \draw [help lines] (0,0) grid (3,2);

  \coordinate (a) at (0,1);
  \coordinate (b) at (3,2);
  \coordinate (c) at (2.5,0);

  \draw (a) -- (b) -- (c) -- cycle;

  \draw[red]    (a) -- ($(b)!(a)!(c)$);
  \draw[orange] (b) -- ($(a)!(b)!(c)$);
  \draw[blue]   (c) -- ($(a)!(c)!(b)$);
\end{tikzpicture}
\end{codeexample}
