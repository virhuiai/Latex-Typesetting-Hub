
\mshowc{section}
\mshowc{subsection}
\mshowc{subsubsection}
\subsubsection{Intersecting a Line and a Circle\\直线和圆的相交}

The next step in the construction is to draw a circle around $B$ through $C$,
which is easy enough to do using the |circle through| option. Extending the
lines $DA$ and $DB$ can be done using partway calculations, but this time with
a part value outside the range $[0,1]$:
%
\begin{codeexample}[preamble={\usetikzlibrary{calc,through}}]
\begin{tikzpicture}
  \coordinate [label=left:$A$]  (A) at (0,0);
  \coordinate [label=right:$B$] (B) at (0.75,0.25);
  \coordinate [label=above:$C$] (C) at (1,1.5);
  \draw (A) -- (B) -- (C);
  \coordinate [label=above:$D$] (D) at
    ($ (A) ! .5 ! (B) ! {sin(60)*2} ! 90:(B) $) {};
  \node (H) [label=135:$H$,draw,circle through=(C)] at (B) {};
  \draw (D) -- ($ (D) ! 3.5 ! (B) $) coordinate [label=below:$F$] (F);
  \draw (D) -- ($ (D) ! 2.5 ! (A) $) coordinate [label=below:$E$] (E);
\end{tikzpicture}
\end{codeexample}

We now face the problem of finding the point $G$, which is the intersection of
the line $BF$ and the circle $H$. One way is to use yet another variant of the
partway computation: Normally, a partway computation has the form
\meta{p}|!|\meta{factor}|!|\meta{q}, resulting in the point
$(1-\meta{factor})\meta{p} + \meta{factor}\meta{q}$. Alternatively, instead of
\meta{factor} you can also use a \meta{dimension} between the points. In this
case, you get the point that is \meta{dimension} away from \meta{p} on the
straight line to \meta{q}.

We know that the point $G$ is on the way from $B$ to $F$. The distance is given
by the radius of the circle~$H$. Here is the code for computing $H$:
%
{\ifpgfmanualexternalize\tikzexternaldisable\fi
\begin{codeexample}[
    preamble={\usetikzlibrary{calc,through}},
    pre={\begin{tikzpicture}
  \coordinate [label=left:$A$]  (A) at (0,0);
  \coordinate [label=right:$B$] (B) at (0.75,0.25);
  \coordinate [label=above:$C$] (C) at (1,1.5);
  \draw (A) -- (B) -- (C);
  \coordinate [label=above:$D$] (D) at
    ($ (A) ! .5 ! (B) ! {sin(60)*2} ! 90:(B) $) {};
  \draw (D) -- ($ (D) ! 3.5 ! (B) $) coordinate [label=below:$F$] (F);
  \draw (D) -- ($ (D) ! 2.5 ! (A) $) coordinate [label=below:$E$] (E);},
    post={\end{tikzpicture}},
]
  \node (H) [label=135:$H$,draw,circle through=(C)] at (B) {};
  \path let \p1 = ($ (B) - (C) $) in
    coordinate [label=left:$G$] (G) at ($ (B) ! veclen(\x1,\y1) ! (F) $);
  \fill[red,opacity=.5] (G) circle (2pt);
\end{codeexample}

However, there is a simpler way: We can simply name the path of the circle and
of the line in question and then use |name intersections| to compute the
intersections.
%
\begin{codeexample}[
    preamble={\usetikzlibrary{calc,intersections,through}},
    pre={\begin{tikzpicture}
  \coordinate [label=left:$A$]  (A) at (0,0);
  \coordinate [label=right:$B$] (B) at (0.75,0.25);
  \coordinate [label=above:$C$] (C) at (1,1.5);
  \draw (A) -- (B) -- (C);
  \coordinate [label=above:$D$] (D) at
    ($ (A) ! .5 ! (B) ! {sin(60)*2} ! 90:(B) $) {};
  \draw (D) -- ($ (D) ! 3.5 ! (B) $) coordinate [label=below:$F$] (F);
  \draw (D) -- ($ (D) ! 2.5 ! (A) $) coordinate [label=below:$E$] (E);},
    post={\end{tikzpicture}},
]
  \node (H) [name path=H,label=135:$H$,draw,circle through=(C)] at (B) {};
  \path [name path=B--F] (B) -- (F);
  \path [name intersections={of=H and B--F,by={[label=left:$G$]G}}];
  \fill[red,opacity=.5] (G) circle (2pt);
\end{codeexample}
}%


\subsubsection{The Complete Code}

\begin{codeexample}[pre={\pgfmathsetseed{1}},preamble={\usetikzlibrary{calc,intersections,through}}]
\begin{tikzpicture}[thick,help lines/.style={thin,draw=black!50}]
  \def\A{\textcolor{orange}{$A$}}   \def\B{\textcolor{input}{$B$}}
  \def\C{\textcolor{input}{$C$}}    \def\D{$D$}
  \def\E{$E$}                       \def\F{$F$}
  \def\G{$G$}                       \def\H{$H$}
  \def\K{$K$}                       \def\L{\textcolor{output}{$L$}}

  \colorlet{input}{blue!80!black}    \colorlet{output}{red!70!black}

  \coordinate [label=left:\A]  (A) at ($ (0,0) + .1*(rand,rand) $);
  \coordinate [label=right:\B] (B) at ($ (1,0.2) + .1*(rand,rand) $);
  \coordinate [label=above:\C] (C) at ($ (1,2) + .1*(rand,rand) $);

  \draw [input] (B) -- (C);
  \draw [help lines] (A) -- (B);

  \coordinate [label=above:\D] (D) at ($ (A)!.5!(B) ! {sin(60)*2} ! 90:(B) $);

  \draw [help lines] (D) -- ($ (D)!3.75!(A) $) coordinate [label=-135:\E] (E);
  \draw [help lines] (D) -- ($ (D)!3.75!(B) $) coordinate [label=-45:\F] (F);

  \node (H) at (B) [name path=H,help lines,circle through=(C),draw,label=135:\H] {};
  \path [name path=B--F] (B) -- (F);
  \path [name intersections={of=H and B--F,by={[label=right:\G]G}}];

  \node (K) at (D) [name path=K,help lines,circle through=(G),draw,label=135:\K] {};
  \path [name path=A--E] (A) -- (E);
  \path [name intersections={of=K and A--E,by={[label=below:\L]L}}];

  \draw [output] (A) -- (L);

  \foreach \point in {A,B,C,D,G,L}
    \fill [black,opacity=.5] (\point) circle (2pt);

  % \node ...
\end{tikzpicture}
\end{codeexample}
