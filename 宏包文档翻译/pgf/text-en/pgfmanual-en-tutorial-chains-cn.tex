% % Copyright 2019 by Till Tantau
% %
% % This file may be distributed and/or modified
% %
% % 1. under the LaTeX Project Public License and/or
% % 2. under the GNU Free Documentation License.
% %
% % See the file doc/generic/pgf/licenses/LICENSE for more details.


% \section{Tutorial: Diagrams as Simple Graphs\\教程:将图表绘制为简单图形}

% In this tutorial we have a look at how graphs and matrices can be used to
% typeset a diagram.

% 在本教程中,我们将探讨如何使用图和矩阵来绘制图表。

% Ilka, who just got tenure for her professorship on Old and Lovable Programming
% Languages, has recently dug up a technical report entitled \emph{The
% Programming Language Pascal} in the dusty cellar of the library of her
% university. Having been created in the good old times using pens and rules, it
% looks like this%
% \footnote{The shown diagram was not scanned, but rather typeset using
% \tikzname. The jittering lines were created using the |random steps|
% decoration.}:


% Ilka刚刚因其对古老且可爱的编程语言的教授职位获得终身教职,并最近在大学图书馆的地下室中挖掘出一份名为《Pascal编程语言》的技术报告。这份报告是在过去使用钢笔和规则制作的,看起来像这样\footnote{所示的图表不是扫描的,而是使用\tikzname\ 排版得到的。抖动的线是使用|random steps|装饰实现的。}:


% {
%   \tikzset{
%     nonterminal/.style={
%       % The shape:
%       rectangle,
%       % The size:
%       minimum size=6mm,
%       % The border:
%       very thick,
%       draw=red!50!black!50,         % 50% red and 50% black,
%                                     % and that mixed with 50% white
%       % The filling:
%       top color=white,              % a shading that is white at the top...
%       bottom color=red!50!black!20, % and something else at the bottom
%       % Font
%       font=\itshape
%     },
%     terminal/.style={
%       % The shape:
%       rounded rectangle,
%       minimum size=6mm,
%       % The rest
%       very thick,draw=black!50,
%       top color=white,bottom color=black!20,
%       font=\ttfamily},
%     skip loop/.style={to path={-- ++(0,#1) -| (\tikztotarget)}}
%   }
%   \tikzset{terminal/.append style={text height=1.5ex,text depth=.25ex}}
%   \tikzset{nonterminal/.append style={text height=1.5ex,text depth=.25ex}}
%   \pgfmathsetseed{1}
% \medskip
% \noindent\begin{tikzpicture}[
%   >=latex,thick,
%   /pgf/every decoration/.style={/tikz/sharp corners},
%   fuzzy/.style={decorate,decoration={random steps,segment length=0.5mm,amplitude=0.15pt}},
%   minimum size=6mm,line join=round,line cap=round,
%   terminal/.style={rectangle,draw,fill=white,fuzzy,rounded corners=3mm},
%   nonterminal/.style={rectangle,draw,fill=white,fuzzy},
%   node distance=4mm]

%   \ttfamily
%   \begin{scope}[start chain,
%                 every node/.style={on chain},
%                 terminal/.append style={join=by {->,shorten >=-1pt,fuzzy,decoration={post length=4pt}}},
%                 nonterminal/.append style={join=by {->,shorten >=-1pt,fuzzy,decoration={post length=4pt}}},
%                 support/.style={coordinate,join=by fuzzy}]
%     \node [support]             (start)        {};
%     \node [nonterminal]                        {unsigned integer};
%     \node [support]             (after ui)     {};
%     \node [terminal]                           {.};
%     \node [support]             (after dot)    {};
%     \node [terminal]                           {digit};
%     \node [support]             (after digit)  {};
%     \node [support]             (skip)         {};
%     \node [support]             (before E)     {};
%     \node [terminal]                           {E};
%     \node [support]             (after E)      {};
%     \node [support,xshift=5mm]  (between)      {};
%     \node [support,xshift=5mm]  (before last)  {};
%     \node [nonterminal]                        {unsigned integer};
%     \node [support]             (after last)   {};
%     \node [coordinate,join=by ->] (end)          {};
%   \end{scope}
%   \node (plus)  [terminal,above=of between] {+};
%   \node (minus) [terminal,below=of between] {-};

%   \begin{scope}[->,decoration={post length=4pt},rounded corners=2mm,every path/.style=fuzzy]
%     \draw (after ui)    -- +(0,.7)  -| (skip);
%     \draw (after digit) -- +(0,-.7) -| (after dot);
%     \draw (before E)    -- +(0,-1.2) -| (after last);
%     \draw (after E)     |- (plus);
%     \draw (plus)        -| (before last);
%     \draw (after E)     |- (minus);
%     \draw (minus)       -| (before last);
%   \end{scope}
% \end{tikzpicture}
% \medskip

% For her next lecture, Ilka decides to redo this diagram, but this time perhaps
% a bit cleaner and perhaps also bit ``cooler''.

% 为了下一堂课,Ilka决定重新制作这个图表,但这次可能会更加清晰,也更加“酷”。

% \medskip
% \noindent\begin{tikzpicture}[point/.style={coordinate},>={Stealth[round]},thick,draw=black!50,
%                     tip/.style={->,shorten >=1pt},every join/.style={rounded corners},
%                     hv path/.style={to path={-| (\tikztotarget)}},
%                     vh path/.style={to path={|- (\tikztotarget)}}]
%   \matrix[column sep=4mm] {
%     % First row:
%     & & & & & & &  & & & & \node (plus) [terminal] {+};\\
%     % Second row:
%     \node (p1) [point]  {}; &    \node (ui1)   [nonterminal] {unsigned integer}; &
%     \node (p2) [point]  {}; &    \node (dot)   [terminal]    {.};                &
%     \node (p3) [point]  {}; &    \node (digit) [terminal]    {digit};            &
%     \node (p4) [point]  {}; &    \node (p5)    [point]  {};                      &
%     \node (p6) [point]  {}; &    \node (e)     [terminal]    {E};                &
%     \node (p7) [point]  {}; &                                                    &
%     \node (p8) [point]  {}; &    \node (ui2)   [nonterminal] {unsigned integer}; &
%     \node (p9) [point]  {}; &    \node (p10)   [point]       {};\\
%     % Third row:
%     & & & & & & &  & & & & \node (minus)[terminal] {-};\\
%   };

%   { [start chain]
%     \chainin (p1);
%     \chainin (ui1)   [join=by tip];
%     \chainin (p2)    [join];
%     \chainin (dot)   [join=by tip];
%     \chainin (p3)    [join];
%     \chainin (digit) [join=by tip];
%     \chainin (p4)    [join];
%     { [start branch=digit loop]
%       \chainin (p3) [join=by {skip loop=-6mm,tip}];
%     }
%     \chainin (p5)    [join,join=with p2 by {skip loop=6mm,tip}];
%     \chainin (p6)    [join];
%     \chainin (e)     [join=by tip];
%     \chainin (p7)    [join];
%     { [start branch=plus]
%       \chainin (plus)  [join=by {vh path,tip}];
%       \chainin (p8)    [join=by {hv path,tip}];
%     }
%     { [start branch=minus]
%       \chainin (minus) [join=by {vh path,tip}];
%       \chainin (p8)    [join=by {hv path,tip}];
%     }
%     \chainin (p8)    [join];
%     \chainin (ui2)   [join=by tip];
%     \chainin (p9)    [join,join=with p6 by {skip loop=-11mm,tip}];
%     \chainin (p10)   [join=by tip];
%   }
% \end{tikzpicture}
% }\medskip

% Having read the previous tutorials, Ilka knows already how to set up the
% environment for her diagram, namely using a |tikzpicture| environment. She
% wonders which libraries she will need. She decides that she will postpone the
% decision and add the necessary libraries as needed as she constructs the
% picture.

% 在阅读了之前的教程后,Ilka已经知道如何设置图表的环境,即使用|tikzpicture|环境。她想知道她将需要哪些库。她决定在构建图表时根据需要添加所需的库。



% \subsection{Styling the Nodes\\为节点设置样式}

% The bulk of this tutorial will be about arranging the nodes and connecting them
% using chains, but let us start with setting up styles for the nodes.

% 本教程的主要部分将涉及安排节点并使用链将它们连接起来,但让我们首先为节点设置样式。

% There are two kinds of nodes in the diagram, namely what theoreticians like to
% call \emph{terminals} and \emph{nonterminals}. For the terminals, Ilka decides
% to use a black color, which visually shows that ``nothing needs to be done
% about them''. The nonterminals, which still need to be ``processed'' further,
% get a bit of red mixed in.

% 图表中有两种类型的节点,即理论家所称的“终端”和“非终端”。对于终端,Ilka决定使用黑色,以视觉上表明“无需对其做任何处理”。非终端仍需要进一步的“处理”,因此会混入一些红色。

% Ilka starts with the simpler nonterminals, as there are no rounded corners
% involved. Naturally, she sets up a style:

% Ilka从较简单的非终端开始,因为它们没有圆角。自然地,她设置了一个样式:

% %
% \begin{codeexample}[preamble={\usetikzlibrary{positioning}}]
% \begin{tikzpicture}[
%     nonterminal/.style={
%       % The shape:
%       rectangle,
%       % The size:
%       minimum size=6mm,
%       % The border:
%       very thick,
%       draw=red!50!black!50,         % 50% red and 50% black,
%                                     % and that mixed with 50% white
%       % The filling:
%       top color=white,              % a shading that is white at the top...
%       bottom color=red!50!black!20, % and something else at the bottom
%       % Font
%       font=\itshape
%     }]
%   \node [nonterminal] {unsigned integer};
% \end{tikzpicture}
% \end{codeexample}
% %
% Ilka is pretty proud of the use of the |minimum size| option: As the name
% suggests, this option ensures that the node is at least 6mm by 6mm, but it will
% expand in size as necessary to accommodate longer text. By giving this option
% to all nodes, they will all have the same height of 6mm.

% 对于|minimum size|选项,Ilka感到非常自豪:正如其名称所示,该选项确保节点至少为6mm乘6mm,但它会根据需要扩展大小以适应更长的文本。通过将此选项应用于所有节点,它们将都具有相同的高度为6mm。

% Styling the terminals is a bit more difficult because of the round corners.
% Ilka has several options how she can achieve them. One way is to use the
% |rounded corners| option. It gets a dimension as parameter and causes all
% corners to be replaced by little arcs with the given dimension as radius. By
% setting the radius to 3mm, she will get exactly what she needs: circles, when
% the shapes are, indeed, exactly 6mm by 6mm and otherwise half circles on the
% sides:

% 对终端的样式设置要困难一些,因为要处理圆角。Ilka有几种方法可以实现它们。一种方法是使用|rounded corners|选项。它接受一个尺寸作为参数,并将所有角替换为具有给定尺寸作为半径的小弧。通过将半径设置为3mm,她将得到所需的效果:当形状确实为6mm乘6mm时,得到圆形,否则在两侧得到半圆:
% %
% \begin{codeexample}[preamble={\usetikzlibrary{positioning}}]
% \begin{tikzpicture}[node distance=5mm,
%                     terminal/.style={
%                       % The shape:
%                       rectangle,minimum size=6mm,rounded corners=3mm,
%                       % The rest
%                       very thick,draw=black!50,
%                       top color=white,bottom color=black!20,
%                       font=\ttfamily}]
%   \node (dot)   [terminal]                {.};
%   \node (digit) [terminal,right=of dot]   {digit};
%   \node (E)     [terminal,right=of digit] {E};
% \end{tikzpicture}
% \end{codeexample}

% Another possibility is to use a shape that is specially made for typesetting
% rectangles with arcs on the sides (she has to use the |shapes.misc| library to
% use it). This shape gives Ilka much more control over the appearance. For
% instance, she could have an arc only on the left side, but she will not need
% this.

% 另一种可能性是使用一种专门用于绘制带有侧边弧的矩形的形状(她必须使用|shapes.misc|库才能使用它)。该形状使Ilka对外观有更多的控制权。例如,她可以仅在左侧使用弧,但她不需要这样做。

% %
% \begin{codeexample}[preamble={\usetikzlibrary{positioning,shapes.misc}}]
% \begin{tikzpicture}[node distance=5mm,
%                     terminal/.style={
%                       % The shape:
%                       rounded rectangle,
%                       minimum size=6mm,
%                       % The rest
%                       very thick,draw=black!50,
%                       top color=white,bottom color=black!20,
%                       font=\ttfamily}]
%   \node (dot)   [terminal]                {.};
%   \node (digit) [terminal,right=of dot]   {digit};
%   \node (E)     [terminal,right=of digit] {E};
% \end{tikzpicture}
% \end{codeexample}
% %
% At this point, she notices a problem. The baseline of the text in the nodes is
% not aligned:

% 此时,她注意到了一个问题。节点中的文本的基线没有对齐:
% %
% \begin{codeexample}[setup code,hidden]
% \tikzset{
%   terminal/.style={
%     % The shape:
%     rounded rectangle,
%     minimum size=6mm,
%     % The rest
%     very thick,draw=black!50,
%     top color=white,bottom color=black!20,
%     font=\ttfamily},
% }
% \end{codeexample}
% %
% \begin{codeexample}[preamble={\usetikzlibrary{calc,positioning,shapes.misc}}]
% \begin{tikzpicture}[node distance=5mm]
%   \node (dot)   [terminal]                {.};
%   \node (digit) [terminal,right=of dot]   {digit};
%   \node (E)     [terminal,right=of digit] {E};

%   \draw [help lines] let \p1 = (dot.base),
%                          \p2 = (digit.base),
%                          \p3 = (E.base)
%                      in (-.5,\y1) -- (3.5,\y1)
%                         (-.5,\y2) -- (3.5,\y2)
%                         (-.5,\y3) -- (3.5,\y3);
% \end{tikzpicture}
% \end{codeexample}
% %
% \noindent (Ilka has moved the style definition to the preamble by saying
% |\tikzset{terminal/.style=...}|, so that she can use it in all pictures.)

% \noindent(Ilka将样式定义移到导言区中,使用|\tikzset{terminal/.style=...}|,以便在所有图表中都可以使用它。)

% For the |digit| and the |E| the difference in the baselines is almost
% imperceptible, but for the dot the problem is quite severe: It looks more like
% a multiplication dot than a period.

% 对于|digit|和|E|,基线之间的差异几乎不可见,但对于点号,问题相当严重:它看起来更像是乘法点而不是句点。

% Ilka toys with the idea of using the |base right=of...| option rather than
% |right=of...| to align the nodes in such a way that the baselines are all on
% the same line (the |base right| option places a node right of something so that
% the baseline is right of the baseline of the other object). However, this does
% not have the desired effect:

% Ilka心生一个想法,即使用|base right=of...|选项而不是|right=of...|选项来对齐节点,使得基线都在同一条线上(|base right|选项将一个节点放置在另一个节点的右侧,以使得其基线位于另一个对象的基线的右侧)。然而,这没有产生期望的效果:

% %
% \begin{codeexample}[preamble={\usetikzlibrary{positioning,shapes.misc}}]
% \begin{tikzpicture}[node distance=5mm]
%   \node (dot)   [terminal]                     {.};
%   \node (digit) [terminal,base right=of dot]   {digit};
%   \node (E)     [terminal,base right=of digit] {E};
% \end{tikzpicture}
% \end{codeexample}
% %

% The nodes suddenly ``dance around''! There is no hope of changing the position
% of text inside a node using anchors. Instead, Ilka must use a trick: The
% problem of mismatching baselines is caused by the fact that |.| and |digit| and
% |E| all have different heights and depth. If they all had the same, they would
% all be positioned vertically in the same manner. So, all Ilka needs to do is to
% use the |text height| and |text depth| options to explicitly specify a height
% and depth for the nodes.

% 节点突然“乱动”起来!无法通过锚点更改节点内文本的位置。相反,Ilka必须使用一个技巧:基线不匹配的问题是由于|.|和|digit|以及|E|的高度和深度不同所导致的。如果它们都相同,它们将以相同的方式在垂直方向上定位。因此,Ilka所需做的就是使用|text height|和|text depth|选项明确指定节点的高度和深度。



% %
% \begin{codeexample}[preamble={\usetikzlibrary{positioning,shapes.misc}}]
% \begin{tikzpicture}[node distance=5mm,
%                     text height=1.5ex,text depth=.25ex]
%   \node (dot)   [terminal]                {.};
%   \node (digit) [terminal,right=of dot]   {digit};
%   \node (E)     [terminal,right=of digit] {E};
% \end{tikzpicture}
% \end{codeexample}


% \subsection{Aligning the Nodes Using Positioning Options\\使用定位选项对齐节点}

% Ilka now has the ``styling'' of the nodes ready. The next problem is to place
% them in the right places. There are several ways to do this. The most
% straightforward is to simply explicitly place the nodes at certain coordinates
% ``calculated by hand''. For very simple graphics this is perfectly alright, but
% it has several disadvantages:

% Ilka现在已经准备好节点的“样式”。下一个问题是将它们放置在正确的位置。有几种方法可以实现。最直接的方法是直接在“手工计算”的某些坐标处放置节点。对于非常简单的图形,这是完全可以的,但它有几个缺点:
% %
% \begin{enumerate}
%     \item For more difficult graphics, the calculation may become
%         complicated.

%         对于更复杂的图形,计算可能变得复杂。
%     \item Changing the text of the nodes may make it necessary to recalculate
%         the coordinates.

%         更改节点的文本可能需要重新计算坐标。
%     \item The source code of the graphic is not very clear since the
%         relationships between the positions of the nodes are not made
%         explicit.

%         图形的源代码不太清晰,因为节点之间的位置关系没有明确表达。
% \end{enumerate}

% For these reasons, Ilka decides to try out different ways of arranging the
% nodes on the page.

% 出于这些原因,Ilka决定尝试在页面上排列节点的不同方式。

% The first method is the use of \emph{positioning options}. To use them, you
% need to load the |positioning| library. This gives you access to advanced
% implementations of options like |above| or |left|, since you can now say
% |above=of some node| in order to place a node above of |some node|, with the
% borders separated by |node distance|.

% 第一种方法是使用\emph{定位选项}。要使用它们,你需要加载|positioning|库。这使得你可以访问高级选项的实现,如|above|或|left|,因为现在你可以说|above=of some node|以将一个节点放置在|some node|的上方,边界由|node distance|分离。

% Ilka can use this to draw the place the nodes in a long row:

% Ilka可以使用这个选项将节点放置在一行中:
% %
% \begin{codeexample}[setup code,hidden]
% \tikzset{
%   nonterminal/.style={
%     % The shape:
%     rectangle,
%     % The size:
%     minimum size=6mm,
%     % The border:
%     very thick,
%     draw=red!50!black!50,         % 50% red and 50% black,
%                                   % and that mixed with 50% white
%     % The filling:
%     top color=white,              % a shading that is white at the top...
%     bottom color=red!50!black!20, % and something else at the bottom
%     % Font
%     font=\itshape,
%   },
% }
% \tikzset{
%   terminal/.append style={text height=1.5ex,text depth=.25ex},
%   nonterminal/.append style={text height=1.5ex,text depth=.25ex},
% }
% \end{codeexample}
% %
% \begin{codeexample}[preamble={\usetikzlibrary{positioning,shapes.misc}}]
% \begin{tikzpicture}[node distance=5mm and 5mm]
%   \node (ui1)   [nonterminal]                     {unsigned integer};
%   \node (dot)   [terminal,right=of ui1]           {.};
%   \node (digit) [terminal,right=of dot]           {digit};
%   \node (E)     [terminal,right=of digit]         {E};
%   \node (plus)  [terminal,above right=of E]       {+};
%   \node (minus) [terminal,below right=of E]       {-};
%   \node (ui2)   [nonterminal,below right=of plus] {unsigned integer};
% \end{tikzpicture}
% \end{codeexample}

% For the plus and minus nodes, Ilka is a bit startled by their placements.
% Shouldn't they be more to the right? The reason they are placed in that manner
% is the following: The |north east| anchor of the |E| node lies at the ``upper
% start of the right arc'', which, a bit unfortunately in this case, happens to
% be the top of the node. Likewise, the |south west| anchor of the |+| node is
% actually at its bottom and, indeed, the horizontal and vertical distances
% between the top of the |E| node and the bottom of the |+| node are both 5mm.

% 对于加号和减号节点,Ilka对它们的位置有些惊讶。它们不应该更靠右吗?它们被放置在那个位置的原因是:|E|节点的|north east|锚点位于“右弧的上部起点”,这在这种情况下有点不幸,恰好是节点的顶部。同样,|+|节点的|south west|锚点实际上位于其底部,并且实际上|E|节点的顶部和|+|节点的底部之间的水平和垂直距离都为5mm。

% There are several ways of fixing this problem. The easiest way is to simply add
% a little bit of horizontal shift by hand:

% 有几种方法可以解决这个问题。最简单的方法是通过手动添加一点水平偏移量来解决:

% %
% \begin{codeexample}[preamble={\usetikzlibrary{positioning,shapes.misc}}]
% \begin{tikzpicture}[node distance=5mm and 5mm]
%   \node (E)     [terminal]                                   {E};
%   \node (plus)  [terminal,above right=of E,xshift=5mm]       {+};
%   \node (minus) [terminal,below right=of E,xshift=5mm]       {-};
%   \node (ui2)   [nonterminal,below right=of plus,xshift=5mm] {unsigned integer};
% \end{tikzpicture}
% \end{codeexample}

% A second way is to revert back to the idea of using a normal rectangle for the
% terminals, but with rounded corners. Since corner rounding does not affect
% anchors, she gets the following result:

% 第二种方法是回到使用普通矩形来表示终端的想法,但带有圆角。由于角的圆角不会影响锚点,因此可以得到以下结果:

% %
% \begin{codeexample}[preamble={\usetikzlibrary{positioning,shapes.misc}}]
% \begin{tikzpicture}[node distance=5mm and 5mm,terminal/.append style={rectangle,rounded corners=3mm}]
%   \node (E)     [terminal]                        {E};
%   \node (plus)  [terminal,above right=of E]       {+};
%   \node (minus) [terminal,below right=of E]       {-};
%   \node (ui2)   [nonterminal,below right=of plus] {unsigned integer};
% \end{tikzpicture}
% \end{codeexample}
% %
% A third way is to use matrices, which we will do later.

% 第三种方法是使用后面将要介绍的矩阵。

% Now that the nodes have been placed, Ilka needs to add connections. Here, some
% connections are more difficult than others. Consider for instance the
% ``repeat'' line around the |digit|. One way of describing this line is to say
% ``it starts a little to the right of |digit| than goes down and then goes to
% the left and finally ends at a point a little to the left of |digit|''. Ilka
% can put this into code as follows:

% 现在节点已经放置好了,Ilka需要添加连接。这里,有些连接比其他连接更难处理。例如,考虑围绕|digit|的“repeat”线条。描述这条线条的一种方法是说“它从|digit|的右侧稍微向右移动,然后向下移动,然后向左移动,最后以稍微在|digit|的左侧的一个点结束”。Ilka可以将其编码如下:

% %
% \begin{codeexample}[preamble={\usetikzlibrary{calc,positioning,shapes.misc}}]
% \begin{tikzpicture}[node distance=5mm and 5mm]
%   \node (dot)   [terminal]                        {.};
%   \node (digit) [terminal,right=of dot]           {digit};
%   \node (E)     [terminal,right=of digit]         {E};

%   \path (dot)   edge[->] (digit)  % simple edges
%         (digit) edge[->] (E);

%   \draw [->]
%      % start right of digit.east, that is, at the point that is the
%      % linear combination of digit.east and the vector (2mm,0pt). We
%      % use the ($ ... $) notation for computing linear combinations
%      ($ (digit.east) + (2mm,0) $)
%      % Now go down
%      -- ++(0,-.5)
%      % And back to the left of digit.west
%      -| ($ (digit.west) - (2mm,0) $);
% \end{tikzpicture}
% \end{codeexample}

% Since Ilka needs this ``go up/down then horizontally and then up/down to a
% target'' several times, it seems sensible to define a special \emph{to-path}
% for this. Whenever the |edge| command is used, it simply adds the current value
% of |to path| to the path. So, Ilka can set up a style that contains the correct
% path:
% %

% 由于Ilka需要多次执行“向上/向下,然后水平,再向上/向下到目标”操作,因此定义一个特殊的\emph{to-path}似乎是合理的。每当使用|edge|命令时,它会将当前的|to path|值添加到路径中。因此,Ilka可以设置一个包含正确路径的样式:


% \begin{codeexample}[preamble={\usetikzlibrary{calc,positioning,shapes.misc}}]
% \begin{tikzpicture}[node distance=5mm and 5mm,
%     skip loop/.style={to path={-- ++(0,-.5) -| (\tikztotarget)}}]
%   \node (dot)   [terminal]                        {.};
%   \node (digit) [terminal,right=of dot]           {digit};
%   \node (E)     [terminal,right=of digit]         {E};

%   \path (dot)   edge[->]           (digit)  % simple edges
%         (digit) edge[->]           (E)
%         ($ (digit.east) + (2mm,0) $)
%                 edge[->,skip loop] ($ (digit.west) - (2mm,0) $);
% \end{tikzpicture}
% \end{codeexample}

% Ilka can even go a step further and make her |skip loop| style parameterized.
% For this, the skip loop's vertical offset is passed as parameter |#1|. Also, in
% the following code Ilka specifies the start and targets differently, namely as
% the positions that are ``in the middle between the nodes''.

% Ilka甚至可以进一步使她的|skip loop|样式参数化。为此,跳过循环的垂直偏移作为参数|#1|传递。此外,在下面的代码中,Ilka将开始和目标指定方式进行了不同的指定,即作为“在节点之间的中间位置”。

% %
% \begin{codeexample}[preamble={\usetikzlibrary{calc,positioning,shapes.misc}}]
% \begin{tikzpicture}[node distance=5mm and 5mm,
%     skip loop/.style={to path={-- ++(0,#1) -| (\tikztotarget)}}]
%   \node (dot)   [terminal]                        {.};
%   \node (digit) [terminal,right=of dot]           {digit};
%   \node (E)     [terminal,right=of digit]         {E};

%   \path (dot)   edge[->]                (digit)  % simple edges
%         (digit) edge[->]                (E)
%         ($ (digit.east)!.5!(E.west) $)
%                 edge[->,skip loop=-5mm] ($ (digit.west)!.5!(dot.east) $);
% \end{tikzpicture}
% \end{codeexample}



% \subsection{Aligning the Nodes Using Matrices\\使用矩阵对节点进行对齐}

% Ilka is still bothered a bit by the placement of the plus and minus nodes.
% Somehow, having to add an explicit |xshift| seems too much like cheating.

% Ilka仍然对加号和减号节点的位置感到困扰。不知何故,需要显式添加|xshift|似乎太像作弊了。

% A perhaps better way of positioning the nodes is to use a \emph{matrix}. In
% \tikzname\ matrices can be used to align quite arbitrary graphical objects in
% rows and columns. The syntax is very similar to the use of arrays and tables in
% \TeX\ (indeed, internally \TeX\ tables are used, but a lot of stuff is going on
% additionally).

% 一种更好的定位节点的方法是使用\emph{矩阵}。在\tikzname 中,矩阵可用于在行和列中对几乎任意图形对象进行对齐。语法与在\TeX 中使用数组和表格非常相似(实际上,内部使用了\TeX 的表格,但还有很多其他操作)。

% In Ilka's graphic, there will be three rows: One row containing only the plus
% node, one row containing the main nodes and one row containing only the minus
% node.

% 在Ilka的图形中,将有三行:一行仅包含加号节点,一行包含主要节点,一行仅包含减号节点。






% \begin{codeexample}[preamble={\usetikzlibrary{shapes.misc}}]
% \begin{tikzpicture}
%   \matrix[row sep=1mm,column sep=5mm] {
%     % First row:
%       & & & & \node [terminal] {+}; & \\
%     % Second row:
%     \node [nonterminal] {unsigned integer}; &
%     \node [terminal]    {.};                &
%     \node [terminal]    {digit};            &
%     \node [terminal]    {E};                &
%                                             &
%     \node [nonterminal] {unsigned integer}; \\
%     % Third row:
%       & & & & \node [terminal] {-}; & \\
%   };
% \end{tikzpicture}
% \end{codeexample}
% %
% That was easy! By toying around with the row and columns separations, Ilka can
% achieve all sorts of pleasing arrangements of the nodes.

% 这很简单!通过调整行和列之间的间隔,Ilka可以实现各种令人愉悦的节点布局。

% Ilka now faces the same connecting problem as before. This time, she has an
% idea: She adds small nodes (they will be turned into coordinates later on and
% be invisible) at all the places where she would like connections to start and
% end.

% 现在,Ilka面临的问题与以前一样。这次,她有了一个主意:她在所有她希望连接开始和结束的位置上添加了小节点(它们将在稍后转换为坐标并且不可见)。


% %
% \begin{codeexample}[preamble={\usetikzlibrary{shapes.misc}}]
% \begin{tikzpicture}[point/.style={circle,inner sep=0pt,minimum size=2pt,fill=red},
%                    skip loop/.style={to path={-- ++(0,#1) -| (\tikztotarget)}}]
%   \matrix[row sep=1mm,column sep=2mm] {
%     % First row:
%     & & & & & & &  & & & & \node (plus) [terminal] {+};\\
%     % Second row:
%     \node (p1) [point]  {}; &    \node (ui1)   [nonterminal] {unsigned integer}; &
%     \node (p2) [point]  {}; &    \node (dot)   [terminal]    {.};                &
%     \node (p3) [point]  {}; &    \node (digit) [terminal]    {digit};            &
%     \node (p4) [point]  {}; &    \node (p5)    [point]  {};                      &
%     \node (p6) [point]  {}; &    \node (e)     [terminal]    {E};                &
%     \node (p7) [point]  {}; &                                                    &
%     \node (p8) [point]  {}; &    \node (ui2)   [nonterminal] {unsigned integer}; &
%     \node (p9) [point]  {}; &    \node (p10)   [point]       {};\\
%     % Third row:
%     & & & & & & &  & & & & \node (minus)[terminal] {-};\\
%   };

%   \path (p4) edge [->,skip loop=-5mm] (p3)
%         (p2) edge [->,skip loop=5mm]  (p6);
% \end{tikzpicture}
% \end{codeexample}
% %
% Now, it's only a small step to add all the missing edges.

% 现在,只需要添加所有缺失的边。


% \subsection{The Diagram as a Graph\\将图表视为图}

% Matrices allow Ilka to align the nodes nicely, but the connections are not
% quite perfect. The problem is that the code does not really reflect the paths
% that underlie the diagram. For this, it seems natural enough to Ilka to use the
% |graphs| library since, after all, connecting nodes by edges is exactly what
% happens in a graph. The |graphs| library can both be used to connect nodes that
% have already been created, but it can also be used to create nodes ``on the
% fly'' and these processes can also be mixed.

% 矩阵使Ilka能够很好地对齐节点,但连接并不完美。问题在于代码实际上并没有反映出底层图表的路径。为此,使用|graphs|库似乎是最自然的选择,毕竟,连接节点的边正是图中发生的事情。|graphs|库既可用于连接已经创建的节点,也可用于“即时”创建节点,而且这两个过程也可以混合使用。



% \subsubsection{Connecting Already Positioned Nodes\\连接已经定位的节点}

% Ilka has already a fine method for positioning her nodes (using a |matrix|), so
% all that she needs is an easy way of specifying the edges. For this, she uses
% the |\graph| command (which is actually just a shorthand for |\path graph|). It
% allows her to write down edges between them in a simple way (the macro
% |\matrixcontent| contains exactly the matrix content from the previous example;
% no need to repeat it here):

% Ilka已经有了一种很好的方法来定位她的节点(使用|matrix|),所以她所需要的就是一种简单的方法来指定边。为此,她使用了|\graph|命令(实际上只是|\path graph|的简写)。它允许她以简单的方式写出它们之间的边(宏|\matrixcontent|包含了前面示例中的矩阵内容,这里不需要重复):

% %
% \begin{codeexample}[setup code,hidden]
% \def\matrixcontent{
%   % First row:
%   \& \& \& \& \& \& \&  \& \& \& \& \node (plus) [terminal] {+};\\
%   % Second row:
%   \node (p1) [point]  {}; \&    \node (ui1)   [nonterminal] {unsigned integer}; \&
%   \node (p2) [point]  {}; \&    \node (dot)   [terminal]    {.};                \&
%   \node (p3) [point]  {}; \&    \node (digit) [terminal]    {digit};            \&
%   \node (p4) [point]  {}; \&    \node (p5)    [point]  {};                      \&
%   \node (p6) [point]  {}; \&    \node (e)     [terminal]    {E};                \&
%   \node (p7) [point]  {}; \&                                                    \&
%   \node (p8) [point]  {}; \&    \node (ui2)   [nonterminal] {unsigned integer}; \&
%   \node (p9) [point]  {}; \&    \node (p10)   [point]       {};\\
%   % Third row:
%   \& \& \& \& \& \& \&  \& \& \& \& \node (minus)[terminal] {-};\\
% }
% \end{codeexample}
% %
% \begin{codeexample}[
%     preamble={\usetikzlibrary{graphs,shapes.misc}},
%     pre={\tikzset{ampersand replacement=\&,point/.style={coordinate}}},
% ]
% \begin{tikzpicture}[skip loop/.style={to path={-- ++(0,#1) -| (\tikztotarget)}},
%                     hv path/.style={to path={-| (\tikztotarget)}},
%                     vh path/.style={to path={|- (\tikztotarget)}}]
%   \matrix[row sep=1mm,column sep=2mm] { \matrixcontent };

%   \graph {
%     (p1) -> (ui1) -- (p2) -> (dot) -- (p3) -> (digit) -- (p4)
%          -- (p5)  -- (p6) -> (e) -- (p7) -- (p8) -> (ui2) -- (p9) -> (p10);
%     (p4) ->[skip loop=-5mm]  (p3);
%     (p2) ->[skip loop=5mm]   (p5);
%     (p6) ->[skip loop=-11mm] (p9);
%     (p7) ->[vh path]         (plus)  -> [hv path] (p8);
%     (p7) ->[vh path]         (minus) -> [hv path] (p8);
%   };
% \end{tikzpicture}
% \end{codeexample}

% This is already pretty near to the desired result, just a few ``finishing
% touches'' are needed to style the edges more nicely.

% 这已经非常接近预期的结果了,只需要一些“最后的修饰”来更好地设置边的样式。

% However, Ilka does not have the feeling that the |graph| command is all that
% hot in the example. It certainly does cut down on the number of characters she
% has to write, but the overall graph structure is not that much clear -- it is
% still mainly a list of paths through the graph. It would be nice to specify
% that, say, there the path from |(p7)| sort of splits to |(plus)| and |(minus)|
% and then merges once more at |(p8)|. Also, all these parentheses are bit hard
% to type.

% 然而,Ilka并不认为|graph|命令在这个例子中十分出色。它确实减少了她需要编写的字符数,但整体的图结构并不是那么明显——它仍然主要是图中路径的列表。可以指定,例如,从|(p7)|到|plus|和|minus|之间的路径在某种程度上分叉,然后在|(p8)|处再次合并。而且,所有这些括号都很难键入。

% It turns out that edges from a node to a whole group of nodes are quite easy to
% specify, as shown in the next example. Additionally, by using the
% |use existing nodes| option, Ilka can also leave out all the parentheses
% (again, some options have been moved outside to keep the examples shorter):

% 事实证明,从一个节点到一组节点的边很容易指定,如下面的示例所示。此外,通过使用|use existing nodes|选项,Ilka还可以省略所有括号(再次,一些选项已经移到外面以缩短示例):

% %
% \begin{codeexample}[
%     preamble={\usetikzlibrary{arrows.meta,graphs,shapes.misc}},
%     pre={\tikzset{
%     ampersand replacement=\&,
%     point/.style={coordinate},
%     skip loop/.style={to path={-- ++(0,##1) -| (\tikztotarget)}},
%     hv path/.style={to path={-| (\tikztotarget)}},
%     vh path/.style={to path={|- (\tikztotarget)}},
% }},
% ]
% \begin{tikzpicture}[>={Stealth[round]},thick,black!50,text=black,
%                     every new ->/.style={shorten >=1pt},
%                     graphs/every graph/.style={edges=rounded corners}]
%   \matrix[column sep=4mm] { \matrixcontent };

%   \graph [use existing nodes] {
%     p1 -> ui1 -- p2 -> dot -- p3 -> digit -- p4 -- p5  -- p6 -> e -- p7 -- p8 -> ui2 -- p9 -> p10;
%     p4 ->[skip loop=-5mm]  p3;
%     p2 ->[skip loop=5mm]   p5;
%     p6 ->[skip loop=-11mm] p9;
%     p7 ->[vh path] { plus, minus } -> [hv path] p8;
%   };
% \end{tikzpicture}
% \end{codeexample}

