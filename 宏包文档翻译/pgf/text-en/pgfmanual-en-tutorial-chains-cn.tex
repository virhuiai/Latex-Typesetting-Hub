% % Copyright 2019 by Till Tantau
% %
% % This file may be distributed and/or modified
% %
% % 1. under the LaTeX Project Public License and/or
% % 2. under the GNU Free Documentation License.
% %
% % See the file doc/generic/pgf/licenses/LICENSE for more details.


% \section{Tutorial: Diagrams as Simple Graphs\\教程:将图表绘制为简单图形}

% In this tutorial we have a look at how graphs and matrices can be used to
% typeset a diagram.

% 在本教程中,我们将探讨如何使用图和矩阵来绘制图表。

% Ilka, who just got tenure for her professorship on Old and Lovable Programming
% Languages, has recently dug up a technical report entitled \emph{The
% Programming Language Pascal} in the dusty cellar of the library of her
% university. Having been created in the good old times using pens and rules, it
% looks like this%
% \footnote{The shown diagram was not scanned, but rather typeset using
% \tikzname. The jittering lines were created using the |random steps|
% decoration.}:


% Ilka刚刚因其对古老且可爱的编程语言的教授职位获得终身教职,并最近在大学图书馆的地下室中挖掘出一份名为《Pascal编程语言》的技术报告。这份报告是在过去使用钢笔和规则制作的,看起来像这样\footnote{所示的图表不是扫描的,而是使用\tikzname\ 排版得到的。抖动的线是使用|random steps|装饰实现的。}:


% {
%   \tikzset{
%     nonterminal/.style={
%       % The shape:
%       rectangle,
%       % The size:
%       minimum size=6mm,
%       % The border:
%       very thick,
%       draw=red!50!black!50,         % 50% red and 50% black,
%                                     % and that mixed with 50% white
%       % The filling:
%       top color=white,              % a shading that is white at the top...
%       bottom color=red!50!black!20, % and something else at the bottom
%       % Font
%       font=\itshape
%     },
%     terminal/.style={
%       % The shape:
%       rounded rectangle,
%       minimum size=6mm,
%       % The rest
%       very thick,draw=black!50,
%       top color=white,bottom color=black!20,
%       font=\ttfamily},
%     skip loop/.style={to path={-- ++(0,#1) -| (\tikztotarget)}}
%   }
%   \tikzset{terminal/.append style={text height=1.5ex,text depth=.25ex}}
%   \tikzset{nonterminal/.append style={text height=1.5ex,text depth=.25ex}}
%   \pgfmathsetseed{1}
% \medskip
% \noindent\begin{tikzpicture}[
%   >=latex,thick,
%   /pgf/every decoration/.style={/tikz/sharp corners},
%   fuzzy/.style={decorate,decoration={random steps,segment length=0.5mm,amplitude=0.15pt}},
%   minimum size=6mm,line join=round,line cap=round,
%   terminal/.style={rectangle,draw,fill=white,fuzzy,rounded corners=3mm},
%   nonterminal/.style={rectangle,draw,fill=white,fuzzy},
%   node distance=4mm]

%   \ttfamily
%   \begin{scope}[start chain,
%                 every node/.style={on chain},
%                 terminal/.append style={join=by {->,shorten >=-1pt,fuzzy,decoration={post length=4pt}}},
%                 nonterminal/.append style={join=by {->,shorten >=-1pt,fuzzy,decoration={post length=4pt}}},
%                 support/.style={coordinate,join=by fuzzy}]
%     \node [support]             (start)        {};
%     \node [nonterminal]                        {unsigned integer};
%     \node [support]             (after ui)     {};
%     \node [terminal]                           {.};
%     \node [support]             (after dot)    {};
%     \node [terminal]                           {digit};
%     \node [support]             (after digit)  {};
%     \node [support]             (skip)         {};
%     \node [support]             (before E)     {};
%     \node [terminal]                           {E};
%     \node [support]             (after E)      {};
%     \node [support,xshift=5mm]  (between)      {};
%     \node [support,xshift=5mm]  (before last)  {};
%     \node [nonterminal]                        {unsigned integer};
%     \node [support]             (after last)   {};
%     \node [coordinate,join=by ->] (end)          {};
%   \end{scope}
%   \node (plus)  [terminal,above=of between] {+};
%   \node (minus) [terminal,below=of between] {-};

%   \begin{scope}[->,decoration={post length=4pt},rounded corners=2mm,every path/.style=fuzzy]
%     \draw (after ui)    -- +(0,.7)  -| (skip);
%     \draw (after digit) -- +(0,-.7) -| (after dot);
%     \draw (before E)    -- +(0,-1.2) -| (after last);
%     \draw (after E)     |- (plus);
%     \draw (plus)        -| (before last);
%     \draw (after E)     |- (minus);
%     \draw (minus)       -| (before last);
%   \end{scope}
% \end{tikzpicture}
% \medskip

% For her next lecture, Ilka decides to redo this diagram, but this time perhaps
% a bit cleaner and perhaps also bit ``cooler''.

% 为了下一堂课,Ilka决定重新制作这个图表,但这次可能会更加清晰,也更加“酷”。

% \medskip
% \noindent\begin{tikzpicture}[point/.style={coordinate},>={Stealth[round]},thick,draw=black!50,
%                     tip/.style={->,shorten >=1pt},every join/.style={rounded corners},
%                     hv path/.style={to path={-| (\tikztotarget)}},
%                     vh path/.style={to path={|- (\tikztotarget)}}]
%   \matrix[column sep=4mm] {
%     % First row:
%     & & & & & & &  & & & & \node (plus) [terminal] {+};\\
%     % Second row:
%     \node (p1) [point]  {}; &    \node (ui1)   [nonterminal] {unsigned integer}; &
%     \node (p2) [point]  {}; &    \node (dot)   [terminal]    {.};                &
%     \node (p3) [point]  {}; &    \node (digit) [terminal]    {digit};            &
%     \node (p4) [point]  {}; &    \node (p5)    [point]  {};                      &
%     \node (p6) [point]  {}; &    \node (e)     [terminal]    {E};                &
%     \node (p7) [point]  {}; &                                                    &
%     \node (p8) [point]  {}; &    \node (ui2)   [nonterminal] {unsigned integer}; &
%     \node (p9) [point]  {}; &    \node (p10)   [point]       {};\\
%     % Third row:
%     & & & & & & &  & & & & \node (minus)[terminal] {-};\\
%   };

%   { [start chain]
%     \chainin (p1);
%     \chainin (ui1)   [join=by tip];
%     \chainin (p2)    [join];
%     \chainin (dot)   [join=by tip];
%     \chainin (p3)    [join];
%     \chainin (digit) [join=by tip];
%     \chainin (p4)    [join];
%     { [start branch=digit loop]
%       \chainin (p3) [join=by {skip loop=-6mm,tip}];
%     }
%     \chainin (p5)    [join,join=with p2 by {skip loop=6mm,tip}];
%     \chainin (p6)    [join];
%     \chainin (e)     [join=by tip];
%     \chainin (p7)    [join];
%     { [start branch=plus]
%       \chainin (plus)  [join=by {vh path,tip}];
%       \chainin (p8)    [join=by {hv path,tip}];
%     }
%     { [start branch=minus]
%       \chainin (minus) [join=by {vh path,tip}];
%       \chainin (p8)    [join=by {hv path,tip}];
%     }
%     \chainin (p8)    [join];
%     \chainin (ui2)   [join=by tip];
%     \chainin (p9)    [join,join=with p6 by {skip loop=-11mm,tip}];
%     \chainin (p10)   [join=by tip];
%   }
% \end{tikzpicture}
% }\medskip

% Having read the previous tutorials, Ilka knows already how to set up the
% environment for her diagram, namely using a |tikzpicture| environment. She
% wonders which libraries she will need. She decides that she will postpone the
% decision and add the necessary libraries as needed as she constructs the
% picture.

% 在阅读了之前的教程后,Ilka已经知道如何设置图表的环境,即使用|tikzpicture|环境。她想知道她将需要哪些库。她决定在构建图表时根据需要添加所需的库。
