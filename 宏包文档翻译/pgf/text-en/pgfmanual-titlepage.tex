

\pgfmathsetseed{1}%设置了随机数生成器的种子为1。
\newbox\mybox%定义了一个新的盒子,它可以用来存储和操作文本、图形等内容。
{
  \parindent0pt%将段落缩进设置为0点 \parindent 是一个长度参数,用于设置段落的首行缩进距离。0pt 表示没有缩进。
  \null%插入了一个空的盒子。在 LaTeX 中,\null 命令通常用于在没有任何可见内容的情况下占据位置。
  \colorlet{mintgreen}{green!50!black!50}%定义了一个新的颜色 mintgreen,它是绿色和黑色的混合,每种颜色各占50%。这是通过 xcolor 包的 \colorlet 命令实现的。

  \thispagestyle{empty}%设置当前页面的样式为empty,这意味着不会显示页眉和页脚。
  \vskip3cm%创建垂直空间。 创建一个高度为3cm的空白区域
  \vfill%创建一个可以填充全部剩余空间的空白区域。
  \hfil%水平空白,它会在水平方向上填充剩余空间。
  \begin{tikzpicture}[overlay]%这是TiKZ包的主要环境,用于创建各种复杂的绘图。
    \coordinate (front) at (0,0);
    \coordinate (horizon) at (0,.31\paperheight);
    \coordinate (bottom) at (0,-.6\paperheight);
    \coordinate (sky) at (0,.57\paperheight);
    \coordinate (left) at (-.51\paperwidth,0);
    \coordinate (right) at (.51\paperwidth,0);%定义了一系列坐标(front,horizon,bottom,sky,left,right)

    \shade [bottom color=blue!30!black!10,top color=blue!30!black!50]
      ([yshift=-5mm]horizon -|  left) rectangle (sky -| right);
    \shade [bottom color=black!70!green!25,top color=black!70!green!10]
      (front -| left) -- (horizon -| left)
      decorate [decoration=random steps] { -- (horizon -| right) }
      -- (front -| right) -- cycle;
    \shade [top color=black!70!green!25,bottom color=black!25]
      ([yshift=-5mm-1pt]front -| left) rectangle ([yshift=1pt]front -| right);%\shade命令用于创建一个颜色渐变;; 画出地面、天空
    \fill [black!25] (bottom -| left) rectangle ([yshift=-5mm]front -| right);%\fill命令则用于填充一个区域。

    \def\nodeshadowed[#1]#2;{\node[scale=2,above,#1]{\global\setbox\mybox=\hbox{#2}\copy\mybox};
      \node[scale=2,above,#1,yscale=-1,scope fading=south,opacity=0.4]{\box\mybox};}%定义了一个新的命令,用于创建带有阴影效果的节点。

    \nodeshadowed [at={(-5,5  )},yslant=0.05] {\Huge Ti\textcolor{orange}{\emph{k}}Z};%绘制文本"TiKZ"
    \nodeshadowed [at={( 0,5.3)}] {\huge \textcolor{mintgreen}{\&}};
    \nodeshadowed [at={( 5,5  )},yslant=-0.05] {\Huge \textsc{PGF}};%绘制文本"PGF"
    % \nodeshadowed [at={( 0,2  )}] {Manual for Version \pgftypesetversion};
    \nodeshadowed [at={( 0,2  )}] {\pgftypesetversion 版手册};
    

    \foreach \where in {-9cm,9cm}%循环  用于在指定的位置创建一系列的节点和形状。
    {\nodeshadowed [at={(\where,5cm)}] {
    % TODO: Nesting tikzpictures is NOT supported
    \tikz \draw [green!20!black, rotate=90]
    [l-system={rule set={F -> FF-[-F+F]+[+F-F]}, axiom=F, order=4,
      step=2pt, randomize step percent=50, angle=30, randomize angle percent=5}]
    lindenmayer system;};}

    \foreach \i in {0.5,0.6,...,2}
      \fill [white,decoration=Koch snowflake,opacity=.9]
            [shift=(horizon),shift={(rand*11,rnd*7)},scale=\i]
            [double copy shadow={opacity=0.2,shadow xshift=0pt,shadow
              yshift=3*\i pt,fill=white,draw=none}]
        decorate {
          decorate {
            decorate {
              (0,0) -- ++(60:1) -- ++(-60:1) -- cycle
            }
          }
        };

  % \node (left text)创建了一个新的节点,包含了一段代码列表,这段代码列表使用了lstlisting环境,这是一个用于显示格式化代码的环境。
  \node (left text) [text width=.5\paperwidth-2cm,below right,at={(-.5\paperwidth+1cm,-1.5cm)}]
  {
    \fontencoding{T1}
    \fontfamily{pcr}
    \def\textbraceleft{\char`\{}
    \def\textbraceright{\char`\}}
    \def\textbackslash{\char`\\}
    \begin{lstlisting}[basicstyle=\scriptsize\color{black},
                       keywordstyle=\bfseries\color{white},
                       identifierstyle=\bfseries\color{black},
                       keywords={tikzpicture,shade,fill,draw,path,node},
                       literate={-}{{-}}1]
\begin{tikzpicture}
  \coordinate (front) at (0,0);
  \coordinate (horizon) at (0,.31\paperheight);
  \coordinate (bottom) at (0,-.6\paperheight);
  \coordinate (sky) at (0,.57\paperheight);
  \coordinate (left) at (-.51\paperwidth,0);
  \coordinate (right) at (.51\paperwidth,0);

  \shade [bottom color=white,
          top color=blue!30!black!50]
              ([yshift=-5mm]horizon -|  left)
    rectangle (sky -| right);

  \shade [bottom color=black!70!green!25,
          top color=black!70!green!10]
    (front -| left) -- (horizon -| left)
    decorate [decoration=random steps] {
      -- (horizon -| right)  }
    -- (front -| right) -- cycle;

  \shade [top color=black!70!green!25,
         bottom color=black!25]
              ([yshift=-5mm-1pt]front -| left)
    rectangle ([yshift=1pt]front -| right);

  \fill [black!25]
              (bottom -| left)
    rectangle ([yshift=-5mm]front -| right);

  \def\nodeshadowed[#1]#2;{
    \node[scale=2,above,#1]{
      \global\setbox\mybox=\hbox{#2}
      \copy\mybox};
    \node[scale=2,above,#1,yscale=-1,
          scope fading=south,opacity=0.4]{\box\mybox};
  }
\end{lstlisting}
};

  \node (right text) [text width=.5\paperwidth-2cm,below right,at={(1cm,-1.5cm)}]
  {
    \fontencoding{T1}
    \fontfamily{pcr}
    \def\textbraceleft{\char`\{}
    \def\textbraceright{\char`\}}
    \def\textbackslash{\char`\\}
    \begin{lstlisting}[basicstyle=\scriptsize\color{black},
                       keywordstyle=\bfseries\color{white},
                       identifierstyle=\bfseries\color{black},
                       keywords={tikzpicture,shade,fill,draw,path,node},
                       literate={-}{{-}}1]
  \nodeshadowed [at={(-5,8  )},yslant=0.05]
    {\Huge Ti\textcolor{orange}{\emph{k}}Z};
  \nodeshadowed [at={( 0,8.3)}]
    {\huge \textcolor{green!50!black!50}{\&}};
  \nodeshadowed [at={( 5,8  )},yslant=-0.05]
    {\Huge \textsc{PGF}};
  \nodeshadowed [at={( 0,5  )}]
    {Manual for Version \pgftypesetversion};

  \foreach \where in {-9cm,9cm} {
    \nodeshadowed [at={(\where,5cm)}] { \tikz
      \draw [green!20!black, rotate=90,
             l-system={rule set={F -> FF-[-F+F]+[+F-F]},
               axiom=F, order=4,step=2pt,
               randomize step percent=50, angle=30,
               randomize angle percent=5}] l-system; }}

  \foreach \i in {0.5,0.6,...,2}
    \fill
      [white,opacity=\i/2,
       decoration=Koch snowflake,
       shift=(horizon),shift={(rand*11,rnd*7)},
       scale=\i,double copy shadow={
         opacity=0.2,shadow xshift=0pt,
         shadow yshift=3*\i pt,fill=white,draw=none}]
      decorate {
        decorate {
          decorate {
            (0,0)- ++(60:1) -- ++(-60:1) -- cycle
          } } };

   \node (left text) ...
   \node (right text) ...

   \fill [decorate,decoration={footprints,foot of=gnome},
          opacity=.5,brown]        (rand*8,-rnd*10)
     to [out=rand*180,in=rand*180] (rand*8,-rnd*10);
\end{tikzpicture}
  \end{lstlisting}
  };

  \fill [decorate,decoration=footprints,
         decoration={footprints,foot of=gnome},
         opacity=.5,brown]        (rand*8,-rnd*10)
    to [out=rand*180,in=rand*180] (rand*8,-rnd*10);
\end{tikzpicture}
\vfill
\vbox{}
\clearpage
} 

