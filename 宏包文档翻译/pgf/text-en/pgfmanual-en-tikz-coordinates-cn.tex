% % Copyright 2019 by Till Tantau
% %
% % This file may be distributed and/or modified
% %
% % 1. under the LaTeX Project Public License and/or
% % 2. under the GNU Free Documentation License.
% %
% % See the file doc/generic/pgf/licenses/LICENSE for more details.


% \section{Specifying Coordinates\\指定坐标}

% \subsection{Overview\\概述}

% A \emph{coordinate} is a position on the canvas on which your picture is drawn.
% \tikzname\ uses a special syntax for specifying coordinates. Coordinates are
% always put in round brackets. The general syntax is
% \declare{|(|\opt{|[|\meta{options}|]|}\meta{coordinate  specification}|)|}.


% \emph{坐标}是在绘图画布上的位置。\tikzname\ 使用一种特殊的语法来指定坐标。坐标总是放在圆括号中。一般的语法格式为
% \declare{|(|\opt{|[|\meta{选项}|]|}\meta{坐标规范}|)|}。


% The \meta{coordinate specification} specifies coordinates using one of many
% different possible \emph{coordinate systems}. Examples are the Cartesian
% coordinate system or polar coordinates or spherical coordinates. No matter
% which coordinate system is used, in the end, a specific point on the canvas is
% represented by the coordinate.

% \meta{坐标规范}使用许多不同的\emph{坐标系}之一来指定坐标。例如,笛卡尔坐标系、极坐标或球面坐标。无论使用哪种坐标系,最终都会用坐标表示画布上的特定点。

% There are two ways of specifying which coordinate system should be used:

% 有两种指定使用的坐标系的方法:
% %
% \begin{description}
%     \item[Explicitly] You can specify the coordinate system explicitly. To do
%         so, you give the name of the coordinate system at the beginning,
%         followed by |cs:|, which stands for ``coordinate system'', followed by
%         a specification of the coordinate using the key--value syntax. Thus,
%         the general syntax for \meta{coordinate specification} in the explicit
%         case is |(|\meta{coordinate system}| cs:|\meta{list of key--value pairs
%         specific to the coordinate system}|)|.

%         {显式} 可以显式指定坐标系。为此,在开头给出坐标系的名称,后面跟着 |cs:|,其代表``坐标系'',然后使用键-值语法来指定坐标。因此,显式情况下 \meta{坐标规范} 的一般语法为 |(|\meta{坐标系}| cs:|\meta{坐标系特定的键-值对列表}|)|。
%     \item[Implicitly] The explicit specification is often too verbose when
%         numerous coordinates should be given. Because of this, for the
%         coordinate systems that you are likely to use often a special syntax
%         is provided. \tikzname\ will notice when you use a coordinate
%         specified in a special syntax and will choose the correct coordinate
%         system automatically.

%         {隐式} 当需要给出大量坐标时,显式指定通常太冗长。因此,对于经常使用的坐标系,提供了一种特殊的语法。当使用特殊语法指定坐标时,\tikzname\ 会自动识别并选择正确的坐标系。
% \end{description}

% Here is an example in which explicit the coordinate systems are specified
% explicitly:

% 下面是一个显示显式坐标系的示例:

% %
% \begin{codeexample}[]
% \begin{tikzpicture}
%   \draw[help lines] (0,0) grid (3,2);
%   \draw (canvas cs:x=0cm,y=2mm)
%      -- (canvas polar cs:radius=2cm,angle=30);
% \end{tikzpicture}
% \end{codeexample}
% %
% In the next example, the coordinate systems are implicit:

% 在下一个示例中,坐标系是隐式指定的:

% %
% \begin{codeexample}[]
% \begin{tikzpicture}
%   \draw[help lines] (0,0) grid (3,2);
%   \draw (0cm,2mm) -- (30:2cm);
% \end{tikzpicture}
% \end{codeexample}

% It is possible to give options that apply only to a single coordinate, although
% this makes sense for transformation options only. To give transformation
% options for a single coordinate, give these options at the beginning in
% brackets:

% 可以为单个坐标提供仅适用于变换选项的选项,尽管这仅对于变换选项有意义。要为单个坐标提供变换选项,请在方括号中的开头给出这些选项:

% %
% \begin{codeexample}[]
% \begin{tikzpicture}
%   \draw[help lines] (0,0) grid (3,2);
%   \draw      (0,0) -- (1,1);
%   \draw[red] (0,0) -- ([xshift=3pt] 1,1);
%   \draw      (1,0) -- +(30:2cm);
%   \draw[red] (1,0) -- +([shift=(135:5pt)] 30:2cm);
% \end{tikzpicture}
% \end{codeexample}


% \subsection{Coordinate Systems\\坐标系}


% \subsubsection{Canvas, XYZ, and Polar Coordinate Systems\\画布、XYZ 和极坐标系}

% Let us start with the basic coordinate systems.

% 让我们从基本的坐标系开始。

% \begin{coordinatesystem}{canvas}
%     The simplest way of specifying a coordinate is to use the |canvas|
%     coordinate system. You provide a dimension $d_x$ using the |x=| option and
%     another dimension $d_y$ using the |y=| option. The position on the canvas
%     is located at the position that is $d_x$ to the right and $d_y$ above the
%     origin.

%     指定坐标的最简单方法是使用 |canvas| 坐标系。使用 |x=| 选项提供一个维度 $d_x$,使用 |y=| 选项提供另一个维度 $d_y$。画布上的位置位于距离原点向右 $d_x$、向上 $d_y$ 的位置。

%     \begin{key}{/tikz/cs/x=\meta{dimension} (initially 0pt)}
%         Distance by which the coordinate is to the right of the origin. You can
%         also write things like |1cm+2pt| since the mathematical engine is used
%         to evaluate the \meta{dimension}.

%         坐标相对于原点向右的距离。可以写成 |1cm+2pt| 这样的形式,因为使用数学引擎来评估 \meta{尺寸}。
%     \end{key}

%     \begin{key}{/tikz/cs/y=\meta{dimension} (initially 0pt)}
%         Distance by which the coordinate is above the origin.

%         坐标相对于原点向上的距离。
%     \end{key}

% \begin{codeexample}[]
% \begin{tikzpicture}
%   \draw[help lines] (0,0) grid (3,2);

%   \fill (canvas cs:x=1cm,y=1.5cm)    circle (2pt);
%   \fill (canvas cs:x=2cm,y=-5mm+2pt) circle (2pt);
% \end{tikzpicture}
% \end{codeexample}

%     To specify a coordinate in the coordinate system implicitly, you use two
%     dimensions that are separated by a comma as in |(0cm,3pt)| or
%     |(2cm,\textheight)|.
    
    
%     要隐式指定坐标系,可以使用用逗号分隔的两个维度,如 |(0cm,3pt)| 或 |(2cm,\textheight)|。%
% \begin{codeexample}[]
% \begin{tikzpicture}
%   \draw[help lines] (0,0) grid (3,2);

%   \fill (1cm,1.5cm)    circle (2pt);
%   \fill (2cm,-5mm+2pt) circle (2pt);
% \end{tikzpicture}
% \end{codeexample}
%     %
% \end{coordinatesystem}

% \begin{coordinatesystem}{xyz}
%     The |xyz| coordinate system allows you to specify a point as a multiple of
%     three vectors called the $x$-, $y$-, and $z$-vectors. By default, the
%     $x$-vector points 1cm to the right, the $y$-vector points 1cm upwards, but
%     this can be changed arbitrarily as explained in Section~\ref{section-xyz}.
%     The default $z$-vector points to
%     $\bigl(-3.85\textrm{mm},-3.85\textrm{mm}\bigr)$.

%     |xyz| 坐标系允许您将点指定为三个向量($x$ 向量、$y$ 向量和 $z$ 向量)的倍数。默认情况下,$x$ 向量指向右侧 1cm,$y$ 向量指向上方 1cm,但可以根据需要进行任意更改,如第~\ref{section-xyz} 节中所述。默认的 $z$ 向量指向 $\bigl(-3.85\textrm{mm},-3.5\textrm{mm}\bigr)$。


%     To specify the factors by which the vectors should be multiplied before
%     being added, you use the following three options:
    
%     要指定向量相加之前的倍数,可以使用以下三个选项:%
%     \begin{key}{/tikz/cs/x=\meta{factor} (initially 0)}
%         Factor by which the $x$-vector is multiplied.
%     \end{key}
%     %
%     \begin{key}{/tikz/cs/y=\meta{factor} (initially 0)}
%         Works like |x|.
%     \end{key}
%     %
%     \begin{key}{/tikz/cs/z=\meta{factor} (initially 0)}
%         Works like |x|.
%     \end{key}

% \begin{codeexample}[]
% \begin{tikzpicture}[->]
%   \draw (0,0) -- (xyz cs:x=1);
%   \draw (0,0) -- (xyz cs:y=1);
%   \draw (0,0) -- (xyz cs:z=1);
% \end{tikzpicture}
% \end{codeexample}

%     This coordinate system can also be selected implicitly. To do so, you just
%     provide two or three comma-separated factors (not dimensions).
    
    
%     这个坐标系也可以隐式选择。为此,只需提供两个或三个以逗号分隔的因子(不是尺寸)。
% %
% \begin{codeexample}[]
% \begin{tikzpicture}[->]
%   \draw (0,0) -- (1,0);
%   \draw (0,0) -- (0,1,0);
%   \draw (0,0) -- (0,0,1);
% \end{tikzpicture}
% \end{codeexample}
%     %
% \end{coordinatesystem}

% \emph{Note:} It is possible to use coordinates like |(1,2cm)|, which are
% neither |canvas| coordinates nor |xyz| coordinates. The rule is the following:
% If a coordinate is of the implicit form |(|\meta{x}|,|\meta{y}|)|, then
% \meta{x} and \meta{y} are checked, independently, whether they have a dimension
% or whether they are dimensionless. If both have a dimension, the |canvas|
% coordinate system is used. If both lack a dimension, the |xyz| coordinate
% system is used. If \meta{x} has a dimension and \meta{y} has not, then the sum
% of two coordinate |(|\meta{x}|,0pt)| and |(0,|\meta{y}|)| is used. If \meta{y}
% has a dimension and \meta{x} has not, then the sum of two coordinate
% |(|\meta{x}|,0)| and |(0pt,|\meta{y}|)| is used.

% \emph{注意:}可以使用诸如 |(1,2cm)| 这样的坐标,它们既不是 |canvas| 坐标也不是 |xyz| 坐标。规则如下:如果坐标的隐式形式为 |(|\meta{x}|,|\meta{y}|)|,则会独立检查 \meta{x} 和 \meta{y} 是否具有尺寸或无尺寸。如果两者都有尺寸,则使用 |canvas| 坐标系。如果两者都没有尺寸,则使用 |xyz| 坐标系。如果 \meta{x} 有尺寸而 \meta{y} 没有,则使用两个坐标 |(|\meta{x}|,0pt)| 和 |(0,|\meta{y}|)| 的和。如果 \meta{y} 有尺寸而 \meta{x} 没有,则使用两个坐标 |(|\meta{x}|,0)| 和 |(0pt,|\meta{y}|)| 的和。


% \emph{Note furthermore:} An expression like |(2+3cm,0)| does \emph{not} mean
% the same as |(2cm+3cm,0)|. Instead, if \meta{x} or \meta{y} internally uses a
% mixture of dimensions and dimensionless values, then all dimensionless values
% are ``upgraded'' to dimensions by interpreting them as |pt|. So, |2+3cm| is the
% same dimension as |2pt+3cm|.

% \emph{此外注意:}表达式 |(2+3cm,0)| 与 |(2cm+3cm,0)| \emph{不}相同。如果 \meta{x} 或 \meta{y} 内部使用尺寸和无尺寸值的混合,则所有无尺寸值都会通过将其解释为 |pt| 而被``升级''为尺寸。因此,|2+3cm| 和 |2pt+3cm| 表示相同的尺寸。


% \begin{coordinatesystem}{canvas polar}
%     The |canvas polar| coordinate system allows you to specify polar
%     coordinates. You provide an angle using the |angle=| option and a radius
%     using the |radius=| option. This yields the point on the canvas that is at
%     the given radius distance from the origin at the given degree. An angle of
%     zero degrees to the right, a degree of 90 upward.
    
%     |canvas polar|坐标系允许您指定极坐标。您可以使用|angle=|选项提供角度,并使用|radius=|选项提供半径。这将给出位于给定角度处距离原点给定半径距离的画布上的点。角度为零度表示向右,90度表示向上。
% %
%     \begin{key}{/tikz/cs/angle=\meta{degrees}}
%         The angle of the coordinate. The angle must always be given in degrees
%         and should be between $-360$ and $720$.

%         坐标的角度。角度必须始终以度为单位,并且应在$-360$到$720$之间。

%     \end{key}
%     %
%     \begin{key}{/tikz/cs/radius=\meta{dimension}}
%         The distance from the origin.

%         距离原点的距离。

%     \end{key}
%     %
%     \begin{key}{/tikz/cs/x radius=\meta{dimension}}
%         A polar coordinate is, after all, just a point on a circle of the given
%         \meta{radius}. When you provide an $x$-radius and also a $y$-radius,
%         you specify an ellipse instead of a circle. The |radius| option has the
%         same effect as specifying identical |x radius| and |y radius| options.

%         极坐标实际上只是给定\meta{radius}的圆上的一个点。当您提供$x$半径和$y$半径时,您指定的是椭圆而不是圆。|radius|选项与指定相同的|x radius|和|y radius|选项具有相同的效果。
%     \end{key}
%     %
%     \begin{key}{/tikz/cs/y radius=\meta{dimension}}
%         Works like |x radius|.

%         与|x radius|相同。

%     \end{key}
%     %
% \begin{codeexample}[]
% \tikz \draw (0,0) -- (canvas polar cs:angle=30,radius=1cm);
% \end{codeexample}

%     The implicit form for canvas polar coordinates is the following: you
%     specify the angle and the distance, separated by a colon as in |(30:1cm)|.
%     %

%     画布极坐标的隐式形式如下:您使用冒号分隔的角度和距离来指定,例如 |(30:1cm)|。

% \begin{codeexample}[]
% \tikz \draw    (0cm,0cm) -- (30:1cm) -- (60:1cm) -- (90:1cm)
%             -- (120:1cm) -- (150:1cm) -- (180:1cm);
% \end{codeexample}

%     Two different radii are specified by writing |(30:1cm and 2cm)|.

%     通过编写 |(30:1cm and 2cm)| 可以指定两个不同的半径。



%     For the implicit form, instead of an angle given as a number you can also
%     use certain words. For example, |up| is the same as |90|, so that you can
%     write |\tikz \draw (0,0) -- (2ex,0pt) -- +(up:1ex);| and get
%     \tikz \draw (0,0) -- (2ex,0pt) -- +(up:1ex);. Apart from |up| you can use
%     |down|, |left|, |right|, |north|, |south|, |west|, |east|, |north east|,
%     |north west|, |south east|, |south west|, all of which have their natural
%     meaning.

%     对于隐式形式,您可以使用某些单词而不是数字来表示角度。例如,|up| 等同于 |90|,因此您可以编写 |\tikz \draw (0,0) -- (2ex,0pt) -- +(up:1ex);| 并获得 \tikz \draw (0,0) -- (2ex,0pt) -- +(up:1ex);。除了 |up|,您还可以使用 |down|、|left|、|right|、|north|、|south|、|west|、|east|、|north east|、|north west|、|south east| 和 |south west|,它们都有其自然的含义。

% \end{coordinatesystem}

% \begin{coordinatesystem}{xyz polar}
%     This coordinate system work similarly to the |canvas polar| system.
%     However, the radius and the angle are interpreted in the $xy$-coordinate
%     system, not in the canvas system. More detailed, consider the circle or
%     ellipse whose half axes are given by the current $x$-vector and the current
%     $y$-vector. Then, consider the point that lies at a given angle on this
%     ellipse, where an angle of zero is the same as the $x$-vector and an angle
%     of 90 is the $y$-vector. Finally, multiply the resulting vector by the
%     given radius factor. Voil\`a.
    
%     这个坐标系与|canvas polar|系统类似。但是,半径和角度是在$xy$坐标系中解释的,而不是在画布坐标系中解释的。更详细地说,考虑由当前$x$向量和当前$y$向量给出的圆或椭圆。然后,考虑位于此椭圆上给定角度处的点,其中角度零与$x$向量相同,角度90与$y$向量相同。最后,将得到的向量乘以给定的半径因子。Voil`a。

% %
%     \begin{key}{/tikz/cs/angle=\meta{degrees}}
%         The angle of the coordinate interpreted in the ellipse whose axes are
%         the $x$-vector and the $y$-vector.

%         坐标的角度,解释为以$x$向量和$y$向量为轴的椭圆中的角度。

%     \end{key}
%     %
%     \begin{key}{/tikz/cs/radius=\meta{factor}}
%         A factor by which the $x$-vector and $y$-vector are multiplied prior to
%         forming the ellipse.

%         在形成椭圆之前,$x$向量和$y$向量乘以的因子。

%     \end{key}
%     %
%     \begin{key}{/tikz/cs/x radius=\meta{dimension}}
%         A specific factor by which only the $x$-vector is multiplied.
    
%         仅将$x$向量乘以的特定因子。
%     \end{key}
%     %
%     \begin{key}{/tikz/cs/y radius=\meta{dimension}}
%         Works like |x radius|.

%         与|x radius|相同。

%     \end{key}
%     %
% \begin{codeexample}[]
% \begin{tikzpicture}[x=1.5cm,y=1cm]
%   \draw[help lines] (0cm,0cm) grid (3cm,2cm);

%   \draw (0,0) -- (xyz polar cs:angle=0,radius=1);
%   \draw (0,0) -- (xyz polar cs:angle=30,radius=1);
%   \draw (0,0) -- (xyz polar cs:angle=60,radius=1);
%   \draw (0,0) -- (xyz polar cs:angle=90,radius=1);

%   \draw (xyz polar cs:angle=0,radius=2)
%      -- (xyz polar cs:angle=30,radius=2)
%      -- (xyz polar cs:angle=60,radius=2)
%      -- (xyz polar cs:angle=90,radius=2);
%  \end{tikzpicture}
% \end{codeexample}

%     The implicit version of this option is the same as the implicit version of
%     |canvas polar|, only you do not provide a unit.

%     这个选项的隐式版本与|canvas polar|的隐式版本相同,只是您不需要提供单位。
% \begin{codeexample}[]
% \tikz[x={(0cm,1cm)},y={(-1cm,0cm)}]
%   \draw  (0,0) -- (30:1) -- (60:1) -- (90:1)
%              -- (120:1) -- (150:1) -- (180:1);
% \end{codeexample}
%     %
% \end{coordinatesystem}

% \begin{coordinatesystem}{xy polar}
%     This is just an alias for |xyz polar|, which some people might prefer as
%     there is no z-coordinate involved in the |xyz polar| coordinates.

%     这只是|xyz polar|的别名,一些人可能更喜欢使用它,因为|xyz polar|坐标中没有涉及z坐标。
% \end{coordinatesystem}


% \subsubsection{Barycentric Systems\\重心系}
% \label{section-barycentric-coordinates}

% In the barycentric coordinate system a point is expressed as the linear
% combination of multiple vectors. The idea is that you specify vectors $v_1$,
% $v_2$, \dots, $v_n$ and numbers $\alpha_1$, $\alpha_2$, \dots, $\alpha_n$. Then
% the barycentric coordinate specified by these vectors and numbers is

% 在重心坐标系中,一个点被表示为多个向量的线性组合。思路是您指定向量$v_1$、$v_2$、\dots、$v_n$和数字$\alpha_1$、$\alpha_2$、\dots、$\alpha_n$。然后由这些向量和数字指定的重心坐标为
% %
% \begin{align*}
%     \frac{\alpha_1 v_1 + \alpha_2 v_2 + \cdots + \alpha_n v_n}{\alpha_1
%         + \alpha_2 + \cdots + \alpha_n}
% \end{align*}

% The |barycentric cs| allows you to specify such coordinates easily.

% |barycentric cs|允许您轻松指定这样的坐标。


% \begin{coordinatesystem}{barycentric}
%     For this coordinate system, the \meta{coordinate specification} should be a
%     comma-separated list of expressions of the form \meta{node
%     name}|=|\meta{number}. Note that (currently) the list should not contain
%     any spaces before or after the \meta{node name} (unlike normal key--value
%     pairs).

%     对于这个坐标系,\meta{coordinate specification}应该是一个由逗号分隔的表达式列表,每个表达式的格式为\meta{node name}|=|\meta{number}。请注意(目前)列表中不应在\meta{node name}之前或之后包含任何空格(与常规键值对不同)。


%     The specified coordinate is now computed as follows: Each pair provides one
%     vector and a number. The vector is the |center| anchor of the \meta{node
%     name}. The number is the \meta{number}. Note that (currently) you cannot
%     specify a different anchor, so that in order to use, say, the |north|
%     anchor of a node you first have to create a new coordinate at this north
%     anchor. (Using for instance \texttt{\string\coordinate (mynorth) at
%     (mynode.north);}.)

%     现在,指定的坐标计算如下:每对提供一个向量和一个数字。向量是\meta{node name}的|center|锚点。数字是\meta{number}。请注意(目前)您不能指定不同的锚点,因此为了使用节点的|north|锚点,您首先必须在该北锚点处创建一个新的坐标(例如,使用\texttt{\string\coordinate (mynorth) at (mynode.north);})。

%     %
% \begin{codeexample}[]
% \begin{tikzpicture}
%   \coordinate (content)   at (90:3cm);
%   \coordinate (structure) at (210:3cm);
%   \coordinate (form)      at (-30:3cm);

%   \node [above]       at (content)   {content oriented};
%   \node [below left]  at (structure) {structure oriented};
%   \node [below right] at (form)      {form oriented};

%   \draw [thick,gray] (content.south) -- (structure.north east) -- (form.north west) -- cycle;

%   \small
%   \node at (barycentric cs:content=0.5,structure=0.1 ,form=1)    {PostScript};
%   \node at (barycentric cs:content=1  ,structure=0   ,form=0.4)  {DVI};
%   \node at (barycentric cs:content=0.5,structure=0.5 ,form=1)    {PDF};
%   \node at (barycentric cs:content=0  ,structure=0.25,form=1)    {CSS};
%   \node at (barycentric cs:content=0.5,structure=1   ,form=0)    {XML};
%   \node at (barycentric cs:content=0.5,structure=1   ,form=0.4)  {HTML};
%   \node at (barycentric cs:content=1  ,structure=0.2 ,form=0.8)  {\TeX};
%   \node at (barycentric cs:content=1  ,structure=0.6 ,form=0.8)  {\LaTeX};
%   \node at (barycentric cs:content=0.8,structure=0.8 ,form=1)    {Word};
%   \node at (barycentric cs:content=1  ,structure=0.05,form=0.05) {ASCII};
% \end{tikzpicture}
% \end{codeexample}
%     %
% \end{coordinatesystem}



% \subsubsection{Node Coordinate System\\节点坐标系}
% \label{section-node-coordinates}

% In \pgfname\ and in \tikzname\ it is quite easy to define a node that you wish
% to reference at a later point. Once you have defined a node, there are
% different ways of referencing points of the node. To do so, you use the
% following coordinate system:

% 在 \pgfname 和 \tikzname 中,定义一个稍后要引用的节点非常容易。一旦定义了一个节点,就有不同的方法来引用节点的点。为此,您可以使用以下坐标系:

% \begin{coordinatesystem}{node}
%     This coordinate system is used to reference a specific point inside or on
%     the border of a previously defined node. It can be used in different ways,
%     so let us go over them one by one.

%     这个坐标系用于引用先前定义的节点内部或边界上的特定点。它可以以不同的方式使用,我们逐一介绍它们。


%     You can use three options to specify which coordinate you mean:
    
%     您可以使用三个选项来指定您指的是哪个坐标:
% %
%     \begin{key}{/tikz/cs/name=\meta{node name}}
%         Specifies the node that you wish to use to specify a coordinate. The
%         \meta{node name} is the name that was previously used to name the node
%         using the |name=|\meta{node name} option or the special node name
%         syntax.

%         指定您希望用于指定坐标的节点。 \meta{node name}是先前使用|name=|\meta{node name}选项或特殊节点名称语法命名节点时使用的名称。

%     \end{key}
%     %
%     \begin{key}{/tikz/anchor=\meta{anchor}}
%         Specifies an anchor of the node. Here is an example:
%         %

%         指定节点的锚点。以下是一个示例:

% \begin{codeexample}[preamble={\usetikzlibrary{arrows.meta}}]
% \begin{tikzpicture}
%   \node (shape)   at (0,2)  [draw] {|class Shape|};
%   \node (rect)    at (-2,0) [draw] {|class Rectangle|};
%   \node (circle)  at (2,0)  [draw] {|class Circle|};
%   \node (ellipse) at (6,0)  [draw] {|class Ellipse|};

%   \draw (node cs:name=circle,anchor=north) |- (0,1);
%   \draw (node cs:name=ellipse,anchor=north) |- (0,1);
%   \draw [arrows = -{Triangle[open, angle=60:3mm]}]
%            (node cs:name=rect,anchor=north)
%         |- (0,1) -| (node cs:name=shape,anchor=south);
% \end{tikzpicture}
% \end{codeexample}
%     \end{key}
%     %
%     \begin{key}{/tikz/cs/angle=\meta{degrees}}
%         It is also possible to provide an angle \emph{instead} of an anchor.
%         This coordinate refers to a point of the node's border where a ray shot
%         from the center in the given angle hits the border. Here is an example:
        
%         也可以提供一个角度\emph{而不是}一个锚点。这个坐标是指节点边界上的一个点,从中心以给定角度射出的射线与边界相交的位置。这是一个例子:
        


% \begin{codeexample}[preamble={\usetikzlibrary{shapes.geometric}}]
% \begin{tikzpicture}
%   \node (start) [draw,shape=ellipse] {start};
%   \foreach \angle in {-90, -80, ..., 90}
%     \draw (node cs:name=start,angle=\angle)
%       .. controls +(\angle:1cm) and +(-1,0) .. (2.5,0);
%   \end{tikzpicture}
% \end{codeexample}
%     \end{key}

%     It is possible to provide \emph{neither} the |anchor=| option nor the
%     |angle=| option. In this case, \tikzname\ will calculate an appropriate
%     border position for you. Here is an example:
    
%     可以\emph{既不}使用 |anchor=| 选项,也不使用 |angle=| 选项。在这种情况下,\tikzname\ 会为您计算一个适当的边界位置。这是一个例子:
% %
% \begin{codeexample}[preamble={\usetikzlibrary{shapes.geometric}}]
% \begin{tikzpicture}
%   \path (0,0)  node(a) [ellipse,rotate=10,draw] {An ellipse}
%         (3,-1) node(b) [circle,draw]            {A circle};
%   \draw[thick] (node cs:name=a) -- (node cs:name=b);
% \end{tikzpicture}
% \end{codeexample}

%     \tikzname\ will be reasonably clever at determining the border points that
%     you ``mean'', but, naturally, this may fail in some situations. If
%     \tikzname\ fails to determine an appropriate border point, the center will
%     be used instead.

%     \tikzname\ 在确定您所“指定”的边界点时会相当聪明,但是自然地,在某些情况下会出现失败。如果\tikzname\ 无法确定适当的边界点,则使用中心点代替。


%     Automatic computation of anchors works only with the line-to operations
%     |--|, the vertical/horizontal versions \verb!|-! and \verb!-|!, and with
%     the curve-to operation |..|. For other path commands, such as |parabola| or
%     |plot|, the center will be used. If this is not desired, you should give a
%     named anchor or an angle anchor.

%     自动计算锚点仅适用于线性操作 |--|,垂直/水平版本 \verb!|-! 和 \verb!-|!,以及曲线操作 |..|。对于其他路径命令(如 |parabola| 或 |plot|),将使用中心点。如果不希望这样,请给出一个命名的锚点或角度锚点。


%     Note that if you use an automatic coordinate for both the start and the end
%     of a line-to, as in |--(node cs:name=b)--|, then \emph{two} border
%     coordinates are computed with a move-to between them. This is usually
%     exactly what you want.

%     请注意,如果您对一条线段的起点和终点都使用自动坐标,例如 |--(node cs:name=b)--|,则将计算出\emph{两个}边界坐标,并在它们之间进行移动。这通常是您想要的。


%     If you use relative coordinates together with automatic anchor coordinates,
%     the relative coordinates are computed relative to the node's center, not
%     relative to the border point. Here is an example:
    
%     如果您同时使用相对坐标和自动锚点坐标,则相对坐标是相对于节点的中心计算的,而不是相对于边界点计算的。这是一个例子:%
% \begin{codeexample}[]
% \tikz \draw (0,0) node(x) [draw] {Text}
%             rectangle (1,1)
%             (node cs:name=x) -- +(1,1);
% \end{codeexample}

%     Similarly, in the following examples both control points are $(1,1)$:
    
%     类似地,在以下示例中,两个控制点都是 $(1,1)$:
% %
% \begin{codeexample}[]
% \tikz \draw (0,0) node(x) [draw] {X}
%             (2,0) node(y) {Y}
%             (node cs:name=x) .. controls +(1,1) and +(-1,1) ..
%             (node cs:name=y);
% \end{codeexample}

%     The implicit way of specifying the node coordinate system is to simply use
%     the name of the node in parentheses as in |(a)| or to specify a name
%     together with an anchor or an angle separated by a dot as in |(a.north)| or
%     |(a.10)|.

%     隐式指定节点坐标系统的方法是简单地使用节点的名称,如 |(a)|,或者使用由点分隔的名称和锚点或角度,如 |(a.north)| 或 |(a.10)|。



%     Here is a more complete example:

%     这是一个更完整的示例:

%     %
% \begin{codeexample}[preamble={\usetikzlibrary{shapes.geometric}}]
% \begin{tikzpicture}[fill=blue!20]
%   \draw[help lines] (-1,-2) grid (6,3);
%   \path (0,0)  node(a) [ellipse,rotate=10,draw,fill]    {An ellipse}
%         (3,-1) node(b) [circle,draw,fill]               {A circle}
%         (2,2)  node(c) [rectangle,rotate=20,draw,fill]  {A rectangle}
%         (5,2)  node(d) [rectangle,rotate=-30,draw,fill] {Another rectangle};
%   \draw[thick] (a.south) -- (b) -- (c) -- (d);
%   \draw[thick,red,->] (a) |- +(1,3) -| (c) |- (b);
%   \draw[thick,blue,<->] (b) .. controls +(right:2cm) and +(down:1cm) .. (d);
% \end{tikzpicture}
% \end{codeexample}
%     %
% \end{coordinatesystem}


% % % -----------------------------------------------------------------------------
% % % Deprecated:
% % % -----------------------------------------------------------------------------
% % %
% \subsubsection{Intersection Coordinate Systems\\Deprecated\\交点坐标系统}

% Often you wish to specify a point that is on the
% intersection of two lines or shapes. For this, the following
% coordinate system is useful:

% 通常情况下,你希望指定一个位于两条线或形状的交点上的点。为此,以下坐标系统很有用:

% \begin{coordinatesystem}{intersection}
%   First, you must specify two objects that should be
%   intersected. These ``objects'' can either be lines or the shapes of
%   nodes. There are two option to specify the first object:

%   首先,你需要指定两个想要相交的对象。这些“对象”可以是线或节点的形状。有两种选项可以指定第一个对象:

%   \begin{key}{/tikz/cs/first line={\ttfamily\char`\{}|(|\meta{first
%           coordinate}|)--(|\meta{second coordinate}|)|{\ttfamily\char`\}}}
%     Specifies that the first object is a line that goes from
%     \meta{first coordinate} to meta{second coordinate}.

%     指定第一个对象是一条从 \meta{第一个坐标} 到 \meta{第二个坐标} 的线。
%   \end{key}
%   Note that you have to write |--| between the coordinate, but this
%   does not mean that anything is added to the path. This is simply a
%   special syntax.

%   注意,在坐标之间你必须使用 |--|,但这并不意味着路径中添加了任何内容。这只是一种特殊的语法。
%   \begin{key}{/tikz/cs/first node=\meta{node}}
%     Specifies that the first object is a previously defined node named
%     \meta{node}.

%     指定第一个对象是一个之前定义的名为 \meta{节点} 的节点。

%   \end{key}

%   To specify the second object, you use one of the following keys:

%   要指定第二个对象,你可以使用以下键之一:
%   \begin{key}{/tikz/cs/second line={\ttfamily\char`\{}|(|\meta{first
%           coordinate}|)--(|\meta{second coordinate}|)|{\ttfamily\char`\}}}
%     As above.
%   \end{key}
%   \begin{key}{/tikz/cs/second node=\meta{node}}
%     Specifies that the second object is a previously defined node
%     named \meta{node}.

%     指定第二个对象是一个之前定义的名为 \meta{节点} 的节点。
%   \end{key}

%   Since it is possible that two objects have multiple intersections,
%   you may need to specify which solution you want:

%   由于两个对象可能有多个交点,你可能需要指定要使用的解决方案:
%   \begin{key}{/tikz/cs/solution=\meta{number} (initially 1)}
%     Specifies which solution should be used. Numbering starts with 1.

%     指定要使用的解决方案。编号从1开始。

%   \end{key}
%   The coordinate specified in this way is the \meta{number}th
%   intersection of the two objects.  If the objects do not intersect,
%   an error may occur.

%   以这种方式指定的坐标是两个对象的第 \meta{数字} 个交点。如果对象不相交,可能会出现错误。


% \begin{codeexample}[]
% \begin{tikzpicture}
%   \draw[help lines] (0,0) grid (3,2);
%   \draw (0,0) coordinate (A) -- (3,2) coordinate (B)
%         (1,2)                -- (3,0);

%   \fill[red] (intersection cs:
%     first line={(A)--(B)},
%     second line={(1,2)--(3,0)}) circle (2pt);
% \end{tikzpicture}
% \end{codeexample}

%   The implicit way of specifying this coordinate system is to write
%   \declare{|(intersection |\opt{\meta{number}}| of |\meta{first
%       object}%
%     | and |\meta{second object}|)|}. Here, \meta{first object} either
%   has the form \meta{$p_1$}|--|\meta{$p_2$} or it is just a node
%   name. Likewise for \meta{second object}. Note that there are \emph{no}
%   parentheses around the $p_i$. Thus, you would write
%   |(intersection of A--B and 1,2--3,0)|  for the intersection of the
%   line through the coordinates |A| and |B| and the line through the
%   points $(1,2)$ and $(3,0)$. You would write
%   |(intersection 2 of c_1 and c_2)| for the second
%   intersection of the node named |c_1| and the node named
%   |c_2|.

%   隐式指定此坐标系统的方法是写作 \declare{|(intersection |\opt{\meta{数字}}| of |\meta{第一个对象}%
% | and |\meta{第二个对象}|)|}。这里,\meta{第一个对象} 的形式可以是 \meta{$p_1$}|--|\meta{$p_2$},或者只是一个节点名称。对于 \meta{第二个对象} 也是如此。请注意,$p_i$ 周围\emph{没有}括号。因此,你可以写 |(intersection of A--B and 1,2--3,0)| 表示通过坐标 |A| 和 |B| 的线与点 $(1,2)$ 和 $(3,0)$ 的线的交点。你可以写 |(intersection 2 of c_1 and c_2)| 表示名为 |c_1| 和 |c_2| 的节点的第二个交点。



%   \tikzname\ needs an explicit algorithm for computing the
%   intersection of two shapes and such an algorithm is available only
%   for few shapes. Currently, the following intersection will be
%   computed correctly:

%   \tikzname 需要明确的算法来计算两个形状的交点,而且目前只有少数形状的交点可以正确计算:

%   \begin{itemize}
%   \item a line and a line

%   一条线和一条线
%   \item a |circle| node and a line (in any order)

%   一个 |circle| 节点和一条线(顺序可以任意)
%   \item a |circle| and a |circle|

%   一个 |circle| 和一个 |circle|
%   \end{itemize}
% \begin{codeexample}[]
% \begin{tikzpicture}[scale=.25]
%   \coordinate [label=-135:$a$] (a) at ($ (0,0)   + (rand,rand) $);
%   \coordinate [label=45:$b$]   (b) at ($ (3,2) + (rand,rand) $);

%   \coordinate [label=-135:$u$] (u) at (-1,1);
%   \coordinate [label=45:$v$]   (v) at (6,0);

%   \draw (a) -- (b)
%         (u) -- (v);

%   \node (c1) at (a) [draw,circle through=(b)] {};
%   \node (c2) at (b) [draw,circle through=(a)] {};

%   \coordinate [label=135:$c$] (c) at (intersection 2 of c1 and c2);
%   \coordinate [label=-45:$d$] (d) at (intersection of u--v and c2);
%   \coordinate [label=135:$e$] (e) at (intersection of u--v and a--b);

%   \foreach \p in {a,b,c,d,e,u,v}
%     \fill [opacity=.5] (\p) circle (8pt);
% \end{tikzpicture}
% \end{codeexample}
% \end{coordinatesystem}
% % -----------------------------------------------------------------------------


% \subsubsection{Tangent Coordinate Systems\\切线坐标系统}

% \begin{coordinatesystem}{tangent}
%     This coordinate system, which is available only when the \tikzname\ library
%     |calc| is loaded, allows you to compute the point that lies tangent to a
%     shape. In detail, consider a \meta{node} and a \meta{point}. Now, draw a
%     straight line from the \meta{point} so that it ``touches'' the \meta{node}
%     (more formally, so that it is \emph{tangent} to this \meta{node}). The
%     point where the line touches the shape is the point referred to by the
%     |tangent| coordinate system.

%     这个坐标系统仅在加载了 \tikzname\ 库 |calc| 后才可用,它允许你计算与形状相切的点。具体而言,考虑一个 \meta{节点} 和一个 \meta{点}。现在,从 \meta{点} 画一条直线,使其“接触” \meta{节点}(更形式化地说,使其与该 \meta{节点} \emph{切线})。线与形状接触的点就是由 |tangent| 坐标系统引用的点。



%     The following options may be given:

%     可以给出以下选项:

%     %
%     \begin{key}{/tikz/cs/node=\meta{node}}
%         This key specifies the node on whose border the tangent should lie.

%         指定切线应位于其边界上的节点。

%     \end{key}
%     %
%     \begin{key}{/tikz/cs/point=\meta{point}}
%         This key specifies the point through which the tangent should go.

%         指定切线应通过的点。
%     \end{key}
%     %
%     \begin{key}{/tikz/cs/solution=\meta{number}}
%         Specifies which solution should be used if there are more than one.

%         如果有多个解决方案,则指定要使用的解决方案。
%     \end{key}

%     A special algorithm is needed in order to compute the tangent for a given
%     shape. Currently, tangents can be computed for nodes whose shape is one of
%     the following:
%     %

%     为了计算给定形状的切线,需要一个特殊的算法。目前,可以计算以下形状的节点的切线:
%     \begin{itemize}
%         \item |coordinate|
%         \item |circle|
%     \end{itemize}
%     %
% \begin{codeexample}[preamble={\usetikzlibrary{calc}}]
% \begin{tikzpicture}
%   \draw[help lines] (0,0) grid (3,2);

%   \coordinate (a) at (3,2);

%   \node [circle,draw] (c) at (1,1) [minimum size=40pt] {$c$};

%   \draw[red] (a)  -- (tangent cs:node=c,point={(a)},solution=1) --
%        (c.center) -- (tangent cs:node=c,point={(a)},solution=2) -- cycle;
% \end{tikzpicture}
% \end{codeexample}

%     There is no implicit syntax for this coordinate system.

%     这个坐标系统没有隐式语法。

% \end{coordinatesystem}


% \subsubsection{Defining New Coordinate Systems\\定义新的坐标系统}

% While the set of coordinate systems that \tikzname\ can parse via their special
% syntax is fixed, it is possible and quite easy to define new explicitly named
% coordinate systems. For this, the following commands are used:

% 尽管 \tikzname\ 可以通过其特殊语法解析一组坐标系统,但是定义新的显式命名的坐标系统是可能且非常容易的。为此,使用以下命令:

% \begin{command}{\tikzdeclarecoordinatesystem\marg{name}\marg{code}}
%     This command declares a new coordinate system named \meta{name} that can
%     later on be used by writing |(|\meta{name}| cs:|\meta{arguments}|)|. When
%     \tikzname\ encounters a coordinate specified in this way, the
%     \meta{arguments} are passed to \meta{code} as argument |#1|.

%     这个命令声明了一个名为 \meta{name} 的新坐标系统,可以通过写 |(|\meta{name}| cs:|\meta{参数}|)| 来使用。当 \tikzname\ 遇到以这种方式指定的坐标时,将把 \meta{参数} 作为参数 |#1| 传递给 \meta{code}。


%     It is now the job of \meta{code} to make sense of the \meta{arguments}. At
%     the end of \meta{code}, the two \TeX\ dimensions |\pgf@x| and |\pgf@y|
%     should be have the $x$- and $y$-canvas coordinate of the coordinate.

%     现在,\meta{code} 的任务是理解 \meta{参数}。在 \meta{code} 的末尾,两个 \TeX\ 尺寸 |\pgf@x| 和 |\pgf@y| 应该具有坐标的 $x$- 和 $y$-画布坐标。


%     It is not necessary, but customary, to parse \meta{arguments} using the
%     key--value syntax. However, you can also parse it in any way you like.

%     不必要,但习惯上,使用键-值语法解析 \meta{参数}。但你也可以以任何你喜欢的方式解析它。


%     In the following example, a coordinate system |cylindrical| is defined.
    
%     在下面的示例中,定义了一个坐标系统 |cylindrical|。
% %
% \begin{codeexample}[]
% \makeatletter
% \define@key{cylindricalkeys}{angle}{\def\myangle{#1}}
% \define@key{cylindricalkeys}{radius}{\def\myradius{#1}}
% \define@key{cylindricalkeys}{z}{\def\myz{#1}}
% \tikzdeclarecoordinatesystem{cylindrical}%
% {%
%   \setkeys{cylindricalkeys}{#1}%
%   \pgfpointadd{\pgfpointxyz{0}{0}{\myz}}{\pgfpointpolarxy{\myangle}{\myradius}}
% }
% \begin{tikzpicture}[z=0.2pt]
%   \draw [->] (0,0,0) -- (0,0,350);
%   \foreach \num in {0,10,...,350}
%     \fill (cylindrical cs:angle=\num,radius=1,z=\num) circle (1pt);
% \end{tikzpicture}
% \end{codeexample}
%     %
% \end{command}

% \begin{command}{\tikzaliascoordinatesystem\marg{new name}\marg{old name}}
%     Creates an alias of \meta{old name}.
% \end{command}

