
\columnratio{0.55}
\begin{paracol}{2}
\subsection{Drivers}\label{sec:drivers}
The package supports drivers |dvips|, |dvipdfm|, |pdftex|, |luatex|,
|xetex| and |vtex|. You can also set |dvipdfm| for \textsf{dvipdfmx} and
\textsf{xdvipdfmx} The options |dvipdfmx| and |xdvipdfmx| are also supported
as aliases for the |dvipdfm| option.
|pdftex| for \textsf{pdflatex}, and |vtex| for
V\TeX{} environment.
The driver options are exclusive. The driver can be set by either
|driver=|\meta{driver name} or any of the drivers directly like |pdftex|.
By default, \Gm\ guesses the driver appropriate to the system
in use. Therefore, you don't have to set a driver in most cases.
However, if you want to use |dvipdfm|, you should specify it explicitly.
\switchcolumn
\subsection{Driver驱动}
这个选项用于设置驱动程序(driver),可以使用的驱动程序有:|dvips|、|dvipdfm|、|pdftex|、|luatex|、|xetex| 和 |vtex|。你也可以将 |dvipdfmx| 用于 \textsf{dvipdfmx} 和 \textsf{xdvipdfmx},选项 |pdftex| 用于 \textsf{pdflatex},选项 |vtex| 用于 V\TeX{}。这些驱动程序选项是互斥的。你可以通过 |driver=|\meta{driver name} 或直接使用驱动程序进行设置,如 |pdftex|。默认情况下,\Gm 会猜测适合当前系统的驱动程序,所以在大多数情况下你不需要设置驱动程序。但是,如果你想使用 |dvipdfm|,你需要明确指定它。
\end{paracol}

\begin{Options}
\columnratio{0.55}
\begin{paracol}{2}
\switchcolumn
\switchcolumn[0]*\item[\onlypre driver] specifies the driver with |driver=|\meta{driver name}. 
|dvips|, |dvipdfm|, |pdftex|, |luatex|, |vtex|, |xetex|, |auto| and |none| are
available as a driver name. The names except for |auto| and |none| can
be specified directly with the name without |driver=|.
|driver=auto| makes the auto-detection work whatever the previous setting is. 
|driver=none| disables the auto-detection and sets no driver, which
may be useful when you want to let other package work out the driver
setting. For example, if you want to use \textsf{crop} package with \Gm,
you should call |\usepackage[driver=none]{geometry}| before
the \textsf{crop} package.
\switchcolumn\item[\onlypre driver] 使用 |driver=|\meta{driver name} 来指定驱动程序。
可用的驱动程序名称有 |dvips|、|dvipdfm|、|pdftex|、|luatex|、|vtex|、|xetex|、|auto| 和 |none|。
除了 |auto| 和 |none| 外的名称可以直接使用不带 |driver=| 的名称指定。
|driver=auto| 会使自动检测驱动程序的功能生效,无论之前的设置是什么。
|driver=none| 会禁用自动检测,并且不设置驱动程序,这在你想让其他包来决定驱动程序设置时可能会有用。
例如,如果你想在使用 \textsf{crop} 宏包时使用 \Gm{},你应该在 \textsf{crop} 宏包之前调用 |\usepackage[driver=none]{geometry}|。
\switchcolumn[0]*\item[\onlypre dvips] writes the paper size in dvi output with the \cs{special}
    macro. If you use \textsl{dvips} as a DVI-to-PS driver,
    for example, to print a document with |\geometry{a3paper,landscape}|
    on A3 paper in landscape orientation, you don't need options
    ``|-t a3 -t landscape|'' to \textsl{dvips}. 
\switchcolumn\item[\onlypre dvips] 使用 \cs{special} 宏来在 dvi 输出中写入纸张大小。
如果你使用 \textsl{dvips} 作为 DVI 到 PS 的驱动程序,例如在 A3 纸上打印 |\geometry{a3paper,landscape}| 的文档时,
你不需要在 \textsl{dvips} 中使用 ``|-t a3 -t landscape|'' 选项。
\switchcolumn[0]*\item[\onlypre dvipdfm] works like |dvips| except for landscape correction.
     You can set this option when using \textsf{dvipdfmx} and
     \textsf{xdvipdfmx} to process the dvi output.
\switchcolumn\item[\onlypre dvipdfm] 与 |dvips| 相同,只是不会进行横向纠正。
当使用 \textsf{dvipdfmx} 和 \textsf{xdvipdfmx} 处理 dvi 输出时,可以设置此选项。
\switchcolumn[0]*\item[\onlypre pdftex] sets \cs{pdfpagewidth} and \cs{pdfpageheight}
     internally.
\switchcolumn\item[\onlypre pdftex] 内部设置 \cs{pdfpagewidth} 和 \cs{pdfpageheight}。
\switchcolumn[0]*\item[\onlypre luatex] sets \cs{pagewidth} and \cs{pageheight} internally.
\switchcolumn\item[\onlypre luatex] 内部设置 \cs{pagewidth} 和 \cs{pageheight}。
\switchcolumn[0]*\item[\onlypre xetex] is the same as |pdftex| except for ignoring
    |\pdf{h,v}origin| undefined in \XeLaTeX{}. This option is introduced in
    the version 5. Note that `geometry.cfg' in \TeX{} Live, which disables the
    auto-detection routine and sets |pdftex|, is no longer necessary,
    but has no problem even though it's left undeleted.
    Instead of |xetex|, you can specify |dvipdfm| with \XeLaTeX{}
    if you want to use specials of dvipdfm \XeTeX{} supports.
\switchcolumn\item[\onlypre xetex] 与 |pdftex| 相同,只是在 \XeLaTeX{} 中忽略 |\pdf{h,v}origin| 未定义。
此选项在版本 5 中引入。
注意,在 \TeX{} Live 中的 `geometry.cfg',它禁用了自动检测例程并设置为 |pdftex|,
虽然不再需要,但即使未删除也没有问题。
如果你想在 \XeLaTeX{} 中使用 dvipdfm \XeTeX{} 支持的特殊命令,
可以用 |dvipdfm| 替代 |xetex|。
\switchcolumn[0]*\item[\onlypre vtex] sets dimensions \cs{mediawidth} and \cs{mediaheight}
    for V\TeX. When this driver is selected (explicitly or
    automatically), \Gm\ will auto-detect which output mode
    (DVI, PDF or PS) is selected in V\TeX, and do proper
    settings for it.
\switchcolumn\item[\onlypre vtex] 为 V\TeX{} 设置 \cs{mediawidth} 和 \cs{mediaheight} 尺寸。
当选择了此驱动程序(显式或自动),\Gm{} 将自动检测在 V\TeX{} 中选择的输出模式(DVI、PDF 或 PS)并进行适当的设置。
\end{paracol}
\end{Options}

\columnratio{0.55}
\begin{paracol}{2}
    If explicit driver setting is mismatched with the typesetting program
    in use, the default driver |dvips| would be selected.
    \switchcolumn 如果显式设置的驱动程序与正在使用的排版程序不匹配,则会选择默认的驱动程序 |dvips|。
\end{paracol}
