\section{Option details\hfill 选项详细信息}
\columnratio{0.55}
\begin{paracol}{2}
This section describes all options available in \Gm.
Options with a dagger $^\dagger$ are not available 
as arguments of \cs{newgeometry} (See Section~\ref{sec:midchange}).
\switchcolumn
本节描述了 \Gm 中所有可用的选项。
带有†标记的选项不能作为 \cs{newgeometry} 的参数使用(参见第~\ref{sec:midchange}节)。
\end{paracol}

% \columnratio{0.55}
\begin{paracol}{2}
\switchcolumn[0]*
\subsection{Paper size}\label{sec:paper}

The options below set paper/media size and orientation.
\switchcolumn
\subsection{纸张大小}
以下选项设置纸张/媒体大小和方向。
\end{paracol}


\begin{Options}
\columnratio{0.55}
\begin{paracol}{2}
\item[\onlypre paper\OR papername] ~\\ 
specifies the paper size by name. |paper=|\meta{paper-name}.
For convenience, you can specify the paper name without |paper=|.
For example, |a4paper| is equivalent to |paper=a4paper|.
\switchcolumn
\item[\onlypre paper\OR papername] ~\\
通过名称指定纸张大小。|paper=|\meta{纸张名称}。
为了方便起见,您可以省略 |paper=| 直接指定纸张名称。
例如,|a4paper| 等同于 |paper=a4paper|。
\switchcolumn[0]*
\item[\onlypre \vtop{
\hbox{a0paper, a1paper, a2paper, a3paper, a4paper, a5paper, a6paper,}
\hbox{b0paper, b1paper, b2paper, b3paper, b4paper, b5paper, b6paper,}
\hbox{c0paper, c1paper, c2paper, c3paper, c4paper, c5paper, c6paper,}
\hbox{b0j, b1j, b2j, b3j, b4j, b5j, b6j,}
\hbox{ansiapaper, ansibpaper, ansicpaper, ansidpaper, ansiepaper,}
\hbox{letterpaper, executivepaper, legalpaper}}]~\\[1ex] 
    specifies paper name.  The value part is ignored even if any.
    For example, the followings have the same effect:
    |a5paper|, |a5paper=true|, |a5paper=false| and so forth.
    |a[0-6]paper|, |b[0-6]paper| and |c[0-6]paper| are ISO A, B and C
    series of paper sizes respectively.
    The JIS (Japanese Industrial Standards) A-series is identical to the
    ISO A-series, but the JIS B-series is different from the ISO B-series.
    |b[0-6]j| should be used for the JIS B-series. 
\switchcolumn
\item[\onlypre \vtop{
\hbox{a0paper, a1paper, a2paper, a3paper, a4paper, a5paper, a6paper,}
\hbox{b0paper, b1paper, b2paper, b3paper, b4paper, b5paper, b6paper,}
\hbox{c0paper, c1paper, c2paper, c3paper, c4paper, c5paper, c6paper,}
\hbox{b0j, b1j, b2j, b3j, b4j, b5j, b6j,}
\hbox{ansiapaper, ansibpaper, ansicpaper, ansidpaper, ansiepaper,}
\hbox{letterpaper, executivepaper, legalpaper}}]~\\[1ex] 
指定纸张名称。即使有值部分,也会被忽略。
例如,下面的选项具有相同的效果:
|a5paper|, |a5paper=true|, |a5paper=false| 等等。
|a[0-6]paper|, |b[0-6]paper| 和 |c[0-6]paper| 分别是 ISO A、B 和 C
系列纸张大小。
JIS(日本工业标准)A 系列与 ISO A 系列相同,但 JIS B 系列与 ISO B 系列不同。
应该使用 |b[0-6]j| 来表示 JIS B 系列。

\switchcolumn[0]*
\item[\onlypre screen] a special paper size with (W,H) = (225mm,180mm).
For presentation with PC and video projector, ``|screen,centering|''
with `slide' documentclass would be useful.
\switchcolumn
\item[\onlypre screen] 一种特殊的纸张尺寸,宽度(W)为225mm,高度(H)为180mm。
对于使用个人电脑和视频投影仪进行演示,使用带有 `slide' 文档类的 ``|screen,centering|'' 会很有用。
\switchcolumn[0]*
\item[\onlypre paperwidth] width of the paper. |paperwidth=|\meta{length}.
\switchcolumn
\item[\onlypre paperwidth] 纸张的宽度。|paperwidth=|\meta{长度}。
\switchcolumn[0]*
\item[\onlypre paperheight] height of the paper. |paperheight=|\meta{length}.
\switchcolumn
\item[\onlypre paperheight] 纸张的高度。|paperheight=|\meta{长度}。
\switchcolumn[0]*
\item[\onlypre papersize] width and height of the paper. 
    |papersize=|\argii{width}{height} or |papersize=|\meta{length}.
\switchcolumn
\item[\onlypre papersize] 纸张的宽度和高度。|papersize=|\argii{宽度}{高度} 或 |papersize=|\meta{长度}。
\switchcolumn[0]*
\item[\onlypre landscape] switches the paper orientation to landscape mode.
\switchcolumn
\item[\onlypre landscape] 将纸张方向切换为横向模式。
\switchcolumn[0]*
\item[\onlypre portrait] switches the paper orientation to portrait mode.
This is equivalent to |landscape=false|.
\switchcolumn
\item[\onlypre portrait] 将纸张方向切换为纵向模式。
这相当于 |landscape=false|。
\end{paracol}
\end{Options}

\columnratio{0.55}
\begin{paracol}{2}
The options for paper names (e.g., |a4paper|) and orientation
(|portrait| and |landscape|) can be set as document class options. 
For example, you can set |\documentclass[a4paper,landscape]{article}|, 
then |a4paper| and |landscape| are processed in \Gm\ as well.
This is also the case for |twoside| and |twocolumn|
(see also Section~\ref{sec:dimension}).
\switchcolumn
纸张名称选项(例如,|a4paper|)和方向选项(|portrait| 和 |landscape|)可以作为文档类选项进行设置。
例如,您可以设置 |\documentclass[a4paper,landscape]{article}|,那么 |a4paper| 和 |landscape| 也会在 \Gm\ 中进行处理。
对于 |twoside| 和 |twocolumn| 也是如此(详见第~\ref{sec:dimension}~节)。
\end{paracol}

% 
\columnratio{0.55}
\begin{paracol}{2}
\switchcolumn[0]*
\subsection{Layout size}
You can specify the layout area with options described in this
section regardless of the paper size.
The options would help to print the specified layout to a different
sized paper.  For example, with |a4paper| and |layout=a5paper|, the
package uses `A5' layout to calculate margins on 'A4' paper.
The layout size defaults to the same as the paper.
The options for the layout size are available in \cs{newgeometry},
so that you can change the layout size in the middle of the document.
The paper size itself can't be changed though.
Figure~\ref{fig:layoutandpaper} shows what the difference between
|layout| and |paper| is.
\switchcolumn
\subsection{布局尺寸}

无论纸张大小如何,您都可以使用本节中描述的选项来指定布局区域。
这些选项可以帮助您将指定的布局打印到不同大小的纸张上。
例如,使用 |a4paper| 和 |layout=a5paper|,该宏包会在“A4”纸上使用“A5”布局来计算边距。
布局尺寸默认与纸张相同。
布局尺寸的选项在 \cs{newgeometry} 中也可用,因此您可以在文档的中间更改布局尺寸。
但是纸张尺寸本身无法更改。
图~\ref{fig:layoutandpaper}~展示了 |layout| 和 |paper| 之间的区别。
\end{paracol}


\begin{Options}
\columnratio{0.55}
\begin{paracol}{2}
\item[layout] specifies the layout size by paper
name. |layout=|\meta{paper-name}. All the paper names defined in \Gm\
are available. See Section~\ref{sec:paper} for details.
\switchcolumn
\item[layout] 按纸张名称指定布局尺寸。|layout=|\meta{纸张名称}。
所有在 \Gm\ 中定义的纸张名称都可用。详情请参阅第~\ref{sec:paper}~节。
\switchcolumn[0]*
\item[layoutwidth] width of the layout. |layoutwidth=|\meta{length}.
\switchcolumn
\item[layoutwidth] 布局的宽度。|layoutwidth=|\meta{长度}。
\switchcolumn[0]*
\item[layoutheight] height of the layout. |layoutheight=|\meta{length}.
\switchcolumn
\item[layoutheight] 布局的高度。|layoutheight=|\meta{长度}。
\switchcolumn[0]*
\item[layoutsize] width and height of the layout.
|layoutsize=|\argii{width}{height} or |layoutsize=|\meta{length}.
\switchcolumn
\item[layoutsize] 布局的宽度和高度。|layoutsize=|\argii{宽度}{高度} 或 |layoutsize=|\meta{长度}。
\switchcolumn[0]*
\item[layouthoffset] specifies the horizontal offset from the left edge of
the paper. |layouthoffset=|\meta{length}.
\switchcolumn
\item[layouthoffset] 指定布局相对于纸张左边缘的水平偏移量。|layouthoffset=|\meta{长度}。
\switchcolumn[0]*
\item[layoutvoffset] specifies the vertical offset from the top edge of
the paper. |layoutvoffset=|\meta{length}.
\switchcolumn
\item[layoutvoffset] 指定布局相对于纸张顶边缘的垂直偏移量。|layoutvoffset=|\meta{长度}。
\switchcolumn[0]*
\item[layoutoffset] specifies both horizontal and vertical offsets. 
|layoutoffset=|\argii{hoffset}{voffset} or |layoutsize=|\meta{length}.
\switchcolumn
\item[layoutoffset] 指定布局的水平和垂直偏移量。|layoutoffset=|\argii{水平偏移量}{垂直偏移量} 或 |layoutsize=|\meta{长度}。
\end{paracol}
\end{Options}
\begin{figure}
 \centering\small
 {\unitlength=.6pt
 \begin{picture}(450,250)(0,-10)
 \put(20,0){\makebox(168,12)[r]{\gpart{paper}}}
 \put(20,0){\framebox(170,230){}}
 \put(21,40){\dashbox{3}(140,189){}}
 \put(21,28){\makebox(140,12)[r]{\gpart{layout}}}
 \put(40,50){\makebox(100,10){\gpart{foot}}}
 \put(40,50){\line(1,0){100}}
 \put(40,65){\framebox(100,125){\gpart{body}}}
 \put(40,200){\framebox(100,10){\gpart{head}}}
 \put(20,230){\makebox(140,20){|layoutwidth|}}
 \put(40,240){\vector(-1,0){20}}
 \put(140,240){\vector(1,0){20}}
 \put(10,145){\vector(0,1){85}}
 \put(15,125){\makebox(0,20)[r]{|layoutheight|}}
 \put(10,125){\vector(0,-1){85}}
 \put(280,0){\makebox(168,12)[r]{\gpart{paper}}}
 \put(280,0){\framebox(170,230){}}
 \put(293,35){\dashbox{3}(140,189){}}
 \put(293,23){\makebox(140,12)[r]{\gpart{layout}}}
 \put(312,45){\makebox(100,10){\gpart{foot}}}
 \put(312,45){\line(1,0){100}}
 \put(312,60){\framebox(100,125){\gpart{body}}}
 \put(312,195){\framebox(100,10){\gpart{head}}}
 \put(235,230){\makebox(80,20)[l]{|layouthoffset|}}
 \put(260,210){\vector(1,0){20}}
 \put(308,210){\vector(-1,0){15}}
 \put(260,210){\line(-1,2){10}}
 \put(355,230){\makebox(100,20){|layoutvoffset|}}
 \put(350,250){\vector(0,-1){20}}
 \put(350,209){\vector(0,1){15}}
 \end{picture}}
 \caption[layout and paper]{%
 \begin{minipage}[t]{.7\textwidth}\raggedright\small
 The dimensions related to the layout size. Note that the layout size
 defaults to the same size as the paper, so you don't have to specify
 layout-related options explicitly in most cases.
 \end{minipage}}
 \label{fig:layoutandpaper}
\end{figure}

 
\columnratio{0.55}
\begin{paracol}{2}
\subsection{Body size}\label{sec:body}

The options specifying the size of \gpart{total body} are described in this
section.
\switchcolumn
\subsection{Body size}
本节描述了指定\gpart{总正文}尺寸的选项。
\end{paracol}

\begin{Options}
    \columnratio{0.55}
\begin{paracol}{2}
\item[hscale]
ratio of width of \gpart{total body} to \cs{paperwidth}. 
|hscale=|\meta{h-scale}, e.g., |hscale=0.8| is equivalent to
|width=0.8|\cs{paperwidth}. (|0.7| by default)
\switchcolumn
\item[hscale] \gpart{总正文}的宽度与 \cs{paperwidth} 的比例。|hscale=|\meta{h-scale},例如,|hscale=0.8| 等同于 |width=0.8|\cs{paperwidth}。(默认为 |0.7|)   
   \switchcolumn[0]*   
\item[vscale]
   ratio of height of \gpart{total body} to \cs{paperheight}, e.g.,
   |vscale=|\meta{v-scale}. (|0.7| by default) |vscale=0.9| is equivalent
   to |height=0.9|\cs{paperheight}.
\switchcolumn
\item[vscale] \gpart{总正文}的高度与 \cs{paperheight} 的比例。例如,|vscale=|\meta{v-scale}。(默认为 |0.7|)
|vscale=0.9| 等同于 |height=0.9|\cs{paperheight}。
\switchcolumn[0]*
\item[scale] ratio of \gpart{total body} to the paper.
   |scale=|\argii{h-scale}{v-scale} or |scale=|\meta{scale}.
   (|0.7| by default)
\switchcolumn
\item[scale] \gpart{总正文}与纸张的比例。|scale=|\argii{h-scale}{v-scale} 或 |scale=|\meta{scale}。(默认为 |0.7|)
\switchcolumn[0]*
\item[width\OR totalwidth] ~\\
width of \gpart{total body}. |width=|\meta{length} or
|totalwidth=|\meta{length}. This dimension defaults to |textwidth|,
but if |includemp| is set to |true|, |width| $\ge$ |textwidth| 
because |width| includes the width of the marginal notes.
If |textwidth| and |width| are specified at the same time, 
|textwidth| takes priority over |width|.
\switchcolumn
\item[width\OR totalwidth] ~\\
\gpart{总正文}的宽度。|width=|\meta{长度} 或 |totalwidth=|\meta{长度}。
该尺寸默认为 |textwidth|,但如果将 |includemp| 设置为 |true|,则 |width| $\ge$ |textwidth|,因为 |width| 包括边注的宽度。
如果同时指定 |textwidth| 和 |width|,则 |textwidth| 优先于 |width|。
\switchcolumn[0]*
\item[height\OR totalheight] ~\\
height of \gpart{total body}, excluding header and footer by default.
If |includehead| or |includefoot| is set, |height| includes
the head or foot of the page as well as |textheight|.
|height=|\meta{length} or |totalheight=|\meta{length}. If both
|textheight| and |height| are specified, |height| will be ignored.
\switchcolumn
\item[height\OR totalheight] ~\\
\gpart{总正文}的高度,默认不包括页眉和页脚。
如果设置了 |includehead| 或 |includefoot|,则 |height| 包括页眉或页脚以及 |textheight|。
|height=|\meta{长度} 或 |totalheight=|\meta{长度}。
如果同时指定了 |textheight| 和 |height|,则会忽略 |height|。
\switchcolumn[0]*
\item[total] width and height of \gpart{total body}.\\
   |total=|\argii{width}{height} or |total=|\meta{length}.
   \switchcolumn
   \item[total] \gpart{总正文}的宽度和高度。|total=|\argii{宽度}{高度} 或 |total=|\meta{长度}。
\switchcolumn[0]*
\item[textwidth] specifies \cs{textwidth}, the width of \gpart{body} 
   (the text area). |textwidth=|\meta{length}.
   \switchcolumn
   \item[textwidth] 指定 \cs{textwidth},即正文(文本区域)的宽度。|textwidth=|\meta{长度}。
   \switchcolumn[0]*
\item[textheight] specifies \cs{textheight}, the height of
   \gpart{body} (the text area). |textheight=|\meta{length}.
   \switchcolumn
   \item[textheight] 指定 \cs{textheight},即正文(文本区域)的高度。|textheight=|\meta{长度}。
\switchcolumn[0]*
\item[text\OR body] specifies both \cs{textwidth} and \cs{textheight}
   of the body of page. |body=|\argii{width}{height} or
   |text=|\meta{length}.
   \switchcolumn
   \item[text\OR body] 指定正文(文本区域)的 \cs{textwidth} 和 \cs{textheight}。|body=|\argii{宽度}{高度} 或 |text=|\meta{长度}。
\switchcolumn[0]*
\item[lines] enables users to specify \cs{textheight} by the number
   of lines. |lines|=\meta{integer}.
   \switchcolumn
   \item[lines] 允许用户通过行数指定 \cs{textheight}。|lines=|\meta{整数}。
\switchcolumn[0]*
\item[includehead] includes the head of the page, \cs{headheight}
   and \cs{headsep}, into \gpart{total body}. It is set to |false| by
   default. It is opposite to |ignorehead|. See
   Figure~\ref{fig:includes} and Figure~\ref{fig:modes}.
   \switchcolumn
   \item[includehead] 将页眉,\cs{headheight} 和 \cs{headsep} 包含到 \gpart{总正文} 中。默认为 |false|。与 |ignorehead| 相反。参见图\ref{fig:includes} 和图\ref{fig:modes}。
\switchcolumn[0]*
\item[includefoot] includes the foot of the page, \cs{footskip},
   into \gpart{total body}. It is opposite to |ignorefoot|.
   It is |false| by default. See Figure~\ref{fig:includes} and
   Figure~\ref{fig:modes}.
   \switchcolumn
   \item[includefoot] 将页脚,\cs{footskip},包含到 \gpart{总正文} 中。与 |ignorefoot| 相反。默认为 |false|。参见图~\ref{fig:includes} 和图~\ref{fig:modes}。
   \switchcolumn[0]*
\item[includeheadfoot]~\\ 
   sets both |includehead| and |includefoot| to |true|, which is opposite
   to |ignoreheadfoot|. See Figure~\ref{fig:includes} and
   Figure~\ref{fig:modes}.
\switchcolumn
\item[includeheadfoot]~\\ 
将 |includehead| 和 |includefoot| 都设置为 |true|,与 |ignoreheadfoot| 相反。参见图\ref{fig:includes} 和图~\ref{fig:modes}。
\switchcolumn[0]*
\item[includemp] includes the margin notes,  \cs{marginparwidth}
   and \cs{marginparsep}, into \gpart{body} when calculating horizontal
   calculation.
   \switchcolumn
   \item[includemp] 当计算水平布局时,将边注,\cs{marginparwidth} 和 \cs{marginparsep},包含到 \gpart{正文} 中。
\switchcolumn[0]*
\item[includeall] sets both |includeheadfoot| and |includemp| to
   |true|. See Figure~\ref{fig:modes}.
   \switchcolumn
   \item[includeall] 将 |includeheadfoot| 和 |includemp| 都设置为 |true|。参见图~\ref{fig:modes}。
\switchcolumn[0]*
\item[ignorehead] disregards the head of the page,
   |headheight| and |headsep|, in determining vertical layout, but does not
   change those lengths. It is equivalent to |includehead=false|. It is set
   to |true| by default. See also |includehead|.
   \switchcolumn
   \item[ignorehead] 在确定垂直布局时忽略页眉,即 \cs{headheight} 和 \cs{headsep},但不更改这些长度。等同于 |includehead=false|。默认为 |true|。参见 |includehead|。
\switchcolumn[0]*
\item[ignorefoot] disregards the foot of page, |footskip|,
   in determining vertical layout, but does not change that length.
   This option defaults to |true|. See also |includefoot|.
   \switchcolumn
   \item[ignorefoot] 在确定垂直布局时忽略页脚,即 \cs{footskip},但不更改该长度。默认为 |true|。参见 |includefoot|。
\switchcolumn[0]*
\item[ignoreheadfoot]~\\ sets both |ignorehead| and |ignorefoot|
   to |true|. See also |includeheadfoot|.
   \switchcolumn
   \item[ignoreheadfoot] ~\\ 将 |ignorehead| 和 |ignorefoot| 都设置为 |true|。参见 |includeheadfoot|。
\switchcolumn[0]*
\item[ignoremp] disregards the marginal notes in determining the
   horizontal margins (defaults to |true|). If marginal notes overrun
   the page, the warning message will be displayed when |verbose=true|.
   See also |includemp| and Figure~\ref{fig:modes}.
   \switchcolumn
   \item[ignoremp] 在确定水平边距时忽略边注(默认为 |true|)。如果边注超出页面,则在 |verbose=true| 时将显示警告消息。参见 |includemp| 和图\ref{fig:modes}。
\switchcolumn[0]*
\item[ignoreall] sets both |ignoreheadfoot| and |ignoremp| to |true|. 
   See also |includeall|.
   \switchcolumn
   \item[ignoreall] 将 |ignoreheadfoot| 和 |ignoremp| 都设置为 |true|。参见 |includeall|。
\switchcolumn[0]*
\item[heightrounded]~\\
   This option rounds \cs{textheight} to \textit{n}-times (\textit{n}:
   an integer) of \cs{baselineskip} plus \cs{topskip} to avoid 
   ``underfull vbox'' in some cases. For example, if \cs{textheight} is
   486pt with \cs{baselineskip} 12pt and \cs{topskip} 10pt, then
   \begin{quote}
     $(39\times12\textrm{pt}+10\textrm{pt}=)\: 478\textrm{pt}
      < 486\textrm{pt} < 
     490\textrm{pt} \:(=40\times12\textrm{pt}+10\textrm{pt})$,
   \end{quote}
   as a result \cs{textheight} is rounded to 490pt. |heightrounded=false|
   by default.
   \switchcolumn
   \item[heightrounded] ~\\
此选项将 \cs{textheight} 舍入为 \textit{n} 倍(\textit{n}:整数)的 \cs{baselineskip} 加上 \cs{topskip},以避免在某些情况下出现``underfull vbox''。
例如,如果 \cs{textheight} 是 486pt,\cs{baselineskip} 是 12pt,\cs{topskip} 是 10pt,则
\begin{quote}
    $(39\times12\textrm{pt}+10\textrm{pt}=)\: 478\textrm{pt}
     < 486\textrm{pt} < 
    490\textrm{pt} \:(=40\times12\textrm{pt}+10\textrm{pt})$,
  \end{quote}
结果 \cs{textheight} 被舍入为 490pt。默认情况下,|heightrounded=false|。
\end{paracol}
\end{Options}

Figure~\ref{fig:modes} illustrates various layouts with different layout
modes. The dimensions for a header and a footer can be controlled by
|nohead| or |nofoot| mode, which sets each length to 0pt directly.
On the other hand, options with the prefix |ignore| do \textit{not}
change the corresponding native dimensions.
\begin{figure}
 \centering\small
 {\unitlength=.65pt
 \begin{picture}(460,525)(0,0)
 \put( 20,310){\framebox(120,170){}}
 \put( 20,507){\makebox(120,0)[bl]%
 {\textbf{(a)}~|includeheadfoot|}}
 \put( 20,460){\line(1,0){120}}\put( 20,450){\line(1,0){120}}
 \put( 20,330){\line(1,0){120}}
 \put( 20,485){\makebox(120,0)[br]{\gpart{total body}}}
 \put( 20,335){\makebox(120,0)[bc]{|textwidth|}}
 \put(150,470){\makebox(0,0)[l]{|headheight|}}
 \put(150,450){\makebox(0,0)[l]{|headsep|}}
 \put(150,390){\makebox(0,0)[l]{|textheight|}}
 \put(150,320){\makebox(0,0)[l]{|footskip|}}
 \put( 10,460){\makebox(120,20)[bc]{\gpart{head}}}
 \put( 10,320){\makebox(120,140)[c]{\gpart{body}}}
 \put( 10,310){\makebox(120,10)[c]{\gpart{foot}}}
 \put(250,310){\framebox(120,170){}}
 \put(250,507){\makebox(120,0)[bl]%
 {\textbf{(b)}~|includeall|}}
 \put(250,460){\line(1,0){95}}\put(250,450){\line(1,0){95}}
 \put(250,330){\line(1,0){95}}\put(345,330){\line(0,1){120}}
 \put(350,330){\line(0,1){120}}\put(350,450){\line(1,0){20}}
 \put(350,330){\line(1,0){20}}
 \put(250,485){\makebox(120,0)[br]{\gpart{total body}}}
 \put(250,460){\makebox(95,20)[bc]{\gpart{head}}}
 \put(250,320){\makebox(95,140)[c]{\gpart{body}}}
 \put(385,390){\makebox(95,0)[cl]%
 {\gpart{\shortstack[l]{marginal\\note}}}}
 \put(250,310){\makebox(95,10)[c]{\gpart{foot}}}
 \put(250,335){\makebox(95,0)[bc]{|textwidth|}}
 \multiput(360, 390)(4,0){6}{\line(1,0){2}}
 \multiput(348,333)(0,-4){12}{\line(0,1){2}}
 \multiput(360,333)(0,-4){8}{\line(0,1){2}}
 \put(355,292){\makebox(0,0)[bl]{|marginparwidth|}}
 \put(345,275){\makebox(0,0)[bl]{|marginparsep|}}
 \put( 20, 40){\framebox(120,170){}}
 \put( 20,237){\makebox(120,0)[bl]%
 {\textbf{(c)}~|includefoot|}}
 \put( 20, 60){\line(1,0){120}}
 \put( 20,215){\makebox(120,0)[br]{\gpart{total body}}}
 \put(150,130){\makebox(0,0)[l]{|textheight|}}
 \put(150, 50){\makebox(0,0)[l]{|footskip|}}
 \put( 20, 50){\makebox(120,160)[c]{\gpart{body}}}
 \put( 20, 40){\makebox(120,10)[c]{\gpart{foot}}}
 \put( 20, 65){\makebox(120,10)[c]{|textwidth|}}
 \put(250, 40){\framebox(120,170){}}
 \put(250,237){\makebox(120,0)[bl]%
 {\textbf{(d)}~|includefoot,includemp|}}
 \put(250, 60){\line(1,0){95}}\put(350, 60){\line(1,0){20}}
 \put(250,215){\makebox(120,0)[br]{\gpart{total body}}}
 \put(250, 50){\makebox(95,160)[c]{\gpart{body}}}
 \put(385,130){\makebox(95,0)[cl]%
 {\gpart{\shortstack[l]{marginal\\note}}}}
 \put(250, 40){\makebox(95,10)[c]{\gpart{foot}}}
 \put(250, 65){\makebox(95,0)[bc]{|textwidth|}}
 \put(345, 60){\line(0,1){150}}\put(350, 60){\line(0,1){150}}
 \multiput(360, 130)(4,0){6}{\line(1,0){2}}
 \multiput(348, 63)(0,-4){12}{\line(0,1){2}}
 \multiput(360, 63)(0,-4){8}{\line(0,1){2}}
 \put(355,22){\makebox(0,0)[bl]{|marginparwidth|}}
 \put(345, 5){\makebox(0,0)[bl]{|marginparsep|}}
 \end{picture}}
 \caption[Sample layouts for \gpart{total body} with different 
    layout modes]{%
 \begin{minipage}[t]{.8\textwidth}\small
   Sample layouts for \gpart{total body} with different switches.
   (a) |includeheadfoot|, (b) |includeall|, (c) |includefoot|
    and (d) |includefoot,includemp|. 
   If |reversemp| is set to |true|, the location of the
   marginal notes are swapped on every page.
   Option |twoside| swaps both margins and marginal notes on verso pages.
   Note that the marginal note, if any, is printed despite
   |ignoremp| or |includemp=false| and overrun the page in some cases.
 \end{minipage}}
 \label{fig:modes}
\end{figure}

The following options can specify body and margins simultaneously with
three comma-separated values in braces.

\begin{Options}
\item[hdivide] horizontal partitions (left,width,right).
  |hdivide=|\argiii{left margin}{width}{right margin}. 
  Note that you should not specify all of the three parameters.
  The best way of using this option is to specify two of three and 
  leave the rest with null(nothing) or `|*|'. For example, when you set
  |hdivide={2cm,15cm, }|, the margin from the right-side edge of page
  will be determined calculating |paperwidth-2cm-15cm|.
\item[vdivide] vertical partitions (top,height,bottom).
  |vdivide=|\argiii{top margin}{height}{bottom margin}.
\item[divide] |divide=|\vargiii{$A$}{$B$}{$C$} is interpreted  as 
  |hdivide=|\vargiii{$A$}{$B$}{$C$} and |vdivide=|\vargiii{$A$}{$B$}{$C$}.
\end{Options}

\subsection{Margin size}\label{sec:margin}

The options specifying the size of the margins are listed below.

\begin{Options}
\item[left\OR lmargin\OR inner]~\\
   left margin (for oneside) or inner margin (for twoside) of 
   \gpart{total body}. In other words, the distance between the left (inner)
   edge of the paper and that of \gpart{total body}. |left=|\meta{length}.
   |inner| has no special meaning, just an alias of |left| and |lmargin|.
\item[right\OR rmargin\OR outer]~\\ 
   right or outer margin of \gpart{total body}. |right=|\meta{length}.
\item[top\OR tmargin] top margin of the page. |top=|\meta{length}.
   Note this option has nothing to do with the native dimension
   \cs{topmargin}.
\item[bottom\OR bmargin]~\\ 
   bottom margin of the page. |bottom=|\meta{length}.
\item[hmargin] left and right margin.
  |hmargin=|\argii{left margin}{right margin} or |hmargin=|\meta{length}.
\item[vmargin] top and bottom margin.
  |vmargin=|\argii{top margin}{bottom margin} or |vmargin=|\meta{length}.
\item[margin] |margin=|\vargii{$A$}{$B$} is equivalent to 
  |hmargin=|\vargii{$A$}{$B$} and |vmargin=|\vargii{$A$}{$B$}.
  |margin=|$A$ is automatically expanded to |hmargin=|$A$ and |vmargin=|$A$.
\item[hmarginratio]
  horizontal margin ratio of |left| (inner) to |right| (outer). 
  The value of \meta{ratio} should be specified with colon-separated 
  two values. Each value should be a positive integer less than 100
  to prevent arithmetic overflow, e.g., |2:3| instead of |1:1.5|.
  The default ratio is |1:1| for oneside, |2:3| for twoside.
\item[vmarginratio]
   vertical margin ratio of |top| to |bottom|. The default ratio is |2:3|.
\item[marginratio\OR ratio]~\\
   horizontal and vertical margin ratios.
  |marginratio=|\argii{horizontal ratio}{vertical ratio} or
  |marginratio=|\meta{ratio}.
\item[hcentering] sets auto-centering horizontally and is
  equivalent to |hmarginratio=1:1|. It is set to |true| by default for
  oneside. See also |hmarginratio|.
\item[vcentering] sets auto-centering vertically and is
  equivalent to |vmarginratio=1:1|. The default is |false|.
  See also |vmarginratio|.
\item[centering] sets auto-centering and is equivalent to
  |marginratio=1:1|. See also |marginratio|. The default is |false|.
  See also |marginratio|.
\item[twoside] switches on twoside mode with left and right margins swapped
  on verso pages. The option sets \cs{@twoside} and \cs{@mparswitch} 
  switches. See also |asymmetric|.
\item[asymmetric] implements a twosided layout in which margins are
  not swapped on alternate pages (by setting \cs{oddsidemargin} to 
  \cs{evensidemargin} |+| |bindingoffset|) and in which the marginal notes
  stay always on the same side. This option can be used as an alternative
  to the twoside option. See also |twoside|.
\item[bindingoffset]~\\ removes a specified space 
  from the lefthand-side of the page for oneside or the inner-side for
  twoside. |bindingoffset=|\meta{length}. This is useful if pages 
  are bound by a press binding (glued, stitched, stapled \ldots).
  See Figure~\ref{fig:bindingoffset}.
\item[hdivide] See description in Section~\ref{sec:body}.
\item[vdivide] See description in Section~\ref{sec:body}.
\item[divide] See description in Section~\ref{sec:body}.
\end{Options}
\begin{figure}
 \centering\small
 {\unitlength=.65pt
 \begin{picture}(500,270)(0,0)
 \put(20,0){\framebox(170,230){}}
 \put(20,255){\makebox(80,20)[l]{\textbf{a)}~every page for oneside or}}
 \put(20,240){\makebox(80,20)[l]{\hspace{3ex}odd pages for twoside}}
 \put(110,225){\makebox(80,20)[r]{\gpart{paper}}}
 \put(55,37){\framebox(110,170)[tc]{\gpart{total body}}}
 \multiput(38,0)(0,7){33}{\line(0,1){4}}
 \put(38,100){\vector(1,0){17}}\put(55,100){\vector(-1,0){17}}
 \put(60,95){\makebox(80,10)[l]{|left|}}
 \put(60,80){\makebox(80,10)[l]{(|inner|)}}
 \put(165,100){\vector(1,0){25}}\put(190,100){\vector(-1,0){25}}
 \put(195,95){\makebox(80,10)[l]{|right|}}
 \put(195,80){\makebox(80,10)[l]{(|outer|)}}
 \put(20,16){\vector(1,0){18}}
 \put(45,10){\makebox(80,10)[bl]{|bindingoffset|}}
 \put(280,255){\makebox(80,20)[l]{\textbf{b)}~even (back) pages for twoside}}
 \put(280,0){\framebox(170,230){}}
 \put(370,225){\makebox(80,20)[r]{\gpart{paper}}}
 \put(305,37){\framebox(110,170)[tc]{\gpart{total body}}}
 \multiput(432,0)(0,7){33}{\line(0,1){4}}
 \put(280,100){\vector(1,0){25}}\put(305,100){\vector(-1,0){25}}
 \put(310,95){\makebox(80,10)[l]{|outer|}}
 \put(310,80){\makebox(80,10)[l]{(|right|)}}
 \put(415,100){\vector(1,0){17}}\put(432,100){\vector(-1,0){17}}
 \put(373,95){\makebox(80,10)[l]{|inner|}}
 \put(373,80){\makebox(80,10)[l]{(|left|)}}
 \put(450,16){\vector(-1,0){18}}
 \put(330,10){\makebox(80,10)[bl]{|bindingoffset|}}
 \end{picture}}
 \caption[\texttt{bindingoffset} option]{%
  \begin{minipage}[t]{.8\textwidth}\raggedright\small
  The option |bindingoffset| adds the specified length to the inner margin.
  Note that |twoside| option swaps the horizontal margins and the
  marginal notes together with |bindingoffset| on even pages (see
  \textbf{b)}), but |asymmetric| option suppresses the swap of the
  margins and marginal notes (but |bindingoffset| is still swapped).
  \end{minipage}}
 \label{fig:bindingoffset}
\end{figure}

\subsection{Native dimensions}\label{sec:dimension}

The options below overwrite \LaTeX\ native dimensions and switches for page
layout (See the right-hand side in Figure~\ref{fig:layout}).

\begin{Options}
\item[headheight\OR head]~\\
   modifies \cs{headheight}, height of header.
   |headheight=|\meta{length} or |head=|\meta{length}.
\item[headsep] modifies \cs{headsep}, separation between header and text
   (body). |headsep=|\meta{length}.
\item[footskip\OR foot]~\\ modifies \cs{footskip}, distance separation
   between baseline of last line of text and baseline of footer.
   |footskip=|\meta{length} or |foot=|\meta{length}.
\item[nohead] eliminates spaces for the head of the page, which is
   equivalent to both \cs{headheight}|=0pt| and \cs{headsep}|=0pt|.
\item[nofoot] eliminates spaces for the foot of the page, which is
   equivalent to \cs{footskip}|=0pt|.
\item[noheadfoot] equivalent to |nohead| and |nofoot|, which means that
   \cs{headheight}, \cs{headsep} and \cs{footskip} are all set to |0pt|.
\item[footnotesep] changes the dimension \cs{skip}\cs{footins}, separation
   between the bottom of text body and the top of footnote text.
\item[marginparwidth\OR marginpar]~\\ 
   modifies \cs{marginparwidth}, width of the marginal notes.
   |marginparwidth=|\meta{length}.
\item[marginparsep] modifies \cs{marginparsep}, separation between
   body and marginal notes. |marginparsep=|\meta{length}.
\item[nomarginpar] shrinks spaces for marginal notes to 0pt, which
   is equivalent to \cs{marginparwidth}|=0pt| and \cs{marginparsep}|=0pt|.
\item[columnsep] modifies \cs{columnsep}, the separation between two
   columns in |twocolumn| mode.
\item[hoffset]  modifies \cs{hoffset}. |hoffset=|\meta{length}.
\item[voffset]  modifies \cs{voffset}. |voffset=|\meta{length}.
\item[offset] horizontal and vertical offset.\\
   |offset=|\argii{hoffset}{voffset} or |offset=|\meta{length}.
\item[twocolumn] sets |twocolumn| mode with \cs{@twocolumntrue}.
  |twocolumn=false| denotes onecolumn mode with\cs{@twocolumnfalse}.
  Instead of |twocolumn=false|, you can specify |onecolumn| (which
  defaults to |true|)
\item[onecolumn] works as |twocolumn=false|. On the other hand,
  |onecolumn=false| is equivalent to |twocolumn|. 
\item[twoside] sets both \cs{@twosidetrue} and \cs{@mparswitchtrue}.
  See Section~\ref{sec:margin}.
\item[textwidth] sets \cs{textwidth} directly. See Section~\ref{sec:body}.
\item[textheight] sets \cs{textheight} directly. See Section~\ref{sec:body}.
\item[reversemp\OR reversemarginpar]~\\
  makes the marginal notes appear in the left (inner) margin with
  \cs{@reversemargintrue}. The option doesn't change |includemp| mode.
  It's set |false| by default.
\end{Options}

\subsection{Drivers}\label{sec:drivers}

The package supports drivers |dvips|, |dvipdfm|, |pdftex|, |luatex|,
|xetex| and |vtex|. You can also set |dvipdfm| for \textsf{dvipdfmx} and
\textsf{xdvipdfmx} The options |dvipdfmx| and |xdvipdfmx| are also supported
as aliases for the |dvipdfm| option.
|pdftex| for \textsf{pdflatex}, and |vtex| for
V\TeX{} environment.
The driver options are exclusive. The driver can be set by either
|driver=|\meta{driver name} or any of the drivers directly like |pdftex|.
By default, \Gm\ guesses the driver appropriate to the system
in use. Therefore, you don't have to set a driver in most cases.
However, if you want to use |dvipdfm|, you should specify it explicitly.

\begin{Options}
\item[\onlypre driver] specifies the driver with |driver=|\meta{driver name}. 
|dvips|, |dvipdfm|, |pdftex|, |luatex|, |vtex|, |xetex|, |auto| and |none| are
available as a driver name. The names except for |auto| and |none| can
be specified directly with the name without |driver=|.
|driver=auto| makes the auto-detection work whatever the previous setting is. 
|driver=none| disables the auto-detection and sets no driver, which
may be useful when you want to let other package work out the driver
setting. For example, if you want to use \textsf{crop} package with \Gm,
you should call |\usepackage[driver=none]{geometry}| before
the \textsf{crop} package.
\item[\onlypre dvips] writes the paper size in dvi output with the \cs{special}
    macro. If you use \textsl{dvips} as a DVI-to-PS driver,
    for example, to print a document with |\geometry{a3paper,landscape}|
    on A3 paper in landscape orientation, you don't need options
    ``|-t a3 -t landscape|'' to \textsl{dvips}. 
\item[\onlypre dvipdfm] works like |dvips| except for landscape correction.
     You can set this option when using \textsf{dvipdfmx} and
     \textsf{xdvipdfmx} to process the dvi output.
\item[\onlypre pdftex] sets \cs{pdfpagewidth} and \cs{pdfpageheight}
     internally.
\item[\onlypre luatex] sets \cs{pagewidth} and \cs{pageheight} internally.
\item[\onlypre xetex] is the same as |pdftex| except for ignoring
    |\pdf{h,v}origin| undefined in \XeLaTeX{}. This option is introduced in
    the version 5. Note that `geometry.cfg' in \TeX{} Live, which disables the
    auto-detection routine and sets |pdftex|, is no longer necessary,
    but has no problem even though it's left undeleted.
    Instead of |xetex|, you can specify |dvipdfm| with \XeLaTeX{}
    if you want to use specials of dvipdfm \XeTeX{} supports.
\item[\onlypre vtex] sets dimensions \cs{mediawidth} and \cs{mediaheight}
    for V\TeX. When this driver is selected (explicitly or
    automatically), \Gm\ will auto-detect which output mode
    (DVI, PDF or PS) is selected in V\TeX, and do proper
    settings for it.
\end{Options}
If explicit driver setting is mismatched with the typesetting program
in use, the default driver |dvips| would be selected.

\subsection{Other options}

 The other useful options are described here.

\begin{Options}
\item[\onlypre verbose] displays the parameter results on the terminal.
  |verbose=false| (default) still puts them into the log file.
\item[\onlypre reset] sets back the layout dimensions and switches to the
  settings before \Gm\ is loaded. Options given in 
  |geometry.cfg| are also cleared.
  Note that this cannot reset |pass| and |mag| with |truedimen|.
  |reset=false| has no effect and cannot cancel the previous
  |reset|(|=true|) if any. For example, when you go
  \begin{quote}
    |\documentclass[landscape]{article}|\\
    |\usepackage[twoside,reset,left=2cm]{geometry}|
  \end{quote}
  with |\ExecuteOptions{scale=0.9}| in |geometry.cfg|,
  then as a result, |landscape| and |left=2cm| remain effective,
  and |scale=0.9| and |twoside| are ineffective.
\item[\onlypre mag] sets magnification value (\cs{mag}) and automatically modifies 
  \cs{hoffset} and \cs{voffset} according to the magnification.
  |mag=|\meta{value}. Note that \meta{value} should be an integer value
  with 1000 as a normal size. For example, |mag=1414| with |a4paper|
  provides an enlarged print fitting in |a3paper|, which is $1.414$
  (=$\sqrt{2}$) times larger than |a4paper|. Font enlargement needs extra
  disk space. \textbf{Note that setting |mag| should precede any other
  settings with `true' dimensions, such as  |1.5truein|, |2truecm|
  and so on.} See also |truedimen| option.
\item[\onlypre truedimen] changes all internal explicit dimension values into 
  \textit{true} dimensions, e.g., |1in| is changed to |1truein|.
  Typically this option will be used together with |mag| option. Note that
  this is ineffective against externally specified dimensions. For example,
  when you set ``\texttt{mag=1440, margin=10pt, truedimen}'', margins are
  not `true' but magnified. If you want to set exact margins, you should
  set like ``\texttt{mag=1440, margin=10truept, truedimen}'' instead.
\item[\onlypre pass] disables all of the geometry options and calculations
  except |verbose| and |showframe|. It is order-independent and can be
  used for checking out the page layout of the documentclass, other
  packages and manual settings without \Gm.
\item[\onlypre showframe] shows visible frames for the text area and page,
  and the lines for the head and foot on the first page.
\item[\onlypre showcrop] prints crop marks at each corner of user-specified
layout area.
\end{Options}
