% \section{Preface to version 5\hfill 第5版前言}
%
% \begin{itemize}
    \columnratio{0.6}
    \begin{paracol}{2}
    
    %  \item \textbf{Changing page layout mid-document.}\par
    %   The new commands \cs{newgeometry\{$\cdots$\}} and
    %   \cs{restoregeometry} allow users to change page dimensions in the
    %   middle of the document. \cs{newgeometry} is almost similar to 
    %   \cs{geometry} except that \cs{newgeometry} disables all the options
    %   specified in the preamble and skips the papersize-related options:
    %    |landscape|, |portrait| and paper size options (such as
    %    |papersize|, |paper=a4paper| and so forth).
    \switchcolumn 
    %  \item \textbf{在文档中间更改页面布局。}\par
    % 新的命令 \cs{newgeometry{$\cdots$}} 和 \cs{restoregeometry} 允许用户在文档的中间更改页面尺寸。 \cs{newgeometry} 基本上与 \cs{geometry} 相似,只是 \cs{newgeometry} 禁用了在导言区指定的所有选项,并跳过与纸张尺寸相关的选项:|landscape|,|portrait| 和纸张大小选项(如 |papersize|,|paper=a4paper| 等)。
    
    \switchcolumn* 
    %  \item \textbf{A new set of options to specify the layout area.}\par
    %   The options specified for the area, in which the page dimensions
    %   are calculated, are added: \textsf{layout}, \textsf{layoutsize}, 
    %   \textsf{layoutwidth}, \textsf{layoutheight} and so forth.
    %   These options would help to print the specified
    %   layout to a different sized paper.  For example, with |a4paper|
    %   and |layout=a5paper|, the \Gm\ package uses `A5' layout to
    %   calculate margins with the paper size still `A4'.
    \switchcolumn
    % \item \textbf{一组新的选项来指定布局区域。}\par
    % 添加了用于计算页面尺寸的区域的选项:\textsf{layout},\textsf{layoutsize},\textsf{layoutwidth},\textsf{layoutheight} 等。这些选项可以帮助将指定的布局打印到不同大小的纸张上。例如,通过 |a4paper| 和 |layout=a5paper|,\Gm 宏包使用“A5”布局计算边距,但纸张尺寸仍然是“A4”。
    
    \switchcolumn*
    %  \item \textbf{A new driver option |xetex|.}\par
    %   The new driver option |xetex| is added. The driver auto-detection
    %   routine has been revised so as to avoid an error with undefined
    %   control sequences. Note that `geometry.cfg' in \TeX{} Live, which
    %   disables the auto-detection routine and sets |pdftex|, is no
    %   longer necessary and has no problem even though it still exists.
    %   To set |xetex| is strongly recommended with \XeLaTeX{}.
    \switchcolumn
    % \item \textbf{新的驱动选项 |xetex|。}\par
    % 添加了新的驱动选项 |xetex|。驱动程序自动检测程序已经进行了修订,以避免出现未定义控制序列的错误。请注意,在 \TeX{} Live 中禁用了自动检测程序并设置了 |pdftex| 的 `geometry.cfg' 不再需要,尽管它仍然存在也没有问题。强烈建议在使用 \XeLaTeX{} 时设置 |xetex|。
    
    \switchcolumn*
    %  \item \textbf{New paper size presets for JIS B-series and ISO C-series.}\par
    %   The papersize presets |b0j| to |b6j| for JIS (Japanese Industrial
    %   Standards) B-series and |c0paper| to |c6paper| for ISO C-series
    %   (v5.4$\sim$) are added.
    \switchcolumn
    % \item \textbf{JIS B 系列和 ISO C 系列的新纸张大小预设。}\par
    % 添加了 JIS(日本工业标准)B 系列的纸张预设 |b0j| 到 |b6j|,以及 ISO C 系列(v5.4$\sim$)的纸张预设 |c0paper| 到 |c6paper|。
    
    
    \switchcolumn*
    %  \item \textbf{Changing default for underspecified margin.}\par
    %   In the previous version, if only one margin was specified,
    %   |bottom=1cm| for example, then \Gm\ set the other margin with
    %   the margin ratio (1:1 by default for the vertical dimensions)
    %   and got |top=1cm| in this case.
    %   The version 5 sets the text-body size with the default |scale|
    %   ($=0.7$) and determine the unspecified margin. (See Section~\ref{sec:rules}) 
    \switchcolumn
    % \item \textbf{更改未指定边距的默认值。}\par
    % 在之前的版本中,若只指定单个边距,例如 |bottom=1cm|,那么 \Gm 会使用边距比例(对于垂直维度,默认为 1:1)设置另一个边距,并在这种情况下得到 |top=1cm|。第5版使用默认的 |scale|($=0.7$)设置文本区域的尺寸,并确定未指定的边距。(见第~\ref{sec:rules} 节)
    
    \switchcolumn*
    %  \item \textbf{The option |showframe| and |showcrop| works on every page.}\par
    %   With |showframe| option, the page frames are shown on every page.
    %   In addition, a new option |showcrop| prints crop marks at each
    %   corner of layout area on every page. Note that the marks would be
    %   invisible without specifying the layout size smaller than paper size.
    %   Version 5.4 introduced a new |\shipout| overloading process using
    %   \textsf{atbegshi} package, so the \textsf{atbegshi} package
    %   is required when showframe or showcrop option is specified.
    \switchcolumn
    % \item \textbf{选项 |showframe| 和 |showcrop| 在每页上都有效。}\par
    % 使用选项 |showframe|,页面框架将在每页上显示。此外,新选项 |showcrop| 在布局区域的每个角落打印裁剪标记。请注意,如果未指定小于纸张尺寸的布局尺寸,则这些标记将不可见。第5.4版引入了使用 \textsf{atbegshi} 宏包的新的 |\shipout| 过载过程,因此在指定 showframe 或 showcrop 选项时需要 \textsf{atbegshi} 宏包。
    
    \switchcolumn*
    %  \item \textbf{Loading geometry.cfg precedes processing class options.}\par
    %    The previous version loaded \textsf{geometry.cfg} after
    %    processing the document class options.  Now that the config file is
    %    loaded before processing the class options, you can change the
    %    behavior specified in \textsf{geometry.cfg} by adding options
    %    into |\documentclass| as well as |\usepackage| and |\geometry|.
    \switchcolumn
    % \item \textbf{加载 geometry.cfg 在处理类选项之前。}\par
    % 旧版本在处理文档类选项后加载 \textsf{geometry.cfg}。现在,在处理类选项之前加载配置文件,可在 |\documentclass|、|\usepackage| 和 |\geometry| 中添加选项来更改 \textsf{geometry.cfg} 中的行为。
    
    \switchcolumn*
    %  \item \textbf{Deleted options: |compat2| and |twosideshift|.}
    %    The version 5 has no longer compatibility with the previous ones.
    %    |compat2| and |twosideshift| are gone for simplicity. 
    \switchcolumn
    \item \textbf{删除的选项:|compat2| 和 |twosideshift|。}
    第5版不再与之前的版本兼容。为简单起见,删除了 |compat2| 和 |twosideshift|。
    \end{paracol}
    % \end{itemize}