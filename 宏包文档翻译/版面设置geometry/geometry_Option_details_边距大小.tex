
\columnratio{0.55}
\begin{paracol}{2}
\subsection{Margin size}\label{sec:margin}
The options specifying the size of the margins are listed below.
\switchcolumn
\subsection{边距大小}
下面列出了指定边距大小的选项。
\end{paracol}

\begin{Options}
\columnratio{0.55}
\begin{paracol}{2}
\switchcolumn
\switchcolumn[0]*\item[left\OR lmargin\OR inner]~\\
left margin (for oneside) or inner margin (for twoside) of 
\gpart{total body}. In other words, the distance between the left (inner)
edge of the paper and that of \gpart{total body}. |left=|\meta{length}.
|inner| has no special meaning, just an alias of |left| and |lmargin|.
\switchcolumn\item[left\OR lmargin\OR inner]~\\
正文的左边距(单页模式)或内部边距(双页模式)。换句话说,纸张左(内)边缘与正文左(内)边缘之间的距离。|left=|\meta{长度}。|inner| 没有特殊含义,只是 |left| 和 |lmargin| 的别名。
%%%
\switchcolumn[0]*\item[right\OR rmargin\OR outer]~\\ 
   right or outer margin of \gpart{total body}. |right=|\meta{length}.
\switchcolumn\item[right\OR rmargin\OR outer]~\\
正文的右边距或外部边距。|right=|\meta{长度}。
\switchcolumn[0]*\item[top\OR tmargin] top margin of the page. |top=|\meta{length}.
   Note this option has nothing to do with the native dimension
   \cs{topmargin}.
\switchcolumn\item[top\OR tmargin] 页面的上边距。|top=|\meta{长度}。
请注意,此选项与原始尺寸 \cs{topmargin} 无关。
\switchcolumn[0]*\item[bottom\OR bmargin]~\\ 
   bottom margin of the page. |bottom=|\meta{length}.
\switchcolumn\item[bottom\OR bmargin]~\\
页面的下边距。|bottom=|\meta{长度}。
\switchcolumn[0]*\item[hmargin] left and right margin.
  |hmargin=|\argii{left margin}{right margin} or |hmargin=|\meta{length}.
\switchcolumn\item[hmargin] 左边距和右边距。
|hmargin=|\argii{左边距}{右边距} 或 |hmargin=|\meta{长度}。
\switchcolumn[0]*\item[vmargin] top and bottom margin.
  |vmargin=|\argii{top margin}{bottom margin} or |vmargin=|\meta{length}.
\switchcolumn\item[vmargin] 上边距和下边距。
|vmargin=|\argii{上边距}{下边距} 或 |vmargin=|\meta{长度}。
\switchcolumn[0]*\item[margin] |margin=|\vargii{$A$}{$B$} is equivalent to 
  |hmargin=|\vargii{$A$}{$B$} and |vmargin=|\vargii{$A$}{$B$}.
  |margin=|$A$ is automatically expanded to |hmargin=|$A$ and |vmargin=|$A$.
\switchcolumn\item[margin] |margin=|\vargii{$A$}{$B$} 等同于
|hmargin=|\vargii{$A$}{$B$} 和 |vmargin=|\vargii{$A$}{$B$}。
|margin=|$A$ 会自动展开为 |hmargin=|$A$ 和 |vmargin=|$A$。
\switchcolumn[0]*\item[hmarginratio]
  horizontal margin ratio of |left| (inner) to |right| (outer). 
  The value of \meta{ratio} should be specified with colon-separated 
  two values. Each value should be a positive integer less than 100
  to prevent arithmetic overflow, e.g., |2:3| instead of |1:1.5|.
  The default ratio is |1:1| for oneside, |2:3| for twoside.
\switchcolumn\item[hmarginratio]
|left|(内)与 |right|(外)的水平边距比例。
\meta{比例} 的值应使用冒号分隔的两个值来指定。
每个值应为小于 100 的正整数,以防止算术溢出,例如使用 |2:3| 而不是 |1:1.5|。
对于单页模式,默认比例是 |1:1|,对于双页模式,默认比例是 |2:3|。
\switchcolumn[0]*\item[vmarginratio]
   vertical margin ratio of |top| to |bottom|. The default ratio is |2:3|.
\switchcolumn\item[vmarginratio]
|top| 与 |bottom| 的垂直边距比例。默认比例是 |2:3|。
\switchcolumn[0]*\item[marginratio\OR ratio]~\\
   horizontal and vertical margin ratios.
  |marginratio=|\argii{horizontal ratio}{vertical ratio} or
  |marginratio=|\meta{ratio}.
\switchcolumn\item[marginratio\OR ratio]\
水平和垂直边距比例。
|marginratio=|\argii{水平比例}{垂直比例} 或
|marginratio=|\meta{比例}。
\switchcolumn[0]*\item[hcentering] sets auto-centering horizontally and is
  equivalent to |hmarginratio=1:1|. It is set to |true| by default for
  oneside. See also |hmarginratio|.
\switchcolumn\item[hcentering] 设置水平自动居中,等同于 |hmarginratio=1:1|。默认情况下,单页模式下设置为 |true|。见 |hmarginratio|。
\switchcolumn[0]*\item[vcentering] sets auto-centering vertically and is
  equivalent to |vmarginratio=1:1|. The default is |false|.
  See also |vmarginratio|.
\switchcolumn\item[vcentering] 设置垂直自动居中,等同于 |vmarginratio=1:1|。默认情况下为 |false|。见 |vmarginratio|。
\switchcolumn[0]*\item[centering] sets auto-centering and is equivalent to
  |marginratio=1:1|. See also |marginratio|. The default is |false|.
  See also |marginratio|.
\switchcolumn\item[centering] 设置自动居中,等同于 |marginratio=1:1|。默认情况下为 |false|。见 |marginratio|。
\switchcolumn[0]*\item[twoside] switches on twoside mode with left and right margins swapped
  on verso pages. The option sets \cs{@twoside} and \cs{@mparswitch} 
  switches. See also |asymmetric|.
\switchcolumn\item[twoside] 打开双面模式,交换背面页面的左右边距。此选项设置 \cs{@twoside} 和 \cs{@mparswitch} 开关。见 |asymmetric|。
\switchcolumn[0]*\item[asymmetric] implements a twosided layout in which margins are
  not swapped on alternate pages (by setting \cs{oddsidemargin} to 
  \cs{evensidemargin} |+| |bindingoffset|) and in which the marginal notes
  stay always on the same side. This option can be used as an alternative
  to the twoside option. See also |twoside|.
\switchcolumn\item[asymmetric] 实现了一个双面布局,其中边距在交替页面上不会交换(通过将 \cs{oddsidemargin} 设置为 \cs{evensidemargin} |+| |bindingoffset|),并且边注始终停留在同一侧。此选项可用作 twoside 选项的替代方案。见 |twoside|。
\switchcolumn[0]*\item[bindingoffset]~\\ removes a specified space 
  from the lefthand-side of the page for oneside or the inner-side for
  twoside. |bindingoffset=|\meta{length}. This is useful if pages 
  are bound by a press binding (glued, stitched, stapled \ldots).
  See Figure~\ref{fig:bindingoffset}.
\switchcolumn\item[bindingoffset]~\\ 从页面的左侧(单页模式或双页模式的内部侧)移除指定的空间。|bindingoffset=|\meta{长度}。如果页面通过压合装订(胶合、缝合、订书钉等)装订,则这很有用。见图\ref{fig:bindingoffset}。

\switchcolumn[0]*\item[hdivide] See description in Section~\ref{sec:body}.
\switchcolumn\item[hdivide] 见第~\ref{sec:body} 节中的描述。
\switchcolumn[0]*\item[vdivide] See description in Section~\ref{sec:body}.
\switchcolumn\item[vdivide] 见第~\ref{sec:body} 节中的描述。
\switchcolumn[0]*\item[divide] See description in Section~\ref{sec:body}.
\switchcolumn\item[divide] 见第~\ref{sec:body} 节中的描述。
\end{paracol}
\end{Options}




\begin{figure}
 \centering\small
 {\unitlength=.65pt
 \begin{picture}(500,270)(0,0)
 \put(20,0){\framebox(170,230){}}
 \put(20,255){\makebox(80,20)[l]{\textbf{a)}~every page for oneside or}}
 \put(20,240){\makebox(80,20)[l]{\hspace{3ex}odd pages for twoside}}
 \put(110,225){\makebox(80,20)[r]{\gpart{paper}}}
 \put(55,37){\framebox(110,170)[tc]{\gpart{total body}}}
 \multiput(38,0)(0,7){33}{\line(0,1){4}}
 \put(38,100){\vector(1,0){17}}\put(55,100){\vector(-1,0){17}}
 \put(60,95){\makebox(80,10)[l]{|left|}}
 \put(60,80){\makebox(80,10)[l]{(|inner|)}}
 \put(165,100){\vector(1,0){25}}\put(190,100){\vector(-1,0){25}}
 \put(195,95){\makebox(80,10)[l]{|right|}}
 \put(195,80){\makebox(80,10)[l]{(|outer|)}}
 \put(20,16){\vector(1,0){18}}
 \put(45,10){\makebox(80,10)[bl]{|bindingoffset|}}
 \put(280,255){\makebox(80,20)[l]{\textbf{b)}~even (back) pages for twoside}}
 \put(280,0){\framebox(170,230){}}
 \put(370,225){\makebox(80,20)[r]{\gpart{paper}}}
 \put(305,37){\framebox(110,170)[tc]{\gpart{total body}}}
 \multiput(432,0)(0,7){33}{\line(0,1){4}}
 \put(280,100){\vector(1,0){25}}\put(305,100){\vector(-1,0){25}}
 \put(310,95){\makebox(80,10)[l]{|outer|}}
 \put(310,80){\makebox(80,10)[l]{(|right|)}}
 \put(415,100){\vector(1,0){17}}\put(432,100){\vector(-1,0){17}}
 \put(373,95){\makebox(80,10)[l]{|inner|}}
 \put(373,80){\makebox(80,10)[l]{(|left|)}}
 \put(450,16){\vector(-1,0){18}}
 \put(330,10){\makebox(80,10)[bl]{|bindingoffset|}}
 \end{picture}}
 \caption[\texttt{bindingoffset} option]{%
  \begin{minipage}[t]{.8\textwidth}\raggedright\small
  The option |bindingoffset| adds the specified length to the inner margin.
  Note that |twoside| option swaps the horizontal margins and the
  marginal notes together with |bindingoffset| on even pages (see
  \textbf{b)}), but |asymmetric| option suppresses the swap of the
  margins and marginal notes (but |bindingoffset| is still swapped).\\
  选项 |bindingoffset| 将指定的长度添加到内边距。请注意,|twoside| 选项会在偶数页上与 |bindingoffset| 一起交换水平边距和边注(参见 \textbf{b)}),但 |asymmetric| 选项会取消交换边距和边注(但 |bindingoffset| 仍然会被交换)。
  \end{minipage}}
 \label{fig:bindingoffset}
\end{figure}
