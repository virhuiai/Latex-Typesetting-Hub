\ProvidesFile{geometry.drv}
\documentclass{ltxdoc}
\usepackage[heading=true
,scheme=chinese%中文方案
,fontset=none%不使用默认的字体设置
,space=auto%自动调整中英文间距
]{ctex}
\setCJKmainfont{FangZhengShuSong-GBK-1.ttf}[Path=/Users/virhuiai/hlProjects/Latex-Typesetting-Hub/font/方正/]%设置文本的中文有衬线字体
\setCJKsansfont{FangZhengHeiTi-GBK-1.ttf}[Path=/Users/virhuiai/hlProjects/Latex-Typesetting-Hub/font/方正/]%设置文本的中文无衬线字体为
\setCJKmonofont{FangZhengFangSong-GBK-1.ttf}[Path=/Users/virhuiai/hlProjects/Latex-Typesetting-Hub/font/方正/] %设置文本的中文等宽字体 
% \setCJKfamilyfont{fontKai}{LXGWWenKai-Regular.ttf}[Path=/Users/virhuiai/hlProjects/Latex-Typesetting-Hub/font/霞鹜文楷/]
\setCJKfamilyfont{fontKai}{FangZhengKaiTi-GBK-1.ttf}[Path=/Users/virhuiai/hlProjects/Latex-Typesetting-Hub/font/方正/]
\newcommand\fontKai{\CJKfamily{fontKai}}
\usepackage[colorlinks, linkcolor=blue]{hyperref}
\usepackage{graphicx}
\usepackage[a3paper, hmargin=2.5cm, vmargin=1cm, 
        includeheadfoot,landscape]{geometry}
\DeclareRobustCommand\XeTeX{%
      X\lower.5ex\hbox{\kern-.07em\reflectbox{E}}%
      \kern-.15em\TeX}
\DeclareRobustCommand\XeLaTeX{%
      X\lower.5ex\hbox{\kern-.07em\reflectbox{E}}%
      \kern-.15em\LaTeX}
\usepackage[all]{tcolorbox}
\usepackage{paracol}
\begin{document}
\parindent=0pt

\MakeShortVerb{|}

\def\OpenB{{\ttfamily\char`\{}}
\def\Comma{{\ttfamily\char`,}}
\def\CloseB{{\ttfamily\char`\}}}
\def\Gm{\textsf{geometry}}
\newcommand\argii[2]{\OpenB\meta{#1}\Comma\meta{#2}\CloseB}
\newcommand\argiii[3]{\OpenB\meta{#1}\Comma\meta{#2}\Comma\meta{#3}\CloseB}
\newcommand\vargii[2]{\OpenB#1\Comma#2\CloseB}
\newcommand\vargiii[3]{\OpenB#1\Comma#2\Comma#3\CloseB}
\newcommand\OR{\ \strut\vrule width .4pt\ }
\newcommand\gpart[1]{\textsf{\textsl{\color[rgb]{.0,.45,.7}#1}}}%
\newcommand\glen[1]{\textsf{#1}}
\newenvironment{key}[2]{\expandafter\macro\expandafter{`#2'}}{\endmacro}
\newenvironment{Options}%
 {\begin{list}{}{%
  \renewcommand{\makelabel}[1]{\texttt{##1}\hfil}%
  \setlength{\itemsep}{-.5\parsep}
  \settowidth{\labelwidth}{\texttt{xxxxxxxxxxx\space}}%
  \setlength{\leftmargin}{\labelwidth}%
  \addtolength{\leftmargin}{\labelsep}}%
  \raggedright}
 {\end{list}}
\newenvironment{Spec}%
 {\begin{list}{}{%
  \renewcommand{\makelabel}[1]{\fbox{##1}\hfil}%
  \setlength{\itemsep}{-.5\parsep}
  \settowidth{\labelwidth}{\texttt{S(x,x)}}%
  \setlength{\leftmargin}{\labelwidth}%
  \addtolength{\leftmargin}{\labelsep}%
  \addtolength{\leftmargin}{2em}%
  \setlength{\rightmargin}{2em}}%
  \raggedright}
 {\end{list}}
\def\Ss(#1,#2){\textsf{S(#1,#2)}}%
\def\Sp(#1,#2,#3){\mbox{|(#1,#2,#3)|}}%
\def\onlypre{\llap{$^{\dagger\:}$}}%

% 
\title{The \textsf{geometry} package}
\date{\filedate\ \fileversion}
\author{Hideo Umeki\\\url{https://github.com/davidcarlisle/geometry} \and 翻译 virhuiai@qq.com\\\url{https://github.com/virhuiai/Latex-Typesetting-Hub/tree/main/宏包文档翻译/版面设置geometry}}

\maketitle
%  

 \section{Introduction\hfill 导言}

\columnratio{0.6}
\begin{paracol}{2}
 To set dimensions for page layout in \LaTeX\ is not straightforward. 
 You need to adjust several \LaTeX{} native dimensions to place a text area
 where you want.
 If you want to center the text area in the paper you use, for example, 
 you have to specify native dimensions as follows:
\switchcolumn 
在 {\LaTeX} 中设置页面布局并不简单。
你需要调整几个 {\LaTeX} 的内部尺寸,以将文本区域放置在你想要的位置。
例如,如果你想要在纸张上居中显示文本区域,你需要指定以下内部尺寸:
\switchcolumn
 \begin{quote}
    |\usepackage{calc}|\\
    |\setlength\textwidth{7in}|\\
    |\setlength\textheight{10in}|\\
    |\setlength\oddsidemargin{(\paperwidth-\textwidth)/2 - 1in}|\\
    |\setlength\topmargin{(\paperheight-\textheight|\\
    |                      -\headheight-\headsep-\footskip)/2 - 1in}|.
 \end{quote}
\switchcolumn
\begin{minted}{latex}
\usepackage{calc}
\setlength\textwidth{7in}
\setlength\textheight{10in}
\setlength\oddsidemargin{(\paperwidth-\textwidth)/2 - 1in}
\setlength\topmargin{(\paperheight-\textheight-\headheight-\headsep-\footskip)/2 - 1in}
\end{minted}
\end{paracol}




 Without package \textsl{calc}, the above example would need
 more tedious settings. Package \Gm\ provides an easy
 way to set page layout parameters. In this case, what you have to do
 is just
 \begin{quote}
    |\usepackage[text={7in,10in},centering]{geometry}|. 
 \end{quote}
 Besides centering problem, setting margins from each edge of the paper is
 also troublesome. But \Gm\ also make it easy.
 If you want to set each margin to 1.5in, you can type
 \begin{quote}
    |\usepackage[margin=1.5in]{geometry}| 
 \end{quote}
 Thus, the geometry package has an auto-completion mechanism, in which
 unspecified dimensions are automatically determined.
 The \Gm\ package will be also useful when you have to set page layout
 obeying the following strict instructions: for example,
 \begin{quote}\slshape
   The total allowable width of the text area is $6.5$ inches wide by $8.75$
   inches high. The top margin on each page should be $1.2$ inches from
   the top edge of the page. The left margin should be $0.9$ inch from 
   the left edge. The footer with page number should be at the bottom
   of the text area.
 \end{quote}
 In this case, using \Gm\ you can type
 \begin{quote}
 |\usepackage[total={6.5in,8.75in},|\\
 |            top=1.2in, left=0.9in, includefoot]{geometry}|.
 \end{quote}

 Setting a text area on the paper in document preparation system has some
 analogy to placing a window on the background in the window system. 
 The name `geometry' comes from the |-geometry| option used for specifying
 a size and location of a window in X Window System.

% \section{Page geometry\hfill 页面布局}

\columnratio{0.55}
\begin{paracol}{2}
Figure~\ref{fig:layout} shows the page layout dimensions defined
in the \Gm\ package.
The page layout contains a \gpart{total body} (printable area) and \gpart{margins}. 
The \gpart{total body} consists of a \gpart{body} (text area) with an
optional \gpart{header}, \gpart{footer} and marginal notes (marginpar).
There are four margins: \gpart{left}, \gpart{right}, \gpart{top} and
\gpart{bottom}. For twosided documents, horizontal
margins should be called \gpart{inner} and \gpart{outer}.
\switchcolumn
图~\ref{fig:layout}展示了\Gm\ 宏包中定义的页面布局尺寸。
页面布局包括一个\gpart{总体正文}(可打印区域)和\gpart{页边距}。
\gpart{总体正文}包括一个\gpart{正文}(文本区域),还可以有可选的\gpart{页眉}、\gpart{页脚}和边注(marginpar)。
有四个边距:\gpart{左边距}、\gpart{右边距}、\gpart{顶边距}和\gpart{底边距}。对于双面文档,水平的边距应称为\gpart{内边距}和\gpart{外边距}。
\switchcolumn[0]*
\begin{quote}
\begin{tabular}{rcl}
\gpart{paper}&:&\gpart{total body} and
\gpart{margins}\\
\gpart{total body}&:&\gpart{body} (text area)\quad
(optional \gpart{head}, \gpart{foot} and \gpart{marginpar})\\
\gpart{margins}&:&\gpart{left} (\gpart{inner}), 
\gpart{right} (\gpart{outer}), \gpart{top} and \gpart{bottom}
\end{tabular}
\end{quote}
\switchcolumn
\begin{quote}
\begin{tabular}{rcl}
\gpart{纸张} & : & \gpart{总体正文}和\gpart{页边距} \\
\gpart{总体正文} & : & \gpart{正文}(文本区域)\quad
(可选的\gpart{页眉}、\gpart{页脚}和\gpart{边注}) \\
\gpart{页边距} & : & \gpart{左边距}(\gpart{内边距})、
\gpart{右边距}(\gpart{外边距})、\gpart{顶边距}和\gpart{底边距}
\end{tabular}
\end{quote}
%%%%%%%%%
\switchcolumn[0]*
Each margin is measured from the corresponding edge of a paper. 
For example, left margin (inner margin) means a horizontal distance
between the left (inner) edge of the paper and that of the total body.
Therefore the left and top margins defined in \Gm\ are different
from the native dimensions \cs{leftmargin} and \cs{topmargin}.
The size of a body (text area) can be modified by \cs{textwidth} and
\cs{textheight}. 
The dimensions for paper, total body and margins have the following
relations.
\switchcolumn
每个边距都是从纸张的相应边缘测量的。例如,左边距(内边距)表示纸张的左(内)边缘与总体正文的左(内)边缘之间的水平距离。因此,在\Gm\ 中定义的左边距和顶边距与原生尺寸\cs{leftmargin}和\cs{topmargin}是不同的。正文(文本区域)的大小可以通过\cs{textwidth}和\cs{textheight}进行修改。纸张、总体正文和边距的尺寸具有以下关系。
\switchcolumn[0]*
\begin{eqnarray}
 \label{eq:paperwidth}
 |paperwidth| &=& |left|+|width|+|right| \\
 |paperheight| &=& |top|+|height|+|bottom|
 \label{eq:paperheight}
\end{eqnarray}
\switchcolumn
\begin{eqnarray}
\label{eq:paperwidth}
|纸张宽度| &=& |左边距|+|正文宽度|+|右边距| \\
|纸张高度| &=& |顶边距|+|正文高度|+|底边距|
\label{eq:paperheight}
\end{eqnarray}


%%%%%%%%%

%%%%%%%%%
\end{paracol}


    
\begin{figure}
 \centering\small
 {\unitlength=.65pt
 \begin{picture}(450,250)(0,-10)
 \put(20,0){\framebox(170,230){}}
 \put(20,235){\makebox(170,230)[br]{\gpart{paper}}}
 \begingroup\thicklines
 \put(40,30){\framebox(120,170){}}\endgroup
 \put(40,30){\makebox(120,165)[tr]{\gpart{total body}~}}
 \put(45,30){\makebox(0,170)[l]{|height|}}
 \put(40,35){\makebox(120,0)[bc]{|width|}}
 \put(50,-20){\makebox(120,0)[bc]{|paperwidth|}}
 \put(10,45){\makebox(0,170)[r]{|paperheight|}}
 \put(90,200){\makebox(0,30)[lc]{|top|}}
 \put(90,0){\makebox(0,30)[lc]{|bottom|}}
 \put(10,70){\makebox(0,0)[r]{|left|}}
 \put(10,55){\makebox(0,0)[r]{(|inner|)}}
 \put(200,70){\makebox(0,0)[l]{|right|}}
 \put(200,55){\makebox(0,0)[l]{(|outer|)}}
 \put(80,230){\vector(0,-1){30}}\put(80,30){\vector(0,-1){30}}
 \put(80,200){\vector(0,1){30}}\put(80,0){\vector(0,1){30}}
 \put(20,70){\vector(1,0){20}}\put(40,70){\vector(-1,0){20}}
 \put(160,70){\vector(1,0){30}}\put(190,70){\vector(-1,0){30}}
 \multiput(160,30)(5,0){24}{\line(1,0){2}}
 \multiput(160,200)(5,0){24}{\line(1,0){2}}
 \begingroup\thicklines
 \put(280,30){\framebox(120,170){}}\endgroup
 \put(283,133){\makebox(0,12)[l]{|textheight|}}
 \put(295,130){\vector(0,-1){100}}\put(295,150){\vector(0,1){50}}
 \multiput(280,220)(5,0){24}{\line(1,0){3}}
 \put(280,208){\makebox(120,20)[bc]{\gpart{head}}}
 \multiput(280,207)(5,0){24}{\line(1,0){3}}
 \put(420,225){\makebox(0,0)[l]{|headheight|}}
 \put(418,225){\line(-2,-1){20}}
 \put(420,213){\makebox(0,0)[l]{|headsep|}}
 \put(418,213){\line(-2,-1){20}}
 \put(420,12){\makebox(0,0)[l]{|footskip|}}
 \put(418,12){\line(-2,1){20}}
 \put(280,40){\makebox(120,140)[c]{\gpart{body}}}
 \put(305,45){\vector(-1,0){25}}\put(375,45){\vector(1,0){25}}
 \put(80,230){\vector(0,-1){30}}\put(80,30){\vector(0,-1){30}}
 \put(280,48){\makebox(120,0)[c]{|textwidth|}}
 \put(280,15){\makebox(120,10)[c]{\gpart{foot}}}
 \multiput(280,14)(5,0){24}{\line(1,0){2}}
 \put(410,30){\dashbox{3}(30,170){}}
 \put(415,30){\makebox(30,170)[l]{\gpart{marginal note}}}
 \put(425,45){\vector(-1,0){15}}\put(425,45){\vector(1,0){15}}
 \put(450,70){\makebox(0,0)[l]{|marginparsep|}}
 \put(448,70){\line(-3,-1){43}}
 \put(450,45){\makebox(0,0)[l]{|marginparwidth|}}
 \end{picture}}
 \caption[Dimension names for \Gm]{%
 \begin{minipage}[t]{.8\textwidth}\raggedright\small
 Dimension names used in the \Gm\ package.
 |width| $=$ |textwidth| and |height| $=$ |textheight| by default.
 |left|, |right|, |top| and |bottom| are margins. 
 If margins on verso pages are swapped by |twoside| option,
 margins specified by |left| and |right| options
 are used for the inside and outside margins respectively.
 |inner| and |outer| are aliases of |left| and |right|
 respectively.\\
 \Gm\ 宏包中使用的尺寸名称如下:
默认情况下,|width| $=$ |textwidth| 和 |height| $=$ |textheight|。
|left|、|right|、|top| 和 |bottom| 是边距。
如果通过|twoside|选项交换了底稿页的边距,
那么通过|left|和|right|选项指定的边距将分别用于内边距和外边距。
|inner| 和 |outer| 分别是 |left| 和 |right| 的别名。
 \end{minipage}}
 \label{fig:layout}
\end{figure}

\columnratio{0.55}
\begin{paracol}{2}
The total body |width| and |height| would be defined:
\begin{eqnarray}
\label{eq:width}
|width| &:=& |textwidth| \quad( +\>  |marginparsep| + |marginparwidth| )\\
|height| &:=& |textheight| \quad(+\> |headheight| + |headsep| + |footskip| )
\label{eq:height}
\end{eqnarray}
\switchcolumn
总体正文的宽度和高度定义如下:
\begin{eqnarray}
\label{eq:width}
|宽度| &:=& |正文宽度| \quad( +> |边注间距| + |边注宽度| )\\
|高度| &:=& |正文高度| \quad(+> |页眉高度| + |页眉与正文的距离| + |页脚与正文的距离| )
\label{eq:height}
\end{eqnarray}

\switchcolumn[0]*
In Equation (\ref{eq:width}) |width:=textwidth| by default, 
while |marginparsep| and |marginparwidth| are included in |width|
if |includemp| option is set |true|. 
In Equation (\ref{eq:height}), |height:=textheight| by default. 
If |includehead| is set to |true|, |headheight| and |headsep| are
considered as a part of |height|.
In the same way, |includefoot| takes |footskip| into |height|. 
Figure~\ref{fig:includes} shows how these options
work in the vertical direction.
\switchcolumn
在方程(\ref{eq:width})中,默认情况下,|width:=textwidth|,而如果将|includemp|选项设置为|true|,则|marginparsep|和|marginparwidth|将包括在|width|中。
在方程(\ref{eq:height})中,默认情况下,|height:=textheight|。
如果将|includehead|设置为|true|,则将考虑|headheight|和|headsep|作为|height|的一部分。
同样,|includefoot|将|footskip|包括在|height|中。
图~\ref{fig:includes}展示了这些选项在垂直方向上的工作方式。
\end{paracol}

\begin{figure}
 \centering\small
 {\unitlength=.65pt
 \begin{picture}(490,280)(0,-10)
 \put(60,250){\makebox(120,0)[bl]{\textbf{(a)}~\textit{default}}}%
 \put(20,0){\framebox(170,230){}}
 \put(20,230){\makebox(170,15)[r]{\gpart{paper}}}
 \begingroup\thicklines
 \put(40,30){\framebox(120,165){}}\endgroup
 \put(70,165){\vector(0,1){30}}
 \put(55,145){\makebox(0,20)[lc]{|textheight|}}
 \put(70,145){\vector(0,-1){115}}
 \multiput(40,203)(5,0){24}{\line(1,0){3}}
 \multiput(40,213)(5,0){24}{\line(1,0){3}}
 \multiput(40,10)(5,0){24}{\line(1,0){3}}
 \put(40,203){\makebox(120,20)[bc]{\gpart{head}}}
 \put(40,40){\makebox(120,140)[c]{\gpart{body}}}
 \put(40,11){\makebox(120,10)[c]{\gpart{foot}}}
 \put(150,230){\vector(0,-1){35}}\put(150,30){\vector(0,-1){30}}
 \put(150,195){\vector(0,1){35}}\put(150,0){\vector(0,1){30}}
 \put(160,197){\makebox(0,30)[lc]{|top|}}
 \put(160,0){\makebox(0,30)[lc]{|bottom|}}
 \multiput(160,30)(5,0){24}{\line(1,0){2}}
 \multiput(160,195)(5,0){24}{\line(1,0){2}}
 \put(255,250){\makebox(120,0)[bl]
     {\textbf{(b)}~|includehead| and |includefoot|}}%
 \put(260,0){\framebox(170,230){}}
 \put(260,230){\makebox(170,15)[r]{\gpart{paper}}}
 \begingroup\thicklines
 \put(280,30){\framebox(120,165){}}\endgroup
 \put(310,152){\vector(0,1){25}}
 \put(295,130){\makebox(0,20)[lc]{|textheight|}}
 \put(310,130){\vector(0,-1){80}}
 \multiput(280,184)(5,0){24}{\line(1,0){3}}
 \multiput(280,177)(5,0){24}{\line(1,0){3}}
 \multiput(280,50)(5,0){24}{\line(1,0){3}}
 \put(280,184){\makebox(120,10)[c]{\gpart{head}}}
 \put(280,40){\makebox(120,140)[c]{\gpart{body}}}
 \put(400,140){\line(1,1){45}}
 \put(437,187){\makebox(50,10)[l]{\gpart{total body}}}
 \put(280,31){\makebox(120,10)[c]{\gpart{foot}}}
 \put(370,230){\vector(0,-1){35}}\put(370,30){\vector(0,-1){30}}
 \put(370,195){\vector(0,1){35}}\put(370,0){\vector(0,1){30}}
 \put(380,197){\makebox(0,30)[lc]{|top|}}
 \put(380,0){\makebox(0,30)[lc]{|bottom|}}
 \end{picture}}
 \caption[An effect of \texttt{includehead} and \texttt{includefoot}.]{%
 \begin{minipage}[t]{.8\textwidth}\raggedright\small
   |includehead| and |includefoot| include the head and foot respectively
   into \gpart{total body}. \textbf{(a)} |height| $=$ |textheight| (default).
   \textbf{(b)} |height| $=$ |textheight| $+$ |headheight| $+$ |headsep| $+$ 
   |footskip| if |includehead| and |includefoot|. If the top and bottom
   margins are specified, |includehead| and |includefoot| result in
   shorter |textheight|.\\
   |includehead|和|includefoot|分别将页眉和页脚包括在\gpart{总体正文}中。\textbf{(a)} |height| $=$ |textheight|(默认情况下)。\textbf{(b)} 如果设置了|includehead|和|includefoot|,则|height| $=$ |textheight| $+$ |headheight| $+$ |headsep| $+$ |footskip|。如果指定了顶部和底部边距,则|includehead|和|includefoot|会导致较短的|textheight|。
 \end{minipage}}
 \label{fig:includes}
\end{figure}

\columnratio{0.55}
\begin{paracol}{2}
Thus, the page layout consists of three parts (lengths) in each
direction: one body and two margins. If the two of them are explicitly
specified, the other length is obvious and no need to be specified.
Figure~\ref{fig:Labc} shows a simple model of page dimensions. 
When a length |L| is given and is partitioned into the body |b|, the
margins |a| and |c|, it's obvious that
\begin{equation}
    |L|=|a|+|b|+|c|  \label{eq:Labc}
\end{equation}
\switchcolumn
因此,页面布局在每个方向上由三个部分(长度)组成:一个正文和两个边距。如果其中两个长度被明确指定,另一个长度是显而易见的,无需指定。图~\ref{fig:Labc}展示了页面尺寸的简单模型。当给定一个长度 |L|,并将其分成正文 |b|、边距 |a| 和 |c| 时,很显然有以下关系:
\switchcolumn[0]*
The specification with two of the three (|a|,|b| and |c|) fixed
explicitly is solvable. If two or more are left unspecified
or `underspecified', Equation~(\ref{eq:Labc}) cannot be solved
without any other relation between them. If all of them are
specified, then it needs to check whether or not they
satisfy Equation~(\ref{eq:Labc}), that is too much specification or
`overspecified'.
\switchcolumn
如果明确指定了三个中的两个(|a|、|b|和|c|),则可以解决规范问题。如果有两个或更多未指定或“不完全规定”,则在没有其他关系的情况下,方程~(\ref{eq:Labc}) 无法解决。如果全部都被指定,那么就需要检查它们是否满足方程~(\ref{eq:Labc}),即是否过度规定或“超规定”。   
\end{paracol}


\begin{figure}
 \centering
 {\unitlength=0.8pt
 \begin{picture}(300,60)(0,-5)
 \begingroup\linethickness{5pt}
 \put(0,5){\textcolor{green}{\line(1,0){60}}}
 \put(60,5){\textcolor{black}{\line(1,0){160}}}
 \put(220,5){\textcolor{green}{\line(1,0){80}}}
 \endgroup
 \put(0,15){\makebox(60,10)[b]{|a|}}
 \put(60,0){\line(0,1){20}}
 \put(60,15){\makebox(160,10)[b]{|b|}}
 \put(220,0){\line(0,1){20}}
 \put(220,15){\makebox(80,10)[b]{|c|}}
 \put(0,0){\line(0,1){50}}
 \put(150,35){\vector(-1,0){150}}
 \put(0,40){\makebox(300,10){|L|}}
 \put(150,35){\vector(1,0){150}}
 \put(300,0){\line(0,1){50}}
 \end{picture}}
 \caption{A simple model of page dimensions.}
 \label{fig:Labc}
\end{figure}

The \Gm\ package has auto-completion mechanism that saves the
trouble of specifying the page layout dimensions. For example,
you can set
\begin{quote}
 |\usepackage[width=14cm, left=3cm]{geometry}|
\end{quote}
on A4 paper. In this case you don't have to set the right margin
The details of auto-completion will be described in
Section~\ref{sec:rules}.%todo 再看下
% \section{User interface\hfill 用户接口}

\subsection{Commands\hfill 命令}

\columnratio{0.55}
\begin{paracol}{2}
The \Gm\ package provides the following commands:
\switchcolumn
\Gm\ 宏包提供以下命令:
\end{paracol}

\begin{itemize}\setlength{\itemsep}{-.5\parsep}
\columnratio{0.55}
\begin{paracol}{2}
\item |\geometry{|\meta{options}|}|
\switchcolumn
\item |\geometry{|\meta{options}|}|

\switchcolumn
\item |\newgeometry{|\meta{options}|}| and |\restoregeometry|
\switchcolumn
\item |\newgeometry{|\meta{options}|}| 和 |\restoregeometry|

\switchcolumn
\item |\savegeometry{|\meta{name}|}| and |\loadgeometry{|\meta{name}|}|
\switchcolumn
\item |\savegeometry{|\meta{name}|}| 和 |\loadgeometry{|\meta{name}|}|
\end{paracol}
\end{itemize}

% \columnratio{0.55}
% \begin{paracol}{2}
% The \Gm\ package provides the following commands:
% \begin{itemize}\setlength{\itemsep}{-.5\parsep}
% \item |\geometry{|\meta{options}|}|
% \item |\newgeometry{|\meta{options}|}| and |\restoregeometry|
% \item |\savegeometry{|\meta{name}|}| and |\loadgeometry{|\meta{name}|}|
% \end{itemize}
% \switchcolumn
% \Gm\ 宏包提供以下命令:
% \begin{itemize}\setlength{\itemsep}{-.5\parsep}
% \item |\geometry{|\meta{options}|}|
% \item |\newgeometry{|\meta{options}|}| 和 |\restoregeometry|
% \item |\savegeometry{|\meta{name}|}| 和 |\loadgeometry{|\meta{name}|}|
% \end{itemize}
% \end{paracol}

\columnratio{0.55}
\begin{paracol}{2}
|\geometry{|\meta{options}|}|
changes the page layout according to the options specified in the
argument. This command, if any, should be placed only in the
preamble (before |\begin{document}|).
\switchcolumn
|\geometry{|\meta{options}|}|根据参数中指定的选项更改页面布局。如果有的话,此命令应该只放在导言区(|\begin{document}|之前)。
%%%
\switchcolumn
The \Gm\ package may be used as part of a class or another package
you use in your document. The command \cs{geometry} can overwrite
some of the settings in the preamble. Multiple use of \cs{geometry}
is allowed and then processed with the options concatenated.
If \Gm\ is not yet loaded, you can use only
|\usepackage[|\meta{options}|]{geometry}| instead of \cs{geometry}.
\switchcolumn
\Gm\ 宏包可以作为文档中使用的类或其他宏包的一部分。命令\cs{geometry}可以覆盖导言区中的一些设置。可以多次使用\cs{geometry}命令,并将选项连接在一起进行处理。如果\Gm\ 尚未加载,可以使用|\usepackage[|\meta{options}|]{geometry}|代替\cs{geometry}。    
\switchcolumn
\medskip
|\newgeometry{|\meta{options}|}|
changes the page layout mid-document. \cs{newgeometry} is almost
similar to \cs{geometry} except that \cs{newgeometry} disables all
the options specified by \cs{usepackage} and \cs{geometry} in
the preamble and skips papersize-related options. 
\cs{restoregeometry}
restores the page layout specified in the preamble. This command
has no arguments. See Section~\ref{sec:midchange} for details.
\switchcolumn
\medskip
|\newgeometry{|\meta{options}|}|可以在文档中更改页面布局。\cs{newgeometry}与\cs{geometry}几乎相同,除了会禁用导言区中由\cs{usepackage}和\cs{geometry}指定的所有选项,并且跳过与页面尺寸相关的选项外。|\restoregeometry|命令将恢复导言区中指定的页面布局。此命令没有参数。详见第~\ref{sec:midchange} 节。
\switchcolumn[0]*
\medskip
|\savegeometry{|\meta{name}|}|
saves the page dimensions as \meta{name} where you put
this command.
|\loadgeometry{|\meta{name}|}|
loads the page dimensions saved as \meta{name}. See
Section~\ref{sec:midchange} for details.
\switchcolumn
\medskip
|\savegeometry{|\meta{name}|}|将页面尺寸保存为\meta{name}。
|\loadgeometry{|\meta{name}|}|加载保存为\meta{name}的页面尺寸。详见第~\ref{sec:midchange} 节。
\end{paracol}


\subsection{Optional argument\hfill 可选参数}

\columnratio{0.55}
\begin{paracol}{2}
The \Gm\ package adopts \textsf{keyval} interface
`\meta{key}=\meta{value}' for the optional argument to
\cs{usepackage}, \cs{geometry} and \cs{newgeometry}.
\switchcolumn
\Gm\ 宏包在\cs{usepackage}、\cs{geometry}和\cs{newgeometry}的可选参数中采用了\textsf{keyval}接口的`\meta{key}=\meta{value}'形式。

\switchcolumn[0]*
The argument includes a list of comma-separated \textsf{keyval}
options and has basic rules as follows:
\switchcolumn
参数包括一个由逗号分隔的\textsf{keyval}选项列表,并有以下基本规则:
\end{paracol}


\begin{itemize}\setlength{\itemsep}{-.5\parsep}
\columnratio{0.55}
\begin{paracol}{2}
\item Multiple lines are allowed, while blank lines are not.
\switchcolumn
\item 允许有多行,但不允许有空行。
\switchcolumn
\item Any spaces between words are ignored.
\switchcolumn\item 单词之间的空格会被忽略。
\switchcolumn
\item Options are basically order-independent.
(There are some exceptions. See Section~\ref{sec:optionorder} for details.)
\switchcolumn
\item 选项基本上是无序的。
(有一些例外情况,请参见第~\ref{sec:optionorder}节了解详细信息。)
\end{paracol}
\end{itemize}

\columnratio{0.55}
\begin{paracol}{2}
For example,
\begin{quote}
|\usepackage[a5paper,hmargin={3cm,.8in},height=10in]{geometry}|
\end{quote}
is equivalent to 
\begin{quote}
|\usepackage[height=10in,a5paper,hmargin={3cm,0.8in}]{geometry}|
\end{quote}
\switchcolumn
例如,
\begin{quote}
\begin{minted}{latex}
\usepackage[a5paper,hmargin={3cm,.8in},height=10in]{geometry}
\end{minted}
\end{quote}
等同于
\begin{quote}
\begin{minted}{latex}
\usepackage[height=10in,a5paper,hmargin={3cm,0.8in}]{geometry}
\end{minted}
\end{quote}


\switchcolumn[0]*
Some options are allowed to have sub-list, e.g. |{3cm,0.8in}|.
Note that the order of values in the sub-list is significant.
The above setting is also equivalent to the followings:
\begin{quote}
  |\usepackage{geometry}|\\
  |\geometry{height=10in,a5paper,hmargin={3cm,0.8in}}|
\end{quote}
or 
\begin{quote}
  |\usepackage[a5paper]{geometry}|\\
  |\geometry{hmargin={3cm,0.8in},height=8in}|\\
  |\geometry{height=10in}|.
\end{quote}
\switchcolumn
某些选项允许有子列表,例如 |{3cm,0.8in}|。
请注意,子列表中的值的顺序是重要的。
上述设置也等同于以下设置:
\begin{quote}
\begin{minted}{latex}
\usepackage{geometry}
\geometry{height=10in,a5paper,hmargin={3cm,0.8in}}
\end{minted}
\end{quote}
或者
\begin{quote}
\begin{minted}{latex}
\usepackage[a5paper]{geometry} 
\geometry{hmargin={3cm,0.8in},height=8in} 
\geometry{height=10in}.
\end{minted}
\end{quote}

\switchcolumn[0]*
Thus, multiple use of \cs{geometry} just appends options.
\switchcolumn
因此,多次使用 \cs{geometry} 只会追加选项。

\switchcolumn[0]*
\Gm\ supports package 
\textsl{calc}\footnote{CTAN:~\texttt{macros/latex/required/tools}}.
For example,
\begin{quote}
|\usepackage{calc}|\\
|\usepackage[textheight=20\baselineskip+10pt]{geometry}|
\end{quote}
\switchcolumn
\Gm\ 支持 \textsl{calc}\footnote{CTAN:~\texttt{macros/latex/required/tools}} 宏包。
例如,
\begin{quote}
\begin{minted}{latex}
\usepackage{calc}
\usepackage[textheight=20\baselineskip+10pt]{geometry}
\end{minted}
\end{quote}
\end{paracol}



\columnratio{0.55}
\begin{paracol}{2}
\subsection{Option types}
\Gm\ options are categorized into four types:
\switchcolumn
\subsection{选项类型}
\Gm\ 的选项被分为四种类型:
\end{paracol}


\begin{enumerate}\itemsep=0pt
\columnratio{0.55}
\begin{paracol}{2}
\item \textbf{Boolean type}
takes a boolean value (|true| or |false|). If no value,
|true| is set by default.
\begin{quote}
\meta{key}|=true|\OR|false|.\\
\meta{key} with no value is equivalent to 
\meta{key}|=true|.
\end{quote}
\textit{Examples:}~ |verbose=true|, |includehead|, 
|twoside=false|.\\
Paper name is the exception. The preferred paper name should be set
with no values. Whatever value is given, it is ignored. For
instance, |a4paper=XXX| is equivalent to |a4paper|.
\switchcolumn
\item \textbf{布尔类型}
可以取布尔值(|true| 或 |false|)。如果没有给出值,则默认为 |true|。
\begin{quote}
\meta{key}|=true|\OR|false|.\\
\meta{key} 没有给出值,等同于
\meta{key}|=true|.
\end{quote}

\textit{示例:}~ |verbose=true|, |includehead|,
|twoside=false|。\\
纸张名称是一个例外。应该使用没有值来设置首选纸张名称。无论给出什么值,都会被忽略。例如,|a4paper=XXX| 等同于 |a4paper|。

\switchcolumn[0]*
\item \textbf{Single-valued type}
takes a mandatory value.
\begin{quote}
\meta{key}|=|\meta{value}.
\end{quote}
\textit{Examples:}~ |width=7in|, |left=1.25in|,
|footskip=1cm|, |height=.86\paperheight|.
\switchcolumn
\item \textbf{单值类型}
必须给出一个值。
\begin{quote}
\meta{key}|=|\meta{value}.
\end{quote}
\textit{示例:}~ |width=7in|, |left=1.25in|,
|footskip=1cm|, |height=.86\paperheight|。

\switchcolumn[0]*
\item \textbf{Double-valued type}

takes a pair of comma-separated values in braces. The two values can
be shortened to one value if they are identical.
\begin{quote}
\meta{key}|=|\argii{value1}{value2}.\\
\meta{key}|=|\meta{value} is equivalent to 
\meta{key}|=|\argii{value}{value}
\end{quote}
\textit{Examples:}~ |hmargin={1.5in,1in}|, |scale=0.8|,
|body={7in,10in}|.
\switchcolumn

\item \textbf{双值类型}

需要用括号括起来的逗号分隔的一对值。如果两个值相同,可以缩写为一个值
\begin{quote}
\meta{key}|=|\argii{value1}{value2}.\\
\meta{key}|=|\meta{value} 等同于
\meta{key}|=|\argii{value}{value}
\end{quote}
\textit{示例:}~ |hmargin={1.5in,1in}|, |scale=0.8|,
|body={7in,10in}|。

\switchcolumn[0]*
\item \textbf{Triple-valued type}

takes three mandatory, comma-separated values in braces.
\begin{quote}
\meta{key}|=|\argiii{value1}{value2}{value3}
\end{quote}
Each value must be a dimension or null. When you give an empty value
or `|*|', it means null and leaves the appropriate value 
to the auto-completion mechanism. You need to specify at least one
dimension, typically two dimensions. You can set nulls for all the 
values, but it makes no sense.
\textit{Examples:}\\
\hspace*{2em} |hdivide={2cm,*,1cm}|, |vdivide={3cm,19cm, }|,
    |divide={1in,*,1in}|.
\switchcolumn
\item \textbf{三值类型}

需要给出三个必填的逗号分隔值。
\begin{quote}
\meta{key}|=|\argiii{value1}{value2}{value3}
\end{quote}
每个值必须是一个长度或空值。当给出空值或 `|*|' 时,表示为空值,并将相应的值留给自动补全机制。您需要至少指定一个长度,通常是两个长度。您可以为所有值设置为空值,但这没有意义。
\textit{示例:}\\
\hspace{2em} |hdivide={2cm,*,1cm}, vdivide={3cm,19cm, }, divide={1in,*,1in}|.
\end{paracol}
\end{enumerate}
 
% \section{Option details\hfill 选项详细信息}
\columnratio{0.55}
\begin{paracol}{2}
This section describes all options available in \Gm.
Options with a dagger $^\dagger$ are not available 
as arguments of \cs{newgeometry} (See Section~\ref{sec:midchange}).
\switchcolumn
本节描述了 \Gm 中所有可用的选项。
带有†标记的选项不能作为 \cs{newgeometry} 的参数使用(参见第~\ref{sec:midchange}节)。
\end{paracol}

% \columnratio{0.55}
\begin{paracol}{2}
\switchcolumn[0]*
\subsection{Paper size}\label{sec:paper}

The options below set paper/media size and orientation.
\switchcolumn
\subsection{纸张大小}
以下选项设置纸张/媒体大小和方向。
\end{paracol}


\begin{Options}
\columnratio{0.55}
\begin{paracol}{2}
\item[\onlypre paper\OR papername] ~\\ 
specifies the paper size by name. |paper=|\meta{paper-name}.
For convenience, you can specify the paper name without |paper=|.
For example, |a4paper| is equivalent to |paper=a4paper|.
\switchcolumn
\item[\onlypre paper\OR papername] ~\\
通过名称指定纸张大小。|paper=|\meta{纸张名称}。
为了方便起见,您可以省略 |paper=| 直接指定纸张名称。
例如,|a4paper| 等同于 |paper=a4paper|。
\switchcolumn[0]*
\item[\onlypre \vtop{
\hbox{a0paper, a1paper, a2paper, a3paper, a4paper, a5paper, a6paper,}
\hbox{b0paper, b1paper, b2paper, b3paper, b4paper, b5paper, b6paper,}
\hbox{c0paper, c1paper, c2paper, c3paper, c4paper, c5paper, c6paper,}
\hbox{b0j, b1j, b2j, b3j, b4j, b5j, b6j,}
\hbox{ansiapaper, ansibpaper, ansicpaper, ansidpaper, ansiepaper,}
\hbox{letterpaper, executivepaper, legalpaper}}]~\\[1ex] 
    specifies paper name.  The value part is ignored even if any.
    For example, the followings have the same effect:
    |a5paper|, |a5paper=true|, |a5paper=false| and so forth.
    |a[0-6]paper|, |b[0-6]paper| and |c[0-6]paper| are ISO A, B and C
    series of paper sizes respectively.
    The JIS (Japanese Industrial Standards) A-series is identical to the
    ISO A-series, but the JIS B-series is different from the ISO B-series.
    |b[0-6]j| should be used for the JIS B-series. 
\switchcolumn
\item[\onlypre \vtop{
\hbox{a0paper, a1paper, a2paper, a3paper, a4paper, a5paper, a6paper,}
\hbox{b0paper, b1paper, b2paper, b3paper, b4paper, b5paper, b6paper,}
\hbox{c0paper, c1paper, c2paper, c3paper, c4paper, c5paper, c6paper,}
\hbox{b0j, b1j, b2j, b3j, b4j, b5j, b6j,}
\hbox{ansiapaper, ansibpaper, ansicpaper, ansidpaper, ansiepaper,}
\hbox{letterpaper, executivepaper, legalpaper}}]~\\[1ex] 
指定纸张名称。即使有值部分,也会被忽略。
例如,下面的选项具有相同的效果:
|a5paper|, |a5paper=true|, |a5paper=false| 等等。
|a[0-6]paper|, |b[0-6]paper| 和 |c[0-6]paper| 分别是 ISO A、B 和 C
系列纸张大小。
JIS(日本工业标准)A 系列与 ISO A 系列相同,但 JIS B 系列与 ISO B 系列不同。
应该使用 |b[0-6]j| 来表示 JIS B 系列。

\switchcolumn[0]*
\item[\onlypre screen] a special paper size with (W,H) = (225mm,180mm).
For presentation with PC and video projector, ``|screen,centering|''
with `slide' documentclass would be useful.
\switchcolumn
\item[\onlypre screen] 一种特殊的纸张尺寸,宽度(W)为225mm,高度(H)为180mm。
对于使用个人电脑和视频投影仪进行演示,使用带有 `slide' 文档类的 ``|screen,centering|'' 会很有用。
\switchcolumn[0]*
\item[\onlypre paperwidth] width of the paper. |paperwidth=|\meta{length}.
\switchcolumn
\item[\onlypre paperwidth] 纸张的宽度。|paperwidth=|\meta{长度}。
\switchcolumn[0]*
\item[\onlypre paperheight] height of the paper. |paperheight=|\meta{length}.
\switchcolumn
\item[\onlypre paperheight] 纸张的高度。|paperheight=|\meta{长度}。
\switchcolumn[0]*
\item[\onlypre papersize] width and height of the paper. 
    |papersize=|\argii{width}{height} or |papersize=|\meta{length}.
\switchcolumn
\item[\onlypre papersize] 纸张的宽度和高度。|papersize=|\argii{宽度}{高度} 或 |papersize=|\meta{长度}。
\switchcolumn[0]*
\item[\onlypre landscape] switches the paper orientation to landscape mode.
\switchcolumn
\item[\onlypre landscape] 将纸张方向切换为横向模式。
\switchcolumn[0]*
\item[\onlypre portrait] switches the paper orientation to portrait mode.
This is equivalent to |landscape=false|.
\switchcolumn
\item[\onlypre portrait] 将纸张方向切换为纵向模式。
这相当于 |landscape=false|。
\end{paracol}
\end{Options}

\columnratio{0.55}
\begin{paracol}{2}
The options for paper names (e.g., |a4paper|) and orientation
(|portrait| and |landscape|) can be set as document class options. 
For example, you can set |\documentclass[a4paper,landscape]{article}|, 
then |a4paper| and |landscape| are processed in \Gm\ as well.
This is also the case for |twoside| and |twocolumn|
(see also Section~\ref{sec:dimension}).
\switchcolumn
纸张名称选项(例如,|a4paper|)和方向选项(|portrait| 和 |landscape|)可以作为文档类选项进行设置。
例如,您可以设置 |\documentclass[a4paper,landscape]{article}|,那么 |a4paper| 和 |landscape| 也会在 \Gm\ 中进行处理。
对于 |twoside| 和 |twocolumn| 也是如此(详见第~\ref{sec:dimension}~节)。
\end{paracol}

% 
\columnratio{0.55}
\begin{paracol}{2}
\switchcolumn[0]*
\subsection{Layout size}
You can specify the layout area with options described in this
section regardless of the paper size.
The options would help to print the specified layout to a different
sized paper.  For example, with |a4paper| and |layout=a5paper|, the
package uses `A5' layout to calculate margins on 'A4' paper.
The layout size defaults to the same as the paper.
The options for the layout size are available in \cs{newgeometry},
so that you can change the layout size in the middle of the document.
The paper size itself can't be changed though.
Figure~\ref{fig:layoutandpaper} shows what the difference between
|layout| and |paper| is.
\switchcolumn
\subsection{布局尺寸}

无论纸张大小如何,您都可以使用本节中描述的选项来指定布局区域。
这些选项可以帮助您将指定的布局打印到不同大小的纸张上。
例如,使用 |a4paper| 和 |layout=a5paper|,该宏包会在“A4”纸上使用“A5”布局来计算边距。
布局尺寸默认与纸张相同。
布局尺寸的选项在 \cs{newgeometry} 中也可用,因此您可以在文档的中间更改布局尺寸。
但是纸张尺寸本身无法更改。
图~\ref{fig:layoutandpaper}~展示了 |layout| 和 |paper| 之间的区别。
\end{paracol}


\begin{Options}
\columnratio{0.55}
\begin{paracol}{2}
\item[layout] specifies the layout size by paper
name. |layout=|\meta{paper-name}. All the paper names defined in \Gm\
are available. See Section~\ref{sec:paper} for details.
\switchcolumn
\item[layout] 按纸张名称指定布局尺寸。|layout=|\meta{纸张名称}。
所有在 \Gm\ 中定义的纸张名称都可用。详情请参阅第~\ref{sec:paper}~节。
\switchcolumn[0]*
\item[layoutwidth] width of the layout. |layoutwidth=|\meta{length}.
\switchcolumn
\item[layoutwidth] 布局的宽度。|layoutwidth=|\meta{长度}。
\switchcolumn[0]*
\item[layoutheight] height of the layout. |layoutheight=|\meta{length}.
\switchcolumn
\item[layoutheight] 布局的高度。|layoutheight=|\meta{长度}。
\switchcolumn[0]*
\item[layoutsize] width and height of the layout.
|layoutsize=|\argii{width}{height} or |layoutsize=|\meta{length}.
\switchcolumn
\item[layoutsize] 布局的宽度和高度。|layoutsize=|\argii{宽度}{高度} 或 |layoutsize=|\meta{长度}。
\switchcolumn[0]*
\item[layouthoffset] specifies the horizontal offset from the left edge of
the paper. |layouthoffset=|\meta{length}.
\switchcolumn
\item[layouthoffset] 指定布局相对于纸张左边缘的水平偏移量。|layouthoffset=|\meta{长度}。
\switchcolumn[0]*
\item[layoutvoffset] specifies the vertical offset from the top edge of
the paper. |layoutvoffset=|\meta{length}.
\switchcolumn
\item[layoutvoffset] 指定布局相对于纸张顶边缘的垂直偏移量。|layoutvoffset=|\meta{长度}。
\switchcolumn[0]*
\item[layoutoffset] specifies both horizontal and vertical offsets. 
|layoutoffset=|\argii{hoffset}{voffset} or |layoutsize=|\meta{length}.
\switchcolumn
\item[layoutoffset] 指定布局的水平和垂直偏移量。|layoutoffset=|\argii{水平偏移量}{垂直偏移量} 或 |layoutsize=|\meta{长度}。
\end{paracol}
\end{Options}
\begin{figure}
 \centering\small
 {\unitlength=.6pt
 \begin{picture}(450,250)(0,-10)
 \put(20,0){\makebox(168,12)[r]{\gpart{paper}}}
 \put(20,0){\framebox(170,230){}}
 \put(21,40){\dashbox{3}(140,189){}}
 \put(21,28){\makebox(140,12)[r]{\gpart{layout}}}
 \put(40,50){\makebox(100,10){\gpart{foot}}}
 \put(40,50){\line(1,0){100}}
 \put(40,65){\framebox(100,125){\gpart{body}}}
 \put(40,200){\framebox(100,10){\gpart{head}}}
 \put(20,230){\makebox(140,20){|layoutwidth|}}
 \put(40,240){\vector(-1,0){20}}
 \put(140,240){\vector(1,0){20}}
 \put(10,145){\vector(0,1){85}}
 \put(15,125){\makebox(0,20)[r]{|layoutheight|}}
 \put(10,125){\vector(0,-1){85}}
 \put(280,0){\makebox(168,12)[r]{\gpart{paper}}}
 \put(280,0){\framebox(170,230){}}
 \put(293,35){\dashbox{3}(140,189){}}
 \put(293,23){\makebox(140,12)[r]{\gpart{layout}}}
 \put(312,45){\makebox(100,10){\gpart{foot}}}
 \put(312,45){\line(1,0){100}}
 \put(312,60){\framebox(100,125){\gpart{body}}}
 \put(312,195){\framebox(100,10){\gpart{head}}}
 \put(235,230){\makebox(80,20)[l]{|layouthoffset|}}
 \put(260,210){\vector(1,0){20}}
 \put(308,210){\vector(-1,0){15}}
 \put(260,210){\line(-1,2){10}}
 \put(355,230){\makebox(100,20){|layoutvoffset|}}
 \put(350,250){\vector(0,-1){20}}
 \put(350,209){\vector(0,1){15}}
 \end{picture}}
 \caption[layout and paper]{%
 \begin{minipage}[t]{.7\textwidth}\raggedright\small
 The dimensions related to the layout size. Note that the layout size
 defaults to the same size as the paper, so you don't have to specify
 layout-related options explicitly in most cases.
 \end{minipage}}
 \label{fig:layoutandpaper}
\end{figure}

 
\columnratio{0.55}
\begin{paracol}{2}
\subsection{Body size}\label{sec:body}

The options specifying the size of \gpart{total body} are described in this
section.
\switchcolumn
\subsection{Body size}
本节描述了指定\gpart{总正文}尺寸的选项。
\end{paracol}

\begin{Options}
    \columnratio{0.55}
\begin{paracol}{2}
\item[hscale]
ratio of width of \gpart{total body} to \cs{paperwidth}. 
|hscale=|\meta{h-scale}, e.g., |hscale=0.8| is equivalent to
|width=0.8|\cs{paperwidth}. (|0.7| by default)
\switchcolumn
\item[hscale] \gpart{总正文}的宽度与 \cs{paperwidth} 的比例。|hscale=|\meta{h-scale},例如,|hscale=0.8| 等同于 |width=0.8|\cs{paperwidth}。(默认为 |0.7|)   
   \switchcolumn[0]*   
\item[vscale]
   ratio of height of \gpart{total body} to \cs{paperheight}, e.g.,
   |vscale=|\meta{v-scale}. (|0.7| by default) |vscale=0.9| is equivalent
   to |height=0.9|\cs{paperheight}.
\switchcolumn
\item[vscale] \gpart{总正文}的高度与 \cs{paperheight} 的比例。例如,|vscale=|\meta{v-scale}。(默认为 |0.7|)
|vscale=0.9| 等同于 |height=0.9|\cs{paperheight}。
\switchcolumn[0]*
\item[scale] ratio of \gpart{total body} to the paper.
   |scale=|\argii{h-scale}{v-scale} or |scale=|\meta{scale}.
   (|0.7| by default)
\switchcolumn
\item[scale] \gpart{总正文}与纸张的比例。|scale=|\argii{h-scale}{v-scale} 或 |scale=|\meta{scale}。(默认为 |0.7|)
\switchcolumn[0]*
\item[width\OR totalwidth] ~\\
width of \gpart{total body}. |width=|\meta{length} or
|totalwidth=|\meta{length}. This dimension defaults to |textwidth|,
but if |includemp| is set to |true|, |width| $\ge$ |textwidth| 
because |width| includes the width of the marginal notes.
If |textwidth| and |width| are specified at the same time, 
|textwidth| takes priority over |width|.
\switchcolumn
\item[width\OR totalwidth] ~\\
\gpart{总正文}的宽度。|width=|\meta{长度} 或 |totalwidth=|\meta{长度}。
该尺寸默认为 |textwidth|,但如果将 |includemp| 设置为 |true|,则 |width| $\ge$ |textwidth|,因为 |width| 包括边注的宽度。
如果同时指定 |textwidth| 和 |width|,则 |textwidth| 优先于 |width|。
\switchcolumn[0]*
\item[height\OR totalheight] ~\\
height of \gpart{total body}, excluding header and footer by default.
If |includehead| or |includefoot| is set, |height| includes
the head or foot of the page as well as |textheight|.
|height=|\meta{length} or |totalheight=|\meta{length}. If both
|textheight| and |height| are specified, |height| will be ignored.
\switchcolumn
\item[height\OR totalheight] ~\\
\gpart{总正文}的高度,默认不包括页眉和页脚。
如果设置了 |includehead| 或 |includefoot|,则 |height| 包括页眉或页脚以及 |textheight|。
|height=|\meta{长度} 或 |totalheight=|\meta{长度}。
如果同时指定了 |textheight| 和 |height|,则会忽略 |height|。
\switchcolumn[0]*
\item[total] width and height of \gpart{total body}.\\
   |total=|\argii{width}{height} or |total=|\meta{length}.
   \switchcolumn
   \item[total] \gpart{总正文}的宽度和高度。|total=|\argii{宽度}{高度} 或 |total=|\meta{长度}。
\switchcolumn[0]*
\item[textwidth] specifies \cs{textwidth}, the width of \gpart{body} 
   (the text area). |textwidth=|\meta{length}.
   \switchcolumn
   \item[textwidth] 指定 \cs{textwidth},即正文(文本区域)的宽度。|textwidth=|\meta{长度}。
   \switchcolumn[0]*
\item[textheight] specifies \cs{textheight}, the height of
   \gpart{body} (the text area). |textheight=|\meta{length}.
   \switchcolumn
   \item[textheight] 指定 \cs{textheight},即正文(文本区域)的高度。|textheight=|\meta{长度}。
\switchcolumn[0]*
\item[text\OR body] specifies both \cs{textwidth} and \cs{textheight}
   of the body of page. |body=|\argii{width}{height} or
   |text=|\meta{length}.
   \switchcolumn
   \item[text\OR body] 指定正文(文本区域)的 \cs{textwidth} 和 \cs{textheight}。|body=|\argii{宽度}{高度} 或 |text=|\meta{长度}。
\switchcolumn[0]*
\item[lines] enables users to specify \cs{textheight} by the number
   of lines. |lines|=\meta{integer}.
   \switchcolumn
   \item[lines] 允许用户通过行数指定 \cs{textheight}。|lines=|\meta{整数}。
\switchcolumn[0]*
\item[includehead] includes the head of the page, \cs{headheight}
   and \cs{headsep}, into \gpart{total body}. It is set to |false| by
   default. It is opposite to |ignorehead|. See
   Figure~\ref{fig:includes} and Figure~\ref{fig:modes}.
   \switchcolumn
   \item[includehead] 将页眉,\cs{headheight} 和 \cs{headsep} 包含到 \gpart{总正文} 中。默认为 |false|。与 |ignorehead| 相反。参见图\ref{fig:includes} 和图\ref{fig:modes}。
\switchcolumn[0]*
\item[includefoot] includes the foot of the page, \cs{footskip},
   into \gpart{total body}. It is opposite to |ignorefoot|.
   It is |false| by default. See Figure~\ref{fig:includes} and
   Figure~\ref{fig:modes}.
   \switchcolumn
   \item[includefoot] 将页脚,\cs{footskip},包含到 \gpart{总正文} 中。与 |ignorefoot| 相反。默认为 |false|。参见图~\ref{fig:includes} 和图~\ref{fig:modes}。
   \switchcolumn[0]*
\item[includeheadfoot]~\\ 
   sets both |includehead| and |includefoot| to |true|, which is opposite
   to |ignoreheadfoot|. See Figure~\ref{fig:includes} and
   Figure~\ref{fig:modes}.
\switchcolumn
\item[includeheadfoot]~\\ 
将 |includehead| 和 |includefoot| 都设置为 |true|,与 |ignoreheadfoot| 相反。参见图\ref{fig:includes} 和图~\ref{fig:modes}。
\switchcolumn[0]*
\item[includemp] includes the margin notes,  \cs{marginparwidth}
   and \cs{marginparsep}, into \gpart{body} when calculating horizontal
   calculation.
   \switchcolumn
   \item[includemp] 当计算水平布局时,将边注,\cs{marginparwidth} 和 \cs{marginparsep},包含到 \gpart{正文} 中。
\switchcolumn[0]*
\item[includeall] sets both |includeheadfoot| and |includemp| to
   |true|. See Figure~\ref{fig:modes}.
   \switchcolumn
   \item[includeall] 将 |includeheadfoot| 和 |includemp| 都设置为 |true|。参见图~\ref{fig:modes}。
\switchcolumn[0]*
\item[ignorehead] disregards the head of the page,
   |headheight| and |headsep|, in determining vertical layout, but does not
   change those lengths. It is equivalent to |includehead=false|. It is set
   to |true| by default. See also |includehead|.
   \switchcolumn
   \item[ignorehead] 在确定垂直布局时忽略页眉,即 \cs{headheight} 和 \cs{headsep},但不更改这些长度。等同于 |includehead=false|。默认为 |true|。参见 |includehead|。
\switchcolumn[0]*
\item[ignorefoot] disregards the foot of page, |footskip|,
   in determining vertical layout, but does not change that length.
   This option defaults to |true|. See also |includefoot|.
   \switchcolumn
   \item[ignorefoot] 在确定垂直布局时忽略页脚,即 \cs{footskip},但不更改该长度。默认为 |true|。参见 |includefoot|。
\switchcolumn[0]*
\item[ignoreheadfoot]~\\ sets both |ignorehead| and |ignorefoot|
   to |true|. See also |includeheadfoot|.
   \switchcolumn
   \item[ignoreheadfoot] ~\\ 将 |ignorehead| 和 |ignorefoot| 都设置为 |true|。参见 |includeheadfoot|。
\switchcolumn[0]*
\item[ignoremp] disregards the marginal notes in determining the
   horizontal margins (defaults to |true|). If marginal notes overrun
   the page, the warning message will be displayed when |verbose=true|.
   See also |includemp| and Figure~\ref{fig:modes}.
   \switchcolumn
   \item[ignoremp] 在确定水平边距时忽略边注(默认为 |true|)。如果边注超出页面,则在 |verbose=true| 时将显示警告消息。参见 |includemp| 和图\ref{fig:modes}。
\switchcolumn[0]*
\item[ignoreall] sets both |ignoreheadfoot| and |ignoremp| to |true|. 
   See also |includeall|.
   \switchcolumn
   \item[ignoreall] 将 |ignoreheadfoot| 和 |ignoremp| 都设置为 |true|。参见 |includeall|。
\switchcolumn[0]*
\item[heightrounded]~\\
   This option rounds \cs{textheight} to \textit{n}-times (\textit{n}:
   an integer) of \cs{baselineskip} plus \cs{topskip} to avoid 
   ``underfull vbox'' in some cases. For example, if \cs{textheight} is
   486pt with \cs{baselineskip} 12pt and \cs{topskip} 10pt, then
   \begin{quote}
     $(39\times12\textrm{pt}+10\textrm{pt}=)\: 478\textrm{pt}
      < 486\textrm{pt} < 
     490\textrm{pt} \:(=40\times12\textrm{pt}+10\textrm{pt})$,
   \end{quote}
   as a result \cs{textheight} is rounded to 490pt. |heightrounded=false|
   by default.
   \switchcolumn
   \item[heightrounded] ~\\
此选项将 \cs{textheight} 舍入为 \textit{n} 倍(\textit{n}:整数)的 \cs{baselineskip} 加上 \cs{topskip},以避免在某些情况下出现``underfull vbox''。
例如,如果 \cs{textheight} 是 486pt,\cs{baselineskip} 是 12pt,\cs{topskip} 是 10pt,则
\begin{quote}
    $(39\times12\textrm{pt}+10\textrm{pt}=)\: 478\textrm{pt}
     < 486\textrm{pt} < 
    490\textrm{pt} \:(=40\times12\textrm{pt}+10\textrm{pt})$,
  \end{quote}
结果 \cs{textheight} 被舍入为 490pt。默认情况下,|heightrounded=false|。
\end{paracol}
\end{Options}

Figure~\ref{fig:modes} illustrates various layouts with different layout
modes. The dimensions for a header and a footer can be controlled by
|nohead| or |nofoot| mode, which sets each length to 0pt directly.
On the other hand, options with the prefix |ignore| do \textit{not}
change the corresponding native dimensions.
\begin{figure}
 \centering\small
 {\unitlength=.65pt
 \begin{picture}(460,525)(0,0)
 \put( 20,310){\framebox(120,170){}}
 \put( 20,507){\makebox(120,0)[bl]%
 {\textbf{(a)}~|includeheadfoot|}}
 \put( 20,460){\line(1,0){120}}\put( 20,450){\line(1,0){120}}
 \put( 20,330){\line(1,0){120}}
 \put( 20,485){\makebox(120,0)[br]{\gpart{total body}}}
 \put( 20,335){\makebox(120,0)[bc]{|textwidth|}}
 \put(150,470){\makebox(0,0)[l]{|headheight|}}
 \put(150,450){\makebox(0,0)[l]{|headsep|}}
 \put(150,390){\makebox(0,0)[l]{|textheight|}}
 \put(150,320){\makebox(0,0)[l]{|footskip|}}
 \put( 10,460){\makebox(120,20)[bc]{\gpart{head}}}
 \put( 10,320){\makebox(120,140)[c]{\gpart{body}}}
 \put( 10,310){\makebox(120,10)[c]{\gpart{foot}}}
 \put(250,310){\framebox(120,170){}}
 \put(250,507){\makebox(120,0)[bl]%
 {\textbf{(b)}~|includeall|}}
 \put(250,460){\line(1,0){95}}\put(250,450){\line(1,0){95}}
 \put(250,330){\line(1,0){95}}\put(345,330){\line(0,1){120}}
 \put(350,330){\line(0,1){120}}\put(350,450){\line(1,0){20}}
 \put(350,330){\line(1,0){20}}
 \put(250,485){\makebox(120,0)[br]{\gpart{total body}}}
 \put(250,460){\makebox(95,20)[bc]{\gpart{head}}}
 \put(250,320){\makebox(95,140)[c]{\gpart{body}}}
 \put(385,390){\makebox(95,0)[cl]%
 {\gpart{\shortstack[l]{marginal\\note}}}}
 \put(250,310){\makebox(95,10)[c]{\gpart{foot}}}
 \put(250,335){\makebox(95,0)[bc]{|textwidth|}}
 \multiput(360, 390)(4,0){6}{\line(1,0){2}}
 \multiput(348,333)(0,-4){12}{\line(0,1){2}}
 \multiput(360,333)(0,-4){8}{\line(0,1){2}}
 \put(355,292){\makebox(0,0)[bl]{|marginparwidth|}}
 \put(345,275){\makebox(0,0)[bl]{|marginparsep|}}
 \put( 20, 40){\framebox(120,170){}}
 \put( 20,237){\makebox(120,0)[bl]%
 {\textbf{(c)}~|includefoot|}}
 \put( 20, 60){\line(1,0){120}}
 \put( 20,215){\makebox(120,0)[br]{\gpart{total body}}}
 \put(150,130){\makebox(0,0)[l]{|textheight|}}
 \put(150, 50){\makebox(0,0)[l]{|footskip|}}
 \put( 20, 50){\makebox(120,160)[c]{\gpart{body}}}
 \put( 20, 40){\makebox(120,10)[c]{\gpart{foot}}}
 \put( 20, 65){\makebox(120,10)[c]{|textwidth|}}
 \put(250, 40){\framebox(120,170){}}
 \put(250,237){\makebox(120,0)[bl]%
 {\textbf{(d)}~|includefoot,includemp|}}
 \put(250, 60){\line(1,0){95}}\put(350, 60){\line(1,0){20}}
 \put(250,215){\makebox(120,0)[br]{\gpart{total body}}}
 \put(250, 50){\makebox(95,160)[c]{\gpart{body}}}
 \put(385,130){\makebox(95,0)[cl]%
 {\gpart{\shortstack[l]{marginal\\note}}}}
 \put(250, 40){\makebox(95,10)[c]{\gpart{foot}}}
 \put(250, 65){\makebox(95,0)[bc]{|textwidth|}}
 \put(345, 60){\line(0,1){150}}\put(350, 60){\line(0,1){150}}
 \multiput(360, 130)(4,0){6}{\line(1,0){2}}
 \multiput(348, 63)(0,-4){12}{\line(0,1){2}}
 \multiput(360, 63)(0,-4){8}{\line(0,1){2}}
 \put(355,22){\makebox(0,0)[bl]{|marginparwidth|}}
 \put(345, 5){\makebox(0,0)[bl]{|marginparsep|}}
 \end{picture}}
 \caption[Sample layouts for \gpart{total body} with different 
    layout modes]{%
 \begin{minipage}[t]{.8\textwidth}\small
   Sample layouts for \gpart{total body} with different switches.
   (a) |includeheadfoot|, (b) |includeall|, (c) |includefoot|
    and (d) |includefoot,includemp|. 
   If |reversemp| is set to |true|, the location of the
   marginal notes are swapped on every page.
   Option |twoside| swaps both margins and marginal notes on verso pages.
   Note that the marginal note, if any, is printed despite
   |ignoremp| or |includemp=false| and overrun the page in some cases.
 \end{minipage}}
 \label{fig:modes}
\end{figure}

The following options can specify body and margins simultaneously with
three comma-separated values in braces.

\begin{Options}
\item[hdivide] horizontal partitions (left,width,right).
  |hdivide=|\argiii{left margin}{width}{right margin}. 
  Note that you should not specify all of the three parameters.
  The best way of using this option is to specify two of three and 
  leave the rest with null(nothing) or `|*|'. For example, when you set
  |hdivide={2cm,15cm, }|, the margin from the right-side edge of page
  will be determined calculating |paperwidth-2cm-15cm|.
\item[vdivide] vertical partitions (top,height,bottom).
  |vdivide=|\argiii{top margin}{height}{bottom margin}.
\item[divide] |divide=|\vargiii{$A$}{$B$}{$C$} is interpreted  as 
  |hdivide=|\vargiii{$A$}{$B$}{$C$} and |vdivide=|\vargiii{$A$}{$B$}{$C$}.
\end{Options}

\subsection{Margin size}\label{sec:margin}

The options specifying the size of the margins are listed below.

\begin{Options}
\item[left\OR lmargin\OR inner]~\\
   left margin (for oneside) or inner margin (for twoside) of 
   \gpart{total body}. In other words, the distance between the left (inner)
   edge of the paper and that of \gpart{total body}. |left=|\meta{length}.
   |inner| has no special meaning, just an alias of |left| and |lmargin|.
\item[right\OR rmargin\OR outer]~\\ 
   right or outer margin of \gpart{total body}. |right=|\meta{length}.
\item[top\OR tmargin] top margin of the page. |top=|\meta{length}.
   Note this option has nothing to do with the native dimension
   \cs{topmargin}.
\item[bottom\OR bmargin]~\\ 
   bottom margin of the page. |bottom=|\meta{length}.
\item[hmargin] left and right margin.
  |hmargin=|\argii{left margin}{right margin} or |hmargin=|\meta{length}.
\item[vmargin] top and bottom margin.
  |vmargin=|\argii{top margin}{bottom margin} or |vmargin=|\meta{length}.
\item[margin] |margin=|\vargii{$A$}{$B$} is equivalent to 
  |hmargin=|\vargii{$A$}{$B$} and |vmargin=|\vargii{$A$}{$B$}.
  |margin=|$A$ is automatically expanded to |hmargin=|$A$ and |vmargin=|$A$.
\item[hmarginratio]
  horizontal margin ratio of |left| (inner) to |right| (outer). 
  The value of \meta{ratio} should be specified with colon-separated 
  two values. Each value should be a positive integer less than 100
  to prevent arithmetic overflow, e.g., |2:3| instead of |1:1.5|.
  The default ratio is |1:1| for oneside, |2:3| for twoside.
\item[vmarginratio]
   vertical margin ratio of |top| to |bottom|. The default ratio is |2:3|.
\item[marginratio\OR ratio]~\\
   horizontal and vertical margin ratios.
  |marginratio=|\argii{horizontal ratio}{vertical ratio} or
  |marginratio=|\meta{ratio}.
\item[hcentering] sets auto-centering horizontally and is
  equivalent to |hmarginratio=1:1|. It is set to |true| by default for
  oneside. See also |hmarginratio|.
\item[vcentering] sets auto-centering vertically and is
  equivalent to |vmarginratio=1:1|. The default is |false|.
  See also |vmarginratio|.
\item[centering] sets auto-centering and is equivalent to
  |marginratio=1:1|. See also |marginratio|. The default is |false|.
  See also |marginratio|.
\item[twoside] switches on twoside mode with left and right margins swapped
  on verso pages. The option sets \cs{@twoside} and \cs{@mparswitch} 
  switches. See also |asymmetric|.
\item[asymmetric] implements a twosided layout in which margins are
  not swapped on alternate pages (by setting \cs{oddsidemargin} to 
  \cs{evensidemargin} |+| |bindingoffset|) and in which the marginal notes
  stay always on the same side. This option can be used as an alternative
  to the twoside option. See also |twoside|.
\item[bindingoffset]~\\ removes a specified space 
  from the lefthand-side of the page for oneside or the inner-side for
  twoside. |bindingoffset=|\meta{length}. This is useful if pages 
  are bound by a press binding (glued, stitched, stapled \ldots).
  See Figure~\ref{fig:bindingoffset}.
\item[hdivide] See description in Section~\ref{sec:body}.
\item[vdivide] See description in Section~\ref{sec:body}.
\item[divide] See description in Section~\ref{sec:body}.
\end{Options}
\begin{figure}
 \centering\small
 {\unitlength=.65pt
 \begin{picture}(500,270)(0,0)
 \put(20,0){\framebox(170,230){}}
 \put(20,255){\makebox(80,20)[l]{\textbf{a)}~every page for oneside or}}
 \put(20,240){\makebox(80,20)[l]{\hspace{3ex}odd pages for twoside}}
 \put(110,225){\makebox(80,20)[r]{\gpart{paper}}}
 \put(55,37){\framebox(110,170)[tc]{\gpart{total body}}}
 \multiput(38,0)(0,7){33}{\line(0,1){4}}
 \put(38,100){\vector(1,0){17}}\put(55,100){\vector(-1,0){17}}
 \put(60,95){\makebox(80,10)[l]{|left|}}
 \put(60,80){\makebox(80,10)[l]{(|inner|)}}
 \put(165,100){\vector(1,0){25}}\put(190,100){\vector(-1,0){25}}
 \put(195,95){\makebox(80,10)[l]{|right|}}
 \put(195,80){\makebox(80,10)[l]{(|outer|)}}
 \put(20,16){\vector(1,0){18}}
 \put(45,10){\makebox(80,10)[bl]{|bindingoffset|}}
 \put(280,255){\makebox(80,20)[l]{\textbf{b)}~even (back) pages for twoside}}
 \put(280,0){\framebox(170,230){}}
 \put(370,225){\makebox(80,20)[r]{\gpart{paper}}}
 \put(305,37){\framebox(110,170)[tc]{\gpart{total body}}}
 \multiput(432,0)(0,7){33}{\line(0,1){4}}
 \put(280,100){\vector(1,0){25}}\put(305,100){\vector(-1,0){25}}
 \put(310,95){\makebox(80,10)[l]{|outer|}}
 \put(310,80){\makebox(80,10)[l]{(|right|)}}
 \put(415,100){\vector(1,0){17}}\put(432,100){\vector(-1,0){17}}
 \put(373,95){\makebox(80,10)[l]{|inner|}}
 \put(373,80){\makebox(80,10)[l]{(|left|)}}
 \put(450,16){\vector(-1,0){18}}
 \put(330,10){\makebox(80,10)[bl]{|bindingoffset|}}
 \end{picture}}
 \caption[\texttt{bindingoffset} option]{%
  \begin{minipage}[t]{.8\textwidth}\raggedright\small
  The option |bindingoffset| adds the specified length to the inner margin.
  Note that |twoside| option swaps the horizontal margins and the
  marginal notes together with |bindingoffset| on even pages (see
  \textbf{b)}), but |asymmetric| option suppresses the swap of the
  margins and marginal notes (but |bindingoffset| is still swapped).
  \end{minipage}}
 \label{fig:bindingoffset}
\end{figure}

\subsection{Native dimensions}\label{sec:dimension}

The options below overwrite \LaTeX\ native dimensions and switches for page
layout (See the right-hand side in Figure~\ref{fig:layout}).

\begin{Options}
\item[headheight\OR head]~\\
   modifies \cs{headheight}, height of header.
   |headheight=|\meta{length} or |head=|\meta{length}.
\item[headsep] modifies \cs{headsep}, separation between header and text
   (body). |headsep=|\meta{length}.
\item[footskip\OR foot]~\\ modifies \cs{footskip}, distance separation
   between baseline of last line of text and baseline of footer.
   |footskip=|\meta{length} or |foot=|\meta{length}.
\item[nohead] eliminates spaces for the head of the page, which is
   equivalent to both \cs{headheight}|=0pt| and \cs{headsep}|=0pt|.
\item[nofoot] eliminates spaces for the foot of the page, which is
   equivalent to \cs{footskip}|=0pt|.
\item[noheadfoot] equivalent to |nohead| and |nofoot|, which means that
   \cs{headheight}, \cs{headsep} and \cs{footskip} are all set to |0pt|.
\item[footnotesep] changes the dimension \cs{skip}\cs{footins}, separation
   between the bottom of text body and the top of footnote text.
\item[marginparwidth\OR marginpar]~\\ 
   modifies \cs{marginparwidth}, width of the marginal notes.
   |marginparwidth=|\meta{length}.
\item[marginparsep] modifies \cs{marginparsep}, separation between
   body and marginal notes. |marginparsep=|\meta{length}.
\item[nomarginpar] shrinks spaces for marginal notes to 0pt, which
   is equivalent to \cs{marginparwidth}|=0pt| and \cs{marginparsep}|=0pt|.
\item[columnsep] modifies \cs{columnsep}, the separation between two
   columns in |twocolumn| mode.
\item[hoffset]  modifies \cs{hoffset}. |hoffset=|\meta{length}.
\item[voffset]  modifies \cs{voffset}. |voffset=|\meta{length}.
\item[offset] horizontal and vertical offset.\\
   |offset=|\argii{hoffset}{voffset} or |offset=|\meta{length}.
\item[twocolumn] sets |twocolumn| mode with \cs{@twocolumntrue}.
  |twocolumn=false| denotes onecolumn mode with\cs{@twocolumnfalse}.
  Instead of |twocolumn=false|, you can specify |onecolumn| (which
  defaults to |true|)
\item[onecolumn] works as |twocolumn=false|. On the other hand,
  |onecolumn=false| is equivalent to |twocolumn|. 
\item[twoside] sets both \cs{@twosidetrue} and \cs{@mparswitchtrue}.
  See Section~\ref{sec:margin}.
\item[textwidth] sets \cs{textwidth} directly. See Section~\ref{sec:body}.
\item[textheight] sets \cs{textheight} directly. See Section~\ref{sec:body}.
\item[reversemp\OR reversemarginpar]~\\
  makes the marginal notes appear in the left (inner) margin with
  \cs{@reversemargintrue}. The option doesn't change |includemp| mode.
  It's set |false| by default.
\end{Options}

\subsection{Drivers}\label{sec:drivers}

The package supports drivers |dvips|, |dvipdfm|, |pdftex|, |luatex|,
|xetex| and |vtex|. You can also set |dvipdfm| for \textsf{dvipdfmx} and
\textsf{xdvipdfmx} The options |dvipdfmx| and |xdvipdfmx| are also supported
as aliases for the |dvipdfm| option.
|pdftex| for \textsf{pdflatex}, and |vtex| for
V\TeX{} environment.
The driver options are exclusive. The driver can be set by either
|driver=|\meta{driver name} or any of the drivers directly like |pdftex|.
By default, \Gm\ guesses the driver appropriate to the system
in use. Therefore, you don't have to set a driver in most cases.
However, if you want to use |dvipdfm|, you should specify it explicitly.

\begin{Options}
\item[\onlypre driver] specifies the driver with |driver=|\meta{driver name}. 
|dvips|, |dvipdfm|, |pdftex|, |luatex|, |vtex|, |xetex|, |auto| and |none| are
available as a driver name. The names except for |auto| and |none| can
be specified directly with the name without |driver=|.
|driver=auto| makes the auto-detection work whatever the previous setting is. 
|driver=none| disables the auto-detection and sets no driver, which
may be useful when you want to let other package work out the driver
setting. For example, if you want to use \textsf{crop} package with \Gm,
you should call |\usepackage[driver=none]{geometry}| before
the \textsf{crop} package.
\item[\onlypre dvips] writes the paper size in dvi output with the \cs{special}
    macro. If you use \textsl{dvips} as a DVI-to-PS driver,
    for example, to print a document with |\geometry{a3paper,landscape}|
    on A3 paper in landscape orientation, you don't need options
    ``|-t a3 -t landscape|'' to \textsl{dvips}. 
\item[\onlypre dvipdfm] works like |dvips| except for landscape correction.
     You can set this option when using \textsf{dvipdfmx} and
     \textsf{xdvipdfmx} to process the dvi output.
\item[\onlypre pdftex] sets \cs{pdfpagewidth} and \cs{pdfpageheight}
     internally.
\item[\onlypre luatex] sets \cs{pagewidth} and \cs{pageheight} internally.
\item[\onlypre xetex] is the same as |pdftex| except for ignoring
    |\pdf{h,v}origin| undefined in \XeLaTeX{}. This option is introduced in
    the version 5. Note that `geometry.cfg' in \TeX{} Live, which disables the
    auto-detection routine and sets |pdftex|, is no longer necessary,
    but has no problem even though it's left undeleted.
    Instead of |xetex|, you can specify |dvipdfm| with \XeLaTeX{}
    if you want to use specials of dvipdfm \XeTeX{} supports.
\item[\onlypre vtex] sets dimensions \cs{mediawidth} and \cs{mediaheight}
    for V\TeX. When this driver is selected (explicitly or
    automatically), \Gm\ will auto-detect which output mode
    (DVI, PDF or PS) is selected in V\TeX, and do proper
    settings for it.
\end{Options}
If explicit driver setting is mismatched with the typesetting program
in use, the default driver |dvips| would be selected.

\subsection{Other options}

 The other useful options are described here.

\begin{Options}
\item[\onlypre verbose] displays the parameter results on the terminal.
  |verbose=false| (default) still puts them into the log file.
\item[\onlypre reset] sets back the layout dimensions and switches to the
  settings before \Gm\ is loaded. Options given in 
  |geometry.cfg| are also cleared.
  Note that this cannot reset |pass| and |mag| with |truedimen|.
  |reset=false| has no effect and cannot cancel the previous
  |reset|(|=true|) if any. For example, when you go
  \begin{quote}
    |\documentclass[landscape]{article}|\\
    |\usepackage[twoside,reset,left=2cm]{geometry}|
  \end{quote}
  with |\ExecuteOptions{scale=0.9}| in |geometry.cfg|,
  then as a result, |landscape| and |left=2cm| remain effective,
  and |scale=0.9| and |twoside| are ineffective.
\item[\onlypre mag] sets magnification value (\cs{mag}) and automatically modifies 
  \cs{hoffset} and \cs{voffset} according to the magnification.
  |mag=|\meta{value}. Note that \meta{value} should be an integer value
  with 1000 as a normal size. For example, |mag=1414| with |a4paper|
  provides an enlarged print fitting in |a3paper|, which is $1.414$
  (=$\sqrt{2}$) times larger than |a4paper|. Font enlargement needs extra
  disk space. \textbf{Note that setting |mag| should precede any other
  settings with `true' dimensions, such as  |1.5truein|, |2truecm|
  and so on.} See also |truedimen| option.
\item[\onlypre truedimen] changes all internal explicit dimension values into 
  \textit{true} dimensions, e.g., |1in| is changed to |1truein|.
  Typically this option will be used together with |mag| option. Note that
  this is ineffective against externally specified dimensions. For example,
  when you set ``\texttt{mag=1440, margin=10pt, truedimen}'', margins are
  not `true' but magnified. If you want to set exact margins, you should
  set like ``\texttt{mag=1440, margin=10truept, truedimen}'' instead.
\item[\onlypre pass] disables all of the geometry options and calculations
  except |verbose| and |showframe|. It is order-independent and can be
  used for checking out the page layout of the documentclass, other
  packages and manual settings without \Gm.
\item[\onlypre showframe] shows visible frames for the text area and page,
  and the lines for the head and foot on the first page.
\item[\onlypre showcrop] prints crop marks at each corner of user-specified
layout area.
\end{Options}

% \columnratio{0.55}
\begin{paracol}{2}
\switchcolumn[0]*
\subsection{Paper size}\label{sec:paper}

The options below set paper/media size and orientation.
\switchcolumn
\subsection{纸张大小}
以下选项设置纸张/媒体大小和方向。
\end{paracol}


\begin{Options}
\columnratio{0.55}
\begin{paracol}{2}
\item[\onlypre paper\OR papername] ~\\ 
specifies the paper size by name. |paper=|\meta{paper-name}.
For convenience, you can specify the paper name without |paper=|.
For example, |a4paper| is equivalent to |paper=a4paper|.
\switchcolumn
\item[\onlypre paper\OR papername] ~\\
通过名称指定纸张大小。|paper=|\meta{纸张名称}。
为了方便起见,您可以省略 |paper=| 直接指定纸张名称。
例如,|a4paper| 等同于 |paper=a4paper|。
\switchcolumn[0]*
\item[\onlypre \vtop{
\hbox{a0paper, a1paper, a2paper, a3paper, a4paper, a5paper, a6paper,}
\hbox{b0paper, b1paper, b2paper, b3paper, b4paper, b5paper, b6paper,}
\hbox{c0paper, c1paper, c2paper, c3paper, c4paper, c5paper, c6paper,}
\hbox{b0j, b1j, b2j, b3j, b4j, b5j, b6j,}
\hbox{ansiapaper, ansibpaper, ansicpaper, ansidpaper, ansiepaper,}
\hbox{letterpaper, executivepaper, legalpaper}}]~\\[1ex] 
    specifies paper name.  The value part is ignored even if any.
    For example, the followings have the same effect:
    |a5paper|, |a5paper=true|, |a5paper=false| and so forth.
    |a[0-6]paper|, |b[0-6]paper| and |c[0-6]paper| are ISO A, B and C
    series of paper sizes respectively.
    The JIS (Japanese Industrial Standards) A-series is identical to the
    ISO A-series, but the JIS B-series is different from the ISO B-series.
    |b[0-6]j| should be used for the JIS B-series. 
\switchcolumn
\item[\onlypre \vtop{
\hbox{a0paper, a1paper, a2paper, a3paper, a4paper, a5paper, a6paper,}
\hbox{b0paper, b1paper, b2paper, b3paper, b4paper, b5paper, b6paper,}
\hbox{c0paper, c1paper, c2paper, c3paper, c4paper, c5paper, c6paper,}
\hbox{b0j, b1j, b2j, b3j, b4j, b5j, b6j,}
\hbox{ansiapaper, ansibpaper, ansicpaper, ansidpaper, ansiepaper,}
\hbox{letterpaper, executivepaper, legalpaper}}]~\\[1ex] 
指定纸张名称。即使有值部分,也会被忽略。
例如,下面的选项具有相同的效果:
|a5paper|, |a5paper=true|, |a5paper=false| 等等。
|a[0-6]paper|, |b[0-6]paper| 和 |c[0-6]paper| 分别是 ISO A、B 和 C
系列纸张大小。
JIS(日本工业标准)A 系列与 ISO A 系列相同,但 JIS B 系列与 ISO B 系列不同。
应该使用 |b[0-6]j| 来表示 JIS B 系列。

\switchcolumn[0]*
\item[\onlypre screen] a special paper size with (W,H) = (225mm,180mm).
For presentation with PC and video projector, ``|screen,centering|''
with `slide' documentclass would be useful.
\switchcolumn
\item[\onlypre screen] 一种特殊的纸张尺寸,宽度(W)为225mm,高度(H)为180mm。
对于使用个人电脑和视频投影仪进行演示,使用带有 `slide' 文档类的 ``|screen,centering|'' 会很有用。
\switchcolumn[0]*
\item[\onlypre paperwidth] width of the paper. |paperwidth=|\meta{length}.
\switchcolumn
\item[\onlypre paperwidth] 纸张的宽度。|paperwidth=|\meta{长度}。
\switchcolumn[0]*
\item[\onlypre paperheight] height of the paper. |paperheight=|\meta{length}.
\switchcolumn
\item[\onlypre paperheight] 纸张的高度。|paperheight=|\meta{长度}。
\switchcolumn[0]*
\item[\onlypre papersize] width and height of the paper. 
    |papersize=|\argii{width}{height} or |papersize=|\meta{length}.
\switchcolumn
\item[\onlypre papersize] 纸张的宽度和高度。|papersize=|\argii{宽度}{高度} 或 |papersize=|\meta{长度}。
\switchcolumn[0]*
\item[\onlypre landscape] switches the paper orientation to landscape mode.
\switchcolumn
\item[\onlypre landscape] 将纸张方向切换为横向模式。
\switchcolumn[0]*
\item[\onlypre portrait] switches the paper orientation to portrait mode.
This is equivalent to |landscape=false|.
\switchcolumn
\item[\onlypre portrait] 将纸张方向切换为纵向模式。
这相当于 |landscape=false|。
\end{paracol}
\end{Options}

\columnratio{0.55}
\begin{paracol}{2}
The options for paper names (e.g., |a4paper|) and orientation
(|portrait| and |landscape|) can be set as document class options. 
For example, you can set |\documentclass[a4paper,landscape]{article}|, 
then |a4paper| and |landscape| are processed in \Gm\ as well.
This is also the case for |twoside| and |twocolumn|
(see also Section~\ref{sec:dimension}).
\switchcolumn
纸张名称选项(例如,|a4paper|)和方向选项(|portrait| 和 |landscape|)可以作为文档类选项进行设置。
例如,您可以设置 |\documentclass[a4paper,landscape]{article}|,那么 |a4paper| 和 |landscape| 也会在 \Gm\ 中进行处理。
对于 |twoside| 和 |twocolumn| 也是如此(详见第~\ref{sec:dimension}~节)。
\end{paracol}
 
% 
\columnratio{0.55}
\begin{paracol}{2}
\switchcolumn[0]*
\subsection{Layout size}
You can specify the layout area with options described in this
section regardless of the paper size.
The options would help to print the specified layout to a different
sized paper.  For example, with |a4paper| and |layout=a5paper|, the
package uses `A5' layout to calculate margins on 'A4' paper.
The layout size defaults to the same as the paper.
The options for the layout size are available in \cs{newgeometry},
so that you can change the layout size in the middle of the document.
The paper size itself can't be changed though.
Figure~\ref{fig:layoutandpaper} shows what the difference between
|layout| and |paper| is.
\switchcolumn
\subsection{布局尺寸}

无论纸张大小如何,您都可以使用本节中描述的选项来指定布局区域。
这些选项可以帮助您将指定的布局打印到不同大小的纸张上。
例如,使用 |a4paper| 和 |layout=a5paper|,该宏包会在“A4”纸上使用“A5”布局来计算边距。
布局尺寸默认与纸张相同。
布局尺寸的选项在 \cs{newgeometry} 中也可用,因此您可以在文档的中间更改布局尺寸。
但是纸张尺寸本身无法更改。
图~\ref{fig:layoutandpaper}~展示了 |layout| 和 |paper| 之间的区别。
\end{paracol}


\begin{Options}
\columnratio{0.55}
\begin{paracol}{2}
\item[layout] specifies the layout size by paper
name. |layout=|\meta{paper-name}. All the paper names defined in \Gm\
are available. See Section~\ref{sec:paper} for details.
\switchcolumn
\item[layout] 按纸张名称指定布局尺寸。|layout=|\meta{纸张名称}。
所有在 \Gm\ 中定义的纸张名称都可用。详情请参阅第~\ref{sec:paper}~节。
\switchcolumn[0]*
\item[layoutwidth] width of the layout. |layoutwidth=|\meta{length}.
\switchcolumn
\item[layoutwidth] 布局的宽度。|layoutwidth=|\meta{长度}。
\switchcolumn[0]*
\item[layoutheight] height of the layout. |layoutheight=|\meta{length}.
\switchcolumn
\item[layoutheight] 布局的高度。|layoutheight=|\meta{长度}。
\switchcolumn[0]*
\item[layoutsize] width and height of the layout.
|layoutsize=|\argii{width}{height} or |layoutsize=|\meta{length}.
\switchcolumn
\item[layoutsize] 布局的宽度和高度。|layoutsize=|\argii{宽度}{高度} 或 |layoutsize=|\meta{长度}。
\switchcolumn[0]*
\item[layouthoffset] specifies the horizontal offset from the left edge of
the paper. |layouthoffset=|\meta{length}.
\switchcolumn
\item[layouthoffset] 指定布局相对于纸张左边缘的水平偏移量。|layouthoffset=|\meta{长度}。
\switchcolumn[0]*
\item[layoutvoffset] specifies the vertical offset from the top edge of
the paper. |layoutvoffset=|\meta{length}.
\switchcolumn
\item[layoutvoffset] 指定布局相对于纸张顶边缘的垂直偏移量。|layoutvoffset=|\meta{长度}。
\switchcolumn[0]*
\item[layoutoffset] specifies both horizontal and vertical offsets. 
|layoutoffset=|\argii{hoffset}{voffset} or |layoutsize=|\meta{length}.
\switchcolumn
\item[layoutoffset] 指定布局的水平和垂直偏移量。|layoutoffset=|\argii{水平偏移量}{垂直偏移量} 或 |layoutsize=|\meta{长度}。
\end{paracol}
\end{Options}
\begin{figure}
 \centering\small
 {\unitlength=.6pt
 \begin{picture}(450,250)(0,-10)
 \put(20,0){\makebox(168,12)[r]{\gpart{paper}}}
 \put(20,0){\framebox(170,230){}}
 \put(21,40){\dashbox{3}(140,189){}}
 \put(21,28){\makebox(140,12)[r]{\gpart{layout}}}
 \put(40,50){\makebox(100,10){\gpart{foot}}}
 \put(40,50){\line(1,0){100}}
 \put(40,65){\framebox(100,125){\gpart{body}}}
 \put(40,200){\framebox(100,10){\gpart{head}}}
 \put(20,230){\makebox(140,20){|layoutwidth|}}
 \put(40,240){\vector(-1,0){20}}
 \put(140,240){\vector(1,0){20}}
 \put(10,145){\vector(0,1){85}}
 \put(15,125){\makebox(0,20)[r]{|layoutheight|}}
 \put(10,125){\vector(0,-1){85}}
 \put(280,0){\makebox(168,12)[r]{\gpart{paper}}}
 \put(280,0){\framebox(170,230){}}
 \put(293,35){\dashbox{3}(140,189){}}
 \put(293,23){\makebox(140,12)[r]{\gpart{layout}}}
 \put(312,45){\makebox(100,10){\gpart{foot}}}
 \put(312,45){\line(1,0){100}}
 \put(312,60){\framebox(100,125){\gpart{body}}}
 \put(312,195){\framebox(100,10){\gpart{head}}}
 \put(235,230){\makebox(80,20)[l]{|layouthoffset|}}
 \put(260,210){\vector(1,0){20}}
 \put(308,210){\vector(-1,0){15}}
 \put(260,210){\line(-1,2){10}}
 \put(355,230){\makebox(100,20){|layoutvoffset|}}
 \put(350,250){\vector(0,-1){20}}
 \put(350,209){\vector(0,1){15}}
 \end{picture}}
 \caption[layout and paper]{%
 \begin{minipage}[t]{.7\textwidth}\raggedright\small
 The dimensions related to the layout size. Note that the layout size
 defaults to the same size as the paper, so you don't have to specify
 layout-related options explicitly in most cases.
 \end{minipage}}
 \label{fig:layoutandpaper}
\end{figure}

% \columnratio{0.55}
\begin{paracol}{2}
\subsection{Body size}\label{sec:body}

The options specifying the size of \gpart{total body} are described in this
section.
\switchcolumn
\subsection{总正文尺寸}
本节描述了指定\gpart{总正文}尺寸的选项。
\end{paracol}

\begin{Options}
    \columnratio{0.55}
\begin{paracol}{2}
\item[hscale]
ratio of width of \gpart{total body} to \cs{paperwidth}. 
|hscale=|\meta{h-scale}, e.g., |hscale=0.8| is equivalent to
|width=0.8|\cs{paperwidth}. (|0.7| by default)
\switchcolumn
\item[hscale] \gpart{总正文}的宽度与 \cs{paperwidth} 的比例。|hscale=|\meta{h-scale},例如,|hscale=0.8| 等同于 |width=0.8|\cs{paperwidth}。(默认为 |0.7|)   
   \switchcolumn[0]*   
\item[vscale]
   ratio of height of \gpart{total body} to \cs{paperheight}, e.g.,
   |vscale=|\meta{v-scale}. (|0.7| by default) |vscale=0.9| is equivalent
   to |height=0.9|\cs{paperheight}.
\switchcolumn
\item[vscale] \gpart{总正文}的高度与 \cs{paperheight} 的比例。例如,|vscale=|\meta{v-scale}。(默认为 |0.7|)
|vscale=0.9| 等同于 |height=0.9|\cs{paperheight}。
\switchcolumn[0]*
\item[scale] ratio of \gpart{total body} to the paper.
   |scale=|\argii{h-scale}{v-scale} or |scale=|\meta{scale}.
   (|0.7| by default)
\switchcolumn
\item[scale] \gpart{总正文}与纸张的比例。|scale=|\argii{h-scale}{v-scale} 或 |scale=|\meta{scale}。(默认为 |0.7|)
\switchcolumn[0]*
\item[width\OR totalwidth] ~\\
width of \gpart{total body}. |width=|\meta{length} or
|totalwidth=|\meta{length}. This dimension defaults to |textwidth|,
but if |includemp| is set to |true|, |width| $\ge$ |textwidth| 
because |width| includes the width of the marginal notes.
If |textwidth| and |width| are specified at the same time, 
|textwidth| takes priority over |width|.
\switchcolumn
\item[width\OR totalwidth] ~\\
\gpart{总正文}的宽度。|width=|\meta{长度} 或 |totalwidth=|\meta{长度}。
该尺寸默认为 |textwidth|,但如果将 |includemp| 设置为 |true|,则 |width| $\ge$ |textwidth|,因为 |width| 包括边注的宽度。
如果同时指定 |textwidth| 和 |width|,则 |textwidth| 优先于 |width|。
\switchcolumn[0]*
\item[height\OR totalheight] ~\\
height of \gpart{total body}, excluding header and footer by default.
If |includehead| or |includefoot| is set, |height| includes
the head or foot of the page as well as |textheight|.
|height=|\meta{length} or |totalheight=|\meta{length}. If both
|textheight| and |height| are specified, |height| will be ignored.
\switchcolumn
\item[height\OR totalheight] ~\\
\gpart{总正文}的高度,默认不包括页眉和页脚。
如果设置了 |includehead| 或 |includefoot|,则 |height| 包括页眉或页脚以及 |textheight|。
|height=|\meta{长度} 或 |totalheight=|\meta{长度}。
如果同时指定了 |textheight| 和 |height|,则会忽略 |height|。
\switchcolumn[0]*
\item[total] width and height of \gpart{total body}.\\
   |total=|\argii{width}{height} or |total=|\meta{length}.
   \switchcolumn
   \item[total] \gpart{总正文}的宽度和高度。|total=|\argii{宽度}{高度} 或 |total=|\meta{长度}。
\switchcolumn[0]*
\item[textwidth] specifies \cs{textwidth}, the width of \gpart{body} 
   (the text area). |textwidth=|\meta{length}.
   \switchcolumn
   \item[textwidth] 指定 \cs{textwidth},即正文(文本区域)的宽度。|textwidth=|\meta{长度}。
   \switchcolumn[0]*
\item[textheight] specifies \cs{textheight}, the height of
   \gpart{body} (the text area). |textheight=|\meta{length}.
   \switchcolumn
   \item[textheight] 指定 \cs{textheight},即正文(文本区域)的高度。|textheight=|\meta{长度}。
\switchcolumn[0]*
\item[text\OR body] specifies both \cs{textwidth} and \cs{textheight}
   of the body of page. |body=|\argii{width}{height} or
   |text=|\meta{length}.
   \switchcolumn
   \item[text\OR body] 指定正文(文本区域)的 \cs{textwidth} 和 \cs{textheight}。|body=|\argii{宽度}{高度} 或 |text=|\meta{长度}。
\switchcolumn[0]*
\item[lines] enables users to specify \cs{textheight} by the number
   of lines. |lines|=\meta{integer}.
   \switchcolumn
   \item[lines] 允许用户通过行数指定 \cs{textheight}。|lines=|\meta{整数}。
\switchcolumn[0]*
\item[includehead] includes the head of the page, \cs{headheight}
   and \cs{headsep}, into \gpart{total body}. It is set to |false| by
   default. It is opposite to |ignorehead|. See
   Figure~\ref{fig:includes} and Figure~\ref{fig:modes}.
   \switchcolumn
   \item[includehead] 将页眉,\cs{headheight} 和 \cs{headsep} 包含到 \gpart{总正文} 中。默认为 |false|。与 |ignorehead| 相反。参见图\ref{fig:includes} 和图\ref{fig:modes}。
\switchcolumn[0]*
\item[includefoot] includes the foot of the page, \cs{footskip},
   into \gpart{total body}. It is opposite to |ignorefoot|.
   It is |false| by default. See Figure~\ref{fig:includes} and
   Figure~\ref{fig:modes}.
   \switchcolumn
   \item[includefoot] 将页脚,\cs{footskip},包含到 \gpart{总正文} 中。与 |ignorefoot| 相反。默认为 |false|。参见图~\ref{fig:includes} 和图~\ref{fig:modes}。
   \switchcolumn[0]*
\item[includeheadfoot]~\\ 
   sets both |includehead| and |includefoot| to |true|, which is opposite
   to |ignoreheadfoot|. See Figure~\ref{fig:includes} and
   Figure~\ref{fig:modes}.
\switchcolumn
\item[includeheadfoot]~\\ 
将 |includehead| 和 |includefoot| 都设置为 |true|,与 |ignoreheadfoot| 相反。参见图\ref{fig:includes} 和图~\ref{fig:modes}。
\switchcolumn[0]*
\item[includemp] includes the margin notes,  \cs{marginparwidth}
   and \cs{marginparsep}, into \gpart{body} when calculating horizontal
   calculation.
   \switchcolumn
   \item[includemp] 当计算水平布局时,将边注,\cs{marginparwidth} 和 \cs{marginparsep},包含到 \gpart{正文} 中。
\switchcolumn[0]*
\item[includeall] sets both |includeheadfoot| and |includemp| to
   |true|. See Figure~\ref{fig:modes}.
   \switchcolumn
   \item[includeall] 将 |includeheadfoot| 和 |includemp| 都设置为 |true|。参见图~\ref{fig:modes}。
\switchcolumn[0]*
\item[ignorehead] disregards the head of the page,
   |headheight| and |headsep|, in determining vertical layout, but does not
   change those lengths. It is equivalent to |includehead=false|. It is set
   to |true| by default. See also |includehead|.
   \switchcolumn
   \item[ignorehead] 在确定垂直布局时忽略页眉,即 \cs{headheight} 和 \cs{headsep},但不更改这些长度。等同于 |includehead=false|。默认为 |true|。参见 |includehead|。
\switchcolumn[0]*
\item[ignorefoot] disregards the foot of page, |footskip|,
   in determining vertical layout, but does not change that length.
   This option defaults to |true|. See also |includefoot|.
   \switchcolumn
   \item[ignorefoot] 在确定垂直布局时忽略页脚,即 \cs{footskip},但不更改该长度。默认为 |true|。参见 |includefoot|。
\switchcolumn[0]*
\item[ignoreheadfoot]~\\ sets both |ignorehead| and |ignorefoot|
   to |true|. See also |includeheadfoot|.
   \switchcolumn
   \item[ignoreheadfoot] ~\\ 将 |ignorehead| 和 |ignorefoot| 都设置为 |true|。参见 |includeheadfoot|。
\switchcolumn[0]*
\item[ignoremp] disregards the marginal notes in determining the
   horizontal margins (defaults to |true|). If marginal notes overrun
   the page, the warning message will be displayed when |verbose=true|.
   See also |includemp| and Figure~\ref{fig:modes}.
   \switchcolumn
   \item[ignoremp] 在确定水平边距时忽略边注(默认为 |true|)。如果边注超出页面,则在 |verbose=true| 时将显示警告消息。参见 |includemp| 和图\ref{fig:modes}。
\switchcolumn[0]*
\item[ignoreall] sets both |ignoreheadfoot| and |ignoremp| to |true|. 
   See also |includeall|.
   \switchcolumn
   \item[ignoreall] 将 |ignoreheadfoot| 和 |ignoremp| 都设置为 |true|。参见 |includeall|。
\switchcolumn[0]*
\item[heightrounded]~\\
   This option rounds \cs{textheight} to \textit{n}-times (\textit{n}:
   an integer) of \cs{baselineskip} plus \cs{topskip} to avoid 
   ``underfull vbox'' in some cases. For example, if \cs{textheight} is
   486pt with \cs{baselineskip} 12pt and \cs{topskip} 10pt, then
   \begin{quote}
     $(39\times12\textrm{pt}+10\textrm{pt}=)\: 478\textrm{pt}
      < 486\textrm{pt} < 
     490\textrm{pt} \:(=40\times12\textrm{pt}+10\textrm{pt})$,
   \end{quote}
   as a result \cs{textheight} is rounded to 490pt. |heightrounded=false|
   by default.
   \switchcolumn
   \item[heightrounded] ~\\
此选项将 \cs{textheight} 舍入为 \textit{n} 倍(\textit{n}:整数)的 \cs{baselineskip} 加上 \cs{topskip},以避免在某些情况下出现``underfull vbox''。
例如,如果 \cs{textheight} 是 486pt,\cs{baselineskip} 是 12pt,\cs{topskip} 是 10pt,则
\begin{quote}
    $(39\times12\textrm{pt}+10\textrm{pt}=)\: 478\textrm{pt}
     < 486\textrm{pt} < 
    490\textrm{pt} \:(=40\times12\textrm{pt}+10\textrm{pt})$,
  \end{quote}
结果 \cs{textheight} 被舍入为 490pt。默认情况下,|heightrounded=false|。
\end{paracol}
\end{Options}

\columnratio{0.55}
\begin{paracol}{2}
Figure~\ref{fig:modes} illustrates various layouts with different layout
modes. The dimensions for a header and a footer can be controlled by
|nohead| or |nofoot| mode, which sets each length to 0pt directly.
On the other hand, options with the prefix |ignore| do \textit{not}
change the corresponding native dimensions.
\switchcolumn
图~\ref{fig:modes} 展示了不同布局模式下的各种布局。页眉和页脚的尺寸可以通过 |nohead| 或 |nofoot| 模式进行控制,这会直接将每个长度设置为 0pt。另一方面,以 |ignore| 为前缀的选项\textit{不会}改变对应的原始尺寸。
\end{paracol}
\begin{figure}
 \centering\small
 {\unitlength=.65pt
 \begin{picture}(460,525)(0,0)
 \put( 20,310){\framebox(120,170){}}
 \put( 20,507){\makebox(120,0)[bl]%
 {\textbf{(a)}~|includeheadfoot|}}
 \put( 20,460){\line(1,0){120}}\put( 20,450){\line(1,0){120}}
 \put( 20,330){\line(1,0){120}}
 \put( 20,485){\makebox(120,0)[br]{\gpart{total body}}}
 \put( 20,335){\makebox(120,0)[bc]{|textwidth|}}
 \put(150,470){\makebox(0,0)[l]{|headheight|}}
 \put(150,450){\makebox(0,0)[l]{|headsep|}}
 \put(150,390){\makebox(0,0)[l]{|textheight|}}
 \put(150,320){\makebox(0,0)[l]{|footskip|}}
 \put( 10,460){\makebox(120,20)[bc]{\gpart{head}}}
 \put( 10,320){\makebox(120,140)[c]{\gpart{body}}}
 \put( 10,310){\makebox(120,10)[c]{\gpart{foot}}}
 \put(250,310){\framebox(120,170){}}
 \put(250,507){\makebox(120,0)[bl]%
 {\textbf{(b)}~|includeall|}}
 \put(250,460){\line(1,0){95}}\put(250,450){\line(1,0){95}}
 \put(250,330){\line(1,0){95}}\put(345,330){\line(0,1){120}}
 \put(350,330){\line(0,1){120}}\put(350,450){\line(1,0){20}}
 \put(350,330){\line(1,0){20}}
 \put(250,485){\makebox(120,0)[br]{\gpart{total body}}}
 \put(250,460){\makebox(95,20)[bc]{\gpart{head}}}
 \put(250,320){\makebox(95,140)[c]{\gpart{body}}}
 \put(385,390){\makebox(95,0)[cl]%
 {\gpart{\shortstack[l]{marginal\\note}}}}
 \put(250,310){\makebox(95,10)[c]{\gpart{foot}}}
 \put(250,335){\makebox(95,0)[bc]{|textwidth|}}
 \multiput(360, 390)(4,0){6}{\line(1,0){2}}
 \multiput(348,333)(0,-4){12}{\line(0,1){2}}
 \multiput(360,333)(0,-4){8}{\line(0,1){2}}
 \put(355,292){\makebox(0,0)[bl]{|marginparwidth|}}
 \put(345,275){\makebox(0,0)[bl]{|marginparsep|}}
 \put( 20, 40){\framebox(120,170){}}
 \put( 20,237){\makebox(120,0)[bl]%
 {\textbf{(c)}~|includefoot|}}
 \put( 20, 60){\line(1,0){120}}
 \put( 20,215){\makebox(120,0)[br]{\gpart{total body}}}
 \put(150,130){\makebox(0,0)[l]{|textheight|}}
 \put(150, 50){\makebox(0,0)[l]{|footskip|}}
 \put( 20, 50){\makebox(120,160)[c]{\gpart{body}}}
 \put( 20, 40){\makebox(120,10)[c]{\gpart{foot}}}
 \put( 20, 65){\makebox(120,10)[c]{|textwidth|}}
 \put(250, 40){\framebox(120,170){}}
 \put(250,237){\makebox(120,0)[bl]%
 {\textbf{(d)}~|includefoot,includemp|}}
 \put(250, 60){\line(1,0){95}}\put(350, 60){\line(1,0){20}}
 \put(250,215){\makebox(120,0)[br]{\gpart{total body}}}
 \put(250, 50){\makebox(95,160)[c]{\gpart{body}}}
 \put(385,130){\makebox(95,0)[cl]%
 {\gpart{\shortstack[l]{marginal\\note}}}}
 \put(250, 40){\makebox(95,10)[c]{\gpart{foot}}}
 \put(250, 65){\makebox(95,0)[bc]{|textwidth|}}
 \put(345, 60){\line(0,1){150}}\put(350, 60){\line(0,1){150}}
 \multiput(360, 130)(4,0){6}{\line(1,0){2}}
 \multiput(348, 63)(0,-4){12}{\line(0,1){2}}
 \multiput(360, 63)(0,-4){8}{\line(0,1){2}}
 \put(355,22){\makebox(0,0)[bl]{|marginparwidth|}}
 \put(345, 5){\makebox(0,0)[bl]{|marginparsep|}}
 \end{picture}}
 \caption[Sample layouts for \gpart{total body} with different 
    layout modes]{%
 \begin{minipage}[t]{.8\textwidth}\small
   Sample layouts for \gpart{total body} with different switches.
   (a) |includeheadfoot|, (b) |includeall|, (c) |includefoot|
    and (d) |includefoot,includemp|. 
   If |reversemp| is set to |true|, the location of the
   marginal notes are swapped on every page.
   Option |twoside| swaps both margins and marginal notes on verso pages.
   Note that the marginal note, if any, is printed despite
   |ignoremp| or |includemp=false| and overrun the page in some cases.\\
   使用不同开关的\gpart{总体正文}的示例布局。
   (a) |includeheadfoot|,(b) |includeall|,(c) |includefoot|
   和 (d) |includefoot,includemp|。
   如果将 |reversemp| 设置为 |true|,则边注的位置将在每一页上交换。
   选项 |twoside| 在背面页面上交换两边的边距和边注。
   请注意,如果有边注,则会在某些情况下打印出来,尽管设置了 |ignoremp| 或 |includemp=false|,并可能超出页面。
 \end{minipage}}
 \label{fig:modes}
\end{figure}

\columnratio{0.55}
\begin{paracol}{2}
The following options can specify body and margins simultaneously with
three comma-separated values in braces.
\switchcolumn
以下选项可以同时指定正文和边距,使用花括号内的三个逗号分隔的值。
\end{paracol}

\begin{Options}
\columnratio{0.55}
\begin{paracol}{2}
\item[hdivide] horizontal partitions (left,width,right).
  |hdivide=|\argiii{left margin}{width}{right margin}. 
  Note that you should not specify all of the three parameters.
  The best way of using this option is to specify two of three and 
  leave the rest with null(nothing) or `|*|'. For example, when you set
  |hdivide={2cm,15cm, }|, the margin from the right-side edge of page
  will be determined calculating |paperwidth-2cm-15cm|.
\switchcolumn
\item[hdivide] 水平分割 (左边距,宽度,右边距)。
|hdivide=|\argiii{左边距}{宽度}{右边距}。
请注意,不应同时指定这三个参数。
使用此选项的最佳方法是指定其中的两个,并将剩下的一个设为 null (空) 或 `|*|'。例如,当设置 |hdivide={2cm,15cm, }| 时,页面右侧边缘的边距将通过计算 |paperwidth-2cm-15cm| 来确定。
\switchcolumn[0]*
\item[vdivide] vertical partitions (top,height,bottom).
|vdivide=|\argiii{top margin}{height}{bottom margin}.
\switchcolumn
\item[vdivide] 垂直分割 (上边距,高度,下边距)。
|vdivide=|\argiii{上边距}{高度}{下边距}。
\switchcolumn[0]*
\item[divide] |divide=|\vargiii{$A$}{$B$}{$C$} is interpreted  as 
|hdivide=|\vargiii{$A$}{$B$}{$C$} and |vdivide=|\vargiii{$A$}{$B$}{$C$}.
\switchcolumn
\item[divide] |divide=|\vargiii{$A$}{$B$}{$C$} 被解释为 |hdivide=|\vargiii{$A$}{$B$}{$C$} 和 |vdivide=|\vargiii{$A$}{$B$}{$C$}。
\end{paracol}
\end{Options}

% 
\columnratio{0.55}
\begin{paracol}{2}
\subsection{Margin size}\label{sec:margin}
The options specifying the size of the margins are listed below.
\switchcolumn
\subsection{边距大小}
下面列出了指定边距大小的选项。
\end{paracol}

\begin{Options}
\columnratio{0.55}
\begin{paracol}{2}
\switchcolumn
\switchcolumn[0]*\item[left\OR lmargin\OR inner]~\\
left margin (for oneside) or inner margin (for twoside) of 
\gpart{total body}. In other words, the distance between the left (inner)
edge of the paper and that of \gpart{total body}. |left=|\meta{length}.
|inner| has no special meaning, just an alias of |left| and |lmargin|.
\switchcolumn\item[left\OR lmargin\OR inner]~\\
正文的左边距(单页模式)或内部边距(双页模式)。换句话说,纸张左(内)边缘与正文左(内)边缘之间的距离。|left=|\meta{长度}。|inner| 没有特殊含义,只是 |left| 和 |lmargin| 的别名。
%%%
\switchcolumn[0]*\item[right\OR rmargin\OR outer]~\\ 
   right or outer margin of \gpart{total body}. |right=|\meta{length}.
\switchcolumn\item[right\OR rmargin\OR outer]~\\
正文的右边距或外部边距。|right=|\meta{长度}。
\switchcolumn[0]*\item[top\OR tmargin] top margin of the page. |top=|\meta{length}.
   Note this option has nothing to do with the native dimension
   \cs{topmargin}.
\switchcolumn\item[top\OR tmargin] 页面的上边距。|top=|\meta{长度}。
请注意,此选项与原始尺寸 \cs{topmargin} 无关。
\switchcolumn[0]*\item[bottom\OR bmargin]~\\ 
   bottom margin of the page. |bottom=|\meta{length}.
\switchcolumn\item[bottom\OR bmargin]~\\
页面的下边距。|bottom=|\meta{长度}。
\switchcolumn[0]*\item[hmargin] left and right margin.
  |hmargin=|\argii{left margin}{right margin} or |hmargin=|\meta{length}.
\switchcolumn\item[hmargin] 左边距和右边距。
|hmargin=|\argii{左边距}{右边距} 或 |hmargin=|\meta{长度}。
\switchcolumn[0]*\item[vmargin] top and bottom margin.
  |vmargin=|\argii{top margin}{bottom margin} or |vmargin=|\meta{length}.
\switchcolumn\item[vmargin] 上边距和下边距。
|vmargin=|\argii{上边距}{下边距} 或 |vmargin=|\meta{长度}。
\switchcolumn[0]*\item[margin] |margin=|\vargii{$A$}{$B$} is equivalent to 
  |hmargin=|\vargii{$A$}{$B$} and |vmargin=|\vargii{$A$}{$B$}.
  |margin=|$A$ is automatically expanded to |hmargin=|$A$ and |vmargin=|$A$.
\switchcolumn\item[margin] |margin=|\vargii{$A$}{$B$} 等同于
|hmargin=|\vargii{$A$}{$B$} 和 |vmargin=|\vargii{$A$}{$B$}。
|margin=|$A$ 会自动展开为 |hmargin=|$A$ 和 |vmargin=|$A$。
\switchcolumn[0]*\item[hmarginratio]
  horizontal margin ratio of |left| (inner) to |right| (outer). 
  The value of \meta{ratio} should be specified with colon-separated 
  two values. Each value should be a positive integer less than 100
  to prevent arithmetic overflow, e.g., |2:3| instead of |1:1.5|.
  The default ratio is |1:1| for oneside, |2:3| for twoside.
\switchcolumn\item[hmarginratio]
|left|(内)与 |right|(外)的水平边距比例。
\meta{比例} 的值应使用冒号分隔的两个值来指定。
每个值应为小于 100 的正整数,以防止算术溢出,例如使用 |2:3| 而不是 |1:1.5|。
对于单页模式,默认比例是 |1:1|,对于双页模式,默认比例是 |2:3|。
\switchcolumn[0]*\item[vmarginratio]
   vertical margin ratio of |top| to |bottom|. The default ratio is |2:3|.
\switchcolumn\item[vmarginratio]
|top| 与 |bottom| 的垂直边距比例。默认比例是 |2:3|。
\switchcolumn[0]*\item[marginratio\OR ratio]~\\
   horizontal and vertical margin ratios.
  |marginratio=|\argii{horizontal ratio}{vertical ratio} or
  |marginratio=|\meta{ratio}.
\switchcolumn\item[marginratio\OR ratio]\
水平和垂直边距比例。
|marginratio=|\argii{水平比例}{垂直比例} 或
|marginratio=|\meta{比例}。
\switchcolumn[0]*\item[hcentering] sets auto-centering horizontally and is
  equivalent to |hmarginratio=1:1|. It is set to |true| by default for
  oneside. See also |hmarginratio|.
\switchcolumn\item[hcentering] 设置水平自动居中,等同于 |hmarginratio=1:1|。默认情况下,单页模式下设置为 |true|。见 |hmarginratio|。
\switchcolumn[0]*\item[vcentering] sets auto-centering vertically and is
  equivalent to |vmarginratio=1:1|. The default is |false|.
  See also |vmarginratio|.
\switchcolumn\item[vcentering] 设置垂直自动居中,等同于 |vmarginratio=1:1|。默认情况下为 |false|。见 |vmarginratio|。
\switchcolumn[0]*\item[centering] sets auto-centering and is equivalent to
  |marginratio=1:1|. See also |marginratio|. The default is |false|.
  See also |marginratio|.
\switchcolumn\item[centering] 设置自动居中,等同于 |marginratio=1:1|。默认情况下为 |false|。见 |marginratio|。
\switchcolumn[0]*\item[twoside] switches on twoside mode with left and right margins swapped
  on verso pages. The option sets \cs{@twoside} and \cs{@mparswitch} 
  switches. See also |asymmetric|.
\switchcolumn\item[twoside] 打开双面模式,交换背面页面的左右边距。此选项设置 \cs{@twoside} 和 \cs{@mparswitch} 开关。见 |asymmetric|。
\switchcolumn[0]*\item[asymmetric] implements a twosided layout in which margins are
  not swapped on alternate pages (by setting \cs{oddsidemargin} to 
  \cs{evensidemargin} |+| |bindingoffset|) and in which the marginal notes
  stay always on the same side. This option can be used as an alternative
  to the twoside option. See also |twoside|.
\switchcolumn\item[asymmetric] 实现了一个双面布局,其中边距在交替页面上不会交换(通过将 \cs{oddsidemargin} 设置为 \cs{evensidemargin} |+| |bindingoffset|),并且边注始终停留在同一侧。此选项可用作 twoside 选项的替代方案。见 |twoside|。
\switchcolumn[0]*\item[bindingoffset]~\\ removes a specified space 
  from the lefthand-side of the page for oneside or the inner-side for
  twoside. |bindingoffset=|\meta{length}. This is useful if pages 
  are bound by a press binding (glued, stitched, stapled \ldots).
  See Figure~\ref{fig:bindingoffset}.
\switchcolumn\item[bindingoffset]~\\ 从页面的左侧(单页模式或双页模式的内部侧)移除指定的空间。|bindingoffset=|\meta{长度}。如果页面通过压合装订(胶合、缝合、订书钉等)装订,则这很有用。见图\ref{fig:bindingoffset}。

\switchcolumn[0]*\item[hdivide] See description in Section~\ref{sec:body}.
\switchcolumn\item[hdivide] 见第~\ref{sec:body} 节中的描述。
\switchcolumn[0]*\item[vdivide] See description in Section~\ref{sec:body}.
\switchcolumn\item[vdivide] 见第~\ref{sec:body} 节中的描述。
\switchcolumn[0]*\item[divide] See description in Section~\ref{sec:body}.
\switchcolumn\item[divide] 见第~\ref{sec:body} 节中的描述。
\end{paracol}
\end{Options}




\begin{figure}
 \centering\small
 {\unitlength=.65pt
 \begin{picture}(500,270)(0,0)
 \put(20,0){\framebox(170,230){}}
 \put(20,255){\makebox(80,20)[l]{\textbf{a)}~every page for oneside or}}
 \put(20,240){\makebox(80,20)[l]{\hspace{3ex}odd pages for twoside}}
 \put(110,225){\makebox(80,20)[r]{\gpart{paper}}}
 \put(55,37){\framebox(110,170)[tc]{\gpart{total body}}}
 \multiput(38,0)(0,7){33}{\line(0,1){4}}
 \put(38,100){\vector(1,0){17}}\put(55,100){\vector(-1,0){17}}
 \put(60,95){\makebox(80,10)[l]{|left|}}
 \put(60,80){\makebox(80,10)[l]{(|inner|)}}
 \put(165,100){\vector(1,0){25}}\put(190,100){\vector(-1,0){25}}
 \put(195,95){\makebox(80,10)[l]{|right|}}
 \put(195,80){\makebox(80,10)[l]{(|outer|)}}
 \put(20,16){\vector(1,0){18}}
 \put(45,10){\makebox(80,10)[bl]{|bindingoffset|}}
 \put(280,255){\makebox(80,20)[l]{\textbf{b)}~even (back) pages for twoside}}
 \put(280,0){\framebox(170,230){}}
 \put(370,225){\makebox(80,20)[r]{\gpart{paper}}}
 \put(305,37){\framebox(110,170)[tc]{\gpart{total body}}}
 \multiput(432,0)(0,7){33}{\line(0,1){4}}
 \put(280,100){\vector(1,0){25}}\put(305,100){\vector(-1,0){25}}
 \put(310,95){\makebox(80,10)[l]{|outer|}}
 \put(310,80){\makebox(80,10)[l]{(|right|)}}
 \put(415,100){\vector(1,0){17}}\put(432,100){\vector(-1,0){17}}
 \put(373,95){\makebox(80,10)[l]{|inner|}}
 \put(373,80){\makebox(80,10)[l]{(|left|)}}
 \put(450,16){\vector(-1,0){18}}
 \put(330,10){\makebox(80,10)[bl]{|bindingoffset|}}
 \end{picture}}
 \caption[\texttt{bindingoffset} option]{%
  \begin{minipage}[t]{.8\textwidth}\raggedright\small
  The option |bindingoffset| adds the specified length to the inner margin.
  Note that |twoside| option swaps the horizontal margins and the
  marginal notes together with |bindingoffset| on even pages (see
  \textbf{b)}), but |asymmetric| option suppresses the swap of the
  margins and marginal notes (but |bindingoffset| is still swapped).\\
  选项 |bindingoffset| 将指定的长度添加到内边距。请注意,|twoside| 选项会在偶数页上与 |bindingoffset| 一起交换水平边距和边注(参见 \textbf{b)}),但 |asymmetric| 选项会取消交换边距和边注(但 |bindingoffset| 仍然会被交换)。
  \end{minipage}}
 \label{fig:bindingoffset}
\end{figure}

% 
\columnratio{0.55}
\begin{paracol}{2}
\subsection{Native dimensions}\label{sec:dimension}

The options below overwrite \LaTeX\ native dimensions and switches for page
layout (See the right-hand side in Figure~\ref{fig:layout}).
\switchcolumn
\subsection{原生尺寸}
下面的选项可以覆盖\LaTeX 的原生尺寸和页面布局开关(见图~\ref{fig:layout}的右侧)。
\end{paracol}

\begin{Options}
\columnratio{0.55}
\begin{paracol}{2}
\switchcolumn[0]*\item[headheight\OR head]~\\
   modifies \cs{headheight}, height of header.
   |headheight=|\meta{length} or |head=|\meta{length}.
\switchcolumn\item[headheight\OR head]~\\
修改\cs{headheight},页眉的高度。
|headheight=|\meta{length} 或 |head=|\meta{length}。
\switchcolumn[0]*\item[headsep] modifies \cs{headsep}, separation between header and text
   (body). |headsep=|\meta{length}.
\switchcolumn
\item[headsep] 修改\cs{headsep},页眉和正文(主体)之间的距离。
|headsep=|\meta{length}。
\switchcolumn[0]*\item[footskip\OR foot]~\\ modifies \cs{footskip}, distance separation
   between baseline of last line of text and baseline of footer.
   |footskip=|\meta{length} or |foot=|\meta{length}.
\switchcolumn\item[footskip\OR foot]~\\ 修改\cs{footskip},正文最后一行的基线与页脚基线之间的距离。
|footskip=|\meta{length} 或 |foot=|\meta{length}。
\switchcolumn[0]*\item[nohead] eliminates spaces for the head of the page, which is
   equivalent to both \cs{headheight}|=0pt| and \cs{headsep}|=0pt|.
\switchcolumn\item[nohead] 消除页头的空白,等效于 \cs{headheight}|=0pt| 和 \cs{headsep}|=0pt|。
\switchcolumn[0]*\item[nofoot] eliminates spaces for the foot of the page, which is
   equivalent to \cs{footskip}|=0pt|.
\switchcolumn\item[nofoot] 消除页脚的空白,等效于 \cs{footskip}|=0pt|。
\switchcolumn[0]*\item[noheadfoot] equivalent to |nohead| and |nofoot|, which means that
   \cs{headheight}, \cs{headsep} and \cs{footskip} are all set to |0pt|.
\switchcolumn\item[noheadfoot] 等效于 |nohead| 和 |nofoot|,即将 \cs{headheight}、\cs{headsep} 和 \cs{footskip} 都设为 |0pt|。
\switchcolumn[0]*\item[footnotesep] changes the dimension \cs{skip}\cs{footins}, separation
   between the bottom of text body and the top of footnote text.
\switchcolumn\item[footnotesep] 修改 \cs{skip}\cs{footins} 的尺寸,即正文底部与脚注文本顶部之间的距离。
\switchcolumn[0]*\item[marginparwidth\OR marginpar]~\\ 
   modifies \cs{marginparwidth}, width of the marginal notes.
   |marginparwidth=|\meta{length}.
\switchcolumn\item[marginparwidth\OR marginpar]~\\
修改\cs{marginparwidth},边注的宽度。
|marginparwidth=|\meta{length}。
\switchcolumn[0]*\item[marginparsep] modifies \cs{marginparsep}, separation between
   body and marginal notes. |marginparsep=|\meta{length}.
\switchcolumn\item[marginparsep] 修改\cs{marginparsep},正文和边注之间的距离。
|marginparsep=|\meta{length}。
\switchcolumn[0]*\item[nomarginpar] shrinks spaces for marginal notes to 0pt, which
   is equivalent to \cs{marginparwidth}|=0pt| and \cs{marginparsep}|=0pt|.
\switchcolumn\item[nomarginpar] 将边注的空白缩小为0pt,等效于 \cs{marginparwidth}|=0pt| 和 \cs{marginparsep}|=0pt|。
\switchcolumn[0]*\item[columnsep] modifies \cs{columnsep}, the separation between two
   columns in |twocolumn| mode.
\switchcolumn\item[columnsep] 修改\cs{columnsep},在双栏模式下两栏之间的距离。
\switchcolumn[0]*\item[hoffset]  modifies \cs{hoffset}. |hoffset=|\meta{length}.
\switchcolumn\item[hoffset] 修改\cs{hoffset}。 |hoffset=|\meta{length}。
\switchcolumn[0]*\item[voffset]  modifies \cs{voffset}. |voffset=|\meta{length}.
\switchcolumn\item[voffset] 修改\cs{voffset}。 |voffset=|\meta{length}。
\switchcolumn[0]*\item[offset] horizontal and vertical offset.\\
   |offset=|\argii{hoffset}{voffset} or |offset=|\meta{length}.
\switchcolumn\item[offset] 水平和垂直偏移。~\\
|offset=|\argii{hoffset}{voffset} 或 |offset=|\meta{length}。
\switchcolumn[0]*\item[twocolumn] sets |twocolumn| mode with \cs{@twocolumntrue}.
  |twocolumn=false| denotes onecolumn mode with\cs{@twocolumnfalse}.
  Instead of |twocolumn=false|, you can specify |onecolumn| (which
  defaults to |true|)
\switchcolumn\item[twocolumn] 使用 \cs{@twocolumntrue} 设置双栏模式。|twocolumn=false| 表示单栏模式,等效于 \cs{@twocolumnfalse}。你也可以使用 |onecolumn| 来代替 |twocolumn=false|(默认为 |true|)。
\switchcolumn[0]*\item[onecolumn] works as |twocolumn=false|. On the other hand,
  |onecolumn=false| is equivalent to |twocolumn|. 
\switchcolumn\item[onecolumn] 等效于 |twocolumn=false|。另一方面,|onecolumn=false| 等效于 |twocolumn|。
\switchcolumn[0]*\item[twoside] sets both \cs{@twosidetrue} and \cs{@mparswitchtrue}.
  See Section~\ref{sec:margin}.
\switchcolumn\item[twoside] 同时设置 \cs{@twosidetrue} 和 \cs{@mparswitchtrue}。见第~\\ref{sec:margin}节。
\switchcolumn[0]*\item[textwidth] sets \cs{textwidth} directly. See Section~\ref{sec:body}.
\switchcolumn\item[textwidth] 直接设置 \cs{textwidth}。见第\ref{sec:body}节。
\switchcolumn[0]*\item[textheight] sets \cs{textheight} directly. See Section~\ref{sec:body}.
\switchcolumn\item[textheight] 直接设置 \cs{textheight}。见第~\\ref{sec:body}节。
\switchcolumn[0]*\item[reversemp\OR reversemarginpar]~\\
  makes the marginal notes appear in the left (inner) margin with
  \cs{@reversemargintrue}. The option doesn't change |includemp| mode.
  It's set |false| by default.
  \switchcolumn\item[reversemp\OR reversemarginpar]~\\ 这个选项使用 \cs{@reversemargintrue} 让边注出现在左侧(内侧)边距。这个选项不会改变 |includemp| 模式。默认情况下,该选项为 |false|。
\end{paracol}
\end{Options}
 
% 
\columnratio{0.55}
\begin{paracol}{2}
\subsection{Drivers}\label{sec:drivers}
The package supports drivers |dvips|, |dvipdfm|, |pdftex|, |luatex|,
|xetex| and |vtex|. You can also set |dvipdfm| for \textsf{dvipdfmx} and
\textsf{xdvipdfmx} The options |dvipdfmx| and |xdvipdfmx| are also supported
as aliases for the |dvipdfm| option.
|pdftex| for \textsf{pdflatex}, and |vtex| for
V\TeX{} environment.
The driver options are exclusive. The driver can be set by either
|driver=|\meta{driver name} or any of the drivers directly like |pdftex|.
By default, \Gm\ guesses the driver appropriate to the system
in use. Therefore, you don't have to set a driver in most cases.
However, if you want to use |dvipdfm|, you should specify it explicitly.
\switchcolumn
\subsection{Driver驱动}
这个选项用于设置驱动程序(driver),可以使用的驱动程序有:|dvips|、|dvipdfm|、|pdftex|、|luatex|、|xetex| 和 |vtex|。你也可以将 |dvipdfmx| 用于 \textsf{dvipdfmx} 和 \textsf{xdvipdfmx},选项 |pdftex| 用于 \textsf{pdflatex},选项 |vtex| 用于 V\TeX{}。这些驱动程序选项是互斥的。你可以通过 |driver=|\meta{driver name} 或直接使用驱动程序进行设置,如 |pdftex|。默认情况下,\Gm 会猜测适合当前系统的驱动程序,所以在大多数情况下你不需要设置驱动程序。但是,如果你想使用 |dvipdfm|,你需要明确指定它。
\end{paracol}

\begin{Options}
\columnratio{0.55}
\begin{paracol}{2}
\switchcolumn
\switchcolumn[0]*\item[\onlypre driver] specifies the driver with |driver=|\meta{driver name}. 
|dvips|, |dvipdfm|, |pdftex|, |luatex|, |vtex|, |xetex|, |auto| and |none| are
available as a driver name. The names except for |auto| and |none| can
be specified directly with the name without |driver=|.
|driver=auto| makes the auto-detection work whatever the previous setting is. 
|driver=none| disables the auto-detection and sets no driver, which
may be useful when you want to let other package work out the driver
setting. For example, if you want to use \textsf{crop} package with \Gm,
you should call |\usepackage[driver=none]{geometry}| before
the \textsf{crop} package.
\switchcolumn\item[\onlypre driver] 使用 |driver=|\meta{driver name} 来指定驱动程序。
可用的驱动程序名称有 |dvips|、|dvipdfm|、|pdftex|、|luatex|、|vtex|、|xetex|、|auto| 和 |none|。
除了 |auto| 和 |none| 外的名称可以直接使用不带 |driver=| 的名称指定。
|driver=auto| 会使自动检测驱动程序的功能生效,无论之前的设置是什么。
|driver=none| 会禁用自动检测,并且不设置驱动程序,这在你想让其他包来决定驱动程序设置时可能会有用。
例如,如果你想在使用 \textsf{crop} 宏包时使用 \Gm{},你应该在 \textsf{crop} 宏包之前调用 |\usepackage[driver=none]{geometry}|。
\switchcolumn[0]*\item[\onlypre dvips] writes the paper size in dvi output with the \cs{special}
    macro. If you use \textsl{dvips} as a DVI-to-PS driver,
    for example, to print a document with |\geometry{a3paper,landscape}|
    on A3 paper in landscape orientation, you don't need options
    ``|-t a3 -t landscape|'' to \textsl{dvips}. 
\switchcolumn\item[\onlypre dvips] 使用 \cs{special} 宏来在 dvi 输出中写入纸张大小。
如果你使用 \textsl{dvips} 作为 DVI 到 PS 的驱动程序,例如在 A3 纸上打印 |\geometry{a3paper,landscape}| 的文档时,
你不需要在 \textsl{dvips} 中使用 ``|-t a3 -t landscape|'' 选项。
\switchcolumn[0]*\item[\onlypre dvipdfm] works like |dvips| except for landscape correction.
     You can set this option when using \textsf{dvipdfmx} and
     \textsf{xdvipdfmx} to process the dvi output.
\switchcolumn\item[\onlypre dvipdfm] 与 |dvips| 相同,只是不会进行横向纠正。
当使用 \textsf{dvipdfmx} 和 \textsf{xdvipdfmx} 处理 dvi 输出时,可以设置此选项。
\switchcolumn[0]*\item[\onlypre pdftex] sets \cs{pdfpagewidth} and \cs{pdfpageheight}
     internally.
\switchcolumn\item[\onlypre pdftex] 内部设置 \cs{pdfpagewidth} 和 \cs{pdfpageheight}。
\switchcolumn[0]*\item[\onlypre luatex] sets \cs{pagewidth} and \cs{pageheight} internally.
\switchcolumn\item[\onlypre luatex] 内部设置 \cs{pagewidth} 和 \cs{pageheight}。
\switchcolumn[0]*\item[\onlypre xetex] is the same as |pdftex| except for ignoring
    |\pdf{h,v}origin| undefined in \XeLaTeX{}. This option is introduced in
    the version 5. Note that `geometry.cfg' in \TeX{} Live, which disables the
    auto-detection routine and sets |pdftex|, is no longer necessary,
    but has no problem even though it's left undeleted.
    Instead of |xetex|, you can specify |dvipdfm| with \XeLaTeX{}
    if you want to use specials of dvipdfm \XeTeX{} supports.
\switchcolumn\item[\onlypre xetex] 与 |pdftex| 相同,只是在 \XeLaTeX{} 中忽略 |\pdf{h,v}origin| 未定义。
此选项在版本 5 中引入。
注意,在 \TeX{} Live 中的 `geometry.cfg',它禁用了自动检测例程并设置为 |pdftex|,
虽然不再需要,但即使未删除也没有问题。
如果你想在 \XeLaTeX{} 中使用 dvipdfm \XeTeX{} 支持的特殊命令,
可以用 |dvipdfm| 替代 |xetex|。
\switchcolumn[0]*\item[\onlypre vtex] sets dimensions \cs{mediawidth} and \cs{mediaheight}
    for V\TeX. When this driver is selected (explicitly or
    automatically), \Gm\ will auto-detect which output mode
    (DVI, PDF or PS) is selected in V\TeX, and do proper
    settings for it.
\switchcolumn\item[\onlypre vtex] 为 V\TeX{} 设置 \cs{mediawidth} 和 \cs{mediaheight} 尺寸。
当选择了此驱动程序(显式或自动),\Gm{} 将自动检测在 V\TeX{} 中选择的输出模式(DVI、PDF 或 PS)并进行适当的设置。
\end{paracol}
\end{Options}

\columnratio{0.55}
\begin{paracol}{2}
    If explicit driver setting is mismatched with the typesetting program
    in use, the default driver |dvips| would be selected.
    \switchcolumn 如果显式设置的驱动程序与正在使用的排版程序不匹配,则会选择默认的驱动程序 |dvips|。
\end{paracol}
 
% 

\columnratio{0.55}
\begin{paracol}{2}
\subsection{Other options}
The other useful options are described here.
\switchcolumn
\subsection{其他选项}
这里介绍了其他有用的选项。
\end{paracol}

\begin{Options}
\columnratio{0.55}
\begin{paracol}{2}
\switchcolumn[0]*\item[\onlypre verbose] displays the parameter results on the terminal.
  |verbose=false| (default) still puts them into the log file.
\switchcolumn\item[\onlypre verbose] 在终端上显示参数结果。|verbose=false|(默认值)仍然会将结果放入日志文件中。
\switchcolumn[0]*\item[\onlypre reset] sets back the layout dimensions and switches to the
  settings before \Gm\ is loaded. Options given in 
  |geometry.cfg| are also cleared.
  Note that this cannot reset |pass| and |mag| with |truedimen|.
  |reset=false| has no effect and cannot cancel the previous
  |reset|(|=true|) if any. For example, when you go
  \begin{quote}
    |\documentclass[landscape]{article}|\\
    |\usepackage[twoside,reset,left=2cm]{geometry}|
  \end{quote}
  with |\ExecuteOptions{scale=0.9}| in |geometry.cfg|,
  then as a result, |landscape| and |left=2cm| remain effective,
  and |scale=0.9| and |twoside| are ineffective.
\switchcolumn\item[\onlypre reset] 将布局尺寸设置回到加载 \Gm\ 之前的设置。
|geometry.cfg| 中给出的选项也会被清除。
注意,这不能重置带有 |truedimen| 的 |pass| 和 |mag|。
|reset=false| 没有效果,也无法取消之前的 |reset|(|=true|)设置。
例如,当你使用 |\ExecuteOptions{scale=0.9}| 在 |geometry.cfg| 中进行如下设置:
\begin{quote}
|\documentclass[landscape]{article}|\\
|\usepackage[twoside,reset,left=2cm]{geometry}|
\end{quote}
那么结果是 |landscape| 和 |left=2cm| 仍然生效,
而 |scale=0.9| 和 |twoside| 不生效。
\switchcolumn[0]*\item[\onlypre mag] sets magnification value (\cs{mag}) and automatically modifies 
  \cs{hoffset} and \cs{voffset} according to the magnification.
  |mag=|\meta{value}. Note that \meta{value} should be an integer value
  with 1000 as a normal size. For example, |mag=1414| with |a4paper|
  provides an enlarged print fitting in |a3paper|, which is $1.414$
  (=$\sqrt{2}$) times larger than |a4paper|. Font enlargement needs extra
  disk space. \textbf{Note that setting |mag| should precede any other
  settings with `true' dimensions, such as  |1.5truein|, |2truecm|
  and so on.} See also |truedimen| option.
\switchcolumn\item[\onlypre mag] 设置放大倍数 (\cs{mag}) 并根据放大倍数自动修改 \cs{hoffset} 和 \cs{voffset}。
|mag=|\meta{value}。注意,\meta{value} 应该是一个整数值,以 1000 作为正常大小。
例如,|mag=1414| 结合 |a4paper| 提供了一个适合于 |a3paper| 的放大打印,即比 |a4paper| 大 $1.414$(=$\sqrt{2}$)倍。
字体的放大需要额外的磁盘空间。
\textbf{注意,设置 |mag| 应该在任何其他带有真实尺寸的设置之前,例如 |1.5truein|、|2truecm| 等。}
还请参见 |truedimen| 选项。
\switchcolumn[0]*\item[\onlypre truedimen] changes all internal explicit dimension values into 
  \textit{true} dimensions, e.g., |1in| is changed to |1truein|.
  Typically this option will be used together with |mag| option. Note that
  this is ineffective against externally specified dimensions. For example,
  when you set ``\texttt{mag=1440, margin=10pt, truedimen}'', margins are
  not `true' but magnified. If you want to set exact margins, you should
  set like ``\texttt{mag=1440, margin=10truept, truedimen}'' instead.
\switchcolumn\item[\onlypre truedimen] 将所有内部显式尺寸值更改为\textit{真实}尺寸,例如将 |1in| 更改为 |1truein|。
通常,此选项将与 |mag| 选项一起使用。注意,这对于外部指定的尺寸是无效的。
例如,当你设置 \texttt{mag=1440, margin=10pt, truedimen}'' 时,边距不是 `true',而是被放大了。   如果你想设置精确的边距,应该改为 \texttt{mag=1440, margin=10truept, truedimen}''。
\switchcolumn[0]*\item[\onlypre pass] disables all of the geometry options and calculations
  except |verbose| and |showframe|. It is order-independent and can be
  used for checking out the page layout of the documentclass, other
  packages and manual settings without \Gm.
\switchcolumn\item[\onlypre pass] 禁用所有的几何选项和计算,除了 |verbose| 和 |showframe|。
它是无序的,可以用于检查文档类、其他包和手动设置的页面布局,而不使用 \Gm。
\switchcolumn[0]*\item[\onlypre showframe] shows visible frames for the text area and page,
  and the lines for the head and foot on the first page.
\switchcolumn\item[\onlypre showframe] 在文本区域和页面上显示可见的框架,以及第一页上的页眉和页脚线条。
\switchcolumn[0]*\item[\onlypre showcrop] prints crop marks at each corner of user-specified
layout area.
\switchcolumn\item[\onlypre showcrop] 在用户指定的布局区域的每个角上打印出裁剪标记。
\end{paracol}
\end{Options}
   
% 
 \section{Processing options\hfill 处理选项}\label{sec:process}


 
% \columnratio{0.55}
\begin{paracol}{2}
\subsection{Order of loading}\label{sec:loadorder}

If there's |geometry.cfg| somewhere \TeX{} can find it, \Gm\ loads
it first. For example, in |geometry.cfg| you may write
|\ExecuteOptions{a4paper}|, which specifies A4 paper as the default
paper. Basically you can use all the options defined in \Gm\ with
|\ExecuteOptions{}|.

The order of loading in the preamble of your document is as follows:
\switchcolumn

\subsection{加载顺序}\label{sec:loadorder}

如果\TeX{}能找到|geometry.cfg|文件,\Gm\ 会首先加载它。例如,在|geometry.cfg|文件中,您可以写上|\ExecuteOptions{a4paper}|,这将把A4纸设置为默认纸张。基本上,您可以使用\Gm\ 中定义的所有选项与|\ExecuteOptions{}|一起使用。

在您的文档的导言部分加载的顺序如下:
\end{paracol}

\begin{enumerate}
\columnratio{0.55}
\begin{paracol}{2}
\switchcolumn[0]*\item |geometry.cfg| if it exists.
\switchcolumn \item 如果存在|geometry.cfg|,加载之。
\switchcolumn[0]*\item Options specified with |\documentclass[|\meta{options}|]{...}|.
\switchcolumn \item 使用|\documentclass[|\meta{options}|]{...}|指定的选项。
\switchcolumn[0]*\item Options specified with |\usepackage[|\meta{options}|]{geometry}|
\switchcolumn \item 使用|\usepackage[|\meta{options}|]{geometry}|指定的选项。
\switchcolumn[0]*\item Options specified with |\geometry{|\meta{options}|}|, which
can be called multiple times. (|reset| option will cancel the
specified options ever given in  |\usepackage{geometry}| or
|\geometry|.)
\switchcolumn
\item 使用|\geometry{|\meta{options}|}|指定的选项,可以多次调用。(|reset|选项将取消在|\usepackage{geometry}|或|\geometry|中指定的选项。)
\end{paracol}
\end{enumerate}
% \columnratio{0.55}
\begin{paracol}{2}
\subsection{Order of options}\label{sec:optionorder}

The specification of \Gm\ options is order-independent,
and overwrites the previous one for the same setting.
For example, 
\switchcolumn
\subsection{选项的顺序}

\Gm\ 选项的指定是无序的,并且会覆盖先前相同设置的选项。
例如,

\switchcolumn[0]*
\begin{center}
|[left=2cm, right=3cm]| is equivalent to 
|[right=3cm, left=2cm]|.
\end{center}
\switchcolumn
\begin{center}
|[left=2cm, right=3cm]| 等同于
|[right=3cm, left=2cm]|.
\end{center}

\switchcolumn[0]*
 The options called multiple times overwrite the previous
 settings. For example, 
 \begin{center}
  |[verbose=true, verbose=false]| results in |verbose=false|. 
 \end{center}
 |[hmargin={3cm,2cm}, left=1cm]| is the same as |hmargin={1cm,2cm}|, 
where the left (or inner) margin is overwritten by |left=1cm|. 
\switchcolumn
多次调用的选项会覆盖先前的设置。例如,
\begin{center}
|[verbose=true, verbose=false]| 的结果是 |verbose=false|。
\end{center}
|[hmargin={3cm,2cm}, left=1cm]| 等同于 |hmargin={1cm,2cm}|,其中左侧(或内侧)边距被 |left=1cm| 覆盖。        

\switchcolumn[0]*
 |reset| and |mag| are exceptions.
 The |reset| option removes all the geometry options (except |pass|)
 before it. If you set
 \begin{quote}
 |\documentclass[landscape]{article}|\\
 |\usepackage[margin=1cm,twoside]{geometry}|\\
 |\geometry{a5paper, reset, left=2cm}|
 \end{quote}
\switchcolumn
|reset| 和 |mag| 是例外。
|reset| 选项会在其之前删除所有的几何选项(除了 |pass|)。如果您设置
\begin{quote}
|\documentclass[landscape]{article}|\\
|\usepackage[margin=1cm,twoside]{geometry}|\\
|\geometry{a5paper, reset, left=2cm}|
\end{quote}

\switchcolumn[0]*
 then |margin=1cm|, |twoside| and |a5paper| are removed, and 
 is eventually equivalent to
 \begin{quote}
 |\documentclass[landscape]{article}|\\
 |\usepackage[left=2cm]{geometry}|
 \end{quote}
 \switchcolumn
 那么 |margin=1cm|、|twoside| 和 |a5paper| 被删除,并最终等同于
\begin{quote}
|\documentclass[landscape]{article}|\\
|\usepackage[left=2cm]{geometry}|
\end{quote}
\switchcolumn[0]*
 The |mag| option should be set in advance of any other settings with
 `true' length, such as |left=1.5truecm|, |width=5truein| and so on.
 The |\mag| primitive can be set before this package is called.
\switchcolumn
|mag| 选项应该在任何其他带有 `true' 长度的设置之前进行设置,例如 |left=1.5truecm|、|width=5truein| 等等。
|\mag| 原语可以在调用此宏包之前进行设置。

\end{paracol}

% 
\columnratio{0.55}
\begin{paracol}{2}
\subsection{Priority}\label{sec:priority}

There are several ways to set dimensions of the \gpart{body}:
|scale|, |total|, |text| and |lines|. The \Gm\ package gives higher
priority to the more concrete specification. Here is the priority
rule for \gpart{body}.

\switchcolumn
\subsection{优先级}

有多种方式可以设置\gpart{正文}的尺寸:|scale|、|total|、|text| 和 |lines|。
\Gm\ 宏包对更具体的规格给予更高的优先级。以下是\gpart{正文}的优先级规则。

\switchcolumn[0]*
 \[\begin{array}{c}
 \textrm{priority:}\qquad\textrm{low}\quad
    \longrightarrow\quad\textrm{high}\\[1em]
 \left\{\begin{array}{l}|hscale|\\|vscale|\\|scale|
        \end{array}\right\} <
 \left\{\begin{array}{l}|width|\\|height|\\|total|
        \end{array}\right\} <
 \left\{\begin{array}{l}|textwidth|\\|textheight|
         \\|text|\end{array}\right\} < |lines|.
 \end{array}\]

\switchcolumn
 \[\begin{array}{c}
 \textrm{优先级:}\qquad\textrm{低}\quad
    \longrightarrow\quad\textrm{高}\\[1em]
 \left\{\begin{array}{l}|hscale|\\|vscale|\\|scale|
        \end{array}\right\} <
 \left\{\begin{array}{l}|width|\\|height|\\|total|
        \end{array}\right\} <
 \left\{\begin{array}{l}|textwidth|\\|textheight|
         \\|text|\end{array}\right\} < |lines|.
 \end{array}\]


 \switchcolumn[0]*
 For example, 
 \begin{quote}
  |\usepackage[hscale=0.8, textwidth=7in, width=18cm]{geometry}|
 \end{quote}
\switchcolumn
例如,
\begin{quote}
\begin{minted}{latex}
\usepackage[hscale=0.8, textwidth=7in, width=18cm]{geometry}
\end{minted}    
\end{quote}

\switchcolumn[0]*
is the same as |\usepackage[textwidth=7in]{geometry}|. Another example:
\begin{quote}
|\usepackage[lines=30, scale=0.8, text=7in]{geometry}|
\end{quote}
\switchcolumn
等同于 |\usepackage[textwidth=7in]{geometry}|。另一个例子:
\begin{quote}
\begin{minted}{latex}
\usepackage[lines=30, scale=0.8, text=7in]{geometry}
\end{minted}    
\end{quote}
\switchcolumn[0]
results in \texttt{[lines=30, textwidth=7in]}.
\switchcolumn
的结果是 \texttt{[lines=30, textwidth=7in]}.
\end{paracol}

 
% \columnratio{0.55}
\begin{paracol}{2}
 \subsection{Defaults}\label{sec:defaults}

 This section sums up the default settings for the auto-completion
 described later.
 
 The default vertical margin ratio is $2/3$, namely,
 \begin{equation}
  |top| : |bottom| = 2 : 3 \qquad\textit{default}.
 \end{equation}
\switchcolumn
\subsection{默认设置}

本节总结了后面描述的自动补全的默认设置。
 
默认的垂直边距比例是 $2/3$,即
 \begin{equation}
  |top| : |bottom| = 2 : 3 \qquad\textit{default}.
 \end{equation}

 \switchcolumn[0]*
 As for the horizontal margin ratio, the default value depends on
 whether the document is onesided or twosided,
 \begin{equation}
  |left|\;(|inner|) : |right|\;(|outer|) 
       = \left\{ \begin{array}{ll}
              1 : 1 \qquad\textit{default for oneside},\\
              2 : 3 \qquad\textit{default for twoside}.
         \end{array}\right.
 \end{equation}
\switchcolumn
至于水平边距比例,默认值取决于文档是单面还是双面的,
\begin{equation}
|left|\;(|inner|) : |right|\;(|outer|) 
    = \left\{ \begin{array}{ll}
            1 : 1 \qquad\textit{单面默认},\\
            2 : 3 \qquad\textit{双面默认}.
        \end{array}\right.
 \end{equation}

 \switchcolumn[0]*
 Obviously the default horizontal margin ratio for oneside is `centering'.

 The \Gm\ package has the following default setting for
 \textit{onesided} documents:

 \switchcolumn
 显然,单面的默认水平边距比例是“居中”。

 对于\Gm\ 宏包,默认的设置对于\textit{单面}文档如下:
\end{paracol}

\begin{itemize}\setlength{\itemsep}{-.5\parsep}
    \columnratio{0.55}
    \begin{paracol}{2}
\item |scale=0.7| (\gpart{body} is $0.7 \times \gpart{paper}$)
\switchcolumn\item |scale=0.7| (\gpart{正文}是 $0.7 \times \gpart{纸张}$)
\switchcolumn[0]*
\item |marginratio={1:1, 2:3}| (1:1 for horizontal and 2:3 for vertical margins)
\switchcolumn\item |marginratio={1:1, 2:3}| (水平边距比例为 1:1,垂直边距比例为 2:3)
\switchcolumn[0]*
\item |ignoreall| (the header, footer, marginal notes are excluded
when calculating the size of \gpart{body}.)
\switchcolumn\item |ignoreall| (在计算\gpart{正文}的大小时,头部、底部和边注都被排除在外。)
\end{paracol}
\end{itemize}

\columnratio{0.55}
\begin{paracol}{2}
 For \textit{twosided} document with |twoside| option, the default
 setting is the same as \textit{onesided} except that the horizontal
 margin ratio is set to |2:3| as well.

 Additional options overwrite the previous specified dimensions. 
\switchcolumn
对于带有 |twoside| 选项的\textit{双面}文档,默认设置与\textit{单面}相同,只是水平边距比例也设置为 |2:3|。

附加选项会覆盖先前指定的尺寸。
\end{paracol} 
% 
\columnratio{0.55}
\begin{paracol}{2}
    \subsection{Auto-completion} \label{sec:rules}
\switchcolumn
\subsection{自动补全} 

\switchcolumn[0]*
Figure~\ref{fig:specrule} shows schematically how many specification
patterns exist and how to solve the ambiguity of the
specifications. Each axis shows the numbers of lengths
explicitly specified for body and margins. \Ss($m$,$b$) presents the
specification with a set of numbers $(\gpart{margin},\gpart{body})=(m,b)$.
\switchcolumn
图~\ref{fig:specrule} 简要展示了有多少种规格模式以及如何解决规格的模糊性。
每个轴表示正文和边距明确指定的长度的数量。$\text{S}_{(m,b)}$ 表示一组数字 $(\gpart{边距},\gpart{正文})=(m,b)$ 的规格。
\switchcolumn[0]*    
For example, the specification |width=14cm, left=3cm| is categorized
into \Ss(1,1), which is an adequate specification. If you add
|right=4cm|, it would be in \Ss(2,1) and overspecified. 
If only |width=14cm| is given, it's in \Ss(0,1), underspecified. 
\switchcolumn 例如,规格 |width=14cm, left=3cm| 归类为 $\text{S}{(1,1)}$,这是一个合适的规格。
如果添加 |right=4cm|,则属于 $\text{S}{(2,1)}$,即过度规定。
如果只给出 |width=14cm|,则属于 $\text{S}_{(0,1)}$,即不足规定。
\switchcolumn[0]*    
The \Gm\ package has the auto-completion mechanism, in which
if the layout parameters are underspecified or overspecified,
\Gm\ works out the ambiguity using the defaults and other
relations. Here are the specifications and the completion rules.
\switchcolumn\Gm\ 宏包具有自动补全机制,如果布局参数不足或过度规定,\Gm\ 会使用默认值和其他关系解决模糊性。以下是规格和补全规则。  
\end{paracol}



 \begin{figure}
  \centering
  {\unitlength=1pt
  \begin{picture}(400,150)(40,0)
  \put(1,49){\makebox(90,49)[r]{\large 0}}
  \put(1,1){\makebox(70,99)[r]{\large \gpart{body}}}
  \put(1,1){\makebox(90,49)[r]{\large 1}}
  \put(100,100){\makebox(99,20){\large 0}}
  \put(100,50){\framebox(99,49){}}
  \put(100,80){\makebox(99,15){\Ss(0,0)}}
  \put(100,50){\makebox(99,49){use |scale|}}
  {\linethickness{1pt}%
  \put(150,35){\line(0,1){30}}
  \put(150,35){\line(-1,3){4}}
  \put(150,35){\line(1,3){4}}}
  \put(100,0){\framebox(99,49){}}
  \put(100,2){\makebox(99,15){\Ss(0,1)}}
  \put(100,0){\makebox(99,49){use |marginratio|}}
  \put(200,120){\makebox(99,12){\large \gpart{margin}}}
  \put(200,100){\makebox(99,20){\large 1}}
  \put(200,50){\framebox(99,49){use |scale|}}
  \put(200,55){\makebox(89,10)[r]{\scriptsize\shortstack[l]{if |ratio|\\specified}}}
  \put(200,2){\makebox(99,15){\Ss(1,1)}}
  {\linethickness{1pt}%
  \put(250,35){\line(0,1){30}}
  \put(250,35){\line(-1,3){4}}
  \put(250,35){\line(1,3){4}}}
  {\linethickness{1pt}%
  \put(225,25){\line(-1,0){35}}
  \put(225,25){\line(-3,-1){12}}
  \put(225,25){\line(-3,1){12}}}
  \put(200,80){\makebox(99,15){\Ss(1,0)}}
  \put(200,0){\framebox(99,49){\textcolor{red}{\textit{solvable}}}}
  \put(300,100){\makebox(99,20){\large 2}}
  \put(300,80){\makebox(99,15){\Ss(2,0)}}
  \put(300,50){\framebox(99,49){\textcolor{red}{\textit{solvable}}}}
  \put(300,0){\framebox(99,49){forget |body|}}
  \put(300,2){\makebox(99,15){\Ss(2,1)}}
  {\linethickness{1pt}%
  \multiput(290,65)(5,0){6}{\line(1,0){3}}
  \put(320,65){\line(-3,-1){12}}
  \put(320,65){\line(-3,1){12}}}
  {\linethickness{1pt}%
  \put(350,65){\line(0,-1){30}}
  \put(350,65){\line(-1,-3){4}}
  \put(350,65){\line(1,-3){4}}}
  \end{picture}}
  \caption[Specifications and completion rules]{%
  \begin{minipage}[t]{.7\textwidth}\raggedright\small
  Specifications \Ss(0,0) to \Ss(2,1) and the completion rules
  (arrows). Column and row numbers denote the number of explicitly
  specified lengths for margin and body respectively. \Ss($m$,$b$) denote a
  specification with a set of the numbers $(\gpart{margin},\gpart{body})=(m,b)$. \\
  规格 \Ss(0,0) 到 \Ss(2,1) 和补全规则(箭头)。列和行号分别表示边距和正文明确指定长度的数量。$\text{S}_{(m,b)}$ 表示一组数字 $(\gpart{边距},\gpart{正文})=(m,b)$ 的规格。
  \end{minipage}}
  \label{fig:specrule}
 \end{figure}

\begin{Spec}
\columnratio{0.55}
\begin{paracol}{2}
\switchcolumn[0]*\item[\Ss(0,0)]
 Nothing is specified. The \Gm\ package sets \gpart{body} with the
 default |scale| ($=0.7$). \\ For example, |width| is set to be
 $|0.7|\times|layoutwidth|$. Note that by default |layoutwidth| 
 and |layoutheight| will be equal to |\paperwidth| and |\paperheight|
 respectively.
 Thus \Ss(0,0) goes to \Ss(0,1). See \Ss(0,1).
 \bigskip
 \switchcolumn
 \item[\Ss(0,0)]
 未指定任何内容。 \Gm\ 宏包使用默认值 |scale|($=0.7$)设置 \gpart{正文}。例如,|width| 被设置为 $|0.7|\times|layoutwidth|$。需要注意的是,默认情况下,|layoutwidth| 和 |layoutheight| 分别等于 |\paperwidth| 和 |\paperheight|。因此,\Ss(0,0) 转向 \Ss(0,1)。请参见 \Ss(0,1)。
 \bigskip

\switchcolumn[0]*\item[\Ss(0,1)]
 Only \gpart{body} is specified, such as |width=7in|, |lines=20|,
 |body={20cm,24cm}|, |scale=0.9| and so forth.
 Then \Gm\ sets margins with the margin ratio.
 If the margin ratio is not specified, the default is used.
 The default vertical margin ratio is defined as
 \begin{equation}
  |top| : |bottom| = 2 : 3 \qquad\textit{default}.
 \end{equation}
 As for the horizontal margin ratio, the default value depends on
 whether the document is onesided or twosided,
 \begin{equation}
  |left|\;(|inner|) : |right|\;(|outer|) 
       = \left\{ \begin{array}{ll}
              1 : 1 \qquad\textit{default for oneside},\\
              2 : 3 \qquad\textit{default for twoside}.
         \end{array}\right.
 \end{equation}
 For example, if |height=22cm| is specified on A4 paper, 
 \Gm\ calculates |top| margin as follows:
 \begin{equation}
   \begin{array}{ll}
   |top| &= ( |layoutheight| - |height| ) \times 2/5 \\
         &= (29.7-22)\times2/5 = 3.08\textrm{(cm)}
   \end{array}
 \end{equation}
 Thus |top| margin and body |height| have been determined, the
 specification for the vertical goes to \Ss(1,1) and
 all the parameters can be solved.
 \bigskip
 \switchcolumn
 \item[\Ss(0,1)]
 只指定了 \gpart{正文},例如 |width=7in|,|lines=20|,|body={20cm,24cm}|,|scale=0.9| 等。然后,\Gm\ 使用边距比例设置边距。如果未指定边距比例,则使用默认值。默认的垂直边距比例定义为
 \begin{equation}
  |top| : |bottom| = 2 : 3 \qquad\textit{默认值}.
 \end{equation}
 至于水平边距比例,其默认值取决于文档是单面还是双面的情况,
 \begin{equation}
  |left|\;(|inner|) : |right|\;(|outer|) 
       = \left\{ \begin{array}{ll}
              1 : 1 \qquad\textit{单面的默认值},\\
              2 : 3 \qquad\textit{双面的默认值}.
         \end{array}\right.
 \end{equation}
 例如,在 A4 纸上指定 |height=22cm|,\Gm\ 计算 |top| 边距如下:
 \begin{equation}
   \begin{array}{ll}
   |top| &= ( |layoutheight| - |height| ) \times 2/5 \\
         &= (29.7-22)\times2/5 = 3.08\textrm{(cm)}
   \end{array}
 \end{equation}
 因此,确定了 |top| 边距和正文 |height|,垂直规格转到 \Ss(1,1),所有参数都可以解决。
 \bigskip

\switchcolumn[0]*\item[\Ss(1,0)]
 Only one margin is specified, such as |bottom=2cm|, |left=1in|,
 |top=3cm|, and so forth.
\switchcolumn
\item[\Ss(1,0)]只指定了一个边距,例如 |bottom=2cm|,|left=1in|,|top=3cm| 等。

\begin{itemize}
\switchcolumn[0]*\item \textbf{If the margin ratio is \textit{not} specified}, \Gm\ sets
 \gpart{body} with the default |scale| ($=0.7$). 
 For example, if |top=2.4cm| is specified, \Gm\ sets
 \begin{center}
     $|height|= |0.7|\times|layoutheight|$ 
            ~~($=|0.7\paperheight|$ by default),
 \end{center}
 then \Ss(1,0) goes to \Ss(1,1), in which |bottom| is calculated
 with $|layoutheight|-(|height|+|top|)$ and results in 6.51cm on A4
 paper if the layout size is equal to the paper size.
 \medskip
 \switchcolumn
 \item \textbf{如果未指定边距比例},\Gm\ 使用默认值 |scale|($=0.7$)设置 \gpart{正文}。例如,如果指定了 |top=2.4cm|,\Gm\ 设置
 \begin{center}
     $|height|= |0.7|\times|layoutheight|$ 
            ~~($=|0.7\paperheight|$ 默认情况下),
 \end{center}
 然后,\Ss(1,0) 转向 \Ss(1,1),在其中使用 $|layoutheight|-(|height|+|top|)$ 计算 |bottom|,如果布局大小等于纸张大小,则在 A4 纸上结果为 6.51cm。
 \medskip

\switchcolumn[0]*\item \textbf{If the margin ratio is specified}, such as
 |hmarginratio={1:2}|, |vratio={3:4}| and so forth,
 \Gm\ sets the other% margin with the specified margin ratio.
 For example, if a set of options ``|top=2.4cm,vratio={3:4}|'' is
 specified, \Gm\ sets |bottom| to be |3.2cm| calculating
 \begin{center}
     $|bottom|= |top|/3\times4 = 3.2\textrm{cm}$
 \end{center}
 Thus \Ss(1,0) goes to \Ss(2,0).
 

 \switchcolumn
 \item \textbf{如果指定了边距比例},例如 |hmarginratio={1:2}|,|vratio={3:4}| 等,\Gm\ 使用指定的边距比例设置另一个边距。例如,如果指定了一组选项 ``|top=2.4cm,vratio={3:4}|'',\Gm\ 将 |bottom| 设置为 |3.2cm|,计算如下:
 \begin{center}
     $|bottom|= |top|/3\times4 = 3.2\textrm{cm}$
 \end{center}
 因此,\Ss(1,0) 转向 \Ss(2,0)。
 \end{itemize}

\switchcolumn[0]*
 Note that the version 4 or earlier used to set the other margin
 with the margin ratio. In the version 5, therefore, with the
 same specification, the result will be different from the one in the
 version 4. For example, if only |top=2.4cm| is specified, 
 you got |bottom=2.4cm| in the version 4 or earlier, but you will get
 |bottom=6.51cm| in the version 5.
 \bigskip
 \switchcolumn
 需要注意的是,版本4或更早的版本通常使用边距比例设置另一个边距。因此,在版本5中,如果使用相同的规格,结果将与版本4中的结果不同。例如,如果只指定 |top=2.4cm|,在版本4或更早的版本中,你将得到 |bottom=2.4cm|,但在版本5中,你将得到 |bottom=6.51cm|。
 \bigskip

\switchcolumn[0]*\item[\Ss(2,1)]
 The \gpart{body} and two \gpart{margins} are all specified, such as
  |vdivide={1in,8in,1.5in}|, ``|left=3cm,width=13cm,right=4cm|'' and
  so forth. Since \Gm\ basically gives priority to \gpart{margins}
  if dimensions are overspecified, \Gm\ forgets and resets
  \gpart{body}. For example, if you specify
 \begin{center}
    |\usepackage[a4paper,left=3cm,width=13cm,right=4cm]{geometry}|,
 \end{center}
 |width| is reset to be 14cm because the width of a A4 paper is 21cm
 long.
 \switchcolumn
 \item[\Ss(2,1)]
指定了 \gpart{正文} 和两个 \gpart{边距},例如 |vdivide={1in,8in,1.5in}|,``|left=3cm,width=13cm,right=4cm|'' 等。由于 \Gm\ 基本上优先考虑 \gpart{边距},如果尺寸被超指定,\Gm\ 将忽略并重新设置 \gpart{正文}。例如,如果你指定
 \begin{center}
    |\usepackage[a4paper,left=3cm,width=13cm,right=4cm]{geometry}|,
 \end{center}
 |width| 将被重置为 14cm,因为 A4 纸的宽度是 21cm。
\end{paracol}
 \end{Spec}
 
% 
\section{Changing layout mid-document\hfill 在文档中间改变页面布局}\label{sec:midchange}
\columnratio{0.55}
\begin{paracol}{2}
The version 5 provides the new commands \cs{newgeometry\{$\cdots$\}}
and \cs{restoregeometry}, which allow you to change page dimensions
in the middle of the document. Unlike \cs{geometry} in the preamble,
\cs{newgeometry} is available only after |\begin{document}|,
resets all the options ever specified except for the
papersize-related options: |landscape|, |portrait|, and paper size
options (such as |papersize|, |paper=a4paper| and so forth), which
can't be changed with \cs{newgeometry}. 
\switchcolumn
版本5提供了新的命令\cs{newgeometry{$\cdots$}}和\cs{restoregeometry},允许您在文档中间更改页面尺寸。与导言区的\cs{geometry}命令不同,\cs{newgeometry}只能在|\begin{document}|之后使用,并且会重置所有以前指定的选项,但保留与纸张尺寸相关的选项:|landscape|、|portrait|和纸张大小选项(如|papersize|、|paper=a4paper|等)。这些选项不能使用\cs{newgeometry}进行更改。
\switchcolumn[0]*
The command \cs{restoregeometry} restores the page layout specified
in the preamble (before |\begin{document}|) with the options to
|\usepackage{geometry}| and \cs{geometry}.
\switchcolumn 
命令\cs{restoregeometry}会恢复在导言区(|\begin{document}|之前)使用|\usepackage{geometry}|和\cs{geometry}指定的页面布局。
\switchcolumn[0]*
Note that both \cs{newgeometry} and \cs{restoregeometry} insert
|\clearpage| where they are called.
\switchcolumn 请注意,\cs{newgeometry}和\cs{restoregeometry}命令都会在调用它们的位置插入|\clearpage|,因此它们会开始一个新的页面。

\switchcolumn[0]*
Below is an example of changing layout mid-document. The layout L1
specified with |hmargin=3cm| (|left| and |right| margins are |3cm|
long) is changed to L2 with |left=3cm|, |right=1cm| and
|bottom=0.1cm|. The layout L1 is restored with
\cs{restoregeometry}. 
\switchcolumn
以下是一个在文档中间改变布局的示例。布局L1使用|hmargin=3cm|(|left|和|right|边距为3cm)指定,然后将其更改为L2,使用|left=3cm|、|right=1cm|和|bottom=0.1cm|。然后使用\cs{restoregeometry}恢复布局L1。
\switchcolumn[0]*
\begin{center}
\begin{minipage}{.8\textwidth}
|\usepackage[hmargin=3cm]{geometry}|\\
|\begin{document}|\\
\medskip
\hspace{1cm}\fbox{Layout L1}\\
\medskip
|\newgeometry{left=3cm,right=1cm,bottom=0.1cm}|\\
\medskip
\hspace{1cm}\fbox{Layout L2 (new)}\\
\medskip
|\restoregeometry|\\
\medskip
\hspace{1cm}\fbox{Layout L1 (restored)}\\
\medskip
|\newgeometry{margin=1cm,includefoot}|\\
\medskip
\hspace{1cm}\fbox{Layout L3 (new)}\\
\medskip
|\end{document}|
\end{minipage}%
\end{center}
\begin{center}
\centering\small
{\unitlength=.8pt
\begin{picture}(450,180)(0,0)
\put(0,165){\makebox(95,12){(saved)}}
\put(15,135){\framebox(65,10){\gpart{head}}}
\put(15,60){\framebox(65,70){\gpart{body}}}
\put(15,45){\makebox(65,12){\gpart{foot}}}
\put(15,45){\line(1,0){65}}
\put(0,20){\framebox(95,140){}}
\put(0,0){\makebox(95,20){L1}}
\put(104,90){\circle*{4}}
\put(110,90){\circle*{4}}
\put(116,90){\circle*{4}}
\put(125,165){\makebox(95,12){\cs{newgeometry}}}
\put(140,135){\framebox(71,10){\gpart{head}}}
\put(140,33){\framebox(71,97){\gpart{body}}}
\put(140,21){\makebox(71,12){\gpart{foot}}}
\put(125,20){\framebox(95,140){}}
\put(125,0){\makebox(95,20){L2 (new)}}
\put(229,90){\circle*{4}}
\put(235,90){\circle*{4}}
\put(241,90){\circle*{4}}
\put(250,165){\makebox(95,12){\cs{restoregeometry}}}
\put(265,135){\framebox(65,10){\gpart{head}}}
\put(265,60){\framebox(65,70){\gpart{body}}}
\put(265,45){\makebox(65,12){\gpart{foot}}}
\put(265,45){\line(1,0){65}}
\put(250,20){\framebox(95,140){}}
\put(250,0){\makebox(95,20){L1 (restored)}}
\put(354,90){\circle*{4}}
\put(360,90){\circle*{4}}
\put(366,90){\circle*{4}}
\put(375,165){\makebox(95,12){\cs{newgeometry}}}
\put(383,41){\framebox(80,111){\gpart{body}}}
\put(383,29){\makebox(80,12){\gpart{foot}}}
\put(383,29){\line(1,0){80}}
\put(375,20){\framebox(95,140){}}
\put(375,0){\makebox(95,20){L3 (new)}}
\end{picture}}
\end{center}
\switchcolumn
\begin{center}
\begin{minipage}{.8\textwidth}
|\usepackage[hmargin=3cm]{geometry}|\\
|\begin{document}|\\
\medskip
\hspace{1cm}\fbox{Layout L1}\\
\medskip
|\newgeometry{left=3cm,right=1cm,bottom=0.1cm}|\\
\medskip
\hspace{1cm}\fbox{Layout L2 (new)}\\
\medskip
|\restoregeometry|\\
\medskip
\hspace{1cm}\fbox{Layout L1 (restored)}\\
\medskip
|\newgeometry{margin=1cm,includefoot}|\\
\medskip
\hspace{1cm}\fbox{Layout L3 (new)}\\
\medskip
|\end{document}|
\end{minipage}%
\end{center}
\begin{center}
\centering\small
{\unitlength=.8pt
\begin{picture}(450,180)(0,0)
\put(0,165){\makebox(95,12){(saved)}}
\put(15,135){\framebox(65,10){\gpart{head}}}
\put(15,60){\framebox(65,70){\gpart{body}}}
\put(15,45){\makebox(65,12){\gpart{foot}}}
\put(15,45){\line(1,0){65}}
\put(0,20){\framebox(95,140){}}
\put(0,0){\makebox(95,20){L1}}
\put(104,90){\circle*{4}}
\put(110,90){\circle*{4}}
\put(116,90){\circle*{4}}
\put(125,165){\makebox(95,12){\cs{newgeometry}}}
\put(140,135){\framebox(71,10){\gpart{head}}}
\put(140,33){\framebox(71,97){\gpart{body}}}
\put(140,21){\makebox(71,12){\gpart{foot}}}
\put(125,20){\framebox(95,140){}}
\put(125,0){\makebox(95,20){L2 (new)}}
\put(229,90){\circle*{4}}
\put(235,90){\circle*{4}}
\put(241,90){\circle*{4}}
\put(250,165){\makebox(95,12){\cs{restoregeometry}}}
\put(265,135){\framebox(65,10){\gpart{head}}}
\put(265,60){\framebox(65,70){\gpart{body}}}
\put(265,45){\makebox(65,12){\gpart{foot}}}
\put(265,45){\line(1,0){65}}
\put(250,20){\framebox(95,140){}}
\put(250,0){\makebox(95,20){L1 (restored)}}
\put(354,90){\circle*{4}}
\put(360,90){\circle*{4}}
\put(366,90){\circle*{4}}
\put(375,165){\makebox(95,12){\cs{newgeometry}}}
\put(383,41){\framebox(80,111){\gpart{body}}}
\put(383,29){\makebox(80,12){\gpart{foot}}}
\put(383,29){\line(1,0){80}}
\put(375,20){\framebox(95,140){}}
\put(375,0){\makebox(95,20){L3 (new)}}
\end{picture}}
\end{center}
\switchcolumn[0]*
A set of commands |\savegeometry{|\meta{name}|}| and
|\loadgeometry{|\meta{name}|}| is handy if you want to
reuse more different layouts in your document.
For example, 
\switchcolumn
如果您想在文档中重用多个不同的布局,一组命令|\savegeometry{|\meta{name}|}|和|\loadgeometry{|\meta{name}|}|会很方便。例如:
\switchcolumn[0]*
\begin{center}
\begin{verbatim}
    \usepackage[hmargin=3cm]{geometry}
    \begin{document}
          L1
    \newgeometry{left=3cm,right=1cm,bottom=0.1cm}
    \savegeometry{L2}
          L2 (new, saved)
    \restoregeometry
          L1 (restored)
    \newgeometry{margin=1cm,includefoot}
          L3 (new)
    \loadgeometry{L2}
          L2 (loaded)
    \end{document}
\end{verbatim}
\end{center}
\switchcolumn
\begin{center}
\begin{verbatim}
\usepackage[hmargin=3cm]{geometry}
\begin{document}
    L1
\newgeometry{left=3cm,right=1cm,bottom=0.1cm}
\savegeometry{L2}
    L2 (new, saved)
\restoregeometry
    L1 (restored)
\newgeometry{margin=1cm,includefoot}
    L3 (new)
\loadgeometry{L2}
    L2 (loaded)
\end{document}
\end{verbatim}
\end{center}
\end{paracol}
% \section{Examples\hfill 示例}

\begin{enumerate}
\columnratio{0.55}
\begin{paracol}{2}
\switchcolumn[0]*\item A onesided page layout with the text area centered in the paper.
The examples below have the same result because the horizontal margin ratio
is set |1:1| for oneside by default.
\switchcolumn
\item 单页布局,文本区域在纸张中央。
以下示例的结果相同,因为默认情况下,单页的水平边距比例设置为 |1:1|。

\begin{itemize}
\switchcolumn[0]*\item |centering|
  \switchcolumn\item |centering|
\switchcolumn[0]*\item |marginratio=1:1|
  \switchcolumn\item |marginratio=1:1|
\switchcolumn[0]*\item |vcentering|
\switchcolumn\item |vcentering|
\end{itemize}


\switchcolumn[0]*\item A twosided page layout with the inside offset for binding set to |1cm|.
\switchcolumn \item 双页布局,内部偏移量设置为 |1cm|。
\begin{itemize}
\switchcolumn[0]*\item |twoside, bindingoffset=1cm|
\switchcolumn \item |twoside, bindingoffset=1cm|
\end{itemize}

\switchcolumn[0]*
In this case, |textwidth| is shorter than that of the default twosided
document by $0.7\times|1cm|$ ($=|0.7cm|$) because the default width of
\gpart{body} is set with |scale=0.7|, which means 
$|width|=|0.7|\times|layoutwidth|$ ($=|0.7\paperwidth|$ by default).
\switchcolumn 在这种情况下,|textwidth| 比默认的双页文档短了 $0.7\times|1cm|$($=|0.7cm|$),因为\gpart{body}的默认宽度设置为 |scale=0.7|,意味着 $|width|=|0.7|\times|layoutwidth|$ (默认为 $=|0.7\paperwidth|$)。

\switchcolumn[0]*\item A layout with the left, right, and top margin |3cm|, |2cm| and
|2.5in| respectively, with textheight of 40 lines, and with the head and
foot of the page included in \gpart{total body}.
The two examples below have the same result.
\switchcolumn \item 左、右和上边距分别设置为 |3cm|、|2cm| 和 |2.5in|,文本高度设置为 40 行,页眉和页脚包括在\gpart{total body}中。
以下两个示例的结果相同。

\begin{itemize}

\switchcolumn[0]*\item |left=3cm, right=2cm, lines=40, top=2.5in, includeheadfoot|
  \switchcolumn \item |left=3cm, right=2cm, lines=40, top=2.5in, includeheadfoot|
\switchcolumn[0]*\item |hmargin={3cm,2cm}, tmargin=2.5in, lines=40, includeheadfoot|
\switchcolumn \item |hmargin={3cm,2cm}, tmargin=2.5in, lines=40, includeheadfoot|
\end{itemize}


\switchcolumn[0]*\item A layout with the height of \gpart{total body} |10in|, the bottom
margin |2cm|, and the default width. The top margin will be calculated
automatically. Each solution below results in the same page layout.
\switchcolumn \item 高度设置为 \gpart{total body} 的高度为 |10in|,底边距设置为 |2cm|,默认宽度。顶边距将自动计算。以下两个解决方案的页面布局相同。
\begin{itemize}
\switchcolumn[0]*\item |vdivide={*, 10in, 2cm}|
    \switchcolumn\item |vdivide={*, 10in, 2cm}|
\switchcolumn[0]*\item |bmargin=2cm, height=10in|
    \switchcolumn\item |bmargin=2cm, height=10in|
\switchcolumn[0]*\item |bottom=2cm, textheight=10in| 
\switchcolumn\item |bottom=2cm, textheight=10in| 
\end{itemize}

\switchcolumn[0]*
Note that dimensions for \gpart{head} and \gpart{foot} are excluded from
|height| of \gpart{total body}. An additional |includefoot| makes
\cs{footskip} included in |totalheight|. Therefore, in the two cases below,
|textheight| in the former layout is shorter than the latter
(with 10in exactly) by \cs{footskip}. In other words, 
|height| = |textheight| + |footskip| when |includefoot=true| in this case.
\switchcolumn
请注意,\gpart{head}和\gpart{foot}的尺寸不包括在\gpart{total body}的高度中。添加 |includefoot| 会将 \cs{footskip} 包含在 |totalheight| 中。因此,在下面的两种情况下,前一种布局中的 |textheight| 比后一种布局中的 |textheight|(确切为 10in)要短 \cs{footskip}。换句话说,当在这种情况下 |includefoot=true| 时,|height| = |textheight| + |footskip|。
\begin{itemize}
    
\switchcolumn[0]*\item |bmargin=2cm, height=10in, includefoot|
    \switchcolumn\item |bmargin=2cm, height=10in, includefoot|
\switchcolumn[0]*\item |bottom=2cm, textheight=10in, includefoot|
\switchcolumn\item |bottom=2cm, textheight=10in, includefoot|
\end{itemize}


\switchcolumn[0]*\item A layout with \glen{textwidth} and \glen{textheight} 90\% of the
paper and with \gpart{body} centered.
Each solution below results in the same page layout as long as
|layoutwidth| and |layoutheight| are not modified from the default.
\switchcolumn \item 宽度为纸张宽度的 90\%,高度为纸张高度的 90\%,\gpart{body} 居中。只要 |layoutwidth| 和 |layoutheight| 没有修改,默认情况下,以下解决方案会得到相同的页面布局。
\begin{itemize}

\switchcolumn[0]*\item |scale=0.9, centering|
  \switchcolumn\item |scale=0.9, centering|
\switchcolumn[0]*\item |text={.9\paperwidth,.9\paperheight}, ratio=1:1|
  \switchcolumn\item |text={.9\paperwidth,.9\paperheight}, ratio=1:1|
\switchcolumn[0]*\item |width=.9\paperwidth, vmargin=.05\paperheight, marginratio=1:1|
  \switchcolumn\item |width=.9\paperwidth, vmargin=.05\paperheight, marginratio=1:1|
\switchcolumn[0]*\item |hdivide={*,0.9\paperwidth,*}, vdivide={*,0.9\paperheight,*}|
  (as for onesided documents)
  \switchcolumn \item |hdivide={*,0.9\paperwidth,*}, vdivide={*,0.9\paperheight,*}|
  (as for onesided documents)
\switchcolumn[0]*\item |margin={0.05\paperwidth,0.05\paperheight}|
\switchcolumn\item |margin={0.05\paperwidth,0.05\paperheight}|
\end{itemize}

\switchcolumn[0]*
You can add |heightrounded| to avoid an ``underfull vbox warning'' like
\begin{quote}\small
|Underfull \vbox (badness 10000) has occurred while \output is active|.
\end{quote}
See Section~\ref{sec:body} for the detailed description about |heightrounded|.
\switchcolumn
您可以添加 |heightrounded| 来避免出现“未填满的 vbox 警告”,例如:
\begin{quote}\small
|Underfull \vbox (badness 10000) has occurred while \output is active|.
\end{quote}
有关 |heightrounded| 的详细描述,请参阅第~\ref{sec:body} 节。


\switchcolumn[0]*\item A layout with the width of marginal notes set to |3cm| and included
in the width of \gpart{total body}. The following examples are the same.
\switchcolumn \item 一个布局,边注的宽度设置为 |3cm|,并包含在 \gpart{total body} 的宽度中。以下示例相同。
\begin{itemize}
  
\switchcolumn[0]*\item |marginparwidth=3cm, includemp|
  \switchcolumn\item |marginparwidth=3cm, includemp|
\switchcolumn[0]*\item |marginpar=3cm, ignoremp=false|
\switchcolumn\item |marginpar=3cm, ignoremp=false|
\end{itemize}


\switchcolumn[0]*\item A layout where \gpart{body} occupies the whole paper with A5 paper in
landscape. The following examples are the same.
\switchcolumn \item 一个布局,其中 \gpart{body} 占据整个纸张,使用横向的 A5 纸张。以下示例相同。

\begin{itemize}
  
\switchcolumn[0]*\item |a5paper, landscape, scale=1.0|
  \switchcolumn\item |a5paper, landscape, scale=1.0|
\switchcolumn[0]*\item |landscape=TRUE, paper=a5paper, margin=0pt|
\switchcolumn\item |landscape=TRUE, paper=a5paper, margin=0pt|
\end{itemize}


\switchcolumn[0]*\item  A screen size layout appropriate for presentation with PC and video
projector.
\begin{verbatim}
\documentclass{slide}
\usepackage[screen,margin=0.8in]{geometry}
...
\begin{slide}
    ...
\end{slide}\end{verbatim}
\switchcolumn    \item 适合使用个人电脑和投影仪进行演示的屏幕尺寸布局。  
\begin{verbatim}
  \documentclass{slide}
  \usepackage[screen,margin=0.8in]{geometry}
  ...
  \begin{slide}
    ...
  \end{slide}\end{verbatim}

\switchcolumn[0]*\item A layout with fonts and spaces both enlarged from A4 to A3.
In the case below, the resulting paper size is A3.
\begin{itemize}
    \switchcolumn
\switchcolumn[0]*\item |a4paper, mag=1414|.
\end{itemize}
If you want to have a layout with two times bigger fonts, but without
changing paper size, you can type
\begin{itemize}
  \switchcolumn
\switchcolumn[0]*\item |letterpaper, mag=2000, truedimen|.
\end{itemize}
You can add |dvips| option, that is useful to preview it with proper
paper size by |dviout| or |xdvi|.

\switchcolumn
\switchcolumn[0]*\item  Changing the layout of the first page and leaving the others
as default before loading \Gm. Use |pass| option, |\newgeometry|
and |\restoregeometry|.
\begin{verbatim}
  \documentclass{book}
  \usepackage[pass]{geometry}
    % 'pass' disregards the package layout,
    %  so the original 'book' layout is memorized here.
  \begin{document}
  \newgeometry{margin=1cm}% changes the first page dimensions.
    Page 1
  \restoregeometry % restores the original 'book' layout.
    Page 2 and more
  \end{document}\end{verbatim}

\switchcolumn
\switchcolumn[0]*\item A complex page layout.
\begin{verbatim}
\usepackage[a5paper, landscape, twocolumn, twoside,
    left=2cm, hmarginratio=2:1, includemp, marginparwidth=43pt, 
    bottom=1cm, foot=.7cm, includefoot, textheight=11cm, heightrounded,
    columnsep=1cm, dvips,  verbose]{geometry}\end{verbatim}
Try typesetting it and checking out the result yourself. |:-)|
\end{paracol}
\end{enumerate}


\section{Known problems\hfill 已知问题}
\begin{itemize}
\columnratio{0.55}
\begin{paracol}{2}
\item With |mag| $\neq 1000$ and |truedimen|,
|paperwidth| and |paperheight| shown in verbose mode are different
from the real size of the resulted PDF. The PDF itself is correct anyway.
\switchcolumn\item 当 |mag| $\neq 1000$ 且 |truedimen| 时,在详细模式下显示的 |paperwidth| 和 |paperheight| 与生成的 PDF 的实际尺寸不同。但是生成的 PDF 本身是正确的。
\switchcolumn[0]*\item With |mag| $\neq 1000$, \textit{no} |truedimen|
and \textsf{hyperref}, \textsf{hyperref} should be loaded before \Gm. 
Otherwise the resulted PDF size will become wrong.
\switchcolumn\item 当 |mag| $\neq 1000$,\textit{没有} |truedimen| 和 \textsf{hyperref} 时,应在加载 \Gm 之前加载 \textsf{hyperref}。否则生成的 PDF 尺寸将错误。
\switchcolumn[0]*\item With \textsf{crop} package and |mag| $\neq 1000$,
|center| option of \textsf{crop} doesn't work well.
\switchcolumn\item 使用 \textsf{crop} 宏包和 |mag| $\neq 1000$ 时,\textsf{crop} 的 |center| 选项无法正常工作。
\end{paracol}
\end{itemize}

% \DocInput{geometry.dtx}
\end{document}   