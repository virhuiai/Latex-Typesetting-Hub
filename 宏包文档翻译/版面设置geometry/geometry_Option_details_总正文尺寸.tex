\columnratio{0.55}
\begin{paracol}{2}
\subsection{Body size}\label{sec:body}

The options specifying the size of \gpart{total body} are described in this
section.
\switchcolumn
\subsection{总正文尺寸}
本节描述了指定\gpart{总正文}尺寸的选项。
\end{paracol}

\begin{Options}
    \columnratio{0.55}
\begin{paracol}{2}
\item[hscale]
ratio of width of \gpart{total body} to \cs{paperwidth}. 
|hscale=|\meta{h-scale}, e.g., |hscale=0.8| is equivalent to
|width=0.8|\cs{paperwidth}. (|0.7| by default)
\switchcolumn
\item[hscale] \gpart{总正文}的宽度与 \cs{paperwidth} 的比例。|hscale=|\meta{h-scale},例如,|hscale=0.8| 等同于 |width=0.8|\cs{paperwidth}。(默认为 |0.7|)   
   \switchcolumn[0]*   
\item[vscale]
   ratio of height of \gpart{total body} to \cs{paperheight}, e.g.,
   |vscale=|\meta{v-scale}. (|0.7| by default) |vscale=0.9| is equivalent
   to |height=0.9|\cs{paperheight}.
\switchcolumn
\item[vscale] \gpart{总正文}的高度与 \cs{paperheight} 的比例。例如,|vscale=|\meta{v-scale}。(默认为 |0.7|)
|vscale=0.9| 等同于 |height=0.9|\cs{paperheight}。
\switchcolumn[0]*
\item[scale] ratio of \gpart{total body} to the paper.
   |scale=|\argii{h-scale}{v-scale} or |scale=|\meta{scale}.
   (|0.7| by default)
\switchcolumn
\item[scale] \gpart{总正文}与纸张的比例。|scale=|\argii{h-scale}{v-scale} 或 |scale=|\meta{scale}。(默认为 |0.7|)
\switchcolumn[0]*
\item[width\OR totalwidth] ~\\
width of \gpart{total body}. |width=|\meta{length} or
|totalwidth=|\meta{length}. This dimension defaults to |textwidth|,
but if |includemp| is set to |true|, |width| $\ge$ |textwidth| 
because |width| includes the width of the marginal notes.
If |textwidth| and |width| are specified at the same time, 
|textwidth| takes priority over |width|.
\switchcolumn
\item[width\OR totalwidth] ~\\
\gpart{总正文}的宽度。|width=|\meta{长度} 或 |totalwidth=|\meta{长度}。
该尺寸默认为 |textwidth|,但如果将 |includemp| 设置为 |true|,则 |width| $\ge$ |textwidth|,因为 |width| 包括边注的宽度。
如果同时指定 |textwidth| 和 |width|,则 |textwidth| 优先于 |width|。
\switchcolumn[0]*
\item[height\OR totalheight] ~\\
height of \gpart{total body}, excluding header and footer by default.
If |includehead| or |includefoot| is set, |height| includes
the head or foot of the page as well as |textheight|.
|height=|\meta{length} or |totalheight=|\meta{length}. If both
|textheight| and |height| are specified, |height| will be ignored.
\switchcolumn
\item[height\OR totalheight] ~\\
\gpart{总正文}的高度,默认不包括页眉和页脚。
如果设置了 |includehead| 或 |includefoot|,则 |height| 包括页眉或页脚以及 |textheight|。
|height=|\meta{长度} 或 |totalheight=|\meta{长度}。
如果同时指定了 |textheight| 和 |height|,则会忽略 |height|。
\switchcolumn[0]*
\item[total] width and height of \gpart{total body}.\\
   |total=|\argii{width}{height} or |total=|\meta{length}.
   \switchcolumn
   \item[total] \gpart{总正文}的宽度和高度。|total=|\argii{宽度}{高度} 或 |total=|\meta{长度}。
\switchcolumn[0]*
\item[textwidth] specifies \cs{textwidth}, the width of \gpart{body} 
   (the text area). |textwidth=|\meta{length}.
   \switchcolumn
   \item[textwidth] 指定 \cs{textwidth},即正文(文本区域)的宽度。|textwidth=|\meta{长度}。
   \switchcolumn[0]*
\item[textheight] specifies \cs{textheight}, the height of
   \gpart{body} (the text area). |textheight=|\meta{length}.
   \switchcolumn
   \item[textheight] 指定 \cs{textheight},即正文(文本区域)的高度。|textheight=|\meta{长度}。
\switchcolumn[0]*
\item[text\OR body] specifies both \cs{textwidth} and \cs{textheight}
   of the body of page. |body=|\argii{width}{height} or
   |text=|\meta{length}.
   \switchcolumn
   \item[text\OR body] 指定正文(文本区域)的 \cs{textwidth} 和 \cs{textheight}。|body=|\argii{宽度}{高度} 或 |text=|\meta{长度}。
\switchcolumn[0]*
\item[lines] enables users to specify \cs{textheight} by the number
   of lines. |lines|=\meta{integer}.
   \switchcolumn
   \item[lines] 允许用户通过行数指定 \cs{textheight}。|lines=|\meta{整数}。
\switchcolumn[0]*
\item[includehead] includes the head of the page, \cs{headheight}
   and \cs{headsep}, into \gpart{total body}. It is set to |false| by
   default. It is opposite to |ignorehead|. See
   Figure~\ref{fig:includes} and Figure~\ref{fig:modes}.
   \switchcolumn
   \item[includehead] 将页眉,\cs{headheight} 和 \cs{headsep} 包含到 \gpart{总正文} 中。默认为 |false|。与 |ignorehead| 相反。参见图\ref{fig:includes} 和图\ref{fig:modes}。
\switchcolumn[0]*
\item[includefoot] includes the foot of the page, \cs{footskip},
   into \gpart{total body}. It is opposite to |ignorefoot|.
   It is |false| by default. See Figure~\ref{fig:includes} and
   Figure~\ref{fig:modes}.
   \switchcolumn
   \item[includefoot] 将页脚,\cs{footskip},包含到 \gpart{总正文} 中。与 |ignorefoot| 相反。默认为 |false|。参见图~\ref{fig:includes} 和图~\ref{fig:modes}。
   \switchcolumn[0]*
\item[includeheadfoot]~\\ 
   sets both |includehead| and |includefoot| to |true|, which is opposite
   to |ignoreheadfoot|. See Figure~\ref{fig:includes} and
   Figure~\ref{fig:modes}.
\switchcolumn
\item[includeheadfoot]~\\ 
将 |includehead| 和 |includefoot| 都设置为 |true|,与 |ignoreheadfoot| 相反。参见图\ref{fig:includes} 和图~\ref{fig:modes}。
\switchcolumn[0]*
\item[includemp] includes the margin notes,  \cs{marginparwidth}
   and \cs{marginparsep}, into \gpart{body} when calculating horizontal
   calculation.
   \switchcolumn
   \item[includemp] 当计算水平布局时,将边注,\cs{marginparwidth} 和 \cs{marginparsep},包含到 \gpart{正文} 中。
\switchcolumn[0]*
\item[includeall] sets both |includeheadfoot| and |includemp| to
   |true|. See Figure~\ref{fig:modes}.
   \switchcolumn
   \item[includeall] 将 |includeheadfoot| 和 |includemp| 都设置为 |true|。参见图~\ref{fig:modes}。
\switchcolumn[0]*
\item[ignorehead] disregards the head of the page,
   |headheight| and |headsep|, in determining vertical layout, but does not
   change those lengths. It is equivalent to |includehead=false|. It is set
   to |true| by default. See also |includehead|.
   \switchcolumn
   \item[ignorehead] 在确定垂直布局时忽略页眉,即 \cs{headheight} 和 \cs{headsep},但不更改这些长度。等同于 |includehead=false|。默认为 |true|。参见 |includehead|。
\switchcolumn[0]*
\item[ignorefoot] disregards the foot of page, |footskip|,
   in determining vertical layout, but does not change that length.
   This option defaults to |true|. See also |includefoot|.
   \switchcolumn
   \item[ignorefoot] 在确定垂直布局时忽略页脚,即 \cs{footskip},但不更改该长度。默认为 |true|。参见 |includefoot|。
\switchcolumn[0]*
\item[ignoreheadfoot]~\\ sets both |ignorehead| and |ignorefoot|
   to |true|. See also |includeheadfoot|.
   \switchcolumn
   \item[ignoreheadfoot] ~\\ 将 |ignorehead| 和 |ignorefoot| 都设置为 |true|。参见 |includeheadfoot|。
\switchcolumn[0]*
\item[ignoremp] disregards the marginal notes in determining the
   horizontal margins (defaults to |true|). If marginal notes overrun
   the page, the warning message will be displayed when |verbose=true|.
   See also |includemp| and Figure~\ref{fig:modes}.
   \switchcolumn
   \item[ignoremp] 在确定水平边距时忽略边注(默认为 |true|)。如果边注超出页面,则在 |verbose=true| 时将显示警告消息。参见 |includemp| 和图\ref{fig:modes}。
\switchcolumn[0]*
\item[ignoreall] sets both |ignoreheadfoot| and |ignoremp| to |true|. 
   See also |includeall|.
   \switchcolumn
   \item[ignoreall] 将 |ignoreheadfoot| 和 |ignoremp| 都设置为 |true|。参见 |includeall|。
\switchcolumn[0]*
\item[heightrounded]~\\
   This option rounds \cs{textheight} to \textit{n}-times (\textit{n}:
   an integer) of \cs{baselineskip} plus \cs{topskip} to avoid 
   ``underfull vbox'' in some cases. For example, if \cs{textheight} is
   486pt with \cs{baselineskip} 12pt and \cs{topskip} 10pt, then
   \begin{quote}
     $(39\times12\textrm{pt}+10\textrm{pt}=)\: 478\textrm{pt}
      < 486\textrm{pt} < 
     490\textrm{pt} \:(=40\times12\textrm{pt}+10\textrm{pt})$,
   \end{quote}
   as a result \cs{textheight} is rounded to 490pt. |heightrounded=false|
   by default.
   \switchcolumn
   \item[heightrounded] ~\\
此选项将 \cs{textheight} 舍入为 \textit{n} 倍(\textit{n}:整数)的 \cs{baselineskip} 加上 \cs{topskip},以避免在某些情况下出现``underfull vbox''。
例如,如果 \cs{textheight} 是 486pt,\cs{baselineskip} 是 12pt,\cs{topskip} 是 10pt,则
\begin{quote}
    $(39\times12\textrm{pt}+10\textrm{pt}=)\: 478\textrm{pt}
     < 486\textrm{pt} < 
    490\textrm{pt} \:(=40\times12\textrm{pt}+10\textrm{pt})$,
  \end{quote}
结果 \cs{textheight} 被舍入为 490pt。默认情况下,|heightrounded=false|。
\end{paracol}
\end{Options}

\columnratio{0.55}
\begin{paracol}{2}
Figure~\ref{fig:modes} illustrates various layouts with different layout
modes. The dimensions for a header and a footer can be controlled by
|nohead| or |nofoot| mode, which sets each length to 0pt directly.
On the other hand, options with the prefix |ignore| do \textit{not}
change the corresponding native dimensions.
\switchcolumn
图~\ref{fig:modes} 展示了不同布局模式下的各种布局。页眉和页脚的尺寸可以通过 |nohead| 或 |nofoot| 模式进行控制,这会直接将每个长度设置为 0pt。另一方面,以 |ignore| 为前缀的选项\textit{不会}改变对应的原始尺寸。
\end{paracol}
\begin{figure}
 \centering\small
 {\unitlength=.65pt
 \begin{picture}(460,525)(0,0)
 \put( 20,310){\framebox(120,170){}}
 \put( 20,507){\makebox(120,0)[bl]%
 {\textbf{(a)}~|includeheadfoot|}}
 \put( 20,460){\line(1,0){120}}\put( 20,450){\line(1,0){120}}
 \put( 20,330){\line(1,0){120}}
 \put( 20,485){\makebox(120,0)[br]{\gpart{total body}}}
 \put( 20,335){\makebox(120,0)[bc]{|textwidth|}}
 \put(150,470){\makebox(0,0)[l]{|headheight|}}
 \put(150,450){\makebox(0,0)[l]{|headsep|}}
 \put(150,390){\makebox(0,0)[l]{|textheight|}}
 \put(150,320){\makebox(0,0)[l]{|footskip|}}
 \put( 10,460){\makebox(120,20)[bc]{\gpart{head}}}
 \put( 10,320){\makebox(120,140)[c]{\gpart{body}}}
 \put( 10,310){\makebox(120,10)[c]{\gpart{foot}}}
 \put(250,310){\framebox(120,170){}}
 \put(250,507){\makebox(120,0)[bl]%
 {\textbf{(b)}~|includeall|}}
 \put(250,460){\line(1,0){95}}\put(250,450){\line(1,0){95}}
 \put(250,330){\line(1,0){95}}\put(345,330){\line(0,1){120}}
 \put(350,330){\line(0,1){120}}\put(350,450){\line(1,0){20}}
 \put(350,330){\line(1,0){20}}
 \put(250,485){\makebox(120,0)[br]{\gpart{total body}}}
 \put(250,460){\makebox(95,20)[bc]{\gpart{head}}}
 \put(250,320){\makebox(95,140)[c]{\gpart{body}}}
 \put(385,390){\makebox(95,0)[cl]%
 {\gpart{\shortstack[l]{marginal\\note}}}}
 \put(250,310){\makebox(95,10)[c]{\gpart{foot}}}
 \put(250,335){\makebox(95,0)[bc]{|textwidth|}}
 \multiput(360, 390)(4,0){6}{\line(1,0){2}}
 \multiput(348,333)(0,-4){12}{\line(0,1){2}}
 \multiput(360,333)(0,-4){8}{\line(0,1){2}}
 \put(355,292){\makebox(0,0)[bl]{|marginparwidth|}}
 \put(345,275){\makebox(0,0)[bl]{|marginparsep|}}
 \put( 20, 40){\framebox(120,170){}}
 \put( 20,237){\makebox(120,0)[bl]%
 {\textbf{(c)}~|includefoot|}}
 \put( 20, 60){\line(1,0){120}}
 \put( 20,215){\makebox(120,0)[br]{\gpart{total body}}}
 \put(150,130){\makebox(0,0)[l]{|textheight|}}
 \put(150, 50){\makebox(0,0)[l]{|footskip|}}
 \put( 20, 50){\makebox(120,160)[c]{\gpart{body}}}
 \put( 20, 40){\makebox(120,10)[c]{\gpart{foot}}}
 \put( 20, 65){\makebox(120,10)[c]{|textwidth|}}
 \put(250, 40){\framebox(120,170){}}
 \put(250,237){\makebox(120,0)[bl]%
 {\textbf{(d)}~|includefoot,includemp|}}
 \put(250, 60){\line(1,0){95}}\put(350, 60){\line(1,0){20}}
 \put(250,215){\makebox(120,0)[br]{\gpart{total body}}}
 \put(250, 50){\makebox(95,160)[c]{\gpart{body}}}
 \put(385,130){\makebox(95,0)[cl]%
 {\gpart{\shortstack[l]{marginal\\note}}}}
 \put(250, 40){\makebox(95,10)[c]{\gpart{foot}}}
 \put(250, 65){\makebox(95,0)[bc]{|textwidth|}}
 \put(345, 60){\line(0,1){150}}\put(350, 60){\line(0,1){150}}
 \multiput(360, 130)(4,0){6}{\line(1,0){2}}
 \multiput(348, 63)(0,-4){12}{\line(0,1){2}}
 \multiput(360, 63)(0,-4){8}{\line(0,1){2}}
 \put(355,22){\makebox(0,0)[bl]{|marginparwidth|}}
 \put(345, 5){\makebox(0,0)[bl]{|marginparsep|}}
 \end{picture}}
 \caption[Sample layouts for \gpart{total body} with different 
    layout modes]{%
 \begin{minipage}[t]{.8\textwidth}\small
   Sample layouts for \gpart{total body} with different switches.
   (a) |includeheadfoot|, (b) |includeall|, (c) |includefoot|
    and (d) |includefoot,includemp|. 
   If |reversemp| is set to |true|, the location of the
   marginal notes are swapped on every page.
   Option |twoside| swaps both margins and marginal notes on verso pages.
   Note that the marginal note, if any, is printed despite
   |ignoremp| or |includemp=false| and overrun the page in some cases.\\
   使用不同开关的\gpart{总体正文}的示例布局。
   (a) |includeheadfoot|,(b) |includeall|,(c) |includefoot|
   和 (d) |includefoot,includemp|。
   如果将 |reversemp| 设置为 |true|,则边注的位置将在每一页上交换。
   选项 |twoside| 在背面页面上交换两边的边距和边注。
   请注意,如果有边注,则会在某些情况下打印出来,尽管设置了 |ignoremp| 或 |includemp=false|,并可能超出页面。
 \end{minipage}}
 \label{fig:modes}
\end{figure}

\columnratio{0.55}
\begin{paracol}{2}
The following options can specify body and margins simultaneously with
three comma-separated values in braces.
\switchcolumn
以下选项可以同时指定正文和边距,使用花括号内的三个逗号分隔的值。
\end{paracol}

\begin{Options}
\columnratio{0.55}
\begin{paracol}{2}
\item[hdivide] horizontal partitions (left,width,right).
  |hdivide=|\argiii{left margin}{width}{right margin}. 
  Note that you should not specify all of the three parameters.
  The best way of using this option is to specify two of three and 
  leave the rest with null(nothing) or `|*|'. For example, when you set
  |hdivide={2cm,15cm, }|, the margin from the right-side edge of page
  will be determined calculating |paperwidth-2cm-15cm|.
\switchcolumn
\item[hdivide] 水平分割 (左边距,宽度,右边距)。
|hdivide=|\argiii{左边距}{宽度}{右边距}。
请注意,不应同时指定这三个参数。
使用此选项的最佳方法是指定其中的两个,并将剩下的一个设为 null (空) 或 `|*|'。例如,当设置 |hdivide={2cm,15cm, }| 时,页面右侧边缘的边距将通过计算 |paperwidth-2cm-15cm| 来确定。
\switchcolumn[0]*
\item[vdivide] vertical partitions (top,height,bottom).
|vdivide=|\argiii{top margin}{height}{bottom margin}.
\switchcolumn
\item[vdivide] 垂直分割 (上边距,高度,下边距)。
|vdivide=|\argiii{上边距}{高度}{下边距}。
\switchcolumn[0]*
\item[divide] |divide=|\vargiii{$A$}{$B$}{$C$} is interpreted  as 
|hdivide=|\vargiii{$A$}{$B$}{$C$} and |vdivide=|\vargiii{$A$}{$B$}{$C$}.
\switchcolumn
\item[divide] |divide=|\vargiii{$A$}{$B$}{$C$} 被解释为 |hdivide=|\vargiii{$A$}{$B$}{$C$} 和 |vdivide=|\vargiii{$A$}{$B$}{$C$}。
\end{paracol}
\end{Options}
