
\columnratio{0.55}
\begin{paracol}{2}
\switchcolumn[0]*
\subsection{Layout size}
You can specify the layout area with options described in this
section regardless of the paper size.
The options would help to print the specified layout to a different
sized paper.  For example, with |a4paper| and |layout=a5paper|, the
package uses `A5' layout to calculate margins on 'A4' paper.
The layout size defaults to the same as the paper.
The options for the layout size are available in \cs{newgeometry},
so that you can change the layout size in the middle of the document.
The paper size itself can't be changed though.
Figure~\ref{fig:layoutandpaper} shows what the difference between
|layout| and |paper| is.
\switchcolumn
\subsection{布局尺寸}

无论纸张大小如何,您都可以使用本节中描述的选项来指定布局区域。
这些选项可以帮助您将指定的布局打印到不同大小的纸张上。
例如,使用 |a4paper| 和 |layout=a5paper|,该宏包会在“A4”纸上使用“A5”布局来计算边距。
布局尺寸默认与纸张相同。
布局尺寸的选项在 \cs{newgeometry} 中也可用,因此您可以在文档的中间更改布局尺寸。
但是纸张尺寸本身无法更改。
图~\ref{fig:layoutandpaper}~展示了 |layout| 和 |paper| 之间的区别。
\end{paracol}


\begin{Options}
\columnratio{0.55}
\begin{paracol}{2}
\item[layout] specifies the layout size by paper
name. |layout=|\meta{paper-name}. All the paper names defined in \Gm\
are available. See Section~\ref{sec:paper} for details.
\switchcolumn
\item[layout] 按纸张名称指定布局尺寸。|layout=|\meta{纸张名称}。
所有在 \Gm\ 中定义的纸张名称都可用。详情请参阅第~\ref{sec:paper}~节。
\switchcolumn[0]*
\item[layoutwidth] width of the layout. |layoutwidth=|\meta{length}.
\switchcolumn
\item[layoutwidth] 布局的宽度。|layoutwidth=|\meta{长度}。
\switchcolumn[0]*
\item[layoutheight] height of the layout. |layoutheight=|\meta{length}.
\switchcolumn
\item[layoutheight] 布局的高度。|layoutheight=|\meta{长度}。
\switchcolumn[0]*
\item[layoutsize] width and height of the layout.
|layoutsize=|\argii{width}{height} or |layoutsize=|\meta{length}.
\switchcolumn
\item[layoutsize] 布局的宽度和高度。|layoutsize=|\argii{宽度}{高度} 或 |layoutsize=|\meta{长度}。
\switchcolumn[0]*
\item[layouthoffset] specifies the horizontal offset from the left edge of
the paper. |layouthoffset=|\meta{length}.
\switchcolumn
\item[layouthoffset] 指定布局相对于纸张左边缘的水平偏移量。|layouthoffset=|\meta{长度}。
\switchcolumn[0]*
\item[layoutvoffset] specifies the vertical offset from the top edge of
the paper. |layoutvoffset=|\meta{length}.
\switchcolumn
\item[layoutvoffset] 指定布局相对于纸张顶边缘的垂直偏移量。|layoutvoffset=|\meta{长度}。
\switchcolumn[0]*
\item[layoutoffset] specifies both horizontal and vertical offsets. 
|layoutoffset=|\argii{hoffset}{voffset} or |layoutsize=|\meta{length}.
\switchcolumn
\item[layoutoffset] 指定布局的水平和垂直偏移量。|layoutoffset=|\argii{水平偏移量}{垂直偏移量} 或 |layoutsize=|\meta{长度}。
\end{paracol}
\end{Options}
\begin{figure}
 \centering\small
 {\unitlength=.6pt
 \begin{picture}(450,250)(0,-10)
 \put(20,0){\makebox(168,12)[r]{\gpart{paper}}}
 \put(20,0){\framebox(170,230){}}
 \put(21,40){\dashbox{3}(140,189){}}
 \put(21,28){\makebox(140,12)[r]{\gpart{layout}}}
 \put(40,50){\makebox(100,10){\gpart{foot}}}
 \put(40,50){\line(1,0){100}}
 \put(40,65){\framebox(100,125){\gpart{body}}}
 \put(40,200){\framebox(100,10){\gpart{head}}}
 \put(20,230){\makebox(140,20){|layoutwidth|}}
 \put(40,240){\vector(-1,0){20}}
 \put(140,240){\vector(1,0){20}}
 \put(10,145){\vector(0,1){85}}
 \put(15,125){\makebox(0,20)[r]{|layoutheight|}}
 \put(10,125){\vector(0,-1){85}}
 \put(280,0){\makebox(168,12)[r]{\gpart{paper}}}
 \put(280,0){\framebox(170,230){}}
 \put(293,35){\dashbox{3}(140,189){}}
 \put(293,23){\makebox(140,12)[r]{\gpart{layout}}}
 \put(312,45){\makebox(100,10){\gpart{foot}}}
 \put(312,45){\line(1,0){100}}
 \put(312,60){\framebox(100,125){\gpart{body}}}
 \put(312,195){\framebox(100,10){\gpart{head}}}
 \put(235,230){\makebox(80,20)[l]{|layouthoffset|}}
 \put(260,210){\vector(1,0){20}}
 \put(308,210){\vector(-1,0){15}}
 \put(260,210){\line(-1,2){10}}
 \put(355,230){\makebox(100,20){|layoutvoffset|}}
 \put(350,250){\vector(0,-1){20}}
 \put(350,209){\vector(0,1){15}}
 \end{picture}}
 \caption[layout and paper]{%
 \begin{minipage}[t]{.7\textwidth}\raggedright\small
 The dimensions related to the layout size. Note that the layout size
 defaults to the same size as the paper, so you don't have to specify
 layout-related options explicitly in most cases.
 \end{minipage}}
 \label{fig:layoutandpaper}
\end{figure}
