\columnratio{0.55}
\begin{paracol}{2}
\switchcolumn[0]*
\subsection{Paper size}\label{sec:paper}

The options below set paper/media size and orientation.
\switchcolumn
\subsection{纸张大小}
以下选项设置纸张/媒体大小和方向。
\end{paracol}


\begin{Options}
\columnratio{0.55}
\begin{paracol}{2}
\item[\onlypre paper\OR papername] ~\\ 
specifies the paper size by name. |paper=|\meta{paper-name}.
For convenience, you can specify the paper name without |paper=|.
For example, |a4paper| is equivalent to |paper=a4paper|.
\switchcolumn
\item[\onlypre paper\OR papername] ~\\
通过名称指定纸张大小。|paper=|\meta{纸张名称}。
为了方便起见,您可以省略 |paper=| 直接指定纸张名称。
例如,|a4paper| 等同于 |paper=a4paper|。
\switchcolumn[0]*
\item[\onlypre \vtop{
\hbox{a0paper, a1paper, a2paper, a3paper, a4paper, a5paper, a6paper,}
\hbox{b0paper, b1paper, b2paper, b3paper, b4paper, b5paper, b6paper,}
\hbox{c0paper, c1paper, c2paper, c3paper, c4paper, c5paper, c6paper,}
\hbox{b0j, b1j, b2j, b3j, b4j, b5j, b6j,}
\hbox{ansiapaper, ansibpaper, ansicpaper, ansidpaper, ansiepaper,}
\hbox{letterpaper, executivepaper, legalpaper}}]~\\[1ex] 
    specifies paper name.  The value part is ignored even if any.
    For example, the followings have the same effect:
    |a5paper|, |a5paper=true|, |a5paper=false| and so forth.
    |a[0-6]paper|, |b[0-6]paper| and |c[0-6]paper| are ISO A, B and C
    series of paper sizes respectively.
    The JIS (Japanese Industrial Standards) A-series is identical to the
    ISO A-series, but the JIS B-series is different from the ISO B-series.
    |b[0-6]j| should be used for the JIS B-series. 
\switchcolumn
\item[\onlypre \vtop{
\hbox{a0paper, a1paper, a2paper, a3paper, a4paper, a5paper, a6paper,}
\hbox{b0paper, b1paper, b2paper, b3paper, b4paper, b5paper, b6paper,}
\hbox{c0paper, c1paper, c2paper, c3paper, c4paper, c5paper, c6paper,}
\hbox{b0j, b1j, b2j, b3j, b4j, b5j, b6j,}
\hbox{ansiapaper, ansibpaper, ansicpaper, ansidpaper, ansiepaper,}
\hbox{letterpaper, executivepaper, legalpaper}}]~\\[1ex] 
指定纸张名称。即使有值部分,也会被忽略。
例如,下面的选项具有相同的效果:
|a5paper|, |a5paper=true|, |a5paper=false| 等等。
|a[0-6]paper|, |b[0-6]paper| 和 |c[0-6]paper| 分别是 ISO A、B 和 C
系列纸张大小。
JIS(日本工业标准)A 系列与 ISO A 系列相同,但 JIS B 系列与 ISO B 系列不同。
应该使用 |b[0-6]j| 来表示 JIS B 系列。

\switchcolumn[0]*
\item[\onlypre screen] a special paper size with (W,H) = (225mm,180mm).
For presentation with PC and video projector, ``|screen,centering|''
with `slide' documentclass would be useful.
\switchcolumn
\item[\onlypre screen] 一种特殊的纸张尺寸,宽度(W)为225mm,高度(H)为180mm。
对于使用个人电脑和视频投影仪进行演示,使用带有 `slide' 文档类的 ``|screen,centering|'' 会很有用。
\switchcolumn[0]*
\item[\onlypre paperwidth] width of the paper. |paperwidth=|\meta{length}.
\switchcolumn
\item[\onlypre paperwidth] 纸张的宽度。|paperwidth=|\meta{长度}。
\switchcolumn[0]*
\item[\onlypre paperheight] height of the paper. |paperheight=|\meta{length}.
\switchcolumn
\item[\onlypre paperheight] 纸张的高度。|paperheight=|\meta{长度}。
\switchcolumn[0]*
\item[\onlypre papersize] width and height of the paper. 
    |papersize=|\argii{width}{height} or |papersize=|\meta{length}.
\switchcolumn
\item[\onlypre papersize] 纸张的宽度和高度。|papersize=|\argii{宽度}{高度} 或 |papersize=|\meta{长度}。
\switchcolumn[0]*
\item[\onlypre landscape] switches the paper orientation to landscape mode.
\switchcolumn
\item[\onlypre landscape] 将纸张方向切换为横向模式。
\switchcolumn[0]*
\item[\onlypre portrait] switches the paper orientation to portrait mode.
This is equivalent to |landscape=false|.
\switchcolumn
\item[\onlypre portrait] 将纸张方向切换为纵向模式。
这相当于 |landscape=false|。
\end{paracol}
\end{Options}

\columnratio{0.55}
\begin{paracol}{2}
The options for paper names (e.g., |a4paper|) and orientation
(|portrait| and |landscape|) can be set as document class options. 
For example, you can set |\documentclass[a4paper,landscape]{article}|, 
then |a4paper| and |landscape| are processed in \Gm\ as well.
This is also the case for |twoside| and |twocolumn|
(see also Section~\ref{sec:dimension}).
\switchcolumn
纸张名称选项(例如,|a4paper|)和方向选项(|portrait| 和 |landscape|)可以作为文档类选项进行设置。
例如,您可以设置 |\documentclass[a4paper,landscape]{article}|,那么 |a4paper| 和 |landscape| 也会在 \Gm\ 中进行处理。
对于 |twoside| 和 |twocolumn| 也是如此(详见第~\ref{sec:dimension}~节)。
\end{paracol}
