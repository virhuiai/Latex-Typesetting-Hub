
\columnratio{0.55}
\begin{paracol}{2}
    \subsection{Auto-completion} \label{sec:rules}
\switchcolumn
\subsection{自动补全} 

\switchcolumn[0]*
Figure~\ref{fig:specrule} shows schematically how many specification
patterns exist and how to solve the ambiguity of the
specifications. Each axis shows the numbers of lengths
explicitly specified for body and margins. \Ss($m$,$b$) presents the
specification with a set of numbers $(\gpart{margin},\gpart{body})=(m,b)$.
\switchcolumn
图~\ref{fig:specrule} 简要展示了有多少种规格模式以及如何解决规格的模糊性。
每个轴表示正文和边距明确指定的长度的数量。$\text{S}_{(m,b)}$ 表示一组数字 $(\gpart{边距},\gpart{正文})=(m,b)$ 的规格。
\switchcolumn[0]*    
For example, the specification |width=14cm, left=3cm| is categorized
into \Ss(1,1), which is an adequate specification. If you add
|right=4cm|, it would be in \Ss(2,1) and overspecified. 
If only |width=14cm| is given, it's in \Ss(0,1), underspecified. 
\switchcolumn 例如,规格 |width=14cm, left=3cm| 归类为 $\text{S}{(1,1)}$,这是一个合适的规格。
如果添加 |right=4cm|,则属于 $\text{S}{(2,1)}$,即过度规定。
如果只给出 |width=14cm|,则属于 $\text{S}_{(0,1)}$,即不足规定。
\switchcolumn[0]*    
The \Gm\ package has the auto-completion mechanism, in which
if the layout parameters are underspecified or overspecified,
\Gm\ works out the ambiguity using the defaults and other
relations. Here are the specifications and the completion rules.
\switchcolumn\Gm\ 宏包具有自动补全机制,如果布局参数不足或过度规定,\Gm\ 会使用默认值和其他关系解决模糊性。以下是规格和补全规则。  
\end{paracol}



 \begin{figure}
  \centering
  {\unitlength=1pt
  \begin{picture}(400,150)(40,0)
  \put(1,49){\makebox(90,49)[r]{\large 0}}
  \put(1,1){\makebox(70,99)[r]{\large \gpart{body}}}
  \put(1,1){\makebox(90,49)[r]{\large 1}}
  \put(100,100){\makebox(99,20){\large 0}}
  \put(100,50){\framebox(99,49){}}
  \put(100,80){\makebox(99,15){\Ss(0,0)}}
  \put(100,50){\makebox(99,49){use |scale|}}
  {\linethickness{1pt}%
  \put(150,35){\line(0,1){30}}
  \put(150,35){\line(-1,3){4}}
  \put(150,35){\line(1,3){4}}}
  \put(100,0){\framebox(99,49){}}
  \put(100,2){\makebox(99,15){\Ss(0,1)}}
  \put(100,0){\makebox(99,49){use |marginratio|}}
  \put(200,120){\makebox(99,12){\large \gpart{margin}}}
  \put(200,100){\makebox(99,20){\large 1}}
  \put(200,50){\framebox(99,49){use |scale|}}
  \put(200,55){\makebox(89,10)[r]{\scriptsize\shortstack[l]{if |ratio|\\specified}}}
  \put(200,2){\makebox(99,15){\Ss(1,1)}}
  {\linethickness{1pt}%
  \put(250,35){\line(0,1){30}}
  \put(250,35){\line(-1,3){4}}
  \put(250,35){\line(1,3){4}}}
  {\linethickness{1pt}%
  \put(225,25){\line(-1,0){35}}
  \put(225,25){\line(-3,-1){12}}
  \put(225,25){\line(-3,1){12}}}
  \put(200,80){\makebox(99,15){\Ss(1,0)}}
  \put(200,0){\framebox(99,49){\textcolor{red}{\textit{solvable}}}}
  \put(300,100){\makebox(99,20){\large 2}}
  \put(300,80){\makebox(99,15){\Ss(2,0)}}
  \put(300,50){\framebox(99,49){\textcolor{red}{\textit{solvable}}}}
  \put(300,0){\framebox(99,49){forget |body|}}
  \put(300,2){\makebox(99,15){\Ss(2,1)}}
  {\linethickness{1pt}%
  \multiput(290,65)(5,0){6}{\line(1,0){3}}
  \put(320,65){\line(-3,-1){12}}
  \put(320,65){\line(-3,1){12}}}
  {\linethickness{1pt}%
  \put(350,65){\line(0,-1){30}}
  \put(350,65){\line(-1,-3){4}}
  \put(350,65){\line(1,-3){4}}}
  \end{picture}}
  \caption[Specifications and completion rules]{%
  \begin{minipage}[t]{.7\textwidth}\raggedright\small
  Specifications \Ss(0,0) to \Ss(2,1) and the completion rules
  (arrows). Column and row numbers denote the number of explicitly
  specified lengths for margin and body respectively. \Ss($m$,$b$) denote a
  specification with a set of the numbers $(\gpart{margin},\gpart{body})=(m,b)$. \\
  规格 \Ss(0,0) 到 \Ss(2,1) 和补全规则(箭头)。列和行号分别表示边距和正文明确指定长度的数量。$\text{S}_{(m,b)}$ 表示一组数字 $(\gpart{边距},\gpart{正文})=(m,b)$ 的规格。
  \end{minipage}}
  \label{fig:specrule}
 \end{figure}

\begin{Spec}
\columnratio{0.55}
\begin{paracol}{2}
\switchcolumn[0]*\item[\Ss(0,0)]
 Nothing is specified. The \Gm\ package sets \gpart{body} with the
 default |scale| ($=0.7$). \\ For example, |width| is set to be
 $|0.7|\times|layoutwidth|$. Note that by default |layoutwidth| 
 and |layoutheight| will be equal to |\paperwidth| and |\paperheight|
 respectively.
 Thus \Ss(0,0) goes to \Ss(0,1). See \Ss(0,1).
 \bigskip
 \switchcolumn
 \item[\Ss(0,0)]
 未指定任何内容。 \Gm\ 宏包使用默认值 |scale|($=0.7$)设置 \gpart{正文}。例如,|width| 被设置为 $|0.7|\times|layoutwidth|$。需要注意的是,默认情况下,|layoutwidth| 和 |layoutheight| 分别等于 |\paperwidth| 和 |\paperheight|。因此,\Ss(0,0) 转向 \Ss(0,1)。请参见 \Ss(0,1)。
 \bigskip

\switchcolumn[0]*\item[\Ss(0,1)]
 Only \gpart{body} is specified, such as |width=7in|, |lines=20|,
 |body={20cm,24cm}|, |scale=0.9| and so forth.
 Then \Gm\ sets margins with the margin ratio.
 If the margin ratio is not specified, the default is used.
 The default vertical margin ratio is defined as
 \begin{equation}
  |top| : |bottom| = 2 : 3 \qquad\textit{default}.
 \end{equation}
 As for the horizontal margin ratio, the default value depends on
 whether the document is onesided or twosided,
 \begin{equation}
  |left|\;(|inner|) : |right|\;(|outer|) 
       = \left\{ \begin{array}{ll}
              1 : 1 \qquad\textit{default for oneside},\\
              2 : 3 \qquad\textit{default for twoside}.
         \end{array}\right.
 \end{equation}
 For example, if |height=22cm| is specified on A4 paper, 
 \Gm\ calculates |top| margin as follows:
 \begin{equation}
   \begin{array}{ll}
   |top| &= ( |layoutheight| - |height| ) \times 2/5 \\
         &= (29.7-22)\times2/5 = 3.08\textrm{(cm)}
   \end{array}
 \end{equation}
 Thus |top| margin and body |height| have been determined, the
 specification for the vertical goes to \Ss(1,1) and
 all the parameters can be solved.
 \bigskip
 \switchcolumn
 \item[\Ss(0,1)]
 只指定了 \gpart{正文},例如 |width=7in|,|lines=20|,|body={20cm,24cm}|,|scale=0.9| 等。然后,\Gm\ 使用边距比例设置边距。如果未指定边距比例,则使用默认值。默认的垂直边距比例定义为
 \begin{equation}
  |top| : |bottom| = 2 : 3 \qquad\textit{默认值}.
 \end{equation}
 至于水平边距比例,其默认值取决于文档是单面还是双面的情况,
 \begin{equation}
  |left|\;(|inner|) : |right|\;(|outer|) 
       = \left\{ \begin{array}{ll}
              1 : 1 \qquad\textit{单面的默认值},\\
              2 : 3 \qquad\textit{双面的默认值}.
         \end{array}\right.
 \end{equation}
 例如,在 A4 纸上指定 |height=22cm|,\Gm\ 计算 |top| 边距如下:
 \begin{equation}
   \begin{array}{ll}
   |top| &= ( |layoutheight| - |height| ) \times 2/5 \\
         &= (29.7-22)\times2/5 = 3.08\textrm{(cm)}
   \end{array}
 \end{equation}
 因此,确定了 |top| 边距和正文 |height|,垂直规格转到 \Ss(1,1),所有参数都可以解决。
 \bigskip

\switchcolumn[0]*\item[\Ss(1,0)]
 Only one margin is specified, such as |bottom=2cm|, |left=1in|,
 |top=3cm|, and so forth.
\switchcolumn
\item[\Ss(1,0)]只指定了一个边距,例如 |bottom=2cm|,|left=1in|,|top=3cm| 等。

\begin{itemize}
\switchcolumn[0]*\item \textbf{If the margin ratio is \textit{not} specified}, \Gm\ sets
 \gpart{body} with the default |scale| ($=0.7$). 
 For example, if |top=2.4cm| is specified, \Gm\ sets
 \begin{center}
     $|height|= |0.7|\times|layoutheight|$ 
            ~~($=|0.7\paperheight|$ by default),
 \end{center}
 then \Ss(1,0) goes to \Ss(1,1), in which |bottom| is calculated
 with $|layoutheight|-(|height|+|top|)$ and results in 6.51cm on A4
 paper if the layout size is equal to the paper size.
 \medskip
 \switchcolumn
 \item \textbf{如果未指定边距比例},\Gm\ 使用默认值 |scale|($=0.7$)设置 \gpart{正文}。例如,如果指定了 |top=2.4cm|,\Gm\ 设置
 \begin{center}
     $|height|= |0.7|\times|layoutheight|$ 
            ~~($=|0.7\paperheight|$ 默认情况下),
 \end{center}
 然后,\Ss(1,0) 转向 \Ss(1,1),在其中使用 $|layoutheight|-(|height|+|top|)$ 计算 |bottom|,如果布局大小等于纸张大小,则在 A4 纸上结果为 6.51cm。
 \medskip

\switchcolumn[0]*\item \textbf{If the margin ratio is specified}, such as
 |hmarginratio={1:2}|, |vratio={3:4}| and so forth,
 \Gm\ sets the other% margin with the specified margin ratio.
 For example, if a set of options ``|top=2.4cm,vratio={3:4}|'' is
 specified, \Gm\ sets |bottom| to be |3.2cm| calculating
 \begin{center}
     $|bottom|= |top|/3\times4 = 3.2\textrm{cm}$
 \end{center}
 Thus \Ss(1,0) goes to \Ss(2,0).
 

 \switchcolumn
 \item \textbf{如果指定了边距比例},例如 |hmarginratio={1:2}|,|vratio={3:4}| 等,\Gm\ 使用指定的边距比例设置另一个边距。例如,如果指定了一组选项 ``|top=2.4cm,vratio={3:4}|'',\Gm\ 将 |bottom| 设置为 |3.2cm|,计算如下:
 \begin{center}
     $|bottom|= |top|/3\times4 = 3.2\textrm{cm}$
 \end{center}
 因此,\Ss(1,0) 转向 \Ss(2,0)。
 \end{itemize}

\switchcolumn[0]*
 Note that the version 4 or earlier used to set the other margin
 with the margin ratio. In the version 5, therefore, with the
 same specification, the result will be different from the one in the
 version 4. For example, if only |top=2.4cm| is specified, 
 you got |bottom=2.4cm| in the version 4 or earlier, but you will get
 |bottom=6.51cm| in the version 5.
 \bigskip
 \switchcolumn
 需要注意的是,版本4或更早的版本通常使用边距比例设置另一个边距。因此,在版本5中,如果使用相同的规格,结果将与版本4中的结果不同。例如,如果只指定 |top=2.4cm|,在版本4或更早的版本中,你将得到 |bottom=2.4cm|,但在版本5中,你将得到 |bottom=6.51cm|。
 \bigskip

\switchcolumn[0]*\item[\Ss(2,1)]
 The \gpart{body} and two \gpart{margins} are all specified, such as
  |vdivide={1in,8in,1.5in}|, ``|left=3cm,width=13cm,right=4cm|'' and
  so forth. Since \Gm\ basically gives priority to \gpart{margins}
  if dimensions are overspecified, \Gm\ forgets and resets
  \gpart{body}. For example, if you specify
 \begin{center}
    |\usepackage[a4paper,left=3cm,width=13cm,right=4cm]{geometry}|,
 \end{center}
 |width| is reset to be 14cm because the width of a A4 paper is 21cm
 long.
 \switchcolumn
 \item[\Ss(2,1)]
指定了 \gpart{正文} 和两个 \gpart{边距},例如 |vdivide={1in,8in,1.5in}|,``|left=3cm,width=13cm,right=4cm|'' 等。由于 \Gm\ 基本上优先考虑 \gpart{边距},如果尺寸被超指定,\Gm\ 将忽略并重新设置 \gpart{正文}。例如,如果你指定
 \begin{center}
    |\usepackage[a4paper,left=3cm,width=13cm,right=4cm]{geometry}|,
 \end{center}
 |width| 将被重置为 14cm,因为 A4 纸的宽度是 21cm。
\end{paracol}
 \end{Spec}
