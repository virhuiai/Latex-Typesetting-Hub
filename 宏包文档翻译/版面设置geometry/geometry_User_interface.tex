\section{User interface\hfill 用户接口}

\subsection{Commands\hfill 命令}

\columnratio{0.55}
\begin{paracol}{2}
The \Gm\ package provides the following commands:
\switchcolumn
\Gm\ 宏包提供以下命令:
\end{paracol}

\begin{itemize}\setlength{\itemsep}{-.5\parsep}
\columnratio{0.55}
\begin{paracol}{2}
\item |\geometry{|\meta{options}|}|
\switchcolumn
\item |\geometry{|\meta{options}|}|

\switchcolumn
\item |\newgeometry{|\meta{options}|}| and |\restoregeometry|
\switchcolumn
\item |\newgeometry{|\meta{options}|}| 和 |\restoregeometry|

\switchcolumn
\item |\savegeometry{|\meta{name}|}| and |\loadgeometry{|\meta{name}|}|
\switchcolumn
\item |\savegeometry{|\meta{name}|}| 和 |\loadgeometry{|\meta{name}|}|
\end{paracol}
\end{itemize}

% \columnratio{0.55}
% \begin{paracol}{2}
% The \Gm\ package provides the following commands:
% \begin{itemize}\setlength{\itemsep}{-.5\parsep}
% \item |\geometry{|\meta{options}|}|
% \item |\newgeometry{|\meta{options}|}| and |\restoregeometry|
% \item |\savegeometry{|\meta{name}|}| and |\loadgeometry{|\meta{name}|}|
% \end{itemize}
% \switchcolumn
% \Gm\ 宏包提供以下命令:
% \begin{itemize}\setlength{\itemsep}{-.5\parsep}
% \item |\geometry{|\meta{options}|}|
% \item |\newgeometry{|\meta{options}|}| 和 |\restoregeometry|
% \item |\savegeometry{|\meta{name}|}| 和 |\loadgeometry{|\meta{name}|}|
% \end{itemize}
% \end{paracol}

\columnratio{0.55}
\begin{paracol}{2}
|\geometry{|\meta{options}|}|
changes the page layout according to the options specified in the
argument. This command, if any, should be placed only in the
preamble (before |\begin{document}|).
\switchcolumn
|\geometry{|\meta{options}|}|根据参数中指定的选项更改页面布局。如果有的话,此命令应该只放在导言区(|\begin{document}|之前)。
%%%
\switchcolumn
The \Gm\ package may be used as part of a class or another package
you use in your document. The command \cs{geometry} can overwrite
some of the settings in the preamble. Multiple use of \cs{geometry}
is allowed and then processed with the options concatenated.
If \Gm\ is not yet loaded, you can use only
|\usepackage[|\meta{options}|]{geometry}| instead of \cs{geometry}.
\switchcolumn
\Gm\ 宏包可以作为文档中使用的类或其他宏包的一部分。命令\cs{geometry}可以覆盖导言区中的一些设置。可以多次使用\cs{geometry}命令,并将选项连接在一起进行处理。如果\Gm\ 尚未加载,可以使用|\usepackage[|\meta{options}|]{geometry}|代替\cs{geometry}。    
\switchcolumn
\medskip
|\newgeometry{|\meta{options}|}|
changes the page layout mid-document. \cs{newgeometry} is almost
similar to \cs{geometry} except that \cs{newgeometry} disables all
the options specified by \cs{usepackage} and \cs{geometry} in
the preamble and skips papersize-related options. 
\cs{restoregeometry}
restores the page layout specified in the preamble. This command
has no arguments. See Section~\ref{sec:midchange} for details.
\switchcolumn
\medskip
|\newgeometry{|\meta{options}|}|可以在文档中更改页面布局。\cs{newgeometry}与\cs{geometry}几乎相同,除了会禁用导言区中由\cs{usepackage}和\cs{geometry}指定的所有选项,并且跳过与页面尺寸相关的选项外。|\restoregeometry|命令将恢复导言区中指定的页面布局。此命令没有参数。详见第~\ref{sec:midchange} 节。
\switchcolumn[0]
\medskip
|\savegeometry{|\meta{name}|}|
saves the page dimensions as \meta{name} where you put
this command.
|\loadgeometry{|\meta{name}|}|
loads the page dimensions saved as \meta{name}. See
Section~\ref{sec:midchange} for details.
\switchcolumn
\medskip
|\savegeometry{|\meta{name}|}|将页面尺寸保存为\meta{name},可以在命令所在的位置使用它。
|\loadgeometry{|\meta{name}|}|加载保存为\meta{name}的页面尺寸。详见第~\ref{sec:midchange} 节。
\end{paracol}


\subsection{Optional argument}

The \Gm\ package adopts \textsf{keyval} interface
`\meta{key}=\meta{value}' for the optional argument to
\cs{usepackage}, \cs{geometry} and \cs{newgeometry}.

The argument includes a list of comma-separated \textsf{keyval}
options and has basic rules as follows:
\begin{itemize}\setlength{\itemsep}{-.5\parsep}
\item Multiple lines are allowed, while blank lines are not.
\item Any spaces between words are ignored.
\item Options are basically order-independent.
(There are some exceptions. See Section~\ref{sec:optionorder} for details.)
\end{itemize}
 For example,
\begin{quote}
|\usepackage[ a5paper ,  hmargin = { 3cm,|\\
|                .8in } , height|\\
|         =  10in ]{geometry}|
\end{quote}
is equivalent to 
\begin{quote}
  |\usepackage[height=10in,a5paper,hmargin={3cm,0.8in}]{geometry}|
\end{quote}
Some options are allowed to have sub-list, e.g. |{3cm,0.8in}|.
Note that the order of values in the sub-list is significant.
The above setting is also equivalent to the followings:
\begin{quote}
  |\usepackage{geometry}|\\
  |\geometry{height=10in,a5paper,hmargin={3cm,0.8in}}|
\end{quote}
or 
\begin{quote}
  |\usepackage[a5paper]{geometry}|\\
  |\geometry{hmargin={3cm,0.8in},height=8in}|\\
  |\geometry{height=10in}|.
\end{quote}
Thus, multiple use of \cs{geometry} just appends options.

\Gm\ supports package 
\textsl{calc}\footnote{CTAN:~\texttt{macros/latex/required/tools}}.
For example,
\begin{quote}
  |\usepackage{calc}|\\
  |\usepackage[textheight=20\baselineskip+10pt]{geometry}|
\end{quote}

\subsection{Option types}
\Gm\ options are categorized into four types:

\begin{enumerate}\itemsep=0pt
\item \textbf{Boolean type}

   takes a boolean value (|true| or |false|). If no value,
   |true| is set by default.
   \begin{quote}
      \meta{key}|=true|\OR|false|.\\
      \meta{key} with no value is equivalent to 
      \meta{key}|=true|.
   \end{quote}
   \textit{Examples:}~ |verbose=true|, |includehead|, 
   |twoside=false|.\\
   Paper name is the exception. The preferred paper name should be set
   with no values. Whatever value is given, it is ignored. For
   instance, |a4paper=XXX| is equivalent to |a4paper|.

\item \textbf{Single-valued type}

   takes a mandatory value.
   \begin{quote}
   \meta{key}|=|\meta{value}.
   \end{quote}
   \textit{Examples:}~ |width=7in|, |left=1.25in|,
   |footskip=1cm|, |height=.86\paperheight|.

\item \textbf{Double-valued type}

   takes a pair of comma-separated values in braces. The two values can
   be shortened to one value if they are identical.
   \begin{quote}
   \meta{key}|=|\argii{value1}{value2}.\\
   \meta{key}|=|\meta{value} is equivalent to 
             \meta{key}|=|\argii{value}{value}.
   \end{quote}
   \textit{Examples:}~ |hmargin={1.5in,1in}|, |scale=0.8|,
   |body={7in,10in}|.

\item \textbf{Triple-valued type}

   takes three mandatory, comma-separated values in braces.
   \begin{quote}
   \meta{key}|=|\argiii{value1}{value2}{value3}
   \end{quote}
   Each value must be a dimension or null. When you give an empty value
   or `|*|', it means null and leaves the appropriate value 
   to the auto-completion mechanism. You need to specify at least one
   dimension, typically two dimensions. You can set nulls for all the 
   values, but it makes no sense.
   \textit{Examples:}\\
   \hspace*{2em} |hdivide={2cm,*,1cm}|, |vdivide={3cm,19cm, }|,
                  |divide={1in,*,1in}|.
\end{enumerate}
