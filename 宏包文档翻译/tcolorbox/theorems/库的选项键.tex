\subsection{Option Keys of the Library\\库的选项键}


\begin{docTcbKey}{separator sign}{=\meta{sign}}{no default, initially |:|}
The given \meta{sign} is used inside the title text of a theorem
as separater between display name combined with number and
the specific title text. It is omitted, if there is no specific title text.

在定理的标题文本中,给定的 \meta{sign} 用作显示名称和编号与特定标题文本之间的分隔符。如果没有特定的标题文本,则省略它。
\begin{dispExample}
% \usepackage{amssymb}
\newtcbtheorem[use counter from=mytheo]{sometheorem}{Theorem}%
  {colback=white,colframe=red!50!black,fonttitle=\bfseries,
   separator sign={\ $\blacktriangleright$}}{theo}
\begin{sometheorem}{My example}{}
我的定理文本。
\end{sometheorem}
\end{dispExample}
\end{docTcbKey}

\begin{docTcbKey}{separator sign colon}{}{style, no value, initially set}
Sets \refKey{/tcb/separator sign} to the default colon |:| sign.

将\refKey{/tcb/separator sign}设置为默认的冒号|:|符号。
\end{docTcbKey}

\begin{docTcbKey}{separator sign dash}{}{style, no value}
Sets \refKey{/tcb/separator sign} to an en-dash sign.

将\refKey{/tcb/separator sign}设置为一个破折号符号。
\begin{dispExample}
\newtcbtheorem[use counter from=mytheo]{sometheorem}{Theorem}%
  {colback=white,colframe=red!50!black,fonttitle=\bfseries,
   separator sign dash}{theo}
\begin{sometheorem}{My example}{}
我的定理文本。
\end{sometheorem}
\end{dispExample}
\end{docTcbKey}

\begin{docTcbKey}{separator sign none}{}{style, no value}
Sets \refKey{/tcb/separator sign} to empty.

将\refKey{/tcb/separator sign}的集合设置为空。
\begin{dispExample}
\newtcbtheorem[use counter from=mytheo]{sometheorem}{Theorem}%
  {colback=white,colframe=red!50!black,fonttitle=\bfseries,
   separator sign none}{theo}
\begin{sometheorem}{My example}{}
我的定理文本。
\end{sometheorem}
\end{dispExample}
\end{docTcbKey}

% \clearpage
\begin{docTcbKey}{description delimiters}{=\marg{left}\marg{right}}{no default, initially empty}
The given \meta{left} and \meta{right} delimiter signs are used to frame
the descriptive title text of a theorem.

给定的\meta{left}和\meta{right}分隔符号用于框定定理的描述性标题文本。
\begin{dispExample}
\newtcbtheorem[use counter from=mytheo]{sometheorem}{Theorem}%
  {colback=white,colframe=red!50!black,fonttitle=\bfseries,
   description delimiters={\flqq}{\frqq}}{theo}
\begin{sometheorem}{My example}{}
我的定理文本。
\end{sometheorem}
\end{dispExample}
\end{docTcbKey}


\begin{docTcbKey}{description delimiters parenthesis}{}{style, no value}
Sets \refKey{/tcb/description delimiters} to |(| and |)|.

将 \refKey{/tcb/description delimiters} 设置为 |(| 和 |)|。
\begin{dispExample}
\newtcbtheorem[use counter from=mytheo]{sometheorem}{Theorem}%
  {colback=white,colframe=red!50!black,fonttitle=\bfseries,
   description delimiters parenthesis}{theo}
\begin{sometheorem}{My example}{}
我的定理文本。
\end{sometheorem}
\end{dispExample}
\end{docTcbKey}


\begin{docTcbKey}{description delimiters none}{}{style, no value, initially set}
Sets \refKey{/tcb/description delimiters} to the default empty texts.

将 \refKey{/tcb/description delimiters} 设置为默认的空文本。
\end{docTcbKey}


\begin{docTcbKey}{description color}{\colOpt{=\meta{color}}}{default empty, initially empty}
Sets the \meta{color} of the descriptive title text deviating from \refKey{/tcb/coltitle}.
The color is reset to \refKey{/tcb/coltitle}, if |description color| is used without value.

设置与 \refKey{/tcb/coltitle} 不同的描述标题文本的 \meta{color}。如果使用不带值的 |description color|,则颜色将重置为 \refKey{/tcb/coltitle}。
\begin{dispExample}
\newtcbtheorem[use counter from=mytheo]{sometheorem}{Theorem}%
  {colback=white,colframe=red!50!black,fonttitle=\bfseries,
   description color=red!25!yellow}{theo}
\begin{sometheorem}{My example}{}
我的定理文本。
\end{sometheorem}
\end{dispExample}
\end{docTcbKey}

% \clearpage
\begin{docTcbKey}{description font}{\colOpt{=\meta{text}}}{default empty, initially empty}
Sets \meta{text} (e.\,g.\ font settings) before the descriptive title text deviating from \refKey{/tcb/fonttitle}.
The \meta{text} is removed, if |description font| is used without value.

在与 \refKey{/tcb/fonttitle} 不同的描述标题文本之前设置 \meta{text}(例如字体设置)。如果使用不带值的 |description font|,则会删除 \meta{text}。
\begin{dispExample}
\newtcbtheorem[use counter from=mytheo]{sometheorem}{Theorem}%
  {colback=white,colframe=red!50!black,fonttitle=\bfseries,
   description delimiters={\glqq}{\grqq},
   description font=\mdseries\itshape}{theo}
\begin{sometheorem}{My example}{}
我的定理文本。
\end{sometheorem}
\end{dispExample}
\end{docTcbKey}


\begin{docTcbKey}{description formatter}{\colOpt{=\meta{macro}}}{default empty, initially empty}
Sets \meta{macro} as formatter for the descriptive title text. The \meta{macro}
has to take one mandatory argument (the description text).\\ 
Note that \refKey{/tcb/description delimiters}, \refKey{/tcb/description color},
and \refKey{/tcb/description font} are ignored, if this option is used.\\
If |description formatter| is used without value, the formatter is reset
to its standard behavior.

将\meta{宏}设置为描述标题文本的格式化程序。 \meta{宏}必须接受一个必需参数(描述文本)。\ 请注意,如果使用此选项,则忽略\refKey{/tcb/description delimiters},\refKey{/tcb/description color}和\refKey{/tcb/description font}。\ 如果使用|description formatter|而没有值,则格式化程序将重置为其标准行为。
\begin{dispExample}
\newtcbox{\formbox}{enhanced,frame empty,size=minimal,boxsep=2pt,arc=1pt,
  on line,interior style image=goldshade.png}

\newtcbtheorem[use counter from=mytheo]{sometheorem}{Theorem}%
  {colback=white,colframe=red!50!black,fonttitle=\bfseries,
   description formatter=\formbox}{theo}
\begin{sometheorem}{My example}{}
我的定理文本。
\end{sometheorem}
\end{dispExample}
\end{docTcbKey}


\begin{docTcbKey}{terminator sign}{=\meta{sign}}{no default, initially empty}
The given \meta{sign} is used as terminator at the end of the title text of a theorem.

给定的\meta{符号}用作定理标题文本末尾的终止符。
\begin{dispExample}
\newtcbtheorem[use counter from=mytheo]{sometheorem}{Theorem}%
  {colback=white,colframe=red!50!black,fonttitle=\bfseries,
   terminator sign={.}}{theo}
\begin{sometheorem}{My example}{}
我的定理文本。
\end{sometheorem}
\end{dispExample}
\end{docTcbKey}

% \clearpage
\begin{docTcbKey}{terminator sign colon}{}{style, no value, initially set}
Sets \refKey{/tcb/terminator sign} to the colon |:| sign.

将\refKey{/tcb/terminator sign}设置为冒号|:|符号。
\begin{dispExample}
\newtcbtheorem[use counter from=mytheo]{sometheorem}{Theorem}%
  {colback=white,colframe=red!50!black,fonttitle=\bfseries,
   separator sign dash,terminator sign colon}{theo}
\begin{sometheorem}{My example}{}
我的定理文本。
\end{sometheorem}
\end{dispExample}
\end{docTcbKey}

\begin{docTcbKey}{terminator sign dash}{}{style, no value}
Sets \refKey{/tcb/terminator sign} to an en-dash sign.

将\refKey{/tcb/terminator sign}设置为短横线符号。
\begin{dispExample}
\newtcbtheorem[use counter from=mytheo]{sometheorem}{Theorem}%
  {colback=white,colframe=red!50!black,fonttitle=\bfseries,
   terminator sign dash}{theo}
\begin{sometheorem}{My example}{}
我的定理文本。
\end{sometheorem}
\end{dispExample}
\end{docTcbKey}

\begin{docTcbKey}{terminator sign none}{}{style, no value}
Sets \refKey{/tcb/terminator sign} to the default empty text.

将\refKey{/tcb/terminator sign}设置为默认的空文本。
\end{docTcbKey}


\begin{docTcbKey}[][doc new=2016-04-19]{label separator}{=\meta{separator}}{no default, initially |:|}
The given \meta{separator} is used for labels created with environments which
are defined themselves by \refCom{newtcbtheorem}. This \meta{separator} is
put between \meta{prefix} (defined by \refCom{newtcbtheorem})
and \meta{marker} (defined by an actual theorem environment).

给定的 \meta{separator} 用于通过 \refCom{newtcbtheorem} 定义的环境创建的标签。这个 \meta{separator} 被放置在 \meta{prefix}(由 \refCom{newtcbtheorem} 定义)和 \meta{marker}(由实际的定理环境定义)之间。
\begin{dispExample}
\newtcbtheorem[use counter from=mytheo]{sometheorem}{Theorem}%
  {colback=white,colframe=red!50!black,fonttitle=\bfseries,
   label separator=*}{theo}
\begin{sometheorem}{My example}{myex}
我的定理文本。
\end{sometheorem}
See Example~\ref{theo*myex}.
\end{dispExample}
\end{docTcbKey}


% \clearpage

\begin{docTcbKey}[][doc new=2018-01-12]{theorem full label supplement}{=\marg{style}}{no default, initially empty}
The given \meta{style} is used in connection with labels created with environments which
are defined themselves by \refCom{newtcbtheorem}.
This \meta{style} uses one argument which is automatically set to the
full label marker of the environment, i.e. a text consisting of
\meta{prefix} (defined by \refCom{newtcbtheorem}),
\refKey{/tcb/label separator},
and \meta{marker} (defined by an actual theorem environment).

所给出的 \meta{style} 用于与由 \refCom{newtcbtheorem} 定义的环境创建的标签相关联。这个 \meta{style} 使用一个参数,该参数自动设置为环境的完整标签标记,即由 \refCom{newtcbtheorem} 定义的 \meta{prefix}、\refKey{/tcb/label separator} 和实际定理环境定义的 \meta{marker} 组成的文本。
\begin{dispExample}
% The following adds a hyper target to all environments
% created with \newtcbtheorem
% 以下代码将为所有使用 \newtcbtheorem 命令创建的环境添加超链接目标。
\tcbset{theorem full label supplement={hypertarget={#1}}}

\newtcbtheorem[use counter from=mytheo]{sometheorem}{Theorem}%
   {colback=white,colframe=red!50!black,fonttitle=\bfseries}{theo}
\begin{sometheorem}{My example}{myex2}
我的定理文本。
\end{sometheorem}
This automated \hyperlink{theo:myex2}{hyper target can be linked to with a
 hyper link}.

这个自动化的\hyperlink{theo:myex2}{超链接目标}可以通过超链接链接。
\end{dispExample}

A second usage of \refKey{/tcb/theorem full label supplement} overwrites
the first setting.

对\refKey{/tcb/theorem full label supplement}的第二次使用会覆盖第一次的设置。
\end{docTcbKey}


\begin{docTcbKey}[][doc new=2018-01-12]{theorem label supplement}{=\marg{style}}{no default, initially empty}
The given \meta{style} is used in connection with labels created with environments which
are defined themselves by \refCom{newtcbtheorem}.
This \meta{style} uses one argument which is automatically set to the
label \meta{marker} defined by an actual theorem environment.\par
A second usage of \refKey{/tcb/theorem label supplement} overwrites
the first setting, but
\refKey{/tcb/theorem full label supplement}
and \refKey{/tcb/theorem label supplement} can be used independently.

给定的\meta{style}与使用\refCom{newtcbtheorem}定义的环境创建的标签相关联。 这个\meta{style}使用一个参数,该参数自动设置为实际定理环境定义的标签\meta{marker}。 第二次使用\refKey{/tcb/theorem label supplement}会覆盖第一次设置,但\refKey{/tcb/theorem full label supplement}和\refKey{/tcb/theorem label supplement}可以独立使用。
\begin{dispExample}
% `marginnote' has to be loaded
\newtcbtheorem[use counter from=mytheo]{sometheorem}{Theorem}%
   {colback=white,colframe=red!50!black,fonttitle=\bfseries,
    theorem label supplement={hypertarget={XYZ-##1}},
    theorem full label supplement={code={\marginnote{##1}}}
   }{theo}
\begin{sometheorem}{My example}{myex3}
我的定理文本。
\end{sometheorem}
This automated \hyperlink{XYZ-myex3}{hyper target can be linked to with a
hyper link}.

此自动化的超链接目标可与超链接链接。
\end{dispExample}
\end{docTcbKey}
% \clearpage

\begin{docTcbKey}[][doc new=2020-10-21]{theorem hanging indent}{\colOpt{=\docValue*{auto}\textbar \meta{length}}}{default \docValue*{auto}, initially \docValue*{auto}}
Sets the hanging indent of the theorem title to \docValue{auto} or the
given \meta{length}.
For \docValue{auto}, the hanging indent matches the display name, number and
separator sign of the theorem.
If \meta{length} is negative, the theorem title is indented positively
without hanging indent.

将定理标题的悬挂缩进设置为\docValue{auto}或给定的\meta{length}。 对于\docValue{auto},悬挂缩进与定理的显示名称、编号和分隔符匹配。 如果\meta{length}为负数,则定理标题会向正方向缩进而不是悬挂缩进。

\begin{dispExample}
\newtcbtheorem[use counter from=mytheo]{sometheorem}{Theorem}%
{colback=white,colframe=red!50!black,fonttitle=\bfseries}{theo}
\begin{sometheorem}{这是一个相当简短且几乎为空的定理的非常长且复杂的标%
题题题题题题题题题题题题题题题题题题题题题题}{myexA1}
我的定理文本。
\end{sometheorem}
\end{dispExample}
%%%%%
\begin{dispExample}
\newtcbtheorem[use counter from=mytheo]{sometheorem}{Theorem}%
{colback=white,colframe=red!50!black,fonttitle=\bfseries}{theo}
\begin{sometheorem}[theorem hanging indent=5mm]{这是一个相当简短且几乎为空的定理的非常长且复杂的标%
    题题题题题题题题题题题题题题题题题题题题题题}{myexA2}
我的定理文本。
\end{sometheorem}
\end{dispExample}
%%%%
\begin{dispExample}
\newtcbtheorem[use counter from=mytheo]{sometheorem}{Theorem}%
{colback=white,colframe=red!50!black,fonttitle=\bfseries}{theo}
\begin{sometheorem}[theorem hanging indent=0pt]{这是一个相当简短且几乎为空的定理的非常长且复杂的标%
    题题题题题题题题题题题题题题题题题题题题题题}{myexA3}
我的定理文本。
\end{sometheorem}
\end{dispExample}

\begin{dispExample}
\newtcbtheorem[use counter from=mytheo]{sometheorem}{Theorem}%
{colback=white,colframe=red!50!black,fonttitle=\bfseries}{theo}
\begin{sometheorem}[theorem hanging indent=-5mm]{这是一个相当简短且几乎为空的定理的非常长且复杂的标%
    题题题题题题题题题题题题题题题题题题题题题题}{myexA4}
我的定理文本。
\end{sometheorem}
\end{dispExample}
\end{docTcbKey}



% \clearpage
\begin{docTcbKey}{theorem name and number}{}{style, no value, initially set}
Prints theorem name followed by theorem number inside the title.

在标题中打印定理名称,后面跟随定理编号。
\begin{dispExample}
\newtcbtheorem[use counter from=mytheo]{sometheorem}{Theorem}%
  {colback=white,colframe=red!50!black,fonttitle=\bfseries,
   theorem name and number}{theo}
\begin{sometheorem}{My example}{}
我的定理文本。
\end{sometheorem}
\end{dispExample}
\end{docTcbKey}


\begin{docTcbKey}{theorem number and name}{}{style, no value}
Prints theorem number followed by theorem name inside the title.

在标题中打印定理编号,后面跟随定理名称。
\begin{dispExample}
\newtcbtheorem[use counter from=mytheo]{sometheorem}{Theorem}%
  {colback=white,colframe=red!50!black,fonttitle=\bfseries,
   theorem number and name}{theo}
\begin{sometheorem}{My example}{}
我的定理文本。
\end{sometheorem}
\end{dispExample}
\end{docTcbKey}

\begin{docTcbKey}{theorem name}{}{style, no value}
Prints theorem name without number inside the title.

在标题中打印定理名称,不包括编号。
\begin{dispExample}
\newtcbtheorem[use counter from=mytheo]{sometheorem}{Theorem}%
  {enhanced,colback=white,colframe=red!50!black,fonttitle=\bfseries,
   theorem name,watermark text={\thetcbcounter}}{theo}
\begin{sometheorem}{My example}{}
我的定理文本。
\end{sometheorem}
\end{dispExample}
\end{docTcbKey}

% \enlargethispage*{20mm}

\begin{docTcbKey}[][doc new=2021-12-03]{theorem number}{}{style, no value}
Prints theorem number without name inside the title.

在标题中打印定理编号,不包括名称。
\begin{dispExample}
\newtcbtheorem[use counter from=mytheo]{sometheorem}{Theorem}%
  {enhanced,colback=white,colframe=red!50!black,fonttitle=\bfseries,
   theorem number}{theo}
\begin{sometheorem}{My example}{}
我的定理文本。
\end{sometheorem}
\end{dispExample}
\end{docTcbKey}


% \clearpage
\begin{docTcbKey}{theorem}{=\marg{display name}\marg{counter}\marg{title}\marg{marker}}{no default}
This key can be used
directly in a |tcolorbox| for a more flexible approach to create a
theorem type box.
The \meta{display name} is used together with the increased \meta{counter} value
and the \meta{title} for the title line of the box. Additionally, a
|\label| with the given \meta{marker} is created.

这个键可以直接用于 |tcolorbox|,以更灵活的方式创建定理类型的框。 \meta{display name} 和增加的 \meta{counter} 值以及 \meta{title} 一起用于框的标题行。此外,还创建了一个带有给定 \meta{marker} 的 |\label|。
\begin{dispExample}
% \newcounter{texercise}%  preamble
\begin{tcolorbox}[colback=green!10,colframe=green!50!black,arc=4mm,
                  theorem={Test}{texercise}{Direct usage}{myMarker}]
Here, we see the test \ref{myMarker}.

在这里,我们看到测试 \ref{myMarker}。
\end{tcolorbox}
\end{dispExample}
For a common appearance inside the document, the key |theorem| should not be
used directly as in the example above, but as part of a new environment
created by hand or using \refCom{newtcbtheorem}.

为了在文档内部具有常见的外观,关键字 |theorem| 不应直接使用,如上例所示,而应作为一个新环境的一部分手动创建或使用 \refCom{newtcbtheorem}。
\end{docTcbKey}

% \clearpage

\begin{docTcbKey}{highlight math}{}{style, no value}
A style which is used for \refCom{tcbhighmath} and which is
predefined as |notitle,nophantom,colframe=red,colback=yellow!25!white|.\par
It can be changed with the usual |pgf| techniques or
with \refKey{/tcb/highlight math style}.

这是用于 \refCom{tcbhighmath} 的样式,预定义为 |notitle,nophantom,colframe=red,colback=yellow!25!white|。\par 可以使用常规的 |pgf| 技术或 \refKey{/tcb/highlight math style} 来更改它。
\begin{dispExample}
\begin{align*}
  \tcbhighmath{1} + 1 &= 2,\\
  \tcbset{highlight math/.append style={left=0mm,right=0mm,top=0mm,bottom=0mm}}
  \tcbhighmath{1} + 1 &= 2.
\end{align*}
\end{dispExample}

\end{docTcbKey}


\begin{docTcbKey}{highlight math style}{=\meta{style definition}}{style, no default}
Changes the definition for \refKey{/tcb/highlight math} to
|notitle,nophantom| plus the given \meta{style definition}.
See \refCom{tcbhighmath} for another example.

将 \refKey{/tcb/highlight math} 的定义更改为 |notitle,nophantom| 加上给定的 \meta{样式定义}。另见 \refCom{tcbhighmath} 的另一个示例。
\begin{dispExample}
% \tcbuselibrary{skins}
\tcbset{highlight math style={enhanced,%<-- needed for the `remember' options
  colframe=red,colback=red!10!white,boxsep=0pt}}
\begin{align*}
\tcbhighmath[remember as=fx]{f(x)}
     &= \int\limits_{1}^{x} \frac{1}{t^2}~dt
      = \left[ -\frac{1}{t} \right]_{1}^{x}\\
     &= -\frac{1}{x} + \frac{1}{1}\\
     &=
\tcbhighmath[remember,overlay={%
    \draw[blue,very thick,->] (fx.south) to[bend right] ([yshift=2mm]frame.west);}]
  {1-\frac{1}{x}.}
\end{align*}
\end{dispExample}
\end{docTcbKey}

% \clearpage
\begin{docTcbKey}{math upper}{}{style, no value}
Sets the upper part to mathematical mode with font |\displaystyle|.

将上部设置为数学模式,字体为|\displaystyle|。
\end{docTcbKey}

\begin{docTcbKey}{math lower}{}{style, no value}
Sets the lower part to mathematical mode with font |\displaystyle|.

将下部设置为数学模式,字体为|\displaystyle|。
\end{docTcbKey}

\begin{docTcbKey}{math}{}{style, no value}
Sets the upper part \emph{and} lower part to mathematical mode with font |\displaystyle|.

将上部和下部同时设置为数学模式,字体为|\displaystyle|。
\begin{dispExample}
\begin{tcolorbox}[math,colback=yellow!10!white,colframe=red!50!black]
  \sum\limits_{n=1}^{\infty} \frac{1}{n} = \infty.
\end{tcolorbox}
\end{dispExample}
\end{docTcbKey}


\begin{marker}
The following styles are only tested to work with the original |amsmath| environments.
If e.g. the |equation| environment is redefined as |gather|, then
\refKey{/tcb/ams equation} should / could not be used. Obviously, you are encouraged
to use \refKey{/tcb/ams gather} in this case.

以下样式仅经过测试能够与原始的 |amsmath| 环境一起使用。如果例如 |equation| 环境被重新定义为 |gather|,那么 \refKey{/tcb/ams equation} 就不能被使用,显然,在这种情况下,鼓励使用 \refKey{/tcb/ams gather}。
\end{marker}

\begin{docTcbKey}[][doc updated=2014-10-30]{ams equation upper}{}{style, no value}
Adds an |amsmath| |equation| environment to the start and end
of the upper part.

在上半部分的开头和结尾添加一个 |amsmath| |equation| 环境。
\end{docTcbKey}

\begin{docTcbKey}[][doc updated=2014-10-30]{ams equation lower}{}{style, no value}
Adds an |amsmath| |equation| environment to the start and end
of the lower part.

在下半部分的开头和结尾添加一个 |amsmath| |equation| 环境。
\end{docTcbKey}

\begin{docTcbKey}[][doc updated=2014-10-30]{ams equation}{}{style, no value}
Adds an |amsmath| |equation| environment to the start and end
of the upper \emph{and} lower part.

在上半部分和下半部分的开头和结尾添加一个 |amsmath| |equation| 环境。
\begin{dispExample}
\begin{tcolorbox}[ams equation,colback=yellow!10!white,colframe=red!50!black]
  \sum\limits_{n=1}^{\infty} \frac{1}{n} = \infty.
\end{tcolorbox}
\end{dispExample}
\end{docTcbKey}

\begin{docTcbKey}[][doc updated=2014-10-30]{ams equation* upper}{}{style, no value}
Adds an |amsmath| |equation*| environment to the start and end
of the upper part.

在上半部分的开头和结尾添加一个 |amsmath| |equation*| 环境。
\end{docTcbKey}

\begin{docTcbKey}[][doc updated=2014-10-30]{ams equation* lower}{}{style, no value}
Adds an |amsmath| |equation*| environment to the start and end
of the lower part.

在下半部分的开头和结尾添加一个 |amsmath| |equation*| 环境。
\end{docTcbKey}

\enlargethispage*{2cm}
\begin{docTcbKey}[][doc updated=2014-10-30]{ams equation*}{}{style, no value}
Adds an |amsmath| |equation*| environment to the start and end
of the upper \emph{and} lower part.

在上半部分和下半部分的开头和结尾添加一个 |amsmath| |equation*| 环境。
\begin{dispExample}
\begin{tcolorbox}[ams equation*,colback=yellow!10!white,colframe=red!50!black]
  \sum\limits_{n=1}^{\infty} \frac{1}{n} = \infty.
\end{tcolorbox}
\end{dispExample}
\end{docTcbKey}

% \clearpage
\begin{docTcbKey}{ams align upper}{}{style, no value}
Adds an |amsmath| |align| environment to the start and end
of the upper part.

在上半部分的开头和结尾添加一个 |amsmath| |align| 环境。
\end{docTcbKey}

\begin{docTcbKey}{ams align lower}{}{style, no value}
Adds an |amsmath| |align| environment to the start and end
of the lower part.

在下部分的开头和结尾添加一个|amsmath| |align|环境。
\end{docTcbKey}

\begin{docTcbKey}{ams align}{}{style, no value}
Adds an |amsmath| |align| environment to the start and end
of the upper \emph{and} lower part.

在上部分和下部分的开头和结尾添加一个|amsmath| |align|环境。
\begin{dispExample}
\begin{tcolorbox}[ams align,colback=yellow!10!white,colframe=red!50!black]
  \sum\limits_{n=1}^{\infty} \frac{1}{n} &= \infty.\\
  \int x^2 ~\text{d}x                     &= \frac13 x^3 + c.
\end{tcolorbox}
\end{dispExample}
\end{docTcbKey}

\begin{docTcbKey}{ams align* upper}{}{style, no value}
Adds an |amsmath| |align*| environment to the start and end
of the upper part.

在上部分的开头和结尾添加一个|amsmath| |align*|环境。
\end{docTcbKey}

\begin{docTcbKey}{ams align* lower}{}{style, no value}
Adds an |amsmath| |align*| environment to the start and end
of the lower part.

在下半部分的开头和结尾添加一个 |amsmath| |align*| 环境。
\end{docTcbKey}

\begin{docTcbKey}{ams align*}{}{style, no value}
Adds an |amsmath| |align*| environment to the start and end
of the upper \emph{and} lower part.

在上半部分和下半部分的开头和结尾添加一个 |amsmath| |align*| 环境。
\begin{dispExample}
\begin{tcolorbox}[ams align*,colback=yellow!10!white,colframe=red!50!black]
  \sum\limits_{n=1}^{\infty} \frac{1}{n} &= \infty.\\
  \int x^2 ~\text{d}x                     &= \frac13 x^3 + c.
\end{tcolorbox}
\end{dispExample}
\end{docTcbKey}

% \clearpage
\begin{docTcbKey}{ams gather upper}{}{style, no value}
Adds an |amsmath| |gather| environment to the start and end
of the upper part.

在上半部分的开头和结尾添加一个 |amsmath| |gather| 环境。
\end{docTcbKey}

\begin{docTcbKey}{ams gather lower}{}{style, no value}
Adds an |amsmath| |gather| environment to the start and end
of the lower part.

在下部分的开头和结尾添加了一个|amsmath| |gather|环境。
\end{docTcbKey}

\begin{docTcbKey}{ams gather}{}{style, no value}
Adds an |amsmath| |gather| environment to the start and end
of the upper \emph{and} lower part.

在上部分和下部分的开头和结尾都添加了一个|amsmath| |gather|环境。
\begin{dispExample}
\begin{tcolorbox}[ams gather,colback=yellow!10!white,colframe=red!50!black]
  \sum\limits_{n=1}^{\infty} \frac{1}{n} = \infty.\\
  \int x^2 ~\text{d}x = \frac13 x^3 + c.
\end{tcolorbox}
\end{dispExample}
\end{docTcbKey}

\begin{docTcbKey}{ams gather* upper}{}{style, no value}
Adds an |amsmath| |gather*| environment to the start and end
of the upper part.

在上部分的开头和结尾添加了一个|amsmath| |gather*|环境。
\end{docTcbKey}

\begin{docTcbKey}{ams gather* lower}{}{style, no value}
Adds an |amsmath| |gather*| environment to the start and end
of the lower part.

在底部的开头和结尾添加一个 |amsmath| |gather*| 环境。
\end{docTcbKey}

\begin{docTcbKey}{ams gather*}{}{style, no value}
Adds an |amsmath| |gather*| environment to the start and end
of the upper \emph{and} lower part.

在上部和底部的开头和结尾都添加一个 |amsmath| |gather*| 环境。
\begin{dispExample}
\begin{tcolorbox}[ams gather*,colback=yellow!10!white,colframe=red!50!black]
  \sum\limits_{n=1}^{\infty} \frac{1}{n} = \infty.\\
  \int x^2 ~\text{d}x = \frac13 x^3 + c.
\end{tcolorbox}
\end{dispExample}
\end{docTcbKey}


% \clearpage
\begin{docTcbKey}{ams nodisplayskip upper}{}{style, no value}
Neutralizes the |\abovedisplayskip| of a following |align| or |gather|
environment for the upper part. Note that the text content has to
start with such a formula.

中和后续 |align| 或 |gather| 环境上部的 |\abovedisplayskip|。注意,文本内容必须以这样的公式开头。
\end{docTcbKey}


\begin{docTcbKey}{ams nodisplayskip lower}{}{style, no value}
Neutralizes the |\abovedisplayskip| of a following |align| or |gather|
environment for the lower part. Note that the text content has to
start with such a formula.

对于下部分的 |align| 或 |gather| 环境中的 |\abovedisplayskip| 进行中和。请注意,文本内容必须以这样的公式开头。
\end{docTcbKey}


\begin{docTcbKey}{ams nodisplayskip}{}{style, no value}
Neutralizes the |\abovedisplayskip| of a following |align| or |gather|
environment for the upper part \emph{and} lower part.
Note that the text content has to start with such a formula.

中和紧跟其后的 |align| 或 |gather| 环境的 |\abovedisplayskip|,包括上部和下部。请注意,文本内容必须以这样的公式开头。
\begin{dispExample}
\begin{tcolorbox}[ams nodisplayskip,colback=yellow!10!white,colframe=red!50!black]
  \begin{gather}
  \sum\limits_{n=1}^{\infty} \frac{1}{n} = \infty.\\
  \int x^2 ~\text{d}x = \frac13 x^3 + c.
  \end{gather}
And now for something completely different.

现在是完全不同的事情。
\end{tcolorbox}
\end{dispExample}
\end{docTcbKey}

\bigskip
New colored mathematical environments are easily created using
\refCom{newtcolorbox}:

可以使用\refCom{newtcolorbox}轻松创建新的彩色数学环境:

\begin{dispExample}
\newtcolorbox{mymath}{ams gather*,colback=yellow!10!white,colframe=red!50!black}

\begin{mymath}
  \sum\limits_{n=1}^{\infty} \frac{1}{n} = \infty.\\
  \int x^2 ~\text{d}x = \frac13 x^3 + c.
\end{mymath}
\end{dispExample}

\bigskip
\begin{marker}
All described options like \refKey{/tcb/ams gather upper}, \refKey{/tcb/ams gather lower},
\refKey{/tcb/ams gather} are (partially) setting (overwritting) the
keys \refKey{/tcb/before upper}, \refKey{/tcb/after upper},
\refKey{/tcb/before lower}, \refKey{/tcb/after lower}.\par
Therefore, e.\,g.\ |\tcbset{ams gather,before upper={\text{Pythagoras:}}}|
produces an invalid result. For this case, you are invited to use\\
|\tcbset{ams gather,before upper app={\text{Pythagoras:}}}|,\\
see \refKey{/tcb/before upper app}.

所有描述的选项,如 \refKey{/tcb/ams gather upper}、\refKey{/tcb/ams gather lower}、\refKey{/tcb/ams gather},都(部分地)设置(覆盖)了键 \refKey{/tcb/before upper}、\refKey{/tcb/after upper}、\refKey{/tcb/before lower}、\refKey{/tcb/after lower}。\par 因此,例如 |\tcbset{ams gather,before upper={\text{Pythagoras:}}}| 会产生无效的结果。对于这种情况,建议使用\ |\tcbset{ams gather,before upper app={\text{Pythagoras:}}}|,参见 \refKey{/tcb/before upper app}。
\end{marker}


% \clearpage
\begin{docTcbKey}{theorem style}{=\meta{name}}{no default, initially |standard|}
Applies a predefined style \meta{name} to the theorem environment. Some of
the feasible \meta{name} values resemble style names from the packages |theorem|
and |ntheorem| to give convenient access to known patterns.

将预定义的样式\meta{name}应用于定理环境。一些可行的\meta{name}值类似于|theorem|和|ntheorem|包中的样式名称,以便方便地访问已知的模式。
\begin{marker}
The styles alter \refKey{/tcb/separator sign}, \refKey{/tcb/description delimiters},
\refKey{/tcb/terminator sign}, and more. Therefore, one should apply such
keys \emph{after} a theorem style.

样式会改变 \refKey{/tcb/separator sign}、\refKey{/tcb/description delimiters}、\refKey{/tcb/terminator sign} 等等。因此,在定理样式之后应该应用这些关键字。
\end{marker}

For the following examples, we use:

对于以下示例,我们使用:

\inputpreamblelisting{J}


The predefined styles are:

预定义的样式有:
\begin{itemize}
%
\item\docValue{standard}: This is the initial value.
\begin{dispExample}
\begin{theorem}[theorem style=standard]{standard}{}
This is my theorem. \begin{equation*} a^2 + b^2 = c^2. \end{equation*}
\end{theorem}
\end{dispExample}
%
\item\docValue{change standard}
\begin{dispExample}
\begin{theorem}[theorem style=change standard]{change standard}{}
This is my theorem. \begin{equation*} a^2 + b^2 = c^2. \end{equation*}
\end{theorem}
\end{dispExample}
%
\item\docValue{plain}
\begin{dispExample}
\begin{theorem}[theorem style=plain]{plain}{}
This is my theorem. \begin{equation*} a^2 + b^2 = c^2. \end{equation*}
\end{theorem}
\end{dispExample}
%
% \clearpage
\item\docValue{break}
\begin{dispExample}
\begin{theorem}[theorem style=break]{break}{}
This is my theorem. \begin{equation*} a^2 + b^2 = c^2. \end{equation*}
\end{theorem}
\end{dispExample}
%
\item\docValue{plain apart}
\begin{dispExample}
\begin{theorem}[theorem style=plain apart]{plain apart}{}
This is my theorem. \begin{equation*} a^2 + b^2 = c^2. \end{equation*}
\end{theorem}
\end{dispExample}
%
\item\docValue{change}
\begin{dispExample}
\begin{theorem}[theorem style=change]{change}{}
This is my theorem. \begin{equation*} a^2 + b^2 = c^2. \end{equation*}
\end{theorem}
\end{dispExample}
%
\item\docValue{change break}
\begin{dispExample}
\begin{theorem}[theorem style=change break]{change break}{}
This is my theorem. \begin{equation*} a^2 + b^2 = c^2. \end{equation*}
\end{theorem}
\end{dispExample}
%
\item\docValue{change apart}
\begin{dispExample}
\begin{theorem}[theorem style=change apart]{change apart}{}
This is my theorem. \begin{equation*} a^2 + b^2 = c^2. \end{equation*}
\end{theorem}
\end{dispExample}
%
% \clearpage
\item\docValue{margin}
\begin{dispExample}
\begin{theorem}[theorem style=margin,left=10mm]{margin}{}
This is my theorem. \begin{equation*} a^2 + b^2 = c^2. \end{equation*}
\end{theorem}
\begin{theorem}[theorem style=margin,left=10mm,oversize]{margin}{}
This is my theorem. \begin{equation*} a^2 + b^2 = c^2. \end{equation*}
\end{theorem}
\end{dispExample}
%
\item\docValue{margin break}
\begin{dispExample}
\begin{theorem}[theorem style=margin break,left=10mm]{margin break}{}
This is my theorem. \begin{equation*} a^2 + b^2 = c^2. \end{equation*}
\end{theorem}
\begin{theorem}[theorem style=margin break,left=10mm,oversize]{margin break}{}
This is my theorem. \begin{equation*} a^2 + b^2 = c^2. \end{equation*}
\end{theorem}
\end{dispExample}
%
\item\docValue{margin apart}
\begin{dispExample}
\begin{theorem}[theorem style=margin apart,left=10mm]{margin apart}{}
This is my theorem. \begin{equation*} a^2 + b^2 = c^2. \end{equation*}
\end{theorem}
\begin{theorem}[theorem style=margin apart,left=10mm,oversize]{margin apart}{}
This is my theorem. \begin{equation*} a^2 + b^2 = c^2. \end{equation*}
\end{theorem}
\end{dispExample}
%
\end{itemize}
\end{docTcbKey}

