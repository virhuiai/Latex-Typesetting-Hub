
\subsection{Macros of the Library\\库中的宏}

\begin{docCommand}[doc updated=2016-06-22]{newtcbtheorem}{\oarg{init options}\marg{name}\marg{display name}\marg{options}\marg{prefix}}
Creates new environments \meta{name} and \meta{name}|*| based on |tcolorbox| to frame a
(mathematical) theorem. The \meta{display name} is used in the title line
with a number, e.\,g. \mbox{\flqq Theorem 5.1\frqq}.
The \meta{options} are given to the underlying |tcolorbox| to control
the appearance.
%The \meta{counter} is used for automatic numbering.
The \meta{init options} allow setting up automatic numbering,
see \Vref{sec:initkeys}.\\
创建基于|tcolorbox|的新环境\meta{name}和\meta{name}||来框定(数学)定理。在标题行中使用\meta{display name}和一个编号,例如\mbox{\flqq 定理5.1\frqq}。\meta{options}用于控制外观的底层|tcolorbox|。 \meta{counter}用于自动编号。\meta{init options}可用于设置自动编号,参见\Vref{sec:initkeys}。\\[0.5em]
The new environment \meta{name} takes one optional and two mandatory
parameters. The optional parameter supplements the options and should be
used only in rare cases.
The first mandatory parameter is the title text for the theorem and
is also set as \refKeyLe{/tcb/nameref} identifier.
The second mandatory parameter is a \meta{marker}. The theorem is
automatically labeled with \meta{prefix}\meta{separator}\meta{marker}
where \meta{separator} is predefined as \enquote{|:|}, see \refKeyLe{/tcb/label separator}.\\
新环境\meta{name}具有一个可选参数和两个强制参数。可选参数补充选项,仅在极少数情况下使用。第一个强制参数是定理的标题文本,也设置为\refKeyLe{/tcb/nameref}标识符。第二个强制参数是一个\meta{marker}。定理自动标记为\meta{prefix}\meta{separator}\meta{marker},其中\meta{separator}预定义为\enquote{|:|},请参见\refKeyLe{/tcb/label separator}。\\[0.5em]
The new environment \meta{name}|*| takes one optional and one mandatory
parameter and represents an unnumbered variant of the environment \meta{name}.
This variant is not labeled and not listed in lists of theorems.\\
新环境\meta{name}|*|具有一个可选参数和一个强制参数,表示环境\meta{name}的未编号变体。此变体未标记,并且不在定理列表中列出。
\enlargethispage*{20mm}
\inputpreamblelisting{F}
\begin{dispExample}
% usage of `\nameref' needs `nameref' or `hyperref' to be loaded
\begin{mytheo}{This is my title}{theoexample}
This is the text of the theorem. The counter is automatically assigned and,
in this example, prefixed with the section number. This theorem is numbered with
\ref{th:theoexample}, it is given on page~\pageref{th:theoexample},
and it is titled \flqq\nameref{th:theoexample}\frqq.

这是定理的文本。计数器会自动分配,并在本例中带有章节编号前缀。此定理编号为%
\ref{th:theoexample},在第\pageref{th:theoexample}页给出,并以%
\flqq\nameref{th:theoexample}\frqq 为标题。
\end{mytheo}
\end{dispExample}

\begin{dispExample}
\begin{mytheo}[label=myownlabel]{This is my title}{}
The label parameter can be left empty without \LaTeX\ error.
Or you may use an own label to reference Theorem \ref{myownlabel}.

标签参数可以为空,不会发生 \LaTeX\ 错误。 或者您可以使用自己的标签来引用定理 \ref{myownlabel}。
\end{mytheo}
\end{dispExample}

\begin{dispExample}
\begin{mytheo}{}{}
The title can also be left empty without problem. Note that the \enquote{:}
vanished magically.

标题也可以为空,没有问题。注意 \enquote{:} 神奇地消失了。
\end{mytheo}
\end{dispExample}

\begin{dispExample}
\begin{mytheo*}{Unnumbered Theorem}
This theorem is not numbered.

这个定理没有编号。
\end{mytheo*}
\end{dispExample}

\begin{dispExample}
\begin{mytheo*}{}
This theorem has no number and no title.

这个定理既没有编号也没有标题。
\end{mytheo*}
\end{dispExample}

\begin{marker}
To switch off the \texttt{nameref} feature permanently, add
\mbox{\texttt{nameref/.style=\{\}}} inside the \meta{options} list.

要永久关闭\texttt{nameref}功能,请在\meta{options}列表中添加\mbox{\texttt{nameref/.style={}}}。
\end{marker}

\end{docCommand}


\begin{docCommand}{renewtcbtheorem}{\oarg{init options}\marg{name}\marg{display name}\marg{options}\marg{prefix}}
Operates like \refComLe{newtcbtheorem}, but based on |\renewenvironment| instead of |\newenvironment|.
An existing environment is redefined.

与 \refComLe{newtcbtheorem} 类似,但基于 |\renewenvironment| 而不是 |\newenvironment| 进行操作。重新定义现有的环境。
\end{docCommand}


% \clearpage
\begin{docCommand}{tcboxmath}{\oarg{options}\marg{mathematical box content}}
Creates a \refEnvLe{tcolorbox} which is fitted to the width of the given
\meta{mathematical box content}. This box is intended to be applied as
part of a larger formula and may be used as replacement for the |\boxed|
macro of |amsmath|.

创建一个\refEnvLe{tcolorbox},该盒子适合给定的\meta{mathematical box content}的宽度。此框旨在作为较大公式的一部分应用,并可用作|amsmath|的|\boxed|宏的替换。
\begin{dispExample}
\begin{equation}
\tcbset{fonttitle=\scriptsize}
\tcboxmath[colback=LightBlue!25!white,colframe=blue]{ a^2 = 16 }
\quad \Rightarrow \quad
\tcboxmath[colback=Salmon!25!white,colframe=red,title=Implication]%
  { a = 4 ~\vee~ a=-4. }
\end{equation}
\end{dispExample}
\end{docCommand}

\begin{docCommand}{tcbhighmath}{\oarg{options}\marg{mathematical box content}}
This is a special case of the \refComLe{tcboxmath} macro which uses
the style \refKeyLe{/tcb/highlight math}.
It is intended to provide context sensitive highlighting of formula parts.
The color settings via \refKeyLe{/tcb/highlight math style} may be different
inside theorems or other colored areas and outside.

这是\refComLe{tcboxmath}宏的一个特殊情况,它使用了样式\refKeyLe{/tcb/highlight math}。它旨在提供上下文敏感的公式部分高亮。通过\refKeyLe{/tcb/highlight math style}的颜色设置可以在定理或其他彩色区域内外不同。
\begin{dispExample}
\tcbset{myformula/.style={colback=yellow!10!white,colframe=red!50!black,
  every box/.style={highlight math style={colback=LightBlue!50!white,colframe=Navy}}
  }}

\begin{align}
  \tcbhighmath{\sum\limits_{n=1}^{\infty} \frac{1}{n}} &= \infty.\\
  \int x^2 ~\text{d}x                     &= \frac13 x^3 + c.
\end{align}

\begin{tcolorbox}[ams align,myformula]
  \tcbhighmath{\sum\limits_{n=1}^{\infty} \frac{1}{n}} &= \infty.\\
  \int x^2 ~\text{d}x                     &= \frac13 x^3 + c.
\end{tcolorbox}
\end{dispExample}

% \clearpage
\refComLe{tcbhighmath} can be used in symbiosis with the |empheq| package
which allows to specify own boxing commands to mark multiline formulas.

\refComLe{tcbhighmath} 可以与 |empheq| 宏包共同使用,该宏包允许指定自己的盒子命令来标记多行公式。
\begin{dispExample}
% \usepackage{empheq}
\begin{empheq}[box=\tcbhighmath]{align}
a&=\sin(z)\\
E&=mc^2 + \int_a^b x\, dx
\end{empheq}

\tcbset{highlight math style={enhanced,
  colframe=red!60!black,colback=yellow!50!white,arc=4pt,boxrule=1pt,
  drop fuzzy shadow}}

\begin{empheq}[box=\tcbhighmath]{align}
a&=\sin(z)\\
E&=mc^2 + \int_a^b x\, dx
\end{empheq}
\end{dispExample}

Besides \refComLe{tcbhighmath}, one can easily define an independent new box
based on \refComLe{tcbox} which acts like \refComLe{tcbhighmath}:

除了\refComLe{tcbhighmath},我们可以轻松地基于\refComLe{tcbox}定义一个独立的新盒子,它的行为类似于\refComLe{tcbhighmath}:
\begin{dispExample}
% \usepackage{empheq}
\newtcbox{\otherbox}[1][]{nobeforeafter,math upper,tcbox raise base,
  enhanced,frame hidden,boxrule=0pt,interior style={top color=green!10!white,
  bottom color=green!10!white,middle color=green!50!yellow},
  fuzzy halo=1pt with green,#1}

\begin{empheq}[box=\otherbox]{align}
a&=\sin(z)\\
E&=mc^2 + \int_a^b x\, dx
\end{empheq}

\begin{equation}
\tcbhighmath{E} = \otherbox{mc^2}
\end{equation}
\end{dispExample}
\end{docCommand}

