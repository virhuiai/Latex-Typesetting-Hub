
\subsection{Examples for Definitions and Theorems\\定义和定理的示例}
In the following, the application of \refComLe{newtcbtheorem}
to highlight mathematical definitions, theorems, or the like is demonstrated.

以下展示了使用\refComLe{newtcbtheorem}突出数学定义、定理或类似内容的应用。

At first, additional |tcb| keys are created for the appearance of
the colored boxes. It is assumed that theorems and corollaries should be
identically colored.

首先,为了出现有色框,需要创建额外的|tcb|键。假设定理和推论应该是同样的颜色。

All following environments are numbered with a common counter, but this
can be changed easily. Here, the counter output is supplemented by
the subsection number.
Further, the |cleveref| package  is used for clever
references.

所有下列环境都使用一个公共计数器进行编号,但这可以很容易地更改。在这里,计数器输出还会加上子节编号。另外,使用 |cleveref| 宏包  进行智能引用。


\inputpreamblelisting{G}

By \refComLe{newtcbtheorem}, commonly numbered theorem environments are
created now. |defstyle| and |theostyle| are used for the appearance.

通过 \refComLe{newtcbtheorem},现在可以创建常见编号的定理环境。|defstyle| 和 |theostyle| 用于外观样式。

Now, everything is prepared for the following examples.

现在,一切都准备就绪,可以进行以下示例。

\begin{dispExample}
The following theorem is numbered as \Cref{theo:diffbarstetig} and
referenced with the marker \texttt{theo:diffbarstetig}.

以下定理标号为\Cref{theo:diffbarstetig},使用标记\texttt{theo:diffbarstetig}进行引用。
\begin{Theorem}{Differenzierbarkeit bedingt Stetigkeit, wobei diese Benennung
  zu Testzwecken ungew\"{o}hnlich lang ist}{diffbarstetig}%
  Eine Funktion $f:I\to\mathbb{R}$ ist in $x_0\in I$ stetig, wenn $f$ in
  $x_0$ differenzierbar ist.
\end{Theorem}
\end{dispExample}


\begin{dispExample}
The following definition is numbered as \Cref{def:diffbarkeit} and
referenced with the marker \texttt{def:diffbarkeit}.\bigskip

下面的定义编号为\Cref{def:diffbarkeit},并使用标记\texttt{def:diffbarkeit}引用。
\begin{Definition}{Differenzierbarkeit}{diffbarkeit}
  Eine Funktion $f:~I\to\mathbb{R}$ auf einem Intervall $I$ hei\ss{}t in
  $x_0\in I$ differenzierbar oder linear approximierbar,
  wenn der Grenzwert
  \begin{equation*}
  \lim\limits_{x\to x_0}\frac{f(x)-f(x_0)}{x-x_0}=
  \lim\limits_{h\to 0}\frac{f(x_0+h)-f(x_0)}{h}
  \end{equation*}
  existiert. Bei Existenz hei\ss{}t dieser Grenzwert Ableitung
  oder Differentialquotient von $f$ in $x_0$ und man
  schreibt f\"{u}r ihn
  \begin{equation*}
  f'(x_0)\quad\text{oder}\quad\frac{df}{dx}(x_0).
  \end{equation*}
\end{Definition}
\end{dispExample}


\begin{dispExample}
The following corollary is numbered as \Cref{cor:nullstellen} and
referenced with the marker \texttt{cor:nullstellen}.

以下推论标号为\Cref{cor:nullstellen},并且使用标记\texttt{cor:nullstellen}进行引用。\bigskip
\begin{Corollary}{Nullstellenexistenz}{nullstellen}
  Ist $f:[a,b]\to\mathbb{R}$ stetig und haben $f(a)$ und $f(b)$ entgegengesetzte
  Vorzeichen, also $f(a)f(b)<0$, so besitzt $f$ eine Nullstelle $x_0\in]a,b[$,
  also $f(x_0)=0$.
\end{Corollary}
\end{dispExample}


\begin{dispExample}
\begin{Theorem}[boxrule=2mm,toptitle=-1.5mm,bottomtitle=-1.5mm]{%
    Hinreichende Bedingung f\"{u}r Wendepunkte}{wendehinreichend}%
  $f$ sei eine auf einem Intervall $]a,b[$ dreimal stetig differenzierbare Funktion.
  Ist $f''(x_0)=0$ in $x_0\in]a,b[$ und $f'''(x_0)\ne 0$, so ist
  $(x_0,f(x_0))$ ein Wendepunkt von $f$.
\end{Theorem}
\end{dispExample}

\begin{dispExample}
% \usepackage{varioref}
% \usepackage{cleveref}
% \tcbuselibrary{skins}
\newtcbtheorem[use counter from=Definition]{YetAnotherTheorem}{Theorem}%
  {theorem style=plain apart,label type=theorem,enhanced,frame hidden,
  boxrule=2mm,titlerule=0mm,toptitle=1mm,bottomtitle=1mm,
  fonttitle=\bfseries\large,fontupper=\normalsize,
  coltitle=green!35!black,colbacktitle=green!15!white,
  colback=green!50!yellow!15!white,borderline={1pt}{0pt}{green!25!blue},
  }{theo}

\begin{YetAnotherTheorem}{Mittelwertsatz f\"{u}r $n$ Variable}{meanvaluetheorem}%
  Es sei $n\in\mathbb{N}$, $D\subseteq\mathbb{R}^n$ eine offene Menge und
  $f\in C^{1}(D,\mathbb{R})$. Dann gibt es auf jeder Strecke
  $[x_0,x]\subset D$ einen Punkt $\xi\in[x_0,x]$, so dass gilt
  \begin{equation*}
  f(x)-f(x_0) = \operatorname{grad} f(\xi)^{\top}(x-x_0)
  \end{equation*}
\end{YetAnotherTheorem}

\medskip
Here, |cleveref| support is used to reference \Cref{theo:meanvaluetheorem}
on \Cpageref{theo:meanvaluetheorem}. This \namecref{theo:meanvaluetheorem}
can also be referenced by |\Vref| resulting in \Vref{theo:meanvaluetheorem}.

这里使用了 |cleveref| 支持来引用\Cref{theo:meanvaluetheorem},它位于 \Cpageref{theo:meanvaluetheorem}。这个\namecref{theo:meanvaluetheorem}也可以通过 |\Vref| 引用,结果为 \Vref{theo:meanvaluetheorem}。
\end{dispExample}

\begin{marker}
Note that \refKeyLe{/tcb/label type} was used in the example above to feed
|cleveref|  with the needed name information.

请注意,上面的示例中使用了 \refKeyLe{/tcb/label type} 来提供所需的名称信息,以便将其提供给 |cleveref|。
\end{marker}

% \clearpage

\begin{dispExample}
Here, using |\Vref| resulting in \Vref{theo:meanvaluetheorem} is more interesting\ldots

在这里,使用 |\Vref| 得到的 \Vref{theo:meanvaluetheorem} 更有趣...
\end{dispExample}


\begin{dispExample}
% \tcbuselibrary{skins}
\newtcbtheorem[use counter from=Definition]{YetAnotherTheorem}{Theorem}%
  {theorem style=change apart,enhanced,arc=0mm,outer arc=0mm,
  boxrule=0mm,toprule=1mm,bottomrule=1mm,left=1mm,right=1mm,
  titlerule=0mm,toptitle=0mm,bottomtitle=1mm,top=0mm,
  colframe=red!50!black,colback=red!5!white,coltitle=red!50!black,
  title style={top color=yellow!50!white,bottom color=red!5!white,
    middle color=yellow!50!white},
  fonttitle=\bfseries\sffamily\normalsize,fontupper=\normalsize\itshape,
  }{theo}

\begin{YetAnotherTheorem}{Mittelwertsatz f\"{u}r $n$ Variable}{mittelwertsatz_n2}%
  Es sei $n\in\mathbb{N}$, $D\subseteq\mathbb{R}^n$ eine offene Menge und
  $f\in C^{1}(D,\mathbb{R})$. Dann gibt es auf jeder Strecke
  $[x_0,x]\subset D$ einen Punkt $\xi\in[x_0,x]$, so dass gilt
  \begin{equation*}
  f(x)-f(x_0) = \operatorname{grad} f(\xi)^{\top}(x-x_0)
  \end{equation*}
\end{YetAnotherTheorem}
\end{dispExample}


\begin{dispExample}
% \usepackage{varwidth}   \tcbuselibrary{skins}
\newtcbtheorem[use counter from=Definition]{YetAnotherTheorem}{Theorem}%
{enhanced,frame empty,interior empty,colframe=ForestGreen!50!white,
  coltitle=ForestGreen!50!black,fonttitle=\bfseries,colbacktitle=ForestGreen!15!white,
  borderline={0.5mm}{0mm}{ForestGreen!15!white},
  borderline={0.5mm}{0mm}{ForestGreen!50!white,dashed},
  attach boxed title to top center={yshift=-2mm},
  boxed title style={boxrule=0.4pt},varwidth boxed title}{theo}

\begin{YetAnotherTheorem}{Mittelwertsatz f\"{u}r $n$ Variable}{mittelwertsatz_n3}%
  Es sei $n\in\mathbb{N}$, $D\subseteq\mathbb{R}^n$ eine offene Menge und
  $f\in C^{1}(D,\mathbb{R})$. Dann gibt es auf jeder Strecke
  $[x_0,x]\subset D$ einen Punkt $\xi\in[x_0,x]$, so dass gilt
  \begin{equation*}
  f(x)-f(x_0) = \operatorname{grad} f(\xi)^{\top}(x-x_0)
  \end{equation*}
\end{YetAnotherTheorem}
\end{dispExample}


% \clearpage
You need more attention for your theorems? Here, you are \ldots

你需要更多关注你的定理吗?在这里,就是这里。...

\begin{dispExample}
% tcbuselibrary{skins}  % preamble
\begin{Theorem}[enhanced,
    fuzzy halo=3mm with yellow,
    fuzzy halo=2mm with red,
    fuzzy halo=1mm with yellow,
    watermark color=red!35!white,
    watermark text={Overacting\\Fundamental Theorem}]%
  {Fundamental Theorem of Theorems}{fundamental}%
  \lipsum[1-2]
\end{Theorem}
\end{dispExample}

Let's try a more conservative approach:

让我们尝试一种更为保守的方法:
\begin{dispExample}
% \tcbuselibrary{skins}
\newtcbtheorem[use counter from=Definition]{YetAnotherTheorem}{Theorem}%
  {theorem style=plain,enhanced,colframe=blue!50!black,colback=yellow!20!white,
   coltitle=red!50!black,fonttitle=\upshape\bfseries,fontupper=\itshape,
   drop fuzzy shadow=blue!50!black!50!white,boxrule=0.4pt}{theo}

\begin{YetAnotherTheorem}{Mittelwertsatz f\"{u}r $n$ Variable}{mittelwertsatz_n4}%
  Es sei $n\in\mathbb{N}$, $D\subseteq\mathbb{R}^n$ eine offene Menge und
  $f\in C^{1}(D,\mathbb{R})$. Dann gibt es auf jeder Strecke
  $[x_0,x]\subset D$ einen Punkt $\xi\in[x_0,x]$, so dass gilt
  \begin{equation*}
  f(x)-f(x_0) = \operatorname{grad} f(\xi)^{\top}(x-x_0)
  \end{equation*}
\end{YetAnotherTheorem}
\end{dispExample}


