
\subsection{Concept of Hooks\\钩子的概念}
A hook is a placeholder in some \LaTeX\ code where additional code
can be added. For example, the \LaTeX\ macro |\AtBeginDocument| adds code to a
hook which is placed at the beginning of every document.

钩子是一些 \LaTeX\ 代码中的占位符,可以在其中添加额外的代码。例如,\LaTeX\ 宏 |\AtBeginDocument| 向钩子添加代码,该钩子位于每个文档的开头。

Several option keys of |tcolorbox| allow providing some code which is
added to specific places of a colored box. For example, \refKeyLe{/tcb/before upper}
places code before the content of the upper part. A following usage of this
key overwrites any prior settings.

|tcolorbox| 的几个选项键允许提供一些代码,该代码添加到有色框的特定位置。例如,\refKeyLe{/tcb/before upper} 在上半部分的内容之前放置代码。对此键的后续使用将覆盖任何先前的设置。

The library \mylib{hooks} extends \refKeyLe{/tcb/before upper} and several more
existing keys to \enquote{hookable} versions, e.\,g.\ 
\refKeyLe{/tcb/before upper app} and \refKeyLe{/tcb/before upper pre}.
The \enquote{hookable} keys don't overwrite prior settings but either \emph{app}end
or \emph{pre}pend the newly given code to the existing code.

库 \mylib{hooks} 扩展了 \refKeyLe{/tcb/before upper} 和其他几个现有键,将其扩展为可“挂钩”的版本,例如 \refKeyLe{/tcb/before upper app} 和 \refKeyLe{/tcb/before upper pre}。这些“挂钩”键不会覆盖以前的设置,而是将新给定的代码附加或预置到现有代码中。

The general naming convention (with some small exceptions) is:

一般的命名约定(有一些小的例外)是:
\begin{itemize}
\item \meta{option key} |app|: works like \meta{option key} but
  \emph{app}ends its code to the existing code.

  与 \meta{option key} 相似,但将其代码附加到现有代码中。
\item \meta{option key} |pre|: works like \meta{option key} but
  \emph{pre}pends its code to the existing code.

  与 \meta{option key} 相似,但将其代码预置到现有代码中。
\end{itemize}
If the original \meta{option key} is used (again), all code will be overwritten.
Therefore, the order of the option key usage is crucial.

如果再次使用原始的 \meta{option key},则所有代码将被覆盖。因此,选项键使用的顺序至关重要。

\begin{dispExample}
% \usepackage{array,tabularx}
\newcolumntype{Y}{>{\raggedleft\arraybackslash}X}% see tabularx
\tcbset{enhanced,fonttitle=\bfseries\large,fontupper=\normalsize\sffamily,
  colback=yellow!10!white,colframe=red!50!black,colbacktitle=Salmon!30!white,
  coltitle=black,center title,
  tabularx={X||Y|Y|Y|Y||Y},% this sets `before upper' and `after upper'
  before upper app={Group & One & Two & Three & Four & Sum\\\hline\hline}  }

\begin{tcolorbox}[title=My table]
Red   & 1000.00 & 2000.00 &  3000.00 &  4000.00 & 10000.00\\\hline
Green & 2000.00 & 3000.00 &  4000.00 &  5000.00 & 14000.00\\\hline
Blue  & 3000.00 & 4000.00 &  5000.00 &  6000.00 & 18000.00\\\hline\hline
Sum   & 6000.00 & 9000.00 & 12000.00 & 15000.00 & 42000.00
\end{tcolorbox}
\end{dispExample}