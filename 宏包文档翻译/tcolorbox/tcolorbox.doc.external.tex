% !TeX root = tcolorbox.tex
% include file of tcolorbox.tex (manual of the LaTeX package tcolorbox)

% \include*{external/external}
% \include*{external/准备文档以进行外部化}
%%\include*{external/标记外部化代码片段}%有问题
% \include*{external/自定义}

\end{document}




% \clearpage
\subsection{Troubleshooting and FAQ}\label{subsec:external_troubleshooting}

\begin{itemize}

\item\textbf{I use the default settings, but the |external| subdirectory is
  not created.}\\
  Depending on operating system and compiler, an |external| subdirectory is
  automatically created or not. If not, create such a directory manually
  or add the following to your document\footnote{The |shellesc| package
  is loaded automatically by the library.}:
\begin{dispListing}
\ShellEscape{mkdir external}
\end{dispListing}
or
\begin{dispListing}
\ShellEscape{mkdir -p external}
\end{dispListing}
  If the combination of \refKeyLe{/tcb/external/prefix} and chosen snippet
  name points to another subdirectory than |external|, this has to be
  adapted.

\item\textbf{I use the |minted| package and I get a cache directory for
  every externalized snippet.}\\
  To avoid this problem, there are several ways.
  \begin{itemize}
  \item If you do not need |minted| inside the snippet code, you may use
    |\usepackage{minted}| \emph{after} \refComLe{tcbEXTERNALIZE}
    or use \refComLe{tcbifexternal} to switch |minted| off for the external code.
    If |minted| is already included by another package, add the following to
    your preamble:
\begin{dispListing}
\tcbset{external/PassOptionsToPackage={draft}{minted}}
\end{dispListing}
  \item If |minted| is needed for the snippet code, caching can be switched
    off by adding the following to your preamble:
\begin{dispListing}
\tcbset{external/PassOptionsToPackage={cache=false}{minted}}
\end{dispListing}
  Alternatively, the |cachedir| option of |minted| may be used to redirect
  the cache.
  \end{itemize}


\end{itemize}

