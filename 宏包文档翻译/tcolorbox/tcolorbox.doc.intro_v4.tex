% !TeX root = tcolorbox.tex

\section{Introduction\\介绍}%
%TODO external/prefix 是啥意思:
\tcbset{external/prefix=external/intro_}%
% external/prefix 是 tikz 宏包的 external 库中的一个键。这个键用于设置 external 库生成的外部图形文件的前缀。
% external 库用于将 tikzpicture 环境中的内容编译为独立的 PDF 文件,这样在重新编译整个文档时,如果这些环境的内容没有改变,LaTeX 可以直接插入已经编译好的 PDF,从而大大加快编译速度。

% \tcbset{external/prefix=external/intro_} 这个命令就是设置了 external 库生成的 PDF 文件的前缀为 external/intro_。这意味着,如果你有一个 tikzpicture 环境,它的标签(label)是 fig1,那么 tikz 将会生成一个名为 external/intro_fig1.pdf 的文件。

% 这个功能在处理包含大量 tikz 图形的大文档时非常有用,可以显著提高编译速度。

% \begin{stripedbox}
The package originates from %
the first edition of my book \flqq{\LaTeX -- Einführung in das Textsatzsystem}%
% {\citetitle{sturm:latex}
\frqq~%\cite{sturm:latex}
in about 2006.%
For the \LaTeX\ examples and tutorials given there, %
I wanted to have accentuated and colored boxes to display source code and
compiled text in combination.%
Since, in my opinion, %
this type of boxes is also quite useful to highlight definitions and theorems,% 
I applied them for my lecture notes in mathematics %\cite{sturm:mathe1,sturm:mathe2,sturm:mathe3}
as well.%
With this package, you are invited to apply these boxes for similar projects.
% 通过这个包,
% \tcblower
% 这个包起源于我2006年出版的\flqq{\LaTeX -- Einführung in das Textsatzsystem}%
% % 这个软件包来源于我在2006年出版的第一版书《 LATEX-Einfühung in das Textsatzsystem 》。
% % {\citetitle{sturm:latex}
% \frqq%~\cite{sturm:latex}
% 一书的第一版。%
% 对于书中给出的 \LaTeX\ 例子和教程,我想用突出的彩色方框来显示源代码和编译后的文本。%
% 因为,在我看来,这种类型的盒子,很适合突出定义和定理, 所以我也把它们应用到我的数学讲义 %\cite{sturm:mathe1,sturm:mathe2,sturm:mathe3} 
% 。%
% 您可以将这个宏包的这些盒子用于类似的项目中。

这个包源于我的书《\LaTeX -- Einführung in das Textsatzsystem》的第一版(约为2006年),其中提供了一些\LaTeX 的示例和教程。为了显示源代码和编译后的文本,我想要有突出和带颜色的框。由于在我看来,这种类型的框对于突出定义和定理也非常有用,因此我也将它们用于我的数学讲义中。通过这个包,您可以将这些框应用于类似的项目中。
% \end{stripedbox}

% \begin{stripedbox}
The breaking news for version 2.00 was the support for breakable boxes.%
This feature allows new applications of the package without affecting the core package too much if you do not need boxes to break automatically.%
With version 2.20, the often requested \enquote{side by side} mode for listings has been added.%
With version 3.00, boxed titles are introduced together with improved customization options for overlays, underlays, finishes, and own code extensions.%
% \tcblower

2.00版的爆炸性新闻是,支持换页盒子了。%
如果你不需要盒子自动换页,这个功能不太影响核心包。%不影响意思,去除
2.20版, 加上了经常被要求加上的,排版代码清单的 \enquote{side by side} 模式。%
在版本3.00中,盒标题会和改进了的,用于定制overlays(覆盖层)、underlays(衬垫)、finishes和自行扩展的选项一起介绍。
% \end{stripedbox}


% 笑脸 TODO 看下
\begin{tcolorbox}[enhanced,%允许我们使用高级特性,例如图形和图片。
boxrule=0mm,boxsep=0mm,%设置盒子边框的宽度和盒子与周围内容的间距为0,使得盒子不可见。
frame hidden,interior hidden,%隐藏盒子的边框和内部,这样盒子就完全透明了。
left=0mm,right=0mm,top=0mm,bottom=0mm,%设置盒子的左、右、上、下边距都为0。
watermark opacity=0.25,watermark zoom=1.2,%设置水印的透明度为25%,并放大水印的尺寸。
before=\par\smallskip,%在盒子之前插入一个新段落和一个小的垂直间距。
clip watermark=false,%不裁剪水印,使其可以超出盒子的边界。
watermark tikz={%
\path[fill=yellow,draw=yellow!75!red] (0,0) circle (1cm);%一个黄色的圆脸(circle (1cm))
\fill[red] (45:5mm) circle (1mm);\fill[red] (135:5mm) circle (1mm);%两个红色的眼睛(circle (1mm))
\draw[line width=1mm,red] (215:5mm) arc (215:325:5mm);}]%一个弧形的笑脸(arc (215:325:5mm))
% \begin{stripedbox}
Since the first public release in 2011, %
I received a lot of feedback from all over the world.%
I want to thank all who wrote me for supporting this package by sending bug reports and ideas for new or better features.
% \tcblower

自从2011年第一次公开发布以来,%
我收到了来自世界各地的大量反馈。%
我想感谢所有写信给我的人,感谢他们通过发送错误报告、和关于新的或更好的功能的想法来支持这个宏包。
% \end{stripedbox}
\end{tcolorbox}
    