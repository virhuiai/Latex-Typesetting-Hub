% !TeX root = tcolorbox.tex

\section{Introduction\\介绍}%
%TODO external/prefix 是啥意思:
\tcbset{external/prefix=external/intro_}%
% external/prefix 是 tikz 宏包的 external 库中的一个键。这个键用于设置 external 库生成的外部图形文件的前缀。
% external 库用于将 tikzpicture 环境中的内容编译为独立的 PDF 文件,这样在重新编译整个文档时,如果这些环境的内容没有改变,LaTeX 可以直接插入已经编译好的 PDF,从而大大加快编译速度。

% \tcbset{external/prefix=external/intro_} 这个命令就是设置了 external 库生成的 PDF 文件的前缀为 external/intro_。这意味着,如果你有一个 tikzpicture 环境,它的标签(label)是 fig1,那么 tikz 将会生成一个名为 external/intro_fig1.pdf 的文件。

% 这个功能在处理包含大量 tikz 图形的大文档时非常有用,可以显著提高编译速度。

% \begin{stripedbox}
The package originates from %
the first edition of my book \flqq{\LaTeX -- Einführung in das Textsatzsystem}%
% {\citetitle{sturm:latex}
\frqq~%\cite{sturm:latex}
in about 2006.%
For the \LaTeX\ examples and tutorials given there, %
I wanted to have accentuated and colored boxes to display source code and
compiled text in combination.%
Since, in my opinion, %
this type of boxes is also quite useful to highlight definitions and theorems,% 
I applied them for my lecture notes in mathematics %\cite{sturm:mathe1,sturm:mathe2,sturm:mathe3}
as well.%
With this package, you are invited to apply these boxes for similar projects.
% 通过这个包,
% \tcblower
% 这个包起源于我2006年出版的\flqq{\LaTeX -- Einführung in das Textsatzsystem}%
% % 这个软件包来源于我在2006年出版的第一版书《 LATEX-Einfühung in das Textsatzsystem 》。
% % {\citetitle{sturm:latex}
% \frqq%~\cite{sturm:latex}
% 一书的第一版。%
% 对于书中给出的 \LaTeX\ 例子和教程,我想用突出的彩色方框来显示源代码和编译后的文本。%
% 因为,在我看来,这种类型的盒子,很适合突出定义和定理, 所以我也把它们应用到我的数学讲义 %\cite{sturm:mathe1,sturm:mathe2,sturm:mathe3} 
% 。%
% 您可以将这个宏包的这些盒子用于类似的项目中。

这个包源于我的书《\LaTeX -- Einführung in das Textsatzsystem》的第一版(约为2006年),其中提供了一些\LaTeX 的示例和教程。为了显示源代码和编译后的文本,我想要有突出和带颜色的框。由于在我看来,这种类型的框对于突出定义和定理也非常有用,因此我也将它们用于我的数学讲义中。通过这个包,您可以将这些框应用于类似的项目中。
% \end{stripedbox}

% \begin{stripedbox}
The breaking news for version 2.00 was the support for breakable boxes.%
This feature allows new applications of the package without affecting the core package too much if you do not need boxes to break automatically.%
With version 2.20, the often requested \enquote{side by side} mode for listings has been added.%
With version 3.00, boxed titles are introduced together with improved customization options for overlays, underlays, finishes, and own code extensions.%
% \tcblower

2.00版的爆炸性新闻是,支持换页盒子了。%
如果你不需要盒子自动换页,这个功能不太影响核心包。%不影响意思,去除
2.20版, 加上了经常被要求加上的,排版代码清单的 \enquote{side by side} 模式。%
在版本3.00中,盒标题会和改进了的,用于定制overlays(覆盖层)、underlays(衬垫)、finishes和自行扩展的选项一起介绍。
% \end{stripedbox}


% 笑脸 TODO 看下
\begin{tcolorbox}[enhanced,%允许我们使用高级特性,例如图形和图片。
boxrule=0mm,boxsep=0mm,%设置盒子边框的宽度和盒子与周围内容的间距为0,使得盒子不可见。
frame hidden,interior hidden,%隐藏盒子的边框和内部,这样盒子就完全透明了。
left=0mm,right=0mm,top=0mm,bottom=0mm,%设置盒子的左、右、上、下边距都为0。
watermark opacity=0.25,watermark zoom=1.2,%设置水印的透明度为25%,并放大水印的尺寸。
before=\par\smallskip,%在盒子之前插入一个新段落和一个小的垂直间距。
clip watermark=false,%不裁剪水印,使其可以超出盒子的边界。
watermark tikz={%
\path[fill=yellow,draw=yellow!75!red] (0,0) circle (1cm);%一个黄色的圆脸(circle (1cm))
\fill[red] (45:5mm) circle (1mm);\fill[red] (135:5mm) circle (1mm);%两个红色的眼睛(circle (1mm))
\draw[line width=1mm,red] (215:5mm) arc (215:325:5mm);}]%一个弧形的笑脸(arc (215:325:5mm))
% \begin{stripedbox}
Since the first public release in 2011, %
I received a lot of feedback from all over the world.%
I want to thank all who wrote me for supporting this package by sending bug reports and ideas for new or better features.
% \tcblower

自从2011年第一次公开发布以来,%
我收到了来自世界各地的大量反馈。%
我想感谢所有写信给我的人,感谢他们通过发送错误报告、和关于新的或更好的功能的想法来支持这个宏包。
% \end{stripedbox}
\end{tcolorbox}
    

% \hfill 
\subsection{Installation\\安装}

% \begin{stripedbox}
Typically, |tcolorbox| will be installed as part of a major \LaTeX\ distribution
and there is nothing special to do for a user.
% \tcblower

通常情况下,|tcolorbox| 会作为%主要的 
\LaTeX\ 发行版的一部分被安装,对用户来说没有什么特别的事情要做。
% \end{stripedbox}

% \begin{stripedbox}
If you intend to make a localinstallation \emph{by hand}, %
see the |README| file of the |tcolorbox| package for some hints. %
The short story is: you have to install not only |tcolorbox.sty|, %
but also all |*.code.tex| files in the local |texmf| tree.
% \tcblower

如果你打算\emph{手工}进行本地安装,%
请参阅tcolorbox软件包的README文件,以获得一些提示。%
简单的说: 你不仅要安装 |tcolorbox.sty| ,还要安装本地texmf目录中的所有 |*.code.tex| 文件。
% \end{stripedbox}

% % 
\subsection{Loading the Package\\加载包}


% \begin{stripedbox}
The base package |tcolorbox| loads the packages
|pgf| %\cite{tantau:tikz_and_pgf}
, |verbatim| %\cite{schoepf:2001a},
|etoolbox| %\cite{lehmann:etoolbox}
, and |environ| %\cite{robertson:2014a}
.
|tcolorbox| itself is loaded in the usual manner in the preamble:
% \tcblower

基础包 |tcolorbox| 加载了以下包:
|pgf| ,%\cite{tantau:tikz_and_pgf}, % todo 看
|verbatim| ,%\cite{schoepf:2001a},
|etoolbox| ,%\cite{lehmann:etoolbox},
和 |environ| .%\cite{robertson:2014a}.
% 
% 导言区中加载|tcolorbox|:
|tcolorbox| 本身则是在导言区(preamble)以常规方式加载的:
% \end{stripedbox}


\begin{dispListing}
\usepackage{tcolorbox}
\end{dispListing}


% \begin{stripedbox}
The package takes option keys in the key-value syntax.%
For example, the key to typeset listings is:
% \tcblower

该包的选项采用键值语法。例如,排版代码列表的键是。
%该包采用键值语法中的选项键。例如,设置列表的键是:
% \end{stripedbox}


\begin{dispListing}
\usepackage[listings]{tcolorbox}
\end{dispListing}

% \begin{stripedbox}
Alternatively, you may use these keys later in the preamble with \refCom{tcbuselibrary} (see there).
% \tcblower

% 可选的,
你也可以在导言区中通过使用 \refComLe{tcbuselibrary} 来设置这些键值。%
% \end{stripedbox}




% % \clearpage

\subsection{Libraries\hfill 库}\label{sec:bibliothek}

% \begin{stripedbox}
The base package |tcolorbox| is extendable by program libraries.%
This is done by using option keys while loading the package or inside
the preamble by applying the following macro with the same set of keys.
% \tcblower

基础的|tcolorbox|包还可以由程序库扩展功能。%
这可以在加载宏包时指定可选选项,或者在导言中使用以下命令宏来实现,传递的参数是相同的选项值:
% \end{stripedbox}

\begin{verbatim}
\begin{docCommand}{tcbuselibrary}{\marg{key list}}
...
\end{docCommand}  
\end{verbatim}


\begin{docCommand}{tcbuselibrary}{\marg{key list}}
% \begin{stripedbox}
Loads the libraries given by the \meta{key list}.
% \tcblower

加载由\makebox[0pt]{~}\meta{key list}\makebox[0pt]{~}给出的库。  
% \end{stripedbox} 

\begin{dispListing}
\tcbuselibrary{listings,theorems}
\end{dispListing}

% \begin{stripedbox}
The following keys are used inside |\tcbuselibrary| respectively
|\usepackage| without the key tree path |/tcb/library/|.
% \tcblower

下面的键值(不含 |/tcb/library/|)可以在|\tcbuselibrary|或|\usepackage|中使用。%%
% \end{stripedbox}

\end{docCommand}  


\begin{verbatim}
\tcbmakedocSubKey{docTcbKey}{tcb}%tcolorbox.doc.s_main.sty
\begin{docTcbKey}[library]{skins}{}{\mylib{skins}}
...
\end{docTcbKey}
\end{verbatim}

\begin{docTcbKey}[library]{skins}{}{\mylib{skins}}
% \begin{stripedbox}
Loads the package |tikz| %\cite{tantau:tikz_and_pgf} 
and provides additional styles (skins) for the appearance of the colored boxes; 
see  Section~\ref{sec:skins} from page~\pageref{sec:skins}.
% \tcblower

加载|tikz|\makebox[0pt]{~}%\cite{tantau:tikz_and_pgf} 
并为彩色框的外观提供其他样式(皮肤);
请参见第\pageref{sec:skins}页的~\ref{sec:skins}小节。
% \end{stripedbox}
\end{docTcbKey}

\begin{docTcbKey}[library]{vignette}{}{\mylib{vignette}}
% \begin{stripedbox}
Provides code for more ornamental; see
Section~\ref{sec:vignette} from page~\pageref{sec:vignette}.
% \tcblower

提供更多装饰性代码; 请参见第\pageref{sec:vignette}页的~\ref{sec:vignette}小节。

% % 提供更多装饰性代码,请参见第\ref{sec:vignette}节,从第\pageref{sec:vignette}页开始。
% \end{stripedbox}
\end{docTcbKey}

\begin{docTcbKey}[library]{raster}{}{\mylib{raster}}
% \begin{stripedbox}
Provides additional macros and options for typesetting 
multiple boxes arranged in a kind of raster;
see Section~\ref{sec:raster} from page~\pageref{sec:raster}.
% \tcblower

提供额外的宏和选项排版多个盒子,以一种栅格\footnotemark%
的形式排列。请参见第\pageref{sec:raster}页的~\ref{sec:raster}小节。
% \end{stripedbox}
\end{docTcbKey}
\footnotetext{栅格系统英文为“grid systems”,也有人翻译为“网格系统”,运用固定的格子设计版面布局,其风格工整简洁,在二战后大受欢迎,已成为今日出版物设计的主流风格之一。}
% 栅 格 (zhà gé)
    
\begin{docTcbKey}[library]{listings}{}{\mylib{listings}}

Loads the package |listings| %\cite{hoffmann:listings}
and provides additional macros for typesetting listings which are described in %Section~\ref{sec:listings} from 
page~\pageref{sec:listings}.
% \tcblower

载入|listings|包,并提供额外的用于代码排版的宏,详见~\pageref{sec:listings}页。%的~\ref{sec:listings}小节。

\end{docTcbKey}

\begin{docTcbKey}[library]{listingsutf8}{}{\mylib{listingsutf8}}

Loads the packages |listings| %\cite{hoffmann:listings}
 and |listingsutf8| %\cite{oberdiek:listingsutf8}
  for UTF-8 support.
This is a variant of the library \mylib{listings}
and is described in %Section \ref{sec:listings} from 
page~\pageref{sec:listings}.
% \tcblower

载入 |listings| 和用于支持UTF-8的 |listingsutf8| 包。这是\mylib{listings}的一个变体。%
详见~\pageref{sec:listings}页%
% 的~\ref{sec:listings}小节
。

\end{docTcbKey}

\begin{docTcbKey}[library]{minted}{}{\mylib{minted}}

Loads the package |minted| %\cite{poore:minted} 
to typeset listings with the |Pygments| %\cite{pygments:web}
 tool, also see \Vref{sec:listings}.

 %\tcblower
加载用 |Pygments| %\cite{pygments:web} 
排版代码的|minted|包。另见\Vref{sec:listings}。

\end{docTcbKey}

\begin{docTcbKey}[library]{theorems}{}{\mylib{theorems}}

Provides additional
macros for typesetting theorems which are described in %Section~\ref{sec:theorems}
% from 
page~\pageref{sec:theorems}.

 %\tcblower
为排版定理提供额外的宏,详见~\pageref{sec:theorems}页%的~\ref{sec:theorems} 小节
。

\end{docTcbKey}

% \tcbmakedocSubKey{docCodeKey}{代码}%tcolorbox.doc.s_main.sty
% \begin{docCodeKey}[]{skins}{}{\mylib{skins}}
% ...
% \end{docCodeKey}

