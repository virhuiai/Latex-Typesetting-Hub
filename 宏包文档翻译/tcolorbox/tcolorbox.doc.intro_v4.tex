% !TeX root = tcolorbox.tex

\section{Introduction\\介绍}%
%TODO external/prefix 是啥意思:
\tcbset{external/prefix=external/intro_}%
% external/prefix 是 tikz 宏包的 external 库中的一个键。这个键用于设置 external 库生成的外部图形文件的前缀。
% external 库用于将 tikzpicture 环境中的内容编译为独立的 PDF 文件,这样在重新编译整个文档时,如果这些环境的内容没有改变,LaTeX 可以直接插入已经编译好的 PDF,从而大大加快编译速度。

% \tcbset{external/prefix=external/intro_} 这个命令就是设置了 external 库生成的 PDF 文件的前缀为 external/intro_。这意味着,如果你有一个 tikzpicture 环境,它的标签(label)是 fig1,那么 tikz 将会生成一个名为 external/intro_fig1.pdf 的文件。

% 这个功能在处理包含大量 tikz 图形的大文档时非常有用,可以显著提高编译速度。

% \begin{stripedbox}
The package originates from %
the first edition of my book \flqq{\LaTeX -- Einführung in das Textsatzsystem}%
% {\citetitle{sturm:latex}
\frqq~%\cite{sturm:latex}
in about 2006.%
For the \LaTeX\ examples and tutorials given there, %
I wanted to have accentuated and colored boxes to display source code and
compiled text in combination.%
Since, in my opinion, %
this type of boxes is also quite useful to highlight definitions and theorems,% 
I applied them for my lecture notes in mathematics %\cite{sturm:mathe1,sturm:mathe2,sturm:mathe3}
as well.%
With this package, you are invited to apply these boxes for similar projects.
% 通过这个包,
% \tcblower
% 这个包起源于我2006年出版的\flqq{\LaTeX -- Einführung in das Textsatzsystem}%
% % 这个软件包来源于我在2006年出版的第一版书《 LATEX-Einfühung in das Textsatzsystem 》。
% % {\citetitle{sturm:latex}
% \frqq%~\cite{sturm:latex}
% 一书的第一版。%
% 对于书中给出的 \LaTeX\ 例子和教程,我想用突出的彩色方框来显示源代码和编译后的文本。%
% 因为,在我看来,这种类型的盒子,很适合突出定义和定理, 所以我也把它们应用到我的数学讲义 %\cite{sturm:mathe1,sturm:mathe2,sturm:mathe3} 
% 。%
% 您可以将这个宏包的这些盒子用于类似的项目中。

这个包源于我的书《\LaTeX -- Einführung in das Textsatzsystem》的第一版(约为2006年),其中提供了一些\LaTeX 的示例和教程。为了显示源代码和编译后的文本,我想要有突出和带颜色的框。由于在我看来,这种类型的框对于突出定义和定理也非常有用,因此我也将它们用于我的数学讲义中。通过这个包,您可以将这些框应用于类似的项目中。
% \end{stripedbox}


