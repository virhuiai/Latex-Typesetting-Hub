% !TeX root = tcolorbox.tex
% include file of tcolorbox.tex (manual of the LaTeX package tcolorbox)
%\clearpage
% Technical Overview and Customization
\section{技术概述和定制}\label{sec:technical}%
\tcbset{external/prefix=external/technical_}%
\begin{stripedbox}
This section provides a technical overview of the skin concept of |tcolorbox|.
For most applications of |tcolorbox|, one will not need to know the bells and
whistles described herein.
You may proceed to \Vref{sec:skins} where the customization options
for most users are documented.
\tcblower
本节提供|tcolorbox|皮肤概念的技术概述。
对于|tcolorbox|的大多数应用,人们不需要知道这里描述的内容%花里胡哨
。
您可以继续到\Vref{sec:skins},其中记录了大多数用户的自定义选项。
\end{stripedbox}

\begin{stripedbox}
The following explanations also cover options and settings from the \mylib{skins} library,
see  \Vref{sec:skins}.
\tcblower
下面的解释还包括\mylib{skins}库中的选项和设置,
请参阅\Vref{sec:skins}。
\end{stripedbox}

% Skins and Drawing Engines
\subsection{皮肤和绘图引擎}\label{sec:skincorekeys}
\begin{stripedbox}
From a technical point of view, a \emph{skin} is a style definition for the
appearance of a |tcolorbox|. The core package provides some additional
option keys for skins but only two skins called \refSkin{standard}
and \refSkin{standard jigsaw}.
The \mylib{skins} library adds several more skins. To change to a skin, only one
option from the core package has to be set.
\tcblower
从技术角度来看,\emph{skin}是|tcolorbox|外观的样式定义。%
核心包为皮肤提供了一些额外的选项键,但只有两个皮肤,分别叫\refSkin{standard}和\refSkin{standard jigsaw}。%
\mylib{skins}库添加了更多的皮肤。%
要更改为皮肤,只需要设置核心包中的一个选项。
\end{stripedbox}

\begin{docTcbKey}{skin}{=\meta{name}}{style, no default, initially \texttt{standard}}
\begin{stripedbox}
Sets the current skin to \meta{name}. This is a style definition which sets all the following
keys, i.\,e.\ for many use cases there is nothing more to do.
\tcblower
将当前皮肤设置为\meta{name}。这是一个样式定义,它设置以下键,
i.\,e.\ 对于许多用例,没有更多的事情要做。
\end{stripedbox}

\begin{dispExample}
\tcbset{colback=Salmon!50!white,colframe=FireBrick!75!black,
  width=(\linewidth-8mm)/2,before=,after=\hfill,equal height group=ske}

\begin{tcolorbox}[adjusted title=adjusted皮肤]
  This is my content.
\end{tcolorbox}
\begin{tcolorbox}[beamer,adjusted title=beamer皮肤]
  This is my content.
\end{tcolorbox}
\end{dispExample}
\end{docTcbKey}

\begin{docTcbKey}{skin}{=\meta{name}}{style, no default, initially \texttt{standard}}
  \begin{stripedbox}
  Sets the current skin to \meta{name}. This is a style definition which sets all the following
  keys, i.\,e.\ for many use cases there is nothing more to do.
  \tcblower
  将当前皮肤设置为\meta{name}。这是一个样式定义,它设置以下键,
  i.\,e.\ 对于许多用例,没有更多的事情要做。
  \end{stripedbox}
  
  \begin{dispExample}
  \tcbset{colback=Salmon!50!white,colframe=FireBrick!75!black,
  width=(\linewidth-8mm)/2,before=,after=\hfill,equal height group=ske}
  
  \begin{tcolorbox}[adjusted title=adjusted皮肤]
  This is my content.
  \end{tcolorbox}
  \begin{tcolorbox}[beamer,adjusted title=beamer皮肤]
  This is my content.
  \end{tcolorbox}
  \end{dispExample}
  \end{docTcbKey}

  \begin{docTcbKey}{skin first}{=\meta{name}}{style, no default, initially \texttt{standard}}
\begin{stripedbox}
If the box is set to be \refKey{/tcb/breakable} and \emph{is} broken actually,
then the skin for the \emph{first} part of the break sequence
is set to \meta{name}, see Subsection \ref{subsec:breaksequence} on page \pageref{subsec:breaksequence}.
Typically, this key is set by a \refKey{/tcb/skin}.
\tcblower
如果盒子被设置为\refKey{/tcb/breakable} 且实际上也是分页了,%
然后break序列的\emph{第一}部分的皮肤被设置为\meta{name},
参见 \pageref{subsec:breaksequence} 页上的 \ref{subsec:breaksequence} 小节。
通常,该键由\refKey{/tcb/skin}设置。
\end{stripedbox}
\end{docTcbKey}


\begin{docTcbKey}{skin middle}{=\meta{name}}{style, no default, initially \texttt{standard}}
\begin{stripedbox}
If the box is set to be \refKey{/tcb/breakable} and \emph{is} broken actually,
then the skin for the \emph{middle} parts (if any) of the break sequence
is set to \meta{name}, see Subsection \ref{subsec:breaksequence} on page \pageref{subsec:breaksequence}.
Typically, this key is set by a \refKey{/tcb/skin}.
\tcblower
如果盒子被设置为\refKey{/tcb/breakable} 且实际上也是分页了,%
然后break序列的\emph{中间}部分的皮肤被设置为\meta{name},
参见 \pageref{subsec:breaksequence} 页上的 \ref{subsec:breaksequence} 小节。
通常,该键由\refKey{/tcb/skin}设置。
\end{stripedbox}
\end{docTcbKey}


\begin{docTcbKey}{skin last}{=\meta{name}}{style, no default, initially \texttt{standard}}
\begin{stripedbox}
If the box is set to be \refKey{/tcb/breakable} and \emph{is} broken actually,
then the skin for the \emph{last} part of the break sequence
is set to \meta{name}, see Subsection \ref{subsec:breaksequence} on page \pageref{subsec:breaksequence}.
Typically, this key is set by a \refKey{/tcb/skin}.
\tcblower
如果盒子被设置为\refKey{/tcb/breakable} 且实际上也是分页了,%
然后break序列的\emph{最后}部分的皮肤被设置为\meta{name},
参见 \pageref{subsec:breaksequence} 页上的 \ref{subsec:breaksequence} 小节。
通常,该键由\refKey{/tcb/skin}设置。
\end{stripedbox}
\end{docTcbKey}

\begin{docTcbKey}{skin first}{=\meta{name}}{style, no default, initially \texttt{standard}}
\begin{stripedbox}
If the box is set to be \refKey{/tcb/breakable} and \emph{is} broken actually,
then the skin for the \emph{first} part of the break sequence
is set to \meta{name}, see Subsection \ref{subsec:breaksequence} on page \pageref{subsec:breaksequence}.
Typically, this key is set by a \refKey{/tcb/skin}.
\tcblower
如果盒子被设置为\refKey{/tcb/breakable} 且实际上也是分页了,%
然后break序列的\emph{第一}部分的皮肤被设置为\meta{name},
参见 \pageref{subsec:breaksequence} 页上的 \ref{subsec:breaksequence} 小节。
通常,该键由\refKey{/tcb/skin}设置。
\end{stripedbox}
\end{docTcbKey}


\begin{docTcbKey}{skin middle}{=\meta{name}}{style, no default, initially \texttt{standard}}
\begin{stripedbox}
If the box is set to be \refKey{/tcb/breakable} and \emph{is} broken actually,
then the skin for the \emph{middle} parts (if any) of the break sequence
is set to \meta{name}, see Subsection \ref{subsec:breaksequence} on page \pageref{subsec:breaksequence}.
Typically, this key is set by a \refKey{/tcb/skin}.
\tcblower
如果盒子被设置为\refKey{/tcb/breakable} 且实际上也是分页了,%
然后break序列的\emph{中间}部分的皮肤被设置为\meta{name},
参见 \pageref{subsec:breaksequence} 页上的 \ref{subsec:breaksequence} 小节。
通常,该键由\refKey{/tcb/skin}设置。
\end{stripedbox}
\end{docTcbKey}


\begin{docTcbKey}{skin last}{=\meta{name}}{style, no default, initially \texttt{standard}}
\begin{stripedbox}
If the box is set to be \refKey{/tcb/breakable} and \emph{is} broken actually,
then the skin for the \emph{last} part of the break sequence
is set to \meta{name}, see Subsection \ref{subsec:breaksequence} on page \pageref{subsec:breaksequence}.
Typically, this key is set by a \refKey{/tcb/skin}.
\tcblower
如果盒子被设置为\refKey{/tcb/breakable} 且实际上也是分页了,%
然后break序列的\emph{最后}部分的皮肤被设置为\meta{name},
参见 \pageref{subsec:breaksequence} 页上的 \ref{subsec:breaksequence} 小节。
通常,该键由\refKey{/tcb/skin}设置。
\end{stripedbox}
\end{docTcbKey}

%\clearpage
\begin{docTcbKey}{graphical environment}{=\meta{name}}{no default, initially \texttt{pgfpicture}}
\begin{stripedbox}
Sets the graphical environment for the |tcolorbox| to \meta{name}.
Feasible values are |pgfpicture| and |tikzpicture| or environments which inherit from one of these two. 
This key is set by a \refKey{/tcb/skin} and may seldom be used directly.
\tcblower
将|tcolorbox|的图形环境设置为\meta{name}。%
可行值为|pgfpicture|和|tikzpicture|或继承这两者之一的环境。%
该键由\refKey{/tcb/skin}设置,可能很少直接使用。
\end{stripedbox}
\end{docTcbKey}

\begin{stripedbox}
The skin of a |tcolorbox| is drawn by up to four \emph{engines}.
Afterwards, the text content is drawn which is not part of a skin.
The four steps are:
\tcblower
|tcolorbox|的皮肤是由多达四个\emph{engines}绘制的。%
然后,绘制不属于皮肤的文本内容。%
这四个步骤是:
\end{stripedbox}

\begin{enumerate}
\item 
~\vspace*{-1em}
\begin{stripedbox}
The \emph{frame} of the box, drawn by \refKey{/tcb/frame engine}.
\tcblower
盒子的\emph{frame},由\refKey{/tcb/frame engine}绘制。
\end{stripedbox}

\item 
~\vspace*{-1em}
\begin{stripedbox}
The \emph{interior} of the box. 
The interior of a box with title is   drawn differently from a box without title.
\refKey{/tcb/interior titled engine} or \refKey{/tcb/interior engine}   is used to draw the interior.
\tcblower
盒子的内部。
有标题的盒子内部与没有标题的盒子内部的绘制方式不同。
\refKey{/tcb/interior title engine}或\refKey{/tcb/interior engine}用于绘制内部。
\end{stripedbox}

\item 
~\vspace*{-1em}
\begin{stripedbox}
The \emph{segmentation} (line) of the box, if there is a lower part;
drawn by \refKey{/tcb/segmentation engine}.
\tcblower
盒子的\emph{segmentation}(line),如果有lower的部分;
由\refKey{/tcb/segmentation engine}绘制。
\end{stripedbox}

\item 
~\vspace*{-1em}
\begin{stripedbox}
The \emph{title area} of the box, if there is a title and
\refKey{/tcb/title filled} is set to |true|; drawn
by \refKey{/tcb/title engine}.
\tcblower
盒子中的\emph{title area},
如果存在标题且 \refKey{/tcb/title filled} 设置为 |true|; 使用 \refKey{/tcb/title engine} 绘制。  
\end{stripedbox}
\end{enumerate}

%Every engine for the up to four steps can be set to one of the following types:
%\begin{enumerate}
%\item\docValue{standard}: the original code from the core package.
%\item\docValue{path}: a |tikz| path which can be controlled by options.
%\item\docValue{pathfirst}: a |tikz| path which can be controlled by options.
%\item\docValue{pathmiddle}: a |tikz| path which can be controlled by options.
%\item\docValue{pathlast}: a |tikz| path which can be controlled by options.
%\item\docValue{freelance}: arbitrary user code.
%\item\docValue{spartan}: a quite spartan code.
%\end{enumerate}
%\end{document}%\clearpage

\begin{docTcbKey}{frame engine}{=\meta{name}}{no default, initially \texttt{standard}}
\begin{stripedbox}
Sets the \emph{frame} drawing engine for a box to \meta{name}.
Typically, this key is set by a \refKey{/tcb/skin}.
Feasible values for \meta{name} are:
\tcblower
将盒子的\emph{frame}绘图引擎设置为\meta{name}。
通常,这是由 \refKey{/tcb/skin} 设置。
\meta{name}的可行值是:
\end{stripedbox}

\begin{itemize}
  \item\docValue{standard}: 
\begin{stripedbox}
the original code from the core package,
\tcblower
核心包里的原始代码,
\end{stripedbox}
  
  \item\docValue{path}: 
\begin{stripedbox}
a |tikz| path which is controlled by \refKey{/tcb/frame style},
\tcblower
由\refKey{/tcb/frame style}控制的|tikz|路径,
\end{stripedbox}

  \item\docValue{pathjigsaw}:
\begin{stripedbox}
a |tikz| path which is controlled by \refKey{/tcb/frame style},
\tcblower
由\refKey{/tcb/frame style}控制的|tikz|路径,
\end{stripedbox}

  \item\docValue{pathfirst}:
\begin{stripedbox}
a |tikz| path which is controlled by \refKey{/tcb/frame style},
\tcblower
由\refKey{/tcb/frame style}控制的|tikz|路径,
\end{stripedbox}

  \item\docValue{pathfirstjigsaw}: 
\begin{stripedbox}
a |tikz| path which is controlled by \refKey{/tcb/frame style},
\tcblower
由\refKey{/tcb/frame style}控制的|tikz|路径,
\end{stripedbox}

  \item\docValue{pathmiddle}:
\begin{stripedbox}
a |tikz| path which is controlled by \refKey{/tcb/frame style},
\tcblower
由\refKey{/tcb/frame style}控制的|tikz|路径,
\end{stripedbox}

  \item\docValue{pathmiddlejigsaw}: 
\begin{stripedbox}
a |tikz| path which is controlled by \refKey{/tcb/frame style},
\tcblower
由\refKey{/tcb/frame style}控制的|tikz|路径,
\end{stripedbox}

  \item\docValue{pathlast}: 
\begin{stripedbox}
a |tikz| path which is controlled by \refKey{/tcb/frame style},
\tcblower
由\refKey{/tcb/frame style}控制的|tikz|路径,
\end{stripedbox}
  \item\docValue{pathlastjigsaw}: 
\begin{stripedbox}
a |tikz| path which is controlled by \refKey{/tcb/frame style},
\tcblower
由\refKey{/tcb/frame style}控制的|tikz|路径,
\end{stripedbox}
  \item\docValue{freelance}: 
\begin{stripedbox}
deprecated.
\tcblower
弃用。
\end{stripedbox}
  \item\docValue{spartan}: 
\begin{stripedbox}
a quite spartan code.
\tcblower
非常简朴的代码。
\end{stripedbox}
  \item\docValue{empty}: 
\begin{stripedbox}
draw nothing.
\tcblower
不绘制。
\end{stripedbox}
\end{itemize}
\end{docTcbKey}

\begin{docTcbKey}{interior titled engine}{=\meta{name}}{no default, initially \texttt{standard}}
\begin{stripedbox}
Sets the \emph{interior} drawing engine for a titled box to \meta{name}.
Typically, this key is set by a \refKey{/tcb/skin}.
Feasible values for \meta{name} are:
\tcblower
将带标题的盒子的\emph{内部}绘图引擎设置为\meta{name}。%
通常,该键由\refKey{/tcb/skin}设置。
\meta{name}的可行值是:
\end{stripedbox}  

  \begin{itemize}
  \item\docValue{standard}: the original code from the core package,
  \item\docValue{path}: a |tikz| path which is controlled by \refKey{/tcb/interior style},
  \item\docValue{pathfirst}: a |tikz| path which is controlled by \refKey{/tcb/interior style},
  \item\docValue{pathmiddle}: a |tikz| path which is controlled by \refKey{/tcb/interior style},
  \item\docValue{pathlast}: a |tikz| path which is controlled by \refKey{/tcb/interior style},
  \item\docValue{freelance}: deprecated.
  \item\docValue{spartan}: a quite spartan code.
  \item\docValue{empty}: draw nothing.
  \end{itemize}
\end{docTcbKey}

%\clearpage
\begin{docTcbKey}{interior engine}{=\meta{name}}{no default, initially \texttt{standard}}
\begin{stripedbox}
Sets the \emph{interior} drawing engine for an untitled box to \meta{name}.
Typically, this key is set by a \refKey{/tcb/skin}.
Feasible values for \meta{name} are:
\tcblower
将不带标题的盒子的\emph{内部}绘图引擎设置为\meta{name}。
通常,该键由\refKey{/tcb/skin}设置。
\meta{name}的可行值是:
    \end{stripedbox}
  \begin{itemize}
  \item\docValue{standard}: the original code from the core package,
  \item\docValue{path}: a |tikz| path which is controlled by \refKey{/tcb/interior style},
  \item\docValue{pathfirst}: a |tikz| path which is controlled by \refKey{/tcb/interior style},
  \item\docValue{pathmiddle}: a |tikz| path which is controlled by \refKey{/tcb/interior style},
  \item\docValue{pathlast}: a |tikz| path which is controlled by \refKey{/tcb/interior style},
  \item\docValue{freelance}: deprecated.
  \item\docValue{spartan}: a quite spartan code.
  \item\docValue{empty}: draw nothing.
  \end{itemize}
\end{docTcbKey}

