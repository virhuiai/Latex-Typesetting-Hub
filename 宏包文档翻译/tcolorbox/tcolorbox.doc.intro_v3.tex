% !TeX root = tcolorbox.tex
% include file of tcolorbox.tex (manual of the LaTeX package tcolorbox)
% \clearpage
% 
\section{Introduction\\介绍}%
%TODO external/prefix 是啥意思?
\tcbset{external/prefix=external/intro_}%

\begin{stripedbox}
The package originates from %
the first edition of my book \flqq{\LaTeX -- Einführung in das Textsatzsystem}%
% {\citetitle{sturm:latex}
\frqq~%\cite{sturm:latex}
in about 2006.%
For the \LaTeX\ examples and tutorials given there, %
I wanted to have accentuated and colored boxes to display source code and
compiled text in combination.%
Since, in my opinion, %
this type of boxes is also quite useful to highlight definitions and theorems,% 
I applied them for my lecture notes in mathematics %\cite{sturm:mathe1,sturm:mathe2,sturm:mathe3}
as well.%
With this package, you are invited to apply these boxes for similar projects.
% 通过这个包,
\tcblower
% 这个包起源于我2006年出版的\flqq{\LaTeX -- Einführung in das Textsatzsystem}%
% % 这个软件包来源于我在2006年出版的第一版书《 LATEX-Einfühung in das Textsatzsystem 》。
% % {\citetitle{sturm:latex}
% \frqq%~\cite{sturm:latex}
% 一书的第一版。%
% 对于书中给出的 \LaTeX\ 例子和教程,我想用突出的彩色方框来显示源代码和编译后的文本。%
% 因为,在我看来,这种类型的盒子,很适合突出定义和定理, 所以我也把它们应用到我的数学讲义 %\cite{sturm:mathe1,sturm:mathe2,sturm:mathe3} 
% 。%
% 您可以将这个宏包的这些盒子用于类似的项目中。

这个包源于我的书《\LaTeX -- Einführung in das Textsatzsystem》的第一版(约为2006年),其中提供了一些\LaTeX 的示例和教程。为了显示源代码和编译后的文本,我想要有突出和带颜色的框。由于在我看来,这种类型的框对于突出定义和定理也非常有用,因此我也将它们用于我的数学讲义中。通过这个包,您可以将这些框应用于类似的项目中。
\end{stripedbox}




\begin{stripedbox}
The breaking news for version 2.00 was the support for breakable boxes.%
This feature allows new applications of the package without affecting the core package too much if you do not need boxes to break automatically.%
With version 2.20, the often requested \enquote{side by side} mode for listings has been added.%
With version 3.00, boxed titles are introduced together with improved customization options for overlays, underlays, finishes, and own code extensions.%
\tcblower
2.00版的爆炸性新闻是,盒子可以支持换页了。%
如果你不需要盒子自动换页,这个功能不太影响核心包。%不影响意思,去除
2.20版, 加上了经常被要求加上的,排版代码清单的 \enquote{side by side} 模式。%
在版本3.00中,盒标题会和改进了的,用于定制overlays(覆盖层)、underlays(衬垫)、finishes和自行扩展的选项一起介绍。
\end{stripedbox}

% 笑脸 TODO 看下
\begin{tcolorbox}[enhanced,boxrule=0mm,boxsep=0mm,frame hidden,interior hidden,
left=0mm,right=0mm,top=0mm,bottom=0mm,watermark opacity=0.25,watermark zoom=1.2,before=\par\smallskip,
clip watermark=false,
watermark tikz={%
\path[fill=yellow,draw=yellow!75!red] (0,0) circle (1cm);
\fill[red] (45:5mm) circle (1mm);
\fill[red] (135:5mm) circle (1mm);
\draw[line width=1mm,red] (215:5mm) arc (215:325:5mm);}]
\begin{stripedbox}
Since the first public release in 2011, %
I received a lot of feedback from all over the world.%
I want to thank all who wrote me for supporting this package by sending bug reports and ideas for new or better features.
\tcblower
自从2011年第一次公开发布以来,%
我收到了来自世界各地的大量反馈。%
我想感谢所有写信给我的人,感谢他们通过发送错误报告、和关于新的或更好的功能的想法来支持这个宏包。
\end{stripedbox}
\end{tcolorbox}

% \hfill 
\subsection{Installation\\安装}

\begin{stripedbox}
Typically, |tcolorbox| will be installed as part of a major \LaTeX\ distribution
and there is nothing special to do for a user.
\tcblower
通常情况下,|tcolorbox| 会作为%主要的 
\LaTeX\ 发行版的一部分被安装。%,对用户来说没有什么特别的事情要做。
\end{stripedbox}

\begin{stripedbox}
If you intend to make a localinstallation \emph{by hand}, %
see the |README| file of the |tcolorbox| package for some hints. %
The short story is: you have to install not only |tcolorbox.sty|, %
but also all |*.code.tex| files in the local |texmf| tree.
\tcblower
如果你打算\emph{手工}进行本地安装,%
请参阅tcolorbox软件包的README文件,以获得一些提示。%
简单的说: 你不仅要安装 |tcolorbox.sty| ,还要安装本地texmf目录中的所有 |*.code.tex| 文件。
\end{stripedbox}

% 
\subsection{Loading the Package\\加载包}


\begin{stripedbox}
The base package |tcolorbox| loads the packages
|pgf| %\cite{tantau:tikz_and_pgf}
, |verbatim| %\cite{schoepf:2001a},
|etoolbox| %\cite{lehmann:etoolbox}
, and |environ| %\cite{robertson:2014a}
.
|tcolorbox| itself is loaded in the usual manner in the preamble:
\tcblower
基础包 |tcolorbox| 加载了以下包:
|pgf| ,%\cite{tantau:tikz_and_pgf}, % todo 看
|verbatim| ,%\cite{schoepf:2001a},
|etoolbox| ,%\cite{lehmann:etoolbox},
和 |environ| .%\cite{robertson:2014a}.
% 
导言区中加载|tcolorbox|:
\end{stripedbox}

\begin{dispListing}
\usepackage{tcolorbox}
\end{dispListing}


\begin{stripedbox}
The package takes option keys in the key-value syntax.%
For example, the key to typeset listings is:
\tcblower
该包的选项采用键值语法。例如,排版代码列表的键是。
% 该包采用键值语法中的选项键。例如,设置列表的键是:
\end{stripedbox}


\begin{dispListing}
\usepackage[listings]{tcolorbox}
\end{dispListing}

\begin{stripedbox}
Alternatively, you may use these keys later in the preamble with \refCom{tcbuselibrary} (see there).
\tcblower
% 可选的,
你也可以在导言区中通过使用 \refCom{tcbuselibrary} 来设置这些键值。%
\end{stripedbox}


% \clearpage
% Libraries\hfill 
\subsection{库}\label{sec:bibliothek}

\begin{stripedbox}
The base package |tcolorbox| is extendable by program libraries.%
This is done by using option keys while loading the package or inside
the preamble by applying the following macro with the same set of keys.
\tcblower
基础的|tcolorbox|包还可以由程序库扩展功能。%
这可以在加载宏包时指定可选选项,或者在导言中使用以下命令宏来实现,传递的参数是相同的选项值:
\end{stripedbox}

\begin{docCommand}{tcbuselibrary}{\marg{key list}}
\begin{stripedbox}
Loads the libraries given by the \meta{key list}.
\tcblower
加载由\makebox[0pt]{~}\meta{key list}\makebox[0pt]{~}给出的库。  
\end{stripedbox} 

\begin{dispListing}
\tcbuselibrary{listings,theorems}
\end{dispListing}

\begin{stripedbox}
The following keys are used inside |\tcbuselibrary| respectively
|\usepackage| without the key tree path |/tcb/library/|.
\tcblower
下面的键值(不含 |/tcb/library/|)可以在|\tcbuselibrary|或|\usepackage|中使用。%%
\end{stripedbox}

\end{docCommand}  

\begin{docTcbKey}[library]{skins}{}{\mylib{skins}}
\begin{stripedbox}
Loads the package |tikz| %\cite{tantau:tikz_and_pgf} 
and provides additional styles (skins) for the appearance of the colored boxes; 
see  Section~\ref{sec:skins} from page~\pageref{sec:skins}.
\tcblower
加载|tikz|\makebox[0pt]{~}%\cite{tantau:tikz_and_pgf} 
并为彩色框的外观提供其他样式(皮肤);
请参见第\pageref{sec:skins}页的~\ref{sec:skins}小节。
\end{stripedbox}
\end{docTcbKey}

\begin{docTcbKey}[library]{vignette}{}{\mylib{vignette}}
\begin{stripedbox}
Provides code for more ornamental; see
Section~\ref{sec:vignette} from page~\pageref{sec:vignette}.
\tcblower
提供更多装饰性代码; 请参见第\pageref{sec:vignette}页的~\ref{sec:vignette}小节。

% 提供更多装饰性代码,请参见第\ref{sec:vignette}节,从第\pageref{sec:vignette}页开始。
\end{stripedbox}
\end{docTcbKey}

\begin{docTcbKey}[library]{raster}{}{\mylib{raster}}
\begin{stripedbox}
Provides additional macros and options for typesetting 
multiple boxes arranged in a kind of raster;
see Section~\ref{sec:raster} from page~\pageref{sec:raster}.
\tcblower
提供额外的宏和选项排版多个盒子,以一种栅格\footnotemark%
的形式排列。请参见第\pageref{sec:raster}页的~\ref{sec:raster}小节。
\end{stripedbox}
\end{docTcbKey}
\footnotetext{栅格系统英文为“grid systems”,也有人翻译为“网格系统”,运用固定的格子设计版面布局,其风格工整简洁,在二战后大受欢迎,已成为今日出版物设计的主流风格之一。}


\begin{docTcbKey}[library]{listings}{}{\mylib{listings}}
\begin{stripedbox}
Loads the package |listings| %\cite{hoffmann:listings}
and provides additional macros for typesetting listings which are described in Section~\ref{sec:listings} from page~\pageref{sec:listings}.
\tcblower
载入|listings|包,并提供额外的用于代码排版的宏,详见~\pageref{sec:listings}页的~\ref{sec:listings}小节。
\end{stripedbox}
\end{docTcbKey}

\begin{docTcbKey}[library]{listingsutf8}{}{\mylib{listingsutf8}}
\begin{stripedbox}
Loads the packages |listings| %\cite{hoffmann:listings}
 and |listingsutf8| %\cite{oberdiek:listingsutf8}
  for UTF-8 support.
This is a variant of the library \mylib{listings}
and is described in Section \ref{sec:listings}
from page~\pageref{sec:listings}.
\tcblower
载入 |listings| 和用于支持UTF-8的 |listingsutf8| 包。这是\mylib{listings}的一个变体。%
详见~\pageref{sec:listings}页的~\ref{sec:listings}小节。
\end{stripedbox}
\end{docTcbKey}



\begin{docTcbKey}[library]{minted}{}{\mylib{minted}}
\begin{stripedbox}
Loads the package |minted| %\cite{poore:minted} 
to typeset listings with the |Pygments| %\cite{pygments:web}
 tool, also see \Vref{sec:listings}.
\tcblower
加载用 |Pygments| %\cite{pygments:web} 
排版代码的|minted|包。另见\Vref{sec:listings}。
\end{stripedbox}
\end{docTcbKey}

\begin{docTcbKey}[library]{theorems}{}{\mylib{theorems}}
\begin{stripedbox}
Provides additional
macros for typesetting theorems which are described in Section~\ref{sec:theorems}
from page~\pageref{sec:theorems}.
\tcblower
为排版定理提供额外的宏,详见~\pageref{sec:theorems}页的~\ref{sec:theorems} 小节。
\end{stripedbox}
\end{docTcbKey}

\begin{docTcbKey}[library]{breakable}{}{\mylib{breakable}}
\begin{stripedbox}
Provides support for automatic box breaking from one page to another;
see \Fullref{sec:breakable}.
\tcblower
为 |tcolorbox| 盒子提供了自动分页的支持。见\Fullref{sec:breakable}。

% 提供支持自动在页面之间进行盒子换行的功能;请参阅\Fullref{sec:breakable}。
\end{stripedbox}



% 仓库
\begin{docTcbKey}[library]{magazine}{}{\mylib{magazine}}
\begin{stripedbox}
Provides support for storing broken box parts to be used later or
in interchanged order, \Fullref{sec:magazine}.
\tcblower
为储存盒子的分开的各个部分提供支持,以便以后使用或互换顺序,见\Fullref{sec:magazine}。
\end{stripedbox}
\end{docTcbKey}
% \end{docTcbKey}
% magazine | BrE maɡəˈziːn, AmE ˈmæɡəˌzin |
% noun
% ① (publication) 杂志 zázhì
% ▸ a literary magazine
% 文学期刊
% ② (on radio, TV) (also magazine programme) 专题节目 zhuāntí jiémù
% ③ (of gun) 弹仓 dàncāng
% ④ (store for arms, ammunition) 弹药库 dànyàokù

\begin{docTcbKey}[library]{poster}{}{\mylib{poster}}
\begin{stripedbox}
Provides support for creating posters, \Fullref{sec:poster}.
\tcblower
为创作海报提供支持, \Fullref{sec:poster}。
\end{stripedbox}
\end{docTcbKey}

\begin{docTcbKey}[library]{fitting}{}{\mylib{fitting}}
\begin{stripedbox}
Provides support for font size adaption of the box content to
the box dimensions;
see Section~\ref{sec:fitting} from page~\pageref{sec:fitting}.
\tcblower
提供对盒子内的字体大小适应盒子尺寸的支持。详见~\pageref{sec:fitting}页的~\ref{sec:fitting}小节。
\end{stripedbox}
\end{docTcbKey}

\begin{docTcbKey}[library]{hooks}{}{\mylib{hooks}}
\begin{stripedbox}
Extends several option keys to \enquote{hookable} keys;
see Section~\ref{sec:hooks} from page~\pageref{sec:hooks}.
\tcblower
将几个选项键扩展为 \enquote{hookable} 键。详见~\pageref{sec:hooks}页的~\ref{sec:hooks}小节。
\end{stripedbox}
\end{docTcbKey}

% \clearpage
\begin{docTcbKey}[library]{xparse}{}{\mylib{xparse}}
\begin{stripedbox}
Provides document command production with |xparse| for |tcolorbox|;
see Section~\ref{sec:xparse} from page~\pageref{sec:xparse}.
\tcblower
为tcolorbox提供来自xparse的文档命令。详见~\pageref{sec:xparse}页的~\ref{sec:xparse}小节。
\end{stripedbox}
\end{docTcbKey}

\begin{docTcbKey}[library]{external}{}{\mylib{external}}
\begin{stripedbox}
Provides externalization support for stand-alone document snippets,
see \Fullref{sec:external}.
\tcblower
为独立的文档片段提供外部化支持。见\Fullref{sec:external}。
\end{stripedbox}
\end{docTcbKey}

\begin{docTcbKey}[library]{documentation}{}{\mylib{documentation}}
\begin{stripedbox}
Provides additional macros for typesetting \LaTeX\ documentations
which are described in Section~\ref{sec:documentation}
from page~\pageref{sec:documentation}. 
\tcblower
为排版 \LaTeX\ 教程提供额外的宏。详见~\pageref{sec:documentation}页的~\ref{sec:documentation}小节。
\end{stripedbox}
\end{docTcbKey}

\begin{docTcbKey}[library]{many}{}{style, no value}
\begin{stripedbox}
Loads the libraries \mylib{skins}, \mylib{breakable}, \mylib{raster}, \mylib{hooks},
\mylib{theorems}, \mylib{fitting}, and \mylib{xparse}.
Use this shortcut, if you want to use all features of |tcolorbox|
with exception of typesetting listings and using
the specialized \mylib{documentation} library.
\tcblower
加载\mylib{skins}、\mylib{breakable}、\mylib{raster}、\mylib{hooks}、%
\mylib{theorems}、\mylib{fitting} 和 \mylib{xparse}.%
如果你除了排版代码和使用专门的\mylib{documentation}包之外,想使用tcolorbox的所有功能,请使用这个快捷选项。
\end{stripedbox}
\end{docTcbKey}

\begin{docTcbKey}[library]{most}{}{style, no value}
\begin{stripedbox}
Loads all libraries except \mylib{minted} and \mylib{documentation}.
Use this shortcut, if you want to use all features of |tcolorbox|
with exception of using the |minted| package and using
the specialized \mylib{documentation} library.
\tcblower
加载除 \mylib{minted} 和 \mylib{documentation} 之外的所有包。如果你想使用tcolorbox的所有功能,除了使用 \mylib{minted} 和 \mylib{documentation} 包之外,请使用这个快捷选项。
\end{stripedbox}
\end{docTcbKey}

\begin{docTcbKey}[library]{all}{}{style, no value}
\begin{stripedbox}
Loads all libraries. Use this shortcut only, if you intend to use the
\mylib{documentation} library.
\tcblower
加载所有的包。只有当你打算使用 \mylib{documentation} 包时使用这个快捷选项。
\end{stripedbox}
\end{docTcbKey}

\begin{extcolorbox}[runs=2]%为代码段设置编译运行的次数。
{intro_packageoverview}
[title={\texttt{tcolorbox}包},center title,fonttitle=\bfseries,arc=0pt,
colback=red!10!white,
interior style={fill tile image*={width=2cm}{pink_marble.png},fill image opacity=0.5},
colframe=red!50!black]
\begin{tcolorbox}[beamer,adjusted title=基本特性,colframe=blue!50!black,colback=blue!10!white]
只包含基础包
\end{tcolorbox}
\begin{tcbitemize}[raster columns=3,raster before skip=2mm,raster after skip=0pt,
  raster equal height,beamer,colframe=blue!50!black,colback=blue!10!white]
\tcbitem[adjusted title=进阶特性]
  \mylib{breakable}\\
  \mylib{external}\\
  \mylib{fitting}\\
  \mylib{hooks}\\
  \mylib{magazine}\\
  \mylib{poster}\\
  \mylib{raster}\\ 
  \mylib{skins}\\
  \mylib{theorems}\\
  \mylib{vignette}\\
  \mylib{xparse}
\tcbitem[adjusted title=进阶的代码列表]
  \mylib{listings}\\
  \mylib{listingsutf8}
  \tcblower
  \mylib{minted}
\tcbitem[adjusted title=编写文档用]
  \mylib{documentation}
\end{tcbitemize}
\end{extcolorbox}