% !TeX root = tcolorbox.tex
% include file of tcolorbox.tex (manual of the LaTeX package tcolorbox)
% \clearpage
% Recording
\setcounter{section}{7}
\section{记录}\label{sec:recording}%
\tcbset{external/prefix=external/recording_}%
\begin{stripedbox}
The package provides some macros and options to take \emph{records} during compilation. 
This is done by \LaTeX\ file operations to save some data to a file for later usage. 
The main application scenario is depicted in \Vref{sec:recording-exercises} where information about example solutions is recorded and read again in \Vref{sec:recording-solutions}.
\tcblower
% 这个包提供了一些宏和选项来在编译期间获取\emph{records}。%
% 这是通过 \LaTeX\ 的文件操作来完成的,以将一些数据保存到文件中以供以后使用。%
% \Vref{sec:recording-solutions}描述了主要的应用场景,其中关于示例解决方案的信息被记录下来,并在\Vref{sec:recording-solutions}中再次读取。

该套件提供了一些宏和选项,用于在编译过程中记录“记录”。这是通过\LaTeX 文件操作将一些数据保存到文件中以供以后使用来完成的。主要应用场景如\Vref{sec:recording-exercises}所示,在此场景中记录示例解决方案的信息,并在\Vref{sec:recording-solutions}中再次读取。
\end{stripedbox}

% Macros
\subsection{宏}\label{sec:recording-makros}
\begin{docCommand}[doc new=2014-11-28]{tcbstartrecording}{\oarg{file name}}
% \begin{stripedbox}
Opens a file denoted by \meta{file name} for writing the records.
The default file name is |\jobname.records|.
See \Vref{sec:recording-exercises} for an example application.

% \tcblower
% 打开一个由\meta{file name}表示的文件用于写入记录。%
% 默认文件名为 |\jobname.records|。%
% 示例应用,请参阅\Vref{sec:recording-exercises}。
打开一个由\meta{文件名}指定的文件,用于写入记录。 默认文件名是|\jobname.records|。 请参见\Vref{sec:recording-exercises}以获取示例应用程序。
% \end{stripedbox}

\begin{marker}
% \begin{stripedbox}
In some situations, a not existing optional parameter may cause parsing problems. 
If this happens (or just for precaution), use
% \tcblower

在某些情况下,不存在的可选参数可能会导致解析问题。%
如果发生这种情况(或只是为了预防),请使用
% \end{stripedbox}

\begin{dispListing}
\tcbstartrecording\relax
\end{dispListing}
  or
\begin{dispListing}
\tcbstartrecording[\jobname.records]
\end{dispListing}
  \end{marker}
\end{docCommand}

\begin{docCommand}[doc new=2014-11-28]{tcbrecord}{\marg{content}}
% \begin{stripedbox}
Records any \meta{content} to the record file.
\refComLe{tcbrecord} is implemented as |\immediate\write|.
\refComLe{tcbstartrecording} has to be called before; otherwise, \refComLe{tcbrecord} is silently ignored.
% \tcblower

将任何\meta{内容}记录到记录文件中。%
\refComLe{tcbrecord}实现为|\immediate\write|。%
必须先调用\refComLe{tcbstartrecording};否则,\refComLe{tcbrecord}将被默默地忽略。
% \end{stripedbox}

\begin{dispListing}
\tcbrecord{\string\solution{\thetcbcounter}{solutions/exercise-\thetcbcounter.tex}}
% \string\solution是一个命令名称,它表示将要记录的tcolorbox环境的名称;
% {\thetcbcounter}表示当前的tcolorbox计数器的值,也就是当前记录的tcolorbox环境的编号;
% {solutions/exercise-\thetcbcounter.tex}表示记录的文件名,
% 即将当前tcolorbox环境的内容保存到名为solutions/exercise-\thetcbcounter.tex的文件中。
% 因此,这段代码的作用是将当前的tcolorbox环境(假设它的名称是\solution)的状态
% 保存到名为solutions/exercise-\thetcbcounter.tex的文件中。
% 这样,在后面就可以从这个文件中取出内容。
\end{dispListing}
\end{docCommand}
%todo 也许,除了\来开启命令的开始,其他会原样保存起来

\begin{docCommand}[doc new=2014-11-28]{tcbstoprecording}{}
% \begin{stripedbox}
Closes the current record file which was opened by \refComLe{tcbstartrecording} before.
% \tcblower

关闭当前由\refComLe{tcbstartrecording}打开的记录文件。
% \end{stripedbox}
\end{docCommand}

\begin{docCommand}[doc new=2014-11-28]{tcbinputrecords}{\oarg{file name}}
% \begin{stripedbox}
Opens a file denoted by \meta{file name} for reading the records via |\input|.
The default file name is the name of the last used record file for saving.
\refComLe{tcbstoprecording} has to be called before.
% \tcblower

打开一个由\meta{文件名}表示的文件,通过|\input|读取记录。%
默认文件名为上一次使用的记录文件的保存名称。%
必须先调用\refComLe{tcbstoprecording}。
% \end{stripedbox}
\end{docCommand}

% Options
\subsection{选项}\label{sec:recording-options}
\begin{docTcbKey}[][doc new=2014-11-28]{record}{=\meta{content}}{style, no default}
\begin{stripedbox}
Records any \meta{content} to the record file, see \refComLe{tcbrecord}.
This key can be used several times to write several lines.
\tcblower
将任何\meta{content}记录到记录文件中,参见\refComLe{tcbrecord}。%
可以多次使用来写多行。
\end{stripedbox}
  
  \begin{dispListing}
  record={\string\solution{\thetcbcounter}{solutions/exercise-\thetcbcounter.tex}}
  \end{dispListing}
\end{docTcbKey}

\begin{docTcbKey}[][doc new=2014-11-28]{no recording}{}{}
\begin{stripedbox}
Disables \refComLe{tcbrecord} and \refKeyLe{/tcb/record} inside the current   group.
\tcblower
在当前组中禁用\refComLe{tcbrecord}和\refKeyLe{/tcb/record}。
\end{stripedbox}
  
\end{docTcbKey}

% \clearpage
% Example: Exercises
\subsection{范例:练习}\label{sec:recording-exercises}
\begin{stripedbox}
The following application example creates exercises and their corresponding solutions. 
Each pair is generated inside a single |tcolorbox| where the solution is given below \refComLe{tcblower}. 
For every example, the solution part is saved by \refKeyLe{/tcb/savelowerto} to a file. 
The saving is recorded using \refKeyLe{/tcb/record}. 
To enlighten the possibilities, the second exercise has no solution. 
Finally, the solutions are input in \Vref{sec:recording-solutions}.
\tcblower
下面的应用程序示例创建练习及其相应的参考答案。%
每对都在单个|tcolorbox|中生成,其中参考答案在\refComLe{tcblower}下面给出。%
对于每个示例,参考答案部分都通过\refKeyLe{/tcb/savelowerto}保存到一个文件中。%
保存记录使用\refKeyLe{/tcb/record}\footnote{可以多次使用来写多行}。%
为了启发可能性,第二个练习没有解决方案。%
最后,\Vref{sec:recording-solutions}将解输入。
\end{stripedbox}


\inputpreamblelisting{L}

\begin{dispListing*}{breakable,before upper=}
\tcbstartrecording\relax

\begin{exercise}
  Compute the derivative of the following function:
  \begin{equation*}
    f(x)=\sin((\sin x)^2)
  \end{equation*}
\tcblower
  The derivative is:
  \begin{align*}
    f'(x) &= \left( \sin((\sin x)^2) \right)'
    =\cos((\sin x)^2) 2\sin x \cos x.
  \end{align*}
\end{exercise}

\begin{exercise}[no solution]
  It holds:
  \begin{equation*}
    \frac{d}{dx}\left(\ln|x|\right) = \frac{1}{x}.
  \end{equation*}
\end{exercise}

\begin{exercise}
  Compute the derivative of the following function:
  \begin{equation*}
    f(x)=(\sin(\sin x))^2
  \end{equation*}
\tcblower
  The derivative is:
  \begin{align*}
    f'(x) &= \left( (\sin(\sin x))^2 \right)'
    =2\sin(\sin x)\cos(\sin x)\cos x.
  \end{align*}
\end{exercise}

\begin{exercise}
  Compute the derivative of the following function:
  \begin{equation*}
    f(x)=\sqrt{x^3-6x^2+2x}
  \end{equation*}
\tcblower
  The derivative is:
  \begin{align*}
    f'(x) &= \left( \sqrt{x^3-6x^2+2x} \right)'
    = \frac{3x^2-12x+2}{2\sqrt{x^3-6x^2+2x}}.
  \end{align*}
\end{exercise}

\begin{exercise}
  Compute the derivative of the following function:
  \begin{equation*}
    f(x)=\left(\frac{2+3x}{1-2x}\right)^3
  \end{equation*}
\tcblower
  The derivative is:
  \begin{align*}
    f'(x) &= \left( \left(\frac{2+3x}{1-2x}\right)^3 \right)'
    = 3 \left(\frac{2+3x}{1-2x}\right)^2 \frac{(1-2x)3-(2+3x)(-2)}{(1-2x)^2}
    = \frac{21(2+3x)^2}{(1-2x)^4}.
  \end{align*}
\end{exercise}

\begin{exercise}
  Compute the derivative of the following function:
  \begin{equation*}
    f(x)=\frac{\cos x}{(\tan 2x)^2}
  \end{equation*}
\tcblower
  The derivative is:
  \begin{align*}
    f'(x) &= \left( \frac{\cos x}{(\tan 2x)^2} \right)'
    = \left( \frac{\cos x (\cos 2x)^2}{(\sin 2x)^2} \right)'\\
    &= \frac{(\sin 2x)^2 [(-\sin x)(\cos 2x)^2+(\cos x)4\cos 2x (-\sin 2x)]
       - \cos x (\cos 2x)^2 4\sin 2x \cos 2x}{(\sin 2x)^4}\\
    &= -\frac{\cos(2x) [\sin x \sin 2x \cos 2x+ 4\cos x(\sin 2x)^2
       + 4 \cos x (\cos 2x)^2]}{(\sin 2x)^3}\\
    &= -\frac{\cos(2x) [\sin x \sin 2x \cos 2x+ 4\cos x]}{(\sin 2x)^3}.
  \end{align*}
\end{exercise}

\begin{exercise}
  Compute the derivative of the following function:
  \begin{equation*}
    f(x)=\cos((2x^2+3)^3)
  \end{equation*}
\tcblower
  The derivative is:
  \begin{align*}
    f'(x) &= \left( \cos((2x^2+3)^3) \right)'
    =-\sin((2x^2+3)^3) 3(2x^2+3)^2 2\cdot 2x\\
    &=-12x(2x^2+3)^2\sin((2x^2+3)^3).
  \end{align*}
\end{exercise}

\begin{exercise}
  Compute the derivative of the following function:
  \begin{equation*}
    f(x)=(x^2+1)\sqrt{x^4+1}
  \end{equation*}
\tcblower
  The derivative is:
  \begin{align*}
    f'(x) &= \left( (x^2+1)\sqrt{x^4+1} \right)'
    = 2x\sqrt{x^4+1} + \frac{2x^3(x^2+1)}{\sqrt{x^4+1}}.
  \end{align*}
\end{exercise}

\tcbstoprecording
\end{dispListing*}
\tcbusetemp

% Example: Solutions
\subsection{范例:参考答案}\label{sec:recording-solutions}
\begin{stripedbox}
This concludes the example given in \Vref{sec:recording-exercises}.
Now, the saved and recorded solutions are included.
\tcblower
这就结束了\Vref{sec:recording-exercises}中给出的示例。%
现在,包含了保存和记录的解决方案。
\end{stripedbox}


\begin{dispListing*}{breakable,before upper=}
\tcbinputrecords
\end{dispListing*}
\tcbusetemp
