
\subsection{Generic Fading Settings\\通用渐变设置}\label{subsec:vignettefading}

The |fadings| library of |tikz|  is loaded
automatically by the \mylib{vignette} library.
Amongst others, the fadings
\docFading{west},
\docFading{east},
\docFading{north}, and
\docFading{south} are defined inside the |fadings| library.
\\|tikz| 的 |fadings| 库会被 \mylib{vignette} 库自动加载。在其中,|fadings| 库定义了许多渐变,包括 \docFading{west}、\docFading{east}、\docFading{north} 和 \docFading{south}。


The \mylib{vignette} library adds some more fadings called
\docFading{semi west},
\docFading{semi east},
\docFading{semi north}, and
\docFading{semi south}.
These fadings are much \emph{weaker} than the normal fadings.\\\mylib{vignette} 库添加了一些称为 \docFading{semi west}、\docFading{semi east}、\docFading{semi north} 和 \docFading{semi south} 的渐变效果。这些渐变效果比普通的渐变效果要 \emph{弱} 得多。

\begin{dispExample*}{sbs,righthand width=3cm,center lower}
\begin{tikzpicture}
\fill [black!20] (0,0) rectangle (1,1);
\path [pattern=checkerboard,pattern color=black!30]
(0,0) rectangle (1,1);
\fill [path fading=semi west,blue] (0,0) rectangle (1,1);
\end{tikzpicture}
\end{dispExample*}



\begin{tcboxedraster}{base example,title=Comparison of the Fadings}
\def\doShadingExample#1{%
\begin{tcolorbox}[sbs,size=fbox,colback=white,lower separated=false,
righthand width=2cm,left=5mm]
\docFading{#1}\tcblower
\begin{tikzpicture}
\fill [black!20] (0,0) rectangle (1,1);
\path [pattern=checkerboard,pattern color=black!30] (0,0) rectangle (1,1);
\fill [path fading=#1,blue] (0,0) rectangle (1,1);
\end{tikzpicture}
\end{tcolorbox}}%
\doShadingExample{west}
\doShadingExample{east}
\doShadingExample{north}
\doShadingExample{south}
\doShadingExample{semi west}
\doShadingExample{semi east}
\doShadingExample{semi north}
\doShadingExample{semi south}
\end{tcboxedraster}


% \clearpage

\begin{vigTcbKey}[][doc new=2016-04-22]{fade in}{\colOpt{=\marg{style}}}{style, default |white|}
Sets the four style options
\refKeyLe{/tcb/vig/north style},
\refKeyLe{/tcb/vig/south style},
\refKeyLe{/tcb/vig/east style}, and
\refKeyLe{/tcb/vig/west style}
such that the paths fade from outside to inside.
\\设置四个样式选项\refKeyLe{/tcb/vig/north style}、\refKeyLe{/tcb/vig/south style}、\refKeyLe{/tcb/vig/east style}和\refKeyLe{/tcb/vig/west style},使路径从外向内逐渐消失。
\begin{dispExample*}{sbs,righthand width=3cm,center lower}
\begin{tikzpicture}
\fill [black!20] (-0.5,-0.5) rectangle (1.5,1.5);
\path [pattern=checkerboard,pattern color=black!30]
(-0.5,-0.5) rectangle (1.5,1.5);
\tcbvignette{fade in=blue}
\end{tikzpicture}
\end{dispExample*}
\end{vigTcbKey}


\begin{vigTcbKey}[][doc new=2016-04-22]{fade out}{\colOpt{=\marg{style}}}{style, default |white|}
Sets the four style options
\refKeyLe{/tcb/vig/north style},
\refKeyLe{/tcb/vig/south style},
\refKeyLe{/tcb/vig/east style}, and
\refKeyLe{/tcb/vig/west style}
such that the paths fade from inside to outside.
\\设置四个样式选项 \refKeyLe{/tcb/vig/north style}、\refKeyLe{/tcb/vig/south style}、\refKeyLe{/tcb/vig/east style} 和 \refKeyLe{/tcb/vig/west style},使路径从内部向外部逐渐消失。
\begin{dispExample*}{sbs,righthand width=3cm,center lower}
\begin{tikzpicture}
\fill [black!20] (-0.5,-0.5) rectangle (1.5,1.5);
\path [pattern=checkerboard,pattern color=black!30]
(-0.5,-0.5) rectangle (1.5,1.5);
\tcbvignette{fade out=blue}
\end{tikzpicture}
\end{dispExample*}
\end{vigTcbKey}


\begin{vigTcbKey}[][doc new=2016-04-22]{semi fade in}{\colOpt{=\marg{style}}}{style, default |white|}
Sets the four style options
\refKeyLe{/tcb/vig/north style},
\refKeyLe{/tcb/vig/south style},
\refKeyLe{/tcb/vig/east style}, and
\refKeyLe{/tcb/vig/west style}
such that the paths fade weak from outside to inside.
\\设置四个样式选项\refKeyLe{/tcb/vig/north style}、\refKeyLe{/tcb/vig/south style}、\refKeyLe{/tcb/vig/east style}和\refKeyLe{/tcb/vig/west style},使路径从外到内逐渐变淡。
\begin{dispExample*}{sbs,righthand width=3cm,center lower}
\begin{tikzpicture}
\fill [black!20] (-0.5,-0.5) rectangle (1.5,1.5);
\path [pattern=checkerboard,pattern color=black!30]
(-0.5,-0.5) rectangle (1.5,1.5);
\tcbvignette{semi fade in=blue}
\end{tikzpicture}
\end{dispExample*}
\end{vigTcbKey}


\begin{vigTcbKey}[][doc new=2016-04-22]{semi fade out}{\colOpt{=\marg{style}}}{style, default |white|}
Sets the four style options
\refKeyLe{/tcb/vig/north style},
\refKeyLe{/tcb/vig/south style},
\refKeyLe{/tcb/vig/east style}, and
\refKeyLe{/tcb/vig/west style}
such that the paths fade weak from inside to outside.
\\设置四个样式选项\refKeyLe{/tcb/vig/north style},\refKeyLe{/tcb/vig/south style},\refKeyLe{/tcb/vig/east style}和\refKeyLe{/tcb/vig/west style},使路径从内部向外部逐渐消失。
\begin{dispExample*}{sbs,righthand width=3cm,center lower}
\begin{tikzpicture}
\fill [black!20] (-0.5,-0.5) rectangle (1.5,1.5);
\path [pattern=checkerboard,pattern color=black!30]
(-0.5,-0.5) rectangle (1.5,1.5);
\tcbvignette{semi fade out=blue}
\end{tikzpicture}
\end{dispExample*}
\end{vigTcbKey}

% \clearpage

It is possible to assign different fadings for each side of the vignette,
if needed. Therefore, the fadings have to be applied individually with
the four style options
\refKeyLe{/tcb/vig/north style},
\refKeyLe{/tcb/vig/south style},
\refKeyLe{/tcb/vig/east style}, and
\refKeyLe{/tcb/vig/west style}.
\\如果需要,可以为每个角落分配不同的淡出效果。因此,必须使用四个样式选项分别应用淡出效果,包括 \refKeyLe{/tcb/vig/north style}、 \refKeyLe{/tcb/vig/south style}、 \refKeyLe{/tcb/vig/east style} 和 \refKeyLe{/tcb/vig/west style}。
\begin{dispExample*}{sbs,righthand width=3cm,center lower}
\begin{tikzpicture}
\fill [black!20] (-0.5,-0.5) rectangle (1.5,1.5);
\path [pattern=checkerboard,pattern color=black!30]
(-0.5,-0.5) rectangle (1.5,1.5);
\tcbvignette{
north style={blue,path fading=south},
east style ={blue,path fading=semi west},
south style={blue,path fading=semi north},
west style ={blue,path fading=east}
}
\end{tikzpicture}
\end{dispExample*}

\begin{dispExample*}{sbs,righthand width=3cm,center lower}
\begin{tikzpicture}
\fill [black!20] (-0.5,-0.5) rectangle (1.5,1.5);
\path [pattern=checkerboard,pattern color=black!30]
(-0.5,-0.5) rectangle (1.5,1.5);
\tcbvignette{
north style={blue,path fading=west},
east style ={blue,path fading=south},
south style={red,path fading=east},
west style ={red,path fading=north}
}
\end{tikzpicture}
\end{dispExample*}

