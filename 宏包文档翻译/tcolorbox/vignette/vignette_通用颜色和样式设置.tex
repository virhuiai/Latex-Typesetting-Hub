\subsection{Generic Color and Style Settings\\通用颜色和样式设置}\label{subsec:vignettestyle}

\begin{vigTcbKey}[][doc new=2016-04-22]{north style}{=\marg{style}}{no default, initially |red!50!white|}
Sets \tikzname\ \meta{style} options for the north \emph{vignette} part.

设置北部 \emph{vignette} 部分的 \tikzname\ \meta{style} 选项。
\begin{dispExample*}{sbs,righthand width=3cm,center lower}
\begin{tikzpicture}
  \tcbvignette{north style=blue}
\end{tikzpicture}
\end{dispExample*}
\end{vigTcbKey}

\begin{vigTcbKey}[][doc new=2016-04-22]{south style}{=\marg{style}}{no default, initially |red!50!black|}
Sets \tikzname\ \meta{style} options for the south \emph{vignette} part.

设置南部 \emph{vignette} 部分的 \tikzname\ \meta{style} 选项。
\begin{dispExample*}{sbs,righthand width=3cm,center lower}
\begin{tikzpicture}
  \tcbvignette{south style={draw=blue,fill=yellow}}
\end{tikzpicture}
\end{dispExample*}
\end{vigTcbKey}


\begin{vigTcbKey}[][doc new=2016-04-22]{east style}{=\marg{style}}{no default, initially |red!75!black|}
Sets \tikzname\ \meta{style} options for the east \emph{vignette} part.

设置东部 \emph{vignette} 部分的 \tikzname\ \meta{style} 选项。
\begin{dispExample*}{sbs,righthand width=3cm,center lower}
\begin{tikzpicture}
  \tcbvignette{east style={left color=yellow!75!black,
    right color=blue!75!black}}
\end{tikzpicture}
\end{dispExample*}
\end{vigTcbKey}

% \clearpage

\begin{vigTcbKey}[][doc new=2016-04-22]{west style}{=\marg{style}}{no default, initially |red!75!white|}
Sets \tikzname\ \meta{style} options for the west \emph{vignette} part.

设置西部 \emph{vignette} 部分的 \tikzname\ \meta{style} 选项。
\begin{dispExample*}{sbs,righthand width=3cm,center lower}
\begin{tikzpicture}
  \tcbvignette{west style={preaction={fill=black!20},
    pattern=checkerboard,
    pattern color=black!30}}
\end{tikzpicture}
\end{dispExample*}
\end{vigTcbKey}


\begin{vigTcbKey}[][doc new=2016-05-24]{scope}{=\marg{style}}{no default, initially empty}
The four \emph{vignette} parts are drawn inside a \tikzname\ |scope|
environment which takes the given \meta{style} as option.

四个“小块”部分是在一个\tikzname 的|scope|环境内绘制的,该环境将给定的\meta{style}作为选项。
\begin{dispExample*}{sbs,righthand width=3cm,center lower}
\begin{tikzpicture}
  \tcbvignette{scope={transparency group,opacity=0.25}}
\end{tikzpicture}
\end{dispExample*}
\end{vigTcbKey}



\begin{vigTcbKey}[][doc new=2016-04-22]{raised color}{=\meta{color}}{no default}
Creates a raised frame impression by setting the four style options
\refKeyLe{/tcb/vig/north style},
\refKeyLe{/tcb/vig/south style},
\refKeyLe{/tcb/vig/east style}, and
\refKeyLe{/tcb/vig/west style}
to darkened and lightened variations of the given \meta{color}.

通过设置四个样式选项\refKeyLe{/tcb/vig/north style}、\refKeyLe{/tcb/vig/south style}、\refKeyLe{/tcb/vig/east style}和\refKeyLe{/tcb/vig/west style},使用给定的\meta{color}的加深和减淡变化创建一个凸起的框架印象。
\begin{dispExample*}{sbs,righthand width=3cm,center lower}
\begin{tikzpicture}
  \tcbvignette{raised color=blue}
\end{tikzpicture}
\end{dispExample*}
\end{vigTcbKey}


\begin{vigTcbKey}[][doc new=2016-04-22]{lowered color}{=\meta{color}}{no default}
Creates a lowered frame impression by setting the four style options
\refKeyLe{/tcb/vig/north style},
\refKeyLe{/tcb/vig/south style},
\refKeyLe{/tcb/vig/east style}, and
\refKeyLe{/tcb/vig/west style}
to darkened and lightened variations of the given \meta{color}.

通过设置四个样式选项\refKeyLe{/tcb/vig/north style}、\refKeyLe{/tcb/vig/south style}、\refKeyLe{/tcb/vig/east style}和\refKeyLe{/tcb/vig/west style}为给定\meta{color}的变暗和变亮的版本,创建一个降低框架印象。
\begin{dispExample*}{sbs,righthand width=3cm,center lower}
\begin{tikzpicture}
  \tcbvignette{lowered color=green!75!black}
\end{tikzpicture}
\end{dispExample*}
\end{vigTcbKey}


\begin{vigTcbKey}[][doc new=2016-04-22]{color from}{=\meta{inner} |to| \meta{outer}}{no default}
  Sets the four style options
  \refKeyLe{/tcb/vig/north style},
  \refKeyLe{/tcb/vig/south style},
  \refKeyLe{/tcb/vig/east style}, and
  \refKeyLe{/tcb/vig/west style}
  such that the color shades from the
  \meta{inner} color to the \meta{outer} color.

设置四个样式选项\refKeyLe{/tcb/vig/north style}、\refKeyLe{/tcb/vig/south style}、\refKeyLe{/tcb/vig/east style}和\refKeyLe{/tcb/vig/west style},使颜色从\meta{内部}颜色渐变到\meta{外部}颜色。
\begin{dispExample*}{sbs,righthand width=3cm,center lower}
\begin{tikzpicture}
  \tcbvignette{color from=red to blue!50}
\end{tikzpicture}
\end{dispExample*}
\end{vigTcbKey}



\begin{vigTcbKey}[][doc new=2016-04-22]{base color}{=\meta{color}}{no default}
Sets the base color for \refKeyLe{/tcb/vig/raised color},
\refKeyLe{/tcb/vig/lowered color}, \refKeyLe{/tcb/finish fading vignette}.
Typically, this value has not to be set directly.

设置\refKeyLe{/tcb/vig/raised color}、\refKeyLe{/tcb/vig/lowered color}、\refKeyLe{/tcb/finish fading vignette}的基础颜色。通常情况下,不需要直接设置这个值。
\end{vigTcbKey}


% \clearpage
\begin{vigTcbKey}[][doc new=2016-04-22]{draw method}{=\docValue{direct}\textbar\docValue{clipped}}{no default, initially |direct|}
Especially, if shadings or fadings are used, the drawn \emph{vignette}
graphs are displayed sometimes not as perfect as expected. Glitches and
imperfections are very dependent on the previewer software.
The \refKeyLe{/tcb/vig/draw method} intends to give a choice of alternative
drawing methods.

特别是,如果使用了阴影或渐变,绘制的“vignette”图形有时显示不如预期的完美。故障和瑕疵非常依赖于预览软件。 \refKeyLe {/ tcb / vig / draw method} 的目的是提供选择备选绘图方法。
\begin{itemize}
\item\docValue{direct}: The \emph{vignette} parts are drawn/filled
by using a single \tikzname\ graph. This is the preferred (and default)
method for solid color graphs.
\\通过使用单个 \tikzname\ 图形来绘制/填充 \emph{vignette} 部分。这是实心颜色图形的首选(也是默认)方法。
\item\docValue{clipped}: The \emph{vignette} parts are drawn somewhat
oversized and are clipped to the intended region.
In combination with shadings and fadings this seems to give a
better/different optical result (depends on the previewer).
\\部分会被绘制得略大一些,然后被裁剪到预定区域。与渐变和淡化效果相结合,这似乎会产生更好/不同的视觉效果(取决于预览器)。
\end{itemize}
\begin{dispExample*}{sbs,righthand width=3cm,center lower}
\begin{tikzpicture}
  \tcbvignette{color from=red to yellow}
\end{tikzpicture}
\end{dispExample*}
\begin{dispExample*}{sbs,righthand width=3cm,center lower}
\begin{tikzpicture}
  \tcbvignette{color from=red to yellow,draw method=clipped}
\end{tikzpicture}
\end{dispExample*}

\begin{marker}
This option is a stopgap and may be changed or preferably removed in
future.

这个选项只是一个权宜之计,未来可能会被更改或最好被移除。
\end{marker}
\end{vigTcbKey}
