\subsection{Option Keys of the Library\\库的选项键}\label{subsec:fit}


The font size for the content of a box with fixed width and fixed height
can be adjusted automatically. This is called the \emph{fitbox capture mode}.
Note that the fit control algorithm
constructs a series of versions for the box and selects the \enquote{best}.
Therefore, the compilation time is quite longer than for a normal box.
The algorithm will fail, if a different selected font size does not change the
overall size of the box content.
The \refComLe{tcboxfit} macro uses this algorithm by default.

具有固定宽度和固定高度的盒子内容的字体大小可以自动调整。这称为\emph{自适应盒子捕获模式}。请注意,适合控制算法构建盒子的一系列版本并选择\enquote{最佳版本}。因此,编译时间比普通盒子要长得多。如果不同的所选字体大小不会改变盒子内容的整体大小,则算法将失败。默认情况下,\refComLe{tcboxfit} 宏使用此算法。
\medskip
\begin{marker}
The fit control keys are only applicable to unbreakable boxes without
a lower part.
The box content should not change counters.

适合控制键仅适用于无法中断且没有下部分的盒子。盒子内容不应更改计数器。
\end{marker}
% \medskip


% \enlargethispage*{5mm}
\begin{docTcbKey}{fit}{}{style, initially unset}
Sets the \refKeyLe{/tcb/capture} mode to |fitbox|, i.\,e.\ enables the
font size adjustment algorithm. Thereby, a \refEnvLe{tcolorbox} acts
like \refComLe{tcboxfit} where the given \meta{box content} is fitted to
the width and height of the box. Therefore, the box has to have a fixed height.
If no fixed height is given, a square box is constructed.
The font dimension \refComLe{tcbfitdim} can also be used to adjust
the margins of the box since a box with a tiny font may not need large
margins. The number of constructed boxes is saved to the macro \docAuxCommand{tcbfitsteps}
for analysis.

将 \refKeyLe{/tcb/capture} 模式设置为 |fitbox|,即启用字体大小调整算法。因此,\refEnvLe{tcolorbox} 的行为类似于 \refComLe{tcboxfit},其中给定的 \meta{box content} 适应于盒子的宽度和高度。因此,盒子必须有固定高度。如果没有给定固定高度,则构造一个正方形盒子。字体尺寸 \refComLe{tcbfitdim} 也可用于调整盒子的边距,因为字体很小的盒子可能不需要大的边距。构造的盒子数保存到宏 \docAuxCommand{tcbfitsteps} 中以供分析。
\begin{dispExample}
% \usepackage{lipsum}
% \tcbuselibrary{skins}
\newtcolorbox{fitting}[2][]{fit,height=#2,boxsep=1pt,valign=center,opacityupper=0.5,
  top=0.4\tcbfitdim,bottom=0.4\tcbfitdim,left=0.75\tcbfitdim,right=0.75\tcbfitdim,
  enhanced,watermark text={\tcbfitsteps},colframe=blue!75!black,colback=white,#1}

\begin{fitting}{4cm}
\lipsum[1]
\end{fitting}

\begin{fitting}{2cm}
\lipsum[2]
\end{fitting}

\begin{fitting}{1cm}
\lipsum[3]
\end{fitting}
\end{dispExample}
\end{docTcbKey}


 
\begin{docTcbKey}{fit to}{=\meta{width} and \meta{height}}{style, initially unset}
Shortcut for using \refKeyLe{/tcb/fit} and setting the \meta{width} and \meta{height} values
separately.

该选项是使用\refKeyLe{/tcb/fit}并分别设置\meta{宽度}和\meta{高度}值的快捷方式。
\begin{dispExample*}{sbs,lefthand ratio=0.6}
\tcbset{colback=red!5!white,colframe=red!75!black}

% This box content is fitted to the given
% dimensions.
\begin{tcolorbox}[fit to=3cm and 2cm]
此盒子的内容将适应给定的尺寸。
\end{tcolorbox}
\end{dispExample*}
\end{docTcbKey}


\begin{docTcbKey}{fit to height}{=\meta{height}}{style, initially unset}
Shortcut for using \refKeyLe{/tcb/fit} and setting the \meta{height} value separately.

该选项是使用\refKeyLe{/tcb/fit}并单独设置\meta{高度}值的快捷方式。
\begin{dispExample*}{sbs,lefthand ratio=0.6}
\tcbset{colback=red!5!white,colframe=red!75!black}

% This box content is fitted to the given
% height.
\begin{tcolorbox}[fit to height=2cm]
此盒子的内容将适应给定的高度。
\end{tcolorbox}
\end{dispExample*}
\end{docTcbKey}

\begin{docTcbKey}{fit basedim}{=\meta{length}}{no default, initially |10pt|}
Sets the starting font dimension for the font size adjustment algorithm
to \meta{length}. The algorithm never enlarges this dimension.
Therefore, the final \refComLe{tcbfitdim} is identical to or small than
\meta{length}.

将字体大小调整算法的起始字体尺寸设置为\meta{长度}。算法永远不会扩大这个尺寸。因此,最终的\refComLe{tcbfitdim}等于或小于\meta{长度}。
\begin{dispExample*}{sbs,lefthand ratio=0.6}
\tcbset{colback=red!5!white,colframe=red!75!black}

% Too few words for the box.
\begin{tcolorbox}[fit to=4cm and 2cm]
此盒子的内容过少。
\end{tcolorbox}

\begin{tcolorbox}[fit to=4cm and 2cm,
  fit basedim=50pt]
此盒子的内容足够多多多多多多多多多。
\end{tcolorbox}

\begin{tcolorbox}[fit to=4cm and 2cm,
fit basedim=50pt]
此盒子的内容足够多多多多多多多多多多多多多多多多多。
\end{tcolorbox}
\end{dispExample*}
\end{docTcbKey}


\begin{docTcbKey}{fit skip}{=\meta{real value}}{no default, initially |1.2|}
Sets the skip value of the selected font to \meta{real value} times \refComLe{tcbfitdim}.

将所选字体的间距设置为\meta{实数值}乘以\refComLe{tcbfitdim}。
\begin{dispExample*}{sbs,lefthand ratio=0.6}
% \usepackage{lipsum}
\tcbset{colback=red!5!white,
  colframe=red!75!black,left=1mm,
  right=1mm,boxsep=0mm}

\begin{tcolorbox}[fit to=5cm and 4cm,
  fit skip=1.0  ]
  \lipsum[1]
\end{tcolorbox}
\end{dispExample*}
\end{docTcbKey}

 
\begin{docTcbKey}{fit fontsize macros}{}{style, initially unset}
Redefines the standard \LaTeX\ font size macros
|\tiny|, |\scriptsize|, |\footnotesize|, |\small|, |\normalsize|,
|\large|, |\Large|, |\LARGE|, |\huge|, and |\Huge|,
to set font sizes relative to
the current \refComLe{tcbfitdim}. Note that the display skip values for
mathematical formulas are respected by the redefined macros.
Also see \refComLe{tcbfontsize}.

重新定义了标准的\LaTeX\ 字体大小宏
|\tiny|,|\scriptsize|,|\footnotesize|,|\small|,|\normalsize|,
|\large|,|\Large|,|\LARGE|,|\huge| 和 |\Huge|,使其相对于当前的\refComLe{tcbfitdim}设置字体大小。请注意,数学公式的显示间距值也会受到重新定义的宏的影响。此外,还可以参考\refComLe{tcbfontsize}。
\begin{dispExample*}{sbs,lefthand ratio=0.6}
% \usepackage{lipsum}
\tcbset{colback=red!5!white,
  colframe=red!75!black,left=1mm,
  right=1mm,boxsep=0mm}

\begin{tcolorbox}[fit to height=4cm]
  {\Large\bfseries This text is
             not adapted:\par}
  \lipsum[2]
\end{tcolorbox}

\begin{tcolorbox}[fit to height=4cm,
  fit fontsize macros ]
  {\Large\bfseries This text is adapted:\par}
  \lipsum[2]
\end{tcolorbox}
\end{dispExample*}

\begin{dispExample*}{sbs,lefthand ratio=0.6}
\tcbset{colback=red!5!white,
  colframe=red!75!black,left=1mm,
  right=1mm,boxsep=0mm}

\let\realHuge=\Huge

\begin{tcolorbox}[fit basedim=7pt,
  fontupper=\normalsize,
  fit fontsize macros]
The relative relative font size macros
are also usable without the
\textit{fit} algorithm.\par
{\Huge Adapted Huge} ---
{\realHuge Original Huge}
\end{tcolorbox}

\end{dispExample*}


\begin{dispExample*}{sbs,lefthand ratio=0.6}
\tcbset{size=fbox,colback=red!5!white,
  colframe=red!75!black}

\tcboxfit[height=5cm,
  fit fontsize macros,
  fonttitle=\normalsize\bfseries,
  title=Adapted title]
{\lipsum[2]}

\end{dispExample*}
\end{docTcbKey}

 
\begin{docTcbKey}{fit height plus}{=\meta{dimension}}{no default, initially |0pt|}
The box is allowed to enlarge the fixed height up to the given \meta{dimension},
before a font size fit is applied. An optional \refKeyLe{/tcb/fit width plus}
is tried after the height adaption.

在应用字体大小调整之前,该盒子允许将固定高度增加至给定的 \meta{dimension}。可选的 \refKeyLe{/tcb/fit width plus} 在高度适应后进行尝试。
\begin{dispExample}
% \usepackage{lipsum}
\tcbset{colback=red!5!white,colframe=red!75!black,left=1mm,top=1mm,bottom=1mm,
  right=1mm,boxsep=0mm,width=3cm,height=3cm,nobeforeafter}

\begin{tcolorbox}[fit]
这是一个tcolorbox。
\end{tcolorbox}
\begin{tcolorbox}[fit,fit height plus=1cm]
这是一个tcolorbox。
\end{tcolorbox}
\begin{tcolorbox}[fit]
\lipsum[2]
\end{tcolorbox}
\begin{tcolorbox}[fit,fit height plus=1cm]
\lipsum[2]
\end{tcolorbox}
\end{dispExample}
\end{docTcbKey}


\begin{docTcbKey}{fit width plus}{=\meta{dimension}}{no default, initially |0pt|}
The box is allowed to enlarge the fixed width up to the given \meta{dimension},
before a font size fit is applied. An optional \refKeyLe{/tcb/fit height plus}
is tried before the width adaption.

在应用字体大小调整之前,该盒子允许将固定宽度增加至给定的 \meta{dimension}。可选的 \refKeyLe{/tcb/fit height plus} 在宽度适应之前进行尝试。
  % \enlargethispage*{1cm}
\begin{dispExample}
% \usepackage{lipsum}
\tcbset{colback=red!5!white,colframe=red!75!black,left=1mm,top=1mm,bottom=1mm,
  right=1mm,boxsep=0mm,width=3cm,height=3cm,nobeforeafter}

\begin{tcolorbox}[fit]
这是一个tcolorbox。
\end{tcolorbox}
\begin{tcolorbox}[fit,fit width plus=1cm]
这是一个tcolorbox。
\end{tcolorbox}
\begin{tcolorbox}[fit]
\lipsum[2]
\end{tcolorbox}
\begin{tcolorbox}[fit,fit width plus=1cm]
\lipsum[2]
\end{tcolorbox}
\end{dispExample}
\end{docTcbKey}


\begin{marker}
Typically but not necessarily, the optional title of a |tcolorbox| is not part of the fit operation.
If a \refKeyLe{/tcb/fit width plus} is applied, the title is also adapted to
the new width. If counters are increased inside the title text, they may be
increased more than one time.
To avoid this, you are encouraged to use \refKeyLe{/tcb/phantom} or \refKeyLe{/tcb/step and label}
to set counters or use automatic numbering, see Subsection \ref{sec:numberedboxes}
from page \pageref{sec:numberedboxes}.

通常,但不一定,|tcolorbox| 的可选标题不是适合操作的一部分。
如果应用了 \refKeyLe{/tcb/fit width plus},则标题也会适应新宽度。
如果计数器在标题文本内增加,则它们可能会增加多次。
为避免这种情况,建议使用 \refKeyLe{/tcb/phantom} 或 \refKeyLe{/tcb/step and label} 来设置计数器或使用自动编号,请参见第 \pageref{sec:numberedboxes} 页的子节 \ref{sec:numberedboxes}。
\end{marker}


\begin{docTcbKey}{fit width from}{=\meta{min} \texttt{to} \meta{max}}{style, no default}
Sets the box width to \meta{min} and allows the width to grow up to \meta{max}.

将盒子宽度设置为 \meta{min} 并允许宽度增长到 \meta{max}。
\begin{dispExample}
% \usepackage{lipsum}
\tcbset{colback=red!5!white,colframe=red!75!black,left=1mm,top=1mm,bottom=1mm,
  right=1mm,boxsep=0mm,height=4cm}

\begin{tcolorbox}[fit,width=\linewidth/2]
\lipsum[2]
\end{tcolorbox}\par
\begin{tcolorbox}[fit width from=\linewidth/2 to \linewidth]
\lipsum[2]
\end{tcolorbox}\par
\end{dispExample}
\end{docTcbKey}

 
\begin{docTcbKey}{fit height from}{=\meta{min} \texttt{to} \meta{max}}{style, no default}
Sets the box height to \meta{min} and allows the height to grow up to \meta{max}.

将盒子高度设置为 \meta{min} 并允许高度增长到 \meta{max}。
\begin{dispExample}
% \usepackage{lipsum}
\newtcolorbox{mybox}{colback=red!5!white,colframe=red!75!black,left=1mm,top=1mm,
  bottom=1mm,right=1mm,boxsep=0mm,width=4cm,nobeforeafter,
  fit height from=1cm to 8cm}

\begin{mybox}
这是一个tcolorbox。
\end{mybox}
\begin{mybox}
这是一个tcolorbox。 这是一个tcolorbox。 这是一个tcolorbox。
\end{mybox}
\begin{mybox}
\lipsum[2]
\end{mybox}
\end{dispExample}
\end{docTcbKey}


\begin{docTcbKey}{fit algorithm}{=\meta{name}}{no default, initially |fontsize|}
Sets the algorithm for the fitting process \emph{after} optionally width and height
are adapted. In the following, adapting the font size means adapting
\refComLe{tcbfitdim}.
  Feasible values for \meta{name} are:

在可选的宽度和高度适应之后,设置适合过程的算法。在下面的说明中,适应字体大小意味着适应 \refComLe{tcbfitdim}。
可行的 \meta{name} 值为:
  \begin{itemize}
  \item\docValue{fontsize} (initial):
    The algorithm is a bisection method that adapts the font size until
    certain stop conditions are fulfilled. This is the most time-consuming
    method but it is robust and gives pleasant results.

    该算法是一个二分法,它调整字体大小,直到满足某些停止条件。这是最耗时的方法,但它是强大而令人愉悦的结果。
    \begin{marker}
    The used font has to be freely scalable for this method!
    Other content than text is not scaled down.
    The aspect ratio is fully garanteed.

    对于这种方法使用的字体必须是自由缩放的!其他内容不会缩小。纵横比完全保证。
    \end{marker}
  \item\docValue{fontsize*}:\tcbdocmarginnote{\tcbdocnew{2014-10-29}}
    First, the \docValue{fontsize} algorithm is applied. If the font was scaled down
    and the resulting height is too small, the box is squeezed to fit the area.

    首先应用 \docValue{fontsize} 算法。如果字体被缩小并且结果高度太小,则会压缩盒子以适应区域。
    \begin{marker}
    The used font has to be freely scalable for this method!
    Other content than text may be slightly rescaled.
    The aspect ratio cannot be fully garanteed.

使用的字体必须可以自由缩放!
与文本不同的其他内容可能会被轻微缩放。
纵横比不能完全保证。
    \end{marker}
  \item\docValue{areasize}:
    The algorithm calculates the area size for the text without scaling the font.
    The text box is shaped for the needed aspect ratio in one or two
    steps. Finally, it is scaled down with a standard |\resizebox| macro.

该算法在不缩放字体的情况下计算文本的面积大小。
文本框根据所需的纵横比在一步或两步中形成,最后使用标准的 |\resizebox| 宏进行缩小。
    \begin{marker}
    The used font has not to be scalable. Every box content is scaled down.
    The aspect ratio cannot be fully garanteed.

使用的字体不必可缩放。每个盒子内容都会被缩小。
纵横比不能完全保证。
    \end{marker}
  \item\docValue{areasize*}:\tcbdocmarginnote{\tcbdocnew{2014-10-29}}
    The \docValue{areasize} algorithm is applied, but if the content was scaled
    down and the resulting height is too small, the box is squeezed to fit the area.

    该算法应用 \docValue{areasize},但如果内容被缩小,结果高度太小,则会压缩盒子以适应该区域。
    \begin{marker}
    The used font has not to be scalable. Every box content is scaled down.
    The aspect ratio cannot be fully garanteed.

使用的字体不必可缩放。每个盒子内容都会被缩小。
纵横比不能完全保证。
    \end{marker}
  \item\docValue{hybrid}:
    First, this algorithm estimates the needed font size in one or two steps.
    Then an \docValue{areasize} fitting as above is a applied.

    该算法首先通过一步或两步估计所需的字体大小,然后应用如上的 \docValue{areasize} 。
    \begin{marker}
    The used font has to be freely scalable for this method!
    Other content than text may be slightly rescaled.
    The aspect ratio cannot be fully garanteed.

使用的字体必须可以自由缩放!
与文本不同的其他内容可能会被轻微缩放。
纵横比不能完全保证。
    \end{marker}
  \item\docValue{hybrid*}:\tcbdocmarginnote{\tcbdocnew{2014-10-29}}
    First, this algorithm estimates the needed font size in one or two steps.
    Then an \docValue{areasize*} fitting as above is a applied.

    该算法首先通过一步或两步估计所需的字体大小,然后应用如上的 \docValue{areasize*} 。
    \begin{marker}
The used font has to be freely scalable for this method!
Other content than text may be slightly rescaled.
The aspect ratio cannot be fully garanteed.

使用的字体必须可以自由缩放!
与文本不同的其他内容可能会被轻微缩放。
纵横比不能完全保证。
    \end{marker}
  \item\docValue{squeeze}:
The text box is brutally scaled down to fit.

文本框被强制缩小以适应。
    \begin{marker}
    The aspect ratio is very likely to be horrible.
    You should not use this method for final documents.

纵横比很可能很糟糕。
您不应将此方法用于最终文档。
    \end{marker}
\end{itemize}


\end{docTcbKey}
\begin{dispExample}
% \usepackage{lipsum}
\newtcboxfit{mybox}[1]{colback=red!5!white,colframe=red!75!black,left=1mm,top=1mm,
  bottom=1mm,right=1mm,boxsep=0mm,width=3.5cm,height=7cm,nobeforeafter,
  before upper=\textcolor{blue}{\rule{5mm}{5mm}}\ ,
  enhanced,watermark text={\tcbfitsteps},
  fonttitle=\bfseries,adjusted title={#1},fit algorithm=#1}

\mybox{fontsize}{\lipsum[2]}\hfill
\mybox{hybrid}{\lipsum[2]}\hfill
\mybox{areasize}{\lipsum[2]}\hfill
\mybox{squeeze}{\lipsum[2]}

Quality \dotfill versus \dotfill Speed
\end{dispExample}


\begin{dispExample}
% \usepackage{lipsum}
\newtcboxfit{mybox}[2]{colback=red!5!white,colframe=red!75!black,left=1mm,top=1mm,
  size=tight,width=7.2cm,height=5cm,nobeforeafter,
  before upper=\textcolor{blue}{\rule{5mm}{5mm}}\ ,
  enhanced,fonttitle=\bfseries,adjusted title={#2},fit algorithm=#1}

\mybox{hybrid}{hybrid (possible gap at end)}{\lipsum[1]}\hfill
\mybox{hybrid*}{hybrid* (no gap but possibly squeezed)}{\lipsum[1]}
\end{dispExample}


 
\begin{marker}
The following options set control parameters for the fit algorithm.
Mainly, they apply to the |fontsize| variant, see \refKeyLe{/tcb/fit algorithm}.
The options should be seen as experimental and are likely to change in future versions,
if necessary.

以下选项设置适合算法的控制参数。主要适用于 |fontsize| 变量,参见 \refKeyLe{/tcb/fit algorithm}。这些选项应视为实验性的,并且可能在未来的版本中更改,如果必要的话。
\end{marker}

\begin{docTcbKey}{fit maxstep}{=\meta{number}}{no default, initially |20|}
Sets the maximal step size for the font size adjustment algorithm.
In normal situations, the algorithm stops before reaching the intial value of 20 steps.
If the box content does not shrink, this value prevents an endless loop.

设置字体大小调整算法的最大步长。在正常情况下,算法在达到初始值 20 步之前就停止。如果框内容不缩小,则该值可以防止无限循环。
\end{docTcbKey}
