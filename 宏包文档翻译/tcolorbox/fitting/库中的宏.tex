
\subsection{Macros of the Library\\库中的宏}

\begin{docCommand}{tcboxfit}{\oarg{options}\marg{box content}}
Creates a colored box where the given \meta{box content} is fitted to
the width and height of the box. A |tcboxfit| has to have a fixed height.
If no fixed height is given, a square box is constructed.
In principle, most \meta{options} for a \refEnvLe{tcolorbox}
can be used for |\tcboxfit| with some restrictions. A |\tcboxfit| cannot have
a lower part and cannot be broken.

创建一个带有颜色的框,其中给定的 \meta{盒子内容} 被适应为框的宽度和高度。一个 |tcboxfit| 必须有一个固定的高度。如果没有给定固定的高度,则构造一个正方形框。原则上,大多数 \refEnvLe{tcolorbox} 的 \meta{选项} 都可以用于带有一些限制的 |\tcboxfit|。|\tcboxfit| 不能有下部并且不能被打破。
% \enlargethispage*{1cm}
% \begin{dispExample}
% % \usepackage{lipsum} \tcbuselibrary{raster}
% \tcbset{colframe=blue!50!black,colback=red!10!white,
%     boxsep=0pt,top=1mm,bottom=1mm,left=1mm,right=1mm,
%     fit algorithm=hybrid*,raster equal skip=1mm}
% \begin{tcbraster}[raster columns=3,raster valign=bottom]
%   \tcboxfit[height=8cm]{\lipsum[1]}
%   \tcboxfit[height=4cm]{\lipsum[1]}
%   \tcboxfit[height=2cm]{\lipsum[1]}
% \end{tcbraster}
% \begin{tcbraster}[colback=green!10!white,boxsep=1mm]
%   \tcboxfit[height=4cm]{\lipsum[2]}
%   \tcboxfit[height=4cm,title=With a title]{\lipsum[2]}
% \end{tcbraster}
% \end{dispExample}
\begin{dispExample}
%\tcbuselibrary{raster}
\tcbset{colframe=blue!50!black,colback=red!10!white,
boxsep=0pt,top=1mm,bottom=1mm,left=1mm,right=1mm,
fit algorithm=hybrid*,raster equal skip=1mm}
\begin{tcbraster}[raster columns=3,raster valign=bottom]
\tcboxfit[height=8em]{绿蚁新醅酒,\\红泥小火炉。\\
晚来天欲雪,\\能饮一杯无?}
\tcboxfit[height=4em]{绿蚁新醅酒,\\红泥小火炉。\\
晚来天欲雪,\\能饮一杯无?}
\tcboxfit[height=1cm]{绿蚁新醅酒,\\红泥小火炉。\\
晚来天欲雪,\\能饮一杯无?}
\end{tcbraster}
\begin{tcbraster}[colback=green!10!white,boxsep=1mm]
\tcboxfit[height=8em]{绿蚁新醅酒,\\红泥小火炉。\\
晚来天欲雪,\\能饮一杯无?}
\tcboxfit[height=8em,title=《问刘十九》]{绿蚁新醅酒,\\红泥小火炉。\\
晚来天欲雪,\\能饮一杯无?}
\end{tcbraster}
\end{dispExample}
\end{docCommand}
% 赏析:“绿蚁”二字绘酒色摹酒状,酒色流香,“红泥”二字表现出冬日温暖的情境,全句显得朴素温馨,表现了温暖如春的诗情。
% 刘二十八就是刘禹锡。刘十九乃其堂兄刘禹铜,系洛阳一富商,与白居易常有应酬。
% 绿蚁:指浮在新酿的没有过滤的米酒上的绿色泡沫。醅(pēi):酿造。
% 绿蚁新醅酒:酒是新酿的酒。新酿酒未滤清时,酒面浮起酒渣,色微绿,细如蚁,称为“绿蚁”。
% 雪:下雪,这里作动词用。
% 无:表示疑问的语气词,相当于“么”或“吗”。

%这里的“绿蚁”指的是在绿色的酒液表面上爬行的细小气泡,就像是绿蚁在爬行一样,这种景象常常出现在新酿制的酒液上面。而“新醅”则指的是刚刚酿制好的酒,还没有经过长时间的陈放和贮存。这种新酒味道鲜美,清香扑鼻,色泽晶莹剔透,让人非常喜欢。


%  
\begin{marker}
See \Vref{subsec:xparse_tcboxfit} for more
elaborate methods to create new commands.

有关创建新命令的更详细方法,请参见 \Vref{subsec:xparse_tcboxfit}。
\end{marker}

% \enlargethispage*{2cm}

\begin{docCommand}{newtcboxfit}{\oarg{init options}\brackets{\texttt{\textbackslash}\rmfamily\meta{name}}\oarg{number}\oarg{default}\marg{options}}
Creates a new macro \texttt{\textbackslash}\meta{name} based on \refComLe{tcboxfit}.
Basically, |\newtcboxfit| operates like |\newcommand|.
The new macro \texttt{\textbackslash}\meta{name} optionally takes \meta{number}$+1$ arguments, where
\meta{default} is the default value for the optional first argument.
The \meta{options} are given to the underlying |tcboxfit|.
The \meta{init options} allow setting up automatic numbering,
see Section \ref{sec:initkeys} from page \pageref{sec:initkeys}.

基于 \refComLe{tcboxfit} 创建一个新的宏 \texttt{\textbackslash}\meta{name}。基本上,|\newtcboxfit| 的操作类似于 |\newcommand|。新宏 \texttt{\textbackslash}\meta{name} 可以选择地接受 \meta{number}$+1$ 个参数,其中 \meta{default} 是可选第一个参数的默认值。\meta{options} 给底层的 |tcboxfit|。\meta{init options} 允许设置自动编号,请参见第 \pageref{sec:initkeys} 页的第 \ref{sec:initkeys} 节。
\begin{dispExample*}{sbs,lefthand ratio=0.6}
\newtcboxfit{\mybox}{colback=red!5!white,
  colframe=red!75!black,width=4cm,
  height=1.5cm,halign=center}

\mybox{This is my own box.}\par
\mybox{This is my own box with more text
       to be written.}
\end{dispExample*}

\begin{dispExample*}{sbs,lefthand ratio=0.6}
% \usepackage{lipsum}
\newtcboxfit{\mybox}[2]{colback=red!5!white,
  colframe=red!75!black,fonttitle=\bfseries,
  boxsep=1mm,left=0mm,right=0mm,top=0mm,
  bottom=0mm,halign=center,valign=center,
  nobeforeafter,width=#1,height=#2}

\mybox{2.5cm}{1cm}{First box}%
\mybox{2.5cm}{1cm}{Second box with more text}\\
\mybox{5cm}{2cm}{Third box with text}\\
\mybox{5cm}{3cm}{\lipsum[1]}
\end{dispExample*}

\begin{dispExample*}{sbs,lefthand ratio=0.6}
% \usepackage{lipsum}
\newtcboxfit{\mybox}[2][]{colback=red!5!white,
  colframe=red!75!black,
  width=#2,height=#2/3*2,#1}

\mybox[colback=yellow]{5cm}%
  {\lipsum[2]}
\end{dispExample*}
\end{docCommand}


\begin{docCommand}{renewtcboxfit}{\oarg{init options}\brackets{\texttt{\textbackslash}\rmfamily\meta{name}}\oarg{number}\oarg{default}\marg{options}}
Operates like \refComLe{newtcboxfit}, but based on |\renewcommand| instead of |\newcommand|.
An existing macro is redefined.

与 \refComLe{newtcboxfit} 类似,但基于 |\renewcommand| 而不是 |\newcommand|。重定义现有的宏。
\end{docCommand}


%  

\begin{docCommands}[
    doc name        = tcbfitdim,
    doc description = {read-only \LaTeX\ length},
    doc updated     = 2020-04-24,
  ]{}
This is a \LaTeX\ length adapted automatically by most variants of
\refKeyLe{/tcb/fit algorithm}. Therefore, it never is to be
changed by the user, but may be applied read-only.
The \refComLe{tcbfitdim} corresponds to the font size and may also
be used to calculate box margins or other distances in dependency.
The initial and maximum value for \refComLe{tcbfitdim} is set by
\refKeyLe{/tcb/fit basedim}.

这是一个 \LaTeX\ 长度,由大多数 \refKeyLe{/tcb/fit algorithm} 变体自动调整。因此,用户不能更改它,但可以只读访问。 \refComLe{tcbfitdim} 对应于字体大小,也可以用于计算框边距或其他依赖距离。 \refComLe{tcbfitdim} 的初始和最大值由 \refKeyLe{/tcb/fit basedim} 设置。
\end{docCommands}


\begin{docCommand}{tcbfontsize}{\marg{factor}}
Selects a font size inside a tcolorbox which is scaled with the given
\meta{factor} relative to \refComLe{tcbfitdim}.
Also see \refKeyLe{/tcb/fit fontsize macros}

在一个 tcolorbox 中选择一个字体大小,其相对于 \refComLe{tcbfitdim} 缩放给定的 \meta{缩放因子}。也参见 \refKeyLe{/tcb/fit fontsize macros}。
\begin{dispExample*}{sbs,lefthand ratio=0.6}
\tcbset{colback=red!5!white,size=small,
  colframe=red!75!black}
\begin{tcolorbox}[fit basedim=10pt]
  {\tcbfontsize{0.25} Very tiny,}\\
  {\tcbfontsize{0.5} Small,}\\
  {\tcbfontsize{1} Normal,}\\
  {\tcbfontsize{2} Large,}\\
  {\tcbfontsize{4} Huge.}
\end{tcolorbox}
\end{dispExample*}
\begin{dispExample*}{sbs,lefthand ratio=0.6}
\tcbset{colback=red!5!white,size=small,
  colframe=red!75!black}
\begin{tcolorbox}[fit basedim=10pt,
    fit to height=2cm]
  {\tcbfontsize{0.25} Very tiny,}\\
  {\tcbfontsize{0.5} Small,}\\
  {\tcbfontsize{1} Normal,}\\
  {\tcbfontsize{2} Large,}\\
  {\tcbfontsize{4} Huge.}
\end{tcolorbox}
\end{dispExample*}
\end{docCommand}
