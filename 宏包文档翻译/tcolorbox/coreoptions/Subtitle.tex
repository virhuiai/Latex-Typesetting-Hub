\subsection{Subtitle\\副标题}

Inside the box content, one or more subtitles can be added.
In general, a subtitle is a further \refEnvLe{tcolorbox} which inherits some color and geometry options from the enclosing box. 
It may be customized just like any other \refEnvLe{tcolorbox}.

% 在盒子内,可以添加一个或多个副标题。%
% 一般来说,副标题是一个从封闭盒子继承了一些颜色和几何选项的|tcolorbox|。%
% 它可以像是一个 |tcolorbox| 一样进行设置。  

在盒子内容内部,可以添加一个或多个副标题。 通常,副标题是一个进一步的\refEnvLe{tcolorbox},它从封闭盒子中继承了一些颜色和几何选项。 它可以像任何其他的\refEnvLe{tcolorbox}一样进行自定义。

\begin{docCommand}[doc new=2014-10-10]{tcbsubtitle}{\oarg{options}\marg{text}}
Used inside a \refEnvLe{tcolorbox} to add a subtitle box with the given \meta{text}.
which is formatted by several inherited properties of the enclosing box
by further settings from \refKey{/tcb/subtitle style}, and by the given \meta{options}.

在 \refEnvLe{tcolorbox} 中使用,用于添加一个带有给定 \meta{text} 的副标题框,该框使用封闭框的多个继承属性进行格式设置,并通过 \refKey{/tcb/subtitle style} 的进一步设置和给定的 \meta{options} 进行格式化。

% 在\refEnvLe{tcolorbox}内部,将\meta{text}添加为子标题盒子。
% 这是一个独立的\refEnvLe{tcolorbox},%
% 一些格式从封闭的盒子中继承过来的属性,%
% 也可以通过给\refKey{/tcb/subtitle style}指定的\meta{options}来设定。
\begin{exdispExample*}{tcbsubtitle_1}{%
sbs%是sidebyside的意思
,lefthand ratio=0.6%upper侧占的比例
}
\begin{tcolorbox}[title=我的标题,
colback=red!5!white,
colframe=red!75!black,
fonttitle=\bfseries]
  This is a \textbf{tcolorbox}.
\tcbsubtitle[before skip=\baselineskip]%
  {我的{\tt 副}标题}
进一步的文本。
\end{tcolorbox}
\end{exdispExample*}

% 这段代码使用了tcolorbox宏包,创建了一个带有标题和副标题的盒子。具体解释如下:
% \begin{exdispExample*}{tcbsubtitle_1}{% sbs%是sidebyside的意思 ,lefthand ratio=0.6%upper侧占的比例 }

% 这部分代码是使用exdispExample环境来显示代码和输出结果的,其中使用了sbs选项表示将代码和输出结果并排显示,使用lefthand ratio选项表示代码窗口占整个窗口的比例为0.6。

% \begin{tcolorbox}[title=我的标题, colback=red!5!white, colframe=red!75!black, fonttitle=\bfseries]

% 这部分代码创建了一个tcolorbox盒子,其中包含了一个标题“我的标题”。colback选项表示背景色为红色和白色混合,colframe选项表示边框颜色为红色和黑色混合,fonttitle选项表示标题的字体为粗体。

% This is a \textbf{tcolorbox}.

% 这部分代码在盒子中插入了一段文本“This is a tcolorbox”,其中\textbf命令让“tcolorbox”加粗。

% \tcbsubtitle[before skip=\baselineskip]% {我的{\tt 副}标题}

% 这部分代码创建了一个副标题“我的副标题”,使用了\tcbsubtitle命令。before skip选项表示副标题前面的垂直距离为一个基准行距(\baselineskip),{\tt 副}使用了typewriter字体。

% 进一步的文本。

% 这部分代码在盒子中插入了进一步的文本。

% \end{tcolorbox}

% 这部分代码表示tcolorbox盒子的结束。

\begin{exdispExample*}{tcbsubtitle_2}{sbs,lefthand ratio=0.6}
\begin{tcolorbox}[title=My title,
    colback=red!5!white,
    colframe=red!75!black,
    colbacktitle=yellow!50!red,
    coltitle=red!25!black,
    fonttitle=\bfseries]
  This is a \textbf{tcolorbox}.
\tcbsubtitle[before skip=\baselineskip]%
{我的{\tt 副}标题}
进一步的文本。
\end{tcolorbox}
\end{exdispExample*}
\end{docCommand}

\begin{docTcbKey}[][doc new=2014-10-10]{subtitle style}{=\meta{options}}{no default, initially empty}
Adds |tcolorbox| \meta{options} to the settings for \refCom{tcbsubtitle}.

向 |tcbsubtitle| 设置的副标题的 |tcolorbox| 的 \meta{options} 中设置选项。

\begin{exdispExample*}{subtitle_style}{sbs,lefthand ratio=0.6}
\begin{tcolorbox}[title=My title,
  colback=red!5!white,
  colframe=red!75!black,
  colbacktitle=yellow!50!red,
  coltitle=red!25!black,
  fonttitle=\bfseries,
  subtitle style={boxrule=0.4pt,
    colback=yellow!50!red!25!white} ]
  This is a \textbf{tcolorbox}.
\tcbsubtitle{我的子标题}
  Further text.
\tcbsubtitle[colback=green!50!red!25!white]%
{第二个子标题}
上面的子标题中的背景色覆盖了 |subtitle style| 的设置。
\end{tcolorbox}
\end{exdispExample*}
\end{docTcbKey}
