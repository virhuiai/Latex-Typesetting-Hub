\begin{docTcbKey}{phantom}{=\meta{code}}{no default, initially unset}
The \meta{code} is put in a box at the upper left corner of the |tcolorbox|.
If the |tcolorbox| is breakable, the \meta{code} is executed for the first box of
the break sequence only. If there already was some phantom code given, the
new \meta{code} is appended.\par
The \meta{code} is intended to be used for counter stepping, labelling, and
related operations which do not produce visible text.

\meta{code}被放在|tcolorbox|的左上角的盒子中。%
如果 |tcolorbox| 是可分的, \meta{code} 将只会在中断序列的第一部分执行。%
如果已经给出了一些phantom代码,新的\meta{code}被追加过去。\par
\meta{code}旨在用于计数器步进, 标签和一些不会产生可见内容的相关操作。
\begin{itemize}
\item 
% The \meta{code} is executed before the title and box content, i.\,e.\ counter
%   values are ensured to be increased before usage.
\meta{code}在标题和盒子内容之前执行, i.\,e.\ 确保计数器的值在使用前增加了。
\item %Labels are ensured to reference the correct page number.
确保标签引用正确的页码。
\item 
% The \meta{code} is executed only once even during fitting operations for
%   title and box content.
\meta{code}只执行一次,即使是在标题和内容的自适应过程中。
\item 
% In combination with the |hyperref| package, the hyper anchor is set
%   to the upper left corner of the |tcolorbox|, i.\,e.\ 
% links inside the pdf document   will jump to the box pleasantly.
%todo 再翻译
结合|hyperref|包,超锚被设置为|tcolorbox|的左上角, i.\,e.\ PDF文档中的链接将友好地跳转到相应盒子。

\item 
% Since the \meta{code} is executed inside a \TeX\ group, only global
%   operations can survive this group.
由于\meta{code}是在\TeX\ 组中执行的,因此仅是全局的在这个群体中,操作可以存活下来。
\end{itemize}
% Examples for the |phantom| usage are given in Section \ref{listing:exercises}
% from page \pageref{listing:exercises}, e.\,g.\
% Example \ref{exe:tabular_example} on page \pageref{exe:tabular_example}.
|phantom| 的使用示例见\pageref{listing:exercises}页的\ref{listing:exercises}小节
, e.\,g.\ 第 \pageref{exe:tabular_example} 页的 \ref{exe:tabular_example}。
\end{docTcbKey}


\begin{docTcbKey}{nophantom}{}{no value, initially set}
% Removes the phantom code if set before.
删除之前设置的phantom代码。
\end{docTcbKey}


\begin{docTcbKey}{label}{=\meta{marker}}{no default, initially unset}
The \meta{marker} is set as label text for a reference with the |\ref| macro.
Typically, this option is used for numbered boxes, see Subsection \ref{sec:numberedboxes}
from page \pageref{sec:numberedboxes}, e.\,g.\ \refKeyLe{/tcb/new/auto counter}.

\meta{marker}被设置为|\ref|宏引用的标签文本。%
通常,这个选项用于编号的盒子,参见,\pageref{sec:numberedboxes}页的 \ref{sec:numberedboxes} 小节%
, e.\,g.\ \refKeyLe{/tcb/new/auto counter}.
\end{docTcbKey}

\begin{docTcbKey}[][doc new=2014-11-28]{phantomlabel}{=\meta{marker}}{no default, initially unset}
Equivalent to \refKeyLe{/tcb/label} for an \emph{unnumbered} box.
A |\phantomsection| from the package |hyperref| \cite{rahtz:hyperref} is used to set a correct
hyperlink target. This is not needed for a numbered box.

等效于\emph{未编号}的盒子的\refKeyLe{/tcb/label}。%
包|hyperref|中的|\phantomsection|用于设置正确的超链接目标。%
对于有编号的盒子,这是不需要的。
\end{docTcbKey}



\begin{docTcbKey}{label type}{=\meta{type}}{no default, initially unset}
This option key can be used only in conjunction with the |cleveref| package
\cite{cubitt:2018a} which has to be loaded separately.
\meta{type} has to be a cross-reference type \emph{known} to |cleveref|
like |theorem|, |algorithm|, |result|, etc. References made with |cleveref|
will use this type. Note that using |label type| will result in compilation
errors, if |cleveref| is not loaded.
For an example, see \Vref{theo:meanvaluetheorem}.

此选项键只能与|cleveref|包一起使用,|cleveref|包必须单独加载。%
\meta{type}必须是|cleveref|的交叉引用类型,如 |theorem|, |algorithm|, |result|, 等。%
使用|cleveref|所做的引用将使用此类型。%
注意的是,如果|cleveref|未加载, 使用 |label type| 将导致编译错误。例子见 \Vref{theo:meanvaluetheorem}。
\end{docTcbKey}

\begin{docTcbKey}{no label type}{}{no value, initially set}
% Removes a \refKeyLe{/tcb/label type}, if set before.
删除\refKeyLe{/tcb/label type},如果之前有设置过。
\end{docTcbKey}

\begin{docTcbKey}{step}{=\meta{counter}}{no default, initially unset}
Shortcut for |phantom={\refstepcounter{#1}}|. The given \meta{counter} is
increased and ready for labelling. This option is not needed when
using the convenient automated numbering introduced with version 2.40,
see Subsection \ref{sec:numberedboxes}
from page \pageref{sec:numberedboxes}.

|phantom={\refstepcounter{#1}}|的快捷方式。%
给定的计数器\meta{counter}被增加并准备好标记。%
当使用2.40版本引入的,方便的,自动编号时,不需要这个选项,%
见\pageref{sec:numberedboxes}页的\ref{sec:numberedboxes}小节。
\end{docTcbKey}



\begin{docTcbKey}{step and label}{=\marg{counter}\marg{marker}}{no default, initially unset}
Shortcut for using \refKeyLe{/tcb/step} and \refKeyLe{/tcb/label}. This option is not needed when
using the convenient automated numbering introduced with version 2.40,
see Subsection \ref{sec:numberedboxes}
from page \pageref{sec:numberedboxes}.

使用\refKeyLe{/tcb/step}和\refKeyLe{/tcb/label}的快捷方式。%
当使用2.40版本引入的方便的自动编号时,不需要这个选项,%
参见\pageref{sec:numberedboxes}页的\ref{sec:numberedboxes}小节。
\end{docTcbKey}


% \clearpage
\begin{docTcbKey}{list entry}{=\meta{text}}{no default, initially unset}
If the \flqq list of tcolorbox(es)\frqq\ feature described in Subsection
\ref{sec:listsof} from page \pageref{sec:listsof} is used, this key
describes the \meta{text} for an entry into the generated list, e.\,g.

如果使用了,在\pageref{sec:listsof}页的\ref{sec:listsof}小节描述的 \flqq tcolorbox(es)列表\frqq\ 特性, 
这个选项描述了生成列表中条目的\meta{text}, e.\,g.
\begin{dispListing}
list entry={\protect\numberline{\thetcbcounter}My beautiful Example}
\end{dispListing}
完整的例子见 \pageref{listing:exercises} 页的 \ref{listing:exercises} 小节。
\end{docTcbKey}


\begin{docTcbKey}[][doc new=2014-09-19]{list text}{=\meta{text}}{style, no default}
This is a shortcut for setting \refKeyLe{/tcb/list entry} to\\
|\protect\numberline{\thetcbcounter}|\meta{text}.
So, the following settings are identical:

这是将 \refKeyLe{/tcb/list entry} 设为\\
|\protect\numberline{\thetcbcounter}|\meta{text} 的快捷方式。
因此,以下设置是相同的:
\begin{dispListing}
list text={My beautiful Example},
list entry={\protect\numberline{\thetcbcounter}My beautiful Example}
\end{dispListing}
See Section \ref{listing:exercises} from page \pageref{listing:exercises}
for a complete example.

完整的例子见 \pageref{listing:exercises} 页的 \ref{listing:exercises} 小节。
\end{docTcbKey}



\begin{docTcbKey}{add to list}{=\marg{list}\marg{type}}{no default, initially unset}
If the \flqq list of tcolorbox(es)\frqq\ feature described in Subsection
\ref{sec:listsof} from page \pageref{sec:listsof} is used, list entries are
generated automatically. With this key, you can enforce an entry to the
given \meta{list} with the given \meta{type}.
This issues:\\
|\addcontentsline|\marg{list}\marg{type}\marg{entry text}

如果使用了,\pageref{sec:listsof}页的\ref{sec:listsof}小节描述的\flqq list of tcolorbox(es)\frqq\ 功能, 列表项会自动生成。使用此键,您可以使用给定的\meta{type}将一个条目强制到给定的\meta{list}。
This issues:\\
|\addcontentsline|\marg{list}\marg{type}\marg{entry text}
\end{docTcbKey}




\begin{docTcbKey}[][doc new and updated={2016-06-22}{2016-11-18}]{nameref}{=\meta{text}}{no default, initially unset}
If the |nameref| package is loaded, the given \meta{text} is used for
corresponding |\nameref| macros. Typically, the \meta{text} will be chosen
to be identical or nearly identical to the one for \refKeyLe{/tcb/title}.

如果加载了|nameref|包,%
给定的\meta{text}作为|\nameref|宏的参数。%
通常,\meta{text}将被选择为与\refKeyLe{/tcb/title}相同或几乎相同。

\inputpreamblelisting{A}


\begin{dispExample}
\begin{pabox}[label={mynamelabel},nameref={Title or anything else}]{Title text}
This is a tcolorbox.
\end{pabox}
This box is automatically numbered with \ref{mynamelabel} on page
\pageref{mynamelabel}.

The box is titled \enquote{\nameref{mynamelabel}}.
\end{dispExample}

\begin{marker}
\refKeyLe{/tcb/nameref} is used automatically inside \refCom{newtcbtheorem}.

\refKeyLe{/tcb/nameref}在\refCom{newtcbtheorem}中自动使用。
\end{marker}

\end{docTcbKey}




% \clearpage
\begin{docTcbKey}[][doc new=2017-02-03]{hypertarget}{=\meta{marker}}{no default, initially unset}
A |\hypertarget| from the package |hyperref| \cite{rahtz:hyperref} is used to
create an internal link of an anchor \meta{marker}.
This \meta{marker} can be referenced by |\hyperlink| or
\refKeyLe{/tcb/hyperlink}.

包|hyperref|中的|\hypertarget|用于创建一个锚\meta{marker}的内部链接。
这个\meta{marker}可以通过|\hyperlink|或\refKeyLe{/tcb/hyperlink}链接引用到。

\begin{dispExample*}{sbs,lefthand ratio=0.7}
% \usepackage{hyperref}%
\begin{tcolorbox}[enhanced,
colback=red!10,colframe=red!50!black,
hypertarget=hypertwinA,
hyperlink=hypertwinB,
title=Box A]
Click me to jump to Box B.
\end{tcolorbox}
\end{dispExample*}
\end{docTcbKey}

\begin{docTcbKey}[][doc new=2017-02-10]{bookmark}{=\meta{text}}{no default, initially unset}
Sets a PDF bookmark with the given \meta{text}, if the package |bookmark| \cite{oberdiek:bookmark}
is loaded. This bookmark is set with an automated destination (the current box)
and is set one level below the current bookmark level.

% Sets a PDF bookmark with the given \meta{text}, if the package |bookmark| is loaded. 
如果加载了包|bookmark|,则使用给定的\meta{text}设置PDF书签。%
此书签使用自动目标(当前盒子)设置,并设置在当前书签级别以下一级。%
\begin{dispExample*}{sbs,lefthand ratio=0.7}
% \usepackage{bookmark}%
\begin{tcolorbox}[colback=blue!10,colframe=blue!50!black,
bookmark=Example for using a bookmark,
title=Example for using a bookmark]
Open the bookmark view of the previewer
to see the bookmark.
\end{tcolorbox}
\end{dispExample*}
\end{docTcbKey}


\begin{docTcbKey}[][doc new=2017-02-10]{bookmark*}{=\marg{options}\marg{text}}{no default, initially unset}
Identical to \refKeyLe{/tcb/bookmark}, but additional \meta{options}
from the package |bookmark| \cite{oberdiek:bookmark} can be given.

与\refKeyLe{/tcb/bookmark}相同,但可以从包|bookmark|中给出额外的\meta{options}。
\begin{dispExample*}{sbs,lefthand ratio=0.7}
% \usepackage{bookmark}%
\begin{tcolorbox}[colback=red!10,colframe=red!50!black,
bookmark*={color=red,italic,bold}%
            {Another bookmark example},
title=Red and bold bookmark]
Open the bookmark view of the previewer
to see the bookmark.
\end{tcolorbox}
\end{dispExample*}
\end{docTcbKey}

\begin{docTcbKey}[][doc new=2018-07-26]{index}{=\meta{entry}}{no default, initially unset}
Adds an index \meta{entry} for the box. This is a shortcut for
setting |\index|\marg{entry} to \refKeyLe{/tcb/phantom}.

为盒子添加索引\meta{entry}。 这是一个将|\index|\marg{entry} 设置为 \refKeyLe{/tcb/phantom}的快捷方式。
\end{docTcbKey}

\begin{docTcbKey}[][doc new=2018-07-26]{index*}{=\marg{name}\marg{entry}}{no default, initially unset}
Adds an \meta{entry} to an index with a specific \meta{name}.
This is a shortcut for  setting |\index|\oarg{name}\marg{entry} to \refKeyLe{/tcb/phantom}.
An index extension package like |imakeidx| has to be loaded to use  this option key.

% Adds an \meta{entry} to an index with a specific \meta{name}.
将\meta{entry}添加到具有特定\meta{name}的索引中。
这是一个将|\index|\oarg{name}\marg{entry}设置为\refKeyLe{/tcb/phantom}的快捷方式。
必须加载像|imakeidx|这样的索引扩展包才能使用此选项键。
\end{docTcbKey}

% \clearpage
% Even and Odd Pages