\subsection{Geometry\\几何形状}

% \subsubsection{Width} 

\begin{docTcbKey}{width}{=\meta{length}}{no default, initially \cs{linewidth}}
Sets the total width of the colored box to \meta{length}.
See also \refKeyLe{/tcb/height}.

将%有色的带框
盒子的总宽度设置为 \meta{length}。另见 \refKeyLe{/tcb/height}。
\begin{exdispExample}{width} 
\tcbset{colback=red!5!white,colframe=red!75!black}

\begin{tcolorbox}[width=\linewidth/2]
这是一个\textbf{tcolorbox}.
\end{tcolorbox}
\end{exdispExample}
\end{docTcbKey}


\begin{docTcbKey}[][doc new=2014-10-31]{text width}{=\meta{length}}{style, no default}
Sets the text width of the upper part to \meta{length}.
See also \refKeyLe{/tcb/text height}.

将upper部分的文本宽度设置为 \meta{length}.
另见 \refKeyLe{/tcb/text height}.
\begin{exdispExample}{text_width}
\tcbset{colback=red!5!white,colframe=red!75!black}

\begin{tcolorbox}[text width=4cm]
This is a \textbf{tcolorbox} where the text has a width of 4cm.
\end{tcolorbox}
\end{exdispExample}
\end{docTcbKey}

\begin{docTcbKey}[][doc new=2014-11-07]{add to width}{=\meta{length}}{style, no default}
Adds \meta{length} to the current total width of the colored box.

将当前盒子的总宽度增加 \meta{length} 。    
\begin{exdispExample*}{add_to_width}{sbs,lefthand ratio=0.6}
\tcbset{width=4cm,colback=red!5!white,
colframe=red!75!black}

\begin{tcolorbox}
这是一个\textbf{tcolorbox}.
\end{tcolorbox}

\begin{tcolorbox}[add to width=1cm]
这是一个\textbf{tcolorbox}.
\end{tcolorbox}
\end{exdispExample*}
\end{docTcbKey}
See \Fullref{sec:heightcontrol} for setting fixed height values.

有关设置固定高度值的问题,请参阅 \Fullref{sec:heightcontrol}。
% \setcounter{section}{4}
% \setcounter{subsection}{7}
% \setcounter{subsubsection}{1}
% \subsubsection{Rules\\线}
\begin{docTcbKey}{toprule}{=\meta{length}}{no default, initially \texttt{0.5mm}}
Sets the line width of the top rule to \meta{length}.

设置顶边框线的宽度为 \meta{length}。
\begin{exdispExample}{toprule}
\tcbset{colback=red!5!white,colframe=red!75!black}

\begin{tcolorbox}[toprule=3mm]
这是一个\textbf{tcolorbox}.
\end{tcolorbox}
\end{exdispExample}
\end{docTcbKey}


\begin{docTcbKey}{bottomrule}{=\meta{length}}{no default, initially \texttt{0.5mm}}
Sets the line width of the bottom rule to \meta{length}.

设置底边框线的宽度为 \meta{length}。
\begin{exdispExample}{bottomrule}
\tcbset{colback=red!5!white,colframe=red!75!black}

\begin{tcolorbox}[bottomrule=3mm]
这是一个\textbf{tcolorbox}.
\end{tcolorbox}
\end{exdispExample}
\end{docTcbKey}

\begin{docTcbKey}{leftrule}{=\meta{length}}{no default, initially \texttt{0.5mm}}
Sets the line width of the left rule to \meta{length}.

设置左边框线的宽度为 \meta{length}。
\begin{exdispExample}{leftrule}
\tcbset{colback=red!5!white,colframe=red!75!black}

\begin{tcolorbox}[leftrule=3mm]
这是一个\textbf{tcolorbox}.
\end{tcolorbox}
\end{exdispExample}
\end{docTcbKey}


\begin{docTcbKey}{rightrule}{=\meta{length}}{no default, initially \texttt{0.5mm}}
Sets the line width of the right rule to \meta{length}.

设置右边框线的宽度为 \meta{length}。
\begin{exdispExample}{rightrule}
\tcbset{colback=red!5!white,colframe=red!75!black}

\begin{tcolorbox}[rightrule=3mm]
这是一个\textbf{tcolorbox}.
\end{tcolorbox}
\end{exdispExample}
\end{docTcbKey}




% \clearpage
\begin{docTcbKey}{titlerule}{=\meta{length}}{no default, initially \texttt{0.5mm}}
Sets the line width of the rule below the title to \meta{length}.

设置标题文本下方的线的宽度为 \meta{length}。
\begin{exdispExample}{titlerule}
\tcbset{enhanced,colback=red!5!white,colframe=red!75!black,
colbacktitle=red!90!black}

\begin{tcolorbox}[titlerule=3mm,title=This is the title]
这是一个\textbf{tcolorbox}.
\end{tcolorbox}
\end{exdispExample}
\end{docTcbKey}


\begin{docTcbKey}{boxrule}{=\meta{length}}{style, no default, initially \texttt{0.5mm}}
Sets all rules of the frame to \meta{length}, i.\,e.\ 
\refKeyLe{/tcb/toprule}, \refKeyLe{/tcb/bottomrule}, \refKeyLe{/tcb/leftrule},
\refKeyLe{/tcb/rightrule}, and \refKeyLe{/tcb/titlerule}.

设置所有的边框线的宽度为 \meta{length}\footnote{i.\,e.\ 
\refKeyLe{/tcb/toprule}, \refKeyLe{/tcb/bottomrule}, \refKeyLe{/tcb/leftrule},
\refKeyLe{/tcb/rightrule}, 和 \refKeyLe{/tcb/titlerule}.}。
\begin{exdispExample}{boxrule}
\tcbset{colback=red!5!white,colframe=red!75!black}

\begin{tcolorbox}[boxrule=3mm]
这是一个\textbf{tcolorbox}.
\end{tcolorbox}
\end{exdispExample}
\end{docTcbKey}

\bigskip
\begin{marker}
More options for drawing a \refKeyLe{/tcb/borderline} are provided by using skins documented in
Section \ref{sec:skins} from page \pageref{sec:skins}.

更多的关于绘制 \refKeyLe{/tcb/borderline} 的选项的描述在 skins 的文档中,详见 \pageref{sec:skins} 页的 \ref{sec:skins} 小节。
\end{marker}


% \setcounter{section}{4}
% \setcounter{subsection}{7}
% \setcounter{subsubsection}{2}
% % Arcs\hfill 
\subsubsection{弧线}
\begin{docTcbKey}{arc}{=\meta{length}}{no default, initially \texttt{1mm}}
Sets the inner radius of the four frame arcs to \meta{length}.

设置边框的四个角落的弧的内半径为 \meta{length}。

% \begin{exdispExample}{arc}
% \tcbset{colback=red!5!white,colframe=red!75!black}
% \begin{tcolorbox}[arc=0mm]
% 这是一个\textbf{tcolorbox}.
% \end{tcolorbox}
% \begin{tcolorbox}[arc=3mm]
% 这是一个\textbf{tcolorbox}.
% \end{tcolorbox}
% \end{exdispExample}

\begin{dispExample*}{sidebyside,lefthand ratio=0.6}
\tcbset{colback=red!5!white,colframe=red!75!black}
\begin{tcolorbox}[arc=0mm]
这是一个\textbf{tcolorbox}.
\end{tcolorbox}
\end{dispExample*}

\begin{dispExample*}{sidebyside,lefthand ratio=0.6}
\tcbset{colback=red!5!white,colframe=red!75!black}
\begin{tcolorbox}[arc=3mm]
这是一个\textbf{tcolorbox}.
\end{tcolorbox}
\end{dispExample*}
\end{docTcbKey}



% \begin{exdispExample*}{arc_default}{sbs,lefthand ratio=0.6}
 
% \end{exdispExample*}

% \begin{exdispExample*}{arc_3cm}{sbs,lefthand ratio=0.6}
 
% \end{exdispExample*}

% gpt4:
% 在 LaTeX 中,颜色的混合可以使用 ! 符号来实现。这种写法叫做 "interpolation of colors"。

% red!5!white 的意思是,颜色由5%的红色和95%的白色混合而成。结果是一个非常浅的红色。

% red!75!black 的意思是,颜色由75%的红色和25%的黑色混合而成。结果是一个较深的红色。

% claude
% !后面的数字表示混合比例,范围是0到100。
% 0表示完全使用第一个颜色。
% 100表示完全使用第二个颜色。
% 所以:

% red!0!white 等同于纯红色red。
% red!100!white 等同于纯白色white。


% \clearpage
\begin{docTcbKey}[][doc new=2015-05-05]{circular arc}{}{style, no value}
  Sets \refKeyLe{/tcb/arc} to match the half of the inner width of the colored box.
  If width and height of the box are identical, this gives a circle.
  
  将 \refKeyLe{/tcb/arc} 设置为盒子内部宽度的一半。如果盒子的宽度和高度相同,就会得到一个圆。
  \begin{marker}
  If the height of the box is smaller than the width, the result will look
  quite ugly.
  
  如果盒子的高度小于宽度,结果看起来会很难看。   
  \end{marker}
  \begin{exdispExample*}{circular_arc}{sbs,lefthand ratio=0.6}
  \begin{tcolorbox}[width=3cm,
  colback=red!5!white,
  colframe=red!75!black,
  halign=center,valign=center,
  square,circular arc]
  这是一个\textbf{tcolorbox}.
  \end{tcolorbox}
  \end{exdispExample*}
  \end{docTcbKey}
  
  
  \begin{docTcbKey}[][doc new=2015-05-05]{bean arc}{}{style, no value}
  Sets \refKeyLe{/tcb/arc} to match the smaller value of the
  half of the inner width and of the inner height of the colored box.
  
%   设置 \refKeyLe{/tcb/arc} 为盒子内宽和内高中较小者值的一半。
  
  将\refKeyLe{/tcb/arc}设置为盒子内宽度一半和高度一半的较小值。

  \begin{marker}
  This only works for a fixed \refKeyLe{/tcb/height}. Also, \refKeyLe{/tcb/bean arc}
  must be used \emph{after} width and height are set by option keys.
  
%   这只适用于 \refKeyLe{/tcb/height} 为固定值的情况。此外,\refKeyLe{/tcb/bean arc} 选项设置需要在设置宽度和高度之后。
  这仅适用于固定的\refKeyLe{/tcb/height}。另外,\refKeyLe{/tcb/bean arc}必须在宽度和高度由选项键设置之后使用。
  \end{marker}
  \begin{exdispExample*}{bean_arc}{sbs,lefthand ratio=0.6}
  \tcbset{size=fbox,boxrule=0.5mm,
  colback=red!5!white,
  colframe=red!75!black,
  halign=center,valign=center}
  
  \begin{tcolorbox}[width=3cm,height=2cm,
  bean arc]
  Box A
  \end{tcolorbox}
  
  \begin{tcolorbox}[width=2cm,height=3cm,
  bean arc]
  Box B
  \end{tcolorbox}
  \end{exdispExample*}
  \end{docTcbKey}
  
  % 八角形
  \begin{docTcbKey}[][doc new=2015-05-05]{octogon arc}{}{style, no value}
  Sets \refKeyLe{/tcb/arc} to match $\frac{1}{2+\sqrt{2}}$ of the inner width
  of the colored box. If width and height of the box are identical,
  the interior is a regular octogon.

将\refKeyLe{/tcb/arc}设置为盒子的内部宽度的$\frac{1}{2+\sqrt{2}}$。如果盒子的宽度和高度相同,则内部是一个正八边形。

% 设置 \refKeyLe{/tcb/arc} 为盒子内部宽度的 $\frac{1}{2+\sqrt{2}}$ 。如果盒子的宽度和高度相同,内部是一个规则的八边形。
  % \begin{tcolorbox}[
  %   width=2.1cm,octogon arc,
  %   ]
  %   STOP
  %   \end{tcolorbox}
\begin{exdispExample*}{octogon_arc}{sbs,lefthand ratio=0.8}
\begin{tcolorbox}[enhanced,% enhanced: 启用高级选项。
size=minimal,% size=minimal: 去除默认边距,仅显示框体。
auto outer arc,% auto outer arc: 自动调整外部圆角半径以适应框体大小。
width=2.1cm,octogon arc,% octogon arc: 外部圆角为八角形。
colback=red,% colback=red: 背景色为红色。
colframe=white,% colframe=white: 边框颜色为白色。
colupper=white,% colupper=white: 文字颜色为白色。
fontupper=\fontsize{7mm}{7mm}\selectfont\bfseries\sffamily,
%文字大小为7毫米,粗体,无衬线字体。
halign=center,valign=center,
% halign=center: 水平居中对齐。% valign=center: 垂直居中对齐。
square,% square: 边框为直角。
arc is angular,% arc is angular: 内部圆角为直角。
borderline={0.2mm}{-1mm}{red}  ]%使用红色的0.2毫米线条作为边框,内部偏移1毫米。
STOP
\end{tcolorbox}
\end{exdispExample*}
\end{docTcbKey}




  

% angular,有角度的
% \clearpage
\begin{docTcbKey}[][doc new=2015-05-05]{arc is angular}{}{no value, initially unset}
  Using this options applies a patch which straightens the corners arcs of
  the boxes. The little arcs are replaced by little straight lines.
  
%   这个选项\footnote{arc is angular}用于对盒子四角弧线的一个补丁。小弧线被小直线代替。

  使用此选项将应用一个补丁,使盒子的角弧线变直。小弧线将被小直线替换。
  \begin{marker}
  This patch is considered as an experimental feature.
  It changes some of the original \tikzname\ code. This change may break
  with future updates of \tikzname.
  
%   这个补丁被认为是一个实验性的特征。它改变了一些原始的 \tikzname\ 代码。此更改可能会随着 \tikzname\ 的未来更新而中断。
  这个补丁被视为一项实验性的功能。它改变了一些原始的\tikzname 代码。这种改变可能会在未来的\tikzname 更新中出现问题。
  \end{marker}
  
  \begin{exdispExample*}{arc_is_angular}{sbs,lefthand ratio=0.6}
  \tcbset{colback=red!5!white,colframe=red!75!black,
  arc=3mm}
  
  \begin{tcolorbox}[arc is angular]
  这是一个\textbf{tcolorbox}.
  \end{tcolorbox}
  \begin{tcolorbox}[arc is curved]
  这是一个\textbf{tcolorbox}.
  \end{tcolorbox}
  \end{exdispExample*}
  
  \end{docTcbKey}
  
  
\begin{docTcbKey}[][doc new=2015-05-05]{arc is curved}{}{no value, initially set}
This option resets the patch from \refKeyLe{/tcb/arc is angular}. The
original \tikzname\ code is activated.

% 此选项将 \refKeyLe{/tcb/arc is angular} 的补丁修改重置为角度。激活原始的 \tikzname\ 代码。

此选项将重置 \refKeyLe{/tcb/arc is angular} 所设置的修补程序。原始的 \tikzname\ 代码将被激活。
\end{docTcbKey}


\begin{docTcbKey}{outer arc}{=\meta{length}}{no default, initially unset}
Sets the outer radius of the four frame arcs to \meta{length}.

设置盒子四个角的弧的外半径为 \meta{length}。
% 将四个框架拱形的外半径设置为\meta{length}。
\begin{exdispExample}{outer_arc}
\tcbset{colback=red!5!white,colframe=red!75!black}

\begin{tcolorbox}[arc=4mm,outer arc=1mm]
这是一个\textbf{tcolorbox}.
\end{tcolorbox}
\begin{tcolorbox}[arc=4mm]
这是一个\textbf{tcolorbox}.
\end{tcolorbox}
\end{exdispExample}
\end{docTcbKey}

\begin{docTcbKey}{auto outer arc}{}{no value, initially set}
Sets the outer radius of the four frame arcs automatically in
dependency of the inner radius given by \refKeyLe{/tcb/arc}.

根据 \refKeyLe{/tcb/arc} 给出的内部半径自动设置外部半径。

\end{docTcbKey}
  
% \setcounter{section}{4}
% \setcounter{subsection}{7}
% \setcounter{subsubsection}{3}
% \setcounter{section}{4}
\setcounter{subsection}{7}
\setcounter{subsubsection}{3}

\subsubsection{Spacing\hfill 间隔}
\begin{docTcbKey}{boxsep}{=\meta{length}}{no default, initially \texttt{1mm}}
Sets a common padding of \meta{length} between the text content and the
frame of the box. This value is added to the key values of
|left|, |right|, |top|, |bottom|, and |middle| at the appropriate places.

% 在文本内容和盒子的框之间设置一个共同的填充宽度为 \meta{length}。 这个值会被添加到
% |left|, |right|, |top|, |bottom|, 和 |middle| 的合适位置。

在文本内容和盒子边框之间设置一个共同的填充 \meta{length}。这个值会添加到 |left|、|right|、|top|、|bottom| 和 |middle| 的关键值中,在适当的位置使用。

% \begin{dispExample*}{}
% \tcbset{colback=red!5!white,colframe=red!75!black,width=(\linewidth-4mm)/2, before=,after=\hfill}
% \begin{tcolorbox}
% hi
% \end{tcolorbox}
% \begin{tcolorbox}[draft]
% hi
% \end{tcolorbox}
% \end{dispExample*}
% %%%%%%%%
% \begin{dispExample*}{}
% \tcbset{colback=red!5!white,colframe=red!75!black,width=(\linewidth-4mm)/2, before=,after=\hfill}
% \begin{tcolorbox}[boxsep=0mm]
% hi
% \end{tcolorbox}
% \begin{tcolorbox}[boxsep=0mm,draft]
% hi
% \end{tcolorbox}
% \end{dispExample*}
% %%%%%%
\begin{dispExample*}{}
\tcbset{colback=red!5!white,colframe=red!75!black,width=(\linewidth-4mm)/2, before=,after=\hfill}

\begin{tcolorbox}[boxsep=5mm,draft]
hi
\end{tcolorbox}
\begin{tcolorbox}[boxsep=8mm,draft]
hi
\end{tcolorbox}
\begin{tcolorbox}[boxsep=5mm]
hi
\end{tcolorbox}
\begin{tcolorbox}[boxsep=8mm]
hi
\end{tcolorbox}
\end{dispExample*}

% \begin{exdispExample}{boxsep}
% \tcbset{colback=red!5!white,colframe=red!75!black  
% \begin{tcolorbox}[boxsep=0mm]
% hi
% \end{tcolorbox}
% \begin{tcolorbox}[boxsep=0mm,draft]
% hi
% \end{tcolorbox}
% \end{exdispExample}
\end{docTcbKey}


\begin{docTcbKey}{left}{=\meta{length}}{style, no default, initially \texttt{4mm}}
Sets the left space between all text parts and frame (additional to |boxsep|).
This is an abbreviation for setting
|lefttitle|, |leftupper|, and |leftlower| to the same value.

设置所有文本部分和盒子间的左边距(除了|boxsep|之外)。 这是设置|lefttitle|、|leftupper|和|leftlower|为相同值的简写方式。


% 设置所有的文本内容同左侧边框的间隔(附加到|boxsep|).
% 这是同时将 |lefttitle|, |leftupper|, 和 |leftlower| 设置为同一个值的简写方式。
\begin{dispExample*}{sbs}
\tcbset{colback=red!5!white%
,colframe=red!75!black
% ,width=(\linewidth-4mm)/2, before=,after=\hfill
}

\begin{tcolorbox}[left=0mm]
指定 \verb|left=0mm|
\end{tcolorbox}

\begin{tcolorbox}
使用默认
\end{tcolorbox}

\begin{tcolorbox}[left=4mm]
指定 \verb|left=4mm|
\end{tcolorbox}


\begin{tcolorbox}[left=10mm]
指定 \verb|left=10mm|
\end{tcolorbox}
\end{dispExample*}
\end{docTcbKey}

% TODO 再看下
\begin{docTcbKey}[][doc new=2017-02-16]{left*}{=\meta{length}}{style, no default}
Sets \refKeyLe{/tcb/left} such that \meta{length} is the distance between
the left bounding box and the text parts.

设置\refKeyLe{/tcb/left},使得\meta{length}为左边界框和文本部分之间的距离。

% 设置 \refKeyLe{/tcb/left} 的值 \meta{length} 为盒子左边界和上下文的文本左侧的距离。

\begin{exdispExample}{left_star}
\tcbset{colback=red!5!white,colframe=red!75!black}

This is some text.
\begin{tcolorbox}[grow to left by=5mm,left*=0mm,
enhanced,show bounding box]
这是一个\textbf{tcolorbox}.
\end{tcolorbox}

\begin{tcolorbox}[left*=0mm,
enhanced,show bounding box]
这是一个\textbf{tcolorbox}.
\end{tcolorbox}

\begin{tcolorbox}[%left*=0mm,
enhanced,show bounding box]
这是一个\textbf{tcolorbox}.
\end{tcolorbox}
\end{exdispExample}
\end{docTcbKey}

% 这段Latex代码使用了tcolorbox宏包,它提供了创建漂亮框框的命令。在这个例子中,代码创建了三个tcolorbox,每个框内都包含了一段文本“这是一个\textbf{tcolorbox}”。

% 第一个tcolorbox使用了参数“grow to left by=5mm”和“left*=0mm”,
% 这意味着这个框会向左边延伸5毫米,并且左边的边框宽度为0毫米。同时,也使用了“enhanced”和“show bounding box”参数,这些参数可以让框框看起来更漂亮,并且显示边框的边界。

% 第二个tcolorbox只使用了“left*=0mm”参数,这意味着这个框的左边边框宽度为0毫米。同样,也使用了“enhanced”和“show bounding box”参数。

% 第三个tcolorbox只使用了“enhanced”和“show bounding box”参数,这意味着这个框没有任何特殊的设置,只是一个普通的tcolorbox。


% \clearpage
\begin{docTcbKey}{lefttitle}{=\meta{length}}{no default, initially \texttt{4mm}}
  Sets the left space between title text and frame (additional to |boxsep|).

设置标题文本的左侧同边框的距离(附加 |boxsep|)。
\begin{exdispExample}{lefttitle}
\tcbset{colback=red!5!white,colframe=red!75!black}

\begin{tcolorbox}[lefttitle=3cm,title=My Title]
这是一个\textbf{tcolorbox}.
\end{tcolorbox}
\end{exdispExample}
\end{docTcbKey}


\begin{docTcbKey}{leftupper}{=\meta{length}}{no default, initially \texttt{4mm}}
  Sets the left space between upper text and frame (additional to |boxsep|).

设置upper部分同左侧边边框的距离(附加 |boxsep|)。
\begin{exdispExample}{leftupper}
\tcbset{colback=red!5!white,colframe=red!75!black}

\begin{tcolorbox}[leftupper=3cm,title=My Title]
这是一个\textbf{tcolorbox}.
\end{tcolorbox}
\end{exdispExample}
\end{docTcbKey}

\begin{docTcbKey}{leftlower}{=\meta{length}}{no default, initially \texttt{4mm}}
  Sets the left space between lower text and frame (additional to |boxsep|).

设置lower部分同左侧边边框的距离(附加 |boxsep|)。
\begin{exdispExample}{leftlower}
\tcbset{colback=red!5!white,colframe=red!75!black}

\begin{tcolorbox}[leftlower=3cm]
这是一个\textbf{tcolorbox}.
\tcblower
这是lower部分。
\end{tcolorbox}
\end{exdispExample}
\end{docTcbKey}

\enlargethispage*{1cm}

\begin{docTcbKey}{right}{=\meta{length}}{style, no default, initially \texttt{4mm}}
  Sets the right space between all text parts and frame (additional to |boxsep|).
  This is an abbreviation for setting
  |righttitle|, |rightupper|, and |rightlower| to the same value.

设置所有文本部分同右侧边框的距离(附加 |boxsep|)。
这是同时将 |righttitle|, |rightupper|, 和 |rightlower| 设置为同一个值的简写方式。
\begin{exdispExample}{right}
\tcbset{colback=red!5!white,colframe=red!75!black}

\begin{tcolorbox}[width=5cm,right=2cm]
这是一个\textbf{tcolorbox}.
\end{tcolorbox}
\end{exdispExample}
\end{docTcbKey}





% \clearpage
  
\begin{docTcbKey}[][doc new=2017-02-16]{right*}{=\meta{length}}{style, no default}
  Sets \refKeyLe{/tcb/right} such that \meta{length} is the distance between
  the right bounding box and the text parts.

设置 \refKeyLe{/tcb/right} 的宽度 \meta{length} 为盒子右边框同上下文文本的右侧的距离。
\begin{exdispExample}{right_star}
\tcbset{colback=red!5!white,colframe=red!75!black}

\flushright This is some text.
\begin{tcolorbox}[grow to right by=5mm,right*=0mm,
  halign=right,enhanced,show bounding box]
这是一个\textbf{tcolorbox}.
\end{tcolorbox}
\end{exdispExample}
\end{docTcbKey}



\begin{docTcbKey}{righttitle}{=\meta{length}}{no default, initially \texttt{4mm}}
  Sets the right space between title text and frame (additional to |boxsep|).

设置标题文本右侧同右边框的距离(附加 |boxsep|)。
  \begin{exdispExample}{righttitle}
\tcbset{colback=red!5!white,colframe=red!75!black}

\begin{tcolorbox}[width=5cm,righttitle=2cm,title=My very long title text]
This is a \textbf{tcolorbox} with standard upper box dimensions.
\end{tcolorbox}
\end{exdispExample}
\end{docTcbKey}


\begin{docTcbKey}{rightupper}{=\meta{length}}{no default, initially \texttt{4mm}}
  Sets the right space between upper text and frame (additional to |boxsep|).

设置upper部分的文本同右边框的距离(附加|boxsep|).
\begin{exdispExample}{rightupper}
\tcbset{colback=red!5!white,colframe=red!75!black}

\begin{tcolorbox}[width=5cm,rightupper=2cm,title=My very long title text]
This is a \textbf{tcolorbox} with compressed upper box dimensions.
\end{tcolorbox}
\end{exdispExample}
\end{docTcbKey}





% \clearpage
\begin{docTcbKey}{rightlower}{=\meta{length}}{no default, initially \texttt{4mm}}
  Sets the right space between lower text and frame (additional to |boxsep|).

设置lower部分的右边同右侧边框的距离(附加 |boxsep|)。
\begin{exdispExample}{rightlower}
\tcbset{colback=red!5!white,colframe=red!75!black}

\begin{tcolorbox}[width=5cm,rightlower=2cm]
This is a \textbf{tcolorbox} with standard upper box dimensions.
\tcblower
This is the lower part with large space at right.
\end{tcolorbox}
\end{exdispExample}
\end{docTcbKey}



\begin{docTcbKey}{top}{=\meta{length}}{no default, initially \texttt{2mm}}
  Sets the top space between text and frame (additional to |boxsep|).

设置文本同上边框的距离(附加 |boxsep|)。
\begin{exdispExample}{top}
\tcbset{colback=red!5!white,colframe=red!75!black}

\begin{tcolorbox}[top=0mm]
这是一个\textbf{tcolorbox}.
\tcblower
这是lower部分。
\end{tcolorbox}
\end{exdispExample}
\end{docTcbKey}


\begin{docTcbKey}{toptitle}{=\meta{length}}{no default, initially \texttt{0mm}}
  Sets the top space between title and frame (additional to |boxsep|).

设置标题文本同上边框的距离(附加 |boxsep|)。    
\begin{exdispExample}{toptitle}
\tcbset{colback=red!5!white,colframe=red!75!black}

\begin{tcolorbox}[toptitle=3mm,title=My title]
这是一个\textbf{tcolorbox}.
\end{tcolorbox}
\end{exdispExample}
\end{docTcbKey}






% \clearpage
\begin{docTcbKey}{bottom}{=\meta{length}}{no default, initially \texttt{2mm}}
  Sets the bottom space between text and frame (additional to |boxsep|).

设置文本底部同边框的距离 (附加 |boxsep|).
\begin{exdispExample}{bottom}
\tcbset{colback=red!5!white,colframe=red!75!black}

\begin{tcolorbox}[bottom=0mm]
这是一个\textbf{tcolorbox}.
\tcblower
这是lower部分。
\end{tcolorbox}
\begin{tcolorbox}
  这是一个\textbf{tcolorbox}.
  \tcblower
  这是lower部分。
  \end{tcolorbox}
\end{exdispExample}
\end{docTcbKey}

\begin{docTcbKey}{bottomtitle}{=\meta{length}}{no default, initially \texttt{0mm}}
  Sets the bottom space between title and frame (additional to |boxsep|).

设置标题同下方的边框的距离(附加 |boxsep|).
\begin{exdispExample}{bottomtitle}
\tcbset{colback=red!5!white,colframe=red!75!black}

\begin{tcolorbox}[bottomtitle=3mm,title=My title]
这是一个\textbf{tcolorbox}.
\end{tcolorbox}
\end{exdispExample}
\end{docTcbKey}


\begin{docTcbKey}{middle}{=\meta{length}}{no default, initially \texttt{2mm}}
Sets the space between upper and lower text to the separation line
(additional to |boxsep|).

% 将上下文本与分隔线之间的距离设置为分隔线(附加到|boxsep|)。

设置上下文本同分隔线的距离(附加 |boxsep|)。
\begin{exdispExample}{middle}
\tcbset{colback=red!5!white,colframe=red!75!black}

\begin{tcolorbox}[middle=0mm,boxsep=0mm]
这是一个\textbf{tcolorbox}.
\tcblower
这是lower部分。
\end{tcolorbox}
\begin{tcolorbox}[boxsep=0mm]
这是一个\textbf{tcolorbox}.
\tcblower
这是lower部分。
\end{tcolorbox}
\begin{tcolorbox}
这是一个\textbf{tcolorbox}.
\tcblower
这是lower部分。
\end{tcolorbox}
\end{exdispExample}
\end{docTcbKey}


% \setcounter{section}{4}
% \setcounter{subsection}{7}
% \setcounter{subsubsection}{4}
% \setcounter{section}{4}
\setcounter{subsection}{7}
\setcounter{subsubsection}{4}

\subsubsection{Size Shortcuts\\调整尺寸的快捷方式}
\begin{docTcbKey}{size}{=\meta{name}}{no default, initially \texttt{normal}}
Sets all geometry keys with exception of \refKeyLe{/tcb/width} to
predefined length values.
For \meta{name}, the following values are feasible:

将除 \refKeyLe{/tcb/width} 外的所有尺寸设置为预定义的值。
可选的 \meta{name} 值有:    


\sbox{\cshDocValueLikeItemBox}{\docValue{minimal}}%%% 调用前设置,以用来指定宽度
\cshDocValueAndTwoDescLikeItem{normal}{normal sized boxes e.g. of width \cs{linewidth}.}{常用的盒子尺寸 e.g. 宽度为 \cs{linewidth}。}
\cshDocValueAndTwoDescLikeItem{title}{title line sized boxes.}{宽度同标题行一致。}
\cshDocValueAndTwoDescLikeItem{small}{ small boxes e.g. for keyword highlighting.}{小一些的盒子 e.g. 用于关键字的高亮。}
\cshDocValueAndTwoDescLikeItem{fbox}{ identical to the standard \cs{fbox}.}{同使用 \cs{fbox} 一样。}
\cshDocValueAndTwoDescLikeItem{tight}{ no padding space at all.}{完全没有填充空间。}
\cshDocValueAndTwoDescLikeItem{minimal}{ no padding space, no box rules.}{没有填充空间,没有边框。}
   
% todo on line 是什么意思
\begin{exdispExample}{size_1}
\tcbset{colback=red!5!white,colframe=red!75!black}

\foreach \s in {normal,title,small,fbox,tight,minimal} {
  \tcbox[size=\s,on line]{\s} }

\foreach \s in {normal,title,small,fbox,tight,minimal} {
  \tcbox[size=\s,on line,title=Test]{\s} }

\foreach \s in {normal,title,small,fbox,tight,minimal} {
  \begin{tcolorbox}[size=\s,on line,title=Test,width=2.2cm]
    \s \tcblower lower\end{tcolorbox} }
\end{exdispExample}

\bigskip

\begin{tcolorbox}[tabularx={l|XXXXXX},title=Predefined values,
enhanced,fonttitle=\small\bfseries,fontupper=\small\ttfamily,
colback=yellow!10!white,colframe=red!50!black,colbacktitle=Salmon!30!white,
coltitle=black,center title
]
            & normal & title  & small & fbox  & tight & minimal\\\hline
boxrule     & 0.5mm  & 0.4mm  & 0.3mm & 0.4pt & 0.4pt & 0.0pt \\
boxsep      & 1.0mm  & 1.0mm  & 1.0mm & 3.0pt & 0.0pt & 0.0pt \\
left        & 4.0mm  & 2.0mm  & 1.0mm & 0.0pt & 0.0pt & 0.0pt \\
right       & 4.0mm  & 2.0mm  & 1.0mm & 0.0pt & 0.0pt & 0.0pt \\
top         & 2.0mm  & 0.25mm & 0.0mm & 0.0pt & 0.0pt & 0.0pt \\
bottom      & 2.0mm  & 0.25mm & 0.0mm & 0.0pt & 0.0pt & 0.0pt \\
toptitle    & 0.0mm  & 0.0mm  & 0.0mm & 0.0pt & 0.0pt & 0.0pt \\
bottomtitle & 0.0mm  & 0.0mm  & 0.0mm & 0.0pt & 0.0pt & 0.0pt \\
middle      & 2.0mm  & 0.75mm & 0.5mm & 1.0pt & 0.2pt & 0.0pt \\
arc         & 1.0mm  & 0.75mm & 0.5mm & 1.0pt & 0.0pt & 0.0pt \\
outer arc   & auto   & auto   & auto  & auto  & 0.0pt & 0.0pt \\
\end{tcolorbox}
\end{docTcbKey}


  

% \clearpage
\begin{docTcbKey}{oversize}{\colOpt{=\meta{length}}}{style, default |0pt|}
Sets the text width of the upper part to the current line width plus an
optional \meta{length}.
This is achieved by changing the keys \refKeyLe{/tcb/width}
\refKeyLe{/tcb/enlarge left by}, and
\refKeyLe{/tcb/enlarge right by} appropriately.
The resulting box is overlapping into the left and right margin of
the page.
Note that this style option has to be given \emph{after} all other
geometry keys!
Also see \refKeyLe{/tcb/grow sidewards by} and \refKeyLe{/tcb/spread sidewards}.

将upper部分的文本宽度设置为当前行宽再加上可选的\meta{length}。这是通过适当地更改键\refKeyLe{/tcb/width}、\refKeyLe{/tcb/enlarge left by}和\refKeyLe{/tcb/enlarge right by}来实现的。结果的框重叠到页面的左右边距上。请注意,这个样式选项必须在所有其他几何键之后给出!还请参见\refKeyLe{/tcb/grow sidewards by}和\refKeyLe{/tcb/spread sidewards}。

% 设置upper部分的文本的宽度为上下文中行宽加上 \meta{length}。这个效果是通过适当的改变 \refKeyLe{/tcb/width} \refKeyLe{/tcb/enlarge left by}, 和 \refKeyLe{/tcb/enlarge right by} 实现的。
% 最终的盒子会向左和右侧的边注伸展。注意,这个选项应该放置在所有其他的尺寸选项\emph{之后}!
% 另见 \refKeyLe{/tcb/grow sidewards by} 和 \refKeyLe{/tcb/spread sidewards}.
\begin{dispListing}
\tcbset{colback=red!5!white,colframe=red!75!black,fonttitle=\bfseries}

\textit{用于比较的普通文本:}\\
\lipsum[2]

\begin{tcolorbox}[oversize,title=Oversized box]
\lipsum[2]
\end{tcolorbox}

\begin{tcolorbox}[title=Normal box]
\lipsum[2]
\end{tcolorbox}
\end{dispListing}
\end{docTcbKey}

{\tcbusetemp}

  
\setcounter{section}{4}
\setcounter{subsection}{7}
\setcounter{subsubsection}{5}
% % \clearpage
% % \hfill 
% \subsubsection{Toggle Left and Right\\左右设置切换}
% \begin{docTcbKey}[][doc updated=2017-02-16]{toggle left and right}{=\meta{toggle preset}}{default |evenpage|, initially |none|}
% According to the \meta{toggle preset}, the left and the right settings
% of the |tcolorbox| are switched or not. Feasible values are:

% 根据 \meta{toggle preset}, |tcolorbox| 的左右设置是否对换。 可选的值有:
%   \begin{DescriptionL}{\docValue{evenpage}}
%   \item[\docValue{none}]no switching.
% 不对换。
%   \item[\docValue{forced}]the values of the left and right rules, spaces, and corners are switched.
% 左右的线、距离和四个角的设置互换。
%   \item[\docValue{evenpage}]
%   if the page is an even page, the values of the left and
%     right rules, spaces, and corners are switched. This value also sets
%     \refKeyLe{/tcb/check odd page} to |true|.
% 偶数页的左右线条数值,距离和四个角的设置互换。 此设置同时将 \refKeyLe{/tcb/check odd page} 设置为 |true|.
%   \end{DescriptionL}
% \begin{marker}
% Horizontal bounding box enlargements are not toggled by this option.
% They can be toggled independently by \refKeyLe{/tcb/toggle enlargement}.
% For example, \refKeyLe{/tcb/oversize} changes the bounding box.

% 此选项不会将盒子的水平边框放大。
% 它们可以通过 \refKeyLe{/tcb/toggle enlargement} 独立地进行切换。
% 例如, \refKeyLe{/tcb/oversize} 更改盒子的的边界框。
% \end{marker}
% \begin{dispListing}
% % \usepackage{lipsum}
% % \usetikzlibrary{patterns}
% % \tcbuselibrary{skins,breakable}
% \begin{tcolorbox}[enhanced,% 启用增强功能,支持更多的选项。
% breakable,%当盒子过长时,可以自动分页。
% toggle left and right,%使盒子可以在左右两侧切换。
% sharp corners,% 使盒子的角变得尖锐。
% boxrule=0mm,top=0mm,bottom=0mm,left=1mm,right=1mm,
% rightrule=1cm,colupper=blue!25!black,
% interior style={fill overzoom image=lichtspiel.jpg,fill image opacity=0.25},
% frame style={pattern=crosshatch dots light steel blue},
% overlay={%定义要在盒子上覆盖的内容,这里是一个填充球和一个交叉标记。
% \begin{tcbclipframe}
% \tcbifoddpage{\coordinate (X) at ([xshift=-5mm]frame.east);}
%             {\coordinate (X) at ([xshift=5mm]frame.west);}
% \fill[shading=ball,ball color=blue!50!white,opacity=0.5] (X) circle (4mm);
% \end{tcbclipframe}}]
% \lipsum[1-6]
% \end{tcolorbox}
% \end{dispListing}
% \medskip

% % 这是一个tcolorbox环境,会在页面上创建一个带有填充图片和交叉标记的盒子。


% % boxrule,top,bottom,left,right:控制盒子边框的宽度和位置。
% % rightrule:在盒子右侧添加一条宽度为1厘米的边框。
% % colupper:定义盒子中的文本颜色。
% % interior style:定义盒子内部的样式,包括填充图片和填充透明度。
% % frame style:定义盒子边框的样式,包括交叉标记和颜色。
% % overlay:定义要在盒子上覆盖的内容,这里是一个填充球和一个交叉标记。
% % 整个盒子中放置了一个lipsum段落,用于测试盒子的分页效果。

% This example switches a |1cm| thick rule from the left to the right side
% depending on the page number. Thereby, the rule is always on the outer side
% of the double-sided paper. Additionally, a ball is drawn on the outer side
% with help of an overlay.

% 这个例子根据页面编号将厚度为|1cm|的标尺从左侧移到右侧。因此,该标尺始终位于双面纸的外侧。此外,通过叠加层在外侧绘制一个球。

% % 此示例根据页码将 |1cm| 宽的线条从左边切换到右边。
% % 因此,线条总是在双面纸的外侧。
% % 此外,一个球形,通过 overlay 绘制在外侧。
% \bigskip

% \tcbusetemp
% \end{docTcbKey}

