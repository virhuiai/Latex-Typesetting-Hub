\setcounter{section}{4}
\setcounter{subsection}{19}
\setcounter{subsubsection}{0}

\subsection{\texttt{\textbackslash tcbox} Specials\\tcbox特殊选项}

The following options are applicable for \refComLe{tcbox} and \refComLe{tcboxmath}
only.

以下选项仅适用于\refComLe{tcbox}和\refComLe{tcboxmath}。

\begin{docTcbKey}{tcbox raise}{=\meta{length}}{no default, initially \texttt{0pt}}
Raises the \refComLe{tcbox} by the given \meta{length}.

将 \refComLe{tcbox} 上移指定的高度 \meta{length}。

Sets the line width of the right rule to \meta{length}.
\begin{exdispExample}{tcbox_raise}
\tcbset{colframe=blue!50!black,colback=white,colupper=red!50!black,
    fonttitle=\bfseries,nobeforeafter,center title}

Test\dotfill
\tcbox[tcbox raise base]{tcbox raise base}\dotfill
\tcbox{Hello World 2}\dotfill
\tcbox[tcbox raise=5mm]{tcbox raise=5mm}
\end{exdispExample}
\end{docTcbKey}


\begin{docTcbKey}{tcbox raise base}{}{style, no value, initially unset}
Raises the \refComLe{tcbox} such that the base of its content matches
the base of the environmental line; see example above.

上移 \refComLe{tcbox} ,使盒子内容的基线匹配所在环境的基线对齐;请参见上面的示例。
% 与环境行的基础相匹配; 

\begin{引述之言}{virhuiai}
可以理解为,外部的基线,默认同tcolorbox盒子的底对齐,设置了tcbox raise base后,对齐上升到同tcolorbox盒子的文本的基线了。
\end{引述之言}
\end{docTcbKey}

\begin{docTcbKey}{on line}{}{style, no value, initially unset}
Combines \refKeyLe{/tcb/tcbox raise base} with \refKeyLe{/tcb/nobeforeafter}.
The resulting box behaves analogue to |\fbox|.

%   analogue
% 美 ['ænə.lɔɡ]
% 英 ['ænə.lɒɡ]
% adj.模拟的;指针式的
% n.相似物;类似事情
% 网络类似物;类比;同源语
组合 \refKeyLe{/tcb/tcbox raise base} 和 \refKeyLe{/tcb/nobeforeafter}.
得到的盒子的行为类似于|\fbox|.
\end{docTcbKey}




% \clearpage
\begin{docTcbKey}[][doc new=2015-03-23]{tcbox width}{=\meta{mode}}{no default, initially \texttt{auto}}
Controls how \refComLe{tcbox} respects a \refKeyLe{/tcb/width} setting.
Feasible values for \meta{mode} are:

控制\refComLe{tcbox}对宽度参数\refKeyLe{/tcb/width}的处理。%
\meta{mode}可以选择的值有:
\begin{itemize}
\item\docValue{auto} 
% (initial setting):
%   ignore \refKeyLe{/tcb/width} and set box width according to its content.
(初始设定) :
忽略\refKeyLe{/tcb/width},根据盒子内容设置宽度。
\item\docValue{auto limited}:
% Set box width according to its content, if it is smaller than \refKeyLe{/tcb/width}.
% Otherwise, the content is set like in a \refEnvLe{tcolorbox} with line breaks.
如果盒子内容的宽度小于\refKeyLe{/tcb/width},则据内容设置盒子的宽度。%
否则,盒子内的效果类似于可换行的\refEnvLe{tcolorbox}。
\item\docValue{forced center}:
% Set box width according to \refKeyLe{/tcb/width}.
% The content is centered and may overlap the box borders.
将盒子的宽度设置为\refKeyLe{/tcb/width}。%
内容居中,可能与盒子两侧重叠。
\item\docValue{forced left}:
% Set box width according to \refKeyLe{/tcb/width}.
% The content is left aligned and may overlap the box borders.
将盒子的宽度设置为\refKeyLe{/tcb/width}。%
内容居{\bf 左},可能与盒子两侧重叠。
\item\docValue{forced right}:
% Set box width according to \refKeyLe{/tcb/width}.
% The content is right aligned and may overlap the box borders.
将盒子的宽度设置为\refKeyLe{/tcb/width}。%
内容居{\bf 右},可能与盒子两侧重叠。
\item\docValue{minimum center}:
% Set box width according to \refKeyLe{/tcb/width}, if the content fits into.
% The content is centered and the box width may grow beyond \refKeyLe{/tcb/width}.
如果内容合适,将盒子的宽度设置为\refKeyLe{/tcb/width}。%
内容是居中的,盒宽可能超出\refKeyLe{/tcb/width}。
\item\docValue{minimum left}:
% Set box width according to \refKeyLe{/tcb/width}, if the content fits into.
% The content is left aligned and the box width may grow beyond \refKeyLe{/tcb/width}.
如果内容合适,将盒子的宽度设置为\refKeyLe{/tcb/width}。%
内容是居{\bf 左}的,盒宽可能超出\refKeyLe{/tcb/width}。
\item\docValue{minimum right}:
如果内容合适,将盒子的宽度设置为\refKeyLe{/tcb/width}。%
内容是居{\bf 右}的,盒宽可能超出\refKeyLe{/tcb/width}。
\end{itemize}


% \enlargethispage*{1cm}

\begin{exdispExample}{tcbox_width}
\tcbset{size=small,on line,before upper=\strut,
colframe=blue!75!black,colback=blue!5!white,
fontupper=\normalsize,width=4cm}

\tcbox[tcbox width=auto]{auto}\qquad
\tcbox[tcbox width=auto limited]{auto limited}\qquad
\tcbox[tcbox width=auto limited]{auto limited遇上长文本}\\
\tcbox[tcbox width=forced center]{forced center}\qquad
\tcbox[tcbox width=forced center]{forced center with long text}\\
\tcbox[tcbox width=forced left]{forced left}\qquad
\tcbox[tcbox width=forced left]{forced left with long text}\\
\tcbox[tcbox width=forced right]{forced right}\qquad
\tcbox[tcbox width=forced right]{forced right with long text}\\
\tcbox[tcbox width=minimum center]{minimum center}\qquad
\tcbox[tcbox width=minimum center]{minimum center with long text}\\
\tcbox[tcbox width=minimum left]{minimum left}\qquad
\tcbox[tcbox width=minimum left]{minimum left with long text}\\
\tcbox[tcbox width=minimum right]{minimum right}\qquad
\tcbox[tcbox width=minimum right]{minimum right with long text}
\end{exdispExample}
\end{docTcbKey}