\setcounter{section}{4}
\setcounter{subsection}{5}
\setcounter{subsubsection}{0}
 
\subsection{Text Alignment文本对齐}
\begin{docTcbKey}[][doc new=2015-05-07]{halign}{=\meta{alignment}}{no default, initially \texttt{justify}}
If there is no lower part, |halign| determines the horizontal \meta{alignment}
of the text content.
Otherwise, |halign| determines the horizontal \meta{alignment}
of the upper part of the box only.
The feasible values for \meta{alignment} are more or less identical to
the corresponding |/tikz/align| settings, even if the implementation differs.

确定盒子upper部分的水平\meta{alignment}方式。可行的\meta{alignment}值与相应的|/tikz/align|设置几乎相同,即使实现不同。

% 如果没有lower部分, |halign| 决定着水平的文本内容的对齐方式为 \meta{alignment}。
% 否则, |halign| 的 \meta{alignment} 只影响到upper部分的水平对齐。
% |halign| 决定着upper部分文本内容的水平对齐方式为 \meta{alignment}。
% \meta{alignment} 可选的值有不少同 |/tikz/align| 的相应设置是一样的, 即使实现有所不同。
\begin{DescriptionR}{\docValue{flush center}}
\item[\docValue{justify}] usual left and right justified type setting.
\par 常用的,左右对齐%的排版设置(两端对齐)。
\item[\docValue{left}]left border justification in analogy to plain \TeX.
\par 类似于plain \TeX 的左边界对齐。
% 向着盒子的左边框对齐,类似于 plain \TeX。
\item[\docValue{flush left}]
left border justification with |\raggedright| of \LaTeX.
\par 向着盒子的左边框对齐。%,使用 \LaTeX 的 |\raggedright|\footnote{ragged是不整齐的意思}。
\item[\docValue{right}]right border justification in analogy to plain \TeX.
\par 向着盒子的右边框对齐。%,类似于 plain \TeX。
\item[\docValue{flush right}]right border justification with |\raggedleft| of \LaTeX.
\par 向着盒子的右边框对齐。%使用 \LaTeX 的 |\raggedleft|。
\item[\docValue{center}]centering in analogy to plain \TeX.
\par 居中对齐.%,使用 plain \TeX。
\item[\docValue{flush center}]centering with |\centering| of \LaTeX.
\par 居中对齐。%,使用 \LaTeX 的 |\centering|。
\end{DescriptionR}
The differences between the flush and non-flush version are explained in
detail in the \tikzname\ manual \cite{tantau:tikz_and_pgf}. The short story is that
the non-flush versions will often look more balanced but with more
hyphenations.

% 在 \tikzname\ 手册 %\cite{tantau:tikz_and_pgf} 
% 中详细介绍了 flush 和 non-flush 版本之间的区别。简而言之 non-flush 版本通常看起来更加平衡,但是有更多的连字符。

在\tikzname\ 手册\cite{tantau:tikz_and_pgf}中详细解释了flush版本和non-flush版本之间的区别。简短的说,non-flush版本通常看起来更平衡,但会有更多的连字。

% \begin{tcolorbox}[adjusted title=flush center,halign=flush center]
%   This is a demonstration text for showing how line breaking works.
%   \end{tcolorbox}
%   \begin{tcolorbox}[adjusted title=flush left,halign=flush left]
%   This is a demonstration text for showing how line breaking works.
%   \end{tcolorbox}
%   \begin{tcolorbox}[adjusted title=flush right,halign=flush right]
%   This is a demonstration text for showing how line breaking works.
%   \end{tcolorbox}
  
%   \begin{tcolorbox}[adjusted title=center,halign=center]
%   This is a demonstration text for showing how line breaking works.
%   \end{tcolorbox}
%   \begin{tcolorbox}[adjusted title=left,halign=left]
%   This is a demonstration text for showing how line breaking works.
%   \end{tcolorbox}
%   \begin{tcolorbox}[adjusted title=right,halign upper=right]
%   This is a demonstration text for showing how line breaking works.
%   \end{tcolorbox}
  

\begin{exdispExample}{halign}
\tcbset{colback=red!5!white,colframe=red!75!black,size=small,
fonttitle=\bfseries,width=3.5cm,box align=top,
nobeforeafter}

\foreach \p in {flush center,flush left,flush right}
{\begin{tcolorbox}[adjusted title=\p,halign=\p]
This is a demonstration text for showing how line breaking works.
\end{tcolorbox}
}

\foreach \q in {center, left, right}
{\begin{tcolorbox}[adjusted title=\q,halign=\q]
This is a demonstration text for showing how line breaking works.
\end{tcolorbox}
} 

\end{exdispExample}
\end{docTcbKey}



\begin{docTcbKey}[][doc new=2015-05-07]{halign upper}{=\meta{alignment}}{no default, initially \texttt{justify}}
% Alias for \refKeyLe{/tcb/halign}.

\refKeyLe{/tcb/halign} 的别名
\end{docTcbKey}



% \newpage
\begin{docTcbKey}[][doc new=2015-05-07]{halign lower}{=\meta{alignment}}{no default, initially \texttt{justify}}
|halign lower| determines the horizontal \meta{alignment} of the lower part of the box.
The feasible values for \meta{alignment} are the same as for \refKeyLe{/tcb/halign}.

% |halign lower| 控制着盒子的lower部分的内容的水平对齐方式为 \meta{alignment}。\meta{alignment} 的可选值同 \refKeyLe{/tcb/halign} 一样。

|halign lower| 确定盒子lower部分的水平 \meta{alignment}。 \meta{alignment} 的可行值与 \refKeyLe{/tcb/halign} 相同。
\begin{exdispExample}{halign_lower}
\begin{tcbraster}[raster columns=3,fonttitle=\bfseries,
colback=red!5!white,colframe=red!75!black]

\begin{tcolorbox}[adjusted title=flush center,halign lower=flush center]
Upper part. \tcblower Lower part.
\end{tcolorbox}
\begin{tcolorbox}[adjusted title=flush left,halign lower=flush left]
Upper part. \tcblower Lower part.
\end{tcolorbox}
\begin{tcolorbox}[adjusted title=flush right,halign lower=flush right]
Upper part. \tcblower Lower part.
\end{tcolorbox}
\begin{tcolorbox}[adjusted title=center,halign lower=center]
Upper part. \tcblower Lower part.
\end{tcolorbox}
\begin{tcolorbox}[adjusted title=left,halign lower=left]
Upper part. \tcblower Lower part.
\end{tcolorbox}
\begin{tcolorbox}[adjusted title=right,halign lower=right]
Upper part. \tcblower Lower part.
\end{tcolorbox}

\end{tcbraster}
\end{exdispExample}
\end{docTcbKey}






% \clearpage
%这儿原来写错了
\begin{docTcbKey}[][doc new=2022-10-30]{halign title}{=\meta{alignment}}{no default, initially \texttt{justify}}
|halign title| determines the horizontal \meta{alignment} of the title of the box.
The feasible values for \meta{alignment} are the same as for \refKeyLe{/tcb/halign}.

|halign title| 设置盒子的标题部分的对齐方式为 \meta{alignment}。
\meta{alignment} 的可选值同 \refKeyLe{/tcb/halign} 一样。

\begin{exdispExample}{halign_title}
\begin{tcbraster}[raster columns=3,fonttitle=\bfseries,
colback=red!5!white,colframe=red!75!black]

\begin{tcolorbox}[adjusted title=flush center,halign title=flush center]
This is a \textbf{tcolorbox}.
\end{tcolorbox}
\begin{tcolorbox}[adjusted title=flush left,halign title=flush left]
This is a \textbf{tcolorbox}.
\end{tcolorbox}
\begin{tcolorbox}[adjusted title=flush right,halign title=flush right]
This is a \textbf{tcolorbox}.
\end{tcolorbox}
\begin{tcolorbox}[adjusted title=center,halign title=center]
This is a \textbf{tcolorbox}.
\end{tcolorbox}
\begin{tcolorbox}[adjusted title=left,halign title=left]
This is a \textbf{tcolorbox}.
\end{tcolorbox}
\begin{tcolorbox}[adjusted title=right,halign title=right]
This is a \textbf{tcolorbox}.
\end{tcolorbox}

\end{tcbraster}
\end{exdispExample}
\end{docTcbKey}




% \enlargethispage*{1cm}

\begin{docTcbKey}[][doc updated=2015-05-07]{flushleft upper}{}{style, no value}
Shortcut for setting \refKeyLe{/tcb/halign} to \docValue{flush left}.
将 \refKeyLe{/tcb/halign} 设置为 \docValue{flush left} 的简写形式。
\end{docTcbKey}

\begin{docTcbKey}[][doc updated=2015-05-07]{center upper}{}{style, no value}
Shortcut for setting \refKeyLe{/tcb/halign} to \docValue{flush center}.

将 \refKeyLe{/tcb/halign} 设置为 \docValue{flush center} 的简写形式。
\end{docTcbKey}

\begin{docTcbKey}[][doc updated=2015-05-07]{flushright upper}{}{style, no value}
Shortcut for setting \refKeyLe{/tcb/halign} to \docValue{flush right}.

将 \refKeyLe{/tcb/halign} 设置为 \docValue{flush right} 的简写形式。
\end{docTcbKey}

\begin{docTcbKey}[][doc updated=2015-05-07]{flushleft lower}{}{style, no value}
Shortcut for setting \refKeyLe{/tcb/halign lower} to \docValue{flush left}.

将 \refKeyLe{/tcb/halign lower} 设置为 \docValue{flush left} 的简写形式。
\end{docTcbKey}

\begin{docTcbKey}[][doc updated=2015-05-07]{center lower}{}{style, no value}
Shortcut for setting \refKeyLe{/tcb/halign lower} to \docValue{flush center}.
将 \refKeyLe{/tcb/halign lower} 设置为 \docValue{flush center} 的简写形式。
\end{docTcbKey}

\begin{docTcbKey}[][doc updated=2015-05-07]{flushright lower}{}{style, no value}
Shortcut for setting \refKeyLe{/tcb/halign lower} to \docValue{flush right}.

将 \refKeyLe{/tcb/halign lower} 设置为 \docValue{flush right} 的简写形式。
\end{docTcbKey}



% \clearpage

\begin{docTcbKey}[][doc updated=2015-05-07]{flushleft title}{}{style, no value}
Shortcut for setting \refKeyLe{/tcb/halign title} to \docValue{flush left}.

将 \refKeyLe{/tcb/halign title} 设置为 \docValue{flush left} 的简写形式。
\end{docTcbKey}

\begin{docTcbKey}[][doc updated=2015-05-07]{center title}{}{style, no value}
Shortcut for setting \refKeyLe{/tcb/halign title} to \docValue{flush center}.

将 \refKeyLe{/tcb/halign title} 设置主 \docValue{flush center} 的简写形式。
\end{docTcbKey}

\begin{docTcbKey}[][doc updated=2015-05-07]{flushright title}{}{style, no value}
Shortcut for setting \refKeyLe{/tcb/halign title} to \docValue{flush right}.

将 \refKeyLe{/tcb/halign title} 设置为 \docValue{flush right} 的简写形式。
\end{docTcbKey}


\begin{marker}
The vertical alignment settings are only relevant for boxes which are larger
than their natural height, see \Fullref{sec:heightcontrol}.

垂直对齐设置只适用于大于其自然高度的盒子。见 \Fullref{sec:heightcontrol}.
\end{marker}

\begin{docTcbKey}[][doc updated=2015-07-16]{valign}{=\meta{alignment}}{no default, initially |top|}
If the height of a |tcolorbox| is not the natural height, |valign|
determines the vertical \meta{alignment} of the upper part.
Feasible values are

如果一个 |tcolorbox| 的盒子的高度不是其自然高度\footnote{译注:指定了高度}, |valign| 控制着盒子的upper部分的对方方式 \meta{alignment} 。
可选的值有:
\begin{itemize}
\item\docValue{top}: %Anchor text at top.
顶部对齐。
\item\docValue{center}: %Anchor text at center.
中间对齐。
\item\docValue{bottom}:% Anchor text at bottom.
底部对齐。
\item\docValue{scale}: 
Scale text vertically to fit into the available space.
  This is brutal and may not look very good. Consider \Fullref{sec:fitting}
  alternatively.
垂直缩放文本以适应可用空间。
这简单粗暴,可能看起来不是很好。或者考虑下 \Fullref{sec:fitting}。


\item\docValue{scale*}: 
Like \docValue{scale}, but scaling is bounded by
  \refKeyLe{/tcb/valign scale limit}.

类似于\docValue{scale}, 但缩放范围受限于 \refKeyLe{/tcb/valign scale limit}.
\end{itemize}
For a box with natural height, these settings are meaningless.

对于具有自然高度的盒子,这些设置毫无意义。
\begin{exdispExample}{valign}
\tcbset{width=(\linewidth-2mm)/4,before=,after=\hfill,
colframe=blue!75!black,colback=white,height=2cm}

\foreach \myalign in {top,center,bottom,scale}
{\begin{tcolorbox}[valign=\myalign]
This is a \textbf{tcolorbox}.
\end{tcolorbox}}
\end{exdispExample}
\end{docTcbKey}






\begin{docTcbKey}[][doc new=2015-05-07]{valign upper}{=\meta{alignment}}{no default, initially \texttt{top}}
Alias for \refKeyLe{/tcb/valign}.

\refKeyLe{/tcb/valign} 的别名。
\end{docTcbKey}

\begin{docTcbKey}{valign lower}{=\meta{alignment}}{no default, initially |top|}
This key has the same meaning for the lower part as |valign|
for the upper part, i.\,e., it determines
the vertical \meta{alignment} of the lower part with feasible values
|top|, |center|, |bottom|, |scale|, and |scale*|.

此项设置含义同upper部分的 |valign| 相同, i.\,e., 它指定了lower部分的竖直方向的对齐方式 \meta{alignment} ,可选的值有 |top|, |center|, |bottom|, |scale|, 和 |scale*|.

\end{docTcbKey}

\begin{docTcbKey}[][doc new=2015-07-16]{valign scale limit}{=\meta{real number}}{no default, initially \texttt{1.1}}
Sets an upper scale limit for the \docValue{scale*} setting in
\refKeyLe{/tcb/valign} and \refKeyLe{/tcb/valign lower}.
Note that this value is not reset by \refKeyLe{/tcb/reset}. So, changes
also apply to embedded boxes.

设置 \docValue{scale*} 在指定 \refKeyLe{/tcb/valign} 和 \refKeyLe{/tcb/valign lower} 时缩放的范围上限。注意,此值不会随着 \refKeyLe{/tcb/reset} 而重置。所以,修改的话也会同时在嵌套的盒子中生效。
\end{docTcbKey}

Also see \refKeyLe{/tcb/sidebyside align} for alignment settings when
upper part and lower part are set side-by-side.

另见 \refKeyLe{/tcb/sidebyside align} 了解当设置为side-by-side左右排布时的对齐选项。