The following options introduce some arbitrary \meta{code} to the content
of a |tcolorbox|. These additions can be given at the beginning or at the ending
of the title, the upper part, or the lower part.

下面的选项向 |tcolorbox| 的内容中添加一些任意的 \meta{code}。这些添加可以在标题、上部分或下部分的开头或结尾给出。

% 下面的选项介绍将一引动任意的 \meta{code} 附加到 |tcolorbox| 的内容中。可以附加在标题、upper部分或lower部分的开头或结尾。

\begin{docTcbKey}{before title}{=\meta{code}}{no default, initially unset}
The given \meta{code} is placed \emph{after} the color and font settings
and \emph{before} the content of the title.

给出的\meta{code}被放置在标题的颜色和字体设置\emph{之后} 、内容\emph{之前}。
\begin{exdispExample}{before_title}
\tcbset{before title={\textcolor{yellow}{\large Important:}~},
colback=red!5!white,colframe=red!75!black,fonttitle=\bfseries}

\begin{tcolorbox}[title=My title]
This is a \textbf{tcolorbox}.
\end{tcolorbox}
\end{exdispExample}
\end{docTcbKey}


\begin{docTcbKey}{after title}{=\meta{code}}{no default, initially unset}
The given \meta{code} is placed \emph{after} the content of the title.

给出的\meta{code}被放置在标题的内容\emph{之后}。
\begin{exdispExample}{after_title}
\tcbset{after title={\hfill\colorbox{Navy}{approved}},
colback=red!5!white,colframe=red!75!black,fonttitle=\bfseries}

\begin{tcolorbox}[title=My title]
This is a \textbf{tcolorbox}.
\end{tcolorbox}
\end{exdispExample}
\end{docTcbKey}




%\clearpage
\begin{docTcbKey}{before upper}{=\meta{code}}{no default, initially empty}
The given \meta{code} is placed \emph{after} the color and font settings
and \emph{before} the content of the upper part.
The \meta{code} is appended by a final |\ignorespaces|.

给出的\meta{code}被放置在upper部分的颜色和字体设置\emph{之后}、内容\emph{之前}。
\meta{code}会被附加一个最后的|\ignorespaces|。

\begin{exdispExample}{before_upper}
\tcbset{before upper={\textit{The story:}\par},
colback=red!5!white,colframe=red!75!black,fonttitle=\bfseries}

\begin{tcolorbox}[title=My title]
This is a \textbf{tcolorbox}.
\end{tcolorbox}
\end{exdispExample}
\end{docTcbKey}


\begin{docTcbKey}[][doc new=2019-02-26]{before upper*}{=\meta{code}}{no default, initially unset}
The given \meta{code} is placed \emph{after} the color and font settings
and \emph{before} the content of the upper part.
In contrast to \refKey{/tcb/before upper}, no |\ignorespaces| is appended.
Use this for situations where |\ignorespaces| is not needed or causes harm.

给出的\meta{code}被放置在upper部分的颜色和字体设置\emph{之后}、内容\emph{之前}。
同\refKey{/tcb/before upper}对比,不附加|\ignorespaces|。
当不需要|\ignorespaces|或|\ignorespaces|会导致问题时使用此项。

\begin{exdispExample}{before_upper_star}
\begin{tcolorbox}[size=small,tile,
colback=yellow!20,colbacktitle=yellow!70!black,
title=My table,hbox,center,center title,
before upper*=\begin{tabular}{cc},
after upper*=\end{tabular},
]
\multicolumn{2}{c}{Title}\\
one & two \\
three & four\\
\end{tcolorbox}
\end{exdispExample}
\end{docTcbKey}

%\clearpage





\begin{docTcbKey}[][doc updated=2016-10-21]{after upper}{=\meta{code}}{no default, initially empty}
The given \meta{code} is placed \emph{after} the content of the upper part.
The \meta{code} is prepended by a leading |\unskip|.

给出的\meta{code}会被放置到upper部分的内容\emph{之后}。
\meta{code}之前会插入|\unskip|。

\begin{exdispExample}{after_upper_1}
\tcbset{after upper={\par\hfill\textit{Read more next week}},
colback=red!5!white,colframe=red!75!black,fonttitle=\bfseries}

\begin{tcolorbox}[title=My title]
This is a \textbf{tcolorbox}.
\end{tcolorbox}

\begin{tcolorbox}[after upper={\par\hfill---\textit{王勃}}]
穷且益坚,不坠青云之志。
\end{tcolorbox}
\end{exdispExample}

\begin{exdispExample}{after_upper_2}
\begin{tcolorbox}[before upper=\flqq,after upper=\frqq,
colback=red!5!white,colframe=red!75!black]
This is a \textbf{tcolorbox}.\footnote{译注:想到可以用命令行盒子,加这个书名号的效果!}
\end{tcolorbox}
\end{exdispExample}
\end{docTcbKey}




\begin{docTcbKey}[][doc new and updated={2016-10-21}{2019-02-28}]{after upper*}{=\meta{code}}{no default, initially unset}
The given \meta{code} is placed \emph{after} the content of the upper part.
In contrast to \refKey{/tcb/after upper}, no |\unskip| is prepended.
Use this for situations where |\unskip| is not needed or causes harm.
See \refKey{/tcb/before upper*} for an example.

给出的\meta{code}会被放置到upper部分的内容\emph{之后}。%
同\refKey{/tcb/after upper}相比,没有在前附加|\unskip|。%
当不需要 |\unskip| 或 |\unskip| 会导致问题时使用此项。%
例见 \refKey{/tcb/before upper*}。

\begin{marker}
从版本 3.80 到 3.94, 此项会将 |\unskip| 添加到 \meta{code} 之前。\\
版本 3.95 到 4.15, 此项不建议使用。\\
版本 4.20 起, 这个选项是用改变的语义重新建立的 (没有 |\unskip|!)
\end{marker}
\end{docTcbKey}





%\clearpage
\begin{docTcbKey}{before lower}{=\meta{code}}{no default, initially empty}
The given \meta{code} is placed \emph{after} the color and font settings
and \emph{before} the content of the lower part.
The \meta{code} is appended by a final |\ignorespaces|.

将 \meta{code} 放到lower部分的颜色和字段设置\emph{之后}、内容\emph{之前}。
\meta{code} 之后附加一个|\ignorespaces|。
\begin{exdispExample}{before_lower}
\tcbset{before lower=\textit{Behold:~},colback=red!5!white,colframe=red!75!black}

\begin{tcolorbox}
This is a \textbf{tcolorbox}.
\tcblower
This is the lower part.
\end{tcolorbox}
\end{exdispExample}
\end{docTcbKey}


\begin{docTcbKey}[][doc new=2019-02-26]{before lower*}{=\meta{code}}{no default, initially unset}
The given \meta{code} is placed \emph{after} the color and font settings
and \emph{before} the content of the lower part.
In contrast to \refKey{/tcb/before lower}, no |\ignorespaces| is appended.
Use this for situations where |\ignorespaces| is not needed or causes harm.

\meta{code} 放置到lower部分的颜色和字体设置\emph{之后}、内容\emph{之前}。
同 \refKey{/tcb/before lower} 相比, 尾部不会附加 |\ignorespaces|。
当不需要 |\ignorespaces| 或它会导致问题时使用此项。
\begin{exdispExample}{before_lower_star}
\begin{tcolorbox}[size=small,bicolor,sidebyside,center lower,
colback=yellow!30,colbacklower=yellow!20,colframe=yellow!80!black,
before lower*=\begin{tabular}{cc},
after lower*=\end{tabular},
]
My table
\tcblower
\multicolumn{2}{c}{Title}\\
one & two \\
three & four\\
\end{tcolorbox}
\end{exdispExample}
\end{docTcbKey}





%\clearpage

\begin{docTcbKey}[][doc updated=2016-10-21]{after lower}{=\meta{code}}{no default, initially empty}
The given \meta{code} is placed \emph{after} the content of the lower part.
The \meta{code} is prepended by a leading |\unskip|.

\meta{code} 放置到lower部分的内容\emph{之后}。
\meta{code} 之前会插入 |\unskip|。

\begin{exdispExample}{after_lower_2}
\begin{tcolorbox}[after lower=\ \textit{This is the end.},
colback=red!5!white,colframe=red!75!black]
This is a \textbf{tcolorbox}.
\tcblower
This is the lower part.
\end{tcolorbox}
\end{exdispExample}
\end{docTcbKey}


\begin{docTcbKey}[][doc new and updated={2016-10-21}{2019-02-28}]{after lower*}{=\meta{code}}{no default, initially unset}
The given \meta{code} is placed \emph{after} the content of the lower part.
In contrast to \refKey{/tcb/after upper}, no |\unskip| is prepended.
Use this for situations where |\unskip| is not needed or causes harm.

\meta{code} 放置到lower部分的内容\emph{之后}。
同 \refKey{/tcb/after upper} 相比, 不会在头部附加 |\unskip|。
当不需要 |\unskip| 或它会导致问题时使用此项。

\begin{exdispExample}{after_lower_1}
\begin{tcolorbox}[before lower*=$,after lower*=$,
colback=red!5!white,colframe=red!75!black]
This is a \textbf{tcolorbox}.
\tcblower
\sin^2(x)+\cos^2(x)=1.
\end{tcolorbox}
\end{exdispExample}

\begin{marker}
From version 3.80 to 3.94, this option prepended an |\unskip| to the given \meta{code}.\\
From version 3.95 to 4.15, this option was deprecated.\\
From version 4.20, this option is re-established with changed semantic (no |\unskip|!)
\end{marker}
\end{docTcbKey}



%\clearpage
\begin{marker}
If \refKey{/tcb/text fill} is used, one cannot have a lower part
and the box is unbreakable.
如果使用了 \refKey{/tcb/text fill} , 则没有 lower 部分,且盒子是不可分割的。
\end{marker}

\begin{docTcbKey}[][doc new=2015-07-15]{text fill}{}{style, no value}
This style sets \refKey{/tcb/before upper} and \refKey{/tcb/after upper}
to embed the upper part with a minipage. If a fixed height was applied
e.g.\ by \refKey{/tcb/height} or \refKey{/tcb/height fill}, this minipage
gets a matching height. This allows to use vertical glue macros like
|\vfill| to act like expected. If the box has no fixed height,
setting \refKey{/tcb/text fill} has no other effect as making the box
unbreakable. 

此项设置 \refKey{/tcb/before upper} 和 \refKey{/tcb/after upper}
以将upper部分包围在 minipage 环境中。 如果高度指定为固定的值
e.g.\ 使用 \refKey{/tcb/height} 或 \refKey{/tcb/height fill}, 此 minipage 环境得到一个匹配的高度。 这允许我们使用竖直的粘连% glue 
宏,如 |\vfill| 能正常工作。如果盒子没有指定固定的高度,%
设置 \refKey{/tcb/text fill} 没有其他作用,但盒子会变成不可分的。
\begin{exdispExample}{text_fill}
\begin{tcolorbox}[colback=red!5!white,colframe=red!75!black,fonttitle=\bfseries,
height=8cm,text fill,
title=My filled box]
This is a \textbf{tcolorbox}.
\par\vfill
\begin{center}
My middle text.
\end{center}
\par\vfill
This is the end of my box.
\end{tcolorbox}
\end{exdispExample}
\end{docTcbKey}



%\clearpage

\begin{docTcbKey}[][doc new={2019-09-19}]{tabulars}{=\meta{preamble}}{style}
This style sets \refKey{/tcb/before upper} and \refKey{/tcb/after upper}
and several geometry keys to support a |tabular*| with the
given \meta{preamble}.
The packages |array| and |colortbl| have to be loaded separately.

此项设置了 \refKey{/tcb/before upper} 和 \refKey{/tcb/after upper} 和一些命令,使得支持使用 |tabular*| 并在 \meta{preamble} 中指定表格头。
需要分别加载宏包 |array| 和 |colortbl|。
\begin{exdispExample}{tabulars_1}
%%\usepackage{array}
%%\usepackage{colortbl} - or - %%\usepackage[table]{xcolor}
\tcbset{enhanced,fonttitle=\bfseries\large,fontupper=\normalsize\sffamily,
colback=yellow!10!white,colframe=red!50!black,colbacktitle=Salmon!30!white,
coltitle=black,center title}

\begin{tcolorbox}[tabulars={@{\extracolsep{\fill}\hspace{5mm}}lrrrrr@{\hspace{5mm}}},
boxrule=0.5pt,title=My table]
Group & One     & Two     & Three    & Four     & Sum\\\hline\hline
Red   & 1000.00 & 2000.00 &  3000.00 &  4000.00 & 10000.00\\\hline
Green & 2000.00 & 3000.00 &  4000.00 &  5000.00 & 14000.00\\\hline
Blue  & 3000.00 & 4000.00 &  5000.00 &  6000.00 & 18000.00\\\hline\hline
Sum   & 6000.00 & 9000.00 & 12000.00 & 15000.00 & 42000.00
\end{tcolorbox}
\end{exdispExample}
\end{docTcbKey}


\begin{docTcbKey}[][doc new={2019-09-19}]{tabulars*}{=\marg{code}\marg{preamble}}{style}
This is a variant of \refKey{/tcb/tabulars} which adds some \meta{code}
before the table starts.

这是 \refKey{/tcb/tabulars} 的变种,可以附加 \meta{code} 到表格的开头。
before the table starts.
\begin{exdispExample}{tabulars_2}
%%\usepackage{array}
%%\usepackage{colortbl} - or - %%\usepackage[table]{xcolor}
\tcbset{enhanced,fonttitle=\bfseries\large,fontupper=\normalsize\sffamily,
colback=yellow!10!white,colframe=red!50!black,colbacktitle=Salmon!30!white,
coltitle=black,center title}

\begin{tcolorbox}[tabulars*={\arrayrulewidth0.5mm\renewcommand\arraystretch{1.4}}%
{@{\extracolsep{\fill}\hspace{20mm}}lll@{\hspace{20mm}}},
title=My table]
One     & Two     & Three \\\hline\hline
1000.00 & 2000.00 &  3000.00\\\hline
2000.00 & 3000.00 &  4000.00
\end{tcolorbox}
\end{exdispExample}
\end{docTcbKey}




%\clearpage
\begin{marker}
If \refKey{/tcb/tabularx} or \refKey{/tcb/tabularx*} are used, one cannot
have a lower part.

如果使用了 \refKey{/tcb/tabularx} 或 \refKey{/tcb/tabularx*} , 将不会有lower部分。
\end{marker}



\begin{docTcbKey}{tabularx}{=\meta{preamble}}{style}
This style sets \refKey{/tcb/before upper} and \refKey{/tcb/after upper}
and several geometry keys to support a |tabularx| with the
given \meta{preamble}.
The packages |tabularx| \cite {carlisle:tabularx}, |array|, and |colortbl|
have to be loaded separately.

此项设置了 \refKey{/tcb/before upper} 和 \refKey{/tcb/after upper} 以及一些命令以支持 |tabularx| 环境,且可以指定表头为 \meta{preamble}。
需要加载宏包 |tabularx| %\cite {carlisle:tabularx}
, |array|, 和 |colortbl|。
\begin{exdispExample}{tabularx_1}
%%\usepackage{array,tabularx}
%%\usepackage{colortbl} - or - %%\usepackage[table]{xcolor}
\newcolumntype{Y}{>{\raggedleft\arraybackslash}X}% see tabularx
\tcbset{enhanced,fonttitle=\bfseries\large,fontupper=\normalsize\sffamily,
colback=yellow!10!white,colframe=red!50!black,colbacktitle=Salmon!30!white,
coltitle=black,center title}

\begin{tcolorbox}[tabularx={X||Y|Y|Y|Y||Y},title=My table]
Group & One     & Two     & Three    & Four     & Sum\\\hline\hline
Red   & 1000.00 & 2000.00 &  3000.00 &  4000.00 & 10000.00\\\hline
Green & 2000.00 & 3000.00 &  4000.00 &  5000.00 & 14000.00\\\hline
Blue  & 3000.00 & 4000.00 &  5000.00 &  6000.00 & 18000.00\\\hline\hline
Sum   & 6000.00 & 9000.00 & 12000.00 & 15000.00 & 42000.00
\end{tcolorbox}
\end{exdispExample}
\end{docTcbKey}


\begin{docTcbKey}{tabularx*}{=\marg{code}\marg{preamble}}{style}
This is a variant of \refKey{/tcb/tabularx} which adds some \meta{code}
before the table starts.

这是 \refKey{/tcb/tabularx} 的变种,附加 \meta{code} 到表格的开头。
\begin{exdispExample}{tabularx_2}
%%\usepackage{array,tabularx}
%%\usepackage{colortbl} - or - %%\usepackage[table]{xcolor}
\tcbset{enhanced,fonttitle=\bfseries\large,fontupper=\normalsize\sffamily,
colback=yellow!10!white,colframe=red!50!black,colbacktitle=Salmon!30!white,
coltitle=black,center title}

\begin{tcolorbox}[tabularx*={\arrayrulewidth0.5mm}{X|X|X},title=My table]
One     & Two     & Three \\\hline\hline
1000.00 & 2000.00 &  3000.00\\\hline
2000.00 & 3000.00 &  4000.00
\end{tcolorbox}
\end{exdispExample}
\end{docTcbKey}




%\clearpage
\begin{docTcbKey}{tikz upper}{\colOpt{=\meta{options}}}{style}
This style adds a centered |tikzpicture| environment to the start and end
of the upper part. The \meta{options} may be given as \tikzname\  picture options.

% 该样式在上部分的开头和结尾添加了一个居中的|tikzpicture|环境。可以将\meta{options}作为\tikzname\ 图片选项给出。

此项将上部分的内容放入一个 |tikzpicture| 环境。给定的选项 \meta{options} 会传递给 |tikzpicture| 环境。 % \tikzname\  picture
\begin{exdispExample}{tikz_upper}
%%\usepackage{tikz}

\begin{tcolorbox}[tikz upper,fonttitle=\bfseries,colback=white,colframe=black,
            title=\tikzname\ 绘制]
\path[fill=yellow,draw=yellow!75!red] (0,0) circle (1cm);
\fill[red] (45:5mm) circle (1mm);
\fill[red] (135:5mm) circle (1mm);
\draw[line width=1mm,red] (215:5mm) arc (215:325:5mm);
\end{tcolorbox}
\end{exdispExample}
\end{docTcbKey}


\begin{docTcbKey}{tikz lower}{\colOpt{=\meta{options}}}{style}
This style adds a centered |tikzpicture| environment to the start and end
of the lower part. The \meta{options} may be given as \tikzname\  picture options.

此项将lower部分的内容放入一个 |tikzpicture| 环境。给定的选项 \meta{options} 会传递给 |tikzpicture| 环境。% \tikzname\  picture 。
\begin{exdispExample}{tikz_lower}
%%\usepackage{tikz}
%\tcbuselibrary{skins,listings}

\begin{tcblisting}{tikz lower%lower 部分放入 tikzpicture 环境
,listing side text% LaTeX源码和效果各一边
,fonttitle=\bfseries,
bicolor,colback=LightBlue!50!white,colbacklower=white,colframe=black,
righthand width=3cm,title=\tikzname\ drawing}
\path[fill=yellow,draw=yellow!75!red]
(0,0) circle (1cm);
\fill[red] (45:5mm) circle (1mm);
\fill[red] (135:5mm) circle (1mm);
\draw[line width=1mm,red]
(215:5mm) arc (215:325:5mm);
\end{tcblisting}
\end{exdispExample}
\end{docTcbKey}




%\clearpage
\begin{docTcbKey}{tikznode upper}{\colOpt{=\meta{options}}}{style}
This style places the upper part content into a centered
\tikzname\  node. The \meta{options} may be given as \tikzname\  node options.
This style is especially useful for boxes with multiline texts which are
fitted to the text width.

此项将upper部分的内容放到一个居中的 \tikzname\  node。选项 \meta{options} 会传递给 \tikzname\  node 。
此项常用于包含多行文本的盒子,可以适应文本宽度。
\begin{exdispExample}{tikznode_upper}
%%\usepackage{tikz}
\newtcbox{\headline}[1][]{enhanced,center,
ignore nobreak,fontupper=\Large\bfseries,
colframe=red!50!black,colback=red!10!white,
drop fuzzy shadow=yellow,tikznode upper,#1}

\headline{Important\\Headline}
\end{exdispExample}
\end{docTcbKey}

\begin{docTcbKey}{tikznode lower}{\colOpt{=\meta{options}}}{style}
This style places the lower part content into a centered
\tikzname\ node. The \meta{options} may be given as \tikzname\  node options.

此项将lower部分的内容放到一个居中的 \tikzname\  node。选项 \meta{options} 会传递给 \tikzname\  node 。
\begin{exdispExample}{tikznode_lower}
%%\usepackage{tikz}
\begin{tcolorbox}[bicolor,colback=LightBlue!50!white,colbacklower=white,
colframe=black,tikznode lower={inner sep=2pt,draw=red,fill=yellow}]
Upper part.
\tcblower
Lower part.
\end{tcolorbox}
\end{exdispExample}
\end{docTcbKey}



\begin{docTcbKey}{tikznode}{\colOpt{=\meta{options}}}{style}
Shortcut for setting \refKey{/tcb/tikznode upper} and \refKey{/tcb/tikznode lower}
the same time.

同时设置 \refKey{/tcb/tikznode upper} 和 \refKey{/tcb/tikznode lower} 的简写形式。
\end{docTcbKey}

\begin{docTcbKey}{varwidth upper}{\colOpt{=\meta{length}}}{style, default \refKey{/tcb/width}}
This style places the upper part content into a |varwidth| environment.
This style needs the |varwidth| package \cite{arseneau:2011a} to be loaded manually.
The resulting box has a maximal width of \meta{length}.
This option is only senseful for a \refCom{tcbox}.


此项将upper部分放到一个 |varwidth| 环境中。需要手动加载 |varwidth| 宏包 %\cite{arseneau:2011a}
。产出的盒子的最大宽度是 \meta{length}。此项只对 \refCom{tcbox} 生效。
\begin{exdispExample*}{varwidth_upper}{sbs,lefthand ratio=0.6}
%%\usepackage{varwidth}
\newtcbox{\varbox}{colframe=red!50!black,
colback=red!10!white,varwidth upper}

\varbox{Short text.}
\varbox{This box contains is a longer text
which is broken.}
\end{exdispExample*}
\end{docTcbKey}