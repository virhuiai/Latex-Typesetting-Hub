
\begin{docTcbKey}{overlay unbroken and first}{=\meta{graphical code}}{no default, initially unset}
 This is an optimized abbreviation for setting
 \refKeyLe{/tcb/overlay unbroken} and
 \refKeyLe{/tcb/overlay first} together.
 \refKeyLe{/tcb/overlay} overwrites this key.

这是同时设置\refKeyLe{/tcb/overlay unbroken}和\refKeyLe{/tcb/overlay first}的优化缩写。
\refKeyLe{/tcb/overlay}会覆盖此项设置。
\end{docTcbKey}

\begin{docTcbKey}{overlay middle and last}{=\meta{graphical code}}{no default, initially unset}
 This is an optimized abbreviation for setting
 \refKeyLe{/tcb/overlay middle} and
 \refKeyLe{/tcb/overlay last} together.
 \refKeyLe{/tcb/overlay} overwrites this key.

这是同时设置\refKeyLe{/tcb/overlay middle}和\refKeyLe{/tcb/overlay last}的优化缩写。
\refKeyLe{/tcb/overlay}会覆盖此项设置。
\end{docTcbKey}

\begin{docTcbKey}{overlay unbroken and last}{=\meta{graphical code}}{no default, initially unset}
 This is an optimized abbreviation for setting
 \refKeyLe{/tcb/overlay unbroken} and
 \refKeyLe{/tcb/overlay last} together.
 \refKeyLe{/tcb/overlay} overwrites this key.

这是同时设置\refKeyLe{/tcb/overlay unbroken}和\refKeyLe{/tcb/overlay last}的优化缩写。
\refKeyLe{/tcb/overlay}会覆盖此项设置。
\end{docTcbKey}





\begin{docTcbKey}[][doc new=2014-09-19]{overlay first and middle}{=\meta{graphical code}}{no default, initially unset}
 This is an optimized abbreviation for setting
 \refKeyLe{/tcb/overlay first} and \refKeyLe{/tcb/overlay middle} together.
 \refKeyLe{/tcb/overlay} overwrites this key.

这是同时设置\refKeyLe{/tcb/overlay first}和\refKeyLe{/tcb/overlay middle}的优化缩写。
\refKeyLe{/tcb/overlay}会覆盖此项设置。
\end{docTcbKey}

% csh_xelatexpdfcrop_example.tex
% \providecommand\cshInputPathVal{/Users/virhuiai/hlProjects/2022-LaTeX2e完全学习手册-第2版/LaTeX-snippets-pure/}
% \input{\cshInputPathVal/csh_AddInputPath.tex}%查找文件的路径
% \input{\cshInputPathVal/csh_xelatexpdfcrop_example.tex}%
\makeatletter
\providecommand\jobnameToRun{\filename@area\filename@base}
\makeatletter
\definecolor{Blue_Dark}{rgb}{0.090196,0.211765,0.364706}
\definecolor{Blue_Bright}{rgb}{0.858824,0.898039,0.945098}
\definecolor{Blue_Title}{rgb}{0.117647,0.211765,0.352941}%RGB值为30, 54, 90对应的比例值。请注意,RGB值在0到1之间进行归一化,因此需要将原始RGB值除以255以获得比例值。
\tcbset{%
csh xelatex example/.style={%
    skin=bicolor,
    colframe=Blue_Dark,
    colback=Blue_Bright,
    colbacklower=white,
    colbacktitle=Blue_Title,
    fonttitle=\tt,
    arc is angular,arc=1mm,
    drop fuzzy shadow,
    % title={来自virhuiai的例子:run system command里调用了pdfcrop},
    listing side comment,
    compilable listing,
    run xelatex={-shell-escape},
    pdf comment,
    freeze pdf,
    comment style={opacityframe=0}%透明
} 
}

\begin{tcblisting}{
    csh xelatex example,sidebyside=false,
    breakable,
    pdf comment={\jobnameToRun.pdf},
    comment style={raster columns=3
    % ,graphics pages={1,2,3}
    },before upper={
This example demonstrates
the application of break sequence specific overlay options.
Here, we define an environment |myexample| based
on |tcolorbox| where the visible drawing is done totally by overlay keys.\par
Here, the first application of |myexample| produces an unbroken |tcolorbox|.
The frame is drawn by the code given with \refKeyLe{/tcb/overlay unbroken}.\par
The second application of |myexample| is broken into several parts which
are drawn by the codes given with
\refKeyLe{/tcb/overlay first}, \refKeyLe{/tcb/overlay middle}, and
\refKeyLe{/tcb/overlay last}.

这个例子演示了应用于断行序列的特定覆盖选项。在这里,我们基于tcolorbox定义了一个名为|myexample|的环境,其中的可见绘图完全由覆盖键完成。\par
在这里,第一个应用于|myexample|的实例生成了一个不中断的|tcolorbox|。边框由给定的代码与\refKeyLe{/tcb/overlay unbroken}一起绘制。\par
第二个应用于|myexample|的实例被分成几个部分,每个部分都由与\refKeyLe{/tcb/overlay first}、\refKeyLe{/tcb/overlay middle}和\refKeyLe{/tcb/overlay last}相对应的代码进行绘制。
\par\bigskip
}
}
\documentclass{article}
\usepackage{ctex}
\usepackage[paperwidth=8em,paperheight=8em,%
text={8em,8em},left=0mm,top=0pt]{geometry}

\usepackage{tikz,lipsum}
\usepackage{tcolorbox}
\tcbuselibrary{skins,breakable}


\begin{document}%%%
\thispagestyle{empty}

\colorlet{colexam}{red!75!black}
\newcounter{example}%计数器使用example
\newtcolorbox[use counter=example]{myexample}{%
empty,%环境内没有默认文本
title={例 \thetcbcounter},% 自定义标题,其中\thetcbcounter 为环境的计数器。
attach boxed title to top left,%标题框线附着在左上角
% 自定义标题框线的样式
boxed title style={empty,size=minimal,toprule=2pt,top=4pt,
    overlay={\draw[colexam,line width=2pt]
    ([yshift=-1pt]frame.north west)--([yshift=-1pt]frame.north east);}},
coltitle=colexam,fonttitle=\Large\bfseries,
before=\par\medskip\noindent,%环境前的文本
parbox=false,%环境内不使用parbox
boxsep=0pt,%边框线与内容的间距为0
left=0pt,right=3mm,top=4pt,%左边距为0,右边距为3mm,上边距为4pt
breakable,%该环境可以跨页
pad at break*=0mm,%跨页时边框线不需要额外的空白
vfill before first,%第一页时环境顶部需要添加额外的空白。
%接下来,定义了四个tcolorbox中的overlay,
%分别表示未跨页、第一页、中间页和最后一页时边框线的样式。
overlay unbroken={%不会被分割时,左边有缺口的框
\draw[colexam,line width=1pt]
([yshift=-1pt]title.north east)%表示标题区域的右上角点,向下移动1pt(点)。
--([xshift=-0.5pt,yshift=-1pt]title.north-|frame.east)
%从标题区域的右上角点,向下移动1pt并向左移动0.5pt(水平方向的一半),再与框架区域的右侧线的交点。
--([xshift=-0.5pt]frame.south east)
% 表示从框架区域的右下角点,向左移动0.5pt,再与上一步得到的交点相连。
--(frame.south west); },
%从框架区域的左下角点,与上一步得到的线条相连。
% 整个绘图过程形成了一个闭合的多边形,线条宽度为1pt(点),
% 颜色为colexam。
overlay first={\draw[colexam,line width=1pt]%指定绘制线条的颜色和宽度。
([yshift=-1pt]title.north east)
%指定线条的起点。在这个例子中,我们将线条起点设置在标题框架的右上角,向下移动1个点(pt)。
--([xshift=-0.5pt,yshift=-1pt]title.north-|frame.east)
% 指定线条从起点到终点的路径。在这个例子中,我们将线条路径设置为从起点沿着标题框架的上边缘向
%右下方偏移0.5个点(pt),然后沿着框架的右边缘向下偏移1个点(pt),直到达到框架的右下角。
--([xshift=-0.5pt]frame.south east); },
% 指定线条的最后一个点。框架的右下角向左偏移0.5个点(pt)。
overlay middle={\draw[colexam,line width=1pt]
([xshift=-0.5pt]frame.north east)
%右上角,[xshift=-0.5pt]表示将线条向左偏移0.5pt,使得线条在框架内部。
--([xshift=-0.5pt]frame.south east); },
%绘制一条直线,终点为框架右下角,同样向左偏移0.5pt,与上面的线条相连成垂直线条。
overlay last={\draw[colexam,line width=1pt]%在frame的右侧绘制一条从顶部到底部的线。
([xshift=-0.5pt]frame.north east)
--([xshift=-0.5pt]frame.south east)--(frame.south west);},
}

\begin{myexample}
每日微笑不放弃,积极生活乐无极!
\end{myexample}
\clearpage

\begin{myexample}
城阙辅三秦,风烟望五津。
与君离别意,同是宦游人。
海内存知己,天涯若比邻。
无为在歧路,儿女共沾巾。
\end{myexample}

\clearpage
\begin{myexample}
江山如此多娇,引无数英雄竞折腰。
惜秦皇汉武,略输文采;唐宗宋祖,稍逊风骚。
一代天骄,成吉思汗,只识弯弓射大雕。
俱往矣,数风流人物,还看今朝。
\end{myexample}

\end{document}  
\end{tcblisting}

\end{document}

% \end{dispListing*}
% {\tcbusetemp}


\begin{dispExample}
% %% \tcbuselibrary{skins}
%% \newcounter{example}
\newtcolorbox[use counter=example]{FancyTitle}[3][]{%
  enhanced,colback=blue!10!white,colframe=orange,top=4mm,
  enlarge top by=\baselineskip/2+1mm,
  enlarge top at break by=0mm,pad at break=2mm,
  fontupper=\normalsize,label={#3},
  overlay unbroken and first={%
    \node[rectangle,rounded corners,draw=black,fill=blue!20!white,
      inner sep=1mm,anchor=west,font=\small]
      at ([xshift=4.5mm]frame.north west)
         {\strut\textbf{Example \thetcbcounter: #2}};},
  #1}%

\begin{FancyTitle}{My fancy title}{fancy:title}
  \lipsum[1]
\end{FancyTitle}
\end{dispExample}