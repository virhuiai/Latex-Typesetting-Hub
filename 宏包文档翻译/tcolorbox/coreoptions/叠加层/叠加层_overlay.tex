
\begin{docTcbKey}{overlay}{=\meta{graphical code}}{no default, initially unset}
 Adds \meta{graphical code} to the box drawing process. This \meta{graphical code}
 is drawn \emph{after} the frame and interior and \emph{before} the text content.

将 \meta{graphical code} 添加到盒子的绘制过程中。\meta{graphical code}将附加到边框和内部的绘制\emph{之后}、文本内容\emph{之前}。

\begin{exdispExample}{overlay_1}
 %% \tcbuselibrary{skins} % preamble
\tcbset{frogbox/.style={enhanced,
colback=green!10,colframe=green!65!black,
enlarge top by=5.5mm,%在上方扩大5.5mm的空间,用于放置青蛙图案;
overlay=%在tcolorbox的上方覆盖一个图案;
{\foreach \x in {2cm,3.5cm} {
%循环两次,分别在距离tcolorbox左上角2cm和3.5cm的位置上放置青蛙图案;
\begin{scope}[shift={([xshift=\x]frame.north west)}]
%进入一个局部坐标系,将坐标系原点移到指定位置;
\path[draw=green!65!black,fill=green!10,line width=1mm] (0,0) arc (0:180:5mm);
%绘制一条绿色的线条,宽度为1mm,半径为5mm,形成一个半圆;
\path[fill=black] (-0.2,0) arc (0:180:1mm);
% 在半圆的左侧绘制一个黑色的小圆,作为青蛙的眼睛;
    \end{scope}}}}}

% 接下来的代码使用了定义好的"frogbox"样式,创建了一个tcolorbox,标题为"My title",并在内部填写了一些文本。
\begin{tcolorbox}[frogbox,title=My title]
This is a \textbf{tcolorbox}.
\end{tcolorbox}
\end{exdispExample}

% 这段latex代码定义了一个名为"frogbox"的tcolorbox样式,具体说明如下:

% enhanced:启用tcolorbox的增强模式;
% colback=green!10:设置背景颜色为绿色的10%;
% colframe=green!65!black:设置边框颜色为绿色的65%和黑色的35%;


\enlargethispage*{5mm}
\begin{exdispExample}{overlay_2}
%%\usetikzlibrary{patterns} % preamble
%% \tcbuselibrary{skins}     % preamble
% 这段代码定义了一个新的tcolorbox风格"ribbonbox",包括以下属性:
\tcbset{ribbonbox/.style={enhanced,%增强选项,允许在盒子中使用更多的TikZ选项。
colback=red!5!white,%背景颜色为红色的5%和白色的95%混合色。
colframe=red!75!black,%框架边框颜色为红色的75%和黑色的25%混合色。
fonttitle=\bfseries,%标题字体粗体。
overlay={%覆盖选项,允许在盒子上覆盖其他图形元素。
%在这里,我们使用TikZ绘制了一个蓝色且带有五角星纹理的黄色丝带,用于装饰盒子的右上角。
% path:绘制路径。
\path[fill=blue!75!white,%填充颜色为蓝色的75%和白色的25%混合色。
draw=blue,%绘制颜色为蓝色。
double=white!85!blue,%双重边框颜色为白色的85%和蓝色的15%混合色。
% preaction:在绘制路径之前执行的操作。
% 在这里,我们将填充颜色设置为蓝色的75%和白色的25%混合色,不透明度为0.6。
preaction={opacity=0.6,fill=blue!75!white},
line width=0.1mm,double distance=0.2mm,
%绘制线宽为0.1毫米。双重边框之间的距离为0.2毫米。
pattern=fivepointed stars,%填充纹理为五角星。
pattern color=white!75!blue]%填充纹理颜色为白色的75%和蓝色的25%混合色。
([xshift=-0.2mm,yshift=-1.02cm]frame.north east)
-- ++(-1,1) -- ++(-0.5,0) -- ++(1.5,-1.5) -- cycle;}}}

\begin{tcolorbox}[ribbonbox,title=My title]
This is a \textbf{tcolorbox}.
\tcblower
This is the lower part.
\end{tcolorbox}
\end{exdispExample}
\end{docTcbKey}


% 然后,我们使用这个新的风格来创建一个tcolorbox,标题为"My title",并在盒子中包含一些文本。下面的"tcblower"选项将文本分成两个部分,分别放置在盒子的上lower部分。


%%%% \clearpage
\begin{docTcbKey}{no overlay}{}{style, no default, initially set}
 Removes the overlay if set before.

% 移除叠加层。
如果之前设置了浮层,将其移除。
\end{docTcbKey}