
\begin{docTcbKey}{overlay broken}{=\meta{graphical code}}{no default, initially unset}
 If the box is set to be \refKeyLe{/tcb/breakable} and \emph{is} broken actually,
 then the \meta{graphical code} is added to the box drawing process.
 \refKeyLe{/tcb/overlay} overwrites this key.

如果盒子被设置为 \refKeyLe{/tcb/breakable} 且分页了, 那么 \meta{graphical code} 将添加到盒子绘制过程中。 \refKeyLe{/tcb/overlay} 将覆盖此键。

% 如果盒子设置为\refKeyLe{/tcb/breakable}且\emph{实际上}分页了,那么\meta{graphical code}会被添加到盒子的绘制过程中。\refKeyLe{/tcb/overlay}会覆盖此项设置。\footnote{译注:即实际上分页时才添加叠加层。}
\end{docTcbKey}

\begin{docTcbKey}{overlay unbroken}{=\meta{graphical code}}{no default, initially unset}
 If the box is set to be \refKeyLe{/tcb/breakable} but \emph{is not} broken actually
 or if the box is set to be \refKeyLe{/tcb/unbreakable},
 then the \meta{graphical code} is added to the box drawing process.
 \refKeyLe{/tcb/overlay} overwrites this key.

如果盒子设置为\refKeyLe{/tcb/breakable},但实际上\emph{没有}被打破,或者盒子设置为\refKeyLe{/tcb/unbreakable},则\meta{graphical code}将被添加到盒子绘制过程中。\refKeyLe{/tcb/overlay}将覆盖此键。

% 如果盒子设置了\refKeyLe{/tcb/breakable}但\emph{实际上没}分页,或盒子设置为\refKeyLe{/tcb/unbreakable},那么\meta{graphical code}会被添加到盒子的绘制过程中。\refKeyLe{/tcb/overlay}会覆盖此项设置。\footnote{译注:即实际上没有分页时才添加叠加层。}
\end{docTcbKey}

