
With an overlay, arbitrary \meta{graphical code} can be added to a
 |tcolorbox|. This code is executed \emph{after} the frame and interior are
 drawn and \emph{before} the text content is drawn. Therefore, you can
 decorate the |tcolorbox| with your own extensions.
 Common special cases are \emph{watermarks} which are implemented using overlays.
 See Subsection \ref{subsec:watermarks} from page \pageref{subsec:watermarks} if
 you want to add \emph{watermarks}.

通过叠加层,可以向 |tcolorbox| 盒子中添加任意的\meta{图形代码}。该代码在边框和内部绘制完毕后执行,文本内容绘制之前执行。因此,您可以使用自己的扩展装饰 |tcolorbox|。
常见的特殊情况是使用叠加层实现的\emph{水印}。如果您想要添加\emph{水印},请参见第 \pageref{subsec:watermarks} 页的第 \ref{subsec:watermarks} 小节。


\begin{marker}
 If you use the core package only, the \meta{graphical code} has to be |pgf| code
 and there is not much assistance for positioning.
 Therefore, the usage of the \refKeyLe{/tcb/enhanced} mode from the library skins
 is recommended which allows |tikz| code and gives access to
 \refKeyLe{/tcb/geometry nodes} for positioning.

如果您仅使用核心包, \meta{graphical code} 必须是 |pgf| 代码,而且也没有太多的定位辅助。因此, 推荐使用 |skins| 库的 \refKeyLe{/tcb/enhanced} 模式,它不仅允许 |tikz| 代码,还允许用 \refKeyLe{/tcb/geometry nodes} 进行定位。
\end{marker}