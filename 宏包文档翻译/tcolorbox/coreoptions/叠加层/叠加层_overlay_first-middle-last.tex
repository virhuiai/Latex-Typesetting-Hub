\begin{docTcbKey}{overlay first}{=\meta{graphical code}}{no default, initially unset}
 If the box is set to be \refKeyLe{/tcb/breakable} and \emph{is} broken actually,
 then the \meta{graphical code} is added to the box drawing process for
 the \emph{first} part of the break sequence.
 \refKeyLe{/tcb/overlay} overwrites this key.

%  如果该盒子被设置为 \refKeyLe{/tcb/breakable} 并且实际上已经被打破, 那么对于断裂序列的 \emph{第一个} 部分,\meta{graphical code} 将添加到盒子绘制过程中。 \refKeyLe{/tcb/overlay} 将覆盖此键。

如果盒子设置了\refKeyLe{/tcb/breakable}且分页了,%
那么\meta{graphical code}会被添加到盒子的分开的\emph{第一}部分的绘制过程中。
\refKeyLe{/tcb/overlay}会覆盖此项设置。
\end{docTcbKey}

\begin{docTcbKey}{overlay middle}{=\meta{graphical code}}{no default, initially unset}
 If the box is set to be \refKeyLe{/tcb/breakable} and \emph{is} broken actually,
 then the \meta{graphical code} is added to the box drawing process for
 the \emph{middle} parts (if any) of the break sequence.
 \refKeyLe{/tcb/overlay} overwrites this key.

%  如果盒子被设置为\refKeyLe{/tcb/breakable}并且实际上已经被分割, 则\meta{graphical code}将被添加到盒子绘制过程中的\emph{中间}部分(如果有的话)的分割序列中。 \refKeyLe{/tcb/overlay}将覆盖此键。

如果盒子设置了\refKeyLe{/tcb/breakable}且分页了,%
那么\meta{graphical code}会被添加到盒子的分开的序列中的\emph{中间}部分(如果有)。 \refKeyLe{/tcb/overlay}会覆盖此项设置。
\end{docTcbKey}

\begin{docTcbKey}{overlay last}{=\meta{graphical code}}{no default, initially unset}
 If the box is set to be \refKeyLe{/tcb/breakable} and \emph{is} broken actually,
 then the \meta{graphical code} is added to the box drawing process for
 the \emph{last} part of the break sequence.
 \refKeyLe{/tcb/overlay} overwrites this key.

%  如果盒子设置为 \refKeyLe{/tcb/breakable} 并且实际上已经被打破, 则 \meta{graphical code} 将被添加到盒子绘制过程的最后一个断裂序列的部分中。 \refKeyLe{/tcb/overlay} 覆盖此键。

如果盒子设置了\refKeyLe{/tcb/breakable}且分页了,%
那么\meta{graphical code}会被添加到盒子的分开的\emph{最后}一部分的绘制过程中。
\refKeyLe{/tcb/overlay}会覆盖此项设置。
\end{docTcbKey}