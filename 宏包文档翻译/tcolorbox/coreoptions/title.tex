\subsection{Title}% 是相对于使用input语句所在文件 todo 

  % \begin{docTcbKey}{title}{=\meta{text}}{no default, initially empty}
\begin{docTcbKey}{title}{=\meta{文本}}{无默认值,初始为空}
Creates a heading line with \meta{text} as content.

创建以 \meta{text} 作为内容的标题行。

\begin{exdispExample*}{title}{sbs,lefthand ratio=0.6}
\begin{tcolorbox}[title=我的标题行]
这是一个\textbf{tcolorbox}。
\end{tcolorbox}
\end{exdispExample*}
\end{docTcbKey}

\begin{docTcbKey}{notitle}{}{no value, initially set}
Removes the title line if set before.

移除之前设置的标题行。

\begin{dispExample*}{sbs,lefthand ratio=0.6}
\begin{tcolorbox}[title=我的标题行,notitle]
这是一个\textbf{tcolorbox}。
\end{tcolorbox}
\end{dispExample*}
\end{docTcbKey}

\begin{docTcbKey}{adjusted title}{=\meta{text}}{style, no default, initially unset, 标题高度等高}
Creates a heading line with \meta{text} as content. The minimal height of
this line is adjusted to fit the text given by \refKeyLe{/tcb/adjust text}.
This option makes sense
for single line headings if boxes are set side by side with equal height.
Note that it is very easy to trick this adjustment.


% 创建一个以 \meta{text} 为内容的标题行。 此行的最小高度根据 \meta{text} 自适应。
% 此选项适用于希望让单行放置多个并排的盒子拥有相同的标题高度。
% 注意,to trick 此调整是非常容易的。

创建一个标题行,其中内容为\meta{text}。此行的最小高度会根据\refKeyLe{/tcb/adjust text}中给定的文本进行调整。如果希望让单行放置多个并排的盒子拥有相同的标题高度,则此选项很有意义。请注意,很容易欺骗此调整。
% 诀窍
% 技巧
% 骗局
% 魔术
% 神奇手法
% 欺骗
% 诡计
% 奥妙
% 狡猾手段
% 玩意儿

% exdispExample 
% 这是一个tcolorbox宏包提供的环境,用于生成一个例子展示。runs=2表示该例子最多运行2次,{adjusted_title}是该例子的标题。
\begin{exdispExample}[runs=2]{adjusted_title}
\tcbset{colback=White,arc=0mm,width=(\linewidth-4pt)/4,
equal height group=AT,before=,after=\hfill,fonttitle=\bfseries}

以下标题是not adjusted:\\
\foreach \n in {xxx,ggg,AAA,\"Agypten}
{\begin{tcolorbox}[title=\n,colframe=red!75!black]
一些文本。\end{tcolorbox}}
现在,我们再次尝试adjusted titles:\\
\foreach \n in {xxx,ggg,AAA,\"Agypten}
{\begin{tcolorbox}[adjusted title=\n,colframe=blue!75!black]
一些文本。\end{tcolorbox}}
\end{exdispExample}
\end{docTcbKey}

% todo 
% \tcbset{colback=White,arc=0mm,width=(\linewidth-4pt)/4,equal height group=AT,before=,after=\hfill,fonttitle=\bfseries}:
%这是tcolorbox宏包提供的一个设置环境,可以设置后面生成的tcolorbox的一些属性。
%其中,colback表示背景颜色,arc表示圆角半径,width表示盒子的宽度,
%equal height group表示将多个盒子的高度设置成相同,before和after表示在盒子前后分别添加一些内容,fonttitle表示标题的字体。

% \foreach \n in {xxx,ggg,AAA,\"Agypten}:
% 这是一个foreach循环,用于循环生成tcolorbox盒子。\n表示循环变量,{xxx,ggg,AAA,\"Agypten}是一个包含了4个元素的列表。
% \begin{tcolorbox}[title=\n,colframe=red!75!black]一些文本。\end{tcolorbox}:这是一个tcolorbox环境,用于生成一个盒子。
%title表示盒子的标题,colframe表示边框颜色,一些文本。是盒子中的内容。

% \begin{tcolorbox}[adjusted title=\n,colframe=blue!75!black]一些文本。\end{tcolorbox}:这是一个tcolorbox环境,用于生成一个盒子。
%adjusted title表示盒子的标题,colframe表示边框颜色,一些文本。是盒子中的内容。




\begin{docTcbKey}{adjust text}{=\meta{text}}{no default, initially \texttt{\"Apgjy}}
This sets the reference text for \refKeyLe{/tcb/adjusted title}. If your texts
never exceed \enquote{\"Apgjy} in depth and height you don't need to care about this option.

用于为\refKeyLe{/tcb/adjusted title}设置参考文本。%
如果你的文本不会超过\enquote{\"Apgjy}的深度和高度,你就不需要关心这个选项。

%%%TODO 弄些例子出来看到底啥意思 
% \begin{exdispExample}[runs=2]{adjust_text}
%   \foreach \n in {xxx,ggg,AAA,\"Agypten}
%   {\begin{tcolorbox}[adjust text=\n,title=test,colframe=blue!75!black,width=(\linewidth-4pt)/4]
%   一些文本。\end{tcolorbox}}
% \end{exdispExample}

\end{docTcbKey}

%squeezed 挤压
\begin{docTcbKey}[][doc new=2014-11-24]{squeezed title}{=\meta{text}}{style, no default, initially unset,挤压标题,宽度限最大宽}
Creates a single heading line with \meta{text} as content.
If the \meta{text} is longer than the available space, the text is
squeezed to fit into the available space.

创建一个标题行,内容为 \meta{text}。%
如果 \meta{text} 比可用空间长, 文本将被压缩以适应可用的空间。
\begin{exdispExample}{squeezed_title}
% \tcbuselibrary{raster}
\begin{tcbitemize}[
raster columns=3,%三列
raster equal height,%等高
colframe=red!75!black,colback=red!5!white,%框与背景色
fonttitle=\bfseries%标题加格式
]
\tcbitem[squeezed title={短标题}]
第一个例子
\tcbitem[squeezed title={这是一个非常非常长的标题}]
第二个例子
\tcbitem[squeezed title={这个标题显然对这个例子来说太长了}]
第三个例子
\end{tcbitemize}
\end{exdispExample}
\end{docTcbKey}



% 这是一个使用tcolorbox宏包的例子,展示了如何使用tcbitemize环境创建一个三列等高的项目列表,并对项目的标题进行压缩。
% 具体解释如下:
% 首先加载tcolorbox宏包的raster库,以便使用raster columns和raster equal height选项。

% 然后创建一个tcbitemize环境,并设置raster columns=3和raster equal height选项,以创建一个三列等高的项目列表。

% 接下来,对每个项目使用\tcbitem命令创建一个项目,并使用squeezed title选项压缩项目标题。在本例中,第一个项目的标题为“短标题”,第二个项目的标题为“这是一个非常非常长的标题”,第三个项目的标题为“这个标题显然对这个例子来说太长了”。

% 最后,为了美化效果,设置colframe和colback选项,分别为项目框和背景色添加红色调,并使用fonttitle选项为项目标题添加粗体格式。



%这个 todo 未定义的命令,可能是  tcbitemize 那边才有
% begin{tcolorbox}[squeezed title={BrE əˈdʒʌst, AmE əˈdʒəst},colframe=red!75!black] 
% 一些文本。
% \end{tcolorbox}
% begin{tcolorbox}[squeezed title={BrE skwiːz, AmE skwiz},colframe=red!75!black] 
% 一些文本。
% \end{tcolorbox}
% begin{tcolorbox}[squeezed title={这个标题显然对这个例子来说太长了},colframe=red!75!black] 
% 一些文本。
% \end{tcolorbox}

%试了音标的字体
% \begin{tcbitemize}[
% raster columns=3,%三列
% raster equal height,%等高
% colframe=red!75!black,colback=red!5!white,%框与背景色
% fonttitle=\bfseries%标题加格式
% ]
% \tcbitem[squeezed title={BrE əˈdʒʌst, AmE əˈdʒəst}]
% adjust
% \tcbitem[squeezed title={BrE skwiːz, AmE skwiz}]
% squeeze
% \tcbitem[squeezed title={这个标题显然对这个例子来说太长了}]
% 第三个例子
% \end{tcbitemize}

\begin{docTcbKey}[][doc new=2014-11-24]{squeezed title*}{=\meta{text}}{style, no default, initially unset}
This is a combination of \refKeyLe{/tcb/adjusted title} and  \refKeyLe{/tcb/squeezed title}.

这是  \refKeyLe{/tcb/adjusted title} 和  \refKeyLe{/tcb/squeezed title} 的组合。即高度和宽度都 \dots
\begin{exdispExample}{squeezed_title_2}
% \tcbuselibrary{raster}
\begin{tcbitemize}[raster columns=3,raster equal height,
  colframe=red!75!black,colback=red!5!white,fonttitle=\bfseries]
\tcbitem[squeezed title*={Short title}]
  First box
\tcbitem[squeezed title*={This is a very very long title}]
  Second box
\tcbitem[squeezed title*={This title is clearly to long for this application}]
  Third box
\end{tcbitemize}
\end{exdispExample}
\end{docTcbKey}

\begin{docTcbKey}[][doc new=2019-03-01]{titlebox}{=\meta{mode}}{no default, initially \texttt{visible}}
Controls the treatment of the title part of the box.
Feasible values for \meta{mode} are:

% 控制盒子的标题部分的处理。可设的 \meta{mode} 值有:
控制盒子的标题部分的处理方式。 \meta{mode} 的可行值为:

\begin{DescriptionR}{\docValue{invisible}}
\item[\docValue{visible}]usual type setting of the title box,\\
对带标题盒子的常用的设置,
\item[\docValue{invisible}]empty space instead of the title contents.\\
使用空白代替标题内容。
\end{DescriptionR}

% \begin{DescriptionL}{\docValue{invisible}}
% \item[\docValue{visible}]usual type setting of the title box,\\
% 对带标题盒子的常用的设置,
% \item[\docValue{invisible}]empty space instead of the title contents.\\
% 使用空白代替标题内容。
% \end{DescriptionL}

% \begin{DescriptionLsqueezed}{\docValue{visible}}
% \item[\docValue{visible}]usual type setting of the title box,\\
% 对带标题盒子的常用的设置,
% \item[\docValue{invisible}]empty space instead of the title contents.\\
% 使用空白代替标题内容。
% \end{DescriptionLsqueezed}

% \begin{DescriptionR}{\docValue{invisible}}
% \item[\docValue{visible}]usual type setting of the title box,\\
% 对带标题盒子的常用的设置,
% \item[\docValue{invisible}]empty space instead of the title contents.\\
% 使用空白代替标题内容。
% \end{DescriptionR}

% \begin{description}
% \item[\docValue{visible}]usual type setting of the title box,\\
% 对带标题盒子的常用的设置,
% \item[\docValue{invisible}]empty space instead of the title contents.\\
% 使用空白代替标题内容。
% \end{description}
\begin{exdispExample}{titlebox}
\begin{tcolorbox}[title=我的不可见标题,
  titlebox=invisible]
这是一个\textbf{tcolorbox}.
\end{tcolorbox}
\end{exdispExample}

\begin{exdispExample}{visible_titlebox}
  \begin{tcolorbox}[title=我的可见标题,
    titlebox=visible]
这是一个\textbf{tcolorbox}.
  \end{tcolorbox}
  \end{exdispExample}
\end{docTcbKey}



% \clearpage
\begin{docTcbKey}{detach title}{}{no value}
Detaches the title from its normal position. The text of the title is
stored into \docAuxCommand{tcbtitletext} and the formatted title is
available by \docAuxCommand{tcbtitle}.
The main application is to move the title from its usual place to another one.
  
将标题从它的正常位置移开。标题的文本存储到 \docAuxCommand{tcbtitletext} 中,格式化的标题可以通过 \docAuxCommand{tcbtitle} 获得。主要的应用是将标题从它的惯例位置移动到另一个位置。

% 将标题从其正常位置移开。标题文本存储在\docAuxCommand{tcbtitletext}中,格式化的标题可通过\docAuxCommand{tcbtitle}获得。 主要应用是将标题从其通常位置移动到另一个位置。
 
\begin{exdispExample}{detach_title}
\newtcolorbox{mybox}[2][]{
colbacktitle=red!10!white,
colback=blue!10!white,
coltitle=red!70!black,
title={#2},fonttitle=\bfseries,#1}

\begin{mybox}{My title}
这是一个\textbf{tcolorbox}.
\end{mybox}
\begin{mybox}[detach title,%暂存标题内容到\tcbtitle
before upper={\tcbtitle\quad}%放到upper的前面
]{detach title}
这是一个\textbf{tcolorbox}。\footnotemark
\end{mybox}
\begin{mybox}[detach title,
after upper={\par\hfill\tcbtitle}%放到upper的后面
]{名人名字}
可以用于名人名言的内容。
\end{mybox}
\end{exdispExample}
\end{docTcbKey}
\footnotetext{译者idea:可以改造了当单条的description。}


  

\begin{docTcbKey}{attach title}{}{no value}
  Attaches the title to its normal position. This option is used to reverse
  \refKeyLe{/tcb/detach title}.

  将标题位置重置到其正常位置。此选项用于反转 \refKeyLe{/tcb/detach title}。
  \end{docTcbKey}
  
  
  \begin{docTcbKey}[][doc updated=2015-07-08]{attach title to upper}{\colOpt{=\meta{text}}}{style, default empty, initially unset}
Attaches the title to the begin of the upper part of the box content.
The optional \meta{text} is set between the formatted title and the box content.

将标题附加到框内容upper部分的开头。
可选的 \meta{text} 插入到标题和upper部分内容之间。
  \begin{exdispExample}{attach_title_to_upper}
  \newtcolorbox{mybox}[2][]{colbacktitle=red!10!white,
    colback=blue!10!white,coltitle=red!70!black,
    title={#2},fonttitle=\bfseries,#1}
   
  \begin{mybox}[attach title to upper={\ ---\ }]{My title}
    attach title to upper加值的情况,不仅将标题放到upper之前,还在标题和upper之间放入破折线。
  \end{mybox}
  \begin{mybox}[attach title to upper,after title={:\ }]{My title}
    attach title to upper不加值时,将标题位置放到upper部分之前,再使用after title在标题后加内容。
  \end{mybox}
  \end{exdispExample}
  \end{docTcbKey}



\bigskip
\begin{marker}
More title options are documented in \Vref{subsec:contentadditions}
and \Vref{subsec:skinboxedtitle}.

更多的标题相关配置文档内容见 \Vref{subsec:contentadditions}
和 \Vref{subsec:skinboxedtitle}.
\end{marker}

