\subsection{Corners\\四角}\label{subsec:corners}

The four corners of any |tcolorbox| can be set individually as
\refKeyLe{/tcb/sharp corners} or as \refKeyLe{/tcb/rounded corners}.
These settings are also reflected in the behavior of \refKeyLe{/tcb/borderline}
and \refKeyLe{/tcb/shadow} as one would expect.

任何 |tcolorbox| 的四个角可以通过 \refKeyLe{/tcb/sharp corners} 或 \refKeyLe{/tcb/rounded corners} 单独设置。
这些设置也影响到 \refKeyLe{/tcb/borderline} 和 \refKeyLe{/tcb/shadow} 的表现。

By default, all four corners are \emph{rounded}. So, only the
\refKeyLe{/tcb/sharp corners} option will be necessary for most use cases.
The \refKeyLe{/tcb/rounded corners} option can be used to revert a \refKeyLe{/tcb/sharp corners}
setting.

默认情况下, 四角都是\emph{圆润的}。 因此,只有
\refKeyLe{/tcb/sharp corners} 选项是需要在大多数用例中显式指定。
\refKeyLe{/tcb/rounded corners} 选项可以用于重置 \refKeyLe{/tcb/sharp corners}
修改过的设定。


\begin{docTcbKey}{sharp corners}{=\meta{position}}{default |all|, initially unset}
The \meta{position} denotes one or more of the four box corners to be set as
\emph{sharp} corners. The not assigned corners will retain their mode.
Feasible values for \meta{position} are:

\meta{position} 用来将盒子的四个角中的一个或多个设置为\emph{sharp}(直角) ,未指定的角将保持原来的模式(圆角)。
可设置的 \meta{position} 值有:

\begin{itemize}
\foreach \p in {northwest,northeast,southwest,southeast,north,south,east,west,downhill,uphill,all}
{
\item\tcbox[on line,size=title,arc=2mm,colframe=red!75!black,colback=red!5!white,
enlarge top by=0.5mm,enlarge bottom by=0.5mm,sharp corners=\p]{\docValue{\p}}
}
\end{itemize}
\begin{exdispExample*}{sharp_corners_1}{sbs,lefthand ratio=0.6}
\begin{tcolorbox}[colback=red!5!white,
colframe=red!75!black,
sharp corners=northwest ]
|sharp corners=|\textbf{northwest},%
北西,左上角为直角。
\end{tcolorbox}
\end{exdispExample*}
\begin{exdispExample*}{sharp_corners_2}{sbs,lefthand ratio=0.6}
\begin{tcolorbox}[colback=red!5!white,
colframe=red!75!black,
sharp corners ]
This is a \textbf{tcolorbox}.
\end{tcolorbox}
\end{exdispExample*}
\end{docTcbKey}




%% \clearpage
\begin{docTcbKey}{rounded corners}{=\meta{position}}{default |all|, initially |all|}
The \refKeyLe{/tcb/rounded corners} can be used to revert a \refKeyLe{/tcb/sharp corners}
setting. The \meta{position} denotes one or more of the four box corners to be set as
\emph{rounded} corners. The not assigned corners will retain their mode.
Feasible values for \meta{position} are\footnote{The graphical examples assume
that the boxes where set to have sharp corners before.}:

\refKeyLe{/tcb/rounded corners} 可以用来重置 \refKeyLe{/tcb/sharp corners} 带来的设置的修改。 \meta{position} 用来将盒子的四个角中的一个或多个设置为 \emph{rounded}(圆角)。未指定的角将保持原来的模式。
\meta{position}可用的值有\footnote{例子中假设设置之前有尖角的盒子。}:
\begin{itemize}
\foreach \p in {northwest,northeast,southwest,southeast,north,south,east,west,downhill,uphill,all}
{
\item\tcbox[on line,size=title,arc=2mm,colframe=red!75!black,colback=red!5!white,
enlarge top by=0.5mm,enlarge bottom by=0.5mm,sharp corners,rounded corners=\p]{\docValue{\p}}
}
\end{itemize}
\begin{exdispExample*}{rounded_corners}{sbs,lefthand ratio=0.6}
\begin{tcolorbox}[colback=red!5!white,
colframe=red!75!black,sharp corners,
rounded corners=northwest ]
|rounded corners=northwest|,北西,左上角为圆角。
\end{tcolorbox}
\end{exdispExample*}
\end{docTcbKey}


\begin{docTcbKey}{sharpish corners}{}{style, no value}
Shortcut for setting \refKeyLe{/tcb/arc} and \refKeyLe{/tcb/outer arc}
to |0pt|. With this setting, rounded corners will appear as quasi-sharp,
but e.\,g.\ the shadow will be somewhat rounder than the shadow
of really sharp corners.

同时设置 \refKeyLe{/tcb/arc} 和 \refKeyLe{/tcb/outer arc} 到 |0pt| 的简写。通过这项设置, 圆角展示得很像直角, 但 e.\,g.\ 阴影会比真正的直角的阴影稍微圆一些。
\begin{marker}
Corners are still of type \emph{rounded} with this option, but appear
\emph{sharp}. To switch back to rounded corners, one has to adapt
\refKeyLe{/tcb/arc} and \refKeyLe{/tcb/outer arc}.

本项设置后,四个角的类型仍然是 \emph{rounded} , 但展现为\emph{sharp}。要切回 rounded corners, 需要修改 \refKeyLe{/tcb/arc} 和 \refKeyLe{/tcb/outer arc}。
\end{marker}
\begin{exdispExample*}{sharpish_corners}{sbs,lefthand ratio=0.6}
\begin{tcolorbox}[colback=red!5!white,
colframe=red!75!black,
sharpish corners ]
This is a \textbf{tcolorbox}.
\end{tcolorbox}
\end{exdispExample*}
\end{docTcbKey}




%% \clearpage

The following examples will show the differences between
\refKeyLe{/tcb/rounded corners}, \refKeyLe{/tcb/sharpish corners}, and \refKeyLe{/tcb/sharp corners}.
The later two give the same core box, but \refKeyLe{/tcb/borderline}
and \refKeyLe{/tcb/shadow} settings are slightly different.
The following examples use \refKeyLe{/tcb/drop fuzzy shadow}.

下面的例子展示了 \refKeyLe{/tcb/rounded corners}, \refKeyLe{/tcb/sharpish corners}, 和 \refKeyLe{/tcb/sharp corners} 的区别。
后两个的内部盒子是相同的, 但 \refKeyLe{/tcb/borderline} 和 \refKeyLe{/tcb/shadow} 的设置有点不同。
下面的例子使用了 \refKeyLe{/tcb/drop fuzzy shadow}。


\begin{extcolorbox}[minipage]{corners_comparison}%
[blankest]%样式为 blankest,这意味着这个盒子没有边框、背景色、标题和水印等。
\foreach \n in {rounded corners,sharpish corners,sharp corners}{%这是一个循环,分别为 rounded corners、sharpish corners 和 sharp corners 执行后面的代码块。
\begin{tcolorbox}[enhanced jigsaw,%使用 enhanced jigsaw 样式
    frame empty,%没有框架
    interior empty,%没有背景
    fuzzy halo,%带有模糊光晕
    halign=center,%
    beforeafter skip=4mm]%在盒子前后加入4毫米的间距
\begin{tcolorbox}[enhanced,%允许使用高级特性,如图形和图片。
    drop fuzzy shadow,%给盒子添加一个模糊的阴影。
    width=\linewidth-1cm,
colback=red!5!white, colframe=red!75!black, fonttitle=\bfseries,
title=My title,\n,
%使用 tikz 创建一个放大镜效果,放大的形状为圆形,放大倍数为8倍,大小为2厘米。
%这段 LaTeX 代码是使用了 TikZ 的 spy 库,用于在图形中创建一个 "放大镜" 效果。
tikz={spy using outlines={circle,%设置 "放大镜" 的形状为圆形。
 magnification=8, %设置放大倍数为8。
 size=2cm, %设置 "放大镜" 的大小为2厘米。
 connect spies}},%在 "放大镜" 和被放大的区域之间画一条连接线。
%在盒子上添加额外的元素,具体内容包括一个蓝色的放大镜和一个文本节点。
overlay={\spy [blue, size=4cm] on (frame.south east)
    in node at ([xshift=-2.5cm,yshift=-2.5cm]frame.south east);
\node[right] at ([xshift=2cm,yshift=-1cm]frame.south west) {\textbf{\Large\ttfamily\n}};
}]
This is a \textbf{tcolorbox}.
\end{tcolorbox}
\end{tcolorbox}}
\end{extcolorbox}