\subsection{Externalization\\外部化}

\begin{marker}
See \Fullref{sec:external} for the \mylib{external} library of |tcolorbox|.

请查看第 \Fullref{sec:external} 节,了解 |tcolorbox| 的 \mylib{external} 库。
% |tcolorbox|的\mylib{external}库请参见\Fullref{sec:external}。
\end{marker}

If the \emph{externalization} library of the \texttt{tikz} package is used
and \refKeyLe{/tcb/graphical environment} is set to |tikzpicture|,
a |tcolorbox| could trigger the externalization process which will arise
a compilation error.

如果使用\texttt{tikz}宏包的\emph{externalization}库,并将\refKeyLe{/tcb/graphical environment}设置为|tikzpicture|,那么|tcolorbox|可以触发外部化进程,这将引发编译错误。

% 如果使用了\texttt{tikz}包的\emph{externalization}库,%
% 且 \refKeyLe{/tcb/graphical environment}被设置为|tikzpicture|,%
% 一个|tcolorbox|盒子可能触发外部进程,从而就可能产生编译错误。


To avoid this, there are two possible strategies:

为了避免这种情况,有两种可能的策略:
\begin{itemize}
\item 
Ensure, that |\tikzexternaldisable| is set before a |tcolorbox| is used.
If you typically use the pattern |\tikzexternalenable| \textit{some picture} |\tikzexternaldisable|,
there is nothing to care about.
\\确保在使用|tcolorbox|之前设置|\tikzexternaldisable|。%
如果你经常使用这种 |\tikzexternalenable| \textit{some picture} |\tikzexternaldisable| 模式,
没什么好担心的。
\item 
If \emph{externalization} is enabled globally, use \refKeyLe{/tcb/shield externalize} to
shield any |tcolorbox|. The preamble code could look like this:
\\如果\emph{externalization}全局启用, 使用\refKeyLe{/tcb/shield externalize}来保护|tcolorbox|。%
导言代码可以是这样的:
\begin{dispListing}
\usetikzlibrary{external}
\tikzexternalize
\tcbset{shield externalize}
\end{dispListing}
\end{itemize}

\begin{docTcbKey}{shield externalize}{\colOpt{=true\textbar false}}{default |true|, initially |false|}
If set to |true|, the drawing part of the |tcolorbox| is not being externalized
which is a good thing at the current state of art. Nevertheless, if the
|tcolorbox| contains a |tikzpicture|, this picture is still externalized.
Pictures drawn with help of \refKeyLe{/tcb/tikz upper} or alike are \emph{not}
externalized.

如果设置为|true|,则|tcolorbox|的绘制部分不会被外部化,这在当前的艺术状态下是一件好事。然而,如果|tcolorbox|包含一个|tikzpicture|,则该图仍然会被外部化。使用\refKeyLe{/tcb/tikz upper}或类似方法绘制的图形\emph{不会}被外部化。

% 如果设置为|true|, |tcolorbox|的绘图部分不会外部化,就目前的技术水平而言,这是一件好事。%
% 尽管如此,如果|tcolorbox|包含|tikzpicture|,这张图片仍然是外部化的。%
% 借助\refKeyLe{/tcb/tikz upper}或类似工具绘制的图片是不外化的。
\end{docTcbKey}

\begin{marker}
If a |tcolorbox| is used inside a node of an encircling |tikzpicture| which is externalized,
do \emph{not} use |\tikzexternaldisable| in front of the |tcolorbox|.
\refKeyLe{/tcb/shield externalize} is deactivated automatically inside a |tikzpicture|.

如果在一个外部化的 |tikzpicture| 的节点中使用了 |tcolorbox|,则不要在 |tcolorbox| 前使用 |\tikzexternaldisable|。|tikzpicture| 中会自动禁用 \refKeyLe{/tcb/shield externalize}。
% 如果|tcolorbox|在外部化的,被|tikzpicture|包围的节点内使用,%
% 在 |tcolorbox| 之前不要使用 |\tikzexternaldisable| 。%
% \refKeyLe{/tcb/shield externalize}在|tikzpicture|中自动停用。
\end{marker}



\begin{marker}
\refKeyLe{/tcb/shield externalize} is applied for every following |tcolorbox|
inside the current \TeX\ group and is not affected by \refKeyLe{/tcb/reset}.

\refKeyLe{/tcb/shield externalize}应用于当前\TeX\ 组内的每个|tcolorbox|,并且不受\refKeyLe{/tcb/reset}的影响。
\end{marker}

\begin{docTcbKey}{external}{=\meta{file name}}{no default, initially unset}
Convenience option which calls |\tikzsetnextfilename|\marg{file name}. Typically,
it may be used inside the option list of a |tcolorbox| to set the
externalization \meta{file name} for the first |tikzpicture| which is discovered
\emph{inside} the box content.
The package |tikz| \cite{tantau:tikz_and_pgf} or the library \mylib{skins} has to be loaded to use this option.
Additionally, |\usetikzlibrary{external}| has to be used.

方便选项,调用|\tikzsetnextfilename|\marg{file name}。%
通常,它可以在|tcolorbox|的选项列表中使用,%
以设置在盒子内容中发现的,第一个|tikzpicture|的外部化\meta{file name}。%

The package |tikz| \cite{tantau:tikz_and_pgf} or the library \mylib{skins} has to be loaded to use this option.
Additionally, |\usetikzlibrary{external}| has to be used.

必须加载包|tikz|或库\mylib{skins}才能使用此选项。%
此外,必须设置|\usetikzlibrary{external}|。
\end{docTcbKey}



\begin{docTcbKey}{remake}{\colOpt{=true\textbar false}}{default |true|, initially |false|}
Convenience option which calls |/tikz/external/remake next|. Typically,
it may be used inside the option list of a |tcolorbox| to force the remake
of the first |tikzpicture| which is discovered \emph{inside} the box content.
The package |tikz| \cite{tantau:tikz_and_pgf} or the library \mylib{skins} has to be loaded to use this option.
Additionally, |\usetikzlibrary{external}| has to be used.

方便选项,调用|/tikz/external/remake next|. %
通常,它可以在|tcolorbox|的选项列表中使用,以强制重绘在盒子内容中发现的第一个|tikzpicture|。%
必须加载包|tikz|或\mylib{skins}库才能使用此选项。%
此外,必须设置|\usetikzlibrary{external}|。
\end{docTcbKey}
