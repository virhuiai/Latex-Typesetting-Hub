 With an overlay, arbitrary \meta{graphical code} can be added to a
 |tcolorbox|. This code is executed \emph{after} the frame and interior are
 drawn and \emph{before} the text content is drawn. Therefore, you can
 decorate the |tcolorbox| with your own extensions.
 Common special cases are \emph{watermarks} which are implemented using overlays.
 See Subsection \ref{subsec:watermarks} from page \pageref{subsec:watermarks} if
 you want to add \emph{watermarks}.

通过覆盖物,可以向 |tcolorbox| 添加任意的\meta{图形代码}。这些代码在边框和内部绘制完毕后执行,而在文本内容绘制之前执行。因此,您可以使用自己的扩展来装饰 |tcolorbox|。
常见的特殊情况是使用覆盖物实现的\emph{水印}。如果您想要添加\emph{水印},请参见第 \pageref{subsec:watermarks} 页的第 \ref{subsec:watermarks} 小节。

% 通过overlay, 可以将任意的 \meta{graphical code} 添加到 |tcolorbox| 中。
% 这部分代码附加到边框和内部%interior
% \emph{之后}、文本内容绘制\emph{之前}。 因此,你可以自行扩展装饰 |tcolorbox| 环境。常见的一个特定情况是使用overlays实现\emph{水印}。
% 如果你想添加\emph{水印},见\pageref{subsec:watermarks}页的\ref{subsec:watermarks}。

\begin{marker}
 If you use the core package only, the \meta{graphical code} has to be |pgf| code
 and there is not much assistance for positioning.
 Therefore, the usage of the \refKeyLe{/tcb/enhanced} mode from the library skins
 is recommended which allows |tikz| code and gives access to
 \refKeyLe{/tcb/geometry nodes} for positioning.

如果您仅使用核心包, \meta{graphical code} 必须是 |pgf| 代码,而且也没有太多的定位辅助。因此, 推荐使用 |skins| 库的 \refKeyLe{/tcb/enhanced} 模式,它不仅允许 |tikz| 代码,还允许用 \refKeyLe{/tcb/geometry nodes} 进行定位。
\end{marker}


\begin{docTcbKey}{overlay}{=\meta{graphical code}}{no default, initially unset}
 Adds \meta{graphical code} to the box drawing process. This \meta{graphical code}
 is drawn \emph{after} the frame and interior and \emph{before} the text content.

将 \meta{graphical code} 添加到盒子的绘制过程中。\meta{graphical code}将附加到边框和内部的绘制\emph{之后}、文本内容\emph{之前}。

\begin{exdispExample}{overlay_1}
 %% \tcbuselibrary{skins} % preamble
\tcbset{frogbox/.style={enhanced,
colback=green!10,colframe=green!65!black,
enlarge top by=5.5mm,%在上方扩大5.5mm的空间,用于放置青蛙图案;
overlay=%在tcolorbox的上方覆盖一个图案;
{\foreach \x in {2cm,3.5cm} {
%循环两次,分别在距离tcolorbox左上角2cm和3.5cm的位置上放置青蛙图案;
\begin{scope}[shift={([xshift=\x]frame.north west)}]
%进入一个局部坐标系,将坐标系原点移到指定位置;
\path[draw=green!65!black,fill=green!10,line width=1mm] (0,0) arc (0:180:5mm);
%绘制一条绿色的线条,宽度为1mm,半径为5mm,形成一个半圆;
\path[fill=black] (-0.2,0) arc (0:180:1mm);
% 在半圆的左侧绘制一个黑色的小圆,作为青蛙的眼睛;
    \end{scope}}}}}

% 接下来的代码使用了定义好的"frogbox"样式,创建了一个tcolorbox,标题为"My title",并在内部填写了一些文本。
\begin{tcolorbox}[frogbox,title=My title]
This is a \textbf{tcolorbox}.
\end{tcolorbox}
\end{exdispExample}

% 这段latex代码定义了一个名为"frogbox"的tcolorbox样式,具体说明如下:

% enhanced:启用tcolorbox的增强模式;
% colback=green!10:设置背景颜色为绿色的10%;
% colframe=green!65!black:设置边框颜色为绿色的65%和黑色的35%;


\enlargethispage*{5mm}
\begin{exdispExample}{overlay_2}
%%\usetikzlibrary{patterns} % preamble
%% \tcbuselibrary{skins}     % preamble
% 这段代码定义了一个新的tcolorbox风格"ribbonbox",包括以下属性:
\tcbset{ribbonbox/.style={enhanced,%增强选项,允许在盒子中使用更多的TikZ选项。
colback=red!5!white,%背景颜色为红色的5%和白色的95%混合色。
colframe=red!75!black,%框架边框颜色为红色的75%和黑色的25%混合色。
fonttitle=\bfseries,%标题字体粗体。
overlay={%覆盖选项,允许在盒子上覆盖其他图形元素。
%在这里,我们使用TikZ绘制了一个蓝色且带有五角星纹理的黄色丝带,用于装饰盒子的右上角。
% path:绘制路径。
\path[fill=blue!75!white,%填充颜色为蓝色的75%和白色的25%混合色。
draw=blue,%绘制颜色为蓝色。
double=white!85!blue,%双重边框颜色为白色的85%和蓝色的15%混合色。
% preaction:在绘制路径之前执行的操作。
% 在这里,我们将填充颜色设置为蓝色的75%和白色的25%混合色,不透明度为0.6。
preaction={opacity=0.6,fill=blue!75!white},
line width=0.1mm,double distance=0.2mm,
%绘制线宽为0.1毫米。双重边框之间的距离为0.2毫米。
pattern=fivepointed stars,%填充纹理为五角星。
pattern color=white!75!blue]%填充纹理颜色为白色的75%和蓝色的25%混合色。
([xshift=-0.2mm,yshift=-1.02cm]frame.north east)
-- ++(-1,1) -- ++(-0.5,0) -- ++(1.5,-1.5) -- cycle;}}}

\begin{tcolorbox}[ribbonbox,title=My title]
This is a \textbf{tcolorbox}.
\tcblower
This is the lower part.
\end{tcolorbox}
\end{exdispExample}
\end{docTcbKey}


% 然后,我们使用这个新的风格来创建一个tcolorbox,标题为"My title",并在盒子中包含一些文本。下面的"tcblower"选项将文本分成两个部分,分别放置在盒子的上下部分。


%%%% \clearpage
\begin{docTcbKey}{no overlay}{}{style, no default, initially set}
 Removes the overlay if set before.

% 移除覆盖层。
如果之前设置了浮层,将其移除。
\end{docTcbKey}

\begin{docTcbKey}{overlay broken}{=\meta{graphical code}}{no default, initially unset}
 If the box is set to be \refKeyLe{/tcb/breakable} and \emph{is} broken actually,
 then the \meta{graphical code} is added to the box drawing process.
 \refKeyLe{/tcb/overlay} overwrites this key.

如果盒子被设置为 \refKeyLe{/tcb/breakable} 并且实际上 \emph{被}打破了, 那么 \meta{graphical code} 将添加到盒子绘制过程中。 \refKeyLe{/tcb/overlay} 将覆盖此键。

% 如果盒子设置为\refKeyLe{/tcb/breakable}且\emph{实际上}分页了,那么\meta{graphical code}会被添加到盒子的绘制过程中。\refKeyLe{/tcb/overlay}会覆盖此项设置。\footnote{译注:即实际上分页时才添加覆盖层。}
\end{docTcbKey}

\begin{docTcbKey}{overlay unbroken}{=\meta{graphical code}}{no default, initially unset}
 If the box is set to be \refKeyLe{/tcb/breakable} but \emph{is not} broken actually
 or if the box is set to be \refKeyLe{/tcb/unbreakable},
 then the \meta{graphical code} is added to the box drawing process.
 \refKeyLe{/tcb/overlay} overwrites this key.

如果盒子设置为\refKeyLe{/tcb/breakable},但实际上\emph{没有}被打破,或者盒子设置为\refKeyLe{/tcb/unbreakable},则\meta{graphical code}将被添加到盒子绘制过程中。\refKeyLe{/tcb/overlay}将覆盖此键。

% 如果盒子设置了\refKeyLe{/tcb/breakable}但\emph{实际上没}分页,或盒子设置为\refKeyLe{/tcb/unbreakable},那么\meta{graphical code}会被添加到盒子的绘制过程中。\refKeyLe{/tcb/overlay}会覆盖此项设置。\footnote{译注:即实际上没有分页时才添加覆盖层。}
\end{docTcbKey}



\begin{docTcbKey}{overlay first}{=\meta{graphical code}}{no default, initially unset}
 If the box is set to be \refKeyLe{/tcb/breakable} and \emph{is} broken actually,
 then the \meta{graphical code} is added to the box drawing process for
 the \emph{first} part of the break sequence.
 \refKeyLe{/tcb/overlay} overwrites this key.

%  如果该盒子被设置为 \refKeyLe{/tcb/breakable} 并且实际上已经被打破, 那么对于断裂序列的 \emph{第一个} 部分,\meta{graphical code} 将添加到盒子绘制过程中。 \refKeyLe{/tcb/overlay} 将覆盖此键。

如果盒子设置了\refKeyLe{/tcb/breakable}且\emph{事实上}分页了,
那么\meta{graphical code}会被添加到盒子的分开的\emph{第一}部分的绘制过程中。
\refKeyLe{/tcb/overlay}会覆盖此项设置。
\end{docTcbKey}

\begin{docTcbKey}{overlay middle}{=\meta{graphical code}}{no default, initially unset}
 If the box is set to be \refKeyLe{/tcb/breakable} and \emph{is} broken actually,
 then the \meta{graphical code} is added to the box drawing process for
 the \emph{middle} parts (if any) of the break sequence.
 \refKeyLe{/tcb/overlay} overwrites this key.

%  如果盒子被设置为\refKeyLe{/tcb/breakable}并且实际上已经被分割, 则\meta{graphical code}将被添加到盒子绘制过程中的\emph{中间}部分(如果有的话)的分割序列中。 \refKeyLe{/tcb/overlay}将覆盖此键。

如果盒子设置了\refKeyLe{/tcb/breakable}且\emph{事实上}分页了,
那么\meta{graphical code}会被添加到盒子的分开的序列中的\emph{中间}部分(如果有)。 \refKeyLe{/tcb/overlay}会覆盖此项设置。
\end{docTcbKey}

\begin{docTcbKey}{overlay last}{=\meta{graphical code}}{no default, initially unset}
 If the box is set to be \refKeyLe{/tcb/breakable} and \emph{is} broken actually,
 then the \meta{graphical code} is added to the box drawing process for
 the \emph{last} part of the break sequence.
 \refKeyLe{/tcb/overlay} overwrites this key.

%  如果盒子设置为 \refKeyLe{/tcb/breakable} 并且实际上已经被打破, 则 \meta{graphical code} 将被添加到盒子绘制过程的最后一个断裂序列的部分中。 \refKeyLe{/tcb/overlay} 覆盖此键。

如果盒子设置了\refKeyLe{/tcb/breakable}且\emph{事实上}分页了,
那么\meta{graphical code}会被添加到盒子的分开的\emph{最后}一部分的绘制过程中。
\refKeyLe{/tcb/overlay}会覆盖此项设置。
\end{docTcbKey}


\begin{docTcbKey}{overlay unbroken and first}{=\meta{graphical code}}{no default, initially unset}
 This is an optimized abbreviation for setting
 \refKeyLe{/tcb/overlay unbroken} and
 \refKeyLe{/tcb/overlay first} together.
 \refKeyLe{/tcb/overlay} overwrites this key.

这是同时设置\refKeyLe{/tcb/overlay unbroken}和\refKeyLe{/tcb/overlay first}的优化缩写。
\refKeyLe{/tcb/overlay}会覆盖此项设置。
\end{docTcbKey}

\begin{docTcbKey}{overlay middle and last}{=\meta{graphical code}}{no default, initially unset}
 This is an optimized abbreviation for setting
 \refKeyLe{/tcb/overlay middle} and
 \refKeyLe{/tcb/overlay last} together.
 \refKeyLe{/tcb/overlay} overwrites this key.

这是同时设置\refKeyLe{/tcb/overlay middle}和\refKeyLe{/tcb/overlay last}的优化缩写。
\refKeyLe{/tcb/overlay}会覆盖此项设置。
\end{docTcbKey}

\begin{docTcbKey}{overlay unbroken and last}{=\meta{graphical code}}{no default, initially unset}
 This is an optimized abbreviation for setting
 \refKeyLe{/tcb/overlay unbroken} and
 \refKeyLe{/tcb/overlay last} together.
 \refKeyLe{/tcb/overlay} overwrites this key.

这是同时设置\refKeyLe{/tcb/overlay unbroken}和\refKeyLe{/tcb/overlay last}的优化缩写。
\refKeyLe{/tcb/overlay}会覆盖此项设置。
\end{docTcbKey}





\begin{docTcbKey}[][doc new=2014-09-19]{overlay first and middle}{=\meta{graphical code}}{no default, initially unset}
 This is an optimized abbreviation for setting
 \refKeyLe{/tcb/overlay first} and \refKeyLe{/tcb/overlay middle} together.
 \refKeyLe{/tcb/overlay} overwrites this key.

这是同时设置\refKeyLe{/tcb/overlay first}和\refKeyLe{/tcb/overlay middle}的优化缩写。
\refKeyLe{/tcb/overlay}会覆盖此项设置。
\end{docTcbKey}

% 这段Latex代码定义了一个新的环境myexample,它是基于tcolorbox的,具有以下特点:
\begin{dispListing*}{breakable,vfill before first,before upper={This example demonstrates
the application of break sequence specific overlay options.
Here, we define an environment |myexample| based
on |tcolorbox| where the visible drawing is done totally by overlay keys.\par
Here, the first application of |myexample| produces an unbroken |tcolorbox|.
The frame is drawn by the code given with \refKeyLe{/tcb/overlay unbroken}.\par
The second application of |myexample| is broken into several parts which
are drawn by the codes given with
\refKeyLe{/tcb/overlay first}, \refKeyLe{/tcb/overlay middle}, and
\refKeyLe{/tcb/overlay last}.
\par\bigskip
}}
    Preamble:
%% \usepackage{tikz,lipsum}
%% \tcbuselibrary{skins,breakable}
%%\newcounter{example}
\colorlet{colexam}{red!75!black}
\newtcolorbox[use counter=example]%%计数器使用example
{myexample}{%
empty,%环境内没有默认文本
title={Example \thetcbcounter},
% 自定义标题:标题格式为Example X,其中X为环境的计数器。
attach boxed title to top left,%标题框线附着在左上角
% 自定义标题框线的样式
boxed title style={empty,size=minimal,toprule=2pt,top=4pt,
    overlay={\draw[colexam,line width=2pt]
    ([yshift=-1pt]frame.north west)--([yshift=-1pt]frame.north east);}},
coltitle=colexam,fonttitle=\Large\bfseries,
before=\par\medskip\noindent,%环境前的文本
parbox=false,%环境内不使用parbox
boxsep=0pt,%边框线与内容的间距为0
left=0pt,right=3mm,top=4pt,
%left=0pt表示左边距为0,right=3mm表示右边距为3mm,top=4pt表示上边距为4pt
breakable,%该环境可以跨页
pad at break*=0mm,%跨页时边框线不需要额外的空白
vfill before first,%第一页时环境顶部需要添加额外的空白。
%接下来,定义了四个tcolorbox中的overlay,
%分别表示未跨页、第一页、中间页和最后一页时边框线的样式。
overlay unbroken={\draw[colexam,line width=1pt]
([yshift=-1pt]title.north east)
%表示标题区域的右上角点,向下移动1pt(点)。
--([xshift=-0.5pt,yshift=-1pt]title.north-|frame.east)
%从标题区域的右上角点,向下移动1pt并向左移动0.5pt(水平方向的一半),再与框架区域的右侧线的交点。
--([xshift=-0.5pt]frame.south east)
% 表示从框架区域的右下角点,向左移动0.5pt,再与上一步得到的交点相连。
--(frame.south west); },
%从框架区域的左下角点,与上一步得到的线条相连。
% 整个绘图过程形成了一个闭合的多边形,线条宽度为1pt(点),
% 颜色为colexam。代码中的overlay unbroken表示这个线条
%会出现在所有页面中,不会被分割。
overlay first={\draw[colexam,line width=1pt]%指定绘制线条的颜色和宽度。
% 在这个例子中,我们将线条颜色设置为“colexam”(可能是在主题中定义的一种颜色),并将线条宽度设置为1个点(pt)。
([yshift=-1pt]title.north east)
%指定线条的起点。在这个例子中,我们将线条起点设置在标题框架的右上角,向下移动1个点(pt)。
--([xshift=-0.5pt,yshift=-1pt]title.north-|frame.east)
% 指定线条从起点到终点的路径。在这个例子中,我们将线条路径设置为从起点沿着标题框架的上边缘向
%右下方偏移0.5个点(pt),然后沿着框架的右边缘向下偏移1个点(pt),直到达到框架的右下角。
--([xshift=-0.5pt]frame.south east); },
% 指定线条的最后一个点。在这个例子中,我们将线条终点设置为框架的右下角向左偏移0.5个点(pt)。
overlay middle={\draw[colexam,line width=1pt] %绘制一条颜色为colexam,线宽为1pt的线条。
([xshift=-0.5pt]frame.north east)
%从框架右上角开始绘制线条,[xshift=-0.5pt]表示将线条向左偏移0.5pt,使得线条在框架内部。
--([xshift=-0.5pt]frame.south east); },
%绘制一条直线,终点为框架右下角,同样向左偏移0.5pt,与上面的线条相连成垂直线条。
overlay last={\draw[colexam,line width=1pt]%绘制一条颜色为colexam、线宽为1pt的线。
 ([xshift=-0.5pt]frame.north east)
% 从frame的东北角向左平移0.5pt作为线的起点
--([xshift=-0.5pt]frame.south east)--(frame.south west);},%
% 从frame的东南角向左平移0.5pt作为线的中间点,最后到frame的西南角作为线的终点。
%因此,这段代码的作用是在frame的右侧绘制一条从顶部到底部的线。
}

\begin{myexample}
\lipsum[1]
\end{myexample}

\begin{myexample}
\lipsum[2-11]
\end{myexample}

\lipsum[12]% following text
\end{dispListing*}
{\tcbusetemp}


\begin{dispExample}
% %% \tcbuselibrary{skins}
%% \newcounter{example}
\newtcolorbox[use counter=example]{FancyTitle}[3][]{%
  enhanced,colback=blue!10!white,colframe=orange,top=4mm,
  enlarge top by=\baselineskip/2+1mm,
  enlarge top at break by=0mm,pad at break=2mm,
  fontupper=\normalsize,label={#3},
  overlay unbroken and first={%
    \node[rectangle,rounded corners,draw=black,fill=blue!20!white,
      inner sep=1mm,anchor=west,font=\small]
      at ([xshift=4.5mm]frame.north west)
         {\strut\textbf{Example \thetcbcounter: #2}};},
  #1}%

\begin{FancyTitle}{My fancy title}{fancy:title}
  \lipsum[1]
\end{FancyTitle}
\end{dispExample}