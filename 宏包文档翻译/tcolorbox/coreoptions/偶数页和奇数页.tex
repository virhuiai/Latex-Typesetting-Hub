
\begin{marker}
Also see \refKeyLe{/tcb/toggle left and right} and \refKeyLe{/tcb/toggle enlargement} for further even/odd options.

也可以参考\refKeyLe{/tcb/toggle left and right}和\refKeyLe{/tcb/toggle enlargement}了解更多的偶数/奇数选项。
\end{marker}

\begin{docTcbKey}[][doc updated=2015-11-13]{check odd page}{\colOpt{=true\textbar false}}{default |true|, initially |false|}
If set to |true|, a precise even/odd page testing for the current box is applied. 
This is done by using labels. If a box moves to another page,
the document has to be compiled twice for the correct settings.
If set to |false|, even/odd page tests may give wrong results for the first box of a page.


如果设置为|true|,则对当前盒子应用精确的偶数/奇数页测试。%
这是通过使用标签来实现的。%
如果一个盒子移动到另一页,%
为了获得正确的设置,必须编译两次文档。%
如果设置为|false|,偶数/奇数页测试可能会对页面的第一个盒子给出错误的结果。

 \refKeyLe{/tcb/toggle left and right},
 \refKeyLe{/tcb/toggle enlargement}, and
 \refKeyLe{/tcb/if odd page}
 automatically set |check odd page|, but for
 \refCom{tcbifoddpage} this option has to be set explicitely.
\refKeyLe{/tcb/toggle left and right},
\refKeyLe{/tcb/toggle enlargement}, 和
\refKeyLe{/tcb/if odd page}
自动设置|check odd page|, 但是对于\refCom{tcbifoddpage},这个选项必须显式设置。
\end{docTcbKey}


%  \enlargethispage*{1cm}
\begin{docTcbKey}[][doc new=2015-11-13]{if odd page}{=\marg{odd options}\marg{even options}}{style, no default}
 If the current box is on an odd page, the \meta{odd options} are applied.
 On an even page, the \meta{even options} are applied.
 \refKeyLe{/tcb/check odd page} is automatically set for precise even/odd page testing.

如果盒子当前位于奇数页,则应用\meta{odd options}。%
如果是偶数页, 则应用\meta{even options}。%
\refKeyLe{/tcb/check odd page}会自动设置,用于精确的偶数/奇数页测试。

\begin{dispExample}
\begin{tcolorbox}[if odd page={colback=yellow!50}{colback=red!50}]
这个盒子在奇数页上是黄色的,在偶数页上是红色的。
\end{tcolorbox}
\end{dispExample}

\begin{marker}
 If a box is \refKeyLe{/tcb/breakable}, using \refKeyLe{/tcb/if odd page}
 only acts upon the \emph{first} box. If the setting should be
 repeated for every partial box of the break sequence, the option should be
 packed into \refKeyLe{/tcb/extras}. In this case, \refKeyLe{/tcb/check odd page}
 has to be set explicitely! Also see \refKeyLe{/tcb/if odd page*}.

如果盒子设置了 \refKeyLe{/tcb/breakable}, 则 \refKeyLe{/tcb/if odd page} 只在\emph{第一}部分盒子生效。%
如果对中断序列的每个部分框都要重复设置,则选项应设到\refKeyLe{/tcb/extras}。%
在这种情况下,\refKeyLe{/tcb/check odd page}必须显式设置! 另见 \refKeyLe{/tcb/if odd page*}.
\end{marker}
\end{docTcbKey}




\begin{docTcbKey}[][doc new=2016-11-18]{if odd page or oneside}{=\marg{odd options}\marg{even options}}{style, no default}
 For onesided documents, the \meta{odd options} are applied always.
 For twosided documents, this style is identical to \refKeyLe{/tcb/if odd page}.

对于单开文档,总是应用\meta{odd选项}。%
对于双面文档,此样式与\refKeyLe{/tcb/if odd page}相同。
\end{docTcbKey}

 %%\clearpage
\begin{docTcbKey}[][doc new=2015-11-13]{if odd page*}{=\marg{odd options}\marg{even options}}{style, no default}
\begin{marker}
 This option needs the \mylib{breakable} library, see \Fullref{sec:breakable}.
这个选项需要\mylib{breakable}库,参见\Fullref{sec:breakable}。
\end{marker}
 For breakable boxes, if the current partial box is on an odd page, the \meta{odd options} are applied.
 On an even page, the \meta{even options} are applied.
 \refKeyLe{/tcb/check odd page} is automatically set for precise even/odd page testing.

对于可分的盒子,如果当前盒子部分位于奇数页,则应用\meta{odd选项}。%
在偶数页上,应用\meta{even选项}。%
\refKeyLe{/tcb/check odd page}会被自动设置,以用于精确的偶数/奇数页测试。

 In contrast to \refKeyLe{/tcb/if odd page}, \refKeyLe{/tcb/if odd page*} is used
 on \emph{every} partial box of a break sequences and not only on the
 \emph{first} box. Another difference is that \refKeyLe{/tcb/if odd page*}
 is applied quite \emph{late} during option processing, while
 \refKeyLe{/tcb/if odd page} is applied immediately.
同\refKeyLe{/tcb/if odd page}对比, \refKeyLe{/tcb/if odd page*} 作用于中断序列的每一个部分的盒子,而不仅仅是对
第一个盒子。%
另一个区别是, \refKeyLe{/tcb/if odd page*} 在选项处理过程中应用较晚,而 \refKeyLe{/tcb/if odd page} 立即生效。

 \refKeyLe{/tcb/if odd page*} is implemented as \refKeyLe{/tcb/if odd page}
 packed into \refKeyLe{/tcb/extras}.

\refKeyLe{/tcb/if odd page*} 是在 \refKeyLe{/tcb/extras} 中的一个 \refKeyLe{/tcb/if odd page} 的实现。

\begin{dispExample}
 %%\tcbuselibrary{breakable}
\begin{tcolorbox}[breakable,if odd page*={colback=yellow!50}{colback=red!50}]
这个可中断盒子在奇数页上是黄色的,在偶数页上是红色的。%
对于每一个部分盒子,测试都会重复执行, i.e. 对于长文本,这会得到黄色,红色,黄色,红色, \ldots\ 这样的序列。
\end{tcolorbox}
\end{dispExample}
\end{docTcbKey}




\begin{docTcbKey}[][doc new=2016-11-18]{if odd page or oneside*}{=\marg{odd options}\marg{even options}}{style, no default}
 For onesided documents, the \meta{odd options} are applied always.
 For twosided documents, this style is identical to \refKeyLe{/tcb/if odd page*}.

对于单开页文档,总是应用\meta{odd选项}。%
对于双开页文档,此项与\refKeyLe{/tcb/if odd page*}相同。
\end{docTcbKey}



 %%\clearpage
\begin{docCommand}[doc new=2015-11-13]{tcbifoddpage}{\marg{odd code}\marg{even code}}
 If the current box is on an odd page, the \meta{odd code} is executed.
 On an even page, the \meta{even code} is executed.
 For precise even/odd page testing, the \refKeyLe{/tcb/check odd page} has to be
 set manually inside the box options.

如果当前盒子位于奇数页上,则执行\meta{odd code}。%
在偶数页上,执行\meta{even code}。%
对于精确的偶数/奇数页测试,\refKeyLe{/tcb/check odd page}必须在盒子的选项中手动设置。

 The macro \refCom{tcbifoddpage} can be used inside underlay, overlay, or watermark code to
 test if the box is on an odd page. This will work also for boxes in a break sequence.
宏\refCom{tcbifoddpage}可以在底层、覆盖或水印代码中使用,以测试框是否在奇数页上。这也适用于中断序列中的盒子。

 The macro can also be used inside the box \textbf{content text}. For unbreakable boxes,
 the correct page test is applied.
 But for \refKeyLe{/tcb/breakable} boxes, \refCom{tcbifoddpage}
 will always give the result for the page of the \emph{first} box inside
 the box \textbf{content text}. If needed, the methods from the packages
 |changepage| or |ifoddpage| could be used here.
宏也可以在盒子的\textbf{内容文本}内使用。对于不可分的盒子,应用正确的页面测试。%
但是对于\refKeyLe{/tcb/breakable}的盒子, \refCom{tcbifodpage}将始终给出盒子的内容文本的第一部分所在页面的结果。 如果需要,这里可以使用包|changepage|或|ifoddpage|中的方法。
To mention it again, for overlays, watermarks, etc, \refCom{tcbifoddpage} gives
the correct page test.

\begin{dispExample}
\tcbset{colframe=blue!75!black,colback=white,fonttitle=\bfseries}

\begin{tcolorbox}[enhanced,check odd page,
title={Example for a box on an \tcbifoddpage{odd}{even} page},
watermark text={\tcbifoddpage{Odd}{Even} page!}]
\lipsum[1]
\end{tcolorbox}
\end{dispExample}
\end{docCommand}




\begin{docCommand}[doc new=2016-11-18]{tcbifoddpageoroneside}{\marg{odd code}\marg{even code}}
 For onesided documents, the \meta{odd code} is executed always.
 For twosided documents, this macro is identical to \refCom{tcbifoddpage}.
对于单开页文档,总是执行\meta{odd code}。%
对于双开页文档,这个宏与\refCom{tcbifoddpage}相同。
\end{docCommand}


 %%\clearpage
\begin{docCommand}[doc new=2015-11-13]{thetcolorboxnumber}{}
 This is a unique identifier (arabic number) for a tcolorbox. It is locally
 defined inside boxes and has no meaning outside. It is used for
 precise even/odd page testing, but may also be valuable for elaborate user
 code.

这是tcolorbox盒子的唯一标识符(阿拉伯数字)。 %
它在盒子内局部定义,在盒子外没有意义。%
它用于精确的偶数/奇数页测试,但对于复杂的用户代码也很有价值。

\begin{dispExample}
\begin{tcolorbox}[colback=yellow!5,title=Box \thetcolorboxnumber]
本盒号\thetcolorboxnumber.
\tcbox[on line,size=fbox]{本盒号\thetcolorboxnumber} 然后
\tcbox[on line,size=fbox]{本盒号\thetcolorboxnumber}.
本盒号 \thetcolorboxnumber.
\end{tcolorbox}
\end{dispExample}
\end{docCommand}



\begin{docCommand}[doc new=2015-11-13]{thetcolorboxpage}{}
 This macro contains the expanded arabic page number of the current tcolorbox.
 It is locally defined inside boxes and has no meaning outside.
 It is precise only, if \refKeyLe{/tcb/check odd page} was set.

这个宏包含当前tcolorbox盒子所在页的page计数器的阿拉伯数字页码。%
它在盒子内局部定义,在盒子外没有意义。%
只有当\refKeyLe{/tcb/check odd page}被设置时,它才是精确的。

\begin{dispExample}
\begin{tcolorbox}[colback=yellow!5,check odd page,
    title=Box on page~\thetcolorboxpage]
这个盒子位于~\thetcolorboxpage~页。
\end{tcolorbox}
\end{dispExample}
\end{docCommand}
