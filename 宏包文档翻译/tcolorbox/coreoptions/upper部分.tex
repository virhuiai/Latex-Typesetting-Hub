The text content of a \refEnv{tcolorbox} may be parted into a mandatory \emph{upper part}
and an optional \emph{lower part}. These parts are separated by
\refCom{tcblower}. If there is no \refCom{tcblower} present, there is no
\emph{lower part} and the \emph{upper part} forms the complete text content.

% \refEnv{tcolorbox}中的文本内容被分为 \emph{upper 部分} 和一个可选的 \emph{lower 部分}。这两部分由命令 \refCom{tcblower}分隔。如果没出现 \refCom{tcblower} ,那么就没有 \emph{lower 部分},\emph{upper 部分}就是所有的内容。

\refEnv{tcolorbox} 的文本内容可以分为一个必需的\emph{上部分}和一个可选的\emph{下部分}。这些部分由\refCom{tcblower}分隔。如果没有\refCom{tcblower},则没有\emph{下部分},而\emph{上部分}构成完整的文本内容。

\begin{docTcbKey}[][doc new=2015-01-06]{upperbox}{=\meta{mode}}{no default, initially \texttt{visible}}
Controls the treatment of the upper part of the box. If there is no lower part,  this is the complete text content.
Feasible values for \meta{mode} are:

控制例子的upper部分的显示处理。如果没有lower部分, upper部分内容即是所有内容。
可设置的 \meta{mode} 值有:
\begin{DescriptionL}{\docValue{invisible}}
\item[\docValue{visible}]usual type setting of the upper part,
\par 可见,upper部分的常用设定
\item[\docValue{invisible}] empty space instead of the upper part contents.
\par 不可见,upper部分内容显示为空白。
\end{DescriptionL}
\begin{exdispExample}{upperbox}
\begin{tcolorbox}[upperbox=invisible,colback=white%
,title=upperbox设置为invisible且没有lower部分]
这是一个\textbf{tcolorbox}(但是是隐形的)。
\end{tcolorbox}

\begin{tcolorbox}[upperbox=invisible,colback=white%
,title=upperbox设置为invisible时只显示lower部分]
这是一个\textbf{tcolorbox}(但是是隐形的)。
\tcblower
这是下部分。
\end{tcolorbox}
\end{exdispExample}
\end{docTcbKey}


\begin{docTcbKey}[][doc new and updated={2015-01-06}{2019-03-01}]{visible}{}{style, no value}
Shortcut for setting \refKey{/tcb/upperbox}, \refKey{/tcb/lowerbox}, and \refKey{/tcb/titlebox}
to be \docValue{visible}.

同时设置 \refKey{/tcb/upperbox}, \refKey{/tcb/lowerbox}, 和 \refKey{/tcb/titlebox} 为 \docValue{visible} 的快捷方式。
\end{docTcbKey}

\begin{docTcbKey}[][doc new and updated={2015-01-06}{2019-03-01}]{invisible}{}{style, no value}
Shortcut for setting \refKey{/tcb/upperbox}, \refKey{/tcb/lowerbox}, and \refKey{/tcb/titlebox}
to be \docValue{invisible}.

设置 \refKey{/tcb/upperbox}, \refKey{/tcb/lowerbox}, 和 \refKey{/tcb/titlebox} 为 \docValue{invisible} 的快捷方式。
\begin{exdispExample}{invisible}
\begin{tcolorbox}[invisible]
这是一个\textbf{tcolorbox}(但是是隐形的)。
\tcblower
这是\textbf{下部分}(但是是隐形的)。
\end{tcolorbox}
\end{exdispExample}
\end{docTcbKey}




% \clearpage
\begin{docTcbKey}[][doc new=2015-05-04]{saveto}{=\meta{file name}}{no default, initially empty}
Saves the content of the box into a file for an optional later usage.
This is the counterpart of \refKey{/tcb/savelowerto}, but is saves not
only the upper part but the whole content. If a lower part is present,
it is also saved including \refCom{tcblower}.

% 将盒子的内容保存到一个文件中,以供以后使用。
% 这和 \refKey{/tcb/savelowerto} 类似, 但它不仅保存了upper部分,是保存了整个内容。
% 如果存在lower部分,也会保存且包含 \refCom{tcblower}。

将框的内容保存到文件中,以备将来需要时使用。这是\refKey{/tcb/savelowerto}的对应项,但不仅保存上部分,而是保存整个内容。如果存在下部分,则也会保存,包括\refCom{tcblower}。

\begin{marker}
This option cannot be combined with \refKey{/tcb/savelowerto}.

此项不能同 \refKey{/tcb/savelowerto} 组合使用.
\end{marker}

\begin{exdispExample}{saveto_1}
\begin{tcolorbox}[invisible%upper、lower都不显示
,saveto=\jobname_mysave1.tex
,colback=white]
这是一个\textbf{tcolorbox},使用invisible后,看着是空的。
它的内容被saveto暂存到指定的文件中,后续再在其他位置包含进来使用。
\end{tcolorbox}

现在,我们包含进来保存好的文本内容:\\
\input{\jobname_mysave1.tex}
\end{exdispExample}

\begin{exdispExample}{saveto_2}
\begin{tcolorbox}[saveto=\jobname_mysave2.tex]
这是一个\textbf{tcolorbox}.
\tcblower
这是lower部分。
\end{tcolorbox}

现在,我们包含进来保存好的文本内容:
\begin{tcolorbox}[colframe=red,colback=red!10,
coltitle=black,colbacktitle=red!20
,sidebyside%从上下改为左右
,title=在这里我们看到保存的内容包括lower部分]
\input{\jobname_mysave2.tex}
\end{tcolorbox}
\end{exdispExample}
\end{docTcbKey}

% 可以看下这个 mysave2.tex 文件,内容是:

% 这是一个\textbf{tcolorbox}.
% \tcblower
% 这是lower部分。
