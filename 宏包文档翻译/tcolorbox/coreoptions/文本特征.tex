\setcounter{section}{4}
\setcounter{subsection}{17}
\setcounter{subsubsection}{0}
\subsection{Text Characteristics\\文本特征}

\begin{docTcbKey}[][doc updated=2015-10-14]{parbox}{\colOpt{=true\textbar false}}{default |true|, initially |true|}
The text inside a |tcolorbox| is formatted using a \LaTeX\ |minipage|
if the box is unbreakable. 
If breakable, the box tries a mimicry of a |minipage|. 
In a |minipage| or |parbox|, paragraphs are formatted slightly different
as the main text. If the key value is set to |false|, the normal main text
behavior is restored. In some situations, this has some unwanted side
effects. It is recommended that you use this experimental setting only
where you really want to have this feature.

如果 |tcolorbox| 是不分页的,那么其中的文本使用 \LaTeX\ 的 |minipage| 进行格式化。如果可分页,这个盒子会模拟 |minipage| 的行为。在 |minipage| 或 |parbox| 中,段落的格式稍有不同。如果将值设置为 |false|,则恢复正常的主文本行为。在某些情况下,这会产生一些不必要的副作用。建议仅在真正需要此功能的情况下使用此实验设置。

% 如果盒子是不可分的,|tcolorbox| 中的文本使用 \LaTeX\ |minipage| 格式化。
% 如果是可分的, 盒子试图模仿一个 |minipage|。%
% 文本在 |minipage| 和 |parbox| 中的格式处理会有略微的不同。%
% 如果这项值设为 |false|, 将恢复正常的主文本行为。%
% 在某些情况下,这会产生一些不必要的副作用。%
% 建议您只在真正希望具有此特性的地方使用此实验性设置。
\end{docTcbKey}

\begin{dispListing}
% \usepackage{lipsum}  % preamble
\tcbset{width=(\linewidth-2mm)/2,nobeforeafter,arc=1mm,
colframe=blue!75!black,colback=white,fonttitle=\bfseries,fontupper=\small,
left=2mm,right=2mm,top=1mm,bottom=1mm,equal height group=parbox}

\begin{tcolorbox}[parbox,adjusted title={parbox=true (normal)}]
\lipsum[1-2]
\end{tcolorbox}\hfill%
\begin{tcolorbox}[parbox=false,adjusted title={parbox=false}]
\lipsum[1-2]
\end{tcolorbox}%
\end{dispListing}
{\tcbusetemp}

\begin{引述之言}{virhuiai}
更喜欢 \verb|parbox=false|
\end{引述之言}



% \clearpage
\begin{docTcbKey}{hyphenationfix}{\colOpt{=true\textbar false}}{default |true|, initially |false|}
Long words at the beginning of paragraphs in very narrow boxes
will not be hyphenated using |pdflatex|. This problem is circumvented by
applying the |hyphenationfix| option.

使用|pdflatex|时,在非常狭窄的盒子中,段落开头的长单词,不会用连字符。%
通过应用|hyphenationfix|选项,可以规避此问题。
\begin{exdispExample*}{hyphenationfix}{sbs,lefthand ratio=0.6}
\tcbset{colframe=blue!75!black,
fontupper=\normalsize,
colback=blue!5!white,width=4cm}

\begin{tcolorbox}
Rechnungsadjunktentochter.\par
Statthaltereikonzipist.
\end{tcolorbox}

\begin{tcolorbox}[hyphenationfix]
Rechnungsadjunktentochter.\par
Statthaltereikonzipist.
\end{tcolorbox}
\end{exdispExample*}

\smallskip
\begin{marker}
|parbox=false| and |hyphenationfix| should not be used together. 
They are targeting different box types and they do not blend very well.

|parbox=false| 和 |hyphenationfix| 不应该一起使用。%
他们的目标是不同的盒子类型。%, 他们不能很好地融合。
\end{marker}
\end{docTcbKey}