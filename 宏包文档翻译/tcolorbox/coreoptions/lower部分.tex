\begin{docTcbKey}{lowerbox}{=\meta{mode}}{no default, initially \texttt{visible}}
Controls the treatment of the lower part of the box.
Feasible values for \meta{mode} are:

控制lower部分的显示情况。可选的 \meta{mode} 值有:
\begin{DescriptionL}{\docValue{invisible}}
\item[\docValue{visible}]usual type setting of the lower part,
\\可见,lower部分的常用设定

\item[\docValue{invisible}]empty space instead of the lower part contents,
\\不可见,lower部分内容显示为空白。

\item[\docValue{ignored}]the lower part is not used (here).
\\忽略,下部分在这儿没有用上。
\end{DescriptionL}

The last two values are usually applied in connection with |savelowerto|.

后两个选项值通常用于与 |savelowerto| 配合使用。

\begin{exdispExample}{lowerbox}
\begin{tcolorbox}[lowerbox=invisible,colback=white]
This is a \textbf{tcolorbox}.
\tcblower
这是lower部分(但不可见)
\end{tcolorbox}

\begin{tcolorbox}[lowerbox=ignored,colback=white]
This is a \textbf{tcolorbox}.
\tcblower
这是lower部分(但ignored)
\end{tcolorbox}
\end{exdispExample}
\end{docTcbKey}


\begin{docTcbKey}[][doc updated=2014-11-28]{savelowerto}{=\meta{file name}}{no default, initially empty}
Saves the content of the lower part into a file for an optional later usage.

将lower部分的内容保存到一个文件中,以备以后使用。
\begin{exdispExample}{savelowerto}
\begin{tcolorbox}[lowerbox=invisible,savelowerto=\jobname_bspsave.tex,colback=white]
This is a \textbf{tcolorbox}.
\tcblower
这是可能相当复杂的lower部分:
$\displaystyle f(x)=\frac{1+x^2}{1-x^2}$.
\end{tcolorbox}

现在,我们加载保存的文本:\\
\input{\jobname_bspsave.tex}
\end{exdispExample}
\end{docTcbKey}



% \clearpage
\begin{docTcbKey}{lower separated}{\colOpt{=true\textbar false}}{default |true|, initially |true|}
If set to |true|, the lower part is visually separated from the upper part.
It depends on the chosen skin how the visualization of the separation is done.

如果设置为 |true|则lower部分与upper部分可见的分隔开来。
分隔的样式依赖于皮肤的选择。
\enlargethispage*{1cm}
\begin{exdispExample}{lower_separated}
% \tcbuselibrary{skins,raster}
\begin{tcbraster}[colback=red!5!white,colframe=red!75!black,
fonttitle=\bfseries,fontlower=\itshape]
%
\begin{tcolorbox}[title=Lower separated]
This is the upper part.
\tcblower
This is the lower part.
\end{tcolorbox}
%
\begin{tcolorbox}[title=Lower not separated,lower separated=false]
这是upper部分。

设置 |lower separated=false|
\tcblower
这是lower部分。两部分的分隔没有线条出现。
\end{tcolorbox}
%
\begin{tcolorbox}[sidebyside,title=Lower separated]
% 这是upper部分。
|sidebyside|
\tcblower
% 这是lower部分。
两部分变成左右分隔了。
\end{tcolorbox}
%
\begin{tcolorbox}[sidebyside,title=Lower not separated,lower separated=false]
This is the upper part.
\tcblower
This is the lower part.
\end{tcolorbox}
%
\begin{tcolorbox}[beamer,title=Lower separated]
% This is the upper part.
|beamer|
\tcblower
This is the lower part.
\end{tcolorbox}
%
\begin{tcolorbox}[beamer,title=Lower not separated,lower separated=false]
This is the upper part.
\tcblower
This is the lower part.
\end{tcolorbox}
%
\end{tcbraster}
\end{exdispExample}
\end{docTcbKey}




% delimiter
% 定界符
% \clearpage
\begin{docTcbKey}{savedelimiter}{=\meta{name}}{no default, initially \texttt{tcolorbox}}
Used in connection with new environment definitions which extend
|tcolorbox| and use or allow the option |savelowerto|.
To catch the end of the new box environment \meta{name} has to be the name of
this environment. Additionally, the environment definition has to use
|\tcolorbox| instead of
|\begin{tcolorbox}| and |\endtcolorbox| instead of |\end{tcolorbox}|.

% 用于关联由 |newenvironment| 自定义的,拓展自 |tcolorbox| 的新环境中的 |savelowerto| 选项。%
% 要捕获新的盒子环境 \meta{name} 的结尾,必须是这个环境的名字。%
% 此外,环境定义必须使用 |\tcolorbox| 代替 |\begin{tcolorbox}|、用 |\endtcolorbox| 代替 |\end{tcolorbox}|。

用于与扩展了|tcolorbox|并使用或允许选项|savelowerto|的新环境定义相关联。为了捕捉新框环境的结尾,\meta{name}必须是此环境的名称。此外,环境定义必须使用|\tcolorbox|而不是|\begin{tcolorbox}|,并且使用|\endtcolorbox|而不是|\end{tcolorbox}|。

\begin{exdispExample}{savedelimiter1}
\newenvironment{mybox}[1]{%
\tcolorbox[savedelimiter=mybox,
            savelowerto=\jobname_bspsave2.tex,lowerbox=ignored,
            colback=red!5!white,colframe=red!75!black,fonttitle=\bfseries,
            title={#1}]}%
{\endtcolorbox}

\begin{mybox}{暂存 savelowerto 的内容的新环境}
Upper部分。
\tcblower
暂存的lower部分!
\end{mybox}

现在,使用之前暂存的部分:
\begin{tcolorbox}[colback=green!5,title=用到 savelowerto 暂存的内容]
\input{\jobname_bspsave2.tex}
\end{tcolorbox}
\end{exdispExample}

\enlargethispage*{1cm}

The |savedelimiter| is used implicitely with \refCom{newtcolorbox} which
allows a more convenient usage:

上面的 |savedelimiter| 隐式使用了 \refCom{newtcolorbox},和下面对比,使用起来更方便:
\begin{exdispExample}{savedelimiter2}
\newtcolorbox{mybox}[1]{%
            savelowerto=\jobname_bspsave2.tex,lowerbox=ignored,
            colback=red!5!white,colframe=red!75!black,fonttitle=\bfseries,
            title={#1}}%

\begin{mybox}{My Example}
Upper part.
\tcblower
Saved lower part!
\end{mybox}

Now, the saved part is used:
\begin{tcolorbox}[colback=green!5]
\input{\jobname_bspsave2.tex}
\end{tcolorbox}
\end{exdispExample}
\end{docTcbKey}