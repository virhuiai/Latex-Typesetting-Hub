
Normally, every |tcolorbox| has a bounding box which fits exactly to the
dimensions of the outer frame. Therefore, \LaTeX\ reserves exactly the space
needed for the box.
This behavior can be changed by enlarging (or shrinking) the bounding box.
If the bounding box is enlarged, the |tcolorbox| will get some clearance
around it. If the bounding box is shrunk, i.\,e.\ enlarged with negative
values, the |tcolorbox| will overlap to other parts of the page.
For example, the |tcolorbox| could be stretched into the page margin.

通常,每个 |tcolorbox| 都有一个边界框,它恰好适合外部框架的尺寸。因此,\LaTeX\ 精确地保留了盒子所需的空间。 可以通过放大(或缩小)边界框来改变这种行为。 如果边界框被放大,|tcolorbox| 将在其周围获得一些间隙。 如果边界框被缩小,即用负值放大,|tcolorbox| 将与页面的其他部分重叠。 例如,|tcolorbox| 可以延伸到页面边缘。

% 通常,每个 |tcolorbox| 盒子有一个与其外框严丝合缝的边界盒子。%
% 因此,\LaTeX\ 保留了盒子所需的空间。%
% 可以通过扩大(或缩小)边界盒子来更改此空间。%
% 如果边界盒子被放大, 那么 |tcolorbox| 的周围将多出一些空间。 如果边界框缩小, i.\,e.\ 扩大一个负值, |tcolorbox| 将重叠到页面的其他部分。%
% 例如,|tcolorbox| 可能会突到页边距去。

\begin{marker}
The following examples use \refKey{/tcb/show bounding box} to display the actual bounding box. For this, the library \mylib{skins} has to be included and \refKey{/tcb/enhanced} has to be set.

以下示例使用\refKey{/tcb/show bounding box}来显示实际边界框。为此,必须包含库\mylib{skins}并设置\refKey{/tcb/enhanced}。
% 以下示例使用 \refKey{/tcb/show bounding box} 来显示实际的边界框。为此,必须包含库 \mylib{skins} 并且必须设置 \refKey{/tcb/enhanced}。
\end{marker}



\subsubsection{Shifting Bounding Box Borders\\移动边界盒子的边框}

\begin{docTcbKey}{enlarge top initially by}{=\meta{length}}{no default, initially |0mm|}
Enlarges the bounding box distance to the top of the box by \meta{length}.
If the box is \emph{breakable}, only the first box of the break sequence
gets enlarged. \refKey{/tcb/enlarge top by} overwrites this key.

将边框顶部的边界框距离增加\meta{length}。如果该盒子是\emph{可分页的},则仅对分页序列的第一个框进行扩展。 \refKey{/tcb/enlarge top by} 覆盖此键。 

% 扩大边界盒子同 |tcolorbox| 盒子的顶部的距离 \meta{length}。
% 如果 |tcolorbox| 盒子是\emph{可分的}, 只有中断序列的第一个盒子的扩大会生效。 
\refKey{/tcb/enlarge top by} 会覆盖这个设置。
\begin{exdispExample}{enlarge_top_initially_by}
\tcbset{colframe=blue!75!black,colback=white}

\begin{tcolorbox}[enlarge top initially by=-5mm]
This is a \textbf{tcolorbox}.
\end{tcolorbox}
\begin{tcolorbox}[enlarge top initially by=5mm,enhanced,show bounding box]
This is a \textbf{tcolorbox}.
\end{tcolorbox}
\end{exdispExample}
\end{docTcbKey}

\begin{tcolorbox}[nobeforeafter]
\textbf{tcolorbox} a.
\end{tcolorbox}
\begin{tcolorbox}[nobeforeafter]
\textbf{tcolorbox} b.
\end{tcolorbox}




\begin{docTcbKey}{enlarge bottom finally by}{=\meta{length}}{no default, initially |0mm|}
Enlarges the bounding box distance to the bottom of the box by \meta{length}.
If the box is \emph{breakable}, only the last box of the break sequence
gets enlarged. \refKey{/tcb/enlarge bottom by} overwrites this key.

通过\meta{length}来增加盒子底部的边界距离。如果该盒子是\emph{可分割}的,那么只有分割序列的最后一个盒子会被放大。 \refKey{/tcb/enlarge bottom by} 会覆盖此关键字。

% 扩大边界盒子同 |tcolorbox| 盒子的底部的距离 \meta{length}。%
% 如果盒子是 \emph{可中断的}, 只有中断序列的最后一部分得到扩大。
% \refKey{/tcb/enlarge bottom by} 会覆盖这个设置。
\begin{exdispExample}{enlarge_bottom_finally_by}
\tcbset{colframe=blue!75!black,colback=white}

\begin{tcolorbox}[enlarge bottom finally by=5mm]
This is a \textbf{tcolorbox}.
\end{tcolorbox}
\begin{tcolorbox}[enlarge bottom finally by=-5mm,enhanced,show bounding box]
This is a \textbf{tcolorbox}.
\end{tcolorbox}
\end{exdispExample}
\end{docTcbKey}

%%\clearpage




\begin{docTcbKey}{enlarge top at break by}{=\meta{length}}{no default, initially \texttt{0mm}}
Enlarges the bounding box distance to the top of the box by \meta{length},
\emph{if} the box is \refKey{/tcb/breakable}.
In this case, it is applied to \emph{middle} and \emph{last} parts in a
break sequence.
\refKey{/tcb/enlarge top by} overwrites this key.

\emph{如果}盒子是 \refKey{/tcb/breakable}的,扩大边界盒子同 |tcolorbox| 盒子的顶部的距离 \meta{length}。这种情况下, 它在 \emph{中间}和\emph{最后}的中断序列部分生效。 \refKey{/tcb/enlarge top by} 会覆盖此项设置。
\end{docTcbKey}


\begin{docTcbKey}{enlarge bottom at break by}{=\meta{length}}{no default, initially \texttt{0mm}}
Enlarges the bounding box distance to the bottom of the box by \meta{length},
\emph{if} the box is \refKey{/tcb/breakable}.
In this case, it is applied to \emph{first} and \emph{middle} parts in a
break sequence. \refKey{/tcb/enlarge bottom by} overwrites this key.

\emph{如果}盒子是 \refKey{/tcb/breakable}的,扩大边界盒子同 |tcolorbox| 盒子的底部的距离 \meta{length}。
这种情况下, 它在 \emph{首个}和\emph{中间}的中断序列部分生效。
\refKey{/tcb/enlarge bottom by} 会覆盖此项设置。
\end{docTcbKey}




\begin{docTcbKey}{enlarge top by}{=\meta{length}}{no default, initially |0mm|}

Enlarges the bounding box distance to the top of the box by \meta{length}.
\refKey{/tcb/enlarge top initially by} and
\refKey{/tcb/enlarge top at break by} are set to \meta{length}.

扩大边界盒子同 |tcolorbox| 盒子的{\bf 顶}部的距离 \meta{length}。
\refKey{/tcb/enlarge top initially by} 和
\refKey{/tcb/enlarge top at break by} 也会被设置为 \meta{length}。
\end{docTcbKey}


\begin{docTcbKey}{enlarge bottom by}{=\meta{length}}{no default, initially |0mm|}
Enlarges the bounding box distance to the bottom of the box by \meta{length}.
\refKey{/tcb/enlarge bottom finally by} and
\refKey{/tcb/enlarge bottom at break by} are set to \meta{length}.

扩大边界盒子同 |tcolorbox| 盒子的{\bf 底}部的距离 \meta{length}。
\refKey{/tcb/enlarge bottom finally by} 和
\refKey{/tcb/enlarge bottom at break by} 也会被设为 \meta{length}.
\end{docTcbKey}



\begin{docTcbKey}{enlarge left by}{=\meta{length}}{no default, initially |0mm|}
Enlarges the bounding box distance to the left side of the box by \meta{length}.

扩大边界盒子同 |tcolorbox| 盒子的{\bf 左侧}的距离 \meta{length}。
\begin{exdispExample}[safety=2cm]{enlarge_left_by}
\tcbset{colframe=blue!75!black,colback=white}

\begin{tcolorbox}[enlarge left by=2cm,width=5cm,enhanced,show bounding box]
This is a \textbf{tcolorbox}.
\end{tcolorbox}
\begin{tcolorbox}[enlarge left by=-2cm,width=\linewidth+2cm]
This is a \textbf{tcolorbox}.
\end{tcolorbox}
\end{exdispExample}
\end{docTcbKey}

\begin{docTcbKey}{enlarge right by}{=\meta{length}}{no default, initially |0mm|}
Enlarges the bounding box distance to the right side of the box by \meta{length}.

扩大边界盒子同 |tcolorbox| 盒子的{\bf 右侧}的距离 \meta{length}。
\begin{exdispExample}[safety=2cm]{enlarge_right_by}
\tcbset{colframe=blue!75!black,colback=white}

\begin{tcolorbox}[enlarge right by=-2cm,width=\linewidth+2cm,
enhanced,show bounding box]
\textbf{tcolorbox}缩小了同右侧的距离到负数,就突出到右边了.
\end{tcolorbox}
\begin{tcolorbox}[enlarge right by=2cm,width=\linewidth-2cm]
This is a \textbf{tcolorbox}.
\end{tcolorbox}
\end{exdispExample}
\end{docTcbKey}




%%\clearpage
\begin{docTcbKey}{enlarge by}{=\meta{length}}{no default, initially |0mm|}
Enlarges the bounding box distance to all sides of the box by \meta{length}.

扩大边界盒子同 |tcolorbox| 盒子的{\bf 四侧}的距离 \meta{length}。
\begin{exdispExample}{enlarge_by}
\tcbset{colframe=blue!75!black,colback=white,width=5cm,nobeforeafter}

\begin{tcolorbox}
This is a \textbf{tcolorbox}.
\end{tcolorbox}
\begin{tcolorbox}[enlarge by=5mm,enhanced,show bounding box]
This is a \textbf{tcolorbox}.
\end{tcolorbox}
\end{exdispExample}
\end{docTcbKey}





\begin{docTcbKey}{grow to left by}{=\meta{length}}{no default, initially |0mm|}
Enlarges the current box width by \meta{length} and
enlarges (shrinks) the bounding box distance to the left side of the box by
$-$\meta{length}. Also see \refKey{/tcb/left skip}.

扩大当前盒子的宽度\meta{length},并扩大边界盒子到左侧的距离为
$-$\meta{length}。\footnote{突出到左侧了} 另见 \refKey{/tcb/left skip}。
\begin{exdispExample}[safety=2cm]{grow_to_left_by}
\tcbset{colframe=blue!75!black,colback=white}

\begin{tcolorbox}[width=5cm,grow to left by=2cm,enhanced,show bounding box]
This is a \textbf{tcolorbox} with a width of 7cm.
\end{tcolorbox}
\end{exdispExample}
\end{docTcbKey}

\begin{docTcbKey}{grow to right by}{=\meta{length}}{no default, initially |0mm|}
Enlarges the current box width by \meta{length} and
enlarges (shrinks) the bounding box distance to the right side of the box by
$-$\meta{length}. Also see \refKey{/tcb/right skip}.

扩大当前盒子的宽度\meta{length},并扩大边界盒子到右侧的距离为
$-$\meta{length}。\footnote{突出到右侧了} 另见 \refKey{/tcb/right skip}。
\begin{exdispExample}[safety=2cm]{grow_to_right_by}
\tcbset{colframe=blue!75!black,colback=white}

\begin{tcolorbox}[grow to right by=2cm,enhanced,show bounding box]
This is a \textbf{tcolorbox}.
\end{tcolorbox}

\bigskip

\begin{tcolorbox}[grow to right by=2cm,grow to left by=1cm,
enhanced,show bounding box]
This is a \textbf{tcolorbox}.
\end{tcolorbox}
\end{exdispExample}
\end{docTcbKey}

%%\clearpage


\begin{docTcbKey}[][doc new=2018-03-22]{grow sidewards by}{=\meta{length}}{no default, initially |0mm|}
Shortcut for setting \refKey{/tcb/grow to left by} and \refKey{/tcb/grow to right by}
to \meta{length}. Also see \refKey{/tcb/oversize} and \refKey{/tcb/spread sidewards}.

同时设置 \refKey{/tcb/grow to left by} 和 \refKey{/tcb/grow to right by} 到\meta{length} 的简写形式。另见 \refKey{/tcb/oversize} 和 \refKey{/tcb/spread sidewards}.
\begin{exdispExample}[safety=2cm]{grow_sidewards_by}
\tcbset{colframe=blue!75!black,colback=white}

\begin{tcolorbox}[grow sidewards by=2cm,enhanced,show bounding box]
This is a \textbf{tcolorbox}.
\end{tcolorbox}
\end{exdispExample}
\end{docTcbKey}




\subsubsection{Box Alignment\\盒子的对齐}

\begin{docTcbKey}[][doc new=2015-11-20]{flush left}{}{style, no value}
Enlarges the bounding box to the right side to fill the line completely.

左对齐效果,扩大边界盒子完全填充到右侧。
\begin{exdispExample}{flush_left}
\tcbset{colframe=blue!75!black,colback=white}

\begin{tcolorbox}[flush left,width=5cm,enhanced,show bounding box]
This is a \textbf{tcolorbox}.
\end{tcolorbox}
\end{exdispExample}
\end{docTcbKey}


\begin{docTcbKey}[][doc new=2015-11-20]{flush right}{}{style, no value}
Enlarges the bounding box to the left side to fill the line completely.

右对齐效果,扩大边界盒子完全填充到左侧。
\begin{exdispExample}{flush_right}
\tcbset{colframe=blue!75!black,colback=white}

\begin{tcolorbox}[flush right,width=5cm,enhanced,show bounding box]
This is a \textbf{tcolorbox}.
\end{tcolorbox}
\end{exdispExample}
\end{docTcbKey}


\begin{docTcbKey}[][doc new=2015-11-20]{center}{}{style, no value}
Enlarges the bounding box equally to both sides to fill the line completely.

居中对齐效果,扩大边界盒子完全填充到两侧。
\begin{exdispExample}{center}
\tcbset{colframe=blue!75!black,colback=white}

\begin{tcolorbox}[center,width=5cm,enhanced,show bounding box]
This is a \textbf{tcolorbox}.
\end{tcolorbox}
\end{exdispExample}
\end{docTcbKey}




%%\clearpage

\subsubsection{Toggle Enlargements}

\begin{docTcbKey}[][doc updated=2015-11-13]{toggle enlargement}{=\meta{toggle preset}}{默认 |evenpage|(偶), initially |none|}
According to the \meta{toggle preset}, the left and the right enlargements of
the bounding box are switched or not. Feasible values are:

依据 \meta{toggle preset} 的值, 对边界盒子的左边和右边的增加空间的设置进行交换或不交换。可设的值有:
\begin{itemize}
\item\docValue{none}: %no switching.
不切换。
\item\docValue{forced}: %the values of the left and right enlargement are switched.
强制将边界盒子的左边和右边的增加空间的设置进行交换。
\item\docValue{evenpage}: 
if the page is an even page, the values of the left and    right enlargement are switched. This value also sets    \refKey{/tcb/check odd page} to |true|.

如果当前页是偶数页, 将边界盒子的左边和右边的增加空间的设置进行交换。这项值也会将  \refKey{/tcb/check odd page} 设置为 |true|.
\end{itemize}
\begin{marker}
See \refKey{/tcb/toggle left and right} to toggle geometry settings.
见 \refKey{/tcb/toggle left and right} 来切换几何设置。
\end{marker}

\begin{dispExample}
\tcbset{colframe=blue!75!black,colback=white,
grow to left by=20mm,%突出左侧
grow to right by=-5mm%右侧凹着
}

\begin{tcolorbox}[toggle enlargement=none
,enhanced,show bounding box]
设置为 |toggle enlargement=none|,不切换
\end{tcolorbox}
\begin{tcolorbox}[toggle enlargement=forced]
设置为 |toggle enlargement=forced|,强制切换
\end{tcolorbox}
\begin{tcolorbox}[toggle enlargement=evenpage]
设置为 |toggle enlargement=evenpage|,偶数页才切换。当前页是 \tcbifoddpage{奇}{偶} 数页。因此, 左边的增加空间的设置 \tcbifoddpage{不会}{会}切换。
\end{tcolorbox}
\end{dispExample}

\begin{dispListing}
\begin{tcolorbox}[colframe=red!60!black,colback=red!15!white,
fonttitle=\bfseries,title=Floating box from \texttt{toggle enlargement},
width=\textwidth
,grow to right by=2cm%突出到右边
,toggle enlargement%默认是偶数页
,float=t]
当前页是\tcbifoddpage{奇}{偶}数页。%
因此, 左边的增加空间的设置 \tcbifoddpage{不会}{会}切换。%
这个盒子,在奇数页是突出到右边,在偶数页是突出到左边。%
本文档是one-sided文档 -- 这项特性只在two-sided%
\footnote{译注:即双面打印,%
奇数和偶数页的文档内容的左右边距是不同的,以用于装订。}%
文档中生效。
\end{tcolorbox}
\end{dispListing}
\tcbusetemp
\end{docTcbKey}





%%\clearpage

\subsubsection{Spread Box to Page Borders\\盒子扩张到页面边缘}

\begin{marker} 
The following border options are \emph{not} applicable to nested boxes, boxes insides tables, etc.
For boxes inside lists, the options \emph{may} work, but not necessarily.
Also, boxes should be set with |\noindent| and full width.

以下的 border 选项对嵌套的盒子, 在表格中的盒子\emph{不}起作用, etc.
对于列表中的盒子,这些选项\emph{可以}工作, 但没必要。
另外,盒子需要设置 |\noindent| 来达到全宽。
\end{marker}

\begin{docTcbKey}[][doc new=2017-02-13]{spread inwards}{\colOpt{=\meta{length}}}{default |0pt|, initially unset}
Enlarges the current box width to match the inner page border (left-handed side for one-sided
documents). If the optional \meta{length} is greater than |0pt|, the box
grows over the border, if \meta{length} is lower than |0pt|, there is a
margin between box and page border.
\refKey{/tcb/toggle enlargement} is set automatically.

扩张当前盒子的宽度到书本的内页边缘(对单面文档是在左侧).如果选项的值\meta{length}是大于 |0pt|, 那么盒子将穿过页面的边缘, 如果\meta{length}是小于|0pt|,那么在盒子和页面边缘就有一段面边空白。会自动设置 \refKey{/tcb/toggle enlargement} 。
\begin{dispListing}
\begin{tcolorbox}[enhanced,spread inwards,
colframe=blue!75!black,colback=white,show bounding box]
扩张当前盒子的宽度到书本的内页边缘 (对单面文档是在左侧).(|spread inwards|)
\end{tcolorbox}

\begin{tcolorbox}[enhanced,spread inwards=2em,
colframe=blue!75!black,colback=white,show bounding box]
前面的内容穿过边缘了(|spread inwards=2em,|)。
\end{tcolorbox}

\begin{tcolorbox}[enhanced,spread inwards=-2em,
colframe=blue!75!black,colback=white,show bounding box]
在盒子和页面边缘就有一段面边空白。(|spread inwards=-2em,|)。
\end{tcolorbox}
\end{dispListing}
{\tcbusetemp}
\end{docTcbKey}



\begin{docTcbKey}[][doc new=2017-02-13]{spread outwards}{\colOpt{=\meta{length}}}{default |0pt|, initially unset}
Enlarges the current box width to match the outer page border (right-handed side for one-sided
documents). If the optional \meta{length} is greater than |0pt|, the box
grows over the border, if \meta{length} is lower than |0pt|, there is a
margin between box and page border.
\refKey{/tcb/toggle enlargement} is set automatically.

扩张当前盒子的宽度到书本的内页边缘(对单面文档是在右侧)。如果选项的值\meta{length}是大于 |0pt|, 那么盒子将穿过页面的边缘, 如果\meta{length}是小于|0pt|,那么在盒子和页面边缘就有一段空白。会自动设置 \refKey{/tcb/toggle enlargement} 。

\begin{dispListing}
\begin{tcolorbox}[enhanced,spread outwards,
colframe=blue!75!black,colback=white,show bounding box]
This is a \textbf{tcolorbox}.
\end{tcolorbox}
\end{dispListing}
{\tcbusetemp}
\end{docTcbKey}


\begin{docTcbKey}[][doc new=2017-02-13]{move upwards}{\colOpt{=\meta{length}}}{default |0pt|, initially unset}
Starts a new page with the box at the very top page border.
If the optional \meta{length} is greater than |0pt|, the box
moves over the border, if \meta{length} is lower than |0pt|, there is a
margin between box and page border.

新起一页,将盒子放在新页面的最顶部。%
如果选项的值\meta{length}是大于 |0pt|, 那么盒子将穿过页面的边缘, 如果\meta{length}是小于|0pt|,那么在盒子和页面边缘就有一段空白。
\end{docTcbKey}


\begin{docTcbKey}[][doc new=2017-02-13]{move upwards*}{\colOpt{=\meta{length}}}{default |0pt|, initially unset}
同\refKey{/tcb/move upwards}一样,但少了新起一页的操作。
\end{docTcbKey}




\begin{docTcbKey}[][doc new=2017-02-13]{fill downwards}{\colOpt{=\meta{length}}}{default |0pt|, initially unset}
Enlarges the height of the box until the very bottom page border.
The library \mylib{breakable} has to be loaded, and
\refKey{/tcb/height fill} is set automatically.
If the optional \meta{length} is greater than |0pt|, the box
moves over the border, if \meta{length} is lower than |0pt|, there is a
margin between box and page border.

扩张当前盒子的宽度到书本的底部边缘。%
需要加载 \mylib{breakable} 库, 且会自动设置\refKey{/tcb/height fill} 。%
如果选项的值\meta{length}是大于 |0pt|, 那么盒子将穿过页面的边缘, 如果\meta{length}是小于|0pt|,那么在盒子和页面边缘就有一段空白。
\begin{dispListing}
\begin{tcolorbox}[enhanced,fill downwards,
colframe=blue!75!black,colback=white,show bounding box]
扩张当前盒子的宽度到书本的底部边缘。
\end{tcolorbox}
\end{dispListing}
{\tcbusetemp}
\end{docTcbKey}


\begin{tcolorbox}[enhanced,spread upwards,sharp corners=north,height=3cm,
colframe=blue!75!black,interior style={top color=blue!50,bottom color=white}]
这是\enquote{spread upwards}的例子。 
\end{tcolorbox}
\begin{docTcbKey}[][doc new=2017-02-13]{spread upwards}{\colOpt{=\meta{length}}}{default |0pt|, initially unset}
组合,同时将\meta{length}设到
\refKey{/tcb/move upwards}, \refKey{/tcb/spread inwards}, 和 \refKey{/tcb/spread outwards}.
\begin{dispListing}
\begin{tcolorbox}[enhanced,spread upwards,sharp corners=north,height=3cm,
colframe=blue!75!black,interior style={top color=blue!50,bottom color=white}]
这是 \enquote{spread upwards} 的例子。
\end{tcolorbox}
\end{dispListing}
\end{docTcbKey}


\begin{docTcbKey}[][doc new=2017-02-13]{spread upwards*}{\colOpt{=\meta{length}}}{default |0pt|, initially unset}
同\refKey{/tcb/move upwards}一样,但不会新起一页。
\end{docTcbKey}




\begin{docTcbKey}[][doc new=2017-02-13]{spread sidewards}{\colOpt{=\meta{length}}}{default |0pt|, initially unset}
Combination of \refKey{/tcb/spread inwards} and \refKey{/tcb/spread outwards}.
The optional \meta{length} is used for all these keys.
Also see \refKey{/tcb/oversize} and \refKey{/tcb/grow sidewards by}.

\meta{length}被同时设置到 \refKey{/tcb/spread inwards} 和 \refKey{/tcb/spread outwards}。另见 \refKey{/tcb/oversize} 和 \refKey{/tcb/grow sidewards by}.
\begin{dispListing}
\begin{tcolorbox}[enhanced,spread sidewards,
colframe=blue!75!black,colback=white,show bounding box]
向左右两侧突出了。
\end{tcolorbox}
\end{dispListing}
{\tcbusetemp}
\end{docTcbKey}


\begin{docTcbKey}[][doc new=2017-02-13]{spread}{\colOpt{=\meta{length}}}{default |0pt|, initially unset}
Combination of
\refKey{/tcb/move upwards}, \refKey{/tcb/fill downwards}, \refKey{/tcb/spread inwards},
and \refKey{/tcb/spread outwards}.
Such, the box fills the whole page.
The optional \meta{length} is used for all these keys.

组合了 \refKey{/tcb/move upwards}, \refKey{/tcb/fill downwards}, \refKey{/tcb/spread inwards},和 \refKey{/tcb/spread outwards}。
这样,盒子就填满了整个页面。
\meta{length} 被同时设置到这些选项。
\end{docTcbKey}




\begin{docTcbKey}[][doc new=2017-02-13]{spread downwards}{\colOpt{=\meta{length}}}{default |0pt|, initially unset}
Combination of
\refKey{/tcb/fill downwards}, \refKey{/tcb/spread inwards}, and \refKey{/tcb/spread outwards}.
The optional \meta{length} is used for all these keys.

组合使用 \refKey{/tcb/fill downwards}, \refKey{/tcb/spread inwards}, 和 \refKey{/tcb/spread outwards}.
\meta{length} 被同时设置到这些选项。
\begin{dispListing}
\begin{tcolorbox}[enhanced,spread downwards,sharp corners=south,
colframe=red!75!black,interior style={top color=white,bottom color=red!50}]
This is an example for \enquote{spread downwards}.
\end{tcolorbox}
\end{dispListing}
\end{docTcbKey}
\begin{tcolorbox}[enhanced,spread downwards,sharp corners=south,
colframe=red!75!black,interior style={top color=white,bottom color=red!50}]
This is an example for \enquote{spread downwards}.
\end{tcolorbox}






%%\clearpage

\subsubsection{Box Extrusion\\挤压盒子}

\begin{marker}
The following keys should not be used with breakable boxes or boxes with a
lower part.

以下选项不应在可分盒子或带有lower部分的盒子内使用。
\end{marker}

\begin{docTcbKey}{shrink tight}{}{style, no value, initially unset}
The total colored box is shrunk to the dimensions of the upper
part. There should be no lower part and no title.
This style sets the \refKey{/tcb/boxsep} to |0pt| and other geometry keys
to fitting values. This option is likely to be used with the following
extrusion keys.

整个盒子缩小到upper部分的尺寸。不应有lower部分和标题。
此样式会将 \refKey{/tcb/boxsep} 设置为 |0pt|,以及一些其他几何设置。此选项可能与以下挤压键一起使用。
\begin{exdispExample}{shrink_tight}
\tcbset{colframe=blue!75!black,colback=white,arc=0mm,boxrule=0.4pt,
    nobeforeafter,tcbox raise base,shrink tight}

\begin{tcolorbox}
This is a \textbf{tcolorbox}.
\end{tcolorbox}

Lorem \tcbox{ipsum} dolor sit amet, consectetuer adipiscing elit.
\end{exdispExample}

\begin{exdispExample}{shrink_tight2}
\tcbset{colframe=blue!75!black,colback=white,arc=0mm,boxrule=0.4pt,
        shrink tight}

\begin{tcolorbox}
This is a \textbf{tcolorbox}.
\end{tcolorbox}

Lorem \tcbox{ipsum} dolor sit amet, consectetuer adipiscing elit.
\end{exdispExample}
\end{docTcbKey}  

% extrude
% ① (force out) 挤出 jǐchū ‹toothpaste, glue, icing›; 压出 yāchū ‹pasta›
% ② (shape) 压制 yāzhì ‹plastic, metal›
\begin{docTcbKey}[][doc updated=2014-09-19]{extrude left by}{=\meta{length}}{style, no default, initially unset}
The (upper part of the) colored box is extruded by the given \meta{length} to the left side.
The inner width and the bounding box is kept unchanged and the operation
is additive!

(upper部分)盒子向左挤出 \meta{length} 空间\footnote{译注:这部分有点像零宽的盒子效果。}。内部宽度和边界盒子保持不变,挤出是额外的!
\begin{exdispExample}{extrude_left_by}
\tcbset{enhanced,colframe=red,colback=yellow!25!white,
frame style={opacity=0.25},interior style={opacity=0.5},
nobeforeafter,tcbox raise base,shrink tight,extrude by=2mm}

Lorem ipsum dolor sit amet, consectetuer adipiscing elit. Ut purus elit,
vestibulum ut, placerat ac, adipiscing vitae, felis.
\tcbox[extrude left by=1cm]{Curabitur} dictum gravida mauris.
Nam arcu libero, nonummy eget, consectetuer id, vulputate a, magna.
\end{exdispExample}

\begin{exdispExample}{extrude_left_by2}
\tcbset{enhanced,colframe=red,colback=yellow!25!white,
frame style={opacity=0.25},interior style={opacity=0.5},
nobeforeafter,tcbox raise base,shrink tight,extrude by=2mm}

Lorem ipsum dolor sit amet, consectetuer adipiscing elit. Ut purus elit,
vestibulum ut, placerat ac, adipiscing vitae, felis.
\tcbox[extrude left by=1cm,,show bounding box]{Curabitur} dictum gravida mauris.
Nam arcu libero, nonummy eget, consectetuer id, vulputate a, magna.
\end{exdispExample}

\end{docTcbKey}

\begin{docTcbKey}[][doc updated=2014-09-19]{extrude right by}{=\meta{length}}{style, no default, initially unset}
The (upper part of the) colored box is extruded by the given \meta{length} to the right side.
The inner width and the bounding box is kept unchanged and the operation
is additive!

(upper部分)盒子向{\bf 右}挤出 \meta{length} 空间%\footnote{译注:这部分有点像零宽的盒子效果。}
。内部宽度和边界盒子保持不变,挤出是额外的!
\begin{exdispExample}{extrude_right_by}
\tcbset{enhanced,colframe=red,colback=yellow!25!white,
frame style={opacity=0.25},interior style={opacity=0.5},
nobeforeafter,tcbox raise base,shrink tight,extrude by=2mm}

Lorem ipsum dolor sit amet, consectetuer adipiscing elit. Ut purus elit,
vestibulum ut, placerat ac, adipiscing vitae, felis.
\tcbox[extrude right by=1cm]{Curabitur} dictum gravida mauris.
Nam arcu libero, nonummy eget, consectetuer id, vulputate a, magna.
\end{exdispExample}
\end{docTcbKey}

%%\clearpage
\begin{docTcbKey}{extrude top by}{=\meta{length}}{style, no default, initially unset}
The (upper part of the) colored box is extruded by the given \meta{length} to the top side.
The inner width and the bounding box is kept unchanged and the operation
is additive!

(upper部分)盒子向{\bf 上}挤出 \meta{length} 空间%\footnote{译注:这部分有点像零宽的盒子效果。}
。内部宽度和边界盒子保持不变,挤出是额外的!
\begin{exdispExample}{extrude_top_by}
\tcbset{enhanced,colframe=red,colback=yellow!25!white,
frame style={opacity=0.25},interior style={opacity=0.5},
nobeforeafter,tcbox raise base,shrink tight,extrude by=2mm}

Lorem ipsum dolor sit amet, consectetuer adipiscing elit. Ut purus elit,
vestibulum ut, placerat ac, adipiscing vitae, felis.
\tcbox[extrude top by=1cm]{Curabitur} dictum gravida mauris.
Nam arcu libero, nonummy eget, consectetuer id, vulputate a, magna.
\end{exdispExample}
\end{docTcbKey}

\begin{docTcbKey}{extrude bottom by}{=\meta{length}}{style, no default, initially unset}
The (upper part of the) colored box is extruded by the given \meta{length} to the bottom side.
The inner width and the bounding box is kept unchanged and the operation
is additive!

(upper部分)盒子向{\bf 下}挤出 \meta{length} 空间%\footnote{译注:这部分有点像零宽的盒子效果。}
。内部宽度和边界盒子保持不变,挤出是额外的!
\begin{exdispExample}[safety=1cm]{extrude_bottom_by}
\tcbset{enhanced,colframe=red,colback=yellow!25!white,
frame style={opacity=0.25},interior style={opacity=0.5},
nobeforeafter,tcbox raise base,shrink tight,extrude by=2mm}

Lorem ipsum dolor sit amet, consectetuer adipiscing elit. Ut purus elit,
vestibulum ut, placerat ac, adipiscing vitae, felis.
\tcbox[extrude bottom by=1cm]{Curabitur} dictum gravida mauris.
Nam arcu libero, nonummy eget, consectetuer id, vulputate a, magna.
\end{exdispExample}
\end{docTcbKey}



\begin{docTcbKey}{extrude by}{=\meta{length}}{style, no default, initially unset}
The (upper part of the) colored box is extruded by the given \meta{length} to all sides.
The inner width and the bounding box is kept unchanged and the operation
is additive!

(upper部分)盒子向{\bf 四周}都挤出 \meta{length} 空间%\footnote{译注:这部分有点像零宽的盒子效果。}
。内部宽度和边界盒子保持不变,挤出是额外的!
\begin{exdispExample}{extrude_by}
\tcbset{enhanced,colframe=red,colback=yellow!25!white,
frame style={opacity=0.25},interior style={opacity=0.5},
nobeforeafter,tcbox raise base,shrink tight,extrude by=2mm}

Lorem ipsum dolor sit amet, consectetuer adipiscing elit. Ut purus elit,
vestibulum ut, placerat ac, adipiscing vitae, felis. \tcbox{Curabitur} dictum
gravida mauris. \tcbox[colframe=Green,interior style={opacity=0.0}]{Nam}
arcu libero, nonummy eget, consectetuer id, \tcbox{vulputate} a, magna. Donec
vehicula augue eu neque. Pellentesque habitant morbi tristique senectus et netus
et malesuada fames ac turpis egestas. \tcbox{Mauris ut leo.}
\end{exdispExample}
\end{docTcbKey}