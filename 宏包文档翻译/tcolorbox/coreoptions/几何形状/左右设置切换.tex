\setcounter{section}{4}
\setcounter{subsection}{7}
\setcounter{subsubsection}{5}

\subsubsection{Toggle Left and Right\\左右设置切换}
\begin{docTcbKey}[][doc updated=2017-02-16]{toggle left and right}{=\meta{toggle preset}}{default |evenpage|, initially |none|}
According to the \meta{toggle preset}, the left and the right settings
of the |tcolorbox| are switched or not. Feasible values are:

根据 \meta{toggle preset}, |tcolorbox| 的左右设置是否对换。 可选的值有:
  % \begin{DescriptionL}{\docValue{evenpage}}

  \sbox{\cshDocValueLikeItemBox}{\docValue{evenpage}}
  \cshDocValueAndTwoDescLikeItem{none}{no switching.}{不对换。}  
  \cshDocValueAndTwoDescLikeItem{forced}{the values of the left and right rules, spaces, and corners are switched.}{左右的线、距离和四个角的设置互换。}  
  \cshDocValueAndTwoDescLikeItem{evenpage}{
  if the page is an even page, the values of the left and
    right rules, spaces, and corners are switched. This value also sets
    \refKeyLe{/tcb/check odd page} to |true|.}
{偶数页的左右线条数值,距离和四个角的设置互换。 此设置同时将 \refKeyLe{/tcb/check odd page} 设置为 |true|.}
  
\begin{marker}
Horizontal bounding box enlargements are not toggled by this option.
They can be toggled independently by \refKeyLe{/tcb/toggle enlargement}.
For example, \refKeyLe{/tcb/oversize} changes the bounding box.

此选项不会将盒子的水平边框放大。
它们可以通过 \refKeyLe{/tcb/toggle enlargement} 独立地进行切换。
例如, \refKeyLe{/tcb/oversize} 更改盒子的的边界框。
\end{marker}
\begin{dispListing}
% \usepackage{lipsum}
% \usetikzlibrary{patterns}
% \tcbuselibrary{skins,breakable}
\begin{tcolorbox}[enhanced,% 启用增强功能,支持更多的选项。
breakable,%当盒子过长时,可以自动分页。
toggle left and right,%使盒子可以在左右两侧切换。
sharp corners,% 使盒子的角变得尖锐。
boxrule=0mm,top=0mm,bottom=0mm,left=1mm,right=1mm,
rightrule=1cm,colupper=blue!25!black,
interior style={fill overzoom image=lichtspiel.jpg,fill image opacity=0.25},
frame style={pattern=crosshatch dots light steel blue},
overlay={%定义要在盒子上覆盖的内容,这里是一个填充球和一个交叉标记。
\begin{tcbclipframe}
\tcbifoddpage{\coordinate (X) at ([xshift=-5mm]frame.east);}
            {\coordinate (X) at ([xshift=5mm]frame.west);}
\fill[shading=ball,ball color=blue!50!white,opacity=0.5] (X) circle (4mm);
\end{tcbclipframe}}]
\lipsum[1-6]
\end{tcolorbox}
\end{dispListing}
\medskip

% 这是一个tcolorbox环境,会在页面上创建一个带有填充图片和交叉标记的盒子。


% boxrule,top,bottom,left,right:控制盒子边框的宽度和位置。
% rightrule:在盒子右侧添加一条宽度为1厘米的边框。
% colupper:定义盒子中的文本颜色。
% interior style:定义盒子内部的样式,包括填充图片和填充透明度。
% frame style:定义盒子边框的样式,包括交叉标记和颜色。
% overlay:定义要在盒子上覆盖的内容,这里是一个填充球和一个交叉标记。
% 整个盒子中放置了一个lipsum段落,用于测试盒子的分页效果。

This example switches a |1cm| thick rule from the left to the right side
depending on the page number. Thereby, the rule is always on the outer side
of the double-sided paper. Additionally, a ball is drawn on the outer side
with help of an overlay.

这个例子根据页面编号将厚度为|1cm|的标尺从左侧移到右侧。因此,该标尺始终位于双面纸的外侧。此外,通过叠加层在外侧绘制一个球。

% 此示例根据页码将 |1cm| 宽的线条从左边切换到右边。
% 因此,线条总是在双面纸的外侧。
% 此外,一个球形,通过 overlay 绘制在外侧。
\bigskip

\tcbusetemp
\end{docTcbKey}

