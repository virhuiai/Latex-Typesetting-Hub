\setcounter{section}{4}
\setcounter{subsection}{7}
\setcounter{subsubsection}{3}

\subsubsection{Spacing\hfill 间隔}
\begin{docTcbKey}{boxsep}{=\meta{length}}{no default, initially \texttt{1mm}}
Sets a common padding of \meta{length} between the text content and the
frame of the box. This value is added to the key values of
|left|, |right|, |top|, |bottom|, and |middle| at the appropriate places.

% 在文本内容和盒子的框之间设置一个共同的填充宽度为 \meta{length}。 这个值会被添加到
% |left|, |right|, |top|, |bottom|, 和 |middle| 的合适位置。

在文本内容和盒子边框之间设置一个共同的填充 \meta{length}。这个值会添加到 |left|、|right|、|top|、|bottom| 和 |middle| 的关键值中,在适当的位置使用。

% \begin{dispExample*}{}
% \tcbset{colback=red!5!white,colframe=red!75!black,width=(\linewidth-4mm)/2, before=,after=\hfill}
% \begin{tcolorbox}
% hi
% \end{tcolorbox}
% \begin{tcolorbox}[draft]
% hi
% \end{tcolorbox}
% \end{dispExample*}
% %%%%%%%%
% \begin{dispExample*}{}
% \tcbset{colback=red!5!white,colframe=red!75!black,width=(\linewidth-4mm)/2, before=,after=\hfill}
% \begin{tcolorbox}[boxsep=0mm]
% hi
% \end{tcolorbox}
% \begin{tcolorbox}[boxsep=0mm,draft]
% hi
% \end{tcolorbox}
% \end{dispExample*}
% %%%%%%
\begin{dispExample*}{}
\tcbset{colback=red!5!white,colframe=red!75!black,width=(\linewidth-4mm)/2, before=,after=\hfill}

\begin{tcolorbox}[boxsep=5mm,draft]
hi
\end{tcolorbox}
\begin{tcolorbox}[boxsep=8mm,draft]
hi
\end{tcolorbox}
\begin{tcolorbox}[boxsep=5mm]
hi
\end{tcolorbox}
\begin{tcolorbox}[boxsep=8mm]
hi
\end{tcolorbox}
\end{dispExample*}

% \begin{exdispExample}{boxsep}
% \tcbset{colback=red!5!white,colframe=red!75!black  
% \begin{tcolorbox}[boxsep=0mm]
% hi
% \end{tcolorbox}
% \begin{tcolorbox}[boxsep=0mm,draft]
% hi
% \end{tcolorbox}
% \end{exdispExample}
\end{docTcbKey}


\begin{docTcbKey}{left}{=\meta{length}}{style, no default, initially \texttt{4mm}}
Sets the left space between all text parts and frame (additional to |boxsep|).
This is an abbreviation for setting
|lefttitle|, |leftupper|, and |leftlower| to the same value.

设置所有文本部分和盒子间的左边距(除了|boxsep|之外)。 这是设置|lefttitle|、|leftupper|和|leftlower|为相同值的简写方式。


% 设置所有的文本内容同左侧边框的间隔(附加到|boxsep|).
% 这是同时将 |lefttitle|, |leftupper|, 和 |leftlower| 设置为同一个值的简写方式。
\begin{dispExample*}{sbs}
\tcbset{colback=red!5!white%
,colframe=red!75!black
% ,width=(\linewidth-4mm)/2, before=,after=\hfill
}

\begin{tcolorbox}[left=0mm]
指定 \verb|left=0mm|
\end{tcolorbox}

\begin{tcolorbox}
使用默认
\end{tcolorbox}

\begin{tcolorbox}[left=4mm]
指定 \verb|left=4mm|
\end{tcolorbox}


\begin{tcolorbox}[left=10mm]
指定 \verb|left=10mm|
\end{tcolorbox}
\end{dispExample*}
\end{docTcbKey}

% TODO 再看下
\begin{docTcbKey}[][doc new=2017-02-16]{left*}{=\meta{length}}{style, no default}
Sets \refKeyLe{/tcb/left} such that \meta{length} is the distance between
the left bounding box and the text parts.

设置\refKeyLe{/tcb/left},使得\meta{length}为左边界框和文本部分之间的距离。

% 设置 \refKeyLe{/tcb/left} 的值 \meta{length} 为盒子左边界和上下文的文本左侧的距离。

\begin{exdispExample}{left_star}
\tcbset{colback=red!5!white,colframe=red!75!black}

This is some text.

\begin{tcolorbox}[
enhanced,show bounding box]
默认情况
\end{tcolorbox}

\begin{tcolorbox}[left=0mm,
enhanced,show bounding box]
\verb|left=0mm|
\end{tcolorbox}

\begin{tcolorbox}[left*=0mm,
enhanced,show bounding box]
\verb|left*=0mm|
\end{tcolorbox}

\begin{tcolorbox}[grow to left by=5mm,
enhanced,show bounding box]
\verb|grow to left by=5mm|
\end{tcolorbox}

\begin{tcolorbox}[grow to left by=5mm,left*=0mm,
enhanced,show bounding box]
\verb|grow to left by=5mm,left*=0mm|
\end{tcolorbox}

\begin{tcolorbox}[grow to left by=5mm,left*=3mm,
enhanced,show bounding box]
\verb|grow to left by=5mm,left*=3mm|
\end{tcolorbox}


\end{exdispExample}
\end{docTcbKey}

% 这段Latex代码使用了tcolorbox宏包,它提供了创建漂亮框框的命令。在这个例子中,代码创建了三个tcolorbox,每个框内都包含了一段文本“这是一个\textbf{tcolorbox}”。

% 第一个tcolorbox使用了参数“grow to left by=5mm”和“left*=0mm”,
% 这意味着这个框会向左边延伸5毫米,并且左边的边框宽度为0毫米。同时,也使用了“enhanced”和“show bounding box”参数,这些参数可以让框框看起来更漂亮,并且显示边框的边界。

% 第二个tcolorbox只使用了“left*=0mm”参数,这意味着这个框的左边边框宽度为0毫米。同样,也使用了“enhanced”和“show bounding box”参数。

% 第三个tcolorbox只使用了“enhanced”和“show bounding box”参数,这意味着这个框没有任何特殊的设置,只是一个普通的tcolorbox。


% \clearpage
\begin{docTcbKey}{lefttitle}{=\meta{length}}{no default, initially \texttt{4mm}}
  Sets the left space between title text and frame (additional to |boxsep|).

设置标题文本的左侧同边框的距离(附加 |boxsep|)。
\begin{exdispExample}{lefttitle}
\tcbset{colback=red!5!white,colframe=red!75!black}

\begin{tcolorbox}[lefttitle=3cm,title=My Title]
这是一个\textbf{tcolorbox}.
\end{tcolorbox}
\end{exdispExample}
\end{docTcbKey}


\begin{docTcbKey}{leftupper}{=\meta{length}}{no default, initially \texttt{4mm}}
  Sets the left space between upper text and frame (additional to |boxsep|).

设置upper部分同左侧边边框的距离(附加 |boxsep|)。
\begin{exdispExample}{leftupper}
\tcbset{colback=red!5!white,colframe=red!75!black}

\begin{tcolorbox}[leftupper=3cm,title=My Title]
这是一个\textbf{tcolorbox}.
\end{tcolorbox}
\end{exdispExample}
\end{docTcbKey}

\begin{docTcbKey}{leftlower}{=\meta{length}}{no default, initially \texttt{4mm}}
  Sets the left space between lower text and frame (additional to |boxsep|).

设置lower部分同左侧边边框的距离(附加 |boxsep|)。
\begin{exdispExample}{leftlower}
\tcbset{colback=red!5!white,colframe=red!75!black}

\begin{tcolorbox}[leftlower=3cm]
这是一个\textbf{tcolorbox}.
\tcblower
这是lower部分。
\end{tcolorbox}
\end{exdispExample}
\end{docTcbKey}

\enlargethispage*{1cm}

\begin{docTcbKey}{right}{=\meta{length}}{style, no default, initially \texttt{4mm}}
  Sets the right space between all text parts and frame (additional to |boxsep|).
  This is an abbreviation for setting
  |righttitle|, |rightupper|, and |rightlower| to the same value.

设置所有文本部分同右侧边框的距离(附加 |boxsep|)。
这是同时将 |righttitle|, |rightupper|, 和 |rightlower| 设置为同一个值的简写方式。
\begin{exdispExample}{right}
\tcbset{colback=red!5!white,colframe=red!75!black}

\begin{tcolorbox}[width=5cm,right=2cm]
这是一个\textbf{tcolorbox}.
\end{tcolorbox}
\end{exdispExample}
\end{docTcbKey}





% \clearpage
  
\begin{docTcbKey}[][doc new=2017-02-16]{right*}{=\meta{length}}{style, no default}
  Sets \refKeyLe{/tcb/right} such that \meta{length} is the distance between
  the right bounding box and the text parts.

设置 \refKeyLe{/tcb/right} 的宽度 \meta{length} 为盒子右边框同上下文文本的右侧的距离。
\begin{exdispExample}{right_star}
\tcbset{colback=red!5!white,colframe=red!75!black}

\flushright This is some text.
\begin{tcolorbox}[grow to right by=5mm,right*=0mm,
  halign=right,enhanced,show bounding box]
这是一个\textbf{tcolorbox}.
\end{tcolorbox}
\end{exdispExample}
\end{docTcbKey}



\begin{docTcbKey}{righttitle}{=\meta{length}}{no default, initially \texttt{4mm}}
  Sets the right space between title text and frame (additional to |boxsep|).

设置标题文本右侧同右边框的距离(附加 |boxsep|)。
  \begin{exdispExample}{righttitle}
\tcbset{colback=red!5!white,colframe=red!75!black}

\begin{tcolorbox}[width=5cm,righttitle=2cm,title=My very long title text]
This is a \textbf{tcolorbox} with standard upper box dimensions.
\end{tcolorbox}
\end{exdispExample}
\end{docTcbKey}


\begin{docTcbKey}{rightupper}{=\meta{length}}{no default, initially \texttt{4mm}}
  Sets the right space between upper text and frame (additional to |boxsep|).

设置upper部分的文本同右边框的距离(附加|boxsep|).
\begin{exdispExample}{rightupper}
\tcbset{colback=red!5!white,colframe=red!75!black}

\begin{tcolorbox}[width=5cm,rightupper=2cm,title=My very long title text]
This is a \textbf{tcolorbox} with compressed upper box dimensions.
\end{tcolorbox}
\end{exdispExample}
\end{docTcbKey}





% \clearpage
\begin{docTcbKey}{rightlower}{=\meta{length}}{no default, initially \texttt{4mm}}
  Sets the right space between lower text and frame (additional to |boxsep|).

设置lower部分的右边同右侧边框的距离(附加 |boxsep|)。
\begin{exdispExample}{rightlower}
\tcbset{colback=red!5!white,colframe=red!75!black}

\begin{tcolorbox}[width=5cm,rightlower=2cm]
This is a \textbf{tcolorbox} with standard upper box dimensions.
\tcblower
This is the lower part with large space at right.
\end{tcolorbox}
\end{exdispExample}
\end{docTcbKey}



\begin{docTcbKey}{top}{=\meta{length}}{no default, initially \texttt{2mm}}
Sets the top space between text and frame (additional to |boxsep|).

设置文本同上边框的距离(附加 |boxsep|)。
\begin{exdispExample}{top}
\tcbset{colback=red!5!white,colframe=red!75!black%
,width=(\linewidth-4mm)/2
% ,nobeforeafter
, before=,after=\hfill
}

\begin{tcolorbox}[]
默认\verb|top=2mm|
\tcblower
这是lower部分。
\end{tcolorbox}
\begin{tcolorbox}[top=0mm]
\verb|top=0mm|
\tcblower
这是lower部分。
\end{tcolorbox}
\end{exdispExample}
\end{docTcbKey}


\begin{docTcbKey}{toptitle}{=\meta{length}}{no default, initially \texttt{0mm}}
  Sets the top space between title and frame (additional to |boxsep|).

设置标题文本同上边框的距离(附加 |boxsep|)。    
\begin{exdispExample}{toptitle}
\tcbset{colback=red!5!white,colframe=red!75!black
,width=(\linewidth-4mm)/2
% ,nobeforeafter
, before=,after=\hfill}

\begin{tcolorbox}[toptitle=3mm,title=My title]
\verb|toptitle=3mm|
\end{tcolorbox}
\tcbset{colback=red!5!white,colframe=red!75!black}
%
\begin{tcolorbox}[title=My title]
默认\verb|toptitle=0mm|
\end{tcolorbox}
\end{exdispExample}
\end{docTcbKey}



% \clearpage
\begin{docTcbKey}{bottom}{=\meta{length}}{no default, initially \texttt{2mm}}
Sets the bottom space between text and frame (additional to |boxsep|).

设置文本底部同边框的距离 (附加 |boxsep|).
\begin{exdispExample}{bottom}
\tcbset{colback=red!5!white,colframe=red!75!black
,width=(\linewidth-4mm)/2
% ,nobeforeafter
, before=,after=\hfill}

\begin{tcolorbox}[bottom=0mm]
\verb|bottom=0mm|
\tcblower
这是lower部分。
\end{tcolorbox}
\begin{tcolorbox}
默认\verb|bottom=2mm|
\tcblower
这是lower部分。
\end{tcolorbox}
\end{exdispExample}
\end{docTcbKey}

\begin{docTcbKey}{bottomtitle}{=\meta{length}}{no default, initially \texttt{0mm}}
  Sets the bottom space between title and frame (additional to |boxsep|).

设置标题同下方的边框的距离(附加 |boxsep|).
\begin{exdispExample}{bottomtitle}
\tcbset{colback=red!5!white,colframe=red!75!black
,width=(\linewidth-4mm)/2
% ,nobeforeafter
, before=,after=\hfill}

\begin{tcolorbox}[bottomtitle=3mm,title=My title]
\verb|bottomtitle=3mm|
\end{tcolorbox}
\begin{tcolorbox}[title=My title]
默认\verb|bottomtitle=0mm|
\end{tcolorbox}
\end{exdispExample}
\end{docTcbKey}


\begin{docTcbKey}{middle}{=\meta{length}}{no default, initially \texttt{2mm}}
Sets the space between upper and lower text to the separation line
(additional to |boxsep|).

% 将上下文本与分隔线之间的距离设置为分隔线(附加到|boxsep|)。

设置上下文本同分隔线的距离(附加 |boxsep|)。
\begin{exdispExample}{middle}
\tcbset{colback=red!5!white,colframe=red!75!black,width=(\linewidth-4mm)/3
% ,nobeforeafter
, before=,after=\hfill}

\begin{tcolorbox}
默认\verb|middle=2mm|
\tcblower
这是lower部分。
\end{tcolorbox}
\begin{tcolorbox}[boxsep=0mm]
\verb|boxsep=0mm|
\tcblower
这是lower部分。
\end{tcolorbox}
\begin{tcolorbox}[middle=0mm,boxsep=0mm]
\verb|middle=0mm,boxsep=0mm|
\tcblower
这是lower部分。
\end{tcolorbox}


\end{exdispExample}
\end{docTcbKey}

