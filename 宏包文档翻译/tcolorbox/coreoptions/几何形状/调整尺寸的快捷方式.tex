\subsubsection{Size Shortcuts\\调整尺寸的快捷方式}
\begin{docTcbKey}{size}{=\meta{name}}{no default, initially \texttt{normal}}
Sets all geometry keys with exception of \refKeyLe{/tcb/width} to
predefined length values.
For \meta{name}, the following values are feasible:

将除 \refKeyLe{/tcb/width} 外的所有尺寸设置为预定义的值。
可选的 \meta{name} 值有:    
  \begin{DescriptionL}{\docValue{minimal}}
  \item[\docValue{normal}]normal sized boxes e.g. of width |\linewidth|.
\\常用的盒子尺寸 e.g. 宽度为 |\linewidth|。
  \item[\docValue{title}]title line sized boxes.
  \\宽度同标题行一致。
  \item[\docValue{small}] small boxes e.g. for keyword highlighting.
  \\小一些的盒子 e.g. 用于关键字的高亮。
  \item[\docValue{fbox}] identical to the standard |\fbox|.
  \\同使用 |\fbox| 一样。
  \item[\docValue{tight}] no padding space at all.
  \\完全没有填充空间。

  \item[\docValue{minimal}] no padding space, no box rules.
  \\没有填充空间,没有边框。
  \end{DescriptionL}
% todo on line 是什么意思
\begin{exdispExample}{size_1}
\tcbset{colback=red!5!white,colframe=red!75!black}

\foreach \s in {normal,title,small,fbox,tight,minimal} {
  \tcbox[size=\s,on line]{\s} }

\foreach \s in {normal,title,small,fbox,tight,minimal} {
  \tcbox[size=\s,on line,title=Test]{\s} }

\foreach \s in {normal,title,small,fbox,tight,minimal} {
  \begin{tcolorbox}[size=\s,on line,title=Test,width=2.2cm]
    \s \tcblower lower\end{tcolorbox} }
\end{exdispExample}

\bigskip

\begin{tcolorbox}[tabularx={l|XXXXXX},title=Predefined values,
enhanced,fonttitle=\small\bfseries,fontupper=\small\ttfamily,
colback=yellow!10!white,colframe=red!50!black,colbacktitle=Salmon!30!white,
coltitle=black,center title
]
            & normal & title  & small & fbox  & tight & minimal\\\hline
boxrule     & 0.5mm  & 0.4mm  & 0.3mm & 0.4pt & 0.4pt & 0.0pt \\
boxsep      & 1.0mm  & 1.0mm  & 1.0mm & 3.0pt & 0.0pt & 0.0pt \\
left        & 4.0mm  & 2.0mm  & 1.0mm & 0.0pt & 0.0pt & 0.0pt \\
right       & 4.0mm  & 2.0mm  & 1.0mm & 0.0pt & 0.0pt & 0.0pt \\
top         & 2.0mm  & 0.25mm & 0.0mm & 0.0pt & 0.0pt & 0.0pt \\
bottom      & 2.0mm  & 0.25mm & 0.0mm & 0.0pt & 0.0pt & 0.0pt \\
toptitle    & 0.0mm  & 0.0mm  & 0.0mm & 0.0pt & 0.0pt & 0.0pt \\
bottomtitle & 0.0mm  & 0.0mm  & 0.0mm & 0.0pt & 0.0pt & 0.0pt \\
middle      & 2.0mm  & 0.75mm & 0.5mm & 1.0pt & 0.2pt & 0.0pt \\
arc         & 1.0mm  & 0.75mm & 0.5mm & 1.0pt & 0.0pt & 0.0pt \\
outer arc   & auto   & auto   & auto  & auto  & 0.0pt & 0.0pt \\
\end{tcolorbox}
\end{docTcbKey}


  

% \clearpage
\begin{docTcbKey}{oversize}{\colOpt{=\meta{length}}}{style, default |0pt|}
Sets the text width of the upper part to the current line width plus an
optional \meta{length}.
This is achieved by changing the keys \refKeyLe{/tcb/width}
\refKeyLe{/tcb/enlarge left by}, and
\refKeyLe{/tcb/enlarge right by} appropriately.
The resulting box is overlapping into the left and right margin of
the page.
Note that this style option has to be given \emph{after} all other
geometry keys!
Also see \refKeyLe{/tcb/grow sidewards by} and \refKeyLe{/tcb/spread sidewards}.

将upper部分的文本宽度设置为当前行宽再加上可选的\meta{length}。这是通过适当地更改键\refKeyLe{/tcb/width}、\refKeyLe{/tcb/enlarge left by}和\refKeyLe{/tcb/enlarge right by}来实现的。结果的框重叠到页面的左右边距上。请注意,这个样式选项必须在所有其他几何键之后给出!还请参见\refKeyLe{/tcb/grow sidewards by}和\refKeyLe{/tcb/spread sidewards}。

% 设置upper部分的文本的宽度为上下文中行宽加上 \meta{length}。这个效果是通过适当的改变 \refKeyLe{/tcb/width} \refKeyLe{/tcb/enlarge left by}, 和 \refKeyLe{/tcb/enlarge right by} 实现的。
% 最终的盒子会向左和右侧的边注伸展。注意,这个选项应该放置在所有其他的尺寸选项\emph{之后}!
% 另见 \refKeyLe{/tcb/grow sidewards by} 和 \refKeyLe{/tcb/spread sidewards}.
\begin{dispListing}
\tcbset{colback=red!5!white,colframe=red!75!black,fonttitle=\bfseries}

\textit{用于比较的普通文本:}\\
\lipsum[2]

\begin{tcolorbox}[oversize,title=Oversized box]
\lipsum[2]
\end{tcolorbox}

\begin{tcolorbox}[title=Normal box]
\lipsum[2]
\end{tcolorbox}
\end{dispListing}
\end{docTcbKey}

{\tcbusetemp}

  