\subsubsection{Rules\\线}
\begin{docTcbKey}{toprule}{=\meta{length}}{no default, initially \texttt{0.5mm}}
Sets the line width of the top rule to \meta{length}.

设置顶边框线的宽度为 \meta{length}。
\begin{exdispExample}{toprule}
\tcbset{colback=red!5!white,colframe=red!75!black}

\begin{tcolorbox}[toprule=3mm]
这是一个\textbf{tcolorbox}.
\end{tcolorbox}
\end{exdispExample}
\end{docTcbKey}


\begin{docTcbKey}{bottomrule}{=\meta{length}}{no default, initially \texttt{0.5mm}}
Sets the line width of the bottom rule to \meta{length}.

设置底边框线的宽度为 \meta{length}。
\begin{exdispExample}{bottomrule}
\tcbset{colback=red!5!white,colframe=red!75!black}

\begin{tcolorbox}[bottomrule=3mm]
这是一个\textbf{tcolorbox}.
\end{tcolorbox}
\end{exdispExample}
\end{docTcbKey}

\begin{docTcbKey}{leftrule}{=\meta{length}}{no default, initially \texttt{0.5mm}}
Sets the line width of the left rule to \meta{length}.

设置左边框线的宽度为 \meta{length}。
\begin{exdispExample}{leftrule}
\tcbset{colback=red!5!white,colframe=red!75!black}

\begin{tcolorbox}[leftrule=3mm]
这是一个\textbf{tcolorbox}.
\end{tcolorbox}
\end{exdispExample}
\end{docTcbKey}


\begin{docTcbKey}{rightrule}{=\meta{length}}{no default, initially \texttt{0.5mm}}
Sets the line width of the right rule to \meta{length}.

设置右边框线的宽度为 \meta{length}。
\begin{exdispExample}{rightrule}
\tcbset{colback=red!5!white,colframe=red!75!black}

\begin{tcolorbox}[rightrule=3mm]
这是一个\textbf{tcolorbox}.
\end{tcolorbox}
\end{exdispExample}
\end{docTcbKey}




% \clearpage
\begin{docTcbKey}{titlerule}{=\meta{length}}{no default, initially \texttt{0.5mm}}
Sets the line width of the rule below the title to \meta{length}.

设置标题文本下方的线的宽度为 \meta{length}。
\begin{exdispExample}{titlerule}
\tcbset{enhanced,colback=red!5!white,colframe=red!75!black,
colbacktitle=red!90!black}

\begin{tcolorbox}[titlerule=3mm,title=This is the title]
这是一个\textbf{tcolorbox}.
\end{tcolorbox}
\end{exdispExample}
\end{docTcbKey}


\begin{docTcbKey}{boxrule}{=\meta{length}}{style, no default, initially \texttt{0.5mm}}
Sets all rules of the frame to \meta{length}, i.\,e.\ 
\refKeyLe{/tcb/toprule}, \refKeyLe{/tcb/bottomrule}, \refKeyLe{/tcb/leftrule},
\refKeyLe{/tcb/rightrule}, and \refKeyLe{/tcb/titlerule}.

设置所有的边框线的宽度为 \meta{length}\footnote{i.\,e.\ 
\refKeyLe{/tcb/toprule}, \refKeyLe{/tcb/bottomrule}, \refKeyLe{/tcb/leftrule},
\refKeyLe{/tcb/rightrule}, 和 \refKeyLe{/tcb/titlerule}.}。
\begin{exdispExample}{boxrule}
\tcbset{colback=red!5!white,colframe=red!75!black}

\begin{tcolorbox}[boxrule=3mm]
这是一个\textbf{tcolorbox}.
\end{tcolorbox}
\end{exdispExample}
\end{docTcbKey}

\bigskip
\begin{marker}
More options for drawing a \refKeyLe{/tcb/borderline} are provided by using skins documented in
Section \ref{sec:skins} from page \pageref{sec:skins}.

更多的关于绘制 \refKeyLe{/tcb/borderline} 的选项的描述在 skins 的文档中,详见 \pageref{sec:skins} 页的 \ref{sec:skins} 小节。
\end{marker}

