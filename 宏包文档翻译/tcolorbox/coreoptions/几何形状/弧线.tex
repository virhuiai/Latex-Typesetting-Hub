\setcounter{section}{4}
\setcounter{subsection}{7}
\setcounter{subsubsection}{2}
% Arcs\hfill 
\subsubsection{弧线}
\begin{docTcbKey}{arc}{=\meta{length}}{no default, initially \texttt{1mm}}
设置边框的四个角落的弧的内半径。\hfill Sets the inner radius of the four frame arcs to \meta{length}.

% \begin{exdispExample}{arc}
% \tcbset{colback=red!5!white,colframe=red!75!black}
% \begin{tcolorbox}[arc=0mm]
% 这是一个\textbf{tcolorbox}.
% \end{tcolorbox}
% \begin{tcolorbox}[arc=3mm]
% 这是一个\textbf{tcolorbox}.
% \end{tcolorbox}
% \end{exdispExample}

\begin{dispExample*}{sidebyside,lefthand ratio=0.6}
\tcbset{colback=red!5!white,colframe=red!75!black}
\begin{tcolorbox}[arc=0mm]
\verb|arc=0mm|的效果
\end{tcolorbox}
\end{dispExample*}

\begin{dispExample*}{sidebyside,lefthand ratio=0.6}
\tcbset{colback=red!5!white,colframe=red!75!black}
\begin{tcolorbox}[arc=3mm]
\verb|arc=3mm|的效果
\end{tcolorbox}
\end{dispExample*}
\end{docTcbKey}



% \begin{exdispExample*}{arc_default}{sbs,lefthand ratio=0.6}
 
% \end{exdispExample*}

% \begin{exdispExample*}{arc_3cm}{sbs,lefthand ratio=0.6}
 
% \end{exdispExample*}

% gpt4:
% 在 LaTeX 中,颜色的混合可以使用 ! 符号来实现。这种写法叫做 "interpolation of colors"。

% red!5!white 的意思是,颜色由5%的红色和95%的白色混合而成。结果是一个非常浅的红色。

% red!75!black 的意思是,颜色由75%的红色和25%的黑色混合而成。结果是一个较深的红色。

% claude
% !后面的数字表示混合比例,范围是0到100。
% 0表示完全使用第一个颜色。
% 100表示完全使用第二个颜色。
% 所以:

% red!0!white 等同于纯红色red。
% red!100!white 等同于纯白色white。


% \clearpage
\begin{docTcbKey}[][doc new=2015-05-05]{circular arc}{}{style, no value}
  Sets \refKeyLe{/tcb/arc} to match the half of the inner width of the colored box.
  If width and height of the box are identical, this gives a circle.
  
  将 \refKeyLe{/tcb/arc} 设置为盒子内部宽度的一半。如果盒子的宽度和高度相同,就会得到一个圆。
  \begin{marker}
  If the height of the box is smaller than the width, the result will look
  quite ugly.
  
  如果盒子的高度小于宽度,结果看起来会很难看。   
  \end{marker}
  \begin{exdispExample*}{circular_arc}{sbs,lefthand ratio=0.6}
  \begin{tcolorbox}[width=3cm,
  colback=red!5!white,
  colframe=red!75!black,
  halign=center,valign=center,
  square,circular arc]
  square,circular arc
  \end{tcolorbox}
  \end{exdispExample*}

\begin{引述之言}{virhuiai}
如果还设置了标题,会怪怪的
\end{引述之言}
\begin{dispExample*}{sbs}
\begin{tcolorbox}[width=3cm,
colback=red!5!white,
colframe=red!75!black,
halign=center,valign=center,
square,circular arc
,title=test]
square,circular arc
\end{tcolorbox}
\end{dispExample*}
\end{docTcbKey}


  % 
% \begin{exdispExample*}{circular_arc}{sbs,lefthand ratio=0.6}
% \begin{tcolorbox}[width=3cm,
% colback=red!5!white,
% colframe=red!75!black,
% halign=center,valign=center,
% square,circular arc
% ,title=test]
% square,circular arc
% \end{tcolorbox}
% \end{exdispExample*}
  
  
  \begin{docTcbKey}[][doc new=2015-05-05]{bean arc}{}{style, no value}
  Sets \refKeyLe{/tcb/arc} to match the smaller value of the
  half of the inner width and of the inner height of the colored box.
  
%   设置 \refKeyLe{/tcb/arc} 为盒子内宽和内高中较小者值的一半。
  
  将\refKeyLe{/tcb/arc}设置为盒子内宽度一半和高度一半的较小值。

  \begin{marker}
  This only works for a fixed \refKeyLe{/tcb/height}. Also, \refKeyLe{/tcb/bean arc}
  must be used \emph{after} width and height are set by option keys.
  
%   这只适用于 \refKeyLe{/tcb/height} 为固定值的情况。此外,\refKeyLe{/tcb/bean arc} 选项设置需要在设置宽度和高度之后。
  这仅适用于固定的\refKeyLe{/tcb/height}。另外,\refKeyLe{/tcb/bean arc}必须在宽度和高度选项设置之后使用。
  \end{marker}
  \begin{exdispExample*}{bean_arc}{sbs,lefthand ratio=0.6}
  \tcbset{size=fbox,boxrule=0.5mm,
  colback=red!5!white,
  colframe=red!75!black,
  halign=center,valign=center}
  
  \begin{tcolorbox}[width=3cm,height=2cm,
  bean arc]
  Box A
  \end{tcolorbox}
  
  \begin{tcolorbox}[width=2cm,height=3cm,
  bean arc]
  Box B
  \end{tcolorbox}
  \end{exdispExample*}
  \end{docTcbKey}
  
  % 八角形
  \begin{docTcbKey}[][doc new=2015-05-05]{octogon arc}{}{style, no value}
  Sets \refKeyLe{/tcb/arc} to match $\frac{1}{2+\sqrt{2}}$ of the inner width
  of the colored box. If width and height of the box are identical,
  the interior is a regular octogon.

将\refKeyLe{/tcb/arc}设置为盒子的内部宽度的$\frac{1}{2+\sqrt{2}}$。如果盒子的宽度和高度相同,则内部是一个正八边形。

% 设置 \refKeyLe{/tcb/arc} 为盒子内部宽度的 $\frac{1}{2+\sqrt{2}}$ 。如果盒子的宽度和高度相同,内部是一个规则的八边形。
  % \begin{tcolorbox}[
  %   width=2.1cm,octogon arc,
  %   ]
  %   STOP
  %   \end{tcolorbox}
\begin{exdispExample*}{octogon_arc}{sbs,lefthand ratio=0.82}
\begin{tcolorbox}[enhanced,% 启用高级选项。
size=minimal,%去除默认边距,仅显示框体。
auto outer arc,% 自动调整外部圆角半径以适应框体大小。
width=2.1cm,
octogon arc,%  外部圆角为八角形。
colback=red,%  背景色为红色。
colframe=white,% 边框颜色为白色。
colupper=white,% 文字颜色为白色。
fontupper=\fontsize{7mm}{7mm}\selectfont\bfseries\sffamily,
%文字大小为7毫米,粗体,无衬线字体。
halign=center,% 水平居中对齐
valign=center,%垂直居中对齐。
square,% square: 边框为直角。
arc is angular,% arc is angular: 内部圆角为直角。
borderline={0.2mm}{-1mm}{red}
%使用红色的0.2毫米线条作为边框,内部偏移1毫米。
]
STOP
\end{tcolorbox}
\end{exdispExample*}
\end{docTcbKey}

% \begin{dispExample*}{sidebyside,lefthand ratio=0.82}
% \begin{tcolorbox}[enhanced,% 启用高级选项。
% size=minimal,%去除默认边距,仅显示框体。
% auto outer arc,% 自动调整外部圆角半径以适应框体大小。
% width=2.1cm,
% % octogon arc,%  外部圆角为八角形。
% colback=red,%  背景色为红色。
% colframe=white,% 边框颜色为白色。
% colupper=white,% 文字颜色为白色。
% fontupper=\fontsize{7mm}{7mm}\selectfont\bfseries\sffamily,
% %文字大小为7毫米,粗体,无衬线字体。
% halign=center,% 水平居中对齐
% valign=center,%垂直居中对齐。
% square,% square: 边框为直角。
% arc is angular,% arc is angular: 内部圆角为直角。
% borderline={0.2mm}{-1mm}{red}
% %使用红色的0.2毫米线条作为边框,内部偏移1毫米。
% ]
% STOP
% \end{tcolorbox}
% \end{dispExample*}


% angular,有角度的
% \clearpage
\begin{docTcbKey}[][doc new=2015-05-05]{arc is angular}{}{no value, initially unset}
  Using this options applies a patch which straightens the corners arcs of
  the boxes. The little arcs are replaced by little straight lines.
  
%   这个选项\footnote{arc is angular}用于对盒子四角弧线的一个补丁。小弧线被小直线代替。

  使用此选项将应用一个补丁,使盒子的角弧线变直。小弧线将被小直线替换。
  \begin{marker}
  This patch is considered as an experimental feature.
  It changes some of the original \tikzname\ code. This change may break
  with future updates of \tikzname.
  
%   这个补丁被认为是一个实验性的特征。它改变了一些原始的 \tikzname\ 代码。此更改可能会随着 \tikzname\ 的未来更新而中断。
  这个补丁被视为一项实验性的功能。它改变了一些原始的\tikzname 代码。这种改变可能会在未来的\tikzname 更新中出现问题。
  \end{marker}
  
\begin{exdispExample*}{arc_is_angular}{sbs,lefthand ratio=0.6}
\tcbset{colback=red!5!white,%
colframe=red!75!black,%
arc=3mm}

\begin{tcolorbox}[arc is angular]
arc is angular.
\end{tcolorbox}
\begin{tcolorbox}[arc is curved]
arc is curved
\end{tcolorbox}
\end{exdispExample*}
  
  \end{docTcbKey}
  
  
\begin{docTcbKey}[][doc new=2015-05-05]{arc is curved}{}{no value, initially set}
This option resets the patch from \refKeyLe{/tcb/arc is angular}. The
original \tikzname\ code is activated.

% 此选项将 \refKeyLe{/tcb/arc is angular} 的补丁修改重置为角度。激活原始的 \tikzname\ 代码。

此选项将重置 \refKeyLe{/tcb/arc is angular} 所设置的修补程序。原始的 \tikzname\ 代码将被激活。
\end{docTcbKey}


\begin{docTcbKey}{outer arc}{=\meta{length}}{no default, initially unset}
Sets the outer radius of the four frame arcs to \meta{length}.

设置盒子四个角的弧的外半径为 \meta{length}。
% 将四个框架拱形的外半径设置为\meta{length}。

\begin{dispExample*}{sidebyside}
\tcbset{colback=red!5!white,%
colframe=red!75!black}
\begin{tcolorbox}[]
这是一个\textbf{tcolorbox}.
\end{tcolorbox}
\end{dispExample*}

\begin{dispExample*}{sidebyside}
\tcbset{colback=red!5!white,%
colframe=red!75!black}
\begin{tcolorbox}[arc=4mm]
arc=4mm
\end{tcolorbox}
\end{dispExample*}

\begin{dispExample*}{sidebyside}
\tcbset{colback=red!5!white,%
colframe=red!75!black}
\begin{tcolorbox}[arc=4mm,%
outer arc=1mm]
arc=4mm,outer arc=1mm
\end{tcolorbox}
\end{dispExample*}
\end{docTcbKey}

\begin{docTcbKey}{auto outer arc}{}{no value, initially set}
Sets the outer radius of the four frame arcs automatically in
dependency of the inner radius given by \refKeyLe{/tcb/arc}.

根据 \refKeyLe{/tcb/arc} 给出的内部半径自动设置外部半径。

\begin{dispExample*}{sidebyside}
\tcbset{colback=red!5!white,%
colframe=red!75!black}
\begin{tcolorbox}[arc=4mm]
arc=4mm
\end{tcolorbox}
\end{dispExample*}

\begin{dispExample*}{sidebyside}
\tcbset{colback=red!5!white,%
colframe=red!75!black} 
\begin{tcolorbox}[arc=4mm,%
outer arc=1mm]
arc=4mm,outer arc=1mm
\end{tcolorbox}
\end{dispExample*}

\begin{dispExample*}{sidebyside}
\tcbset{colback=red!5!white,%
colframe=red!75!black}
\begin{tcolorbox}[arc=4mm,%
auto outer arc]
\verb|arc=4mm,auto outer arc|
\end{tcolorbox}
\end{dispExample*}

\begin{dispExample*}{sidebyside}
\tcbset{colback=red!5!white,%
colframe=red!75!black}
\begin{tcolorbox}[arc=4mm,%
outer arc=9mm]
\verb|arc=4mm,outer arc=9mm|
\end{tcolorbox}
\end{dispExample*}

\begin{dispExample*}{sidebyside}
\tcbset{colback=red!5!white,%
colframe=red!75!black}
\begin{tcolorbox}[arc=4mm,%
outer arc=0.1mm]
\verb|arc=4mm,outer arc=0.1mm|
\end{tcolorbox}
\end{dispExample*}
\end{docTcbKey}
  