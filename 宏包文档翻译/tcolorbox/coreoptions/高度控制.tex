In a typical usage scenario, the height of a |tcolorbox| is computed automatically
to fit the content. Nevertheless, the height can be set to a fixed value
or to fit commonly for several boxes, e.\,g. if boxes are set side by side.

在典型的使用场景中,一个 |tcolorbox| 的高度是根据内容自动计算的。 
尽管如此,其高度是可以设置为一个固定的值,或自适就能放置指定个数的盒子, e.\,g. 如果例子设置为并排。

\bigskip
\begin{marker}
The height control keys are only applicable to unbreakable boxes.
If a box is set to be \refKey{/tcb/breakable}, the height is always
computed according to the \emph{natural height}.

高度控制只在不可分页的例子上使用。
如果一个盒子设置了 \refKey{/tcb/breakable}, 其他高度将终究使用自然高度(\emph{natural height}).
\end{marker}
\bigskip


\begin{docTcbKey}{natural height}{}{no value, initially set}
Sets the total height of the colored box to its natural height depending
on the box content.

根据盒子的内容将%有色
盒子的总高度设置为其自然高度。
\end{docTcbKey}

\begin{docTcbKey}{height}{=\meta{length}}{no default}
Sets the total height of the colored box to \meta{length} independent
of the box content. \meta{length} is the minimum height of the box, if
\refKey{/tcb/height plus} is larger than zero.

将%有色
盒子的总高度设置为与内容无关的长度 \meta{length} 。 \meta{length} 是盒子的最小高度, 如果
\refKey{/tcb/height plus} 大于零。
\begin{exdispExample}{height}
\tcbset{width=(\linewidth-2mm)/3,before=,after=\hfill,
colframe=blue!75!black,colback=white}

\begin{tcolorbox}[height=1cm,valign=center]
This box has a height of 1cm.
\end{tcolorbox}
\begin{tcolorbox}[height=2cm,valign=center]
This box has a height of 2cm.
\end{tcolorbox}
\begin{tcolorbox}[height=3cm,split=0.5,valign=center,valign lower=center]
This box has a height of 3cm.
\tcblower
Lower part.
\end{tcolorbox}
\end{exdispExample}

todo 设置  before=,after=\hfill 就可以一行放多个了
\begin{exdispExample}{height2}
\tcbset{width=(\linewidth-2mm)/3,%before=,after=\hfill,
colframe=blue!75!black,colback=white}

\begin{tcolorbox}[height=1cm,valign=center]
This box has a height of 1cm.
\end{tcolorbox}
\begin{tcolorbox}[height=2cm,valign=center]
This box has a height of 2cm.
\end{tcolorbox}
\begin{tcolorbox}[height=3cm,split=0.5,valign=center,valign lower=center]
This box has a height of 3cm.
\tcblower
Lower part.
\end{tcolorbox}
\end{exdispExample}
\end{docTcbKey}



% \enlargethispage*{10mm}
\begin{docTcbKey}{height plus}{=\meta{length}}{no default, initially |0pt|}
The box may extend a given fixed \refKey{/tcb/height} up to the given \meta{length}.

盒子将高度最大可扩展到 \refKey{/tcb/height} 加上给定的长度\footnote{译注:类似于定义的弹性长度。} \meta{length}。
\begin{exdispExample}{height_plus}
\tcbset{colback=red!5!white,colframe=red!75!black,left=1mm,top=1mm,bottom=1mm,
right=1mm,boxsep=0mm,width=3cm,nobeforeafter}

\begin{tcolorbox}[height=1cm]
This is a tcolorbox.
\end{tcolorbox}
\begin{tcolorbox}[height=1cm,height plus=1cm]
This is a tcolorbox.
\end{tcolorbox}
\begin{tcolorbox}[height=1cm,height plus=1cm]
This is a tcolorbox. This is a tcolorbox. This is a tcolorbox.
\end{tcolorbox}
\end{exdispExample}
\end{docTcbKey}


\begin{docTcbKey}{height from}{=\meta{min} \texttt{to} \meta{max}}{style, no default}
Sets the box height to a dimension between \meta{min} and \meta{max}.

将盒子高度范围设置为从\meta{min}到\meta{max}。
\begin{exdispExample}{height_from}
%\usepackage{lipsum}
\newtcolorbox{mybox}{colback=red!5!white,colframe=red!75!black,left=1mm,top=1mm,
bottom=1mm,right=1mm,boxsep=0mm,width=4.5cm,nobeforeafter,
height from=2cm to 8cm}

\begin{mybox}
This is a tcolorbox.
\end{mybox}
\begin{mybox}
This is a tcolorbox. This is a tcolorbox. This is a tcolorbox.
\end{mybox}
\begin{mybox}
\lipsum[2]
\end{mybox}
\end{exdispExample}
\end{docTcbKey}



\begin{docTcbKey}[][doc new=2014-10-31]{text height}{=\meta{length}}{style, no default}
Sets the text height to \meta{length}. This is the length from the top 
of the upper part to the bottom of the optional lower part.
See also \refKey{/tcb/text width}.

将文本高度设置为 \meta{length}。这是从upper部分的顶部到可选的lower部分的底部的长度。另见 \refKey{/tcb/text width}.


\begin{exdispExample}{text_height}
\tcbset{colback=red!5!white,colframe=red!75!black}

\begin{tcolorbox}[text height=2cm]
This is a \textbf{tcolorbox} where the text area has a height of 2cm.
\end{tcolorbox}
\end{exdispExample}
\end{docTcbKey}




 %%\clearpage

\begin{docTcbKey}[][doc new=2014-11-07]{add to height}{=\meta{length}}{style, no default}
Adds \meta{length} to the current height of the colored box.
\refKey{/tcb/height} has to be set before this key is used!
If this option is used several times, then the \refKey{/tcb/height} is
also increased several times.

为盒子的当前高度添加\meta{length}。%
在此命令之前设定的 \refKey{/tcb/height} 将被用上!%
如果此项设置了多次, 那么 \refKey{/tcb/height} 也会被增加{\bf 多次}。
\begin{exdispExample}{add_to_height}
\tcbset{height=2cm,
valign=center,width=(\linewidth-2mm)/2,
before=,after=\hfill,colframe=blue!75!black,colback=white}

\begin{tcolorbox}
This box has a height of 2cm.
\end{tcolorbox}
\begin{tcolorbox}[add to height=1cm]
This box has a height of 3cm.
\end{tcolorbox}
\end{exdispExample}
\end{docTcbKey}


\begin{docTcbKey}[][doc new=2016-02-16]{add to natural height}{=\meta{length}}{style, no default}
The application of this option generates a box with natural height plus
the given \meta{length}. If this option is used several times, then the
last setting of \meta{length} wins. The resulting box is not considered
a fixed height box and the implementation is quite different to
\refKey{/tcb/add to height}.

应用本项设置,将会生成一个盒子,高度为自然高度加上给定的 \meta{length}。
如果多次设置,{\bf 最后一次}设置的\meta{length}生效。 生成的盒子不是固定的高度且其实现也同 \refKey{/tcb/add to height} 很不同。
\begin{exdispExample}{add_to_natural_height}
\tcbset{valign=center,width=(\linewidth-2mm)/2,
before=,after=\hfill,colframe=blue!75!black,colback=white}

\begin{tcolorbox}
This box has natural height.
\end{tcolorbox}
\begin{tcolorbox}[add to natural height=1cm]
This box has natural height plus 1 cm.
\end{tcolorbox}
\end{exdispExample}
\end{docTcbKey}



 %\clearpage
\begin{docTcbKey}[][doc new and updated={2014-09-22}{2016-02-17}]{height fill}{\colOpt{=true\textbar false\textbar maximum}}{default |true|, initially |false|}
If set to \docValue*{true}, the height of the |tcolorbox| is set to the rest of the
available vertical space of the current page.
If set to \docValue{maximum}, the page is compressed as much as possible.
Note that the |tcolorbox|
is always set as its own paragraph using this option.
Also see \refKey{/tcb/text fill}.

如果设置为\docValue*{true},则|tcolorbox|的高度将设置为当前页面剩余的可用垂直空间。 如果设置为\docValue{maximum},则页面会尽可能地压缩。 %
注意, |tcolorbox| 中的段落始终设置上了这个选项。
% 请注意,使用此选项始终将|tcolorbox|设置为自己的段落。 
另请参见\refKey{/tcb/text fill}。

% 如果设置为 \docValue*{true}, |tcolorbox| 的高度将会设置为当前页的剩余空间的高度。%
% 如果设置为 \docValue{maximum}, 页面被尽可能多地压缩。%
% 注意, |tcolorbox| 中的段落始终设置上了这个选项。%
% 另见 \refKey{/tcb/text fill}.%
\begin{marker}
Note that the library \mylib{breakable} has to be loaded to use this key!

注意,如果使用此设置,需要加载 \mylib{breakable} 库!
\end{marker}
This height control key is only applicable to unbreakable boxes, but it
uses code from the library \mylib{breakable}.
The counterpart for breakable boxes is \refKey{/tcb/height fixed for}.

此项高度控制只作用在不可分页的盒子上, 但它复用了 \mylib{breakable} 库中的代码。
The counterpart for breakable boxes is \refKey{/tcb/height fixed for}.

This option can and should not be used for boxes in boxes, but it can be
used for boxes inside a \refEnv{tcbraster}.

此项不能也不应用在盒子中的盒子, 但它可以用在 \refEnv{tcbraster} 中的盒子。

\begin{dispListing}
%\usepackage{lipsum}
%\tcbuselibrary{breakable}
\begin{tcolorbox}[height fill,
colback=red!5!white,colframe=red!75!black,fonttitle=\bfseries,
title=填充页面剩余部分的盒子]
\lipsum[1]
\end{tcolorbox}
\end{dispListing}
\end{docTcbKey}
{\tcbusetemp}




 %\clearpage
\begin{docTcbKey}[][doc new={2017-06-28}]{inherit height}{\colOpt{=\meta{fraction}}}{default |1|, initially unset}
If this option is used for a |tcolorbox| which is embedded inside
another (outer) |tcolorbox| \emph{and} if this outer |tcolorbox| has
a fixed height, then the given \meta{fraction} of the available text height
of the outer |tcolorbox| is used as \refKey{/tcb/height} for the current
|tcolorbox|.
Otherwise, \refKey{/tcb/natural height} is applied for the current
|tcolorbox|.

如果此项所设置的 |tcolorbox| 是个嵌入在另一个%(外部)
|tcolorbox| 中,\emph{且}外部分的这个 |tcolorbox| 有着固定的高度, 那么,设定的 \meta{fraction} 乘以外部的这个 |tcolorbox| 盒子的剩余高度用作当前盒子的高度 \refKey{/tcb/height}。否则,当前 |tcolorbox| 盒子将应用自然高度 \refKey{/tcb/natural height} 。

\begin{exdispExample}{inherit_height}
\tcbset{colframe=blue!75!black,colback=white,fonttitle=\bfseries}

\begin{tcolorbox}[title=外部盒子指定高度为4cm,height=4cm]
\begin{tcolorbox}[title=Inner box,nobeforeafter,inherit height]
这个内部盒子的高度使用外盒的剩余高度空间。
\end{tcolorbox}
\end{tcolorbox}

\begin{tcolorbox}[title=外部盒子使用自然高度]
\begin{tcolorbox}[title=Inner box,nobeforeafter,inherit height]
这个盒子使用自身的自然高度。
\end{tcolorbox}
\end{tcolorbox}

Deeply nested box using 60 percent of the available space.
Deeply nested box using 40 percent of the available space.
\begin{tcolorbox}[title=外部盒子指定高度为5cm,height=5cm]
\begin{tcolorbox}[title=内部盒子,nobeforeafter,inherit height]
\begin{tcolorbox}[colframe=red,beforeafter skip=0pt,inherit height=0.6]
    使用60\%可用空间的嵌套盒子。
\end{tcolorbox}
\begin{tcolorbox}[colframe=red,beforeafter skip=0pt,inherit height=0.4]
    使用40\%可用空间的嵌套盒子。
\end{tcolorbox}
\end{tcolorbox}
\end{tcolorbox}
\end{exdispExample}
\end{docTcbKey}





 %\clearpage

\begin{docTcbKey}[][doc new=2015-05-05]{square}{}{style, no value}
Sets \refKey{/tcb/height} to match the width of the colored box.

设置盒子的高度(\refKey{/tcb/height})为盒子的宽度。
\begin{exdispExample*}{square}{sbs,lefthand ratio=0.6}
\begin{tcolorbox}[width=3cm,
colback=red!5!white,
colframe=red!75!black,
halign=center,valign=center,
square]
This is a \textbf{tcolorbox}.
\end{tcolorbox}
\end{exdispExample*}
\end{docTcbKey}



\begin{docTcbKey}{space}{=\meta{fraction}}{no default, initially 0}
If the height of a |tcolorbox| is not the natural height, the space
difference between the forced and the natural size is distributed
between the upper and the lower part of the box. This space could also
be negative.
\meta{fraction} with a value between 0 and 1 is the amount of space
which is added to the upper part, the rest is added to the lower part.
If there is no lower part, then all of the space is added to
the upper part always.

如果 |tcolorbox| 的高度不是自然高度,强制大小和自然大小之间的空间差异分配在盒子的上部和下部之间。这个空间也可能是负数。 取值介于 0 和 1 的 \meta{fraction} 是添加到上部的空间量比例,其余部分添加到下部。如果没有下部,则所有空间始终添加到上部。

% 如果 |tcolorbox| 的高度不是自然高度, 指定的高度和自然尺寸的的高度差分布在盒子的上下两部分中。 高度差可以是负值。
% \meta{fraction} 是0到1之前的数值,此值指定的比例添加到upper部分,剩余的高度添加到下部分。
% 如果不存在lower部分, 那么所有的空间将添加到upper部分。
\begin{exdispExample}{fraction}
\tcbset{width=(\linewidth-2mm)/4,before=,after=\hfill,
colframe=blue!75!black,colback=white,height=3cm}

\begin{tcolorbox}
upper部分
\tcblower
lower部分
\end{tcolorbox}
\foreach \f in {0.2,0.4,0.7}
{\begin{tcolorbox}[space=\f]
upper部分
\tcblower
lower部分
\end{tcolorbox}}
\end{exdispExample}
\end{docTcbKey}

\begin{docTcbKey}{space to upper}{}{style}
This is an abbreviation for |space=1|, i.\,e. all extra space is added
to the upper part.

这是指定|space=1|的简写形式, i.\,e. 所有额外的空间都被添加到上部。(此法可读性更好)。
\end{docTcbKey}

\begin{docTcbKey}{space to lower}{}{style, initially set}
This is an abbreviation for |space=0|, i.\,e. all extra space is added
to the lower part (if there is any).

这是指定|space=0|的简写形式, i.\,e. 所有额外的空间都被添加到下部。(此法可读性更好)。
\end{docTcbKey}




 %\clearpage
\begin{docTcbKey}{space to both}{}{style}
This is an abbreviation for |space=0.5|, i.\,e. the extra space
equally distributed between the upper and the lower part.
这是指定|space=0.5|的简写形式, i.\,e. 额外的空间将平均分布到upper部分和lower部分。
\begin{exdispExample}{space_to_both}
\tcbset{width=(\linewidth-2mm)/3,before=,after=\hfill,
colframe=blue!75!black,colback=white,height=3cm}

\foreach \myspace in {space to upper,space to both,space to lower}
{\begin{tcolorbox}[\myspace]
This is the upper part.
\tcblower
This is the lower part.
\end{tcolorbox}}
\end{exdispExample}
\end{docTcbKey}



\begin{docTcbKey}[][doc new and updated={2015-02-15}{2020-07-30}]{space to}{=\meta{macro}}{no default, initially unset}
If the height of a |tcolorbox| is not the natural height, the space
difference between the forced and the natural size is saved into the
given local \meta{macro}. This \meta{macro} can and should be used inside
the box content to add content which is vertically sized to match \meta{macro}.

如果|tcolorbox|盒子的高度不是自然高度, 指定的高度同自然高度差的数值保存到给出的宏命令 \meta{macro}。 这个 \meta{macro} 可以在盒子中使用以用来控制内容的高度恰好同这高度差一致。
\begin{marker}
\begin{itemize}
\item 
The actual length saved into \meta{macro} is adapted dynamically
during several compilations -- at least two, but maybe more.
实际保存到 \meta{macro} 的值在多次编译期间是自适应 --- 至少2次, 可能更多次。
\item %
Due to the adaption algorithm, objects can be sized with
\meta{macro} plus any offset length.
根据自适应算法, 对象尺寸可能在 \meta{macro} 之上添加额外的偏移量。
\item 
Never ever use \meta{macro} multiplied with a factor. The only
exception to this rule is that the space can be split into parts which
sum to \meta{macro}.
永远不要使用 \meta{macro} 乘以一个因子。这个规则的唯一例外是,
分开的几个部分的高度和为\meta{macro}(即多个因子的和为1)。
\item %Never use this in combination with \refKey{/tcb/fit}.
不要同 \refKey{/tcb/fit} 组合使用。
\end{itemize}
\end{marker}
\begin{exdispExample}[runs=3]{space_to_1}
\begin{tcolorbox}[colframe=blue!75!black,colback=white,height=3cm,
space to=\myspace]
这是我的盒子高3cm。指定高度和自然高度差填充了图片    :\\[2mm]
\includegraphics[width=\linewidth,height=\myspace]{goldshade.png}\\[1mm]
这是其他一些文字。译注:图片的高度使用我们指定的 |\myspace|。
\end{tcolorbox}
\end{exdispExample}

\begin{exdispExample}[runs=3]{space_to_2}
\begin{tcolorbox}[colframe=blue!75!black,colback=white,height=3cm,
space to=\myspace]
\includegraphics[width=\linewidth,
height=0.33\dimexpr\myspace]{blueshade.png}\\[1mm]
这是我的盒子高3cm。\\[2mm]
\includegraphics[width=\linewidth,
height=0.67\dimexpr\myspace]{goldshade.png}
\end{tcolorbox}
\end{exdispExample}
\end{docTcbKey}



\begin{docTcbKey}{split}{=\meta{fraction}}{no default}
If the height of a |tcolorbox| is not the natural height, the
\meta{fraction} with a value between 0 and 1 determines the positioning
of the segmentation between the upper and the lower part. Here, 0 stands
for top and 1 for bottom. Note that the box is split regardless of
the actual dimensions of the text parts!

如果 |tcolorbox| 的高度不是自然高度, 取值0到1的
\meta{fraction} 决定了上下两部分的分割位置。在这里,0代表顶部,1代表底部。
注意,不论文本部分的实际尺寸如何,盒子都会被分割!
\begin{exdispExample}{split}
\tcbset{width=(\linewidth-2mm)/3,before=,after=\hfill,height=3cm,
colback=white,colframe=blue!75!black,valign=center,valign lower=center}

\foreach \f in {0,0.1,0.5,0.8,0.9,1}
{\begin{tcolorbox}[split=\f]
上,split: \f
\tcblower
This is the lower part with a lot of text in several lines.
\end{tcolorbox}}
\end{exdispExample}
\begin{exdispExample}{split2}
\tcbset{width=(\linewidth-2mm)/3,before=,after=\hfill,height=1.8cm,
colback=white,colframe=blue!75!black,valign=center,valign lower=center}

\foreach \f in {0,0.1,0.5,0.8,0.9,1}
{\begin{tcolorbox}[split=\f]
上,split: \f
\tcblower
This is the lower part with a lot of text in several lines.
\end{tcolorbox}}
\end{exdispExample}
\end{docTcbKey}


 %\clearpage
\begin{docTcbKey}[][doc updated=2014-11-07]{equal height group}{=\meta{id}}{no default}
Boxes which are members of an |equal height group| will all get the
same height, i.\,e. the maximum of all their natural heights. The
\meta{id} serves to distinguish between different height groups.
This \meta{id} should contain only characters which are feasible
for \TeX\ macro names, typically alphabetic characters but no numerals
and spaces.
Note that you have to compile twice to see changes and
that height groups are global definitions.

属于“等高组”的盒子将获得相同的高度,即它们所有自然高度的最大值。这里的 \meta{id} 用于区分不同的高度组。这个 \meta{id} 应该只包含可行的 \TeX\ 宏名称字符,通常是字母字符,但不包括数字和空格。请注意,您需要编译两次才能看到更改,而高度组是全局定义。

% 同一|equal height group|的成员将拥有相同的高度, i.\,e. 它们自然高度的最大者的值。
% \meta{id} 用来区分不同的身高组别。
% This \meta{id} should contain only characters which are feasible
% for \TeX\ macro names, typically alphabetic characters but no numerals
% and spaces.
% 注意,您必须编译两次才能看到更改,并且高度组是全局定义。

\begin{exdispExample}[runs=2]{equal_height_group}
\tcbset{width=(\linewidth-2mm)/3,before=,after=\hfill,arc=0mm,
colframe=blue!75!black,colback=white,fonttitle=\bfseries}

\begin{tcolorbox}[equal height group=A,adjusted title={一}]
这组最小的盒子
\end{tcolorbox}%
\begin{tcolorbox}[equal height group=A,adjusted title={二}]
这个盒子也小
\tcblower
但有lower部分。
\end{tcolorbox}%
\begin{tcolorbox}[equal height group=A,adjusted title={三}]
This box contains a lot of text just to fill the space
with word flowing and flowing and flowing until the box
is filled with all of it.
\end{tcolorbox}\linebreak

\tcbset{width=(\linewidth-1mm)/2,before=,after=\hfill,arc=0mm,
colframe=red!75!black,colback=white}

\begin{tcolorbox}[equal height group=B]
接着,我们使用另一个等高盒子组。
\end{tcolorbox}%
\begin{tcolorbox}[equal height group=B,after=]
\begin{equation*}
\int\limits_{0}^{1} x^2 = \frac13.
\end{equation*}
\end{tcolorbox}
\end{exdispExample}
\end{docTcbKey}

\medskip
\begin{marker}
See \Vref{sec:raster} for more equal height options.
另见 \Vref{sec:raster} 了解更多等高组相关选项。
\end{marker}



 %\clearpage
\begin{docTcbKey}{minimum for equal height group}{=\meta{id}:\meta{length}}{no default, initially unset}
Plants a \meta{length} into the equal height group with
the given \meta{id}. This ensures that the height will not drop below
\meta{length}. 
Note that you cannot reduce a computed height value by using this key with a small value.
The difference to applying \refKey{/tcb/height} directly is that the boxes
are never too small for their content.

指定值 \meta{length} 到等高组 \meta{id}。这确保高度不会小于 \meta{length}。
指定等高组 \meta{id} 的最小高度不小于 \meta{length}。%
注意,不能通过使用小值来减少计算出的高度值。%
同使用 \refKey{/tcb/height} 相比,此项设置不会使盒子小于它们的内容高度。

\begin{dispExample}
\tcbset{colframe=blue!75!black,colback=white,arc=0mm,
before=,after=\hfill,fonttitle=\bfseries,left=2mm,right=2mm,
width=3.5cm,
equal height group=C,
minimum for equal height group=C:3.5cm}

\begin{tcolorbox}
My first box. All boxes will get 3.5cm times 3.5cm
if the content height is not too large.
\end{tcolorbox}%
\begin{tcolorbox}
My second box.
\tcblower
This is the lower part.
\end{tcolorbox}%
\begin{tcblisting}{}
\textbf{Mixed}
with a listing.
\end{tcblisting}
\begin{tcolorbox}[title={Fourth box}]
My final box.
\end{tcolorbox}%
\end{dispExample}
\end{docTcbKey}

todo 再看看
\begin{docTcbKey}[][doc new=2016-03-24]{minimum for current equal height group}{=\meta{length}}{no default, initially unset}
Sets \refKey{/tcb/minimum for equal height group} for the current equal height
group. Apparently, this only works for an already known equal height group, i.e.
\refKey{/tcb/equal height group} has to be set \emph{before} this option is used.
This option is likely to be used in combination with \refKey{/tcb/raster equal height}

为当前等高组设置 \refKey{/tcb/minimum for equal height group}。
显然, 这只适用于已知的等高组, i.e.
\refKey{/tcb/equal height group}已经在此设置\emph{之前}设置。
此项学与 \refKey{/tcb/raster equal height} 组合使用。
\begin{exdispExample}[runs=2]{minimum_for_current_equal_height_group}
%\tcbuselibrary{raster}
\begin{tcbitemize}[raster equal height,colframe=blue!75!black,colback=white,
raster every box/.style={minimum for current equal height group=2cm}]
\tcbitem A
\tcbitem B
\end{tcbitemize}
\end{exdispExample}

\end{docTcbKey}




 %\clearpage
\begin{docTcbKey}[][doc new and updated={2015-11-27}{2016-02-22}]{use height from group}{\colOpt{=\meta{id}}}{style, default current group}
Sets the current box to a fixed \refKey{/tcb/height} which is copied from
an equal height group with the given \meta{id}. If this height is not
available during the current compilation, no fixed height setting is used.
If \meta{id} is omitted, the current equal height group is used which has
to be set before by \refKey{/tcb/equal height group}.\par
Note that the natural height of the current box is not considered for
computation of the group height. The main application for
\refKey{/tcb/use height from group} is that the height can be adapted
further by \refKey{/tcb/add to height}.

设置当前盒子的高度为一个固定 \refKey{/tcb/height} 值,值来自一个等高组\meta{id}。如果在当前编译期间此高度不可用,则不使用固定高度设置。
如果省略了\meta{id}, 则使用此前的\refKey{/tcb/equal height group}设置的。\par
请注意,在计算组高度时不考虑当前盒子的自然高度。
\refKey{/tcb/use height from group}主要用在当高度可以通过 \refKey{/tcb/add to height} 进一步调整。

\begin{dispExample}
\begin{tcolorbox}[use height from group=C,add to height=-2cm,
colframe=blue!75!black,colback=white]
Height from group \enquote{C} of the previous example, but reduced by 2cm.
\end{tcolorbox}%
\end{dispExample}

\begin{exdispExample}[runs=2]{use_height_from_group}
%\tcbuselibrary{raster}
Every line is inside an equal height group:
\begin{tcbraster}[raster equal height=rows,
title=Box \thetcbrasternum,
enhanced,size=small,colframe=red!50!black,colback=red!10!white]
\begin{tcolorbox}First line\\second line\\
The height of this box rules.\end{tcolorbox}
\begin{tcolorbox}[use height from group]Test\end{tcolorbox}
\begin{tcolorbox}[use height from group]
First line\\second line\end{tcolorbox}
\begin{tcolorbox}The height of this box rules.\end{tcolorbox}
\end{tcbraster}
\end{exdispExample}
\end{docTcbKey}



\begin{docCommand}[doc new=2015-11-27]{tcbheightfromgroup}{\marg{macro}\marg{id}}
Saves the height from an equal height group with the given \meta{id}
to a \meta{macro}. If this height is not available during the current compilation,
\meta{macro} is set to |0pt|.

保存等高组 \meta{id} 的高度到 \meta{macro}。如果在当前编译期间此高度不可用,
\meta{macro} 设为|0pt|.
\end{docCommand}