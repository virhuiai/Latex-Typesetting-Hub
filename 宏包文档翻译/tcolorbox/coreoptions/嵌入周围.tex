\subsection{Embedding into the Surroundings\\嵌入周围}\label{subsec:surroundings}

Typically, but not necessarily, a |tcolorbox| is put inside a separate paragraph and has some vertical space before and after it.
This behavior is controlled by the keys \refKeyLe{/tcb/before} and \refKeyLe{/tcb/after}.

通常情况下,但不一定,|tcolorbox| 放置在单独的段落中,并且在它前后有一些垂直空间。这种行为由键 \refKeyLe{/tcb/before} 和 \refKeyLe{/tcb/after} 控制。

% 通常情况下,但不是必须地,一个|tcolorbox|盒子被放置在单独的段落中,在它%之前和之后
% 的前后有一些垂直空间。%
% 这种行为是由\refKeyLe{/tcb/before}和\refKeyLe{/tcb/after}控制。

\begin{marker}
Before version 4.40, the default setting for \refKeyLe{/tcb/before}
and \refKeyLe{/tcb/after} was given by \refKeyLe{/tcb/autoparskip}.
Starting with version 4.40, the default setting is given by
\refKeyLe{/tcb/before skip balanced} and \refKeyLe{/tcb/after skip balanced}.\par
Note that old documents may need adaptions of page breaks.\par
Alternatively, the old default setting can be restored by using

% 版本4.40之前,\refKeyLe{/tcb/before}和\refKeyLe{/tcb/after}的默认设置由\refKeyLe{/tcb/autoparskip}控制。%
% 从4.40开始, 默认设置由\refKeyLe{/tcb/before skip balanced}和\refKeyLe{/tcb/after skip balanced}控制。\par
% 请注意,旧文档可能需要调整分页。\par
% 或者,可以使用以下命令恢复旧的默认设置

在版本4.40之前,\refKeyLe{/tcb/before}和\refKeyLe{/tcb/after}的默认设置由\refKeyLe{/tcb/autoparskip}给出。从版本4.40开始,默认设置由\refKeyLe{/tcb/before skip balanced}和\refKeyLe{/tcb/after skip balanced}给出。请注意,旧文档可能需要调整分页。或者,可以通过使用以下方式恢复旧的默认设置:
\begin{dispListing}
\tcbsetforeverylayer{autoparskip}
\end{dispListing}
inside the document preamble.

在文档导言区中。
\end{marker}

\begin{docTcbKey}{before}{=\meta{code}}{no default, initially see \refKeyLe{/tcb/before skip balanced}}
Sets the \meta{code} which is executed before the colored box.
It is not used for floating boxes.
Also, it is not used, if the box follows a heading immediately
and \refKeyLe{/tcb/ignore nobreak} is set to \docValue{false}.

设置在盒子绘制之前执行的 \meta{code}。它不用于浮动框。如果盒子紧接着标题而来并且 \refKeyLe{/tcb/ignore nobreak} 被设置为 \docValue{false},也不会使用。

% \meta{code}将在盒子绘制前执行。此项不用于浮动盒子。另外一个不生效的情况:如果盒子之前紧跟着标题且\refKeyLe{/tcb/ignore nobreak}设为\docValue{false}.
\end{docTcbKey}

\begin{docTcbKey}{after}{=\meta{code}}{no default, initially see \refKeyLe{/tcb/after skip balanced}}
Sets the \meta{code} which is executed after the colored box.
It is not used for floating boxes.

\meta{code}将在盒子绘制后执行。此项不用于浮动盒子。
\end{docTcbKey}


\begin{docTcbKey}{nobeforeafter}{}{style, no value}
Abbreviation for clearing the keys |before| and |after|. The colored box
is not put into a paragraph and there is no space before or after the box.

% 清除“之前”和“之后”密钥的缩写。彩色框不放入段落中,并且框前后没有空格。

清除|before|和|after|的简写形式。盒子没有放入段落中,盒子之前或之后没有空间。\footnote{译注:用上后,多个盒子就不会分行了}
\begin{exdispExample}{nobeforeafter}
\tcbset{myone/.style={colback=LightGreen,colframe=DarkGreen,
equal height group=nobefaf,width=\linewidth/4,nobeforeafter}}
\begin{tcolorbox}[myone,title=Box 1]Box 1\end{tcolorbox}%
\begin{tcolorbox}[myone,title=Box 2]Box 2\end{tcolorbox}%
\begin{tcolorbox}[myone,title=Box 3]Box 3\end{tcolorbox}%
\begin{tcolorbox}[myone,title=Box 4]Box 4\end{tcolorbox}
\end{exdispExample}
\end{docTcbKey}



\begin{docTcbKey}{force nobeforeafter}{}{style, no value}
Forces the setting of \refKeyLe{/tcb/nobeforeafter} even if
\refKeyLe{/tcb/before} and \refKeyLe{/tcb/after} are set to other values
later. Do not use this option globally unless you \emph{really} know what you do.
Note that embedded boxes do not inherit this forced clearance.

强制设置\refKeyLe{/tcb/nobeforeafter},甚至\refKeyLe{/tcb/before}和\refKeyLe{/tcb/after}是在这个选项设置之后又设置的。不要全局使用此选项,除非您\emph{真}的知道在做什么。注意,嵌套的盒子不会继承此项设置。
\end{docTcbKey}


%%\clearpage

\begin{docTcbKey}[][doc new={2020-09-25}]{before skip balanced}{=\meta{glue}}{no default, initially |0.5\textbackslash baselineskip plus 2pt|}
Inserts some vertical space before the colored box. This style sets \refKeyLe{/tcb/before}.\par
在盒子之前插入一些垂直空间。该样式设置 \refKeyLe{/tcb/before}。\par
% 在 tcolorbox 盒子之前,插入一些竖直的空白。此项也会设置 \refKeyLe{/tcb/before}。\par
If the depth of the
preceeding \TeX\ box is between |0pt| and |0.3\baselineskip|,
the distance between the \emph{baseline} of the preceeding \TeX\ box and the tcolorbox
ist set to \meta{glue}$+$|0.3\baselineskip|.\par
如果前面的 \TeX\ 盒子的深度在 |0pt| 和 |0.3\baselineskip| 之间,前面的 \TeX\ 盒子的 \emph{基线} 与 tcolorbox 之间的距离设置为 \meta{glue}$+$|0.3\baselineskip|。\par
% 如果要处理的 \TeX\ 盒子的深度介于 |0pt| 到 |0.3\baselineskip|,
% \TeX\ 盒子的 \emph{基线} 到 tcolorbox 盒子间的空白高为 \meta{glue}$+$|0.3\baselineskip|.\par
If the depth is larger, the distance of the preceeding \TeX\ box and the tcolorbox
ist set to \meta{glue}.\par
如果深度更大,则前面的 \TeX\ 盒子与 tcolorbox 之间的距离设置为 \meta{glue}。\par
Alternatively, see \refKeyLe{/tcb/before skip} which ignores the \emph{baseline}.

 或者,查看 \refKeyLe{/tcb/before skip},该选项忽略了 \emph{基线}。
% 如果深度更大些, 则 \TeX\ 盒子和 tcolorbox 盒子间的空白设为 \meta{glue}.\par
% 也可以参阅 \refKeyLe{/tcb/before skip},它忽略 \emph{baseline}.

\begin{exdispExample*}{before_skip_balanced}{sbs,lefthand ratio=0.6}
Some text.
\begin{tcolorbox}[before skip balanced=1cm,
colframe=red!50!white]
This is a \textbf{tcolorbox}.
\end{tcolorbox}
\end{exdispExample*}
\end{docTcbKey}


\begin{docTcbKey}[][doc new={2020-09-25}]{after skip balanced}{=\meta{glue}}{no default, initially |0.5\textbackslash baselineskip plus 2pt|}
Inserts some vertical space of the given \meta{glue} after the colored box.
This style sets \refKeyLe{/tcb/after}.
Additionally, |\prevdepth| is set to |0.3\baselineskip|. The following
\TeX\ box may enlarge the space by further glue to adjust its \emph{baseline}.
Alternatively, see \refKeyLe{/tcb/after skip} which ignores the \emph{baseline}.

插入指定的竖直空白到盒子后,高度为 \meta{glue}。%
此项会设置 \refKeyLe{/tcb/after}。%
此外, |\prevdepth| 设为 |0.3\baselineskip|。下面的 \TeX\ 盒子可以通过弹性%进一步弹性来
扩大空间以调整其\emph{基线}。%
另见 \refKeyLe{/tcb/after skip},它忽略 \emph{baseline}。

\begin{exdispExample*}{after_skip_balanced}{sbs,lefthand ratio=0.6}
\begin{tcolorbox}[after skip balanced=1cm,
colframe=red!50!white]
This is a \textbf{tcolorbox}.
\end{tcolorbox}
Some text.
\end{exdispExample*}
\end{docTcbKey}



\begin{docTcbKey}[][doc new={2020-09-25}]{beforeafter skip balanced}{=\meta{glue}}{no default, initially |0.5\textbackslash baselineskip plus 2pt|}
插入一些指定高度为 \meta{glue} 的垂直空间到盒子的前面和后面。
此项会同时设置 \refKeyLe{/tcb/before skip balanced} 和 \refKeyLe{/tcb/after skip balanced}。
todo tikzpicture
\begin{exdispExample*}{beforeafter_skip_balanced}{sbs,lefthand ratio=0.6}
\newtcolorbox{doubleline}[1][]{
beforeafter skip balanced=0pt,
height=1.8\baselineskip,
enlarge top by=.1\baselineskip,
enlarge bottom by=.1\baselineskip,
colframe=blue!20,colback=blue!5,
size=small,valign upper=center,#1 }

\noindent\begin{tikzpicture}
\path[use as bounding box] (0,0)
rectangle (0.1,0.1);
\foreach \y in {0,1,...,9}  {
\draw[very thin,red]
(-0.2,-\y*\baselineskip) --
(\linewidth+0.2cm,-\y*\baselineskip); }
\end{tikzpicture}
line 1\par
\begin{doubleline}  Abc  \end{doubleline}
\begin{doubleline}  Def  \end{doubleline}
line 2g\par
\begin{doubleline}  Ghi  \end{doubleline}
line 3\par
line 4 g
\end{exdispExample*}
\end{docTcbKey}



%%\clearpage

\begin{docTcbKey}[][doc new and updated={2020-09-25}{2015-03-16}]{before skip}{=\meta{glue}}{style, no default}
在盒子之前插入指定高度\meta{glue}的垂直空间。%
此项会设置 \refKeyLe{/tcb/before}。%
同 \refKeyLe{/tcb/before skip balanced} 相比, 这个 \meta{glue} upper 部分的边缘,并不是到基线位置。
\begin{exdispExample*}{before_skip}{sbs,lefthand ratio=0.6}
Some text.
\begin{tcolorbox}[before skip=1cm,
colframe=red!50!white]
This is a \textbf{tcolorbox}.
\end{tcolorbox}

Some text.
\begin{tcolorbox}[before skip=0cm,
colframe=red!50!white]
This is a \textbf{tcolorbox}.
\end{tcolorbox}
\end{exdispExample*}
\end{docTcbKey}

\begin{docTcbKey}[][doc new and updated={2020-09-25}{2017-02-01}]{after skip}{=\meta{glue}}{style, no default}
在盒子之后插入指定高度\meta{glue}的垂直空间。%
此项会设置 \refKeyLe{/tcb/after}.
同 \refKeyLe{/tcb/after skip balanced} 相比, %
这个 \meta{glue} 是相对于 lower 部分的边缘,并不是到基线位置。%\footnote{译注:后面有空想下英文原文是否有误?}
\begin{exdispExample*}{after_skip}{sbs,lefthand ratio=0.6}
\begin{tcolorbox}[after skip=1cm,
colframe=red!50!white]
This is a \textbf{tcolorbox}.
\end{tcolorbox}
Some text.
\end{exdispExample*}
\end{docTcbKey}

\begin{docTcbKey}[][doc new=2014-10-10]{beforeafter skip}{=\meta{glue}}{style, no default}
Inserts some vertical space of the given \meta{glue} before \emph{and} after the colored box.
This style sets \refKeyLe{/tcb/before skip} and \refKeyLe{/tcb/after skip}.

在盒子的前后插入指定高度\meta{glue}的垂直空间。%
此项会设置 \refKeyLe{/tcb/before skip} 和 \refKeyLe{/tcb/after skip}。


\begin{exdispExample*}{beforeafter_skip}{sbs,lefthand ratio=0.6}
\tcbset{beforeafter skip=0pt,
colframe=red!50!white}

text before
\begin{tcolorbox}
This is a \textbf{tcolorbox}.
\end{tcolorbox}
\begin{tcolorbox}
Second box.
\end{tcolorbox}
text after
\end{exdispExample*}
\end{docTcbKey}



%%\clearpage

\begin{docTcbKey}[][doc new=2014-11-07]{left skip}{=\meta{length}}{style, no default, initially |0mm|}
Inserts some horizontal space of the given \meta{length} before the colored box.
This style sets \refKeyLe{/tcb/grow to left by} with the negated \meta{length},
i.e. the bounding box and box width are changed.

在盒子前插入给定 \meta{length} 的水平空间。
此项会设定 \refKeyLe{/tcb/grow to left by} 为 \meta{length} 的相反数,
i.e. 边界框和盒子宽度已更改。
\begin{exdispExample*}{left_skip}{sbs,lefthand ratio=0.6}
\noindent\rule{\linewidth}{2pt}

\begin{tcolorbox}[left skip=1cm,
colframe=red!50!white]
This is a \textbf{tcolorbox}.
\end{tcolorbox}
\end{exdispExample*}
\end{docTcbKey}

\begin{docTcbKey}[][doc new=2014-11-07]{right skip}{=\meta{length}}{style, no default, initially |0mm|}
Inserts some horizontal space of the given \meta{length} after the colored box.
This style sets \refKeyLe{/tcb/grow to right by} with the negated \meta{length},
i.e. the bounding box and box width are changed.

在盒子{\bf 后}插入给定 \meta{length} 的水平空间。%
此项会设定  \refKeyLe{/tcb/grow to right by} 为 \meta{length} 的相反数,
i.e. 边界框和盒子宽度已更改。
\begin{exdispExample*}{right_skip}{sbs,lefthand ratio=0.6}
\noindent\rule{\linewidth}{2pt}

\begin{tcolorbox}[right skip=1cm,
colframe=red!50!white]
This is a \textbf{tcolorbox}.
\end{tcolorbox}
\end{exdispExample*}
\end{docTcbKey}



\begin{docTcbKey}[][doc new=2014-10-10]{leftright skip}{=\meta{length}}{style, no default}
Inserts some horizontal space of the given \meta{length} before \emph{and} after the colored box.
This style changes the bounding box and the box width.

在盒子前后插入给定长度为 \meta{length} 的水平空间。
此样式更改边界盒子和盒子宽度。

\begin{exdispExample*}{leftright_skip}{sbs,lefthand ratio=0.6}
\noindent\rule{\linewidth}{2pt}

\begin{tcolorbox}[leftright skip=1cm,
colframe=red!50!white]
This is a \textbf{tcolorbox}.
\end{tcolorbox}
\end{exdispExample*}
\end{docTcbKey}


%%\clearpage

\begin{docTcbKey}[][doc updated=2017-02-01]{parskip}{}{style, no value}
This options is considered to be superseded by
\refKeyLe{/tcb/before skip balanced} and \refKeyLe{/tcb/after skip balanced}
(see note on page~\pageref{subsec:surroundings}).\par
Sets the keys |before| and |after| to values which are
recommended, if the package |parskip| \emph{is} used and there is no better
idea for |before| and |after|. This is similar to:

此项有考虑使用 \refKeyLe{/tcb/before skip balanced} 和 \refKeyLe{/tcb/after skip balanced} 取代。
(见~\pageref{subsec:surroundings}页).\par
将 |before| 和 |after| 的值设置为推荐的值,如果{使用}了 |parskip| 包且 |before| 和 |after| 的值没有更好的主意。效果类似于:
\begin{dispListing}
\tcbset{parskip/.style={before={\par\pagebreak[0]\parindent=0pt},
                after={\par}}}
\end{dispListing}
\end{docTcbKey}

\begin{docTcbKey}[][doc updated=2017-02-01]{noparskip}{}{style, no value}
This options is considered to be superseded by
\refKeyLe{/tcb/before skip balanced} and \refKeyLe{/tcb/after skip balanced}
(see note on page~\pageref{subsec:surroundings}).\par
Sets the keys |before| and |after| to values which are
recommended, if the package |parskip| is \emph{not} used and there is no better
idea for |before| and |after|. This is similar to:

此项有考虑使用 \refKeyLe{/tcb/before skip balanced} 和 \refKeyLe{/tcb/after skip balanced} 取代。
(见~\pageref{subsec:surroundings}页).\par
将 |before| 和 |after| 的值设置为推荐的值,如果{使用}了 |parskip| 包且 |before| 和 |after| 的值没有更好的主意。效果类似于:
\begin{dispListing}
\tcbset{noparskip/.style={before={\par\pagebreak[0]\smallskip\parindent=0pt},
                    after={\par\smallskip}}}
\end{dispListing}
\end{docTcbKey}



\begin{docTcbKey}{autoparskip}{}{style, no value}
This options is considered to be superseded by
\refKeyLe{/tcb/before skip balanced} and \refKeyLe{/tcb/after skip balanced}
(see note on page~\pageref{subsec:surroundings}).\par
Tries to detect the usage of the package |parskip| and sets
the keys |before| and |after| accordingly. Actually, the following is done:

这项可以考虑改用 \refKeyLe{/tcb/before skip balanced} 和 \refKeyLe{/tcb/after skip balanced} 替换(另见~\pageref{subsec:surroundings}~页)。\par
尝试检测 |parskip| 的使用情况, 并相应的设置 |before| 和 |after|。 实际上,完成了以下操作:

\begin{itemize}
\item 
If the length of |\parskip| is greater than |0pt| at the beginning of the document,
\refKeyLe{/tcb/parskip} is executed. Here, the usage of package |parskip| is \emph{assumed}.

todo
如果在文档的开头 |\parskip| 的值大于|0pt|,%
\refKeyLe{/tcb/parskip} 将会执行。 这里, 假定 |parskip| 引入\footnote{Here, the usage of package |parskip| is \emph{assumed}}。

\item 
Otherwise, if the length of |\parskip| is not greater than |0pt| at the beginning of the document,
\refKeyLe{/tcb/noparskip} is executed. Here, the absence of package |parskip| is \emph{assumed}.
另外,如果在文档的开头 |\parskip| 的值不大于|0pt|,%
\refKeyLe{/tcb/noparskip} 将会执行。 这里, 假定 |parskip| 没有使用\footnote{Here, the absence of package |parskip| is \emph{assumed}}。
\end{itemize}
\end{docTcbKey}




%%\clearpage

\begin{docTcbKey}{baseline}{=\meta{length}}{no default, initially |0pt|}
Used to set the |\pgfsetbaseline| value of the resulting |tcolorbox|.
设置结果盒子的%,同其他 \TeX\ 对象对齐用的
基线的值(|\pgfsetbaseline|)。

\begin{exdispExample}{baseline}
\tcbset{colframe=red!50!white,width=4cm,nobeforeafter}
Some text\dotfill
\begin{tcolorbox}[baseline=3mm]
第一行
\end{tcolorbox}
\begin{tcolorbox}[baseline=3mm]
第一行\\第二行
\end{tcolorbox}
\begin{tcolorbox}[baseline=4mm]
第一行\\第二行\\第三行
\end{tcolorbox}
\end{exdispExample}
\end{docTcbKey}




\begin{docTcbKey}[][doc new=2014-10-10]{box align}{=\meta{alignment}}{style, no default, initially |bottom|}
Used to set the \refKeyLe{/tcb/baseline} value of the resulting |tcolorbox|.
Feasible values for \meta{alignment} are:

Used to set the \refKeyLe{/tcb/baseline} value of the resulting |tcolorbox|.
Feasible values for \meta{alignment} are:  
\begin{itemize}
\item\docValue{bottom}: %alignment with the box bottom,
与盒子底部对齐,
\item\docValue{top}: %alignment with the box top,
与盒子顶部对齐,
\item\docValue{center}: %alignment with the box center,
与盒子中心对齐,
\item\docValue{base}: 
alignment with the box content base. This option
is not applicable for a \refEnvLe{tcolorbox} but for a \refComLe{tcbox} only.
It is an alias for \refKeyLe{/tcb/tcbox raise base}.
与盒子内容的基线对齐。此项不在 \refEnvLe{tcolorbox} 使用,仅在 \refComLe{tcbox} 生效。
这是 \refKeyLe{/tcb/tcbox raise base} 的别名。
\end{itemize}

\begin{exdispExample}{box_align_1}
\tcbset{colframe=red!50!white,width=4cm,nobeforeafter}
Some text\dotfill
\begin{tcolorbox}[box align=bottom]
bottom
\end{tcolorbox}
\begin{tcolorbox}[box align=bottom]
bottom\\bottom
\end{tcolorbox}
\begin{tcolorbox}
第一行\\第二行\\第三行
\end{tcolorbox}
\end{exdispExample}

\begin{exdispExample}{box_align_2}
\tcbset{colframe=red!50!white,width=4cm,nobeforeafter}
Some text\dotfill
\begin{tcolorbox}[box align=top]
top
\end{tcolorbox}
\begin{tcolorbox}[box align=top]
top\\top
\end{tcolorbox}
\begin{tcolorbox}
第一行\\第二行\\第三行
\end{tcolorbox}
\end{exdispExample}

\begin{exdispExample}{box_align_3}
\tcbset{colframe=red!50!white,width=4cm,nobeforeafter}
Some text\dotfill
\begin{tcolorbox}[box align=center]
center
\end{tcolorbox}
\begin{tcolorbox}[box align=center]
center\\center
\end{tcolorbox}
\begin{tcolorbox}
第一行\\第二行\\第三行
\end{tcolorbox}
\end{exdispExample}

\begin{exdispExample}{box_align_4}
\tcbset{colframe=red!50!white,nobeforeafter}
Some text\dotfill
\tcbox[nobeforeafter,box align=base]{base}
\tcbox[nobeforeafter,box align=base,size=fbox]{base}
\tcbox[nobeforeafter]{未设置}
\end{exdispExample}
\end{docTcbKey}





\begin{docTcbKey}[][doc new=2014-12-11]{ignore nobreak}{\colOpt{=true\textbar false}}{default |true|, initially |false|}
After a heading, \LaTeX\ tries to avoid a break by setting a |nobreak| boolean value.
Starting from version |3.33|, the \refKeyLe{/tcb/before} respectively \refKeyLe{/tcb/before skip}
settings are not used after a heading if \refKeyLe{/tcb/ignore nobreak} is
set to \docValue{false}. For an unbreakable box, \refKeyLe{/tcb/before nobreak} is used instead.
Further, a \refKeyLe{/tcb/breakable} box will also try to
avoid a break between a heading and a directly following first part of a
break sequence.

在标题之后, 通过设置 |nobreak| ,\LaTeX\ 会尝试避免分页。
从版本 |3.33| 开始, 如果 \refKeyLe{/tcb/ignore nobreak} 设置为 \docValue{false}, 那么 \refKeyLe{/tcb/before} 和 \refKeyLe{/tcb/before skip}
的设置在标题之后将不生效。%
对于一个不可分的盒子, 将使用 \refKeyLe{/tcb/before nobreak} 替代使用。
将来, 一个设置了 \refKeyLe{/tcb/breakable} 的盒子,将会尝试避免在标题和紧随其后的中断序列的第一部分之间中断。

Set \refKeyLe{/tcb/ignore nobreak} to \docValue{true}, if |nobreak| should be
ignored as prior to version |3.33|. Also, such a setting may be used locally to
enforce the \refKeyLe{/tcb/before} setting.
在版本 |3.33|,如果需要保留这个忽略 |nobreak| 的效果,将 \refKeyLe{/tcb/ignore nobreak} 设置为 \docValue{true}。 此外,这样的设置可以在 locally 使用以强制设置 \refKeyLe{/tcb/before} 。
\end{docTcbKey}

\begin{docTcbKey}[][doc new=2014-12-16]{before nobreak}{=\meta{code}}{no default, initially \cs{noindent}}
Sets the \meta{code} which is executed before the colored box if it
is unbreakable, if \refKeyLe{/tcb/ignore nobreak} is not set, and if
the box follows a heading.

如果是不可以分的,设置 \meta{code} 在盒子之前执行, 如果没有设置 \refKeyLe{/tcb/ignore nobreak} , 或如果盒子是跟随在标题之后。
\end{docTcbKey}



\begin{docTcbKey}[][doc new=2017-02-23]{parfillskip restore}{\colOpt{=true\textbar false}}{default |true|, initially |true|}
If this option is set to be |true|, the minimum value of |\parfillskip| is
tested at specific spots, if it is greater than |0pt|.
If so, |\parfillskip| is restored to |\@flushglue| which happens to be
the default value.

如果此项设置为 |true|, 则在特定点测试 |\parfillskip| 的最小值, 如果它大于 |0pt|.
如果是这样,|\parfillskip| 恢复到 |\@flushglue|, 这恰好是默认值。

These tests are executed for
这些判断将会在以下位置执行:
\refKeyLe{/tcb/parskip},
\refKeyLe{/tcb/noparskip},
\refKeyLe{/tcb/after skip},
\refKeyLe{/tcb/breakable}, and
\refEnvLe{tcbraster}.

This option was created to automatically
avoid overfull box warnings with |\parfillskip| changing packages.

创建此选项是为了自动避免 |\parfillskip| 改变包裹带来的 |overfull box| 警告 。
\end{docTcbKey}