\setcounter{section}{4}
\setcounter{subsection}{8}
\setcounter{subsubsection}{0}
 
\subsection{Transparency\\透明度}

\begin{marker}
Transparency effects are likely to be used in conjunction with \emph{jigsaw}
skin variants, see \Vref{subsec:skinjigsaw}.

透明效果常常同skin的\emph{jigsaw}变量配合使用, 见 \Vref{subsec:skinjigsaw}.
\end{marker}

\begin{docTcbKey}{opacityframe}{=\meta{fraction}}{no default, initially \texttt{1.0}}
Sets the frame opacity of the box to the given \meta{fraction}.

根据给定的 \meta{fraction} 值设置盒子的边框%线
的不透明度。
\begin{exdispExample*}{opacityframe}{sbs,lefthand ratio=0.6,
segmentation style={preaction={fill=white},pattern=checkerboard,pattern color=gray!40}
}
\begin{tcolorbox}[opacityframe=0.25
,title=设置了边框的颜色和透明度,
colframe=red]
设置了边框的颜色为红色,透明度为0.25
\end{tcolorbox}
\begin{tcolorbox}[colframe=red
,title=设置了边框的颜色]
只设置了边框的颜色为红色。
\end{tcolorbox}
\end{exdispExample*}
\end{docTcbKey}
% 这段代码使用了tcolorbox宏包来创建两个带边框的文本框。其中,第一个文本框设置了边框的颜色为红色,透明度为0.25,第二个文本框只设置了边框的颜色为红色。

% 代码中的sbs选项表示将两个文本框并排展示,lefthand ratio=0.6表示左侧文本框占总宽度的60%,右侧文本框占40%。
%segmentation style用于设置两个文本框之间的分割线样式,这里使用了一个棋盘格样式,填充色为灰色。

\begin{docTcbKey}{opacityback}{=\meta{fraction}}{no default, initially \texttt{1.0}}
Sets the background opacity of the box to the given \meta{fraction}.

根据给定的 \meta{fraction} 值设置盒子的背景色的不透明度。
\begin{exdispExample*}{opacityback}{sbs,lefthand ratio=0.6,segmentation style={preaction={fill=white},pattern=checkerboard,pattern color=gray!40}}
\begin{tcolorbox}[standard jigsaw,colframe=red,
opacityframe=0.5, opacityback=0.5]
This is a \textbf{tcolorbox}.
\end{tcolorbox}
\end{exdispExample*}
\end{docTcbKey}

另见 \mylib{skins} 库的 \refKeyLe{/tcb/opacitybacklower}。



\begin{docTcbKey}{opacitybacktitle}{=\meta{fraction}}{no default, initially \texttt{1.0}}
Sets the title background opacity of the box to the given \meta{fraction}.

根据给定的 \meta{fraction} 值设置盒子的标题的背景色的不透明度。
\begin{exdispExample*}{opacitybacktitle}{sbs,lefthand ratio=0.6,segmentation style={preaction={fill=white},pattern=checkerboard,pattern color=gray!40}}
\begin{tcolorbox}[standard jigsaw,colframe=red,
opacityframe=0.5, opacitybacktitle=0.5,
title filled, title=This is a title]
This is a \textbf{tcolorbox}.
\end{tcolorbox}
\end{exdispExample*}
\end{docTcbKey}


\begin{docTcbKey}{opacityfill}{=\meta{fraction}}{style, no default, initially \texttt{1.0}}
Sets the fill opacity for frame, interior and optionally the title background
to the given \meta{fraction}.

根据给定的 \meta{fraction} 值设置盒子的边框、内部和可选的标题背景的填充不透明度。
\begin{exdispExample*}{opacityfill}{sbs,lefthand ratio=0.6,segmentation style={preaction={fill=white},pattern=checkerboard,pattern color=gray!40}}
\begin{tcolorbox}[standard jigsaw,colframe=red,
opacityfill=0.7, title=This is a title]
This is a \textbf{tcolorbox}.
\end{tcolorbox}
\begin{tcolorbox}[standard jigsaw,colframe=red,
title=This is a title]
This is a \textbf{tcolorbox}.
\end{tcolorbox}
\end{exdispExample*}
\end{docTcbKey}





 \clearpage
\begin{docTcbKey}{opacityupper}{=\meta{fraction}}{no default, initially \texttt{1.0}}
Sets the text opacity of the upper box part to the given \meta{fraction}.

根据给定的 \meta{fraction},设置upper部分的文本的不透明度。
\begin{exdispExample*}{opacityupper}{sbs,lefthand ratio=0.6}
\begin{tcolorbox}[enhanced,opacityupper=0.5
,interior style={preaction={fill=white}
,pattern=checkerboard
,pattern color=gray!40}]
This is a \textbf{tcolorbox}.
\end{tcolorbox}
\end{exdispExample*}
\end{docTcbKey}



\begin{docTcbKey}{opacitylower}{=\meta{fraction}}{no default, initially \texttt{1.0}}
Sets the text opacity of the lower box part to the given \meta{fraction}.

根据给定的 \meta{fraction},设置lower部分的文本的不透明度。
\begin{exdispExample*}{opacitylower}{sbs,lefthand ratio=0.6}
\begin{tcolorbox}[enhanced,opacitylower=0.5,
interior style={preaction={fill=white},pattern=checkerboard,pattern color=gray!40}]
This is a \textbf{tcolorbox}.
\tcblower
This is the lower part.
\end{tcolorbox}
\end{exdispExample*}
\end{docTcbKey}

\begin{docTcbKey}{opacitytext}{=\meta{fraction}}{no default, initially \texttt{1.0}}
Sets the text opacity of the upper and the lower box part to the given \meta{fraction}.

根据给定的 \meta{fraction},设置upper和lower半两部分的文本的不透明度。
\begin{exdispExample*}{opacitytext}{sbs,lefthand ratio=0.6}
\begin{tcolorbox}[enhanced,opacitytext=0.5,
interior style={preaction={fill=white},pattern=checkerboard,pattern color=gray!40}]
This is a \textbf{tcolorbox}.
\tcblower
This is the lower part.
\end{tcolorbox}
\end{exdispExample*}
\end{docTcbKey}


\begin{docTcbKey}{opacitytitle}{=\meta{fraction}}{no default, initially \texttt{1.0}}
Sets the text opacity of the box title to the given \meta{fraction}.

根据给定的 \meta{fraction},设置标题的文本的不透明度。
\begin{exdispExample*}{opacitytitle}{sbs,lefthand ratio=0.6}
\begin{tcolorbox}[enhanced,opacitytitle=0.7,
coltitle=black,
fonttitle=\bfseries,title=This is a title,
title style={preaction={fill=white},pattern=checkerboard,pattern color=gray!40}]
This is a \textbf{tcolorbox}.
\end{tcolorbox}
\end{exdispExample*}
\end{docTcbKey}


\begin{exdispExample*}{opacity_general}{segmentation style={preaction={fill=white},pattern=checkerboard,pattern color=gray!40}}
\begin{tcolorbox}[enhanced jigsaw,fonttitle=\bfseries,title=This is a title,
opacityframe=0.5,opacityback=0.25,opacitybacktitle=0.25,opacitytext=0.8,
colback=red!5!white,colframe=red!75!black,colbacktitle=yellow!20!red]
This is a \textbf{tcolorbox}.
\end{tcolorbox}
\end{exdispExample*}