
\begin{docTcbKey}{space}{=\meta{fraction}}{no default, initially 0}
If the height of a |tcolorbox| is not the natural height, the space
difference between the forced and the natural size is distributed
between the upper and the lower part of the box. This space could also
be negative.
\meta{fraction} with a value between 0 and 1 is the amount of space
which is added to the upper part, the rest is added to the lower part.
If there is no lower part, then all of the space is added to
the upper part always.

如果 |tcolorbox| 的高度不是自然高度,强制大小和自然大小之间的空间差异分配在盒子的上部和下部之间。这个空间也可能是负数。 取值介于 0 和 1 的 \meta{fraction} 是添加到上部的空间量比例,其余部分添加到下部。如果没有下部,则所有空间始终添加到上部。

% 如果 |tcolorbox| 的高度不是自然高度, 指定的高度和自然尺寸的的高度差分布在盒子的上下两部分中。 高度差可以是负值。
% \meta{fraction} 是0到1之前的数值,此值指定的比例添加到upper部分,剩余的高度添加到lower部分。
% 如果不存在lower部分, 那么所有的空间将添加到upper部分。
\begin{exdispExample}{fraction}
\tcbset{width=(\linewidth-2mm)/4,before=,after=\hfill,
colframe=blue!75!black,colback=white,height=3cm}

\begin{tcolorbox}
upper部分
\tcblower
lower部分
\end{tcolorbox}
\foreach \f in {0.2,0.4,0.7}
{\begin{tcolorbox}[space=\f]
\verb|space=|{\tt\f} upper部分
\tcblower
lower部分
\end{tcolorbox}}
\end{exdispExample}
\end{docTcbKey}


\begin{docTcbKey}{space to upper}{}{style}
This is an abbreviation for |space=1|, i.\,e. all extra space is added
to the upper part.

这是指定|space=1|的简写形式, i.\,e. 所有额外的空间都被添加到上部。(此法可读性更好)。
\end{docTcbKey}

\begin{docTcbKey}{space to lower}{}{style, initially set}
This is an abbreviation for |space=0|, i.\,e. all extra space is added
to the lower part (if there is any).

这是指定|space=0|的简写形式, i.\,e. 所有额外的空间都被添加到下部。(此法可读性更好)。
\end{docTcbKey}




 %\clearpage
\begin{docTcbKey}{space to both}{}{style}
This is an abbreviation for |space=0.5|, i.\,e. the extra space
equally distributed between the upper and the lower part.
这是指定|space=0.5|的简写形式, i.\,e. 额外的空间将平均分布到upper部分和lower部分。
\begin{exdispExample}{space_to_both}
\tcbset{width=(\linewidth-2mm)/3,before=,after=\hfill,
colframe=blue!75!black,colback=white,height=3cm}

\foreach \myspace in {space to upper,space to both,space to lower}
{\begin{tcolorbox}[\myspace]
\tt \myspace
\tcblower
This is the lower part.
\end{tcolorbox}}
\end{exdispExample}
\end{docTcbKey}

% \begin{tcolorbox}[space to lower,upperbox=invisible]
% hideThis
% \tcblower
% This is the lower part.
% \end{tcolorbox}

\begin{docTcbKey}[][doc new and updated={2015-02-15}{2020-07-30}]{space to}{=\meta{macro}}{no default, initially unset}
If the height of a |tcolorbox| is not the natural height, the space
difference between the forced and the natural size is saved into the
given local \meta{macro}. This \meta{macro} can and should be used inside
the box content to add content which is vertically sized to match \meta{macro}.

如果|tcolorbox|盒子的高度不是自然高度, 指定的高度同自然高度差的数值保存到给出的宏命令 \meta{macro}。 这个 \meta{macro} 可以在盒子中使用以用来控制内容的高度恰好同这高度差一致。
\begin{marker}
\begin{itemize}
\item 
The actual length saved into \meta{macro} is adapted dynamically
during several compilations -- at least two, but maybe more.
实际保存到 \meta{macro} 的值在多次编译期间是自适应 --- 至少2次, 可能更多次。
\item %
Due to the adaption algorithm, objects can be sized with
\meta{macro} plus any offset length.
根据自适应算法, 对象尺寸可能在 \meta{macro} 之上添加额外的偏移量。
\item 
Never ever use \meta{macro} multiplied with a factor. The only
exception to this rule is that the space can be split into parts which
sum to \meta{macro}.
永远不要使用 \meta{macro} 乘以一个因子。这个规则的唯一例外是,
分开的几个部分的高度和为\meta{macro}(即多个因子的和为1)。
\item %Never use this in combination with \refKeyLe{/tcb/fit}.
不要同 \refKeyLe{/tcb/fit} 组合使用。
\end{itemize}
\end{marker}
\begin{exdispExample}[runs=3]{space_to_1}
\begin{tcolorbox}[colframe=blue!75!black,colback=white,height=3cm,
space to=\myspace]
这是我的盒子高3cm。指定高度和自然高度差填充了图片    :\\[2mm]
\includegraphics[width=\linewidth,height=\myspace]{goldshade.png}\\[1mm]
这是其他一些文字。译注:图片的高度使用我们指定的 |\myspace|。
\end{tcolorbox}
\end{exdispExample}

\begin{exdispExample}[runs=3]{space_to_2}
\begin{tcolorbox}[colframe=blue!75!black,colback=white,height=3cm,
space to=\myspace]
\includegraphics[width=\linewidth,
height=0.33\dimexpr\myspace]{blueshade.png}\\[1mm]
这是我的盒子高3cm。\\[2mm]
\includegraphics[width=\linewidth,
height=0.67\dimexpr\myspace]{goldshade.png}
\end{tcolorbox}
\end{exdispExample}
\end{docTcbKey}