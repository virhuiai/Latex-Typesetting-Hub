
\begin{docTcbKey}{split}{=\meta{fraction}}{no default}
If the height of a |tcolorbox| is not the natural height, the
\meta{fraction} with a value between 0 and 1 determines the positioning
of the segmentation between the upper and the lower part. Here, 0 stands
for top and 1 for bottom. Note that the box is split regardless of
the actual dimensions of the text parts!

如果 |tcolorbox| 的高度不是自然高度, 取值0到1的
\meta{fraction} 决定了上下两部分的分割位置。在这里,0代表顶部,1代表底部。
注意,不论文本部分的实际尺寸如何,盒子都会被分割!
\begin{exdispExample}{split}
\tcbset{width=(\linewidth-2mm)/3,before=,after=\hfill,height=3cm,
colback=white,colframe=blue!75!black,valign=center,valign lower=center}

\foreach \f in {0,0.1,0.5,0.8,0.9,1}
{\begin{tcolorbox}[split=\f]
上,split: \f
\tcblower
This is the lower part with a lot of text in several lines.
\end{tcolorbox}}
\end{exdispExample}
\begin{exdispExample}{split2}
\tcbset{width=(\linewidth-2mm)/3,before=,after=\hfill,height=1.8cm,
colback=white,colframe=blue!75!black,valign=center,valign lower=center}

\foreach \f in {0,0.1,0.5,0.8,0.9,1}
{\begin{tcolorbox}[split=\f]
上,split: \f
\tcblower
This is the lower part with a lot of text in several lines.
\end{tcolorbox}}
\end{exdispExample}
\end{docTcbKey}