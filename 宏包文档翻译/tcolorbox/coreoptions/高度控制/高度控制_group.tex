

 %\clearpage
\begin{docTcbKey}[][doc updated=2014-11-07]{equal height group}{=\meta{id}}{no default}
Boxes which are members of an |equal height group| will all get the
same height, i.\,e. the maximum of all their natural heights. The
\meta{id} serves to distinguish between different height groups.
This \meta{id} should contain only characters which are feasible
for \TeX\ macro names, typically alphabetic characters but no numerals
and spaces.
Note that you have to compile twice to see changes and
that height groups are global definitions.

属于“等高组”的盒子将获得相同的高度,即它们所有自然高度的最大值。这里的 \meta{id} 用于区分不同的高度组。这个 \meta{id} 应该只包含可行的 \TeX\ 宏名称字符,通常是字母字符,但不包括数字和空格。请注意,您需要编译两次才能看到更改,而高度组是全局定义。

% 同一|equal height group|的成员将拥有相同的高度, i.\,e. 它们自然高度的最大者的值。
% \meta{id} 用来区分不同的身高组别。
% This \meta{id} should contain only characters which are feasible
% for \TeX\ macro names, typically alphabetic characters but no numerals
% and spaces.
% 注意,您必须编译两次才能看到更改,并且高度组是全局定义。

\begin{exdispExample}[runs=2]{equal_height_group}
\tcbset{width=(\linewidth-2mm)/3,before=,after=\hfill,arc=0mm,
colframe=blue!75!black,colback=white,fonttitle=\bfseries}

\begin{tcolorbox}[equal height group=A,adjusted title={一}]
这组最小的盒子
\end{tcolorbox}%
\begin{tcolorbox}[equal height group=A,adjusted title={二}]
这个盒子也小
\tcblower
但有lower部分。
\end{tcolorbox}%
\begin{tcolorbox}[equal height group=A,adjusted title={三}]
This box contains a lot of text just to fill the space
with word flowing and flowing and flowing until the box
is filled with all of it.
\end{tcolorbox}\linebreak

\tcbset{width=(\linewidth-1mm)/2,before=,after=\hfill,arc=0mm,
colframe=red!75!black,colback=white}

\begin{tcolorbox}[equal height group=B]
接着,我们使用另一个等高盒子组。
\end{tcolorbox}%
\begin{tcolorbox}[equal height group=B,after=]
\begin{equation*}
\int\limits_{0}^{1} x^2 = \frac13.
\end{equation*}
\end{tcolorbox}
\end{exdispExample}
\end{docTcbKey}

\medskip
\begin{marker}
See \Vref{sec:raster} for more equal height options.
另见 \Vref{sec:raster} 了解更多等高组相关选项。
\end{marker}



 %\clearpage
\begin{docTcbKey}{minimum for equal height group}{=\meta{id}:\meta{length}}{no default, initially unset}
Plants a \meta{length} into the equal height group with
the given \meta{id}. This ensures that the height will not drop below
\meta{length}. 
Note that you cannot reduce a computed height value by using this key with a small value.
The difference to applying \refKeyLe{/tcb/height} directly is that the boxes
are never too small for their content.

指定值 \meta{length} 到等高组 \meta{id}。这确保高度不会小于 \meta{length}。
指定等高组 \meta{id} 的最小高度不小于 \meta{length}。%
注意,不能通过使用小值来减少计算出的高度值。%
同使用 \refKeyLe{/tcb/height} 相比,此项设置不会使盒子小于它们的内容高度。

\begin{dispExample}
\tcbset{colframe=blue!75!black,colback=white,arc=0mm,
before=,after=\hfill,fonttitle=\bfseries,left=2mm,right=2mm,
width=3.5cm,
equal height group=C,
minimum for equal height group=C:3.5cm}

\begin{tcolorbox}
My first box. All boxes will get 3.5cm times 3.5cm
if the content height is not too large.
\end{tcolorbox}%
\begin{tcolorbox}
My second box.
\tcblower
This is the lower part.
\end{tcolorbox}%
\begin{tcblisting}{}
\textbf{Mixed}
with a listing.
\end{tcblisting}
\begin{tcolorbox}[title={Fourth box}]
My final box.
\end{tcolorbox}%
\end{dispExample}
\end{docTcbKey}

% todo 再看看
\begin{docTcbKey}[][doc new=2016-03-24]{minimum for current equal height group}{=\meta{length}}{no default, initially unset}
Sets \refKeyLe{/tcb/minimum for equal height group} for the current equal height
group. Apparently, this only works for an already known equal height group, i.e.
\refKeyLe{/tcb/equal height group} has to be set \emph{before} this option is used.
This option is likely to be used in combination with \refKeyLe{/tcb/raster equal height}

为当前等高组设置 \refKeyLe{/tcb/minimum for equal height group}。
显然, 这只适用于已知的等高组, i.e.
\refKeyLe{/tcb/equal height group}已经在此设置\emph{之前}设置。
此项学与 \refKeyLe{/tcb/raster equal height} 组合使用。
\begin{exdispExample}[runs=2]{minimum_for_current_equal_height_group}
%\tcbuselibrary{raster}
\begin{tcbitemize}[raster equal height,colframe=blue!75!black,colback=white,
raster every box/.style={minimum for current equal height group=2cm}]
\tcbitem A
\tcbitem B
\end{tcbitemize}
\end{exdispExample}

\end{docTcbKey}




 %\clearpage
\begin{docTcbKey}[][doc new and updated={2015-11-27}{2016-02-22}]{use height from group}{\colOpt{=\meta{id}}}{style, default current group}
Sets the current box to a fixed \refKeyLe{/tcb/height} which is copied from
an equal height group with the given \meta{id}. If this height is not
available during the current compilation, no fixed height setting is used.
If \meta{id} is omitted, the current equal height group is used which has
to be set before by \refKeyLe{/tcb/equal height group}.\par
Note that the natural height of the current box is not considered for
computation of the group height. The main application for
\refKeyLe{/tcb/use height from group} is that the height can be adapted
further by \refKeyLe{/tcb/add to height}.

设置当前盒子的高度为一个固定 \refKeyLe{/tcb/height} 值,值来自一个等高组\meta{id}。如果在当前编译期间此高度不可用,则不使用固定高度设置。
如果省略了\meta{id}, 则使用此前的\refKeyLe{/tcb/equal height group}设置的。\par
请注意,在计算组高度时不考虑当前盒子的自然高度。
\refKeyLe{/tcb/use height from group}主要用在当高度可以通过 \refKeyLe{/tcb/add to height} 进一步调整。

\begin{dispExample}
\begin{tcolorbox}[use height from group=C,add to height=-2cm,
colframe=blue!75!black,colback=white]
Height from group \enquote{C} of the previous example, but reduced by 2cm.
\end{tcolorbox}%
\end{dispExample}

\begin{exdispExample}[runs=2]{use_height_from_group}
%\tcbuselibrary{raster}
Every line is inside an equal height group:
\begin{tcbraster}[raster equal height=rows,
title=Box \thetcbrasternum,
enhanced,size=small,colframe=red!50!black,colback=red!10!white]
\begin{tcolorbox}First line\\second line\\
The height of this box rules.\end{tcolorbox}
\begin{tcolorbox}[use height from group]Test\end{tcolorbox}
\begin{tcolorbox}[use height from group]
First line\\second line\end{tcolorbox}
\begin{tcolorbox}The height of this box rules.\end{tcolorbox}
\end{tcbraster}
\end{exdispExample}
\end{docTcbKey}



\begin{docCommand}[doc new=2015-11-27]{tcbheightfromgroup}{\marg{macro}\marg{id}}
Saves the height from an equal height group with the given \meta{id}
to a \meta{macro}. If this height is not available during the current compilation,
\meta{macro} is set to |0pt|.

保存等高组 \meta{id} 的高度到 \meta{macro}。如果在当前编译期间此高度不可用,
\meta{macro} 设为|0pt|.
\end{docCommand}