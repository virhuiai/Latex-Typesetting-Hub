\PassOptionsToPackage{no-math}{fontspec}%禁用了使用fontspec宏包中的数学字体功能。
\PassOptionsToPackage{AutoFakeBold=true,AutoFakeSlant=true}{xeCJK}%让xeCJK宏包自动产生伪粗体和伪斜体效果。

\documentclass{book}
\usepackage[heading=true
,scheme=chinese%中文方案
,fontset=none%不使用默认的字体设置
,space=auto%自动调整中英文间距
]{ctex}
\setCJKmainfont{FangZhengShuSong-GBK-1.ttf}[Path=/Users/virhuiai/hlProjects/Latex-Typesetting-Hub/font/方正/]%设置文本的中文有衬线字体
\setCJKsansfont{FangZhengHeiTi-GBK-1.ttf}[Path=/Users/virhuiai/hlProjects/Latex-Typesetting-Hub/font/方正/]%设置文本的中文无衬线字体为
\setCJKmonofont{FangZhengFangSong-GBK-1.ttf}[Path=/Users/virhuiai/hlProjects/Latex-Typesetting-Hub/font/方正/] %设置文本的中文等宽字体 

\usepackage[all]{tcolorbox}
\begin{document}



\makeatletter
\providecommand\jobnameToRun{\filename@area\filename@base}
\makeatletter
\definecolor{Blue_Dark}{rgb}{0.090196,0.211765,0.364706}
\definecolor{Blue_Bright}{rgb}{0.858824,0.898039,0.945098}
\definecolor{Blue_Title}{rgb}{0.117647,0.211765,0.352941}%RGB值为30, 54, 90对应的比例值。请注意,RGB值在0到1之间进行归一化,因此需要将原始RGB值除以255以获得比例值。
\tcbset{%
csh xelatexpdfcrop example/.style={%
    skin=bicolor,
    colframe=Blue_Dark,
    colback=Blue_Bright,
    colbacklower=white,
    colbacktitle=Blue_Title,
    fonttitle=\tt,
    arc is angular,arc=1mm,
    drop fuzzy shadow,
    % title={来自virhuiai的例子:run system command里调用了pdfcrop},
    listing side comment,
    compilable listing,
    run xelatex={-shell-escape},
    % run system command={%
    % pdfcrop\space\jobnameToRun.pdf\space\jobnameToRun.pdf},
    % pdf comment,
    % freeze pdf,
    comment style={opacityframe=0}%透明
} 
}

\begin{tcblisting}{
    csh xelatexpdfcrop example,before upper={\%hi},
    pdf comment={\jobnameToRun.pdf},
    comment style={raster columns=3,graphics pages={1,2,3}},
    sidebyside=false
}
\documentclass{minimal}
\begin{document}
\thispagestyle{empty}
test2\TeX

\newpage
1


\newpage
2

\end{document}  
\end{tcblisting}



\end{document}
