% \clearpage
{\let\xdocEnvironment\docEnvironment
\let\endxdocEnvironment\enddocEnvironment
\begin{xdocEnvironment}[doclang/environment content=environment description,doc updated=2020-04-22]
  {docEnvironment}{\oarg{options}\marg{name}\marg{parameters}}
Documents a \LaTeX\ environment with given \meta{name}.
The given \meta{options} are set with \refComLe{tcbset}.
This environment takes mandatory or optional \meta{parameters}.
It is automatically indexed and can be referenced with
\refComLe{refEnv}\marg{name}.

记录带有给定名称 \meta{name} 的 \LaTeX\ 环境。 给定的 \meta{options} 通过 |tcbset| 进行设置。 该环境可以带有强制或可选的 \meta{parameters}。 它会自动编制索引,并可以通过 |\refEnv|\marg{name} 进行引用。

\begin{dispExample}
\begin{docEnvironment}{foocolorbox}{\oarg{options}}
This is the main environment to create an accentuated colored text box with
rounded corners and, optionally, two parts.

这是创建带有突出颜色的圆角文本框的主要环境,还可以选择性地分为两个部分。
\end{docEnvironment}
\end{dispExample}
\begin{dispExample}
\begin{docEnvironment}%
  [doclang/environment content=My content text]%
  {foocolorbox*}{\oarg{options}}
This is the main environment to create an accentuated colored text box with
rounded corners and, optionally, two parts.

这是创建一个强调着色的文本框的主要环境,它具有圆角,并可选地分为两个部分。
\end{docEnvironment}
\end{dispExample}
\end{xdocEnvironment}}

{\let\xdocEnvironment\docEnvironment
\let\endxdocEnvironment\enddocEnvironment
\begin{xdocEnvironment}[doclang/environment content=environment description,doc updated=2020-04-22]
  {docEnvironment*}{\oarg{options}\marg{name}\marg{parameters}}
Identical to \refEnvLe{docEnvironment}, but without index entry.

与\refEnvLe{docEnvironment}相同,但没有索引条目。
\end{xdocEnvironment}}

%% \clearpage
{\let\xdocEnvironment\docEnvironment
\let\endxdocEnvironment\enddocEnvironment
\begin{xdocEnvironment}[doclang/environment content=environment description,doc new=2020-04-22]
  {docEnvironments}{\oarg{options}\brackets{\marg{variant1},\marg{variant2},...}}
Documents several (similar) \LaTeX\ environment variants simultaneously.
The given \meta{options} are set with \refComLe{tcbset} and are valid for
all variants and the documentation text.
Every variant is described by an option set \meta{variant1}, \meta{variant2}, and so on.
The most crucial options are \refKeyLe{/tcb/doc name} and \refKeyLe{/tcb/doc parameter}.

同时记录了几个(相似的)\LaTeX\ 环境变体。 给定的 \meta{options} 是通过 |\tcbset| 设置的,对于所有变体和文档文本都有效。 每个变体都由选项集 \meta{variant1}、\meta{variant2} 等描述。 最关键的选项是 |/tcb/doc name| 和 |/tcb/doc parameter|。
\begin{dispExample}
\begin{docEnvironments}[
  doc no index,   %  no index entries for this example
  doc parameter = \oarg{options}\marg{title},
  doclang/environment content = box content,
]
{
  {
    doc name        = redbox,
    doc description = a red colored box,
  },
  {
    doc name        = greenbox,
    doc description = a green colored box,
  },
  {
    doc name        = bluebox,
    doc description = a blue colored box,
  },
  {
    doc name        = custombox,
    doc parameter   = \oarg{options}\marg{color}\marg{title},
    doc description = a colored box,
  },
}
example
\end{docEnvironments}
\end{dispExample}
\end{xdocEnvironment}}