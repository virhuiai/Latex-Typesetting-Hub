\begin{docTcbKey}[][doc new={2020-04-22}]{doc name}{=\meta{name}}{no default, initially empty}
Sets the \meta{name} of the entry to document, i.e. the \meta{name} of the
command, environment, key, etc. For \refEnv{docCommand}, \refEnv{docEnvironment}, etc.
the \meta{name} is set by a mandatory parameter, but can also be set
by \refKey{/tcb/doc name}.
\refKey{/tcb/doc name} also sets \meta{name} to
\refKey{/tcb/doc label}, \refKey{/tcb/doc index},
and \refKey{/tcb/doc sort index}.

设置要记录的条目的\meta{name},即命令、环境、键等的\meta{name}。对于\refEnv{docCommand}、\refEnv{docEnvironment}等,\meta{name}由必需参数设置,但也可以通过\refKey{/tcb/doc name}设置。

\refKey{/tcb/doc name}还将\meta{name}设置为\refKey{/tcb/doc label}、\refKey{/tcb/doc index}和\refKey{/tcb/doc sort index}。
\begin{dispExample}
\begin{docCommands}[
doc no index,  %  no index entries for this example
doc name      = bfseries,
] {}
Font setting to bold face.
\end{docCommands}
\end{dispExample}
\end{docTcbKey}


\begin{docTcbKey}[][doc new={2020-04-22}]{doc parameter}{=\meta{parameters}}{no default, initially empty}
Sets the \meta{parameters} of the entry to document, i.e. the \meta{parameters} of the
command, environment, key, etc. For \refEnv{docCommand}, \refEnv{docEnvironment}, etc.
the \meta{parameters} is set by a mandatory option, but can also be set
by \refKey{/tcb/doc parameter}.

设置文档条目的 \meta{参数},即命令、环境、键等的 \meta{参数}。对于 \refEnv{docCommand}、\refEnv{docEnvironment} 等,\meta{参数}由必选选项设置,但也可以通过 \refKey{/tcb/doc parameter} 设置。
\begin{dispExample}
\begin{docCommands}[
doc no index,  %  no index entries for this example
doc name      = textbf,
doc parameter = \marg{text},
] {}
Sets \meta{text} in bold face.
\end{docCommands}
\end{dispExample}
\end{docTcbKey}



\begin{docTcbKey}[][doc new={2020-04-22}]{doc keypath}{=\meta{key path}}{no default, initially empty}
Sets the \meta{key path} of the key to document. For \refEnv{docKey}
and \refEnv{docKey*} the \meta{key path}  is set by a specialized option,
but can also be set by \refKey{/tcb/doc keypath}.

将键的\meta{key path}设置为文档中的值。对于\refEnv{docKey}和\refEnv{docKey*},\meta{key path}是通过专门的选项设置的,但也可以通过\refKey{/tcb/doc keypath}设置。
\begin{dispExample}
\begin{docKeys}[
doc no index,  %  no index entries for this example
doc keypath     = tikz,
doc name        = fill,
doc parameter   = \colOpt{=\meta{color}},
doc description = default is scope's color setting,
] {}
This option causes the path to be filled.
\end{docKeys}
\end{dispExample}
\end{docTcbKey}

% \clearpage

\begin{docTcbKey}{doc description}{=\meta{description}}{no default, initially empty}
Sets a (short!) additional \meta{description} for
\refEnv{docCommand}, \refEnv{docEnvironment}, or \refEnv{docPathOperation}.
Such a description is
mandatory for \refEnv{docKey}.

为\refEnv{docCommand}、\refEnv{docEnvironment}或\refEnv{docPathOperation}设置一个(简短的!)附加的\meta{description}。对于\refEnv{docKey},这样的描述是必需的。
\begin{dispExample}
\begin{docCommand*}[doc description=my description]{myCommandF}{\marg{argument}}
This is the documentation of \refCom{myCommandF} which takes one \meta{argument}.
\refCom{myCommandF} does some funny things with its \meta{argument}.

这是 \refCom{myCommandF} 的文档,它接受一个 \meta{argument}。 \refCom{myCommandF} 会对其 \meta{argument} 进行一些有趣的操作。
\end{docCommand*}
\end{dispExample}
\begin{marker}
Note that the description \meta{text} may overlap with the text on the left
hand side if too long. Linebreaks can be used inside the \meta{text}.

请注意,如果\meta{text}太长,描述文本可能会与左侧文本重叠。可以在\meta{text}中使用换行符。
\end{marker}
\end{docTcbKey}


\begin{docTcbKey}[][doc new={2019-09-18}]{doc label}{=\meta{text}}{no default, initially unset}
If used inside the option list of \refEnv{docCommand}, \refEnv{docEnvironment},
\refEnv{docKey}, etc, then \meta{text} is used
for labeling instead of the name of the definition.

如果在 \refEnv{docCommand}、\refEnv{docEnvironment}、\refEnv{docKey} 等选项列表中使用,那么 \meta{text} 被用于标记,而不是定义的名称。
\begin{dispExample}
\begin{docPathOperation*}[doc label=pathline]{-{}-}{\meta{coordinate or cycle}}
This is the documentation of \refPathOperation{pathline}.

这是 \refPathOperation{pathline} 的文档。
\end{docPathOperation*}
\end{dispExample}
\end{docTcbKey}

\begin{docTcbKey}[][doc new={2020-01-07}]{doc index}{=\meta{text}}{no default, initially unset}
If used inside the option list of \refEnv{docCommand}, \refEnv{docEnvironment},
\refEnv{docKey}, etc, then \meta{text} is used
for the index instead of the name of the definition.

如果在\refEnv{docCommand}、\refEnv{docEnvironment}、\refEnv{docKey}等选项列表中使用,那么\meta{text}将用于索引而不是定义的名称。
\begin{dispExample}
\begin{docPathOperation}[doc index=foo path (horizontal then vertical),
doc label=pathline2]{-\textbar}{\meta{coordinate or cycle}}
This is the documentation of \refPathOperation{pathline2}.

这是\refPathOperation{pathline2}的文档。
\end{docPathOperation}
\end{dispExample}
\end{docTcbKey}


\begin{docTcbKey}[][doc new={2020-04-23}]{doc sort index}{=\meta{text}}{no default, initially unset}
If used inside the option list of \refEnv{docCommand}, \refEnv{docEnvironment},
\refEnv{docKey}, etc, then \meta{text} is used
for as sort key for the index instead of the name of the definition.

如果在\refEnv{docCommand}、\refEnv{docEnvironment}、\refEnv{docKey}等选项列表中使用,那么\meta{text}将用作索引的排序关键字,而不是定义名称。
\begin{dispListing}
\begin{docCommands}[
doc name        = l_tcobox_example_tl,
doc sort index  = example_tl,  % sorted unter e like example
]{}
\end{docCommands}
\end{dispListing}
\end{docTcbKey}

% \clearpage

\begin{docTcbKey}{doc into index}{\colOpt{=true\textbar false}}{default |true|, initially |true|}
If set to |false|, no index entries are written for the main documentation
environments. The same effect is achieved by using e.\,g.\ \refEnv{docCommand*}
instead of \refEnv{docCommand}.

如果设置为|false|,则不会为主要文档环境编写索引条目。使用例如\refEnv{docCommand*}而不是\refEnv{docCommand}可以达到相同的效果。
\end{docTcbKey}


\begin{docTcbKey}[][doc new={2020-04-22}]{doc no index}{}{style, initially unset}
If set, no index entries are written for the main documentation
environments. This is a shortcut for using \refKey{/tcb/doc into index}|=false|.

如果设置了此选项,则不会为主要文档环境编写索引条目。这是使用\refKey{/tcb/doc into index}|=false|的快捷方式。
\end{docTcbKey}

\begin{docTcbKey}[][doc new=2014-09-19]{doc marginnote}{=\meta{options}}{no default, initially empty}
Sets style \meta{options} for the displayed box of the \refCom{tcbdocmarginnote} command.

为\refCom{tcbdocmarginnote}命令的显示框设置样式选项\meta{options}。
\begin{dispExample}
\tcbset{doc marginnote={colframe=blue!50!white,colback=blue!5!white}}%
This is some text\tcbdocmarginnote{Note A}
which is commented by a note inside the margin.

这是一些文本\tcbdocmarginnote{注释A}, 其中注释在边缘内部。
\end{dispExample}
\end{docTcbKey}

\begin{docTcbKey}[][doc new=2014-09-19]{doc new}{=\meta{date}}{style, no default}
Adds a a marginnote with a \enquote{New: \meta{date}} message at the beginning of
the upper box part. The intended use is inside the option list of
\refEnv{docCommand}, \refEnv{docEnvironment}, etc.

在上方盒子的开头添加一个带有 \enquote{New: \meta{date}} 信息的边注。旨在用于 \refEnv{docCommand}、\refEnv{docEnvironment} 等选项列表内。 
\makeatletter\renewcommand*{\tcbdocnew}[1]{\kvtcb@text@new: #1}\makeatother%
\begin{dispExample}
\begin{docCommand}[doc new=2000-01-01]{foosomething}{\marg{text}}
Some command for something.

一些用于某事的命令。
\end{docCommand}
\end{dispExample}
\end{docTcbKey}


\begin{docTcbKey}[][doc new=2014-09-19]{doc updated}{=\meta{date}}{style, no default}
Adds a marginnote with a \enquote{Updated: \meta{date}} message at the beginning of
the upper box part. See \refKey{/tcb/doc new}.

在上方盒子部分的开头添加一个带有\enquote{更新于:\meta{日期}}消息的边注。请参见\refKey{/tcb/doc new}。
\end{docTcbKey}


\begin{docTcbKey}[][doc new=2014-09-19]{doc new and updated}{=\marg{new date}\marg{update date}}{style, no default}
Adds a marginnote with \enquote{New: \meta{new date}} and \enquote{Updated: \meta{update date}} messages at the beginning of
the upper box part. See \refKey{/tcb/doc new}.

在上部框的开始处添加带有\enquote{New: \meta{new date}}和\enquote{Updated: \meta{update date}}消息的marginnote。请参见\refKey{/tcb/doc new}。
\end{docTcbKey}

