\begin{docTcbKey}[][doc updated=2015-03-16]{docexample}{}{style, no value}
Sets the style for \refEnvLe{dispExample} and \refEnvLe{dispListing}
with the colors |ExampleBack| and |ExampleFrame|.
To change the appearance of the examples, this style can be
redefined.

使用颜色 |ExampleBack| 和 |ExampleFrame| 设置 \refEnvLe{dispExample} 和 \refEnvLe{dispListing} 的样式。要更改示例的外观,可以重新定义此样式。
\begin{dispListing}
% Predefined style:
\tcbset{
docexample/.style={colframe=ExampleFrame,colback=ExampleBack,
before skip=\medskipamount,after skip=\medskipamount,
fontlower=\footnotesize}
}
\end{dispListing}
\end{docTcbKey}

\begin{docTcbKey}{documentation listing options}{=\meta{key list}}{no default,\\\hspace*{\fill} initially |style=tcbdocumentation|}
Sets the options from the package |listings| \cite{hoffmann:listings}.
They are used inside \refEnvLe{dispExample} and \refEnvLe{dispListing} to typeset
the listings. Note that this is not identical to the key
\refKeyLe{/tcb/listing options} which is used for \enquote{normal} listings.\\
Used for \refKeyLe{/tcb/listing engine}|=listings| only.

从包|listings|\cite{hoffmann:listings}中设置选项。它们用于在\refEnvLe{dispExample}和\refEnvLe{dispListing}中排版代码清单。请注意,这与用于“常规”清单的键\refKeyLe{/tcb/listing options}不完全相同。仅用于\refKeyLe{/tcb/listing engine}|=listings|。
\end{docTcbKey}

\begin{docTcbKey}{documentation listing style}{=\meta{listing style}}{no default, initially |tcbdocumentation|}
Abbreviation for |documentation listing options={style=...}|.
This key sets a \meta{style}
for the |listings| package, see \cite{hoffmann:listings}.
Note that this is not identical to the key
\refKeyLe{/tcb/listing style} which is used for \enquote{normal} listings.\\
Used for \refKeyLe{/tcb/listing engine}|=listings| only.

缩写为|documentation listing options={style=...}|。 此键设置|listings|包的\meta{style},参见\cite{hoffmann:listings}。 请注意,这与用于“正常”列表的键\refKeyLe{/tcb/listing style}不完全相同。\ 仅用于\refKeyLe{/tcb/listing engine}|=listings|。
\end{docTcbKey}

\begin{docTcbKey}{documentation minted options}{=\meta{key list}}{no default,\\\hspace*{\fill} initially |tabsize=2,fontsize=\textbackslash small|}
Sets the options from the package |minted| \cite{poore:minted}
which are used during typesetting of the listing, if used.
Note that this is not identical to the key
\refKeyLe{/tcb/minted options} which is used for \enquote{normal} listings.\\
Used for \refKeyLe{/tcb/listing engine}|=minted| only.

如果使用,从包 |minted| \cite{poore:minted} 设置用于清单排版的选项。请注意,这与用于“常规”清单的键 \refKeyLe{/tcb/minted options} 不完全相同。仅用于 \refKeyLe{/tcb/listing engine}|=minted|。
\end{docTcbKey}

\begin{docTcbKey}{documentation minted style}{=\meta{key list}}{no default, initially unset}
Sets a \meta{style} known to |Pygments| \cite{pygments:web} for
the package |minted| \cite{poore:minted}, if used.
Note that this is not identical to the key
\refKeyLe{/tcb/minted style} which is used for \enquote{normal} listings.\\
Used for \refKeyLe{/tcb/listing engine}|=minted| only.

如果使用了 |minted| \cite{poore:minted} 包,则为其设置在 |Pygments| \cite{pygments:web} 中已知的 \meta{style}。请注意,这与用于“常规”列表的关键字 \refKeyLe{/tcb/minted style} 不同。仅用于 \refKeyLe{/tcb/listing engine}|=minted|。
\end{docTcbKey}

\begin{docTcbKey}[][doc new=2017-04-24]{documentation minted language}{=\meta{programming language}}{no default, initially |latex|}
Sets a \meta{programming language} known to |Pygments| \cite{pygments:web}
for the package |minted| \cite{poore:minted}, if used.
Note that this is not identical to the key
\refKeyLe{/tcb/minted language} which is used for \enquote{normal} listings.\\
Used for \refKeyLe{/tcb/listing engine}|=minted| only.

如果使用了包|minted|\cite{poore:minted},则设置已知于|Pygments|\cite{pygments:web}的\meta{编程语言}。 请注意,这与用于“普通”列表的关键字\refKeyLe{/tcb/minted language}不同。仅用于\refKeyLe{/tcb/listing engine}|=minted|。
\end{docTcbKey}


\begin{marker}
The following two keys are deprecated and without function (v3.50 and above).
Use \refKeyLe{/tcb/before} and \refKeyLe{/tcb/after} with appropriate values
instead. Also see \refKeyLe{/tcb/docexample}.

以下两个键已被弃用且无功能(v3.50及以上版本)。 请使用适当的值替换\refKeyLe{/tcb/before}和\refKeyLe{/tcb/after}。 另请参阅\refKeyLe{/tcb/docexample}。
\end{marker}

\begin{docTcbKey}[][doc updated=2015-03-16]{before example}{=\meta{macros}}{no default, initially empty}
\smallskip\begin{deprecated}
Sets the \meta{macros} which are executed before \refEnvLe{dispExample} and \refEnvLe{dispListing}
additional to \refKeyLe{/tcb/before}.

设置在 \refEnvLe{dispExample} 和 \refEnvLe{dispListing} 之前执行的 \meta{macros},除了 \refKeyLe{/tcb/before}。
\end{deprecated}
\end{docTcbKey}

% \enlargethispage*{1cm}

\begin{docTcbKey}{after example}{=\meta{macros}}{no default, initially empty}
\smallskip\begin{deprecated}
Sets the \meta{macros} which are executed after \refEnvLe{dispExample} and \refEnvLe{dispListing}
additional to \refKeyLe{/tcb/after}.

设置在\refEnvLe{dispExample}和\refEnvLe{dispListing}之后执行的\meta{宏},除了\refKeyLe{/tcb/after}之外的附加内容。
\end{deprecated}
\end{docTcbKey}

%%\clearpage
\begin{docTcbKey}[][doc new=2017-04-25]{keywords bold}{\colOpt{=true\textbar false}}{default |true|, initially |true|}
Keyword used in \refEnvLe{docEnvironment}, \refEnvLe{docCommand}, etc. are printed
boldface (or not). Since the typewriter font is used, the effect may be
invisible with Computer Modern fonts or similar which do not
have a bold variant. Note that references to keywords are not printed boldface at all.

在\refEnvLe{docEnvironment},\refEnvLe{docCommand}等中使用的关键字以粗体(或不以粗体)打印。由于使用了打字机字体,因此在没有粗体变体的计算机现代字体或类似字体中效果可能看不见。请注意,对关键字的引用根本不以粗体打印。
\begin{dispExample*}{sidebyside}
\LARGE
\docAuxCommand{fooaux}, \refComLe{tcbset}

\tcbset{keywords bold=false}
\docAuxCommand{fooaux}, \refComLe{tcbset}
\end{dispExample*}
\end{docTcbKey}



\begin{docTcbKey}[][doc new=2015-01-09]{index command}{=\meta{macro}}{no default, initially \cs{index}}
Replaces the internally used \cs{index} macro by the given \meta{macro}.
The \meta{macro} has to take one mandatory argument like \cs{index}.
This option is mutually exclusive with \refKeyLe{/tcb/index command name}.

将内部使用的\cs{index}宏替换为给定的\meta{宏}。 \meta{宏}必须像\cs{index}一样需要一个必选参数。 此选项与\refKeyLe{/tcb/index command name}互斥。
\begin{dispListing}
\tcbset{index command=\myindexcommand}
\end{dispListing}
\end{docTcbKey}


\begin{docTcbKey}[][doc new=2015-01-09]{index command name}{=\meta{name}}{no default, initially unset}
Replaces the internally used \cs{index} macro by
\mbox{\cs{index}\texttt{[\meta{name}]}}, i.e.\ 
\mbox{\cs{index}\texttt{\textbraceleft\ldots\textbraceright}} is replaced by
\mbox{\cs{index}\texttt{[\meta{name}]\textbraceleft\ldots\textbraceright}}.
This option is intended to be used with |imakeidx| and is
mutually exclusive with \refKeyLe{/tcb/index command}.

将内部使用的\cs{index}宏替换为\mbox{\cs{index}\texttt{[\meta{name}]}},即\mbox{\cs{index}\texttt{\textbraceleft\ldots\textbraceright}}被替换为\mbox{\cs{index}\texttt{[\meta{name}]\textbraceleft\ldots\textbraceright}}。此选项旨在与|imakeidx|一起使用,并且与\refKeyLe{/tcb/index command}互斥。
\begin{dispListing}
\tcbset{index command name=mydoc}
\end{dispListing}
\end{docTcbKey}



\begin{docTcbKey}{index format}{=\meta{format}}{no default, initially |pgf|}
Determines the basic \meta{format} of the generated index.
Feasible values are:

确定生成索引的基本\meta{格式}。 可行的值有:
\begin{itemize}
\item\docValue{pgfsection}: The index is formatted like in the |pgf| documentation (as a section).
\item\docValue{pgfchapter}: The index is formatted like in the |pgf| documentation (as a chapter).
\item\docValue{pgf}: Alias for |pgfsection|.
\item\docValue{doc}: The index is assumed to be formatted by |doc| or |ltxdoc|. The usage of |makeindex|
with |-s gind.ist| is assumed. The package |hypdoc| has to be loaded
\emph{before} |tcolorbox|. Only a limited set of customizations will
work! This option cannot be unset when used!
\item\docValue{off}: The index is not formatted by |tcolorbox|. Use this, if
the index is formatted by other package like |imakeidx|.
\end{itemize}
\end{docTcbKey}

% \begin{itemize} \item \docValue{pgfsection}:索引的格式与|pgf|文档一样(作为一个章节)。 \item \docValue{pgfchapter}:索引的格式与|pgf|文档一样(作为一个章)。 \item \docValue{pgf}:是|pgfsection|的别名。 \item \docValue{doc}:假定索引由|doc|或|ltxdoc|格式化。假定使用|makeindex|和|-s gind.ist|。必须在|tcolorbox|之前加载|hypdoc|宏包。仅有限的自定义将起作用!使用此选项时无法取消! \item \docValue{off}:索引不由|tcolorbox|格式化。如果索引由其他宏包(如|imakeidx|)格式化,则使用此选项。 \end{itemize}


\begin{docTcbKey}{index actual}{=\meta{character}}{no default, initially |@|}
Sets the character for \enquote{actual} in automatic indexing.

设置自动索引中“实际”的字符。
\end{docTcbKey}

\begin{docTcbKey}{index quote}{=\meta{character}}{no default, initially |"|}
Sets the character for \enquote{quote} in automatic indexing.

设置自动索引中引用符号的字符。
\end{docTcbKey}

\begin{docTcbKey}{index level}{=\meta{character}}{no default, initially |!|}
Sets the character for \enquote{level} in automatic indexing.

在自动索引中设置“级别”字符。
\end{docTcbKey}

\begin{docTcbKey}{index default settings}{}{style, no value}
Sets the |makeindex| default values for
\refKeyLe{/tcb/index actual},
\refKeyLe{/tcb/index quote}, and
\refKeyLe{/tcb/index level}.

设置 |makeindex| 的默认值,包括 \refKeyLe{/tcb/index actual}、\refKeyLe{/tcb/index quote} 和 \refKeyLe{/tcb/index level}。
\end{docTcbKey}

% \enlargethispage*{1cm}

\begin{docTcbKey}{index german settings}{}{style, no value}
Sets the |makeindex| values recommended for German language texts.
This is identical to setting the following:

设置适用于德语文本的 |makeindex| 值建议。这与以下设置相同:
\begin{dispListing}
\tcbset{index actual={=},index quote={!},index level={>}}
\end{dispListing}
\end{docTcbKey}

% \clearpage

\begin{docTcbKey}{index annotate}{\colOpt{=true\textbar false}}{default |true|, initially |true|}
If set to |true|, the index entries are annotated with short descriptions
given by \refKeyLe{/tcb/doclang/environment}, \refKeyLe{/tcb/doclang/key},
and others.

如果设置为|true|,索引条目将用\refKeyLe{/tcb/doclang/environment}、\refKeyLe{/tcb/doclang/key}和其他短描述进行注释。
\end{docTcbKey}

\begin{docTcbKey}{index colorize}{\colOpt{=true\textbar false}}{default |true|, initially |false|}
If set to |true|, the index entries colorized according to the color
settings given by \refKeyLe{/tcb/color environment}, \refKeyLe{/tcb/color key},
and others.

如果设置为|true|,则索引条目将根据\refKeyLe{/tcb/color environment}、\refKeyLe{/tcb/color key}和其他颜色设置进行着色。
\end{docTcbKey}


\begin{docTcbKey}{color command}{=\meta{color}}{no default, initially |Definition|}
Sets the highlight color used by macro definitions.

设置宏定义使用的高亮颜色。
\end{docTcbKey}

\begin{docTcbKey}{color environment}{=\meta{color}}{no default, initially |Definition|}
Sets the highlight color used by environment definitions.

设置环境定义中使用的高亮颜色。
\end{docTcbKey}

\begin{docTcbKey}{color key}{=\meta{color}}{no default, initially |Definition|}
Sets the highlight color used by key definitions.

设置键定义使用的高亮颜色。
\end{docTcbKey}

\begin{docTcbKey}[][doc new={2019-09-18}]{color path}{=\meta{color}}{no default, initially |Definition|}
Sets the highlight color used by \tikzname\ path operation definitions.

设置\tikzname 路径操作定义使用的高亮颜色。
\end{docTcbKey}

\begin{docTcbKey}{color value}{=\meta{color}}{no default, initially |Definition|}
Sets the highlight color used by value definitions.

设置值定义使用的高亮颜色。
\end{docTcbKey}

\begin{docTcbKey}[][doc new={2015-01-08}]{color counter}{=\meta{color}}{no default, initially |Definition|}
Sets the highlight color used by counter definitions.

设置计数器定义使用的高亮颜色。
\end{docTcbKey}

\begin{docTcbKey}[][doc new={2015-01-08}]{color length}{=\meta{color}}{no default, initially |Definition|}
Sets the highlight color used by length definitions.

设置长度定义中使用的高亮颜色。
\end{docTcbKey}

\begin{docTcbKey}{color color}{=\meta{color}}{no default, initially |Definition|}
Sets the highlight color used by color definitions.

设置颜色定义使用的高亮颜色。
\end{docTcbKey}

\begin{docTcbKey}[][doc updated={2019-09-18}]{color definition}{=\meta{color}}{no default, initially |Definition|}
Sets the highlight color for \refKeyLe{/tcb/color command}, \refKeyLe{/tcb/color environment},
\refKeyLe{/tcb/color key}, \refKeyLe{/tcb/color path}, \refKeyLe{/tcb/color value}, \refKeyLe{/tcb/color counter},
\refKeyLe{/tcb/color length}, and \refKeyLe{/tcb/color color}.

设置 \refKeyLe{/tcb/color command}、\refKeyLe{/tcb/color environment}、\refKeyLe{/tcb/color key}、\refKeyLe{/tcb/color path}、\refKeyLe{/tcb/color value}、\refKeyLe{/tcb/color counter}、\refKeyLe{/tcb/color length} 和 \refKeyLe{/tcb/color color} 的高亮颜色。
\end{docTcbKey}

\begin{docTcbKey}{color option}{=\meta{color}}{no default, initially |Option|}
Sets the color used for optional arguments.

设置用于可选参数的颜色。
\end{docTcbKey}

\begin{docTcbKey}{color fade}{=\meta{color}}{no default, initially |Fade|}
Sets the color used for faded text like \colFade{\textbackslash path}
in \refEnvLe{docPathOperation}.

设置在\refEnvLe{docPathOperation} 中使用\colFade{\textbackslash path} 的淡化文本的颜色。
\end{docTcbKey}


\begin{docTcbKey}{color hyperlink}{=\meta{color}}{no default, initially |Hyperlink|}
Sets the color for all hyper-links, i.\,e. all internal and external links.

设置所有超链接的颜色,即所有内部和外部链接。
\end{docTcbKey}
