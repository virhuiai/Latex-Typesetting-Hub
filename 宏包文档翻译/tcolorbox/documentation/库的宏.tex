% \begin{docEnvironment}[doclang/environment content=command description,doc updated=2020-04-22]
  {docCommand}{\oarg{options}\marg{name}\marg{parameters}}
Documents a \LaTeX\ macro with given \meta{name} where \meta{name} is
written without backslash. The given \meta{options} are set with \refCom{tcbset}.
This macro takes mandatory or optional \meta{parameters}.
It is automatically indexed and can be referenced with
\refCom{refCom}\marg{name}.

记录一个具有给定 \meta{name} 的 \LaTeX\ 宏,其中 \meta{name} 不带反斜杠。给定的 \meta{options} 通过 \refCom{tcbset} 设置。这个宏需要强制或可选的 \meta{parameters}。它会自动索引,并可以通过 \refCom{refCom}\marg{name} 引用。
\begin{dispExample}
\begin{docCommand}{foomakedocSubKey}{\marg{name}\marg{key path}}
Creates a new environment \meta{name} based on \refEnv{docKey} for the
documentation of keys with the given \meta{key path}.

基于\refEnv{docKey},为给定的\meta{key path}创建一个名为\meta{name}的新环境,用于记录关键字的文档。
\end{docCommand}
\end{dispExample}
\begin{dispExample}
\begin{docCommand}[color definition=blue]{foomakedocSubKey*}%
  {\marg{name}\marg{key path}}
Creates a new environment \meta{name} based on \refEnv{docKey} for the
documentation of keys with the given \meta{key path}.

基于\refEnv{docKey},为给定的\meta{key path}创建一个名为\meta{name}的新环境,用于键的文档化。
\end{docCommand}
\end{dispExample}
\end{docEnvironment}

\begin{docEnvironment}[doclang/environment content=command description,doc updated=2020-04-22]
  {docCommand*}{\oarg{options}\marg{name}\marg{parameters}}
Identical to \refEnv{docCommand}, but without index entry.

与 |docCommand| 相同,但不包含索引条目。
\end{docEnvironment}

\begin{docEnvironment}[doclang/environment content=command description,doc new=2020-04-22]
  {docCommands}{\oarg{options}\brackets{\marg{variant1},\marg{variant2},...}}
Documents several (similar) \LaTeX\ macro variants simultaneously.
The given \meta{options} are set with \refCom{tcbset} and are valid for
all variants and the documentation text.
Every variant is described by an option set \meta{variant1}, \meta{variant2}, and so on.
The most crucial options are \refKeyLe{/tcb/doc name} and \refKeyLe{/tcb/doc parameter}.

同时记录了几个类似的 \LaTeX\ 宏变量。 给定的 \meta{选项} 是通过 |tcbset| 设置的,对于所有变体和文档文本都有效。 每个变体都由选项集 \meta{variant1}、\meta{variant2} 等描述。 最关键的选项是 |/tcb/doc name| 和 |/tcb/doc parameter|。
\begin{dispExample}
\begin{docCommands}[
  doc no index,  %  no index entries for this example
  doc name      = newtheorem,
  doc parameter = \marg{envname},
]
{
  {  },
  { doc parameter = \marg{envname}\oarg{numbered within} },
  { doc parameter = \oarg{numbered like}\marg{envname} },
  { doc name      = newtheorem* },
}
example
\end{docCommands}
\end{dispExample}
\end{docEnvironment}
% % \clearpage
{\let\xdocEnvironment\docEnvironment
\let\endxdocEnvironment\enddocEnvironment
\begin{xdocEnvironment}[doclang/environment content=environment description,doc updated=2020-04-22]
  {docEnvironment}{\oarg{options}\marg{name}\marg{parameters}}
Documents a \LaTeX\ environment with given \meta{name}.
The given \meta{options} are set with \refComLe{tcbset}.
This environment takes mandatory or optional \meta{parameters}.
It is automatically indexed and can be referenced with
\refComLe{refEnv}\marg{name}.

记录带有给定名称 \meta{name} 的 \LaTeX\ 环境。 给定的 \meta{options} 通过 |tcbset| 进行设置。 该环境可以带有强制或可选的 \meta{parameters}。 它会自动编制索引,并可以通过 |\refEnv|\marg{name} 进行引用。

\begin{dispExample}
\begin{docEnvironment}{foocolorbox}{\oarg{options}}
This is the main environment to create an accentuated colored text box with
rounded corners and, optionally, two parts.

这是创建带有突出颜色的圆角文本框的主要环境,还可以选择性地分为两个部分。
\end{docEnvironment}
\end{dispExample}
\begin{dispExample}
\begin{docEnvironment}%
  [doclang/environment content=My content text]%
  {foocolorbox*}{\oarg{options}}
This is the main environment to create an accentuated colored text box with
rounded corners and, optionally, two parts.

这是创建一个强调着色的文本框的主要环境,它具有圆角,并可选地分为两个部分。
\end{docEnvironment}
\end{dispExample}
\end{xdocEnvironment}}

{\let\xdocEnvironment\docEnvironment
\let\endxdocEnvironment\enddocEnvironment
\begin{xdocEnvironment}[doclang/environment content=environment description,doc updated=2020-04-22]
  {docEnvironment*}{\oarg{options}\marg{name}\marg{parameters}}
Identical to \refEnvLe{docEnvironment}, but without index entry.

与\refEnvLe{docEnvironment}相同,但没有索引条目。
\end{xdocEnvironment}}

%% \clearpage
{\let\xdocEnvironment\docEnvironment
\let\endxdocEnvironment\enddocEnvironment
\begin{xdocEnvironment}[doclang/environment content=environment description,doc new=2020-04-22]
  {docEnvironments}{\oarg{options}\brackets{\marg{variant1},\marg{variant2},...}}
Documents several (similar) \LaTeX\ environment variants simultaneously.
The given \meta{options} are set with \refComLe{tcbset} and are valid for
all variants and the documentation text.
Every variant is described by an option set \meta{variant1}, \meta{variant2}, and so on.
The most crucial options are \refKeyLe{/tcb/doc name} and \refKeyLe{/tcb/doc parameter}.

同时记录了几个(相似的)\LaTeX\ 环境变体。 给定的 \meta{options} 是通过 |\tcbset| 设置的,对于所有变体和文档文本都有效。 每个变体都由选项集 \meta{variant1}、\meta{variant2} 等描述。 最关键的选项是 |/tcb/doc name| 和 |/tcb/doc parameter|。
\begin{dispExample}
\begin{docEnvironments}[
  doc no index,   %  no index entries for this example
  doc parameter = \oarg{options}\marg{title},
  doclang/environment content = box content,
]
{
  {
    doc name        = redbox,
    doc description = a red colored box,
  },
  {
    doc name        = greenbox,
    doc description = a green colored box,
  },
  {
    doc name        = bluebox,
    doc description = a blue colored box,
  },
  {
    doc name        = custombox,
    doc parameter   = \oarg{options}\marg{color}\marg{title},
    doc description = a colored box,
  },
}
example
\end{docEnvironments}
\end{dispExample}
\end{xdocEnvironment}}

% %% \clearpage
\begin{docEnvironment}[doclang/environment content=key description,doc updated=2020-04-22]
  {docKey}{\oarg{key path}\oarg{options}\marg{name}\marg{parameters}\marg{description}}
Documents a key with given \meta{name} and an optional \meta{key path}.
The given \meta{options} are set with \refCom{tcbset}.
This key takes mandatory or optional \meta{parameters} as value
with a short \meta{description}.
It is automatically indexed and can be referenced with
\refCom{refKey}\marg{name}.

记录一个带有给定的 \meta{name} 和可选的 \meta{key path} 的关键字。 给定的 \meta{options} 由 |\tcbset| 设置。 该关键字以强制或可选的 \meta{parameters} 作为值, 并带有简短的 \meta{description}。 它会自动索引,并可以通过 |\refKey|\marg{name} 引用。
\begin{dispExample}
\begin{docKey}[foo]{footitle}{=\meta{text}}{no default, initially empty}
Creates a heading line with \meta{text} as content.
\end{docKey}
\end{dispExample}
\end{docEnvironment}


\begin{docEnvironment}[doclang/environment content=key description,doc updated=2020-04-22]
  {docKey*}{\oarg{key path}\oarg{options}\marg{name}\marg{parameters}\marg{description}}
Identical to \refEnv{docKey}, but without index entry.

与 \refEnv{docKey} 相同,但没有索引条目。
\end{docEnvironment}

\begin{docEnvironment}[doclang/environment content=key description,doc new=2020-04-22]
  {docKeys}{\oarg{options}\brackets{\marg{variant1},\marg{variant2},...}}
Documents several (similar) key variants simultaneously.
The given \meta{options} are set with \refCom{tcbset} and are valid for
all variants and the documentation text.
Every variant is described by an option set \meta{variant1}, \meta{variant2}, and so on.
The most crucial options are
\refKeyLe{/tcb/doc keypath}, \refKeyLe{/tcb/doc name}, \refKeyLe{/tcb/doc parameter},
and \refKeyLe{/tcb/doc description}.

同时记录几个(相似的)关键变量。 给定的\meta{options}是通过|\tcbset|设置的,对所有变量和文档文本有效。 每个变量由一个选项集\meta{variant1},\meta{variant2}等描述。 最关键的选项是|/tcb/doc keypath|,|/tcb/doc name|,|/tcb/doc parameter|和|/tcb/doc description|。
\begin{dispExample}
\begin{docKeys}[
  doc no index,   %  no index entries for this example
  doc keypath   = mykeyroot,
  doc parameter = {=\meta{length}},
]
{
  {
    doc name        = width,
    doc description = initially \texttt{10cm},
  },
  {
    doc name        = height,
    doc description = initially \texttt{7cm},
  },
}
example
\end{docKeys}
\end{dispExample}
\end{docEnvironment}







%% \clearpage
\begin{docEnvironment}[doclang/environment content=operation description,
  doc new and updated={2019-09-18}{2020-04-22}]{docPathOperation}{\oarg{options}\marg{name}\marg{parameters}}
Documents a \tikzname\ path operation with given \meta{name}.
The given \meta{options} are set with \refCom{tcbset}.
This \tikzname\ path operation takes mandatory or optional \meta{parameters}.
It is automatically indexed and can be referenced with
\refCom{refPathOperation}\marg{name}.

使用给定的\meta{name}记录\tikzname\ 路径操作。给定的\meta{options}使用\refCom{tcbset}设置。这个\tikzname\ 路径操作接受必选或可选的\meta{parameters}。它会自动索引,并可以通过|\refPathOperation|\marg{name}进行引用。
\begin{dispExample}
\begin{docPathOperation}{fooop}{\oarg{opt}(\meta{name})\colOpt{at(\meta{coord})}}
Imaginary path operation for illustration.
\end{docPathOperation}
\end{dispExample}
\end{docEnvironment}


\begin{docEnvironment}[doclang/environment content=command description,
  doc new and updated={2019-09-18}{2020-04-22}]{docPathOperation*}{\oarg{options}\marg{name}\marg{parameters}}
Identical to \refEnv{docPathOperation}, but without index entry.

虚构路径操作的插图。
\end{docEnvironment}


% \begin{docEnvironment}[doclang/environment content=command description,
%   doc new={2020-04-22}]{docPathOperations}{\oarg{options}\brackets{\marg{variant1},\marg{variant2},...}}
% Documents several (similar) \tikzname\ path operation variants simultaneously.
% The given \meta{options} are set with \refCom{tcbset} and are valid for
% all variants and the documentation text.
% Every variant is described by an option set \meta{variant1}, \meta{variant2}, and so on.
% The most crucial options are \refKey{/tcb/doc name} and \refKey{/tcb/doc parameter}.
% \begin{dispExample}
% \begin{docPathOperations}[
%   doc no index,   %  no index entries for this example
% ]
% {
%   {
%     doc name      = rectangle,
%     doc parameter = \meta{corner or cycle},
%   },
%   {
%     doc name      = circle,
%     doc parameter = \oarg{options},
%   },
%   {
%     doc name      = ellipse,
%     doc parameter = \oarg{options},
%   },
% }
% example
% \end{docPathOperations}
% \end{dispExample}
% \end{docEnvironment}


% \clearpage
% \begin{docCommands}[doc parameter=\oarg{options}\marg{name}]
% {
%   {
%     doc name = docValue,
%     doc updated=2020-04-23,
%   },
%   {
%     doc name = docValue*,
%   },
% }
% Documents a value with given \meta{name}. Typically, this is a value for a key.
% The given \meta{options} are set with \refCom{tcbset}.
% This value is automatically indexed for \refCom{docValue}
% and has no index entry for \refCom{docValue*}.
% \begin{dispExample}
% A feasible value for \refKey{/foo/footitle} is \docValue*{foovalue}.
% \end{dispExample}
% \end{docCommands}



% \begin{docCommands}[doc parameter=\oarg{options}\marg{name}]
% {
%   {
%     doc name    = docAuxCommand,
%     doc updated = 2020-04-23,
%   },
%   {
%     doc name = docAuxCommand*,
%   },
% }
% Documents an auxiliary or minor \LaTeX\ macro with given \meta{name}
% where \meta{name} is written without backslash.
% The given \meta{options} are set with \refCom{tcbset}.
% This macro is automatically indexed for \refCom{docAuxCommand}
% and has no index entry for \refCom{docAuxCommand*}.
% \begin{dispExample}
% The macro \docAuxCommand{fooaux} holds some interesting data.
% \end{dispExample}
% \end{docCommands}



% \begin{docCommands}[doc parameter=\oarg{options}\marg{name}]
% {
%   {
%     doc name    = docAuxEnvironment,
%     doc updated = 2020-04-23,
%   },
%   {
%     doc name = docAuxEnvironment*,
%   },
% }
% Documents an auxiliary or minor \LaTeX\ environment with given \meta{name}.
% The given \meta{options} are set with \refCom{tcbset}.
% This macro is automatically indexed indexed for \refCom{docAuxEnvironment}
% and has no index entry for \refCom{docAuxEnvironment*}.
% \begin{dispExample}
% The environment \docAuxEnvironment{fooauxenv} holds some interesting data.
% \end{dispExample}
% \end{docCommands}


% \begin{docCommands}[doc parameter=\oarg{key path}\oarg{options}\marg{name}]
% {
%   {
%     doc name    = docAuxKey,
%     doc updated = 2020-04-23,
%   },
%   {
%     doc name = docAuxKey*,
%   },
% }
% Documents an auxiliary key with given \meta{name} and an optional \meta{key path}.
% The given \meta{options} are set with \refCom{tcbset}.
% It is automatically indexed for \refCom{docAuxKey}
% and has no index entry for \refCom{docAuxKey*}.
% \begin{dispExample}
% The key \docAuxKey[foo]{fooaux} holds some interesting data.
% \end{dispExample}
% \end{docCommands}



% \begin{docCommands}[doc parameter=\oarg{options}\marg{name}]
% {
%   {
%     doc name    = docCounter,
%     doc updated = 2020-04-23,
%   },
%   {
%     doc name = docCounter*,
%   },
% }
% Documents a counter with given \meta{name}.
% The given \meta{options} are set with \refCom{tcbset}.
% The counter is automatically indexed for \refCom{docCounter}
% and has no index entry for \refCom{docCounter*}.
% \begin{dispExample}
% The counter \docCounter{foocounter} can be used for computation.
% \end{dispExample}
% \end{docCommands}


% \clearpage
% \begin{docCommands}[doc parameter=\oarg{options}\marg{name}]
% {
%   {
%     doc name    = docLength,
%     doc updated = 2020-04-23,
%   },
%   {
%     doc name = docLength*,
%   },
% }
% Documents a length with given \meta{name}.
% The given \meta{options} are set with \refCom{tcbset}.
% The length is automatically indexed for \refCom{docLength}
% and has no index entry for \refCom{docLength*}.
% \begin{dispExample}
% The length \docLength{foolength} can be used for computation.
% \end{dispExample}
% \end{docCommands}


% \begin{docCommands}[doc parameter=\oarg{options}\marg{name}]
% {
%   {
%     doc name    = docColor,
%     doc updated = 2020-04-23,
%   },
%   {
%     doc name = docColor*,
%   },
% }
% Documents a color with given \meta{name}.
% The given \meta{options} are set with \refCom{tcbset}.
% The color is automatically indexed for \refCom{docColor}
% and has no index entry for \refCom{docColor*}.
% \begin{dispExample}
% The color \docColor{foocolor} is available.
% \end{dispExample}
% \end{docCommands}



% \begin{docCommand}{cs}{\marg{name}}
% Macro from |ltxdoc| \cite{carlisle:ltxdoc} to typeset a command word \meta{name}
% where the backslash is prefixed. The library overwrites the original macro.
% \begin{dispExample}
% This is a \cs{foocommand}.
% \end{dispExample}
% \end{docCommand}

% \begin{docCommand}{meta}{\marg{text}}
% Macro from |doc| \cite{mittelbach:2011a} to typeset a meta \meta{text}.
% The library overwrites the original macro.
% \begin{dispExample}
% This is a \meta{text}.
% \end{dispExample}
% \end{docCommand}


% \begin{docCommand}{marg}{\marg{text}}
% Macro from |ltxdoc| \cite{carlisle:ltxdoc} to typeset a \meta{text} with
% curly brackets as a mandatory argument. The library overwrites the original macro.
% \begin{dispExample}
% This is a mandatory \marg{argument}.
% \end{dispExample}
% \end{docCommand}

% \begin{docCommand}{oarg}{\marg{text}}
% Macro from |ltxdoc| \cite{carlisle:ltxdoc} to typeset a \meta{text} with
% square brackets as an optional argument. The library overwrites the original macro.
% \begin{dispExample}
% This is an optional \oarg{argument}.
% \end{dispExample}
% \end{docCommand}

% \clearpage

% \begin{docCommand}{brackets}{\marg{text}}
% Sets the given \meta{text} with curly brackets.
% \begin{dispExample}
% Here we use \brackets{some text}.
% \end{dispExample}
% \end{docCommand}


% {\let\xdispExample\dispExample
% \let\endxdispExample\enddispExample
% \begin{docEnvironment}[doc updated=2014-10-10]{dispExample}{}
% Creates a colored box based on a \refEnv{tcolorbox}.
% It displays the environment content as source code in the upper part
% and as compiled text in the lower part of the box.
% The appearance is controlled by \refKey{/tcb/documentation listing style}
% and the style \refKey{/tcb/docexample}. It may be
% changed by redefining this style.
% {
% %\tcbset{before lower app={\tcbset{docexample/.style={docexample original}}}}
% %\tcbset{docexample/.style={docexample original}}%
% \begin{xdispExample}
% \begin{dispExample}
% This is a \LaTeX\ example.
% \end{dispExample}
% \end{xdispExample}
% }
% \end{docEnvironment}}


% {\let\xdispExample\dispExample
% \let\endxdispExample\enddispExample
% \begin{docEnvironment}[doc updated=2014-10-10]{dispExample*}{\marg{options}}
% The starred version of \refEnv{dispExample} takes \refEnv{tcolorbox} \meta{options}
% as parameter. These \meta{options} are executed after \refKey{/tcb/docexample}.
% \begin{xdispExample}
% \begin{dispExample*}{sidebyside}
% This is a \LaTeX\ example.
% \end{dispExample*}
% \end{xdispExample}
% \end{docEnvironment}}


% \clearpage
% \begin{docEnvironment}{dispListing}{}
% Creates a colored box based on a \refEnv{tcolorbox}.
% It displays the environment content as source code.
% The appearance is controlled by \refKey{/tcb/documentation listing style}
% and the style \refKey{/tcb/docexample}. It may be
% changed by redefining this style.
% \begin{dispExample}
% \begin{dispListing}
% This is a \LaTeX\ example.
% \end{dispListing}
% \end{dispExample}
% \end{docEnvironment}

% \begin{docEnvironment}{dispListing*}{\marg{options}}
% The starred version of \refEnv{dispListing} takes \refEnv{tcolorbox} \meta{options}
% as parameter. These \meta{options} are executed after \refKey{/tcb/docexample}.
% \begin{dispExample}
% \begin{dispListing*}{title=My listing}
% This is a \LaTeX\ example.
% \end{dispListing*}
% \end{dispExample}
% \end{docEnvironment}


% \begin{docEnvironment}{absquote}{}
% Used to typeset an abstract as quoted and small text.
% \begin{dispExample}
% \begin{absquote}
% |tcolorbox| provides an environment for colored and framed text boxes with a
% heading line. Optionally, such a box can be split in an upper and a lower part.
% \end{absquote}
% \end{dispExample}
% \end{docEnvironment}

% \clearpage
% \begin{docCommand}[doc updated=2020-04-22]{tcbmakedocSubKey}{\marg{name}\marg{key path}}
% Creates a new environment \meta{name} based on \refEnv{docKey} for the
% documentation of keys with the given \meta{key path} as root.
% The new environment \meta{name} takes the same para\-meters as \refEnv{docKey} itself.
% A second starred environment \meta{name} is also created, which is identical
% to \meta{name} but without index entry.
% \begin{dispExample}
% \tcbmakedocSubKey{docFooKey}{foo}

% \begin{docFooKey}{foodummy}{=\meta{nothing}}{no default, initially empty}
% Some key.
% \end{docFooKey}

% \begin{docFooKey*}{foo another dummy}{=\meta{nothing}}{no default, initially empty}
% Some key (not indexed).
% \end{docFooKey*}
% \end{dispExample}
% \end{docCommand}


% \begin{docCommand}[doc new=2020-04-22]{tcbmakedocSubKeys}{\marg{name}\marg{key path}}
% Creates a new environment \meta{name} based on \refEnv{docKeys} for the
% documentation of keys with the given \meta{key path} as root.
% The new environment \meta{name} takes the same para\-meters as \refEnv{docKeys} itself.
% \begin{dispExample}
% \tcbmakedocSubKeys{docFooKeys}{foo}

% \begin{docFooKeys}[
%   doc parameter   = {=\meta{nothing}},
%   doc description = {no default, initially empty},
% ]
% {
%   {
%     doc name = foodummy 2,
%   },
%   {
%     doc name = foo another dummy 2,
%     doc no index,
%   }
% }
% Some description.
% \end{docFooKeys}
% \end{dispExample}
% \end{docCommand}


% \clearpage

% \begin{docCommand}{refCom}{\marg{name}}
% References a documented \LaTeX\ macro with given \meta{name} where \meta{name} is
% written without backslash. The page reference is suppressed if it links
% to the same page.
% \begin{dispExample}
% We have created \refCom{foomakedocSubKey} as an example.
% \end{dispExample}
% \end{docCommand}

% \begin{docCommand}{refCom*}{\marg{name}}
% References a documented \LaTeX\ macro with given \meta{name} where \meta{name} is
% written without backslash. There is no page reference.
% \begin{dispExample}
% We have created \refCom*{foomakedocSubKey} as an example.
% \end{dispExample}
% \end{docCommand}


% \begin{docCommand}{refEnv}{\marg{name}}
% References a documented \LaTeX\ environment with given \meta{name}.
% The page reference is suppressed if it links to the same page.
% \begin{dispExample}
% We have created \refEnv{foocolorbox} as an example.
% \end{dispExample}
% \end{docCommand}

% \begin{docCommand}{refEnv*}{\marg{name}}
% References a documented \LaTeX\ environment with given \meta{name}.
% There is no page reference.
% \begin{dispExample}
% We have created \refEnv*{foocolorbox} as an example.
% \end{dispExample}
% \end{docCommand}


% \begin{docCommand}{refKey}{\marg{name}}
% References a documented key with given \meta{name} where \meta{name}
% is the full path name of the key.
% The page reference is suppressed if it links to the same page.
% \begin{dispExample}
% We have created \refKey{/foo/footitle} as an example.
% \end{dispExample}
% \end{docCommand}

% \begin{docCommand}{refKey*}{\marg{name}}
% References a documented key with given \meta{name} where \meta{name}
% is the full path name of the key.
% There is no page reference.
% \begin{dispExample}
% We have created \refKey*{/foo/footitle} as an example.
% \end{dispExample}
% \end{docCommand}

% \clearpage

% \begin{docCommand}[doc new=2019-09-17]{refPathOperation}{\marg{name}}
% References a documented \tikzname\ path operation with given \meta{name}.
% The page reference is suppressed if it links to the same page.
% \begin{dispExample}
% We have created \refPathOperation{fooop} as an example.
% \end{dispExample}
% \end{docCommand}

% \begin{docCommand}[doc new=2019-09-17]{refPathOperation*}{\marg{name}}
% References a documented \tikzname\ path operation with given \meta{name}.
% There is no page reference.
% \begin{dispExample}
% We have created \refPathOperation*{fooop} as an example.
% \end{dispExample}
% \end{docCommand}



% \begin{docCommand}[doc updated=2020-02-11]{refAux}{\marg{name}}
% References some auxiliary environment, key, value, or color.
% The \meta{name} is colored according to \refKey{/tcb/color hyperlink},
% if |hyperref| colorlinks are set, but there is no real link.
% \begin{dispExample}
% Some pages back, one can see \refAux{/foo/footitle} as an example.
% \end{dispExample}
% \end{docCommand}

% \begin{docCommand}[doc updated=2020-02-11]{refAuxcs}{\marg{name}}
% References some auxiliary macro \meta{name} where \meta{name} is
% written without backslash.
% The \meta{name} is colored according to \refKey{/tcb/color hyperlink},
% if |hyperref| colorlinks are set, but there is no real link.
% \begin{dispExample}
% Some pages back, one can see \refAuxcs{fooaux} as an example.
% \end{dispExample}
% \end{docCommand}


% \begin{docCommand}{colDef}{\marg{text}}
% Sets \meta{text} with the command color, see \refKey{/tcb/color command}.
% \begin{dispExample}
% This is my \colDef{text}.
% \end{dispExample}
% \end{docCommand}

% \begin{docCommand}{colOpt}{\marg{text}}
% Sets \meta{text} with the option color, see \refKey{/tcb/color option}.
% \begin{dispExample}
% This is my \colOpt{text}.
% \end{dispExample}
% \end{docCommand}

% \clearpage

% \begin{docCommand}[doc new=2019-09-18]{colFade}{\marg{text}}
% Sets \meta{text} with the fade color, see \refKey{/tcb/color fade}.
% \begin{dispExample}
% This is my \colFade{text}.
% \end{dispExample}
% \end{docCommand}


% \begin{docCommand}[doc new=2014-09-19]{tcbdocmarginnote}{\oarg{options}\marg{text}}
% Creates a |tcolorbox| note with the given \meta{text} inside the margin using
% the |marginnote| package. The style of the |tcolorbox| is predefined and can be
% altered by \refKey{/tcb/doc marginnote} and the given \meta{options}.
% \begin{dispExample}
% Some text\tcbdocmarginnote{Note A}
% which is commented by a note inside the margin.
% Alternatively to |\tcbdocmarginnote|, you can always use
% |\marginnote| with a |tcolorbox| directly.\par
% This is further text%
% \tcbdocmarginnote[colframe=blue!50!white,colback=blue!5!white]{Note B}
% with another note.
% \end{dispExample}
% \end{docCommand}

% \begin{docCommand}[doc new=2014-09-19]{tcbdocnew}{\marg{date}}
% Auxiliary macro which typesets the \refKey{/tcb/doclang/new} text with
% the given \meta{date}. It may be redefined for customization.
% \makeatletter\renewcommand*{\tcbdocnew}[1]{\kvtcb@text@new: #1}\makeatother%
% \begin{dispExample*}{sidebyside}
% \tcbdocnew{1981-10-29}.
% % Next one is displayed in the margin:
% \tcbdocmarginnote{\tcbdocnew{1978-02-09}}
% \end{dispExample*}
% \end{docCommand}

% \begin{docCommand}[doc new=2014-09-19]{tcbdocupdated}{\marg{date}}
% Auxiliary macro which typesets the \refKey{/tcb/doclang/updated} text with
% the given \meta{date}. It may be redefined for customization.
% \makeatletter\renewcommand*{\tcbdocupdated}[1]{\kvtcb@text@updated: #1}\makeatother%
% \begin{dispExample*}{sidebyside}
% \tcbdocupdated{2014-09-19}.
% \end{dispExample*}
% \end{docCommand}

