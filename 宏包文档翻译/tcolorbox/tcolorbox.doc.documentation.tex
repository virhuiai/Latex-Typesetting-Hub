% % !TeX root = tcolorbox.tex
% % include file of tcolorbox.tex (manual of the LaTeX package tcolorbox)
% \clearpage
\section{Library \mylib{documentation}}\label{sec:documentation}%
\tcbset{external/prefix=external/documentation_}%
This library has the single purpose to support \LaTeX\ package documentations
like this one. Actually, the visual nature follows the approach from
Till Tantau's |pgf| \cite{tantau:tikz_and_pgf} documentation.
Typically, this library is assumed to be used in conjunction with the
class |ltxdoc| or alike.
Denis Bitouz\'e, Muzimuzhi, and many others provided very valuable input for this library.

这个库的唯一目的是支持像这个一样的\LaTeX\ 包文档。实际上,其视觉风格遵循了Till Tantau的|pgf|\cite{tantau:tikz_and_pgf}文档的方法。通常,这个库被认为是与类|ltxdoc|或类似的类一起使用的。Denis Bitouz'e,Muzimuzhi和许多其他人为这个库提供了非常有价值的输入。

The library is loaded by a package option or inside the preamble by:

该库可以通过软件包选项或者在导言部分中被加载:
\begin{dispListing}
  \tcbuselibrary{documentation}
\end{dispListing}
This also loads
the library \mylib{skins}, see \Vref{sec:skins},
the library \mylib{raster}, see \Vref{sec:raster},
the library \mylib{listings}, see \Vref{sec:listings},
the library \mylib{xparse}, see \Vref{sec:xparse},
and a bunch of packages, namely
|makeidx|, |marginnote|, |refcount|, and |hyperref|.
The packages |pifont| and |marvosym| should be installed for some symbols, but
need not to be loaded.

这也会加载库\mylib{skins},参见\Vref{sec:skins}, 库\mylib{raster},参见\Vref{sec:raster}, 库\mylib{listings},参见\Vref{sec:listings}, 库\mylib{xparse},参见\Vref{sec:xparse}, 以及一堆包,即 |makeidx|,|marginnote|,|refcount|和|hyperref|。 对于一些符号,应安装包|pifont|和|marvosym|,但不需要加载。

\begin{marker}
The package |makeidx| is loaded only, if \docAuxCommand*{printindex} is
\emph{not} already defined. Therefore, one can include an alternative to |makeidx| like
|imakeidx| \emph{before} the library |documentation| is used.

只有当\docAuxCommand*{printindex}没有被定义时,才会加载|makeidx|包。因此,在使用库|documentation|之前,可以在其之前包含一个替代|makeidx|的包,例如|imakeidx|。
\end{marker}
\begin{marker}
The package |marginnote| is loaded only, if \docAuxCommand*{marginnote} is
\emph{not} already defined.

只有当 \docAuxCommand*{marginnote} 没有被定义时,才会加载 |marginnote| 包。
\end{marker}
\begin{marker}
In contrast to other |tcolorbox| options, the option
settings for \mylib{documentation} are typically not
getting reset by \refKeyLe{/tcb/reset}, i.e. they keep their
values for embedded boxes.

与其他 |tcolorbox| 选项不同的是,\mylib{documentation} 的选项设置通常不会被 \refKeyLe{/tcb/reset} 重置,即它们会保持嵌入箱子的值不变。
\end{marker}
\begin{marker}
In combination with DocStrip, \refKeyLe{/tcb/verbatim ignore percent} may be helpful.

结合使用DocStrip,\refKeyLe{/tcb/verbatim ignore percent}可能会有所帮助。
\end{marker}

For UTF-8 support load (ignore this when using Xe\LaTeX):

对于UTF-8支持加载(当使用Xe\LaTeX 时请忽略此项):
\begin{dispListing}
  \tcbuselibrary{listingsutf8,documentation}
\end{dispListing}

For |minted| \cite{poore:minted} support, load:

要使用 |minted| \cite{poore:minted},请加载以下内容:
\begin{dispListing}
  \tcbuselibrary{documentation,minted}
  \tcbset{listing engine=minted}
\end{dispListing}


%% todo 以上后面放开

%\clearpage
%-------------------------------------------------------------------------------
\subsection{Macros of the Library\\库的宏}
% \begin{docEnvironment}[doclang/environment content=command description,doc updated=2020-04-22]
  {docCommand}{\oarg{options}\marg{name}\marg{parameters}}
Documents a \LaTeX\ macro with given \meta{name} where \meta{name} is
written without backslash. The given \meta{options} are set with \refCom{tcbset}.
This macro takes mandatory or optional \meta{parameters}.
It is automatically indexed and can be referenced with
\refCom{refCom}\marg{name}.

记录一个具有给定 \meta{name} 的 \LaTeX\ 宏,其中 \meta{name} 不带反斜杠。给定的 \meta{options} 通过 \refCom{tcbset} 设置。这个宏需要强制或可选的 \meta{parameters}。它会自动索引,并可以通过 \refCom{refCom}\marg{name} 引用。
\begin{dispExample}
\begin{docCommand}{foomakedocSubKey}{\marg{name}\marg{key path}}
Creates a new environment \meta{name} based on \refEnv{docKey} for the
documentation of keys with the given \meta{key path}.

基于\refEnv{docKey},为给定的\meta{key path}创建一个名为\meta{name}的新环境,用于记录关键字的文档。
\end{docCommand}
\end{dispExample}
\begin{dispExample}
\begin{docCommand}[color definition=blue]{foomakedocSubKey*}%
  {\marg{name}\marg{key path}}
Creates a new environment \meta{name} based on \refEnv{docKey} for the
documentation of keys with the given \meta{key path}.

基于\refEnv{docKey},为给定的\meta{key path}创建一个名为\meta{name}的新环境,用于键的文档化。
\end{docCommand}
\end{dispExample}
\end{docEnvironment}

\begin{docEnvironment}[doclang/environment content=command description,doc updated=2020-04-22]
  {docCommand*}{\oarg{options}\marg{name}\marg{parameters}}
Identical to \refEnv{docCommand}, but without index entry.

与 |docCommand| 相同,但不包含索引条目。
\end{docEnvironment}

\begin{docEnvironment}[doclang/environment content=command description,doc new=2020-04-22]
  {docCommands}{\oarg{options}\brackets{\marg{variant1},\marg{variant2},...}}
Documents several (similar) \LaTeX\ macro variants simultaneously.
The given \meta{options} are set with \refCom{tcbset} and are valid for
all variants and the documentation text.
Every variant is described by an option set \meta{variant1}, \meta{variant2}, and so on.
The most crucial options are \refKeyLe{/tcb/doc name} and \refKeyLe{/tcb/doc parameter}.

同时记录了几个类似的 \LaTeX\ 宏变量。 给定的 \meta{选项} 是通过 |tcbset| 设置的,对于所有变体和文档文本都有效。 每个变体都由选项集 \meta{variant1}、\meta{variant2} 等描述。 最关键的选项是 |/tcb/doc name| 和 |/tcb/doc parameter|。
\begin{dispExample}
\begin{docCommands}[
  doc no index,  %  no index entries for this example
  doc name      = newtheorem,
  doc parameter = \marg{envname},
]
{
  {  },
  { doc parameter = \marg{envname}\oarg{numbered within} },
  { doc parameter = \oarg{numbered like}\marg{envname} },
  { doc name      = newtheorem* },
}
example
\end{docCommands}
\end{dispExample}
\end{docEnvironment}
% % \clearpage
{\let\xdocEnvironment\docEnvironment
\let\endxdocEnvironment\enddocEnvironment
\begin{xdocEnvironment}[doclang/environment content=environment description,doc updated=2020-04-22]
  {docEnvironment}{\oarg{options}\marg{name}\marg{parameters}}
Documents a \LaTeX\ environment with given \meta{name}.
The given \meta{options} are set with \refComLe{tcbset}.
This environment takes mandatory or optional \meta{parameters}.
It is automatically indexed and can be referenced with
\refComLe{refEnv}\marg{name}.

记录带有给定名称 \meta{name} 的 \LaTeX\ 环境。 给定的 \meta{options} 通过 |tcbset| 进行设置。 该环境可以带有强制或可选的 \meta{parameters}。 它会自动编制索引,并可以通过 |\refEnv|\marg{name} 进行引用。

\begin{dispExample}
\begin{docEnvironment}{foocolorbox}{\oarg{options}}
This is the main environment to create an accentuated colored text box with
rounded corners and, optionally, two parts.

这是创建带有突出颜色的圆角文本框的主要环境,还可以选择性地分为两个部分。
\end{docEnvironment}
\end{dispExample}
\begin{dispExample}
\begin{docEnvironment}%
  [doclang/environment content=My content text]%
  {foocolorbox*}{\oarg{options}}
This is the main environment to create an accentuated colored text box with
rounded corners and, optionally, two parts.

这是创建一个强调着色的文本框的主要环境,它具有圆角,并可选地分为两个部分。
\end{docEnvironment}
\end{dispExample}
\end{xdocEnvironment}}

{\let\xdocEnvironment\docEnvironment
\let\endxdocEnvironment\enddocEnvironment
\begin{xdocEnvironment}[doclang/environment content=environment description,doc updated=2020-04-22]
  {docEnvironment*}{\oarg{options}\marg{name}\marg{parameters}}
Identical to \refEnvLe{docEnvironment}, but without index entry.

与\refEnvLe{docEnvironment}相同,但没有索引条目。
\end{xdocEnvironment}}

%% \clearpage
{\let\xdocEnvironment\docEnvironment
\let\endxdocEnvironment\enddocEnvironment
\begin{xdocEnvironment}[doclang/environment content=environment description,doc new=2020-04-22]
  {docEnvironments}{\oarg{options}\brackets{\marg{variant1},\marg{variant2},...}}
Documents several (similar) \LaTeX\ environment variants simultaneously.
The given \meta{options} are set with \refComLe{tcbset} and are valid for
all variants and the documentation text.
Every variant is described by an option set \meta{variant1}, \meta{variant2}, and so on.
The most crucial options are \refKeyLe{/tcb/doc name} and \refKeyLe{/tcb/doc parameter}.

同时记录了几个(相似的)\LaTeX\ 环境变体。 给定的 \meta{options} 是通过 |\tcbset| 设置的,对于所有变体和文档文本都有效。 每个变体都由选项集 \meta{variant1}、\meta{variant2} 等描述。 最关键的选项是 |/tcb/doc name| 和 |/tcb/doc parameter|。
\begin{dispExample}
\begin{docEnvironments}[
  doc no index,   %  no index entries for this example
  doc parameter = \oarg{options}\marg{title},
  doclang/environment content = box content,
]
{
  {
    doc name        = redbox,
    doc description = a red colored box,
  },
  {
    doc name        = greenbox,
    doc description = a green colored box,
  },
  {
    doc name        = bluebox,
    doc description = a blue colored box,
  },
  {
    doc name        = custombox,
    doc parameter   = \oarg{options}\marg{color}\marg{title},
    doc description = a colored box,
  },
}
example
\end{docEnvironments}
\end{dispExample}
\end{xdocEnvironment}}
% %% \clearpage
\begin{docEnvironment}[doclang/environment content=key description,doc updated=2020-04-22]
  {docKey}{\oarg{key path}\oarg{options}\marg{name}\marg{parameters}\marg{description}}
Documents a key with given \meta{name} and an optional \meta{key path}.
The given \meta{options} are set with \refCom{tcbset}.
This key takes mandatory or optional \meta{parameters} as value
with a short \meta{description}.
It is automatically indexed and can be referenced with
\refCom{refKey}\marg{name}.

记录一个带有给定的 \meta{name} 和可选的 \meta{key path} 的关键字。 给定的 \meta{options} 由 |\tcbset| 设置。 该关键字以强制或可选的 \meta{parameters} 作为值, 并带有简短的 \meta{description}。 它会自动索引,并可以通过 |\refKey|\marg{name} 引用。
\begin{dispExample}
\begin{docKey}[foo]{footitle}{=\meta{text}}{no default, initially empty}
Creates a heading line with \meta{text} as content.
\end{docKey}
\end{dispExample}
\end{docEnvironment}


\begin{docEnvironment}[doclang/environment content=key description,doc updated=2020-04-22]
  {docKey*}{\oarg{key path}\oarg{options}\marg{name}\marg{parameters}\marg{description}}
Identical to \refEnv{docKey}, but without index entry.

与 \refEnv{docKey} 相同,但没有索引条目。
\end{docEnvironment}

\begin{docEnvironment}[doclang/environment content=key description,doc new=2020-04-22]
  {docKeys}{\oarg{options}\brackets{\marg{variant1},\marg{variant2},...}}
Documents several (similar) key variants simultaneously.
The given \meta{options} are set with \refCom{tcbset} and are valid for
all variants and the documentation text.
Every variant is described by an option set \meta{variant1}, \meta{variant2}, and so on.
The most crucial options are
\refKeyLe{/tcb/doc keypath}, \refKeyLe{/tcb/doc name}, \refKeyLe{/tcb/doc parameter},
and \refKeyLe{/tcb/doc description}.

同时记录几个(相似的)关键变量。 给定的\meta{options}是通过|\tcbset|设置的,对所有变量和文档文本有效。 每个变量由一个选项集\meta{variant1},\meta{variant2}等描述。 最关键的选项是|/tcb/doc keypath|,|/tcb/doc name|,|/tcb/doc parameter|和|/tcb/doc description|。
\begin{dispExample}
\begin{docKeys}[
  doc no index,   %  no index entries for this example
  doc keypath   = mykeyroot,
  doc parameter = {=\meta{length}},
]
{
  {
    doc name        = width,
    doc description = initially \texttt{10cm},
  },
  {
    doc name        = height,
    doc description = initially \texttt{7cm},
  },
}
example
\end{docKeys}
\end{dispExample}
\end{docEnvironment}


% %% \clearpage
\begin{docEnvironment}[doclang/environment content=operation description,
  doc new and updated={2019-09-18}{2020-04-22}]{docPathOperation}{\oarg{options}\marg{name}\marg{parameters}}
Documents a \tikzname\ path operation with given \meta{name}.
The given \meta{options} are set with \refCom{tcbset}.
This \tikzname\ path operation takes mandatory or optional \meta{parameters}.
It is automatically indexed and can be referenced with
\refCom{refPathOperation}\marg{name}.

使用给定的\meta{name}记录\tikzname\ 路径操作。给定的\meta{options}使用|\tcbset|设置。这个\tikzname\ 路径操作接受必选或可选的\meta{parameters}。它会自动索引,并可以通过|\refPathOperation|\marg{name}进行引用。
\begin{dispExample}
\begin{docPathOperation}{fooop}{\oarg{opt}(\meta{name})\colOpt{at(\meta{coord})}}
Imaginary path operation for illustration.
\end{docPathOperation}
\end{dispExample}
\end{docEnvironment}


\begin{docEnvironment}[doclang/environment content=command description,
  doc new and updated={2019-09-18}{2020-04-22}]{docPathOperation*}{\oarg{options}\marg{name}\marg{parameters}}
Identical to \refEnv{docPathOperation}, but without index entry.

\refEnv{docPathOperation} 的副本,但不含索引条目。
\end{docEnvironment}


\begin{docEnvironment}[doclang/environment content=command description,
  doc new={2020-04-22}]{docPathOperations}{\oarg{options}\brackets{\marg{variant1},\marg{variant2},...}}
Documents several (similar) \tikzname\ path operation variants simultaneously.
The given \meta{options} are set with \refCom{tcbset} and are valid for
all variants and the documentation text.
Every variant is described by an option set \meta{variant1}, \meta{variant2}, and so on.
The most crucial options are \refKey{/tcb/doc name} and \refKey{/tcb/doc parameter}.

同时记录了几个(类似的)\tikzname\ 路径操作变体。 给定的\meta{选项}通过|\tcbset|进行设置,并适用于所有变体和文档文本。 每个变体由一个选项集\meta{variant1}、\meta{variant2}等描述。 最关键的选项是|/tcb/doc name|和|/tcb/doc parameter|。

\begin{dispExample}
\begin{docPathOperations}[
  doc no index,   %  no index entries for this example
]
{
  {
    doc name      = rectangle,
    doc parameter = \meta{corner or cycle},
  },
  {
    doc name      = circle,
    doc parameter = \oarg{options},
  },
  {
    doc name      = ellipse,
    doc parameter = \oarg{options},
  },
}
example
\end{docPathOperations}
\end{dispExample}
\end{docEnvironment}
% % \clearpage
\begin{docCommands}[doc parameter=\oarg{options}\marg{name}]
{
  {
    doc name = docValue,
    doc updated=2020-04-23,
  },
  {
    doc name = docValue*,
  },
}
Documents a value with given \meta{name}. Typically, this is a value for a key.
The given \meta{options} are set with \refComLe{tcbset}.
This value is automatically indexed for \refComLe{docValue}
and has no index entry for \refComLe{docValue*}.

使用给定的 \meta{name} 记录一个值。通常,这是一个键的值。 给定的 \meta{options} 使用 |\tcbset| 进行设置。 此值会自动索引到 |docValue| 中,但不会在 |docValue*| 中有索引条目。
\begin{dispExample}
A feasible value for \refKeyLe{/foo/footitle} is \docValue*{foovalue}.
\end{dispExample}
\end{docCommands}
% \begin{docCommands}[doc parameter=\oarg{options}\marg{name}]
{
  {
    doc name    = docAuxCommand,
    doc updated = 2020-04-23,
  },
  {
    doc name = docAuxCommand*,
  },
}
Documents an auxiliary or minor \LaTeX\ macro with given \meta{name}
where \meta{name} is written without backslash.
The given \meta{options} are set with \refComLe{tcbset}.
This macro is automatically indexed for \refComLe{docAuxCommand}
and has no index entry for \refComLe{docAuxCommand*}.

记录一个带有给定\meta{name}的辅助或次要\LaTeX\ 宏,\meta{name}不带反斜杠。给定的\meta{options}由\refComLe{tcbset}设置。此宏将自动索引为\refComLe{docAuxCommand},而\refComLe{docAuxCommand*}没有索引条目。
\begin{dispExample}
The macro \docAuxCommand{fooaux} holds some interesting data.
\end{dispExample}
\end{docCommands}


\begin{docCommands}[doc parameter=\oarg{options}\marg{name}]
{
  {
    doc name    = docAuxEnvironment,
    doc updated = 2020-04-23,
  },
  {
    doc name = docAuxEnvironment*,
  },
}
Documents an auxiliary or minor \LaTeX\ environment with given \meta{name}.
The given \meta{options} are set with \refComLe{tcbset}.
This macro is automatically indexed indexed for \refComLe{docAuxEnvironment}
and has no index entry for \refComLe{docAuxEnvironment*}.

记录一个带有给定\meta{name}的辅助或次要的\LaTeX 环境。 给定的\meta{options}使用\refComLe{tcbset}设置。 此宏自动索引索引用于\refComLe{docAuxEnvironment},并且对于\refComLe{docAuxEnvironment*}没有索引条目。
\begin{dispExample}
The environment \docAuxEnvironment{fooauxenv} holds some interesting data.
\end{dispExample}
\end{docCommands}


\begin{docCommands}[doc parameter=\oarg{key path}\oarg{options}\marg{name}]
{
  {
    doc name    = docAuxKey,
    doc updated = 2020-04-23,
  },
  {
    doc name = docAuxKey*,
  },
}
Documents an auxiliary key with given \meta{name} and an optional \meta{key path}.
The given \meta{options} are set with \refComLe{tcbset}.
It is automatically indexed for \refComLe{docAuxKey}
and has no index entry for \refComLe{docAuxKey*}.

记录带有给定的 \meta{name} 和可选的 \meta{key path} 的辅助键。给定的 \meta{options} 通过 \refComLe{tcbset} 设置。它会自动被 \refComLe{docAuxKey} 索引,但不会有 \refComLe{docAuxKey*} 的索引条目。
\begin{dispExample}
The key \docAuxKey[foo]{fooaux} holds some interesting data.
\end{dispExample}
\end{docCommands}


\begin{docCommands}[doc parameter=\oarg{options}\marg{name}]
{
  {
    doc name    = docCounter,
    doc updated = 2020-04-23,
  },
  {
    doc name = docCounter*,
  },
}
Documents a counter with given \meta{name}.
The given \meta{options} are set with \refCom{tcbset}.
The counter is automatically indexed for \refCom{docCounter}
and has no index entry for \refCom{docCounter*}.

记录一个给定\meta{name}的计数器。 给定的\meta{options}由\refCom{tcbset}设置。 计数器会自动为\refCom{docCounter}索引,但对于\refCom{docCounter*}没有索引条目。
\begin{dispExample}
The counter \docCounter{foocounter} can be used for computation.
\end{dispExample}
\end{docCommands}


%% \clearpage
\begin{docCommands}[doc parameter=\oarg{options}\marg{name}]
{
  {
    doc name    = docLength,
    doc updated = 2020-04-23,
  },
  {
    doc name = docLength*,
  },
}
Documents a length with given \meta{name}.
The given \meta{options} are set with \refCom{tcbset}.
The length is automatically indexed for \refCom{docLength}
and has no index entry for \refCom{docLength*}.

使用给定的\meta{name}记录长度。 使用\refCom{tcbset}设置给定的\meta{options}。 该长度将自动索引\refCom{docLength}, 并且\refCom{docLength*}没有索引条目。
\begin{dispExample}
The length \docLength{foolength} can be used for computation.
\end{dispExample}
\end{docCommands}


\begin{docCommands}[doc parameter=\oarg{options}\marg{name}]
{
  {
    doc name    = docColor,
    doc updated = 2020-04-23,
  },
  {
    doc name = docColor*,
  },
}
Documents a color with given \meta{name}.
The given \meta{options} are set with \refCom{tcbset}.
The color is automatically indexed for \refCom{docColor}
and has no index entry for \refCom{docColor*}.

记录给定\meta{name}的颜色。 使用\refCom{tcbset}设置给定的\meta{options}。 该颜色将自动为\refCom{docColor}建立索引, 但对于\refCom{docColor*}没有索引条目。
\begin{dispExample}
The color \docColor{foocolor} is available.
\end{dispExample}
\end{docCommands}



\begin{docCommand}{cs}{\marg{name}}
Macro from |ltxdoc| \cite{carlisle:ltxdoc} to typeset a command word \meta{name}
where the backslash is prefixed. The library overwrites the original macro.

从 |ltxdoc| \cite{carlisle:ltxdoc} 中的宏,用于排版以反斜杠为前缀的命令词 \meta{name}。该库会覆盖原始宏。
\begin{dispExample}
This is a \cs{foocommand}.
\end{dispExample}
\end{docCommand}

\begin{docCommand}{meta}{\marg{text}}
Macro from |doc| \cite{mittelbach:2011a} to typeset a meta \meta{text}.
The library overwrites the original macro.

从 |doc| \cite{mittelbach:2011a} 到排版元 \meta{text} 的宏。该库覆盖了原始的宏。
\begin{dispExample}
This is a \meta{text}.
\end{dispExample}
\end{docCommand}


\begin{docCommand}{marg}{\marg{text}}
Macro from |ltxdoc| \cite{carlisle:ltxdoc} to typeset a \meta{text} with
curly brackets as a mandatory argument. The library overwrites the original macro.

从 |ltxdoc| \cite{carlisle:ltxdoc} 中获取宏,用花括号作为强制参数来排版 \meta{text}。该库覆盖了原始宏。
\begin{dispExample}
This is a mandatory \marg{argument}.
\end{dispExample}
\end{docCommand}

\begin{docCommand}{oarg}{\marg{text}}
Macro from |ltxdoc| \cite{carlisle:ltxdoc} to typeset a \meta{text} with
square brackets as an optional argument. The library overwrites the original macro.

从 |ltxdoc| \cite{carlisle:ltxdoc} 中的宏开始,使用方括号作为可选参数排版 \meta{text}。该库会覆盖原始宏。
\begin{dispExample}
This is an optional \oarg{argument}.
\end{dispExample}
\end{docCommand}

%% \clearpage

\begin{docCommand}{brackets}{\marg{text}}
Sets the given \meta{text} with curly brackets.

将给定的 \meta{text} 使用花括号设置。
\begin{dispExample}
Here we use \brackets{some text}.
\end{dispExample}
\end{docCommand}


{\let\xdispExample\dispExample
\let\endxdispExample\enddispExample
\begin{docEnvironment}[doc updated=2014-10-10]{dispExample}{}
Creates a colored box based on a \refEnv{tcolorbox}.
It displays the environment content as source code in the upper part
and as compiled text in the lower part of the box.
The appearance is controlled by \refKey{/tcb/documentation listing style}
and the style \refKey{/tcb/docexample}. It may be
changed by redefining this style.

基于\refEnv{tcolorbox}创建一个带颜色的框。它将环境内容分别显示在框的上部源代码和下部编译文本中。外观由\refKey{/tcb/documentation listing style} 和样式\refKey{/tcb/docexample}控制。可以通过重新定义此样式来改变它。
{
%\tcbset{before lower app={\tcbset{docexample/.style={docexample original}}}}
%\tcbset{docexample/.style={docexample original}}%
\begin{xdispExample}
\begin{dispExample}
This is a \LaTeX\ example.
\end{dispExample}
\end{xdispExample}
}
\end{docEnvironment}}


{\let\xdispExample\dispExample
\let\endxdispExample\enddispExample
\begin{docEnvironment}[doc updated=2014-10-10]{dispExample*}{\marg{options}}
The starred version of \refEnv{dispExample} takes \refEnv{tcolorbox} \meta{options}
as parameter. These \meta{options} are executed after \refKey{/tcb/docexample}.

\refEnv{dispExample} 的带星号版本接受 \refEnv{tcolorbox} 的 \meta{选项} 作为参数。这些 \meta{选项} 在 \refKey{/tcb/docexample} 之后执行。
\begin{xdispExample}
\begin{dispExample*}{sidebyside}
This is a \LaTeX\ example.
\end{dispExample*}
\end{xdispExample}
\end{docEnvironment}}


%%\clearpage
\begin{docEnvironment}{dispListing}{}
Creates a colored box based on a \refEnv{tcolorbox}.
It displays the environment content as source code.
The appearance is controlled by \refKey{/tcb/documentation listing style}
and the style \refKey{/tcb/docexample}. It may be
changed by redefining this style.

基于 \refEnv{tcolorbox} 创建一个有颜色的盒子。它将环境内容以源代码形式显示出来。外观由 \refKey{/tcb/documentation listing style} 和样式 \refKey{/tcb/docexample} 控制。可以通过重新定义此样式来进行更改。
\begin{dispExample}
\begin{dispListing}
This is a \LaTeX\ example.
\end{dispListing}
\end{dispExample}
\end{docEnvironment}

\begin{docEnvironment}{dispListing*}{\marg{options}}
The starred version of \refEnv{dispListing} takes \refEnv{tcolorbox} \meta{options}
as parameter. These \meta{options} are executed after \refKey{/tcb/docexample}.

\refEnv{dispListing} 的加星版本以 \refEnv{tcolorbox} 的 \meta{选项} 作为参数。这些 \meta{选项} 在 \refKey{/tcb/docexample} 之后执行。
\begin{dispExample}
\begin{dispListing*}{title=My listing}
This is a \LaTeX\ example.
\end{dispListing*}
\end{dispExample}
\end{docEnvironment}


% \begin{docEnvironment}{absquote}{}
% Used to typeset an abstract as quoted and small text.
% \begin{dispExample}
% \begin{absquote}
% |tcolorbox| provides an environment for colored and framed text boxes with a
% heading line. Optionally, such a box can be split in an upper and a lower part.
% \end{absquote}
% \end{dispExample}
% \end{docEnvironment}

% \clearpage
% \begin{docCommand}[doc updated=2020-04-22]{tcbmakedocSubKey}{\marg{name}\marg{key path}}
% Creates a new environment \meta{name} based on \refEnv{docKey} for the
% documentation of keys with the given \meta{key path} as root.
% The new environment \meta{name} takes the same para\-meters as \refEnv{docKey} itself.
% A second starred environment \meta{name} is also created, which is identical
% to \meta{name} but without index entry.
% \begin{dispExample}
% \tcbmakedocSubKey{docFooKey}{foo}

% \begin{docFooKey}{foodummy}{=\meta{nothing}}{no default, initially empty}
% Some key.
% \end{docFooKey}

% \begin{docFooKey*}{foo another dummy}{=\meta{nothing}}{no default, initially empty}
% Some key (not indexed).
% \end{docFooKey*}
% \end{dispExample}
% \end{docCommand}


% \begin{docCommand}[doc new=2020-04-22]{tcbmakedocSubKeys}{\marg{name}\marg{key path}}
% Creates a new environment \meta{name} based on \refEnv{docKeys} for the
% documentation of keys with the given \meta{key path} as root.
% The new environment \meta{name} takes the same para\-meters as \refEnv{docKeys} itself.
% \begin{dispExample}
% \tcbmakedocSubKeys{docFooKeys}{foo}

% \begin{docFooKeys}[
%   doc parameter   = {=\meta{nothing}},
%   doc description = {no default, initially empty},
% ]
% {
%   {
%     doc name = foodummy 2,
%   },
%   {
%     doc name = foo another dummy 2,
%     doc no index,
%   }
% }
% Some description.
% \end{docFooKeys}
% \end{dispExample}
% \end{docCommand}


% \clearpage

% \begin{docCommand}{refCom}{\marg{name}}
% References a documented \LaTeX\ macro with given \meta{name} where \meta{name} is
% written without backslash. The page reference is suppressed if it links
% to the same page.
% \begin{dispExample}
% We have created \refCom{foomakedocSubKey} as an example.
% \end{dispExample}
% \end{docCommand}

% \begin{docCommand}{refCom*}{\marg{name}}
% References a documented \LaTeX\ macro with given \meta{name} where \meta{name} is
% written without backslash. There is no page reference.
% \begin{dispExample}
% We have created \refCom*{foomakedocSubKey} as an example.
% \end{dispExample}
% \end{docCommand}


% \begin{docCommand}{refEnv}{\marg{name}}
% References a documented \LaTeX\ environment with given \meta{name}.
% The page reference is suppressed if it links to the same page.
% \begin{dispExample}
% We have created \refEnv{foocolorbox} as an example.
% \end{dispExample}
% \end{docCommand}

% \begin{docCommand}{refEnv*}{\marg{name}}
% References a documented \LaTeX\ environment with given \meta{name}.
% There is no page reference.
% \begin{dispExample}
% We have created \refEnv*{foocolorbox} as an example.
% \end{dispExample}
% \end{docCommand}


% \begin{docCommand}{refKey}{\marg{name}}
% References a documented key with given \meta{name} where \meta{name}
% is the full path name of the key.
% The page reference is suppressed if it links to the same page.
% \begin{dispExample}
% We have created \refKey{/foo/footitle} as an example.
% \end{dispExample}
% \end{docCommand}

% \begin{docCommand}{refKey*}{\marg{name}}
% References a documented key with given \meta{name} where \meta{name}
% is the full path name of the key.
% There is no page reference.
% \begin{dispExample}
% We have created \refKey*{/foo/footitle} as an example.
% \end{dispExample}
% \end{docCommand}

% \clearpage

% \begin{docCommand}[doc new=2019-09-17]{refPathOperation}{\marg{name}}
% References a documented \tikzname\ path operation with given \meta{name}.
% The page reference is suppressed if it links to the same page.
% \begin{dispExample}
% We have created \refPathOperation{fooop} as an example.
% \end{dispExample}
% \end{docCommand}

% \begin{docCommand}[doc new=2019-09-17]{refPathOperation*}{\marg{name}}
% References a documented \tikzname\ path operation with given \meta{name}.
% There is no page reference.
% \begin{dispExample}
% We have created \refPathOperation*{fooop} as an example.
% \end{dispExample}
% \end{docCommand}



% \begin{docCommand}[doc updated=2020-02-11]{refAux}{\marg{name}}
% References some auxiliary environment, key, value, or color.
% The \meta{name} is colored according to \refKey{/tcb/color hyperlink},
% if |hyperref| colorlinks are set, but there is no real link.
% \begin{dispExample}
% Some pages back, one can see \refAux{/foo/footitle} as an example.
% \end{dispExample}
% \end{docCommand}

% \begin{docCommand}[doc updated=2020-02-11]{refAuxcs}{\marg{name}}
% References some auxiliary macro \meta{name} where \meta{name} is
% written without backslash.
% The \meta{name} is colored according to \refKey{/tcb/color hyperlink},
% if |hyperref| colorlinks are set, but there is no real link.
% \begin{dispExample}
% Some pages back, one can see \refAuxcs{fooaux} as an example.
% \end{dispExample}
% \end{docCommand}


% \begin{docCommand}{colDef}{\marg{text}}
% Sets \meta{text} with the command color, see \refKey{/tcb/color command}.
% \begin{dispExample}
% This is my \colDef{text}.
% \end{dispExample}
% \end{docCommand}

% \begin{docCommand}{colOpt}{\marg{text}}
% Sets \meta{text} with the option color, see \refKey{/tcb/color option}.
% \begin{dispExample}
% This is my \colOpt{text}.
% \end{dispExample}
% \end{docCommand}

% \clearpage

% \begin{docCommand}[doc new=2019-09-18]{colFade}{\marg{text}}
% Sets \meta{text} with the fade color, see \refKey{/tcb/color fade}.
% \begin{dispExample}
% This is my \colFade{text}.
% \end{dispExample}
% \end{docCommand}


% \begin{docCommand}[doc new=2014-09-19]{tcbdocmarginnote}{\oarg{options}\marg{text}}
% Creates a |tcolorbox| note with the given \meta{text} inside the margin using
% the |marginnote| package. The style of the |tcolorbox| is predefined and can be
% altered by \refKey{/tcb/doc marginnote} and the given \meta{options}.
% \begin{dispExample}
% Some text\tcbdocmarginnote{Note A}
% which is commented by a note inside the margin.
% Alternatively to |\tcbdocmarginnote|, you can always use
% |\marginnote| with a |tcolorbox| directly.\par
% This is further text%
% \tcbdocmarginnote[colframe=blue!50!white,colback=blue!5!white]{Note B}
% with another note.
% \end{dispExample}
% \end{docCommand}

% \begin{docCommand}[doc new=2014-09-19]{tcbdocnew}{\marg{date}}
% Auxiliary macro which typesets the \refKey{/tcb/doclang/new} text with
% the given \meta{date}. It may be redefined for customization.
% \makeatletter\renewcommand*{\tcbdocnew}[1]{\kvtcb@text@new: #1}\makeatother%
% \begin{dispExample*}{sidebyside}
% \tcbdocnew{1981-10-29}.
% % Next one is displayed in the margin:
% \tcbdocmarginnote{\tcbdocnew{1978-02-09}}
% \end{dispExample*}
% \end{docCommand}

% \begin{docCommand}[doc new=2014-09-19]{tcbdocupdated}{\marg{date}}
% Auxiliary macro which typesets the \refKey{/tcb/doclang/updated} text with
% the given \meta{date}. It may be redefined for customization.
% \makeatletter\renewcommand*{\tcbdocupdated}[1]{\kvtcb@text@updated: #1}\makeatother%
% \begin{dispExample*}{sidebyside}
% \tcbdocupdated{2014-09-19}.
% \end{dispExample*}
% \end{docCommand}



\clearpage
%-------------------------------------------------------------------------------
\subsection{Entry Content Option Keys\\条目内容选项键}
\begin{docTcbKey}[][doc new={2020-04-22}]{doc name}{=\meta{name}}{no default, initially empty}
Sets the \meta{name} of the entry to document, i.e. the \meta{name} of the
command, environment, key, etc. For \refEnv{docCommand}, \refEnv{docEnvironment}, etc.
the \meta{name} is set by a mandatory parameter, but can also be set
by \refKeyLe{/tcb/doc name}.
\refKeyLe{/tcb/doc name} also sets \meta{name} to
\refKeyLe{/tcb/doc label}, \refKeyLe{/tcb/doc index},
and \refKeyLe{/tcb/doc sort index}.

设置要记录的条目的\meta{name},即命令、环境、键等的\meta{name}。对于\refEnv{docCommand}、\refEnv{docEnvironment}等,\meta{name}由必需参数设置,但也可以通过\refKeyLe{/tcb/doc name}设置。

\refKeyLe{/tcb/doc name}还将\meta{name}设置为\refKeyLe{/tcb/doc label}、\refKeyLe{/tcb/doc index}和\refKeyLe{/tcb/doc sort index}。
\begin{dispExample}
\begin{docCommands}[
doc no index,  %  no index entries for this example
doc name      = bfseries,
] {}
Font setting to bold face.
\end{docCommands}
\end{dispExample}
\end{docTcbKey}


\begin{docTcbKey}[][doc new={2020-04-22}]{doc parameter}{=\meta{parameters}}{no default, initially empty}
Sets the \meta{parameters} of the entry to document, i.e. the \meta{parameters} of the
command, environment, key, etc. For \refEnv{docCommand}, \refEnv{docEnvironment}, etc.
the \meta{parameters} is set by a mandatory option, but can also be set
by \refKeyLe{/tcb/doc parameter}.

设置文档条目的 \meta{参数},即命令、环境、键等的 \meta{参数}。对于 \refEnv{docCommand}、\refEnv{docEnvironment} 等,\meta{参数}由必选选项设置,但也可以通过 \refKeyLe{/tcb/doc parameter} 设置。
\begin{dispExample}
\begin{docCommands}[
doc no index,  %  no index entries for this example
doc name      = textbf,
doc parameter = \marg{text},
] {}
Sets \meta{text} in bold face.
\end{docCommands}
\end{dispExample}
\end{docTcbKey}



\begin{docTcbKey}[][doc new={2020-04-22}]{doc keypath}{=\meta{key path}}{no default, initially empty}
Sets the \meta{key path} of the key to document. For \refEnv{docKey}
and \refEnv{docKey*} the \meta{key path}  is set by a specialized option,
but can also be set by \refKeyLe{/tcb/doc keypath}.

将键的\meta{key path}设置为文档中的值。对于\refEnv{docKey}和\refEnv{docKey*},\meta{key path}是通过专门的选项设置的,但也可以通过\refKeyLe{/tcb/doc keypath}设置。
\begin{dispExample}
\begin{docKeys}[
doc no index,  %  no index entries for this example
doc keypath     = tikz,
doc name        = fill,
doc parameter   = \colOpt{=\meta{color}},
doc description = default is scope's color setting,
] {}
This option causes the path to be filled.
\end{docKeys}
\end{dispExample}
\end{docTcbKey}

% \clearpage

\begin{docTcbKey}{doc description}{=\meta{description}}{no default, initially empty}
Sets a (short!) additional \meta{description} for
\refEnv{docCommand}, \refEnv{docEnvironment}, or \refEnv{docPathOperation}.
Such a description is
mandatory for \refEnv{docKey}.

为\refEnv{docCommand}、\refEnv{docEnvironment}或\refEnv{docPathOperation}设置一个(简短的!)附加的\meta{description}。对于\refEnv{docKey},这样的描述是必需的。
\begin{dispExample}
\begin{docCommand*}[doc description=my description]{myCommandF}{\marg{argument}}
This is the documentation of \refCom{myCommandF} which takes one \meta{argument}.
\refCom{myCommandF} does some funny things with its \meta{argument}.

这是 \refCom{myCommandF} 的文档,它接受一个 \meta{argument}。 \refCom{myCommandF} 会对其 \meta{argument} 进行一些有趣的操作。
\end{docCommand*}
\end{dispExample}
\begin{marker}
Note that the description \meta{text} may overlap with the text on the left
hand side if too long. Linebreaks can be used inside the \meta{text}.

请注意,如果\meta{text}太长,描述文本可能会与左侧文本重叠。可以在\meta{text}中使用换行符。
\end{marker}
\end{docTcbKey}


\begin{docTcbKey}[][doc new={2019-09-18}]{doc label}{=\meta{text}}{no default, initially unset}
If used inside the option list of \refEnv{docCommand}, \refEnv{docEnvironment},
\refEnv{docKey}, etc, then \meta{text} is used
for labeling instead of the name of the definition.

如果在 \refEnv{docCommand}、\refEnv{docEnvironment}、\refEnv{docKey} 等选项列表中使用,那么 \meta{text} 被用于标记,而不是定义的名称。
\begin{dispExample}
\begin{docPathOperation*}[doc label=pathline]{-{}-}{\meta{coordinate or cycle}}
This is the documentation of \refPathOperation{pathline}.

这是 \refPathOperation{pathline} 的文档。
\end{docPathOperation*}
\end{dispExample}
\end{docTcbKey}

\begin{docTcbKey}[][doc new={2020-01-07}]{doc index}{=\meta{text}}{no default, initially unset}
If used inside the option list of \refEnv{docCommand}, \refEnv{docEnvironment},
\refEnv{docKey}, etc, then \meta{text} is used
for the index instead of the name of the definition.

如果在\refEnv{docCommand}、\refEnv{docEnvironment}、\refEnv{docKey}等选项列表中使用,那么\meta{text}将用于索引而不是定义的名称。
\begin{dispExample}
\begin{docPathOperation}[doc index=foo path (horizontal then vertical),
doc label=pathline2]{-\textbar}{\meta{coordinate or cycle}}
This is the documentation of \refPathOperation{pathline2}.

这是\refPathOperation{pathline2}的文档。
\end{docPathOperation}
\end{dispExample}
\end{docTcbKey}


\begin{docTcbKey}[][doc new={2020-04-23}]{doc sort index}{=\meta{text}}{no default, initially unset}
If used inside the option list of \refEnv{docCommand}, \refEnv{docEnvironment},
\refEnv{docKey}, etc, then \meta{text} is used
for as sort key for the index instead of the name of the definition.

如果在\refEnv{docCommand}、\refEnv{docEnvironment}、\refEnv{docKey}等选项列表中使用,那么\meta{text}将用作索引的排序关键字,而不是定义名称。
\begin{dispListing}
\begin{docCommands}[
doc name        = l_tcobox_example_tl,
doc sort index  = example_tl,  % sorted unter e like example
]{}
\end{docCommands}
\end{dispListing}
\end{docTcbKey}

% \clearpage

\begin{docTcbKey}{doc into index}{\colOpt{=true\textbar false}}{default |true|, initially |true|}
If set to |false|, no index entries are written for the main documentation
environments. The same effect is achieved by using e.\,g.\ \refEnv{docCommand*}
instead of \refEnv{docCommand}.

如果设置为|false|,则不会为主要文档环境编写索引条目。使用例如\refEnv{docCommand*}而不是\refEnv{docCommand}可以达到相同的效果。
\end{docTcbKey}


\begin{docTcbKey}[][doc new={2020-04-22}]{doc no index}{}{style, initially unset}
If set, no index entries are written for the main documentation
environments. This is a shortcut for using \refKeyLe{/tcb/doc into index}|=false|.

如果设置了此选项,则不会为主要文档环境编写索引条目。这是使用\refKeyLe{/tcb/doc into index}|=false|的快捷方式。
\end{docTcbKey}

\begin{docTcbKey}[][doc new=2014-09-19]{doc marginnote}{=\meta{options}}{no default, initially empty}
Sets style \meta{options} for the displayed box of the \refCom{tcbdocmarginnote} command.

为\refCom{tcbdocmarginnote}命令的显示框设置样式选项\meta{options}。
\begin{dispExample}
\tcbset{doc marginnote={colframe=blue!50!white,colback=blue!5!white}}%
This is some text\tcbdocmarginnote{Note A}
which is commented by a note inside the margin.

这是一些文本\tcbdocmarginnote{注释A}, 其中注释在边缘内部。
\end{dispExample}
\end{docTcbKey}

\begin{docTcbKey}[][doc new=2014-09-19]{doc new}{=\meta{date}}{style, no default}
Adds a a marginnote with a \enquote{New: \meta{date}} message at the beginning of
the upper box part. The intended use is inside the option list of
\refEnv{docCommand}, \refEnv{docEnvironment}, etc.

在上方盒子的开头添加一个带有 \enquote{New: \meta{date}} 信息的边注。旨在用于 \refEnv{docCommand}、\refEnv{docEnvironment} 等选项列表内。 
\makeatletter\renewcommand*{\tcbdocnew}[1]{\kvtcb@text@new: #1}\makeatother%
\begin{dispExample}
\begin{docCommand}[doc new=2000-01-01]{foosomething}{\marg{text}}
Some command for something.

一些用于某事的命令。
\end{docCommand}
\end{dispExample}
\end{docTcbKey}


\begin{docTcbKey}[][doc new=2014-09-19]{doc updated}{=\meta{date}}{style, no default}
Adds a marginnote with a \enquote{Updated: \meta{date}} message at the beginning of
the upper box part. See \refKeyLe{/tcb/doc new}.

在上方盒子部分的开头添加一个带有\enquote{更新于:\meta{日期}}消息的边注。请参见\refKeyLe{/tcb/doc new}。
\end{docTcbKey}


\begin{docTcbKey}[][doc new=2014-09-19]{doc new and updated}{=\marg{new date}\marg{update date}}{style, no default}
Adds a marginnote with \enquote{New: \meta{new date}} and \enquote{Updated: \meta{update date}} messages at the beginning of
the upper box part. See \refKeyLe{/tcb/doc new}.

在上部框的开始处添加带有\enquote{New: \meta{new date}}和\enquote{Updated: \meta{update date}}消息的marginnote。请参见\refKeyLe{/tcb/doc new}。
\end{docTcbKey}




\clearpage
%-------------------------------------------------------------------------------
\subsection{Entry Customization Option Keys\\条目自定义选项键}
\begin{docTcbKey}{doc left}{=\meta{length}}{no default, initially |2em|}
Sets the left hand offset of the documentation texts from
\refEnvLe{docCommand}, \refEnvLe{docEnvironment}, \refEnvLe{docKey}, etc, to \meta{length}.

将文档文本从\refEnvLe{docCommand}、\refEnvLe{docEnvironment}、\refEnvLe{docKey}等的左侧偏移设置为\meta{长度}。
\begin{dispExample}
\begin{docCommand*}[doc left=2cm,doc left indent=-2cm]{myCommandA}{\marg{argument}}
This is the documentation of \refComLe{myCommandA} which takes one \meta{argument}.
\refComLe{myCommandA} does some funny things with its \meta{argument}.

这是关于 \refComLe{myCommandA} 的文档,它接收一个 \meta{argument} 参数。 \refComLe{myCommandA} 会对它的 \meta{argument} 参数进行一些有趣的操作。
\end{docCommand*}
\end{dispExample}

\begin{dispExample}
\begin{docCommand*}[doc left indent=-2cm]{myCommandA}{\marg{argument}}
This is the documentation of \refComLe{myCommandA} which takes one \meta{argument}.
\refComLe{myCommandA} does some funny things with its \meta{argument}.
\end{docCommand*}
\end{dispExample}
\end{docTcbKey}

\begin{docTcbKey}{doc right}{=\meta{length}}{no default, initially |0em|}
Sets the right hand offset of the documentation texts from
\refEnvLe{docCommand}, \refEnvLe{docEnvironment}, \refEnvLe{docKey}, etc, to \meta{length}.

将文档文本中\refEnvLe{docCommand}、\refEnvLe{docEnvironment}、\refEnvLe{docKey}等右侧的偏移量设置为\meta{length}。
\begin{dispExample}
\begin{docCommand*}[doc right=2cm]{myCommandB}{\marg{argument}}
This is the documentation of \refComLe{myCommandB} which takes one \meta{argument}.
\refComLe{myCommandB} does some funny things with its \meta{argument}.

这是关于 \refComLe{myCommandB} 的文档,它需要一个 \meta{参数}。 \refComLe{myCommandB} 会对它的 \meta{参数} 进行一些有趣的操作。
\end{docCommand*}
\end{dispExample}
\end{docTcbKey}

\begin{docTcbKey}{doc left indent}{=\meta{length}}{no default, initially |-2em|}
Sets the left hand indent of documentation heads from
\refEnvLe{docCommand}, \refEnvLe{docEnvironment}, \refEnvLe{docKey}, etc, to \meta{length}.

将文档标题中\refEnvLe{docCommand}、\refEnvLe{docEnvironment}、\refEnvLe{docKey}等的左缩进设置为\meta{length}。
\begin{dispExample}
\begin{docCommand*}[doc left indent=2cm]{myCommandC}{\marg{argument}}
This is the documentation of \refComLe{myCommandC} which takes one \meta{argument}.
\refComLe{myCommandC} does some funny things with its \meta{argument}.

这是 \refComLe{myCommandC} 的文档,它需要一个 \meta{参数}。 \refComLe{myCommandC} 会对其 \meta{参数} 进行一些有趣的操作。
\end{docCommand*}
\end{dispExample}
\end{docTcbKey}


\begin{docTcbKey}{doc right indent}{=\meta{length}}{no default, initially |0pt|}
Sets the right hand indent of documentation heads from
\refEnvLe{docCommand}, \refEnvLe{docEnvironment}, \refEnvLe{docKey}, etc, to \meta{length}.

将文档标题(如\refEnvLe{docCommand}、\refEnvLe{docEnvironment}、\refEnvLe{docKey}等)的右缩进设置为\meta{length}。
\begin{dispExample}
\begin{docCommand*}[doc right indent=-10mm,doc right=10mm,
doc description=test value]{myCommandD}{\marg{argument}}
This is the documentation of \refComLe{myCommandD} which takes one \meta{argument}.
\refComLe{myCommandD} does some funny things with its \meta{argument}.

这是关于 \refComLe{myCommandD} 的文档,它接受一个 \meta{argument}。 \refComLe{myCommandD} 会对它的 \meta{argument} 进行一些有趣的操作。
\end{docCommand*}
\end{dispExample}
\end{docTcbKey}

% \clearpage
The head lines of the main documentation environments \refEnvLe{docCommand},
\refEnvLe{docEnvironment}, \refEnvLe{docKey}, etc, are |tcolorbox|es inside a
\refEnvLe{tcbraster}.
Options to the surrounding |tcbraster|s and the embedded
|tcolorbox|es can be given using the following keys.

主要文档环境的标题 \refEnvLe{docCommand}、\refEnvLe{docEnvironment}、\refEnvLe{docKey} 等是位于 \refEnvLe{tcbraster} 中的 |tcolorbox|。 可以使用以下键来给周围的 |tcbraster| 和嵌入的 |tcolorbox| 指定选项。
\begin{docTcbKeys}[
doc name        = doc raster command,
doc parameter   = {=\meta{options}},
doc description = {no default, initially empty},
doc new         = 2020-04-24,
]{}
Sets \meta{options} for the surrounding \refEnvLe{tcbraster} of\\
\refEnvLe{docCommand}, \refEnvLe{docCommand*}, and \refEnvLe{docCommands}.

为 \refEnvLe{docCommand}、\refEnvLe{docCommand*} 和 \refEnvLe{docCommands} 所包围的 \refEnvLe{tcbraster} 设置 \meta{options}。
\begin{dispExample}
\tcbset{doc raster command={raster before skip=7mm,raster after skip=0mm}}

The is an example text.

\begin{docCommand*}{myCommandI}{\marg{argument}}
This is the documentation of \refComLe{myCommandI} which takes one \meta{argument}.
\refComLe{myCommandI} does some funny things with its \meta{argument}.

这是关于 \refComLe{myCommandI} 的文档,它需要一个 \meta{argument} 作为参数。 \refComLe{myCommandI} 会对它的 \meta{argument} 进行一些有趣的处理。
\end{docCommand*}
\end{dispExample}

\end{docTcbKeys}


\begin{docTcbKeys}[
doc name        = doc raster environment,
doc parameter   = {=\meta{options}},
doc description = {no default, initially empty},
doc new         = 2020-04-24,
]{}
Sets \meta{options} for the surrounding \refEnvLe{tcbraster} of\\
\refEnvLe{docEnvironment}, \refEnvLe{docEnvironment*}, and \refEnvLe{docEnvironments}.

为 \refEnvLe{docEnvironment}、\refEnvLe{docEnvironment*} 和 \refEnvLe{docEnvironments} 的周围 \refEnvLe{tcbraster} 集合 \meta{options}。
\end{docTcbKeys}


\begin{docTcbKeys}[
doc name        = doc raster key,
doc parameter   = {=\meta{options}},
doc description = {no default, initially empty},
doc new         = 2020-04-24,
]{}
Sets \meta{options} for the surrounding \refEnvLe{tcbraster} of\\
\refEnvLe{docKey}, \refEnvLe{docKey*}, and \refEnvLe{docKeys}.

为 \refEnvLe{docKey}、\refEnvLe{docKey*} 和 \refEnvLe{docKeys} 的周围的 \refEnvLe{tcbraster} 设置 \meta{options}。
\end{docTcbKeys}


\begin{docTcbKeys}[
doc name        = doc raster path,
doc parameter   = {=\meta{options}},
doc description = {no default, initially empty},
doc new         = 2020-04-24,
]{}
Sets \meta{options} for the surrounding \refEnvLe{tcbraster} of\\
\refEnvLe{docPathOperation}, \refEnvLe{docPathOperation*}, and \refEnvLe{docPathOperations}.

为\refEnvLe{docPathOperation}、\refEnvLe{docPathOperation*}和\refEnvLe{docPathOperations}周围的\refEnvLe{tcbraster}设置\meta{选项}。
\end{docTcbKeys}


\begin{docTcbKeys}[
doc name        = doc raster,
doc parameter   = {=\meta{options}},
doc description = {no default, initially empty},
doc new         = 2020-04-24,
]{}
Shortcut for setting the same \meta{options} for
\refKeyLe{/tcb/doc raster command}, \refKeyLe{/tcb/doc raster environment},
\refKeyLe{/tcb/doc raster key}, and \refKeyLe{/tcb/doc raster path}.

设置相同\meta{选项}的快捷方式,适用于\refKeyLe{/tcb/doc raster command}、\refKeyLe{/tcb/doc raster environment}、\refKeyLe{/tcb/doc raster key}和\refKeyLe{/tcb/doc raster path}。
\end{docTcbKeys}


\begin{docTcbKey}{doc head command}{=\meta{options}}{no default, initially empty}
Sets \meta{options} for the head line of
\refEnvLe{docCommand}, \refEnvLe{docCommand*}, and \refEnvLe{docCommands}.

为 \refEnvLe{docCommand}、\refEnvLe{docCommand*} 和 \refEnvLe{docCommands} 的页眉设置 \meta{options}。
\begin{dispExample}
\tcbset{doc head command={interior style={fill,left color=red!20!white,
right color=blue!20!white}}}

\begin{docCommand*}{myCommandE}{\marg{argument}}
This is the documentation of \refComLe{myCommandE} which takes one \meta{argument}.
\refComLe{myCommandE} does some funny things with its \meta{argument}.

这是 \refComLe{myCommandE} 的文档,它接受一个 \meta{argument} 参数。 \refComLe{myCommandE} 对它的 \meta{argument} 做了一些有趣的事情。
\end{docCommand*}
\end{dispExample}
\end{docTcbKey}


% \clearpage

\begin{docTcbKey}{doc head environment}{=\meta{options}}{no default, initially empty}
Sets \meta{options} for the head line of
\refEnvLe{docEnvironment}, \refEnvLe{docEnvironment*}, and \refEnvLe{docEnvironments}.

为 \refEnvLe{docEnvironment}、\refEnvLe{docEnvironment*} 和 \refEnvLe{docEnvironments} 的页眉设置 \meta{选项}。
\begin{dispExample}
\tcbset{doc head environment={beamer,boxsep=2pt,arc=2pt,colback=green!20!white}}

\begin{docEnvironment*}{myEnvironment}{\marg{argument}}
This is the documentation of \refEnvLe{myEnvironment} which
takes one \meta{argument}.

这是 \refEnvLe{myEnvironment} 的文档,它接受一个 \meta{argument}。
\end{docEnvironment*}
\end{dispExample}
\end{docTcbKey}

\begin{docTcbKey}{doc head key}{=\meta{options}}{no default, initially empty}
Sets \meta{options} for the head line of
\refEnvLe{docKey}, \refEnvLe{docKey*}, and \refEnvLe{docKeys}.

为 \refEnvLe{docKey}、\refEnvLe{docKey*} 和 \refEnvLe{docKeys} 的页眉设置 \meta{选项}。
\begin{dispExample}
\tcbset{doc head key={boxsep=4pt,arc=4pt,boxrule=0.6pt,
frame style=fill,interior style=fill,colframe=green!50!black}}

\begin{docKey}{/foo/myKey}{}{no value}
This is the documentation of \refKeyLe{/foo/myKey}.

这是 \refKeyLe{/foo/myKey} 的文档。
\end{docKey}
\end{dispExample}
\end{docTcbKey}


\begin{docTcbKey}[][doc new=2019-09-18]{doc head path}{=\meta{options}}{no default, initially empty}
Sets \meta{options} for the head line of
\refEnvLe{docPathOperation}, \refEnvLe{docPathOperation*}, and \refEnvLe{docPathOperations}.

为 \refEnvLe{docPathOperation}、\refEnvLe{docPathOperation*} 和 \refEnvLe{docPathOperations} 的标题行设置 \meta{options}。
\begin{dispExample}
\tcbset{doc head command={interior style={fill,left color=red!7!white,
right color=blue!7!white}}}

\begin{docPathOperation*}{-{}-}{\meta{coordinate or cycle}}
This is the documentation of \refPathOperation{-{}-}.

这是 \refPathOperation{-{}-} 的文档。
\end{docPathOperation*}
\end{dispExample}
\end{docTcbKey}


\begin{docTcbKey}[][doc updated=2019-09-18]{doc head}{=\meta{options}}{no default, initially empty}
Shortcut for setting the same \meta{options} for
\refKeyLe{/tcb/doc head command}, \refKeyLe{/tcb/doc head environment},
\refKeyLe{/tcb/doc head key}, and \refKeyLe{/tcb/doc head path}.

设置相同的\meta{选项}的快捷方式,适用于\refKeyLe{/tcb/doc head command}、\refKeyLe{/tcb/doc head environment}、\refKeyLe{/tcb/doc head key}和\refKeyLe{/tcb/doc head path}。
\end{docTcbKey}


% \clearpage

The description texts of the main documentation environments \refEnvLe{docCommand},
\refEnvLe{docEnvironment}, \refEnvLe{docKey}, etc, are set in a compact form without
indention and |parskip=0pt|. This settings can overruled by using the following
keys to insert code before (or after) the description texts.

主要文档环境 \refEnvLe{docCommand}、\refEnvLe{docEnvironment}、\refEnvLe{docKey} 等的描述文本以紧凑形式设置,没有缩进和 |parskip=0pt|。可以使用以下键在描述文本之前(或之后)插入代码来覆盖此设置。
\begin{docTcbKey}[][doc new=2015-10-09]{before doc body command}{=\meta{code}}{no default, initially empty}
Executes \meta{code} before the description texts
of \refEnvLe{docCommand} and \refEnvLe{docCommand*}.

在 \refEnvLe{docCommand} 和 \refEnvLe{docCommand*} 的描述文本之前执行 \meta{code}。
\begin{dispExample}
\tcbset{before doc body command={%
\setlength{\parindent}{2.5em}%
\setlength{\parskip}{1ex plus 0.75ex minus 0.25ex}%
}}

\begin{docCommand*}{myCommandG}{\marg{argument}}
This is the documentation of \refComLe{myCommandG} which takes one \meta{argument}.
\refComLe{myCommandG} does some funny things with its \meta{argument}.

这是 \refComLe{myCommandG} 的文档,它接受一个 \meta{argument} 参数。 \refComLe{myCommandG} 会对其 \meta{argument} 参数进行一些有趣的操作。
\end{docCommand*}
\end{dispExample}
\end{docTcbKey}


\begin{docTcbKey}[][doc new=2015-10-09]{after doc body command}{=\meta{code}}{no default, initially empty}
Executes \meta{code} after the description texts
of \refEnvLe{docCommand} and \refEnvLe{docCommand*}.

在\refEnvLe{docCommand}和\refEnvLe{docCommand*}的描述文本后执行\meta{code}。
\begin{dispExample}
\tcbset{after doc body command={%
\hfill\nolinebreak[1]\hspace*{\fill}\textcolor{red}{$\diamondsuit$}%
}}

\begin{docCommand*}{myCommandH}{\marg{argument}}
This is the documentation of \refComLe{myCommandH} which takes one \meta{argument}.
\refComLe{myCommandH} does some funny things with its \meta{argument}.

这是\refComLe{myCommandH}的文档,它需要一个\meta{argument}参数。 \refComLe{myCommandH}会对它的\meta{argument}参数做一些有趣的事情。
\end{docCommand*}
\end{dispExample}
\end{docTcbKey}


\begin{docTcbKey}[][doc new=2015-10-09]{before doc body environment}{=\meta{code}}{no default, initially empty}
Executes \meta{code} before the description texts
of \refEnvLe{docEnvironment} and \refEnvLe{docEnvironment*}.

在\refEnvLe{docEnvironment}和\refEnvLe{docEnvironment*}的描述文本之前执行\meta{code}。
\end{docTcbKey}

\begin{docTcbKey}[][doc new=2015-10-09]{after doc body environment}{=\meta{code}}{no default, initially empty}
Executes \meta{code} after the description texts
of \refEnvLe{docEnvironment} and \refEnvLe{docEnvironment*}.

在 \refEnvLe{docEnvironment} 和 \refEnvLe{docEnvironment*} 的描述文本之后执行 \meta{code}。
\end{docTcbKey}


\begin{docTcbKey}[][doc new=2015-10-09]{before doc body key}{=\meta{code}}{no default, initially empty}
Executes \meta{code} before the description texts
of \refEnvLe{docKey} and \refEnvLe{docKey*}.

在\refEnvLe{docKey}和\refEnvLe{docKey*}的描述文本之前执行\meta{code}。
\end{docTcbKey}

\begin{docTcbKey}[][doc new=2015-10-09]{after doc body key}{=\meta{code}}{no default, initially empty}
Executes \meta{code} after the description texts
of \refEnvLe{docKey} and \refEnvLe{docKey*}.

在 \refEnvLe{docKey} 和 \refEnvLe{docKey*} 的描述文本之后执行 \meta{code}。
\end{docTcbKey}

% \clearpage

\begin{docTcbKey}[][doc new=2019-09-18]{before doc body path}{=\meta{code}}{no default, initially empty}
Executes \meta{code} before the description texts
of \refEnvLe{docPathOperation} and \refEnvLe{docPathOperation*}.

在 \refEnvLe{docPathOperation} 和 \refEnvLe{docPathOperation*} 的描述文本之前执行 \meta{code}。
\end{docTcbKey}

\begin{docTcbKey}[][doc new=2019-09-18]{after doc body path}{=\meta{code}}{no default, initially empty}
Executes \meta{code} after the description texts
of \refEnvLe{docPathOperation} and \refEnvLe{docPathOperation*}.

在\refEnvLe{docPathOperation}和\refEnvLe{docPathOperation*}的描述文本之后执行\meta{code}。
\end{docTcbKey}


\begin{docTcbKey}[][doc new and updated={2015-10-09}{2019-09-18}]{before doc body}{=\meta{options}}{no default, initially empty}
Shortcut for setting the same \meta{options} for
\refKeyLe{/tcb/before doc body command}, \refKeyLe{/tcb/before doc body environment},
\refKeyLe{/tcb/before doc body key}, and \refKeyLe{/tcb/before doc body path}.

设置相同\meta{选项}的快捷方式,适用于\refKeyLe{/tcb/before doc body command}、\refKeyLe{/tcb/before doc body environment}、\refKeyLe{/tcb/before doc body key}和\refKeyLe{/tcb/before doc body path}。
\end{docTcbKey}

\begin{docTcbKey}[][doc new and updated={2015-10-09}{2019-09-18}]{after doc body}{=\meta{options}}{no default, initially empty}
Shortcut for setting the same \meta{options} for
\refKeyLe{/tcb/after doc body command}, \refKeyLe{/tcb/after doc body environment},
\refKeyLe{/tcb/after doc body key}, and \refKeyLe{/tcb/after doc body path}.

为 \refKeyLe{/tcb/after doc body command}、\refKeyLe{/tcb/after doc body environment}、\refKeyLe{/tcb/after doc body key} 和 \refKeyLe{/tcb/after doc body path} 设置相同的 \meta{选项} 的快捷方式。
\end{docTcbKey}

\clearpage
\subsection{General Customization Option Keys\\常规自定义选项键}
\begin{docTcbKey}[][doc updated=2015-03-16]{docexample}{}{style, no value}
Sets the style for \refEnvLe{dispExample} and \refEnvLe{dispListing}
with the colors |ExampleBack| and |ExampleFrame|.
To change the appearance of the examples, this style can be
redefined.

使用颜色 |ExampleBack| 和 |ExampleFrame| 设置 \refEnvLe{dispExample} 和 \refEnvLe{dispListing} 的样式。要更改示例的外观,可以重新定义此样式。
\begin{dispListing}
% Predefined style:
\tcbset{
docexample/.style={colframe=ExampleFrame,colback=ExampleBack,
before skip=\medskipamount,after skip=\medskipamount,
fontlower=\footnotesize}
}
\end{dispListing}
\end{docTcbKey}

\begin{docTcbKey}{documentation listing options}{=\meta{key list}}{no default,\\\hspace*{\fill} initially |style=tcbdocumentation|}
Sets the options from the package |listings| \cite{hoffmann:listings}.
They are used inside \refEnvLe{dispExample} and \refEnvLe{dispListing} to typeset
the listings. Note that this is not identical to the key
\refKeyLe{/tcb/listing options} which is used for \enquote{normal} listings.\\
Used for \refKeyLe{/tcb/listing engine}|=listings| only.

从包|listings|\cite{hoffmann:listings}中设置选项。它们用于在\refEnvLe{dispExample}和\refEnvLe{dispListing}中排版代码清单。请注意,这与用于“常规”清单的键\refKeyLe{/tcb/listing options}不完全相同。仅用于\refKeyLe{/tcb/listing engine}|=listings|。
\end{docTcbKey}

\begin{docTcbKey}{documentation listing style}{=\meta{listing style}}{no default, initially |tcbdocumentation|}
Abbreviation for |documentation listing options={style=...}|.
This key sets a \meta{style}
for the |listings| package, see \cite{hoffmann:listings}.
Note that this is not identical to the key
\refKeyLe{/tcb/listing style} which is used for \enquote{normal} listings.\\
Used for \refKeyLe{/tcb/listing engine}|=listings| only.

缩写为|documentation listing options={style=...}|。 此键设置|listings|包的\meta{style},参见\cite{hoffmann:listings}。 请注意,这与用于“正常”列表的键\refKeyLe{/tcb/listing style}不完全相同。\ 仅用于\refKeyLe{/tcb/listing engine}|=listings|。
\end{docTcbKey}

\begin{docTcbKey}{documentation minted options}{=\meta{key list}}{no default,\\\hspace*{\fill} initially |tabsize=2,fontsize=\textbackslash small|}
Sets the options from the package |minted| \cite{poore:minted}
which are used during typesetting of the listing, if used.
Note that this is not identical to the key
\refKeyLe{/tcb/minted options} which is used for \enquote{normal} listings.\\
Used for \refKeyLe{/tcb/listing engine}|=minted| only.

如果使用,从包 |minted| \cite{poore:minted} 设置用于清单排版的选项。请注意,这与用于“常规”清单的键 \refKeyLe{/tcb/minted options} 不完全相同。仅用于 \refKeyLe{/tcb/listing engine}|=minted|。
\end{docTcbKey}

\begin{docTcbKey}{documentation minted style}{=\meta{key list}}{no default, initially unset}
Sets a \meta{style} known to |Pygments| \cite{pygments:web} for
the package |minted| \cite{poore:minted}, if used.
Note that this is not identical to the key
\refKeyLe{/tcb/minted style} which is used for \enquote{normal} listings.\\
Used for \refKeyLe{/tcb/listing engine}|=minted| only.

如果使用了 |minted| \cite{poore:minted} 包,则为其设置在 |Pygments| \cite{pygments:web} 中已知的 \meta{style}。请注意,这与用于“常规”列表的关键字 \refKeyLe{/tcb/minted style} 不同。仅用于 \refKeyLe{/tcb/listing engine}|=minted|。
\end{docTcbKey}

\begin{docTcbKey}[][doc new=2017-04-24]{documentation minted language}{=\meta{programming language}}{no default, initially |latex|}
Sets a \meta{programming language} known to |Pygments| \cite{pygments:web}
for the package |minted| \cite{poore:minted}, if used.
Note that this is not identical to the key
\refKeyLe{/tcb/minted language} which is used for \enquote{normal} listings.\\
Used for \refKeyLe{/tcb/listing engine}|=minted| only.

如果使用了包|minted|\cite{poore:minted},则设置已知于|Pygments|\cite{pygments:web}的\meta{编程语言}。 请注意,这与用于“普通”列表的关键字\refKeyLe{/tcb/minted language}不同。仅用于\refKeyLe{/tcb/listing engine}|=minted|。
\end{docTcbKey}


\begin{marker}
The following two keys are deprecated and without function (v3.50 and above).
Use \refKeyLe{/tcb/before} and \refKeyLe{/tcb/after} with appropriate values
instead. Also see \refKeyLe{/tcb/docexample}.

以下两个键已被弃用且无功能(v3.50及以上版本)。 请使用适当的值替换\refKeyLe{/tcb/before}和\refKeyLe{/tcb/after}。 另请参阅\refKeyLe{/tcb/docexample}。
\end{marker}

\begin{docTcbKey}[][doc updated=2015-03-16]{before example}{=\meta{macros}}{no default, initially empty}
\smallskip\begin{deprecated}
Sets the \meta{macros} which are executed before \refEnvLe{dispExample} and \refEnvLe{dispListing}
additional to \refKeyLe{/tcb/before}.

设置在 \refEnvLe{dispExample} 和 \refEnvLe{dispListing} 之前执行的 \meta{macros},除了 \refKeyLe{/tcb/before}。
\end{deprecated}
\end{docTcbKey}

% \enlargethispage*{1cm}

\begin{docTcbKey}{after example}{=\meta{macros}}{no default, initially empty}
\smallskip\begin{deprecated}
Sets the \meta{macros} which are executed after \refEnvLe{dispExample} and \refEnvLe{dispListing}
additional to \refKeyLe{/tcb/after}.

设置在\refEnvLe{dispExample}和\refEnvLe{dispListing}之后执行的\meta{宏},除了\refKeyLe{/tcb/after}之外的附加内容。
\end{deprecated}
\end{docTcbKey}

%%\clearpage
\begin{docTcbKey}[][doc new=2017-04-25]{keywords bold}{\colOpt{=true\textbar false}}{default |true|, initially |true|}
Keyword used in \refEnvLe{docEnvironment}, \refEnvLe{docCommand}, etc. are printed
boldface (or not). Since the typewriter font is used, the effect may be
invisible with Computer Modern fonts or similar which do not
have a bold variant. Note that references to keywords are not printed boldface at all.

在\refEnvLe{docEnvironment},\refEnvLe{docCommand}等中使用的关键字以粗体(或不以粗体)打印。由于使用了打字机字体,因此在没有粗体变体的计算机现代字体或类似字体中效果可能看不见。请注意,对关键字的引用根本不以粗体打印。
\begin{dispExample*}{sidebyside}
\LARGE
\docAuxCommand{fooaux}, \refComLe{tcbset}

\tcbset{keywords bold=false}
\docAuxCommand{fooaux}, \refComLe{tcbset}
\end{dispExample*}
\end{docTcbKey}



\begin{docTcbKey}[][doc new=2015-01-09]{index command}{=\meta{macro}}{no default, initially \cs{index}}
Replaces the internally used \cs{index} macro by the given \meta{macro}.
The \meta{macro} has to take one mandatory argument like \cs{index}.
This option is mutually exclusive with \refKeyLe{/tcb/index command name}.

将内部使用的\cs{index}宏替换为给定的\meta{宏}。 \meta{宏}必须像\cs{index}一样需要一个必选参数。 此选项与\refKeyLe{/tcb/index command name}互斥。
\begin{dispListing}
\tcbset{index command=\myindexcommand}
\end{dispListing}
\end{docTcbKey}


\begin{docTcbKey}[][doc new=2015-01-09]{index command name}{=\meta{name}}{no default, initially unset}
Replaces the internally used \cs{index} macro by
\mbox{\cs{index}\texttt{[\meta{name}]}}, i.e.\ 
\mbox{\cs{index}\texttt{\textbraceleft\ldots\textbraceright}} is replaced by
\mbox{\cs{index}\texttt{[\meta{name}]\textbraceleft\ldots\textbraceright}}.
This option is intended to be used with |imakeidx| and is
mutually exclusive with \refKeyLe{/tcb/index command}.

将内部使用的\cs{index}宏替换为\mbox{\cs{index}\texttt{[\meta{name}]}},即\mbox{\cs{index}\texttt{\textbraceleft\ldots\textbraceright}}被替换为\mbox{\cs{index}\texttt{[\meta{name}]\textbraceleft\ldots\textbraceright}}。此选项旨在与|imakeidx|一起使用,并且与\refKeyLe{/tcb/index command}互斥。
\begin{dispListing}
\tcbset{index command name=mydoc}
\end{dispListing}
\end{docTcbKey}



\begin{docTcbKey}{index format}{=\meta{format}}{no default, initially |pgf|}
Determines the basic \meta{format} of the generated index.
Feasible values are:

确定生成索引的基本\meta{格式}。 可行的值有:
\begin{itemize}
\item\docValue{pgfsection}: The index is formatted like in the |pgf| documentation (as a section).
\item\docValue{pgfchapter}: The index is formatted like in the |pgf| documentation (as a chapter).
\item\docValue{pgf}: Alias for |pgfsection|.
\item\docValue{doc}: The index is assumed to be formatted by |doc| or |ltxdoc|. The usage of |makeindex|
with |-s gind.ist| is assumed. The package |hypdoc| has to be loaded
\emph{before} |tcolorbox|. Only a limited set of customizations will
work! This option cannot be unset when used!
\item\docValue{off}: The index is not formatted by |tcolorbox|. Use this, if
the index is formatted by other package like |imakeidx|.
\end{itemize}
\end{docTcbKey}

% \begin{itemize} \item \docValue{pgfsection}:索引的格式与|pgf|文档一样(作为一个章节)。 \item \docValue{pgfchapter}:索引的格式与|pgf|文档一样(作为一个章)。 \item \docValue{pgf}:是|pgfsection|的别名。 \item \docValue{doc}:假定索引由|doc|或|ltxdoc|格式化。假定使用|makeindex|和|-s gind.ist|。必须在|tcolorbox|之前加载|hypdoc|宏包。仅有限的自定义将起作用!使用此选项时无法取消! \item \docValue{off}:索引不由|tcolorbox|格式化。如果索引由其他宏包(如|imakeidx|)格式化,则使用此选项。 \end{itemize}


\begin{docTcbKey}{index actual}{=\meta{character}}{no default, initially |@|}
Sets the character for \enquote{actual} in automatic indexing.

设置自动索引中“实际”的字符。
\end{docTcbKey}

\begin{docTcbKey}{index quote}{=\meta{character}}{no default, initially |"|}
Sets the character for \enquote{quote} in automatic indexing.

设置自动索引中引用符号的字符。
\end{docTcbKey}

\begin{docTcbKey}{index level}{=\meta{character}}{no default, initially |!|}
Sets the character for \enquote{level} in automatic indexing.

在自动索引中设置“级别”字符。
\end{docTcbKey}

\begin{docTcbKey}{index default settings}{}{style, no value}
Sets the |makeindex| default values for
\refKeyLe{/tcb/index actual},
\refKeyLe{/tcb/index quote}, and
\refKeyLe{/tcb/index level}.

设置 |makeindex| 的默认值,包括 \refKeyLe{/tcb/index actual}、\refKeyLe{/tcb/index quote} 和 \refKeyLe{/tcb/index level}。
\end{docTcbKey}

% \enlargethispage*{1cm}

\begin{docTcbKey}{index german settings}{}{style, no value}
Sets the |makeindex| values recommended for German language texts.
This is identical to setting the following:

设置适用于德语文本的 |makeindex| 值建议。这与以下设置相同:
\begin{dispListing}
\tcbset{index actual={=},index quote={!},index level={>}}
\end{dispListing}
\end{docTcbKey}

% \clearpage

\begin{docTcbKey}{index annotate}{\colOpt{=true\textbar false}}{default |true|, initially |true|}
If set to |true|, the index entries are annotated with short descriptions
given by \refKeyLe{/tcb/doclang/environment}, \refKeyLe{/tcb/doclang/key},
and others.

如果设置为|true|,索引条目将用\refKeyLe{/tcb/doclang/environment}、\refKeyLe{/tcb/doclang/key}和其他短描述进行注释。
\end{docTcbKey}

\begin{docTcbKey}{index colorize}{\colOpt{=true\textbar false}}{default |true|, initially |false|}
If set to |true|, the index entries colorized according to the color
settings given by \refKeyLe{/tcb/color environment}, \refKeyLe{/tcb/color key},
and others.

如果设置为|true|,则索引条目将根据\refKeyLe{/tcb/color environment}、\refKeyLe{/tcb/color key}和其他颜色设置进行着色。
\end{docTcbKey}


\begin{docTcbKey}{color command}{=\meta{color}}{no default, initially |Definition|}
Sets the highlight color used by macro definitions.

设置宏定义使用的高亮颜色。
\end{docTcbKey}

\begin{docTcbKey}{color environment}{=\meta{color}}{no default, initially |Definition|}
Sets the highlight color used by environment definitions.

设置环境定义中使用的高亮颜色。
\end{docTcbKey}

\begin{docTcbKey}{color key}{=\meta{color}}{no default, initially |Definition|}
Sets the highlight color used by key definitions.

设置键定义使用的高亮颜色。
\end{docTcbKey}

\begin{docTcbKey}[][doc new={2019-09-18}]{color path}{=\meta{color}}{no default, initially |Definition|}
Sets the highlight color used by \tikzname\ path operation definitions.

设置\tikzname 路径操作定义使用的高亮颜色。
\end{docTcbKey}

\begin{docTcbKey}{color value}{=\meta{color}}{no default, initially |Definition|}
Sets the highlight color used by value definitions.

设置值定义使用的高亮颜色。
\end{docTcbKey}

\begin{docTcbKey}[][doc new={2015-01-08}]{color counter}{=\meta{color}}{no default, initially |Definition|}
Sets the highlight color used by counter definitions.

设置计数器定义使用的高亮颜色。
\end{docTcbKey}

\begin{docTcbKey}[][doc new={2015-01-08}]{color length}{=\meta{color}}{no default, initially |Definition|}
Sets the highlight color used by length definitions.

设置长度定义中使用的高亮颜色。
\end{docTcbKey}

\begin{docTcbKey}{color color}{=\meta{color}}{no default, initially |Definition|}
Sets the highlight color used by color definitions.

设置颜色定义使用的高亮颜色。
\end{docTcbKey}

\begin{docTcbKey}[][doc updated={2019-09-18}]{color definition}{=\meta{color}}{no default, initially |Definition|}
Sets the highlight color for \refKeyLe{/tcb/color command}, \refKeyLe{/tcb/color environment},
\refKeyLe{/tcb/color key}, \refKeyLe{/tcb/color path}, \refKeyLe{/tcb/color value}, \refKeyLe{/tcb/color counter},
\refKeyLe{/tcb/color length}, and \refKeyLe{/tcb/color color}.

设置 \refKeyLe{/tcb/color command}、\refKeyLe{/tcb/color environment}、\refKeyLe{/tcb/color key}、\refKeyLe{/tcb/color path}、\refKeyLe{/tcb/color value}、\refKeyLe{/tcb/color counter}、\refKeyLe{/tcb/color length} 和 \refKeyLe{/tcb/color color} 的高亮颜色。
\end{docTcbKey}

\begin{docTcbKey}{color option}{=\meta{color}}{no default, initially |Option|}
Sets the color used for optional arguments.

设置用于可选参数的颜色。
\end{docTcbKey}

\begin{docTcbKey}{color fade}{=\meta{color}}{no default, initially |Fade|}
Sets the color used for faded text like \colFade{\textbackslash path}
in \refEnvLe{docPathOperation}.

设置在\refEnvLe{docPathOperation} 中使用\colFade{\textbackslash path} 的淡化文本的颜色。
\end{docTcbKey}


\begin{docTcbKey}{color hyperlink}{=\meta{color}}{no default, initially |Hyperlink|}
Sets the color for all hyper-links, i.\,e. all internal and external links.

设置所有超链接的颜色,即所有内部和外部链接。
\end{docTcbKey}




% \clearpage
% %-------------------------------------------------------------------------------
\subsection{Language Option Keys\\语言选项键}
The following keys are provided for language specific settings.
The English language is predefined.

以下键可用于设置特定语言。英语是预定义的。
\begin{docTcbKey}{english language}{}{style, no value}
将所有语言特定设置设置为英语。

Sets all language specific settings to English.
\end{docTcbKey}

\begin{langTcbKey}{color}{=\meta{text}}{no default, initially |color|}
用于颜色索引的文本。

Text used in the index for colors.
\end{langTcbKey}

\begin{langTcbKey}{colors}{=\meta{text}}{no default, initially |Colors|}
颜色索引中的标题文本。

Heading text in the index for colors.
\end{langTcbKey}

\begin{langTcbKey}[][doc new={2015-01-08}]{counter}{=\meta{text}}{no default, initially |counter|}
用于计数器索引的文本。

Text used in the index for counters.
\end{langTcbKey}

\begin{langTcbKey}[][doc new={2015-01-08}]{counters}{=\meta{text}}{no default, initially |Counters|}
计数器索引中的标题文本。

Heading text in the index for counters.
\end{langTcbKey}

\begin{langTcbKey}{environment}{=\meta{text}}{no default, initially |environment|}
用于环境索引的文本。

Text used in the index for environments.
\end{langTcbKey}

\begin{langTcbKey}{environments}{=\meta{text}}{no default, initially |Environments|}
环境索引中的标题文本。

Heading text in the index for environments.
\end{langTcbKey}

\begin{langTcbKey}{environment content}{=\meta{text}}{no default, initially |environment content|}
\refEnvLe{docEnvironment} 中使用的文本。

Text used in \refEnvLe{docEnvironment}.
\end{langTcbKey}

\begin{langTcbKey}{index}{=\meta{text}}{no default, initially |Index|}
索引的标题文本。

Heading text for the index.
\end{langTcbKey}

\begin{langTcbKey}{key}{=\meta{text}}{no default, initially |key|}
用于键索引的文本。

Text used in the index for keys.
\end{langTcbKey}

\begin{langTcbKey}{keys}{=\meta{text}}{no default, initially |Keys|}
用于键索引中的标题文本。

Heading text used in the index for keys.
\end{langTcbKey}

\begin{langTcbKey}[][doc new={2015-01-08}]{length}{=\meta{text}}{no default, initially |length|}
长度索引中使用的文本。

Text used in the index for lengths.
\end{langTcbKey}

\begin{langTcbKey}[][doc new={2015-01-08}]{lengths}{=\meta{text}}{no default, initially |Lengths|}
长度索引中的标题文本。

Heading text in the index for lengths.
\end{langTcbKey}

\begin{langTcbKey}[][doc new={2014-09-19}]{new}{=\meta{text}}{no default, initially |New|}
新内容的公告文本。

Announcement text for new content.
\end{langTcbKey}

\begin{langTcbKey}[][doc new={2019-09-18}]{path}{=\meta{text}}{no default, initially |path operation|}
路径操作索引中使用的文本。

Text used in the index for path operations.
\end{langTcbKey}

\begin{langTcbKey}[][doc new={2019-09-18}]{paths}{=\meta{text}}{no default, initially |Path operations|}
路径操作索引中的标题文本。

Heading text in the index for path operations.
\end{langTcbKey}

\begin{langTcbKey}{pageshort}{=\meta{text}}{no default, initially |P.|}
页面引用的简短文本。

Short text for page references.
\end{langTcbKey}

\begin{langTcbKey}[][doc new={2014-09-19}]{updated}{=\meta{text}}{no default, initially |Updated|}
更新内容的公告文本。

Announcement text for updated content.
\end{langTcbKey}

\begin{langTcbKey}{value}{=\meta{text}}{no default, initially |value|}
用于值索引的文本。

Text used in the index for values.
\end{langTcbKey}

\begin{langTcbKey}{values}{=\meta{text}}{no default, initially |Values|}
值的索引标题文本。

Heading text in the index for values.
\end{langTcbKey}

% \clearpage



\subsection{Predefined Colors of the Library\\库中预定义的颜色}\tcbdocmarginnote{\tcbdocupdated{2019-09-18}}
The following colors are predefined. They are used as default colors
in some library commands.

以下颜色是预定义的。它们在某些库命令中用作默认颜色。

\def\dispColor#1{\docColor{#1}~\tikz[baseline=1mm]\path[fill=#1,draw] (0,0) rectangle (0.4,0.4);~}

\dispColor{Option},
\dispColor{Definition},
\dispColor{ExampleFrame},
\dispColor{ExampleBack},
\dispColor{Hyperlink},
\dispColor{Fade}.


