
%% \clearpage
\begin{docCommand}{newtcbinputlisting}{\oarg{init options}\brackets{\texttt{\textbackslash}\rmfamily\meta{name}}\oarg{number}\oarg{default}\marg{options}}
Creates a new macro \texttt{\textbackslash}\meta{name} based on \refComLe{tcbinputlisting}.
% Basically, |\newtcbinputlisting| operates like |\newcommand|.
The new macro \texttt{\textbackslash}\meta{name} optionally takes \meta{number} arguments, where
\meta{default} is the default value for the optional first argument.
The \meta{options} are given to the underlying |tcbinputlisting|.
The \meta{init options} allow setting up automatic numbering,
see Section \ref{sec:initkeys} from page \pageref{sec:initkeys}.

基于\refComLe{tcbinputlisting}创建一个新的宏\texttt{\textbackslash}\meta{name}。%
% 基本上,|\newtcbinputlisting| 的操作类似于 |\newcommand|。 
新的宏\texttt{\textbackslash}\meta{name}可以选择性地带有\meta{number}个参数,其中\meta{default}是可选第一个参数的默认值。 \meta{options}被赋予基础的|tcbinputlisting|。 \meta{init options}可用于设置自动编号,参见第\pageref{sec:initkeys}页的第\ref{sec:initkeys}节。
\begin{dispExample}
\newtcbinputlisting[use counter from=mycbox]{\mylisting}[2][]{%
listing file={#2},
title=Listing (\thetcbcounter) of \texttt{#2},
colback=red!5!white,colframe=red!75!black,fonttitle=\bfseries,
listing only,breakable,#1}

\mylisting[before upper=\textit{This is the included file content:}]
        {\jobname.tcbtemp}
\end{dispExample}

\begin{dispExample}
\newtcbinputlisting[use counter from=mycbox]{\mylisting}[2][]{%
listing engine=minted,minted language=latex,minted style=colorful,
listing file={#2},
title=Listing (\thetcbcounter) of \texttt{#2},
colback=red!5!white,colframe=red!75!black,fonttitle=\bfseries,
listing only,breakable,#1}

\mylisting[before upper=\textit{This is the included file content:}]
        {\jobname.tcbtemp}
\end{dispExample}
\end{docCommand}


\begin{docCommand}{renewtcbinputlisting}{\oarg{init options}\brackets{\texttt{\textbackslash}\rmfamily\meta{name}}\oarg{number}\oarg{default}\marg{options}}
An existing macro is redefined.Operates like \refComLe{newtcbinputlisting}.%, but based on |\renewcommand| instead of |\newcommand|.


现有的宏被重新定义。类似于\refComLe{newtcbinputlisting}。%,但是基于 |\renewcommand| 而不是 |\newcommand|。 
\end{docCommand}
