The following example shows how to apply the first method.
There already is a file |tcolorbox-example.tex| and a PDF file
|tcolorbox-example.pdf|. Both of them are input partly by the following:

下面的示例展示了如何使用第一种方法。 已经有了一个名为 |tcolorbox-example.tex| 和一个名为 |tcolorbox-example.pdf| 的文件。它们都是通过下列方式部分输入的:
\begin{dispListing}
% \tcbuselibrary{breakable,skins,raster}
\tcbinputlisting{
enhanced jigsaw,breakable,pad at break*=2mm,height fixed for=first and middle,
lower separated=false,
leftlower=0pt,rightlower=0pt,middle=0pt,
colframe=red!50!black,colback=yellow!10!white,
listing and comment,
listing file={tcolorbox-example},
listing options=
{style=tcblatex,texcsstyle=*\color{red!70!black},firstline=20,lastline=40},
after upper={\par\bigskip\texttt{\ldots}\par},
pdf comment,
comment style={drop lifted shadow,graphics pages={1,...,2}},
}
\end{dispListing}
{\tcbusetemp}