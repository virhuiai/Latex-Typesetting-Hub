\begin{docTcbKey}[][doc new=2014-11-14]{run pdflatex}{\colOpt{=\meta{arguments}}}{style, no default, initially unset}
Issues a |pdflatex| compilation of the listing with the given \meta{arguments}.

使用给定的 \meta{arguments} 发行一个 |pdflatex| 编译清单。
\begin{itemize}
\item The main document has to be compiled with the |-shell-escape| permission.
\\主文档必须使用 |-shell-escape| 权限进行编译。
\item The \refKeyLe{/tcb/listing file} has to be unique for the listing.
\\\refKeyLe{/tcb/listing file} 必须对于该代码清单是唯一的。
\item If the listing has to be compiled twice, add |run pdflatex| two times to
the option list.
\\如果需要对代码清单进行两次编译,需要将 |run pdflatex| 添加两次。
\end{itemize}
% \begin{itemize} \item 主文档必须以 |-shell-escape| 权限进行编译。 \item \refKeyLe{/tcb/listing file} 必须对于清单是唯一的。 \item 如果需要编译两次清单,则将 |run pdflatex| 添加到选项列表中两次。 \end{itemize}

\begin{dispListing*}{breakable,enhanced,pad at break*=2mm,before upper=,after skip=3mm}
\begin{tcblisting}{enhanced jigsaw,lower separated=false,
leftlower=0pt,rightlower=0pt,
colframe=red!50!black,colback=yellow!10!white,
listing options={style=tcblatex,texcsstyle=*\color{red!70!black}},
listing and comment,
pdf comment,freeze pdf,
compilable listing,
run pdflatex
}
\documentclass{beamer}
\usetheme{Warsaw}
\begin{document}
\begin{frame}{Beamer example}
\begin{block}{Hello World}
\begin{itemize}[<+->]
    \item One
    \item Two
\end{itemize}
\end{block}

\begin{alertblock}{Integral}
\begin{equation}
    \visible<3->{\int\limits_1^x \frac{1}{t}~dt}
    \visible<4->{ = \ln(x).}
\end{equation}
\end{alertblock}
\end{frame}
\end{document}
\end{tcblisting}
\end{dispListing*}
{\tcbusetemp}
\end{docTcbKey}


% \clearpage
\begin{docTcbKey}[][doc new=2014-11-14]{run xelatex}{\colOpt{=\meta{arguments}}}{style, no default, initially unset}
Issues a |xelatex| compilation of the listing with the given \meta{arguments}.

% 使用给定的\meta{参数}对列表进行|xelatex|编译。
使用给定的\meta{参数}对源码清单进行|xelatex|编译。
\end{docTcbKey}

\begin{docTcbKey}[][doc new=2014-11-14]{run lualatex}{\colOpt{=\meta{arguments}}}{style, no default, initially unset}
Issues a |lualatex| compilation of the listing with the given \meta{arguments}.

使用给定的\meta{参数}对源码清单进行|lualatex|编译。
\end{docTcbKey}

\begin{docTcbKey}[][doc new=2014-11-14]{run makeindex}{\colOpt{=\meta{arguments}}}{style, no default, initially unset}
Issues a |makeindex| compilation of the listing with the given \meta{arguments}.

使用给定的\meta{参数}对源码清单进行|makeindex|编译。
\end{docTcbKey}

\begin{docTcbKey}[][doc new=2014-11-14]{run bibtex}{\colOpt{=\meta{arguments}}}{style, no default, initially unset}
Issues a |bibtex| compilation of the listing with the given \meta{arguments}.

使用给定的\meta{参数}编译清单的|bibtex|问题。
\end{docTcbKey}

\begin{docTcbKey}[][doc new=2014-11-14]{run biber}{\colOpt{=\meta{arguments}}}{style, no default, initially unset}
Issues a |biber| compilation of the listing with the given \meta{arguments}.

使用给定的\meta{参数}对清单进行|biber|编译。
\end{docTcbKey}

\begin{docTcbKey}[][doc new=2014-11-14]{run arara}{\colOpt{=\meta{arguments}}}{style, no default, initially unset}
Issues an |arara| compilation of the listing with the given \meta{arguments}.

使用给定的\meta{参数},发布一个列表的|arara|编译。
\end{docTcbKey}

% arara是一种基于Java编写的编译工具,用于自动化LaTeX文档的编译。它是一个开源软件,可以在不同的操作系统上运行,包括Windows、Mac和Linux。

% 使用arara,用户可以在LaTeX源文件中添加特定的注释命令,用于指定编译器、编译参数和运行顺序等信息。然后,用户只需要运行arara命令,arara就会根据注释命令自动进行编译,并输出编译结果。

% 与其他LaTeX编译工具相比,arara具有以下优点:

% 简单易用:用户只需要添加少量注释命令即可完成编译过程。

% 灵活性:用户可以根据自己的需要定制编译过程,并可以自定义编译规则和命令。

% 可靠性:arara能够自动识别源文件的依赖关系,并根据需要自动进行多次编译,确保最终的输出结果正确。

% 总之,arara是一种强大的编译工具,可以帮助LaTeX用户简化编译过程,提高工作效率。

\begin{docTcbKey}[][doc new=2014-11-14]{run latex}{\colOpt{=\meta{arguments}}}{style, no default, initially unset}
Issues a |latex| compilation of the listing with the given \meta{arguments}.

使用给定的\meta{参数}对清单进行|latex|编译。
\end{docTcbKey}

\begin{docTcbKey}[][doc new=2014-11-14]{run dvips}{\colOpt{=\meta{arguments}}}{style, no default, initially unset}
Issues a |dvips| compilation of the listing with the given \meta{arguments}.

使用给定的\meta{参数},编译清单并生成|dvips|版。
\end{docTcbKey}

\enlargethispage*{1cm}
\begin{docTcbKey}[][doc new=2014-11-14]{run ps2pdf}{\colOpt{=\meta{arguments}}}{style, no default, initially unset}
Issues a |ps2pdf| compilation of the listing with the given \meta{arguments}.

使用给定的\meta{参数},将清单编译成|ps2pdf|格式。
\end{docTcbKey}

\begin{dispListing*}{breakable,enhanced,pad at break*=2mm,before upper=,after skip=3mm}
\begin{tcblisting}{enhanced jigsaw,
title={PSTricks with pdflatex},fonttitle=\bfseries,
colframe=red!50!black,colback=yellow!10!white,
listing options={style=tcblatex,texcsstyle=*\color{red!70!black}},
lower separated=false,middle=0pt,
listing side comment,righthand width=4cm,
compilable listing,
run latex,run dvips,run ps2pdf,
pdf comment,freeze pdf,
comment style={raster columns=1,
graphics options={viewport=0.5in 7.7in 3.5in 10.5in,clip}},
}
\documentclass{article}
\usepackage{pstricks,multido}
\begin{document}
\psset{unit=3}%
\multido{\nHue=0.01+0.01}{100}{%
\definecolor{MyColor}{hsb}{\nHue,1,1}%
\pscircle[linewidth=0.01,linecolor=MyColor]{\nHue}}
\end{document}
\end{tcblisting}
\end{dispListing*}
{\tcbusetemp}

% %\clearpage

\begin{marker}
For most applications, you will like to add \refKeyLe{/tcb/freeze pdf} as option,
since the included |pdf| file is only refreshed, if the source for this file
has changed.

对于大多数应用,您会喜欢将\refKeyLe{/tcb/freeze pdf}作为选项添加, 因为所包含的|pdf|文件仅在此文件的源发生更改时进行刷新。
\end{marker}