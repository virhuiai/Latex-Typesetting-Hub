\begin{marker}
The \mylib{listingsutf8} library is not needed (and troublesome) when using Xe\LaTeX\ or Lua\LaTeX.
Therefore, loading this library is automatically replaced by loading
\mylib{listings} only, if pdf\LaTeX\ is \emph{not} used.

当使用 Xe\LaTeX\ 或 Lua\LaTeX 时,不需要(而且可能会有问题)加载 \mylib{listingsutf8} 库。 因此,如果不使用 pdf\LaTeX,加载此库会自动替换为仅加载 \mylib{listings}。
\end{marker}

The \mylib{listingsutf8} library is an extension of the
\mylib{listings} library, so
all options from \Vref{sec:speclistingkeys} are applicable.

\mylib{listingsutf8} 库是 \mylib{listings} 库的扩展,因此 \Vref{sec:speclistingkeys} 中的所有选项都适用。
\begin{docTcbKey}{listing utf8}{=\meta{one-byte-encoding}}{style, no default, initially |latin1|}
  Abbreviation for using \refKeyLe{/tcb/listing inputencoding}
  together with UTF-8 support from the package |listingsutf8| .
  This option is available only for the library variant \mylib{listingsutf8}.
  The \meta{one-byte-encoding} is one of
  the applicable encodings from , e.\,g.\ |latin1|
  which is the default.\par

  使用\refKeyLe{/tcb/listing inputencoding}的缩写,结合来自包|listingsutf8|的UTF-8支持。此选项仅适用于库变体\mylib{listingsutf8}。\meta{one-byte-encoding}是来自的适用编码之一,例如|latin1|是默认值。
  
  Be aware that this means restriction to this specific \meta{one-byte-encoding}:
  e.\,g.\ |latin1| comprises umlauts and other accented characters, but not
  the Euro sign. If you want to use the |listings| package \emph{and} \flqq real\frqq\
  UTF-8 source code, then do \emph{not} use \mylib{listingsutf8} but \mylib{listings}
  with
  \refKeyLe{/tcb/listing inputencoding}|=utf8|
  \emph{and} with specific manual hacks for specific UTF-8-encoded characters.



请注意,这意味着限制在特定的\meta{one-byte-encoding}上:例如|latin1|包括umlauts和其他重音字符,但不包括欧元符号。如果您想同时使用|listings|包和“真正的”UTF-8源代码,则不要使用\mylib{listingsutf8},而是使用\mylib{listings},并使用\refKeyLe{/tcb/listing inputencoding}|=utf8|以及针对特定UTF-8编码字符的手动修补。
\end{docTcbKey}

See further options in \Vref{sec:commonlistingkeys}.

请参阅 \Vref{sec:commonlistingkeys} 中的其他选项。