\begin{docTcbKey}{minted language}{=\meta{programming language}}{no default, initially |latex|}
Sets a \meta{programming language} known to |Pygments| \cite{pygments:web}.

设置一个被 |Pygments| \cite{pygments:web} 所知的 \meta{编程语言}。
\begin{dispExample}
\begin{tcblisting}{listing engine=minted,minted style=trac,
minted language=java,
colback=red!5!white,colframe=red!75!black,listing only}
public class HelloWorld {
    // A `Hello World' in Java
    public static void main(String[] args) {
    System.out.println("Hello World!");
    }
}
\end{tcblisting}
\end{dispExample}
\end{docTcbKey}


\begin{docTcbKey}[][doc updated={2021-12-15}]{minted options}{=\meta{key list}}{no default, initially
\linebreak see \refKey{/tcb/default minted options}}
Sets the options from the package |minted| \cite{poore:minted}
which are used during typesetting of the listing.
Also see \refKey{/tcb/minted options app} and \refKey{/tcb/minted options pre}.

设置使用 |minted| \cite{poore:minted} 包在列表排版期间使用的选项。另请参见 \refKey{/tcb/minted options app} 和 \refKey{/tcb/minted options pre}。
\begin{dispExample}
% \tcbuselibrary{skins}
\newtcblisting{myjava}{listing engine=minted,
minted style=colorful,
minted language=java,
minted options={fontsize=\small,breaklines,autogobble,linenos,numbersep=3mm},
colback=blue!5!white,colframe=blue!75!black,listing only,
left=5mm,enhanced,
overlay={\begin{tcbclipinterior}\fill[red!20!blue!20!white] (frame.south west)
rectangle ([xshift=5mm]frame.north west);\end{tcbclipinterior}}}

\begin{myjava}
public class HelloWorld {
// A `Hello World' in Java
public static void main(String[] args) {
    System.out.println("Hello World!");
}
}
\end{myjava}
\end{dispExample}
\end{docTcbKey}


% \clearpage
\begin{docTcbKey}[][doc new={2021-12-15}]{default minted options}{=\meta{key list}}{no default, initially
|tabsize=2,fontsize=\textbackslash small,|\linebreak|breaklines,autogobble|}
Sets the options from the package |minted| \cite{poore:minted}
which are used during typesetting of the listing, if
\refKey{/tcb/minted options} are \emph{not} used. The intended use is
inside the preamble to change the default behavior.
Note that setting \refKey{/tcb/default minted options} also resets \refKey{/tcb/minted options}.

如果未使用\refKey{/tcb/minted options},则从|minted|\cite{poore:minted}包设置选项,用于列表的排版。预期的用途是在导言中更改默认行为。请注意,设置\refKey{/tcb/default minted options}也会重置\refKey{/tcb/minted options}。
\begin{dispListing}
% inside the preamble
\tcbset{%
default minted options={tabsize=4,fontsize=\normalsize},
}
\end{dispListing}
\end{docTcbKey}



\begin{docTcbKey}{minted style}{=\meta{style}}{no default, initially unset}
Sets a \meta{style} known to |Pygments| \cite{pygments:web}. This is
independent from \refKey{/tcb/minted options}. Note that styles are always
applied globally; all following examples will be set in the given \meta{style}
until a new style is set. Also note that
setting |\usemintedstyle|\marg{style} only once per document is more economic, if
all styles in a document are the same.
For examples of different styles, see
\refKey{/tcb/minted language} and \refKey{/tcb/minted options}.

设置一个已知于 |Pygments| \cite{pygments:web} 的 \meta{style}。这与 \refKey{/tcb/minted options} 是独立的。请注意,样式始终是全局应用的;所有后续的示例都将设置在给定的 \meta{style} 中,直到设置新样式。还请注意,如果文档中的所有样式都相同,则每个文档仅需设置一次 |\usemintedstyle|\marg{style} 更为经济。有关不同样式的示例,请参见 \refKey{/tcb/minted language} 和 \refKey{/tcb/minted options}。
\end{docTcbKey}

See further options in \Vref{sec:commonlistingkeys}.

请参见\ref{sec:commonlistingkeys} 中的更多选项。