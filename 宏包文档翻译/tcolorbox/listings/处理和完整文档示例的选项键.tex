\subsection{Option Keys for Processing and Full Document Examples\\处理选项键和完整文档示例}\label{sec:proclistingkeys}
% \include*{listings/处理和完整文档示例的选项键_处理完整的LaTeX文档有两种方法}
% \include*{listings/处理和完整文档示例的选项键_包含源文件和生成的PDF文件}
% \include*{listings/处理和完整文档示例的选项键_process_code}
%% \clearpage
% \include*{listings/处理和完整文档示例的选项键_run_system_command}
% \include*{listings/处理和完整文档示例的选项键_compilable_listing}
\end{document}



\begin{docTcbKey}[][doc new=2014-11-14]{run pdflatex}{\colOpt{=\meta{arguments}}}{style, no default, initially unset}
Issues a |pdflatex| compilation of the listing with the given \meta{arguments}.

使用给定的 \meta{arguments} 发行一个 |pdflatex| 编译清单。
\begin{itemize}
\item The main document has to be compiled with the |-shell-escape| permission.
\\主文档必须使用 |-shell-escape| 权限进行编译。
\item The \refKeyLe{/tcb/listing file} has to be unique for the listing.
\\\refKeyLe{/tcb/listing file} 必须对于该代码清单是唯一的。
\item If the listing has to be compiled twice, add |run pdflatex| two times to
the option list.
\\如果需要对代码清单进行两次编译,需要将 |run pdflatex| 添加两次。
\end{itemize}
% \begin{itemize} \item 主文档必须以 |-shell-escape| 权限进行编译。 \item \refKeyLe{/tcb/listing file} 必须对于清单是唯一的。 \item 如果需要编译两次清单,则将 |run pdflatex| 添加到选项列表中两次。 \end{itemize}

\begin{dispListing*}{breakable,enhanced,pad at break*=2mm,before upper=,after skip=3mm}
\begin{tcblisting}{enhanced jigsaw,lower separated=false,
leftlower=0pt,rightlower=0pt,
colframe=red!50!black,colback=yellow!10!white,
listing options={style=tcblatex,texcsstyle=*\color{red!70!black}},
listing and comment,
pdf comment,freeze pdf,
compilable listing,
run pdflatex
}
\documentclass{beamer}
\usetheme{Warsaw}
\begin{document}
\begin{frame}{Beamer example}
\begin{block}{Hello World}
\begin{itemize}[<+->]
    \item One
    \item Two
\end{itemize}
\end{block}

\begin{alertblock}{Integral}
\begin{equation}
    \visible<3->{\int\limits_1^x \frac{1}{t}~dt}
    \visible<4->{ = \ln(x).}
\end{equation}
\end{alertblock}
\end{frame}
\end{document}
\end{tcblisting}
\end{dispListing*}
{\tcbusetemp}
\end{docTcbKey}


% \clearpage
\begin{docTcbKey}[][doc new=2014-11-14]{run xelatex}{\colOpt{=\meta{arguments}}}{style, no default, initially unset}
Issues a |xelatex| compilation of the listing with the given \meta{arguments}.

% 使用给定的\meta{参数}对列表进行|xelatex|编译。
使用给定的\meta{参数}对源码清单进行|xelatex|编译。
\end{docTcbKey}

\begin{docTcbKey}[][doc new=2014-11-14]{run lualatex}{\colOpt{=\meta{arguments}}}{style, no default, initially unset}
Issues a |lualatex| compilation of the listing with the given \meta{arguments}.

使用给定的\meta{参数}对源码清单进行|lualatex|编译。
\end{docTcbKey}

\begin{docTcbKey}[][doc new=2014-11-14]{run makeindex}{\colOpt{=\meta{arguments}}}{style, no default, initially unset}
Issues a |makeindex| compilation of the listing with the given \meta{arguments}.

使用给定的\meta{参数}对源码清单进行|makeindex|编译。
\end{docTcbKey}

\begin{docTcbKey}[][doc new=2014-11-14]{run bibtex}{\colOpt{=\meta{arguments}}}{style, no default, initially unset}
Issues a |bibtex| compilation of the listing with the given \meta{arguments}.

使用给定的\meta{参数}编译清单的|bibtex|问题。
\end{docTcbKey}

\begin{docTcbKey}[][doc new=2014-11-14]{run biber}{\colOpt{=\meta{arguments}}}{style, no default, initially unset}
Issues a |biber| compilation of the listing with the given \meta{arguments}.

使用给定的\meta{参数}对清单进行|biber|编译。
\end{docTcbKey}

\begin{docTcbKey}[][doc new=2014-11-14]{run arara}{\colOpt{=\meta{arguments}}}{style, no default, initially unset}
Issues an |arara| compilation of the listing with the given \meta{arguments}.

使用给定的\meta{参数},发布一个列表的|arara|编译。
\end{docTcbKey}

\begin{docTcbKey}[][doc new=2014-11-14]{run latex}{\colOpt{=\meta{arguments}}}{style, no default, initially unset}
Issues a |latex| compilation of the listing with the given \meta{arguments}.

使用给定的\meta{参数}对清单进行|latex|编译。
\end{docTcbKey}

\begin{docTcbKey}[][doc new=2014-11-14]{run dvips}{\colOpt{=\meta{arguments}}}{style, no default, initially unset}
Issues a |dvips| compilation of the listing with the given \meta{arguments}.

使用给定的\meta{参数},编译清单并生成|dvips|版。
\end{docTcbKey}

\enlargethispage*{1cm}
\begin{docTcbKey}[][doc new=2014-11-14]{run ps2pdf}{\colOpt{=\meta{arguments}}}{style, no default, initially unset}
Issues a |ps2pdf| compilation of the listing with the given \meta{arguments}.

使用给定的\meta{参数},将清单编译成|ps2pdf|格式。
\end{docTcbKey}

\begin{dispListing*}{breakable,enhanced,pad at break*=2mm,before upper=,after skip=3mm}
\begin{tcblisting}{enhanced jigsaw,
title={PSTricks with pdflatex},fonttitle=\bfseries,
colframe=red!50!black,colback=yellow!10!white,
listing options={style=tcblatex,texcsstyle=*\color{red!70!black}},
lower separated=false,middle=0pt,
listing side comment,righthand width=4cm,
compilable listing,
run latex,run dvips,run ps2pdf,
pdf comment,freeze pdf,
comment style={raster columns=1,
graphics options={viewport=0.5in 7.7in 3.5in 10.5in,clip}},
}
\documentclass{article}
\usepackage{pstricks,multido}
\begin{document}
\psset{unit=3}%
\multido{\nHue=0.01+0.01}{100}{%
\definecolor{MyColor}{hsb}{\nHue,1,1}%
\pscircle[linewidth=0.01,linecolor=MyColor]{\nHue}}
\end{document}
\end{tcblisting}
\end{dispListing*}
{\tcbusetemp}

% %\clearpage

\begin{marker}
For most applications, you will like to add \refKeyLe{/tcb/freeze pdf} as option,
since the included |pdf| file is only refreshed, if the source for this file
has changed.

对于大多数应用,您会喜欢将\refKeyLe{/tcb/freeze pdf}作为选项添加, 因为所包含的|pdf|文件仅在此文件的源发生更改时进行刷新。
\end{marker}

\begin{docTcbKey}[][doc new=2016-07-14]{freeze file}{=\meta{file}}{no default, initially unset}
Observes some \meta{file}, usually the final file produced by \refKeyLe{/tcb/process code},
\refKeyLe{/tcb/run system command}, \refKeyLe{/tcb/run pdflatex}, etc.
If the MD5 checksum of the current \refKeyLe{/tcb/listing file} is unchanged
and \meta{file} exists, the processing is skipped and
the \meta{file} is kept (frozen).
Typically, the style \refKeyLe{/tcb/freeze pdf} can be used for
convenience.

观察一些\meta{文件},通常是由\refKeyLe{/tcb/process code},\refKeyLe{/tcb/run system command},\refKeyLe{/tcb/run pdflatex}等生成的最终文件。 如果当前\refKeyLe{/tcb/listing file}的MD5校验和未更改且\meta{文件}存在,则跳过处理并保留\meta{文件}(冻结)。 通常,可以使用样式\refKeyLe{/tcb/freeze pdf}以方便处理。
\end{docTcbKey}

\begin{docTcbKey}[][doc new=2016-07-14]{freeze none}{}{no default, initially set}
Freeze no file and always execute the given process commands.

不要冻结任何文件,并始终执行给定的进程命令。
\end{docTcbKey}

\begin{docTcbKey}[][doc new=2016-07-14]{freeze extension}{=\meta{text}}{style, no default}
Calls \refKeyLe{/tcb/freeze file} with the current \refKeyLe{/tcb/listing file}
stripped with its extension plus \meta{text} as new extension.

使用当前的\refKeyLe{/tcb/listing file},将其剥离扩展名并加上\meta{text}作为新的扩展名,调用\refKeyLe{/tcb/freeze file}。
\begin{dispListing}
...
listing file=myfile.tex,
freeze extension=-modified.pdf,    % ->   myfile-modified.pdf   is observed
...
\end{dispListing}
\end{docTcbKey}

\begin{docTcbKey}[][doc new=2016-07-14]{freeze pdf}{}{no value}
Calls \refKeyLe{/tcb/freeze file} with the current \refKeyLe{/tcb/listing file}
stripped with its extension plus |.pdf| as new extension.

使用当前的 \refKeyLe{/tcb/listing file} 去调用 \refKeyLe{/tcb/freeze file},并将其扩展名去除,然后加上新的扩展名 |.pdf|。
\end{docTcbKey}

\begin{docTcbKey}[][doc new=2016-07-14]{freeze png}{}{no value}
Calls \refKeyLe{/tcb/freeze file} with the current \refKeyLe{/tcb/listing file}
stripped with its extension plus |.png| as new extension.
See the examples for \refKeyLe{/tcb/run pdflatex} and \refKeyLe{/tcb/run ps2pdf}.

使用当前的\refKeyLe{/tcb/listing file},去掉其扩展名,再加上|.png|作为新的扩展名来调用\refKeyLe{/tcb/freeze file}。请参见\refKeyLe{/tcb/run pdflatex}和\refKeyLe{/tcb/run ps2pdf}的示例。
\end{docTcbKey}

\begin{docTcbKey}[][doc new=2016-07-14]{freeze jpg}{}{no value}
Calls \refKeyLe{/tcb/freeze file} with the current \refKeyLe{/tcb/listing file}
stripped with its extension plus |.jpg| as new extension.

调用\refKeyLe{/tcb/freeze file},其中参数为当前\refKeyLe{/tcb/listing file},去除文件扩展名并加上|.jpg|作为新的扩展名。
\end{docTcbKey}