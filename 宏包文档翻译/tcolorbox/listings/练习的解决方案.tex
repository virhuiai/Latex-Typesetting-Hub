\subsection{Solutions for the given \LaTeX\ Exercises\\\LaTeX\ 练习的解决方案}

For all solutions, a macro |\processsol| was written to the file |\jobname.records|.
Now, we need a definition for this macro to use the solutions.

对于所有的解决方案,一个名为 |\processsol| 的宏已经被写入到文件 |\jobname.records| 中。 现在,我们需要一个定义来使用这个宏来处理解决方案。
\begin{dispListing}
% \usepackage{hyperref} % for phantomlabel
\newtcbinputlisting{\processsol}[2]{%
  texercisestyle,
  listing only,
  listing file={#1},
  phantomlabel={sol:#2},%
  title={Solution for Exercise \ref{exe:#2} on page \pageref{exe:#2}},
}
\end{dispListing}
\tcbusetemp

The loading of all solutions is done by:

所有解决方案的加载是通过以下方式完成的:

\begin{dispListing}
\tcbinputrecords
\end{dispListing}

With this, we get:

通过这个,我们得到:

\tcbusetemp



