
\begin{docCommand}{tcbinputlisting}{\marg{options}}
Creates a colored boxed based on a |tcolorbox|. The text content is read
from a file named by the key value of |listing file|. Apart from that,
the function is equal to that of \refEnvLe{tcblisting}.

根据 |tcolorbox| 创建一个彩色盒子。文本内容从名为 |listing file| 的键值对应的文件中读取。除此之外,该函数与 \refEnvLe{tcblisting} 的功能相同。
\begin{dispExample}
\tcbinputlisting{colback=red!5!white,colframe=red!75!black,text only,title={text only}}
\tcbinputlisting{colback=green!5,colframe=green!75!black,listing only,title={listing only}}
\end{dispExample}
\end{docCommand}

\begin{docCommand}{tcbuselistingtext}{}
Loads text from a file named by the key value of |listing file|.

从名为|listing file|的键值指定的文件中加载文本。
\begin{dispExample}
\tcbuselistingtext
\end{dispExample}
\end{docCommand}


\begin{docCommand}{tcbuselistinglisting}{}
Typesets text as listing from a file named by the key value of |listing file|.

将文本设置为源码清单,列表文件的名称由|listing file|的键值命名。
\begin{dispExample}
\tcbuselistinglisting
\end{dispExample}
\end{docCommand}

%\enlargethispage*{5mm}
\begin{docCommand}{tcbusetemplisting}{}
Typesets text as listing from a temporary file which was written by \refEnvLe{tcbwritetemp}.

将文本设置为源码清单,该列表来自由\refEnvLe{tcbwritetemp}写入的临时文件。
\end{docCommand}


% \clearpage
\begin{marker}
See \Vref{subsec:xparse_listing} and \Vref{subsec:xparse_inputlisting} for more
elaborate methods to create new environments and commands.

请参见 \Vref{subsec:xparse_listing} 和 \Vref{subsec:xparse_inputlisting},了解更详细的创建新环境和命令的方法。
\end{marker}
\begin{marker}
If a new sort of |tcblisting| environments should be created with
one optional argument only, one is highly recommended to use
\refComLe{DeclareTCBListing} or \refComLe{NewTCBListing}
instead of \refComLe{newtcblisting} to
avoid content scanning problems.

如果要创建一种只有一个可选参数的新 |tcblisting| 环境,强烈建议使用 \refComLe{DeclareTCBListing} 或 \refComLe{NewTCBListing} 而不是 \refComLe{newtcblisting},以避免内容扫描问题。
\end{marker}