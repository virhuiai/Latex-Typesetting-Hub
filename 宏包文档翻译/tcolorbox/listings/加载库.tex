\subsection{Loading the Libraries\\加载库}

In contrast to other |tcolorbox| libraries, the libraries
\mylib{listings}, \mylib{listingsutf8}, and \mylib{minted} are concurrent in the sense that
they all do the same thing, i.\,e.\ displaying listings with or without typesetting
the listing in \LaTeX\ parallel.
The difference is the underlying \LaTeX\ package which does the core job for
displaying a listing. So, typically, you need just \emph{one} of these
libraries. If you do not have a clue which one of them you should use
and you are using |pdflatex|, you should take \mylib{listingsutf8}.
If you are using |xelatex| or |lualatex|, you should take \mylib{listings}
as |xelatex| and |lualatex| are not compatible with \mylib{listingsutf8}.

与其他 |tcolorbox| 库不同的是,库 \mylib{listings}、\mylib{listingsutf8} 和 \mylib{minted} 在并行方面是相互的,也就是说,它们都可以在 \LaTeX\ 中并行地显示代码,无论是否排版代码。 它们的区别在于执行显示代码的基础 \LaTeX\ 包。因此,通常情况下,您只需要使用其中的 \emph{一个} 库。如果您不知道该使用哪个库,并且您使用的是 |pdflatex|,那么您应该选择 \mylib{listingsutf8}。如果您使用的是 |xelatex| 或 |lualatex|,则应该选择 \mylib{listings},因为 |xelatex| 和 |lualatex| 与 \mylib{listingsutf8} 不兼容。



\begin{marker}
The order in which the libraries are included influences the default settings and
the \refKeyLe{/tcb/reset} behavior. The settings of a later loaded library overwrite
the settings of a previous loaded library. A library is never loaded twice.

库的包含顺序影响默认设置和\refKeyLe{/tcb/reset}行为。后加载的库的设置会覆盖先前加载的库的设置。库永远{\bf 不会被重复加载}。
\end{marker}
