
\subsection{Rasters inside Rasters\\栅格内的栅格}\label{subsec:raster_insideraster}

A \emph{raster} inside a \emph{raster} cannot be used directly, because
a \emph{raster} can only contain a \emph{tcolorbox} or something
derived from a \emph{tcolorbox}. So, a \emph{raster} can be put inside
a \emph{tcolorbox} inside a \emph{raster}.

栅格内的栅格不能直接使用,因为栅格只能包含一个\emph{tcolorbox}或者从\emph{tcolorbox}派生出的东西。因此,可以将一个栅格放在另一个\emph{tcolorbox}内部的栅格中。

Some examples for such constructions can be found at \refEnvLe{tcboxedraster},
\refKeyLe{/tcb/raster multicolumn},
\refKeyLe{/tcb/raster multirow}.

这种结构的一些示例可以在 \refEnvLe{tcboxedraster},\refKeyLe{/tcb/raster multicolumn},\refKeyLe{/tcb/raster multirow} 中找到。

\subsubsection{Raster Setup\\光栅设置}
The intermediating \refEnvLe{tcolorbox} can be made invisible by using
\refKeyLe{/tcb/blankest}.

通过使用\refKeyLe{/tcb/blankest},可以使中间的\refEnvLe{tcolorbox}不可见。
\begin{dispExample}
\begin{tcbraster}[raster equal height=rows,
raster every box/.style={colframe=red!50!black,colback=red!10!white}]
\begin{tcolorbox}[blankest]
\begin{tcbraster}[raster columns=1]
\begin{tcolorbox}One\end{tcolorbox}
\begin{tcolorbox}Two\end{tcolorbox}
\end{tcbraster}
\end{tcolorbox}
\begin{tcolorbox}raster+tcolorbox+raster\end{tcolorbox}
\end{tcbraster}
\end{dispExample}

% \enlargethispage*{1cm}
\begin{dispExample}
\begin{tcbraster}[raster equal height=rows,
raster every box/.style={colframe=red!50!black,colback=red!10!white}]
\begin{tcboxedraster}[raster columns=1]{blankest}
\begin{tcolorbox}One\end{tcolorbox}
\begin{tcolorbox}Two\end{tcolorbox}
\end{tcboxedraster}
\begin{tcolorbox}raster+tcboxedraster\end{tcolorbox}
\end{tcbraster}
\end{dispExample}


\begin{dispExample}
\begin{tcbitemize}[raster equal height=rows,
raster every box/.style={colframe=red!50!black,colback=red!10!white}]
\tcbitem[blankest]
\begin{tcbitemize}[raster columns=1]
\tcbitem One
\tcbitem Two
\end{tcbitemize}
\tcbitem tcbitemize+tcbitem+tcbitemize
\end{tcbitemize}
\end{dispExample}


\subsubsection{Placing Spaces\\放置空间}
If the heights of boxes inside staggered rasters should be matched, the
space has to be distributed accordingly.

如果需要匹配交错排列內部的盒子的高度,则必须相应地分配空间。
\begin{itemize}
\item For fixed height boxes/rasters using \refKeyLe{/tcb/raster height},
the height of boxes is available by \refComLe{tcbtextheight}. This can be
used to size deeper layered boxes/rasters.
\\对于使用\refKeyLe{/tcb/raster height}的固定高度盒子/光栅,可以通过\refComLe{tcbtextheight}获取盒子的高度。这可用于调整更深层的盒子/光栅的大小。
\item For boxes/rasters layed out using \refKeyLe{/tcb/raster equal height},
space can be distributed by \refKeyLe{/tcb/space to}.
It can take several compilations until all spaces are distributed correctly.
\\对于使用\refKeyLe{/tcb/raster equal height}布局的盒子/光栅,可以通过\refKeyLe{/tcb/space to}分配间距。可能需要多次编译才能正确分配所有空间。
\end{itemize}


\begin{dispExample}
\begin{tcbitemize}[raster rows=2,raster height=6cm,
raster every box/.style={colframe=red!50!black,colback=red!10!white}]
\tcbitem[blankest]
\begin{tcbitemize}[raster columns=1,raster rows=2,raster height=\tcbtextheight]
\tcbitem One
\tcbitem Two
\end{tcbitemize}
\tcbitem This is a fixed height box.
\tcbitem Three
\tcbitem Four
\end{tcbitemize}
\end{dispExample}


\begin{dispExample}
\begin{tcbitemize}[raster columns=4,raster rows=4,raster height=0.8\linewidth,
raster every box/.style={size=small,beamer,
colframe=blue!75!yellow,colback=red!75!yellow!20,
center title,title=Box}]
\tcbitem One
\tcbitem Two
\tcbitem Three
\tcbitem Four
\tcbitem[raster multirow=2,blankest]
\begin{tcbitemize}[raster columns=1,raster rows=2,raster height=\tcbtextheight]
\tcbitem Twelve
\tcbitem Eleven
\end{tcbitemize}
\tcbitem[raster multirow=2,raster multicolumn=2,
colframe=red!75!yellow,colback=blue!75!yellow!20]
This is an example with fixed height boxes.
\tcbitem[raster multirow=2,blankest]
\begin{tcbitemize}[raster columns=1,raster rows=2,raster height=\tcbtextheight]
\tcbitem Five
\tcbitem Six
\end{tcbitemize}
\tcbitem Ten
\tcbitem Nine
\tcbitem Eight
\tcbitem Seven
\end{tcbitemize}
\end{dispExample}


\begin{dispExample}
\begin{tcbitemize}[raster equal height=rows,
raster every box/.style={colframe=red!50!black,colback=red!10!white}]
\tcbitem[blankest,space to=\myspace]
\begin{tcbitemize}[raster columns=1]
\tcbitem One
\tcbitem[add to natural height=\myspace]
This box will adapt its height.
\end{tcbitemize}
\tcbitem This is a flexible height box.
\tcbitem \lipsum[4]
\tcbitem[blankest,space to=\myspace]
\begin{tcbitemize}[raster columns=1]
\tcbitem One
\tcbitem[add to natural height=\myspace]
This box will adapt its height.
\end{tcbitemize}
\end{tcbitemize}
\end{dispExample}



\begin{dispExample}
\begin{tcbitemize}[raster equal height=rows,
raster every box/.style={colframe=red!50!black,colback=red!10!white}]
\tcbitem[blankest,space to=\myspace]
\begin{tcbitemize}[raster columns=1]
\tcbitem One
\tcbitem[add to natural height=\myspace]
This box will adapt its height.
\tcbitem \lipsum[4]
\end{tcbitemize}
\tcbitem[blankest,space to=\myspace]
\begin{tcbitemize}[raster columns=1]
\tcbitem[blankest]\includegraphics[width=\linewidth]{goldshade.png}
\tcbitem[add to natural height=\myspace]
This box will adapt its height.
\end{tcbitemize}
\end{tcbitemize}
\end{dispExample}



