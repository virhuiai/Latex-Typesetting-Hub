
\subsection{Combining Columns or Rows\\合并列或行}\label{subsec:raster_multicolrow}

\begin{docTcbKey}[][doc new=2016-02-19]{raster multicolumn}{=\meta{number}}{no default, initially unset}
This option has to be set inside the option list of a \refEnvLe{tcolorbox}
inside a \refEnvLe{tcbraster} or inside \refComLe{tcbitem} inside \refEnvLe{tcbitemize}.
It merges the given \meta{number} of boxes into one single box
on the same line. The resulting box gets the \docAuxCommand{thetcbrasternum}
of the first box.
If there are not enough boxes available on the current line, this option
is ignored and a warning is given.

该选项必须在\refEnvLe{tcbraster}内的\refEnvLe{tcolorbox}的选项列表中或在\refEnvLe{tcbitemize}内的\refComLe{tcbitem}中设置。它将给定的\meta{number}个盒子合并成同一行的一个单独的盒子。结果的盒子将获得第一个盒子的\docAuxCommand{thetcbrasternum}。如果当前行上没有足够的盒子可用,则忽略此选项并发出警告。
\begin{dispExample}
\begin{tcbitemize}[raster equal height=rows,raster columns=3,
title=\thetcbrasternum,colframe=red!50!black,colback=red!10!white]
\tcbitem[colframe=blue!50!black,colback=blue!10!white,raster multicolumn=1]
multicolumn=1
\tcbitem
\tcbitem
\tcbitem[colframe=blue!50!black,colback=blue!10!white,raster multicolumn=2]
multicolumn=2
\tcbitem
\tcbitem[colframe=blue!50!black,colback=blue!10!white,raster multicolumn=3]
multicolumn=3
\tcbitem
\tcbitem[colframe=blue!50!black,colback=blue!10!white,raster multicolumn=2]
multicolumn=2
\end{tcbitemize}
\end{dispExample}
\end{docTcbKey}



\begin{docTcbKey}[][doc new=2016-02-19]{raster multirow}{=\meta{number}}{no default, initially unset}
This option has to be set inside the option list of a \refEnvLe{tcolorbox}
inside a \refEnvLe{tcbraster} or inside \refComLe{tcbitem} inside \refEnvLe{tcbitemize}.
This option not really merges boxes, but simply sizes the current box to fit
the space of \meta{number} rows.

这个选项必须在\refEnvLe{tcolorbox}的选项列表中设置,要么在\refEnvLe{tcbraster}中,要么在\refEnvLe{tcbitemize}中的\refComLe{tcbitem}中。这个选项并不是真正地合并盒子,而是简单地调整当前盒子的大小,以适应\meta{number}行的空间。
\begin{marker}
\refKeyLe{/tcb/raster multirow} needs \refKeyLe{/tcb/raster height} to be set.
How to achieve a similar result for boxes without fixed \refKeyLe{/tcb/raster height}
is shown afterwards.

\refKeyLe{/tcb/raster multirow}需要设置\refKeyLe{/tcb/raster height}。如何为没有固定\refKeyLe{/tcb/raster height}的盒子实现类似的结果将在后面展示。
\end{marker}

\begin{dispExample}
\begin{tcbitemize}[raster rows=3,raster columns=3,raster height=6cm,
raster every box/.style={colframe=red!50!black,colback=red!10!white}]
\tcbitem
\tcbitem
\tcbitem
\tcbitem[colframe=blue!50!black,colback=blue!10!white,raster multirow=2]
multirow=2
\tcbitem[raster multicolumn=2,raster multirow=2,blankest]
\begin{tcbitemize}[raster rows=2,raster columns=2,raster height=\tcbtextheight]
\tcbitem
\tcbitem
\tcbitem
\tcbitem
\end{tcbitemize}
\end{tcbitemize}
\end{dispExample}



For rasters without fixed \refKeyLe{/tcb/raster height}, \refKeyLe{/tcb/raster multirow}
cannot be used. Note that \refComLe{tcbtextheight} also cannot be used like
in the previous example.

对于没有固定\refKeyLe{/tcb/raster height}的栅格,无法使用\refKeyLe{/tcb/raster multirow}。请注意,与之前的示例不同,也不能像之前的示例中那样使用\refComLe{tcbtextheight}。

But, with combination of \refKeyLe{/tcb/raster equal height} and
\refKeyLe{/tcb/space to}, a similar effect can be created:

但是,结合\refKeyLe{/tcb/raster equal height}和\refKeyLe{/tcb/space to},可以创建类似的效果:
\begin{dispExample}
\begin{tcbitemize}[raster columns=3,raster equal height=rows,
raster every box/.style={colframe=red!50!black,colback=red!10!white}]
\tcbitem
\tcbitem
\tcbitem
\tcbitem[colframe=blue!50!black,colback=blue!10!white]
\lipsum[2]
\tcbitem[raster multicolumn=2,blankest,space to=\myspace]
\begin{tcbitemize}[raster columns=2]
\tcbitem This is a box of the inner raster.
\tcbitem
\tcbitem[height=\myspace]
\tcbitem[height=\myspace]
\end{tcbitemize}
\end{tcbitemize}
\end{dispExample}

\end{docTcbKey}



