

\subsection{Concept of Rasters\\栅格的概念}\label{subsec:raster_overview}

A \emph{raster} is used to align several colored boxes in a regular way.
It can be seen as a far related counterpart to the |matrix| construct
of \tikzname, but it differs in many aspects.
\\\emph{栅格}用于以规则的方式对齐多个彩色框。它可以看作是\tikzname 中 |matrix| 结构的远亲,但在许多方面有所不同。


In principle, |tcolorbox|es are arranged in rows and columns when put
inside a \refEnvLe{tcbraster} environment. The boxes are fluently added to the
raster like adding text to a paragraph.
Especially, line/row breaks are done
automatically and one cannot end a line/row ahead of schedule.
Further, a \emph{raster} is not restricted to a single page but may
break into an arbitrary series of pages.
\\原则上,当放置在 \refEnvLe{tcbraster} 环境中时,|tcolorbox| 以行和列的形式排列。这些盒子被流畅地添加到栅格中,就像将文本添加到段落中一样。特别地,行/列换行是自动完成的,不能提前结束行/列。此外,\emph{栅格}不仅限于单个页面,还可以分成任意一系列页面。

\bigskip
\begin{tcolorbox}[enhanced,size=tight,boxrule=0pt,frame hidden,
top=5mm,bottom=5mm,colback=yellow!20!white,
borderline={0.4pt}{0pt}{dashed,yellow!50!black},
finish={
\draw[thick,<->] ([yshift=-1.3cm]frame.north west)-- node[below]{\refKeyLe{/tcb/raster width}}
([yshift=-1.3cm]frame.north east);
\draw[thick,<->] ([xshift=5mm,yshift=-5mm]frame.north east)-- node[left,pos=.75]{\refKeyLe{/tcb/raster height}}
([xshift=5mm,yshift=5mm]frame.south east);
}]
\begin{tcbitemize}[enhanced,fontupper=\tiny,
title={Box \#\thetcbrasternum},
colframe=red!50!black!30!white,colback=red!10!white!40!white,
colupper=black!20!white,
raster equal height=rows,
raster left skip=5mm,raster right skip=5mm,
raster before skip=5mm,raster after skip=5mm,
raster row skip=3mm,raster column skip=3mm,
]
\tcbitem[finish={%
\draw[thick,<->] (frame.west)-- node[below right]{\refKeyLe{/tcb/raster left skip}}+(-0.5,0);
\draw[thick,<->] (frame.north)-- node[right]{\refKeyLe{/tcb/raster before skip}} +(0,0.5);
\draw[thick,<->] (frame.south)-- node[above]{\refKeyLe{/tcb/raster row skip}} +(0,-0.3);
}] \lipsum[1]
\tcbitem[finish={
\draw[thick,<->] (frame.east)-- node[below left]{\refKeyLe{/tcb/raster right skip}}+(0.5,0);
\draw[thick,<->] ([yshift=5mm]frame.south west)-- node[above]{\refKeyLe{/tcb/raster column skip}}+(-0.3,0);
}] \lipsum[2]
\tcbitem[finish={%
\draw[thick,<->] (frame.south)-- node[right]{\refKeyLe{/tcb/raster after skip}} +(0,-0.5);
}] \lipsum[3]
\tcbitem \lipsum[4]
\end{tcbitemize}
\end{tcolorbox}

% \clearpage
\begin{tcboutputlisting}
\begin{tcbraster}[raster columns=3,raster rows=3,raster height=\linewidth,
enhanced,size=small,sharp corners,arc=8mm,colframe=red!50!black,
colback=yellow!10!white,watermark overzoom=1.0,fit algorithm=hybrid* ]
\begin{tcolorbox}[rounded corners=northwest,boxrule=0pt,
watermark graphics=lichtspiel.jpg]\end{tcolorbox}
\tcboxfit{\lipsum[1]}
\begin{tcolorbox}[rounded corners=northeast,boxrule=0pt,
watermark graphics=goldshade.png]\end{tcolorbox}
\tcboxfit{\lipsum[2]}
\begin{tcolorbox}[valign=center,halign=center]Nine Boxes.\end{tcolorbox}
\tcboxfit{\lipsum[3]}
\begin{tcolorbox}[rounded corners=southwest,boxrule=0pt,
watermark graphics=goldshade.png]\end{tcolorbox}
\tcboxfit{\lipsum[4]}
\begin{tcolorbox}[rounded corners=southeast,boxrule=0pt,
watermark graphics=lichtspiel.jpg]\end{tcolorbox}
\end{tcbraster}
\end{tcboutputlisting}

\tcbinputlisting{base example,listing only,listing style=mydocumentation}

\bigskip
{\tcbuselistingtext}

