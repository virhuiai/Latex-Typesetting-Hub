\subsection{Macros of the Library\\库的宏}\label{subsec:raster_macros}

\begin{docEnvironment}[doc new and updated={2014-11-10}{2017-02-01}]{tcbraster}{\oarg{options}}
A raster arranges enclosed boxes in a regular way, mainly into rows and
columns. The \meta{options} are used to control the raster
parameters and to set the properties for the enclosed boxes.

栅格将封闭的盒子以规则的方式排列,主要是按行和列排列。 \meta{options} 用于控制栅格参数并设置封闭盒子的属性。
\begin{itemize}
\item The \emph{raster} is only allowed to contain a series of
\refEnvLe{tcolorbox} environments or derived constructs.
With some small restrictions, boxes created with \refComLe{tcboxfit} can also be added.
Boxes created with \refComLe{tcbox} are not reasonable here, but may be
used to a certain degree.
\\\emph{光栅}只允许包含一系列\refEnvLe{tcolorbox}环境或衍生结构。 在一些小限制下,使用\refComLe{tcboxfit}创建的盒子也可以被添加。 在此处不合理使用\refComLe{tcbox}创建的盒子,但可以在一定程度上使用。
\item Do not add anything else between the boxes inside the raster with
exception of white\-space. Especially, do not use |\\| or |\par| to end
a row; row breaks are done automatically.
\\在光栅内的盒子之间不要添加任何其他东西,除了空格符。 特别是不要使用|\\|或|\par|结束一行;行断自动完成。
\item The boxes inside a raster are numbered automatically.
\docAuxCommand{thetcbrasternum} may be used inside a box to access
this number.
The \LaTeX\ counter \docCounter{tcbrastercolumn} holds the current column,
the counter \docCounter{tcbrasterrow} holds the current row,
and the counter \docCounter{tcbrasternum} holds the current box number.
\\光栅内的盒子会自动编号。 在盒子内部使用\docAuxCommand{thetcbrasternum}可访问此编号。 \LaTeX 计数器\docCounter{tcbrastercolumn}保存当前列, 计数器\docCounter{tcbrasterrow}保存当前行, 计数器\docCounter{tcbrasternum}保存当前盒子编号。
\end{itemize}

% \enlargethispage*{1cm}

% \begin{dispExample}
% \begin{tcbraster}[raster columns=3, raster equal height,
% size=small,colframe=red!50!black,colback=red!10!white,colbacktitle=red!50!white,
% title={Box \# \thetcbrasternum}]
% \begin{tcolorbox}First box\end{tcolorbox}
% \begin{tcolorbox}Second box\end{tcolorbox}
% \begin{tcolorbox}This is a box\\with a second line\end{tcolorbox}
% \begin{tcolorbox}Another box\end{tcolorbox}
% \begin{tcolorbox}A box again\end{tcolorbox}
% \end{tcbraster}
% \end{dispExample}

\begin{dispExample}
\begin{tcbraster}[raster columns=3, raster equal height,
size=small,colframe=red!50!black,colback=red!10!white,colbacktitle=red!50!white,
title={{\tt 盒\makebox{}\#\thetcbrasternum}}]
\begin{tcolorbox}第一个盒子\end{tcolorbox}
\begin{tcolorbox}第二个盒子\end{tcolorbox}
\begin{tcolorbox}这是一个带有\\第二行的盒子\end{tcolorbox}
\begin{tcolorbox}另一个盒子\end{tcolorbox}
\begin{tcolorbox}再次一个盒子\end{tcolorbox}
\end{tcbraster}
\end{dispExample}

\begin{dispExample}
\begin{tcbraster}[raster columns=2, raster equal height=rows,
size=small,colframe=red!50!black,colback=red!10!white,colbacktitle=red!50!white,
title={Box \# \thetcbrasternum}]
\begin{tcolorbox}First box\end{tcolorbox}
\begin{tcolorbox}Second box\end{tcolorbox}
\begin{tcolorbox}This is a box\\with a second line\end{tcolorbox}
\begin{tcolorbox}Another box\end{tcolorbox}
\begin{tcolorbox}A box again\end{tcolorbox}
\end{tcbraster}
\end{dispExample}
\end{docEnvironment}


% \clearpage

\begin{docEnvironment}[doc new=2014-11-10]{tcbitemize}{\oarg{options}}
This is a special case of a \refEnvLe{tcbraster} with the given \meta{options}.
\\这是具有给定选项的\refEnvLe{tcbraster}的特殊情况。
\begin{itemize}
\item Here, the enclosed boxes are created using \refComLe{tcbitem}.
\\在这里,封闭的框使用\refComLe{tcbitem}创建。
\item There has to be at least one \refComLe{tcbitem}.
\\必须至少有一个\refComLe{tcbitem}。
\item One cannot use anything else than \refComLe{tcbitem} to add something
to the \emph{raster}.
\\不能使用除\refComLe{tcbitem}之外的任何东西来添加内容到\emph{raster}中。
\end{itemize}
This leads to a very compact syntax.

这导致了非常紧凑的语法。

\begin{dispExample}
\begin{tcbitemize}[raster columns=2, raster equal height=rows,
size=small,colframe=red!50!black,colback=red!10!white,colbacktitle=red!50!white,
title={Box \# \thetcbrasternum}]
\tcbitem First box
\tcbitem Second box
\tcbitem This is a box\\with a second line
\tcbitem[colback=yellow,colbacktitle=yellow!50!black] Another box
\tcbitem A box again
\end{tcbitemize}
\end{dispExample}

\begin{marker}
\refEnvLe{tcbitemize} has more restrictions than \refEnvLe{tcbraster}.
Especially, the \refKeyLe{/tcb/capture} mode has to be \docValue{minipage}.
For example, \refKeyLe{/tcb/fit} cannot be used safely.
If \refKeyLe{/tcb/fit} should be used, turn over to \refEnvLe{tcbraster}.

\refEnvLe{tcbitemize} 比 \refEnvLe{tcbraster} 有更多的限制。 特别地,\refKeyLe{/tcb/capture} 模式必须是 \docValue{minipage}。 例如,\refKeyLe{/tcb/fit} 不能安全地使用。 如果必须使用 \refKeyLe{/tcb/fit},请转到 \refEnvLe{tcbraster}。
\end{marker}
\end{docEnvironment}


\begin{docCommand}[doc new=2014-11-10]{tcbitem}{\oarg{options}}
Used inside \refEnvLe{tcbitemize} to create a new \refEnvLe{tcolorbox}
with the given \meta{options}.

在 \refEnvLe{tcbitemize} 中使用,用给定的 \meta{options} 创建一个新的 \refEnvLe{tcolorbox}。
\end{docCommand}


% \clearpage
\begin{docEnvironment}[doc new=2016-02-19]{tcboxedraster}{\oarg{raster options}\marg{box options}}
This is a convenience environment which combines a \refEnvLe{tcolorbox} with
an embedded \refEnvLe{tcbraster}. The \meta{box options} are given to the
outer \refEnvLe{tcolorbox}, while the \meta{raster options} are given to the
embedded \refEnvLe{tcbraster}.
This environment is especially useful for rasters inside rasters.

这是一个方便的环境,它将\refEnvLe{tcolorbox}和嵌入式\refEnvLe{tcbraster}组合起来。\meta{box options}应用于外部的\refEnvLe{tcolorbox},而\meta{raster options}应用于嵌入式的\refEnvLe{tcbraster}。这个环境对于嵌套的栅格特别有用。
\begin{dispExample}
\begin{tcboxedraster}[raster columns=3, raster equal height,
size=small,colframe=red!50!black,colback=red!10!white,colbacktitle=red!50!white,
title={Box \# \thetcbrasternum}]
{colback=yellow!10,fonttitle=\bfseries,title=Boxed Raster}
\begin{tcolorbox}First box\end{tcolorbox}
\begin{tcolorbox}Second box\end{tcolorbox}
\begin{tcolorbox}This is a box\\with a second line\end{tcolorbox}
\begin{tcolorbox}Another box\end{tcolorbox}
\begin{tcolorbox}A box again\end{tcolorbox}
\end{tcboxedraster}
\end{dispExample}

\begin{dispExample}
% \tcbuselibrary{skins}
\begin{tcbraster}[raster columns=2, raster equal height,
raster every box/.style={size=small,colframe=red!50!black,colback=red!10!white,
valign=center,halign=center}]
\begin{tcolorbox}One\end{tcolorbox}
\begin{tcolorbox}Two\end{tcolorbox}
\begin{tcboxedraster}{blankest}
    \begin{tcolorbox}Three\end{tcolorbox}
    \begin{tcolorbox}Four\end{tcolorbox}
    \begin{tcolorbox}Five\end{tcolorbox}
    \begin{tcolorbox}Six\end{tcolorbox}
\end{tcboxedraster}
\begin{tcolorbox}Seven\end{tcolorbox}
\end{tcbraster}
\end{dispExample}
\end{docEnvironment}


% \clearpage
\begin{docEnvironment}[doc new=2016-04-27]{tcboxeditemize}{\oarg{raster options}\marg{box options}}
This is a convenience environment which combines a \refEnvLe{tcolorbox} with
an embedded \refEnvLe{tcbitemize}. The \meta{box options} are given to the
outer \refEnvLe{tcolorbox}, while the \meta{raster options} are given to the
embedded \refEnvLe{tcbitemize}.
This environment is especially useful for rasters inside rasters.

这是一个便利的环境,它将 \refEnvLe{tcolorbox} 和嵌入的 \refEnvLe{tcbitemize} 结合在一起。\meta{box options} 用于外部的 \refEnvLe{tcolorbox},而 \meta{raster options} 用于嵌入的 \refEnvLe{tcbitemize}。这个环境在嵌套的栅格中非常有用。
\begin{dispExample}
\begin{tcboxeditemize}[raster columns=3, raster equal height,
size=small,colframe=red!50!black,colback=red!10!white,colbacktitle=red!50!white,
title={Box \# \thetcbrasternum}]
{colback=yellow!10,fonttitle=\bfseries,title=Boxed Itemize}
\tcbitem First box
\tcbitem Second box
\tcbitem This is a box\\with a second line
\tcbitem Another box
\tcbitem A box again
\end{tcboxeditemize}
\end{dispExample}
\end{docEnvironment}

