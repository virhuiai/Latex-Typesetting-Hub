
\subsection{Technical Overview\\技术概述}
The library \mylib{breakable} supports the automatic breaking of a |tcolorbox|.
This feature is enabled by \refKeyLe{/tcb/breakable}
and disabled by \refKeyLe{/tcb/unbreakable}.

库\mylib{breakable}支持自动断行|tcolorbox|。此功能由\refKeyLe{/tcb/breakable}启用,由\refKeyLe{/tcb/unbreakable}禁用。
{
\tcbset{colframe=Navy,colback=AliceBlue,fonttitle=\bfseries,
  watermark color=AliceBlue!85!Navy,enhanced}

If a |tcolorbox| is set to be \refKeyLe{/tcb/breakable}, then the following
algorithm is executed:

如果一个 |tcolorbox| 被设置为 \refKeyLe{/tcb/breakable},则执行以下算法:
\begin{enumerate}
\item The box content is read to a box register similar but not identical
  to the unbreakable case.
\\盒子的内容被读取到一个类似但不完全相同的盒子注册中,这个注册类似于不可破坏的情况。
\item If the total box fits into the current page, it is shipped out visibly
  unbroken and the algorithm stops.
\\如果总的盒子能够完整地放入当前页面,它将被清晰地运送出去,算法停止。
  \begin{tcolorbox}[title=Unbroken Box,watermark text=unbroken]
  The box.
  \end{tcolorbox}
\item Otherwise, it is checked if at least \refKeyLe{/tcb/lines before break}
  of the upper box can be placed on the current page.
  If not, a page break is inserted and the algorithm goes back to Step 2.
\\否则,将检查是否可以将上部盒子中至少\refKeyLe{/tcb/lines before break}行放置在当前页面上。如果不能,则插入一个分页符,算法返回到步骤2。
\item Now, the \emph{break sequence} starts.
  The upper box part or the lower box part is split such that it fits
  into the current page. The fitting part is named \emph{first part} of
  the \emph{break sequence} and shipped out.
\\现在,\emph{断点序列}开始。将上部或下部的箱子分割,以适应当前页面。适合的部分被命名为\emph{断点序列的第一部分}并发送出去。
  \begin{tcolorbox}[title=Broken Box,watermark text=first,skin=enhancedfirst]
  The box.
  \end{tcolorbox}
\item
  If the remaining content of the total box
  fits into the current page, the algorithm continues with Step 7, else
  with Step 6.
\\如果总框内剩余的内容适合当前页面,则算法继续进行步骤7,否则进行步骤6。
\item
  The upper box part or the lower box part is split such that it fits
  into the current page. The fitting part is named \emph{middle part} of
  the \emph{break sequence} and shipped out.
  Then, the algorithm goes back to Step 5.
\\上部或下部的方框被分割,以适应当前页面。适配部分称为“断点序列”的“中间部分”,并被导出。然后,算法返回到第5步。
  \begin{tcolorbox}[watermark text=middle,skin=enhancedmiddle]
  The box.
  \end{tcolorbox}
\item
  The remaining part is named \emph{last part} of
  the \emph{break sequence} and shipped out. The algorithm stops.
\\剩余的部分被称为“断点序列”的“最后一部分”,并被发出。算法停止。
  \begin{tcolorbox}[watermark text=last,skin=enhancedlast]
  The box.
  \end{tcolorbox}
\end{enumerate}
}

The algorithm takes care that the optional segmentation line never appears at
the end of a box. The optional lower box part is also checked to
have at least \refKeyLe{/tcb/lines before break}.

该算法确保可选的分割线永远不会出现在盒子的末尾。也要检查可选的下部盒子部分,以确保至少有\refKeyLe{/tcb/lines before break}。

% \clearpage
In principle, all boxes of the \emph{break sequence} share the same geometric
parameters. The differences are:

原则上,所有“断点序列”的盒子共享相同的几何参数。区别在于:
\begin{itemize}
\item The given \refKeyLe{/tcb/before} and \refKeyLe{/tcb/after} values are
  used only before the \emph{first} and after the \emph{last} part
  of the \emph{break sequence}.
\\给定的 \refKeyLe{/tcb/before} 和 \refKeyLe{/tcb/after} 值仅在 \emph{断行序列} 的\emph{第一个}部分之前和\emph{最后一个}部分之后使用。
\item A special behavior between the parts of the \emph{break sequence} can
  be given by \refKeyLe{/tcb/toprule at break},
  \refKeyLe{/tcb/bottomrule at break},
  \refKeyLe{/tcb/enlarge top at break by}, and
  \refKeyLe{/tcb/enlarge bottom at break by}.
\\\emph{断行序列}的各个部分之间可以通过\refKeyLe{/tcb/toprule at break}、\refKeyLe{/tcb/bottomrule at break}、\refKeyLe{/tcb/enlarge top at break by}和\refKeyLe{/tcb/enlarge bottom at break by}来实现特定的行为。
\item The \refKeyLe{/tcb/skin} decides \emph{how} the \emph{first}, \emph{middle},
  and \emph{last} part look like. Actually, every part type has its own
  skin given by the options  \refKeyLe{/tcb/skin first}, \refKeyLe{/tcb/skin middle}, and
  \refKeyLe{/tcb/skin last}. Typically, these options are set automatically by
  the main skin, see Subsection \ref{subsec:breaksequence} from
  page \pageref{subsec:breaksequence}.
\\\refKeyLe{/tcb/skin}决定了第一部分、中间部分和最后部分的外观。实际上,每种部分类型都有自己的皮肤,由选项\refKeyLe{/tcb/skin first}、\refKeyLe{/tcb/skin middle}和\refKeyLe{/tcb/skin last}给出。通常,这些选项会被主要的皮肤自动设置,参见第\ref{subsec:breaksequence}小节,第\pageref{subsec:breaksequence}页。
\end{itemize}