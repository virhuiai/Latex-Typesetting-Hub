\subsection{Limitations and Known Bugs\\限制和已知问题}
\begin{itemize}
\item  The maximal total height of the upper and of the lower part
of normal breakable |tcolorbox|es is about 65536pt (ca.\,2300cm)
apiece. If such a part gets longer, the output will get buggy
without warning.

普通易碎的|tcolorbox|的上部和下部的最大总高度约为65536pt(约为2300cm)。如果这样的部分变得更长,输出将会出现错误,而没有任何警告。

For very oversized boxes which are longer than 65536pt, use
the \docValue{unlimited} value for  \refKeyLe{/tcb/breakable}.
With the \docValue{unlimited} setting,
the applied algorithm has (virtually) no height limit for boxes, but
very likely the compiler memory will have to be increased for boxes longer
than 300 pages (depending on compiler settings and box content).

对于超出长度为65536pt的非常超大的盒子,请使用\refKeyLe{/tcb/breakable}的\docValue{unlimited}值。使用\docValue{unlimited}设置时,应用的算法对于盒子几乎没有高度限制,但是对于超过300页的盒子(取决于编译器设置和盒子内容),很可能需要增加编译器内存。

But it is recommended to use \docValue{unlimited} for critical large boxes only.
%,since there \emph{may} be a single interline space deviation (lost glue) around
%every 2300cm, e.\,g.\ a \refComLe{tcbline*} \emph{may} get lost.

但建议仅在关键的大盒子中使用 \docValue{unlimited}。因为每 2300cm 可能会出现单个行间距偏差(失去粘合剂),例如,\refComLe{tcbline*} \emph{可能会} 丢失。

\item You can nest an unbreakable |tcolorbox| inside another |tcolorbox|,
even inside a breakable one.
But you cannot not nest a breakable box inside a breakable box.

您可以在另一个不可打破的 |tcolorbox| 中嵌套一个不可打破的 |tcolorbox|, 即使在可打破的盒子中也可以。 但您不能在可打破的盒子中嵌套一个可打破的盒子。

The \refKeyLe{/tcb/breakable} key for a nested box is ignored
automatically\footnote{Until |tcolorbox| 3.04, the \refKeyLe{/tcb/breakable} key
was not ignored for nested boxes.}, i.\,e.\ inner
boxes are always unbreakable.

对于嵌套框的\refKeyLe{/tcb/breakable}关键字将自动被忽略\footnote{直到|tcolorbox| 3.04,\refKeyLe{/tcb/breakable}关键字对于嵌套框不会被忽略。},即内部框始终不可分割。

After all, in the unlikely case you really want to have the nested box to be breakable,
use \refKeyLe{/tcb/enforce breakable} for the nested
box\footnote{\refKeyLe{/tcb/enforce breakable} acts like \refKeyLe{/tcb/breakable} until |tcolorbox| 3.04.}.
\textbf{But, a breakable box inside a breakable box will usually give a mess.}

毕竟,如果您真的想让嵌套框可打破,可以在嵌套框中使用\refKeyLe{/tcb/enforce breakable}(\refKeyLe{/tcb/enforce breakable}在|tcolorbox| 3.04之前的版本中类似于\refKeyLe{/tcb/breakable})。但是,可打破的框内部通常会变得混乱。
\item\tcbdocmarginnote{\tcbdocnew{2020-09-17}}
Depending on the \LaTeX\ engine, if your text content contains some text
color changing commands, your color may not survive the break to the next box.
See the documentation for \refKeyLe{/tcb/use color stack} for more information.

根据 \LaTeX\ 引擎的不同,如果您的文本内容包含一些文本颜色变化命令,则颜色可能无法保留到下一个框中。请参阅 \refKeyLe{/tcb/use color stack} 的文档以获取更多信息。
\item\tcbdocmarginnote{\tcbdocnew{2014-10-30}}
The |perpage| option of the |footmisc| package is deliberately deactivated
inside a breakable box since all footnotes are placed at the end
of the box (possibly far away from the reference point).

|footmisc| 包的 |perpage| 选项在可断行的盒子内故意被禁用,因为所有脚注都放置在盒子的末尾(可能远离参考点)。
\item\tcbdocmarginnote{\tcbdocnew{2016-02-15}}
Making a box \refKeyLe{/tcb/breakable} which actually is not broken creates
a box which acts \emph{almost} like an unbreakable box. Visual differences
are kept as indiscernible as possible, but can appear with certain
\refKeyLe{/tcb/before} and \refKeyLe{/tcb/after} settings, especially, if there
is an automatic page break before the box.

创建一个实际上不会被打破的盒子 \refKeyLe{/tcb/breakable},会创建一个行为 \emph{几乎} 像不可打破的盒子。视觉差异尽可能保持不可辨别,但在某些 \refKeyLe{/tcb/before} 和 \refKeyLe{/tcb/after} 设置下可能会出现,特别是在盒子之前有自动分页的情况下。
\item\tcbdocmarginnote{\tcbdocnew{2016-05-25}}
Lua\TeX\ version 0.95 changes the behavior of the basic |\vsplit| (a bug?!)
resulting in badly broken boxes. Thanks to Jeremy Engel,
the \mylib{breakable} library contains a patch for this which
also loads the the |ifluatex| package.

Lua\TeX\ 版本0.95更改了基本的|\vsplit|的行为(这是一个错误?!),导致盒子严重破碎。感谢Jeremy Engel,\mylib{breakable}库包含了一个修补程序,还加载了|ifluatex|包。
\end{itemize}

