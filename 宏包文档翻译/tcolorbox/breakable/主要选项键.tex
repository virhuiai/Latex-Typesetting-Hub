
\subsection{Main Option Keys\\主要选项键}
\begin{docTcbKey}[][doc updated=2017-02-01]{breakable}{\colOpt{=true\textbar false\textbar unlimited}}{default |true|, initially |false|}
Allows the |tcolorbox| to be breakable. If the box is larger than the
available space at the current page, the box is automatically broken
and continued to the next page. All sorts of |tcolorbox| can be made
breakable. It depends on the skin how the breaking looks like.
If you do not know better, use \refKeyLe{/tcb/enhanced} for breaking a box.
The parts of the \emph{break sequence} are numbered
by the counter |tcbbreakpart|.

允许 |tcolorbox| 可以被分页。如果盒子比当前页可用空间大,则盒子会自动分页并继续到下一页。所有类型的 |tcolorbox| 都可以被分页。这取决于皮肤如何进行分页。 如果您不知道更好的方法,请使用 \refKeyLe{/tcb/enhanced} 选项来分页盒子。 \emph{分页序列} 的部分由计数器 |tcbbreakpart| 编号。
\begin{itemize}
\item\docValue{false}: Sets the |tcolorbox| to be unbreakable.
\\将 |tcolorbox| 设置为不可分页。
\item\docValue{true}: Breaks the |tcolorbox| from one page to another.
The maximal total height of the upper and of the lower part is
about 65536pt (ca.\,2300cm or ca.\,90 pages) apiece.
\\将 |tcolorbox| 从一页分页到另一页。上部分和下部分的最大总高度分别约为65536pt(约为2300cm或约为90页)。
\item\docValue{unlimited}: Experimental code for unlimited total height of
breakable boxes.
%There \emph{may} be a single interline space deviation (lost glue) around every 2300cm.
For boxes longer than 300 pages (or even shorter ones) the
compiler memory will have to be increased.
\\用于无限制分页盒子的实验性代码。对于超过300页(甚至更短)的盒子,需要增加编译器内存。可能会有单个行间距偏差(丢失粘合剂)约为2300cm左右。
\end{itemize}

\begin{dispListing}
% \usepackage{lipsum}  % preamble
\tcbset{enhanced jigsaw,colback=red!5!white,colframe=red!75!black,
  watermark color=yellow!25!white,watermark text=\arabic{tcbbreakpart},
  fonttitle=\bfseries}

\begin{tcolorbox}[breakable,title=My breakable box]
\lipsum[1-6]
\end{tcolorbox}
\end{dispListing}
\end{docTcbKey}
{\tcbusetemp}


\begin{docTcbKey}{unbreakable}{}{no value, initially set}
Sets the |tcolorbox| to be unbreakable.

将 |tcolorbox| 设置为不可分割。
\end{docTcbKey}


\begin{docTcbKey}{enforce breakable}{}{no value}
A |tcolorbox| inside a |tcolorbox| is automatically set to be unbreakable.
Using \refKeyLe{/tcb/breakable} on such an inner box has no effect.
If one \emph{really} wants the inner box to be breakable, use \refKeyLe{/tcb/enforce breakable}.
\textbf{This will usually give a mess of shattered boxes. You are advised to not use this option.}\\
Note that \refKeyLe{/tcb/enforce breakable} has the functionality
that \refKeyLe{/tcb/breakable} had until package version 3.04
and exists for backward compatibility.

|tcolorbox|中的内部|tcolorbox|会自动设置为不可断开的。 在这样的内部盒子上使用\refKeyLe{/tcb/breakable}没有任何效果。 如果您\emph{真的}想要内部盒子是可断开的,请使用\refKeyLe{/tcb/enforce breakable}。 \textbf{这通常会导致一堆粉碎的盒子。建议您不要使用此选项。}\ 请注意,\refKeyLe{/tcb/enforce breakable}具有\refKeyLe{/tcb/breakable}在3.04版本之前的功能,用于向后兼容。
\end{docTcbKey}


\begin{docTcbKey}[][doc updated=2018-07-26]{title after break}{=\meta{text}}{no default, initially empty}
The \refKeyLe{/tcb/title} is used only for the \emph{first} part of a
\emph{break sequence}. Use |title after break| to create a heading line
with \meta{text} as content for all following parts.
Also see \refKeyLe{/tcb/extras title after break} for formatting the title text.

\refKeyLe{/tcb/title} 仅用于一个 \emph{分段序列} 的 \emph{第一部分}。 使用 |title after break| 创建一个标题行,其中 \meta{text} 作为所有后续部分的内容。另请参见 \refKeyLe{/tcb/extras title after break} 以格式化标题文本。
\end{docTcbKey}


\begin{docTcbKey}{notitle after break}{}{no value, initially set}
  Removes the title line or following parts in a  \emph{break sequence} if set before.
\end{docTcbKey}


\begin{docTcbKey}{adjusted title after break}{=\meta{text}}{style, no default, initially unset}
  Works like \refKeyLe{/tcb/adjusted title} but applied to \refKeyLe{/tcb/title after break}.
\end{docTcbKey}


\begin{docTcbKey}{lines before break}{=\meta{number}}{no default, initially |2|}
  Assures that the given \meta{number} of lines of the upper box part or
  the lower box part are placed before a break happens.
\end{docTcbKey}

\clearpage
\begin{docTcbKey}[][doc updated=2017-07-05]{break at}{=\meta{length}\colOpt{/\meta{length}/\ldots/\meta{length}}}{no default, initially |0pt|}
  Defines break points at the given \meta{length} values.
  The first \meta{length} defines the (maximal) height of the first partial box,
  the second \meta{length} defines the (maximal) height of the second partial box,
  and so on. The last \meta{length} value is applied to all following partial boxes if any.
  \begin{itemize}
  \item Setting a \meta{length} to |0pt| means that the naturally available
        space is used for breaking.
  \item Setting a \meta{length} to a negative value means that
        the sum of this negative value and the naturally available space is used
        for breaking (boxes will shrink in height).
        Note that before version 4.10 negative values were treated like |0pt|.
  \end{itemize}
\begin{dispExample}
% \usepackage{multicol,lipsum}
\begin{multicols}{3}\footnotesize
Breakable boxes inside a |multicols| environment need special attendance.
They are broken by default at |\textheight|.
The |break at| option can be used to insert better break points by hand.
\begin{tcolorbox}[enhanced jigsaw,size=small,vfill before first,
  colframe=red,colback=yellow!10!white,before title=\raggedright,
  title={Broken box inside a |multicols| environment},fonttitle=\bfseries,
  enforce breakable,% use only breakable in the real world!
  pad at break=1mm,break at=3cm/6.3cm ]
\lipsum[1]
\end{tcolorbox}
\refKeyLe{/tcb/height fixed for} may also be considered for |multicols| environments.
\end{multicols}
\end{dispExample}
\end{docTcbKey}

\enlargethispage*{1cm}

\begin{docTcbKey}{enlargepage}{=\meta{length}\colOpt{/\meta{length}/\ldots/\meta{length}}}{no default, initially |0pt|}
  Inserts a |\enlargethispage|\marg{length} to the pages of the break sequence,
  i.\,e.\ allows one to enlarge (or shrink) partial boxes. The first \meta{length} is applied
  to the first partial box, the second \meta{length} is applied
  to the second partial box, and so on. The last \meta{length} value is applied
  to all following partial boxes if any. Note that floating boxes will not be enlarged.
\begin{dispListing}
\begin{tcolorbox}[breakable,enlargepage=0mm/\baselineskip/2\baselineskip/0mm,...
\end{dispListing}
  The example code enlarged the second partial box by one line, the third
  partial box by two lines, and all following parts are not enlarged.
  \begin{marker}
  If an automated page break occures before the first partial box, the
  page enlargement is applied to the page before the first partial box \emph{and}
  again to the page of the first partial box. Insert a manual break to prevent this.\\
  In general, |enlargepage| should be used at the final stage of a document
  for fine-tuning only.
  \end{marker}
\end{docTcbKey}

\clearpage
\begin{docTcbKey}{enlargepage flexible}{=\meta{length}}{no default, initially |0pt|}
  This allows an automated page enlargement for up to \meta{length}.
  The algorithm can use this to avoid breaking a box, if there is anough room
  after enlargement. Also, the \emph{last} partial box of a break sequence
  may be enlarged to avoid further breaking.\\
  Note that this potential enlargement is \emph{additive} to settings of
  \refKeyLe{/tcb/enlargepage}.
  But \refKeyLe{/tcb/enlargepage flexible} overwrites settings of
  \refKeyLe{/tcb/pad before break*} or \refKeyLe{/tcb/pad at break*}.
\begin{dispListing}
% The following setting hinders orphan lines for the last partial box
\tcbset{enlargepage flexible=\baselineskip}
\end{dispListing}
\end{docTcbKey}


\begin{docTcbKey}[][doc new=2014-12-15]{compress page}{\colOpt{=\meta{option}}}{default |all|, initially |baselineskip|}
  This option controls the space management on the page which contains the
  unbroken box or the first part of a \emph{break sequence}.
  Feasible \meta{option} values are:
  \begin{itemize}
  \item\docValue{all} (default value):
    All shrinkable glue on the page is potentially used for the
    unbroken box or the first part of a \emph{break sequence}. Thus, all
    vertical spaces on the page will potentially be reduced to their
    minimal values.
  \item\docValue{baselineskip} (initial value):
    Shrinkable glue up to one |\baselineskip| on the page is potentially used for the
    unbroken box or the first part of a \emph{break sequence}.
  \item\docValue{none}:
    The break algorithm respects the target size of the given glue values
    on the page. This was the inital value before version |3.34|.
  \end{itemize}
  \begin{marker}
  Note that the box \emph{content} is not influenced by this option.
  \end{marker}
\end{docTcbKey}


\begin{docTcbKey}{shrink break goal}{=\meta{length}}{no default, initially |0pt|}
  This is an emergency parameter if the break algorithm produces unpleasant
  breaks.
  It shrinks the goal height of the current box part by \meta{length}
  which may result in smaller boxes. Never use negative values.
  \emph{Usually, this option will never be needed at all.}
\end{docTcbKey}



\begin{docTcbKey}[][doc new=2020-10-09]{use color stack}{\colOpt{=true\textbar false}}{default |true|, initially |false|}
  Depending on the \LaTeX\ engine and loaded packages, if your text contains some
  color changing commands, your color may not survive the break to the next box.
  For some engines, there is support for additional color stacks which
  allow colors to survive breaks. Such an color stack  can be activated
  by \refKeyLe{/tcb/use color stack} with help of the |pdfcol| package.
  This can be done globally or per box.
  \begin{marker}
  Note that activating \refKeyLe{/tcb/use color stack} inserts a color command with a \emph{whatsit}
  at the begin of the upper part and of the lower part of a \refEnvLe{tcolorbox}.
  This \emph{may} add additional vertical space, e.g. if your box text starts
  with a list like \emph{enumerate}!
  \end{marker}
  \begin{itemize}
  \item pdf\TeX: color stacks supported.
  \item Lua\TeX: color stacks supported, but you should consider loading the
     |luacolor| package \emph{instead} which avoids the spacing problem.
  \item Xe\TeX: color stacks not supported (yet?). From hearsay,
    with the |fontspec| package, you may use |\addfontfeatures{Color=mycolor}|
    to add a font color which survives the break.
  \end{itemize}
  If |pdfcol| cannot initialize an additional color stack for the used engine,
  \refKeyLe{/tcb/use color stack} is silently ignored.

\clearpage
\begin{dispExample}
% \usepackage{multicol,lipsum}
\begin{multicols}{2}\footnotesize
Breakable box without color stack.
\begin{tcolorbox}[enhanced jigsaw,
  size=small, colframe=gray, colback=yellow!10!white, colupper=blue,
  enforce breakable,% use only breakable in the real world!
  vfill before first, pad at break=1mm, break at=3.3cm ]
    \begin{itemize}\item Some blue text.\end{itemize}
    {\color{red}\itshape\lipsum[2]}\par
    More blue text.
\end{tcolorbox}
Text after box.
\end{multicols}
\end{dispExample}

We do again with \refKeyLe{/tcb/use color stack}. Observe the additional spacing
at the begin of the box:

\begin{dispExample}
% \usepackage{multicol,lipsum}
\begin{multicols}{2}\footnotesize
Breakable box with color stack.
\begin{tcolorbox}[enhanced jigsaw, use color stack,
  size=small, colframe=gray, colback=yellow!10!white, colupper=blue,
  enforce breakable,% use only breakable in the real world!
  vfill before first, pad at break=1mm, break at=3.3cm ]
    \begin{itemize}\item Some blue text.\end{itemize}
    {\color{red}\itshape\lipsum[2]}\par
    More blue text.
\end{tcolorbox}
Text after box.
\end{multicols}
\end{dispExample}

\end{docTcbKey}