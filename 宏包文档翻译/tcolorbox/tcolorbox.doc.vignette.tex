% !TeX root = tcolorbox.tex
% include file of tcolorbox.tex (manual of the LaTeX package tcolorbox)
%\clearpage
% 
% \include*{vignette/vignette}
% \include*{vignette/vignette_通用几何设置}
\include*{vignette/vignette_通用颜色和样式设置}

\end{document}

% % 


\subsection{Generic Fading Settings}\label{subsec:vignettefading}

The |fadings| library of |tikz|  is loaded
automatically by the \mylib{vignette} library.
Amongst others, the fadings
\docFading{west},
\docFading{east},
\docFading{north}, and
\docFading{south} are defined inside the |fadings| library.

The \mylib{vignette} library adds some more fadings called
\docFading{semi west},
\docFading{semi east},
\docFading{semi north}, and
\docFading{semi south}.
These fadings are much \emph{weaker} than the normal fadings.

\begin{dispExample*}{sbs,righthand width=3cm,center lower}
\begin{tikzpicture}
  \fill [black!20] (0,0) rectangle (1,1);
  \path [pattern=checkerboard,pattern color=black!30]
    (0,0) rectangle (1,1);
  \fill [path fading=semi west,blue] (0,0) rectangle (1,1);
\end{tikzpicture}
\end{dispExample*}



\begin{tcboxedraster}{base example,title=Comparison of the Fadings}
  \def\doShadingExample#1{%
    \begin{tcolorbox}[sbs,size=fbox,colback=white,lower separated=false,
      righthand width=2cm,left=5mm]
    \docFading{#1}\tcblower
    \begin{tikzpicture}
        \fill [black!20] (0,0) rectangle (1,1);
        \path [pattern=checkerboard,pattern color=black!30] (0,0) rectangle (1,1);
        \fill [path fading=#1,blue] (0,0) rectangle (1,1);
    \end{tikzpicture}
    \end{tcolorbox}}%
  \doShadingExample{west}
  \doShadingExample{east}
  \doShadingExample{north}
  \doShadingExample{south}
  \doShadingExample{semi west}
  \doShadingExample{semi east}
  \doShadingExample{semi north}
  \doShadingExample{semi south}
\end{tcboxedraster}


\clearpage

\begin{vigTcbKey}[][doc new=2016-04-22]{fade in}{\colOpt{=\marg{style}}}{style, default |white|}
  Sets the four style options
  \refKeyLe{/tcb/vig/north style},
  \refKeyLe{/tcb/vig/south style},
  \refKeyLe{/tcb/vig/east style}, and
  \refKeyLe{/tcb/vig/west style}
  such that the paths fade from outside to inside.
\begin{dispExample*}{sbs,righthand width=3cm,center lower}
\begin{tikzpicture}
  \fill [black!20] (-0.5,-0.5) rectangle (1.5,1.5);
  \path [pattern=checkerboard,pattern color=black!30]
    (-0.5,-0.5) rectangle (1.5,1.5);
  \tcbvignette{fade in=blue}
\end{tikzpicture}
\end{dispExample*}
\end{vigTcbKey}


\begin{vigTcbKey}[][doc new=2016-04-22]{fade out}{\colOpt{=\marg{style}}}{style, default |white|}
  Sets the four style options
  \refKeyLe{/tcb/vig/north style},
  \refKeyLe{/tcb/vig/south style},
  \refKeyLe{/tcb/vig/east style}, and
  \refKeyLe{/tcb/vig/west style}
  such that the paths fade from inside to outside.
\begin{dispExample*}{sbs,righthand width=3cm,center lower}
\begin{tikzpicture}
  \fill [black!20] (-0.5,-0.5) rectangle (1.5,1.5);
  \path [pattern=checkerboard,pattern color=black!30]
    (-0.5,-0.5) rectangle (1.5,1.5);
  \tcbvignette{fade out=blue}
\end{tikzpicture}
\end{dispExample*}
\end{vigTcbKey}


\begin{vigTcbKey}[][doc new=2016-04-22]{semi fade in}{\colOpt{=\marg{style}}}{style, default |white|}
  Sets the four style options
  \refKeyLe{/tcb/vig/north style},
  \refKeyLe{/tcb/vig/south style},
  \refKeyLe{/tcb/vig/east style}, and
  \refKeyLe{/tcb/vig/west style}
  such that the paths fade weak from outside to inside.
\begin{dispExample*}{sbs,righthand width=3cm,center lower}
\begin{tikzpicture}
  \fill [black!20] (-0.5,-0.5) rectangle (1.5,1.5);
  \path [pattern=checkerboard,pattern color=black!30]
    (-0.5,-0.5) rectangle (1.5,1.5);
  \tcbvignette{semi fade in=blue}
\end{tikzpicture}
\end{dispExample*}
\end{vigTcbKey}


\begin{vigTcbKey}[][doc new=2016-04-22]{semi fade out}{\colOpt{=\marg{style}}}{style, default |white|}
  Sets the four style options
  \refKeyLe{/tcb/vig/north style},
  \refKeyLe{/tcb/vig/south style},
  \refKeyLe{/tcb/vig/east style}, and
  \refKeyLe{/tcb/vig/west style}
  such that the paths fade weak from inside to outside.
\begin{dispExample*}{sbs,righthand width=3cm,center lower}
\begin{tikzpicture}
  \fill [black!20] (-0.5,-0.5) rectangle (1.5,1.5);
  \path [pattern=checkerboard,pattern color=black!30]
    (-0.5,-0.5) rectangle (1.5,1.5);
  \tcbvignette{semi fade out=blue}
\end{tikzpicture}
\end{dispExample*}
\end{vigTcbKey}

\clearpage

It is possible to assign different fadings for each side of the vignette,
if needed. Therefore, the fadings have to be applied individually with
the four style options
  \refKeyLe{/tcb/vig/north style},
  \refKeyLe{/tcb/vig/south style},
  \refKeyLe{/tcb/vig/east style}, and
  \refKeyLe{/tcb/vig/west style}.
\begin{dispExample*}{sbs,righthand width=3cm,center lower}
\begin{tikzpicture}
  \fill [black!20] (-0.5,-0.5) rectangle (1.5,1.5);
  \path [pattern=checkerboard,pattern color=black!30]
    (-0.5,-0.5) rectangle (1.5,1.5);
  \tcbvignette{
    north style={blue,path fading=south},
    east style ={blue,path fading=semi west},
    south style={blue,path fading=semi north},
    west style ={blue,path fading=east}
  }
\end{tikzpicture}
\end{dispExample*}

\begin{dispExample*}{sbs,righthand width=3cm,center lower}
\begin{tikzpicture}
  \fill [black!20] (-0.5,-0.5) rectangle (1.5,1.5);
  \path [pattern=checkerboard,pattern color=black!30]
    (-0.5,-0.5) rectangle (1.5,1.5);
  \tcbvignette{
    north style={blue,path fading=west},
    east style ={blue,path fading=south},
    south style={red,path fading=east},
    west style ={red,path fading=north}
  }
\end{tikzpicture}
\end{dispExample*}


% \clearpage
\subsection{Vignette as Underlay}\label{subsec:vignetteunderlay}

\begin{docTcbKey}[][doc new=2016-04-22]{underlay vignette}{\colOpt{=\marg{options}}}{style, no default}
  This puts a \refComLe{tcbvignette} with the given \meta{options}
  as \refKeyLe{/tcb/underlay} to a \refEnvLe{tcolorbox}.
  The dimensions of the \emph{vignette} are matched to the dimensions of
  the \refEnvLe{tcolorbox}. For example, \refKeyLe{/tcb/leftrule} is used as
  \refKeyLe{/tcb/vig/west size}. Also, \refKeyLe{/tcb/colframe} is used as
  \refKeyLe{/tcb/vig/raised color}.

  For a \refKeyLe{/tcb/breakable} tcolorbox, the \emph{vignette} is also
  been broken.
  Alternatively, \refComLe{tcbvignette} could be used directly inside
  an \refKeyLe{/tcb/underlay} with appropriate settings.

\begin{dispExample*}{sbs,righthand width=3cm,center lower}
\begin{tcolorbox}[enhanced,size=small,sharp corners,
  colback=green!10,colframe=green!50!black,
  boxrule=2mm,titlerule=0mm,
  title=My title,center title,fonttitle=\bfseries,
  underlay vignette]
    This is a tcolorbox.
\end{tcolorbox}
\end{dispExample*}

\begin{dispExample*}{sbs,righthand width=3cm,center lower}
\begin{tcolorbox}[enhanced,size=small,arc=0pt,
  colback=blue!10,colframe=blue,boxrule=2mm,
  underlay vignette={size=1.5mm}]
    This is a tcolorbox.
\end{tcolorbox}
\end{dispExample*}

\begin{dispExample*}{sbs,righthand width=3cm,center lower}
\begin{tcolorbox}[enhanced,size=small,sharp corners,
  colframe=red,interior hidden,boxrule=2mm,
  colupper=white,center upper,fontupper=\bfseries,
  underlay vignette]
    This is a tcolorbox.
\end{tcolorbox}
\end{dispExample*}

\begin{dispExample*}{sbs,righthand width=3cm,center lower}
\begin{tcolorbox}[enhanced,size=small,sharp corners,
  colback=red!50!yellow,frame hidden,boxrule=2mm,
  underlay vignette={color from=red!50!yellow to white,
     draw method=clipped,size=2.1mm}]
    This is a tcolorbox.
\end{tcolorbox}
\end{dispExample*}

\begin{dispExample*}{sbs,righthand width=3cm,center lower}
\tcbox[enhanced,sharp corners,colback=red!10,colframe=red]
  {Test}

\tcbox[enhanced,sharp corners,colback=red!10,colframe=red,
  underlay vignette]{Test}
\end{dispExample*}

\end{docTcbKey}


\clearpage
\begin{docTcbKey}[][doc new=2016-04-22]{underlay raised shading vignette}{\colOpt{=\marg{options}}}{style, no default}
  This is a special style derived from \refKeyLe{/tcb/underlay vignette},
  where the frame color is shaded to create a soft raised frame impression.

\begin{dispExample*}{sbs,righthand width=3cm,center lower}
\begin{tcolorbox}[enhanced,sharp corners,
  colback=green!10,
  colframe=green!50!black,
  size=small,boxrule=2mm,titlerule=0mm,
  title=My title,center title,fonttitle=\bfseries,
  underlay raised shading vignette]
    This is a tcolorbox.
\end{tcolorbox}
\end{dispExample*}
\end{docTcbKey}



\begin{docTcbKey}[][doc new=2016-04-22]{underlay raised fading vignette}{\colOpt{=\marg{options}}}{style, no default}
  This style gives a similar effect as \refKeyLe{/tcb/underlay raised shading vignette},
  but a path fading is used here. Different optical impression are very
  previewer-dependent.
\begin{dispExample*}{sbs,righthand width=3cm,center lower}
\begin{tcolorbox}[enhanced,sharp corners,
  colback=green!10,
  colframe=green!50!black,
  size=small,boxrule=2mm,titlerule=0mm,
  title=My title,center title,fonttitle=\bfseries,
  underlay raised fading vignette]
    This is a tcolorbox.
\end{tcolorbox}
\end{dispExample*}
\end{docTcbKey}

\begin{docTcbKey}[][doc new=2016-04-22]{underlay shade in vignette}{\colOpt{=\marg{options}}}{style, no default}
  This is a special style derived from \refKeyLe{/tcb/underlay vignette},
  where the frame color is shaded into the interior color.
\begin{dispExample*}{sbs,righthand width=3cm,center lower}
\begin{tcolorbox}[enhanced,sharp corners,frame hidden,
  colback=green!10,
  colframe=green!50!black,
  size=small,boxrule=2mm,titlerule=0mm,
  underlay shade in vignette]
    This is a tcolorbox.
\end{tcolorbox}
\end{dispExample*}
\end{docTcbKey}


\clearpage
\subsection{Vignette as Finish}\label{subsec:vignettefinish}


\begin{docTcbKey}[][doc new=2016-04-22]{finish vignette}{\colOpt{=\marg{options}}}{style, no default}
  This puts a \refComLe{tcbvignette} with the given \meta{options}
  as \refKeyLe{/tcb/finish} to a \refEnvLe{tcolorbox}.
  The default style settings create a raised frame impression by
  drawing black and white color parts with reduced opacity.

\begin{dispExample*}{sbs,righthand width=3cm,center lower}
\begin{tcolorbox}[enhanced,size=small,
  colback=green!10,colframe=green!50!black,
  boxrule=0.5mm,titlerule=0mm,
  title=My title,center title,fonttitle=\bfseries,
  finish vignette={size=1mm}]
    This is a tcolorbox.
\end{tcolorbox}
\end{dispExample*}

\begin{dispExample*}{sbs,righthand width=3cm,center lower}
\tcbincludegraphics[blankest,width=3cm,
  finish vignette={size=3mm}]{pink_marble.png}
\end{dispExample*}
\end{docTcbKey}


\begin{docTcbKey}[][doc new=2016-04-22]{finish raised fading vignette}{\colOpt{=\marg{options}}}{style, no default}
  This puts a \refComLe{tcbvignette} with the given \meta{options}
  as \refKeyLe{/tcb/finish} to a \refEnvLe{tcolorbox}.
  The default style settings create a soft raised frame impression by
  drawing fading black and white color parts.

\begin{dispExample*}{sbs,righthand width=3cm,center lower}
\begin{tcolorbox}[enhanced,size=small,
  colback=green!10,colframe=green!50!black,
  boxrule=0.5mm,titlerule=0mm,
  title=My title,center title,fonttitle=\bfseries,
  finish raised fading vignette={size=1mm}]
    This is a tcolorbox.
\end{tcolorbox}
\end{dispExample*}

\begin{dispExample*}{sbs,righthand width=3cm,center lower}
\tcbincludegraphics[blankest,width=3cm,
  finish raised fading vignette={size=3mm}]{pink_marble.png}
\end{dispExample*}

\end{docTcbKey}


\clearpage
\begin{docTcbKey}[][doc new=2016-04-22]{finish fading vignette}{\colOpt{=\marg{options}}}{style, no default}
  This puts a \refComLe{tcbvignette} with the given \meta{options}
  as \refKeyLe{/tcb/finish} to a \refEnvLe{tcolorbox}.
  The default style settings fade the box into white from inside to outside.
  Note that \refKeyLe{/tcb/vig/over node} is used here.
  \refKeyLe{/tcb/vig/over node offset} can be adapted to overlap the box
  more or less. The fade color can be set using
  \refKeyLe{/tcb/vig/base color}.

\begin{dispExample*}{sbs,righthand width=3cm,center lower}
\begin{tcolorbox}[enhanced,size=small,
  colback=green!10,colframe=green!50!black,
  boxrule=0.5mm,titlerule=0mm,
  title=My title,center title,fonttitle=\bfseries,
  finish fading vignette={size=2mm}]
    This is a tcolorbox.
\end{tcolorbox}
\end{dispExample*}

\begin{dispExample*}{sbs,righthand width=3cm,center lower}
\tcbincludegraphics[blankest,width=3cm,
  finish fading vignette={size=3mm}]{pink_marble.png}
\end{dispExample*}

\begin{dispExample*}{sbs,righthand width=3cm,center lower}
\begin{tcolorbox}[colback=blue!50!black,size=small,
  title=Example]
\tcbincludegraphics[blankest,
  finish fading vignette={base color=blue!50!black,size=3mm,
    over node offset=0.2mm}]{pink_marble.png}
\end{tcolorbox}
\end{dispExample*}

\end{docTcbKey}


\begin{dispExample*}{}
\begin{tcbitemize}[raster columns=3,bicolor,
  raster equal height,sharp corners,boxrule=2mm,
  colframe=red,colback=yellow!5,colbacklower=yellow!25!red!20]
\tcbitem A
\tcbitem[underlay vignette] B
\tcbitem[underlay={\tcbvignette{inside node=interior,
  lowered color=red,size=1mm}}] C
\tcbitem[underlay vignette,
  underlay={\tcbvignette{inside node=interior,
  lowered color=red,size=1mm}}] D
\tcbitem[boxrule=3mm,underlay vignette={size=2mm},
  underlay={\tcbvignette{inside node=interior,
  lowered color=red,size=1mm}}] E
\tcbitem[underlay raised shading vignette] F
\tcbitem[underlay raised shading vignette,
  underlay={\tcbvignette{inside node=interior,
  lowered color=red,size=1mm}}] G
\tcbitem[title=H1,underlay={\tcbvignette{inside node=interior,
  lowered color=red,size=1mm}},finish vignette] H2
\tcbitem[boxrule=0.25mm,colback=red!30,finish vignette] I1 \tcblower I2
\tcbitem[tile,colback=red!30,finish raised fading vignette] J1 \tcblower J2
\tcbitem[boxrule=1mm,underlay={\tcbvignette{inside node=interior,
  raised color=red,size=1mm}}] K
\tcbitem[boxrule=1mm,title=L1,underlay={\tcbvignette{inside node=title,
  lowered color=red,size=0.5mm}}] L2
\end{tcbitemize}
\end{dispExample*}
