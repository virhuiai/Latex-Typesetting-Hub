% \clearpage
\begin{docCommand}{newtcbox}{\oarg{init options}\brackets{\texttt{\textbackslash}\meta{name}}\oarg{number}\oarg{default}\marg{options}}
Creates a new macro \texttt{\textbackslash}\meta{name} based on \refComLe{tcbox}.
Basically, |\newtcbox| operates like |\newcommand|.
The new macro \texttt{\textbackslash}\meta{name} optionally takes \meta{number}$+1$ arguments, where
\meta{default} is the default value for the optional first argument.
The \meta{options} are given to the underlying |tcbox|.
The \meta{init options} allow setting up automatic numbering,
see Section \ref{sec:initkeys} from page \pageref{sec:initkeys}.


% 基于 \refComLe{tcbox} 创建一个新的宏命令 \texttt{\textbackslash}\meta{name}。%
% \meta{default} 是第一个可选参数的默认值。%
% \meta{options}指定的选项传递到底层的 |tcbox|。%
% 可以在 \meta{init options} 中设置自动编号,%
% 见 \pageref{sec:initkeys} 页的 \ref{sec:initkeys} 小节。

基于\refComLe{tcbox}创建一个新的宏\texttt{\textbackslash}\meta{name}。 基本上,|\newtcbox| 的操作类似于|\newcommand|。 新的宏\texttt{\textbackslash}\meta{name}可选地接受\meta{number}$+1$个参数,其中 \meta{default}是可选第一个参数的默认值。 \meta{options}应用于底层的|tcbox|。 \meta{init options}允许设置自动编号, 参见第\pageref{sec:initkeys}页的第\ref{sec:initkeys}节。
\begin{dispExample*}{sbs,lefthand ratio=0.6}
\newtcbox{\mybox}{colback=red!5!white,
  colframe=red!75!black}

\mybox{这是我的例子。}
\end{dispExample*}

\begin{dispExample*}{sbs,lefthand ratio=0.6}
\newtcbox{\mybox}[1]{colback=red!5!white,
  colframe=red!75!black,fonttitle=\bfseries,
  title={#1}}

\mybox{必选的标题}{这是我的盒子。}
\end{dispExample*}

\begin{dispExample*}{sbs,lefthand ratio=0.6}
\newtcbox{\mybox}[2][]{colback=red!5!white,
  colframe=red!75!black,fonttitle=\bfseries,
  title={#2},#1}

\mybox[colback=yellow]{Hello there}%
  {首参可选,次参必填!}
\end{dispExample*}

\inputpreamblelisting{B}

\begin{dispExample*}{sbs,lefthand ratio=0.6}
\pbbox[colback=yellow]{Hello there}%
  {标题中使用pabox的计数器。}
\end{dispExample*}

% todo 可以将所有的项加上注释
\begin{dispExample}
\newtcbox{\mybox}[1][red]{on line,
arc=0pt,outer arc=0pt,
colback=#1!10!white,colframe=#1!50!black,
boxsep=0pt,
left=1pt,right=1pt,top=2pt,bottom=2pt,
boxrule=0pt,bottomrule=1pt,toprule=1pt%
}
\newtcbox{\xmybox}[1][red]{on line,
arc=7pt,
colback=#1!10!white,colframe=#1!50!black,
before upper={\rule[-3pt]{0pt}{10pt}},
boxrule=1pt,
boxsep=0pt,
left=6pt,right=6pt,top=2pt,bottom=2pt}

The \mybox[green]{quick} brown \mybox{fox} \mybox[blue]{jumps} over the
\mybox[green]{lazy} \mybox{dog}.\par 
The \xmybox[green]{quick} brown \xmybox{fox} \xmybox[blue]{jumps} over the
\xmybox[green]{lazy} \xmybox{dog}.
\end{dispExample}

\end{docCommand}


% \enlargethispage*{1cm}
\begin{docCommand}{renewtcbox}{\oarg{init options}\brackets{\texttt{\textbackslash}\rmfamily\meta{name}}\oarg{number}\oarg{default}\marg{options}}
Operates like \refComLe{newtcbox}, but based on |\renewcommand| instead of |\newcommand|.
An existing macro is redefined.

类似于 \refComLe{newtcbox}, 基于 |\renewcommand| %而非 |\newcommand|。
重新定义一个已经存在的宏。
\end{docCommand}