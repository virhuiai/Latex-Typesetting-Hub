\begin{docCommand}{newtcolorbox}{\oarg{init options}\marg{name}\oarg{number}\oarg{default}\marg{options}}
Creates a new environment \meta{name} based on \refEnvLe{tcolorbox}.
Basically, |\newtcolorbox| operates like |\newenvironment|. This means,
the new environment \meta{name} optionally takes \meta{number} arguments, where
\meta{default} is the default value for the optional first argument.
The \meta{options} are given to the underlying |tcolorbox|.
Note that \refKeyLe{/tcb/savedelimiter} is set to the given \meta{name}
automatically.
The \meta{init options} allow setting up automatic numbering,
see Section \ref{sec:initkeys} from page \pageref{sec:initkeys}.

% 基于\refEnvLe{tcolorbox}创建一个新的\meta{name}环境。%
% 基本上,|\newtcolorbox| 类似于 |\newenvironment|。%
% % 这意味着,
% 新环境 \meta{name} 拥有 \meta{number} 个参数,%
% 其中 \meta{default} 是可选的第一个参数的默认值。%
% \meta{options} 会被传递给底层的 |tcolorbox|。%
% 请注意 \refKeyLe{/tcb/savedelimiter} 将自动设置给 \meta{name}。%
% 可以在\meta{init options}中设置自动编号,%
% 请参阅\pageref{sec:initkeys}页的\ref{sec:initkeys}小节。

基于\refEnvLe{tcolorbox}创建一个新的环境\meta{name}。 基本上,|\newtcolorbox|的操作方式类似于|\newenvironment|。这意味着, 新环境\meta{name}可以选择性地使用\meta{number}个参数,其中 \meta{default}是可选第一个参数的默认值。 \meta{options}是提供给底层|tcolorbox|的选项。 请注意,\refKeyLe{/tcb/savedelimiter}会自动设置为给定的\meta{name}。 \meta{init options}允许设置自动编号, 请参见第\pageref{sec:initkeys}页的第\ref{sec:initkeys}节。
\begin{dispExample*}{sbs,lefthand ratio=0.8,title=只指定了两色颜色属性的例子}
\newtcolorbox{mybox}{colback=red!5!white,colframe=red!75!black}
\begin{mybox}
这是我的盒子。
\end{mybox}
\end{dispExample*}

\begin{dispExample*}{sbs,lefthand ratio=0.6,title=定义一个带有必选参数的环境}
\newtcolorbox{mybox}[1]{colback=red!5!white,
colframe=red!75!black,fonttitle=\bfseries,
title={#1}}
\begin{mybox}{你好呀}
这是我定义的盒子,带有必选的标题参数。%
\footnote{译者发现,dispExample中定义的盒子,%
在环境外失效。}
\end{mybox}
\end{dispExample*}

\begin{dispExample*}{sbs,lefthand ratio=0.6,title=定义一个带有可选参数的环境}
\newtcolorbox{mybox}[2][]{colback=red!5!white,
colframe=red!75!black,fonttitle=\bfseries,
colbacktitle=red!85!black,enhanced,
attach boxed title to top center={yshift=-2mm},
title={#2},#1}
\begin{mybox}[colback=yellow]{Hello there}
这是我定义的盒子,带有必选的标题参数和%
可选的配置参数。
\end{mybox}
\end{dispExample*}

\inputpreamblelisting{A}

\begin{tcolorbox}[breakable, title=译者对上面这个盒子的分析]
  上面由一条命令生成:

  \begin{minted}{latex}
  \inputpreamblelisting{A}
  \end{minted}

  \tcbline  

  这条命令定义为:
  \begin{minted}{latex}
  \newcommand{\inputpreamblelisting}[1]{%
  \tcbinputlisting{title=导言中的定义:,
    base example,coltitle=black,fonttitle=\itshape,titlerule=0pt,
    colbacktitle=Navy!15!ExampleBack,
    top=0mm,
    %before=\par\smallskip,%
    before skip balanced=4pt plus 2pt minus 1pt,
    after skip balanced=5pt plus 2pt minus 1pt,
    listing style=mydocumentation,
    listing only,listing file={\jobname_preamble_#1.tex}}%
  }
  \end{minted}

  \tcbline  

  可以看出代码内容来自 |\jobname_preamble_A.tex|\footnotemark:
  \begin{minted}{latex}
  \begin{tcbverbatimwrite}{\jobname_preamble_A.tex}
    \newtcolorbox[auto counter,number within=section]{pabox}[2][]{%
      colback=red!5!white,colframe=red!75!black,fonttitle=\bfseries,
      title=Examp.~\thetcbcounter: #2,#1}
    \end{tcbverbatimwrite}
  \end{minted}
\end{tcolorbox}
\footnotetext{{\tt \string\jobname} 的值就是当前被编译的文件名\jobname。}

%todo dispExample 的这几个参数后面加上引用
\begin{dispExample*}{sbs,lefthand ratio=0.6}
\begin{pabox}[colback=yellow]{你好哇!}
这是我的盒子,带一个可选参数%
和一个必填的带有编号的标题。
\end{pabox}
\end{dispExample*}
\end{docCommand}


\begin{docCommand}{renewtcolorbox}{\oarg{init options}\marg{name}\oarg{number}\oarg{default}\marg{options}}
Operates like \refComLe{newtcolorbox}, but based on |\renewenvironment|\footnote{instead of |\newenvironment|} .%.
An existing environment is redefined.

操作类似 \refComLe{newtcolorbox},但是是基于 |\renewenvironment|\footnote{而不是 |\newenvironment|}。一个已存在的环境被重新定义。
\end{docCommand}