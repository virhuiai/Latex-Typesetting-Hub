% \setcounter{section}{3}
% \setcounter{subsection}{0}
% \setcounter{subsubsection}{0}
\begin{docCommand}{tcblower}{}
Used inside \refEnvLe{tcolorbox} to separate the upper box part from
the optional lower box part. The upper and the lower part are treated
as separate functional units. If you only want to draw a line, see \refComLe{tcbline}.

% 用在 \refEnvLe{tcolorbox} 内部,用于将盒子upper和lower部分分开。upper和上lower被视为独立的功能单元。如果您只想画一条线,请参阅 \refComLe{tcbline}。
在\refEnvLe{tcolorbox}中用于将{\bf upper}与{\bf 可选的lower}分离。upper和lower被视为独立的功能单元。如果你只想画一条线,请参见\refComLe{tcbline}。

% \begin{引述之言}{virhuiai}
\begin{dispExample*}{sbs,title={{\tt 我们试试同时有tcbline和tcblower的情况}\hfill virhuiai}}
\begin{tcolorbox}
第1部分\_1
\tcbline
第1部分\_2
\tcblower
第2部分
\end{tcolorbox}
\end{dispExample*}
% \end{引述之言}
\end{docCommand} 