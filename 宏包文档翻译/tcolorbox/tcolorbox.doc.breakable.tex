% !TeX root = tcolorbox.tex
% include file of tcolorbox.tex (manual of the LaTeX package tcolorbox)
\setcounter{section}{18}
\section{Library \mylib{breakable}}\label{sec:breakable}%
\tcbset{external/prefix=external/breakable_}%
The library is loaded by a package option or inside the preamble by:

该库可以通过包选项或在导言区中加载:
\begin{dispListing}
\tcbuselibrary{breakable}
\end{dispListing}
This also loads the package |pdfcol|.

这还会加载 |pdfcol| 包。

% \include*{breakable/技术概述}
% \include*{breakable/限制和已知问题}
\include*{breakable/主要选项键}

\end{document}

\subsection{Option Keys for the Break Appearance}

\begin{docTcbKey}{toprule at break}{=\meta{length}}{no default, initially \texttt{0.5mm}}
  Sets the line width of the top rule to \meta{length} \emph{if} the box is \refKeyLe{/tcb/breakable}.
  In this case, it is applied to \emph{middle} and \emph{last} parts in a
  break sequence. Note that \refKeyLe{/tcb/toprule} overwrites this value
  if used afterwards.
\end{docTcbKey}


\begin{docTcbKey}{bottomrule at break}{=\meta{length}}{no default, initially \texttt{0.5mm}}
  Sets the line width of the bottom rule to \meta{length} \emph{if} the box is \refKeyLe{/tcb/breakable}.
  In this case, it is applied to \emph{first} and \emph{middle} parts in a
  break sequence. Note that \refKeyLe{/tcb/bottomrule} overwrites this value
  if used afterwards.
\end{docTcbKey}


\begin{docTcbKey}{topsep at break}{=\meta{length}}{no default, initially \texttt{0mm}}
  Additional vertical space of \meta{length} which is added at the top of
  \emph{middle} and \emph{last} parts in a break sequence. In general,
  it is not advisable to change this value if these parts start with a rule or a title.
\end{docTcbKey}

\begin{docTcbKey}{bottomsep at break}{=\meta{length}}{no default, initially \texttt{0mm}}
  Additional vertical space of \meta{length} which is added at the bottom of
  \emph{first} and \emph{middle} parts in a break sequence.
  In general, it is not advisable to change this value if these parts end with a rule.
\end{docTcbKey}

\begin{docTcbKey}{pad before break}{=\meta{length}}{style, no default, initially \texttt{3.5mm}}
  Sets the total amount of vertical space after the text content and before the
  break point to \meta{length}. This style sets \refKeyLe{/tcb/toprule at break} to |0pt|
  and changes \refKeyLe{/tcb/topsep at break} as required.
  In general, it is not advisable to change this value if the
  \emph{middle} and \emph{last} parts in a break sequence start with a rule or a title.
\end{docTcbKey}

\begin{docTcbKey}{pad before break*}{=\meta{length}}{style, no default}
  Sets \refKeyLe{/tcb/pad before break} to \meta{length} and
  \refKeyLe{/tcb/enlargepage flexible} to an appropriate value such that
  empty closing frames are avoided.
\end{docTcbKey}

\begin{docTcbKey}{pad after break}{=\meta{length}}{style, no default, initially \texttt{3.5mm}}
  Sets the total amount of vertical space after the break point and before the
  text content to \meta{length}. This style sets \refKeyLe{/tcb/bottomrule at break} to |0pt|
  and changes \refKeyLe{/tcb/bottomsep at break} as required.
  In general, it is not advisable to change this value if the
  \emph{first} and \emph{middle} parts in a break sequence end with a rule.
\end{docTcbKey}

\begin{docTcbKey}{pad at break}{=\meta{length}}{style, no default, initially \texttt{3.5mm}}
  Abbreviation for setting \meta{length} to \refKeyLe{/tcb/pad before break}
  and \refKeyLe{/tcb/pad after break}.
\end{docTcbKey}

\enlargethispage*{5mm}

\begin{docTcbKey}{pad at break*}{=\meta{length}}{style, no default}
  Sets \refKeyLe{/tcb/pad at break} to \meta{length} and
  \refKeyLe{/tcb/enlargepage flexible} to an appropriate value such that
  empty closing frames are avoided.
\end{docTcbKey}

\begin{dispListing}
% \usepackage{lipsum}  % preamble
\tcbset{colback=red!5!white,colframe=red!75!black,fonttitle=\bfseries}

\begin{tcolorbox}[enhanced jigsaw,breakable,pad at break*=0mm,
  title={For this box, the pad space at the break point is set to 0mm}]
  \lipsum[1-2]
\end{tcolorbox}
\end{dispListing}
{\tcbusetemp}


\begin{marker}
\refKeyLe{/tcb/pad at break} or \refKeyLe{/tcb/pad at break*}
should be used as very last option in an option list, because
they adapt other settings.
\end{marker}


\begin{marker}
Also see \refKeyLe{/tcb/enlarge top at break by}
and \refKeyLe{/tcb/enlarge bottom at break by}.
\end{marker}


\begin{docTcbKey}{height fixed for}{=\meta{part}}{no default, initially |none|}
  When certain amount of space is available for a partial box of a
  break sequence, the partial box typically is smaller than this space
  (depending on the box content). For given \meta{part}(s), the height can be
  set to all available space.
  \begin{itemize}
  \item\docValue{none}: Every partial |tcolorbox| is set with its natural height.
  \item\docValue{first}: The \emph{first} partial box is set to a height which matches the available space.
  \item\docValue{middle}: All \emph{middle} partial boxes are set to a height which matches the available space.
  \item\docValue{last}: The \emph{last} partial box is set to a height which matches
    the available space.
  \item\docValue{first and middle}: The \emph{first} and
    all \emph{middle} partial boxes are set to a height which matches the available space.
  \item\docValue{middle and last}: All \emph{middle} partial boxes and the \emph{last} partial box
    are set to a height which matches the available space.
  \item\docValue{all}: All partial boxes are set to a height which matches the available space.
  \end{itemize}
\begin{marker}
  If the box keeps unbroken, this option is not applied.
  See \refKeyLe{/tcb/height} for setting a fixed height for unbroken boxes.
  See \refKeyLe{/tcb/height fill} for giving unbroken boxes maximum height.
\end{marker}
\end{docTcbKey}


\begin{docTcbKey}{vfill before first}{\colOpt{=true\textbar false}}{default |true|, initially |false|}
  Inserts a |\vfill| at the begin of the \emph{first} partial box to move this
  partial box to the end of the current page. This may be used as an alternative
  to \refKeyLe{/tcb/height fixed for}|=|\docValue{first} to get justified
  columns or pages. The |\vfill| is not inserted, if the box gets not
  actually broken.
\end{docTcbKey}


\begin{docTcbKey}[][doc new=2017-03-20]{segmentation at break}{\colOpt{=true\textbar false}}{default |true|, initially |true|}
  If a breakable box contains an \emph{upper part} and a \emph{lower part} and
  the break happens at the \emph{segmentation} between both parts, then
  \begin{itemize}
  \item the segmenation line (or similar) is drawn as first element of the
    partial box containing the \emph{lower part}, if \refKeyLe{/tcb/segmentation at break}
    is set to be |true|.
  \item the segmenation line (or similar) is not drawn at all, if
    \refKeyLe{/tcb/segmentation at break} is set to be |false|.
    This may be preferable for skins like \refSkinLe{bicolor}, \refSkinLe{tile},
    or \refSkinLe{beamer}.
  \end{itemize}
\end{docTcbKey}


\clearpage
\subsection{Extra Options for Partial Boxes}\label{subsec:extras}


\begin{docTcbKey}[][doc new=2015-07-16]{extras}{=\marg{options}}{no default, initially unset}
  Adds |tcolorbox| \meta{options} to every box of a \emph{break sequence}
  after skin settings are done. This is quite late in box processing.
  Geometry and break settings should \emph{not be used} here, because they
  will either be ignored or have unexpected negative results. But it is possible
  to change most colors, skin effects, shadows, borders, frame code, etc.
  Note that using \refKeyLe{/tcb/extras} for every box is very seldom an
  advantage over setting the options directly. Usually, \refKeyLe{/tcb/extras first},
  \refKeyLe{/tcb/extras middle}, etc.\ are sensible to apply.
\end{docTcbKey}


\begin{docTcbKey}[][doc new=2015-07-16]{no extras}{}{style, no default, initially set}
  Removes all extras if set before.
\end{docTcbKey}


\begin{docTcbKey}[][doc new=2015-07-16]{extras broken}{=\marg{options}}{no default, initially unset}
  If the box is set to be \refKeyLe{/tcb/breakable} and \emph{is} broken actually,
  then the \meta{options} are added to every box of the \emph{break sequence}.
  \refKeyLe{/tcb/extras} overwrites this key.
\end{docTcbKey}

\begin{docTcbKey}[][doc new=2015-07-16]{extras unbroken}{=\marg{options}}{no default, initially unset}
  If the box is set to be \refKeyLe{/tcb/breakable} but \emph{is not} broken actually
  or if the box is set to be \refKeyLe{/tcb/unbreakable},
  then the \meta{options} are added to the box.
  \refKeyLe{/tcb/extras} overwrites this key.
\end{docTcbKey}

\begin{docTcbKey}[][doc new=2015-07-16]{no extras unbroken}{}{style, no default, initially set}
  Removes the unbroken extras if set before.
\end{docTcbKey}

\begin{docTcbKey}[][doc new=2015-07-16]{extras first}{=\marg{options}}{no default, initially unset}
  If the box is set to be \refKeyLe{/tcb/breakable} and \emph{is} broken actually,
  then the \meta{options} are added to the \emph{first} box of the break sequence.
  \refKeyLe{/tcb/extras} overwrites this key.
\end{docTcbKey}

\begin{docTcbKey}[][doc new=2015-07-16]{no extras first}{}{style, no default, initially set}
  Removes the first extras if set before.
\end{docTcbKey}

\begin{docTcbKey}[][doc new=2015-07-16]{extras middle}{=\marg{options}}{no default, initially unset}
  If the box is set to be \refKeyLe{/tcb/breakable} and \emph{is} broken actually,
  then the \meta{options} are added to every \emph{middle} box (if any) of the break sequence.
  \refKeyLe{/tcb/extras} overwrites this key.
\end{docTcbKey}

\begin{docTcbKey}[][doc new=2015-07-16]{no extras middle}{}{style, no default, initially set}
  Removes the middle extras if set before.
\end{docTcbKey}

\begin{docTcbKey}[][doc new=2015-07-16]{extras last}{=\marg{options}}{no default, initially unset}
  If the box is set to be \refKeyLe{/tcb/breakable} and \emph{is} broken actually,
  then the \meta{options} are added to the \emph{last} box of the break sequence.
  \refKeyLe{/tcb/extras} overwrites this key.
\end{docTcbKey}

\begin{docTcbKey}[][doc new=2015-07-16]{no extras last}{}{style, no default, initially set}
  Removes the last extras if set before.
\end{docTcbKey}

\begin{docTcbKey}[][doc new=2015-07-16]{extras unbroken and first}{=\marg{options}}{no default, initially unset}
  This is an abbreviation for setting
  \refKeyLe{/tcb/extras unbroken} and
  \refKeyLe{/tcb/extras first} together.
  \refKeyLe{/tcb/extras} overwrites this key.
\end{docTcbKey}

\begin{docTcbKey}[][doc new=2015-07-16]{extras middle and last}{=\marg{options}}{no default, initially unset}
  This is an abbreviation for setting
  \refKeyLe{/tcb/extras middle} and
  \refKeyLe{/tcb/extras last} together.
  \refKeyLe{/tcb/extras} overwrites this key.
\end{docTcbKey}

\begin{docTcbKey}[][doc new=2015-07-16]{extras unbroken and last}{=\marg{options}}{no default, initially unset}
  This is an abbreviation for setting
  \refKeyLe{/tcb/extras unbroken} and
  \refKeyLe{/tcb/extras last} together.
  \refKeyLe{/tcb/extras} overwrites this key.
\end{docTcbKey}

\clearpage

\begin{docTcbKey}[][doc new=2015-07-16]{extras first and middle}{=\marg{options}}{no default, initially unset}
  This is an abbreviation for setting
  \refKeyLe{/tcb/extras first} and
  \refKeyLe{/tcb/extras middle} together.
  \refKeyLe{/tcb/extras} overwrites this key.
\end{docTcbKey}


\begin{docTcbKey}[][doc new=2018-07-26]{extras title after break}{=\marg{options}}{no default, initially unset}
  If the box has a \refKeyLe{/tcb/title after break}, then the \meta{options}
  are added for all titles after the first break, i.e.\ all middle and last.
  The color, font, and alignment of titles after break can be adapted choosing
  \meta{options}, e.g.\ by \refKeyLe{/tcb/coltitle}, \refKeyLe{/tcb/fonttitle},
  \refKeyLe{/tcb/halign title}.
  Note that \refKeyLe{/tcb/colbacktitle} has to be placed into
  \refKeyLe{/tcb/extras middle and last}.
\end{docTcbKey}

\begin{docTcbKey}[][doc new=2018-07-26]{no extras title after break}{}{style, no default, initially set}
  Removes the title after break extras if set before.
\end{docTcbKey}

\bigskip

\begin{exdispExample}{extras}
% \usepackage{lipsum,multicol}
% \usetikzlibrary{decorations.pathmorphing}
% \tcbuselibrary{skins}
\newtcolorbox{mybox}[1][]{
  tile,
  colback=green!7,coltitle=blue!50!black,colbacktitle=blue!5,
  center title,
  toprule=1.25mm,bottomrule=1.25mm,
  extras unbroken and first={
    borderline north={0.25mm}{0.5mm}{blue,decoration={zigzag,amplitude=0.5mm},decorate}},
  extras unbroken and last={
    borderline south={0.25mm}{0.5mm}{blue,decoration={zigzag,amplitude=0.5mm},decorate}},
  #1
}

\begin{mybox}[title=My unbroken box]
\lipsum[1]
\end{mybox}

\begin{multicols}{3}
  \begin{mybox}[title=My broken box,
    enforce breakable,% use only breakable in the real world!
    break at=4.2cm,pad at break=2mm,
    height fixed for=first and middle,  ]
  \lipsum[2]
  \end{mybox}
\end{multicols}
\end{exdispExample}




\clearpage
\subsection{Breakable boxes and the \texttt{multicol} package}\label{subsec:multicol}
\begin{marker}
With version 4.10, the algorithm for detecting the available height
for a |tcolorbox| inside a |multicol| environment was improved with help
of Frank Mittelbach. This change \emph{may} impact existing user
code which \emph{may} have to be adapted.
\end{marker}

\begin{multicols}{2}
\begin{tcolorbox}[enhanced jigsaw,size=small,breakable,colback=yellow!10!white,
  colframe=red!50!white,break at=3cm,height fixed for=all]
Unbreakable |tcolorbox|es can be used without special care inside a
|multicols| environment from the |multicol| package \cite{mittelbach:multicol}.

Since version 3.10, a breakable |tcolorbox| detects, if it is used inside
a |multicols| environment. But choosing break points for a breakable box
cannot be done by the balancing routine of |multicols|. By default, boxes
will break at maximum column height. To get pleasant results, use the
\refKeyLe{/tcb/break at} and \refKeyLe{/tcb/height fixed for} options.
\end{tcolorbox}
\end{multicols}

\enlargethispage{\baselineskip}
\begin{dispListing}
% \usepackage{lipsum,multicol}  % preamble
\footnotesize
\begin{multicols}{2}
  \lipsum[1]
  \begin{tcolorbox}[enhanced jigsaw,breakable,size=title,
    colback=red!5!white,colframe=red!75!black,fonttitle=\bfseries,
    title=My breakable box,pad at break=1mm, break at=-\baselineskip/0pt ]
  \lipsum[2-4]
  \end{tcolorbox}
  \lipsum[4]
\end{multicols}
\end{dispListing}
{\tcbusetemp}

\clearpage

\begin{multicols}{2}
\small
This example is already set inside a |multicols| environment.
This time, a \emph{middle} part has full column height (here |\textheight|).
\refKeyLe{/tcb/height fixed for} is used to spread this box part over the full
height to align with neighboring columns.
\begin{dispListing}
% \usepackage{lipsum,multicol}
\lipsum[1]
\begin{tcolorbox}[enhanced jigsaw,
  breakable,
  size=title,
  colback=red!5!white,
  colframe=red!75!black,
  fonttitle=\bfseries,
  title=My breakable box,
  pad at break=2mm,
  break at=-\baselineskip/0pt,
  height fixed for=middle ]
\lipsum[2-7]
\end{tcolorbox}
\lipsum[8]
\end{dispListing}
{\tcbusetemp}
\end{multicols}


The following example has a |\tcolorbox| which fills the |\multicols|
environment completely. Here, \refKeyLe{/tcb/height fixed for} is used
to give all three columns the full height.
Note that the appropriate \refKeyLe{/tcb/break at} value is not computed
automatically but set manually.

\begin{dispListing}
% \usepackage{lipsum,multicol}  % preamble
\small
\begin{multicols}{3}
  \begin{tcolorbox}[enhanced jigsaw,breakable,size=small,
    colback=red!5!white,colframe=red!75!black,fonttitle=\bfseries,
    title=My breakable box,pad at break=2mm,drop fuzzy shadow,
    height fixed for=all, break at=11.4cm ]
  \lipsum[1-3]
  \end{tcolorbox}
\end{multicols}
\end{dispListing}
{\tcbusetemp}

\clearpage
\subsection{Break Point Insertion}\label{subsec:breakpoints}

\begin{docCommand}[doc new=2017-07-05]{tcbbreak}{}
  A \emph{breakable} box is not broken, if there is enough
  space on the current page or column.
  Therefore, typical penalty insertion with
  |\break|, |\pagebreak|, |\columnbreak|, \ldots \emph{may} only work as
  expected, if the box is broken at least into two parts
  \emph{without} inserting the penalties.\par\smallskip
  To \emph{force} a page or column break, \refComLe{tcbbreak}
  starts a new paragraph and inserts an insane tall rule which causes a
  break and which is immediately discarded. You may ignore this technical
  information and just use it as you would use |\pagebreak|.\par\smallskip
  For an \emph{unbreakable box}, \refComLe{tcbbreak} is identical to insert |\par|,
  i.e.\ it just starts a new paragraph.\par\smallskip
  Also see \refKeyLe{/tcb/break at} for defining height dependend breaks.

\begin{dispListing}
% \usepackage{lipsum,multicol}  % preamble
\begin{multicols}{3}
  \begin{tcolorbox}[breakable,enhanced jigsaw,size=small,
    colback=red!5!white,colframe=red!75!black,fonttitle=\bfseries,
    title=Break into parts
  ]
  First part\tcbbreak
  Second part\tcbbreak
  Third part
  \end{tcolorbox}
\end{multicols}

\begin{multicols}{3}
  \begin{tcolorbox}[enhanced jigsaw,size=small,
    colback=red!5!white,colframe=red!75!black,fonttitle=\bfseries,
    title=You shall not break
  ]
  First part\tcbbreak
  Second part\tcbbreak
  Third part
  \end{tcolorbox}
\end{multicols}

\end{dispListing}
{\tcbusetemp}

\end{docCommand}



\clearpage
\subsection{Break Sequence for the Skins}\label{subsec:breaksequence}
The following diagrams document the \emph{break sequence} for different
skins. Depending on the main skin of a |tcolorbox|, the actual skins of
the \emph{break sequence} parts are displayed.

\def\tcbbreakskininto#1#2#3#4#5{%
\begin{center}\begin{tikzpicture}
\tcbset{width=7cm,colframe=Navy,colback=AliceBlue,fonttitle=\bfseries,
  watermark color=AliceBlue!85!Navy,#5
  }
\node[above] (unbroken) at (0,0) {\begin{tcolorbox}[title=Unbroken Box,skin=#1,watermark text=unbroken,height=3.8cm]
\texttt{skin=#1}
\end{tcolorbox}};
\node[above] (first) at (8.7,2.4) {\begin{tcolorbox}[title=Broken Boxes,skin=#2,watermark text=first,height=1.4cm]
\texttt{skin=#2}
\end{tcolorbox}};
\node[above] (middle) at (8.7,1.2) {\begin{tcolorbox}[skin=#3,watermark text=middle,height=1cm]
\texttt{skin=#3}
\end{tcolorbox}};
\node[above] (last) at (8.7,0) {\begin{tcolorbox}[skin=#4,watermark text=last,height=1cm]
\texttt{skin=#4}
\end{tcolorbox}};
\path[draw=FireBrick,line width=2pt,->] (unbroken) edge (first.west) edge (middle.west) edge (last.west);
\end{tikzpicture}\end{center}}

\tcbbreakskininto{standard}{standard}{standard}{standard}{watermark text/.style={}}
\tcbbreakskininto{standard jigsaw}{standard jigsaw}{standard jigsaw}{standard jigsaw}{watermark text/.style={}}
\tcbbreakskininto{spartan}{spartan}{spartan}{spartan}{}
\clearpage
\tcbbreakskininto{enhanced}{enhancedfirst}{enhancedmiddle}{enhancedlast}{}
\tcbbreakskininto{enhancedfirst}{enhancedfirst}{enhancedmiddle}{enhancedmiddle}{}
\tcbbreakskininto{enhancedmiddle}{enhancedmiddle}{enhancedmiddle}{enhancedmiddle}{}
\tcbbreakskininto{enhancedlast}{enhancedmiddle}{enhancedmiddle}{enhancedlast}{}
\clearpage
\tcbbreakskininto{enhanced jigsaw}{enhancedfirst jigsaw}{enhancedmiddle jigsaw}{enhancedlast jigsaw}{}
\tcbbreakskininto{enhancedfirst jigsaw}{enhancedfirst jigsaw}{enhancedmiddle jigsaw}{enhancedmiddle jigsaw}{}
\tcbbreakskininto{enhancedmiddle jigsaw}{enhancedmiddle jigsaw}{enhancedmiddle jigsaw}{enhancedmiddle jigsaw}{}
\tcbbreakskininto{enhancedlast jigsaw}{enhancedmiddle jigsaw}{enhancedmiddle jigsaw}{enhancedlast jigsaw}{}
\clearpage
{\tcbset{borderline={2pt}{0pt}{black!10!white}}%
\tcbbreakskininto{empty}{emptyfirst}{emptymiddle}{emptylast}{}
\tcbbreakskininto{emptyfirst}{emptyfirst}{emptymiddle}{emptymiddle}{}
\tcbbreakskininto{emptymiddle}{emptymiddle}{emptymiddle}{emptymiddle}{}
\tcbbreakskininto{emptylast}{emptymiddle}{emptymiddle}{emptylast}{}
}
\clearpage
\tcbbreakskininto{bicolor}{bicolorfirst}{bicolormiddle}{bicolorlast}{bicolor}
\tcbbreakskininto{bicolorfirst}{bicolorfirst}{bicolormiddle}{bicolormiddle}{bicolor}
\tcbbreakskininto{bicolormiddle}{bicolormiddle}{bicolormiddle}{bicolormiddle}{bicolor}
\tcbbreakskininto{bicolorlast}{bicolormiddle}{bicolormiddle}{bicolorlast}{bicolor}
\clearpage
\tcbbreakskininto{bicolor jigsaw}{bicolorfirst jigsaw}{bicolormiddle jigsaw}{bicolorlast jigsaw}{bicolor jigsaw}
\tcbbreakskininto{bicolorfirst jigsaw}{bicolorfirst jigsaw}{bicolormiddle jigsaw}{bicolormiddle jigsaw}{bicolor jigsaw}
\tcbbreakskininto{bicolormiddle jigsaw}{bicolormiddle jigsaw}{bicolormiddle jigsaw}{bicolormiddle jigsaw}{bicolor jigsaw}
\tcbbreakskininto{bicolorlast jigsaw}{bicolormiddle jigsaw}{bicolormiddle jigsaw}{bicolorlast jigsaw}{bicolor jigsaw}
\clearpage
\tcbbreakskininto{tile}{tilefirst}{tilemiddle}{tilelast}{tile,colbacktitle=Navy}
\tcbbreakskininto{tilefirst}{tilefirst}{tilemiddle}{tilemiddle}{tile,colbacktitle=Navy}
\tcbbreakskininto{tilemiddle}{tilemiddle}{tilemiddle}{tilemiddle}{tile,colbacktitle=Navy}
\tcbbreakskininto{tilelast}{tilemiddle}{tilemiddle}{tilelast}{tile,colbacktitle=Navy}
\clearpage
\tcbbreakskininto{beamer}{beamerfirst}{beamermiddle}{beamerlast}{beamer}
\tcbbreakskininto{beamerfirst}{beamerfirst}{beamermiddle}{beamermiddle}{beamer}
\tcbbreakskininto{beamermiddle}{beamermiddle}{beamermiddle}{beamermiddle}{beamer}
\tcbbreakskininto{beamerlast}{beamermiddle}{beamermiddle}{beamerlast}{beamer}
\clearpage
\tcbbreakskininto{widget}{widgetfirst}{widgetmiddle}{widgetlast}{widget}
\tcbbreakskininto{widgetfirst}{widgetfirst}{widgetmiddle}{widgetmiddle}{widget}
\tcbbreakskininto{widgetmiddle}{widgetmiddle}{widgetmiddle}{widgetmiddle}{widget}
\tcbbreakskininto{widgetlast}{widgetmiddle}{widgetmiddle}{widgetlast}{widget}
\tcbbreakskininto{draft}{draft}{draft}{draft}{draft}
\clearpage
\tcbbreakskininto{freelance}{freelancefirst}{freelancemiddle}{freelancelast}{}
\tcbbreakskininto{freelancefirst}{freelancefirst}{freelancemiddle}{freelancemiddle}{}
\tcbbreakskininto{freelancemiddle}{freelancemiddle}{freelancemiddle}{freelancemiddle}{}
\tcbbreakskininto{freelancelast}{freelancemiddle}{freelancemiddle}{freelancelast}{}




\clearpage
\subsection{Break by Hand (Faked Break)}

\begin{marker}
See \Vref{subsec:multicol} for \emph{real} column breaks.
\end{marker}

Since the appearance of broken boxes is done by skins, it is quite easy
to 'fake a break'. For this, you actually don't need the
\mylib{breakable} library at
all.

\begin{dispExample}
\tcbset{enhanced,equal height group=fakedbreak,
  colback=LightGreen,colframe=DarkGreen,
  width=(\linewidth-6mm)/3,nobeforeafter,
  left=1mm,right=1mm,top=1mm,bottom=1mm,middle=1mm}
%
\begin{tcolorbox}[title=My broken box,skin=enhancedfirst]
This is a box which breaks from one column to another
\end{tcolorbox}\hfill
\begin{tcolorbox}[skin=enhancedmiddle]
column. I am sorry to say that this is a trick.
Nevertheless, you may use this trick for your
\end{tcolorbox}\hfill
\begin{tcolorbox}[skin=enhancedlast]
own purposes.
\end{tcolorbox}
\end{dispExample}


