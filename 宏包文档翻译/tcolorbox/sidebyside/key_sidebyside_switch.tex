


% \clearpage
\begin{docTcbKey}[][doc new=2015-11-20]{sidebyside switch}{\colOpt{=true\textbar false}}{default |true|, initially |false|}

If set to |true|, the
\meta{left-handed content} and \meta{right-handed content}
of \refComLe{tcbsidebyside} are switched.
Obviously, this option is only valid inside \refComLe{tcbsidebyside}.

如果设置为|true|,%
\refComLe{tcbsidebyside}的 \meta{left-handed content} 和 \meta{right-handed content} 会被切换。%
显然,这个选项只在\refComLe{tcbsidebyside}内有效。



The side switching can be made even/odd page sensitive, 
if used inside \refKeyLe{/tcb/if odd page}.

如果指定了了 \refKeyLe{/tcb/if odd page},两侧切换对奇/偶页敏感。



\begin{dispExample}
% \tcbuselibrary{skins,xparse}
\tcbsidebyside{Left}{Right}

\tcbsidebyside[sidebyside switch]{Left}{Right}

\tcbsidebyside[title=Very important table,
  if odd page={sidebyside switch,sidebyside adapt=right,flushright title}%
              {sidebyside adapt=left},
  beamer,colframe=blue!50!black,colback=blue!10,
  lower separated=false,sidebyside gap=5mm
]{%
  \begin{tabular}{|l|c|r|}\hline
    left & center & right\\\hline
    A & B & C\\\hline
    D & E & F\\\hline
  \end{tabular}
}{%
  This table contains the most important figures for
  all future actions. You may notice that B follows A,
  C follows B, and so on.
}
\end{dispExample}


\end{docTcbKey}