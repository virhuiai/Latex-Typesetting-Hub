\begin{docTcbKey}{sidebyside}{=\colOpt{true\textbar false}}{默认值|true|,初始设置为|false|}
% \begin{docTcbKey}{sidebyside}{\colOpt{=true\textbar false}}{default |true|, initially |false|}
% \begin{stripedbox}
Normally, the upper part and the lower part of the box have their positions
as their names suggest. If |sidebyside| is set to |true|, the upper part
is drawn \emph{left-handed} and the lower part is drawn \emph{right-handed}.
Both parts are drawn together with the geometry settings of the upper part but the
space is divided horizontally according to the following options.
Colors, fonts, and box content additions are used individually.
The resulting box is unbreakable.

% Normally, the upper part and the lower part of the box have their positions as their names suggest. 
% 通常,盒子的upper部分和lower部分的位置,同它们的名字一致。%
% If |sidebyside| is set to |true|, the upper part
% is drawn \emph{left-handed} and the lower part is drawn \emph{right-handed}.
% 如果 |sidebyside| 设置为 true, 则upper部分会放置在左边,lower部分放置在右边。%

% Both parts are drawn together with the geometry settings of the upper part but the
% space is divided horizontally according to the following options.
% 两个部分都使用上部分的几何设置进行绘制,但空间按照以下选项水平分割。
% Colors, fonts, and box content additions are used individually.
% The resulting box is unbreakable.
% 颜色、字体和盒子内容的添加都是独立使用的。生成的盒子是\csh渐变盒子V[red]{不可分割}的。


通常,盒子的upper部分和lower部分的位置,同它们的名字一致。
如果 sidebyside 设置为 true, 则upper部分会放置在左边,lower部分放置在右边。
两个部分都使用上部分的几何设置进行绘制,但空间按照以下选项水平分割。颜色、字体和盒子内容的添加都是独立使用的。生成的盒子是\csh渐变盒子V[red]{不可分割}的。


\begin{dispExample}
\tcbset{colback=red!5!white%
,colframe=red!75!black%
,fonttitle=\bfseries}

\begin{tcolorbox}[title=我的标题,sidebyside]
这是 upper (\textit{左侧}) 部分。
\tcblower
这是 lower (\textit{右侧}) 部分。
\end{tcolorbox}
\end{dispExample}


%todo bicolor 是啥
\begin{dispExample}
% \usepackage{lipsum}
% \tcbuselibrary{skins}
\begin{tcolorbox}[bicolor% 表示创建一个双色盒子,即上下两部分的颜色可以不同。
,sidebyside%并排
,righthand width=3cm%指定右侧的宽度
,sharp corners,boxrule=.4pt,colback=green!5,colbacklower=green!50!black!50]
\lipsum[2]
\tcblower
\includegraphics[width=\linewidth]{goldshade}%
\end{tcolorbox}
\end{dispExample}
\end{docTcbKey}