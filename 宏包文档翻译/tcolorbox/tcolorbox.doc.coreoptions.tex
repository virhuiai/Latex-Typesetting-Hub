% !TeX root = tcolorbox.tex
% include file of tcolorbox.tex (manual of the LaTeX package tcolorbox)
% \clearpage
\section{Option Keys}\label{sec:optkeys}%
\tcbset{external/prefix=external/coreoptions_}%
For the \meta{options} in \refEnv{tcolorbox} respectively \refCom{tcbset}
the following |pgf| keys can be applied. The key tree path |/tcb/| is not to
be used inside these macros. It is easy to add your own style keys using
the syntax for |pgf| keys, see \cite{tantau:tikz_and_pgf,sturm:latex} or the examples
starting from page~\pageref{sec:latextutorial}.

% For the \meta{options} in  respectively 
以下的选项可以在\refEnv{tcolorbox}和\refCom{tcbset}中使用。当使用这些命令时,选项路径|/tcb/|是不需要要。
使用 |pgf| 选项的语法可以很容易添加您自己的样式, 例子见~\pageref{sec:latextutorial}页的 \cite{tantau:tikz_and_pgf,sturm:latex} .


% \subsection{Title}% 是相对于使用input语句所在文件 todo 
% \subsection{Title}% 是相对于使用input语句所在文件 todo 

  % \begin{docTcbKey}{title}{=\meta{text}}{no default, initially empty}
\begin{docTcbKey}{title}{=\meta{文本}}{无默认值,初始为空}
Creates a heading line with \meta{text} as content.

创建以 \meta{text} 作为内容的标题行。

\begin{exdispExample*}{title}{sbs,lefthand ratio=0.6}
\begin{tcolorbox}[title=我的标题行]
这是一个\textbf{tcolorbox}。
\end{tcolorbox}
\end{exdispExample*}
\end{docTcbKey}

\begin{docTcbKey}{notitle}{}{no value, initially set}
Removes the title line if set before.

移除之前设置的标题行。
\end{docTcbKey}

\begin{docTcbKey}{adjusted title}{=\meta{text}}{style, no default, initially unset, 标题高度等高}
Creates a heading line with \meta{text} as content. The minimal height of
this line is adjusted to fit the text given by \refKey{/tcb/adjust text}.
This option makes sense
for single line headings if boxes are set side by side with equal height.
Note that it is very easy to trick this adjustment.


% 创建一个以 \meta{text} 为内容的标题行。 此行的最小高度根据 \meta{text} 自适应。
% 此选项适用于希望让单行放置多个并排的盒子拥有相同的标题高度。
% 注意,to trick 此调整是非常容易的。

创建一个标题行,其中内容为\meta{text}。此行的最小高度会根据\refKey{/tcb/adjust text}中给定的文本进行调整。如果希望让单行放置多个并排的盒子拥有相同的标题高度,则此选项很有意义。请注意,很容易欺骗此调整。
% 诀窍
% 技巧
% 骗局
% 魔术
% 神奇手法
% 欺骗
% 诡计
% 奥妙
% 狡猾手段
% 玩意儿

% exdispExample 
% 这是一个tcolorbox宏包提供的环境,用于生成一个例子展示。runs=2表示该例子最多运行2次,{adjusted_title}是该例子的标题。
\begin{exdispExample}[runs=2]{adjusted_title}
\tcbset{colback=White,arc=0mm,width=(\linewidth-4pt)/4,
equal height group=AT,before=,after=\hfill,fonttitle=\bfseries}

以下标题是not adjusted:\\
\foreach \n in {xxx,ggg,AAA,\"Agypten}
{\begin{tcolorbox}[title=\n,colframe=red!75!black]
一些文本。\end{tcolorbox}}
现在,我们再次尝试adjusted titles:\\
\foreach \n in {xxx,ggg,AAA,\"Agypten}
{\begin{tcolorbox}[adjusted title=\n,colframe=blue!75!black]
一些文本。\end{tcolorbox}}
\end{exdispExample}
\end{docTcbKey}

% todo 
% \tcbset{colback=White,arc=0mm,width=(\linewidth-4pt)/4,equal height group=AT,before=,after=\hfill,fonttitle=\bfseries}:
%这是tcolorbox宏包提供的一个设置环境,可以设置后面生成的tcolorbox的一些属性。
%其中,colback表示背景颜色,arc表示圆角半径,width表示盒子的宽度,
%equal height group表示将多个盒子的高度设置成相同,before和after表示在盒子前后分别添加一些内容,fonttitle表示标题的字体。

% \foreach \n in {xxx,ggg,AAA,\"Agypten}:
% 这是一个foreach循环,用于循环生成tcolorbox盒子。\n表示循环变量,{xxx,ggg,AAA,\"Agypten}是一个包含了4个元素的列表。
% \begin{tcolorbox}[title=\n,colframe=red!75!black]一些文本。\end{tcolorbox}:这是一个tcolorbox环境,用于生成一个盒子。
%title表示盒子的标题,colframe表示边框颜色,一些文本。是盒子中的内容。

% \begin{tcolorbox}[adjusted title=\n,colframe=blue!75!black]一些文本。\end{tcolorbox}:这是一个tcolorbox环境,用于生成一个盒子。
%adjusted title表示盒子的标题,colframe表示边框颜色,一些文本。是盒子中的内容。




\begin{docTcbKey}{adjust text}{=\meta{text}}{no default, initially \texttt{\"Apgjy}}
This sets the reference text for \refKey{/tcb/adjusted title}. If your texts
never exceed \enquote{\"Apgjy} in depth and height you don't need to care about this option.

用于为\refKey{/tcb/adjusted title}设置参考文本。%
如果你的文本不会超过\enquote{\"Apgjy}的深度和高度,你就不需要关心这个选项。

%%%TODO 弄些例子出来看到底啥意思 
% \begin{exdispExample}[runs=2]{adjust_text}
%   \foreach \n in {xxx,ggg,AAA,\"Agypten}
%   {\begin{tcolorbox}[adjust text=\n,title=test,colframe=blue!75!black,width=(\linewidth-4pt)/4]
%   一些文本。\end{tcolorbox}}
% \end{exdispExample}

\end{docTcbKey}

%squeezed 挤压
\begin{docTcbKey}[][doc new=2014-11-24]{squeezed title}{=\meta{text}}{style, no default, initially unset,挤压标题,宽度限最大宽}
Creates a single heading line with \meta{text} as content.
If the \meta{text} is longer than the available space, the text is
squeezed to fit into the available space.

创建一个标题行,内容为 \meta{text}。%
如果 \meta{text} 比可用空间长, 文本将被压缩以适应可用的空间。
\begin{exdispExample}{squeezed_title}
% \tcbuselibrary{raster}
\begin{tcbitemize}[
raster columns=3,%三列
raster equal height,%等高
colframe=red!75!black,colback=red!5!white,%框与背景色
fonttitle=\bfseries%标题加格式
]
\tcbitem[squeezed title={短标题}]
第一个例子
\tcbitem[squeezed title={这是一个非常非常长的标题}]
第二个例子
\tcbitem[squeezed title={这个标题显然对这个例子来说太长了}]
第三个例子
\end{tcbitemize}
\end{exdispExample}
\end{docTcbKey}



% 这是一个使用tcolorbox宏包的例子,展示了如何使用tcbitemize环境创建一个三列等高的项目列表,并对项目的标题进行压缩。
% 具体解释如下:
% 首先加载tcolorbox宏包的raster库,以便使用raster columns和raster equal height选项。

% 然后创建一个tcbitemize环境,并设置raster columns=3和raster equal height选项,以创建一个三列等高的项目列表。

% 接下来,对每个项目使用\tcbitem命令创建一个项目,并使用squeezed title选项压缩项目标题。在本例中,第一个项目的标题为“短标题”,第二个项目的标题为“这是一个非常非常长的标题”,第三个项目的标题为“这个标题显然对这个例子来说太长了”。

% 最后,为了美化效果,设置colframe和colback选项,分别为项目框和背景色添加红色调,并使用fonttitle选项为项目标题添加粗体格式。



%这个 todo 未定义的命令,可能是  tcbitemize 那边才有
% begin{tcolorbox}[squeezed title={BrE əˈdʒʌst, AmE əˈdʒəst},colframe=red!75!black] 
% 一些文本。
% \end{tcolorbox}
% begin{tcolorbox}[squeezed title={BrE skwiːz, AmE skwiz},colframe=red!75!black] 
% 一些文本。
% \end{tcolorbox}
% begin{tcolorbox}[squeezed title={这个标题显然对这个例子来说太长了},colframe=red!75!black] 
% 一些文本。
% \end{tcolorbox}

%试了音标的字体
% \begin{tcbitemize}[
% raster columns=3,%三列
% raster equal height,%等高
% colframe=red!75!black,colback=red!5!white,%框与背景色
% fonttitle=\bfseries%标题加格式
% ]
% \tcbitem[squeezed title={BrE əˈdʒʌst, AmE əˈdʒəst}]
% adjust
% \tcbitem[squeezed title={BrE skwiːz, AmE skwiz}]
% squeeze
% \tcbitem[squeezed title={这个标题显然对这个例子来说太长了}]
% 第三个例子
% \end{tcbitemize}

\begin{docTcbKey}[][doc new=2014-11-24]{squeezed title*}{=\meta{text}}{style, no default, initially unset}
This is a combination of \refKey{/tcb/adjusted title} and  \refKey{/tcb/squeezed title}.

这是  \refKey{/tcb/adjusted title} 和  \refKey{/tcb/squeezed title} 的组合。即高度和宽度都 \dots
\begin{exdispExample}{squeezed_title_2}
% \tcbuselibrary{raster}
\begin{tcbitemize}[raster columns=3,raster equal height,
  colframe=red!75!black,colback=red!5!white,fonttitle=\bfseries]
\tcbitem[squeezed title*={Short title}]
  First box
\tcbitem[squeezed title*={This is a very very long title}]
  Second box
\tcbitem[squeezed title*={This title is clearly to long for this application}]
  Third box
\end{tcbitemize}
\end{exdispExample}
\end{docTcbKey}

\begin{docTcbKey}[][doc new=2019-03-01]{titlebox}{=\meta{mode}}{no default, initially \texttt{visible}}
Controls the treatment of the title part of the box.
Feasible values for \meta{mode} are:

% 控制盒子的标题部分的处理。可设的 \meta{mode} 值有:
控制盒子的标题部分的处理方式。 \meta{mode} 的可行值为:

\begin{DescriptionR}{\docValue{invisible}}
\item[\docValue{visible}]usual type setting of the title box,\\
对带标题盒子的常用的设置,
\item[\docValue{invisible}]empty space instead of the title contents.\\
使用空白代替标题内容。
\end{DescriptionR}

% \begin{DescriptionL}{\docValue{invisible}}
% \item[\docValue{visible}]usual type setting of the title box,\\
% 对带标题盒子的常用的设置,
% \item[\docValue{invisible}]empty space instead of the title contents.\\
% 使用空白代替标题内容。
% \end{DescriptionL}

% \begin{DescriptionLsqueezed}{\docValue{visible}}
% \item[\docValue{visible}]usual type setting of the title box,\\
% 对带标题盒子的常用的设置,
% \item[\docValue{invisible}]empty space instead of the title contents.\\
% 使用空白代替标题内容。
% \end{DescriptionLsqueezed}

% \begin{DescriptionR}{\docValue{invisible}}
% \item[\docValue{visible}]usual type setting of the title box,\\
% 对带标题盒子的常用的设置,
% \item[\docValue{invisible}]empty space instead of the title contents.\\
% 使用空白代替标题内容。
% \end{DescriptionR}

% \begin{description}
% \item[\docValue{visible}]usual type setting of the title box,\\
% 对带标题盒子的常用的设置,
% \item[\docValue{invisible}]empty space instead of the title contents.\\
% 使用空白代替标题内容。
% \end{description}
\begin{exdispExample}{titlebox}
\begin{tcolorbox}[title=我的不可见标题,
  titlebox=invisible]
这是一个\textbf{tcolorbox}.
\end{tcolorbox}
\end{exdispExample}

\begin{exdispExample}{visible_titlebox}
  \begin{tcolorbox}[title=我的可见标题,
    titlebox=visible]
这是一个\textbf{tcolorbox}.
  \end{tcolorbox}
  \end{exdispExample}
\end{docTcbKey}



% \clearpage
\begin{docTcbKey}{detach title}{}{no value}
Detaches the title from its normal position. The text of the title is
stored into \docAuxCommand{tcbtitletext} and the formatted title is
available by \docAuxCommand{tcbtitle}.
The main application is to move the title from its usual place to another one.
  
将标题从它的正常位置移开。标题的文本存储到 \docAuxCommand{tcbtitletext} 中,格式化的标题可以通过 \docAuxCommand{tcbtitle} 获得。主要的应用是将标题从它的惯例位置移动到另一个位置。

% 将标题从其正常位置移开。标题文本存储在\docAuxCommand{tcbtitletext}中,格式化的标题可通过\docAuxCommand{tcbtitle}获得。 主要应用是将标题从其通常位置移动到另一个位置。
 
\begin{exdispExample}{detach_title}
\newtcolorbox{mybox}[2][]{
colbacktitle=red!10!white,
colback=blue!10!white,
coltitle=red!70!black,
title={#2},fonttitle=\bfseries,#1}

\begin{mybox}{My title}
这是一个\textbf{tcolorbox}.
\end{mybox}
\begin{mybox}[detach title,%暂存标题内容到\tcbtitle
before upper={\tcbtitle\quad}%放到upper的前面
]{detach title}
这是一个\textbf{tcolorbox}。\footnotemark
\end{mybox}
\begin{mybox}[detach title,
after upper={\par\hfill\tcbtitle}%放到upper的后面
]{名人名字}
可以用于名人名言的内容。
\end{mybox}
\end{exdispExample}
\end{docTcbKey}
\footnotetext{译者idea:可以改造了当单条的description。}


  

\begin{docTcbKey}{attach title}{}{no value}
  Attaches the title to its normal position. This option is used to reverse
  \refKey{/tcb/detach title}.

  将标题位置重置到其正常位置。此选项用于反转 \refKey{/tcb/detach title}。
  \end{docTcbKey}
  
  
  \begin{docTcbKey}[][doc updated=2015-07-08]{attach title to upper}{\colOpt{=\meta{text}}}{style, default empty, initially unset}
Attaches the title to the begin of the upper part of the box content.
The optional \meta{text} is set between the formatted title and the box content.

将标题附加到框内容upper部分的开头。
可选的 \meta{text} 插入到标题和upper部分内容之间。
  \begin{exdispExample}{attach_title_to_upper}
  \newtcolorbox{mybox}[2][]{colbacktitle=red!10!white,
    colback=blue!10!white,coltitle=red!70!black,
    title={#2},fonttitle=\bfseries,#1}
   
  \begin{mybox}[attach title to upper={\ ---\ }]{My title}
    attach title to upper加值的情况,不仅将标题放到upper之前,还在标题和upper之间放入破折线。
  \end{mybox}
  \begin{mybox}[attach title to upper,after title={:\ }]{My title}
    attach title to upper不加值时,将标题位置放到upper部分之前,再使用after title在标题后加内容。
  \end{mybox}
  \end{exdispExample}
  \end{docTcbKey}



\bigskip
\begin{marker}
More title options are documented in \Vref{subsec:contentadditions}
and \Vref{subsec:skinboxedtitle}.

更多的标题相关配置文档内容见 \Vref{subsec:contentadditions}
和 \Vref{subsec:skinboxedtitle}.
\end{marker}


% % \clearpage
% \subsection{Subtitle\\副标题}
% Inside the box content, one or more subtitles can be added.
In general, a subtitle is a further \refEnv{tcolorbox} which inherits some color and geometry options from the enclosing box. 
It may be customized just like any other \refEnv{tcolorbox}.

% 在盒子内,可以添加一个或多个副标题。%
% 一般来说,副标题是一个从封闭盒子继承了一些颜色和几何选项的|tcolorbox|。%
% 它可以像是一个 |tcolorbox| 一样进行设置。  

在盒子内容内部,可以添加一个或多个副标题。 通常,副标题是一个进一步的\refEnv{tcolorbox},它从封闭盒子中继承了一些颜色和几何选项。 它可以像任何其他的\refEnv{tcolorbox}一样进行自定义。

\begin{docCommand}[doc new=2014-10-10]{tcbsubtitle}{\oarg{options}\marg{text}}
Used inside a \refEnv{tcolorbox} to add a subtitle box with the given \meta{text}.
which is formatted by several inherited properties of the enclosing box
by further settings from \refKey{/tcb/subtitle style}, and by the given \meta{options}.

在 \refEnv{tcolorbox} 中使用,用于添加一个带有给定 \meta{text} 的副标题框,该框使用封闭框的多个继承属性进行格式设置,并通过 \refKey{/tcb/subtitle style} 的进一步设置和给定的 \meta{options} 进行格式化。

% 在\refEnv{tcolorbox}内部,将\meta{text}添加为子标题盒子。
% 这是一个独立的\refEnv{tcolorbox},%
% 一些格式从封闭的盒子中继承过来的属性,%
% 也可以通过给\refKey{/tcb/subtitle style}指定的\meta{options}来设定。
\begin{exdispExample*}{tcbsubtitle_1}{%
sbs%是sidebyside的意思
,lefthand ratio=0.6%upper侧占的比例
}
\begin{tcolorbox}[title=我的标题,
colback=red!5!white,
colframe=red!75!black,
fonttitle=\bfseries]
  This is a \textbf{tcolorbox}.
\tcbsubtitle[before skip=\baselineskip]%
  {我的{\tt 副}标题}
进一步的文本。
\end{tcolorbox}
\end{exdispExample*}

% 这段代码使用了tcolorbox宏包,创建了一个带有标题和副标题的盒子。具体解释如下:
% \begin{exdispExample*}{tcbsubtitle_1}{% sbs%是sidebyside的意思 ,lefthand ratio=0.6%upper侧占的比例 }

% 这部分代码是使用exdispExample环境来显示代码和输出结果的,其中使用了sbs选项表示将代码和输出结果并排显示,使用lefthand ratio选项表示代码窗口占整个窗口的比例为0.6。

% \begin{tcolorbox}[title=我的标题, colback=red!5!white, colframe=red!75!black, fonttitle=\bfseries]

% 这部分代码创建了一个tcolorbox盒子,其中包含了一个标题“我的标题”。colback选项表示背景色为红色和白色混合,colframe选项表示边框颜色为红色和黑色混合,fonttitle选项表示标题的字体为粗体。

% This is a \textbf{tcolorbox}.

% 这部分代码在盒子中插入了一段文本“This is a tcolorbox”,其中\textbf命令让“tcolorbox”加粗。

% \tcbsubtitle[before skip=\baselineskip]% {我的{\tt 副}标题}

% 这部分代码创建了一个副标题“我的副标题”,使用了\tcbsubtitle命令。before skip选项表示副标题前面的垂直距离为一个基准行距(\baselineskip),{\tt 副}使用了typewriter字体。

% 进一步的文本。

% 这部分代码在盒子中插入了进一步的文本。

% \end{tcolorbox}

% 这部分代码表示tcolorbox盒子的结束。

\begin{exdispExample*}{tcbsubtitle_2}{sbs,lefthand ratio=0.6}
\begin{tcolorbox}[title=My title,
    colback=red!5!white,
    colframe=red!75!black,
    colbacktitle=yellow!50!red,
    coltitle=red!25!black,
    fonttitle=\bfseries]
  This is a \textbf{tcolorbox}.
\tcbsubtitle[before skip=\baselineskip]%
{我的{\tt 副}标题}
进一步的文本。
\end{tcolorbox}
\end{exdispExample*}
\end{docCommand}

\begin{docTcbKey}[][doc new=2014-10-10]{subtitle style}{=\meta{options}}{no default, initially empty}
Adds |tcolorbox| \meta{options} to the settings for \refCom{tcbsubtitle}.

向 |tcbsubtitle| 设置的副标题的 |tcolorbox| 的 \meta{options} 中设置选项。

\begin{exdispExample*}{subtitle_style}{sbs,lefthand ratio=0.6}
\begin{tcolorbox}[title=My title,
  colback=red!5!white,
  colframe=red!75!black,
  colbacktitle=yellow!50!red,
  coltitle=red!25!black,
  fonttitle=\bfseries,
  subtitle style={boxrule=0.4pt,
    colback=yellow!50!red!25!white} ]
  This is a \textbf{tcolorbox}.
\tcbsubtitle{我的子标题}
  Further text.
\tcbsubtitle[colback=green!50!red!25!white]%
{第二个子标题}
上面的子标题中的背景色覆盖了 |subtitle style| 的设置。
\end{tcolorbox}
\end{exdispExample*}
\end{docTcbKey}

% % \clearpage
% \subsection{Upper Part\\upper部分}
% \setcounter{section}{4}
\setcounter{subsection}{2}
\setcounter{subsubsection}{0}
\subsection{Upper Part\\upper部分} 

The text content of a \refEnvLe{tcolorbox} may be parted into a mandatory \emph{upper part}
and an optional \emph{lower part}. These parts are separated by
\refComLe{tcblower}. If there is no \refComLe{tcblower} present, there is no
\emph{lower part} and the \emph{upper part} forms the complete text content.


\refEnvLe{tcolorbox} 的文本内容包括一个必需的 \cshTBlineBbox[blue]{upper} 部分和一个可选的
\cshTBlineBbox[gray]{lower} 部分。它们由\refComLe{tcblower}分隔。如果没有\refComLe{tcblower},则没有\emph{lower}部分,而\emph{upper}部分构成完整的文本内容。

\begin{tcblisting}{%title={Snapshot of the staging area},
csh texdef result={
\mintinline[breaklines]{shell}|latexdef --tempdir /Volumes/RamDisk -p '[all]tcolorbox' --texoptions '-shell-escape' --before '\begin{tcolorbox}' --after '\end{tcolorbox}'  -s tcblower|
}
}
\tcblower:
macro:->\tcb@insert@after@part \end {tcb@savebox}\tcb@set@color {tcbcollower}\unless \iftcb@sidebyside \tcbdimto \tcb@w@lower {\tcb@innerwidth -\kvtcb@boxsep *2-\kvtcb@leftlower -\kvtcb@rightlower }\fi \tcb@hasLowertrue \let \tcb@insert@after@part =\tcb@insert@after@lower \ifx \kvtcb@savelowerto \@empty \let \tcb@startbox \tcb@savelowerbox \let \endtcolorbox \tcb@endboxanddraw \else \let \tcb@startbox \tcb@lowerverbatim \expandafter \let \csname end\kvtcb@savedelimiter \expandafter \endcsname \csname tcb@endlowerverbatimanddraw\endcsname \fi \tcb@startbox 
\end{tcblisting}

\begin{docTcbKey}[][doc new=2015-01-06]{upperbox}{=\meta{mode}}{no default, initially \texttt{visible}}
Controls the treatment of the upper part of the box. If there is no lower part,  this is the complete text content.
Feasible values for \meta{mode} are:

控制例子的upper部分的显示处理。如果没有lower部分, upper部分内容即是所有内容。
可设置的 \meta{mode} 值有:
\begin{DescriptionL}{\docValue{invisible}}
\item[\docValue{visible}]usual type setting of the upper part,
\par 可见,upper部分的常用设定
\item[\docValue{invisible}] empty space instead of the upper part contents.
\par 不可见,upper部分内容显示为空白。
\end{DescriptionL}
% \begin{exdispExample}{upperbox}
% \begin{tcolorbox}[upperbox=invisible,colback=white%
% ,title=upperbox设置为invisible且没有lower部分]
% 这是一个\textbf{tcolorbox}(但是是隐形的)。
% \end{tcolorbox}

% \begin{tcolorbox}[upperbox=invisible,colback=white%
% ,title=upperbox设置为invisible时只显示lower部分]
% 这是一个\textbf{tcolorbox}(但是是隐形的)。
% \tcblower
% 这是lower部分。
% \end{tcolorbox}
% \end{exdispExample}

\begin{dispExample*}{sbs}
\begin{tcolorbox}[upperbox=invisible%
,colback=white%
,title=upperbox设置为invisible%
且没有lower部分]
这是一个\textbf{tcolorbox}(但是是隐形的)。
\end{tcolorbox}
\end{dispExample*}

\begin{dispExample*}{sbs}
\begin{tcolorbox}[upperbox=invisible%
,colback=white%
,title=upperbox设置为invisible%
时只显示lower部分]
这是一个\textbf{tcolorbox}(但是是隐形的)。
\tcblower
这是lower部分。
\end{tcolorbox}
\end{dispExample*}
 
\end{docTcbKey}


\begin{docTcbKey}[][doc new and updated={2015-01-06}{2019-03-01}]{visible}{}{style, no value}
Shortcut for setting \refKeyLe{/tcb/upperbox}, \refKeyLe{/tcb/lowerbox}, and \refKeyLe{/tcb/titlebox}
to be \docValue{visible}.

同时设置 \refKeyLe{/tcb/upperbox}, \refKeyLe{/tcb/lowerbox}, 和 \refKeyLe{/tcb/titlebox} 为 \docValue{visible} 的快捷方式。
\end{docTcbKey}

\begin{docTcbKey}[][doc new and updated={2015-01-06}{2019-03-01}]{invisible}{}{style, no value}
Shortcut for setting \refKeyLe{/tcb/upperbox}, \refKeyLe{/tcb/lowerbox}, and \refKeyLe{/tcb/titlebox}
to be \docValue{invisible}.

设置 \refKeyLe{/tcb/upperbox}, \refKeyLe{/tcb/lowerbox}, 和 \refKeyLe{/tcb/titlebox} 为 \docValue{invisible} 的快捷方式。
\begin{exdispExample}{invisible}
\begin{tcolorbox}[invisible]
这是一个\textbf{tcolorbox}(但是是隐形的)。
\tcblower
这是\textbf{lower部分}(但是是隐形的)。
\end{tcolorbox}
\end{exdispExample}
\end{docTcbKey}




% \clearpage
\begin{docTcbKey}[][doc new=2015-05-04]{saveto}{=\meta{file name}}{no default, initially empty}
Saves the content of the box into a file for an optional later usage.
This is the counterpart of \refKeyLe{/tcb/savelowerto}, but is saves not
only the upper part but the whole content. If a lower part is present,
it is also saved including \refComLe{tcblower}.

% 将盒子的内容保存到一个文件中,以供以后使用。
% 这和 \refKeyLe{/tcb/savelowerto} 类似, 但它不仅保存了upper部分,是保存了整个内容。
% 如果存在lower部分,也会保存且包含 \refComLe{tcblower}。

将盒子的内容保存到文件中,以备将来需要时使用。这是\refKeyLe{/tcb/savelowerto}的对应项,但不仅保存upper部分,而是保存整个内容。如果存在lower部分,则也会保存,包括\refComLe{tcblower}。

\begin{marker}
This option cannot be combined with \refKeyLe{/tcb/savelowerto}.

此项不能同 \refKeyLe{/tcb/savelowerto} 组合使用.
\end{marker}

\begin{exdispExample}{saveto_1}
\begin{tcolorbox}[invisible%upper、lower都不显示
,saveto=\jobname_mysave1.tex
,colback=white
,before upper={before upper}]
这是一个\textbf{tcolorbox},使用invisible后,看着是空的。
它的内容被saveto暂存到指定的文件中,后续再在其他位置包含进来使用。
\end{tcolorbox}

现在,我们包含进来保存好的文本内容:\\
\input{\jobname_mysave1.tex}


\end{exdispExample}

% before upper

\begin{引述之言}{virhuiai}
在LaTeX中,\verb|\jobname| 代表当前文档的名称(不含扩展名)。
\end{引述之言}

\begin{引述之言}{virhuiai}
如果有\verb|before upper|,但设置了invisible,这部分内容不会写入saveto指定的文件中哦!
\end{引述之言}

% \begin{tcolorbox}[%invisible%upper、lower都不显示
% ,colback=white
% ,before upper={before upper}]
% 这是一个\textbf{tcolorbox},使用invisible后,看着是空的。
% 它的内容被saveto暂存到指定的文件中,后续再在其他位置包含进来使用。
% \end{tcolorbox}

% 在LaTeX中,\jobname代表当前文档的名称(不含扩展名):

% \jobname会展开为去除文件扩展名后的文档名称。

% 例如源文件名为paper.tex,\jobname会展开为paper。

% 如果文件名包含路径,也会去掉路径部分。

% \jobname不包含空格及特殊字符,全部转为小写字母。

% \jobname根据TeX引擎和运行方式有细微差异:

% tex命令默认为文件名。

% latex命令默认为文件名。

% xelatex默认为不含扩展名的文件名。

% \jobname通常用于生成与文档相关的外部文件。

% 也可在导言区重新定义\jobname的值。

% \jobname在标题、摘要、目录等地方也有应用。

% 综上,\jobname可以动态获取文档的名称,对文档编译和内容引用非常有用。它可以确保输出文件名与源文件同步更新。

\begin{exdispExample}{saveto_2}
\begin{tcolorbox}[saveto=\jobname_mysave2.tex]
这是一个\textbf{tcolorbox}.
\tcblower
这是lower部分。
\end{tcolorbox}

现在,我们包含进来保存好的文本内容:
\begin{tcolorbox}[colframe=red,colback=red!10,
coltitle=black,colbacktitle=red!20
,sidebyside%从上下改为左右
,title=在这里我们看到保存的内容包括lower部分]
\input{\jobname_mysave2.tex}
\end{tcolorbox}
\end{exdispExample}
\end{docTcbKey}

% 可以看下这个 mysave2.tex 文件,内容是:

% 这是一个\textbf{tcolorbox}.
% \tcblower
% 这是lower部分。

% % \clearpage
% % Lower Part\hfill 
% \subsection{lower部分}
% \setcounter{section}{4}
\setcounter{subsection}{3}
\setcounter{subsubsection}{0}

% Lower Part\hfill 
\subsection{lower部分}

\begin{docTcbKey}{lowerbox}{=\meta{mode}}{no default, initially \texttt{visible}}
Controls the treatment of the lower part of the box.
Feasible values for \meta{mode} are:

控制lower部分的显示情况。可选的 \meta{mode} 值有:
\begin{DescriptionL}{\docValue{invisible}}
\item[\docValue{visible}]usual type setting of the lower part,
\\可见,lower部分的常用设定

\item[\docValue{invisible}]empty space instead of the lower part contents,
\\不可见,lower部分内容显示为空白。

\item[\docValue{ignored}]the lower part is not used (here).
\\忽略,lower部分在这儿没有用上。
\end{DescriptionL}

The last two values are usually applied in connection with |savelowerto|.

\docValue{invisible}、\docValue{ignored} 通常用于与 |savelowerto| 配合使用。

\begin{exdispExample}{lowerbox}
\begin{tcolorbox}[lowerbox=invisible,colback=white]
This is a \textbf{tcolorbox}.
\tcblower
这是lower部分(但不可见)
\end{tcolorbox}

\begin{tcolorbox}[lowerbox=ignored,colback=white]
This is a \textbf{tcolorbox}.
\tcblower
这是lower部分(但ignored)
\end{tcolorbox}
\end{exdispExample}
\end{docTcbKey}


\begin{docTcbKey}[][doc updated=2014-11-28]{savelowerto}{=\meta{file name}}{no default, initially empty}
Saves the content of the lower part into a file for an optional later usage.

将lower部分的内容保存到一个文件中,以备以后使用。
\begin{exdispExample}{savelowerto}
\begin{tcolorbox}[lowerbox=invisible,savelowerto=\jobname_bspsave.tex,colback=white]
This is a \textbf{tcolorbox}.
\tcblower
这是可能相当复杂的lower部分:
$\displaystyle f(x)=\frac{1+x^2}{1-x^2}$.
\end{tcolorbox}

现在,我们加载保存的文本:\\
\input{\jobname_bspsave.tex}
\end{exdispExample}
\end{docTcbKey}



% \clearpage
\begin{docTcbKey}{lower separated}{\colOpt{=true\textbar false}}{default |true|, initially |true|}
If set to |true|, the lower part is visually separated from the upper part.
It depends on the chosen skin how the visualization of the separation is done.

如果设置为 |true|则lower部分与upper部分可见的分隔开来。
分隔的样式依赖于皮肤的选择。
% \enlargethispage*{1cm}


% \begin{exdispExample}{lower_separated_默认}
% \begin{tcolorbox}[title=Lower separated]
% This is the upper part.
% \tcblower
% This is the lower part.
% \end{tcolorbox}
% \end{exdispExample}

% sidebyside,righthand ratio=0.25

\begin{dispExample*}{sidebyside,righthand ratio=0.4}
\begin{tcolorbox}[title=Lower separated]
This is the upper part.
\tcblower
This is the lower part.
\end{tcolorbox}
\end{dispExample*}


\begin{dispExample*}{sidebyside,righthand ratio=0.4}
\begin{tcolorbox}[title=Lower not separated,%
lower separated=false]
upper部分。

设|lower separated=false|
\tcblower
这是lower部分。两部分的分隔没有线条出现。
\end{tcolorbox}    
\end{dispExample*}

\begin{dispExample*}{sidebyside,righthand ratio=0.4}
\begin{tcolorbox}[sidebyside,title={sidebyside}]
upper
\tcblower
lower
\end{tcolorbox}
\end{dispExample*}

\begin{dispExample*}{sidebyside,righthand ratio=0.4}
\begin{tcolorbox}[sidebyside,title={sidebyside}%
,lower separated=false]
upper
\tcblower
lower
\end{tcolorbox}
\end{dispExample*}

\begin{dispExample*}{sidebyside,righthand ratio=0.4}
\begin{tcolorbox}[beamer,title=Lower separated]
upper
\tcblower
lower
\end{tcolorbox}
\end{dispExample*}

\begin{dispExample*}{sidebyside,righthand ratio=0.4}
\begin{tcolorbox}[beamer,title=Lower not separated%
,lower separated=false]
upper
\tcblower
lower
\end{tcolorbox}
\end{dispExample*}


\end{docTcbKey}




% delimiter
% 定界符
% \clearpage
\begin{docTcbKey}{savedelimiter}{=\meta{name}}{no default, initially \texttt{tcolorbox}}
Used in connection with new environment definitions which extend
|tcolorbox| and use or allow the option |savelowerto|.
To catch the end of the new box environment \meta{name} has to be the name of
this environment. Additionally, the environment definition has to use
|\tcolorbox| instead of
|\begin{tcolorbox}| and |\endtcolorbox| instead of |\end{tcolorbox}|.

% 用于关联由 |newenvironment| 自定义的,拓展自 |tcolorbox| 的新环境中的 |savelowerto| 选项。%
% 要捕获新的盒子环境 \meta{name} 的结尾,必须是这个环境的名字。%
% 此外,环境定义必须使用 |\tcolorbox| 代替 |\begin{tcolorbox}|、用 |\endtcolorbox| 代替 |\end{tcolorbox}|。

用于与扩展了|tcolorbox|并使用或允许选项|savelowerto|的新环境定义相关联。为了捕捉新框环境的结尾,\meta{name}必须是此环境的名称。此外,环境定义必须使用|\tcolorbox|而不是|\begin{tcolorbox}|,并且使用|\endtcolorbox|而不是|\end{tcolorbox}|。

\begin{exdispExample}{savedelimiter1}
\newenvironment{mybox}[1]{%
\tcolorbox[savedelimiter=mybox,
            savelowerto=\jobname_bspsave2.tex,lowerbox=ignored,
            colback=red!5!white,colframe=red!75!black,fonttitle=\bfseries,
            title={#1}]}%
{\endtcolorbox}

\begin{mybox}{暂存 savelowerto 的内容的新环境}
Upper部分。
\tcblower
暂存的lower部分!
\end{mybox}

现在,使用之前暂存的部分:
\begin{tcolorbox}[colback=green!5,title=用到 savelowerto 暂存的内容]
\input{\jobname_bspsave2.tex}
\end{tcolorbox}
\end{exdispExample}

\begin{引述之言}{GPT}
在LaTeX中,tcolorbox是一个功能强大的宏包,用于创建漂亮的盒子和框架。其中,savedelimiter是tcolorbox宏包提供的一个选项,用于保存和恢复当前的分隔符。
\\[0.5em]
分隔符在tcolorbox中用于定义盒子的起始和结束位置。当使用tcolorbox环境时,可以指定不同的分隔符,如\verb|[]|、\verb|{}|等。然而,在某些情况下,可能需要在盒子内部使用其他环境或命令,这些环境或命令也使用相同的分隔符,这可能会导致分隔符冲突的问题。
\\[0.5em]
为了解决这个问题,tcolorbox提供了savedelimiter选项。使用savedelimiter选项,可以在tcolorbox中保存当前的分隔符状态,并在需要时恢复它。这样可以确保在盒子内部使用其他环境或命令时,分隔符不会产生冲突。
\end{引述之言}


% \enlargethispage*{1cm}

The |savedelimiter| is used implicitely with \refComLe{newtcolorbox} which
allows a more convenient usage:

\refComLe{newtcolorbox} 隐式使用了 |savedelimiter|,使用起来更方便:
\begin{exdispExample}{savedelimiter2}
\newtcolorbox{mybox}[1]{%
            savelowerto=\jobname_bspsave2.tex,lowerbox=ignored,
            colback=red!5!white,colframe=red!75!black,fonttitle=\bfseries,
            title={#1}}%

\begin{mybox}{My Example}
Upper part.
\tcblower
Saved lower part!
\end{mybox}

Now, the saved part is used:
\begin{tcolorbox}[colback=green!5]
\input{\jobname_bspsave2.tex}
\end{tcolorbox}
\end{exdispExample}
\end{docTcbKey}
% % \clearpage
% \subsection{Colors and Fonts\\颜色和字体}
% \subsection{Colors and Fonts\\颜色和字体}

\begin{docTcbKey}{colframe}{=\meta{color}}{no default, initially \texttt{black!75!white}}
Sets the frame \meta{color} of the box.

% \hspace{-0.35ex}
设置盒子边框的\meta{color}。
\begin{exdispExample*}{colframe}{sbs,lefthand ratio=0.6}
\begin{tcolorbox}[colframe=red!50!white]
This is a \textbf{tcolorbox}.
\end{tcolorbox}
\end{exdispExample*}
\end{docTcbKey}

\begin{docTcbKey}{colback}{=\meta{color}}{no default, initially \texttt{black!5!white}}
Sets the background \meta{color} of the box.

设置盒子的背景\meta{color}。
\begin{exdispExample*}{colback}{sbs,lefthand ratio=0.6}
\begin{tcolorbox}[colback=red!50!white]
This is a \textbf{tcolorbox}.
\end{tcolorbox}
\end{exdispExample*}
\end{docTcbKey}

另见 \mylib{skins} 库的 \refKeyLe{/tcb/colbacklower}。

\begin{docTcbKey}{title filled}{\colOpt{=true\textbar false}}{default |true|, initially |false|}
Switches the drawing of the title background according to the given value.
This option is set to |true| automatically by \refKeyLe{/tcb/colbacktitle},
\refKeyLe{/tcb/opacitybacktitle}, and \refKeyLe{/tcb/title style},
and \refKeyLe{/tcb/title code}.

根据给定的值切换标题部分背景色的绘制。
\refKeyLe{/tcb/colbacktitle},\refKeyLe{/tcb/opacitybacktitle}, \refKeyLe{/tcb/title style},
和 \refKeyLe{/tcb/title code} 设定时,会自动将此选项设置为 |true|。

\begin{exdispExample*}{title_filled}{sbs,lefthand ratio=0.6}
\begin{tcolorbox}[title=My title,title filled]
This is a \textbf{tcolorbox}.
\end{tcolorbox}
\begin{tcolorbox}[title=My title,
title filled=false]
This is a \textbf{tcolorbox}.
\end{tcolorbox}
\end{exdispExample*}
\end{docTcbKey}


\begin{docTcbKey}{colbacktitle}{=\meta{color}}{no default, initially \texttt{black!50!white}}
Sets the background \meta{color} of the title area of the box.

设置盒子的的标题区域的背景颜色。
\begin{exdispExample*}{colbacktitle}{sbs,lefthand ratio=0.6}
\begin{tcolorbox}[colbacktitle=red!50!white,
title=My title,coltitle=black,
fonttitle=\bfseries]
This is a \textbf{tcolorbox}.
\end{tcolorbox}
\end{exdispExample*}
\end{docTcbKey}



% \clearpage

\begin{docTcbKey}{colupper}{=\meta{color}}{no default, initially \texttt{black}}
Sets the text \meta{color} of the upper part.

设置upper部分的文本的颜色。
\begin{exdispExample*}{colupper}{sbs,lefthand ratio=0.6}
\begin{tcolorbox}[colupper=red!75!black]
This is a \textbf{tcolorbox}.
\tcblower
This is the lower part.
\end{tcolorbox}
\end{exdispExample*}
\end{docTcbKey}


\begin{docTcbKey}{collower}{=\meta{color}}{no default, initially \texttt{black}}
Sets the text \meta{color} of the lower part.

设置lower部分的文本的颜色。
\begin{exdispExample*}{collower}{sbs,lefthand ratio=0.6}
\begin{tcolorbox}[collower=red!75!black]
This is a \textbf{tcolorbox}.
\tcblower
This is the lower part.
\end{tcolorbox}
\end{exdispExample*}
\end{docTcbKey}


\begin{docTcbKey}{coltext}{=\meta{color}}{style, no default, initially \texttt{black}}
Sets the text \meta{color} of the box. This is an abbreviation for setting
|colupper| and |collower| to the same value.

设置盒子中文本的颜色。这是同时将 |colupper| 和 |collower| 的值设置为一个颜色。
\begin{exdispExample*}{coltext}{sbs,lefthand ratio=0.6}
\begin{tcolorbox}[coltext=red!75!black]
This is a \textbf{tcolorbox}.
\tcblower
This is the lower part.
\end{tcolorbox}
\end{exdispExample*}
\end{docTcbKey}


\begin{docTcbKey}{coltitle}{=\meta{color}}{no default, initially \texttt{white}}
Sets the title text \meta{color} of the box.

设置标题的文本的颜色。
\begin{exdispExample*}{coltitle}{sbs,lefthand ratio=0.6}
\begin{tcolorbox}[coltitle=red!75!black,
colbacktitle=black!10!white,title=Test]
This is a \textbf{tcolorbox}.
\end{tcolorbox}
\end{exdispExample*}
\end{docTcbKey}





% \clearpage

\begin{docTcbKey}{fontupper}{=\meta{text}}{no default, initially empty}
Sets \meta{text} before the content of the upper part (e.\,g.\ font settings).

附加 \meta{text} 到{\bf 上}部分的内容之前(例如字体设置)。

% 在upper部分的内容之前设置\meta{text}(例如字体设置)。
\begin{exdispExample}{fontupper}
\begin{tcolorbox}[fontupper=Hello!~\sffamily]
This is a \textbf{tcolorbox}.
\end{tcolorbox}
\end{exdispExample}
\begin{exdispExample}{fontupper2}
\begin{tcolorbox}[fontupper=Hello!~]
This is a \textbf{tcolorbox}.
\end{tcolorbox}
\end{exdispExample}
\end{docTcbKey}


\begin{docTcbKey}{fontlower}{=\meta{text}}{no default, initially empty}
Sets \meta{text} before the content of the lower part (e.\,g.\ font settings).

附加 \meta{text} 到lower部分的内容之前(e.\,g.\ 字体设置)。
\begin{exdispExample}{fontlower}
\begin{tcolorbox}[fontlower=\sffamily\bfseries]
This is a \textbf{tcolorbox}.
\tcblower
This is the lower part.
\end{tcolorbox}
\end{exdispExample}
\end{docTcbKey}


\begin{docTcbKey}{fonttitle}{=\meta{text}}{no default, initially empty}
Sets \meta{text} before the content of the title text (e.\,g.\ font settings).

附加 \meta{text} 到标题文本的内容之前(e.\,g.\ 字体设置)。
\begin{exdispExample}{fonttitle}
\begin{tcolorbox}[fonttitle=\sffamily\bfseries\large,title=Hello]
This is a \textbf{tcolorbox}.
\end{tcolorbox}
\end{exdispExample}
\begin{exdispExample}{fonttitle2}
\begin{tcolorbox}[fonttitle=只加了文字,title=Hello]
This is a \textbf{tcolorbox}.
\end{tcolorbox}
\end{exdispExample}
\end{docTcbKey}

\bigskip
\begin{marker}
More color options are provided by using skins documented in
Section \ref{sec:skins} from page \pageref{sec:skins}.

更多的颜色选项是通过使用 \pageref{sec:skins} 页第 \ref{sec:skins} 节中介绍的 |skins| 提供的。
\end{marker}
% % \clearpage
% % 
% \subsection{Text Alignment文本对齐}
% \setcounter{section}{4}
\setcounter{subsection}{5}
\setcounter{subsubsection}{0}
 
\subsection{Text Alignment文本对齐}
\begin{docTcbKey}[][doc new=2015-05-07]{halign}{=\meta{alignment}}{no default, initially \texttt{justify}}
If there is no lower part, |halign| determines the horizontal \meta{alignment}
of the text content.
Otherwise, |halign| determines the horizontal \meta{alignment}
of the upper part of the box only.
The feasible values for \meta{alignment} are more or less identical to
the corresponding |/tikz/align| settings, even if the implementation differs.

确定盒子upper部分的水平\meta{alignment}方式。可行的\meta{alignment}值与相应的|/tikz/align|设置几乎相同,即使实现不同。

% 如果没有lower部分, |halign| 决定着水平的文本内容的对齐方式为 \meta{alignment}。
% 否则, |halign| 的 \meta{alignment} 只影响到upper部分的水平对齐。
% |halign| 决定着upper部分文本内容的水平对齐方式为 \meta{alignment}。
% \meta{alignment} 可选的值有不少同 |/tikz/align| 的相应设置是一样的, 即使实现有所不同。
\begin{DescriptionR}{\docValue{flush center}}
\item[\docValue{justify}] usual left and right justified type setting.
\par 常用的,左右对齐%的排版设置(两端对齐)。
\item[\docValue{left}]left border justification in analogy to plain \TeX.
\par 类似于plain \TeX 的左边界对齐。
% 向着盒子的左边框对齐,类似于 plain \TeX。
\item[\docValue{flush left}]
left border justification with |\raggedright| of \LaTeX.
\par 向着盒子的左边框对齐。%,使用 \LaTeX 的 |\raggedright|\footnote{ragged是不整齐的意思}。
\item[\docValue{right}]right border justification in analogy to plain \TeX.
\par 向着盒子的右边框对齐。%,类似于 plain \TeX。
\item[\docValue{flush right}]right border justification with |\raggedleft| of \LaTeX.
\par 向着盒子的右边框对齐。%使用 \LaTeX 的 |\raggedleft|。
\item[\docValue{center}]centering in analogy to plain \TeX.
\par 居中对齐.%,使用 plain \TeX。
\item[\docValue{flush center}]centering with |\centering| of \LaTeX.
\par 居中对齐。%,使用 \LaTeX 的 |\centering|。
\end{DescriptionR}
The differences between the flush and non-flush version are explained in
detail in the \tikzname\ manual \cite{tantau:tikz_and_pgf}. The short story is that
the non-flush versions will often look more balanced but with more
hyphenations.

% 在 \tikzname\ 手册 %\cite{tantau:tikz_and_pgf} 
% 中详细介绍了 flush 和 non-flush 版本之间的区别。简而言之 non-flush 版本通常看起来更加平衡,但是有更多的连字符。

在\tikzname\ 手册\cite{tantau:tikz_and_pgf}中详细解释了flush版本和non-flush版本之间的区别。简短的说,non-flush版本通常看起来更平衡,但会有更多的连字。

% \begin{tcolorbox}[adjusted title=flush center,halign=flush center]
%   This is a demonstration text for showing how line breaking works.
%   \end{tcolorbox}
%   \begin{tcolorbox}[adjusted title=flush left,halign=flush left]
%   This is a demonstration text for showing how line breaking works.
%   \end{tcolorbox}
%   \begin{tcolorbox}[adjusted title=flush right,halign=flush right]
%   This is a demonstration text for showing how line breaking works.
%   \end{tcolorbox}
  
%   \begin{tcolorbox}[adjusted title=center,halign=center]
%   This is a demonstration text for showing how line breaking works.
%   \end{tcolorbox}
%   \begin{tcolorbox}[adjusted title=left,halign=left]
%   This is a demonstration text for showing how line breaking works.
%   \end{tcolorbox}
%   \begin{tcolorbox}[adjusted title=right,halign upper=right]
%   This is a demonstration text for showing how line breaking works.
%   \end{tcolorbox}
  

\begin{exdispExample}{halign}
\tcbset{colback=red!5!white,colframe=red!75!black,size=small,
fonttitle=\bfseries,width=3.5cm,box align=top,
nobeforeafter}

\foreach \p in {flush center,flush left,flush right}
{\begin{tcolorbox}[adjusted title=\p,halign=\p]
This is a demonstration text for showing how line breaking works.
\end{tcolorbox}
}

\foreach \q in {center, left, right}
{\begin{tcolorbox}[adjusted title=\q,halign=\q]
This is a demonstration text for showing how line breaking works.
\end{tcolorbox}
} 

\end{exdispExample}
\end{docTcbKey}



\begin{docTcbKey}[][doc new=2015-05-07]{halign upper}{=\meta{alignment}}{no default, initially \texttt{justify}}
% Alias for \refKeyLe{/tcb/halign}.

\refKeyLe{/tcb/halign} 的别名
\end{docTcbKey}



% \newpage
\begin{docTcbKey}[][doc new=2015-05-07]{halign lower}{=\meta{alignment}}{no default, initially \texttt{justify}}
|halign lower| determines the horizontal \meta{alignment} of the lower part of the box.
The feasible values for \meta{alignment} are the same as for \refKeyLe{/tcb/halign}.

% |halign lower| 控制着盒子的lower部分的内容的水平对齐方式为 \meta{alignment}。\meta{alignment} 的可选值同 \refKeyLe{/tcb/halign} 一样。

|halign lower| 确定盒子lower部分的水平 \meta{alignment}。 \meta{alignment} 的可行值与 \refKeyLe{/tcb/halign} 相同。
\begin{exdispExample}{halign_lower}
\begin{tcbraster}[raster columns=3,fonttitle=\bfseries,
colback=red!5!white,colframe=red!75!black]

\begin{tcolorbox}[adjusted title=flush center,halign lower=flush center]
Upper part. \tcblower Lower part.
\end{tcolorbox}
\begin{tcolorbox}[adjusted title=flush left,halign lower=flush left]
Upper part. \tcblower Lower part.
\end{tcolorbox}
\begin{tcolorbox}[adjusted title=flush right,halign lower=flush right]
Upper part. \tcblower Lower part.
\end{tcolorbox}
\begin{tcolorbox}[adjusted title=center,halign lower=center]
Upper part. \tcblower Lower part.
\end{tcolorbox}
\begin{tcolorbox}[adjusted title=left,halign lower=left]
Upper part. \tcblower Lower part.
\end{tcolorbox}
\begin{tcolorbox}[adjusted title=right,halign lower=right]
Upper part. \tcblower Lower part.
\end{tcolorbox}

\end{tcbraster}
\end{exdispExample}
\end{docTcbKey}






% \clearpage
%这儿原来写错了
\begin{docTcbKey}[][doc new=2022-10-30]{halign title}{=\meta{alignment}}{no default, initially \texttt{justify}}
|halign title| determines the horizontal \meta{alignment} of the title of the box.
The feasible values for \meta{alignment} are the same as for \refKeyLe{/tcb/halign}.

|halign title| 设置盒子的标题部分的对齐方式为 \meta{alignment}。
\meta{alignment} 的可选值同 \refKeyLe{/tcb/halign} 一样。

\begin{exdispExample}{halign_title}
\begin{tcbraster}[raster columns=3,fonttitle=\bfseries,
colback=red!5!white,colframe=red!75!black]

\begin{tcolorbox}[adjusted title=flush center,halign title=flush center]
This is a \textbf{tcolorbox}.
\end{tcolorbox}
\begin{tcolorbox}[adjusted title=flush left,halign title=flush left]
This is a \textbf{tcolorbox}.
\end{tcolorbox}
\begin{tcolorbox}[adjusted title=flush right,halign title=flush right]
This is a \textbf{tcolorbox}.
\end{tcolorbox}
\begin{tcolorbox}[adjusted title=center,halign title=center]
This is a \textbf{tcolorbox}.
\end{tcolorbox}
\begin{tcolorbox}[adjusted title=left,halign title=left]
This is a \textbf{tcolorbox}.
\end{tcolorbox}
\begin{tcolorbox}[adjusted title=right,halign title=right]
This is a \textbf{tcolorbox}.
\end{tcolorbox}

\end{tcbraster}
\end{exdispExample}
\end{docTcbKey}




% \enlargethispage*{1cm}

\begin{docTcbKey}[][doc updated=2015-05-07]{flushleft upper}{}{style, no value}
Shortcut for setting \refKeyLe{/tcb/halign} to \docValue{flush left}.
将 \refKeyLe{/tcb/halign} 设置为 \docValue{flush left} 的简写形式。
\end{docTcbKey}

\begin{docTcbKey}[][doc updated=2015-05-07]{center upper}{}{style, no value}
Shortcut for setting \refKeyLe{/tcb/halign} to \docValue{flush center}.

将 \refKeyLe{/tcb/halign} 设置为 \docValue{flush center} 的简写形式。
\end{docTcbKey}

\begin{docTcbKey}[][doc updated=2015-05-07]{flushright upper}{}{style, no value}
Shortcut for setting \refKeyLe{/tcb/halign} to \docValue{flush right}.

将 \refKeyLe{/tcb/halign} 设置为 \docValue{flush right} 的简写形式。
\end{docTcbKey}

\begin{docTcbKey}[][doc updated=2015-05-07]{flushleft lower}{}{style, no value}
Shortcut for setting \refKeyLe{/tcb/halign lower} to \docValue{flush left}.

将 \refKeyLe{/tcb/halign lower} 设置为 \docValue{flush left} 的简写形式。
\end{docTcbKey}

\begin{docTcbKey}[][doc updated=2015-05-07]{center lower}{}{style, no value}
Shortcut for setting \refKeyLe{/tcb/halign lower} to \docValue{flush center}.
将 \refKeyLe{/tcb/halign lower} 设置为 \docValue{flush center} 的简写形式。
\end{docTcbKey}

\begin{docTcbKey}[][doc updated=2015-05-07]{flushright lower}{}{style, no value}
Shortcut for setting \refKeyLe{/tcb/halign lower} to \docValue{flush right}.

将 \refKeyLe{/tcb/halign lower} 设置为 \docValue{flush right} 的简写形式。
\end{docTcbKey}



% \clearpage

\begin{docTcbKey}[][doc updated=2015-05-07]{flushleft title}{}{style, no value}
Shortcut for setting \refKeyLe{/tcb/halign title} to \docValue{flush left}.

将 \refKeyLe{/tcb/halign title} 设置为 \docValue{flush left} 的简写形式。
\end{docTcbKey}

\begin{docTcbKey}[][doc updated=2015-05-07]{center title}{}{style, no value}
Shortcut for setting \refKeyLe{/tcb/halign title} to \docValue{flush center}.

将 \refKeyLe{/tcb/halign title} 设置主 \docValue{flush center} 的简写形式。
\end{docTcbKey}

\begin{docTcbKey}[][doc updated=2015-05-07]{flushright title}{}{style, no value}
Shortcut for setting \refKeyLe{/tcb/halign title} to \docValue{flush right}.

将 \refKeyLe{/tcb/halign title} 设置为 \docValue{flush right} 的简写形式。
\end{docTcbKey}


\begin{marker}
The vertical alignment settings are only relevant for boxes which are larger
than their natural height, see \Fullref{sec:heightcontrol}.

垂直对齐设置只适用于大于其自然高度的盒子。见 \Fullref{sec:heightcontrol}.
\end{marker}

\begin{docTcbKey}[][doc updated=2015-07-16]{valign}{=\meta{alignment}}{no default, initially |top|}
If the height of a |tcolorbox| is not the natural height, |valign|
determines the vertical \meta{alignment} of the upper part.
Feasible values are

如果一个 |tcolorbox| 的盒子的高度不是其自然高度\footnote{译注:指定了高度}, |valign| 控制着盒子的upper部分的对方方式 \meta{alignment} 。
可选的值有:
\begin{itemize}
\item\docValue{top}: %Anchor text at top.
顶部对齐。
\item\docValue{center}: %Anchor text at center.
中间对齐。
\item\docValue{bottom}:% Anchor text at bottom.
底部对齐。
\item\docValue{scale}: 
Scale text vertically to fit into the available space.
  This is brutal and may not look very good. Consider \Fullref{sec:fitting}
  alternatively.
垂直缩放文本以适应可用空间。
这简单粗暴,可能看起来不是很好。或者考虑下 \Fullref{sec:fitting}。


\item\docValue{scale*}: 
Like \docValue{scale}, but scaling is bounded by
  \refKeyLe{/tcb/valign scale limit}.

类似于\docValue{scale}, 但缩放范围受限于 \refKeyLe{/tcb/valign scale limit}.
\end{itemize}
For a box with natural height, these settings are meaningless.

对于具有自然高度的盒子,这些设置毫无意义。
\begin{exdispExample}{valign}
\tcbset{width=(\linewidth-2mm)/4,before=,after=\hfill,
colframe=blue!75!black,colback=white,height=2cm}

\foreach \myalign in {top,center,bottom,scale}
{\begin{tcolorbox}[valign=\myalign]
This is a \textbf{tcolorbox}.
\end{tcolorbox}}
\end{exdispExample}
\end{docTcbKey}






\begin{docTcbKey}[][doc new=2015-05-07]{valign upper}{=\meta{alignment}}{no default, initially \texttt{top}}
Alias for \refKeyLe{/tcb/valign}.

\refKeyLe{/tcb/valign} 的别名。
\end{docTcbKey}

\begin{docTcbKey}{valign lower}{=\meta{alignment}}{no default, initially |top|}
This key has the same meaning for the lower part as |valign|
for the upper part, i.\,e., it determines
the vertical \meta{alignment} of the lower part with feasible values
|top|, |center|, |bottom|, |scale|, and |scale*|.

此项设置含义同upper部分的 |valign| 相同, i.\,e., 它指定了lower部分的竖直方向的对齐方式 \meta{alignment} ,可选的值有 |top|, |center|, |bottom|, |scale|, 和 |scale*|.

\end{docTcbKey}

\begin{docTcbKey}[][doc new=2015-07-16]{valign scale limit}{=\meta{real number}}{no default, initially \texttt{1.1}}
Sets an upper scale limit for the \docValue{scale*} setting in
\refKeyLe{/tcb/valign} and \refKeyLe{/tcb/valign lower}.
Note that this value is not reset by \refKeyLe{/tcb/reset}. So, changes
also apply to embedded boxes.

设置 \docValue{scale*} 在指定 \refKeyLe{/tcb/valign} 和 \refKeyLe{/tcb/valign lower} 时缩放的范围上限。注意,此值不会随着 \refKeyLe{/tcb/reset} 而重置。所以,修改的话也会同时在嵌套的盒子中生效。
\end{docTcbKey}

Also see \refKeyLe{/tcb/sidebyside align} for alignment settings when
upper part and lower part are set side-by-side.

另见 \refKeyLe{/tcb/sidebyside align} 了解当设置为side-by-side左右排布时的对齐选项。




% % \clearpage
% % Geometry\hfill 
\subsection{Geometry\\几何形状}
\setcounter{section}{4}
\setcounter{subsection}{6}
\setcounter{subsubsection}{0}
\subsection{Geometry\\几何属性}

% \subsubsection{Width} 

\begin{docTcbKey}{width}{=\meta{length}}{no default, initially \cs{linewidth}}
Sets the total width of the colored box to \meta{length}.
See also \refKeyLe{/tcb/height}.

将%有色的带框
盒子的总宽度设置为 \meta{length}。另见 \refKeyLe{/tcb/height}。
\begin{exdispExample}{width} 
\tcbset{colback=red!5!white,colframe=red!75!black}

\begin{tcolorbox}[width=\linewidth/2]
这是一个\textbf{tcolorbox}.
\end{tcolorbox}
\end{exdispExample}
\end{docTcbKey}


\begin{docTcbKey}[][doc new=2014-10-31]{text width}{=\meta{length}}{style, no default}
Sets the text width of the upper part to \meta{length}.
See also \refKeyLe{/tcb/text height}.

设置upper部分的文本宽度为 \meta{length}.
另见 \refKeyLe{/tcb/text height}.
\begin{exdispExample}{text_width}
\tcbset{colback=red!5!white,colframe=red!75!black}

\begin{tcolorbox}[text width=4cm]
This is a \textbf{tcolorbox} where the text has a width of 4cm.
\end{tcolorbox}
\end{exdispExample}
\end{docTcbKey}

\begin{docTcbKey}[][doc new=2014-11-07]{add to width}{=\meta{length}}{style, no default}
Adds \meta{length} to the current total width of the colored box.

将当前盒子的总宽度增加 \meta{length} 。    
\begin{exdispExample*}{add_to_width}{sbs,lefthand ratio=0.6}
\tcbset{width=4cm,colback=red!5!white,
colframe=red!75!black}

\begin{tcolorbox}
这是一个\textbf{tcolorbox}.
\end{tcolorbox}

\begin{tcolorbox}[add to width=1cm]
这是一个\textbf{tcolorbox}.
\end{tcolorbox}
\end{exdispExample*}
\end{docTcbKey}
See \Fullref{sec:heightcontrol} for setting fixed height values.

有关设置固定高度值的问题,请参阅 \Fullref{sec:heightcontrol}。
% \setcounter{section}{4}
\setcounter{subsection}{7}
\setcounter{subsubsection}{1}

\subsubsection{Rules\\线}
\begin{docTcbKey}{toprule}{=\meta{length}}{no default, initially \texttt{0.5mm}}

设置顶边框线的宽度为 \meta{length}。\hfill Sets the line width of the top rule to \meta{length}.
\begin{exdispExample}{toprule}
\tcbset{colback=red!5!white,colframe=red!75!black}

\begin{tcolorbox}[toprule=3mm]
这是一个\textbf{tcolorbox}.
\end{tcolorbox}
\end{exdispExample}
\end{docTcbKey}


\begin{docTcbKey}{bottomrule}{=\meta{length}}{no default, initially \texttt{0.5mm}}
设置底边框线的宽度为 \meta{length}。\hfill Sets the line width of the bottom rule to \meta{length}.
\begin{exdispExample}{bottomrule}
\tcbset{colback=red!5!white,colframe=red!75!black}

\begin{tcolorbox}[bottomrule=3mm]
这是一个\textbf{tcolorbox}.
\end{tcolorbox}
\end{exdispExample}
\end{docTcbKey}

\begin{docTcbKey}{leftrule}{=\meta{length}}{no default, initially \texttt{0.5mm}}
设置左边框线的宽度为 \meta{length}。\hfill Sets the line width of the left rule to \meta{length}.
\begin{exdispExample}{leftrule}
\tcbset{colback=red!5!white,colframe=red!75!black}

\begin{tcolorbox}[leftrule=3mm]
这是一个\textbf{tcolorbox}.
\end{tcolorbox}
\end{exdispExample}
\end{docTcbKey}


\begin{docTcbKey}{rightrule}{=\meta{length}}{no default, initially \texttt{0.5mm}}
设置右边框线的宽度为 \meta{length}。\hfill Sets the line width of the right rule to \meta{length}.
\begin{exdispExample}{rightrule}
\tcbset{colback=red!5!white,colframe=red!75!black}

\begin{tcolorbox}[rightrule=3mm]
这是一个\textbf{tcolorbox}.
\end{tcolorbox}
\end{exdispExample}
\end{docTcbKey}




% \clearpage
\begin{docTcbKey}{titlerule}{=\meta{length}}{no default, initially \texttt{0.5mm}}
Sets the line width of the rule below the title to \meta{length}.

设置标题文本下方的线的宽度为 \meta{length}。
\begin{exdispExample}{titlerule}
\tcbset{enhanced,colback=red!5!white,colframe=red!75!black,
colbacktitle=red!90!black}

\begin{tcolorbox}[titlerule=3mm,title=This is the title]
这是一个\textbf{tcolorbox}.
\end{tcolorbox}
\end{exdispExample}
\end{docTcbKey}


\begin{docTcbKey}{boxrule}{=\meta{length}}{style, no default, initially \texttt{0.5mm}}
Sets all rules of the frame to \meta{length}, i.\,e.\ 
\refKeyLe{/tcb/toprule}, \refKeyLe{/tcb/bottomrule}, \refKeyLe{/tcb/leftrule},
\refKeyLe{/tcb/rightrule}, and \refKeyLe{/tcb/titlerule}.

设置所有的边框线的宽度为 \meta{length}\footnote{i.\,e.\ 
\refKeyLe{/tcb/toprule}, \refKeyLe{/tcb/bottomrule}, \refKeyLe{/tcb/leftrule},
\refKeyLe{/tcb/rightrule}, 和 \refKeyLe{/tcb/titlerule}.}。
\begin{exdispExample}{boxrule}
\tcbset{colback=red!5!white,colframe=red!75!black}

\begin{tcolorbox}[boxrule=3mm]
这是一个\textbf{tcolorbox}.
\end{tcolorbox}
\end{exdispExample}
\end{docTcbKey}

\bigskip
\begin{marker}
More options for drawing a \refKeyLe{/tcb/borderline} are provided by using skins documented in
Section \ref{sec:skins} from page \pageref{sec:skins}.

更多的关于绘制 \refKeyLe{/tcb/borderline} 的选项的描述在 skins 的文档中,详见 \pageref{sec:skins} 页的 \ref{sec:skins} 小节。
\end{marker}

 
% % Arcs\hfill 
\subsubsection{弧线}
\begin{docTcbKey}{arc}{=\meta{length}}{no default, initially \texttt{1mm}}
Sets the inner radius of the four frame arcs to \meta{length}.

设置边框的四个角落的弧的内半径为 \meta{length}。

% \begin{exdispExample}{arc}
% \tcbset{colback=red!5!white,colframe=red!75!black}
% \begin{tcolorbox}[arc=0mm]
% 这是一个\textbf{tcolorbox}.
% \end{tcolorbox}
% \begin{tcolorbox}[arc=3mm]
% 这是一个\textbf{tcolorbox}.
% \end{tcolorbox}
% \end{exdispExample}

\begin{dispExample*}{sidebyside,lefthand ratio=0.6}
\tcbset{colback=red!5!white,colframe=red!75!black}
\begin{tcolorbox}[arc=0mm]
这是一个\textbf{tcolorbox}.
\end{tcolorbox}
\end{dispExample*}

\begin{dispExample*}{sidebyside,lefthand ratio=0.6}
\tcbset{colback=red!5!white,colframe=red!75!black}
\begin{tcolorbox}[arc=3mm]
这是一个\textbf{tcolorbox}.
\end{tcolorbox}
\end{dispExample*}
\end{docTcbKey}



% \begin{exdispExample*}{arc_default}{sbs,lefthand ratio=0.6}
 
% \end{exdispExample*}

% \begin{exdispExample*}{arc_3cm}{sbs,lefthand ratio=0.6}
 
% \end{exdispExample*}

% gpt4:
% 在 LaTeX 中,颜色的混合可以使用 ! 符号来实现。这种写法叫做 "interpolation of colors"。

% red!5!white 的意思是,颜色由5%的红色和95%的白色混合而成。结果是一个非常浅的红色。

% red!75!black 的意思是,颜色由75%的红色和25%的黑色混合而成。结果是一个较深的红色。

% claude
% !后面的数字表示混合比例,范围是0到100。
% 0表示完全使用第一个颜色。
% 100表示完全使用第二个颜色。
% 所以:

% red!0!white 等同于纯红色red。
% red!100!white 等同于纯白色white。


% \clearpage
\begin{docTcbKey}[][doc new=2015-05-05]{circular arc}{}{style, no value}
  Sets \refKeyLe{/tcb/arc} to match the half of the inner width of the colored box.
  If width and height of the box are identical, this gives a circle.
  
  将 \refKeyLe{/tcb/arc} 设置为盒子内部宽度的一半。如果盒子的宽度和高度相同,就会得到一个圆。
  \begin{marker}
  If the height of the box is smaller than the width, the result will look
  quite ugly.
  
  如果盒子的高度小于宽度,结果看起来会很难看。   
  \end{marker}
  \begin{exdispExample*}{circular_arc}{sbs,lefthand ratio=0.6}
  \begin{tcolorbox}[width=3cm,
  colback=red!5!white,
  colframe=red!75!black,
  halign=center,valign=center,
  square,circular arc]
  这是一个\textbf{tcolorbox}.
  \end{tcolorbox}
  \end{exdispExample*}
  \end{docTcbKey}
  
  
  \begin{docTcbKey}[][doc new=2015-05-05]{bean arc}{}{style, no value}
  Sets \refKeyLe{/tcb/arc} to match the smaller value of the
  half of the inner width and of the inner height of the colored box.
  
%   设置 \refKeyLe{/tcb/arc} 为盒子内宽和内高中较小者值的一半。
  
  将\refKeyLe{/tcb/arc}设置为盒子内宽度一半和高度一半的较小值。

  \begin{marker}
  This only works for a fixed \refKeyLe{/tcb/height}. Also, \refKeyLe{/tcb/bean arc}
  must be used \emph{after} width and height are set by option keys.
  
%   这只适用于 \refKeyLe{/tcb/height} 为固定值的情况。此外,\refKeyLe{/tcb/bean arc} 选项设置需要在设置宽度和高度之后。
  这仅适用于固定的\refKeyLe{/tcb/height}。另外,\refKeyLe{/tcb/bean arc}必须在宽度和高度选项设置之后使用。
  \end{marker}
  \begin{exdispExample*}{bean_arc}{sbs,lefthand ratio=0.6}
  \tcbset{size=fbox,boxrule=0.5mm,
  colback=red!5!white,
  colframe=red!75!black,
  halign=center,valign=center}
  
  \begin{tcolorbox}[width=3cm,height=2cm,
  bean arc]
  Box A
  \end{tcolorbox}
  
  \begin{tcolorbox}[width=2cm,height=3cm,
  bean arc]
  Box B
  \end{tcolorbox}
  \end{exdispExample*}
  \end{docTcbKey}
  
  % 八角形
  \begin{docTcbKey}[][doc new=2015-05-05]{octogon arc}{}{style, no value}
  Sets \refKeyLe{/tcb/arc} to match $\frac{1}{2+\sqrt{2}}$ of the inner width
  of the colored box. If width and height of the box are identical,
  the interior is a regular octogon.

将\refKeyLe{/tcb/arc}设置为盒子的内部宽度的$\frac{1}{2+\sqrt{2}}$。如果盒子的宽度和高度相同,则内部是一个正八边形。

% 设置 \refKeyLe{/tcb/arc} 为盒子内部宽度的 $\frac{1}{2+\sqrt{2}}$ 。如果盒子的宽度和高度相同,内部是一个规则的八边形。
  % \begin{tcolorbox}[
  %   width=2.1cm,octogon arc,
  %   ]
  %   STOP
  %   \end{tcolorbox}
\begin{exdispExample*}{octogon_arc}{sbs,lefthand ratio=0.82}
\begin{tcolorbox}[enhanced,% 启用高级选项。
size=minimal,%去除默认边距,仅显示框体。
auto outer arc,% 自动调整外部圆角半径以适应框体大小。
width=2.1cm,
octogon arc,%  外部圆角为八角形。
colback=red,%  背景色为红色。
colframe=white,% 边框颜色为白色。
colupper=white,% 文字颜色为白色。
fontupper=\fontsize{7mm}{7mm}\selectfont\bfseries\sffamily,
%文字大小为7毫米,粗体,无衬线字体。
halign=center,% 水平居中对齐
valign=center,%垂直居中对齐。
square,% square: 边框为直角。
arc is angular,% arc is angular: 内部圆角为直角。
borderline={0.2mm}{-1mm}{red}
%使用红色的0.2毫米线条作为边框,内部偏移1毫米。
]
STOP
\end{tcolorbox}
\end{exdispExample*}
\end{docTcbKey}

% \begin{dispExample*}{sidebyside,lefthand ratio=0.82}
% \begin{tcolorbox}[enhanced,% 启用高级选项。
% size=minimal,%去除默认边距,仅显示框体。
% auto outer arc,% 自动调整外部圆角半径以适应框体大小。
% width=2.1cm,
% % octogon arc,%  外部圆角为八角形。
% colback=red,%  背景色为红色。
% colframe=white,% 边框颜色为白色。
% colupper=white,% 文字颜色为白色。
% fontupper=\fontsize{7mm}{7mm}\selectfont\bfseries\sffamily,
% %文字大小为7毫米,粗体,无衬线字体。
% halign=center,% 水平居中对齐
% valign=center,%垂直居中对齐。
% square,% square: 边框为直角。
% arc is angular,% arc is angular: 内部圆角为直角。
% borderline={0.2mm}{-1mm}{red}
% %使用红色的0.2毫米线条作为边框,内部偏移1毫米。
% ]
% STOP
% \end{tcolorbox}
% \end{dispExample*}


% angular,有角度的
% \clearpage
\begin{docTcbKey}[][doc new=2015-05-05]{arc is angular}{}{no value, initially unset}
  Using this options applies a patch which straightens the corners arcs of
  the boxes. The little arcs are replaced by little straight lines.
  
%   这个选项\footnote{arc is angular}用于对盒子四角弧线的一个补丁。小弧线被小直线代替。

  使用此选项将应用一个补丁,使盒子的角弧线变直。小弧线将被小直线替换。
  \begin{marker}
  This patch is considered as an experimental feature.
  It changes some of the original \tikzname\ code. This change may break
  with future updates of \tikzname.
  
%   这个补丁被认为是一个实验性的特征。它改变了一些原始的 \tikzname\ 代码。此更改可能会随着 \tikzname\ 的未来更新而中断。
  这个补丁被视为一项实验性的功能。它改变了一些原始的\tikzname 代码。这种改变可能会在未来的\tikzname 更新中出现问题。
  \end{marker}
  
\begin{exdispExample*}{arc_is_angular}{sbs,lefthand ratio=0.6}
\tcbset{colback=red!5!white,%
colframe=red!75!black,%
arc=3mm}

\begin{tcolorbox}[arc is angular]
arc is angular.
\end{tcolorbox}
\begin{tcolorbox}[arc is curved]
arc is curved
\end{tcolorbox}
\end{exdispExample*}
  
  \end{docTcbKey}
  
  
\begin{docTcbKey}[][doc new=2015-05-05]{arc is curved}{}{no value, initially set}
This option resets the patch from \refKeyLe{/tcb/arc is angular}. The
original \tikzname\ code is activated.

% 此选项将 \refKeyLe{/tcb/arc is angular} 的补丁修改重置为角度。激活原始的 \tikzname\ 代码。

此选项将重置 \refKeyLe{/tcb/arc is angular} 所设置的修补程序。原始的 \tikzname\ 代码将被激活。
\end{docTcbKey}


\begin{docTcbKey}{outer arc}{=\meta{length}}{no default, initially unset}
Sets the outer radius of the four frame arcs to \meta{length}.

设置盒子四个角的弧的外半径为 \meta{length}。
% 将四个框架拱形的外半径设置为\meta{length}。

\begin{dispExample*}{sidebyside}
\tcbset{colback=red!5!white,%
colframe=red!75!black}
\begin{tcolorbox}[]
这是一个\textbf{tcolorbox}.
\end{tcolorbox}
\end{dispExample*}

\begin{dispExample*}{sidebyside}
\tcbset{colback=red!5!white,%
colframe=red!75!black}
\begin{tcolorbox}[arc=4mm]
arc=4mm
\end{tcolorbox}
\end{dispExample*}

\begin{dispExample*}{sidebyside}
\tcbset{colback=red!5!white,%
colframe=red!75!black}
\begin{tcolorbox}[arc=4mm,%
outer arc=1mm]
arc=4mm,outer arc=1mm
\end{tcolorbox}
\end{dispExample*}
\end{docTcbKey}

\begin{docTcbKey}{auto outer arc}{}{no value, initially set}
Sets the outer radius of the four frame arcs automatically in
dependency of the inner radius given by \refKeyLe{/tcb/arc}.

根据 \refKeyLe{/tcb/arc} 给出的内部半径自动设置外部半径。

\begin{dispExample*}{sidebyside}
\tcbset{colback=red!5!white,%
colframe=red!75!black}
\begin{tcolorbox}[arc=4mm]
arc=4mm
\end{tcolorbox}
\end{dispExample*}

\begin{dispExample*}{sidebyside}
\tcbset{colback=red!5!white,%
colframe=red!75!black} 
\begin{tcolorbox}[arc=4mm,%
outer arc=1mm]
arc=4mm,outer arc=1mm
\end{tcolorbox}
\end{dispExample*}

\begin{dispExample*}{sidebyside}
\tcbset{colback=red!5!white,%
colframe=red!75!black}
\begin{tcolorbox}[arc=4mm,%
auto outer arc]
\verb|arc=4mm,auto outer arc|
\end{tcolorbox}
\end{dispExample*}

\begin{dispExample*}{sidebyside}
\tcbset{colback=red!5!white,%
colframe=red!75!black}
\begin{tcolorbox}[arc=4mm,%
outer arc=9mm]
\verb|arc=4mm,outer arc=9mm|
\end{tcolorbox}
\end{dispExample*}

\begin{dispExample*}{sidebyside}
\tcbset{colback=red!5!white,%
colframe=red!75!black}
\begin{tcolorbox}[arc=4mm,%
outer arc=0.1mm]
\verb|arc=4mm,outer arc=0.1mm|
\end{tcolorbox}
\end{dispExample*}
\end{docTcbKey}
  
% \setcounter{section}{4}
\setcounter{subsection}{7}
\setcounter{subsubsection}{3}

\subsubsection{Spacing\hfill 间隔}
\begin{docTcbKey}{boxsep}{=\meta{length}}{no default, initially \texttt{1mm}}
Sets a common padding of \meta{length} between the text content and the
frame of the box. This value is added to the key values of
|left|, |right|, |top|, |bottom|, and |middle| at the appropriate places.

% 在文本内容和盒子的框之间设置一个共同的填充宽度为 \meta{length}。 这个值会被添加到
% |left|, |right|, |top|, |bottom|, 和 |middle| 的合适位置。

在文本内容和盒子边框之间设置一个共同的填充 \meta{length}。这个值会添加到 |left|、|right|、|top|、|bottom| 和 |middle| 的关键值中,在适当的位置使用。

% \begin{dispExample*}{}
% \tcbset{colback=red!5!white,colframe=red!75!black,width=(\linewidth-4mm)/2, before=,after=\hfill}
% \begin{tcolorbox}
% hi
% \end{tcolorbox}
% \begin{tcolorbox}[draft]
% hi
% \end{tcolorbox}
% \end{dispExample*}
% %%%%%%%%
% \begin{dispExample*}{}
% \tcbset{colback=red!5!white,colframe=red!75!black,width=(\linewidth-4mm)/2, before=,after=\hfill}
% \begin{tcolorbox}[boxsep=0mm]
% hi
% \end{tcolorbox}
% \begin{tcolorbox}[boxsep=0mm,draft]
% hi
% \end{tcolorbox}
% \end{dispExample*}
% %%%%%%
\begin{dispExample*}{}
\tcbset{colback=red!5!white,colframe=red!75!black,width=(\linewidth-4mm)/2, before=,after=\hfill}

\begin{tcolorbox}[boxsep=5mm,draft]
hi
\end{tcolorbox}
\begin{tcolorbox}[boxsep=8mm,draft]
hi
\end{tcolorbox}
\begin{tcolorbox}[boxsep=5mm]
hi
\end{tcolorbox}
\begin{tcolorbox}[boxsep=8mm]
hi
\end{tcolorbox}
\end{dispExample*}

% \begin{exdispExample}{boxsep}
% \tcbset{colback=red!5!white,colframe=red!75!black  
% \begin{tcolorbox}[boxsep=0mm]
% hi
% \end{tcolorbox}
% \begin{tcolorbox}[boxsep=0mm,draft]
% hi
% \end{tcolorbox}
% \end{exdispExample}
\end{docTcbKey}


\begin{docTcbKey}{left}{=\meta{length}}{style, no default, initially \texttt{4mm}}
Sets the left space between all text parts and frame (additional to |boxsep|).
This is an abbreviation for setting
|lefttitle|, |leftupper|, and |leftlower| to the same value.

设置所有文本部分和盒子间的左边距(除了|boxsep|之外)。 这是设置|lefttitle|、|leftupper|和|leftlower|为相同值的简写方式。


% 设置所有的文本内容同左侧边框的间隔(附加到|boxsep|).
% 这是同时将 |lefttitle|, |leftupper|, 和 |leftlower| 设置为同一个值的简写方式。
\begin{dispExample*}{sbs}
\tcbset{colback=red!5!white%
,colframe=red!75!black
% ,width=(\linewidth-4mm)/2, before=,after=\hfill
}

\begin{tcolorbox}[left=0mm]
指定 \verb|left=0mm|
\end{tcolorbox}

\begin{tcolorbox}
使用默认
\end{tcolorbox}

\begin{tcolorbox}[left=4mm]
指定 \verb|left=4mm|
\end{tcolorbox}


\begin{tcolorbox}[left=10mm]
指定 \verb|left=10mm|
\end{tcolorbox}
\end{dispExample*}
\end{docTcbKey}

% TODO 再看下
\begin{docTcbKey}[][doc new=2017-02-16]{left*}{=\meta{length}}{style, no default}
Sets \refKeyLe{/tcb/left} such that \meta{length} is the distance between
the left bounding box and the text parts.

设置\refKeyLe{/tcb/left},使得\meta{length}为左边界框和文本部分之间的距离。

% 设置 \refKeyLe{/tcb/left} 的值 \meta{length} 为盒子左边界和上下文的文本左侧的距离。

\begin{exdispExample}{left_star}
\tcbset{colback=red!5!white,colframe=red!75!black}

This is some text.
\begin{tcolorbox}[grow to left by=5mm,left*=0mm,
enhanced,show bounding box]
\verb|grow to left by=5mm,left*=0mm|
\end{tcolorbox}

\begin{tcolorbox}[left*=0mm,
enhanced,show bounding box]
\verb|left*=0mm|
\end{tcolorbox}

\begin{tcolorbox}[%left*=0mm,
enhanced,show bounding box]
这是一个\textbf{tcolorbox}.
\end{tcolorbox}
\end{exdispExample}
\end{docTcbKey}

% 这段Latex代码使用了tcolorbox宏包,它提供了创建漂亮框框的命令。在这个例子中,代码创建了三个tcolorbox,每个框内都包含了一段文本“这是一个\textbf{tcolorbox}”。

% 第一个tcolorbox使用了参数“grow to left by=5mm”和“left*=0mm”,
% 这意味着这个框会向左边延伸5毫米,并且左边的边框宽度为0毫米。同时,也使用了“enhanced”和“show bounding box”参数,这些参数可以让框框看起来更漂亮,并且显示边框的边界。

% 第二个tcolorbox只使用了“left*=0mm”参数,这意味着这个框的左边边框宽度为0毫米。同样,也使用了“enhanced”和“show bounding box”参数。

% 第三个tcolorbox只使用了“enhanced”和“show bounding box”参数,这意味着这个框没有任何特殊的设置,只是一个普通的tcolorbox。


% \clearpage
\begin{docTcbKey}{lefttitle}{=\meta{length}}{no default, initially \texttt{4mm}}
  Sets the left space between title text and frame (additional to |boxsep|).

设置标题文本的左侧同边框的距离(附加 |boxsep|)。
\begin{exdispExample}{lefttitle}
\tcbset{colback=red!5!white,colframe=red!75!black}

\begin{tcolorbox}[lefttitle=3cm,title=My Title]
这是一个\textbf{tcolorbox}.
\end{tcolorbox}
\end{exdispExample}
\end{docTcbKey}


\begin{docTcbKey}{leftupper}{=\meta{length}}{no default, initially \texttt{4mm}}
  Sets the left space between upper text and frame (additional to |boxsep|).

设置upper部分同左侧边边框的距离(附加 |boxsep|)。
\begin{exdispExample}{leftupper}
\tcbset{colback=red!5!white,colframe=red!75!black}

\begin{tcolorbox}[leftupper=3cm,title=My Title]
这是一个\textbf{tcolorbox}.
\end{tcolorbox}
\end{exdispExample}
\end{docTcbKey}

\begin{docTcbKey}{leftlower}{=\meta{length}}{no default, initially \texttt{4mm}}
  Sets the left space between lower text and frame (additional to |boxsep|).

设置lower部分同左侧边边框的距离(附加 |boxsep|)。
\begin{exdispExample}{leftlower}
\tcbset{colback=red!5!white,colframe=red!75!black}

\begin{tcolorbox}[leftlower=3cm]
这是一个\textbf{tcolorbox}.
\tcblower
这是lower部分。
\end{tcolorbox}
\end{exdispExample}
\end{docTcbKey}

\enlargethispage*{1cm}

\begin{docTcbKey}{right}{=\meta{length}}{style, no default, initially \texttt{4mm}}
  Sets the right space between all text parts and frame (additional to |boxsep|).
  This is an abbreviation for setting
  |righttitle|, |rightupper|, and |rightlower| to the same value.

设置所有文本部分同右侧边框的距离(附加 |boxsep|)。
这是同时将 |righttitle|, |rightupper|, 和 |rightlower| 设置为同一个值的简写方式。
\begin{exdispExample}{right}
\tcbset{colback=red!5!white,colframe=red!75!black}

\begin{tcolorbox}[width=5cm,right=2cm]
这是一个\textbf{tcolorbox}.
\end{tcolorbox}
\end{exdispExample}
\end{docTcbKey}





% \clearpage
  
\begin{docTcbKey}[][doc new=2017-02-16]{right*}{=\meta{length}}{style, no default}
  Sets \refKeyLe{/tcb/right} such that \meta{length} is the distance between
  the right bounding box and the text parts.

设置 \refKeyLe{/tcb/right} 的宽度 \meta{length} 为盒子右边框同上下文文本的右侧的距离。
\begin{exdispExample}{right_star}
\tcbset{colback=red!5!white,colframe=red!75!black}

\flushright This is some text.
\begin{tcolorbox}[grow to right by=5mm,right*=0mm,
  halign=right,enhanced,show bounding box]
这是一个\textbf{tcolorbox}.
\end{tcolorbox}
\end{exdispExample}
\end{docTcbKey}



\begin{docTcbKey}{righttitle}{=\meta{length}}{no default, initially \texttt{4mm}}
  Sets the right space between title text and frame (additional to |boxsep|).

设置标题文本右侧同右边框的距离(附加 |boxsep|)。
  \begin{exdispExample}{righttitle}
\tcbset{colback=red!5!white,colframe=red!75!black}

\begin{tcolorbox}[width=5cm,righttitle=2cm,title=My very long title text]
This is a \textbf{tcolorbox} with standard upper box dimensions.
\end{tcolorbox}
\end{exdispExample}
\end{docTcbKey}


\begin{docTcbKey}{rightupper}{=\meta{length}}{no default, initially \texttt{4mm}}
  Sets the right space between upper text and frame (additional to |boxsep|).

设置upper部分的文本同右边框的距离(附加|boxsep|).
\begin{exdispExample}{rightupper}
\tcbset{colback=red!5!white,colframe=red!75!black}

\begin{tcolorbox}[width=5cm,rightupper=2cm,title=My very long title text]
This is a \textbf{tcolorbox} with compressed upper box dimensions.
\end{tcolorbox}
\end{exdispExample}
\end{docTcbKey}





% \clearpage
\begin{docTcbKey}{rightlower}{=\meta{length}}{no default, initially \texttt{4mm}}
  Sets the right space between lower text and frame (additional to |boxsep|).

设置lower部分的右边同右侧边框的距离(附加 |boxsep|)。
\begin{exdispExample}{rightlower}
\tcbset{colback=red!5!white,colframe=red!75!black}

\begin{tcolorbox}[width=5cm,rightlower=2cm]
This is a \textbf{tcolorbox} with standard upper box dimensions.
\tcblower
This is the lower part with large space at right.
\end{tcolorbox}
\end{exdispExample}
\end{docTcbKey}



\begin{docTcbKey}{top}{=\meta{length}}{no default, initially \texttt{2mm}}
  Sets the top space between text and frame (additional to |boxsep|).

设置文本同上边框的距离(附加 |boxsep|)。
\begin{exdispExample}{top}
\tcbset{colback=red!5!white,colframe=red!75!black}

\begin{tcolorbox}[top=0mm]
这是一个\textbf{tcolorbox}.
\tcblower
这是lower部分。
\end{tcolorbox}
\end{exdispExample}
\end{docTcbKey}


\begin{docTcbKey}{toptitle}{=\meta{length}}{no default, initially \texttt{0mm}}
  Sets the top space between title and frame (additional to |boxsep|).

设置标题文本同上边框的距离(附加 |boxsep|)。    
\begin{exdispExample}{toptitle}
\tcbset{colback=red!5!white,colframe=red!75!black}

\begin{tcolorbox}[toptitle=3mm,title=My title]
这是一个\textbf{tcolorbox}.
\end{tcolorbox}
\end{exdispExample}
\end{docTcbKey}






% \clearpage
\begin{docTcbKey}{bottom}{=\meta{length}}{no default, initially \texttt{2mm}}
  Sets the bottom space between text and frame (additional to |boxsep|).

设置文本底部同边框的距离 (附加 |boxsep|).
\begin{exdispExample}{bottom}
\tcbset{colback=red!5!white,colframe=red!75!black}

\begin{tcolorbox}[bottom=0mm]
这是一个\textbf{tcolorbox}.
\tcblower
这是lower部分。
\end{tcolorbox}
\begin{tcolorbox}
  这是一个\textbf{tcolorbox}.
  \tcblower
  这是lower部分。
  \end{tcolorbox}
\end{exdispExample}
\end{docTcbKey}

\begin{docTcbKey}{bottomtitle}{=\meta{length}}{no default, initially \texttt{0mm}}
  Sets the bottom space between title and frame (additional to |boxsep|).

设置标题同下方的边框的距离(附加 |boxsep|).
\begin{exdispExample}{bottomtitle}
\tcbset{colback=red!5!white,colframe=red!75!black}

\begin{tcolorbox}[bottomtitle=3mm,title=My title]
这是一个\textbf{tcolorbox}.
\end{tcolorbox}
\end{exdispExample}
\end{docTcbKey}


\begin{docTcbKey}{middle}{=\meta{length}}{no default, initially \texttt{2mm}}
Sets the space between upper and lower text to the separation line
(additional to |boxsep|).

% 将上下文本与分隔线之间的距离设置为分隔线(附加到|boxsep|)。

设置上下文本同分隔线的距离(附加 |boxsep|)。
\begin{exdispExample}{middle}
\tcbset{colback=red!5!white,colframe=red!75!black}

\begin{tcolorbox}[middle=0mm,boxsep=0mm]
这是一个\textbf{tcolorbox}.
\tcblower
这是lower部分。
\end{tcolorbox}
\begin{tcolorbox}[boxsep=0mm]
这是一个\textbf{tcolorbox}.
\tcblower
这是lower部分。
\end{tcolorbox}
\begin{tcolorbox}
这是一个\textbf{tcolorbox}.
\tcblower
这是lower部分。
\end{tcolorbox}
\end{exdispExample}
\end{docTcbKey}


% \subsubsection{Size Shortcuts\\调整尺寸的快捷方式}
\begin{docTcbKey}{size}{=\meta{name}}{no default, initially \texttt{normal}}
Sets all geometry keys with exception of \refKeyLe{/tcb/width} to
predefined length values.
For \meta{name}, the following values are feasible:

将除 \refKeyLe{/tcb/width} 外的所有尺寸设置为预定义的值。
可选的 \meta{name} 值有:    
  \begin{DescriptionL}{\docValue{minimal}}
  \item[\docValue{normal}]normal sized boxes e.g. of width |\linewidth|.
\\常用的盒子尺寸 e.g. 宽度为 |\linewidth|。
  \item[\docValue{title}]title line sized boxes.
  \\宽度同标题行一致。
  \item[\docValue{small}] small boxes e.g. for keyword highlighting.
  \\小一些的盒子 e.g. 用于关键字的高亮。
  \item[\docValue{fbox}] identical to the standard |\fbox|.
  \\同使用 |\fbox| 一样。
  \item[\docValue{tight}] no padding space at all.
  \\完全没有填充空间。

  \item[\docValue{minimal}] no padding space, no box rules.
  \\没有填充空间,没有边框。
  \end{DescriptionL}
% todo on line 是什么意思
\begin{exdispExample}{size_1}
\tcbset{colback=red!5!white,colframe=red!75!black}

\foreach \s in {normal,title,small,fbox,tight,minimal} {
  \tcbox[size=\s,on line]{\s} }

\foreach \s in {normal,title,small,fbox,tight,minimal} {
  \tcbox[size=\s,on line,title=Test]{\s} }

\foreach \s in {normal,title,small,fbox,tight,minimal} {
  \begin{tcolorbox}[size=\s,on line,title=Test,width=2.2cm]
    \s \tcblower lower\end{tcolorbox} }
\end{exdispExample}

\bigskip

\begin{tcolorbox}[tabularx={l|XXXXXX},title=Predefined values,
enhanced,fonttitle=\small\bfseries,fontupper=\small\ttfamily,
colback=yellow!10!white,colframe=red!50!black,colbacktitle=Salmon!30!white,
coltitle=black,center title
]
            & normal & title  & small & fbox  & tight & minimal\\\hline
boxrule     & 0.5mm  & 0.4mm  & 0.3mm & 0.4pt & 0.4pt & 0.0pt \\
boxsep      & 1.0mm  & 1.0mm  & 1.0mm & 3.0pt & 0.0pt & 0.0pt \\
left        & 4.0mm  & 2.0mm  & 1.0mm & 0.0pt & 0.0pt & 0.0pt \\
right       & 4.0mm  & 2.0mm  & 1.0mm & 0.0pt & 0.0pt & 0.0pt \\
top         & 2.0mm  & 0.25mm & 0.0mm & 0.0pt & 0.0pt & 0.0pt \\
bottom      & 2.0mm  & 0.25mm & 0.0mm & 0.0pt & 0.0pt & 0.0pt \\
toptitle    & 0.0mm  & 0.0mm  & 0.0mm & 0.0pt & 0.0pt & 0.0pt \\
bottomtitle & 0.0mm  & 0.0mm  & 0.0mm & 0.0pt & 0.0pt & 0.0pt \\
middle      & 2.0mm  & 0.75mm & 0.5mm & 1.0pt & 0.2pt & 0.0pt \\
arc         & 1.0mm  & 0.75mm & 0.5mm & 1.0pt & 0.0pt & 0.0pt \\
outer arc   & auto   & auto   & auto  & auto  & 0.0pt & 0.0pt \\
\end{tcolorbox}
\end{docTcbKey}


  

% \clearpage
\begin{docTcbKey}{oversize}{\colOpt{=\meta{length}}}{style, default |0pt|}
Sets the text width of the upper part to the current line width plus an
optional \meta{length}.
This is achieved by changing the keys \refKeyLe{/tcb/width}
\refKeyLe{/tcb/enlarge left by}, and
\refKeyLe{/tcb/enlarge right by} appropriately.
The resulting box is overlapping into the left and right margin of
the page.
Note that this style option has to be given \emph{after} all other
geometry keys!
Also see \refKeyLe{/tcb/grow sidewards by} and \refKeyLe{/tcb/spread sidewards}.

将upper部分的文本宽度设置为当前行宽再加上可选的\meta{length}。这是通过适当地更改键\refKeyLe{/tcb/width}、\refKeyLe{/tcb/enlarge left by}和\refKeyLe{/tcb/enlarge right by}来实现的。结果的框重叠到页面的左右边距上。请注意,这个样式选项必须在所有其他几何键之后给出!还请参见\refKeyLe{/tcb/grow sidewards by}和\refKeyLe{/tcb/spread sidewards}。

% 设置upper部分的文本的宽度为上下文中行宽加上 \meta{length}。这个效果是通过适当的改变 \refKeyLe{/tcb/width} \refKeyLe{/tcb/enlarge left by}, 和 \refKeyLe{/tcb/enlarge right by} 实现的。
% 最终的盒子会向左和右侧的边注伸展。注意,这个选项应该放置在所有其他的尺寸选项\emph{之后}!
% 另见 \refKeyLe{/tcb/grow sidewards by} 和 \refKeyLe{/tcb/spread sidewards}.
\begin{dispListing}
\tcbset{colback=red!5!white,colframe=red!75!black,fonttitle=\bfseries}

\textit{用于比较的普通文本:}\\
\lipsum[2]

\begin{tcolorbox}[oversize,title=Oversized box]
\lipsum[2]
\end{tcolorbox}

\begin{tcolorbox}[title=Normal box]
\lipsum[2]
\end{tcolorbox}
\end{dispListing}
\end{docTcbKey}

{\tcbusetemp}

  % 2023-1029
% \clearpage
% \hfill 
\subsubsection{Toggle Left and Right\\左右设置切换}
\begin{docTcbKey}[][doc updated=2017-02-16]{toggle left and right}{=\meta{toggle preset}}{default |evenpage|, initially |none|}
According to the \meta{toggle preset}, the left and the right settings
of the |tcolorbox| are switched or not. Feasible values are:

根据 \meta{toggle preset}, |tcolorbox| 的左右设置是否对换。 可选的值有:
  \begin{DescriptionL}{\docValue{evenpage}}
  \item[\docValue{none}]no switching.
不对换。
  \item[\docValue{forced}]the values of the left and right rules, spaces, and corners are switched.
左右的线、距离和四个角的设置互换。
  \item[\docValue{evenpage}]
  if the page is an even page, the values of the left and
    right rules, spaces, and corners are switched. This value also sets
    \refKeyLe{/tcb/check odd page} to |true|.
偶数页的左右线条数值,距离和四个角的设置互换。 此设置同时将 \refKeyLe{/tcb/check odd page} 设置为 |true|.
  \end{DescriptionL}
\begin{marker}
Horizontal bounding box enlargements are not toggled by this option.
They can be toggled independently by \refKeyLe{/tcb/toggle enlargement}.
For example, \refKeyLe{/tcb/oversize} changes the bounding box.

此选项不会将盒子的水平边框放大。
它们可以通过 \refKeyLe{/tcb/toggle enlargement} 独立地进行切换。
例如, \refKeyLe{/tcb/oversize} 更改盒子的的边界框。
\end{marker}
\begin{dispListing}
% \usepackage{lipsum}
% \usetikzlibrary{patterns}
% \tcbuselibrary{skins,breakable}
\begin{tcolorbox}[enhanced,% 启用增强功能,支持更多的选项。
breakable,%当盒子过长时,可以自动分页。
toggle left and right,%使盒子可以在左右两侧切换。
sharp corners,% 使盒子的角变得尖锐。
boxrule=0mm,top=0mm,bottom=0mm,left=1mm,right=1mm,
rightrule=1cm,colupper=blue!25!black,
interior style={fill overzoom image=lichtspiel.jpg,fill image opacity=0.25},
frame style={pattern=crosshatch dots light steel blue},
overlay={%定义要在盒子上覆盖的内容,这里是一个填充球和一个交叉标记。
\begin{tcbclipframe}
\tcbifoddpage{\coordinate (X) at ([xshift=-5mm]frame.east);}
            {\coordinate (X) at ([xshift=5mm]frame.west);}
\fill[shading=ball,ball color=blue!50!white,opacity=0.5] (X) circle (4mm);
\end{tcbclipframe}}]
\lipsum[1-6]
\end{tcolorbox}
\end{dispListing}
\medskip

% 这是一个tcolorbox环境,会在页面上创建一个带有填充图片和交叉标记的盒子。


% boxrule,top,bottom,left,right:控制盒子边框的宽度和位置。
% rightrule:在盒子右侧添加一条宽度为1厘米的边框。
% colupper:定义盒子中的文本颜色。
% interior style:定义盒子内部的样式,包括填充图片和填充透明度。
% frame style:定义盒子边框的样式,包括交叉标记和颜色。
% overlay:定义要在盒子上覆盖的内容,这里是一个填充球和一个交叉标记。
% 整个盒子中放置了一个lipsum段落,用于测试盒子的分页效果。

This example switches a |1cm| thick rule from the left to the right side
depending on the page number. Thereby, the rule is always on the outer side
of the double-sided paper. Additionally, a ball is drawn on the outer side
with help of an overlay.

这个例子根据页面编号将厚度为|1cm|的标尺从左侧移到右侧。因此,该标尺始终位于双面纸的外侧。此外,通过叠加层在外侧绘制一个球。

% 此示例根据页码将 |1cm| 宽的线条从左边切换到右边。
% 因此,线条总是在双面纸的外侧。
% 此外,一个球形,通过 overlay 绘制在外侧。
\bigskip

\tcbusetemp
\end{docTcbKey}

%todo 可以调整下





% % \clearpage
% % Corners\hfill 
% \subsection{四角}\label{subsec:corners}

% % The four corners of any |tcolorbox| can be set individually as
% % \refKey{/tcb/sharp corners} or as \refKey{/tcb/rounded corners}.
% % These settings are also reflected in the behavior of \refKey{/tcb/borderline}
% % and \refKey{/tcb/shadow} as one would expect.

% 任何 |tcolorbox| 的四个角可以通过 \refKey{/tcb/sharp corners} 或 \refKey{/tcb/rounded corners} 单独设置。
% 这些设置也影响到 \refKey{/tcb/borderline} 和 \refKey{/tcb/shadow} 的表现。

% % By default, all four corners are \emph{rounded}. So, only the
% % \refKey{/tcb/sharp corners} option will be necessary for most use cases.
% % The \refKey{/tcb/rounded corners} option can be used to revert a \refKey{/tcb/sharp corners}
% % setting.

% 默认情况下, 四角都是\emph{圆润的}。 因此,只有
% \refKey{/tcb/sharp corners} 选项是需要在大多数用例中显式指定。
% \refKey{/tcb/rounded corners} 选项可以用于重置 \refKey{/tcb/sharp corners}
% 修改过的设定。


% \begin{docTcbKey}{sharp corners}{=\meta{position}}{default |all|, initially unset}
% % The \meta{position} denotes one or more of the four box corners to be set as
% % \emph{sharp} corners. The not assigned corners will retain their mode.
% % Feasible values for \meta{position} are:

% \meta{position} 用来将盒子的四个角中的一个或多个设置为\emph{sharp}(直角) ,未指定的角将保持原来的模式(圆角)。
% 可设置的 \meta{position} 值有:

% \begin{itemize}
% \foreach \p in {northwest,northeast,southwest,southeast,north,south,east,west,downhill,uphill,all}
% {
% \item\tcbox[on line,size=title,arc=2mm,colframe=red!75!black,colback=red!5!white,
%   enlarge top by=0.5mm,enlarge bottom by=0.5mm,sharp corners=\p]{\docValue{\p}}
% }
% \end{itemize}
% \begin{exdispExample*}{sharp_corners_1}{sbs,lefthand ratio=0.6}
% \begin{tcolorbox}[colback=red!5!white,
%   colframe=red!75!black,
%   sharp corners=northwest ]
% |sharp corners=|\textbf{northwest},%
% 北西,左上角为直角。
% \end{tcolorbox}
% \end{exdispExample*}
% \begin{exdispExample*}{sharp_corners_2}{sbs,lefthand ratio=0.6}
% \begin{tcolorbox}[colback=red!5!white,
%   colframe=red!75!black,
%   sharp corners ]
% This is a \textbf{tcolorbox}.
% \end{tcolorbox}
% \end{exdispExample*}
% \end{docTcbKey}



  
% % \clearpage
% \begin{docTcbKey}{rounded corners}{=\meta{position}}{default |all|, initially |all|}
%   % The \refKey{/tcb/rounded corners} can be used to revert a \refKey{/tcb/sharp corners}
%   % setting. The \meta{position} denotes one or more of the four box corners to be set as
%   % \emph{rounded} corners. The not assigned corners will retain their mode.
%   % Feasible values for \meta{position} are\footnote{The graphical examples assume
%   %   that the boxes where set to have sharp corners before.}:
  
%   \refKey{/tcb/rounded corners} 可以用来重置 \refKey{/tcb/sharp corners} 带来的设置的修改。 \meta{position} 用来将盒子的四个角中的一个或多个设置为 \emph{rounded}(圆角)。未指定的角将保持原来的模式。
%   \meta{position}可用的值有\footnote{例子中假设设置之前有尖角的盒子。}:
%   \begin{itemize}
%   \foreach \p in {northwest,northeast,southwest,southeast,north,south,east,west,downhill,uphill,all}
%   {
%   \item\tcbox[on line,size=title,arc=2mm,colframe=red!75!black,colback=red!5!white,
%     enlarge top by=0.5mm,enlarge bottom by=0.5mm,sharp corners,rounded corners=\p]{\docValue{\p}}
%   }
%   \end{itemize}
%   \begin{exdispExample*}{rounded_corners}{sbs,lefthand ratio=0.6}
%   \begin{tcolorbox}[colback=red!5!white,
%     colframe=red!75!black,sharp corners,
%     rounded corners=northwest ]
%   |rounded corners=northwest|,北西,左上角为圆角。
%   \end{tcolorbox}
%   \end{exdispExample*}
%   \end{docTcbKey}
  
  
%   \begin{docTcbKey}{sharpish corners}{}{style, no value}
%     % Shortcut for setting \refKey{/tcb/arc} and \refKey{/tcb/outer arc}
%     % to |0pt|. With this setting, rounded corners will appear as quasi-sharp,
%     % but e.\,g.\ the shadow will be somewhat rounder than the shadow
%     % of really sharp corners.
  
%   同时设置 \refKey{/tcb/arc} 和 \refKey{/tcb/outer arc} 到 |0pt| 的简写。通过这项设置, 圆角展示得很像直角, 但 e.\,g.\ 阴影会比真正的直角的阴影稍微圆一些。
%     \begin{marker}
%     % Corners are still of type \emph{rounded} with this option, but appear
%     % \emph{sharp}. To switch back to rounded corners, one has to adapt
%     % \refKey{/tcb/arc} and \refKey{/tcb/outer arc}.
  
%   本项设置后,四个角的类型仍然是 \emph{rounded} , 但展现为\emph{sharp}。要切回 rounded corners, 需要修改 \refKey{/tcb/arc} 和 \refKey{/tcb/outer arc}。
%     \end{marker}
%   \begin{exdispExample*}{sharpish_corners}{sbs,lefthand ratio=0.6}
%   \begin{tcolorbox}[colback=red!5!white,
%     colframe=red!75!black,
%     sharpish corners ]
%   This is a \textbf{tcolorbox}.
%   \end{tcolorbox}
%   \end{exdispExample*}
%   \end{docTcbKey}

  


% % \clearpage
  
% % The following examples will show the differences between
% % \refKey{/tcb/rounded corners}, \refKey{/tcb/sharpish corners}, and \refKey{/tcb/sharp corners}.
% % The later two give the same core box, but \refKey{/tcb/borderline}
% % and \refKey{/tcb/shadow} settings are slightly different.
% % The following examples use \refKey{/tcb/drop fuzzy shadow}.

% 下面的例子展示了 \refKey{/tcb/rounded corners}, \refKey{/tcb/sharpish corners}, 和 \refKey{/tcb/sharp corners} 的区别。
% 后两个的内部盒子是相同的, 但 \refKey{/tcb/borderline} 和 \refKey{/tcb/shadow} 的设置有点不同。
% 下面的例子使用了 \refKey{/tcb/drop fuzzy shadow}。


% \begin{extcolorbox}[minipage]{corners_comparison}[blankest]
% \foreach \n in {rounded corners,sharpish corners,sharp corners}{
% \begin{tcolorbox}[enhanced jigsaw,frame empty,interior empty,fuzzy halo,halign=center,beforeafter skip=4mm]
% \begin{tcolorbox}[enhanced,drop fuzzy shadow,width=\linewidth-1cm,
%   colback=red!5!white, colframe=red!75!black, fonttitle=\bfseries,
%   title=My title,\n,
%   tikz={spy using outlines={circle, magnification=8, size=2cm, connect spies}},
%   overlay={\spy [blue, size=4cm] on (frame.south east)
%       in node at ([xshift=-2.5cm,yshift=-2.5cm]frame.south east);
%   \node[right] at ([xshift=2cm,yshift=-1cm]frame.south west) {\textbf{\Large\ttfamily\n}};
%   }]
% This is a \textbf{tcolorbox}.
% \end{tcolorbox}
% \end{tcolorbox}}
% \end{extcolorbox}



% % \clearpage
% % Transparency \hfill 
% \subsection{透明度}

% \begin{marker}
% % Transparency effects are likely to be used in conjunction with \emph{jigsaw}
% % skin variants, see \Vref{subsec:skinjigsaw}.

% 透明效果常常同skin的\emph{jigsaw}变量配合使用, 见 \Vref{subsec:skinjigsaw}.
% \end{marker}

% \begin{docTcbKey}{opacityframe}{=\meta{fraction}}{no default, initially \texttt{1.0}}
%   % Sets the frame opacity of the box to the given \meta{fraction}.

% 根据给定的 \meta{fraction} 值设置盒子的边框%线
% 的不透明度。
% \begin{exdispExample*}{opacityframe}{sbs,lefthand ratio=0.6,
%   segmentation style={preaction={fill=white},pattern=checkerboard,pattern color=gray!40}
%   }
% \begin{tcolorbox}[opacityframe=0.25
%   ,title=设置了边框的颜色和透明度,
%   colframe=red]
% This is a \textbf{tcolorbox}.
% \end{tcolorbox}
% \begin{tcolorbox}[colframe=red
%   ,title=设置了边框的颜色]
% This is a \textbf{tcolorbox}.
% \end{tcolorbox}
% \end{exdispExample*}
% \end{docTcbKey}

% \begin{docTcbKey}{opacityback}{=\meta{fraction}}{no default, initially \texttt{1.0}}
%   % Sets the background opacity of the box to the given \meta{fraction}.

% 根据给定的 \meta{fraction} 值设置盒子的背景色的不透明度。
% \begin{exdispExample*}{opacityback}{sbs,lefthand ratio=0.6,segmentation style={preaction={fill=white},pattern=checkerboard,pattern color=gray!40}}
% \begin{tcolorbox}[standard jigsaw,colframe=red,
%   opacityframe=0.5, opacityback=0.5]
% This is a \textbf{tcolorbox}.
% \end{tcolorbox}
% \end{exdispExample*}
% \end{docTcbKey}

% 另见 \mylib{skins} 库的 \refKey{/tcb/opacitybacklower}。

 

% \begin{docTcbKey}{opacitybacktitle}{=\meta{fraction}}{no default, initially \texttt{1.0}}
%   % Sets the title background opacity of the box to the given \meta{fraction}.

% 根据给定的 \meta{fraction} 值设置盒子的标题的背景色的不透明度。
% \begin{exdispExample*}{opacitybacktitle}{sbs,lefthand ratio=0.6,segmentation style={preaction={fill=white},pattern=checkerboard,pattern color=gray!40}}
% \begin{tcolorbox}[standard jigsaw,colframe=red,
%   opacityframe=0.5, opacitybacktitle=0.5,
%   title filled, title=This is a title]
% This is a \textbf{tcolorbox}.
% \end{tcolorbox}
% \end{exdispExample*}
% \end{docTcbKey}


% \begin{docTcbKey}{opacityfill}{=\meta{fraction}}{style, no default, initially \texttt{1.0}}
%   % Sets the fill opacity for frame, interior and optionally the title background
%   % to the given \meta{fraction}.

% 根据给定的 \meta{fraction} 值设置盒子的边框、内部和可选的标题背景的填充不透明度。
% \begin{exdispExample*}{opacityfill}{sbs,lefthand ratio=0.6,segmentation style={preaction={fill=white},pattern=checkerboard,pattern color=gray!40}}
% \begin{tcolorbox}[standard jigsaw,colframe=red,
%   opacityfill=0.7, title=This is a title]
% This is a \textbf{tcolorbox}.
% \end{tcolorbox}
% \begin{tcolorbox}[standard jigsaw,colframe=red,
% title=This is a title]
% This is a \textbf{tcolorbox}.
% \end{tcolorbox}
% \end{exdispExample*}
% \end{docTcbKey}





% % \clearpage
% \begin{docTcbKey}{opacityupper}{=\meta{fraction}}{no default, initially \texttt{1.0}}
%   % Sets the text opacity of the upper box part to the given \meta{fraction}.

% 根据给定的 \meta{fraction},设置upper部分的文本的不透明度。
% \begin{exdispExample*}{opacityupper}{sbs,lefthand ratio=0.6}
% \begin{tcolorbox}[enhanced,opacityupper=0.5
%   ,interior style={preaction={fill=white}
%   ,pattern=checkerboard
%   ,pattern color=gray!40}]
% This is a \textbf{tcolorbox}.
% \end{tcolorbox}
% \end{exdispExample*}
% \end{docTcbKey}



% \begin{docTcbKey}{opacitylower}{=\meta{fraction}}{no default, initially \texttt{1.0}}
%   % Sets the text opacity of the lower box part to the given \meta{fraction}.

% 根据给定的 \meta{fraction},设置lower部分的文本的不透明度。
% \begin{exdispExample*}{opacitylower}{sbs,lefthand ratio=0.6}
% \begin{tcolorbox}[enhanced,opacitylower=0.5,
%   interior style={preaction={fill=white},pattern=checkerboard,pattern color=gray!40}]
% This is a \textbf{tcolorbox}.
% \tcblower
% This is the lower part.
% \end{tcolorbox}
% \end{exdispExample*}
% \end{docTcbKey}

% \begin{docTcbKey}{opacitytext}{=\meta{fraction}}{no default, initially \texttt{1.0}}
%   % Sets the text opacity of the upper and the lower box part to the given \meta{fraction}.

% 根据给定的 \meta{fraction},设置upper和lower半两部分的文本的不透明度。
% \begin{exdispExample*}{opacitytext}{sbs,lefthand ratio=0.6}
% \begin{tcolorbox}[enhanced,opacitytext=0.5,
%   interior style={preaction={fill=white},pattern=checkerboard,pattern color=gray!40}]
% This is a \textbf{tcolorbox}.
% \tcblower
% This is the lower part.
% \end{tcolorbox}
% \end{exdispExample*}
% \end{docTcbKey}


% \begin{docTcbKey}{opacitytitle}{=\meta{fraction}}{no default, initially \texttt{1.0}}
%   % Sets the text opacity of the box title to the given \meta{fraction}.

% 根据给定的 \meta{fraction},设置标题的文本的不透明度。
% \begin{exdispExample*}{opacitytitle}{sbs,lefthand ratio=0.6}
% \begin{tcolorbox}[enhanced,opacitytitle=0.7,
%   coltitle=black,
%   fonttitle=\bfseries,title=This is a title,
%   title style={preaction={fill=white},pattern=checkerboard,pattern color=gray!40}]
% This is a \textbf{tcolorbox}.
% \end{tcolorbox}
% \end{exdispExample*}
% \end{docTcbKey}


% \begin{exdispExample*}{opacity_general}{segmentation style={preaction={fill=white},pattern=checkerboard,pattern color=gray!40}}
% \begin{tcolorbox}[enhanced jigsaw,fonttitle=\bfseries,title=This is a title,
%   opacityframe=0.5,opacityback=0.25,opacitybacktitle=0.25,opacitytext=0.8,
%   colback=red!5!white,colframe=red!75!black,colbacktitle=yellow!20!red]
% This is a \textbf{tcolorbox}.
% \end{tcolorbox}
% \end{exdispExample*}



 


% % \clearpage
% % Height Control\hfill 
% \subsection{高度控制}\label{sec:heightcontrol}
% % In a typical usage scenario, the height of a |tcolorbox| is computed automatically
% % to fit the content. Nevertheless, the height can be set to a fixed value
% % or to fit commonly for several boxes, e.\,g. if boxes are set side by side.

% 在典型的使用场景中,一个 |tcolorbox| 的高度是根据内容自动计算的。 
% 尽管如此,其高度是可以设置为一个固定的值,或自适就能放置指定个数的盒子, e.\,g. 如果例子设置为并排。

% \bigskip
% \begin{marker}
%   % The height control keys are only applicable to unbreakable boxes.
%   % If a box is set to be \refKey{/tcb/breakable}, the height is always
%   % computed according to the \emph{natural height}.

% 高度控制只在不可分页的例子上使用。
% 如果一个盒子设置了 \refKey{/tcb/breakable}, 其他高度将终究使用自然高度(\emph{natural height}).
% \end{marker}
% \bigskip


% \begin{docTcbKey}{natural height}{}{no value, initially set}
%   % Sets the total height of the colored box to its natural height depending
%   % on the box content.

% 根据盒子的内容将%有色
% 盒子的总高度设置为其自然高度。
% \end{docTcbKey}

% \begin{docTcbKey}{height}{=\meta{length}}{no default}
% % Sets the total height of the colored box to \meta{length} independent
% % of the box content. \meta{length} is the minimum height of the box, if
% % \refKey{/tcb/height plus} is larger than zero.

% 将%有色
% 盒子的总高度设置为与内容无关的长度 \meta{length} 。 \meta{length} 是盒子的最小高度, 如果
% \refKey{/tcb/height plus} 大于零。
% \begin{exdispExample}{height}
% \tcbset{width=(\linewidth-2mm)/3,before=,after=\hfill,
% colframe=blue!75!black,colback=white}

% \begin{tcolorbox}[height=1cm,valign=center]
%   This box has a height of 1cm.
% \end{tcolorbox}
% \begin{tcolorbox}[height=2cm,valign=center]
%   This box has a height of 2cm.
% \end{tcolorbox}
% \begin{tcolorbox}[height=3cm,split=0.5,valign=center,valign lower=center]
%   This box has a height of 3cm.
%   \tcblower
%   Lower part.
% \end{tcolorbox}
% \end{exdispExample}

% % todo 设置  before=,after=\hfill 就可以一行放多个了
% \begin{exdispExample}{height2}
%   \tcbset{width=(\linewidth-2mm)/3,%before=,after=\hfill,
%   colframe=blue!75!black,colback=white}
  
%   \begin{tcolorbox}[height=1cm,valign=center]
%     This box has a height of 1cm.
%   \end{tcolorbox}
%   \begin{tcolorbox}[height=2cm,valign=center]
%     This box has a height of 2cm.
%   \end{tcolorbox}
%   \begin{tcolorbox}[height=3cm,split=0.5,valign=center,valign lower=center]
%     This box has a height of 3cm.
%     \tcblower
%     Lower part.
%   \end{tcolorbox}
%   \end{exdispExample}
% \end{docTcbKey}



% % \enlargethispage*{10mm}
% \begin{docTcbKey}{height plus}{=\meta{length}}{no default, initially |0pt|}
%   % The box may extend a given fixed \refKey{/tcb/height} up to the given \meta{length}.

% 盒子将高度最大可扩展到 \refKey{/tcb/height} 加上给定的长度\footnote{译注:类似于定义的弹性长度。} \meta{length}。
% \begin{exdispExample}{height_plus}
% \tcbset{colback=red!5!white,colframe=red!75!black,left=1mm,top=1mm,bottom=1mm,
%   right=1mm,boxsep=0mm,width=3cm,nobeforeafter}

% \begin{tcolorbox}[height=1cm]
% This is a tcolorbox.
% \end{tcolorbox}
% \begin{tcolorbox}[height=1cm,height plus=1cm]
% This is a tcolorbox.
% \end{tcolorbox}
% \begin{tcolorbox}[height=1cm,height plus=1cm]
% This is a tcolorbox. This is a tcolorbox. This is a tcolorbox.
% \end{tcolorbox}
% \end{exdispExample}
% \end{docTcbKey}


% \begin{docTcbKey}{height from}{=\meta{min} \texttt{to} \meta{max}}{style, no default}
%   % Sets the box height to a dimension between \meta{min} and \meta{max}.

% 将盒子高度范围设置为从\meta{min}到\meta{max}。
% \begin{exdispExample}{height_from}
% % \usepackage{lipsum}
% \newtcolorbox{mybox}{colback=red!5!white,colframe=red!75!black,left=1mm,top=1mm,
%   bottom=1mm,right=1mm,boxsep=0mm,width=4.5cm,nobeforeafter,
%   height from=2cm to 8cm}

% \begin{mybox}
% This is a tcolorbox.
% \end{mybox}
% \begin{mybox}
% This is a tcolorbox. This is a tcolorbox. This is a tcolorbox.
% \end{mybox}
% \begin{mybox}
% \lipsum[2]
% \end{mybox}
% \end{exdispExample}
% \end{docTcbKey}



% \begin{docTcbKey}[][doc new=2014-10-31]{text height}{=\meta{length}}{style, no default}
%   % Sets the text height to \meta{length}. This is the length from the top 
%   % of the upper part to the bottom of the optional lower part.
%   % See also \refKey{/tcb/text width}.

% 将文本高度设置为 \meta{length}。这是从upper部分的顶部到可选的lower部分的底部的长度。另见 \refKey{/tcb/text width}.


% \begin{exdispExample}{text_height}
% \tcbset{colback=red!5!white,colframe=red!75!black}

% \begin{tcolorbox}[text height=2cm]
% This is a \textbf{tcolorbox} where the text area has a height of 2cm.
% \end{tcolorbox}
% \end{exdispExample}
% \end{docTcbKey}




% % \clearpage

% \begin{docTcbKey}[][doc new=2014-11-07]{add to height}{=\meta{length}}{style, no default}
%   % Adds \meta{length} to the current height of the colored box.
%   % \refKey{/tcb/height} has to be set before this key is used!
%   % If this option is used several times, then the \refKey{/tcb/height} is
%   % also increased several times.

% 为盒子的当前高度添加\meta{length}。%
% 在此命令之前设定的 \refKey{/tcb/height} 将被用上!%
% 如果此项设置了多次, 那么 \refKey{/tcb/height} 也会被增加{\bf 多次}。
% \begin{exdispExample}{add_to_height}
% \tcbset{height=2cm,
%   valign=center,width=(\linewidth-2mm)/2,
%   before=,after=\hfill,colframe=blue!75!black,colback=white}

% \begin{tcolorbox}
%   This box has a height of 2cm.
% \end{tcolorbox}
% \begin{tcolorbox}[add to height=1cm]
%   This box has a height of 3cm.
% \end{tcolorbox}
% \end{exdispExample}
% \end{docTcbKey}


% \begin{docTcbKey}[][doc new=2016-02-16]{add to natural height}{=\meta{length}}{style, no default}
%   % The application of this option generates a box with natural height plus
%   % the given \meta{length}. If this option is used several times, then the
%   % last setting of \meta{length} wins. The resulting box is not considered
%   % a fixed height box and the implementation is quite different to
%   % \refKey{/tcb/add to height}.

% 应用本项设置,将会生成一个盒子,高度为自然高度加上给定的 \meta{length}。
% 如果多次设置,{\bf 最后一次}设置的\meta{length}生效。 生成的盒子不是固定的高度且其实现也同 \refKey{/tcb/add to height} 很不同。
% \begin{exdispExample}{add_to_natural_height}
% \tcbset{valign=center,width=(\linewidth-2mm)/2,
%   before=,after=\hfill,colframe=blue!75!black,colback=white}

% \begin{tcolorbox}
%   This box has natural height.
% \end{tcolorbox}
% \begin{tcolorbox}[add to natural height=1cm]
%   This box has natural height plus 1 cm.
% \end{tcolorbox}
% \end{exdispExample}
% \end{docTcbKey}



% % \clearpage
% \begin{docTcbKey}[][doc new and updated={2014-09-22}{2016-02-17}]{height fill}{\colOpt{=true\textbar false\textbar maximum}}{default |true|, initially |false|}
%   % If set to \docValue*{true}, the height of the |tcolorbox| is set to the rest of the
%   % available vertical space of the current page.
%   % If set to \docValue{maximum}, the page is compressed as much as possible.
%   % Note that the |tcolorbox|
%   % is always set as its own paragraph using this option.
%   % Also see \refKey{/tcb/text fill}.
  
%   如果设置为 \docValue*{true}, |tcolorbox| 的高度将会设置为当前页的剩余空间的高度。%
%   如果设置为 \docValue{maximum}, 页面被尽可能多地压缩。%
%   注意, |tcolorbox| 中的段落始终设置上了这个选项。%
%   另见 \refKey{/tcb/text fill}.%
%   \begin{marker}
%   % Note that the library \mylib{breakable} has to be loaded to use this key!
  
%   注意,如果使用此设置,需要加载 \mylib{breakable} 库!
%   \end{marker}
%   % This height control key is only applicable to unbreakable boxes, but it
%   % uses code from the library \mylib{breakable}.
%   % The counterpart for breakable boxes is \refKey{/tcb/height fixed for}.
  
%   此项高度控制只作用在不可分页的盒子上, 但它复用了 \mylib{breakable} 库中的代码。
%   The counterpart for breakable boxes is \refKey{/tcb/height fixed for}.
  
%   % This option can and should not be used for boxes in boxes, but it can be
%   % used for boxes inside a \refEnv{tcbraster}.
  
%   此项不能也不应用在盒子中的盒子, 但它可以用在 \refEnv{tcbraster} 中的盒子。
  
%   \begin{dispListing}
%   % \usepackage{lipsum}
%   % \tcbuselibrary{breakable}
%   \begin{tcolorbox}[height fill,
%     colback=red!5!white,colframe=red!75!black,fonttitle=\bfseries,
%     title=填充页面剩余部分的盒子]
%   \lipsum[1]
%   \end{tcolorbox}
%   \end{dispListing}
%   \end{docTcbKey}
%   {\tcbusetemp}
  



% % \clearpage
% \begin{docTcbKey}[][doc new={2017-06-28}]{inherit height}{\colOpt{=\meta{fraction}}}{default |1|, initially unset}
%   % If this option is used for a |tcolorbox| which is embedded inside
%   % another (outer) |tcolorbox| \emph{and} if this outer |tcolorbox| has
%   % a fixed height, then the given \meta{fraction} of the available text height
%   % of the outer |tcolorbox| is used as \refKey{/tcb/height} for the current
%   % |tcolorbox|.
%   % Otherwise, \refKey{/tcb/natural height} is applied for the current
%   % |tcolorbox|.

% 如果此项所设置的 |tcolorbox| 是个嵌入在另一个%(外部)
%  |tcolorbox| 中,\emph{且}外部分的这个 |tcolorbox| 有着固定的高度, 那么,设定的 \meta{fraction} 乘以外部的这个 |tcolorbox| 盒子的剩余高度用作当前盒子的高度 \refKey{/tcb/height}。否则,当前 |tcolorbox| 盒子将应用自然高度 \refKey{/tcb/natural height} 。

% \begin{exdispExample}{inherit_height}
% \tcbset{colframe=blue!75!black,colback=white,fonttitle=\bfseries}

% \begin{tcolorbox}[title=外部盒子指定高度为4cm,height=4cm]
%   \begin{tcolorbox}[title=Inner box,nobeforeafter,inherit height]
%     这个内部盒子的高度使用外盒的剩余高度空间。
%   \end{tcolorbox}
% \end{tcolorbox}

% \begin{tcolorbox}[title=外部盒子使用自然高度]
%   \begin{tcolorbox}[title=Inner box,nobeforeafter,inherit height]
%     这个盒子使用自身的自然高度。
%   \end{tcolorbox}
% \end{tcolorbox}

% % Deeply nested box using 60 percent of the available space.
% % Deeply nested box using 40 percent of the available space.
% \begin{tcolorbox}[title=外部盒子指定高度为5cm,height=5cm]
%   \begin{tcolorbox}[title=内部盒子,nobeforeafter,inherit height]
%     \begin{tcolorbox}[colframe=red,beforeafter skip=0pt,inherit height=0.6]
%       使用60\%可用空间的嵌套盒子。
%     \end{tcolorbox}
%     \begin{tcolorbox}[colframe=red,beforeafter skip=0pt,inherit height=0.4]
%       使用40\%可用空间的嵌套盒子。
%     \end{tcolorbox}
%   \end{tcolorbox}
% \end{tcolorbox}
% \end{exdispExample}
% \end{docTcbKey}
 




% % \clearpage

% \begin{docTcbKey}[][doc new=2015-05-05]{square}{}{style, no value}
%   % Sets \refKey{/tcb/height} to match the width of the colored box.

% 设置盒子的高度(\refKey{/tcb/height})为盒子的宽度。
% \begin{exdispExample*}{square}{sbs,lefthand ratio=0.6}
% \begin{tcolorbox}[width=3cm,
%   colback=red!5!white,
%   colframe=red!75!black,
%   halign=center,valign=center,
%   square]
% This is a \textbf{tcolorbox}.
% \end{tcolorbox}
% \end{exdispExample*}
% \end{docTcbKey}



% \begin{docTcbKey}{space}{=\meta{fraction}}{no default, initially 0}
%   % If the height of a |tcolorbox| is not the natural height, the space
%   % difference between the forced and the natural size is distributed
%   % between the upper and the lower part of the box. This space could also
%   % be negative.
%   % \meta{fraction} with a value between 0 and 1 is the amount of space
%   % which is added to the upper part, the rest is added to the lower part.
%   % If there is no lower part, then all of the space is added to
%   % the upper part always.

% 如果 |tcolorbox| 的高度不是自然高度, 指定的高度和自然尺寸的的高度差分布在盒子的上下两部分中。 高度差可以是负值。
% \meta{fraction} 是0到1之前的数值,此值指定的比例添加到upper部分,剩余的高度添加到下部分。
% 如果不存在lower部分, 那么所有的空间将添加到upper部分。
% \begin{exdispExample}{fraction}
% \tcbset{width=(\linewidth-2mm)/4,before=,after=\hfill,
% colframe=blue!75!black,colback=white,height=3cm}

% \begin{tcolorbox}
% upper部分
% \tcblower
% lower部分
% \end{tcolorbox}
% \foreach \f in {0.2,0.4,0.7}
% {\begin{tcolorbox}[space=\f]
% upper部分
% \tcblower
% lower部分
% \end{tcolorbox}}
% \end{exdispExample}
% \end{docTcbKey}

% \begin{docTcbKey}{space to upper}{}{style}
%   % This is an abbreviation for |space=1|, i.\,e. all extra space is added
%   % to the upper part.

% 这是指定|space=1|的简写形式, i.\,e. 所有额外的空间都被添加到上部。(此法可读性更好)。
% \end{docTcbKey}

% \begin{docTcbKey}{space to lower}{}{style, initially set}
%   % This is an abbreviation for |space=0|, i.\,e. all extra space is added
%   % to the lower part (if there is any).

% 这是指定|space=0|的简写形式, i.\,e. 所有额外的空间都被添加到下部。(此法可读性更好)。
% \end{docTcbKey}




% % \clearpage
% \begin{docTcbKey}{space to both}{}{style}
%   % This is an abbreviation for |space=0.5|, i.\,e. the extra space
%   % equally distributed between the upper and the lower part.
% 这是指定|space=0.5|的简写形式, i.\,e. 额外的空间将平均分布到upper部分和lower部分。
% \begin{exdispExample}{space_to_both}
% \tcbset{width=(\linewidth-2mm)/3,before=,after=\hfill,
% colframe=blue!75!black,colback=white,height=3cm}

% \foreach \myspace in {space to upper,space to both,space to lower}
% {\begin{tcolorbox}[\myspace]
%     This is the upper part.
%     \tcblower
%     This is the lower part.
% \end{tcolorbox}}
% \end{exdispExample}
% \end{docTcbKey}



% \begin{docTcbKey}[][doc new and updated={2015-02-15}{2020-07-30}]{space to}{=\meta{macro}}{no default, initially unset}
%   % If the height of a |tcolorbox| is not the natural height, the space
%   % difference between the forced and the natural size is saved into the
%   % given local \meta{macro}. This \meta{macro} can and should be used inside
%   % the box content to add content which is vertically sized to match \meta{macro}.

% 如果|tcolorbox|盒子的高度不是自然高度, 指定的高度同自然高度差的数值保存到给出的宏命令 \meta{macro}。 这个 \meta{macro} 可以在盒子中使用以用来控制内容的高度恰好同这高度差一致。
%   \begin{marker}
%     \begin{itemize}
%     \item 
%     % The actual length saved into \meta{macro} is adapted dynamically
%     %   during several compilations -- at least two, but maybe more.
% 实际保存到 \meta{macro} 的值在多次编译期间是自适应 --- 至少2次, 可能更多次。
%     \item %
%     % Due to the adaption algorithm, objects can be sized with
%     %   \meta{macro} plus any offset length.
% 根据自适应算法, 对象尺寸可能在 \meta{macro} 之上添加额外的偏移量。
%     \item 
%     % Never ever use \meta{macro} multiplied with a factor. The only
%     %   exception to this rule is that the space can be split into parts which
%     %   sum to \meta{macro}.
% 永远不要使用 \meta{macro} 乘以一个因子。这个规则的唯一例外是,
% 分开的几个部分的高度和为\meta{macro}(即多个因子的和为1)。
%     \item %Never use this in combination with \refKey{/tcb/fit}.
% 不要同 \refKey{/tcb/fit} 组合使用。
%     \end{itemize}
%   \end{marker}
% \begin{exdispExample}[runs=3]{space_to_1}
% \begin{tcolorbox}[colframe=blue!75!black,colback=white,height=3cm,
%     space to=\myspace]
%     这是我的盒子高3cm。指定高度和自然高度差填充了图片    :\\[2mm]
%   \includegraphics[width=\linewidth,height=\myspace]{goldshade.png}\\[1mm]
%   这是其他一些文字。译注:图片的高度使用我们指定的 |\myspace|。
% \end{tcolorbox}
% \end{exdispExample}

% \begin{exdispExample}[runs=3]{space_to_2}
% \begin{tcolorbox}[colframe=blue!75!black,colback=white,height=3cm,
%     space to=\myspace]
%   \includegraphics[width=\linewidth,
%     height=0.33\dimexpr\myspace]{blueshade.png}\\[1mm]
%     这是我的盒子高3cm。\\[2mm]
%   \includegraphics[width=\linewidth,
%     height=0.67\dimexpr\myspace]{goldshade.png}
% \end{tcolorbox}
% \end{exdispExample}
% \end{docTcbKey}



% \begin{docTcbKey}{split}{=\meta{fraction}}{no default}
%   % If the height of a |tcolorbox| is not the natural height, the
%   % \meta{fraction} with a value between 0 and 1 determines the positioning
%   % of the segmentation between the upper and the lower part. Here, 0 stands
%   % for top and 1 for bottom. Note that the box is split regardless of
%   % the actual dimensions of the text parts!

% 如果 |tcolorbox| 的高度不是自然高度, 取值0到1的
%   \meta{fraction} 决定了上下两部分的分割位置。在这里,0代表顶部,1代表底部。
% 注意,不论文本部分的实际尺寸如何,盒子都会被分割!
% \begin{exdispExample}{split}
% \tcbset{width=(\linewidth-2mm)/3,before=,after=\hfill,height=3cm,
% colback=white,colframe=blue!75!black,valign=center,valign lower=center}

% \foreach \f in {0,0.1,0.5,0.8,0.9,1}
% {\begin{tcolorbox}[split=\f]
% 上,split: \f
% \tcblower
% This is the lower part with a lot of text in several lines.
% \end{tcolorbox}}
% \end{exdispExample}
% \begin{exdispExample}{split2}
%   \tcbset{width=(\linewidth-2mm)/3,before=,after=\hfill,height=1.8cm,
%   colback=white,colframe=blue!75!black,valign=center,valign lower=center}
  
%   \foreach \f in {0,0.1,0.5,0.8,0.9,1}
%   {\begin{tcolorbox}[split=\f]
%   上,split: \f
%   \tcblower
%   This is the lower part with a lot of text in several lines.
%   \end{tcolorbox}}
%   \end{exdispExample}
% \end{docTcbKey}


% % \clearpage
% \begin{docTcbKey}[][doc updated=2014-11-07]{equal height group}{=\meta{id}}{no default}
%   % Boxes which are members of an |equal height group| will all get the
%   % same height, i.\,e. the maximum of all their natural heights. The
%   % \meta{id} serves to distinguish between different height groups.
%   %This \meta{id} should contain only characters which are feasible
%   %for \TeX\ macro names, typically alphabetic characters but no numerals
%   %and spaces.
%   % Note that you have to compile twice to see changes and
%   % that height groups are global definitions.

% 同一|equal height group|的成员将拥有相同的高度, i.\,e. 它们自然高度的最大者的值。
% \meta{id} 用来区分不同的身高组别。
% %This \meta{id} should contain only characters which are feasible
% %for \TeX\ macro names, typically alphabetic characters but no numerals
% %and spaces.
% 注意,您必须编译两次才能看到更改,并且高度组是全局定义。

% \begin{exdispExample}[runs=2]{equal_height_group}
% \tcbset{width=(\linewidth-2mm)/3,before=,after=\hfill,arc=0mm,
% colframe=blue!75!black,colback=white,fonttitle=\bfseries}

% \begin{tcolorbox}[equal height group=A,adjusted title={一}]
% 这组最小的盒子
% \end{tcolorbox}%
% \begin{tcolorbox}[equal height group=A,adjusted title={二}]
% 这个盒子也小
%   \tcblower
% 但有lower部分。
% \end{tcolorbox}%
% \begin{tcolorbox}[equal height group=A,adjusted title={三}]
%   This box contains a lot of text just to fill the space
%   with word flowing and flowing and flowing until the box
%   is filled with all of it.
% \end{tcolorbox}\linebreak
% %
% \tcbset{width=(\linewidth-1mm)/2,before=,after=\hfill,arc=0mm,
% colframe=red!75!black,colback=white}
% %
% \begin{tcolorbox}[equal height group=B]
% 接着,我们使用另一个等高盒子组。
% \end{tcolorbox}%
% \begin{tcolorbox}[equal height group=B,after=]
%   \begin{equation*}
%     \int\limits_{0}^{1} x^2 = \frac13.
%   \end{equation*}
% \end{tcolorbox}
% \end{exdispExample}
% \end{docTcbKey}

% \medskip
% \begin{marker}
% % See \Vref{sec:raster} for more equal height options.
% 另见 \Vref{sec:raster} 了解更多等高组相关选项。
% \end{marker}



% % \clearpage
% \begin{docTcbKey}{minimum for equal height group}{=\meta{id}:\meta{length}}{no default, initially unset}
%   % Plants a \meta{length} into the equal height group with
%   % the given \meta{id}. This ensures that the height will not drop below
%   % \meta{length}. 
%   % Note that you cannot reduce a computed height value by using this key with a small value.
%   % The difference to applying \refKey{/tcb/height} directly is that the boxes
%   % are never too small for their content.

% % 指定值 \meta{length} 到等高组 \meta{id}。这确保高度不会小于 \meta{length}。
% 指定等高组 \meta{id} 的最小高度不小于 \meta{length}。%
% % 注意,不能通过使用小值来减少计算出的高度值。%
% 同使用 \refKey{/tcb/height} 相比,此项设置不会使盒子小于它们的内容高度。
 
% \begin{dispExample}
% \tcbset{colframe=blue!75!black,colback=white,arc=0mm,
%   before=,after=\hfill,fonttitle=\bfseries,left=2mm,right=2mm,
%   width=3.5cm,
%   equal height group=C,
%   minimum for equal height group=C:3.5cm}

% \begin{tcolorbox}
%   My first box. All boxes will get 3.5cm times 3.5cm
%   if the content height is not too large.
% \end{tcolorbox}%
% \begin{tcolorbox}
%   My second box.
%   \tcblower
%   This is the lower part.
% \end{tcolorbox}%
% \begin{tcblisting}{}
% \textbf{Mixed}
% with a listing.
% \end{tcblisting}
% \begin{tcolorbox}[title={Fourth box}]
%   My final box.
% \end{tcolorbox}%
% \end{dispExample}
% \end{docTcbKey}

% %todo 再看看
% \begin{docTcbKey}[][doc new=2016-03-24]{minimum for current equal height group}{=\meta{length}}{no default, initially unset}
%   % Sets \refKey{/tcb/minimum for equal height group} for the current equal height
%   % group. Apparently, this only works for an already known equal height group, i.e.
%   % \refKey{/tcb/equal height group} has to be set \emph{before} this option is used.
%   % This option is likely to be used in combination with \refKey{/tcb/raster equal height}

% 为当前等高组设置 \refKey{/tcb/minimum for equal height group}。
% 显然, 这只适用于已知的等高组, i.e.
% \refKey{/tcb/equal height group}已经在此设置\emph{之前}设置。
% 此项学与 \refKey{/tcb/raster equal height} 组合使用。
% \begin{exdispExample}[runs=2]{minimum_for_current_equal_height_group}
% % \tcbuselibrary{raster}
% \begin{tcbitemize}[raster equal height,colframe=blue!75!black,colback=white,
%   raster every box/.style={minimum for current equal height group=2cm}]
%   \tcbitem A
%   \tcbitem B
% \end{tcbitemize}
% \end{exdispExample}

% \end{docTcbKey}




% % \clearpage
% \begin{docTcbKey}[][doc new and updated={2015-11-27}{2016-02-22}]{use height from group}{\colOpt{=\meta{id}}}{style, default current group}
%   % Sets the current box to a fixed \refKey{/tcb/height} which is copied from
%   % an equal height group with the given \meta{id}. If this height is not
%   % available during the current compilation, no fixed height setting is used.
%   % If \meta{id} is omitted, the current equal height group is used which has
%   % to be set before by \refKey{/tcb/equal height group}.\par
%   % Note that the natural height of the current box is not considered for
%   % computation of the group height. The main application for
%   % \refKey{/tcb/use height from group} is that the height can be adapted
%   % further by \refKey{/tcb/add to height}.

% 设置当前盒子的高度为一个固定 \refKey{/tcb/height} 值,值来自一个等高组\meta{id}。如果在当前编译期间此高度不可用,则不使用固定高度设置。
% 如果省略了\meta{id}, 则使用此前的\refKey{/tcb/equal height group}设置的。\par
% 请注意,在计算组高度时不考虑当前盒子的自然高度。
% \refKey{/tcb/use height from group}主要用在当高度可以通过 \refKey{/tcb/add to height} 进一步调整。

% \begin{dispExample}
% \begin{tcolorbox}[use height from group=C,add to height=-2cm,
%   colframe=blue!75!black,colback=white]
% Height from group \enquote{C} of the previous example, but reduced by 2cm.
% \end{tcolorbox}%
% \end{dispExample}

% \begin{exdispExample}[runs=2]{use_height_from_group}
% % \tcbuselibrary{raster}
% Every line is inside an equal height group:
% \begin{tcbraster}[raster equal height=rows,
%     title=Box \thetcbrasternum,
%     enhanced,size=small,colframe=red!50!black,colback=red!10!white]
%   \begin{tcolorbox}First line\\second line\\
%     The height of this box rules.\end{tcolorbox}
%   \begin{tcolorbox}[use height from group]Test\end{tcolorbox}
%   \begin{tcolorbox}[use height from group]
%     First line\\second line\end{tcolorbox}
%   \begin{tcolorbox}The height of this box rules.\end{tcolorbox}
% \end{tcbraster}
% \end{exdispExample}
% \end{docTcbKey}



% \begin{docCommand}[doc new=2015-11-27]{tcbheightfromgroup}{\marg{macro}\marg{id}}
%   % Saves the height from an equal height group with the given \meta{id}
%   % to a \meta{macro}. If this height is not available during the current compilation,
%   % \meta{macro} is set to |0pt|.

% 保存等高组 \meta{id} 的高度到 \meta{macro}。如果在当前编译期间此高度不可用,
%  \meta{macro} 设为|0pt|.
% \end{docCommand}






% % \clearpage
% % Box Content Additions\hfill 
% \subsection{盒子内容添加}\label{subsec:contentadditions}
% % The following options introduce some arbitrary \meta{code} to the content
% % of a |tcolorbox|. These additions can be given at the beginning or at the ending
% % of the title, the upper part, or the lower part.

% 下面的选项介绍将一引动任意的 \meta{code} 附加到 |tcolorbox| 的内容中。可以附加在标题、upper部分或lower部分的开头或结尾。

% \begin{docTcbKey}{before title}{=\meta{code}}{no default, initially unset}
%   % The given \meta{code} is placed \emph{after} the color and font settings
%   % and \emph{before} the content of the title.

% 给出的\meta{code}被放置在标题的颜色和字体设置\emph{之后} 、内容\emph{之前}。
% \begin{exdispExample}{before_title}
% \tcbset{before title={\textcolor{yellow}{\large Important:}~},
%   colback=red!5!white,colframe=red!75!black,fonttitle=\bfseries}

% \begin{tcolorbox}[title=My title]
% This is a \textbf{tcolorbox}.
% \end{tcolorbox}
% \end{exdispExample}
% \end{docTcbKey}


% \begin{docTcbKey}{after title}{=\meta{code}}{no default, initially unset}
%   % The given \meta{code} is placed \emph{after} the content of the title.

% 给出的\meta{code}被放置在标题的内容\emph{之后}。
% \begin{exdispExample}{after_title}
% \tcbset{after title={\hfill\colorbox{Navy}{approved}},
%   colback=red!5!white,colframe=red!75!black,fonttitle=\bfseries}

% \begin{tcolorbox}[title=My title]
% This is a \textbf{tcolorbox}.
% \end{tcolorbox}
% \end{exdispExample}
% \end{docTcbKey}




% % \clearpage
% \begin{docTcbKey}{before upper}{=\meta{code}}{no default, initially empty}
%   % The given \meta{code} is placed \emph{after} the color and font settings
%   % and \emph{before} the content of the upper part.
%   % The \meta{code} is appended by a final |\ignorespaces|.

% 给出的\meta{code}被放置在upper部分的颜色和字体设置\emph{之后}、内容\emph{之前}。
% \meta{code}会被附加一个最后的|\ignorespaces|。

% \begin{exdispExample}{before_upper}
% \tcbset{before upper={\textit{The story:}\par},
%   colback=red!5!white,colframe=red!75!black,fonttitle=\bfseries}

% \begin{tcolorbox}[title=My title]
% This is a \textbf{tcolorbox}.
% \end{tcolorbox}
% \end{exdispExample}
% \end{docTcbKey}


% \begin{docTcbKey}[][doc new=2019-02-26]{before upper*}{=\meta{code}}{no default, initially unset}
%   % The given \meta{code} is placed \emph{after} the color and font settings
%   % and \emph{before} the content of the upper part.
%   % In contrast to \refKey{/tcb/before upper}, no |\ignorespaces| is appended.
%   % Use this for situations where |\ignorespaces| is not needed or causes harm.

% 给出的\meta{code}被放置在upper部分的颜色和字体设置\emph{之后}、内容\emph{之前}。
% 同\refKey{/tcb/before upper}对比,不附加|\ignorespaces|。
% 当不需要|\ignorespaces|或|\ignorespaces|会导致问题时使用此项。

% \begin{exdispExample}{before_upper_star}
% \begin{tcolorbox}[size=small,tile,
%   colback=yellow!20,colbacktitle=yellow!70!black,
%   title=My table,hbox,center,center title,
%   before upper*=\begin{tabular}{cc},
%   after upper*=\end{tabular},
% ]
%   \multicolumn{2}{c}{Title}\\
%   one & two \\
%   three & four\\
% \end{tcolorbox}
% \end{exdispExample}
% \end{docTcbKey}

% % \clearpage





% \begin{docTcbKey}[][doc updated=2016-10-21]{after upper}{=\meta{code}}{no default, initially empty}
%   % The given \meta{code} is placed \emph{after} the content of the upper part.
%   % The \meta{code} is prepended by a leading |\unskip|.

% 给出的\meta{code}会被放置到upper部分的内容\emph{之后}。
% \meta{code}之前会插入|\unskip|。

% \begin{exdispExample}{after_upper_1}
% \tcbset{after upper={\par\hfill\textit{Read more next week}},
%   colback=red!5!white,colframe=red!75!black,fonttitle=\bfseries}

% \begin{tcolorbox}[title=My title]
% This is a \textbf{tcolorbox}.
% \end{tcolorbox}

% \begin{tcolorbox}[after upper={\par\hfill---\textit{王勃}}]
% 穷且益坚,不坠青云之志。
% \end{tcolorbox}
% \end{exdispExample}

% \begin{exdispExample}{after_upper_2}
% \begin{tcolorbox}[before upper=\flqq,after upper=\frqq,
%   colback=red!5!white,colframe=red!75!black]
% This is a \textbf{tcolorbox}.\footnote{译注:想到可以用命令行盒子,加这个书名号的效果!}
% \end{tcolorbox}
% \end{exdispExample}
% \end{docTcbKey}




% \begin{docTcbKey}[][doc new and updated={2016-10-21}{2019-02-28}]{after upper*}{=\meta{code}}{no default, initially unset}
%   % The given \meta{code} is placed \emph{after} the content of the upper part.
%   % In contrast to \refKey{/tcb/after upper}, no |\unskip| is prepended.
%   % Use this for situations where |\unskip| is not needed or causes harm.
%   % See \refKey{/tcb/before upper*} for an example.

% 给出的\meta{code}会被放置到upper部分的内容\emph{之后}。%
% 同\refKey{/tcb/after upper}相比,没有在前附加|\unskip|。%
% 当不需要 |\unskip| 或 |\unskip| 会导致问题时使用此项。%
% 例见 \refKey{/tcb/before upper*}。

% \begin{marker}
% 从版本 3.80 到 3.94, 此项会将 |\unskip| 添加到 \meta{code} 之前。\\
% 版本 3.95 到 4.15, 此项不建议使用。\\
% 版本 4.20 起, 这个选项是用改变的语义重新建立的 (没有 |\unskip|!)
% \end{marker}
% \end{docTcbKey}





% % \clearpage
% \begin{docTcbKey}{before lower}{=\meta{code}}{no default, initially empty}
%   % The given \meta{code} is placed \emph{after} the color and font settings
%   % and \emph{before} the content of the lower part.
%   % The \meta{code} is appended by a final |\ignorespaces|.

% 将 \meta{code} 放到lower部分的颜色和字段设置\emph{之后}、内容\emph{之前}。
% \meta{code} 之后附加一个|\ignorespaces|。
% \begin{exdispExample}{before_lower}
% \tcbset{before lower=\textit{Behold:~},colback=red!5!white,colframe=red!75!black}

% \begin{tcolorbox}
% This is a \textbf{tcolorbox}.
% \tcblower
% This is the lower part.
% \end{tcolorbox}
% \end{exdispExample}
% \end{docTcbKey}


% \begin{docTcbKey}[][doc new=2019-02-26]{before lower*}{=\meta{code}}{no default, initially unset}
%   % The given \meta{code} is placed \emph{after} the color and font settings
%   % and \emph{before} the content of the lower part.
%   % In contrast to \refKey{/tcb/before lower}, no |\ignorespaces| is appended.
%   % Use this for situations where |\ignorespaces| is not needed or causes harm.

% \meta{code} 放置到lower部分的颜色和字体设置\emph{之后}、内容\emph{之前}。
% 同 \refKey{/tcb/before lower} 相比, 尾部不会附加 |\ignorespaces|。
% 当不需要 |\ignorespaces| 或它会导致问题时使用此项。
% \begin{exdispExample}{before_lower_star}
% \begin{tcolorbox}[size=small,bicolor,sidebyside,center lower,
%   colback=yellow!30,colbacklower=yellow!20,colframe=yellow!80!black,
%   before lower*=\begin{tabular}{cc},
%   after lower*=\end{tabular},
% ]
% My table
% \tcblower
%   \multicolumn{2}{c}{Title}\\
%   one & two \\
%   three & four\\
% \end{tcolorbox}
% \end{exdispExample}
% \end{docTcbKey}





% % \clearpage

% \begin{docTcbKey}[][doc updated=2016-10-21]{after lower}{=\meta{code}}{no default, initially empty}
%   % The given \meta{code} is placed \emph{after} the content of the lower part.
%   % The \meta{code} is prepended by a leading |\unskip|.

% \meta{code} 放置到lower部分的内容\emph{之后}。
% \meta{code} 之前会插入 |\unskip|。

% \begin{exdispExample}{after_lower_2}
% \begin{tcolorbox}[after lower=\ \textit{This is the end.},
%   colback=red!5!white,colframe=red!75!black]
% This is a \textbf{tcolorbox}.
% \tcblower
% This is the lower part.
% \end{tcolorbox}
% \end{exdispExample}
% \end{docTcbKey}


% \begin{docTcbKey}[][doc new and updated={2016-10-21}{2019-02-28}]{after lower*}{=\meta{code}}{no default, initially unset}
%   % The given \meta{code} is placed \emph{after} the content of the lower part.
%   % In contrast to \refKey{/tcb/after upper}, no |\unskip| is prepended.
%   % Use this for situations where |\unskip| is not needed or causes harm.

% \meta{code} 放置到lower部分的内容\emph{之后}。
% 同 \refKey{/tcb/after upper} 相比, 不会在头部附加 |\unskip|。
% 当不需要 |\unskip| 或它会导致问题时使用此项。

% \begin{exdispExample}{after_lower_1}
% \begin{tcolorbox}[before lower*=$,after lower*=$,
%   colback=red!5!white,colframe=red!75!black]
% This is a \textbf{tcolorbox}.
% \tcblower
% \sin^2(x)+\cos^2(x)=1.
% \end{tcolorbox}
% \end{exdispExample}

% \begin{marker}
%   From version 3.80 to 3.94, this option prepended an |\unskip| to the given \meta{code}.\\
%   From version 3.95 to 4.15, this option was deprecated.\\
%   From version 4.20, this option is re-established with changed semantic (no |\unskip|!)
% \end{marker}
% \end{docTcbKey}



% % \clearpage
% \begin{marker}
%   % If \refKey{/tcb/text fill} is used, one cannot have a lower part
%   % and the box is unbreakable.
%   如果使用了 \refKey{/tcb/text fill} , 则没有 lower 部分,且盒子是不可分割的。
%   \end{marker}
  
%   \begin{docTcbKey}[][doc new=2015-07-15]{text fill}{}{style, no value}
%     % This style sets \refKey{/tcb/before upper} and \refKey{/tcb/after upper}
%     % to embed the upper part with a minipage. If a fixed height was applied
%     % e.g.\ by \refKey{/tcb/height} or \refKey{/tcb/height fill}, this minipage
%     % gets a matching height. This allows to use vertical glue macros like
%     % |\vfill| to act like expected. If the box has no fixed height,
%     % setting \refKey{/tcb/text fill} has no other effect as making the box
%     % unbreakable. 
  
%   此项设置 \refKey{/tcb/before upper} 和 \refKey{/tcb/after upper}
%   以将upper部分包围在 minipage 环境中。 如果高度指定为固定的值
%   e.g.\ 使用 \refKey{/tcb/height} 或 \refKey{/tcb/height fill}, 此 minipage 环境得到一个匹配的高度。 这允许我们使用竖直的粘连% glue 
%   宏,如 |\vfill| 能正常工作。如果盒子没有指定固定的高度,%
%   设置 \refKey{/tcb/text fill} 没有其他作用,但盒子会变成不可分的。
%   \begin{exdispExample}{text_fill}
%   \begin{tcolorbox}[colback=red!5!white,colframe=red!75!black,fonttitle=\bfseries,
%     height=8cm,text fill,
%     title=My filled box]
%   This is a \textbf{tcolorbox}.
%   \par\vfill
%   \begin{center}
%     My middle text.
%   \end{center}
%   \par\vfill
%   This is the end of my box.
%   \end{tcolorbox}
%   \end{exdispExample}
%   \end{docTcbKey}

  

% % \clearpage

% \begin{docTcbKey}[][doc new={2019-09-19}]{tabulars}{=\meta{preamble}}{style}
%   % This style sets \refKey{/tcb/before upper} and \refKey{/tcb/after upper}
%   % and several geometry keys to support a |tabular*| with the
%   % given \meta{preamble}.
%   % The packages |array| and |colortbl| have to be loaded separately.

% 此项设置了 \refKey{/tcb/before upper} 和 \refKey{/tcb/after upper} 和一些命令,使得支持使用 |tabular*| 并在 \meta{preamble} 中指定表格头。
% 需要分别加载宏包 |array| 和 |colortbl|。
% \begin{exdispExample}{tabulars_1}
% % \usepackage{array}
% % \usepackage{colortbl} - or - \usepackage[table]{xcolor}
% \tcbset{enhanced,fonttitle=\bfseries\large,fontupper=\normalsize\sffamily,
%   colback=yellow!10!white,colframe=red!50!black,colbacktitle=Salmon!30!white,
%   coltitle=black,center title}

% \begin{tcolorbox}[tabulars={@{\extracolsep{\fill}\hspace{5mm}}lrrrrr@{\hspace{5mm}}},
%   boxrule=0.5pt,title=My table]
% Group & One     & Two     & Three    & Four     & Sum\\\hline\hline
% Red   & 1000.00 & 2000.00 &  3000.00 &  4000.00 & 10000.00\\\hline
% Green & 2000.00 & 3000.00 &  4000.00 &  5000.00 & 14000.00\\\hline
% Blue  & 3000.00 & 4000.00 &  5000.00 &  6000.00 & 18000.00\\\hline\hline
% Sum   & 6000.00 & 9000.00 & 12000.00 & 15000.00 & 42000.00
% \end{tcolorbox}
% \end{exdispExample}
% \end{docTcbKey}


% \begin{docTcbKey}[][doc new={2019-09-19}]{tabulars*}{=\marg{code}\marg{preamble}}{style}
%   % This is a variant of \refKey{/tcb/tabulars} which adds some \meta{code}
%   % before the table starts.

% 这是 \refKey{/tcb/tabulars} 的变种,可以附加 \meta{code} 到表格的开头。
% % before the table starts.
% \begin{exdispExample}{tabulars_2}
% % \usepackage{array}
% % \usepackage{colortbl} - or - \usepackage[table]{xcolor}
% \tcbset{enhanced,fonttitle=\bfseries\large,fontupper=\normalsize\sffamily,
%   colback=yellow!10!white,colframe=red!50!black,colbacktitle=Salmon!30!white,
%   coltitle=black,center title}

% \begin{tcolorbox}[tabulars*={\arrayrulewidth0.5mm\renewcommand\arraystretch{1.4}}%
%     {@{\extracolsep{\fill}\hspace{20mm}}lll@{\hspace{20mm}}},
%   title=My table]
% One     & Two     & Three \\\hline\hline
% 1000.00 & 2000.00 &  3000.00\\\hline
% 2000.00 & 3000.00 &  4000.00
% \end{tcolorbox}
% \end{exdispExample}
% \end{docTcbKey}




% % \clearpage
% \begin{marker}
% % If \refKey{/tcb/tabularx} or \refKey{/tcb/tabularx*} are used, one cannot
% % have a lower part.

% 如果使用了 \refKey{/tcb/tabularx} 或 \refKey{/tcb/tabularx*} , 将不会有lower部分。
% \end{marker}



% \begin{docTcbKey}{tabularx}{=\meta{preamble}}{style}
%   % This style sets \refKey{/tcb/before upper} and \refKey{/tcb/after upper}
%   % and several geometry keys to support a |tabularx| with the
%   % given \meta{preamble}.
%   % The packages |tabularx| \cite {carlisle:tabularx}, |array|, and |colortbl|
%   % have to be loaded separately.

% 此项设置了 \refKey{/tcb/before upper} 和 \refKey{/tcb/after upper} 以及一些命令以支持 |tabularx| 环境,且可以指定表头为 \meta{preamble}。
% 需要加载宏包 |tabularx| %\cite {carlisle:tabularx}
% , |array|, 和 |colortbl|。
% \begin{exdispExample}{tabularx_1}
% % \usepackage{array,tabularx}
% % \usepackage{colortbl} - or - \usepackage[table]{xcolor}
% \newcolumntype{Y}{>{\raggedleft\arraybackslash}X}% see tabularx
% \tcbset{enhanced,fonttitle=\bfseries\large,fontupper=\normalsize\sffamily,
%   colback=yellow!10!white,colframe=red!50!black,colbacktitle=Salmon!30!white,
%   coltitle=black,center title}

% \begin{tcolorbox}[tabularx={X||Y|Y|Y|Y||Y},title=My table]
% Group & One     & Two     & Three    & Four     & Sum\\\hline\hline
% Red   & 1000.00 & 2000.00 &  3000.00 &  4000.00 & 10000.00\\\hline
% Green & 2000.00 & 3000.00 &  4000.00 &  5000.00 & 14000.00\\\hline
% Blue  & 3000.00 & 4000.00 &  5000.00 &  6000.00 & 18000.00\\\hline\hline
% Sum   & 6000.00 & 9000.00 & 12000.00 & 15000.00 & 42000.00
% \end{tcolorbox}
% \end{exdispExample}
% \end{docTcbKey}
 

% \begin{docTcbKey}{tabularx*}{=\marg{code}\marg{preamble}}{style}
%   % This is a variant of \refKey{/tcb/tabularx} which adds some \meta{code}
%   % before the table starts.

% 这是 \refKey{/tcb/tabularx} 的变种,附加 \meta{code} 到表格的开头。
% \begin{exdispExample}{tabularx_2}
% % \usepackage{array,tabularx}
% % \usepackage{colortbl} - or - \usepackage[table]{xcolor}
% \tcbset{enhanced,fonttitle=\bfseries\large,fontupper=\normalsize\sffamily,
%   colback=yellow!10!white,colframe=red!50!black,colbacktitle=Salmon!30!white,
%   coltitle=black,center title}

% \begin{tcolorbox}[tabularx*={\arrayrulewidth0.5mm}{X|X|X},title=My table]
% One     & Two     & Three \\\hline\hline
% 1000.00 & 2000.00 &  3000.00\\\hline
% 2000.00 & 3000.00 &  4000.00
% \end{tcolorbox}
% \end{exdispExample}
% \end{docTcbKey}

 


% % \clearpage
% \begin{docTcbKey}{tikz upper}{\colOpt{=\meta{options}}}{style}
%   % This style adds a centered |tikzpicture| environment to the start and end
%   % of the upper part. The \meta{options} may be given as \tikzname\  picture options.

% 此项将upper部分的内容放入一个 |tikzpicture| 环境。给定的选项 \meta{options} 会传递给 |tikzpicture| 环境。 % \tikzname\  picture
% \begin{exdispExample}{tikz_upper}
% % \usepackage{tikz}

% \begin{tcolorbox}[tikz upper,fonttitle=\bfseries,colback=white,colframe=black,
%                   title=\tikzname\ 绘制]
%   \path[fill=yellow,draw=yellow!75!red] (0,0) circle (1cm);
%   \fill[red] (45:5mm) circle (1mm);
%   \fill[red] (135:5mm) circle (1mm);
%   \draw[line width=1mm,red] (215:5mm) arc (215:325:5mm);
% \end{tcolorbox}
% \end{exdispExample}
% \end{docTcbKey}


% \begin{docTcbKey}{tikz lower}{\colOpt{=\meta{options}}}{style}
%   % This style adds a centered |tikzpicture| environment to the start and end
%   % of the lower part. The \meta{options} may be given as \tikzname\  picture options.

% 此项将lower部分的内容放入一个 |tikzpicture| 环境。给定的选项 \meta{options} 会传递给 |tikzpicture| 环境。% \tikzname\  picture 。
% \begin{exdispExample}{tikz_lower}
% % \usepackage{tikz}
% % \tcbuselibrary{skins,listings}

% \begin{tcblisting}{tikz lower%lower 部分放入 tikzpicture 环境
%   ,listing side text% LaTeX源码和效果各一边
%   ,fonttitle=\bfseries,
%   bicolor,colback=LightBlue!50!white,colbacklower=white,colframe=black,
%   righthand width=3cm,title=\tikzname\ drawing}
% \path[fill=yellow,draw=yellow!75!red]
%     (0,0) circle (1cm);
% \fill[red] (45:5mm) circle (1mm);
% \fill[red] (135:5mm) circle (1mm);
% \draw[line width=1mm,red]
%     (215:5mm) arc (215:325:5mm);
% \end{tcblisting}
% \end{exdispExample}
% \end{docTcbKey}




% % \clearpage
% \begin{docTcbKey}{tikznode upper}{\colOpt{=\meta{options}}}{style}
%   % This style places the upper part content into a centered
%   % \tikzname\  node. The \meta{options} may be given as \tikzname\  node options.
%   % This style is especially useful for boxes with multiline texts which are
%   % fitted to the text width.

% 此项将upper部分的内容放到一个居中的 \tikzname\  node。选项 \meta{options} 会传递给 \tikzname\  node 。
% 此项常用于包含多行文本的盒子,可以适应文本宽度。
% \begin{exdispExample}{tikznode_upper}
% % \usepackage{tikz}
% \newtcbox{\headline}[1][]{enhanced,center,
%   ignore nobreak,fontupper=\Large\bfseries,
%   colframe=red!50!black,colback=red!10!white,
%   drop fuzzy shadow=yellow,tikznode upper,#1}

% \headline{Important\\Headline}
% \end{exdispExample}
% \end{docTcbKey}

% \begin{docTcbKey}{tikznode lower}{\colOpt{=\meta{options}}}{style}
%   % This style places the lower part content into a centered
%   % \tikzname\ node. The \meta{options} may be given as \tikzname\  node options.

% 此项将lower部分的内容放到一个居中的 \tikzname\  node。选项 \meta{options} 会传递给 \tikzname\  node 。
% \begin{exdispExample}{tikznode_lower}
% % \usepackage{tikz}
% \begin{tcolorbox}[bicolor,colback=LightBlue!50!white,colbacklower=white,
%   colframe=black,tikznode lower={inner sep=2pt,draw=red,fill=yellow}]
% Upper part.
% \tcblower
% Lower part.
% \end{tcolorbox}
% \end{exdispExample}
% \end{docTcbKey}



% \begin{docTcbKey}{tikznode}{\colOpt{=\meta{options}}}{style}
%   % Shortcut for setting \refKey{/tcb/tikznode upper} and \refKey{/tcb/tikznode lower}
%   % the same time.

% 同时设置 \refKey{/tcb/tikznode upper} 和 \refKey{/tcb/tikznode lower} 的简写形式。
% \end{docTcbKey}

% \begin{docTcbKey}{varwidth upper}{\colOpt{=\meta{length}}}{style, default \refKey{/tcb/width}}
%   % This style places the upper part content into a |varwidth| environment.
%   % This style needs the |varwidth| package \cite{arseneau:2011a} to be loaded manually.
%   % The resulting box has a maximal width of \meta{length}.
%   % This option is only senseful for a \refCom{tcbox}.


% 此项将upper部分放到一个 |varwidth| 环境中。需要手动加载 |varwidth| 宏包 %\cite{arseneau:2011a}
% 。产出的盒子的最大宽度是 \meta{length}。此项只对 \refCom{tcbox} 生效。
% \begin{exdispExample*}{varwidth_upper}{sbs,lefthand ratio=0.6}
% % \usepackage{varwidth}
% \newtcbox{\varbox}{colframe=red!50!black,
%   colback=red!10!white,varwidth upper}

% \varbox{Short text.}
% \varbox{This box contains is a longer text
%   which is broken.}
% \end{exdispExample*}
% \end{docTcbKey}





% % \clearpage
% % Overlays\hfill 
% \subsection{覆盖物}\label{subsec:overlays}
% % With an overlay, arbitrary \meta{graphical code} can be added to a
% % |tcolorbox|. This code is executed \emph{after} the frame and interior are
% % drawn and \emph{before} the text content is drawn. Therefore, you can
% % decorate the |tcolorbox| with your own extensions.
% % Common special cases are \emph{watermarks} which are implemented using overlays.
% % See Subsection \ref{subsec:watermarks} from page \pageref{subsec:watermarks} if
% % you want to add \emph{watermarks}.
% % todo interior 是
% 通过overlay, 可以将任意的 \meta{graphical code} 添加到 |tcolorbox| 中。
% 这部分代码附加到边框和内部%interior
% \emph{之后}、文本内容绘制\emph{之前}。 因此,你可以自行扩展装饰 |tcolorbox| 环境。常见的一个特定情况是使用overlays实现\emph{水印}。
% 如果你想添加\emph{水印},见\pageref{subsec:watermarks}页的\ref{subsec:watermarks}。



% \begin{marker}
%   % If you use the core package only, the \meta{graphical code} has to be |pgf| code
%   % and there is not much assistance for positioning.
%   % Therefore, the usage of the \refKey{/tcb/enhanced} mode from the library skins
%   % is recommended which allows |tikz| code and gives access to
%   % \refKey{/tcb/geometry nodes} for positioning.
  
%   如果您仅使用核心包, \meta{graphical code} 必须是 |pgf| 代码,而且也没有太多的定位辅助。因此, 推荐使用 |skins| 库的 \refKey{/tcb/enhanced} 模式,它不仅允许 |tikz| 代码,还允许用 \refKey{/tcb/geometry nodes} 进行定位。
%   \end{marker}
 
   
%   \begin{docTcbKey}{overlay}{=\meta{graphical code}}{no default, initially unset}
%     % Adds \meta{graphical code} to the box drawing process. This \meta{graphical code}
%     % is drawn \emph{after} the frame and interior and \emph{before} the text content.
  
%   将 \meta{graphical code} 添加到盒子的绘制过程中。\meta{graphical code}将附加到边框和内部的绘制\emph{之后}、文本内容\emph{之前}。
  
%   \begin{exdispExample}{overlay_1}
%   % \tcbuselibrary{skins} % preamble
%   \tcbset{frogbox/.style={enhanced,colback=green!10,colframe=green!65!black,
%     enlarge top by=5.5mm,
%     overlay={\foreach \x in {2cm,3.5cm} {
%       \begin{scope}[shift={([xshift=\x]frame.north west)}]
%         \path[draw=green!65!black,fill=green!10,line width=1mm] (0,0) arc (0:180:5mm);
%         \path[fill=black] (-0.2,0) arc (0:180:1mm);
%       \end{scope}}}}}
  
%   \begin{tcolorbox}[frogbox,title=My title]
%   This is a \textbf{tcolorbox}.
%   \end{tcolorbox}
%   \end{exdispExample}
  
%   \enlargethispage*{5mm}
%   \begin{exdispExample}{overlay_2}
%   % \usetikzlibrary{patterns} % preamble
%   % \tcbuselibrary{skins}     % preamble
%   \tcbset{ribbonbox/.style={enhanced,colback=red!5!white,colframe=red!75!black,
%     fonttitle=\bfseries,
%     overlay={\path[fill=blue!75!white,draw=blue,double=white!85!blue,
%       preaction={opacity=0.6,fill=blue!75!white},
%       line width=0.1mm,double distance=0.2mm,
%       pattern=fivepointed stars,pattern color=white!75!blue]
%       ([xshift=-0.2mm,yshift=-1.02cm]frame.north east)
%       -- ++(-1,1) -- ++(-0.5,0) -- ++(1.5,-1.5) -- cycle;}}}
  
%   \begin{tcolorbox}[ribbonbox,title=My title]
%   This is a \textbf{tcolorbox}.
%   \tcblower
%   This is the lower part.
%   \end{tcolorbox}
%   \end{exdispExample}
%   \end{docTcbKey}




% % \clearpage
% \begin{docTcbKey}{no overlay}{}{style, no default, initially set}
%   % Removes the overlay if set before.

% 移除覆盖层。
% \end{docTcbKey}

% \begin{docTcbKey}{overlay broken}{=\meta{graphical code}}{no default, initially unset}
%   % If the box is set to be \refKey{/tcb/breakable} and \emph{is} broken actually,
%   % then the \meta{graphical code} is added to the box drawing process.
%   % \refKey{/tcb/overlay} overwrites this key.

% 如果盒子设置为\refKey{/tcb/breakable}且\emph{实际上}分页了,那么\meta{graphical code}会被添加到盒子的绘制过程中。\refKey{/tcb/overlay}会覆盖此项设置。\footnote{译注:即实际上分页时才添加覆盖层。}
% \end{docTcbKey}

% \begin{docTcbKey}{overlay unbroken}{=\meta{graphical code}}{no default, initially unset}
%   % If the box is set to be \refKey{/tcb/breakable} but \emph{is not} broken actually
%   % or if the box is set to be \refKey{/tcb/unbreakable},
%   % then the \meta{graphical code} is added to the box drawing process.
%   % \refKey{/tcb/overlay} overwrites this key.

% 如果盒子设置了\refKey{/tcb/breakable}但\emph{实际上没}分页,或盒子设置为\refKey{/tcb/unbreakable},那么\meta{graphical code}会被添加到盒子的绘制过程中。\refKey{/tcb/overlay}会覆盖此项设置。\footnote{译注:即实际上没有分页时才添加覆盖层。}
% \end{docTcbKey}



% \begin{docTcbKey}{overlay first}{=\meta{graphical code}}{no default, initially unset}
%   % If the box is set to be \refKey{/tcb/breakable} and \emph{is} broken actually,
%   % then the \meta{graphical code} is added to the box drawing process for
%   % the \emph{first} part of the break sequence.
%   % \refKey{/tcb/overlay} overwrites this key.

% 如果盒子设置了\refKey{/tcb/breakable}且\emph{事实上}分页了,
% 那么\meta{graphical code}会被添加到盒子的分开的\emph{第一}部分的绘制过程中。
% \refKey{/tcb/overlay}会覆盖此项设置。
% \end{docTcbKey}

% \begin{docTcbKey}{overlay middle}{=\meta{graphical code}}{no default, initially unset}
%   % If the box is set to be \refKey{/tcb/breakable} and \emph{is} broken actually,
%   % then the \meta{graphical code} is added to the box drawing process for
%   % the \emph{middle} parts (if any) of the break sequence.
%   % \refKey{/tcb/overlay} overwrites this key.

% 如果盒子设置了\refKey{/tcb/breakable}且\emph{事实上}分页了,
% 那么\meta{graphical code}会被添加到盒子的分开的序列中的\emph{中间}部分(如果有)。 \refKey{/tcb/overlay}会覆盖此项设置。
% \end{docTcbKey}

% \begin{docTcbKey}{overlay last}{=\meta{graphical code}}{no default, initially unset}
%   % If the box is set to be \refKey{/tcb/breakable} and \emph{is} broken actually,
%   % then the \meta{graphical code} is added to the box drawing process for
%   % the \emph{last} part of the break sequence.
%   % \refKey{/tcb/overlay} overwrites this key.

% 如果盒子设置了\refKey{/tcb/breakable}且\emph{事实上}分页了,
% 那么\meta{graphical code}会被添加到盒子的分开的\emph{最后}一部分的绘制过程中。
% \refKey{/tcb/overlay}会覆盖此项设置。
% \end{docTcbKey}




% \begin{docTcbKey}{overlay unbroken and first}{=\meta{graphical code}}{no default, initially unset}
%   % This is an optimized abbreviation for setting
%   % \refKey{/tcb/overlay unbroken} and
%   % \refKey{/tcb/overlay first} together.
%   % \refKey{/tcb/overlay} overwrites this key.

% 这是同时设置\refKey{/tcb/overlay unbroken}和\refKey{/tcb/overlay first}的优化缩写。
% \refKey{/tcb/overlay}会覆盖此项设置。
% \end{docTcbKey}

% \begin{docTcbKey}{overlay middle and last}{=\meta{graphical code}}{no default, initially unset}
%   % This is an optimized abbreviation for setting
%   % \refKey{/tcb/overlay middle} and
%   % \refKey{/tcb/overlay last} together.
%   % \refKey{/tcb/overlay} overwrites this key.

% 这是同时设置\refKey{/tcb/overlay middle}和\refKey{/tcb/overlay last}的优化缩写。
% \refKey{/tcb/overlay}会覆盖此项设置。
% \end{docTcbKey}

% \begin{docTcbKey}{overlay unbroken and last}{=\meta{graphical code}}{no default, initially unset}
%   % This is an optimized abbreviation for setting
%   % \refKey{/tcb/overlay unbroken} and
%   % \refKey{/tcb/overlay last} together.
%   % \refKey{/tcb/overlay} overwrites this key.

% 这是同时设置\refKey{/tcb/overlay unbroken}和\refKey{/tcb/overlay last}的优化缩写。
% \refKey{/tcb/overlay}会覆盖此项设置。
% \end{docTcbKey}





% \begin{docTcbKey}[][doc new=2014-09-19]{overlay first and middle}{=\meta{graphical code}}{no default, initially unset}
%   % This is an optimized abbreviation for setting
%   % \refKey{/tcb/overlay first} and \refKey{/tcb/overlay middle} together.
%   % \refKey{/tcb/overlay} overwrites this key.

% 这是同时设置\refKey{/tcb/overlay first}和\refKey{/tcb/overlay middle}的优化缩写。
% \refKey{/tcb/overlay}会覆盖此项设置。
% \end{docTcbKey}



% % todo 再看
% \begin{dispListing*}{breakable,vfill before first,before upper={This example demonstrates
%   the application of break sequence specific overlay options.
%   Here, we define an environment |myexample| based
%   on |tcolorbox| where the visible drawing is done totally by overlay keys.\par
%   Here, the first application of |myexample| produces an unbroken |tcolorbox|.
%   The frame is drawn by the code given with \refKey{/tcb/overlay unbroken}.\par
%   The second application of |myexample| is broken into several parts which
%   are drawn by the codes given with
%   \refKey{/tcb/overlay first}, \refKey{/tcb/overlay middle}, and
%   \refKey{/tcb/overlay last}.
%   \par\bigskip
%   }}
%   % Preamble:
%   %\usepackage{tikz,lipsum}
%   %\tcbuselibrary{skins,breakable}
%   %\newcounter{example}
%   \colorlet{colexam}{red!75!black}
%   \newtcolorbox[use counter=example]{myexample}{%
%     empty,title={Example \thetcbcounter},attach boxed title to top left,
%     boxed title style={empty,size=minimal,toprule=2pt,top=4pt,
%       overlay={\draw[colexam,line width=2pt]
%         ([yshift=-1pt]frame.north west)--([yshift=-1pt]frame.north east);}},
%     coltitle=colexam,fonttitle=\Large\bfseries,
%     before=\par\medskip\noindent,parbox=false,boxsep=0pt,left=0pt,right=3mm,top=4pt,
%     breakable,pad at break*=0mm,vfill before first,
%     overlay unbroken={\draw[colexam,line width=1pt]
%       ([yshift=-1pt]title.north east)--([xshift=-0.5pt,yshift=-1pt]title.north-|frame.east)
%       --([xshift=-0.5pt]frame.south east)--(frame.south west); },
%     overlay first={\draw[colexam,line width=1pt]
%       ([yshift=-1pt]title.north east)--([xshift=-0.5pt,yshift=-1pt]title.north-|frame.east)
%       --([xshift=-0.5pt]frame.south east); },
%     overlay middle={\draw[colexam,line width=1pt] ([xshift=-0.5pt]frame.north east)
%       --([xshift=-0.5pt]frame.south east); },
%     overlay last={\draw[colexam,line width=1pt] ([xshift=-0.5pt]frame.north east)
%       --([xshift=-0.5pt]frame.south east)--(frame.south west);},%
%   }
  
%   \begin{myexample}
%   \lipsum[1]
%   \end{myexample}
  
%   \begin{myexample}
%   \lipsum[2-11]
%   \end{myexample}
  
%   \lipsum[12]% following text
%   \end{dispListing*}
%   {\tcbusetemp}
  
  
%   %\begin{dispExample}
%   %% \tcbuselibrary{skins}
%   %% \newcounter{example}
%   %\newtcolorbox[use counter=example]{FancyTitle}[3][]{%
%   %  enhanced,colback=blue!10!white,colframe=orange,top=4mm,
%   %  enlarge top by=\baselineskip/2+1mm,
%   %  enlarge top at break by=0mm,pad at break=2mm,
%   %  fontupper=\normalsize,label={#3},
%   %  overlay unbroken and first={%
%   %    \node[rectangle,rounded corners,draw=black,fill=blue!20!white,
%   %      inner sep=1mm,anchor=west,font=\small]
%   %      at ([xshift=4.5mm]frame.north west)
%   %         {\strut\textbf{Example \thetcbcounter: #2}};},
%   %  #1}%
  
%   %\begin{FancyTitle}{My fancy title}{fancy:title}
%   %  \lipsum[1]
%   %\end{FancyTitle}
%   %\end{dispExample}




%   % \clearpage
%   % Floating Objects\hfill 
%   \subsection{浮动对象}
%   \begin{docTcbKey}{floatplacement}{=\meta{values}}{no default, initially \texttt{htb}}
%     % Sets \meta{values} as default values for the usage of \refKey{/tcb/float}
%     % and \refKey{/tcb/float*}.
%     % Feasible are the usual parameters for floating objects.
  
%   设置\meta{values}为\refKey{/tcb/float}和\refKey{/tcb/float*}的默认值。可选值是浮动对象的常用参数。
%   \begin{dispListing}
%   \tcbset{enhanced,colback=red!5!white,colframe=red!75!black,
%       watermark color=red!15!white}
  
%   \begin{tcolorbox}[floatplacement=t,float,
%                     title=Floating box from |floatplacement|,
%                     watermark text={I am floating}]
%     This floating box is placed at the top of a page.
%   \end{tcolorbox}
%   \end{dispListing}
%   \end{docTcbKey}
%   {\tcbusetemp}
  
  
%   \begin{docTcbKey}{float}{\colOpt{=\meta{values}}}{default from \texttt{floatplacement}}
%     % Turns the box to a floating object where \meta{values} are the
%     % usual parameters for such floating objects.
%     % If they are not used, the placement uses the default values given by
%     % |floatplacement|.
  
%   将盒子转为浮动对象,\meta{values}是其浮动位置的参数。如果没有指定,那么将使用 |floatplacement| 设置的默认值。
%   \begin{dispListing}
%   \begin{tcolorbox}[float, title=Floating box from |float|,
%       enhanced,watermark text={I'm also floating}]
%     This box floats to a feasible place automatically. You do not have to
%     use a numbering for this floating object.
%   \end{tcolorbox}
%   \end{dispListing}
%   \end{docTcbKey}
%   {\tcbusetemp}
  
  
%   \begin{docTcbKey}{float*}{\colOpt{=\meta{values}}}{default from \texttt{floatplacement}}
%     % Identical to \refKey{/tcb/float}, but for wide boxes spanning the whole page
%     % width of two column documents or in conjunction with the packages
%     % |multicol| or |paracol|. Note that you have to set |width=\textwidth|
%     % additionally, if the box should span the whole page width in these cases!
  
%   同\refKey{/tcb/float}一样,但用在|multicol|或|paracol|的双栏中排版横跨页面的盒子。 
%   注意,你需要额外设置 |width=\textwidth|, 如果需要让盒子在这些情况下横跨页面的话!
%   \begin{dispListing}
%   \begin{tcolorbox}[float*=b, title=Floating box from |float*|,width=\textwidth,
%       enhanced,watermark text={I'm also floating}]
%     In this single column document, you will see no difference to |float|.
%   \end{tcolorbox}
%   \end{dispListing}
%   \end{docTcbKey}
%   {\tcbusetemp}




% \begin{docTcbKey}{nofloat}{}{style, initially set}
%   % Turns the floating behavior off.

% 关闭浮动行为。
% \end{docTcbKey}


% \begin{docTcbKey}[][doc new=2014-09-19]{every float}{=\meta{code}}{no default, initially empty}
%   % For floating objects, the \refKey{/tcb/before} and \refKey{/tcb/after}
%   % settings are ignored. Instead, the given \meta{code} is inserted before
%   % a floating box. If the box is \refKey{/tcb/breakable}, the given \meta{code} is
%   % inserted before every part of the break sequence.
%   % The most common use case is |every float=\centering|.

% 浮动对象会忽略\refKey{/tcb/before}和\refKey{/tcb/after}的设置。
% 取而代之的是, \meta{code}会被插入到浮动盒子之前。
% 如果盒子设置为\refKey{/tcb/breakable}, 那么\meta{code}会被插入每个分开的部分的前部。
% 最常见的用例是|every float=\centering|.

% \begin{dispListing}
% \tcbox[float=htb,title={Floating box},every float=\centering,
%   colback=blue!50!black,colframe=blue!50!white,colbacktitle=blue!10!white,
%   coltitle=black,center title]
%   {\includegraphics[height=6cm]{lichtspiel.jpg}}
% \end{dispListing}
% {\tcbset{reset}\tcbusetemp}

% \end{docTcbKey}

% % \clearpage
% % Embedding into the Surroundings\hfill 
% \subsection{嵌入周围}\label{subsec:surroundings}
% % Typically, but not necessarily, a |tcolorbox| is put inside a separate paragraph and has some vertical space before and after it.
% % This behavior is controlled by the keys \refKey{/tcb/before} and \refKey{/tcb/after}.

% 通常情况下,但不是必须地,一个|tcolorbox|盒子被放置在单独的段落中,在它%之前和之后
% 的前后有一些垂直空间。%
% 这种行为是由\refKey{/tcb/before}和\refKey{/tcb/after}控制。

% \begin{marker}
% % Before version 4.40, the default setting for \refKey{/tcb/before}
% % and \refKey{/tcb/after} was given by \refKey{/tcb/autoparskip}.
% % Starting with version 4.40, the default setting is given by
% % \refKey{/tcb/before skip balanced} and \refKey{/tcb/after skip balanced}.\par
% % Note that old documents may need adaptions of page breaks.\par
% % Alternatively, the old default setting can be restored by using

% 版本4.40之前,\refKey{/tcb/before}和\refKey{/tcb/after}的默认设置由\refKey{/tcb/autoparskip}控制。%
% 从4.40开始, 默认设置由\refKey{/tcb/before skip balanced}和\refKey{/tcb/after skip balanced}控制。\par
% 请注意,旧文档可能需要调整分页。\par
% 或者,可以使用以下命令恢复旧的默认设置
% \begin{dispListing}
% \tcbsetforeverylayer{autoparskip}
% \end{dispListing}
% % inside the document preamble.

% 在文档导言区中。
% \end{marker}

% \begin{docTcbKey}{before}{=\meta{code}}{no default, initially see \refKey{/tcb/before skip balanced}}
%   % Sets the \meta{code} which is executed before the colored box.
%   % It is not used for floating boxes.
%   % Also, it is not used, if the box follows a heading immediately
%   % and \refKey{/tcb/ignore nobreak} is set to \docValue{false}.

% \meta{code}将在盒子绘制前执行。此项不用于浮动盒子。另外一个不生效的情况:如果盒子之前紧跟着标题且\refKey{/tcb/ignore nobreak}设为\docValue{false}.
% \end{docTcbKey}

% \begin{docTcbKey}{after}{=\meta{code}}{no default, initially see \refKey{/tcb/after skip balanced}}
%   % Sets the \meta{code} which is executed after the colored box.
%   % It is not used for floating boxes.

% \meta{code}将在盒子绘制后执行。此项不用于浮动盒子。
% \end{docTcbKey}


% \begin{docTcbKey}{nobeforeafter}{}{style, no value}
%   % Abbreviation for clearing the keys |before| and |after|. The colored box
%   % is not put into a paragraph and there is no space before or after the box.

% 清除|before|和|after|的简写形式。盒子没有放入段落中,盒子之前或之后没有空间。\footnote{译注:用上后,多个盒子就不会分行了}
% \begin{exdispExample}{nobeforeafter}
% \tcbset{myone/.style={colback=LightGreen,colframe=DarkGreen,
%   equal height group=nobefaf,width=\linewidth/4,nobeforeafter}}
% \begin{tcolorbox}[myone,title=Box 1]Box 1\end{tcolorbox}%
% \begin{tcolorbox}[myone,title=Box 2]Box 2\end{tcolorbox}%
% \begin{tcolorbox}[myone,title=Box 3]Box 3\end{tcolorbox}%
% \begin{tcolorbox}[myone,title=Box 4]Box 4\end{tcolorbox}
% \end{exdispExample}
% \end{docTcbKey}



% \begin{docTcbKey}{force nobeforeafter}{}{style, no value}
%   % Forces the setting of \refKey{/tcb/nobeforeafter} even if
%   % \refKey{/tcb/before} and \refKey{/tcb/after} are set to other values
%   % later. Do not use this option globally unless you \emph{really} know what you do.
%   % Note that embedded boxes do not inherit this forced clearance.

% 强制设置\refKey{/tcb/nobeforeafter},甚至\refKey{/tcb/before}和\refKey{/tcb/after}是在这个选项设置之后又设置的。不要全局使用此选项,除非您\emph{真}的知道在做什么。注意,嵌套的盒子不会继承此项设置。
% \end{docTcbKey}





% % \clearpage

% \begin{docTcbKey}[][doc new={2020-09-25}]{before skip balanced}{=\meta{glue}}{no default, initially |0.5\textbackslash baselineskip plus 2pt|}
%   % Inserts some vertical space before the colored box. This style sets \refKey{/tcb/before}.\par
%   % If the depth of the
%   % preceeding \TeX\ box is between |0pt| and |0.3\baselineskip|,
%   % the distance between the \emph{baseline} of the preceeding \TeX\ box and the tcolorbox
%   % ist set to \meta{glue}$+$|0.3\baselineskip|.\par
%   % If the depth is larger, the distance of the preceeding \TeX\ box and the tcolorbox
%   % ist set to \meta{glue}.\par
%   % Alternatively, see \refKey{/tcb/before skip} which ignores the \emph{baseline}.

% 在 tcolorbox 盒子之前,插入一些竖直的空白。此项也会设置 \refKey{/tcb/before}。\par
% 如果要处理的 \TeX\ 盒子的深度介于 |0pt| 到 |0.3\baselineskip|,
% \TeX\ 盒子的 \emph{基线} 到 tcolorbox 盒子间的空白高为 \meta{glue}$+$|0.3\baselineskip|.\par
% 如果深度更大些, 则 \TeX\ 盒子和 tcolorbox 盒子间的空白设为 \meta{glue}.\par
% 也可以参阅 \refKey{/tcb/before skip},它忽略 \emph{baseline}.

% \begin{exdispExample*}{before_skip_balanced}{sbs,lefthand ratio=0.6}
% Some text.
% \begin{tcolorbox}[before skip balanced=1cm,
%     colframe=red!50!white]
%   This is a \textbf{tcolorbox}.
% \end{tcolorbox}
% \end{exdispExample*}
% \end{docTcbKey}


% \begin{docTcbKey}[][doc new={2020-09-25}]{after skip balanced}{=\meta{glue}}{no default, initially |0.5\textbackslash baselineskip plus 2pt|}
%   % Inserts some vertical space of the given \meta{glue} after the colored box.
%   % This style sets \refKey{/tcb/after}.
%   % Additionally, |\prevdepth| is set to |0.3\baselineskip|. The following
%   % \TeX\ box may enlarge the space by further glue to adjust its \emph{baseline}.
%   % Alternatively, see \refKey{/tcb/after skip} which ignores the \emph{baseline}.

% 插入指定的竖直空白到盒子后,高度为 \meta{glue}。%
% 此项会设置 \refKey{/tcb/after}。%
% 此外, |\prevdepth| 设为 |0.3\baselineskip|。下面的 \TeX\ 盒子可以通过弹性%进一步弹性来
% 扩大空间以调整其\emph{基线}。%
% 另见 \refKey{/tcb/after skip},它忽略 \emph{baseline}。

% \begin{exdispExample*}{after_skip_balanced}{sbs,lefthand ratio=0.6}
% \begin{tcolorbox}[after skip balanced=1cm,
%     colframe=red!50!white]
%   This is a \textbf{tcolorbox}.
% \end{tcolorbox}
% Some text.
% \end{exdispExample*}
% \end{docTcbKey}



% \begin{docTcbKey}[][doc new={2020-09-25}]{beforeafter skip balanced}{=\meta{glue}}{no default, initially |0.5\textbackslash baselineskip plus 2pt|}
%   插入一些指定高度为 \meta{glue} 的垂直空间到盒子的前面和后面。
%   此项会同时设置 \refKey{/tcb/before skip balanced} 和 \refKey{/tcb/after skip balanced}。
%   % todo tikzpicture
%   \begin{exdispExample*}{beforeafter_skip_balanced}{sbs,lefthand ratio=0.6}
%   \newtcolorbox{doubleline}[1][]{
%     beforeafter skip balanced=0pt,
%     height=1.8\baselineskip,
%     enlarge top by=.1\baselineskip,
%     enlarge bottom by=.1\baselineskip,
%     colframe=blue!20,colback=blue!5,
%     size=small,valign upper=center,#1 }
  
%   \noindent\begin{tikzpicture}
%   \path[use as bounding box] (0,0)
%     rectangle (0.1,0.1);
%   \foreach \y in {0,1,...,9}  {
%     \draw[very thin,red]
%       (-0.2,-\y*\baselineskip) --
%       (\linewidth+0.2cm,-\y*\baselineskip); }
%   \end{tikzpicture}
%   line 1\par
%   \begin{doubleline}  Abc  \end{doubleline}
%   \begin{doubleline}  Def  \end{doubleline}
%   line 2g\par
%   \begin{doubleline}  Ghi  \end{doubleline}
%   line 3\par
%   line 4 g
%   \end{exdispExample*}
%   \end{docTcbKey}

  

% % \clearpage

% \begin{docTcbKey}[][doc new and updated={2020-09-25}{2015-03-16}]{before skip}{=\meta{glue}}{style, no default}
%   在盒子之前插入指定高度\meta{glue}的垂直空间。%
%   此项会设置 \refKey{/tcb/before}。%
%   同 \refKey{/tcb/before skip balanced} 相比, 这个 \meta{glue} upper 部分的边缘,并不是到基线位置。
%   \begin{exdispExample*}{before_skip}{sbs,lefthand ratio=0.6}
%   Some text.
%   \begin{tcolorbox}[before skip=1cm,
%       colframe=red!50!white]
%     This is a \textbf{tcolorbox}.
%   \end{tcolorbox}
  
%   Some text.
%   \begin{tcolorbox}[before skip=0cm,
%     colframe=red!50!white]
%   This is a \textbf{tcolorbox}.
%   \end{tcolorbox}
%   \end{exdispExample*}
%   \end{docTcbKey}
  
%   \begin{docTcbKey}[][doc new and updated={2020-09-25}{2017-02-01}]{after skip}{=\meta{glue}}{style, no default}
%   在盒子之后插入指定高度\meta{glue}的垂直空间。%
%   此项会设置 \refKey{/tcb/after}.
%   同 \refKey{/tcb/after skip balanced} 相比, %
%   这个 \meta{glue} 是相对于 lower 部分的边缘,并不是到基线位置。%\footnote{译注:后面有空想下英文原文是否有误?}
%   \begin{exdispExample*}{after_skip}{sbs,lefthand ratio=0.6}
%   \begin{tcolorbox}[after skip=1cm,
%       colframe=red!50!white]
%     This is a \textbf{tcolorbox}.
%   \end{tcolorbox}
%   Some text.
%   \end{exdispExample*}
%   \end{docTcbKey}
  
%   \begin{docTcbKey}[][doc new=2014-10-10]{beforeafter skip}{=\meta{glue}}{style, no default}
%     % Inserts some vertical space of the given \meta{glue} before \emph{and} after the colored box.
%     % This style sets \refKey{/tcb/before skip} and \refKey{/tcb/after skip}.
  
%   在盒子的前后插入指定高度\meta{glue}的垂直空间。%
%   此项会设置 \refKey{/tcb/before skip} 和 \refKey{/tcb/after skip}。
  
  
%     \begin{exdispExample*}{beforeafter_skip}{sbs,lefthand ratio=0.6}
%   \tcbset{beforeafter skip=0pt,
%     colframe=red!50!white}
  
%   text before
%   \begin{tcolorbox}
%     This is a \textbf{tcolorbox}.
%   \end{tcolorbox}
%   \begin{tcolorbox}
%     Second box.
%   \end{tcolorbox}
%   text after
%   \end{exdispExample*}
%   \end{docTcbKey}
 
  

% % \clearpage

% \begin{docTcbKey}[][doc new=2014-11-07]{left skip}{=\meta{length}}{style, no default, initially |0mm|}
%   % Inserts some horizontal space of the given \meta{length} before the colored box.
%   % This style sets \refKey{/tcb/grow to left by} with the negated \meta{length},
%   % i.e. the bounding box and box width are changed.
  
% 在盒子前插入给定 \meta{length} 的水平空间。
% 此项会设定 \refKey{/tcb/grow to left by} 为 \meta{length} 的相反数,
%   i.e. 边界框和盒子宽度已更改。
% \begin{exdispExample*}{left_skip}{sbs,lefthand ratio=0.6}
% \noindent\rule{\linewidth}{2pt}

% \begin{tcolorbox}[left skip=1cm,
%     colframe=red!50!white]
%   This is a \textbf{tcolorbox}.
% \end{tcolorbox}
% \end{exdispExample*}
% \end{docTcbKey}

% \begin{docTcbKey}[][doc new=2014-11-07]{right skip}{=\meta{length}}{style, no default, initially |0mm|}
%   % Inserts some horizontal space of the given \meta{length} after the colored box.
%   % This style sets \refKey{/tcb/grow to right by} with the negated \meta{length},
%   % i.e. the bounding box and box width are changed.

% 在盒子{\bf 后}插入给定 \meta{length} 的水平空间。%
% 此项会设定  \refKey{/tcb/grow to right by} 为 \meta{length} 的相反数,
%   i.e. 边界框和盒子宽度已更改。
% \begin{exdispExample*}{right_skip}{sbs,lefthand ratio=0.6}
% \noindent\rule{\linewidth}{2pt}

% \begin{tcolorbox}[right skip=1cm,
%     colframe=red!50!white]
%   This is a \textbf{tcolorbox}.
% \end{tcolorbox}
% \end{exdispExample*}
% \end{docTcbKey}



% \begin{docTcbKey}[][doc new=2014-10-10]{leftright skip}{=\meta{length}}{style, no default}
%   % Inserts some horizontal space of the given \meta{length} before \emph{and} after the colored box.
%   % This style changes the bounding box and the box width.

%   在盒子前后插入给定长度为 \meta{length} 的水平空间。
%   此样式更改边界盒子和盒子宽度。

%   \begin{exdispExample*}{leftright_skip}{sbs,lefthand ratio=0.6}
% \noindent\rule{\linewidth}{2pt}

% \begin{tcolorbox}[leftright skip=1cm,
%     colframe=red!50!white]
%   This is a \textbf{tcolorbox}.
% \end{tcolorbox}
% \end{exdispExample*}
% \end{docTcbKey}


% % \clearpage

% \begin{docTcbKey}[][doc updated=2017-02-01]{parskip}{}{style, no value}
%   % This options is considered to be superseded by
%   % \refKey{/tcb/before skip balanced} and \refKey{/tcb/after skip balanced}
%   % (see note on page~\pageref{subsec:surroundings}).\par
%   % Sets the keys |before| and |after| to values which are
%   % recommended, if the package |parskip| \emph{is} used and there is no better
%   % idea for |before| and |after|. This is similar to:

%   此项有考虑使用 \refKey{/tcb/before skip balanced} 和 \refKey{/tcb/after skip balanced} 取代。
%   (见~\pageref{subsec:surroundings}页).\par
%   将 |before| 和 |after| 的值设置为推荐的值,如果{使用}了 |parskip| 包且 |before| 和 |after| 的值没有更好的主意。效果类似于:
% \begin{dispListing}
% \tcbset{parskip/.style={before={\par\pagebreak[0]\parindent=0pt},
%                         after={\par}}}
% \end{dispListing}
% \end{docTcbKey}

% \begin{docTcbKey}[][doc updated=2017-02-01]{noparskip}{}{style, no value}
%   % This options is considered to be superseded by
%   % \refKey{/tcb/before skip balanced} and \refKey{/tcb/after skip balanced}
%   % (see note on page~\pageref{subsec:surroundings}).\par
%   % Sets the keys |before| and |after| to values which are
%   % recommended, if the package |parskip| is \emph{not} used and there is no better
%   % idea for |before| and |after|. This is similar to:

%   此项有考虑使用 \refKey{/tcb/before skip balanced} 和 \refKey{/tcb/after skip balanced} 取代。
%   (见~\pageref{subsec:surroundings}页).\par
%   将 |before| 和 |after| 的值设置为推荐的值,如果{使用}了 |parskip| 包且 |before| 和 |after| 的值没有更好的主意。效果类似于:
% \begin{dispListing}
% \tcbset{noparskip/.style={before={\par\pagebreak[0]\smallskip\parindent=0pt},
%                           after={\par\smallskip}}}
% \end{dispListing}
% \end{docTcbKey}



% \begin{docTcbKey}{autoparskip}{}{style, no value}
%   % This options is considered to be superseded by
%   % \refKey{/tcb/before skip balanced} and \refKey{/tcb/after skip balanced}
%   % (see note on page~\pageref{subsec:surroundings}).\par
%   % Tries to detect the usage of the package |parskip| and sets
%   % the keys |before| and |after| accordingly. Actually, the following is done:

%   这项可以考虑改用 \refKey{/tcb/before skip balanced} 和 \refKey{/tcb/after skip balanced} 替换(另见~\pageref{subsec:surroundings}~页)。\par
%   尝试检测 |parskip| 的使用情况, 并相应的设置 |before| 和 |after|。 实际上,完成了以下操作:

%   \begin{itemize}
%   \item 
%   % If the length of |\parskip| is greater than |0pt| at the beginning of the document,
%   %   \refKey{/tcb/parskip} is executed. Here, the usage of package |parskip| is \emph{assumed}.

%   % todo
% 如果在文档的开头 |\parskip| 的值大于|0pt|,%
% \refKey{/tcb/parskip} 将会执行。 这里, 假定 |parskip| 引入\footnote{Here, the usage of package |parskip| is \emph{assumed}}。

%   \item 
%   % Otherwise, if the length of |\parskip| is not greater than |0pt| at the beginning of the document,
%   %   \refKey{/tcb/noparskip} is executed. Here, the absence of package |parskip| is \emph{assumed}.
% 另外,如果在文档的开头 |\parskip| 的值不大于|0pt|,%
% \refKey{/tcb/noparskip} 将会执行。 这里, 假定 |parskip| 没有使用\footnote{Here, the absence of package |parskip| is \emph{assumed}}。
%   \end{itemize}
% \end{docTcbKey}




% % \clearpage

% \begin{docTcbKey}{baseline}{=\meta{length}}{no default, initially |0pt|}
%   % Used to set the |\pgfsetbaseline| value of the resulting |tcolorbox|.
% 设置结果盒子的%,同其他 \TeX\ 对象对齐用的
% 基线的值(|\pgfsetbaseline|)。

% \begin{exdispExample}{baseline}
% \tcbset{colframe=red!50!white,width=4cm,nobeforeafter}
% Some text\dotfill
% \begin{tcolorbox}[baseline=3mm]
% 第一行
% \end{tcolorbox}
% \begin{tcolorbox}[baseline=3mm]
% 第一行\\第二行
% \end{tcolorbox}
% \begin{tcolorbox}[baseline=4mm]
% 第一行\\第二行\\第三行
%   \end{tcolorbox}
% \end{exdispExample}
% \end{docTcbKey}




% \begin{docTcbKey}[][doc new=2014-10-10]{box align}{=\meta{alignment}}{style, no default, initially |bottom|}
%   % Used to set the \refKey{/tcb/baseline} value of the resulting |tcolorbox|.
%   % Feasible values for \meta{alignment} are:

%   Used to set the \refKey{/tcb/baseline} value of the resulting |tcolorbox|.
%   Feasible values for \meta{alignment} are:  
%   \begin{itemize}
%   \item\docValue{bottom}: %alignment with the box bottom,
%   与盒子底部对齐,
%   \item\docValue{top}: %alignment with the box top,
%   与盒子顶部对齐,
%   \item\docValue{center}: %alignment with the box center,
%   与盒子中心对齐,
%   \item\docValue{base}: 
%   % alignment with the box content base. This option
%   %   is not applicable for a \refEnv{tcolorbox} but for a \refCom{tcbox} only.
%   %   It is an alias for \refKey{/tcb/tcbox raise base}.
%   与盒子内容的基线对齐。此项不在 \refEnv{tcolorbox} 使用,仅在 \refCom{tcbox} 生效。
% 这是 \refKey{/tcb/tcbox raise base} 的别名。
%   \end{itemize}

% \begin{exdispExample}{box_align_1}
% \tcbset{colframe=red!50!white,width=4cm,nobeforeafter}
% Some text\dotfill
% \begin{tcolorbox}[box align=bottom]
%   bottom
% \end{tcolorbox}
% \begin{tcolorbox}[box align=bottom]
%   bottom\\bottom
% \end{tcolorbox}
% \begin{tcolorbox}
% 第一行\\第二行\\第三行
% \end{tcolorbox}
% \end{exdispExample}

% \begin{exdispExample}{box_align_2}
% \tcbset{colframe=red!50!white,width=4cm,nobeforeafter}
% Some text\dotfill
% \begin{tcolorbox}[box align=top]
%   top
% \end{tcolorbox}
% \begin{tcolorbox}[box align=top]
%   top\\top
% \end{tcolorbox}
% \begin{tcolorbox}
% 第一行\\第二行\\第三行
% \end{tcolorbox}
% \end{exdispExample}

% \begin{exdispExample}{box_align_3}
% \tcbset{colframe=red!50!white,width=4cm,nobeforeafter}
% Some text\dotfill
% \begin{tcolorbox}[box align=center]
%   center
% \end{tcolorbox}
% \begin{tcolorbox}[box align=center]
%   center\\center
% \end{tcolorbox}
% \begin{tcolorbox}
%   第一行\\第二行\\第三行
%     \end{tcolorbox}
% \end{exdispExample}

% \begin{exdispExample}{box_align_4}
% \tcbset{colframe=red!50!white,nobeforeafter}
% Some text\dotfill
% \tcbox[nobeforeafter,box align=base]{base}
% \tcbox[nobeforeafter,box align=base,size=fbox]{base}
% \tcbox[nobeforeafter]{未设置}
% \end{exdispExample}
% \end{docTcbKey}





% \begin{docTcbKey}[][doc new=2014-12-11]{ignore nobreak}{\colOpt{=true\textbar false}}{default |true|, initially |false|}
%   % After a heading, \LaTeX\ tries to avoid a break by setting a |nobreak| boolean value.
%   % Starting from version |3.33|, the \refKey{/tcb/before} respectively \refKey{/tcb/before skip}
%   % settings are not used after a heading if \refKey{/tcb/ignore nobreak} is
%   % set to \docValue{false}. For an unbreakable box, \refKey{/tcb/before nobreak} is used instead.
%   % Further, a \refKey{/tcb/breakable} box will also try to
%   % avoid a break between a heading and a directly following first part of a
%   % break sequence.
  
%   在标题之后, 通过设置 |nobreak| ,\LaTeX\ 会尝试避免分页。
%   从版本 |3.33| 开始, 如果 \refKey{/tcb/ignore nobreak} 设置为 \docValue{false}, 那么 \refKey{/tcb/before} 和 \refKey{/tcb/before skip}
%   的设置在标题之后将不生效。%
%   对于一个不可分的盒子, 将使用 \refKey{/tcb/before nobreak} 替代使用。
%   将来, 一个设置了 \refKey{/tcb/breakable} 的盒子,将会尝试避免在标题和紧随其后的中断序列的第一部分之间中断。
  
%   % Set \refKey{/tcb/ignore nobreak} to \docValue{true}, if |nobreak| should be
%   % ignored as prior to version |3.33|. Also, such a setting may be used locally to
%   % enforce the \refKey{/tcb/before} setting.
%   在版本 |3.33|,如果需要保留这个忽略 |nobreak| 的效果,将 \refKey{/tcb/ignore nobreak} 设置为 \docValue{true}。 此外,这样的设置可以在 locally 使用以强制设置 \refKey{/tcb/before} 。
%   \end{docTcbKey}
  
%   \begin{docTcbKey}[][doc new=2014-12-16]{before nobreak}{=\meta{code}}{no default, initially \cs{noindent}}
%     % Sets the \meta{code} which is executed before the colored box if it
%     % is unbreakable, if \refKey{/tcb/ignore nobreak} is not set, and if
%     % the box follows a heading.
  
%   如果是不可以分的,设置 \meta{code} 在盒子之前执行, 如果没有设置 \refKey{/tcb/ignore nobreak} , 或如果盒子是跟随在标题之后。
%   \end{docTcbKey}

  

% \begin{docTcbKey}[][doc new=2017-02-23]{parfillskip restore}{\colOpt{=true\textbar false}}{default |true|, initially |true|}
%   % If this option is set to be |true|, the minimum value of |\parfillskip| is
%   % tested at specific spots, if it is greater than |0pt|.
%   % If so, |\parfillskip| is restored to |\@flushglue| which happens to be
%   % the default value.

% 如果此项设置为 |true|, 则在特定点测试 |\parfillskip| 的最小值, 如果它大于 |0pt|.
% 如果是这样,|\parfillskip| 恢复到 |\@flushglue|, 这恰好是默认值。

%   % These tests are executed for
% 这些判断将会在以下位置执行:
%   \refKey{/tcb/parskip},
%   \refKey{/tcb/noparskip},
%   \refKey{/tcb/after skip},
%   \refKey{/tcb/breakable}, and
%   \refEnv{tcbraster}.

%   % This option was created to automatically
%   % avoid overfull box warnings with |\parfillskip| changing packages.

% 创建此选项是为了自动避免 |\parfillskip| 改变包裹带来的 |overfull box| 警告 。
% \end{docTcbKey}




% % \clearpage
% % Bounding Box
% \subsection{边界盒子}
% % Normally, every |tcolorbox| has a bounding box which fits exactly to the
% % dimensions of the outer frame. Therefore, \LaTeX\ reserves exactly the space
% % needed for the box.
% % This behavior can be changed by enlarging (or shrinking) the bounding box.
% % If the bounding box is enlarged, the |tcolorbox| will get some clearance
% % around it. If the bounding box is shrunk, i.\,e.\ enlarged with negative
% % values, the |tcolorbox| will overlap to other parts of the page.
% % For example, the |tcolorbox| could be stretched into the page margin.

% 通常,每个 |tcolorbox| 盒子有一个与其外框严丝合缝的边界盒子。%
% 因此,\LaTeX\ 保留了盒子所需的空间。%
% 可以通过扩大(或缩小)边界盒子来更改此空间。%
% 如果边界盒子被放大, 那么 |tcolorbox| 的周围将多出一些空间。 如果边界框缩小, i.\,e.\ 扩大一个负值, |tcolorbox| 将重叠到页面的其他部分。%
% 例如,|tcolorbox| 可能会突到页边距去。

% \begin{marker}
% % The following examples use \refKey{/tcb/show bounding box} to display the actual bounding box. For this, the library \mylib{skins} has to be included and \refKey{/tcb/enhanced} has to be set.

% 以下示例使用 \refKey{/tcb/show bounding box} 来显示实际的边界框。为此,必须包含库 \mylib{skins} 并且必须设置 \refKey{/tcb/enhanced}。
% \end{marker}


% % Shifting Bounding Box Borders
% \subsubsection{移动边界盒子的边框}

% \begin{docTcbKey}{enlarge top initially by}{=\meta{length}}{no default, initially |0mm|}
%   % Enlarges the bounding box distance to the top of the box by \meta{length}.
%   % If the box is \emph{breakable}, only the first box of the break sequence
%   % gets enlarged. \refKey{/tcb/enlarge top by} overwrites this key.

% 扩大边界盒子同 |tcolorbox| 盒子的顶部的距离 \meta{length}。
% 如果 |tcolorbox| 盒子是\emph{可分的}, 只有中断序列的第一个盒子的扩大会生效。 
% \refKey{/tcb/enlarge top by} 会覆盖这个设置。
% \begin{exdispExample}{enlarge_top_initially_by}
% \tcbset{colframe=blue!75!black,colback=white}

% \begin{tcolorbox}[enlarge top initially by=-5mm]
% This is a \textbf{tcolorbox}.
% \end{tcolorbox}
% \begin{tcolorbox}[enlarge top initially by=5mm,enhanced,show bounding box]
% This is a \textbf{tcolorbox}.
% \end{tcolorbox}
% \end{exdispExample}
% \end{docTcbKey}

% % \begin{tcolorbox}[nobeforeafter]
% % \textbf{tcolorbox} a.
% % \end{tcolorbox}
% % \begin{tcolorbox}[nobeforeafter]
% % \textbf{tcolorbox} b.
% % \end{tcolorbox}




% \begin{docTcbKey}{enlarge bottom finally by}{=\meta{length}}{no default, initially |0mm|}
%   % Enlarges the bounding box distance to the bottom of the box by \meta{length}.
%   % If the box is \emph{breakable}, only the last box of the break sequence
%   % gets enlarged. \refKey{/tcb/enlarge bottom by} overwrites this key.
% 扩大边界盒子同 |tcolorbox| 盒子的底部的距离 \meta{length}。%
% 如果盒子是 \emph{可中断的}, 只有中断序列的最后一部分得到扩大。
% \refKey{/tcb/enlarge bottom by} 会覆盖这个设置。
% \begin{exdispExample}{enlarge_bottom_finally_by}
% \tcbset{colframe=blue!75!black,colback=white}

% \begin{tcolorbox}[enlarge bottom finally by=5mm]
% This is a \textbf{tcolorbox}.
% \end{tcolorbox}
% \begin{tcolorbox}[enlarge bottom finally by=-5mm,enhanced,show bounding box]
% This is a \textbf{tcolorbox}.
% \end{tcolorbox}
% \end{exdispExample}
% \end{docTcbKey}

% % \clearpage



 
% \begin{docTcbKey}{enlarge top at break by}{=\meta{length}}{no default, initially \texttt{0mm}}
%   % Enlarges the bounding box distance to the top of the box by \meta{length},
%   % \emph{if} the box is \refKey{/tcb/breakable}.
%   % In this case, it is applied to \emph{middle} and \emph{last} parts in a
%   % break sequence.
%   % \refKey{/tcb/enlarge top by} overwrites this key.

% \emph{如果}盒子是 \refKey{/tcb/breakable}的,扩大边界盒子同 |tcolorbox| 盒子的顶部的距离 \meta{length}。这种情况下, 它在 \emph{中间}和\emph{最后}的中断序列部分生效。 \refKey{/tcb/enlarge top by} 会覆盖此项设置。
% \end{docTcbKey}


% \begin{docTcbKey}{enlarge bottom at break by}{=\meta{length}}{no default, initially \texttt{0mm}}
%   % Enlarges the bounding box distance to the bottom of the box by \meta{length},
%   % \emph{if} the box is \refKey{/tcb/breakable}.
%   % In this case, it is applied to \emph{first} and \emph{middle} parts in a
%   % break sequence. \refKey{/tcb/enlarge bottom by} overwrites this key.

% \emph{如果}盒子是 \refKey{/tcb/breakable}的,扩大边界盒子同 |tcolorbox| 盒子的底部的距离 \meta{length}。
% 这种情况下, 它在 \emph{首个}和\emph{中间}的中断序列部分生效。
% \refKey{/tcb/enlarge bottom by} 会覆盖此项设置。
% \end{docTcbKey}




% \begin{docTcbKey}{enlarge top by}{=\meta{length}}{no default, initially |0mm|}

%   % Enlarges the bounding box distance to the top of the box by \meta{length}.
%   % \refKey{/tcb/enlarge top initially by} and
%   % \refKey{/tcb/enlarge top at break by} are set to \meta{length}.

%   扩大边界盒子同 |tcolorbox| 盒子的{\bf 顶}部的距离 \meta{length}。
%   \refKey{/tcb/enlarge top initially by} 和
%   \refKey{/tcb/enlarge top at break by} 也会被设置为 \meta{length}。
% \end{docTcbKey}


% \begin{docTcbKey}{enlarge bottom by}{=\meta{length}}{no default, initially |0mm|}
%   % Enlarges the bounding box distance to the bottom of the box by \meta{length}.
%   % \refKey{/tcb/enlarge bottom finally by} and
%   % \refKey{/tcb/enlarge bottom at break by} are set to \meta{length}.

%   扩大边界盒子同 |tcolorbox| 盒子的{\bf 底}部的距离 \meta{length}。
%   \refKey{/tcb/enlarge bottom finally by} 和
%   \refKey{/tcb/enlarge bottom at break by} 也会被设为 \meta{length}.
% \end{docTcbKey}



% \begin{docTcbKey}{enlarge left by}{=\meta{length}}{no default, initially |0mm|}
%   % Enlarges the bounding box distance to the left side of the box by \meta{length}.
% 扩大边界盒子同 |tcolorbox| 盒子的{\bf 左侧}的距离 \meta{length}。
% \begin{exdispExample}[safety=2cm]{enlarge_left_by}
% \tcbset{colframe=blue!75!black,colback=white}

% \begin{tcolorbox}[enlarge left by=2cm,width=5cm,enhanced,show bounding box]
% This is a \textbf{tcolorbox}.
% \end{tcolorbox}
% \begin{tcolorbox}[enlarge left by=-2cm,width=\linewidth+2cm]
% This is a \textbf{tcolorbox}.
% \end{tcolorbox}
% \end{exdispExample}
% \end{docTcbKey}

% \begin{docTcbKey}{enlarge right by}{=\meta{length}}{no default, initially |0mm|}
%   % Enlarges the bounding box distance to the right side of the box by \meta{length}.
%   扩大边界盒子同 |tcolorbox| 盒子的{\bf 右侧}的距离 \meta{length}。
% \begin{exdispExample}[safety=2cm]{enlarge_right_by}
% \tcbset{colframe=blue!75!black,colback=white}

% \begin{tcolorbox}[enlarge right by=-2cm,width=\linewidth+2cm,
%   enhanced,show bounding box]
% \textbf{tcolorbox}缩小了同右侧的距离到负数,就突出到右边了.
% \end{tcolorbox}
% \begin{tcolorbox}[enlarge right by=2cm,width=\linewidth-2cm]
% This is a \textbf{tcolorbox}.
% \end{tcolorbox}
% \end{exdispExample}
% \end{docTcbKey}




% % \clearpage
% \begin{docTcbKey}{enlarge by}{=\meta{length}}{no default, initially |0mm|}
%   % Enlarges the bounding box distance to all sides of the box by \meta{length}.

%   扩大边界盒子同 |tcolorbox| 盒子的{\bf 四侧}的距离 \meta{length}。
% \begin{exdispExample}{enlarge_by}
% \tcbset{colframe=blue!75!black,colback=white,width=5cm,nobeforeafter}

% \begin{tcolorbox}
% This is a \textbf{tcolorbox}.
% \end{tcolorbox}
% \begin{tcolorbox}[enlarge by=5mm,enhanced,show bounding box]
% This is a \textbf{tcolorbox}.
% \end{tcolorbox}
% \end{exdispExample}
% \end{docTcbKey}





% \begin{docTcbKey}{grow to left by}{=\meta{length}}{no default, initially |0mm|}
%   % Enlarges the current box width by \meta{length} and
%   % enlarges (shrinks) the bounding box distance to the left side of the box by
%   % $-$\meta{length}. Also see \refKey{/tcb/left skip}.

% 扩大当前盒子的宽度\meta{length},并扩大边界盒子到左侧的距离为
%   $-$\meta{length}。\footnote{突出到左侧了} 另见 \refKey{/tcb/left skip}。
% \begin{exdispExample}[safety=2cm]{grow_to_left_by}
% \tcbset{colframe=blue!75!black,colback=white}

% \begin{tcolorbox}[width=5cm,grow to left by=2cm,enhanced,show bounding box]
% This is a \textbf{tcolorbox} with a width of 7cm.
% \end{tcolorbox}
% \end{exdispExample}
% \end{docTcbKey}

% \begin{docTcbKey}{grow to right by}{=\meta{length}}{no default, initially |0mm|}
%   % Enlarges the current box width by \meta{length} and
%   % enlarges (shrinks) the bounding box distance to the right side of the box by
%   % $-$\meta{length}. Also see \refKey{/tcb/right skip}.

%   扩大当前盒子的宽度\meta{length},并扩大边界盒子到右侧的距离为
%   $-$\meta{length}。\footnote{突出到右侧了} 另见 \refKey{/tcb/right skip}。
% \begin{exdispExample}[safety=2cm]{grow_to_right_by}
% \tcbset{colframe=blue!75!black,colback=white}

% \begin{tcolorbox}[grow to right by=2cm,enhanced,show bounding box]
% This is a \textbf{tcolorbox}.
% \end{tcolorbox}

% \bigskip

% \begin{tcolorbox}[grow to right by=2cm,grow to left by=1cm,
%   enhanced,show bounding box]
% This is a \textbf{tcolorbox}.
% \end{tcolorbox}
% \end{exdispExample}
% \end{docTcbKey}

% % \clearpage


% \begin{docTcbKey}[][doc new=2018-03-22]{grow sidewards by}{=\meta{length}}{no default, initially |0mm|}
%   % Shortcut for setting \refKey{/tcb/grow to left by} and \refKey{/tcb/grow to right by}
%   % to \meta{length}. Also see \refKey{/tcb/oversize} and \refKey{/tcb/spread sidewards}.

% 同时设置 \refKey{/tcb/grow to left by} 和 \refKey{/tcb/grow to right by} 到\meta{length} 的简写形式。另见 \refKey{/tcb/oversize} 和 \refKey{/tcb/spread sidewards}.
% \begin{exdispExample}[safety=2cm]{grow_sidewards_by}
% \tcbset{colframe=blue!75!black,colback=white}

% \begin{tcolorbox}[grow sidewards by=2cm,enhanced,show bounding box]
% This is a \textbf{tcolorbox}.
% \end{tcolorbox}
% \end{exdispExample}
% \end{docTcbKey}



% %  Box Alignment
% \subsubsection{盒子的对齐}

% \begin{docTcbKey}[][doc new=2015-11-20]{flush left}{}{style, no value}
%   % Enlarges the bounding box to the right side to fill the line completely.

% 左对齐效果,扩大边界盒子完全填充到右侧。
% \begin{exdispExample}{flush_left}
% \tcbset{colframe=blue!75!black,colback=white}

% \begin{tcolorbox}[flush left,width=5cm,enhanced,show bounding box]
% This is a \textbf{tcolorbox}.
% \end{tcolorbox}
% \end{exdispExample}
% \end{docTcbKey}


% \begin{docTcbKey}[][doc new=2015-11-20]{flush right}{}{style, no value}
%   % Enlarges the bounding box to the left side to fill the line completely.

%   右对齐效果,扩大边界盒子完全填充到左侧。
% \begin{exdispExample}{flush_right}
% \tcbset{colframe=blue!75!black,colback=white}

% \begin{tcolorbox}[flush right,width=5cm,enhanced,show bounding box]
% This is a \textbf{tcolorbox}.
% \end{tcolorbox}
% \end{exdispExample}
% \end{docTcbKey}


% \begin{docTcbKey}[][doc new=2015-11-20]{center}{}{style, no value}
%   % Enlarges the bounding box equally to both sides to fill the line completely.

%   居中对齐效果,扩大边界盒子完全填充到两侧。
% \begin{exdispExample}{center}
% \tcbset{colframe=blue!75!black,colback=white}

% \begin{tcolorbox}[center,width=5cm,enhanced,show bounding box]
% This is a \textbf{tcolorbox}.
% \end{tcolorbox}
% \end{exdispExample}
% \end{docTcbKey}




% % \clearpage

% \subsubsection{Toggle Enlargements}

% \begin{docTcbKey}[][doc updated=2015-11-13]{toggle enlargement}{=\meta{toggle preset}}{默认 |evenpage|(偶), initially |none|}
%   % According to the \meta{toggle preset}, the left and the right enlargements of
%   % the bounding box are switched or not. Feasible values are:

% 依据 \meta{toggle preset} 的值, 对边界盒子的左边和右边的增加空间的设置进行交换或不交换。可设的值有:
%   \begin{itemize}
%   \item\docValue{none}: %no switching.
%   不切换。
%   \item\docValue{forced}: %the values of the left and right enlargement are switched.
%   强制将边界盒子的左边和右边的增加空间的设置进行交换。
%   \item\docValue{evenpage}: 
%   % if the page is an even page, the values of the left and    right enlargement are switched. This value also sets    \refKey{/tcb/check odd page} to |true|.
% 如果当前页是偶数页, 将边界盒子的左边和右边的增加空间的设置进行交换。这项值也会将  \refKey{/tcb/check odd page} 设置为 |true|.
%   \end{itemize}
% \begin{marker}
% % See \refKey{/tcb/toggle left and right} to toggle geometry settings.
% 见 \refKey{/tcb/toggle left and right} 来切换几何设置。
% \end{marker}

% \begin{dispExample}
% \tcbset{colframe=blue!75!black,colback=white,
%   grow to left by=20mm,%突出左侧
%   grow to right by=-5mm%右侧凹着
%   }

% \begin{tcolorbox}[toggle enlargement=none
%   ,enhanced,show bounding box]
% 设置为 |toggle enlargement=none|,不切换
% \end{tcolorbox}
% \begin{tcolorbox}[toggle enlargement=forced]
%   设置为 |toggle enlargement=forced|,强制切换
% \end{tcolorbox}
% \begin{tcolorbox}[toggle enlargement=evenpage]
% 设置为 |toggle enlargement=evenpage|,偶数页才切换。当前页是 \tcbifoddpage{奇}{偶} 数页。因此, 左边的增加空间的设置 \tcbifoddpage{不会}{会}切换。
% \end{tcolorbox}
% \end{dispExample}

% \begin{dispListing}
% \begin{tcolorbox}[colframe=red!60!black,colback=red!15!white,
%   fonttitle=\bfseries,title=Floating box from \texttt{toggle enlargement},
%   width=\textwidth
% ,grow to right by=2cm%突出到右边
% ,toggle enlargement%默认是偶数页
% ,float=t]
% 当前页是\tcbifoddpage{奇}{偶}数页。%
% 因此, 左边的增加空间的设置 \tcbifoddpage{不会}{会}切换。%
% 这个盒子,在奇数页是突出到右边,在偶数页是突出到左边。%
% 本文档是one-sided文档 -- 这项特性只在two-sided%
% \footnote{译注:即双面打印,%
% 奇数和偶数页的文档内容的左右边距是不同的,以用于装订。}%
% 文档中生效。
% \end{tcolorbox}
% \end{dispListing}
% \tcbusetemp
% \end{docTcbKey}





% % \clearpage
% % Spread Box to Page Borders
% \subsubsection{盒子扩张到页面边缘}

% \begin{marker} 
% % The following border options are \emph{not} applicable to nested boxes, boxes insides tables, etc.
% % For boxes inside lists, the options \emph{may} work, but not necessarily.
% % Also, boxes should be set with |\noindent| and full width.

% 以下的 border 选项对嵌套的盒子, 在表格中的盒子\emph{不}起作用, etc.
% 对于列表中的盒子,这些选项\emph{可以}工作, 但没必要。
% 另外,盒子需要设置 |\noindent| 来达到全宽。
% \end{marker}

% \begin{docTcbKey}[][doc new=2017-02-13]{spread inwards}{\colOpt{=\meta{length}}}{default |0pt|, initially unset}
% % Enlarges the current box width to match the inner page border (left-handed side for one-sided
% % documents). If the optional \meta{length} is greater than |0pt|, the box
% % grows over the border, if \meta{length} is lower than |0pt|, there is a
% % margin between box and page border.
% % \refKey{/tcb/toggle enlargement} is set automatically.

% 扩张当前盒子的宽度到书本的内页边缘(对单面文档是在左侧).如果选项的值\meta{length}是大于 |0pt|, 那么盒子将穿过页面的边缘, 如果\meta{length}是小于|0pt|,那么在盒子和页面边缘就有一段面边空白。会自动设置 \refKey{/tcb/toggle enlargement} 。
% \begin{dispListing}
% \begin{tcolorbox}[enhanced,spread inwards,
%   colframe=blue!75!black,colback=white,show bounding box]
% 扩张当前盒子的宽度到书本的内页边缘 (对单面文档是在左侧).(|spread inwards|)
% \end{tcolorbox}

% \begin{tcolorbox}[enhanced,spread inwards=2em,
%   colframe=blue!75!black,colback=white,show bounding box]
% 前面的内容穿过边缘了(|spread inwards=2em,|)。
% \end{tcolorbox}

% \begin{tcolorbox}[enhanced,spread inwards=-2em,
%   colframe=blue!75!black,colback=white,show bounding box]
% 在盒子和页面边缘就有一段面边空白。(|spread inwards=-2em,|)。
% \end{tcolorbox}
% \end{dispListing}
% {\tcbusetemp}
% \end{docTcbKey}



% \begin{docTcbKey}[][doc new=2017-02-13]{spread outwards}{\colOpt{=\meta{length}}}{default |0pt|, initially unset}
%   % Enlarges the current box width to match the outer page border (right-handed side for one-sided
%   % documents). If the optional \meta{length} is greater than |0pt|, the box
%   % grows over the border, if \meta{length} is lower than |0pt|, there is a
%   % margin between box and page border.
%   % \refKey{/tcb/toggle enlargement} is set automatically.
  
%   扩张当前盒子的宽度到书本的内页边缘(对单面文档是在右侧)。如果选项的值\meta{length}是大于 |0pt|, 那么盒子将穿过页面的边缘, 如果\meta{length}是小于|0pt|,那么在盒子和页面边缘就有一段空白。会自动设置 \refKey{/tcb/toggle enlargement} 。
  
%   \begin{dispListing}
%   \begin{tcolorbox}[enhanced,spread outwards,
%     colframe=blue!75!black,colback=white,show bounding box]
%   This is a \textbf{tcolorbox}.
%   \end{tcolorbox}
%   \end{dispListing}
%   {\tcbusetemp}
%   \end{docTcbKey}


%   \begin{docTcbKey}[][doc new=2017-02-13]{move upwards}{\colOpt{=\meta{length}}}{default |0pt|, initially unset}
%     % Starts a new page with the box at the very top page border.
%     % If the optional \meta{length} is greater than |0pt|, the box
%     % moves over the border, if \meta{length} is lower than |0pt|, there is a
%     % margin between box and page border.
%     新起一页,将盒子放在新页面的最顶部。%
%     如果选项的值\meta{length}是大于 |0pt|, 那么盒子将穿过页面的边缘, 如果\meta{length}是小于|0pt|,那么在盒子和页面边缘就有一段空白。
%     \end{docTcbKey}
    
    
%     \begin{docTcbKey}[][doc new=2017-02-13]{move upwards*}{\colOpt{=\meta{length}}}{default |0pt|, initially unset}
%     同\refKey{/tcb/move upwards}一样,但少了新起一页的操作。
%     \end{docTcbKey}



 
% \begin{docTcbKey}[][doc new=2017-02-13]{fill downwards}{\colOpt{=\meta{length}}}{default |0pt|, initially unset}
%   % Enlarges the height of the box until the very bottom page border.
%   % The library \mylib{breakable} has to be loaded, and
%   % \refKey{/tcb/height fill} is set automatically.
%   % If the optional \meta{length} is greater than |0pt|, the box
%   % moves over the border, if \meta{length} is lower than |0pt|, there is a
%   % margin between box and page border.
%   扩张当前盒子的宽度到书本的底部边缘。%
%   需要加载 \mylib{breakable} 库, 且会自动设置\refKey{/tcb/height fill} 。%
%   如果选项的值\meta{length}是大于 |0pt|, 那么盒子将穿过页面的边缘, 如果\meta{length}是小于|0pt|,那么在盒子和页面边缘就有一段空白。
%   \begin{dispListing}
%   \begin{tcolorbox}[enhanced,fill downwards,
%     colframe=blue!75!black,colback=white,show bounding box]
%   扩张当前盒子的宽度到书本的底部边缘。
%   \end{tcolorbox}
%   \end{dispListing}
%   {\tcbusetemp}
%   \end{docTcbKey}


% \begin{tcolorbox}[enhanced,spread upwards,sharp corners=north,height=3cm,
%   colframe=blue!75!black,interior style={top color=blue!50,bottom color=white}]
% 这是\enquote{spread upwards}的例子。 
% \end{tcolorbox}
% \begin{docTcbKey}[][doc new=2017-02-13]{spread upwards}{\colOpt{=\meta{length}}}{default |0pt|, initially unset}
% 组合,同时将\meta{length}设到
% \refKey{/tcb/move upwards}, \refKey{/tcb/spread inwards}, 和 \refKey{/tcb/spread outwards}.
% \begin{dispListing}
% \begin{tcolorbox}[enhanced,spread upwards,sharp corners=north,height=3cm,
%   colframe=blue!75!black,interior style={top color=blue!50,bottom color=white}]
% 这是 \enquote{spread upwards} 的例子。
% \end{tcolorbox}
% \end{dispListing}
% \end{docTcbKey}


% \begin{docTcbKey}[][doc new=2017-02-13]{spread upwards*}{\colOpt{=\meta{length}}}{default |0pt|, initially unset}
% 同\refKey{/tcb/move upwards}一样,但不会新起一页。
% \end{docTcbKey}



 
% \begin{docTcbKey}[][doc new=2017-02-13]{spread sidewards}{\colOpt{=\meta{length}}}{default |0pt|, initially unset}
%   % Combination of \refKey{/tcb/spread inwards} and \refKey{/tcb/spread outwards}.
%   % The optional \meta{length} is used for all these keys.
%   % Also see \refKey{/tcb/oversize} and \refKey{/tcb/grow sidewards by}.
  
%   \meta{length}被同时设置到 \refKey{/tcb/spread inwards} 和 \refKey{/tcb/spread outwards}。另见 \refKey{/tcb/oversize} 和 \refKey{/tcb/grow sidewards by}.
%   \begin{dispListing}
%   \begin{tcolorbox}[enhanced,spread sidewards,
%     colframe=blue!75!black,colback=white,show bounding box]
%   向左右两侧突出了。
%   \end{tcolorbox}
%   \end{dispListing}
%   {\tcbusetemp}
%   \end{docTcbKey}
  
  
%   \begin{docTcbKey}[][doc new=2017-02-13]{spread}{\colOpt{=\meta{length}}}{default |0pt|, initially unset}
%   % Combination of
%   % \refKey{/tcb/move upwards}, \refKey{/tcb/fill downwards}, \refKey{/tcb/spread inwards},
%   % and \refKey{/tcb/spread outwards}.
%   % Such, the box fills the whole page.
%   % The optional \meta{length} is used for all these keys.
  
%   组合了 \refKey{/tcb/move upwards}, \refKey{/tcb/fill downwards}, \refKey{/tcb/spread inwards},和 \refKey{/tcb/spread outwards}。
%   这样,盒子就填满了整个页面。
%   \meta{length} 被同时设置到这些选项。
%   \end{docTcbKey}




% \begin{docTcbKey}[][doc new=2017-02-13]{spread downwards}{\colOpt{=\meta{length}}}{default |0pt|, initially unset}
%   % Combination of
%   % \refKey{/tcb/fill downwards}, \refKey{/tcb/spread inwards}, and \refKey{/tcb/spread outwards}.
%   % The optional \meta{length} is used for all these keys.
  
%   组合使用 \refKey{/tcb/fill downwards}, \refKey{/tcb/spread inwards}, 和 \refKey{/tcb/spread outwards}.
%   \meta{length} 被同时设置到这些选项。
%   \begin{dispListing}
%   \begin{tcolorbox}[enhanced,spread downwards,sharp corners=south,
%     colframe=red!75!black,interior style={top color=white,bottom color=red!50}]
%   This is an example for \enquote{spread downwards}.
%   \end{tcolorbox}
%   \end{dispListing}
%   \end{docTcbKey}
%   \begin{tcolorbox}[enhanced,spread downwards,sharp corners=south,
%     colframe=red!75!black,interior style={top color=white,bottom color=red!50}]
%   This is an example for \enquote{spread downwards}.
%   \end{tcolorbox}
  





% % \clearpage
% % Box Extrusion
% \subsubsection{挤压盒子}

% \begin{marker}
% % The following keys should not be used with breakable boxes or boxes with a
% % lower part.

% 以下选项不应在可分盒子或带有lower部分的盒子内使用。
% \end{marker}

% \begin{docTcbKey}{shrink tight}{}{style, no value, initially unset}
%   % The total colored box is shrunk to the dimensions of the upper
%   % part. There should be no lower part and no title.
%   % This style sets the \refKey{/tcb/boxsep} to |0pt| and other geometry keys
%   % to fitting values. This option is likely to be used with the following
%   % extrusion keys.

% 整个盒子缩小到upper部分的尺寸。不应有lower部分和标题。
% 此样式会将 \refKey{/tcb/boxsep} 设置为 |0pt|,以及一些其他几何设置。此选项可能与以下挤压键一起使用。
% \begin{exdispExample}{shrink_tight}
% \tcbset{colframe=blue!75!black,colback=white,arc=0mm,boxrule=0.4pt,
%         nobeforeafter,tcbox raise base,shrink tight}

% \begin{tcolorbox}
% This is a \textbf{tcolorbox}.
% \end{tcolorbox}

% Lorem \tcbox{ipsum} dolor sit amet, consectetuer adipiscing elit.
% \end{exdispExample}

% \begin{exdispExample}{shrink_tight2}
%   \tcbset{colframe=blue!75!black,colback=white,arc=0mm,boxrule=0.4pt,
%           shrink tight}
  
%   \begin{tcolorbox}
%   This is a \textbf{tcolorbox}.
%   \end{tcolorbox}
  
%   Lorem \tcbox{ipsum} dolor sit amet, consectetuer adipiscing elit.
%   \end{exdispExample}
% \end{docTcbKey}  

% % extrude
% % ① (force out) 挤出 jǐchū ‹toothpaste, glue, icing›; 压出 yāchū ‹pasta›
% % ② (shape) 压制 yāzhì ‹plastic, metal›
% \begin{docTcbKey}[][doc updated=2014-09-19]{extrude left by}{=\meta{length}}{style, no default, initially unset}
%   % The (upper part of the) colored box is extruded by the given \meta{length} to the left side.
%   % The inner width and the bounding box is kept unchanged and the operation
%   % is additive!

%   (upper部分)盒子向左挤出 \meta{length} 空间\footnote{译注:这部分有点像零宽的盒子效果。}。内部宽度和边界盒子保持不变,挤出是额外的!
% \begin{exdispExample}{extrude_left_by}
% \tcbset{enhanced,colframe=red,colback=yellow!25!white,
%   frame style={opacity=0.25},interior style={opacity=0.5},
%   nobeforeafter,tcbox raise base,shrink tight,extrude by=2mm}

% Lorem ipsum dolor sit amet, consectetuer adipiscing elit. Ut purus elit,
% vestibulum ut, placerat ac, adipiscing vitae, felis.
% \tcbox[extrude left by=1cm]{Curabitur} dictum gravida mauris.
% Nam arcu libero, nonummy eget, consectetuer id, vulputate a, magna.
% \end{exdispExample}

% \begin{exdispExample}{extrude_left_by2}
%   \tcbset{enhanced,colframe=red,colback=yellow!25!white,
%     frame style={opacity=0.25},interior style={opacity=0.5},
%     nobeforeafter,tcbox raise base,shrink tight,extrude by=2mm}
  
%   Lorem ipsum dolor sit amet, consectetuer adipiscing elit. Ut purus elit,
%   vestibulum ut, placerat ac, adipiscing vitae, felis.
%   \tcbox[extrude left by=1cm,,show bounding box]{Curabitur} dictum gravida mauris.
%   Nam arcu libero, nonummy eget, consectetuer id, vulputate a, magna.
%   \end{exdispExample}

% \end{docTcbKey}

% \begin{docTcbKey}[][doc updated=2014-09-19]{extrude right by}{=\meta{length}}{style, no default, initially unset}
%   % The (upper part of the) colored box is extruded by the given \meta{length} to the right side.
%   % The inner width and the bounding box is kept unchanged and the operation
%   % is additive!

%   (upper部分)盒子向{\bf 右}挤出 \meta{length} 空间%\footnote{译注:这部分有点像零宽的盒子效果。}
%   。内部宽度和边界盒子保持不变,挤出是额外的!
% \begin{exdispExample}{extrude_right_by}
% \tcbset{enhanced,colframe=red,colback=yellow!25!white,
%   frame style={opacity=0.25},interior style={opacity=0.5},
%   nobeforeafter,tcbox raise base,shrink tight,extrude by=2mm}

% Lorem ipsum dolor sit amet, consectetuer adipiscing elit. Ut purus elit,
% vestibulum ut, placerat ac, adipiscing vitae, felis.
% \tcbox[extrude right by=1cm]{Curabitur} dictum gravida mauris.
% Nam arcu libero, nonummy eget, consectetuer id, vulputate a, magna.
% \end{exdispExample}
% \end{docTcbKey}

% % \clearpage
% \begin{docTcbKey}{extrude top by}{=\meta{length}}{style, no default, initially unset}
%   % The (upper part of the) colored box is extruded by the given \meta{length} to the top side.
%   % The inner width and the bounding box is kept unchanged and the operation
%   % is additive!

%   (upper部分)盒子向{\bf 上}挤出 \meta{length} 空间%\footnote{译注:这部分有点像零宽的盒子效果。}
%   。内部宽度和边界盒子保持不变,挤出是额外的!
% \begin{exdispExample}{extrude_top_by}
% \tcbset{enhanced,colframe=red,colback=yellow!25!white,
%   frame style={opacity=0.25},interior style={opacity=0.5},
%   nobeforeafter,tcbox raise base,shrink tight,extrude by=2mm}

% Lorem ipsum dolor sit amet, consectetuer adipiscing elit. Ut purus elit,
% vestibulum ut, placerat ac, adipiscing vitae, felis.
% \tcbox[extrude top by=1cm]{Curabitur} dictum gravida mauris.
% Nam arcu libero, nonummy eget, consectetuer id, vulputate a, magna.
% \end{exdispExample}
% \end{docTcbKey}

% \begin{docTcbKey}{extrude bottom by}{=\meta{length}}{style, no default, initially unset}
%   % The (upper part of the) colored box is extruded by the given \meta{length} to the bottom side.
%   % The inner width and the bounding box is kept unchanged and the operation
%   % is additive!
%   (upper部分)盒子向{\bf 下}挤出 \meta{length} 空间%\footnote{译注:这部分有点像零宽的盒子效果。}
%   。内部宽度和边界盒子保持不变,挤出是额外的!
% \begin{exdispExample}[safety=1cm]{extrude_bottom_by}
% \tcbset{enhanced,colframe=red,colback=yellow!25!white,
%   frame style={opacity=0.25},interior style={opacity=0.5},
%   nobeforeafter,tcbox raise base,shrink tight,extrude by=2mm}

% Lorem ipsum dolor sit amet, consectetuer adipiscing elit. Ut purus elit,
% vestibulum ut, placerat ac, adipiscing vitae, felis.
% \tcbox[extrude bottom by=1cm]{Curabitur} dictum gravida mauris.
% Nam arcu libero, nonummy eget, consectetuer id, vulputate a, magna.
% \end{exdispExample}
% \end{docTcbKey}



% \begin{docTcbKey}{extrude by}{=\meta{length}}{style, no default, initially unset}
%   % The (upper part of the) colored box is extruded by the given \meta{length} to all sides.
%   % The inner width and the bounding box is kept unchanged and the operation
%   % is additive!

%   (upper部分)盒子向{\bf 四周}都挤出 \meta{length} 空间%\footnote{译注:这部分有点像零宽的盒子效果。}
%   。内部宽度和边界盒子保持不变,挤出是额外的!
% \begin{exdispExample}{extrude_by}
% \tcbset{enhanced,colframe=red,colback=yellow!25!white,
%   frame style={opacity=0.25},interior style={opacity=0.5},
%   nobeforeafter,tcbox raise base,shrink tight,extrude by=2mm}

% Lorem ipsum dolor sit amet, consectetuer adipiscing elit. Ut purus elit,
% vestibulum ut, placerat ac, adipiscing vitae, felis. \tcbox{Curabitur} dictum
% gravida mauris. \tcbox[colframe=Green,interior style={opacity=0.0}]{Nam}
% arcu libero, nonummy eget, consectetuer id, \tcbox{vulputate} a, magna. Donec
% vehicula augue eu neque. Pellentesque habitant morbi tristique senectus et netus
% et malesuada fames ac turpis egestas. \tcbox{Mauris ut leo.}
% \end{exdispExample}
% \end{docTcbKey}



% % \clearpage
% % Layered Boxes and Every Box Settings
% \subsection{分层盒子和所有的盒子设置}\label{subsec:everybox}
% % A |tcolorbox| may contain another |tcolorbox| and so on. The package
% % takes track of the nesting level using a counter |tcblayer|. Counter values
% % may be used for doing some fancy things, but you should never change
% % the counter value yourself.

% 一个 |tcolorbox| 盒子可能包含着另一个 |tcolorbox| 盒子,像俄罗斯套娃。本包使用计数器 |tcblayer| 记录盒子是处于嵌套的第几层。 可以用这个计数器值来做一些花哨的事情, 但你永远不应该改变这个计数器的值。

% % The package takes special care for the first four layers or nesting levels,
% % called managed layers.
% % Here, footnote texts are administrated to find their intended place
% % and specific layer dependent options may be set by changing
% % \refKey{/tcb/every box on layer n}.
% % If needed, the number of managed layers can be increased by setting
% % \refCom{tcbsetmanagedlayers} to a higher value than~4.

% %todo 再次翻译
% 该包对前四层或嵌套层会特殊处理, 称为管理层。%
% 在这些层,脚注文本 are administrated to find their 预期位置 specific layer dependent options 可以通过 更改 \refKey{/tcb/every box on layer n} 来设置。
% 如果需要,可以通过将 \refCom{tcbsetmanagedlayers} 设置为高于~4 的值来增加管理层的数量。


% % The following styles have a considerable influence on how layered boxes
% % are processed. Note especially that nested boxes are getting a
% % \refKey{/tcb/reset} by default. You can change this, but be prepared for
% % surprises if you do.

% 以下样式对多层盒子的处理方式有相当大的影响。特别注意,嵌套的盒子默认会被设置 \refKey{/tcb/reset} 。您可以更改此设置,但如果你这样做,要做好惊讶的准备。

% % If the defaults are \emph{not changed}, a |tcolorbox| gets its options
% % in the following order. Following options overwrite preceding options.

% 如果默认值\emph{没有被改变}, 一个|tcolorbox|按以下顺序获取其选项。后出现的选项会覆盖前面的选项:


% \begin{enumerate}
%   \item %On package load, all options are set to default values.
%   在包加载时,所有选项都设置为默认值。
%   \item %Every \refCom{tcbset} command adds or changes options for the following boxes inside the current \TeX\ group.
%   每个 \refCom{tcbset} 命令添加或更改当前 \TeX\ 组中后续盒子的选项。
%   \item 
%   %While entering a |tcolorbox|, a \refKey{/tcb/every box on layer n} or  \refKey{/tcb/every box on higher layers} option list is applied.  With default settings this means:
  
%   进入一个 |tcolorbox| 盒子, 会应用 \refKey{/tcb/every box on layer n} 或  \refKey{/tcb/every box on higher layers} 的选项列表。使用默认设置,这意味着:
%     \begin{itemize}
%     \item %
%     % For layer 1 (lowest layer), the \refKey{/tcb/every box} option list is applied.
%     %   Not overwritten options given by a preceding \refCom{tcbset} survive.
%   对于第 1 层(最低层), 会应用 \refKey{/tcb/every box} 的选项列表。%
%   未被 \refCom{tcbset} 覆盖的选项仍然存在。
%     \item 
%     % For layer 2 and above (nested boxes), a \refKey{/tcb/reset} followed by \refKey{/tcb/every box} option list is applied.  Every resettable options given by a preceding \refCom{tcbset} and by the sourrounding box(es) are reset.
%     % todo 重新翻译
%     对于第 2 层及以上层(嵌套盒子),在 \refKey{/tcb/every box} 选项列表之后会应用 \refKey{/tcb/reset}。 所有能被重置的,由 \refCom{tcbset} 以及外层盒子给出的选项,会被重置。
%     \end{itemize}
%   \item 
%   % The \meta{options} given to the |tcolorbox| are applied.
%   %   Or, if the box was generated by \refCom{newtcolorbox} or friends,
%   %   the \meta{options} given there are applied.
%   直接在 |tcolorbox| 环境参数中设置的 \meta{选项} 被设置。或者,如果盒子生成是由 \refCom{newtcolorbox} 或类似命令, 那边给出的 \meta{选项} 被设置。
%   \item 
%   % If the box was generated by \refCom{newtcolorbox} or friends,  some automated options are applied.
%   如果盒子生成是由 \refCom{newtcolorbox} 或类似命令, 会自动被设置一些选项。
%   \end{enumerate}
  
  
% \begin{docTcbKey}{every box}{}{style}
%   % By default, this style is empty.
%   默认情况下,此样式为空。
%   \begin{dispListing}
%   % default setting:
%   \tcbset{every box/.style={}}
%   \end{dispListing}
%   % It may be changed by redefining this style.
%   可以通过重新定义此样式来更改它。
%   \begin{dispListing}
%   % setting all boxes to be enhanced:
%   \tcbset{every box/.style={enhanced}}
%   \end{dispListing}
  
%   \medskip
%   \begin{marker}
%   % The alternative for setting something for every box (on every layer) is\\
%   为每个盒子(在每一层)设置一些东西的替代方法是\\
%    \refCom{tcbsetforeverylayer}:
%   \begin{dispListing}
%   % setting all boxes to be enhanced:
%   \tcbsetforeverylayer{enhanced}
%   \end{dispListing}
%   \end{marker}
%   \end{docTcbKey}




% % \clearpage
% \begin{docTcbKey}{every box on layer n}{}{style}
%   % Here, |n| has to be replaced by a number ranging from 1 to the highest
%   % managed layer number (4 by default).
%   在这里,|n|为从 1 到最高的托管层编号数字(默认为 4)。
%   \begin{dispListing}
%   % default settings:
%   \tcbset{
%     every box on layer 1/.style={every box},
%     every box on layer 2/.style={reset,every box},
%     every box on layer 3/.style={reset,every box},
%     every box on layer 4/.style={reset,every box},
%     }
%   \end{dispListing}
%   \end{docTcbKey}
  
  
%   \begin{docTcbKey}{every box on higher layers}{}{style}
%   % Higher layers are layers above the highest
%   % managed layer number (4 by default).
  
%   更高层是最高托管层数(默认为 4)之上的层。
%   \begin{dispListing}
%   \tcbset{every box on higher layers/.style={reset,every box}}
%   \end{dispListing}
%   \end{docTcbKey}

  


% \begin{docCommand}{tcbsetmanagedlayers}{\marg{number}}
%   % Replaces the highest managed layer number by \meta{number} where 4 is
%   % the default. This macro can only be used inside the preamble.
%   % Using a \meta{number} lower than 4 typically makes no sense, but is
%   % not forbidden.
  
%   用 \meta{number} 替换最高管理层编号,其中 4 是默认值。该宏只能在序言内使用。使用小于 4 的 \meta{number} 通常没有意义,但是不禁止。
%   \end{docCommand}
  
%   \begin{tcboutputlisting}
%   % \usepackage{lipsum}
%   % \tcbuselibrary{skins,breakable}
%   \tcbset{colframe=red!75!black,fonttitle=\bfseries,
%     colback=red!5!white,
%     every box/.style={enhanced,watermark text=\thetcblayer,
%       before=\par\smallskip,after=\par\smallskip},
%     every box on layer 2/.style={reset,every box,colback=yellow!10!white,
%       drop fuzzy shadow}}
%   \begin{tcolorbox}[enhanced jigsaw,breakable,title=第1层盒子]
%     这里有一个脚注\footnote{第1层的脚注}。
%   \lipsum[2]
%     \begin{tcolorbox}[title=第2层盒子]
%     abc\footnote{第2层的脚注}
%     \end{tcolorbox}
%     \begin{tcolorbox}[title=Another Box,ams equation]
%       \tcbhighmath{\sum\limits_{n=1}^{\infty} \frac{1}{n}} = \infty.
%     \end{tcolorbox}
%   Some text\footnote{Footnote from some text}.
%     \begin{tcolorbox}[title=Yet Another Box]%第2层
%       第2层
%       \tcboxfit[height=2cm]{\lipsum[1]}
%       \begin{tcolorbox}
%         第3层\footnote{第3层的脚注}. \lipsum[3]
%         \begin{tcolorbox}[title=Layer 4,colframe=blue,colback=white]
%           Layer 4\footnote{第4层}
%         \end{tcolorbox}
%         The End\footnote{第4层的脚注}.
%       \end{tcolorbox}
%     \end{tcolorbox}
%   \end{tcolorbox}
%   \end{tcboutputlisting}
  
%   \tcbinputlisting{base example,listing only,listing style=mydocumentation}
  
%   {\tcbuselistingtext}
  

  
% % \clearpage
% % figurative (record) 刻画 kèhuà
% % (take by force) 俘获 fúhuò ‹person›; 捕获 bǔhuò ‹animal›; 占领 zhànlǐng ‹place›
% \subsection{Capture Mode}\label{subsec:capture}

% \begin{docTcbKey}{capture}{=\meta{mode}}{no default, initially |minipage|}
%   % The capture \meta{mode} defines how the box content is processed.
% capture \meta{mode} 定义如何处理盒子内容。

  

% % Feasible values for \meta{mode} are:
% \meta{mode} 的可选值是:
% \begin{itemize}
% \item\docValue{minipage}:\\
%   % This is the default \meta{mode} for \refEnv{tcolorbox}.
%   % The content may have an upper and a lower part. 
%   %Optionally, the box
%   % can be \refKey{/tcb/breakable}. The box content is put into a
%   % minipage or into something similar to a minipage.
% 这是 \refEnv{tcolorbox} 的默认 \meta{mode} 。%
% 内容可能有一个 |upper|和一个|低|部分。%
% 可选地,盒子可以是 \refKey{/tcb/breakable} 可分的。
% 盒子内容被放入一个 |minipage| 或类似于 |minipage| 的东西。
% \item\docValue{hbox}:\\
%   % This is the default \meta{mode} for \refCom{tcbox}. The content cannot have
%   % a lower part and cannot be broken. The colored box is sized according
%   % to the dimensions of the content.
%   % A shortcut to set this mode is \refKey{/tcb/hbox}.
% 这是 \refCom{tcbox} 的默认 \meta{mode} 。%
% 内容将没有lower部分,也不可分。%
% 盒子的大小根据内容的尺寸而定。%
% 设置此模式的快捷方式是 \refKey{/tcb/hbox}.
% \item\docValue{fitbox}:%
% %  (needs the \mylib{fitting} library)\\
%  (需要启用 \mylib{fitting} 库)\\
%   % This is the default \meta{mode} for \refCom{tcboxfit}. The content cannot have
%   % a lower part and cannot be broken.
%   % The content is sized according to the dimensions of the colored box.
%   % A shortcut to set this mode is \refKey{/tcb/fit}.
% 这是 \refCom{tcboxfit} 的默认 \meta{mode}。 %
% 盒子内没有lower部分,也不可分。
% 盒子的大小根据内容的尺寸而定。%
% 设置此模式的快捷方式是 \refKey{/tcb/fit}.
% \end{itemize}

% \begin{exdispExample}{capture}
% \tcbset{colframe=blue!75!black,colback=white}

% \begin{tcolorbox}[capture=minipage]
% |capture=minipage|\\
% 这是 \refEnv{tcolorbox} 的默认 \meta{mode} 。%
% 内容可能有一个 |upper|和一个|低|部分。%
% \tcblower
% 可选地,盒子可以是 \refKey{/tcb/breakable} 可分的。
% 盒子内容被放入一个 |minipage| 或类似于 |minipage| 的东西。
% \end{tcolorbox}

% \begin{tcolorbox}[capture=hbox]
% |capture=hbox|\\
% 内容将没有lower部分,也不可分。%
% % \tcblower 使用会报错
% 盒子的大小根据内容的尺寸而定。而定而定而定而定而定%
% \end{tcolorbox}

% \begin{tcolorbox}[capture=fitbox,height=9mm]% needs the `fitting' library
% |capture=fitbox|\\
% 内容将没有lower部分,也不可分。
% %\footnote{译注,看这效果,是内容适应指定的高度等}%
% \end{tcolorbox}
% \end{exdispExample}
% \end{docTcbKey}




% \begin{docTcbKey}{hbox}{}{style, no default}
%   % Shortcut for |capture=hbox|.
% |capture=hbox| 的快捷方式。
% \begin{exdispExample}{hbox}
% \tcbset{colframe=blue!75!black,colback=white}

% \begin{tcolorbox}[hbox]
% This is a tcolorbox.
% \end{tcolorbox}
% \end{exdispExample}
% \end{docTcbKey}


% \begin{docTcbKey}{minipage}{}{style, no default}
%   % Shortcut for |capture=minipage|.
%   |capture=minipage| 的快捷方式。
% \end{docTcbKey}




% % \clearpage
% % Text Characteristics
% % ① (trait) (of person) 特征 tèzhēng ; (of place, work) 特性 tèxìng
% % ▸ a family characteristic
% % 家族特征
% % ② Mathematics [对数的] 首数 shǒushù
% % 典型的 diǎnxíng de ‹feature›; 独特的 dútè de ‹behaviour, appearance, quality›
% \subsection{Text Characteristics}
% \begin{docTcbKey}[][doc updated=2015-10-14]{parbox}{\colOpt{=true\textbar false}}{default |true|, initially |true|}
%   % The text inside a |tcolorbox| is formatted using a \LaTeX\ |minipage|
%   % if the box is unbreakable. 
%   % If breakable, the box tries a mimicry of a |minipage|. 
%   % In a |minipage| or |parbox|, paragraphs are formatted slightly different
%   % as the main text. If the key value is set to |false|, the normal main text
%   % behavior is restored. In some situations, this has some unwanted side
%   % effects. It is recommended that you use this experimental setting only
%   % where you really want to have this feature.

% 如果盒子是不可分的,|tcolorbox| 中的文本使用 \LaTeX\ |minipage| 格式化。
% 如果是可分的, 盒子试图模仿一个 |minipage|。%
% 文本在 |minipage| 和 |parbox| 中的格式处理会有略微的不同。%
% 如果这项值设为 |false|, 将恢复正常的主文本行为。%
% 在某些情况下,这会产生一些不必要的副作用。%
% 建议您只在真正希望具有此特性的地方使用此实验性设置。
% \end{docTcbKey}

% \begin{dispListing}
% % \usepackage{lipsum}  % preamble
% \tcbset{width=(\linewidth-2mm)/2,nobeforeafter,arc=1mm,
%   colframe=blue!75!black,colback=white,fonttitle=\bfseries,fontupper=\small,
%   left=2mm,right=2mm,top=1mm,bottom=1mm,equal height group=parbox}

% \begin{tcolorbox}[parbox,adjusted title={parbox=true (normal)}]
%   \lipsum[1-2]
% \end{tcolorbox}\hfill%
% \begin{tcolorbox}[parbox=false,adjusted title={parbox=false}]
%   \lipsum[1-2]
% \end{tcolorbox}%
% \end{dispListing}
% {\tcbusetemp}




% % \clearpage
% \begin{docTcbKey}{hyphenationfix}{\colOpt{=true\textbar false}}{default |true|, initially |false|}
%   % Long words at the beginning of paragraphs in very narrow boxes
%   % will not be hyphenated using |pdflatex|. This problem is circumvented by
%   % applying the |hyphenationfix| option.

% 使用|pdflatex|时,在非常狭窄的盒子中,段落开头的长单词,不会用连字符。%
% 通过应用|hyphenationfix|选项,可以规避此问题。
% \begin{exdispExample*}{hyphenationfix}{sbs,lefthand ratio=0.6}
% \tcbset{colframe=blue!75!black,
%   fontupper=\normalsize,
%   colback=blue!5!white,width=4cm}

% \begin{tcolorbox}
% Rechnungsadjunktentochter.\par
% Statthaltereikonzipist.
% \end{tcolorbox}

% \begin{tcolorbox}[hyphenationfix]
% Rechnungsadjunktentochter.\par
% Statthaltereikonzipist.
% \end{tcolorbox}
% \end{exdispExample*}

% \smallskip
% \begin{marker}
% % |parbox=false| and |hyphenationfix| should not be used together. 
% % They are targeting different box types and they do not blend very well.

% |parbox=false| 和 |hyphenationfix| 不应该一起使用。%
% 他们的目标是不同的盒子类型。%, 他们不能很好地融合。
% \end{marker}
% \end{docTcbKey}


% % Files
% \subsection{文件}
% \begin{docTcbKey}{tempfile}{=\meta{file name}}{no default, initially \cs{jobname.tcbtemp}}
%   % Sets \meta{file name} as name for the temporary file which is used inside
%   % \refEnv{tcbwritetemp} and \refCom{tcbusetemp} implicitely.
% 隐式地将 \meta{file name}  设置为在 \refEnv{tcbwritetemp} 和 \refCom{tcbusetemp} 中使用的临时文件的名称
% \end{docTcbKey}

% % applicable
% % 美 [əˈplɪkəb(ə)l]
% % 英 [əˈplɪkəb(ə)l]
% % adj.适用;合适
% % 网络可应用的;适当的;适用的
% \subsection{\texttt{\textbackslash tcbox} Specials}
% % The following options are applicable for \refCom{tcbox} and \refCom{tcboxmath}
% % only.

% 以下选项仅适用于 \refCom{tcbox} 和 \refCom{tcboxmath}。
% \begin{docTcbKey}{tcbox raise}{=\meta{length}}{no default, initially \texttt{0pt}}
%   % Raises the \refCom{tcbox} by the given \meta{length}.
%   将 \refCom{tcbox} 上移指定的高度 \meta{length}。
%   % Sets the line width of the right rule to \meta{length}.
% \begin{exdispExample}{tcbox_raise}
% \tcbset{colframe=blue!50!black,colback=white,colupper=red!50!black,
%         fonttitle=\bfseries,nobeforeafter,center title}

% Test\dotfill
% \tcbox[tcbox raise base]{Hello World 1}\dotfill
% \tcbox{Hello World 2}\dotfill
% \tcbox[tcbox raise=5mm]{上移5mm}
% \end{exdispExample}
% \end{docTcbKey}



% \begin{docTcbKey}{tcbox raise base}{}{style, no value, initially unset}
%   % Raises the \refCom{tcbox} such that the base of its content matches
%   % the base of the environmental line; see example above.

% 上移 \refCom{tcbox} ,使盒子内容的基线匹配所在环境的基线对齐;请参见上面的示例。
%   % 与环境行的基础相匹配; 
% \end{docTcbKey}

% \begin{docTcbKey}{on line}{}{style, no value, initially unset}
%   % Combines \refKey{/tcb/tcbox raise base} with \refKey{/tcb/nobeforeafter}.
%   % The resulting box behaves analogue to |\fbox|.

% %   analogue
% % 美 ['ænə.lɔɡ]
% % 英 ['ænə.lɒɡ]
% % adj.模拟的;指针式的
% % n.相似物;类似事情
% % 网络类似物;类比;同源语
% 组合 \refKey{/tcb/tcbox raise base} 和 \refKey{/tcb/nobeforeafter}.
% 得到的盒子的行为类似于|\fbox|.
% \end{docTcbKey}




% % \clearpage
% \begin{docTcbKey}[][doc new=2015-03-23]{tcbox width}{=\meta{mode}}{no default, initially \texttt{auto}}
%   % Controls how \refCom{tcbox} respects a \refKey{/tcb/width} setting.
%   % Feasible values for \meta{mode} are:
  
%   控制\refCom{tcbox}对宽度参数\refKey{/tcb/width}的处理。%
%   \meta{mode}可以选择的值有:
%   \begin{itemize}
%   \item\docValue{auto} 
%   % (initial setting):
%   %   ignore \refKey{/tcb/width} and set box width according to its content.
%   (初始设定) :
%   忽略\refKey{/tcb/width},根据盒子内容设置宽度。
%   \item\docValue{auto limited}:
%     % Set box width according to its content, if it is smaller than \refKey{/tcb/width}.
%     % Otherwise, the content is set like in a \refEnv{tcolorbox} with line breaks.
%   如果盒子内容的宽度小于\refKey{/tcb/width},则据内容设置盒子的宽度。%
%   否则,盒子内的效果类似于可换行的\refEnv{tcolorbox}。
%   \item\docValue{forced center}:
%     % Set box width according to \refKey{/tcb/width}.
%     % The content is centered and may overlap the box borders.
%   将盒子的宽度设置为\refKey{/tcb/width}。%
%   内容居中,可能与盒子两侧重叠。
%   \item\docValue{forced left}:
%     % Set box width according to \refKey{/tcb/width}.
%     % The content is left aligned and may overlap the box borders.
%   将盒子的宽度设置为\refKey{/tcb/width}。%
%   内容居{\bf 左},可能与盒子两侧重叠。
%   \item\docValue{forced right}:
%     % Set box width according to \refKey{/tcb/width}.
%     % The content is right aligned and may overlap the box borders.
%   将盒子的宽度设置为\refKey{/tcb/width}。%
%   内容居{\bf 右},可能与盒子两侧重叠。
%   \item\docValue{minimum center}:
%     % Set box width according to \refKey{/tcb/width}, if the content fits into.
%     % The content is centered and the box width may grow beyond \refKey{/tcb/width}.
%   如果内容合适,将盒子的宽度设置为\refKey{/tcb/width}。%
%   内容是居中的,盒宽可能超出\refKey{/tcb/width}。
%   \item\docValue{minimum left}:
%     % Set box width according to \refKey{/tcb/width}, if the content fits into.
%     % The content is left aligned and the box width may grow beyond \refKey{/tcb/width}.
%   如果内容合适,将盒子的宽度设置为\refKey{/tcb/width}。%
%   内容是居{\bf 左}的,盒宽可能超出\refKey{/tcb/width}。
%   \item\docValue{minimum right}:
%   如果内容合适,将盒子的宽度设置为\refKey{/tcb/width}。%
%   内容是居{\bf 右}的,盒宽可能超出\refKey{/tcb/width}。
%   \end{itemize}

  
% % \enlargethispage*{1cm}

% \begin{exdispExample}{tcbox_width}
%   \tcbset{size=small,on line,before upper=\strut,
%     colframe=blue!75!black,colback=blue!5!white,
%     fontupper=\normalsize,width=4cm}
  
%   \tcbox[tcbox width=auto]{auto}\qquad
%   \tcbox[tcbox width=auto limited]{auto limited}\qquad
%   \tcbox[tcbox width=auto limited]{auto limited遇上长文本}\\
%   \tcbox[tcbox width=forced center]{forced center}\qquad
%   \tcbox[tcbox width=forced center]{forced center with long text}\\
%   \tcbox[tcbox width=forced left]{forced left}\qquad
%   \tcbox[tcbox width=forced left]{forced left with long text}\\
%   \tcbox[tcbox width=forced right]{forced right}\qquad
%   \tcbox[tcbox width=forced right]{forced right with long text}\\
%   \tcbox[tcbox width=minimum center]{minimum center}\qquad
%   \tcbox[tcbox width=minimum center]{minimum center with long text}\\
%   \tcbox[tcbox width=minimum left]{minimum left}\qquad
%   \tcbox[tcbox width=minimum left]{minimum left with long text}\\
%   \tcbox[tcbox width=minimum right]{minimum right}\qquad
%   \tcbox[tcbox width=minimum right]{minimum right with long text}
%   \end{exdispExample}
%   \end{docTcbKey}



% %\subsection{Skins}
% %There are additional option keys which change the appearance of a |tcolorbox|.
% %If only the core package is used, there is only one \emph{skin} and these
% %keys are meaningless.
% %The library \mylib{skins} adds more skins. The appropriate option keys for skins of
% %the core package are therefore described in \Vref{sec:skincorekeys} from
% %page \pageref{sec:skincorekeys}.

% % \clearpage
% % Counters, Labels, and References
% \subsection{计数器、标签和引用}

% \begin{docTcbKey}{phantom}{=\meta{code}}{no default, initially unset}
% % The \meta{code} is put in a box at the upper left corner of the |tcolorbox|.
% % If the |tcolorbox| is breakable, the \meta{code} is executed for the first box of
% % the break sequence only. If there already was some phantom code given, the
% % new \meta{code} is appended.\par
% % The \meta{code} is intended to be used for counter stepping, labelling, and
% % related operations which do not produce visible text.
% \meta{code}被放在|tcolorbox|的左上角的盒子中。%
% 如果 |tcolorbox| 是可分的, \meta{code} 将只会在中断序列的第一部分执行。%
% 如果已经给出了一些phantom代码,新的\meta{code}被追加过去。\par
% \meta{code}旨在用于计数器步进, 标签和一些不会产生可见内容的相关操作。
% \begin{itemize}
% \item 
% % The \meta{code} is executed before the title and box content, i.\,e.\ counter
% %   values are ensured to be increased before usage.
% \meta{code}在标题和盒子内容之前执行, i.\,e.\ 确保计数器的值在使用前增加了。
% \item %Labels are ensured to reference the correct page number.
% 确保标签引用正确的页码。
% \item 
% % The \meta{code} is executed only once even during fitting operations for
% %   title and box content.
% \meta{code}只执行一次,即使是在标题和内容的自适应过程中。
% \item 
% % In combination with the |hyperref| package, the hyper anchor is set
% %   to the upper left corner of the |tcolorbox|, i.\,e.\ 
% % links inside the pdf document   will jump to the box pleasantly.
% %todo 再翻译
% 结合|hyperref|包,超锚被设置为|tcolorbox|的左上角, i.\,e.\ PDF文档中的链接将友好地跳转到相应盒子。

% \item 
% % Since the \meta{code} is executed inside a \TeX\ group, only global
% %   operations can survive this group.
% 由于\meta{code}是在\TeX\ 组中执行的,因此仅是全局的在这个群体中,操作可以存活下来。
% \end{itemize}
% % Examples for the |phantom| usage are given in Section \ref{listing:exercises}
% % from page \pageref{listing:exercises}, e.\,g.\
% % Example \ref{exe:tabular_example} on page \pageref{exe:tabular_example}.
% |phantom| 的使用示例见\pageref{listing:exercises}页的\ref{listing:exercises}小节
% , e.\,g.\ 第 \pageref{exe:tabular_example} 页的 \ref{exe:tabular_example}。
% \end{docTcbKey}


% \begin{docTcbKey}{nophantom}{}{no value, initially set}
% % Removes the phantom code if set before.
% 删除之前设置的phantom代码。
% \end{docTcbKey}
  

% \begin{docTcbKey}{label}{=\meta{marker}}{no default, initially unset}
%   % The \meta{marker} is set as label text for a reference with the |\ref| macro.
%   % Typically, this option is used for numbered boxes, see Subsection \ref{sec:numberedboxes}
%   % from page \pageref{sec:numberedboxes}, e.\,g.\ \refKey{/tcb/new/auto counter}.
  
%   \meta{marker}被设置为|\ref|宏引用的标签文本。%
%   通常,这个选项用于编号的盒子,参见,\pageref{sec:numberedboxes}页的 \ref{sec:numberedboxes} 小节%
%    , e.\,g.\ \refKey{/tcb/new/auto counter}.
%   \end{docTcbKey}
  
%   \begin{docTcbKey}[][doc new=2014-11-28]{phantomlabel}{=\meta{marker}}{no default, initially unset}
%   % Equivalent to \refKey{/tcb/label} for an \emph{unnumbered} box.
%   % A |\phantomsection| from the package |hyperref| \cite{rahtz:hyperref} is used to set a correct
%   % hyperlink target. This is not needed for a numbered box.
%   等效于\emph{未编号}的盒子的\refKey{/tcb/label}。%
%   包|hyperref|中的|\phantomsection|用于设置正确的超链接目标。%
%   对于有编号的盒子,这是不需要的。
%   \end{docTcbKey}



% \begin{docTcbKey}{label type}{=\meta{type}}{no default, initially unset}
%   % This option key can be used only in conjunction with the |cleveref| package
%   % \cite{cubitt:2018a} which has to be loaded separately.
%   % \meta{type} has to be a cross-reference type \emph{known} to |cleveref|
%   % like |theorem|, |algorithm|, |result|, etc. References made with |cleveref|
%   % will use this type. Note that using |label type| will result in compilation
%   % errors, if |cleveref| is not loaded.
%   % For an example, see \Vref{theo:meanvaluetheorem}.
  
%   此选项键只能与|cleveref|包一起使用,|cleveref|包必须单独加载。%
%   \meta{type}必须是|cleveref|的交叉引用类型,如 |theorem|, |algorithm|, |result|, 等。%
%   使用|cleveref|所做的引用将使用此类型。%
%   注意的是,如果|cleveref|未加载, 使用 |label type| 将导致编译错误。例子见 \Vref{theo:meanvaluetheorem}。
%   \end{docTcbKey}
  
%   \begin{docTcbKey}{no label type}{}{no value, initially set}
%   % Removes a \refKey{/tcb/label type}, if set before.
%   删除\refKey{/tcb/label type},如果之前有设置过。
%   \end{docTcbKey}
  
%   \begin{docTcbKey}{step}{=\meta{counter}}{no default, initially unset}
%   % Shortcut for |phantom={\refstepcounter{#1}}|. The given \meta{counter} is
%   % increased and ready for labelling. This option is not needed when
%   % using the convenient automated numbering introduced with version 2.40,
%   % see Subsection \ref{sec:numberedboxes}
%   % from page \pageref{sec:numberedboxes}.
%   |phantom={\refstepcounter{#1}}|的快捷方式。%
%   给定的计数器\meta{counter}被增加并准备好标记。%
%   当使用2.40版本引入的,方便的,自动编号时,不需要这个选项,%
%   见\pageref{sec:numberedboxes}页的\ref{sec:numberedboxes}小节。
%   \end{docTcbKey}



% \begin{docTcbKey}{step and label}{=\marg{counter}\marg{marker}}{no default, initially unset}
%   % Shortcut for using \refKey{/tcb/step} and \refKey{/tcb/label}. This option is not needed when
%   % using the convenient automated numbering introduced with version 2.40,
%   % see Subsection \ref{sec:numberedboxes}
%   % from page \pageref{sec:numberedboxes}.
%   使用\refKey{/tcb/step}和\refKey{/tcb/label}的快捷方式。%
%   当使用2.40版本引入的方便的自动编号时,不需要这个选项,%
%   参见\pageref{sec:numberedboxes}页的\ref{sec:numberedboxes}小节。
%   \end{docTcbKey}
  

% % \clearpage
% \begin{docTcbKey}{list entry}{=\meta{text}}{no default, initially unset}
%   % If the \flqq list of tcolorbox(es)\frqq\ feature described in Subsection
%   % \ref{sec:listsof} from page \pageref{sec:listsof} is used, this key
%   % describes the \meta{text} for an entry into the generated list, e.\,g.
%   如果使用了,在\pageref{sec:listsof}页的\ref{sec:listsof}小节描述的 \flqq tcolorbox(es)列表\frqq\ 特性, 
%   这个选项描述了生成列表中条目的\meta{text}, e.\,g.
%   \begin{dispListing}
%   list entry={\protect\numberline{\thetcbcounter}My beautiful Example}
%   \end{dispListing}
%   完整的例子见 \pageref{listing:exercises} 页的 \ref{listing:exercises} 小节。
%   \end{docTcbKey}

  
% \begin{docTcbKey}[][doc new=2014-09-19]{list text}{=\meta{text}}{style, no default}
%   % This is a shortcut for setting \refKey{/tcb/list entry} to\\
%   % |\protect\numberline{\thetcbcounter}|\meta{text}.
%   % So, the following settings are identical:
%   这是将 \refKey{/tcb/list entry} 设为\\
%   |\protect\numberline{\thetcbcounter}|\meta{text} 的快捷方式。
%   因此,以下设置是相同的:
%   \begin{dispListing}
%   list text={My beautiful Example},
%   list entry={\protect\numberline{\thetcbcounter}My beautiful Example}
%   \end{dispListing}
%   % See Section \ref{listing:exercises} from page \pageref{listing:exercises}
%   % for a complete example.
%   完整的例子见 \pageref{listing:exercises} 页的 \ref{listing:exercises} 小节。
%   \end{docTcbKey}



% \begin{docTcbKey}{add to list}{=\marg{list}\marg{type}}{no default, initially unset}
%   % If the \flqq list of tcolorbox(es)\frqq\ feature described in Subsection
%   % \ref{sec:listsof} from page \pageref{sec:listsof} is used, list entries are
%   % generated automatically. With this key, you can enforce an entry to the
%   % given \meta{list} with the given \meta{type}.
%   % This issues:\\
%   % |\addcontentsline|\marg{list}\marg{type}\marg{entry text}
%   如果使用了,\pageref{sec:listsof}页的\ref{sec:listsof}小节描述的\flqq list of tcolorbox(es)\frqq\ 功能, 列表项会自动生成。使用此键,您可以使用给定的\meta{type}将一个条目强制到给定的\meta{list}。
%   This issues:\\
%   |\addcontentsline|\marg{list}\marg{type}\marg{entry text}
%   \end{docTcbKey}




% \begin{docTcbKey}[][doc new and updated={2016-06-22}{2016-11-18}]{nameref}{=\meta{text}}{no default, initially unset}
%   % If the |nameref| package is loaded, the given \meta{text} is used for
%   % corresponding |\nameref| macros. Typically, the \meta{text} will be chosen
%   % to be identical or nearly identical to the one for \refKey{/tcb/title}.
  
%   如果加载了|nameref|包,%
%   给定的\meta{text}作为|\nameref|宏的参数。%
%   通常,\meta{text}将被选择为与\refKey{/tcb/title}相同或几乎相同。
  
%   \inputpreamblelisting{A}


% \begin{dispExample}
%   \begin{pabox}[label={mynamelabel},nameref={Title or anything else}]{Title text}
%   This is a tcolorbox.
%   \end{pabox}
%   This box is automatically numbered with \ref{mynamelabel} on page
%   \pageref{mynamelabel}.
  
%   The box is titled \enquote{\nameref{mynamelabel}}.
%   \end{dispExample}
  
%   \begin{marker}
%   % \refKey{/tcb/nameref} is used automatically inside \refCom{newtcbtheorem}.
%   \refKey{/tcb/nameref}在\refCom{newtcbtheorem}中自动使用。
%   \end{marker}
  
%   \end{docTcbKey}




% % \clearpage
% \begin{docTcbKey}[][doc new=2017-02-03]{hypertarget}{=\meta{marker}}{no default, initially unset}
%   % A |\hypertarget| from the package |hyperref| \cite{rahtz:hyperref} is used to
%   % create an internal link of an anchor \meta{marker}.
%   % This \meta{marker} can be referenced by |\hyperlink| or
%   % \refKey{/tcb/hyperlink}.
%   包|hyperref|中的|\hypertarget|用于创建一个锚\meta{marker}的内部链接。
%   这个\meta{marker}可以通过|\hyperlink|或\refKey{/tcb/hyperlink}链接引用到。
  
%     \begin{dispExample*}{sbs,lefthand ratio=0.7}
%   % \usepackage{hyperref}%
%   \begin{tcolorbox}[enhanced,
%     colback=red!10,colframe=red!50!black,
%     hypertarget=hypertwinA,
%     hyperlink=hypertwinB,
%     title=Box A]
%   Click me to jump to Box B.
%   \end{tcolorbox}
%     \end{dispExample*}
%   \end{docTcbKey}

%   \begin{docTcbKey}[][doc new=2017-02-10]{bookmark}{=\meta{text}}{no default, initially unset}
%     % Sets a PDF bookmark with the given \meta{text}, if the package |bookmark| \cite{oberdiek:bookmark}
%     % is loaded. This bookmark is set with an automated destination (the current box)
%     % and is set one level below the current bookmark level.
    
%   % Sets a PDF bookmark with the given \meta{text}, if the package |bookmark| is loaded. 
%   如果加载了包|bookmark|,则使用给定的\meta{text}设置PDF书签。%
%   此书签使用自动目标(当前盒子)设置,并设置在当前书签级别以下一级。%
%     \begin{dispExample*}{sbs,lefthand ratio=0.7}
%   % \usepackage{bookmark}%
%   \begin{tcolorbox}[colback=blue!10,colframe=blue!50!black,
%     bookmark=Example for using a bookmark,
%     title=Example for using a bookmark]
%   Open the bookmark view of the previewer
%   to see the bookmark.
%   \end{tcolorbox}
%     \end{dispExample*}
%   \end{docTcbKey}
  
  
%   \begin{docTcbKey}[][doc new=2017-02-10]{bookmark*}{=\marg{options}\marg{text}}{no default, initially unset}
%     % Identical to \refKey{/tcb/bookmark}, but additional \meta{options}
%     % from the package |bookmark| \cite{oberdiek:bookmark} can be given.
%     与\refKey{/tcb/bookmark}相同,但可以从包|bookmark|中给出额外的\meta{options}。
%     \begin{dispExample*}{sbs,lefthand ratio=0.7}
%   % \usepackage{bookmark}%
%   \begin{tcolorbox}[colback=red!10,colframe=red!50!black,
%     bookmark*={color=red,italic,bold}%
%               {Another bookmark example},
%     title=Red and bold bookmark]
%   Open the bookmark view of the previewer
%   to see the bookmark.
%   \end{tcolorbox}
%     \end{dispExample*}
%   \end{docTcbKey}

%   \begin{docTcbKey}[][doc new=2018-07-26]{index}{=\meta{entry}}{no default, initially unset}
%     % Adds an index \meta{entry} for the box. This is a shortcut for
%     % setting |\index|\marg{entry} to \refKey{/tcb/phantom}.
  
%   为盒子添加索引\meta{entry}。 这是一个将|\index|\marg{entry} 设置为 \refKey{/tcb/phantom}的快捷方式。
%   \end{docTcbKey}
  
%   \begin{docTcbKey}[][doc new=2018-07-26]{index*}{=\marg{name}\marg{entry}}{no default, initially unset}
%     % Adds an \meta{entry} to an index with a specific \meta{name}.
%     % This is a shortcut for  setting |\index|\oarg{name}\marg{entry} to \refKey{/tcb/phantom}.
%     % An index extension package like |imakeidx| has to be loaded to use  this option key.
  
%     % Adds an \meta{entry} to an index with a specific \meta{name}.
%     将\meta{entry}添加到具有特定\meta{name}的索引中。
%     这是一个将|\index|\oarg{name}\marg{entry}设置为\refKey{/tcb/phantom}的快捷方式。
%     必须加载像|imakeidx|这样的索引扩展包才能使用此选项键。
%   \end{docTcbKey}
  
%   % \clearpage
% % Even and Odd Pages
% \subsection{偶数页和奇数页}

% \begin{marker}
% % Also see \refKey{/tcb/toggle left and right} and \refKey{/tcb/toggle enlargement} for further even/odd options.
% 也可以参考\refKey{/tcb/toggle left and right}和\refKey{/tcb/toggle enlargement}了解更多的偶数/奇数选项。
% \end{marker}

% \begin{docTcbKey}[][doc updated=2015-11-13]{check odd page}{\colOpt{=true\textbar false}}{default |true|, initially |false|}
% % If set to |true|, a precise even/odd page testing for the current box is applied. 
% % This is done by using labels. If a box moves to another page,
% % the document has to be compiled twice for the correct settings.
% % If set to |false|, even/odd page tests may give wrong results for the first box of a page.


% 如果设置为|true|,则对当前盒子应用精确的偶数/奇数页测试。%
% 这是通过使用标签来实现的。%
% 如果一个盒子移动到另一页,%
% 为了获得正确的设置,必须编译两次文档。%
% 如果设置为|false|,偶数/奇数页测试可能会对页面的第一个盒子给出错误的结果。

% % \refKey{/tcb/toggle left and right},
% % \refKey{/tcb/toggle enlargement}, and
% % \refKey{/tcb/if odd page}
% % automatically set |check odd page|, but for
% % \refCom{tcbifoddpage} this option has to be set explicitely.
% \refKey{/tcb/toggle left and right},
% \refKey{/tcb/toggle enlargement}, 和
% \refKey{/tcb/if odd page}
% 自动设置|check odd page|, 但是对于\refCom{tcbifoddpage},这个选项必须显式设置。
% \end{docTcbKey}


% % \enlargethispage*{1cm}
% \begin{docTcbKey}[][doc new=2015-11-13]{if odd page}{=\marg{odd options}\marg{even options}}{style, no default}
%   % If the current box is on an odd page, the \meta{odd options} are applied.
%   % On an even page, the \meta{even options} are applied.
%   % \refKey{/tcb/check odd page} is automatically set for precise even/odd page testing.
  
%   如果盒子当前位于奇数页,则应用\meta{odd options}。%
%   如果是偶数页, 则应用\meta{even options}。%
%   \refKey{/tcb/check odd page}会自动设置,用于精确的偶数/奇数页测试。
  
%   \begin{dispExample}
%   \begin{tcolorbox}[if odd page={colback=yellow!50}{colback=red!50}]
%   这个盒子在奇数页上是黄色的,在偶数页上是红色的。
%   \end{tcolorbox}
%   \end{dispExample}
  
%   \begin{marker}
%   % If a box is \refKey{/tcb/breakable}, using \refKey{/tcb/if odd page}
%   % only acts upon the \emph{first} box. If the setting should be
%   % repeated for every partial box of the break sequence, the option should be
%   % packed into \refKey{/tcb/extras}. In this case, \refKey{/tcb/check odd page}
%   % has to be set explicitely! Also see \refKey{/tcb/if odd page*}.
  
%   如果盒子设置了 \refKey{/tcb/breakable}, 则 \refKey{/tcb/if odd page} 只在\emph{第一}部分盒子生效。%
%   如果对中断序列的每个部分框都要重复设置,则选项应设到\refKey{/tcb/extras}。%
%   在这种情况下,\refKey{/tcb/check odd page}必须显式设置! 另见 \refKey{/tcb/if odd page*}.
%   \end{marker}
%   \end{docTcbKey}

  


% \begin{docTcbKey}[][doc new=2016-11-18]{if odd page or oneside}{=\marg{odd options}\marg{even options}}{style, no default}
%   % For onesided documents, the \meta{odd options} are applied always.
%   % For twosided documents, this style is identical to \refKey{/tcb/if odd page}.

% 对于单开文档,总是应用\meta{odd选项}。%
% 对于双面文档,此样式与\refKey{/tcb/if odd page}相同。
% \end{docTcbKey}

% % \clearpage
% \begin{docTcbKey}[][doc new=2015-11-13]{if odd page*}{=\marg{odd options}\marg{even options}}{style, no default}
%   \begin{marker}
%   % This option needs the \mylib{breakable} library, see \Fullref{sec:breakable}.
%   这个选项需要\mylib{breakable}库,参见\Fullref{sec:breakable}。
%   \end{marker}
%   % For breakable boxes, if the current partial box is on an odd page, the \meta{odd options} are applied.
%   % On an even page, the \meta{even options} are applied.
%   % \refKey{/tcb/check odd page} is automatically set for precise even/odd page testing.
  
%   对于可分的盒子,如果当前盒子部分位于奇数页,则应用\meta{odd选项}。%
%   在偶数页上,应用\meta{even选项}。%
%   \refKey{/tcb/check odd page}会被自动设置,以用于精确的偶数/奇数页测试。
  
%   % In contrast to \refKey{/tcb/if odd page}, \refKey{/tcb/if odd page*} is used
%   % on \emph{every} partial box of a break sequences and not only on the
%   % \emph{first} box. Another difference is that \refKey{/tcb/if odd page*}
%   % is applied quite \emph{late} during option processing, while
%   % \refKey{/tcb/if odd page} is applied immediately.
%   同\refKey{/tcb/if odd page}对比, \refKey{/tcb/if odd page*} 作用于中断序列的每一个部分的盒子,而不仅仅是对
%   第一个盒子。%
%   另一个区别是, \refKey{/tcb/if odd page*} 在选项处理过程中应用较晚,而 \refKey{/tcb/if odd page} 立即生效。
  
%   % \refKey{/tcb/if odd page*} is implemented as \refKey{/tcb/if odd page}
%   % packed into \refKey{/tcb/extras}.
  
%   \refKey{/tcb/if odd page*} 是在 \refKey{/tcb/extras} 中的一个 \refKey{/tcb/if odd page} 的实现。
  
%   \begin{dispExample}
%   % \tcbuselibrary{breakable}
%   \begin{tcolorbox}[breakable,if odd page*={colback=yellow!50}{colback=red!50}]
%   这个可中断盒子在奇数页上是黄色的,在偶数页上是红色的。%
%   对于每一个部分盒子,测试都会重复执行, i.e. 对于长文本,这会得到黄色,红色,黄色,红色, \ldots\ 这样的序列。
%   \end{tcolorbox}
%   \end{dispExample}
%   \end{docTcbKey}

  


% \begin{docTcbKey}[][doc new=2016-11-18]{if odd page or oneside*}{=\marg{odd options}\marg{even options}}{style, no default}
%   % For onesided documents, the \meta{odd options} are applied always.
%   % For twosided documents, this style is identical to \refKey{/tcb/if odd page*}.

% 对于单开页文档,总是应用\meta{odd选项}。%
% 对于双开页文档,此项与\refKey{/tcb/if odd page*}相同。
% \end{docTcbKey}

  

% % \clearpage
% \begin{docCommand}[doc new=2015-11-13]{tcbifoddpage}{\marg{odd code}\marg{even code}}
%   % If the current box is on an odd page, the \meta{odd code} is executed.
%   % On an even page, the \meta{even code} is executed.
%   % For precise even/odd page testing, the \refKey{/tcb/check odd page} has to be
%   % set manually inside the box options.
  
%   如果当前盒子位于奇数页上,则执行\meta{odd code}。%
%   在偶数页上,执行\meta{even code}。%
%   对于精确的偶数/奇数页测试,\refKey{/tcb/check odd page}必须在盒子的选项中手动设置。
  
%   % The macro \refCom{tcbifoddpage} can be used inside underlay, overlay, or watermark code to
%   % test if the box is on an odd page. This will work also for boxes in a break sequence.
%   宏\refCom{tcbifoddpage}可以在底层、覆盖或水印代码中使用,以测试框是否在奇数页上。这也适用于中断序列中的盒子。
  
%   % The macro can also be used inside the box \textbf{content text}. For unbreakable boxes,
%   % the correct page test is applied.
%   % But for \refKey{/tcb/breakable} boxes, \refCom{tcbifoddpage}
%   % will always give the result for the page of the \emph{first} box inside
%   % the box \textbf{content text}. If needed, the methods from the packages
%   % |changepage| or |ifoddpage| could be used here.
%   宏也可以在盒子的\textbf{内容文本}内使用。对于不可分的盒子,应用正确的页面测试。%
%   但是对于\refKey{/tcb/breakable}的盒子, \refCom{tcbifodpage}将始终给出盒子的内容文本的第一部分所在页面的结果。 如果需要,这里可以使用包|changepage|或|ifoddpage|中的方法。
%   %To mention it again, for overlays, watermarks, etc, \refCom{tcbifoddpage} gives
%   %the correct page test.
  
%   \begin{dispExample}
%   \tcbset{colframe=blue!75!black,colback=white,fonttitle=\bfseries}
  
%   \begin{tcolorbox}[enhanced,check odd page,
%     title={Example for a box on an \tcbifoddpage{odd}{even} page},
%     watermark text={\tcbifoddpage{Odd}{Even} page!}]
%   \lipsum[1]
%   \end{tcolorbox}
%   \end{dispExample}
%   \end{docCommand}

  


% \begin{docCommand}[doc new=2016-11-18]{tcbifoddpageoroneside}{\marg{odd code}\marg{even code}}
%   % For onesided documents, the \meta{odd code} is executed always.
%   % For twosided documents, this macro is identical to \refCom{tcbifoddpage}.
% 对于单开页文档,总是执行\meta{odd code}。%
% 对于双开页文档,这个宏与\refCom{tcbifoddpage}相同。
% \end{docCommand}


% % \clearpage
% \begin{docCommand}[doc new=2015-11-13]{thetcolorboxnumber}{}
%   % This is a unique identifier (arabic number) for a tcolorbox. It is locally
%   % defined inside boxes and has no meaning outside. It is used for
%   % precise even/odd page testing, but may also be valuable for elaborate user
%   % code.
  
%   这是tcolorbox盒子的唯一标识符(阿拉伯数字)。 %
%   它在盒子内局部定义,在盒子外没有意义。%
%   它用于精确的偶数/奇数页测试,但对于复杂的用户代码也很有价值。
  
%   \begin{dispExample}
%   \begin{tcolorbox}[colback=yellow!5,title=Box \thetcolorboxnumber]
%   本盒号\thetcolorboxnumber.
%     \tcbox[on line,size=fbox]{本盒号\thetcolorboxnumber} 然后
%     \tcbox[on line,size=fbox]{本盒号\thetcolorboxnumber}.
%     本盒号 \thetcolorboxnumber.
%   \end{tcolorbox}
%   \end{dispExample}
%   \end{docCommand}

  

% \begin{docCommand}[doc new=2015-11-13]{thetcolorboxpage}{}
%   % This macro contains the expanded arabic page number of the current tcolorbox.
%   % It is locally defined inside boxes and has no meaning outside.
%   % It is precise only, if \refKey{/tcb/check odd page} was set.
  
%   这个宏包含当前tcolorbox盒子所在页的page计数器的阿拉伯数字页码。%
%   它在盒子内局部定义,在盒子外没有意义。%
%   只有当\refKey{/tcb/check odd page}被设置时,它才是精确的。
  
%   \begin{dispExample}
%   \begin{tcolorbox}[colback=yellow!5,check odd page,
%       title=Box on page~\thetcolorboxpage]
%   这个盒子位于~\thetcolorboxpage~页。
%   \end{tcolorbox}
%   \end{dispExample}
%   \end{docCommand}
  

%   % \clearpage
% % Externalization
% \subsection{Externalization}
% \begin{marker}
% % See \Fullref{sec:external} for the \mylib{external} library of |tcolorbox|.
% |tcolorbox|的\mylib{external}库请参见\Fullref{sec:external}。
% \end{marker}

% % If the \emph{externalization} library of the \texttt{tikz} package is used
% % and \refKey{/tcb/graphical environment} is set to |tikzpicture|,
% % a |tcolorbox| could trigger the externalization process which will arise
% % a compilation error.

% 如果使用了\texttt{tikz}包的\emph{externalization}库,%
% 且 \refKey{/tcb/graphical environment}被设置为|tikzpicture|,%
% 一个|tcolorbox|盒子可能触发外部进程,从而就可能产生编译错误。


% % To avoid this, there are two possible strategies:
% 为了避免这种情况,有两种可能的策略:
% \begin{itemize}
% \item 
% % Ensure, that |\tikzexternaldisable| is set before a |tcolorbox| is used.
% %   If you typically use the pattern |\tikzexternalenable| \textit{some picture} |\tikzexternaldisable|,
% %   there is nothing to care about.
% 确保在使用|tcolorbox|之前设置|\tikzexternaldisable|。%
% 如果你经常使用这种 |\tikzexternalenable| \textit{some picture} |\tikzexternaldisable| 模式,
% 没什么好担心的。
% \item 
% % If \emph{externalization} is enabled globally, use \refKey{/tcb/shield externalize} to
% %   shield any |tcolorbox|. The preamble code could look like this:
% 如果\emph{externalization}全局启用, 使用\refKey{/tcb/shield externalize}来保护|tcolorbox|。%
% 导言代码可以是这样的:
% \begin{dispListing}
% \usetikzlibrary{external}
% \tikzexternalize
% \tcbset{shield externalize}
% \end{dispListing}
% \end{itemize}

% \begin{docTcbKey}{shield externalize}{\colOpt{=true\textbar false}}{default |true|, initially |false|}
%   % If set to |true|, the drawing part of the |tcolorbox| is not being externalized
%   % which is a good thing at the current state of art. Nevertheless, if the
%   % |tcolorbox| contains a |tikzpicture|, this picture is still externalized.
%   % Pictures drawn with help of \refKey{/tcb/tikz upper} or alike are \emph{not}
%   % externalized.
  
%   如果设置为|true|, |tcolorbox|的绘图部分不会外部化,就目前的技术水平而言,这是一件好事。%
%   尽管如此,如果|tcolorbox|包含|tikzpicture|,这张图片仍然是外部化的。%
%   借助\refKey{/tcb/tikz upper}或类似工具绘制的图片是不外化的。
%   \end{docTcbKey}
  
%   \begin{marker}
%   % If a |tcolorbox| is used inside a node of an encircling |tikzpicture| which is externalized,
%   % do \emph{not} use |\tikzexternaldisable| in front of the |tcolorbox|.
%   % \refKey{/tcb/shield externalize} is deactivated automatically inside a |tikzpicture|.
%   如果|tcolorbox|在外部化的,被|tikzpicture|包围的节点内使用,%
%   在 |tcolorbox| 之前不要使用 |\tikzexternaldisable| 。%
%   \refKey{/tcb/shield externalize}在|tikzpicture|中自动停用。
%   \end{marker}

  

% \begin{marker}
%   % \refKey{/tcb/shield externalize} is applied for every following |tcolorbox|
%   % inside the current \TeX\ group and is not affected by \refKey{/tcb/reset}.
%   \refKey{/tcb/shield externalize}应用于当前\TeX\ 组内的每个|tcolorbox|,并且不受\refKey{/tcb/reset}的影响。
%   \end{marker}
  
%   \begin{docTcbKey}{external}{=\meta{file name}}{no default, initially unset}
%     % Convenience option which calls |\tikzsetnextfilename|\marg{file name}. Typically,
%     % it may be used inside the option list of a |tcolorbox| to set the
%     % externalization \meta{file name} for the first |tikzpicture| which is discovered
%     % \emph{inside} the box content.
%     % The package |tikz| \cite{tantau:tikz_and_pgf} or the library \mylib{skins} has to be loaded to use this option.
%     % Additionally, |\usetikzlibrary{external}| has to be used.
  
%   方便选项,调用|\tikzsetnextfilename|\marg{file name}。%
%   通常,它可以在|tcolorbox|的选项列表中使用,%
%   以设置在盒子内容中发现的,第一个|tikzpicture|的外部化\meta{file name}。%
  
%     % The package |tikz| \cite{tantau:tikz_and_pgf} or the library \mylib{skins} has to be loaded to use this option.
%     % Additionally, |\usetikzlibrary{external}| has to be used.
  
%   必须加载包|tikz|或库\mylib{skins}才能使用此选项。%
%   此外,必须设置|\usetikzlibrary{external}|。
%   \end{docTcbKey}



% \begin{docTcbKey}{remake}{\colOpt{=true\textbar false}}{default |true|, initially |false|}
%   % Convenience option which calls |/tikz/external/remake next|. Typically,
%   % it may be used inside the option list of a |tcolorbox| to force the remake
%   % of the first |tikzpicture| which is discovered \emph{inside} the box content.
%   % The package |tikz| \cite{tantau:tikz_and_pgf} or the library \mylib{skins} has to be loaded to use this option.
%   % Additionally, |\usetikzlibrary{external}| has to be used.

% 方便选项,调用|/tikz/external/remake next|. %
% 通常,它可以在|tcolorbox|的选项列表中使用,以强制重绘在盒子内容中发现的第一个|tikzpicture|。%
% 必须加载包|tikz|或\mylib{skins}库才能使用此选项。%
% 此外,必须设置|\usetikzlibrary{external}|。
% \end{docTcbKey}




% % \clearpage
% % Miscellaneous
% \subsection{杂项}
% \begin{docTcbKey}{reset}{}{no value, initially set}
% % Sets (nearly) all |tcolorbox| settings (including loaded libraries) back to their default values
% % \emph{plus} any settings given by \refCom{tcbsetforeverylayer}.
% % \refKey{/tcb/savedelimiter}, \refKey{/tcb/capture}, and
% % \refKey{/tcb/shield externalize} keep their values.
% % Also, all raster values (see \Vref{sec:raster}) are not resetted.

% 设置(几乎)所有|tcolorbox|设置(包括已加载的库)回到它们的默认值,\emph{加上} 由\refCom{tcbsetforeverylayer}给出的任何设置.%
% \refKey{/tcb/savedelimiter}, \refKey{/tcb/capture}, 和 \refKey{/tcb/shield externalize} 保持他们的值不变。%
% 此外,所有raster值(见\Vref{sec:raster})都不会重置。

% % This option is useful for boxes in boxes where the inner box should not inherit
% % the settings of the outer box.
% % Note that for boxes inside boxes the |reset| is done automatically, if the
% % standard settings of the package are used (v2.40 and above), see
% % Section \ref{subsec:everybox} from page \pageref{subsec:everybox}.
% 此项在嵌套的盒子中使用,可以重置从外部盒子继承过来的设置。%
% 注意,如果使用包的标准设置(v2.40及以上),嵌套的内部盒子, |reset|会自动完成, 见\pageref{subsec:everybox}页的\ref{subsec:everybox}小节。
% %See \refCom{tcbhighmath} for an example.
% \end{docTcbKey}




% \begin{docTcbKey}{code}{=\meta{code}}{no default, initially unset}
%   % The given \meta{code} is executed immediately. This option is useful
%   % to place some arbitrary code into an option list.
% 给定的\meta{code}立即执行。这个选项很有用,可以将一些任意代码放入选项列表中。
% \begin{exdispExample}{code}
% \tcbset{colback=red!5!white,colframe=red!75!black,
%   code={Useless at this spot but functional.},
%   fonttitle=\bfseries}

% \begin{tcolorbox}[code={\newcommand{\mycommand}{\textit{working}}},
%   title=My \mycommand\ title]
% This is a \textbf{tcolorbox}.
% \end{tcolorbox}
% \end{exdispExample}
% \end{docTcbKey}



% % \clearpage
% \begin{docTcbKey}[][doc new=2016-10-21]{void}{}{no value, initially unset}
%   % Annihilates the current |tcolorbox| as far as possible.
%   % Basically, this comments out the whole |tcolorbox| by using a key.
%   % If the option list of the current |tcolorbox| contains arbitrary code with global
%   % impact (like counter settings), these actions are not undone automatically.
%   % Nevertheless, the effects of \refKey{/tcb/phantom}, \refKey{/tcb/step},
%   % \refKey{/tcb/new/auto counter}, etc., are removed by \refKey{/tcb/void}.
% %TODO 再翻译
% 尽可能消灭当前|tcolorbox|。%
% 基本上,就是一个设置,注释了整个|tcolorbox|盒子。%
% 如果当前|tcolorbox|的选项列表包含具有全局影响的任意代码(如计数器设置),这些操作不会自动撤消。%
% 不过,\refKey{/tcb/phantom}, \refKey{/tcb/step},   \refKey{/tcb/new/auto counter}等的效果会被\refKey{/tcb/void}删除。

% \begin{exdispExample}{void}
% A%
%   \begin{tcolorbox}[
%       title=This box is completely removed by the following key,
%       void
%     ]
%   This is a \textbf{tcolorbox}.
%   \end{tcolorbox}
% B
% \end{exdispExample}

% \begin{marker}
%   % This option key cannot be applied for every situation.
%   % For example, if several box environments with the same environment name
%   % are nested, for the outer environment \refKey{/tcb/void} cannot be used,
%   % since the end of the inner environment will be misinterpreted as
%   % end of the outer environment. Also, \refKey{/tcb/void} cannot be used
%   % for environments wrapped with \refCom{tcolorboxenvironment}.

% 此选项键不能应用于每种情况。%
% 例如,如果嵌套了几个具有相同环境名称的盒子环境,外部环境不能使用\refKey{/tcb/void}。%
% 因为内部环境的结束会被误认为外部环境的终结。%
% 同样,\refKey{/tcb/void}不能用于由\refCom{tcolorboxenvironment}包装的环境。
% \end{marker}
% \end{docTcbKey}



% % nirvana | BrE nɪəˈvɑːnə, AmE nərˈvɑnə,nɪrˈvɑnə |
% % noun
% % ① uncountable Religion 涅槃 nièpán
% % ② countable (idyllic place) 极乐世界 jílè shìjiè ; (state) 无忧无虑的境界 wú yōu wú lǜ de jìngjiè

% \begin{docTcbKey}[][doc new=2019-03-01]{nirvana}{}{no value, initially unset}
%   % The contents of the current |tcolorbox| are processed including counter
%   % settings, but the box is just not drawn.
%   % Therefore, \refKey{/tcb/nirvana} is less radical than \refKey{/tcb/void}
%   % and several box environments can be nested without problems.

% 当前|tcolorbox|的内容被处理,包括计数器设置,只有盒子没有绘制。%
% 因此, \refKey{/tcb/nirvana} 没有 \refKey{/tcb/void} 那么彻底,%
% 而且应用在嵌套的多个盒子环境上,不出问题。
% \begin{exdispExample}{nirvana}
% A%
%   \begin{tcolorbox}[
%       title=This box is completely removed by the following key,
%       nirvana
%     ]
%   This is a \textbf{tcolorbox}.
%     \begin{tcolorbox}
%     Nested Box
%     \end{tcolorbox}
%   \end{tcolorbox}%
% B
% \end{exdispExample}
% \end{docTcbKey}
