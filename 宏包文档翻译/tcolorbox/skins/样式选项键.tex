
\setcounter{section}{10}
\setcounter{subsection}{0}
\setcounter{subsubsection}{0} 
\subsection{Style Option Keys\\样式选项键}\label{subsec:addstyleoptions}

The following style options are applicable for all skins which
use engines of type |path|, |pathfirst|, |pathmiddle|, or |pathlast|.
Especially, the skin \refSkinLe{enhanced} supports \emph{all} of them
and \refSkinLe{standard} \emph{none}.

以下样式选项适用于所有使用类型为 |path|、|pathfirst|、|pathmiddle| 或 |pathlast| 的引擎的皮肤。特别地,皮肤 \refSkinLe{enhanced} 支持 \emph{所有} 这些选项,而皮肤 \refSkinLe{standard} 则不支持任何一个。


\begin{docTcbKey}{frame style}{=\meta{\texttt{\upshape tikz} keys}}{style, no default}
The \meta{\texttt{\upshape tikz} keys} are used inside the |tikz| path command
for drawing the \emph{frame} of the box.\\
This option is available if the \refKeyLe{/tcb/frame engine} is set to
|path|, |pathfirst|, |pathmiddle|, or |pathlast|.
It is \emph{not} available for |standard|.

\meta{\texttt{\upshape tikz} keys}被用于在|tikz|路径命令中绘制盒子的\emph{框架}。\ 如果\refKeyLe{/tcb/frame engine}设置为|path|、|pathfirst|、|pathmiddle|或|pathlast|,则此选项可用。对于|standard|,它是\emph{不可用}的。
\begin{exdispExample*}{frame_style}{sbs,lefthand ratio=0.66}
\tcbset{colback=red!5!white,fonttitle=\bfseries}

\begin{tcolorbox}[enhanced,title=My title,
frame style={left color=red!75!black,
              right color=blue!75!black}]
This is a \textbf{tcolorbox}.
\tcblower
This is the lower part.
\end{tcolorbox}
\end{exdispExample*}
\end{docTcbKey}

\begin{docTcbKey}{frame style image}{=\meta{file name}}{no default, initially unset}
Fills the frame with an external image referenced by \meta{file name}.
For advanced features like blending of a picture with the background,
use \refKeyLe{/tcb/frame style} together with \refKeyLe{/tikz/fill stretch image}.

使用 \meta{文件名} 引用外部图像来填充帧。要使用高级功能,例如将图像与背景混合,可以同时使用 \refKeyLe{/tcb/frame style} 和 \refKeyLe{/tikz/fill stretch image}。
\begin{exdispExample*}{frame_style_image}{sbs,lefthand ratio=0.66}
\tcbset{colback=red!5!white,fonttitle=\bfseries}

\begin{tcolorbox}[enhanced,title=My title,
  frame style image=blueshade.png]
This is a \textbf{tcolorbox}.
\tcblower
This is the lower part.
\end{tcolorbox}
\end{exdispExample*}
\end{docTcbKey}

% %\clearpage
\begin{docTcbKey}{frame style tile}{=\marg{graphics options}\marg{file name}}{no default, initially unset}
Fills the frame with a tile pattern based on an external image referenced by \meta{file name}.
The \meta{graphics options} are given to the underlying \docAuxCommand*{includegraphics} command.
For advanced features like blending of a picture with the background,
use \refKeyLe{/tcb/frame style} together with \refKeyLe{/tikz/fill tile image}.

根据外部引用的文件名为\meta{file name}的图像,填充框架以形成平铺图案。 \meta{graphics options}提供给底层的\docAuxCommand*{includegraphics}命令。对于高级功能,如将图片与背景混合,可使用\refKeyLe{/tcb/frame style}和\refKeyLe{/tikz/fill tile image}。
\begin{exdispExample*}{frame_style_tile}{sbs,lefthand ratio=0.66}
\tcbset{colback=red!5!white,coltitle=red!30!black,
  opacityback=0.75,fonttitle=\bfseries}

\begin{tcolorbox}[enhanced,title=My title,
  frame style tile={width=1cm}{pink_marble.png}]
This is a \textbf{tcolorbox}.
\tcblower
This is the lower part.
\end{tcolorbox}
\end{exdispExample*}
\end{docTcbKey}

\begin{docTcbKey}{frame hidden}{}{style, no value}
This is a shortcut for |frame style={draw=none,fill=none}|.
Depending on the skin, this option switches off the drawing of the
frame.
Alternatively, use \refKeyLe{/tcb/frame empty}.

这是一个快捷方式,用于设置 |frame style={draw=none,fill=none}|。 根据皮肤不同,此选项可以关闭框架的绘制。 或者,可以使用 \refKeyLe{/tcb/frame empty}。
\begin{exdispExample*}{frame_hidden}{sbs,lefthand ratio=0.66}
\tcbset{colback=red!5!white,colframe=red!75!black,
fonttitle=\bfseries,coltitle=black}

\begin{tcolorbox}[enhanced,title=My title,
frame hidden]
This is a \textbf{tcolorbox}.
\tcblower
This is the lower part.
\end{tcolorbox}
\end{exdispExample*}
\end{docTcbKey}


\begin{docTcbKey}{interior style}{=\meta{\texttt{\upshape tikz} keys}}{style, no default}
The \meta{\texttt{\upshape tikz} keys} are used inside the |tikz| path command
for drawing the \emph{interior} of the box. They are used for the titled
and for the untitled version as well.\\
如果设置了 \refKeyLe{/tcb/interior titled engine} 或 \refKeyLe{/tcb/interior engine} 为 |path|、|pathfirst|、|pathmiddle| 或 |pathlast|,则可使用此选项。但对于 |standard|,此选项\emph{不可用}。

This option is available if the \refKeyLe{/tcb/interior titled engine}
or \refKeyLe{/tcb/interior engine} is set to
|path|, |pathfirst|, |pathmiddle|, or |pathlast|.
It is \emph{not} available for |standard|.\\
\meta{\texttt{\upshape tikz} keys} 用于在 |tikz| 路径命令中绘制盒子的\emph{内部}。无论是有标题还是无标题版本都会用到它们。
\begin{exdispExample*}{interior_style}{sbs,lefthand ratio=0.66}
\tcbset{colframe=red!75!black,fonttitle=\bfseries}

\begin{tcolorbox}[enhanced,title=My title,
interior style={left color=red!20!white,
                right color=yellow!50!white}]
This is a \textbf{tcolorbox}.
\tcblower
This is the lower part.
\end{tcolorbox}
\end{exdispExample*}
\end{docTcbKey}

% %\clearpage
\begin{docTcbKey}{interior style image}{=\meta{file name}}{no default, initially unset}
Fills the interior with an external image referenced by \meta{file name}.
For advanced features like blending of a picture with the background,
use \refKeyLe{/tcb/interior style} together with \refKeyLe{/tikz/fill stretch image}.

使用 \meta{文件名} 引用外部图像来填充内部。 对于像将图片与背景混合的高级功能, 请使用 \refKeyLe{/tcb/interior style} 与 \refKeyLe{/tikz/fill stretch image}。
\begin{exdispExample*}{interior_style_image}{sbs,lefthand ratio=0.66}
\tcbset{colframe=red!75!black,fonttitle=\bfseries}

\begin{tcolorbox}[enhanced,title=My title,
interior style image=goldshade.png]
This is a \textbf{tcolorbox}.
\tcblower
This is the lower part.
\end{tcolorbox}
\end{exdispExample*}
\end{docTcbKey}


\begin{docTcbKey}{interior style tile}{=\marg{graphics options}\marg{file name}}{no default, initially unset}
Fills the interior with a tile pattern based on an external image referenced by \meta{file name}.
The \meta{graphics options} are given to the underlying \docAuxCommand*{includegraphics} command.
For advanced features like blending of a picture with the background,
use \refKeyLe{/tcb/interior style} together with \refKeyLe{/tikz/fill tile image}.

根据外部的图像文件名\meta{file name},填充内部具有瓷砖图案。 \meta{graphics options}将传递给底层的\docAuxCommand*{includegraphics}命令。对于高级功能,例如将图片与背景混合,可以使用\refKeyLe{/tcb/interior style}和\refKeyLe{/tikz/fill tile image}。
\begin{exdispExample*}{interior_style_tile}{sbs,lefthand ratio=0.66}
\tcbset{colframe=red!75!black,fonttitle=\bfseries}

\begin{tcolorbox}[enhanced,title=My title,
interior style tile={width=2cm}{crinklepaper.png}]
This is a \textbf{tcolorbox}.
\tcblower
This is the lower part.
\end{tcolorbox}
\end{exdispExample*}
\end{docTcbKey}

\begin{docTcbKey}{interior hidden}{}{style, no value}
This is a shortcut for |interior style={draw=none,fill=none}|.
Depending on the skin, this option switches off the drawing of the
interior.
Alternatively, use \refKeyLe{/tcb/interior empty} and/or \refKeyLe{/tcb/interior titled empty}.

这是一种快捷方式,用于设置|interior style={draw=none,fill=none}|。根据不同的皮肤,此选项将关闭内部的绘制。或者,可以使用\refKeyLe{/tcb/interior empty}和/或\refKeyLe{/tcb/interior titled empty}。
\begin{exdispExample*}{interior_hidden}{sbs,lefthand ratio=0.66}
\tcbset{frame style={top color=red!20!white,
bottom color=red!20!white!75!black},
fonttitle=\bfseries,coltitle=black}

\begin{tcolorbox}[enhanced,title=My title,
interior hidden]
This is a \textbf{tcolorbox}.
\tcblower
This is the lower part.
\end{tcolorbox}
\end{exdispExample*}
\end{docTcbKey}

%\clearpage
\begin{docTcbKey}{segmentation style}{=\meta{\texttt{\upshape tikz} keys}}{style, no default}
The \meta{\texttt{\upshape tikz} keys} are used inside the |tikz| path command
for drawing the \emph{segmentation} line of the box.\\
This option is available if the \refKeyLe{/tcb/segmentation engine}
is set to |path|.
It is \emph{not} available for |standard|.

\meta{\texttt{\upshape tikz} keys}被用于|tikz|路径命令中,用于绘制盒子的\emph{分割}线。\ 如果设置了\refKeyLe{/tcb/segmentation engine}为|path|,则可用此选项。但是对于|standard|不可用。
\begin{exdispExample*}{segmentation_style}{sbs,lefthand ratio=0.66}
\tcbset{colback=red!5!white,colframe=red!75!black,
fonttitle=\bfseries}

\begin{tcolorbox}[enhanced,title=My title,
segmentation style={double=white,draw=blue,
                double distance=1pt,solid}]
This is a \textbf{tcolorbox}.
\tcblower
This is the lower part.
\end{tcolorbox}
\end{exdispExample*}
\end{docTcbKey}

\begin{docTcbKey}{segmentation hidden}{}{style, no value}
This is a shortcut for |segmentation style={draw=none,fill=none}|.
Depending on the skin, this option switches off the drawing of the
segmentation line. See also \refKeyLe{/tcb/lower separated} which
has the same effect for most skins.
Alternatively, use \refKeyLe{/tcb/segmentation empty}.

这是一个快捷方式,用于设置 |segmentation style={draw=none,fill=none}|。 根据皮肤不同,此选项将关闭分隔线的绘制。另请参见 \refKeyLe{/tcb/lower separated},该选项对大多数皮肤具有相同的效果。 或者,使用 \refKeyLe{/tcb/segmentation empty}。
\begin{exdispExample*}{segmentation_hidden}{sbs,lefthand ratio=0.66}
\tcbset{colback=red!5!white,colframe=red!75!black,
fonttitle=\bfseries}

\begin{tcolorbox}[title=My title,
enhanced,segmentation hidden]
This is a \textbf{tcolorbox}.
\tcblower
This is the lower part.
\end{tcolorbox}
\end{exdispExample*}
\end{docTcbKey}


\begin{docTcbKey}{title style}{=\meta{\texttt{\upshape tikz} keys}}{style, no default}
The \meta{\texttt{\upshape tikz} keys} are used inside the |tikz| path command
for drawing the \emph{title area} of the box.\\
This option is available if the \refKeyLe{/tcb/title engine} is set to
|path|, |pathfirst|, |pathmiddle|, or |pathlast|.
It is \emph{not} available for |standard|.

\meta{\texttt{\upshape tikz} keys} 用于在 |tikz| 路径命令中绘制盒子的 \emph{标题区域}。\\ 如果将 \refKeyLe{/tcb/title engine} 设置为 |path|、|pathfirst|、|pathmiddle| 或 |pathlast|,则可以使用此选项。对于 |standard|,则\emph{不可用}。
\begin{exdispExample*}{title_style}{sbs,lefthand ratio=0.66}
\tcbset{colback=red!5!white,colframe=red!75!black,
coltitle=blue!50!black,fonttitle=\bfseries}

\begin{tcolorbox}[enhanced,title=My title,
title style={left color=blue!15!yellow,
              right color=red!85!black}]
This is a \textbf{tcolorbox}.
\tcblower
This is the lower part.
\end{tcolorbox}
\end{exdispExample*}
\end{docTcbKey}

%\clearpage
\begin{docTcbKey}{title style image}{=\meta{file name}}{no default, initially unset}
Fills the title area with an external image referenced by \meta{file name}.
For advanced features like blending of a picture with the background,
use \refKeyLe{/tcb/title style} together with \refKeyLe{/tikz/fill stretch image}.

使用\meta{文件名}引用的外部图像填充标题区域。 对于高级功能,例如将图片与背景混合,可以使用\refKeyLe{/tcb/title style}和\refKeyLe{/tikz/fill stretch image}。
\begin{exdispExample*}{title_style_image}{sbs,lefthand ratio=0.66}
\tcbset{colback=blue!5!white,colframe=blue!75!black,
fonttitle=\bfseries}

\begin{tcolorbox}[enhanced,title=My title,
title style image=blueshade.png]
This is a \textbf{tcolorbox}.
\tcblower
This is the lower part.
\end{tcolorbox}
\end{exdispExample*}
\end{docTcbKey}

\begin{docTcbKey}{title style tile}{=\marg{graphics options}\marg{file name}}{no default, initially unset}
Fills the title area with a tile pattern based on an external image referenced by \meta{file name}.
The \meta{graphics options} are given to the underlying \docAuxCommand*{includegraphics} command.
For advanced features like blending of a picture with the background,
use \refKeyLe{/tcb/title style} together with \refKeyLe{/tikz/fill tile image}.

使用外部图像的平铺图案填充标题区域,该图像由\meta{文件名}引用。 给底层的\docAuxCommand*{includegraphics}命令提供\meta{图形选项}。 对于像将图片与背景混合等高级功能,请使用\refKeyLe{/tcb/title style}和\refKeyLe{/tikz/fill tile image}。
\begin{exdispExample*}{title_style_tile}{sbs,lefthand ratio=0.66}
\tcbset{colback=red!5!white,colframe=red!75!black,
coltitle=blue!50!black,fonttitle=\bfseries}

\begin{tcolorbox}[enhanced,title=My title,
title style tile={width=1cm}{pink_marble.png}]
This is a \textbf{tcolorbox}.
\tcblower
This is the lower part.
\end{tcolorbox}
\end{exdispExample*}
\end{docTcbKey}


\begin{docTcbKey}{title hidden}{}{style, no value}
This is a shortcut for |title style={draw=none,fill=none}|.
Depending on the skin, this option switches off the drawing of the
title background. See also \refKeyLe{/tcb/title filled} for a similar effect.
Alternatively, use \refKeyLe{/tcb/title empty}.

这是一个 |title style={draw=none,fill=none}| 的快捷方式。 根据皮肤不同,此选项可以关闭标题背景的绘制。也可以参见 \refKeyLe{/tcb/title filled},具有类似的效果。 或者,使用 \refKeyLe{/tcb/title empty}。
\begin{exdispExample*}{title_hidden}{sbs,lefthand ratio=0.66}
\tcbset{colback=red!5!white,colframe=red!75!black,
fonttitle=\bfseries}
\begin{tcolorbox}[title=My title,
enhanced,title hidden]
This is a \textbf{tcolorbox}.
\tcblower
This is the lower part.
\end{tcolorbox}
\end{exdispExample*}
\end{docTcbKey}

\begin{docTcbKey}[][doc new=2015-01-14]{titlerule style}{=\meta{\texttt{\upshape tikz} keys}}{style, no default}
The \meta{\texttt{\upshape tikz} keys} are used to draw a title rule,
i.e.\ a rule below the optional title. The width of the rule is controlled
by \refKeyLe{/tcb/titlerule}. It may be set directly to a smaller width
to create mixed effects with the standard rule.
This option is implemented as an \refKeyLe{/tcb/underlay}. Thus, it is not
available for \refSkinLe{standard} and \refSkinLe{standard jigsaw}, but for
all other skins, e.g.\ \refSkinLe{enhanced}.
As an underlay, this option can be used multiple times and is removed
by \refKeyLe{/tcb/no underlay}.

\meta{\texttt{\upshape tikz} keys} 用于绘制标题的线,即可选标题下方的线。线的宽度由 \refKeyLe{/tcb/titlerule} 控制。它可以直接设置为较小的宽度,以创建标准线的混合效果。此选项实现为 \refKeyLe{/tcb/underlay}。因此,它不适用于 \refSkinLe{standard} 和 \refSkinLe{standard jigsaw},但适用于所有其他皮肤,例如 \refSkinLe{enhanced}。作为底层,此选项可以多次使用,并通过 \refKeyLe{/tcb/no underlay} 移除。
\begin{exdispExample*}{titlerule_style_1}{sbs,lefthand ratio=0.66}
\begin{tcolorbox}[enhanced,
colback=red!5!white,colframe=red!75!black,
colbacktitle=red!50!yellow,fonttitle=\bfseries,
title=My title,
titlerule=1mm,
titlerule style=yellow  ]
This is a \textbf{tcolorbox}.
\end{tcolorbox}
\end{exdispExample*}

\begin{exdispExample*}{titlerule_style_2}{sbs,lefthand ratio=0.66}
\begin{tcolorbox}[enhanced,
colback=red!5!white,colframe=red!75!black,
colbacktitle=red!50!yellow,fonttitle=\bfseries,
title=My title,
titlerule=1mm,
titlerule style={yellow,line width=0.5mm}  ]
This is a \textbf{tcolorbox}.
\end{tcolorbox}
\end{exdispExample*}

\begin{exdispExample*}{titlerule_style_3}{sbs,lefthand ratio=0.66}
\begin{tcolorbox}[enhanced,
colback=red!10!white,colframe=red!75!black,
colbacktitle=red!50!yellow,fonttitle=\bfseries,
frame hidden,
title=My title,
boxrule=0pt,titlerule=1mm,
titlerule style=red!50!black  ]
This is a \textbf{tcolorbox}.
\end{tcolorbox}
\end{exdispExample*}

\begin{exdispExample*}{titlerule_style_4}{sbs,lefthand ratio=0.66}
%\usetikzlibrary{arrows.meta}
\begin{tcolorbox}[empty,
  coltitle=red!75!black,fonttitle=\bfseries,
  borderline horizontal={0.5mm}{0pt}{red!50!white},
  title=My title,
  titlerule style={red,
    arrows = {Hooks[arc=270]-Hooks[arc=270]}} ]
This is a \textbf{tcolorbox}.
\end{tcolorbox}
\end{exdispExample*}
\end{docTcbKey}

The combined \tikzname\ style applied to frame, interior, and title
background can used by authors in customizing code.

作者可以使用组合的\tikzname\ 样式来自定义代码中的边框、内部和标题背景。

\begin{docTikzKey}{tcb fill frame}{}{style, no value}
This is a \tikzname\ style which is finally applied to the \emph{frame}
of the box.

这是一个 \tikzname 的样式,最终应用于盒子的框架。
\begin{exdispExample*}{tcb_fill_frame}{sbs,lefthand ratio=0.66}
% \tcbuselibrary{hooks}
\tcbset{enhanced,colback=red!5!white,
  colframe=red!75!black,fonttitle=\bfseries,
  frame code app={\path[tcb fill frame]
    ([yshift=-2mm]frame.north)
        circle (8mm); }  }

\begin{tcolorbox}[title=My title]
This is a \textbf{tcolorbox}.
\tcblower
This is the lower part.
\end{tcolorbox}
\end{exdispExample*}
\end{docTikzKey}


\begin{docTikzKey}{tcb fill interior}{}{style, no value}
This is a \tikzname\ style which is finally applied to the \emph{interior}
of the box.

这是一个\tikzname 的样式,最终应用在盒子的\emph{内部}。
\begin{exdispExample*}{tcb_fill_interior}{sbs,lefthand ratio=0.66}
% \tcbuselibrary{hooks}
\tcbset{enhanced,colback=red!5!white,
colframe=red!75!black,fonttitle=\bfseries,
interior titled code app={\path[tcb fill interior]
    ([yshift=-0.1pt]interior.north east)
  --([yshift=3pt]interior.north)
  --([yshift=-0.1pt]interior.north west)
  --cycle;}  }

\begin{tcolorbox}[title=My title]
This is a \textbf{tcolorbox}.
\tcblower
This is the lower part.
\end{tcolorbox}
\end{exdispExample*}
\end{docTikzKey}


\begin{docTikzKey}{tcb fill title}{}{style, no value}
This is a \tikzname\ style which is finally applied to the \emph{title area}
of the box.

这是一种 \tikzname\ 样式,最终应用于盒子的\emph{标题区域}。
\begin{exdispExample*}{tcb_fill_title}{sbs,lefthand ratio=0.66}
% \tcbuselibrary{hooks}
\tcbset{enhanced,colback=red!5!white,
colframe=red!75!black,fonttitle=\bfseries,
colbacktitle=blue!75!black,
title code app={\path[tcb fill title]
  (title) circle (5mm); }  }

\begin{tcolorbox}[title=My title]
This is a \textbf{tcolorbox}.
\tcblower
This is the lower part.
\end{tcolorbox}
\end{exdispExample*}
\end{docTikzKey}