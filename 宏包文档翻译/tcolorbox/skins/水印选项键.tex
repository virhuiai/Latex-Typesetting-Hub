The following watermark options are applicable for all skins which
use |tikzpicture| as \refKeyLe{/tcb/graphical environment}.
Therefore, the skin \refSkin{standard} does not support these watermarks,
but all other skins, e.\,g.\ \refSkin{enhanced}.

以下水印选项适用于所有使用 |tikzpicture| 作为 \refKeyLe{/tcb/graphical environment} 的皮肤。因此,皮肤 \refSkin{standard} 不支持这些水印,但所有其他皮肤,如 \refSkin{enhanced},都支持。

\begin{marker}
The watermark options rely on the more general overlay options described in
Section \ref{subsec:overlays} from page \pageref{subsec:overlays}.
Therefore, \emph{watermarks} and \emph{overlays} cannot be used mixed.
But a mixture is possible with the \mylib{hooks} library, see Section \ref{sec:hooks}.

水印选项依赖于第\pageref{subsec:overlays}页的第\ref{subsec:overlays}节中描述的更通用的叠加选项。因此,“水印”和“叠加”不能混用。但可以使用\mylib{hooks}库进行混合,有关详细信息,请参见第\ref{sec:hooks}节。
\end{marker}


\begin{docTcbKey}{watermark text}{=\meta{text}}{no default, initially unset}
Writes some \meta{text} in the center of the interior region of a |tcolorbox|.
This \meta{text} is written \emph{after} the
frame and interior are drawn and \emph{before} the text content is drawn.
It is zoomed or stretched according the values of
\refKeyLe{/tcb/watermark zoom} or \refKeyLe{/tcb/watermark stretch}.

在 |tcolorbox| 的内部区域中央写入一些 \meta{text}。 这个 \meta{text} 是在边框和内部绘制之后、文本内容绘制之前写入的。 它根据 \refKeyLe{/tcb/watermark zoom} 或 \refKeyLe{/tcb/watermark stretch} 的值进行缩放或拉伸。
\begin{dispExample}
\tcbset{colback=red!5!white,colframe=red!75!black,fonttitle=\bfseries}

\begin{tcolorbox}[enhanced,title=My title,watermark text=My Watermark]
\lipsum[1]
\tcblower
\lipsum[2]
\end{tcolorbox}
\end{dispExample}
\end{docTcbKey}

\enlargethispage*{1cm}

\begin{docTcbKey}{watermark text on}{=\meta{part} is \meta{text}}{no default, initially unset}
This option writes some \meta{text} in the center of the interior region of a |tcolorbox|
as described for \refKeyLe{/tcb/watermark text}.
But this is done only for boxes named \meta{part} of a break sequence, see
\refKeyLe{/tcb/breakable}.\\ 
Feasible values for \meta{part} are:

该选项在|tcolorbox|的内部区域的中心写入一些\meta{text},如\refKeyLe{/tcb/watermark text}所述。但只对命名为\meta{part}的分页序列的盒子执行此操作,请参见\refKeyLe{/tcb/breakable}。可行的\meta{part}值为
\begin{DescriptionL}{\docValue{first and middle}}
% \item[\docValue{broken}] all broken box parts,
% \item[\docValue{unbroken}] unbroken boxes only,
% \item[\docValue{first}] first parts of a break sequence,
% \item[\docValue{middle}] middle parts of a break sequence,
% \item[\docValue{last}] last parts of a break sequence,
% \item[\docValue{unbroken and first}] unbroken boxes and first parts of a break sequence,
% \item[\docValue{middle and last}] middle and last parts of a break sequence.
% \item[\docValue{first and middle}] first and middle parts of a break sequence.

\item\docValue{broken}: all broken box parts,
\\所有破损的盒子部分,
\item\docValue{unbroken}: unbroken boxes only,
\\仅未破损的盒子,
\item\docValue{first}: first parts of a break sequence,
\\分页序列的第一部分,
\item\docValue{middle}: middle parts of a break sequence,
\\分页序列的中间部分,
\item\docValue{last}: last parts of a break sequence,
\\分页序列的最后部分,
\item\docValue{unbroken and first}: unbroken boxes and first parts of a break sequence,
\\未分页的盒子和分页序列的第一部分,
\item[\docValue{middle and last}]: middle and last parts of a break sequence.
\\分页序列的中间和最后部分。
\item[\docValue{first and middle}] first and middle parts of a break sequence.
\\分页序列的第一部分和中间部分。
\end{DescriptionL}
\end{docTcbKey}


%\clearpage


\begin{docTcbKey}{watermark graphics}{=\meta{file name}}{no default, initially unset}
Draws an external picture referenced by \meta{file name}
in the center of the interior region of a |tcolorbox|.
The picture is drawn \emph{after} the
frame and interior are drawn and \emph{before} the text content is drawn.
It is zoomed or stretched according the values of
\refKeyLe{/tcb/watermark zoom} or \refKeyLe{/tcb/watermark stretch}.

在 |tcolorbox| 内部区域的中心位置绘制一个外部图片,该图片由 \meta{file name} 引用。这个图片是在框架和内部区域绘制之后,文本内容绘制之前绘制的。它会根据 \refKeyLe{/tcb/watermark zoom} 或 \refKeyLe{/tcb/watermark stretch} 的值进行缩放或拉伸。
\begin{dispExample}
\tcbset{colback=red!5!white,colframe=red!75!black,fonttitle=\bfseries}

\begin{tcolorbox}[enhanced,title=My title,watermark graphics=Basilica_5.png,
watermark opacity=0.15]
\lipsum[1-2]
\tcblower
This example uses a public domain picture from\\
\url{http://commons.wikimedia.org/wiki/File:Basilica_5.png}
\end{tcolorbox}
\end{dispExample}
\end{docTcbKey}


\begin{docTcbKey}{watermark graphics on}{=\meta{part} is \meta{file name}}{no default, initially unset}
This option draws a picture referenced by \meta{file name} in the center of the interior region of a |tcolorbox|
as described for \refKeyLe{/tcb/watermark graphics}.
But this is done only for boxes named \meta{part} of a break sequence, see
\refKeyLe{/tcb/breakable}.\\ 
Feasible values for \meta{part} are:
\begin{itemize}
\item\docValue{broken}: all broken box parts,
\item\docValue{unbroken}: unbroken boxes only,
\item\docValue{first}: first parts of a break sequence,
\item\docValue{middle}: middle parts of a break sequence,
\item\docValue{last}: last parts of a break sequence,
\item\docValue{unbroken and first}: unbroken boxes and first parts of a break sequence,
\item\docValue{middle and last}: middle and last parts of a break sequence.
\end{itemize}
\end{docTcbKey}



%\clearpage
\begin{docTcbKey}{watermark tikz}{=\meta{graphical code}}{no default, initially unset}
Draws the given |tikz| \meta{graphical code}
in the center of the interior region of a |tcolorbox|.
The code is executed \emph{after} the
frame and interior are drawn and \emph{before} the text content is drawn.
The result is zoomed or stretched according the values of
\refKeyLe{/tcb/watermark zoom} or \refKeyLe{/tcb/watermark stretch}.

在 |tcolorbox| 的内部区域中心绘制给定的 |tikz| \meta{图形代码}。 该代码在边框和内部绘制之后执行,但在文本内容绘制之前执行。 结果根据 \refKeyLe{/tcb/watermark zoom} 或 \refKeyLe{/tcb/watermark stretch} 的值进行缩放或拉伸。
\begin{dispExample}
\tcbset{colback=red!5!white,colframe=red!75!black,fonttitle=\bfseries}

\begin{tcolorbox}[enhanced,title=My title,
watermark tikz={\draw[line width=2mm] circle (1cm)
node{\fontfamily{ptm}\fontseries{b}\fontsize{20mm}{20mm}\selectfont ?};}]
\lipsum[1]
\tcblower
\lipsum[2]
\end{tcolorbox}
\end{dispExample}
\end{docTcbKey}



\begin{docTcbKey}{watermark tikz on}{=\meta{part} is \meta{graphical code}}{no default, initially unset}
This option draws the given |tikz| \meta{graphical code} in the center of the interior region of a |tcolorbox|
as described for \refKeyLe{/tcb/watermark tikz}.
But this is done only for boxes named \meta{part} of a break sequence, see
\refKeyLe{/tcb/breakable}.\\ 
Feasible values for \meta{part} are:

该选项将给定的 |tikz| \meta{图形代码} 绘制在 |tcolorbox| 的内部区域中心,就像 \refKeyLe{/tcb/watermark tikz} 中描述的那样。但是,这仅适用于分页序列的名称为 \meta{part} 的盒子,参见 \refKeyLe{/tcb/breakable}。\ 可行的 \meta{part} 值包括:
\begin{itemize}
\item\docValue{broken}: all broken box parts,
\item\docValue{unbroken}: unbroken boxes only,
\item\docValue{first}: first parts of a break sequence,
\item\docValue{middle}: middle parts of a break sequence,
\item\docValue{last}: last parts of a break sequence,
\item\docValue{unbroken and first}: unbroken boxes and first parts of a break sequence,
\item\docValue{middle and last}: middle and last parts of a break sequence.
\end{itemize}
\end{docTcbKey}


\begin{docTcbKey}{no watermark}{}{style, no default, initially set}
Removes the watermark if set before. This is an alias for
\refKeyLe{/tcb/no overlay}.

如果之前设置了水印,则移除水印。这是 \refKeyLe{/tcb/no overlay} 的别名。
\end{docTcbKey}


%\clearpage
\begin{docTcbKey}{watermark opacity}{=\meta{fraction}}{no default, initially |1.00|}
Sets the opacity value $\in[0,1]$ for a watermark.

为水印设置不透明度值,范围为$[0,1]$。
\begin{dispExample}
\tcbset{enhanced,colback=red!5!white,colframe=red!75!black,fonttitle=\bfseries,
watermark text=Watermark,nobeforeafter,width=(\linewidth-2mm)/2}

\begin{tcolorbox}[title=Opacity 1.00,watermark opacity=1.00]
\lipsum[2]
\end{tcolorbox}\hfill%
\begin{tcolorbox}[title=Opacity 0.50,watermark opacity=0.50]
\lipsum[2]
\end{tcolorbox}%
\end{dispExample}
\end{docTcbKey}

% \enlargethispage*{1cm}

\begin{docTcbKey}{watermark zoom}{=\meta{fraction}}{no default, initially |0.75|}
Sets the zoom value for a watermark. The zoom respects the aspect ratio.
The value $1.0$ means to fill the whole box until the watermark touches the frame.

设置水印的缩放值。缩放会保持宽高比。 $1.0$ 的值意味着填满整个框,直到水印触碰到边框。
\begin{dispExample}
\tcbset{enhanced,colback=red!5!white,colframe=red!75!black,fonttitle=\bfseries,
watermark text=Watermark,nobeforeafter,width=(\linewidth-2mm)/2}

\begin{tcolorbox}[title=Zoom 1.0,watermark zoom=1.0]
\lipsum[2]
\end{tcolorbox}\hfill%
\begin{tcolorbox}[title=Zoom 0.5,watermark zoom=0.5]
\lipsum[2]
\end{tcolorbox}%
\end{dispExample}
\end{docTcbKey}

%\clearpage

\begin{docTcbKey}{watermark shrink}{=\meta{fraction}}{no default, initially unset}
Identically to \refKeyLe{/tcb/watermark zoom}, but the watermark
never gets enlarged. Thus, the watermark keeps its original size or is shrunk.

与\refKeyLe{/tcb/watermark zoom}完全相同,但水印永远不会被放大。因此,水印保持其原始大小或被缩小。
\end{docTcbKey}


\begin{docTcbKey}{watermark overzoom}{=\meta{fraction}}{no default, initially unset}
Sets the overzoom value for a watermark. The overzoom respects the aspect ratio.
The value $1.0$ means to fill the whole box until the watermark touches
all four sides of the frame.

设置水印的过度缩放值。过度缩放会保持长宽比例。当值为$1.0$时,水印将填满整个框架,直到触碰到框架的四个边缘。
\begin{dispExample}
\tcbset{enhanced,colback=white,colframe=blue!50!black,fonttitle=\bfseries,
watermark opacity=0.5,
watermark graphics=lichtspiel.jpg,nobeforeafter,width=(\linewidth-2mm)/2}

\begin{tcolorbox}[title=Zoom 1.0,watermark zoom=1.0]
\lipsum[1]
\end{tcolorbox}\hfill%
\begin{tcolorbox}[title=Overzoom 1.0,watermark overzoom=1.0]
\lipsum[1]
\end{tcolorbox}%
\end{dispExample}
\end{docTcbKey}

\begin{marker}
If a \refKeyLe{/tcb/watermark overzoom} value of |1.0| is used in connection
with invisible top and bottom rules which still have a thickness greater than |0pt|,
the space of these invisible rules may not be covered by the watermark.
For example, this situation may occur during the breaking of \refKeyLe{/tcb/enhanced} boxes.
To avoid this optical glitch, just set \refKeyLe{/tcb/pad at break} to any desired value.

如果在使用不可见的顶部和底部规则时,使用 \refKeyLe{/tcb/watermark overzoom} 值为 |1.0|,而这些规则的厚度仍大于 |0pt|,则这些不可见规则的空间可能不会被水印覆盖。例如,在断开 \refKeyLe{/tcb/enhanced} 盒子时可能会出现这种情况。为避免这种光学故障,只需将 \refKeyLe{/tcb/pad at break} 设置为任何所需的值。
\end{marker}

%\clearpage
\begin{docTcbKey}{watermark stretch}{=\meta{fraction}}{no default, initially unset}
Sets the stretch value for a watermark. The stretch value is applied to width
and height in relation to the box dimensions. It does not respect the aspect ratio.
The value $1.0$ means to fill the whole box.

设置水印的拉伸值。拉伸值是针对框尺寸的宽度和高度应用的,不考虑宽高比。值为$1.0$表示填满整个框。
\begin{dispExample}
\tcbset{enhanced,colback=white,colframe=blue!50!black,fonttitle=\bfseries,
watermark graphics=lichtspiel.jpg,watermark opacity=0.5,
nobeforeafter,width=(\linewidth-2mm)/2}

\begin{tcolorbox}[title=Stretch 1.00,watermark stretch=1.00]
\lipsum[2]
\end{tcolorbox}\hfill%
\begin{tcolorbox}[title=Stretch 0.50,watermark stretch=0.50]
\lipsum[2]
\end{tcolorbox}%
\end{dispExample}
\end{docTcbKey}

\begin{docTcbKey}{watermark color}{=\meta{color}}{no default, initially mixed background and frame color}
Sets the color for the watermark.

设置水印的颜色。
\begin{dispExample}
\tcbset{colback=red!5!white,colframe=red!75!black,fonttitle=\bfseries}

\begin{tcolorbox}[enhanced,title=My title,watermark text=My Watermark,
watermark color=yellow!50!red]
\lipsum[1]
\end{tcolorbox}
\end{dispExample}
\end{docTcbKey}

%\clearpage

\begin{docTcbKey}{clip watermark}{\colOpt{=true\textbar false}}{default |true|, initially |true|}
Sets the watermark to be clipped to the interior area.

将水印设置为被剪裁到内部区域。
\begin{dispExample}
\tcbset{enhanced,colback=white,colframe=blue!50!white,fonttitle=\bfseries,
watermark opacity=0.5,watermark stretch=1.00,arc=3mm,
watermark graphics=lichtspiel.jpg}

\begin{tcolorbox}[title=Clip (default),clip watermark]
\lipsum[1]
\end{tcolorbox}

\begin{tcolorbox}[title=No clip,clip watermark=false]
\lipsum[1]
\end{tcolorbox}%
\end{dispExample}
\end{docTcbKey}
