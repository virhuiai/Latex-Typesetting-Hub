\subsubsection{Options for the Boxed Title Box\\方框标题框的选项}

\begin{marker}
The boxed title options are implemented as an underlay, see \Vref{subsec:skinunderlay}.
Therefore, a boxed title is not drawn, if a skin does not support underlays
like \refSkin{standard}. Still, the room for the boxed
titles gets reserved in these cases.

方框标题选项是作为底层实现的,参见\Vref{subsec:skinunderlay}。因此,如果皮肤不支持底层实现,例如\refSkin{standard},则不会绘制方框标题。但在这些情况下,方框标题的空间仍被保留。
\end{marker}

\begin{marker}
A \tikzname\ node |title| is produced by a boxed title which can be used
inside \refKeyLe{/tcb/frame code}, \refKeyLe{/tcb/interior code},
underlays, overlays, and finishes.

一个 \tikzname 节点 |title| 是由一个带框标题产生的,可以在 \refKeyLe{/tcb/frame code}、\refKeyLe{/tcb/interior code}、下层、上层和结束中使用。
\end{marker}

\begin{marker}
A boxed title is almost always the first underlay. The only exceptions are
underlays defined by \refKeyLe{/tcb/underlay boxed title} which are drawn
before. Additionally, underlays defined by \refKeyLe{/tcb/underlay boxed title}
are only drawn, if a boxed title is actually set. They are ignored, if
there is no boxed title.

盒子标题几乎总是第一个底纹。唯一的例外是由\refKeyLe{/tcb/underlay boxed title}定义的底纹,在此之前绘制。此外,由\refKeyLe{/tcb/underlay boxed title}定义的底纹只有在实际设置了盒子标题时才绘制。如果没有盒子标题,它们将被忽略。
\end{marker}



\begin{docTcbKey}[][doc new=2016-02-26]{boxed title size}{=\meta{size}}{no default, initially |title|}
This setting defines the basic size for the title box. Further settings
can be applied using \refKeyLe{/tcb/boxed title style}.
Feasible values for \meta{size} are:

该设置定义标题框的基本大小。可以使用\refKeyLe{/tcb/boxed title style}进行其他设置。可行的\meta{size}值为: 
  
\begin{DescriptionL}{\docValue{standard}}
\item[\docValue{title}] Sets the size according to \refKeyLe{/tcb/size}|=|\docValue{title}.
\\根据\refKeyLe{/tcb/size}|=|\docValue{title}设置大小。
\item[\docValue{standard}] No size setting. Typically, this is identical to
\refKeyLe{/tcb/size}|=|\docValue{normal}.
\\没有大小设置。通常,这与\refKeyLe{/tcb/size}|=|\docValue{normal}相同。
\item[\docValue{copy}] The size values for a title of the base box are copied
for the title box.
\\复制基本框架的标题大小值到标题框中。
\end{DescriptionL}

\begin{dispExample}
% \tcbuselibrary{raster}
\begin{tcbraster}[raster columns=3,enhanced,boxrule=0.4pt,
title=My title,attach boxed title to top center]
\begin{tcolorbox}[boxed title size=title]
This is a \textbf{tcolorbox}.
\end{tcolorbox}
\begin{tcolorbox}[boxed title size=standard]
This is a \textbf{tcolorbox}.
\end{tcolorbox}
\begin{tcolorbox}[boxed title size=copy]
This is a \textbf{tcolorbox}.
\end{tcolorbox}
\end{tcbraster}
\end{dispExample}

\end{docTcbKey}


%\clearpage
\begin{docTcbKey}[][doc updated=2016-02-26]{boxed title style}{=\meta{options}}{style, initially empty}
By default, a boxed title is dimensioned with \refKeyLe{/tcb/size}|=|\docValue*{title}
and inherits the \refKeyLe{/tcb/skin} and \refKeyLe{/tcb/colframe} of the main box.
Also, the \refKeyLe{/tcb/colback} is inherited from the main \refKeyLe{/tcb/colbacktitle}.
Font and color of the title text are set as usual.
All other \meta{options} are set by the \refKeyLe{/tcb/boxed title style} key.
Since a boxed title is set by \refCom{tcbox}, all |tcolorbox| options are
applicable here. If \refKeyLe{/tcb/boxed title style} is used several times,
the \meta{options} are is appended.

默认情况下,带框标题的尺寸为 \refKeyLe{/tcb/size}|=|\docValue*{title},并继承主框的 \refKeyLe{/tcb/skin} 和 \refKeyLe{/tcb/colframe}。同时,\refKeyLe{/tcb/colback} 也从主框的 \refKeyLe{/tcb/colbacktitle} 继承。标题文本的字体和颜色设置与平常一样。所有其他的 \meta{options} 都由 \refKeyLe{/tcb/boxed title style} 键设置。由于带框标题是通过 \refCom{tcbox} 设置的,因此所有 |tcolorbox| 选项均适用于此处。如果多次使用 \refKeyLe{/tcb/boxed title style},则会附加 \meta{options}。

\begin{exdispExample*}{boxed_title_style_1}{sbs,lefthand ratio=0.66}
\begin{tcolorbox}[enhanced,title=My title,
fonttitle=\bfseries,coltitle=green!25!black,
attach boxed title to top center=
{yshift=-2mm,yshifttext=-1mm},
boxed title style={colframe=green!75!black,
colback=yellow!50!green}]
This is a \textbf{tcolorbox}.
\end{tcolorbox}
\end{exdispExample*}


\begin{exdispExample*}{boxed_title_style_2}{sbs,lefthand ratio=0.66}
\begin{tcolorbox}[enhanced,title=My title,
colframe=red!50!black,colback=red!10!white,
arc=1mm,colbacktitle=red!10!white,
fonttitle=\bfseries,coltitle=red!50!black,
attach boxed title to top text left=
{yshift=-0.50mm},
boxed title style={skin=enhancedfirst jigsaw,
size=small,arc=1mm,bottom=-1mm,
interior style={fill=none,
    top color=red!30!white,
    bottom color=red!20!white}}]
This is a \textbf{tcolorbox}.
\end{tcolorbox}
\end{exdispExample*}

\begin{exdispExample}{boxed_title_style_3}
\begin{tcolorbox}[enhanced,title=My title,
colframe=blue!50!black,colback=blue!10!white,colbacktitle=blue!5!yellow!10!white,
fonttitle=\bfseries,coltitle=black,attach boxed title to top center=
{yshift=-0.25mm-\tcboxedtitleheight/2,yshifttext=2mm-\tcboxedtitleheight/2},
boxed title style={boxrule=0.5mm,
frame code={ \path[tcb fill frame] ([xshift=-4mm]frame.west)
-- (frame.north west) -- (frame.north east) -- ([xshift=4mm]frame.east)
-- (frame.south east) -- (frame.south west) -- cycle; },
interior code={ \path[tcb fill interior] ([xshift=-2mm]interior.west)
-- (interior.north west) -- (interior.north east)
-- ([xshift=2mm]interior.east) -- (interior.south east) -- (interior.south west)
-- cycle;}  }]
\lipsum[2]
\end{tcolorbox}
\end{exdispExample}


\begin{exdispExample}{boxed_title_style_4}
% \usepackage{varwidth}
\newtcolorbox{mybox}[2][]{enhanced,skin=enhancedlast jigsaw,
attach boxed title to top left={xshift=-4mm,yshift=-0.5mm},
fonttitle=\bfseries\sffamily,varwidth boxed title=0.7\linewidth,
colbacktitle=blue!45!white,colframe=red!50!black,
interior style={top color=blue!10!white,bottom color=red!10!white},
boxed title style={empty,arc=0pt,outer arc=0pt,boxrule=0pt},
underlay boxed title={
\fill[blue!45!white] (title.north west) -- (title.north east)
    -- +(\tcboxedtitleheight-1mm,-\tcboxedtitleheight+1mm)
    -- ([xshift=4mm,yshift=0.5mm]frame.north east) -- +(0mm,-1mm)
    -- (title.south west) -- cycle;
\fill[blue!45!white!50!black] ([yshift=-0.5mm]frame.north west)
    -- +(-0.4,0) -- +(0,-0.3) -- cycle;
\fill[blue!45!white!50!black] ([yshift=-0.5mm]frame.north east)
    -- +(0,-0.3) -- +(0.4,0) -- cycle;  },
title={#2},#1}

\begin{mybox}{My title}
\lipsum[2]
\end{mybox}
\end{exdispExample}


\begin{exdispExample}{boxed_title_style_5}
% \usepackage{varwidth}
\newtcolorbox{mybox}[2][]{enhanced,
attach boxed title to top left={xshift=1cm,yshift=-2mm},
fonttitle=\bfseries,varwidth boxed title=0.7\linewidth,
colbacktitle=green!45!white,coltitle=green!10!black,colframe=green!50!black,
interior style={top color=yellow!10!white,bottom color=green!10!white},
boxed title style={boxrule=0.75mm,colframe=white,
borderline={0.1mm}{0mm}{green!50!black},
borderline={0.1mm}{0.75mm}{green!50!black},
interior style={top color=green!10!white,bottom color=green!10!white,
    middle color=green!50!white},
drop fuzzy shadow},
title={#2},#1}

\begin{mybox}{My title}
\lipsum[2]
\end{mybox}
\end{exdispExample}


\begin{exdispExample}{boxed_title_style_6}
\newtcolorbox{flipbox}[2][]{
enhanced,colframe=blue!50!black,colback=yellow!5,fonttitle=\bfseries,
flip title={interior hidden},title={#2},#1}

\begin{flipbox}{My title}
\lipsum[2]
\end{flipbox}
\end{exdispExample}


\begin{exdispExample}{boxed_title_style_7}
% \usepackage{varwidth}
\newtcolorbox{mybox}[2][]{skin=enhancedlast jigsaw,interior hidden,
boxsep=0pt,top=0pt,colframe=red,coltitle=red!50!black,
fonttitle=\bfseries\sffamily,
attach boxed title to bottom center,
boxed title style={empty,boxrule=0.5mm},
varwidth boxed title=0.5\linewidth,
underlay boxed title={
\draw[white,line width=0.5mm]
    ([xshift=0.3mm-\tcboxedtitleheight*2,yshift=0.3mm]title.north west)
    --([xshift=-0.3mm+\tcboxedtitleheight*2,yshift=0.3mm]title.north east);
\path[draw=red,top color=white,bottom color=red!50!white,line width=0.5mm]
([xshift=0.25mm-\tcboxedtitleheight*2,yshift=0.25mm]title.north west)
cos +(\tcboxedtitleheight,-\tcboxedtitleheight/2)
sin +(\tcboxedtitleheight,-\tcboxedtitleheight/2)
-- ([xshift=0.25mm,yshift=0.25mm]title.south west)
-- ([yshift=0.25mm]title.south east)
cos +(\tcboxedtitleheight,\tcboxedtitleheight/2)
sin +(\tcboxedtitleheight,\tcboxedtitleheight/2); },
title={#2},#1}

\begin{mybox}{My title}
\lipsum[2]
\end{mybox}
\end{exdispExample}


\begin{exdispExample}{boxed_title_style_8}
% \usepackage{varwidth}
\newtcolorbox{mybox}[2][]{enhanced,
before skip=2mm,after skip=2mm,
colback=black!5,colframe=black!50,boxrule=0.2mm,
attach boxed title to top left={xshift=1cm,yshift*=1mm-\tcboxedtitleheight},
varwidth boxed title*=-3cm,
boxed title style={frame code={
\path[fill=tcbcolback!30!black]
    ([yshift=-1mm,xshift=-1mm]frame.north west)
    arc[start angle=0,end angle=180,radius=1mm]
    ([yshift=-1mm,xshift=1mm]frame.north east)
    arc[start angle=180,end angle=0,radius=1mm];
\path[left color=tcbcolback!60!black,right color=tcbcolback!60!black,
    middle color=tcbcolback!80!black]
    ([xshift=-2mm]frame.north west) -- ([xshift=2mm]frame.north east)
    [rounded corners=1mm]-- ([xshift=1mm,yshift=-1mm]frame.north east)
    -- (frame.south east) -- (frame.south west)
    -- ([xshift=-1mm,yshift=-1mm]frame.north west)
    [sharp corners]-- cycle;
},interior engine=empty,
},
fonttitle=\bfseries,
title={#2},#1}

\begin{mybox}[colbacktitle=green]{My title}
\lipsum[2]
\end{mybox}
\begin{mybox}[colbacktitle=red]{My title}
\lipsum[3]
\end{mybox}
\end{exdispExample}
\end{docTcbKey}



\begin{docTcbKey}[][doc new=2016-02-26]{no boxed title style}{}{style, initially set}
Removes all options which were set by \refKeyLe{/tcb/boxed title style}.

删除所有由\refKeyLe{/tcb/boxed title style}设置的选项。
\end{docTcbKey}


%\clearpage
\begin{docTcbKey}{hbox boxed title}{}{no value, initially set}
The title text content is captured with a horizontal box.
Especially, there are no linebreak possible.

标题文本内容是通过水平框捕捉的。 特别地,不能换行。
\begin{exdispExample*}{hbox_boxed_title}{sbs,lefthand ratio=0.66}
\newtcolorbox{mybox}[1]{hbox boxed title,
enhanced,attach boxed title to top center=
{yshift=-3mm,yshifttext=-1mm},
boxed title style={size=small,colback=red},
title={#1}}

\begin{mybox}{Short title}
This is a \textbf{tcolorbox}.
\end{mybox}\bigskip

\begin{mybox}{This title is not really very short}
This is a \textbf{tcolorbox}.
\end{mybox}
\end{exdispExample*}
\end{docTcbKey}


\begin{docTcbKey}{minipage boxed title}{\colOpt{=\meta{length}}}{initially unset}
The title text content is captured with a minipage with a width of \meta{length}.
By default, the resulting boxed title is somewhat smaller than the main box.

标题文本内容使用宽度为\meta{length}的minipage捕获。 默认情况下,结果为带框标题的大小略小于主框。
\begin{exdispExample*}{minipage_boxed_title}{sbs,lefthand ratio=0.66}
\newtcolorbox{mybox}[1]{minipage boxed title,
enhanced,attach boxed title to top center=
{yshift=-3mm,yshifttext=-1mm},
boxed title style={size=small,colback=red},
center title,title={#1}}

\begin{mybox}{Short title}
This is a \textbf{tcolorbox}.
\end{mybox}\bigskip

\begin{mybox}{This title is not really very short}
This is a \textbf{tcolorbox}.
\end{mybox}
\end{exdispExample*}
\end{docTcbKey}


\begin{docTcbKey}{minipage boxed title*}{\colOpt{=\meta{length}}}{initially unset}
The title text content is captured with a minipage with a width of main box width plus \meta{length}.
By default, the resulting boxed title is somewhat smaller than the main box.

标题文本内容使用宽度为主框宽度加上\meta{长度}的minipage来捕获。 默认情况下,生成的标题框比主框略小。
\begin{exdispExample*}{minipage_boxed_title_star}{sbs,lefthand ratio=0.66}
\newtcolorbox{mybox}[1]{minipage boxed title*=-2cm,
enhanced,attach boxed title to top center=
{yshift=-3mm,yshifttext=-1mm},
boxed title style={size=small,colback=red},
center title,title={#1}}

\begin{mybox}{Short title}
This is a \textbf{tcolorbox}.
\end{mybox}\bigskip

\begin{mybox}{This title is not really very short}
This is a \textbf{tcolorbox}.
\end{mybox}
\end{exdispExample*}
\end{docTcbKey}


%\clearpage
\begin{docTcbKey}{tikznode boxed title}{=\meta{options}}{initially unset}
The title text content is captured with a \tikzname\ node with given \tikzname\ \meta{options}.
The text is centered by default

标题文本内容使用给定的\tikzname\ \meta{选项}捕获到一个\tikzname\ 节点中。 默认情况下,文本居中显示。
\begin{exdispExample*}{tikznode_boxed_title}{sbs,lefthand ratio=0.66}
\newtcolorbox{mybox}[1]{tikznode boxed title,
enhanced,attach boxed title to top center=
{yshift=-3mm,yshifttext=-1mm},
boxed title style={size=small,colback=red},
title={#1}}

\begin{mybox}{Short title}
This is a \textbf{tcolorbox}.
\end{mybox}\bigskip

\begin{mybox}{This title\\is not really\\very short}
This is a \textbf{tcolorbox}.
\end{mybox}
\end{exdispExample*}
\end{docTcbKey}


\begin{docTcbKey}{varwidth boxed title}{\colOpt{=\meta{length}}}{initially unset}
The title text content is captured with a |varwidth| environment with a width of \meta{length}.
This style needs the |varwidth| package \cite{arseneau:2011a} to be loaded manually.
By default, the resulting boxed title is somewhat smaller than the main box.

标题文本内容通过宽度为\meta{length}的|varwidth|环境捕获。 这种样式需要手动加载|varwidth|包\cite{arseneau:2011a}。 默认情况下,生成的标题框比主框稍小。
\begin{exdispExample*}{varwidth_boxed_title}{sbs,lefthand ratio=0.66}
% \usepackage{varwidth}
\newtcolorbox{mybox}[1]{varwidth boxed title,
enhanced,attach boxed title to top center=
{yshift=-3mm,yshifttext=-1mm},
boxed title style={size=small,colback=red},
center title,title={#1}}

\begin{mybox}{Short title}
This is a \textbf{tcolorbox}.
\end{mybox}\bigskip

\begin{mybox}{This title is not really very short}
This is a \textbf{tcolorbox}.
\end{mybox}
\end{exdispExample*}
\end{docTcbKey}


\begin{docTcbKey}{varwidth boxed title*}{\colOpt{=\meta{length}}}{initially unset}
The title text content is captured with a |varwidth| environment with a width of main box width plus \meta{length}.
This style needs the |varwidth| package \cite{arseneau:2011a} to be loaded manually.
By default, the resulting boxed title is somewhat smaller than the main box.

标题文本内容使用宽度为主盒子宽度加上\meta{length}的|varwidth|环境来捕捉。这种样式需要手动加载|varwidth|包\cite{arseneau:2011a}。默认情况下,生成的带框标题比主盒子略小。
\begin{exdispExample*}{varwidth_boxed_title_star}{sbs,lefthand ratio=0.66}
% \usepackage{varwidth}
\newtcolorbox{mybox}[1]{varwidth boxed title*=-2cm,
enhanced,attach boxed title to top center=
{yshift=-3mm,yshifttext=-1mm},
boxed title style={size=small,colback=red},
center title,title={#1}}

\begin{mybox}{Short title}
This is a \textbf{tcolorbox}.
\end{mybox}\bigskip

\begin{mybox}{This title is not really very short}
This is a \textbf{tcolorbox}.
\end{mybox}
\end{exdispExample*}
\end{docTcbKey}