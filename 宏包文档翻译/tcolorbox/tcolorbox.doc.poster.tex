% !TeX root = tcolorbox.tex
% include file of tcolorbox.tex (manual of the LaTeX package tcolorbox)
% %
\section{Library \mylib{poster}}\label{sec:poster}%
\tcbset{external/prefix=external/poster_}%

The main purpose of this library is to support creation of single page posters
with |tcolorbox|es.

这个库的主要目的是支持使用|tcolorbox|创建单页海报。

A \refEnvLe{tcbposter} is a |tikzpicture| where |tcolorbox|es can be
placed in a column oriented manner using \refComLe{posterbox} commands.
This base concept is more or less copied from the great |baposter| package.

\refEnvLe{tcbposter}是一个|tikzpicture|,可以使用\refComLe{posterbox}命令以列为导向的方式将|tcolorbox|放置在其中。
这个基本概念或多或少地从伟大的|baposter|包中复制而来。

The \mylib{raster} library, see \Fullref{sec:raster}, can produce
similar looking results and may be more appropriate
depending on the actual project.

\mylib{raster}库(见\Fullref{sec:raster})可以产生类似的外观效果,并且可能更适合实际项目。
\begin{itemize}
\item The \mylib{raster} library has a flow oriented concept, just like a
  convential text flow. The text flow (box flow) is a merely endless ribbon
  which gets broken into lines (and paragraphs) and the lines are broken
  into pages. \mylib{raster} shapes the boxes to convenient sizes to fill
  lines and pages in a pleasant way.

  \mylib{raster}库具有一种流式概念,就像传统文本流一样。文本流(框流)仅是一个无限长的带子,被分成行(和段落),行被分成页面。\mylib{raster}将框体塑造成便于填充行和页面的方便尺寸。
\item The \mylib{tcbposter} library supports a quite free placement of
  boxes inside a page.
  Basically, boxes are placed like |node|s are placed inside a |tikzpicture|.
  In contrast to \mylib{raster}, this is a \emph{single} page
  and not a flow of pages.
  The poster is divided into columns and rows.
  There is a more or less gentle force to use the columns (or spans of columns)
  for positioning and sizing while the row placement is completely optional.

\mylib{tcbposter}库支持相当自由的框的位置放置。基本上,框像|tikzpicture|中的|node|一样放置。与\mylib{raster}相比,这是一个\emph{单}页而不是页面流。海报被分成列和行。有一种或多种强制使用列(或列的跨度)进行定位和调整大小,而行定位完全是可选的。
\end{itemize}
The creation of this library was motivated by Ignasi.

创建这个库的灵感来自于Ignasi。
\begin{marker}
Inside a |tikzpicture| there should be no embedded |tikzpicture|s.
This rule is violated by the \mylib{poster} library. Be aware that there
may be some unwanted interactions between the main |tikzpicture| and
the embedded ones inside the |tcolorbox|es.

在|tikzpicture|中,不应该有嵌入式|tikzpicture|。这个规则被\mylib{poster}库违反了。请注意,主|tikzpicture|和嵌入在|tcolorbox|中的那些之间可能存在一些不需要的相互作用。
\end{marker}

The library is loaded by a package option or inside the preamble by:

该库可以通过包选项或在导言区内加载:
\begin{dispListing}
\tcbuselibrary{poster}
\end{dispListing}
This also loads the libraries
\mylib{skins}, see \Fullref{sec:skins},
\mylib{breakable}, see \Fullref{sec:breakable},
\mylib{magazine}, see \Fullref{sec:magazine}, and
\mylib{fitting}, see \Fullref{sec:fitting}.

这也加载了库\mylib{skins}(见\Fullref{sec:skins})、\mylib{breakable}(见\Fullref{sec:breakable})、\mylib{magazine}(见\Fullref{sec:magazine})和\mylib{fitting}(见\Fullref{sec:fitting})。
%--------------------------
\subsection{Overview\\概述}\label{subsec:poster_overview}


\begin{tcolorbox}[base example,hyperurl={tcolorbox-tutorial-poster.pdf},title=Click me to see the tutorial]
You get the best overview of the \mylib{poster} library and its facilities,
if you look at the \textbf{Poster Tutorial} which is part of the |tcolorbox|
documentation:\par
\texttt{tcolorbox-tutorial-poster.pdf}

要了解\mylib{poster}库及其功能的最佳概述,请参阅\textbf{Poster Tutorial},它是|tcolorbox|文档的一部分:\par
\texttt{tcolorbox-tutorial-poster.pdf}
\end{tcolorbox}



%
%--------------------------
\subsection{Main Poster Environment\\主要海报环境}\label{subsec:poster_environment}

\begin{docEnvironment}[doc new=2017-07-03]{tcbposter}{\oarg{options}}
This creates a |tikzpicture| environment with suitable additional
settings defined by the given \meta{options}.
Basically, \refComLe{posterbox} and \refEnvLe{posterboxenv} are
used to place |tcolorboxes| as nodes into the environment,
but additional \tikzname\ code can also be used.
As \meta{options} all |/tcb/posterset/| keys may be applied, namely:

这将创建一个|tikzpicture|环境,并使用给定的\meta{options}定义适当的附加设置。
基本上,使用\refComLe{posterbox}和\refEnvLe{posterboxenv}将|tcolorbox|作为节点放置在环境中,
但也可以使用其他\tikzname\ 代码。
作为\meta{options},可以应用所有|/tcb/posterset/|关键字,即:
\begin{itemize}
\item\refKeyLe{/tcb/posterset/poster}: poster settings like columns, rows, sizes\ldots

海报设置,如列、行、大小等。
\item\refKeyLe{/tcb/posterset/coverage} and \refKeyLe{/tcb/posterset/no coverage}:
  settings for a surrounding |tcolorbox| for background and margins.

\refKeyLe{/tcb/posterset/coverage}和\refKeyLe{/tcb/posterset/no coverage}:
用于背景和边距的周围|tcolorbox|的设置。
\item\refKeyLe{/tcb/posterset/boxes}: style of the |tcolorbox|es used for the poster.

用于海报的|tcolorbox|的样式。
\item\refKeyLe{/tcb/posterset/fontsize}: scaling of used fonts.

所使用字体的缩放。
\end{itemize}

\begin{exdispExample}{tcbposter}
  \begin{tcbposter}[
    poster = {showframe,height=10cm,spacing=2mm},
    boxes  = {beamer,colframe=blue!50!black,colback=blue!50,colupper=yellow!50},
  ]
  \posterbox{name=A,column=3,row=2}{My first box}
  \posterbox[adjusted title=Second box]
            {name=B,column=2,span=2,below=A}{My second box}
  \posterbox[adjusted title=Third box]
            {name=C,column=2,between=B and bottom}{My third box}
  \end{tcbposter}
\end{exdispExample}
\end{docEnvironment}

%
Inside \refEnvLe{tcbposter}, there are several predefined \tikzname\ nodes.
These nodes share a common \refKeyLe{/tcb/poster/prefix} which is
|TCBPOSTER@| by default. This prefix is used to discriminate the
poster nodes from local nodes of any embedded |tikzpicture| environment.
You will never need this prefix using \refComLe{posterbox} and its
placement options, but if you want to refer to a predefined node using
pure \tikzname\ code.
The predefined nodes (shown without prefix) are:
  
在\refEnvLe{tcbposter}内,有几个预定义的\tikzname\ 节点。
这些节点共享一个通用的\refKeyLe{/tcb/poster/prefix},默认情况下为|TCBPOSTER@|。
此前缀用于将海报节点与任何嵌入的|tikzpicture|环境的本地节点区分开来。
使用\refComLe{posterbox}和其放置选项时,您永远不需要此前缀,
但是如果要使用纯\tikzname\ 代码引用预定义的节点。
预定义节点(不带前缀)为:
\begin{itemize}
\item|poster|: defines the bounding box of the poster (without the coverage).

定义海报的边界框(不包括边距)。
\item|top|: top position plus row spacing

顶部位置加上行间距。
\item|bottom|: bottom position minus row spacing

底部位置减去行间距。
\item|middle|: vertical middle position

竖直中间位置。
\item|col1|, |col2|, \ldots: bounding box of column~1, column~2, \ldots

列1、列2、$\ldots$的边界框。
\item|row1|, |row2|, \ldots: bounding box of row~1, row~2, \ldots

行1、行2、$\ldots$的边界框。
\end{itemize}
Further nodes are defined using the \refKeyLe{/tcb/posterloc/name} option.

使用\refKeyLe{/tcb/posterloc/name}选项可以定义更多的节点。
\begin{marker}
Never use a \refEnvLe{tcbposter} inside a \refEnvLe{tcbposter}.
But, if you do anyway, use a different \refKeyLe{/tcb/poster/prefix} for
the embedded poster or you surely get a total mess.

永远不要在\refEnvLe{tcbposter}内部使用另一个\refEnvLe{tcbposter}。
但是,如果您确实这么做了,请为嵌套海报使用不同的\refKeyLe{/tcb/poster/prefix},否则您肯定会陷入糟糕的境地。
\end{marker}

There are several properties inside a \refEnvLe{tcbposter} which may be useful
for advanced code (skip the following on first reading):

在\refEnvLe{tcbposter}内部有几个属性对于高级代码可能有用(第一次阅读时请跳过以下内容):
\begin{itemize}
\item\docAuxCommand{tcbposterwidth}: Width of the poster (without margins).

海报的宽度(不包括边距)。
\item\docAuxCommand{tcbposterheight}: Height of the poster (without margins).

海报的高度(不包括边距)。
\item\docAuxCommand{tcbpostercolspacing}: Column distance.

列距离。
\item\docAuxCommand{tcbposterrowspacing}: Row distance.

行距离。
\item\docAuxCommand{tcbpostercolumns}: Column quantity.

列数。
\item\docAuxCommand{tcbposterrows}: Row quantity.

行数。
\item\docAuxCommand{tcbpostercolwidth}: Width of a column.

列的宽度。
\item\docAuxCommand{tcbposterrowheight}: Height of a row.

行的高度。
\end{itemize}

% \medskip
\begin{docCommand}[doc new=2017-07-03]{tcbposterset}{\marg{options}}
Sets options for every following \refEnvLe{tcbposter} inside the current \TeX\ group.
For example, the numbers for rows and columns may be defined for the whole document by this:

设置当前\TeX\ 组内后续所有\refEnvLe{tcbposter}的选项。
例如,可以通过以下方式为整个文档定义行数和列数:
\begin{dispListing}
\tcbposterset{poster={columns=2,rows=3}}
\end{dispListing}
See \refEnvLe{tcbposter} for all feasible options.

有关所有可行选项,请参见\refEnvLe{tcbposter}。
\end{docCommand}


%
%--------------------------
\subsection{Poster Settings\\海报设置}\label{subsec:poster_settings}

\begin{postersetTcbKey}[][doc new=2017-07-03]{poster}{=\marg{option list}}{style, no default}
This option can be applied inside \refEnvLe{tcbposter} and \refComLe{tcbposterset}
to set the given poster \meta{option list}, e.g.

可以在\refEnvLe{tcbposter}和\refComLe{tcbposterset}内应用此选项以设置给定的海报\meta{option list},例如
\begin{dispListing}
\tcbposterset{poster={width=20cm,height=15cm}}
\end{dispListing}
For the \meta{option list}, see the following keys.

有关\meta{option list},请参见以下键。
\end{postersetTcbKey}


\begin{posterTcbKey}[][doc new=2017-07-03]{columns}{=\meta{number}}{no default, initially |3|}
Sets the \meta{number} of columns for a |tcbposter|.

设置 |tcbposter| 的列数为\meta{number}。
\begin{exdispExample}{columns}
  \begin{tcbposter}[
    poster = {showframe,columns=5,rows=2,spacing=1mm,height=4cm},
  ]
  \end{tcbposter}
\end{exdispExample}
\end{posterTcbKey}

\begin{posterTcbKey}[][doc new=2017-07-03]{rows}{=\meta{number}}{no default, initially |4|}
Sets the \meta{number} of rows for a |tcbposter|.

设置 |tcbposter| 的行数为\meta{number}。
\end{posterTcbKey}


\begin{posterTcbKey}[][doc new=2017-07-03]{colspacing}{=\meta{length}}{no default, initially |4mm|}
Sets \meta{length} as distance between columns.

设置列之间的距离为\meta{length}。
\end{posterTcbKey}

\begin{posterTcbKey}[][doc new=2017-07-03]{rowspacing}{=\meta{length}}{no default, initially |4mm|}
Sets \meta{length} as distance between rows.

设置行之间的距离为\meta{length}。
\end{posterTcbKey}

\begin{posterTcbKey}[][doc new=2017-07-03]{spacing}{=\meta{length}}{style, no default, initially |4mm|}
Sets \meta{length} as distance between columns and rows.

设置行和列之间的距离为\meta{length}。
\end{posterTcbKey}


\begin{posterTcbKey}[][doc new=2017-07-03]{showframe}{\colOpt{=true\textbar false}}{default |true|, initially |false|}
Displays a red auxiliary mesh as optical support during poster creation.
Also, every \refKeyLe{/tcb/posterloc/name} is displayed.

在海报创建过程中显示红色辅助网格作为视觉支持。还会显示每个\refKeyLe{/tcb/posterloc/name}。
\end{posterTcbKey}


\begin{posterTcbKey}[][doc new=2017-07-03]{width}{=\meta{length}}{no default, initially \cs{linewidth}}
Sets \meta{length} as width of the poster. For a typical poster, this has not
to be set manually. Especially, if \refKeyLe{/tcb/posterset/coverage} is present,
use |coverage={width=|\meta{length}|}| instead to change the overall width.

将\meta{length}设置为海报的宽度。对于典型的海报,不需要手动设置该值。特别是,如果存在\refKeyLe{/tcb/posterset/coverage},请改用|coverage={width=|\meta{length}|}|来更改整体宽度。
\end{posterTcbKey}

% \enlargethispage*{1cm}

\begin{posterTcbKey}[][doc new=2017-07-03]{height}{=\meta{length}}{no default, initially unset}
Sets \meta{length} as height of the poster. For a typical poster, this has not
to be set manually, but is set automatically to an appropriate value.
If \refKeyLe{/tcb/posterset/coverage} is present, use only one if any option
|coverage={height=|\meta{length}|}| or |poster={height=|\meta{length}|}|.

将 \meta{length} 设置为海报的高度。对于典型的海报,这不需要手动设置,但会自动设置为适当的值。如果存在 \refKeyLe{/tcb/posterset/coverage},则仅使用其中一个选项 |coverage={height=|\meta{length}|}| 或 |poster={height=|\meta{length}|}|。
\end{posterTcbKey}


\begin{posterTcbKey}[][doc new=2017-07-03]{prefix}{=\meta{name}}{no default, initially |TCBPOSTER@|}
\meta{name} is set as prefix for any \tikzname\ node which is generated
automatically by the \mylib{poster} library. This encompasses predefined
nodes like |top|, |bottom|, \ldots, and nodes defined by using
\refKeyLe{/tcb/posterloc/name}. Also, see~\Fullref{subsec:poster_environment}.
For a typical poster, this value can stay as it is.

将 \meta{name} 设置为 \tikzname\ 生成的任何节点的前缀,该节点是由 \mylib{poster} 库自动生成的。这包括预定义的节点,如 |top|,|bottom| 等,以及使用 \refKeyLe{/tcb/posterloc/name} 定义的节点。另请参见~\Fullref{subsec:poster_environment}。对于典型的海报,此值可以保持不变。
\end{posterTcbKey}


%--------------------------
\subsection{Coverage\\覆盖}\label{subsec:poster_coverage}

\begin{postersetTcbKey}[][doc new=2017-07-03]{coverage}{=\marg{option list}}{style, no default}
This option can be applied inside \refEnvLe{tcbposter} and \refComLe{tcbposterset}
and it adds an optional coverage for the poster which is a surrounding |tcolorbox|
with the given \meta{option list}. Here, margins and background settings
for the poster can be given.
The \emph{coverage} has several default |tcolorbox| settings
suitable for the purpose:

可以在 \refEnvLe{tcbposter} 和 \refComLe{tcbposterset} 中应用此选项,并将其添加为海报的可选覆盖层,这是一个带有给定 \meta{option list} 的环绕 |tcolorbox|。在此处,可以给出海报的边距和背景设置。 \emph{覆盖} 具有适用于此目的的多个默认 |tcolorbox| 设置:
\begin{dispListing}
enhanced, frame hidden, sharp corners, boxsep=0pt, boxrule=0pt,
top=4mm, bottom=4mm, left=4mm, right=4mm,
toptitle=2mm, bottomtitle=2mm, colback=white
\end{dispListing}

The \meta{option list} can contain any |tcolorbox| option.

\meta{option list} 可以包含任何 |tcolorbox| 选项。
\begin{exdispExample}{coverage}
\begin{tcbposter}[
  poster   = {showframe,spacing=1mm},
  coverage = {height=5cm,
              interior style={top color=yellow,bottom color=yellow!50!red},
              watermark text={My Poster},watermark color=white,
             },
]
\end{tcbposter}
\end{exdispExample}

\begin{itemize}
\item For a typical poster, the option \refKeyLe{/tcb/spread} will use the
whole page for the poster coverage.

对于典型的海报,选项 \refKeyLe{/tcb/spread} 将使用整个页面来覆盖海报。
\item Poster margins can be adapted by \refKeyLe{/tcb/left}, \refKeyLe{/tcb/right},
\refKeyLe{/tcb/top}, \refKeyLe{/tcb/bottom}.

海报边距可以通过 \refKeyLe{/tcb/left},\refKeyLe{/tcb/right},\refKeyLe{/tcb/top},\refKeyLe{/tcb/bottom} 调整。
\item Poster background can be changed by \refKeyLe{/tcb/colback},
\refKeyLe{/tcb/interior style}, \refKeyLe{/tcb/interior style image}, etc.

海报的背景可以通过\refKeyLe{/tcb/colback}、\refKeyLe{/tcb/interior style}、\refKeyLe{/tcb/interior style image}等方式进行更改。
\item Do not use \refKeyLe{/tcb/poster/width} and \refKeyLe{/tcb/poster/height}
  in combination with a \emph{coverage}. Note that you may use
  \refKeyLe{/tcb/width} and \refKeyLe{/tcb/height} inside
  the \emph{coverage} \meta{option list}. Note that this also is not
  necessary when \refKeyLe{/tcb/spread} is applied.

不要在使用\emph{coverage}时结合使用\refKeyLe{/tcb/poster/width}和\refKeyLe{/tcb/poster/height}。请注意,您可以在\emph{coverage} \meta{option list}内使用\refKeyLe{/tcb/width}和\refKeyLe{/tcb/height}。请注意,当应用\refKeyLe{/tcb/spread}时,这也是不必要的。
\end{itemize}
\end{postersetTcbKey}


\begin{postersetTcbKey}[][doc new=2017-07-03]{no coverage}{}{style, no value, initially set}
Removes the surrounding |tcolorbox| completely.

完全删除周围的|tcolorbox|。
\end{postersetTcbKey}

%
%--------------------------
\subsection{Common Box Settings\\通用框设置}\label{subsec:poster_boxsettings}


\begin{postersetTcbKey}[][doc new=2017-07-03]{boxes}{=\marg{option list}}{style, no default}
This option can be applied inside \refEnvLe{tcbposter} and \refComLe{tcbposterset}
and it is used to set up the style of the |tcolorbox|es inside the poster.
The \meta{option list} can contain any |tcolorbox| option, but box size
options are not assumed to be useful here, because the size will be
determined by the placement options.

此选项可以在\refEnvLe{tcbposter}和\refComLe{tcbposterset}内应用,用于设置海报内|tcolorbox|的样式。 \meta{option list}可以包含任何|tcolorbox|选项,但是框的大小选项不被认为在此有用,因为大小将由放置选项确定。
\begin{exdispExample}{boxes}
\begin{tcbposter}[
  poster   = {spacing=2mm,columns=3,rows=2},
  coverage = {height=5cm,
              interior style={top color=yellow,bottom color=yellow!50!red},
             },
  boxes    = {sharp corners=downhill,arc=3mm,boxrule=1mm,
              colback=white,colframe=cyan,
              title style={left color=black,right color=cyan},
              fonttitle=\bfseries}
]
  \posterbox[adjusted title=First]{column=1,row=1,span=2}{First box}
  \posterbox[adjusted title=Second]{column=1,row=2,span=2}{Second box}
  \posterbox[adjusted title=Third]{column=3,row=1,rowspan=2}{Third box}
\end{tcbposter}
\end{exdispExample}

\end{postersetTcbKey}


%--------------------------
\subsection{Font Scaling\\字体缩放}\label{subsec:poster_fontsize}

\begin{postersetTcbKey}[][doc new=2017-07-03]{fontsize}{=\meta{length}}{style, no default, initially unset}
This option can be applied inside \refEnvLe{tcbposter} and \refComLe{tcbposterset}.
It uses \refKeyLe{/tcb/fit basedim} and \refKeyLe{/tcb/fit fontsize macros}
to redefine |\normalsize| to \meta{length} and all other standard
font size macros like |\small| and |\large| accordingly.\par
This needs a freely scalable font family like |lmodern| to work.
If \refKeyLe{/tcb/posterset/fontsize} is not applied, there standard
font size macros are not changed in any way.

此选项可以在\refEnvLe{tcbposter}和\refComLe{tcbposterset}内应用。它使用\refKeyLe{/tcb/fit basedim}和\refKeyLe{/tcb/fit fontsize macros}重新定义|\normalsize|为\meta{length},并相应地重新定义所有其他标准字体大小宏,如|\small|和|\large|。\par
这需要一个自由缩放的字体系列,如|lmodern|才能工作。
如果未应用\refKeyLe{/tcb/posterset/fontsize},则标准字体大小宏不会以任何方式更改。
\begin{dispListing}
\begin{tcbposter}[
  poster   = {spacing=2mm,columns=3,rows=2},
  coverage = {height=5cm,
              interior style={top color=yellow,bottom color=yellow!50!red},
             },
  fontsize = 15pt,   % <--- \normalsize is now 15pt
]
...
\end{dispListing}
\end{postersetTcbKey}


%
%--------------------------
\subsection{Box Placement\\盒子放置}\label{subsec:poster_boxplacement}

\begin{docCommand}[doc new=2017-07-03]{posterbox}{\oarg{options}\marg{placement}\marg{box content}}
Inside a \refEnvLe{tcbposter} environment, this places a |tcolorbox| with
additional |tcolorbox| \meta{options} and the given \meta{box content}
at a place determined by \meta{placement}.
All \meta{placement} options are described in the following.
Note that \meta{box content} cannot contain \emph{verbatim} material,
see \refEnvLe{posterboxenv}.

在\refEnvLe{tcbposter}环境中,此命令可以将带有附加|tcolorbox| \meta{options}和给定的\meta{box content}的|tcolorbox|放置在由\meta{placement}确定的位置。所有\meta{placement}选项如下所述。请注意,\meta{box content}不能包含\emph{verbatim}材料,请参见\refEnvLe{posterboxenv}。
\begin{exdispExample}{posterbox}
\begin{tcbposter}[
  poster = {showframe,height=4cm,spacing=2mm,rows=2},
  boxes  = {beamer,colframe=blue!50!black,colback=blue!50,colupper=yellow!50},
]
\posterbox[title=My title]{name=A,column=2,row=2}{My first box}
\end{tcbposter}
\end{exdispExample}
\end{docCommand}

\begin{docEnvironment}[doc new=2017-07-03]{posterboxenv}{\oarg{options}\marg{placement}}
This is the environment version of \refComLe{posterbox}, i.e.\ inside a
\refEnvLe{tcbposter} environment, this places a |tcolorbox| with
additional |tcolorbox| \meta{options} and the given \meta{environment content}
at a place determined by \meta{placement}.
In contrast to \refComLe{posterbox}, the \meta{environment content} is
allowed to contain \emph{verbatim} material. Note that the implementation
of \refComLe{posterbox} is more efficient than the implementation of \refEnvLe{posterboxenv}.

这是\refComLe{posterbox}的环境版本,即在\refEnvLe{tcbposter}环境中,此命令可以将带有附加|tcolorbox| \meta{options}和给定的\meta{environment content}的|tcolorbox|放置在由\meta{placement}确定的位置。与\refComLe{posterbox}相比,\meta{environment content}允许包含\emph{verbatim}材料。请注意,\refComLe{posterbox}的实现比\refEnvLe{posterboxenv}的实现更高效。
% \enlargethispage*{1cm}
\begin{exdispExample}{posterboxenv}
\begin{tcbposter}[
  poster = {showframe,height=4cm,spacing=2mm,rows=2},
  boxes  = {size=small,beamer,
            colframe=blue!50!black,colback=blue!50,colupper=yellow!50},
]
\begin{posterboxenv}[title=My title]{name=A,column=2,between=top and bottom}
  My first box.
  \begin{tcblisting}{size=small,colback=yellow!10}
My \textbf{first}
poster listing.
  \end{tcblisting}
\end{posterboxenv}
\end{tcbposter}
\end{exdispExample}

\end{docEnvironment}


%
\begin{posterlocTcbKey}[][doc new=2017-07-03]{name}{=\meta{name}}{no default, initially |@|}
Sets \meta{name} as reference for the current \refComLe{posterbox} or
\refEnvLe{posterboxenv}.
A \tikzname\ shape name is constructed automatically as combination
of \refKeyLe{/tcb/poster/prefix} and \meta{name}.

将当前的 \refComLe{posterbox} 或 \refEnvLe{posterboxenv} 设置为 \meta{name} 的参考。
一个 \tikzname 的形状名称会自动构建为 \refKeyLe{/tcb/poster/prefix} 和 \meta{name} 的组合。
\begin{exdispExample}{name}
\begin{tcbposter}[
  poster = {showframe,height=2.5cm,spacing=2mm,rows=2},
  boxes  = {beamer,colframe=blue!50!black,colback=blue!50,colupper=yellow!50},
]
\posterbox{name=A,column=2,row=2}{My first box}
\node[below right=4mm,fill=yellow] (X) at (TCBPOSTER@poster.north west) {Example A};
\draw[blue,very thick,->] (X) |- (TCBPOSTER@A);
\end{tcbposter}
\end{exdispExample}
\end{posterlocTcbKey}


\begin{posterlocTcbKey}[][doc new=2017-07-03]{column}{=\meta{number}}{no default, initially |1|}
Places the box at the column denoted by \meta{number}. If \refKeyLe{/tcb/posterloc/span}
is not |1|, the box is aligned to the left side of column \meta{number}.

将盒子放置在第 \meta{number} 列。如果 \refKeyLe{/tcb/posterloc/span} 不是 |1|,则盒子对齐到第 \meta{number} 列的左侧。
\begin{exdispExample}{column}
\begin{tcbposter}[
  poster = {showframe,height=2.5cm,spacing=2mm,rows=2},
  boxes  = {beamer,colframe=blue!50!black,colback=blue!50,colupper=yellow!50},
]
\posterbox{row=1,column=2,span=2}{First box}
\posterbox{row=2,column=2,span=0.8}{Second box}
\end{tcbposter}
\end{exdispExample}
\end{posterlocTcbKey}

\enlargethispage*{1cm}
\begin{posterlocTcbKey}[][doc new=2017-07-03]{column*}{=\meta{number}}{no default, initially unset}
Places the box at the column denoted by \meta{number}. If \refKeyLe{/tcb/posterloc/span}
is not |1|, the box is aligned to the right side of column \meta{number}.

将盒子放置在第 \meta{number} 列。如果 \refKeyLe{/tcb/posterloc/span} 不是 |1|,则盒子对齐到第 \meta{number} 列的右侧。
\begin{exdispExample}{columnstar}
\begin{tcbposter}[
  poster = {showframe,height=2.5cm,spacing=2mm,rows=2},
  boxes  = {beamer,colframe=blue!50!black,colback=blue!50,colupper=yellow!50},
]
\posterbox{row=1,column*=2,span=2}{First box}
\posterbox{row=2,column*=2,span=0.8}{Second box}
\end{tcbposter}
\end{exdispExample}
\end{posterlocTcbKey}


%
\begin{posterlocTcbKey}[][doc new=2017-07-03]{span}{=\meta{number}}{no default, initially |1|}
Sets the width of the current box to span \meta{number} columns.
\meta{number} is also allowed to be a real number like |0.5| or |1.7|.
See \refKeyLe{/tcb/posterloc/column} and \refKeyLe{/tcb/posterloc/column*}
for examples.

将当前盒子的高度设置为跨越 \meta{number} 行。
\meta{number} 也可以是实数,如 |0.5| 或 |1.7|。
请参见 \refKeyLe{/tcb/posterloc/column} 和 \refKeyLe{/tcb/posterloc/column*} 的示例。
\end{posterlocTcbKey}

\begin{posterlocTcbKey}[][doc new=2017-07-03]{row}{=\meta{number}}{no default, initially unset}
If this option is applied, the box is placed at the row denoted by \meta{number}.
Also, the height is set as fixed according to \refKeyLe{/tcb/posterloc/rowspan}.

如果应用了此选项,则盒子放置在由 \meta{number} 指定的行上。
此外,高度根据 \refKeyLe{/tcb/posterloc/rowspan} 设置为固定值。
\begin{exdispExample}{row}
\begin{tcbposter}[
poster = {showframe,height=2.5cm,spacing=2mm,rows=2},
boxes  = {beamer,colframe=blue!50!black,colback=blue!50,colupper=yellow!50},
]
\posterbox{row=1,column=1}{First box}
\posterbox{row=1,column=2,rowspan=2}{Second box}
\posterbox[natural height]{row=1,column=3}{Third box}
\end{tcbposter}
\end{exdispExample}
\end{posterlocTcbKey}

\begin{posterlocTcbKey}[][doc new=2017-07-03]{rowspan}{=\meta{number}}{no default, initially |1|}
Sets the height of the current box to span \meta{number} rows.
\meta{number} is also allowed to be a real number like |0.5| or |1.7|.

将当前盒子的高度设置为跨越 \meta{number} 行。\meta{number} 也可以是实数,如 |0.5| 或 |1.7|。
\begin{exdispExample}{rowspan}
\begin{tcbposter}[
poster = {showframe,height=2.5cm,spacing=2mm,rows=2},
boxes  = {beamer,colframe=blue!50!black,colback=blue!50,colupper=yellow!50},
]
\posterbox{row=1,column=1,rowspan=0.9}{First box}
\posterbox{row=1,column=2,rowspan=1.5}{Second box}
\posterbox{row=1,column=3,rowspan=2}{Third box}
\end{tcbposter}
\end{exdispExample}
\end{posterlocTcbKey}

\begin{posterlocTcbKey}[][doc new=2017-07-03]{fixed height}{}{no value, initially |0pt|}
Sets the height of the current box span rows as denoted by
\refKeyLe{/tcb/posterloc/rowspan}.
This can be used, if not \refKeyLe{/tcb/posterloc/row}, but another
height placement option is applied.

此选项将当前盒子的高度设置为由 \refKeyLe{/tcb/posterloc/rowspan} 指定的跨越行数。如果不使用 \refKeyLe{/tcb/posterloc/row},而使用其他高度定位选项,则可以使用此选项。
\end{posterlocTcbKey}


%
\begin{posterlocTcbKey}[][doc new=2017-07-03]{below}{=\meta{name}}{no default, initially |top|}
The box is placed below another box with the given \meta{name}. Also,
\meta{name} can be a predefined node, see \Fullref{subsec:poster_environment}.

该选项将在给定的名称 \meta{name} 的盒子下方放置盒子。此外,\meta{name} 也可以是预定义节点,参见 \Fullref{subsec:poster_environment}。
\begin{exdispExample}{below}
\begin{tcbposter}[
poster = {showframe,height=3cm,spacing=2mm,rows=2},
boxes  = {beamer,colframe=blue!50!black,colback=blue!50,colupper=yellow!50},
]
\posterbox{name=A,column=1,below=top}{First box}
\posterbox{name=B,column=1,below=A}{Second box}
\posterbox{name=C,column=2,below=B}{Third box}
\posterbox{name=D,column=3,below=row1}{Fourth box}
\end{tcbposter}
\end{exdispExample}
\end{posterlocTcbKey}


\begin{posterlocTcbKey}[][doc new=2017-07-03]{above}{=\meta{name}}{no default, initially unset}
The box is placed above another box with the given \meta{name}. Also,
\meta{name} can be a predefined node, see \Fullref{subsec:poster_environment}.

这个框被放置在另一个具有给定\meta{name}的框的上方。同时,\meta{name}可以是预定义的节点,见\Fullref{subsec:poster_environment}。
\begin{exdispExample}{above}
\begin{tcbposter}[
poster = {showframe,height=3cm,spacing=2mm,rows=2},
boxes  = {beamer,colframe=blue!50!black,colback=blue!50,colupper=yellow!50},
]
\posterbox{name=A,column=1,above=bottom}{First box}
\posterbox{name=B,column=1,above=A}{Second box}
\posterbox{name=C,column=2,above=B}{Third box}
\posterbox{name=D,column=3,above=row2}{Fourth box}
\end{tcbposter}
\end{exdispExample}
\end{posterlocTcbKey}


%
\begin{posterlocTcbKey}[][doc new=2017-07-03]{at}{=\meta{name}}{no default, initially unset}
The box is placed at the position with the given \meta{name}. This is
quite likely a predefined node, see \Fullref{subsec:poster_environment}.

该框被放置在给定的 \meta{name} 位置。这很可能是一个预定义的节点,参见\Fullref{subsec:poster_environment}。
\begin{exdispExample}{at}
\begin{tcbposter}[
  poster = {showframe,height=3cm,spacing=2mm,rows=2},
  boxes  = {beamer,colframe=blue!50!black,colback=blue!50,colupper=yellow!50},
]
\posterbox{name=A,column=1,at=middle}{First box}
\posterbox{name=B,column=2,at=row1}{Second box}
\end{tcbposter}
\end{exdispExample}
\end{posterlocTcbKey}


\begin{posterlocTcbKey}[][doc new=2017-07-03]{between}{=\meta{name1} and \meta{name2}}{no default, initially unset}
The box is placed below a box \meta{name1} and above another box \meta{name2}. Also,
\meta{name1} and \meta{name2} can be predefined nodes, see \Fullref{subsec:poster_environment}.

此选项将盒子放置在给定的 \meta{name1} 盒子下面且给定的 \meta{name2} 盒子上面。此外,\meta{name1} 和 \meta{name2} 可以是预定义的节点,详见\Fullref{subsec:poster_environment}。
\begin{exdispExample}{between}
\begin{tcbposter}[
  poster = {showframe,height=3cm,spacing=2mm,rows=2},
  boxes  = {beamer,colframe=blue!50!black,colback=blue!50,colupper=yellow!50},
]
\posterbox{name=A,column=1,below=top}{First box}
\posterbox{name=B,column=1,between=A and bottom}{Second box}
\posterbox{name=C,column=2,above=bottom}{Third box}
\posterbox{name=D,column=2,between=top and C,span=2}{Fourth box}
\posterbox{name=E,column=3,between=D and bottom}{Fifth box}
\end{tcbposter}
\end{exdispExample}
\end{posterlocTcbKey}


%
\begin{posterlocTcbKey}[][doc new=2017-07-03]{sequence}{=\meta{sequence}}{no default, initially unset}
The box is broken into partial boxes. These partial boxes are placed
following the given \meta{sequence} of placements.
The feasible syntax for the \meta{sequence} is:\par\medskip
\meta{column a} |between| \meta{name a1} |and| \meta{name a2} |then|\\
\meta{column b} |between| \meta{name b1} |and| \meta{name b2} |then|\\
\meta{column c} |between| \meta{name c1} |and| \meta{name c2} |then|\ldots\par\medskip
Obviously, this places the first part box at \meta{column a} between
\meta{name a2} and \meta{name a2}. The second box part is placed
at \meta{column b} between
\meta{name b2} and \meta{name b2}, and so on.

该选项将框架分成多个部分,并按照给定的\meta{sequence}序列放置。
\meta{sequence}的语法格式如下:\par\medskip
\meta{column a} |between| \meta{name a1} |and| \meta{name a2} |then|\
\meta{column b} |between| \meta{name b1} |and| \meta{name b2} |then|\
\meta{column c} |between| \meta{name c1} |and| \meta{name c2} |then|\ldots\par\medskip
显然,这将第一个部分框放置在\meta{column a}中,位于\meta{name a2}和\meta{name a2}之间。第二个部分框被放置在\meta{column b}中,位于\meta{name b2}和\meta{name b2}之间,以此类推。
\begin{exdispExample}{sequence}
\begin{tcbposter}[
  poster = {showframe,height=6cm,spacing=2mm,rows=2},
  boxes  = {beamer,colframe=blue!50!black,colback=blue!50,colupper=yellow!50},
]
\posterbox[adjusted title=A]{name=A,column=1,below=top,span=2}{First box}
\posterbox{name=B,column=2,above=bottom,span=2}{Second box}
\posterbox[adjusted title=C,colframe=red!50!black,colback=red!50]{
  name=C, sequence=1 between A and bottom then
                   2 between A and B then
                   3 between top and B
  }{\lipsum[2]}
\end{tcbposter}
\end{exdispExample}
\end{posterlocTcbKey}

%

\begin{docTcbKey}[][doc new=2017-07-03]{placeholder}{}{style, no value}
If the box content of a \refKeyLe{/tcb/posterloc/sequence} is too short
to fill all reserved box parts, the empty boxes are drawn with
the \refKeyLe{/tcb/placeholder} style. This style can be redefined, e.g.
to \refKeyLe{/tcb/blankest}, if nothing should be drawn for empty boxes.

如果一个 \refKeyLe{/tcb/posterloc/sequence} 的盒子内容太短而不能填满所有保留的盒子部分,那么空的盒子部分将用 \refKeyLe{/tcb/placeholder} 样式绘制。如果不希望绘制空的盒子,则可以重新定义此样式,例如使用 \refKeyLe{/tcb/blankest}。
\begin{exdispExample}{placeholder}
\begin{tcbposter}[
  poster = {showframe,height=2.5cm,spacing=2mm,rows=2},
  boxes  = {beamer,colframe=blue!50!black,colback=blue!50,colupper=yellow!50},
]
\posterbox{name=A,column=1,below=top,span=2}{First box}
\posterbox[colframe=red!50!black,colback=red!50]{
  name=B, sequence=1 between A and bottom then
                   2 between A and bottom then
                   3 between top and bottom
  }{Second box followed by placeholder boxes}
\end{tcbposter}
\end{exdispExample}
\end{docTcbKey}



\begin{posterlocTcbKey}[][doc new=2017-07-03]{xshift}{=\meta{length}}{no default, initially |0pt|}
Horizontal shift of a box by \meta{length}.

将一个框向水平方向移动\meta{length}长度。
\begin{exdispExample}{xshift}
\begin{tcbposter}[
  poster = {showframe,height=3cm,spacing=2mm,rows=2},
  boxes  = {beamer,colframe=blue!50!black,colback=blue!50,colupper=yellow!50},
]
\posterbox{name=A,column=1,row=1,xshift=6mm}{First box}
\posterbox{name=B,column=2,row=2,xshift=-6mm}{Second box}
\end{tcbposter}
\end{exdispExample}
\end{posterlocTcbKey}

%

\begin{posterlocTcbKey}[][doc new=2017-07-03]{yshift}{=\meta{length}}{no default, initially |0pt|}
Vertical shift of a box by \meta{length}.

将一个盒子垂直移动 \meta{length} 的距离。
\begin{exdispExample}{yshift}
\begin{tcbposter}[
  poster = {showframe,height=3cm,spacing=2mm,rows=2},
  boxes  = {beamer,colframe=blue!50!black,colback=blue!50,colupper=yellow!50},
]
\posterbox{name=A,column=1,row=1,yshift=-4mm}{First box}
\posterbox{name=B,column=2,row=2,yshift=4mm}{Second box}
\end{tcbposter}
\end{exdispExample}
\end{posterlocTcbKey}

