% !TeX root = /Users/virhuiai/hlProjects/Latex-Typesetting-Hub/宏包文档翻译/tcolorbox/tcolorbox.tex

\section{Macros for Box Creation\\创建盒子的宏命令}%
\tcbset{external/prefix=external/coremacros_}%

% \include*{coremacros/tcolorboxEnv}
% \include*{coremacros/tcblowerCmd}
% \include*{coremacros/tcbsetCmd}
% \include*{coremacros/tcbsetforeverylayerCmd}
% \include*{coremacros/tcboxCmd}
% \include*{coremacros/newtcolorboxCmd}
\include*{coremacros/newtcboxCmd}
\end{document}




% \clearpage

\begin{docCommand}[doc new=2014-10-20]{tcolorboxenvironment}{\marg{name}\marg{options}}
  An existing environment \meta{name} is redefined to be boxed inside a
  |tcolorbox| with the given \meta{options}.

将原环境 \meta{name}嵌入到用给定选项 \meta{options} 的 |tcolorbox| 环境中。
\begin{dispExample*}{sbs,lefthand ratio=0.6}
% tcbuselibrary{skins}
\newenvironment{myitemize}{%
  \begin{itemize}}{\end{itemize}}

% blanker是啥 todo
\tcolorboxenvironment{myitemize}{blanker,
before skip=6pt,after skip=6pt,
% 西边(左)的线
borderline west={3mm}{0pt}{red}
  }

一些文本。
\begin{myitemize}
\item 甲
\item 乙
\item 丙
\end{myitemize}
更多的文本。
\end{dispExample*}

\medskip
另外的例子见 \Vref{subsec:theorems_other}.
\end{docCommand}

