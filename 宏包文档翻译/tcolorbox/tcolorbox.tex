% \LaTeX-Main\
%% The LaTeX package tcolorbox - version 5.0.2 (2022/01/07)
%% tcolorbox.tex: Manual
% 编译:
% xelatex -shell-escape tcolorbox
%%
%% -------------------------------------------------------------------------------------------
%% Copyright (c) 2006-2021 by Prof. Dr. Dr. Thomas F. Sturm <thomas dot sturm at unibw dot de>
%% -------------------------------------------------------------------------------------------
%%
%% This work may be distributed and/or modified under the
%% conditions of the LaTeX Project Public License, either version 1.3
%% of this license or (at your option) any later version.
%% The latest version of this license is in
%%   http://www.latex-project.org/lppl.txt
%% and version 1.3 or later is part of all distributions of LaTeX
%% version 2005/12/01 or later.
%%
%% This work has the LPPL maintenance status `author-maintained'.
%%
%% This work consists of all files listed in README
%%
% arara: pdflatex: { shell: yes, synctex: yes }
% arara: biber
% arara: pdflatex: { shell: yes, synctex: yes }
% arara: pdflatex: { shell: yes, synctex: yes }
% arara: pdflatex: { shell: yes, synctex: yes }
% arara: pdflatex: { shell: yes, synctex: yes }
%
% \PassOptionsToPackage{pdftex,bookmarks,raiselinks,pageanchor,hyperindex,colorlinks}{hyperref}
\PassOptionsToPackage{CJKbookmarks,bookmarksnumbered,bookmarksopen,
% pdftitle={\pdffilename},pdfauthor=virhuiai,
colorlinks=true, pdfstartview=FitH,citecolor=blue,linktocpage,
linkcolor=blue,urlcolor=blue,hyperindex=true}{hyperref}
\documentclass[a4paper,11pt]{ltxdoc}
\makeatletter
\providecommand*\input@path{}
\newcommand*\addinputpath[1]{\expandafter\def\expandafter\input@path\expandafter{\input@path#1}}
\makeatother

\makeatletter
\newcommand\ref@data{}
\newcommand\ref@type@cs{com}

\newcommand\ref@type@input{com}
\NewDocumentCommand\tcb@ref@doc@Le{msm}{%
\renewcommand\ref@type@input{#1}
\ifx\ref@type@input\ref@type@cs
\renewcommand\ref@data{\cs #3}
 \else
\renewcommand\ref@data{#3}
\fi
\hyperref[#1:#3]{\texttt{\ref@Le{#1:#3}}%
  \IfBooleanTF{#2}{}{%
    \ifnum\getpagerefnumber{#1:#3}=\thepage\relax%
    \else%
      \textsuperscript{{\fontfamily{pzd}\fontencoding{U}\fontseries{m}\fontshape{n}\selectfont\char213}%
      \,\kvtcb@text@pageshort\,\pageref*{#1:#3}}%
    \fi}}%
}

\def\refComLe{\tcb@ref@doc@Le{com}}
\def\refEnvLe{\tcb@ref@doc@Le{env}}
\def\refKeyLe{\tcb@ref@doc@Le{key}}
\def\refSkinLe{\tcb@ref@doc@Le{skin}}


\def\VrefLe{\tcb@ref@doc@Le}

\def\ref@Le#1{\expandafter\@setref@Le\csname r@#1\endcsname\@firstoftwo{#1}}

% \ifx#1\relax
%    \protect\G@refundefinedtrue
%    \nfss@text{\reset@font\bfseries \ref@data}%
%    \@latex@warning{Reference `#3' on page \thepage \space
%              undefined}%
%   \else
%    \expandafter#2#1\null
%   \fi
\def\@setref@Le#1#2#3{%
\protect\G@refundefinedtrue
\nfss@text{\reset@font\bfseries \ref@data}%
\@latex@warning{Reference `#3' on page \thepage \space
undefined}%
}
\makeatother

\newcommand\counterSetTmp[2]{\setcounter{#1}{#2}}
% \newcommand\counterSetTmp[2]{}

\addinputpath{%
{/Users/virhuiai/hlProjects/Latex-Typesetting-Hub/宏包文档翻译/tcolorbox}%
}




%xelatex -shell-escape tcolorbox
\PassOptionsToPackage{no-math}{fontspec}
\PassOptionsToPackage{AutoFakeBold=true,AutoFakeSlant=true}{xeCJK}
% \usepackage{ctex}
\usepackage[scheme=plain,fontset=none]{ctex}
\setCJKmainfont{FangZhengShuSong-GBK-1.ttf}[Path=/Users/virhuiai/hlProjects/Latex-Typesetting-Hub/font/方正/]%设置文本的中文有衬线字体
\setCJKsansfont{FangZhengHeiTi-GBK-1.ttf}[Path=/Users/virhuiai/hlProjects/Latex-Typesetting-Hub/font/方正/]%设置文本的中文无衬线字体为
\setCJKmonofont{FangZhengFangSong-GBK-1.ttf}[Path=/Users/virhuiai/hlProjects/Latex-Typesetting-Hub/font/方正/] %设置文本的中文等宽字体 

\usepackage{tcolorbox.doc.s_main}
\tcbEXTERNALIZE
\usepackage{tcolorbox.doc.s_snippet}
%\tcbset{external/PassOptionsToPackage={cache=false}{minted}}
\tcbAUXmkdir{external}
\tcbAUXmkdir{solutions}

\RequirePackage{csquotes}
\RequirePackage[style=numeric-comp,sorting=nyt,
  maxnames=8,minnames=8,abbreviate=false,backend=biber,
  bibencoding=latin1,texencoding=ascii]{biblatex}
\DeclareFieldFormat{url}{\newline\url{#1}}%
\DeclareListFormat{language}{}%
\setlength{\bibitemsep}{\smallskipamount}
\addbibresource{tcolorbox.doc.bib}

\def\version{5.0.2}%
\def\datum{2022/01/07}%
\makeindex

\hypersetup{
  pdftitle={Manual for the tcolorbox package},
  pdfauthor={Thomas F. Sturm},
  pdfsubject={colored boxes},
  pdfkeywords={colored boxes, LaTeX examples, theorems}
}

\usepackage{pgfplots}
\pgfplotsset{compat=1.17}

\nocite{breitenlohner:1998a}

%\geometry{showframe}
%\tcbset{draftmode}
\tcbset{/tcb/external/-}% for final run
%\inputonly{tcolorbox.doc.listings}

%\hypersetup{colorlinks=false}

\newcounter{stripedbox}
% \definecolor{colstriped}{rgb}{0.98, 0.98, 0.98}
\newenvironment{stripedbox}[1][
  % left*=0pt,right*=0pt,
]{%
\refstepcounter{stripedbox}
\ifodd\value{stripedbox}%奇数
\tcolorbox[colback=black,opacityback=0.15,%
,boxsep=0pt,boxrule=0pt,arc=0mm%
% ,opacityback=0
% % ,colframe=red!75!black
,opacityframe=0%0是完全透明了,1是完全不透明
,breakable
,nobeforeafter
,savedelimiter=stripedbox,#1]
\else
\tcolorbox[colback=gray,opacityback=0.05,%
,boxsep=0pt,boxrule=0pt,arc=0mm%
% ,opacityback=0
% % ,colframe=red!75!black
,opacityframe=0%0是完全透明了,1是完全不透明
,breakable
,nobeforeafter
,savedelimiter=stripedbox,#1]
\fi
}%
{\endtcolorbox}

\usepackage{parskip}
\newcommand{\parpar}{\\[0.5ex]}
\usepackage{newclude}%后面 \include* 引用就不会自动分页了
% \includeonly{coremacros/tcolorboxenvironmentCmd}
%%%%%%%%%%%%%%%%%%%%%%%%%%%%%%%%%%%%%%%%%%%%%%%%%

\graphicspath{
{/Users/virhuiai/hlProjects/Latex-Typesetting-Hub/宏包文档翻译/tcolorbox/}
}
\begin{document}
\VerbatimFootnotes

\newminted[latexcode]{latex}{autogobble,breaklines,breakanywhere}

\newenvironment{DescriptionR}[1]
 {\begin{list}
  {}
  {\renewcommand\makelabel[1]{\hfil##1}
   \settowidth\labelwidth{\makelabel{#1}}
   \itemsep=-5pt%
   \setlength\leftmargin{%
   \labelwidth+\labelsep}}
}
{\end{list}}

\newenvironment{DescriptionL}[1]
 {\begin{list}
  {}
  {\renewcommand\makelabel[1]{##1\hfil}
   \settowidth\labelwidth{\makelabel{#1}}
   \itemsep=-5pt%
   \setlength\leftmargin{%
   \labelwidth+\labelsep}}
}
{\end{list}}

\newenvironment{DescriptionLsqueezed}[1]
 {\begin{list}
  {}
  {
   \settowidth\labelwidth{\makelabel{#1}}
   \renewcommand\makelabel[1]{\resizebox{\labelwidth}{!}{##1}}
   \itemsep=-5pt%
   \setlength\leftmargin{%
   \labelwidth+\labelsep}}
}
{\end{list}}

% \newcommand{\meta}[1]{\ensuremath{\langle}\textit{#1}\ensuremath{\rangle}} 
\renewcommand{\meta}[1]{\ensuremath{\langle}%
\makebox[0pt]{}%
\textit{#1}%
\makebox[0pt]{}%
\ensuremath{\rangle}}

%%%%调用系统命令示例加标题STAR TODO 可以优化下
\makeatletter
\newcommand\jobnameToRun{\filename@area\filename@base}
\makeatletter
\newtcblisting{调用系统命令的例子}[2][]{
arc=0mm,
left=0pt,right=0pt,top=0pt,bottom=0pt,
boxrule=0pt,boxsep=0pt,
opacityframe=0,
listing side comment,
compilable listing,
run xelatex={-shell-escape},run system command={pdfcrop\space\jobnameToRun.pdf\space\jobnameToRun_croped.pdf},
pdf comment={\jobnameToRun_croped.pdf},#1}

\newtcolorbox[auto counter,number within=section]{带标题示例}[2][]{% 
left=0pt,right=0pt,top=0pt,bottom=0pt,
fonttitle=\bfseries, title=例.~\thetcbcounter: #2,#1}
%%%%调用系统命令示例加标题END

% \begin{带标题示例}{你好哇!}
% \begin{调用系统命令的例子}{}
% \documentclass{minimal}
% \usepackage[all]{tcolorbox}
% \begin{document}
% \thispagestyle{empty}
% test
% \end{document}  
% \end{调用系统命令的例子}
% \end{带标题示例}

\newcommand{\mshowc}[1]{\noindent\tt\string\setcounter\{#1\}\{\arabic{#1}\}\newline}

\newtcolorbox{引述之言}[1]{%
enhanced,colback=blue!10!white,colframe=orange
,after upper={\par\hfill---\textit{#1}}
} 

% % !TeX root = tcolorbox.tex
% include file of tcolorbox.tex (manual of the LaTeX package tcolorbox)
\begin{tcboutputlisting}
% \usepackage{incgraph}
\begin{inctext}
\begin{tikzpicture}
\definecolorseries{boxcol}{rgb}{last}{blue}{red}
\resetcolorseries[28]{boxcol}
\coordinate (A) at (0,0); \coordinate (B) at (21,29.7);
\path[use as bounding box] (A) rectangle coordinate (C) (B);
\node[transform shape,xslant=0.7,rotate=-10,xshift=0cm] at (C) {%
  \begin{tcbraster}[raster columns=4,title=tcolorbox \version,
    fonttitle=\small\bfseries,raster width=50cm]
  \foreach \b in {1,...,28} {\begin{tcolorbox}[enhanced,
    watermark text=\thetcbrasternum,
    colframe=boxcol!30!white,
    colback=boxcol!25!white!30!white,
    colbacktitle=boxcol!!+!50!black!30!white,
    colupper=black!30!white]\lipsum[2]\end{tcolorbox}}
  \end{tcbraster}%
};
\node at (C) {%
  \begin{tcbitemize}[title=tcolorbox \version,fonttitle=\small\bfseries,
    enhanced jigsaw,opacityback=0.5,opacitybacktitle=0.75,
    halign=center,valign=center,arc=5mm,
    raster width=16cm,raster column skip=8mm,raster halign=center,
    raster force size=false,
    raster row 1/.style={height=6cm},
    raster row 2/.style={width=6cm,height=4cm},
    raster column 1/.style={flushright title,
      frame style={left color=yellow!50!black,right color=green!50!black},
      title style={left color=yellow!50!blue,right color=blue!50!green!50!black},
      interior style={left color=yellow!70,right color=green!70},
      underlay={\draw[line width=6mm,line cap=round,black!60]
        ([shift={(0.4,-0.15)}]frame.north east)
        --([shift={(0.4,0.15)}]frame.south east); }},
    raster column 2/.style={
      frame style={left color=green!50!black,right color=yellow!50!black},
      title style={left color=blue!50!green!50!black,right color=yellow!50!blue},
      interior style={left color=green!70,right color=yellow!70}}]
  \tcbitem[fontupper=\Huge\bfseries,sharp corners=east,
    underlay={\draw[line width=6mm,line cap=round,black!60]
      ([shift={(0.4,0.30)}]frame.north east)-- coordinate(A) +(0,0.2);
      \draw[line width=1mm,line cap=round,black!60](A) -- +(30:1.5cm);
      \draw[line width=1mm,line cap=round,black!60](A) -- +(150:1.5cm);}]
    tcolorbox
  \tcbitem[fontupper=\large\bfseries,sharp corners=west]
    Manual for\\ version\\ \version\\(\datum)
  \tcbitem[sharp corners=northeast]
  \tcbitem[sharp corners=northwest] Thomas F.~Sturm
  \end{tcbitemize}%
};
\end{tikzpicture}
\end{inctext}
\end{tcboutputlisting}
\tcbuselistingtext
% \tcbinputlisting{title=Cover code,
%   base example,coltitle=black,fonttitle=\itshape,titlerule=0pt,
%   colbacktitle=Navy!15!ExampleBack,top=0mm,before=\par\smallskip,%
%   listing style=mydocumentation,listing only}

%\bigskip
%\begin{marker}
%If you have trouble printing this document, the reason is quite likely the
%cover page. Printing the pages starting with page 2 or page 3 should work.
%\end{marker}

\clearpage
\begin{center}
\begin{tcolorbox}[enhanced,hbox,tikznode,left=8mm,right=8mm,boxrule=0.4pt,
  colback=white,colframe=black!50!yellow,
  %drop fuzzy midday shadow=black!50!yellow,
  drop lifted shadow=black!50!yellow,arc is angular,
  before=\par\vspace*{5mm},after=\par\bigskip]
{\bfseries\LARGE The \texttt{tcolorbox} package}\\[3mm]
{\large Manual for version \version\ (\datum)}
\end{tcolorbox}
{\large Thomas F.~Sturm%
  \footnote{Prof.~Dr.~Dr.~Thomas F.~Sturm, Institut f\"{u}r Mathematik und Informatik,
    Universit\"{a}t der Bundeswehr M\"{u}nchen, D-85577 Neubiberg, Germany;
     email: \href{mailto:thomas.sturm@unibw.de}{thomas.sturm@unibw.de}}\par\medskip
\normalsize\url{https://www.ctan.org/pkg/tcolorbox}\par
\url{https://github.com/T-F-S/tcolorbox}}
\end{center}
\bigskip
\begin{absquote}
  \begin{center}\bfseries Abstract摘要\end{center}


|tcolorbox| provides an environment for colored and framed text boxes with a
heading line. Optionally, such a box can be split in an upper and a lower
part. The package |tcolorbox| can be used for the setting of \LaTeX\ examples where
one part of the box displays the source code and the other part shows the
output. Another common use case is the setting of theorems. The package supports
saving and reuse of source code and text parts.

% theorems




% |tcolorbox| 提供了一个环境,用于生成彩色的,带有标题行和边框的文本盒子。%
% 这个文本盒子环境的upper和lower两部分是可选的。
% |tcolorbox|包可用于排版 \LaTeX\ 示例,其中一部分显示 \LaTeX\ 源代码,另一部分显示对应的排版结果。
% 另外,此包也常常用于排版定理(theorems)。%
% 该包支持暂存源代码和文本,后面再重新引入重用。

|tcolorbox| 提供了一个带有标题行的彩色和带框文本框的环境。可选地,这样的框可以分为上下两个部分。|tcolorbox| 包可用于设置 \LaTeX\ 示例,其中框的一部分显示源代码,另一部分显示输出结果。另一个常见的用例是设置定理。该包支持保存和重用源代码和文本部分。
\end{absquote}


% 以下目录吧,TODO 解释以下latex代码:
% 这段LaTeX代码主要是使用 tcolorbox 宏包创建一个带有装饰和样式的框,其中包含文档的目录。
\begin{tcolorbox}[breakable,%这个选项使得该盒子可以在页面之间断开。
enhanced jigsaw,%这个选项允许盒子的形状和样式更复杂和灵活。 todo 
title={Contents目录},fonttitle=\bfseries\Large,
colback=yellow!10!white,colframe=red!50!black,%设置盒子的背景色和边框色。
before=\par\bigskip\noindent,%在盒子之前插入一个新段落、一个大空格,并取消缩进。
interior style={fill overzoom image=goldshade.png,fill image opacity=0.25},%设置盒子内部的样式,包括一个填充的图像和图像的不透明度。
colbacktitle=red!50!yellow!75!black,%设置标题背景的颜色。
enlargepage flexible=\baselineskip,pad at break*=3mm,%设置如何在页面之间断开盒子。 todo
height fixed for=first and middle,%设置第一行和中间行的高度固定。
watermark color=yellow!75!red!25!white,watermark text={\bfseries\Large Contents目录},%设置水印的颜色和文本。
attach boxed title to top center={yshift=-0.25mm-\tcboxedtitleheight/2,yshifttext=2mm-\tcboxedtitleheight/2},%将标题框架定位到顶部中心。
boxed title style={enhanced,boxrule=0.5mm,
frame code={ \path[tcb fill frame] ([xshift=-4mm]frame.west) -- (frame.north west)
-- (frame.north east) -- ([xshift=4mm]frame.east)
-- (frame.south east) -- (frame.south west) -- cycle; },
interior code={ \path[tcb fill interior] ([xshift=-2mm]interior.west)
-- (interior.north west) -- (interior.north east)
-- ([xshift=2mm]interior.east) -- (interior.south east) -- (interior.south west)
-- cycle;}  },%设置标题框的样式。
drop fuzzy shadow]%为盒子添加模糊阴影效果。
\makeatletter
\@starttoc{toc}%插入目录的内容。 @ 符号通常在 LaTeX 中被视为特殊字符,\makeatletter 和 \makeatother 允许我们在此上下文中使用它。
\makeatother
\end{tcolorbox}
%译 done 摘要和目录 v5 20230918


% % !TeX root = tcolorbox.tex

\section{Introduction\\介绍}%
%TODO external/prefix 是啥意思:
\tcbset{external/prefix=external/intro_}%
% external/prefix 是 tikz 宏包的 external 库中的一个键。这个键用于设置 external 库生成的外部图形文件的前缀。
% external 库用于将 tikzpicture 环境中的内容编译为独立的 PDF 文件,这样在重新编译整个文档时,如果这些环境的内容没有改变,LaTeX 可以直接插入已经编译好的 PDF,从而大大加快编译速度。

% \tcbset{external/prefix=external/intro_} 这个命令就是设置了 external 库生成的 PDF 文件的前缀为 external/intro_。这意味着,如果你有一个 tikzpicture 环境,它的标签(label)是 fig1,那么 tikz 将会生成一个名为 external/intro_fig1.pdf 的文件。

% 这个功能在处理包含大量 tikz 图形的大文档时非常有用,可以显著提高编译速度。

% \begin{stripedbox}
The package originates from %
the first edition of my book \flqq{\LaTeX -- Einführung in das Textsatzsystem}%
% {\citetitle{sturm:latex}
\frqq~%\cite{sturm:latex}
in about 2006.%
For the \LaTeX\ examples and tutorials given there, %
I wanted to have accentuated and colored boxes to display source code and
compiled text in combination.%
Since, in my opinion, %
this type of boxes is also quite useful to highlight definitions and theorems,% 
I applied them for my lecture notes in mathematics %\cite{sturm:mathe1,sturm:mathe2,sturm:mathe3}
as well.%
With this package, you are invited to apply these boxes for similar projects.
% 通过这个包,
% \tcblower
% 这个包起源于我2006年出版的\flqq{\LaTeX -- Einführung in das Textsatzsystem}%
% % 这个软件包来源于我在2006年出版的第一版书《 LATEX-Einfühung in das Textsatzsystem 》。
% % {\citetitle{sturm:latex}
% \frqq%~\cite{sturm:latex}
% 一书的第一版。%
% 对于书中给出的 \LaTeX\ 例子和教程,我想用突出的彩色方框来显示源代码和编译后的文本。%
% 因为,在我看来,这种类型的盒子,很适合突出定义和定理, 所以我也把它们应用到我的数学讲义 %\cite{sturm:mathe1,sturm:mathe2,sturm:mathe3} 
% 。%
% 您可以将这个宏包的这些盒子用于类似的项目中。

这个包源于我的书《\LaTeX -- Einführung in das Textsatzsystem》的第一版(约为2006年),其中提供了一些\LaTeX 的示例和教程。为了显示源代码和编译后的文本,我想要有突出和带颜色的框。由于在我看来,这种类型的框对于突出定义和定理也非常有用,因此我也将它们用于我的数学讲义中。通过这个包,您可以将这些框应用于类似的项目中。
% \end{stripedbox}

% \begin{stripedbox}
The breaking news for version 2.00 was the support for breakable boxes.%
This feature allows new applications of the package without affecting the core package too much if you do not need boxes to break automatically.%
With version 2.20, the often requested \enquote{side by side} mode for listings has been added.%
With version 3.00, boxed titles are introduced together with improved customization options for overlays, underlays, finishes, and own code extensions.%
% \tcblower

2.00版的爆炸性新闻是,支持换页盒子了。%
如果你不需要盒子自动换页,这个功能不太影响核心包。%不影响意思,去除
2.20版, 加上了经常被要求加上的,排版代码清单的 \enquote{side by side} 模式。%
在版本3.00中,盒标题会和改进了的,用于定制overlays(覆盖层)、underlays(衬垫)、finishes和自行扩展的选项一起介绍。
% \end{stripedbox}


% 笑脸 TODO 看下
\begin{tcolorbox}[enhanced,%允许我们使用高级特性,例如图形和图片。
boxrule=0mm,boxsep=0mm,%设置盒子边框的宽度和盒子与周围内容的间距为0,使得盒子不可见。
frame hidden,interior hidden,%隐藏盒子的边框和内部,这样盒子就完全透明了。
left=0mm,right=0mm,top=0mm,bottom=0mm,%设置盒子的左、右、上、下边距都为0。
watermark opacity=0.25,watermark zoom=1.2,%设置水印的透明度为25%,并放大水印的尺寸。
before=\par\smallskip,%在盒子之前插入一个新段落和一个小的垂直间距。
clip watermark=false,%不裁剪水印,使其可以超出盒子的边界。
watermark tikz={%
\path[fill=yellow,draw=yellow!75!red] (0,0) circle (1cm);%一个黄色的圆脸(circle (1cm))
\fill[red] (45:5mm) circle (1mm);\fill[red] (135:5mm) circle (1mm);%两个红色的眼睛(circle (1mm))
\draw[line width=1mm,red] (215:5mm) arc (215:325:5mm);}]%一个弧形的笑脸(arc (215:325:5mm))
% \begin{stripedbox}
Since the first public release in 2011, %
I received a lot of feedback from all over the world.%
I want to thank all who wrote me for supporting this package by sending bug reports and ideas for new or better features.
% \tcblower

自从2011年第一次公开发布以来,%
我收到了来自世界各地的大量反馈。%
我想感谢所有写信给我的人,感谢他们通过发送错误报告、和关于新的或更好的功能的想法来支持这个宏包。
% \end{stripedbox}
\end{tcolorbox}
    

% \hfill 
\subsection{Installation\\安装}

% \begin{stripedbox}
Typically, |tcolorbox| will be installed as part of a major \LaTeX\ distribution
and there is nothing special to do for a user.
% \tcblower

通常情况下,|tcolorbox| 会作为%主要的 
\LaTeX\ 发行版的一部分被安装,对用户来说没有什么特别的事情要做。
% \end{stripedbox}

% \begin{stripedbox}
If you intend to make a localinstallation \emph{by hand}, %
see the |README| file of the |tcolorbox| package for some hints. %
The short story is: you have to install not only |tcolorbox.sty|, %
but also all |*.code.tex| files in the local |texmf| tree.
% \tcblower

如果你打算\emph{手工}进行本地安装,%
请参阅tcolorbox软件包的README文件,以获得一些提示。%
简单的说: 你不仅要安装 |tcolorbox.sty| ,还要安装本地texmf目录中的所有 |*.code.tex| 文件。
% \end{stripedbox}

% % 
\subsection{Loading the Package\\加载包}


% \begin{stripedbox}
The base package |tcolorbox| loads the packages
|pgf| %\cite{tantau:tikz_and_pgf}
, |verbatim| %\cite{schoepf:2001a},
|etoolbox| %\cite{lehmann:etoolbox}
, and |environ| %\cite{robertson:2014a}
.
|tcolorbox| itself is loaded in the usual manner in the preamble:
% \tcblower

基础包 |tcolorbox| 加载了以下包:
|pgf| ,%\cite{tantau:tikz_and_pgf}, % todo 看
|verbatim| ,%\cite{schoepf:2001a},
|etoolbox| ,%\cite{lehmann:etoolbox},
和 |environ| .%\cite{robertson:2014a}.
% 
% 导言区中加载|tcolorbox|:
|tcolorbox| 本身则是在导言区(preamble)以常规方式加载的:
% \end{stripedbox}


\begin{dispListing}
\usepackage{tcolorbox}
\end{dispListing}


% \begin{stripedbox}
The package takes option keys in the key-value syntax.%
For example, the key to typeset listings is:
% \tcblower

该包的选项采用键值语法。例如,排版代码列表的键是。
%该包采用键值语法中的选项键。例如,设置列表的键是:
% \end{stripedbox}


\begin{dispListing}
\usepackage[listings]{tcolorbox}
\end{dispListing}

% \begin{stripedbox}
Alternatively, you may use these keys later in the preamble with \refCom{tcbuselibrary} (see there).
% \tcblower

% 可选的,
你也可以在导言区中通过使用 \refComLe{tcbuselibrary} 来设置这些键值。%
% \end{stripedbox}




% % \clearpage

\subsection{Libraries\hfill 库}\label{sec:bibliothek}

% \begin{stripedbox}
The base package |tcolorbox| is extendable by program libraries.%
This is done by using option keys while loading the package or inside
the preamble by applying the following macro with the same set of keys.
% \tcblower

基础的|tcolorbox|包还可以由程序库扩展功能。%
这可以在加载宏包时指定可选选项,或者在导言中使用以下命令宏来实现,传递的参数是相同的选项值:
% \end{stripedbox}

\begin{verbatim}
\begin{docCommand}{tcbuselibrary}{\marg{key list}}
...
\end{docCommand}  
\end{verbatim}


\begin{docCommand}{tcbuselibrary}{\marg{key list}}
% \begin{stripedbox}
Loads the libraries given by the \meta{key list}.
% \tcblower

加载由\makebox[0pt]{~}\meta{key list}\makebox[0pt]{~}给出的库。  
% \end{stripedbox} 

\begin{dispListing}
\tcbuselibrary{listings,theorems}
\end{dispListing}

% \begin{stripedbox}
The following keys are used inside |\tcbuselibrary| respectively
|\usepackage| without the key tree path |/tcb/library/|.
% \tcblower

下面的键值(不含 |/tcb/library/|)可以在|\tcbuselibrary|或|\usepackage|中使用。%%
% \end{stripedbox}

\end{docCommand}  


\begin{verbatim}
\tcbmakedocSubKey{docTcbKey}{tcb}%tcolorbox.doc.s_main.sty
\begin{docTcbKey}[library]{skins}{}{\mylib{skins}}
...
\end{docTcbKey}
\end{verbatim}

\begin{docTcbKey}[library]{skins}{}{\mylib{skins}}
% \begin{stripedbox}
Loads the package |tikz| %\cite{tantau:tikz_and_pgf} 
and provides additional styles (skins) for the appearance of the colored boxes; 
see  Section~\ref{sec:skins} from page~\pageref{sec:skins}.
% \tcblower

加载|tikz|\makebox[0pt]{~}%\cite{tantau:tikz_and_pgf} 
并为彩色框的外观提供其他样式(皮肤);
请参见第\pageref{sec:skins}页的~\ref{sec:skins}小节。
% \end{stripedbox}
\end{docTcbKey}

\begin{docTcbKey}[library]{vignette}{}{\mylib{vignette}}
% \begin{stripedbox}
Provides code for more ornamental; see
Section~\ref{sec:vignette} from page~\pageref{sec:vignette}.
% \tcblower

提供更多装饰性代码; 请参见第\pageref{sec:vignette}页的~\ref{sec:vignette}小节。

% % 提供更多装饰性代码,请参见第\ref{sec:vignette}节,从第\pageref{sec:vignette}页开始。
% \end{stripedbox}
\end{docTcbKey}

\begin{docTcbKey}[library]{raster}{}{\mylib{raster}}
% \begin{stripedbox}
Provides additional macros and options for typesetting 
multiple boxes arranged in a kind of raster;
see Section~\ref{sec:raster} from page~\pageref{sec:raster}.
% \tcblower

提供额外的宏和选项排版多个盒子,以一种栅格\footnotemark%
的形式排列。请参见第\pageref{sec:raster}页的~\ref{sec:raster}小节。
% \end{stripedbox}
\end{docTcbKey}
\footnotetext{栅格系统英文为“grid systems”,也有人翻译为“网格系统”,运用固定的格子设计版面布局,其风格工整简洁,在二战后大受欢迎,已成为今日出版物设计的主流风格之一。}
% 栅 格 (zhà gé)
    
\begin{docTcbKey}[library]{listings}{}{\mylib{listings}}

Loads the package |listings| %\cite{hoffmann:listings}
and provides additional macros for typesetting listings which are described in %Section~\ref{sec:listings} from 
page~\pageref{sec:listings}.
% \tcblower

载入|listings|包,并提供额外的用于代码排版的宏,详见~\pageref{sec:listings}页。%的~\ref{sec:listings}小节。

\end{docTcbKey}

\begin{docTcbKey}[library]{listingsutf8}{}{\mylib{listingsutf8}}

Loads the packages |listings| %\cite{hoffmann:listings}
 and |listingsutf8| %\cite{oberdiek:listingsutf8}
  for UTF-8 support.
This is a variant of the library \mylib{listings}
and is described in %Section \ref{sec:listings} from 
page~\pageref{sec:listings}.
% \tcblower

载入 |listings| 和用于支持UTF-8的 |listingsutf8| 包。这是\mylib{listings}的一个变体。%
详见~\pageref{sec:listings}页%
% 的~\ref{sec:listings}小节
。

\end{docTcbKey}

\begin{docTcbKey}[library]{minted}{}{\mylib{minted}}

Loads the package |minted| %\cite{poore:minted} 
to typeset listings with the |Pygments| %\cite{pygments:web}
 tool, also see \Vref{sec:listings}.

 %\tcblower
加载用 |Pygments| %\cite{pygments:web} 
排版代码的|minted|包。另见\Vref{sec:listings}。

\end{docTcbKey}

\begin{docTcbKey}[library]{theorems}{}{\mylib{theorems}}

Provides additional
macros for typesetting theorems which are described in %Section~\ref{sec:theorems}
% from 
page~\pageref{sec:theorems}.

 %\tcblower
为排版定理提供额外的宏,详见~\pageref{sec:theorems}页%的~\ref{sec:theorems} 小节
。

\end{docTcbKey}

% \tcbmakedocSubKey{docCodeKey}{代码}%tcolorbox.doc.s_main.sty
% \begin{docCodeKey}[]{skins}{}{\mylib{skins}}
% ...
% \end{docCodeKey}

\begin{docTcbKey}[library]{breakable}{}{\mylib{breakable}}

Provides support for automatic box breaking from one page to another;
see \Fullref{sec:breakable}.

%  %\tcblower
% 为 |tcolorbox| 盒子提供了自动分页的支持。见\Fullref{sec:breakable}。

提供自动在页面之间进行盒子换行的功能;参见\Fullref{sec:breakable}。

% 提供自动分页框的断行支持;参见\Fullref{sec:breakable}。
\end{docTcbKey}


% % 仓库
\begin{docTcbKey}[library]{magazine}{}{\mylib{magazine}}

Provides support for storing broken box parts to be used later or
in interchanged order, \Fullref{sec:magazine}.

% \tcblower
% 为储存盒子的分开的各个部分提供支持,以便以后使用或互换顺序,见\Fullref{sec:magazine}。
提供支持存储分页盒子的各个部分,以便后续重用或调整顺序,详见\Fullref{sec:magazine}。

\end{docTcbKey}

% tcolorbox宏包的magazine功能可以实现多页文档中盒子(box)的断页和重组。

% 其主要特点包括:

% 支持盒子跨页断行,并自动存储断行后的各个盒子片段。

% 可以调整跨页盒子片段的顺序,实现盒子内容的重组。

% 断行时可以在页尾和次页页眉插入自定义内容。

% 盒子片段之间可以插入新内容或其他盒子。

% 支持重复打印盒子片段。

% 允许跨页盒子自由浮动。

% 断行点可以精确指定。

% 支持两列和多列盒子的断行。

% 可灵活控制新页的样式。

% 与其他tcolorbox功能集成良好。

% magazine功能强大而灵活,非常适合制作杂志、报纸、文学作品等多页印刷品。用户可以充分发挥创意,自由安排页面元素。

% magazine | BrE maɡəˈziːn, AmE ˈmæɡəˌzin |
% noun
% ① (publication) 杂志 zázhì
% ▸ a literary magazine
% 文学期刊
% ② (on radio, TV) (also magazine programme) 专题节目 zhuāntí jiémù
% ③ (of gun) 弹仓 dàncāng
% ④ (store for arms, ammunition) 弹药库 dànyàokù


\begin{docTcbKey}[library]{poster}{}{\mylib{poster}}

Provides support for creating posters, \Fullref{sec:poster}.

% \tcblower
为创作海报提供支持, \Fullref{sec:poster}。

\end{docTcbKey}

% poster是tcolorbox提供的一种特殊框样式,用于生成科学海报或宣传海报。其主要特征和功能如下:

% 提供了专门的海报样式和布局,如大标题、摘要部分、图文内容等。

% 支持多栏布局,可以自由调整各部分的位置。

% 可插入大量图片和图表。提供了专门的样式控制图片大小、边框、标题等。

% 支持插入代码语法高亮。

% 可以添加参考文献列表。

% 提供了不同的配色方案。支持大面积背景颜色。

% 自带页眉页脚样式。

% 支持生成高清PDF输出。

% 可添加QR码链接到在线资源。

% 与Beamer风格集成,可直接生成海报展示文稿。

% 使用poster可以快速生成精美的学术海报。它极大地简化了海报制作的工作,使得正常的LaTeX文稿可以一键转换为海报格式。这为学生和学者制作学术会议海报提供了很大的便利。



\begin{docTcbKey}[library]{fitting}{}{\mylib{fitting}}

Provides support for font size adaption of the box content to
the box dimensions;
see Section~\ref{sec:fitting} from page~\pageref{sec:fitting}.

% \tcblower
提供对盒子内的字体大小适应盒子尺寸的支持。详见~\pageref{sec:fitting}页的~\ref{sec:fitting}小节。

\end{docTcbKey}

% tcolorbox宏包中的fitting功能用于让框自动适应内部内容的大小。

% fitting的主要特点包括:

% 自动扩大或缩小框的大小,使其刚好容纳内容。

% 支持最小宽度和最大宽度设置,避免框的大小超出合理范围。

% 可选择仅适应内容宽度或同时适应高度。

% 可以立即适应内容变化,也可以延迟到下一页适应。

% 可配置各边距的自动收缩和扩展。

% 配合其他样式使用,如列表、定理环境等。

% 支持内嵌框的fitting。

% 可以自动插入分页,使过长内容分页显示。

% 可控制分页时的最小框高。

% 适应算法高效,即使大内容也能快速计算。

% fitting功能极大地方便了框的大小调整,用户无需手动设置尺寸,框可以自适应地包裹内容。这对讲义、解题等场合特别有用。



\begin{docTcbKey}[library]{hooks}{}{\mylib{hooks}}

Extends several option keys to \enquote{hookable} keys;
see Section~\ref{sec:hooks} from page~\pageref{sec:hooks}.

% \tcblower
将几个选项键扩展为 \enquote{hookable} 键。详见~\pageref{sec:hooks}页的~\ref{sec:hooks}小节。

\end{docTcbKey}

% tcolorbox中的hooks功能允许用户在框的不同阶段执行代码钩子(hook)。

% 主要的hooks包括:

% before upper:框开始之前;
% after upper:框结束内容之后;
% before lower:框开始内容之前;
% after lower:框结束之前;
% before title:标题开始之前;
% after title:标题结束之后;
% at begin document:文档开始时;
% at end document:文档结束时。
% 用户可以通过这些钩子实现如下功能:

% 在框周围插入内容
% 框前后修改计数器
% 保存框内容以备重用
% 在特定位置插入deco
% 框内创建样式或长度命令
% 定制页面样式
% 记录框的位置信息
% 在文档结束时打印所有框的列表
% hooks极大提高了tcolorbox的灵活性,用户可以在任何阶段进行干预和修改。这对实现特定的定制效果非常有用。


\begin{docTcbKey}[library]{xparse}{}{\mylib{xparse}}

Provides document command production with |xparse| for |tcolorbox|;
see Section~\ref{sec:xparse} from page~\pageref{sec:xparse}.

% \tcblower
为tcolorbox提供来自xparse的文档命令。详见~\pageref{sec:xparse}页的~\ref{sec:xparse}小节。

\end{docTcbKey}

% xparse是一个可扩展的LaTeX语法解析包,它提供了定制新命令语法的能力。主要功能包括:

% 可以方便地定义全新语法构造,如新的选项参数风格、环境等。

% 支持连字符、星号等符号作为关键字标记。

% 参数可以有默认值,从而定义短命令形式。

% 参数可以标记为必须提供或可选提供。

% 支持参数processor,在解析阶段处理参数。

% 定制参数解析错误的反馈信息。

% 定制语法方式,如Math模式下的语法解析。

% 提供了许多预定义的有用句法。

% 易于全局修改和调试语法。

% 语法定义仅需要很少的代码。

% xparse让LaTeX的语法可定制化,用户可以方便地开发出富有表达力的新命令,大大提高了LaTeX的可扩展性。

% 许多宏包如tcolorbox、float都依赖xparse来定制语法,xparse可以说是一个非常强大和重要的LaTeX3语法解析工具。



\begin{docTcbKey}[library]{external}{}{\mylib{external}}

Provides externalization support for stand-alone document snippets,
see \Fullref{sec:external}.

% \tcblower
为独立的文档片段提供外部化支持。见\Fullref{sec:external}。

\end{docTcbKey}

% tcolorbox宏包的external库提供了将框内容保存在外部文件的功能。

% 其主要特点包括:

% 框内容可以存储为单独的tex文件,从而避免多次编译相同内容。

% 支持部分页的内容外化,可以大大减少编译用时。

% 修改外化内容后可选择增量编译。

% 外部文件存储为轻量级的plain tex。

% 可指定存储目录,整理框文件。

% 可配置仅外化特定类别的框。

% 支持代码库的协作和版本控制。

% 定制文件名和目录结构。

% 可在preamble中预编译外化内容。

% 提供完善的编译信息和日志。

% external功能对编译大文档和课程笔记尤为有用。它实现了内容和样式的分离,使维护文档更加容易。



\begin{docTcbKey}[library]{documentation}{}{\mylib{documentation}}

Provides additional macros for typesetting \LaTeX\ documentations
which are described in Section~\ref{sec:documentation}
from page~\pageref{sec:documentation}. 

% \tcblower
为排版 \LaTeX\ 教程提供额外的宏。详见~\pageref{sec:documentation}页的~\ref{sec:documentation}小节。

\end{docTcbKey}

% tcolorbox的documentation库提供了生成包文档的功能。主要特点包括:

% 以示例驱动的文档风格。每个示例都是一个可编译的小框。

% 框自动编号,可插入标题和描述。

% 支持categorize将示例划分为组。

% 可定制索引、表格和列表。

% 内置样式支持参数表、代码和图片。

% 可提取文档源码,生成独立的示例文档。

% 支持文档版本和修订记录。

% 可导入外部文件和预定义示例。

% 示例可包含动画帧演示功能。

% 无缝衔接到tcolorbox样式和排版。

% documentation能大大降低写文档的工作量。用户可以用示例驱动的方式组织文档,无需重复描述功能。这是tcolorbox宏包的标准文档体例。

% documentation还可以用于学术写作中的示例环节,以示例代替泛泛的描述,提高文章质量。



\begin{docTcbKey}[library]{many}{}{style, no value}

Loads the libraries \mylib{skins}, \mylib{breakable}, \mylib{raster}, \mylib{hooks},
\mylib{theorems}, \mylib{fitting}, and \mylib{xparse}.
Use this shortcut, if you want to use all features of |tcolorbox|
with exception of typesetting listings and using
the specialized \mylib{documentation} library.

% \tcblower
加载\mylib{skins}、\mylib{breakable}、\mylib{raster}、\mylib{hooks}、%
\mylib{theorems}、\mylib{fitting} 和 \mylib{xparse}.%
如果你除了排版代码和使用专门的\mylib{documentation}包之外,想使用tcolorbox的所有功能,请使用这个快捷选项。

\end{docTcbKey}

% tcolorbox宏包的many库提供了高效创建多个相似框的功能。主要特点包括:

% 用一个命令同时创建多框,避免重复代码。

% 框可自动计数,插入标题。

% 支持统一修改所有框的样式。

% 可基于循环自动生成框内容。

% 框内容可来自外部文件。

% 可交替两种样式创建框。

% 在文档结构中正确放置多框。

% 支持列表和矩阵形式排列多框。

% 可分页和分栏插入多框。

% 配合其他功能使用,如exam卷生成多题。

% many功能对创建讲义、练习题、图表矩阵等多框场合非常方便高效。它极大地减少了重复工作,使内容生成系统化。

% 也可通过many批量创建一致风格的框,再手工调整其中的少数框。这比全部手工创建提高了效率。

\begin{docTcbKey}[library]{most}{}{style, no value}

Loads all libraries except \mylib{minted} and \mylib{documentation}.
Use this shortcut, if you want to use all features of |tcolorbox|
with exception of using the |minted| package and using
the specialized \mylib{documentation} library.

% \tcblower
加载除 \mylib{minted} 和 \mylib{documentation} 之外的所有包。如果你想使用tcolorbox的所有功能,除了使用 \mylib{minted} 和 \mylib{documentation} 包之外,请使用这个快捷选项。
\end{docTcbKey}

\begin{docTcbKey}[library]{all}{}{style, no value}

Loads all libraries. Use this shortcut only, if you intend to use the
\mylib{documentation} library.

% \tcblower
加载所有的包。只有当你打算使用 \mylib{documentation} 包时使用这个快捷选项。

\end{docTcbKey}%译  done v4 2023.3.11 todo 有额外说明,后面可以整理出来
% \setcounter{section}{1}
% \setcounter{subsection}{3}
% \setcounter{subsubsection}{0}
% \setcounter{page}{5}
% \input{tcolorbox.doc.quickref}%译  done 2023.1.8 

% \setcounter{section}{2}
% \setcounter{subsection}{0}
% \setcounter{subsubsection}{0}
% \setcounter{page}{5}
% % !TeX root = /Users/virhuiai/hlProjects/Latex-Typesetting-Hub/宏包文档翻译/tcolorbox/tcolorbox.tex
\setcounter{section}{2}
\section{Macros for Box Creation\\创建盒子的宏命令}%
\tcbset{external/prefix=external/coremacros_}%

% todo 左右的,按内容自动分配
% \begin{tcolorbox}[before skip=\baselineskip,after skip=\baselineskip,sidebyside
% ,boxsep=0pt,boxrule=0pt,arc=0mm
% ,nobeforeafter
% % ,breakable %会干掉 sidebyside ,变成上下的
% ,opacityframe=0%0是完全透明了,1是完全不透明
% % ,opacityback=0
% ]
% a
% \tcblower
% b
% \end{tcolorbox}


% % 
% % % ,colframe=red!75!black

% 
 
% \include*{coremacros/tcolorboxEnv}%2023-0920 %v2 2023 10 09                    /
% \include*{coremacros/tcblowerCmd}%2023-0920 %v2 2023 10 09                    /
% \include*{coremacros/tcbsetCmd}%2023-0920 %v2 2023 10 09       /v3 2023 1029            /
% \include*{coremacros/tcbsetforeverylayerCmd}%【v3-20231029】%2023-0920  %v2 2023 10 09 /v3-20231029
% \include*{coremacros/tcboxCmd}%【v3-20231029】%2023-0921 %v2 2023 10 09                    /v3-20231029
\include*{coremacros/newtcolorboxCmd}%2023-0921 todo savedelimiter 后过来加                    /
% \include*{coremacros/newtcboxCmd}%2023-0921 %v2 2023 10 09                    /
% \include*{coremacros/tcolorboxenvironmentCmd}%2023-0921 %v2 2023 10 09                    /





% %译 done v3 2023.3.11  v4 2023.0322 v5 2023.0921
 
% \setcounter{section}{3}
% \setcounter{subsection}{0}
% \setcounter{subsubsection}{0}
% \setcounter{page}{13} 
% % !TeX root = tcolorbox.tex
% include file of tcolorbox.tex (manual of the LaTeX package tcolorbox)
% \clearpage
\section{Option Keys}\label{sec:optkeys}%
\tcbset{external/prefix=external/coreoptions_}%
For the \meta{options} in \refEnv{tcolorbox} respectively \refCom{tcbset}
the following |pgf| keys can be applied. The key tree path |/tcb/| is not to
be used inside these macros. It is easy to add your own style keys using
the syntax for |pgf| keys, see \cite{tantau:tikz_and_pgf,sturm:latex} or the examples
starting from page~\pageref{sec:latextutorial}.

% For the \meta{options} in  respectively 
以下的选项可以在\refEnv{tcolorbox}和\refCom{tcbset}中使用。当使用这些命令时,选项路径|/tcb/|是不需要要。
使用 |pgf| 选项的语法可以很容易添加您自己的样式, 例子见~\pageref{sec:latextutorial}页的 \cite{tantau:tikz_and_pgf,sturm:latex} .


% \subsection{Title}% 是相对于使用input语句所在文件 todo 
% \subsection{Title}% 是相对于使用input语句所在文件 todo 

  % \begin{docTcbKey}{title}{=\meta{text}}{no default, initially empty}
\begin{docTcbKey}{title}{=\meta{文本}}{无默认值,初始为空}
Creates a heading line with \meta{text} as content.

创建以 \meta{text} 作为内容的标题行。

\begin{exdispExample*}{title}{sbs,lefthand ratio=0.6}
\begin{tcolorbox}[title=我的标题行]
这是一个\textbf{tcolorbox}。
\end{tcolorbox}
\end{exdispExample*}
\end{docTcbKey}

\begin{docTcbKey}{notitle}{}{no value, initially set}
Removes the title line if set before.

移除之前设置的标题行。
\end{docTcbKey}

\begin{docTcbKey}{adjusted title}{=\meta{text}}{style, no default, initially unset, 标题高度等高}
Creates a heading line with \meta{text} as content. The minimal height of
this line is adjusted to fit the text given by \refKey{/tcb/adjust text}.
This option makes sense
for single line headings if boxes are set side by side with equal height.
Note that it is very easy to trick this adjustment.


% 创建一个以 \meta{text} 为内容的标题行。 此行的最小高度根据 \meta{text} 自适应。
% 此选项适用于希望让单行放置多个并排的盒子拥有相同的标题高度。
% 注意,to trick 此调整是非常容易的。

创建一个标题行,其中内容为\meta{text}。此行的最小高度会根据\refKey{/tcb/adjust text}中给定的文本进行调整。如果希望让单行放置多个并排的盒子拥有相同的标题高度,则此选项很有意义。请注意,很容易欺骗此调整。
% 诀窍
% 技巧
% 骗局
% 魔术
% 神奇手法
% 欺骗
% 诡计
% 奥妙
% 狡猾手段
% 玩意儿

% exdispExample 
% 这是一个tcolorbox宏包提供的环境,用于生成一个例子展示。runs=2表示该例子最多运行2次,{adjusted_title}是该例子的标题。
\begin{exdispExample}[runs=2]{adjusted_title}
\tcbset{colback=White,arc=0mm,width=(\linewidth-4pt)/4,
equal height group=AT,before=,after=\hfill,fonttitle=\bfseries}

以下标题是not adjusted:\\
\foreach \n in {xxx,ggg,AAA,\"Agypten}
{\begin{tcolorbox}[title=\n,colframe=red!75!black]
一些文本。\end{tcolorbox}}
现在,我们再次尝试adjusted titles:\\
\foreach \n in {xxx,ggg,AAA,\"Agypten}
{\begin{tcolorbox}[adjusted title=\n,colframe=blue!75!black]
一些文本。\end{tcolorbox}}
\end{exdispExample}
\end{docTcbKey}

% todo 
% \tcbset{colback=White,arc=0mm,width=(\linewidth-4pt)/4,equal height group=AT,before=,after=\hfill,fonttitle=\bfseries}:
%这是tcolorbox宏包提供的一个设置环境,可以设置后面生成的tcolorbox的一些属性。
%其中,colback表示背景颜色,arc表示圆角半径,width表示盒子的宽度,
%equal height group表示将多个盒子的高度设置成相同,before和after表示在盒子前后分别添加一些内容,fonttitle表示标题的字体。

% \foreach \n in {xxx,ggg,AAA,\"Agypten}:
% 这是一个foreach循环,用于循环生成tcolorbox盒子。\n表示循环变量,{xxx,ggg,AAA,\"Agypten}是一个包含了4个元素的列表。
% \begin{tcolorbox}[title=\n,colframe=red!75!black]一些文本。\end{tcolorbox}:这是一个tcolorbox环境,用于生成一个盒子。
%title表示盒子的标题,colframe表示边框颜色,一些文本。是盒子中的内容。

% \begin{tcolorbox}[adjusted title=\n,colframe=blue!75!black]一些文本。\end{tcolorbox}:这是一个tcolorbox环境,用于生成一个盒子。
%adjusted title表示盒子的标题,colframe表示边框颜色,一些文本。是盒子中的内容。




\begin{docTcbKey}{adjust text}{=\meta{text}}{no default, initially \texttt{\"Apgjy}}
This sets the reference text for \refKey{/tcb/adjusted title}. If your texts
never exceed \enquote{\"Apgjy} in depth and height you don't need to care about this option.

用于为\refKey{/tcb/adjusted title}设置参考文本。%
如果你的文本不会超过\enquote{\"Apgjy}的深度和高度,你就不需要关心这个选项。

%%%TODO 弄些例子出来看到底啥意思 
% \begin{exdispExample}[runs=2]{adjust_text}
%   \foreach \n in {xxx,ggg,AAA,\"Agypten}
%   {\begin{tcolorbox}[adjust text=\n,title=test,colframe=blue!75!black,width=(\linewidth-4pt)/4]
%   一些文本。\end{tcolorbox}}
% \end{exdispExample}

\end{docTcbKey}

%squeezed 挤压
\begin{docTcbKey}[][doc new=2014-11-24]{squeezed title}{=\meta{text}}{style, no default, initially unset,挤压标题,宽度限最大宽}
Creates a single heading line with \meta{text} as content.
If the \meta{text} is longer than the available space, the text is
squeezed to fit into the available space.

创建一个标题行,内容为 \meta{text}。%
如果 \meta{text} 比可用空间长, 文本将被压缩以适应可用的空间。
\begin{exdispExample}{squeezed_title}
% \tcbuselibrary{raster}
\begin{tcbitemize}[
raster columns=3,%三列
raster equal height,%等高
colframe=red!75!black,colback=red!5!white,%框与背景色
fonttitle=\bfseries%标题加格式
]
\tcbitem[squeezed title={短标题}]
第一个例子
\tcbitem[squeezed title={这是一个非常非常长的标题}]
第二个例子
\tcbitem[squeezed title={这个标题显然对这个例子来说太长了}]
第三个例子
\end{tcbitemize}
\end{exdispExample}
\end{docTcbKey}



% 这是一个使用tcolorbox宏包的例子,展示了如何使用tcbitemize环境创建一个三列等高的项目列表,并对项目的标题进行压缩。
% 具体解释如下:
% 首先加载tcolorbox宏包的raster库,以便使用raster columns和raster equal height选项。

% 然后创建一个tcbitemize环境,并设置raster columns=3和raster equal height选项,以创建一个三列等高的项目列表。

% 接下来,对每个项目使用\tcbitem命令创建一个项目,并使用squeezed title选项压缩项目标题。在本例中,第一个项目的标题为“短标题”,第二个项目的标题为“这是一个非常非常长的标题”,第三个项目的标题为“这个标题显然对这个例子来说太长了”。

% 最后,为了美化效果,设置colframe和colback选项,分别为项目框和背景色添加红色调,并使用fonttitle选项为项目标题添加粗体格式。



%这个 todo 未定义的命令,可能是  tcbitemize 那边才有
% begin{tcolorbox}[squeezed title={BrE əˈdʒʌst, AmE əˈdʒəst},colframe=red!75!black] 
% 一些文本。
% \end{tcolorbox}
% begin{tcolorbox}[squeezed title={BrE skwiːz, AmE skwiz},colframe=red!75!black] 
% 一些文本。
% \end{tcolorbox}
% begin{tcolorbox}[squeezed title={这个标题显然对这个例子来说太长了},colframe=red!75!black] 
% 一些文本。
% \end{tcolorbox}

%试了音标的字体
% \begin{tcbitemize}[
% raster columns=3,%三列
% raster equal height,%等高
% colframe=red!75!black,colback=red!5!white,%框与背景色
% fonttitle=\bfseries%标题加格式
% ]
% \tcbitem[squeezed title={BrE əˈdʒʌst, AmE əˈdʒəst}]
% adjust
% \tcbitem[squeezed title={BrE skwiːz, AmE skwiz}]
% squeeze
% \tcbitem[squeezed title={这个标题显然对这个例子来说太长了}]
% 第三个例子
% \end{tcbitemize}

\begin{docTcbKey}[][doc new=2014-11-24]{squeezed title*}{=\meta{text}}{style, no default, initially unset}
This is a combination of \refKey{/tcb/adjusted title} and  \refKey{/tcb/squeezed title}.

这是  \refKey{/tcb/adjusted title} 和  \refKey{/tcb/squeezed title} 的组合。即高度和宽度都 \dots
\begin{exdispExample}{squeezed_title_2}
% \tcbuselibrary{raster}
\begin{tcbitemize}[raster columns=3,raster equal height,
  colframe=red!75!black,colback=red!5!white,fonttitle=\bfseries]
\tcbitem[squeezed title*={Short title}]
  First box
\tcbitem[squeezed title*={This is a very very long title}]
  Second box
\tcbitem[squeezed title*={This title is clearly to long for this application}]
  Third box
\end{tcbitemize}
\end{exdispExample}
\end{docTcbKey}

\begin{docTcbKey}[][doc new=2019-03-01]{titlebox}{=\meta{mode}}{no default, initially \texttt{visible}}
Controls the treatment of the title part of the box.
Feasible values for \meta{mode} are:

% 控制盒子的标题部分的处理。可设的 \meta{mode} 值有:
控制盒子的标题部分的处理方式。 \meta{mode} 的可行值为:

\begin{DescriptionR}{\docValue{invisible}}
\item[\docValue{visible}]usual type setting of the title box,\\
对带标题盒子的常用的设置,
\item[\docValue{invisible}]empty space instead of the title contents.\\
使用空白代替标题内容。
\end{DescriptionR}

% \begin{DescriptionL}{\docValue{invisible}}
% \item[\docValue{visible}]usual type setting of the title box,\\
% 对带标题盒子的常用的设置,
% \item[\docValue{invisible}]empty space instead of the title contents.\\
% 使用空白代替标题内容。
% \end{DescriptionL}

% \begin{DescriptionLsqueezed}{\docValue{visible}}
% \item[\docValue{visible}]usual type setting of the title box,\\
% 对带标题盒子的常用的设置,
% \item[\docValue{invisible}]empty space instead of the title contents.\\
% 使用空白代替标题内容。
% \end{DescriptionLsqueezed}

% \begin{DescriptionR}{\docValue{invisible}}
% \item[\docValue{visible}]usual type setting of the title box,\\
% 对带标题盒子的常用的设置,
% \item[\docValue{invisible}]empty space instead of the title contents.\\
% 使用空白代替标题内容。
% \end{DescriptionR}

% \begin{description}
% \item[\docValue{visible}]usual type setting of the title box,\\
% 对带标题盒子的常用的设置,
% \item[\docValue{invisible}]empty space instead of the title contents.\\
% 使用空白代替标题内容。
% \end{description}
\begin{exdispExample}{titlebox}
\begin{tcolorbox}[title=我的不可见标题,
  titlebox=invisible]
这是一个\textbf{tcolorbox}.
\end{tcolorbox}
\end{exdispExample}

\begin{exdispExample}{visible_titlebox}
  \begin{tcolorbox}[title=我的可见标题,
    titlebox=visible]
这是一个\textbf{tcolorbox}.
  \end{tcolorbox}
  \end{exdispExample}
\end{docTcbKey}



% \clearpage
\begin{docTcbKey}{detach title}{}{no value}
Detaches the title from its normal position. The text of the title is
stored into \docAuxCommand{tcbtitletext} and the formatted title is
available by \docAuxCommand{tcbtitle}.
The main application is to move the title from its usual place to another one.
  
将标题从它的正常位置移开。标题的文本存储到 \docAuxCommand{tcbtitletext} 中,格式化的标题可以通过 \docAuxCommand{tcbtitle} 获得。主要的应用是将标题从它的惯例位置移动到另一个位置。

% 将标题从其正常位置移开。标题文本存储在\docAuxCommand{tcbtitletext}中,格式化的标题可通过\docAuxCommand{tcbtitle}获得。 主要应用是将标题从其通常位置移动到另一个位置。
 
\begin{exdispExample}{detach_title}
\newtcolorbox{mybox}[2][]{
colbacktitle=red!10!white,
colback=blue!10!white,
coltitle=red!70!black,
title={#2},fonttitle=\bfseries,#1}

\begin{mybox}{My title}
这是一个\textbf{tcolorbox}.
\end{mybox}
\begin{mybox}[detach title,%暂存标题内容到\tcbtitle
before upper={\tcbtitle\quad}%放到upper的前面
]{detach title}
这是一个\textbf{tcolorbox}。\footnotemark
\end{mybox}
\begin{mybox}[detach title,
after upper={\par\hfill\tcbtitle}%放到upper的后面
]{名人名字}
可以用于名人名言的内容。
\end{mybox}
\end{exdispExample}
\end{docTcbKey}
\footnotetext{译者idea:可以改造了当单条的description。}


  

\begin{docTcbKey}{attach title}{}{no value}
  Attaches the title to its normal position. This option is used to reverse
  \refKey{/tcb/detach title}.

  将标题位置重置到其正常位置。此选项用于反转 \refKey{/tcb/detach title}。
  \end{docTcbKey}
  
  
  \begin{docTcbKey}[][doc updated=2015-07-08]{attach title to upper}{\colOpt{=\meta{text}}}{style, default empty, initially unset}
Attaches the title to the begin of the upper part of the box content.
The optional \meta{text} is set between the formatted title and the box content.

将标题附加到框内容upper部分的开头。
可选的 \meta{text} 插入到标题和upper部分内容之间。
  \begin{exdispExample}{attach_title_to_upper}
  \newtcolorbox{mybox}[2][]{colbacktitle=red!10!white,
    colback=blue!10!white,coltitle=red!70!black,
    title={#2},fonttitle=\bfseries,#1}
   
  \begin{mybox}[attach title to upper={\ ---\ }]{My title}
    attach title to upper加值的情况,不仅将标题放到upper之前,还在标题和upper之间放入破折线。
  \end{mybox}
  \begin{mybox}[attach title to upper,after title={:\ }]{My title}
    attach title to upper不加值时,将标题位置放到upper部分之前,再使用after title在标题后加内容。
  \end{mybox}
  \end{exdispExample}
  \end{docTcbKey}



\bigskip
\begin{marker}
More title options are documented in \Vref{subsec:contentadditions}
and \Vref{subsec:skinboxedtitle}.

更多的标题相关配置文档内容见 \Vref{subsec:contentadditions}
和 \Vref{subsec:skinboxedtitle}.
\end{marker}


% % \clearpage
% \subsection{Subtitle\\副标题}
% Inside the box content, one or more subtitles can be added.
In general, a subtitle is a further \refEnv{tcolorbox} which inherits some color and geometry options from the enclosing box. 
It may be customized just like any other \refEnv{tcolorbox}.

% 在盒子内,可以添加一个或多个副标题。%
% 一般来说,副标题是一个从封闭盒子继承了一些颜色和几何选项的|tcolorbox|。%
% 它可以像是一个 |tcolorbox| 一样进行设置。  

在盒子内容内部,可以添加一个或多个副标题。 通常,副标题是一个进一步的\refEnv{tcolorbox},它从封闭盒子中继承了一些颜色和几何选项。 它可以像任何其他的\refEnv{tcolorbox}一样进行自定义。

\begin{docCommand}[doc new=2014-10-10]{tcbsubtitle}{\oarg{options}\marg{text}}
Used inside a \refEnv{tcolorbox} to add a subtitle box with the given \meta{text}.
which is formatted by several inherited properties of the enclosing box
by further settings from \refKey{/tcb/subtitle style}, and by the given \meta{options}.

在 \refEnv{tcolorbox} 中使用,用于添加一个带有给定 \meta{text} 的副标题框,该框使用封闭框的多个继承属性进行格式设置,并通过 \refKey{/tcb/subtitle style} 的进一步设置和给定的 \meta{options} 进行格式化。

% 在\refEnv{tcolorbox}内部,将\meta{text}添加为子标题盒子。
% 这是一个独立的\refEnv{tcolorbox},%
% 一些格式从封闭的盒子中继承过来的属性,%
% 也可以通过给\refKey{/tcb/subtitle style}指定的\meta{options}来设定。
\begin{exdispExample*}{tcbsubtitle_1}{%
sbs%是sidebyside的意思
,lefthand ratio=0.6%upper侧占的比例
}
\begin{tcolorbox}[title=我的标题,
colback=red!5!white,
colframe=red!75!black,
fonttitle=\bfseries]
  This is a \textbf{tcolorbox}.
\tcbsubtitle[before skip=\baselineskip]%
  {我的{\tt 副}标题}
进一步的文本。
\end{tcolorbox}
\end{exdispExample*}

% 这段代码使用了tcolorbox宏包,创建了一个带有标题和副标题的盒子。具体解释如下:
% \begin{exdispExample*}{tcbsubtitle_1}{% sbs%是sidebyside的意思 ,lefthand ratio=0.6%upper侧占的比例 }

% 这部分代码是使用exdispExample环境来显示代码和输出结果的,其中使用了sbs选项表示将代码和输出结果并排显示,使用lefthand ratio选项表示代码窗口占整个窗口的比例为0.6。

% \begin{tcolorbox}[title=我的标题, colback=red!5!white, colframe=red!75!black, fonttitle=\bfseries]

% 这部分代码创建了一个tcolorbox盒子,其中包含了一个标题“我的标题”。colback选项表示背景色为红色和白色混合,colframe选项表示边框颜色为红色和黑色混合,fonttitle选项表示标题的字体为粗体。

% This is a \textbf{tcolorbox}.

% 这部分代码在盒子中插入了一段文本“This is a tcolorbox”,其中\textbf命令让“tcolorbox”加粗。

% \tcbsubtitle[before skip=\baselineskip]% {我的{\tt 副}标题}

% 这部分代码创建了一个副标题“我的副标题”,使用了\tcbsubtitle命令。before skip选项表示副标题前面的垂直距离为一个基准行距(\baselineskip),{\tt 副}使用了typewriter字体。

% 进一步的文本。

% 这部分代码在盒子中插入了进一步的文本。

% \end{tcolorbox}

% 这部分代码表示tcolorbox盒子的结束。

\begin{exdispExample*}{tcbsubtitle_2}{sbs,lefthand ratio=0.6}
\begin{tcolorbox}[title=My title,
    colback=red!5!white,
    colframe=red!75!black,
    colbacktitle=yellow!50!red,
    coltitle=red!25!black,
    fonttitle=\bfseries]
  This is a \textbf{tcolorbox}.
\tcbsubtitle[before skip=\baselineskip]%
{我的{\tt 副}标题}
进一步的文本。
\end{tcolorbox}
\end{exdispExample*}
\end{docCommand}

\begin{docTcbKey}[][doc new=2014-10-10]{subtitle style}{=\meta{options}}{no default, initially empty}
Adds |tcolorbox| \meta{options} to the settings for \refCom{tcbsubtitle}.

向 |tcbsubtitle| 设置的副标题的 |tcolorbox| 的 \meta{options} 中设置选项。

\begin{exdispExample*}{subtitle_style}{sbs,lefthand ratio=0.6}
\begin{tcolorbox}[title=My title,
  colback=red!5!white,
  colframe=red!75!black,
  colbacktitle=yellow!50!red,
  coltitle=red!25!black,
  fonttitle=\bfseries,
  subtitle style={boxrule=0.4pt,
    colback=yellow!50!red!25!white} ]
  This is a \textbf{tcolorbox}.
\tcbsubtitle{我的子标题}
  Further text.
\tcbsubtitle[colback=green!50!red!25!white]%
{第二个子标题}
上面的子标题中的背景色覆盖了 |subtitle style| 的设置。
\end{tcolorbox}
\end{exdispExample*}
\end{docTcbKey}

% % \clearpage
% \subsection{Upper Part\\upper部分}
% \setcounter{section}{4}
\setcounter{subsection}{2}
\setcounter{subsubsection}{0}
\subsection{Upper Part\\upper部分} 

The text content of a \refEnvLe{tcolorbox} may be parted into a mandatory \emph{upper part}
and an optional \emph{lower part}. These parts are separated by
\refComLe{tcblower}. If there is no \refComLe{tcblower} present, there is no
\emph{lower part} and the \emph{upper part} forms the complete text content.


\refEnvLe{tcolorbox} 的文本内容包括一个必需的 \cshTBlineBbox[blue]{upper} 部分和一个可选的
\cshTBlineBbox[gray]{lower} 部分。它们由\refComLe{tcblower}分隔。如果没有\refComLe{tcblower},则没有\emph{lower}部分,而\emph{upper}部分构成完整的文本内容。

\begin{tcblisting}{%title={Snapshot of the staging area},
csh texdef result={
\mintinline[breaklines]{shell}|latexdef --tempdir /Volumes/RamDisk -p '[all]tcolorbox' --texoptions '-shell-escape' --before '\begin{tcolorbox}' --after '\end{tcolorbox}'  -s tcblower|
}
}
\tcblower:
macro:->\tcb@insert@after@part \end {tcb@savebox}\tcb@set@color {tcbcollower}\unless \iftcb@sidebyside \tcbdimto \tcb@w@lower {\tcb@innerwidth -\kvtcb@boxsep *2-\kvtcb@leftlower -\kvtcb@rightlower }\fi \tcb@hasLowertrue \let \tcb@insert@after@part =\tcb@insert@after@lower \ifx \kvtcb@savelowerto \@empty \let \tcb@startbox \tcb@savelowerbox \let \endtcolorbox \tcb@endboxanddraw \else \let \tcb@startbox \tcb@lowerverbatim \expandafter \let \csname end\kvtcb@savedelimiter \expandafter \endcsname \csname tcb@endlowerverbatimanddraw\endcsname \fi \tcb@startbox 
\end{tcblisting}

\begin{docTcbKey}[][doc new=2015-01-06]{upperbox}{=\meta{mode}}{no default, initially \texttt{visible}}
Controls the treatment of the upper part of the box. If there is no lower part,  this is the complete text content.
Feasible values for \meta{mode} are:

控制例子的upper部分的显示处理。如果没有lower部分, upper部分内容即是所有内容。
可设置的 \meta{mode} 值有:
\begin{DescriptionL}{\docValue{invisible}}
\item[\docValue{visible}]usual type setting of the upper part,
\par 可见,upper部分的常用设定
\item[\docValue{invisible}] empty space instead of the upper part contents.
\par 不可见,upper部分内容显示为空白。
\end{DescriptionL}
% \begin{exdispExample}{upperbox}
% \begin{tcolorbox}[upperbox=invisible,colback=white%
% ,title=upperbox设置为invisible且没有lower部分]
% 这是一个\textbf{tcolorbox}(但是是隐形的)。
% \end{tcolorbox}

% \begin{tcolorbox}[upperbox=invisible,colback=white%
% ,title=upperbox设置为invisible时只显示lower部分]
% 这是一个\textbf{tcolorbox}(但是是隐形的)。
% \tcblower
% 这是lower部分。
% \end{tcolorbox}
% \end{exdispExample}

\begin{dispExample*}{sbs}
\begin{tcolorbox}[upperbox=invisible%
,colback=white%
,title=upperbox设置为invisible%
且没有lower部分]
这是一个\textbf{tcolorbox}(但是是隐形的)。
\end{tcolorbox}
\end{dispExample*}

\begin{dispExample*}{sbs}
\begin{tcolorbox}[upperbox=invisible%
,colback=white%
,title=upperbox设置为invisible%
时只显示lower部分]
这是一个\textbf{tcolorbox}(但是是隐形的)。
\tcblower
这是lower部分。
\end{tcolorbox}
\end{dispExample*}
 
\end{docTcbKey}


\begin{docTcbKey}[][doc new and updated={2015-01-06}{2019-03-01}]{visible}{}{style, no value}
Shortcut for setting \refKeyLe{/tcb/upperbox}, \refKeyLe{/tcb/lowerbox}, and \refKeyLe{/tcb/titlebox}
to be \docValue{visible}.

同时设置 \refKeyLe{/tcb/upperbox}, \refKeyLe{/tcb/lowerbox}, 和 \refKeyLe{/tcb/titlebox} 为 \docValue{visible} 的快捷方式。
\end{docTcbKey}

\begin{docTcbKey}[][doc new and updated={2015-01-06}{2019-03-01}]{invisible}{}{style, no value}
Shortcut for setting \refKeyLe{/tcb/upperbox}, \refKeyLe{/tcb/lowerbox}, and \refKeyLe{/tcb/titlebox}
to be \docValue{invisible}.

设置 \refKeyLe{/tcb/upperbox}, \refKeyLe{/tcb/lowerbox}, 和 \refKeyLe{/tcb/titlebox} 为 \docValue{invisible} 的快捷方式。
\begin{exdispExample}{invisible}
\begin{tcolorbox}[invisible]
这是一个\textbf{tcolorbox}(但是是隐形的)。
\tcblower
这是\textbf{lower部分}(但是是隐形的)。
\end{tcolorbox}
\end{exdispExample}
\end{docTcbKey}




% \clearpage
\begin{docTcbKey}[][doc new=2015-05-04]{saveto}{=\meta{file name}}{no default, initially empty}
Saves the content of the box into a file for an optional later usage.
This is the counterpart of \refKeyLe{/tcb/savelowerto}, but is saves not
only the upper part but the whole content. If a lower part is present,
it is also saved including \refComLe{tcblower}.

% 将盒子的内容保存到一个文件中,以供以后使用。
% 这和 \refKeyLe{/tcb/savelowerto} 类似, 但它不仅保存了upper部分,是保存了整个内容。
% 如果存在lower部分,也会保存且包含 \refComLe{tcblower}。

将盒子的内容保存到文件中,以备将来需要时使用。这是\refKeyLe{/tcb/savelowerto}的对应项,但不仅保存upper部分,而是保存整个内容。如果存在lower部分,则也会保存,包括\refComLe{tcblower}。

\begin{marker}
This option cannot be combined with \refKeyLe{/tcb/savelowerto}.

此项不能同 \refKeyLe{/tcb/savelowerto} 组合使用.
\end{marker}

\begin{exdispExample}{saveto_1}
\begin{tcolorbox}[invisible%upper、lower都不显示
,saveto=\jobname_mysave1.tex
,colback=white
,before upper={before upper}]
这是一个\textbf{tcolorbox},使用invisible后,看着是空的。
它的内容被saveto暂存到指定的文件中,后续再在其他位置包含进来使用。
\end{tcolorbox}

现在,我们包含进来保存好的文本内容:\\
\input{\jobname_mysave1.tex}


\end{exdispExample}

% before upper

\begin{引述之言}{virhuiai}
在LaTeX中,\verb|\jobname| 代表当前文档的名称(不含扩展名)。
\end{引述之言}

\begin{引述之言}{virhuiai}
如果有\verb|before upper|,但设置了invisible,这部分内容不会写入saveto指定的文件中哦!
\end{引述之言}

% \begin{tcolorbox}[%invisible%upper、lower都不显示
% ,colback=white
% ,before upper={before upper}]
% 这是一个\textbf{tcolorbox},使用invisible后,看着是空的。
% 它的内容被saveto暂存到指定的文件中,后续再在其他位置包含进来使用。
% \end{tcolorbox}

% 在LaTeX中,\jobname代表当前文档的名称(不含扩展名):

% \jobname会展开为去除文件扩展名后的文档名称。

% 例如源文件名为paper.tex,\jobname会展开为paper。

% 如果文件名包含路径,也会去掉路径部分。

% \jobname不包含空格及特殊字符,全部转为小写字母。

% \jobname根据TeX引擎和运行方式有细微差异:

% tex命令默认为文件名。

% latex命令默认为文件名。

% xelatex默认为不含扩展名的文件名。

% \jobname通常用于生成与文档相关的外部文件。

% 也可在导言区重新定义\jobname的值。

% \jobname在标题、摘要、目录等地方也有应用。

% 综上,\jobname可以动态获取文档的名称,对文档编译和内容引用非常有用。它可以确保输出文件名与源文件同步更新。

\begin{exdispExample}{saveto_2}
\begin{tcolorbox}[saveto=\jobname_mysave2.tex]
这是一个\textbf{tcolorbox}.
\tcblower
这是lower部分。
\end{tcolorbox}

现在,我们包含进来保存好的文本内容:
\begin{tcolorbox}[colframe=red,colback=red!10,
coltitle=black,colbacktitle=red!20
,sidebyside%从上下改为左右
,title=在这里我们看到保存的内容包括lower部分]
\input{\jobname_mysave2.tex}
\end{tcolorbox}
\end{exdispExample}
\end{docTcbKey}

% 可以看下这个 mysave2.tex 文件,内容是:

% 这是一个\textbf{tcolorbox}.
% \tcblower
% 这是lower部分。

% % \clearpage
% % Lower Part\hfill 
% \subsection{lower部分}
% \setcounter{section}{4}
\setcounter{subsection}{3}
\setcounter{subsubsection}{0}

% Lower Part\hfill 
\subsection{lower部分}

\begin{docTcbKey}{lowerbox}{=\meta{mode}}{no default, initially \texttt{visible}}
Controls the treatment of the lower part of the box.
Feasible values for \meta{mode} are:

控制lower部分的显示情况。可选的 \meta{mode} 值有:
\begin{DescriptionL}{\docValue{invisible}}
\item[\docValue{visible}]usual type setting of the lower part,
\\可见,lower部分的常用设定

\item[\docValue{invisible}]empty space instead of the lower part contents,
\\不可见,lower部分内容显示为空白。

\item[\docValue{ignored}]the lower part is not used (here).
\\忽略,lower部分在这儿没有用上。
\end{DescriptionL}

The last two values are usually applied in connection with |savelowerto|.

\docValue{invisible}、\docValue{ignored} 通常用于与 |savelowerto| 配合使用。

\begin{exdispExample}{lowerbox}
\begin{tcolorbox}[lowerbox=invisible,colback=white]
This is a \textbf{tcolorbox}.
\tcblower
这是lower部分(但不可见)
\end{tcolorbox}

\begin{tcolorbox}[lowerbox=ignored,colback=white]
This is a \textbf{tcolorbox}.
\tcblower
这是lower部分(但ignored)
\end{tcolorbox}
\end{exdispExample}
\end{docTcbKey}


\begin{docTcbKey}[][doc updated=2014-11-28]{savelowerto}{=\meta{file name}}{no default, initially empty}
Saves the content of the lower part into a file for an optional later usage.

将lower部分的内容保存到一个文件中,以备以后使用。
\begin{exdispExample}{savelowerto}
\begin{tcolorbox}[lowerbox=invisible,savelowerto=\jobname_bspsave.tex,colback=white]
This is a \textbf{tcolorbox}.
\tcblower
这是可能相当复杂的lower部分:
$\displaystyle f(x)=\frac{1+x^2}{1-x^2}$.
\end{tcolorbox}

现在,我们加载保存的文本:\\
\input{\jobname_bspsave.tex}
\end{exdispExample}
\end{docTcbKey}



% \clearpage
\begin{docTcbKey}{lower separated}{\colOpt{=true\textbar false}}{default |true|, initially |true|}
If set to |true|, the lower part is visually separated from the upper part.
It depends on the chosen skin how the visualization of the separation is done.

如果设置为 |true|则lower部分与upper部分可见的分隔开来。
分隔的样式依赖于皮肤的选择。
% \enlargethispage*{1cm}


% \begin{exdispExample}{lower_separated_默认}
% \begin{tcolorbox}[title=Lower separated]
% This is the upper part.
% \tcblower
% This is the lower part.
% \end{tcolorbox}
% \end{exdispExample}

% sidebyside,righthand ratio=0.25

\begin{dispExample*}{sidebyside,righthand ratio=0.4}
\begin{tcolorbox}[title=Lower separated]
This is the upper part.
\tcblower
This is the lower part.
\end{tcolorbox}
\end{dispExample*}


\begin{dispExample*}{sidebyside,righthand ratio=0.4}
\begin{tcolorbox}[title=Lower not separated,%
lower separated=false]
upper部分。

设|lower separated=false|
\tcblower
这是lower部分。两部分的分隔没有线条出现。
\end{tcolorbox}    
\end{dispExample*}

\begin{dispExample*}{sidebyside,righthand ratio=0.4}
\begin{tcolorbox}[sidebyside,title={sidebyside}]
upper
\tcblower
lower
\end{tcolorbox}
\end{dispExample*}

\begin{dispExample*}{sidebyside,righthand ratio=0.4}
\begin{tcolorbox}[sidebyside,title={sidebyside}%
,lower separated=false]
upper
\tcblower
lower
\end{tcolorbox}
\end{dispExample*}

\begin{dispExample*}{sidebyside,righthand ratio=0.4}
\begin{tcolorbox}[beamer,title=Lower separated]
upper
\tcblower
lower
\end{tcolorbox}
\end{dispExample*}

\begin{dispExample*}{sidebyside,righthand ratio=0.4}
\begin{tcolorbox}[beamer,title=Lower not separated%
,lower separated=false]
upper
\tcblower
lower
\end{tcolorbox}
\end{dispExample*}


\end{docTcbKey}




% delimiter
% 定界符
% \clearpage
\begin{docTcbKey}{savedelimiter}{=\meta{name}}{no default, initially \texttt{tcolorbox}}
Used in connection with new environment definitions which extend
|tcolorbox| and use or allow the option |savelowerto|.
To catch the end of the new box environment \meta{name} has to be the name of
this environment. Additionally, the environment definition has to use
|\tcolorbox| instead of
|\begin{tcolorbox}| and |\endtcolorbox| instead of |\end{tcolorbox}|.

% 用于关联由 |newenvironment| 自定义的,拓展自 |tcolorbox| 的新环境中的 |savelowerto| 选项。%
% 要捕获新的盒子环境 \meta{name} 的结尾,必须是这个环境的名字。%
% 此外,环境定义必须使用 |\tcolorbox| 代替 |\begin{tcolorbox}|、用 |\endtcolorbox| 代替 |\end{tcolorbox}|。

用于与扩展了|tcolorbox|并使用或允许选项|savelowerto|的新环境定义相关联。为了捕捉新框环境的结尾,\meta{name}必须是此环境的名称。此外,环境定义必须使用|\tcolorbox|而不是|\begin{tcolorbox}|,并且使用|\endtcolorbox|而不是|\end{tcolorbox}|。

\begin{exdispExample}{savedelimiter1}
\newenvironment{mybox}[1]{%
\tcolorbox[savedelimiter=mybox,
            savelowerto=\jobname_bspsave2.tex,lowerbox=ignored,
            colback=red!5!white,colframe=red!75!black,fonttitle=\bfseries,
            title={#1}]}%
{\endtcolorbox}

\begin{mybox}{暂存 savelowerto 的内容的新环境}
Upper部分。
\tcblower
暂存的lower部分!
\end{mybox}

现在,使用之前暂存的部分:
\begin{tcolorbox}[colback=green!5,title=用到 savelowerto 暂存的内容]
\input{\jobname_bspsave2.tex}
\end{tcolorbox}
\end{exdispExample}

\begin{引述之言}{GPT}
在LaTeX中,tcolorbox是一个功能强大的宏包,用于创建漂亮的盒子和框架。其中,savedelimiter是tcolorbox宏包提供的一个选项,用于保存和恢复当前的分隔符。
\\[0.5em]
分隔符在tcolorbox中用于定义盒子的起始和结束位置。当使用tcolorbox环境时,可以指定不同的分隔符,如\verb|[]|、\verb|{}|等。然而,在某些情况下,可能需要在盒子内部使用其他环境或命令,这些环境或命令也使用相同的分隔符,这可能会导致分隔符冲突的问题。
\\[0.5em]
为了解决这个问题,tcolorbox提供了savedelimiter选项。使用savedelimiter选项,可以在tcolorbox中保存当前的分隔符状态,并在需要时恢复它。这样可以确保在盒子内部使用其他环境或命令时,分隔符不会产生冲突。
\end{引述之言}


% \enlargethispage*{1cm}

The |savedelimiter| is used implicitely with \refComLe{newtcolorbox} which
allows a more convenient usage:

\refComLe{newtcolorbox} 隐式使用了 |savedelimiter|,使用起来更方便:
\begin{exdispExample}{savedelimiter2}
\newtcolorbox{mybox}[1]{%
            savelowerto=\jobname_bspsave2.tex,lowerbox=ignored,
            colback=red!5!white,colframe=red!75!black,fonttitle=\bfseries,
            title={#1}}%

\begin{mybox}{My Example}
Upper part.
\tcblower
Saved lower part!
\end{mybox}

Now, the saved part is used:
\begin{tcolorbox}[colback=green!5]
\input{\jobname_bspsave2.tex}
\end{tcolorbox}
\end{exdispExample}
\end{docTcbKey}
% % \clearpage
% \subsection{Colors and Fonts\\颜色和字体}
% \subsection{Colors and Fonts\\颜色和字体}

\begin{docTcbKey}{colframe}{=\meta{color}}{no default, initially \texttt{black!75!white}}
Sets the frame \meta{color} of the box.

% \hspace{-0.35ex}
设置盒子边框的\meta{color}。
\begin{exdispExample*}{colframe}{sbs,lefthand ratio=0.6}
\begin{tcolorbox}[colframe=red!50!white]
This is a \textbf{tcolorbox}.
\end{tcolorbox}
\end{exdispExample*}
\end{docTcbKey}

\begin{docTcbKey}{colback}{=\meta{color}}{no default, initially \texttt{black!5!white}}
Sets the background \meta{color} of the box.

设置盒子的背景\meta{color}。
\begin{exdispExample*}{colback}{sbs,lefthand ratio=0.6}
\begin{tcolorbox}[colback=red!50!white]
This is a \textbf{tcolorbox}.
\end{tcolorbox}
\end{exdispExample*}
\end{docTcbKey}

另见 \mylib{skins} 库的 \refKeyLe{/tcb/colbacklower}。

\begin{docTcbKey}{title filled}{\colOpt{=true\textbar false}}{default |true|, initially |false|}
Switches the drawing of the title background according to the given value.
This option is set to |true| automatically by \refKeyLe{/tcb/colbacktitle},
\refKeyLe{/tcb/opacitybacktitle}, and \refKeyLe{/tcb/title style},
and \refKeyLe{/tcb/title code}.

根据给定的值切换标题部分背景色的绘制。
\refKeyLe{/tcb/colbacktitle},\refKeyLe{/tcb/opacitybacktitle}, \refKeyLe{/tcb/title style},
和 \refKeyLe{/tcb/title code} 设定时,会自动将此选项设置为 |true|。

\begin{exdispExample*}{title_filled}{sbs,lefthand ratio=0.6}
\begin{tcolorbox}[title=My title,title filled]
This is a \textbf{tcolorbox}.
\end{tcolorbox}
\begin{tcolorbox}[title=My title,
title filled=false]
This is a \textbf{tcolorbox}.
\end{tcolorbox}
\end{exdispExample*}
\end{docTcbKey}


\begin{docTcbKey}{colbacktitle}{=\meta{color}}{no default, initially \texttt{black!50!white}}
Sets the background \meta{color} of the title area of the box.

设置盒子的的标题区域的背景颜色。
\begin{exdispExample*}{colbacktitle}{sbs,lefthand ratio=0.6}
\begin{tcolorbox}[colbacktitle=red!50!white,
title=My title,coltitle=black,
fonttitle=\bfseries]
This is a \textbf{tcolorbox}.
\end{tcolorbox}
\end{exdispExample*}
\end{docTcbKey}



% \clearpage

\begin{docTcbKey}{colupper}{=\meta{color}}{no default, initially \texttt{black}}
Sets the text \meta{color} of the upper part.

设置upper部分的文本的颜色。
\begin{exdispExample*}{colupper}{sbs,lefthand ratio=0.6}
\begin{tcolorbox}[colupper=red!75!black]
This is a \textbf{tcolorbox}.
\tcblower
This is the lower part.
\end{tcolorbox}
\end{exdispExample*}
\end{docTcbKey}


\begin{docTcbKey}{collower}{=\meta{color}}{no default, initially \texttt{black}}
Sets the text \meta{color} of the lower part.

设置lower部分的文本的颜色。
\begin{exdispExample*}{collower}{sbs,lefthand ratio=0.6}
\begin{tcolorbox}[collower=red!75!black]
This is a \textbf{tcolorbox}.
\tcblower
This is the lower part.
\end{tcolorbox}
\end{exdispExample*}
\end{docTcbKey}


\begin{docTcbKey}{coltext}{=\meta{color}}{style, no default, initially \texttt{black}}
Sets the text \meta{color} of the box. This is an abbreviation for setting
|colupper| and |collower| to the same value.

设置盒子中文本的颜色。这是同时将 |colupper| 和 |collower| 的值设置为一个颜色。
\begin{exdispExample*}{coltext}{sbs,lefthand ratio=0.6}
\begin{tcolorbox}[coltext=red!75!black]
This is a \textbf{tcolorbox}.
\tcblower
This is the lower part.
\end{tcolorbox}
\end{exdispExample*}
\end{docTcbKey}


\begin{docTcbKey}{coltitle}{=\meta{color}}{no default, initially \texttt{white}}
Sets the title text \meta{color} of the box.

设置标题的文本的颜色。
\begin{exdispExample*}{coltitle}{sbs,lefthand ratio=0.6}
\begin{tcolorbox}[coltitle=red!75!black,
colbacktitle=black!10!white,title=Test]
This is a \textbf{tcolorbox}.
\end{tcolorbox}
\end{exdispExample*}
\end{docTcbKey}





% \clearpage

\begin{docTcbKey}{fontupper}{=\meta{text}}{no default, initially empty}
Sets \meta{text} before the content of the upper part (e.\,g.\ font settings).

附加 \meta{text} 到{\bf 上}部分的内容之前(例如字体设置)。

% 在upper部分的内容之前设置\meta{text}(例如字体设置)。
\begin{exdispExample}{fontupper}
\begin{tcolorbox}[fontupper=Hello!~\sffamily]
This is a \textbf{tcolorbox}.
\end{tcolorbox}
\end{exdispExample}
\begin{exdispExample}{fontupper2}
\begin{tcolorbox}[fontupper=Hello!~]
This is a \textbf{tcolorbox}.
\end{tcolorbox}
\end{exdispExample}
\end{docTcbKey}


\begin{docTcbKey}{fontlower}{=\meta{text}}{no default, initially empty}
Sets \meta{text} before the content of the lower part (e.\,g.\ font settings).

附加 \meta{text} 到lower部分的内容之前(e.\,g.\ 字体设置)。
\begin{exdispExample}{fontlower}
\begin{tcolorbox}[fontlower=\sffamily\bfseries]
This is a \textbf{tcolorbox}.
\tcblower
This is the lower part.
\end{tcolorbox}
\end{exdispExample}
\end{docTcbKey}


\begin{docTcbKey}{fonttitle}{=\meta{text}}{no default, initially empty}
Sets \meta{text} before the content of the title text (e.\,g.\ font settings).

附加 \meta{text} 到标题文本的内容之前(e.\,g.\ 字体设置)。
\begin{exdispExample}{fonttitle}
\begin{tcolorbox}[fonttitle=\sffamily\bfseries\large,title=Hello]
This is a \textbf{tcolorbox}.
\end{tcolorbox}
\end{exdispExample}
\begin{exdispExample}{fonttitle2}
\begin{tcolorbox}[fonttitle=只加了文字,title=Hello]
This is a \textbf{tcolorbox}.
\end{tcolorbox}
\end{exdispExample}
\end{docTcbKey}

\bigskip
\begin{marker}
More color options are provided by using skins documented in
Section \ref{sec:skins} from page \pageref{sec:skins}.

更多的颜色选项是通过使用 \pageref{sec:skins} 页第 \ref{sec:skins} 节中介绍的 |skins| 提供的。
\end{marker}
% % \clearpage
% % 
% \subsection{Text Alignment文本对齐}
% \setcounter{section}{4}
\setcounter{subsection}{5}
\setcounter{subsubsection}{0}
 
\subsection{Text Alignment文本对齐}
\begin{docTcbKey}[][doc new=2015-05-07]{halign}{=\meta{alignment}}{no default, initially \texttt{justify}}
If there is no lower part, |halign| determines the horizontal \meta{alignment}
of the text content.
Otherwise, |halign| determines the horizontal \meta{alignment}
of the upper part of the box only.
The feasible values for \meta{alignment} are more or less identical to
the corresponding |/tikz/align| settings, even if the implementation differs.

确定盒子upper部分的水平\meta{alignment}方式。可行的\meta{alignment}值与相应的|/tikz/align|设置几乎相同,即使实现不同。

% 如果没有lower部分, |halign| 决定着水平的文本内容的对齐方式为 \meta{alignment}。
% 否则, |halign| 的 \meta{alignment} 只影响到upper部分的水平对齐。
% |halign| 决定着upper部分文本内容的水平对齐方式为 \meta{alignment}。
% \meta{alignment} 可选的值有不少同 |/tikz/align| 的相应设置是一样的, 即使实现有所不同。
\begin{DescriptionR}{\docValue{flush center}}
\item[\docValue{justify}] usual left and right justified type setting.
\par 常用的,左右对齐%的排版设置(两端对齐)。
\item[\docValue{left}]left border justification in analogy to plain \TeX.
\par 类似于plain \TeX 的左边界对齐。
% 向着盒子的左边框对齐,类似于 plain \TeX。
\item[\docValue{flush left}]
left border justification with |\raggedright| of \LaTeX.
\par 向着盒子的左边框对齐。%,使用 \LaTeX 的 |\raggedright|\footnote{ragged是不整齐的意思}。
\item[\docValue{right}]right border justification in analogy to plain \TeX.
\par 向着盒子的右边框对齐。%,类似于 plain \TeX。
\item[\docValue{flush right}]right border justification with |\raggedleft| of \LaTeX.
\par 向着盒子的右边框对齐。%使用 \LaTeX 的 |\raggedleft|。
\item[\docValue{center}]centering in analogy to plain \TeX.
\par 居中对齐.%,使用 plain \TeX。
\item[\docValue{flush center}]centering with |\centering| of \LaTeX.
\par 居中对齐。%,使用 \LaTeX 的 |\centering|。
\end{DescriptionR}
The differences between the flush and non-flush version are explained in
detail in the \tikzname\ manual \cite{tantau:tikz_and_pgf}. The short story is that
the non-flush versions will often look more balanced but with more
hyphenations.

% 在 \tikzname\ 手册 %\cite{tantau:tikz_and_pgf} 
% 中详细介绍了 flush 和 non-flush 版本之间的区别。简而言之 non-flush 版本通常看起来更加平衡,但是有更多的连字符。

在\tikzname\ 手册\cite{tantau:tikz_and_pgf}中详细解释了flush版本和non-flush版本之间的区别。简短的说,non-flush版本通常看起来更平衡,但会有更多的连字。

% \begin{tcolorbox}[adjusted title=flush center,halign=flush center]
%   This is a demonstration text for showing how line breaking works.
%   \end{tcolorbox}
%   \begin{tcolorbox}[adjusted title=flush left,halign=flush left]
%   This is a demonstration text for showing how line breaking works.
%   \end{tcolorbox}
%   \begin{tcolorbox}[adjusted title=flush right,halign=flush right]
%   This is a demonstration text for showing how line breaking works.
%   \end{tcolorbox}
  
%   \begin{tcolorbox}[adjusted title=center,halign=center]
%   This is a demonstration text for showing how line breaking works.
%   \end{tcolorbox}
%   \begin{tcolorbox}[adjusted title=left,halign=left]
%   This is a demonstration text for showing how line breaking works.
%   \end{tcolorbox}
%   \begin{tcolorbox}[adjusted title=right,halign upper=right]
%   This is a demonstration text for showing how line breaking works.
%   \end{tcolorbox}
  

\begin{exdispExample}{halign}
\tcbset{colback=red!5!white,colframe=red!75!black,size=small,
fonttitle=\bfseries,width=3.5cm,box align=top,
nobeforeafter}

\foreach \p in {flush center,flush left,flush right}
{\begin{tcolorbox}[adjusted title=\p,halign=\p]
This is a demonstration text for showing how line breaking works.
\end{tcolorbox}
}

\foreach \q in {center, left, right}
{\begin{tcolorbox}[adjusted title=\q,halign=\q]
This is a demonstration text for showing how line breaking works.
\end{tcolorbox}
} 

\end{exdispExample}
\end{docTcbKey}



\begin{docTcbKey}[][doc new=2015-05-07]{halign upper}{=\meta{alignment}}{no default, initially \texttt{justify}}
% Alias for \refKeyLe{/tcb/halign}.

\refKeyLe{/tcb/halign} 的别名
\end{docTcbKey}



% \newpage
\begin{docTcbKey}[][doc new=2015-05-07]{halign lower}{=\meta{alignment}}{no default, initially \texttt{justify}}
|halign lower| determines the horizontal \meta{alignment} of the lower part of the box.
The feasible values for \meta{alignment} are the same as for \refKeyLe{/tcb/halign}.

% |halign lower| 控制着盒子的lower部分的内容的水平对齐方式为 \meta{alignment}。\meta{alignment} 的可选值同 \refKeyLe{/tcb/halign} 一样。

|halign lower| 确定盒子lower部分的水平 \meta{alignment}。 \meta{alignment} 的可行值与 \refKeyLe{/tcb/halign} 相同。
\begin{exdispExample}{halign_lower}
\begin{tcbraster}[raster columns=3,fonttitle=\bfseries,
colback=red!5!white,colframe=red!75!black]

\begin{tcolorbox}[adjusted title=flush center,halign lower=flush center]
Upper part. \tcblower Lower part.
\end{tcolorbox}
\begin{tcolorbox}[adjusted title=flush left,halign lower=flush left]
Upper part. \tcblower Lower part.
\end{tcolorbox}
\begin{tcolorbox}[adjusted title=flush right,halign lower=flush right]
Upper part. \tcblower Lower part.
\end{tcolorbox}
\begin{tcolorbox}[adjusted title=center,halign lower=center]
Upper part. \tcblower Lower part.
\end{tcolorbox}
\begin{tcolorbox}[adjusted title=left,halign lower=left]
Upper part. \tcblower Lower part.
\end{tcolorbox}
\begin{tcolorbox}[adjusted title=right,halign lower=right]
Upper part. \tcblower Lower part.
\end{tcolorbox}

\end{tcbraster}
\end{exdispExample}
\end{docTcbKey}






% \clearpage
%这儿原来写错了
\begin{docTcbKey}[][doc new=2022-10-30]{halign title}{=\meta{alignment}}{no default, initially \texttt{justify}}
|halign title| determines the horizontal \meta{alignment} of the title of the box.
The feasible values for \meta{alignment} are the same as for \refKeyLe{/tcb/halign}.

|halign title| 设置盒子的标题部分的对齐方式为 \meta{alignment}。
\meta{alignment} 的可选值同 \refKeyLe{/tcb/halign} 一样。

\begin{exdispExample}{halign_title}
\begin{tcbraster}[raster columns=3,fonttitle=\bfseries,
colback=red!5!white,colframe=red!75!black]

\begin{tcolorbox}[adjusted title=flush center,halign title=flush center]
This is a \textbf{tcolorbox}.
\end{tcolorbox}
\begin{tcolorbox}[adjusted title=flush left,halign title=flush left]
This is a \textbf{tcolorbox}.
\end{tcolorbox}
\begin{tcolorbox}[adjusted title=flush right,halign title=flush right]
This is a \textbf{tcolorbox}.
\end{tcolorbox}
\begin{tcolorbox}[adjusted title=center,halign title=center]
This is a \textbf{tcolorbox}.
\end{tcolorbox}
\begin{tcolorbox}[adjusted title=left,halign title=left]
This is a \textbf{tcolorbox}.
\end{tcolorbox}
\begin{tcolorbox}[adjusted title=right,halign title=right]
This is a \textbf{tcolorbox}.
\end{tcolorbox}

\end{tcbraster}
\end{exdispExample}
\end{docTcbKey}




% \enlargethispage*{1cm}

\begin{docTcbKey}[][doc updated=2015-05-07]{flushleft upper}{}{style, no value}
Shortcut for setting \refKeyLe{/tcb/halign} to \docValue{flush left}.
将 \refKeyLe{/tcb/halign} 设置为 \docValue{flush left} 的简写形式。
\end{docTcbKey}

\begin{docTcbKey}[][doc updated=2015-05-07]{center upper}{}{style, no value}
Shortcut for setting \refKeyLe{/tcb/halign} to \docValue{flush center}.

将 \refKeyLe{/tcb/halign} 设置为 \docValue{flush center} 的简写形式。
\end{docTcbKey}

\begin{docTcbKey}[][doc updated=2015-05-07]{flushright upper}{}{style, no value}
Shortcut for setting \refKeyLe{/tcb/halign} to \docValue{flush right}.

将 \refKeyLe{/tcb/halign} 设置为 \docValue{flush right} 的简写形式。
\end{docTcbKey}

\begin{docTcbKey}[][doc updated=2015-05-07]{flushleft lower}{}{style, no value}
Shortcut for setting \refKeyLe{/tcb/halign lower} to \docValue{flush left}.

将 \refKeyLe{/tcb/halign lower} 设置为 \docValue{flush left} 的简写形式。
\end{docTcbKey}

\begin{docTcbKey}[][doc updated=2015-05-07]{center lower}{}{style, no value}
Shortcut for setting \refKeyLe{/tcb/halign lower} to \docValue{flush center}.
将 \refKeyLe{/tcb/halign lower} 设置为 \docValue{flush center} 的简写形式。
\end{docTcbKey}

\begin{docTcbKey}[][doc updated=2015-05-07]{flushright lower}{}{style, no value}
Shortcut for setting \refKeyLe{/tcb/halign lower} to \docValue{flush right}.

将 \refKeyLe{/tcb/halign lower} 设置为 \docValue{flush right} 的简写形式。
\end{docTcbKey}



% \clearpage

\begin{docTcbKey}[][doc updated=2015-05-07]{flushleft title}{}{style, no value}
Shortcut for setting \refKeyLe{/tcb/halign title} to \docValue{flush left}.

将 \refKeyLe{/tcb/halign title} 设置为 \docValue{flush left} 的简写形式。
\end{docTcbKey}

\begin{docTcbKey}[][doc updated=2015-05-07]{center title}{}{style, no value}
Shortcut for setting \refKeyLe{/tcb/halign title} to \docValue{flush center}.

将 \refKeyLe{/tcb/halign title} 设置主 \docValue{flush center} 的简写形式。
\end{docTcbKey}

\begin{docTcbKey}[][doc updated=2015-05-07]{flushright title}{}{style, no value}
Shortcut for setting \refKeyLe{/tcb/halign title} to \docValue{flush right}.

将 \refKeyLe{/tcb/halign title} 设置为 \docValue{flush right} 的简写形式。
\end{docTcbKey}


\begin{marker}
The vertical alignment settings are only relevant for boxes which are larger
than their natural height, see \Fullref{sec:heightcontrol}.

垂直对齐设置只适用于大于其自然高度的盒子。见 \Fullref{sec:heightcontrol}.
\end{marker}

\begin{docTcbKey}[][doc updated=2015-07-16]{valign}{=\meta{alignment}}{no default, initially |top|}
If the height of a |tcolorbox| is not the natural height, |valign|
determines the vertical \meta{alignment} of the upper part.
Feasible values are

如果一个 |tcolorbox| 的盒子的高度不是其自然高度\footnote{译注:指定了高度}, |valign| 控制着盒子的upper部分的对方方式 \meta{alignment} 。
可选的值有:
\begin{itemize}
\item\docValue{top}: %Anchor text at top.
顶部对齐。
\item\docValue{center}: %Anchor text at center.
中间对齐。
\item\docValue{bottom}:% Anchor text at bottom.
底部对齐。
\item\docValue{scale}: 
Scale text vertically to fit into the available space.
  This is brutal and may not look very good. Consider \Fullref{sec:fitting}
  alternatively.
垂直缩放文本以适应可用空间。
这简单粗暴,可能看起来不是很好。或者考虑下 \Fullref{sec:fitting}。


\item\docValue{scale*}: 
Like \docValue{scale}, but scaling is bounded by
  \refKeyLe{/tcb/valign scale limit}.

类似于\docValue{scale}, 但缩放范围受限于 \refKeyLe{/tcb/valign scale limit}.
\end{itemize}
For a box with natural height, these settings are meaningless.

对于具有自然高度的盒子,这些设置毫无意义。
\begin{exdispExample}{valign}
\tcbset{width=(\linewidth-2mm)/4,before=,after=\hfill,
colframe=blue!75!black,colback=white,height=2cm}

\foreach \myalign in {top,center,bottom,scale}
{\begin{tcolorbox}[valign=\myalign]
This is a \textbf{tcolorbox}.
\end{tcolorbox}}
\end{exdispExample}
\end{docTcbKey}






\begin{docTcbKey}[][doc new=2015-05-07]{valign upper}{=\meta{alignment}}{no default, initially \texttt{top}}
Alias for \refKeyLe{/tcb/valign}.

\refKeyLe{/tcb/valign} 的别名。
\end{docTcbKey}

\begin{docTcbKey}{valign lower}{=\meta{alignment}}{no default, initially |top|}
This key has the same meaning for the lower part as |valign|
for the upper part, i.\,e., it determines
the vertical \meta{alignment} of the lower part with feasible values
|top|, |center|, |bottom|, |scale|, and |scale*|.

此项设置含义同upper部分的 |valign| 相同, i.\,e., 它指定了lower部分的竖直方向的对齐方式 \meta{alignment} ,可选的值有 |top|, |center|, |bottom|, |scale|, 和 |scale*|.

\end{docTcbKey}

\begin{docTcbKey}[][doc new=2015-07-16]{valign scale limit}{=\meta{real number}}{no default, initially \texttt{1.1}}
Sets an upper scale limit for the \docValue{scale*} setting in
\refKeyLe{/tcb/valign} and \refKeyLe{/tcb/valign lower}.
Note that this value is not reset by \refKeyLe{/tcb/reset}. So, changes
also apply to embedded boxes.

设置 \docValue{scale*} 在指定 \refKeyLe{/tcb/valign} 和 \refKeyLe{/tcb/valign lower} 时缩放的范围上限。注意,此值不会随着 \refKeyLe{/tcb/reset} 而重置。所以,修改的话也会同时在嵌套的盒子中生效。
\end{docTcbKey}

Also see \refKeyLe{/tcb/sidebyside align} for alignment settings when
upper part and lower part are set side-by-side.

另见 \refKeyLe{/tcb/sidebyside align} 了解当设置为side-by-side左右排布时的对齐选项。




% % \clearpage
% % Geometry\hfill 
\subsection{Geometry\\几何形状}
\setcounter{section}{4}
\setcounter{subsection}{6}
\setcounter{subsubsection}{0}
\subsection{Geometry\\几何属性}

% \subsubsection{Width} 

\begin{docTcbKey}{width}{=\meta{length}}{no default, initially \cs{linewidth}}
Sets the total width of the colored box to \meta{length}.
See also \refKeyLe{/tcb/height}.

将%有色的带框
盒子的总宽度设置为 \meta{length}。另见 \refKeyLe{/tcb/height}。
\begin{exdispExample}{width} 
\tcbset{colback=red!5!white,colframe=red!75!black}

\begin{tcolorbox}[width=\linewidth/2]
这是一个\textbf{tcolorbox}.
\end{tcolorbox}
\end{exdispExample}
\end{docTcbKey}


\begin{docTcbKey}[][doc new=2014-10-31]{text width}{=\meta{length}}{style, no default}
Sets the text width of the upper part to \meta{length}.
See also \refKeyLe{/tcb/text height}.

设置upper部分的文本宽度为 \meta{length}.
另见 \refKeyLe{/tcb/text height}.
\begin{exdispExample}{text_width}
\tcbset{colback=red!5!white,colframe=red!75!black}

\begin{tcolorbox}[text width=4cm]
This is a \textbf{tcolorbox} where the text has a width of 4cm.
\end{tcolorbox}
\end{exdispExample}
\end{docTcbKey}

\begin{docTcbKey}[][doc new=2014-11-07]{add to width}{=\meta{length}}{style, no default}
Adds \meta{length} to the current total width of the colored box.

将当前盒子的总宽度增加 \meta{length} 。    
\begin{exdispExample*}{add_to_width}{sbs,lefthand ratio=0.6}
\tcbset{width=4cm,colback=red!5!white,
colframe=red!75!black}

\begin{tcolorbox}
这是一个\textbf{tcolorbox}.
\end{tcolorbox}

\begin{tcolorbox}[add to width=1cm]
这是一个\textbf{tcolorbox}.
\end{tcolorbox}
\end{exdispExample*}
\end{docTcbKey}
See \Fullref{sec:heightcontrol} for setting fixed height values.

有关设置固定高度值的问题,请参阅 \Fullref{sec:heightcontrol}。
% \setcounter{section}{4}
\setcounter{subsection}{7}
\setcounter{subsubsection}{1}

\subsubsection{Rules\\线}
\begin{docTcbKey}{toprule}{=\meta{length}}{no default, initially \texttt{0.5mm}}

设置顶边框线的宽度为 \meta{length}。\hfill Sets the line width of the top rule to \meta{length}.
\begin{exdispExample}{toprule}
\tcbset{colback=red!5!white,colframe=red!75!black}

\begin{tcolorbox}[toprule=3mm]
这是一个\textbf{tcolorbox}.
\end{tcolorbox}
\end{exdispExample}
\end{docTcbKey}


\begin{docTcbKey}{bottomrule}{=\meta{length}}{no default, initially \texttt{0.5mm}}
设置底边框线的宽度为 \meta{length}。\hfill Sets the line width of the bottom rule to \meta{length}.
\begin{exdispExample}{bottomrule}
\tcbset{colback=red!5!white,colframe=red!75!black}

\begin{tcolorbox}[bottomrule=3mm]
这是一个\textbf{tcolorbox}.
\end{tcolorbox}
\end{exdispExample}
\end{docTcbKey}

\begin{docTcbKey}{leftrule}{=\meta{length}}{no default, initially \texttt{0.5mm}}
设置左边框线的宽度为 \meta{length}。\hfill Sets the line width of the left rule to \meta{length}.
\begin{exdispExample}{leftrule}
\tcbset{colback=red!5!white,colframe=red!75!black}

\begin{tcolorbox}[leftrule=3mm]
这是一个\textbf{tcolorbox}.
\end{tcolorbox}
\end{exdispExample}
\end{docTcbKey}


\begin{docTcbKey}{rightrule}{=\meta{length}}{no default, initially \texttt{0.5mm}}
设置右边框线的宽度为 \meta{length}。\hfill Sets the line width of the right rule to \meta{length}.
\begin{exdispExample}{rightrule}
\tcbset{colback=red!5!white,colframe=red!75!black}

\begin{tcolorbox}[rightrule=3mm]
这是一个\textbf{tcolorbox}.
\end{tcolorbox}
\end{exdispExample}
\end{docTcbKey}




% \clearpage
\begin{docTcbKey}{titlerule}{=\meta{length}}{no default, initially \texttt{0.5mm}}
Sets the line width of the rule below the title to \meta{length}.

设置标题文本下方的线的宽度为 \meta{length}。
\begin{exdispExample}{titlerule}
\tcbset{enhanced,colback=red!5!white,colframe=red!75!black,
colbacktitle=red!90!black}

\begin{tcolorbox}[titlerule=3mm,title=This is the title]
这是一个\textbf{tcolorbox}.
\end{tcolorbox}
\end{exdispExample}
\end{docTcbKey}


\begin{docTcbKey}{boxrule}{=\meta{length}}{style, no default, initially \texttt{0.5mm}}
Sets all rules of the frame to \meta{length}, i.\,e.\ 
\refKeyLe{/tcb/toprule}, \refKeyLe{/tcb/bottomrule}, \refKeyLe{/tcb/leftrule},
\refKeyLe{/tcb/rightrule}, and \refKeyLe{/tcb/titlerule}.

设置所有的边框线的宽度为 \meta{length}\footnote{i.\,e.\ 
\refKeyLe{/tcb/toprule}, \refKeyLe{/tcb/bottomrule}, \refKeyLe{/tcb/leftrule},
\refKeyLe{/tcb/rightrule}, 和 \refKeyLe{/tcb/titlerule}.}。
\begin{exdispExample}{boxrule}
\tcbset{colback=red!5!white,colframe=red!75!black}

\begin{tcolorbox}[boxrule=3mm]
这是一个\textbf{tcolorbox}.
\end{tcolorbox}
\end{exdispExample}
\end{docTcbKey}

\bigskip
\begin{marker}
More options for drawing a \refKeyLe{/tcb/borderline} are provided by using skins documented in
Section \ref{sec:skins} from page \pageref{sec:skins}.

更多的关于绘制 \refKeyLe{/tcb/borderline} 的选项的描述在 skins 的文档中,详见 \pageref{sec:skins} 页的 \ref{sec:skins} 小节。
\end{marker}

 
% % Arcs\hfill 
\subsubsection{弧线}
\begin{docTcbKey}{arc}{=\meta{length}}{no default, initially \texttt{1mm}}
Sets the inner radius of the four frame arcs to \meta{length}.

设置边框的四个角落的弧的内半径为 \meta{length}。

% \begin{exdispExample}{arc}
% \tcbset{colback=red!5!white,colframe=red!75!black}
% \begin{tcolorbox}[arc=0mm]
% 这是一个\textbf{tcolorbox}.
% \end{tcolorbox}
% \begin{tcolorbox}[arc=3mm]
% 这是一个\textbf{tcolorbox}.
% \end{tcolorbox}
% \end{exdispExample}

\begin{dispExample*}{sidebyside,lefthand ratio=0.6}
\tcbset{colback=red!5!white,colframe=red!75!black}
\begin{tcolorbox}[arc=0mm]
这是一个\textbf{tcolorbox}.
\end{tcolorbox}
\end{dispExample*}

\begin{dispExample*}{sidebyside,lefthand ratio=0.6}
\tcbset{colback=red!5!white,colframe=red!75!black}
\begin{tcolorbox}[arc=3mm]
这是一个\textbf{tcolorbox}.
\end{tcolorbox}
\end{dispExample*}
\end{docTcbKey}



% \begin{exdispExample*}{arc_default}{sbs,lefthand ratio=0.6}
 
% \end{exdispExample*}

% \begin{exdispExample*}{arc_3cm}{sbs,lefthand ratio=0.6}
 
% \end{exdispExample*}

% gpt4:
% 在 LaTeX 中,颜色的混合可以使用 ! 符号来实现。这种写法叫做 "interpolation of colors"。

% red!5!white 的意思是,颜色由5%的红色和95%的白色混合而成。结果是一个非常浅的红色。

% red!75!black 的意思是,颜色由75%的红色和25%的黑色混合而成。结果是一个较深的红色。

% claude
% !后面的数字表示混合比例,范围是0到100。
% 0表示完全使用第一个颜色。
% 100表示完全使用第二个颜色。
% 所以:

% red!0!white 等同于纯红色red。
% red!100!white 等同于纯白色white。


% \clearpage
\begin{docTcbKey}[][doc new=2015-05-05]{circular arc}{}{style, no value}
  Sets \refKeyLe{/tcb/arc} to match the half of the inner width of the colored box.
  If width and height of the box are identical, this gives a circle.
  
  将 \refKeyLe{/tcb/arc} 设置为盒子内部宽度的一半。如果盒子的宽度和高度相同,就会得到一个圆。
  \begin{marker}
  If the height of the box is smaller than the width, the result will look
  quite ugly.
  
  如果盒子的高度小于宽度,结果看起来会很难看。   
  \end{marker}
  \begin{exdispExample*}{circular_arc}{sbs,lefthand ratio=0.6}
  \begin{tcolorbox}[width=3cm,
  colback=red!5!white,
  colframe=red!75!black,
  halign=center,valign=center,
  square,circular arc]
  这是一个\textbf{tcolorbox}.
  \end{tcolorbox}
  \end{exdispExample*}
  \end{docTcbKey}
  
  
  \begin{docTcbKey}[][doc new=2015-05-05]{bean arc}{}{style, no value}
  Sets \refKeyLe{/tcb/arc} to match the smaller value of the
  half of the inner width and of the inner height of the colored box.
  
%   设置 \refKeyLe{/tcb/arc} 为盒子内宽和内高中较小者值的一半。
  
  将\refKeyLe{/tcb/arc}设置为盒子内宽度一半和高度一半的较小值。

  \begin{marker}
  This only works for a fixed \refKeyLe{/tcb/height}. Also, \refKeyLe{/tcb/bean arc}
  must be used \emph{after} width and height are set by option keys.
  
%   这只适用于 \refKeyLe{/tcb/height} 为固定值的情况。此外,\refKeyLe{/tcb/bean arc} 选项设置需要在设置宽度和高度之后。
  这仅适用于固定的\refKeyLe{/tcb/height}。另外,\refKeyLe{/tcb/bean arc}必须在宽度和高度选项设置之后使用。
  \end{marker}
  \begin{exdispExample*}{bean_arc}{sbs,lefthand ratio=0.6}
  \tcbset{size=fbox,boxrule=0.5mm,
  colback=red!5!white,
  colframe=red!75!black,
  halign=center,valign=center}
  
  \begin{tcolorbox}[width=3cm,height=2cm,
  bean arc]
  Box A
  \end{tcolorbox}
  
  \begin{tcolorbox}[width=2cm,height=3cm,
  bean arc]
  Box B
  \end{tcolorbox}
  \end{exdispExample*}
  \end{docTcbKey}
  
  % 八角形
  \begin{docTcbKey}[][doc new=2015-05-05]{octogon arc}{}{style, no value}
  Sets \refKeyLe{/tcb/arc} to match $\frac{1}{2+\sqrt{2}}$ of the inner width
  of the colored box. If width and height of the box are identical,
  the interior is a regular octogon.

将\refKeyLe{/tcb/arc}设置为盒子的内部宽度的$\frac{1}{2+\sqrt{2}}$。如果盒子的宽度和高度相同,则内部是一个正八边形。

% 设置 \refKeyLe{/tcb/arc} 为盒子内部宽度的 $\frac{1}{2+\sqrt{2}}$ 。如果盒子的宽度和高度相同,内部是一个规则的八边形。
  % \begin{tcolorbox}[
  %   width=2.1cm,octogon arc,
  %   ]
  %   STOP
  %   \end{tcolorbox}
\begin{exdispExample*}{octogon_arc}{sbs,lefthand ratio=0.82}
\begin{tcolorbox}[enhanced,% 启用高级选项。
size=minimal,%去除默认边距,仅显示框体。
auto outer arc,% 自动调整外部圆角半径以适应框体大小。
width=2.1cm,
octogon arc,%  外部圆角为八角形。
colback=red,%  背景色为红色。
colframe=white,% 边框颜色为白色。
colupper=white,% 文字颜色为白色。
fontupper=\fontsize{7mm}{7mm}\selectfont\bfseries\sffamily,
%文字大小为7毫米,粗体,无衬线字体。
halign=center,% 水平居中对齐
valign=center,%垂直居中对齐。
square,% square: 边框为直角。
arc is angular,% arc is angular: 内部圆角为直角。
borderline={0.2mm}{-1mm}{red}
%使用红色的0.2毫米线条作为边框,内部偏移1毫米。
]
STOP
\end{tcolorbox}
\end{exdispExample*}
\end{docTcbKey}

% \begin{dispExample*}{sidebyside,lefthand ratio=0.82}
% \begin{tcolorbox}[enhanced,% 启用高级选项。
% size=minimal,%去除默认边距,仅显示框体。
% auto outer arc,% 自动调整外部圆角半径以适应框体大小。
% width=2.1cm,
% % octogon arc,%  外部圆角为八角形。
% colback=red,%  背景色为红色。
% colframe=white,% 边框颜色为白色。
% colupper=white,% 文字颜色为白色。
% fontupper=\fontsize{7mm}{7mm}\selectfont\bfseries\sffamily,
% %文字大小为7毫米,粗体,无衬线字体。
% halign=center,% 水平居中对齐
% valign=center,%垂直居中对齐。
% square,% square: 边框为直角。
% arc is angular,% arc is angular: 内部圆角为直角。
% borderline={0.2mm}{-1mm}{red}
% %使用红色的0.2毫米线条作为边框,内部偏移1毫米。
% ]
% STOP
% \end{tcolorbox}
% \end{dispExample*}


% angular,有角度的
% \clearpage
\begin{docTcbKey}[][doc new=2015-05-05]{arc is angular}{}{no value, initially unset}
  Using this options applies a patch which straightens the corners arcs of
  the boxes. The little arcs are replaced by little straight lines.
  
%   这个选项\footnote{arc is angular}用于对盒子四角弧线的一个补丁。小弧线被小直线代替。

  使用此选项将应用一个补丁,使盒子的角弧线变直。小弧线将被小直线替换。
  \begin{marker}
  This patch is considered as an experimental feature.
  It changes some of the original \tikzname\ code. This change may break
  with future updates of \tikzname.
  
%   这个补丁被认为是一个实验性的特征。它改变了一些原始的 \tikzname\ 代码。此更改可能会随着 \tikzname\ 的未来更新而中断。
  这个补丁被视为一项实验性的功能。它改变了一些原始的\tikzname 代码。这种改变可能会在未来的\tikzname 更新中出现问题。
  \end{marker}
  
\begin{exdispExample*}{arc_is_angular}{sbs,lefthand ratio=0.6}
\tcbset{colback=red!5!white,%
colframe=red!75!black,%
arc=3mm}

\begin{tcolorbox}[arc is angular]
arc is angular.
\end{tcolorbox}
\begin{tcolorbox}[arc is curved]
arc is curved
\end{tcolorbox}
\end{exdispExample*}
  
  \end{docTcbKey}
  
  
\begin{docTcbKey}[][doc new=2015-05-05]{arc is curved}{}{no value, initially set}
This option resets the patch from \refKeyLe{/tcb/arc is angular}. The
original \tikzname\ code is activated.

% 此选项将 \refKeyLe{/tcb/arc is angular} 的补丁修改重置为角度。激活原始的 \tikzname\ 代码。

此选项将重置 \refKeyLe{/tcb/arc is angular} 所设置的修补程序。原始的 \tikzname\ 代码将被激活。
\end{docTcbKey}


\begin{docTcbKey}{outer arc}{=\meta{length}}{no default, initially unset}
Sets the outer radius of the four frame arcs to \meta{length}.

设置盒子四个角的弧的外半径为 \meta{length}。
% 将四个框架拱形的外半径设置为\meta{length}。

\begin{dispExample*}{sidebyside}
\tcbset{colback=red!5!white,%
colframe=red!75!black}
\begin{tcolorbox}[]
这是一个\textbf{tcolorbox}.
\end{tcolorbox}
\end{dispExample*}

\begin{dispExample*}{sidebyside}
\tcbset{colback=red!5!white,%
colframe=red!75!black}
\begin{tcolorbox}[arc=4mm]
arc=4mm
\end{tcolorbox}
\end{dispExample*}

\begin{dispExample*}{sidebyside}
\tcbset{colback=red!5!white,%
colframe=red!75!black}
\begin{tcolorbox}[arc=4mm,%
outer arc=1mm]
arc=4mm,outer arc=1mm
\end{tcolorbox}
\end{dispExample*}
\end{docTcbKey}

\begin{docTcbKey}{auto outer arc}{}{no value, initially set}
Sets the outer radius of the four frame arcs automatically in
dependency of the inner radius given by \refKeyLe{/tcb/arc}.

根据 \refKeyLe{/tcb/arc} 给出的内部半径自动设置外部半径。

\begin{dispExample*}{sidebyside}
\tcbset{colback=red!5!white,%
colframe=red!75!black}
\begin{tcolorbox}[arc=4mm]
arc=4mm
\end{tcolorbox}
\end{dispExample*}

\begin{dispExample*}{sidebyside}
\tcbset{colback=red!5!white,%
colframe=red!75!black} 
\begin{tcolorbox}[arc=4mm,%
outer arc=1mm]
arc=4mm,outer arc=1mm
\end{tcolorbox}
\end{dispExample*}

\begin{dispExample*}{sidebyside}
\tcbset{colback=red!5!white,%
colframe=red!75!black}
\begin{tcolorbox}[arc=4mm,%
auto outer arc]
\verb|arc=4mm,auto outer arc|
\end{tcolorbox}
\end{dispExample*}

\begin{dispExample*}{sidebyside}
\tcbset{colback=red!5!white,%
colframe=red!75!black}
\begin{tcolorbox}[arc=4mm,%
outer arc=9mm]
\verb|arc=4mm,outer arc=9mm|
\end{tcolorbox}
\end{dispExample*}

\begin{dispExample*}{sidebyside}
\tcbset{colback=red!5!white,%
colframe=red!75!black}
\begin{tcolorbox}[arc=4mm,%
outer arc=0.1mm]
\verb|arc=4mm,outer arc=0.1mm|
\end{tcolorbox}
\end{dispExample*}
\end{docTcbKey}
  
% \setcounter{section}{4}
\setcounter{subsection}{7}
\setcounter{subsubsection}{3}

\subsubsection{Spacing\hfill 间隔}
\begin{docTcbKey}{boxsep}{=\meta{length}}{no default, initially \texttt{1mm}}
Sets a common padding of \meta{length} between the text content and the
frame of the box. This value is added to the key values of
|left|, |right|, |top|, |bottom|, and |middle| at the appropriate places.

% 在文本内容和盒子的框之间设置一个共同的填充宽度为 \meta{length}。 这个值会被添加到
% |left|, |right|, |top|, |bottom|, 和 |middle| 的合适位置。

在文本内容和盒子边框之间设置一个共同的填充 \meta{length}。这个值会添加到 |left|、|right|、|top|、|bottom| 和 |middle| 的关键值中,在适当的位置使用。

% \begin{dispExample*}{}
% \tcbset{colback=red!5!white,colframe=red!75!black,width=(\linewidth-4mm)/2, before=,after=\hfill}
% \begin{tcolorbox}
% hi
% \end{tcolorbox}
% \begin{tcolorbox}[draft]
% hi
% \end{tcolorbox}
% \end{dispExample*}
% %%%%%%%%
% \begin{dispExample*}{}
% \tcbset{colback=red!5!white,colframe=red!75!black,width=(\linewidth-4mm)/2, before=,after=\hfill}
% \begin{tcolorbox}[boxsep=0mm]
% hi
% \end{tcolorbox}
% \begin{tcolorbox}[boxsep=0mm,draft]
% hi
% \end{tcolorbox}
% \end{dispExample*}
% %%%%%%
\begin{dispExample*}{}
\tcbset{colback=red!5!white,colframe=red!75!black,width=(\linewidth-4mm)/2, before=,after=\hfill}

\begin{tcolorbox}[boxsep=5mm,draft]
hi
\end{tcolorbox}
\begin{tcolorbox}[boxsep=8mm,draft]
hi
\end{tcolorbox}
\begin{tcolorbox}[boxsep=5mm]
hi
\end{tcolorbox}
\begin{tcolorbox}[boxsep=8mm]
hi
\end{tcolorbox}
\end{dispExample*}

% \begin{exdispExample}{boxsep}
% \tcbset{colback=red!5!white,colframe=red!75!black  
% \begin{tcolorbox}[boxsep=0mm]
% hi
% \end{tcolorbox}
% \begin{tcolorbox}[boxsep=0mm,draft]
% hi
% \end{tcolorbox}
% \end{exdispExample}
\end{docTcbKey}


\begin{docTcbKey}{left}{=\meta{length}}{style, no default, initially \texttt{4mm}}
Sets the left space between all text parts and frame (additional to |boxsep|).
This is an abbreviation for setting
|lefttitle|, |leftupper|, and |leftlower| to the same value.

设置所有文本部分和盒子间的左边距(除了|boxsep|之外)。 这是设置|lefttitle|、|leftupper|和|leftlower|为相同值的简写方式。


% 设置所有的文本内容同左侧边框的间隔(附加到|boxsep|).
% 这是同时将 |lefttitle|, |leftupper|, 和 |leftlower| 设置为同一个值的简写方式。
\begin{dispExample*}{sbs}
\tcbset{colback=red!5!white%
,colframe=red!75!black
% ,width=(\linewidth-4mm)/2, before=,after=\hfill
}

\begin{tcolorbox}[left=0mm]
指定 \verb|left=0mm|
\end{tcolorbox}

\begin{tcolorbox}
使用默认
\end{tcolorbox}

\begin{tcolorbox}[left=4mm]
指定 \verb|left=4mm|
\end{tcolorbox}


\begin{tcolorbox}[left=10mm]
指定 \verb|left=10mm|
\end{tcolorbox}
\end{dispExample*}
\end{docTcbKey}

% TODO 再看下
\begin{docTcbKey}[][doc new=2017-02-16]{left*}{=\meta{length}}{style, no default}
Sets \refKeyLe{/tcb/left} such that \meta{length} is the distance between
the left bounding box and the text parts.

设置\refKeyLe{/tcb/left},使得\meta{length}为左边界框和文本部分之间的距离。

% 设置 \refKeyLe{/tcb/left} 的值 \meta{length} 为盒子左边界和上下文的文本左侧的距离。

\begin{exdispExample}{left_star}
\tcbset{colback=red!5!white,colframe=red!75!black}

This is some text.
\begin{tcolorbox}[grow to left by=5mm,left*=0mm,
enhanced,show bounding box]
\verb|grow to left by=5mm,left*=0mm|
\end{tcolorbox}

\begin{tcolorbox}[left*=0mm,
enhanced,show bounding box]
\verb|left*=0mm|
\end{tcolorbox}

\begin{tcolorbox}[%left*=0mm,
enhanced,show bounding box]
这是一个\textbf{tcolorbox}.
\end{tcolorbox}
\end{exdispExample}
\end{docTcbKey}

% 这段Latex代码使用了tcolorbox宏包,它提供了创建漂亮框框的命令。在这个例子中,代码创建了三个tcolorbox,每个框内都包含了一段文本“这是一个\textbf{tcolorbox}”。

% 第一个tcolorbox使用了参数“grow to left by=5mm”和“left*=0mm”,
% 这意味着这个框会向左边延伸5毫米,并且左边的边框宽度为0毫米。同时,也使用了“enhanced”和“show bounding box”参数,这些参数可以让框框看起来更漂亮,并且显示边框的边界。

% 第二个tcolorbox只使用了“left*=0mm”参数,这意味着这个框的左边边框宽度为0毫米。同样,也使用了“enhanced”和“show bounding box”参数。

% 第三个tcolorbox只使用了“enhanced”和“show bounding box”参数,这意味着这个框没有任何特殊的设置,只是一个普通的tcolorbox。


% \clearpage
\begin{docTcbKey}{lefttitle}{=\meta{length}}{no default, initially \texttt{4mm}}
  Sets the left space between title text and frame (additional to |boxsep|).

设置标题文本的左侧同边框的距离(附加 |boxsep|)。
\begin{exdispExample}{lefttitle}
\tcbset{colback=red!5!white,colframe=red!75!black}

\begin{tcolorbox}[lefttitle=3cm,title=My Title]
这是一个\textbf{tcolorbox}.
\end{tcolorbox}
\end{exdispExample}
\end{docTcbKey}


\begin{docTcbKey}{leftupper}{=\meta{length}}{no default, initially \texttt{4mm}}
  Sets the left space between upper text and frame (additional to |boxsep|).

设置upper部分同左侧边边框的距离(附加 |boxsep|)。
\begin{exdispExample}{leftupper}
\tcbset{colback=red!5!white,colframe=red!75!black}

\begin{tcolorbox}[leftupper=3cm,title=My Title]
这是一个\textbf{tcolorbox}.
\end{tcolorbox}
\end{exdispExample}
\end{docTcbKey}

\begin{docTcbKey}{leftlower}{=\meta{length}}{no default, initially \texttt{4mm}}
  Sets the left space between lower text and frame (additional to |boxsep|).

设置lower部分同左侧边边框的距离(附加 |boxsep|)。
\begin{exdispExample}{leftlower}
\tcbset{colback=red!5!white,colframe=red!75!black}

\begin{tcolorbox}[leftlower=3cm]
这是一个\textbf{tcolorbox}.
\tcblower
这是lower部分。
\end{tcolorbox}
\end{exdispExample}
\end{docTcbKey}

\enlargethispage*{1cm}

\begin{docTcbKey}{right}{=\meta{length}}{style, no default, initially \texttt{4mm}}
  Sets the right space between all text parts and frame (additional to |boxsep|).
  This is an abbreviation for setting
  |righttitle|, |rightupper|, and |rightlower| to the same value.

设置所有文本部分同右侧边框的距离(附加 |boxsep|)。
这是同时将 |righttitle|, |rightupper|, 和 |rightlower| 设置为同一个值的简写方式。
\begin{exdispExample}{right}
\tcbset{colback=red!5!white,colframe=red!75!black}

\begin{tcolorbox}[width=5cm,right=2cm]
这是一个\textbf{tcolorbox}.
\end{tcolorbox}
\end{exdispExample}
\end{docTcbKey}





% \clearpage
  
\begin{docTcbKey}[][doc new=2017-02-16]{right*}{=\meta{length}}{style, no default}
  Sets \refKeyLe{/tcb/right} such that \meta{length} is the distance between
  the right bounding box and the text parts.

设置 \refKeyLe{/tcb/right} 的宽度 \meta{length} 为盒子右边框同上下文文本的右侧的距离。
\begin{exdispExample}{right_star}
\tcbset{colback=red!5!white,colframe=red!75!black}

\flushright This is some text.
\begin{tcolorbox}[grow to right by=5mm,right*=0mm,
  halign=right,enhanced,show bounding box]
这是一个\textbf{tcolorbox}.
\end{tcolorbox}
\end{exdispExample}
\end{docTcbKey}



\begin{docTcbKey}{righttitle}{=\meta{length}}{no default, initially \texttt{4mm}}
  Sets the right space between title text and frame (additional to |boxsep|).

设置标题文本右侧同右边框的距离(附加 |boxsep|)。
  \begin{exdispExample}{righttitle}
\tcbset{colback=red!5!white,colframe=red!75!black}

\begin{tcolorbox}[width=5cm,righttitle=2cm,title=My very long title text]
This is a \textbf{tcolorbox} with standard upper box dimensions.
\end{tcolorbox}
\end{exdispExample}
\end{docTcbKey}


\begin{docTcbKey}{rightupper}{=\meta{length}}{no default, initially \texttt{4mm}}
  Sets the right space between upper text and frame (additional to |boxsep|).

设置upper部分的文本同右边框的距离(附加|boxsep|).
\begin{exdispExample}{rightupper}
\tcbset{colback=red!5!white,colframe=red!75!black}

\begin{tcolorbox}[width=5cm,rightupper=2cm,title=My very long title text]
This is a \textbf{tcolorbox} with compressed upper box dimensions.
\end{tcolorbox}
\end{exdispExample}
\end{docTcbKey}





% \clearpage
\begin{docTcbKey}{rightlower}{=\meta{length}}{no default, initially \texttt{4mm}}
  Sets the right space between lower text and frame (additional to |boxsep|).

设置lower部分的右边同右侧边框的距离(附加 |boxsep|)。
\begin{exdispExample}{rightlower}
\tcbset{colback=red!5!white,colframe=red!75!black}

\begin{tcolorbox}[width=5cm,rightlower=2cm]
This is a \textbf{tcolorbox} with standard upper box dimensions.
\tcblower
This is the lower part with large space at right.
\end{tcolorbox}
\end{exdispExample}
\end{docTcbKey}



\begin{docTcbKey}{top}{=\meta{length}}{no default, initially \texttt{2mm}}
  Sets the top space between text and frame (additional to |boxsep|).

设置文本同上边框的距离(附加 |boxsep|)。
\begin{exdispExample}{top}
\tcbset{colback=red!5!white,colframe=red!75!black}

\begin{tcolorbox}[top=0mm]
这是一个\textbf{tcolorbox}.
\tcblower
这是lower部分。
\end{tcolorbox}
\end{exdispExample}
\end{docTcbKey}


\begin{docTcbKey}{toptitle}{=\meta{length}}{no default, initially \texttt{0mm}}
  Sets the top space between title and frame (additional to |boxsep|).

设置标题文本同上边框的距离(附加 |boxsep|)。    
\begin{exdispExample}{toptitle}
\tcbset{colback=red!5!white,colframe=red!75!black}

\begin{tcolorbox}[toptitle=3mm,title=My title]
这是一个\textbf{tcolorbox}.
\end{tcolorbox}
\end{exdispExample}
\end{docTcbKey}






% \clearpage
\begin{docTcbKey}{bottom}{=\meta{length}}{no default, initially \texttt{2mm}}
  Sets the bottom space between text and frame (additional to |boxsep|).

设置文本底部同边框的距离 (附加 |boxsep|).
\begin{exdispExample}{bottom}
\tcbset{colback=red!5!white,colframe=red!75!black}

\begin{tcolorbox}[bottom=0mm]
这是一个\textbf{tcolorbox}.
\tcblower
这是lower部分。
\end{tcolorbox}
\begin{tcolorbox}
  这是一个\textbf{tcolorbox}.
  \tcblower
  这是lower部分。
  \end{tcolorbox}
\end{exdispExample}
\end{docTcbKey}

\begin{docTcbKey}{bottomtitle}{=\meta{length}}{no default, initially \texttt{0mm}}
  Sets the bottom space between title and frame (additional to |boxsep|).

设置标题同下方的边框的距离(附加 |boxsep|).
\begin{exdispExample}{bottomtitle}
\tcbset{colback=red!5!white,colframe=red!75!black}

\begin{tcolorbox}[bottomtitle=3mm,title=My title]
这是一个\textbf{tcolorbox}.
\end{tcolorbox}
\end{exdispExample}
\end{docTcbKey}


\begin{docTcbKey}{middle}{=\meta{length}}{no default, initially \texttt{2mm}}
Sets the space between upper and lower text to the separation line
(additional to |boxsep|).

% 将上下文本与分隔线之间的距离设置为分隔线(附加到|boxsep|)。

设置上下文本同分隔线的距离(附加 |boxsep|)。
\begin{exdispExample}{middle}
\tcbset{colback=red!5!white,colframe=red!75!black}

\begin{tcolorbox}[middle=0mm,boxsep=0mm]
这是一个\textbf{tcolorbox}.
\tcblower
这是lower部分。
\end{tcolorbox}
\begin{tcolorbox}[boxsep=0mm]
这是一个\textbf{tcolorbox}.
\tcblower
这是lower部分。
\end{tcolorbox}
\begin{tcolorbox}
这是一个\textbf{tcolorbox}.
\tcblower
这是lower部分。
\end{tcolorbox}
\end{exdispExample}
\end{docTcbKey}


% \subsubsection{Size Shortcuts\\调整尺寸的快捷方式}
\begin{docTcbKey}{size}{=\meta{name}}{no default, initially \texttt{normal}}
Sets all geometry keys with exception of \refKeyLe{/tcb/width} to
predefined length values.
For \meta{name}, the following values are feasible:

将除 \refKeyLe{/tcb/width} 外的所有尺寸设置为预定义的值。
可选的 \meta{name} 值有:    
  \begin{DescriptionL}{\docValue{minimal}}
  \item[\docValue{normal}]normal sized boxes e.g. of width |\linewidth|.
\\常用的盒子尺寸 e.g. 宽度为 |\linewidth|。
  \item[\docValue{title}]title line sized boxes.
  \\宽度同标题行一致。
  \item[\docValue{small}] small boxes e.g. for keyword highlighting.
  \\小一些的盒子 e.g. 用于关键字的高亮。
  \item[\docValue{fbox}] identical to the standard |\fbox|.
  \\同使用 |\fbox| 一样。
  \item[\docValue{tight}] no padding space at all.
  \\完全没有填充空间。

  \item[\docValue{minimal}] no padding space, no box rules.
  \\没有填充空间,没有边框。
  \end{DescriptionL}
% todo on line 是什么意思
\begin{exdispExample}{size_1}
\tcbset{colback=red!5!white,colframe=red!75!black}

\foreach \s in {normal,title,small,fbox,tight,minimal} {
  \tcbox[size=\s,on line]{\s} }

\foreach \s in {normal,title,small,fbox,tight,minimal} {
  \tcbox[size=\s,on line,title=Test]{\s} }

\foreach \s in {normal,title,small,fbox,tight,minimal} {
  \begin{tcolorbox}[size=\s,on line,title=Test,width=2.2cm]
    \s \tcblower lower\end{tcolorbox} }
\end{exdispExample}

\bigskip

\begin{tcolorbox}[tabularx={l|XXXXXX},title=Predefined values,
enhanced,fonttitle=\small\bfseries,fontupper=\small\ttfamily,
colback=yellow!10!white,colframe=red!50!black,colbacktitle=Salmon!30!white,
coltitle=black,center title
]
            & normal & title  & small & fbox  & tight & minimal\\\hline
boxrule     & 0.5mm  & 0.4mm  & 0.3mm & 0.4pt & 0.4pt & 0.0pt \\
boxsep      & 1.0mm  & 1.0mm  & 1.0mm & 3.0pt & 0.0pt & 0.0pt \\
left        & 4.0mm  & 2.0mm  & 1.0mm & 0.0pt & 0.0pt & 0.0pt \\
right       & 4.0mm  & 2.0mm  & 1.0mm & 0.0pt & 0.0pt & 0.0pt \\
top         & 2.0mm  & 0.25mm & 0.0mm & 0.0pt & 0.0pt & 0.0pt \\
bottom      & 2.0mm  & 0.25mm & 0.0mm & 0.0pt & 0.0pt & 0.0pt \\
toptitle    & 0.0mm  & 0.0mm  & 0.0mm & 0.0pt & 0.0pt & 0.0pt \\
bottomtitle & 0.0mm  & 0.0mm  & 0.0mm & 0.0pt & 0.0pt & 0.0pt \\
middle      & 2.0mm  & 0.75mm & 0.5mm & 1.0pt & 0.2pt & 0.0pt \\
arc         & 1.0mm  & 0.75mm & 0.5mm & 1.0pt & 0.0pt & 0.0pt \\
outer arc   & auto   & auto   & auto  & auto  & 0.0pt & 0.0pt \\
\end{tcolorbox}
\end{docTcbKey}


  

% \clearpage
\begin{docTcbKey}{oversize}{\colOpt{=\meta{length}}}{style, default |0pt|}
Sets the text width of the upper part to the current line width plus an
optional \meta{length}.
This is achieved by changing the keys \refKeyLe{/tcb/width}
\refKeyLe{/tcb/enlarge left by}, and
\refKeyLe{/tcb/enlarge right by} appropriately.
The resulting box is overlapping into the left and right margin of
the page.
Note that this style option has to be given \emph{after} all other
geometry keys!
Also see \refKeyLe{/tcb/grow sidewards by} and \refKeyLe{/tcb/spread sidewards}.

将upper部分的文本宽度设置为当前行宽再加上可选的\meta{length}。这是通过适当地更改键\refKeyLe{/tcb/width}、\refKeyLe{/tcb/enlarge left by}和\refKeyLe{/tcb/enlarge right by}来实现的。结果的框重叠到页面的左右边距上。请注意,这个样式选项必须在所有其他几何键之后给出!还请参见\refKeyLe{/tcb/grow sidewards by}和\refKeyLe{/tcb/spread sidewards}。

% 设置upper部分的文本的宽度为上下文中行宽加上 \meta{length}。这个效果是通过适当的改变 \refKeyLe{/tcb/width} \refKeyLe{/tcb/enlarge left by}, 和 \refKeyLe{/tcb/enlarge right by} 实现的。
% 最终的盒子会向左和右侧的边注伸展。注意,这个选项应该放置在所有其他的尺寸选项\emph{之后}!
% 另见 \refKeyLe{/tcb/grow sidewards by} 和 \refKeyLe{/tcb/spread sidewards}.
\begin{dispListing}
\tcbset{colback=red!5!white,colframe=red!75!black,fonttitle=\bfseries}

\textit{用于比较的普通文本:}\\
\lipsum[2]

\begin{tcolorbox}[oversize,title=Oversized box]
\lipsum[2]
\end{tcolorbox}

\begin{tcolorbox}[title=Normal box]
\lipsum[2]
\end{tcolorbox}
\end{dispListing}
\end{docTcbKey}

{\tcbusetemp}

  % 2023-1029
% \clearpage
% \hfill 
\subsubsection{Toggle Left and Right\\左右设置切换}
\begin{docTcbKey}[][doc updated=2017-02-16]{toggle left and right}{=\meta{toggle preset}}{default |evenpage|, initially |none|}
According to the \meta{toggle preset}, the left and the right settings
of the |tcolorbox| are switched or not. Feasible values are:

根据 \meta{toggle preset}, |tcolorbox| 的左右设置是否对换。 可选的值有:
  \begin{DescriptionL}{\docValue{evenpage}}
  \item[\docValue{none}]no switching.
不对换。
  \item[\docValue{forced}]the values of the left and right rules, spaces, and corners are switched.
左右的线、距离和四个角的设置互换。
  \item[\docValue{evenpage}]
  if the page is an even page, the values of the left and
    right rules, spaces, and corners are switched. This value also sets
    \refKeyLe{/tcb/check odd page} to |true|.
偶数页的左右线条数值,距离和四个角的设置互换。 此设置同时将 \refKeyLe{/tcb/check odd page} 设置为 |true|.
  \end{DescriptionL}
\begin{marker}
Horizontal bounding box enlargements are not toggled by this option.
They can be toggled independently by \refKeyLe{/tcb/toggle enlargement}.
For example, \refKeyLe{/tcb/oversize} changes the bounding box.

此选项不会将盒子的水平边框放大。
它们可以通过 \refKeyLe{/tcb/toggle enlargement} 独立地进行切换。
例如, \refKeyLe{/tcb/oversize} 更改盒子的的边界框。
\end{marker}
\begin{dispListing}
% \usepackage{lipsum}
% \usetikzlibrary{patterns}
% \tcbuselibrary{skins,breakable}
\begin{tcolorbox}[enhanced,% 启用增强功能,支持更多的选项。
breakable,%当盒子过长时,可以自动分页。
toggle left and right,%使盒子可以在左右两侧切换。
sharp corners,% 使盒子的角变得尖锐。
boxrule=0mm,top=0mm,bottom=0mm,left=1mm,right=1mm,
rightrule=1cm,colupper=blue!25!black,
interior style={fill overzoom image=lichtspiel.jpg,fill image opacity=0.25},
frame style={pattern=crosshatch dots light steel blue},
overlay={%定义要在盒子上覆盖的内容,这里是一个填充球和一个交叉标记。
\begin{tcbclipframe}
\tcbifoddpage{\coordinate (X) at ([xshift=-5mm]frame.east);}
            {\coordinate (X) at ([xshift=5mm]frame.west);}
\fill[shading=ball,ball color=blue!50!white,opacity=0.5] (X) circle (4mm);
\end{tcbclipframe}}]
\lipsum[1-6]
\end{tcolorbox}
\end{dispListing}
\medskip

% 这是一个tcolorbox环境,会在页面上创建一个带有填充图片和交叉标记的盒子。


% boxrule,top,bottom,left,right:控制盒子边框的宽度和位置。
% rightrule:在盒子右侧添加一条宽度为1厘米的边框。
% colupper:定义盒子中的文本颜色。
% interior style:定义盒子内部的样式,包括填充图片和填充透明度。
% frame style:定义盒子边框的样式,包括交叉标记和颜色。
% overlay:定义要在盒子上覆盖的内容,这里是一个填充球和一个交叉标记。
% 整个盒子中放置了一个lipsum段落,用于测试盒子的分页效果。

This example switches a |1cm| thick rule from the left to the right side
depending on the page number. Thereby, the rule is always on the outer side
of the double-sided paper. Additionally, a ball is drawn on the outer side
with help of an overlay.

这个例子根据页面编号将厚度为|1cm|的标尺从左侧移到右侧。因此,该标尺始终位于双面纸的外侧。此外,通过叠加层在外侧绘制一个球。

% 此示例根据页码将 |1cm| 宽的线条从左边切换到右边。
% 因此,线条总是在双面纸的外侧。
% 此外,一个球形,通过 overlay 绘制在外侧。
\bigskip

\tcbusetemp
\end{docTcbKey}

%todo 可以调整下





% % \clearpage
% % Corners\hfill 
% \subsection{四角}\label{subsec:corners}

% % The four corners of any |tcolorbox| can be set individually as
% % \refKey{/tcb/sharp corners} or as \refKey{/tcb/rounded corners}.
% % These settings are also reflected in the behavior of \refKey{/tcb/borderline}
% % and \refKey{/tcb/shadow} as one would expect.

% 任何 |tcolorbox| 的四个角可以通过 \refKey{/tcb/sharp corners} 或 \refKey{/tcb/rounded corners} 单独设置。
% 这些设置也影响到 \refKey{/tcb/borderline} 和 \refKey{/tcb/shadow} 的表现。

% % By default, all four corners are \emph{rounded}. So, only the
% % \refKey{/tcb/sharp corners} option will be necessary for most use cases.
% % The \refKey{/tcb/rounded corners} option can be used to revert a \refKey{/tcb/sharp corners}
% % setting.

% 默认情况下, 四角都是\emph{圆润的}。 因此,只有
% \refKey{/tcb/sharp corners} 选项是需要在大多数用例中显式指定。
% \refKey{/tcb/rounded corners} 选项可以用于重置 \refKey{/tcb/sharp corners}
% 修改过的设定。


% \begin{docTcbKey}{sharp corners}{=\meta{position}}{default |all|, initially unset}
% % The \meta{position} denotes one or more of the four box corners to be set as
% % \emph{sharp} corners. The not assigned corners will retain their mode.
% % Feasible values for \meta{position} are:

% \meta{position} 用来将盒子的四个角中的一个或多个设置为\emph{sharp}(直角) ,未指定的角将保持原来的模式(圆角)。
% 可设置的 \meta{position} 值有:

% \begin{itemize}
% \foreach \p in {northwest,northeast,southwest,southeast,north,south,east,west,downhill,uphill,all}
% {
% \item\tcbox[on line,size=title,arc=2mm,colframe=red!75!black,colback=red!5!white,
%   enlarge top by=0.5mm,enlarge bottom by=0.5mm,sharp corners=\p]{\docValue{\p}}
% }
% \end{itemize}
% \begin{exdispExample*}{sharp_corners_1}{sbs,lefthand ratio=0.6}
% \begin{tcolorbox}[colback=red!5!white,
%   colframe=red!75!black,
%   sharp corners=northwest ]
% |sharp corners=|\textbf{northwest},%
% 北西,左上角为直角。
% \end{tcolorbox}
% \end{exdispExample*}
% \begin{exdispExample*}{sharp_corners_2}{sbs,lefthand ratio=0.6}
% \begin{tcolorbox}[colback=red!5!white,
%   colframe=red!75!black,
%   sharp corners ]
% This is a \textbf{tcolorbox}.
% \end{tcolorbox}
% \end{exdispExample*}
% \end{docTcbKey}



  
% % \clearpage
% \begin{docTcbKey}{rounded corners}{=\meta{position}}{default |all|, initially |all|}
%   % The \refKey{/tcb/rounded corners} can be used to revert a \refKey{/tcb/sharp corners}
%   % setting. The \meta{position} denotes one or more of the four box corners to be set as
%   % \emph{rounded} corners. The not assigned corners will retain their mode.
%   % Feasible values for \meta{position} are\footnote{The graphical examples assume
%   %   that the boxes where set to have sharp corners before.}:
  
%   \refKey{/tcb/rounded corners} 可以用来重置 \refKey{/tcb/sharp corners} 带来的设置的修改。 \meta{position} 用来将盒子的四个角中的一个或多个设置为 \emph{rounded}(圆角)。未指定的角将保持原来的模式。
%   \meta{position}可用的值有\footnote{例子中假设设置之前有尖角的盒子。}:
%   \begin{itemize}
%   \foreach \p in {northwest,northeast,southwest,southeast,north,south,east,west,downhill,uphill,all}
%   {
%   \item\tcbox[on line,size=title,arc=2mm,colframe=red!75!black,colback=red!5!white,
%     enlarge top by=0.5mm,enlarge bottom by=0.5mm,sharp corners,rounded corners=\p]{\docValue{\p}}
%   }
%   \end{itemize}
%   \begin{exdispExample*}{rounded_corners}{sbs,lefthand ratio=0.6}
%   \begin{tcolorbox}[colback=red!5!white,
%     colframe=red!75!black,sharp corners,
%     rounded corners=northwest ]
%   |rounded corners=northwest|,北西,左上角为圆角。
%   \end{tcolorbox}
%   \end{exdispExample*}
%   \end{docTcbKey}
  
  
%   \begin{docTcbKey}{sharpish corners}{}{style, no value}
%     % Shortcut for setting \refKey{/tcb/arc} and \refKey{/tcb/outer arc}
%     % to |0pt|. With this setting, rounded corners will appear as quasi-sharp,
%     % but e.\,g.\ the shadow will be somewhat rounder than the shadow
%     % of really sharp corners.
  
%   同时设置 \refKey{/tcb/arc} 和 \refKey{/tcb/outer arc} 到 |0pt| 的简写。通过这项设置, 圆角展示得很像直角, 但 e.\,g.\ 阴影会比真正的直角的阴影稍微圆一些。
%     \begin{marker}
%     % Corners are still of type \emph{rounded} with this option, but appear
%     % \emph{sharp}. To switch back to rounded corners, one has to adapt
%     % \refKey{/tcb/arc} and \refKey{/tcb/outer arc}.
  
%   本项设置后,四个角的类型仍然是 \emph{rounded} , 但展现为\emph{sharp}。要切回 rounded corners, 需要修改 \refKey{/tcb/arc} 和 \refKey{/tcb/outer arc}。
%     \end{marker}
%   \begin{exdispExample*}{sharpish_corners}{sbs,lefthand ratio=0.6}
%   \begin{tcolorbox}[colback=red!5!white,
%     colframe=red!75!black,
%     sharpish corners ]
%   This is a \textbf{tcolorbox}.
%   \end{tcolorbox}
%   \end{exdispExample*}
%   \end{docTcbKey}

  


% % \clearpage
  
% % The following examples will show the differences between
% % \refKey{/tcb/rounded corners}, \refKey{/tcb/sharpish corners}, and \refKey{/tcb/sharp corners}.
% % The later two give the same core box, but \refKey{/tcb/borderline}
% % and \refKey{/tcb/shadow} settings are slightly different.
% % The following examples use \refKey{/tcb/drop fuzzy shadow}.

% 下面的例子展示了 \refKey{/tcb/rounded corners}, \refKey{/tcb/sharpish corners}, 和 \refKey{/tcb/sharp corners} 的区别。
% 后两个的内部盒子是相同的, 但 \refKey{/tcb/borderline} 和 \refKey{/tcb/shadow} 的设置有点不同。
% 下面的例子使用了 \refKey{/tcb/drop fuzzy shadow}。


% \begin{extcolorbox}[minipage]{corners_comparison}[blankest]
% \foreach \n in {rounded corners,sharpish corners,sharp corners}{
% \begin{tcolorbox}[enhanced jigsaw,frame empty,interior empty,fuzzy halo,halign=center,beforeafter skip=4mm]
% \begin{tcolorbox}[enhanced,drop fuzzy shadow,width=\linewidth-1cm,
%   colback=red!5!white, colframe=red!75!black, fonttitle=\bfseries,
%   title=My title,\n,
%   tikz={spy using outlines={circle, magnification=8, size=2cm, connect spies}},
%   overlay={\spy [blue, size=4cm] on (frame.south east)
%       in node at ([xshift=-2.5cm,yshift=-2.5cm]frame.south east);
%   \node[right] at ([xshift=2cm,yshift=-1cm]frame.south west) {\textbf{\Large\ttfamily\n}};
%   }]
% This is a \textbf{tcolorbox}.
% \end{tcolorbox}
% \end{tcolorbox}}
% \end{extcolorbox}



% % \clearpage
% % Transparency \hfill 
% \subsection{透明度}

% \begin{marker}
% % Transparency effects are likely to be used in conjunction with \emph{jigsaw}
% % skin variants, see \Vref{subsec:skinjigsaw}.

% 透明效果常常同skin的\emph{jigsaw}变量配合使用, 见 \Vref{subsec:skinjigsaw}.
% \end{marker}

% \begin{docTcbKey}{opacityframe}{=\meta{fraction}}{no default, initially \texttt{1.0}}
%   % Sets the frame opacity of the box to the given \meta{fraction}.

% 根据给定的 \meta{fraction} 值设置盒子的边框%线
% 的不透明度。
% \begin{exdispExample*}{opacityframe}{sbs,lefthand ratio=0.6,
%   segmentation style={preaction={fill=white},pattern=checkerboard,pattern color=gray!40}
%   }
% \begin{tcolorbox}[opacityframe=0.25
%   ,title=设置了边框的颜色和透明度,
%   colframe=red]
% This is a \textbf{tcolorbox}.
% \end{tcolorbox}
% \begin{tcolorbox}[colframe=red
%   ,title=设置了边框的颜色]
% This is a \textbf{tcolorbox}.
% \end{tcolorbox}
% \end{exdispExample*}
% \end{docTcbKey}

% \begin{docTcbKey}{opacityback}{=\meta{fraction}}{no default, initially \texttt{1.0}}
%   % Sets the background opacity of the box to the given \meta{fraction}.

% 根据给定的 \meta{fraction} 值设置盒子的背景色的不透明度。
% \begin{exdispExample*}{opacityback}{sbs,lefthand ratio=0.6,segmentation style={preaction={fill=white},pattern=checkerboard,pattern color=gray!40}}
% \begin{tcolorbox}[standard jigsaw,colframe=red,
%   opacityframe=0.5, opacityback=0.5]
% This is a \textbf{tcolorbox}.
% \end{tcolorbox}
% \end{exdispExample*}
% \end{docTcbKey}

% 另见 \mylib{skins} 库的 \refKey{/tcb/opacitybacklower}。

 

% \begin{docTcbKey}{opacitybacktitle}{=\meta{fraction}}{no default, initially \texttt{1.0}}
%   % Sets the title background opacity of the box to the given \meta{fraction}.

% 根据给定的 \meta{fraction} 值设置盒子的标题的背景色的不透明度。
% \begin{exdispExample*}{opacitybacktitle}{sbs,lefthand ratio=0.6,segmentation style={preaction={fill=white},pattern=checkerboard,pattern color=gray!40}}
% \begin{tcolorbox}[standard jigsaw,colframe=red,
%   opacityframe=0.5, opacitybacktitle=0.5,
%   title filled, title=This is a title]
% This is a \textbf{tcolorbox}.
% \end{tcolorbox}
% \end{exdispExample*}
% \end{docTcbKey}


% \begin{docTcbKey}{opacityfill}{=\meta{fraction}}{style, no default, initially \texttt{1.0}}
%   % Sets the fill opacity for frame, interior and optionally the title background
%   % to the given \meta{fraction}.

% 根据给定的 \meta{fraction} 值设置盒子的边框、内部和可选的标题背景的填充不透明度。
% \begin{exdispExample*}{opacityfill}{sbs,lefthand ratio=0.6,segmentation style={preaction={fill=white},pattern=checkerboard,pattern color=gray!40}}
% \begin{tcolorbox}[standard jigsaw,colframe=red,
%   opacityfill=0.7, title=This is a title]
% This is a \textbf{tcolorbox}.
% \end{tcolorbox}
% \begin{tcolorbox}[standard jigsaw,colframe=red,
% title=This is a title]
% This is a \textbf{tcolorbox}.
% \end{tcolorbox}
% \end{exdispExample*}
% \end{docTcbKey}





% % \clearpage
% \begin{docTcbKey}{opacityupper}{=\meta{fraction}}{no default, initially \texttt{1.0}}
%   % Sets the text opacity of the upper box part to the given \meta{fraction}.

% 根据给定的 \meta{fraction},设置upper部分的文本的不透明度。
% \begin{exdispExample*}{opacityupper}{sbs,lefthand ratio=0.6}
% \begin{tcolorbox}[enhanced,opacityupper=0.5
%   ,interior style={preaction={fill=white}
%   ,pattern=checkerboard
%   ,pattern color=gray!40}]
% This is a \textbf{tcolorbox}.
% \end{tcolorbox}
% \end{exdispExample*}
% \end{docTcbKey}



% \begin{docTcbKey}{opacitylower}{=\meta{fraction}}{no default, initially \texttt{1.0}}
%   % Sets the text opacity of the lower box part to the given \meta{fraction}.

% 根据给定的 \meta{fraction},设置lower部分的文本的不透明度。
% \begin{exdispExample*}{opacitylower}{sbs,lefthand ratio=0.6}
% \begin{tcolorbox}[enhanced,opacitylower=0.5,
%   interior style={preaction={fill=white},pattern=checkerboard,pattern color=gray!40}]
% This is a \textbf{tcolorbox}.
% \tcblower
% This is the lower part.
% \end{tcolorbox}
% \end{exdispExample*}
% \end{docTcbKey}

% \begin{docTcbKey}{opacitytext}{=\meta{fraction}}{no default, initially \texttt{1.0}}
%   % Sets the text opacity of the upper and the lower box part to the given \meta{fraction}.

% 根据给定的 \meta{fraction},设置upper和lower半两部分的文本的不透明度。
% \begin{exdispExample*}{opacitytext}{sbs,lefthand ratio=0.6}
% \begin{tcolorbox}[enhanced,opacitytext=0.5,
%   interior style={preaction={fill=white},pattern=checkerboard,pattern color=gray!40}]
% This is a \textbf{tcolorbox}.
% \tcblower
% This is the lower part.
% \end{tcolorbox}
% \end{exdispExample*}
% \end{docTcbKey}


% \begin{docTcbKey}{opacitytitle}{=\meta{fraction}}{no default, initially \texttt{1.0}}
%   % Sets the text opacity of the box title to the given \meta{fraction}.

% 根据给定的 \meta{fraction},设置标题的文本的不透明度。
% \begin{exdispExample*}{opacitytitle}{sbs,lefthand ratio=0.6}
% \begin{tcolorbox}[enhanced,opacitytitle=0.7,
%   coltitle=black,
%   fonttitle=\bfseries,title=This is a title,
%   title style={preaction={fill=white},pattern=checkerboard,pattern color=gray!40}]
% This is a \textbf{tcolorbox}.
% \end{tcolorbox}
% \end{exdispExample*}
% \end{docTcbKey}


% \begin{exdispExample*}{opacity_general}{segmentation style={preaction={fill=white},pattern=checkerboard,pattern color=gray!40}}
% \begin{tcolorbox}[enhanced jigsaw,fonttitle=\bfseries,title=This is a title,
%   opacityframe=0.5,opacityback=0.25,opacitybacktitle=0.25,opacitytext=0.8,
%   colback=red!5!white,colframe=red!75!black,colbacktitle=yellow!20!red]
% This is a \textbf{tcolorbox}.
% \end{tcolorbox}
% \end{exdispExample*}



 


% % \clearpage
% % Height Control\hfill 
% \subsection{高度控制}\label{sec:heightcontrol}
% % In a typical usage scenario, the height of a |tcolorbox| is computed automatically
% % to fit the content. Nevertheless, the height can be set to a fixed value
% % or to fit commonly for several boxes, e.\,g. if boxes are set side by side.

% 在典型的使用场景中,一个 |tcolorbox| 的高度是根据内容自动计算的。 
% 尽管如此,其高度是可以设置为一个固定的值,或自适就能放置指定个数的盒子, e.\,g. 如果例子设置为并排。

% \bigskip
% \begin{marker}
%   % The height control keys are only applicable to unbreakable boxes.
%   % If a box is set to be \refKey{/tcb/breakable}, the height is always
%   % computed according to the \emph{natural height}.

% 高度控制只在不可分页的例子上使用。
% 如果一个盒子设置了 \refKey{/tcb/breakable}, 其他高度将终究使用自然高度(\emph{natural height}).
% \end{marker}
% \bigskip


% \begin{docTcbKey}{natural height}{}{no value, initially set}
%   % Sets the total height of the colored box to its natural height depending
%   % on the box content.

% 根据盒子的内容将%有色
% 盒子的总高度设置为其自然高度。
% \end{docTcbKey}

% \begin{docTcbKey}{height}{=\meta{length}}{no default}
% % Sets the total height of the colored box to \meta{length} independent
% % of the box content. \meta{length} is the minimum height of the box, if
% % \refKey{/tcb/height plus} is larger than zero.

% 将%有色
% 盒子的总高度设置为与内容无关的长度 \meta{length} 。 \meta{length} 是盒子的最小高度, 如果
% \refKey{/tcb/height plus} 大于零。
% \begin{exdispExample}{height}
% \tcbset{width=(\linewidth-2mm)/3,before=,after=\hfill,
% colframe=blue!75!black,colback=white}

% \begin{tcolorbox}[height=1cm,valign=center]
%   This box has a height of 1cm.
% \end{tcolorbox}
% \begin{tcolorbox}[height=2cm,valign=center]
%   This box has a height of 2cm.
% \end{tcolorbox}
% \begin{tcolorbox}[height=3cm,split=0.5,valign=center,valign lower=center]
%   This box has a height of 3cm.
%   \tcblower
%   Lower part.
% \end{tcolorbox}
% \end{exdispExample}

% % todo 设置  before=,after=\hfill 就可以一行放多个了
% \begin{exdispExample}{height2}
%   \tcbset{width=(\linewidth-2mm)/3,%before=,after=\hfill,
%   colframe=blue!75!black,colback=white}
  
%   \begin{tcolorbox}[height=1cm,valign=center]
%     This box has a height of 1cm.
%   \end{tcolorbox}
%   \begin{tcolorbox}[height=2cm,valign=center]
%     This box has a height of 2cm.
%   \end{tcolorbox}
%   \begin{tcolorbox}[height=3cm,split=0.5,valign=center,valign lower=center]
%     This box has a height of 3cm.
%     \tcblower
%     Lower part.
%   \end{tcolorbox}
%   \end{exdispExample}
% \end{docTcbKey}



% % \enlargethispage*{10mm}
% \begin{docTcbKey}{height plus}{=\meta{length}}{no default, initially |0pt|}
%   % The box may extend a given fixed \refKey{/tcb/height} up to the given \meta{length}.

% 盒子将高度最大可扩展到 \refKey{/tcb/height} 加上给定的长度\footnote{译注:类似于定义的弹性长度。} \meta{length}。
% \begin{exdispExample}{height_plus}
% \tcbset{colback=red!5!white,colframe=red!75!black,left=1mm,top=1mm,bottom=1mm,
%   right=1mm,boxsep=0mm,width=3cm,nobeforeafter}

% \begin{tcolorbox}[height=1cm]
% This is a tcolorbox.
% \end{tcolorbox}
% \begin{tcolorbox}[height=1cm,height plus=1cm]
% This is a tcolorbox.
% \end{tcolorbox}
% \begin{tcolorbox}[height=1cm,height plus=1cm]
% This is a tcolorbox. This is a tcolorbox. This is a tcolorbox.
% \end{tcolorbox}
% \end{exdispExample}
% \end{docTcbKey}


% \begin{docTcbKey}{height from}{=\meta{min} \texttt{to} \meta{max}}{style, no default}
%   % Sets the box height to a dimension between \meta{min} and \meta{max}.

% 将盒子高度范围设置为从\meta{min}到\meta{max}。
% \begin{exdispExample}{height_from}
% % \usepackage{lipsum}
% \newtcolorbox{mybox}{colback=red!5!white,colframe=red!75!black,left=1mm,top=1mm,
%   bottom=1mm,right=1mm,boxsep=0mm,width=4.5cm,nobeforeafter,
%   height from=2cm to 8cm}

% \begin{mybox}
% This is a tcolorbox.
% \end{mybox}
% \begin{mybox}
% This is a tcolorbox. This is a tcolorbox. This is a tcolorbox.
% \end{mybox}
% \begin{mybox}
% \lipsum[2]
% \end{mybox}
% \end{exdispExample}
% \end{docTcbKey}



% \begin{docTcbKey}[][doc new=2014-10-31]{text height}{=\meta{length}}{style, no default}
%   % Sets the text height to \meta{length}. This is the length from the top 
%   % of the upper part to the bottom of the optional lower part.
%   % See also \refKey{/tcb/text width}.

% 将文本高度设置为 \meta{length}。这是从upper部分的顶部到可选的lower部分的底部的长度。另见 \refKey{/tcb/text width}.


% \begin{exdispExample}{text_height}
% \tcbset{colback=red!5!white,colframe=red!75!black}

% \begin{tcolorbox}[text height=2cm]
% This is a \textbf{tcolorbox} where the text area has a height of 2cm.
% \end{tcolorbox}
% \end{exdispExample}
% \end{docTcbKey}




% % \clearpage

% \begin{docTcbKey}[][doc new=2014-11-07]{add to height}{=\meta{length}}{style, no default}
%   % Adds \meta{length} to the current height of the colored box.
%   % \refKey{/tcb/height} has to be set before this key is used!
%   % If this option is used several times, then the \refKey{/tcb/height} is
%   % also increased several times.

% 为盒子的当前高度添加\meta{length}。%
% 在此命令之前设定的 \refKey{/tcb/height} 将被用上!%
% 如果此项设置了多次, 那么 \refKey{/tcb/height} 也会被增加{\bf 多次}。
% \begin{exdispExample}{add_to_height}
% \tcbset{height=2cm,
%   valign=center,width=(\linewidth-2mm)/2,
%   before=,after=\hfill,colframe=blue!75!black,colback=white}

% \begin{tcolorbox}
%   This box has a height of 2cm.
% \end{tcolorbox}
% \begin{tcolorbox}[add to height=1cm]
%   This box has a height of 3cm.
% \end{tcolorbox}
% \end{exdispExample}
% \end{docTcbKey}


% \begin{docTcbKey}[][doc new=2016-02-16]{add to natural height}{=\meta{length}}{style, no default}
%   % The application of this option generates a box with natural height plus
%   % the given \meta{length}. If this option is used several times, then the
%   % last setting of \meta{length} wins. The resulting box is not considered
%   % a fixed height box and the implementation is quite different to
%   % \refKey{/tcb/add to height}.

% 应用本项设置,将会生成一个盒子,高度为自然高度加上给定的 \meta{length}。
% 如果多次设置,{\bf 最后一次}设置的\meta{length}生效。 生成的盒子不是固定的高度且其实现也同 \refKey{/tcb/add to height} 很不同。
% \begin{exdispExample}{add_to_natural_height}
% \tcbset{valign=center,width=(\linewidth-2mm)/2,
%   before=,after=\hfill,colframe=blue!75!black,colback=white}

% \begin{tcolorbox}
%   This box has natural height.
% \end{tcolorbox}
% \begin{tcolorbox}[add to natural height=1cm]
%   This box has natural height plus 1 cm.
% \end{tcolorbox}
% \end{exdispExample}
% \end{docTcbKey}



% % \clearpage
% \begin{docTcbKey}[][doc new and updated={2014-09-22}{2016-02-17}]{height fill}{\colOpt{=true\textbar false\textbar maximum}}{default |true|, initially |false|}
%   % If set to \docValue*{true}, the height of the |tcolorbox| is set to the rest of the
%   % available vertical space of the current page.
%   % If set to \docValue{maximum}, the page is compressed as much as possible.
%   % Note that the |tcolorbox|
%   % is always set as its own paragraph using this option.
%   % Also see \refKey{/tcb/text fill}.
  
%   如果设置为 \docValue*{true}, |tcolorbox| 的高度将会设置为当前页的剩余空间的高度。%
%   如果设置为 \docValue{maximum}, 页面被尽可能多地压缩。%
%   注意, |tcolorbox| 中的段落始终设置上了这个选项。%
%   另见 \refKey{/tcb/text fill}.%
%   \begin{marker}
%   % Note that the library \mylib{breakable} has to be loaded to use this key!
  
%   注意,如果使用此设置,需要加载 \mylib{breakable} 库!
%   \end{marker}
%   % This height control key is only applicable to unbreakable boxes, but it
%   % uses code from the library \mylib{breakable}.
%   % The counterpart for breakable boxes is \refKey{/tcb/height fixed for}.
  
%   此项高度控制只作用在不可分页的盒子上, 但它复用了 \mylib{breakable} 库中的代码。
%   The counterpart for breakable boxes is \refKey{/tcb/height fixed for}.
  
%   % This option can and should not be used for boxes in boxes, but it can be
%   % used for boxes inside a \refEnv{tcbraster}.
  
%   此项不能也不应用在盒子中的盒子, 但它可以用在 \refEnv{tcbraster} 中的盒子。
  
%   \begin{dispListing}
%   % \usepackage{lipsum}
%   % \tcbuselibrary{breakable}
%   \begin{tcolorbox}[height fill,
%     colback=red!5!white,colframe=red!75!black,fonttitle=\bfseries,
%     title=填充页面剩余部分的盒子]
%   \lipsum[1]
%   \end{tcolorbox}
%   \end{dispListing}
%   \end{docTcbKey}
%   {\tcbusetemp}
  



% % \clearpage
% \begin{docTcbKey}[][doc new={2017-06-28}]{inherit height}{\colOpt{=\meta{fraction}}}{default |1|, initially unset}
%   % If this option is used for a |tcolorbox| which is embedded inside
%   % another (outer) |tcolorbox| \emph{and} if this outer |tcolorbox| has
%   % a fixed height, then the given \meta{fraction} of the available text height
%   % of the outer |tcolorbox| is used as \refKey{/tcb/height} for the current
%   % |tcolorbox|.
%   % Otherwise, \refKey{/tcb/natural height} is applied for the current
%   % |tcolorbox|.

% 如果此项所设置的 |tcolorbox| 是个嵌入在另一个%(外部)
%  |tcolorbox| 中,\emph{且}外部分的这个 |tcolorbox| 有着固定的高度, 那么,设定的 \meta{fraction} 乘以外部的这个 |tcolorbox| 盒子的剩余高度用作当前盒子的高度 \refKey{/tcb/height}。否则,当前 |tcolorbox| 盒子将应用自然高度 \refKey{/tcb/natural height} 。

% \begin{exdispExample}{inherit_height}
% \tcbset{colframe=blue!75!black,colback=white,fonttitle=\bfseries}

% \begin{tcolorbox}[title=外部盒子指定高度为4cm,height=4cm]
%   \begin{tcolorbox}[title=Inner box,nobeforeafter,inherit height]
%     这个内部盒子的高度使用外盒的剩余高度空间。
%   \end{tcolorbox}
% \end{tcolorbox}

% \begin{tcolorbox}[title=外部盒子使用自然高度]
%   \begin{tcolorbox}[title=Inner box,nobeforeafter,inherit height]
%     这个盒子使用自身的自然高度。
%   \end{tcolorbox}
% \end{tcolorbox}

% % Deeply nested box using 60 percent of the available space.
% % Deeply nested box using 40 percent of the available space.
% \begin{tcolorbox}[title=外部盒子指定高度为5cm,height=5cm]
%   \begin{tcolorbox}[title=内部盒子,nobeforeafter,inherit height]
%     \begin{tcolorbox}[colframe=red,beforeafter skip=0pt,inherit height=0.6]
%       使用60\%可用空间的嵌套盒子。
%     \end{tcolorbox}
%     \begin{tcolorbox}[colframe=red,beforeafter skip=0pt,inherit height=0.4]
%       使用40\%可用空间的嵌套盒子。
%     \end{tcolorbox}
%   \end{tcolorbox}
% \end{tcolorbox}
% \end{exdispExample}
% \end{docTcbKey}
 




% % \clearpage

% \begin{docTcbKey}[][doc new=2015-05-05]{square}{}{style, no value}
%   % Sets \refKey{/tcb/height} to match the width of the colored box.

% 设置盒子的高度(\refKey{/tcb/height})为盒子的宽度。
% \begin{exdispExample*}{square}{sbs,lefthand ratio=0.6}
% \begin{tcolorbox}[width=3cm,
%   colback=red!5!white,
%   colframe=red!75!black,
%   halign=center,valign=center,
%   square]
% This is a \textbf{tcolorbox}.
% \end{tcolorbox}
% \end{exdispExample*}
% \end{docTcbKey}



% \begin{docTcbKey}{space}{=\meta{fraction}}{no default, initially 0}
%   % If the height of a |tcolorbox| is not the natural height, the space
%   % difference between the forced and the natural size is distributed
%   % between the upper and the lower part of the box. This space could also
%   % be negative.
%   % \meta{fraction} with a value between 0 and 1 is the amount of space
%   % which is added to the upper part, the rest is added to the lower part.
%   % If there is no lower part, then all of the space is added to
%   % the upper part always.

% 如果 |tcolorbox| 的高度不是自然高度, 指定的高度和自然尺寸的的高度差分布在盒子的上下两部分中。 高度差可以是负值。
% \meta{fraction} 是0到1之前的数值,此值指定的比例添加到upper部分,剩余的高度添加到下部分。
% 如果不存在lower部分, 那么所有的空间将添加到upper部分。
% \begin{exdispExample}{fraction}
% \tcbset{width=(\linewidth-2mm)/4,before=,after=\hfill,
% colframe=blue!75!black,colback=white,height=3cm}

% \begin{tcolorbox}
% upper部分
% \tcblower
% lower部分
% \end{tcolorbox}
% \foreach \f in {0.2,0.4,0.7}
% {\begin{tcolorbox}[space=\f]
% upper部分
% \tcblower
% lower部分
% \end{tcolorbox}}
% \end{exdispExample}
% \end{docTcbKey}

% \begin{docTcbKey}{space to upper}{}{style}
%   % This is an abbreviation for |space=1|, i.\,e. all extra space is added
%   % to the upper part.

% 这是指定|space=1|的简写形式, i.\,e. 所有额外的空间都被添加到上部。(此法可读性更好)。
% \end{docTcbKey}

% \begin{docTcbKey}{space to lower}{}{style, initially set}
%   % This is an abbreviation for |space=0|, i.\,e. all extra space is added
%   % to the lower part (if there is any).

% 这是指定|space=0|的简写形式, i.\,e. 所有额外的空间都被添加到下部。(此法可读性更好)。
% \end{docTcbKey}




% % \clearpage
% \begin{docTcbKey}{space to both}{}{style}
%   % This is an abbreviation for |space=0.5|, i.\,e. the extra space
%   % equally distributed between the upper and the lower part.
% 这是指定|space=0.5|的简写形式, i.\,e. 额外的空间将平均分布到upper部分和lower部分。
% \begin{exdispExample}{space_to_both}
% \tcbset{width=(\linewidth-2mm)/3,before=,after=\hfill,
% colframe=blue!75!black,colback=white,height=3cm}

% \foreach \myspace in {space to upper,space to both,space to lower}
% {\begin{tcolorbox}[\myspace]
%     This is the upper part.
%     \tcblower
%     This is the lower part.
% \end{tcolorbox}}
% \end{exdispExample}
% \end{docTcbKey}



% \begin{docTcbKey}[][doc new and updated={2015-02-15}{2020-07-30}]{space to}{=\meta{macro}}{no default, initially unset}
%   % If the height of a |tcolorbox| is not the natural height, the space
%   % difference between the forced and the natural size is saved into the
%   % given local \meta{macro}. This \meta{macro} can and should be used inside
%   % the box content to add content which is vertically sized to match \meta{macro}.

% 如果|tcolorbox|盒子的高度不是自然高度, 指定的高度同自然高度差的数值保存到给出的宏命令 \meta{macro}。 这个 \meta{macro} 可以在盒子中使用以用来控制内容的高度恰好同这高度差一致。
%   \begin{marker}
%     \begin{itemize}
%     \item 
%     % The actual length saved into \meta{macro} is adapted dynamically
%     %   during several compilations -- at least two, but maybe more.
% 实际保存到 \meta{macro} 的值在多次编译期间是自适应 --- 至少2次, 可能更多次。
%     \item %
%     % Due to the adaption algorithm, objects can be sized with
%     %   \meta{macro} plus any offset length.
% 根据自适应算法, 对象尺寸可能在 \meta{macro} 之上添加额外的偏移量。
%     \item 
%     % Never ever use \meta{macro} multiplied with a factor. The only
%     %   exception to this rule is that the space can be split into parts which
%     %   sum to \meta{macro}.
% 永远不要使用 \meta{macro} 乘以一个因子。这个规则的唯一例外是,
% 分开的几个部分的高度和为\meta{macro}(即多个因子的和为1)。
%     \item %Never use this in combination with \refKey{/tcb/fit}.
% 不要同 \refKey{/tcb/fit} 组合使用。
%     \end{itemize}
%   \end{marker}
% \begin{exdispExample}[runs=3]{space_to_1}
% \begin{tcolorbox}[colframe=blue!75!black,colback=white,height=3cm,
%     space to=\myspace]
%     这是我的盒子高3cm。指定高度和自然高度差填充了图片    :\\[2mm]
%   \includegraphics[width=\linewidth,height=\myspace]{goldshade.png}\\[1mm]
%   这是其他一些文字。译注:图片的高度使用我们指定的 |\myspace|。
% \end{tcolorbox}
% \end{exdispExample}

% \begin{exdispExample}[runs=3]{space_to_2}
% \begin{tcolorbox}[colframe=blue!75!black,colback=white,height=3cm,
%     space to=\myspace]
%   \includegraphics[width=\linewidth,
%     height=0.33\dimexpr\myspace]{blueshade.png}\\[1mm]
%     这是我的盒子高3cm。\\[2mm]
%   \includegraphics[width=\linewidth,
%     height=0.67\dimexpr\myspace]{goldshade.png}
% \end{tcolorbox}
% \end{exdispExample}
% \end{docTcbKey}



% \begin{docTcbKey}{split}{=\meta{fraction}}{no default}
%   % If the height of a |tcolorbox| is not the natural height, the
%   % \meta{fraction} with a value between 0 and 1 determines the positioning
%   % of the segmentation between the upper and the lower part. Here, 0 stands
%   % for top and 1 for bottom. Note that the box is split regardless of
%   % the actual dimensions of the text parts!

% 如果 |tcolorbox| 的高度不是自然高度, 取值0到1的
%   \meta{fraction} 决定了上下两部分的分割位置。在这里,0代表顶部,1代表底部。
% 注意,不论文本部分的实际尺寸如何,盒子都会被分割!
% \begin{exdispExample}{split}
% \tcbset{width=(\linewidth-2mm)/3,before=,after=\hfill,height=3cm,
% colback=white,colframe=blue!75!black,valign=center,valign lower=center}

% \foreach \f in {0,0.1,0.5,0.8,0.9,1}
% {\begin{tcolorbox}[split=\f]
% 上,split: \f
% \tcblower
% This is the lower part with a lot of text in several lines.
% \end{tcolorbox}}
% \end{exdispExample}
% \begin{exdispExample}{split2}
%   \tcbset{width=(\linewidth-2mm)/3,before=,after=\hfill,height=1.8cm,
%   colback=white,colframe=blue!75!black,valign=center,valign lower=center}
  
%   \foreach \f in {0,0.1,0.5,0.8,0.9,1}
%   {\begin{tcolorbox}[split=\f]
%   上,split: \f
%   \tcblower
%   This is the lower part with a lot of text in several lines.
%   \end{tcolorbox}}
%   \end{exdispExample}
% \end{docTcbKey}


% % \clearpage
% \begin{docTcbKey}[][doc updated=2014-11-07]{equal height group}{=\meta{id}}{no default}
%   % Boxes which are members of an |equal height group| will all get the
%   % same height, i.\,e. the maximum of all their natural heights. The
%   % \meta{id} serves to distinguish between different height groups.
%   %This \meta{id} should contain only characters which are feasible
%   %for \TeX\ macro names, typically alphabetic characters but no numerals
%   %and spaces.
%   % Note that you have to compile twice to see changes and
%   % that height groups are global definitions.

% 同一|equal height group|的成员将拥有相同的高度, i.\,e. 它们自然高度的最大者的值。
% \meta{id} 用来区分不同的身高组别。
% %This \meta{id} should contain only characters which are feasible
% %for \TeX\ macro names, typically alphabetic characters but no numerals
% %and spaces.
% 注意,您必须编译两次才能看到更改,并且高度组是全局定义。

% \begin{exdispExample}[runs=2]{equal_height_group}
% \tcbset{width=(\linewidth-2mm)/3,before=,after=\hfill,arc=0mm,
% colframe=blue!75!black,colback=white,fonttitle=\bfseries}

% \begin{tcolorbox}[equal height group=A,adjusted title={一}]
% 这组最小的盒子
% \end{tcolorbox}%
% \begin{tcolorbox}[equal height group=A,adjusted title={二}]
% 这个盒子也小
%   \tcblower
% 但有lower部分。
% \end{tcolorbox}%
% \begin{tcolorbox}[equal height group=A,adjusted title={三}]
%   This box contains a lot of text just to fill the space
%   with word flowing and flowing and flowing until the box
%   is filled with all of it.
% \end{tcolorbox}\linebreak
% %
% \tcbset{width=(\linewidth-1mm)/2,before=,after=\hfill,arc=0mm,
% colframe=red!75!black,colback=white}
% %
% \begin{tcolorbox}[equal height group=B]
% 接着,我们使用另一个等高盒子组。
% \end{tcolorbox}%
% \begin{tcolorbox}[equal height group=B,after=]
%   \begin{equation*}
%     \int\limits_{0}^{1} x^2 = \frac13.
%   \end{equation*}
% \end{tcolorbox}
% \end{exdispExample}
% \end{docTcbKey}

% \medskip
% \begin{marker}
% % See \Vref{sec:raster} for more equal height options.
% 另见 \Vref{sec:raster} 了解更多等高组相关选项。
% \end{marker}



% % \clearpage
% \begin{docTcbKey}{minimum for equal height group}{=\meta{id}:\meta{length}}{no default, initially unset}
%   % Plants a \meta{length} into the equal height group with
%   % the given \meta{id}. This ensures that the height will not drop below
%   % \meta{length}. 
%   % Note that you cannot reduce a computed height value by using this key with a small value.
%   % The difference to applying \refKey{/tcb/height} directly is that the boxes
%   % are never too small for their content.

% % 指定值 \meta{length} 到等高组 \meta{id}。这确保高度不会小于 \meta{length}。
% 指定等高组 \meta{id} 的最小高度不小于 \meta{length}。%
% % 注意,不能通过使用小值来减少计算出的高度值。%
% 同使用 \refKey{/tcb/height} 相比,此项设置不会使盒子小于它们的内容高度。
 
% \begin{dispExample}
% \tcbset{colframe=blue!75!black,colback=white,arc=0mm,
%   before=,after=\hfill,fonttitle=\bfseries,left=2mm,right=2mm,
%   width=3.5cm,
%   equal height group=C,
%   minimum for equal height group=C:3.5cm}

% \begin{tcolorbox}
%   My first box. All boxes will get 3.5cm times 3.5cm
%   if the content height is not too large.
% \end{tcolorbox}%
% \begin{tcolorbox}
%   My second box.
%   \tcblower
%   This is the lower part.
% \end{tcolorbox}%
% \begin{tcblisting}{}
% \textbf{Mixed}
% with a listing.
% \end{tcblisting}
% \begin{tcolorbox}[title={Fourth box}]
%   My final box.
% \end{tcolorbox}%
% \end{dispExample}
% \end{docTcbKey}

% %todo 再看看
% \begin{docTcbKey}[][doc new=2016-03-24]{minimum for current equal height group}{=\meta{length}}{no default, initially unset}
%   % Sets \refKey{/tcb/minimum for equal height group} for the current equal height
%   % group. Apparently, this only works for an already known equal height group, i.e.
%   % \refKey{/tcb/equal height group} has to be set \emph{before} this option is used.
%   % This option is likely to be used in combination with \refKey{/tcb/raster equal height}

% 为当前等高组设置 \refKey{/tcb/minimum for equal height group}。
% 显然, 这只适用于已知的等高组, i.e.
% \refKey{/tcb/equal height group}已经在此设置\emph{之前}设置。
% 此项学与 \refKey{/tcb/raster equal height} 组合使用。
% \begin{exdispExample}[runs=2]{minimum_for_current_equal_height_group}
% % \tcbuselibrary{raster}
% \begin{tcbitemize}[raster equal height,colframe=blue!75!black,colback=white,
%   raster every box/.style={minimum for current equal height group=2cm}]
%   \tcbitem A
%   \tcbitem B
% \end{tcbitemize}
% \end{exdispExample}

% \end{docTcbKey}




% % \clearpage
% \begin{docTcbKey}[][doc new and updated={2015-11-27}{2016-02-22}]{use height from group}{\colOpt{=\meta{id}}}{style, default current group}
%   % Sets the current box to a fixed \refKey{/tcb/height} which is copied from
%   % an equal height group with the given \meta{id}. If this height is not
%   % available during the current compilation, no fixed height setting is used.
%   % If \meta{id} is omitted, the current equal height group is used which has
%   % to be set before by \refKey{/tcb/equal height group}.\par
%   % Note that the natural height of the current box is not considered for
%   % computation of the group height. The main application for
%   % \refKey{/tcb/use height from group} is that the height can be adapted
%   % further by \refKey{/tcb/add to height}.

% 设置当前盒子的高度为一个固定 \refKey{/tcb/height} 值,值来自一个等高组\meta{id}。如果在当前编译期间此高度不可用,则不使用固定高度设置。
% 如果省略了\meta{id}, 则使用此前的\refKey{/tcb/equal height group}设置的。\par
% 请注意,在计算组高度时不考虑当前盒子的自然高度。
% \refKey{/tcb/use height from group}主要用在当高度可以通过 \refKey{/tcb/add to height} 进一步调整。

% \begin{dispExample}
% \begin{tcolorbox}[use height from group=C,add to height=-2cm,
%   colframe=blue!75!black,colback=white]
% Height from group \enquote{C} of the previous example, but reduced by 2cm.
% \end{tcolorbox}%
% \end{dispExample}

% \begin{exdispExample}[runs=2]{use_height_from_group}
% % \tcbuselibrary{raster}
% Every line is inside an equal height group:
% \begin{tcbraster}[raster equal height=rows,
%     title=Box \thetcbrasternum,
%     enhanced,size=small,colframe=red!50!black,colback=red!10!white]
%   \begin{tcolorbox}First line\\second line\\
%     The height of this box rules.\end{tcolorbox}
%   \begin{tcolorbox}[use height from group]Test\end{tcolorbox}
%   \begin{tcolorbox}[use height from group]
%     First line\\second line\end{tcolorbox}
%   \begin{tcolorbox}The height of this box rules.\end{tcolorbox}
% \end{tcbraster}
% \end{exdispExample}
% \end{docTcbKey}



% \begin{docCommand}[doc new=2015-11-27]{tcbheightfromgroup}{\marg{macro}\marg{id}}
%   % Saves the height from an equal height group with the given \meta{id}
%   % to a \meta{macro}. If this height is not available during the current compilation,
%   % \meta{macro} is set to |0pt|.

% 保存等高组 \meta{id} 的高度到 \meta{macro}。如果在当前编译期间此高度不可用,
%  \meta{macro} 设为|0pt|.
% \end{docCommand}






% % \clearpage
% % Box Content Additions\hfill 
% \subsection{盒子内容添加}\label{subsec:contentadditions}
% % The following options introduce some arbitrary \meta{code} to the content
% % of a |tcolorbox|. These additions can be given at the beginning or at the ending
% % of the title, the upper part, or the lower part.

% 下面的选项介绍将一引动任意的 \meta{code} 附加到 |tcolorbox| 的内容中。可以附加在标题、upper部分或lower部分的开头或结尾。

% \begin{docTcbKey}{before title}{=\meta{code}}{no default, initially unset}
%   % The given \meta{code} is placed \emph{after} the color and font settings
%   % and \emph{before} the content of the title.

% 给出的\meta{code}被放置在标题的颜色和字体设置\emph{之后} 、内容\emph{之前}。
% \begin{exdispExample}{before_title}
% \tcbset{before title={\textcolor{yellow}{\large Important:}~},
%   colback=red!5!white,colframe=red!75!black,fonttitle=\bfseries}

% \begin{tcolorbox}[title=My title]
% This is a \textbf{tcolorbox}.
% \end{tcolorbox}
% \end{exdispExample}
% \end{docTcbKey}


% \begin{docTcbKey}{after title}{=\meta{code}}{no default, initially unset}
%   % The given \meta{code} is placed \emph{after} the content of the title.

% 给出的\meta{code}被放置在标题的内容\emph{之后}。
% \begin{exdispExample}{after_title}
% \tcbset{after title={\hfill\colorbox{Navy}{approved}},
%   colback=red!5!white,colframe=red!75!black,fonttitle=\bfseries}

% \begin{tcolorbox}[title=My title]
% This is a \textbf{tcolorbox}.
% \end{tcolorbox}
% \end{exdispExample}
% \end{docTcbKey}




% % \clearpage
% \begin{docTcbKey}{before upper}{=\meta{code}}{no default, initially empty}
%   % The given \meta{code} is placed \emph{after} the color and font settings
%   % and \emph{before} the content of the upper part.
%   % The \meta{code} is appended by a final |\ignorespaces|.

% 给出的\meta{code}被放置在upper部分的颜色和字体设置\emph{之后}、内容\emph{之前}。
% \meta{code}会被附加一个最后的|\ignorespaces|。

% \begin{exdispExample}{before_upper}
% \tcbset{before upper={\textit{The story:}\par},
%   colback=red!5!white,colframe=red!75!black,fonttitle=\bfseries}

% \begin{tcolorbox}[title=My title]
% This is a \textbf{tcolorbox}.
% \end{tcolorbox}
% \end{exdispExample}
% \end{docTcbKey}


% \begin{docTcbKey}[][doc new=2019-02-26]{before upper*}{=\meta{code}}{no default, initially unset}
%   % The given \meta{code} is placed \emph{after} the color and font settings
%   % and \emph{before} the content of the upper part.
%   % In contrast to \refKey{/tcb/before upper}, no |\ignorespaces| is appended.
%   % Use this for situations where |\ignorespaces| is not needed or causes harm.

% 给出的\meta{code}被放置在upper部分的颜色和字体设置\emph{之后}、内容\emph{之前}。
% 同\refKey{/tcb/before upper}对比,不附加|\ignorespaces|。
% 当不需要|\ignorespaces|或|\ignorespaces|会导致问题时使用此项。

% \begin{exdispExample}{before_upper_star}
% \begin{tcolorbox}[size=small,tile,
%   colback=yellow!20,colbacktitle=yellow!70!black,
%   title=My table,hbox,center,center title,
%   before upper*=\begin{tabular}{cc},
%   after upper*=\end{tabular},
% ]
%   \multicolumn{2}{c}{Title}\\
%   one & two \\
%   three & four\\
% \end{tcolorbox}
% \end{exdispExample}
% \end{docTcbKey}

% % \clearpage





% \begin{docTcbKey}[][doc updated=2016-10-21]{after upper}{=\meta{code}}{no default, initially empty}
%   % The given \meta{code} is placed \emph{after} the content of the upper part.
%   % The \meta{code} is prepended by a leading |\unskip|.

% 给出的\meta{code}会被放置到upper部分的内容\emph{之后}。
% \meta{code}之前会插入|\unskip|。

% \begin{exdispExample}{after_upper_1}
% \tcbset{after upper={\par\hfill\textit{Read more next week}},
%   colback=red!5!white,colframe=red!75!black,fonttitle=\bfseries}

% \begin{tcolorbox}[title=My title]
% This is a \textbf{tcolorbox}.
% \end{tcolorbox}

% \begin{tcolorbox}[after upper={\par\hfill---\textit{王勃}}]
% 穷且益坚,不坠青云之志。
% \end{tcolorbox}
% \end{exdispExample}

% \begin{exdispExample}{after_upper_2}
% \begin{tcolorbox}[before upper=\flqq,after upper=\frqq,
%   colback=red!5!white,colframe=red!75!black]
% This is a \textbf{tcolorbox}.\footnote{译注:想到可以用命令行盒子,加这个书名号的效果!}
% \end{tcolorbox}
% \end{exdispExample}
% \end{docTcbKey}




% \begin{docTcbKey}[][doc new and updated={2016-10-21}{2019-02-28}]{after upper*}{=\meta{code}}{no default, initially unset}
%   % The given \meta{code} is placed \emph{after} the content of the upper part.
%   % In contrast to \refKey{/tcb/after upper}, no |\unskip| is prepended.
%   % Use this for situations where |\unskip| is not needed or causes harm.
%   % See \refKey{/tcb/before upper*} for an example.

% 给出的\meta{code}会被放置到upper部分的内容\emph{之后}。%
% 同\refKey{/tcb/after upper}相比,没有在前附加|\unskip|。%
% 当不需要 |\unskip| 或 |\unskip| 会导致问题时使用此项。%
% 例见 \refKey{/tcb/before upper*}。

% \begin{marker}
% 从版本 3.80 到 3.94, 此项会将 |\unskip| 添加到 \meta{code} 之前。\\
% 版本 3.95 到 4.15, 此项不建议使用。\\
% 版本 4.20 起, 这个选项是用改变的语义重新建立的 (没有 |\unskip|!)
% \end{marker}
% \end{docTcbKey}





% % \clearpage
% \begin{docTcbKey}{before lower}{=\meta{code}}{no default, initially empty}
%   % The given \meta{code} is placed \emph{after} the color and font settings
%   % and \emph{before} the content of the lower part.
%   % The \meta{code} is appended by a final |\ignorespaces|.

% 将 \meta{code} 放到lower部分的颜色和字段设置\emph{之后}、内容\emph{之前}。
% \meta{code} 之后附加一个|\ignorespaces|。
% \begin{exdispExample}{before_lower}
% \tcbset{before lower=\textit{Behold:~},colback=red!5!white,colframe=red!75!black}

% \begin{tcolorbox}
% This is a \textbf{tcolorbox}.
% \tcblower
% This is the lower part.
% \end{tcolorbox}
% \end{exdispExample}
% \end{docTcbKey}


% \begin{docTcbKey}[][doc new=2019-02-26]{before lower*}{=\meta{code}}{no default, initially unset}
%   % The given \meta{code} is placed \emph{after} the color and font settings
%   % and \emph{before} the content of the lower part.
%   % In contrast to \refKey{/tcb/before lower}, no |\ignorespaces| is appended.
%   % Use this for situations where |\ignorespaces| is not needed or causes harm.

% \meta{code} 放置到lower部分的颜色和字体设置\emph{之后}、内容\emph{之前}。
% 同 \refKey{/tcb/before lower} 相比, 尾部不会附加 |\ignorespaces|。
% 当不需要 |\ignorespaces| 或它会导致问题时使用此项。
% \begin{exdispExample}{before_lower_star}
% \begin{tcolorbox}[size=small,bicolor,sidebyside,center lower,
%   colback=yellow!30,colbacklower=yellow!20,colframe=yellow!80!black,
%   before lower*=\begin{tabular}{cc},
%   after lower*=\end{tabular},
% ]
% My table
% \tcblower
%   \multicolumn{2}{c}{Title}\\
%   one & two \\
%   three & four\\
% \end{tcolorbox}
% \end{exdispExample}
% \end{docTcbKey}





% % \clearpage

% \begin{docTcbKey}[][doc updated=2016-10-21]{after lower}{=\meta{code}}{no default, initially empty}
%   % The given \meta{code} is placed \emph{after} the content of the lower part.
%   % The \meta{code} is prepended by a leading |\unskip|.

% \meta{code} 放置到lower部分的内容\emph{之后}。
% \meta{code} 之前会插入 |\unskip|。

% \begin{exdispExample}{after_lower_2}
% \begin{tcolorbox}[after lower=\ \textit{This is the end.},
%   colback=red!5!white,colframe=red!75!black]
% This is a \textbf{tcolorbox}.
% \tcblower
% This is the lower part.
% \end{tcolorbox}
% \end{exdispExample}
% \end{docTcbKey}


% \begin{docTcbKey}[][doc new and updated={2016-10-21}{2019-02-28}]{after lower*}{=\meta{code}}{no default, initially unset}
%   % The given \meta{code} is placed \emph{after} the content of the lower part.
%   % In contrast to \refKey{/tcb/after upper}, no |\unskip| is prepended.
%   % Use this for situations where |\unskip| is not needed or causes harm.

% \meta{code} 放置到lower部分的内容\emph{之后}。
% 同 \refKey{/tcb/after upper} 相比, 不会在头部附加 |\unskip|。
% 当不需要 |\unskip| 或它会导致问题时使用此项。

% \begin{exdispExample}{after_lower_1}
% \begin{tcolorbox}[before lower*=$,after lower*=$,
%   colback=red!5!white,colframe=red!75!black]
% This is a \textbf{tcolorbox}.
% \tcblower
% \sin^2(x)+\cos^2(x)=1.
% \end{tcolorbox}
% \end{exdispExample}

% \begin{marker}
%   From version 3.80 to 3.94, this option prepended an |\unskip| to the given \meta{code}.\\
%   From version 3.95 to 4.15, this option was deprecated.\\
%   From version 4.20, this option is re-established with changed semantic (no |\unskip|!)
% \end{marker}
% \end{docTcbKey}



% % \clearpage
% \begin{marker}
%   % If \refKey{/tcb/text fill} is used, one cannot have a lower part
%   % and the box is unbreakable.
%   如果使用了 \refKey{/tcb/text fill} , 则没有 lower 部分,且盒子是不可分割的。
%   \end{marker}
  
%   \begin{docTcbKey}[][doc new=2015-07-15]{text fill}{}{style, no value}
%     % This style sets \refKey{/tcb/before upper} and \refKey{/tcb/after upper}
%     % to embed the upper part with a minipage. If a fixed height was applied
%     % e.g.\ by \refKey{/tcb/height} or \refKey{/tcb/height fill}, this minipage
%     % gets a matching height. This allows to use vertical glue macros like
%     % |\vfill| to act like expected. If the box has no fixed height,
%     % setting \refKey{/tcb/text fill} has no other effect as making the box
%     % unbreakable. 
  
%   此项设置 \refKey{/tcb/before upper} 和 \refKey{/tcb/after upper}
%   以将upper部分包围在 minipage 环境中。 如果高度指定为固定的值
%   e.g.\ 使用 \refKey{/tcb/height} 或 \refKey{/tcb/height fill}, 此 minipage 环境得到一个匹配的高度。 这允许我们使用竖直的粘连% glue 
%   宏,如 |\vfill| 能正常工作。如果盒子没有指定固定的高度,%
%   设置 \refKey{/tcb/text fill} 没有其他作用,但盒子会变成不可分的。
%   \begin{exdispExample}{text_fill}
%   \begin{tcolorbox}[colback=red!5!white,colframe=red!75!black,fonttitle=\bfseries,
%     height=8cm,text fill,
%     title=My filled box]
%   This is a \textbf{tcolorbox}.
%   \par\vfill
%   \begin{center}
%     My middle text.
%   \end{center}
%   \par\vfill
%   This is the end of my box.
%   \end{tcolorbox}
%   \end{exdispExample}
%   \end{docTcbKey}

  

% % \clearpage

% \begin{docTcbKey}[][doc new={2019-09-19}]{tabulars}{=\meta{preamble}}{style}
%   % This style sets \refKey{/tcb/before upper} and \refKey{/tcb/after upper}
%   % and several geometry keys to support a |tabular*| with the
%   % given \meta{preamble}.
%   % The packages |array| and |colortbl| have to be loaded separately.

% 此项设置了 \refKey{/tcb/before upper} 和 \refKey{/tcb/after upper} 和一些命令,使得支持使用 |tabular*| 并在 \meta{preamble} 中指定表格头。
% 需要分别加载宏包 |array| 和 |colortbl|。
% \begin{exdispExample}{tabulars_1}
% % \usepackage{array}
% % \usepackage{colortbl} - or - \usepackage[table]{xcolor}
% \tcbset{enhanced,fonttitle=\bfseries\large,fontupper=\normalsize\sffamily,
%   colback=yellow!10!white,colframe=red!50!black,colbacktitle=Salmon!30!white,
%   coltitle=black,center title}

% \begin{tcolorbox}[tabulars={@{\extracolsep{\fill}\hspace{5mm}}lrrrrr@{\hspace{5mm}}},
%   boxrule=0.5pt,title=My table]
% Group & One     & Two     & Three    & Four     & Sum\\\hline\hline
% Red   & 1000.00 & 2000.00 &  3000.00 &  4000.00 & 10000.00\\\hline
% Green & 2000.00 & 3000.00 &  4000.00 &  5000.00 & 14000.00\\\hline
% Blue  & 3000.00 & 4000.00 &  5000.00 &  6000.00 & 18000.00\\\hline\hline
% Sum   & 6000.00 & 9000.00 & 12000.00 & 15000.00 & 42000.00
% \end{tcolorbox}
% \end{exdispExample}
% \end{docTcbKey}


% \begin{docTcbKey}[][doc new={2019-09-19}]{tabulars*}{=\marg{code}\marg{preamble}}{style}
%   % This is a variant of \refKey{/tcb/tabulars} which adds some \meta{code}
%   % before the table starts.

% 这是 \refKey{/tcb/tabulars} 的变种,可以附加 \meta{code} 到表格的开头。
% % before the table starts.
% \begin{exdispExample}{tabulars_2}
% % \usepackage{array}
% % \usepackage{colortbl} - or - \usepackage[table]{xcolor}
% \tcbset{enhanced,fonttitle=\bfseries\large,fontupper=\normalsize\sffamily,
%   colback=yellow!10!white,colframe=red!50!black,colbacktitle=Salmon!30!white,
%   coltitle=black,center title}

% \begin{tcolorbox}[tabulars*={\arrayrulewidth0.5mm\renewcommand\arraystretch{1.4}}%
%     {@{\extracolsep{\fill}\hspace{20mm}}lll@{\hspace{20mm}}},
%   title=My table]
% One     & Two     & Three \\\hline\hline
% 1000.00 & 2000.00 &  3000.00\\\hline
% 2000.00 & 3000.00 &  4000.00
% \end{tcolorbox}
% \end{exdispExample}
% \end{docTcbKey}




% % \clearpage
% \begin{marker}
% % If \refKey{/tcb/tabularx} or \refKey{/tcb/tabularx*} are used, one cannot
% % have a lower part.

% 如果使用了 \refKey{/tcb/tabularx} 或 \refKey{/tcb/tabularx*} , 将不会有lower部分。
% \end{marker}



% \begin{docTcbKey}{tabularx}{=\meta{preamble}}{style}
%   % This style sets \refKey{/tcb/before upper} and \refKey{/tcb/after upper}
%   % and several geometry keys to support a |tabularx| with the
%   % given \meta{preamble}.
%   % The packages |tabularx| \cite {carlisle:tabularx}, |array|, and |colortbl|
%   % have to be loaded separately.

% 此项设置了 \refKey{/tcb/before upper} 和 \refKey{/tcb/after upper} 以及一些命令以支持 |tabularx| 环境,且可以指定表头为 \meta{preamble}。
% 需要加载宏包 |tabularx| %\cite {carlisle:tabularx}
% , |array|, 和 |colortbl|。
% \begin{exdispExample}{tabularx_1}
% % \usepackage{array,tabularx}
% % \usepackage{colortbl} - or - \usepackage[table]{xcolor}
% \newcolumntype{Y}{>{\raggedleft\arraybackslash}X}% see tabularx
% \tcbset{enhanced,fonttitle=\bfseries\large,fontupper=\normalsize\sffamily,
%   colback=yellow!10!white,colframe=red!50!black,colbacktitle=Salmon!30!white,
%   coltitle=black,center title}

% \begin{tcolorbox}[tabularx={X||Y|Y|Y|Y||Y},title=My table]
% Group & One     & Two     & Three    & Four     & Sum\\\hline\hline
% Red   & 1000.00 & 2000.00 &  3000.00 &  4000.00 & 10000.00\\\hline
% Green & 2000.00 & 3000.00 &  4000.00 &  5000.00 & 14000.00\\\hline
% Blue  & 3000.00 & 4000.00 &  5000.00 &  6000.00 & 18000.00\\\hline\hline
% Sum   & 6000.00 & 9000.00 & 12000.00 & 15000.00 & 42000.00
% \end{tcolorbox}
% \end{exdispExample}
% \end{docTcbKey}
 

% \begin{docTcbKey}{tabularx*}{=\marg{code}\marg{preamble}}{style}
%   % This is a variant of \refKey{/tcb/tabularx} which adds some \meta{code}
%   % before the table starts.

% 这是 \refKey{/tcb/tabularx} 的变种,附加 \meta{code} 到表格的开头。
% \begin{exdispExample}{tabularx_2}
% % \usepackage{array,tabularx}
% % \usepackage{colortbl} - or - \usepackage[table]{xcolor}
% \tcbset{enhanced,fonttitle=\bfseries\large,fontupper=\normalsize\sffamily,
%   colback=yellow!10!white,colframe=red!50!black,colbacktitle=Salmon!30!white,
%   coltitle=black,center title}

% \begin{tcolorbox}[tabularx*={\arrayrulewidth0.5mm}{X|X|X},title=My table]
% One     & Two     & Three \\\hline\hline
% 1000.00 & 2000.00 &  3000.00\\\hline
% 2000.00 & 3000.00 &  4000.00
% \end{tcolorbox}
% \end{exdispExample}
% \end{docTcbKey}

 


% % \clearpage
% \begin{docTcbKey}{tikz upper}{\colOpt{=\meta{options}}}{style}
%   % This style adds a centered |tikzpicture| environment to the start and end
%   % of the upper part. The \meta{options} may be given as \tikzname\  picture options.

% 此项将upper部分的内容放入一个 |tikzpicture| 环境。给定的选项 \meta{options} 会传递给 |tikzpicture| 环境。 % \tikzname\  picture
% \begin{exdispExample}{tikz_upper}
% % \usepackage{tikz}

% \begin{tcolorbox}[tikz upper,fonttitle=\bfseries,colback=white,colframe=black,
%                   title=\tikzname\ 绘制]
%   \path[fill=yellow,draw=yellow!75!red] (0,0) circle (1cm);
%   \fill[red] (45:5mm) circle (1mm);
%   \fill[red] (135:5mm) circle (1mm);
%   \draw[line width=1mm,red] (215:5mm) arc (215:325:5mm);
% \end{tcolorbox}
% \end{exdispExample}
% \end{docTcbKey}


% \begin{docTcbKey}{tikz lower}{\colOpt{=\meta{options}}}{style}
%   % This style adds a centered |tikzpicture| environment to the start and end
%   % of the lower part. The \meta{options} may be given as \tikzname\  picture options.

% 此项将lower部分的内容放入一个 |tikzpicture| 环境。给定的选项 \meta{options} 会传递给 |tikzpicture| 环境。% \tikzname\  picture 。
% \begin{exdispExample}{tikz_lower}
% % \usepackage{tikz}
% % \tcbuselibrary{skins,listings}

% \begin{tcblisting}{tikz lower%lower 部分放入 tikzpicture 环境
%   ,listing side text% LaTeX源码和效果各一边
%   ,fonttitle=\bfseries,
%   bicolor,colback=LightBlue!50!white,colbacklower=white,colframe=black,
%   righthand width=3cm,title=\tikzname\ drawing}
% \path[fill=yellow,draw=yellow!75!red]
%     (0,0) circle (1cm);
% \fill[red] (45:5mm) circle (1mm);
% \fill[red] (135:5mm) circle (1mm);
% \draw[line width=1mm,red]
%     (215:5mm) arc (215:325:5mm);
% \end{tcblisting}
% \end{exdispExample}
% \end{docTcbKey}




% % \clearpage
% \begin{docTcbKey}{tikznode upper}{\colOpt{=\meta{options}}}{style}
%   % This style places the upper part content into a centered
%   % \tikzname\  node. The \meta{options} may be given as \tikzname\  node options.
%   % This style is especially useful for boxes with multiline texts which are
%   % fitted to the text width.

% 此项将upper部分的内容放到一个居中的 \tikzname\  node。选项 \meta{options} 会传递给 \tikzname\  node 。
% 此项常用于包含多行文本的盒子,可以适应文本宽度。
% \begin{exdispExample}{tikznode_upper}
% % \usepackage{tikz}
% \newtcbox{\headline}[1][]{enhanced,center,
%   ignore nobreak,fontupper=\Large\bfseries,
%   colframe=red!50!black,colback=red!10!white,
%   drop fuzzy shadow=yellow,tikznode upper,#1}

% \headline{Important\\Headline}
% \end{exdispExample}
% \end{docTcbKey}

% \begin{docTcbKey}{tikznode lower}{\colOpt{=\meta{options}}}{style}
%   % This style places the lower part content into a centered
%   % \tikzname\ node. The \meta{options} may be given as \tikzname\  node options.

% 此项将lower部分的内容放到一个居中的 \tikzname\  node。选项 \meta{options} 会传递给 \tikzname\  node 。
% \begin{exdispExample}{tikznode_lower}
% % \usepackage{tikz}
% \begin{tcolorbox}[bicolor,colback=LightBlue!50!white,colbacklower=white,
%   colframe=black,tikznode lower={inner sep=2pt,draw=red,fill=yellow}]
% Upper part.
% \tcblower
% Lower part.
% \end{tcolorbox}
% \end{exdispExample}
% \end{docTcbKey}



% \begin{docTcbKey}{tikznode}{\colOpt{=\meta{options}}}{style}
%   % Shortcut for setting \refKey{/tcb/tikznode upper} and \refKey{/tcb/tikznode lower}
%   % the same time.

% 同时设置 \refKey{/tcb/tikznode upper} 和 \refKey{/tcb/tikznode lower} 的简写形式。
% \end{docTcbKey}

% \begin{docTcbKey}{varwidth upper}{\colOpt{=\meta{length}}}{style, default \refKey{/tcb/width}}
%   % This style places the upper part content into a |varwidth| environment.
%   % This style needs the |varwidth| package \cite{arseneau:2011a} to be loaded manually.
%   % The resulting box has a maximal width of \meta{length}.
%   % This option is only senseful for a \refCom{tcbox}.


% 此项将upper部分放到一个 |varwidth| 环境中。需要手动加载 |varwidth| 宏包 %\cite{arseneau:2011a}
% 。产出的盒子的最大宽度是 \meta{length}。此项只对 \refCom{tcbox} 生效。
% \begin{exdispExample*}{varwidth_upper}{sbs,lefthand ratio=0.6}
% % \usepackage{varwidth}
% \newtcbox{\varbox}{colframe=red!50!black,
%   colback=red!10!white,varwidth upper}

% \varbox{Short text.}
% \varbox{This box contains is a longer text
%   which is broken.}
% \end{exdispExample*}
% \end{docTcbKey}





% % \clearpage
% % Overlays\hfill 
% \subsection{覆盖物}\label{subsec:overlays}
% % With an overlay, arbitrary \meta{graphical code} can be added to a
% % |tcolorbox|. This code is executed \emph{after} the frame and interior are
% % drawn and \emph{before} the text content is drawn. Therefore, you can
% % decorate the |tcolorbox| with your own extensions.
% % Common special cases are \emph{watermarks} which are implemented using overlays.
% % See Subsection \ref{subsec:watermarks} from page \pageref{subsec:watermarks} if
% % you want to add \emph{watermarks}.
% % todo interior 是
% 通过overlay, 可以将任意的 \meta{graphical code} 添加到 |tcolorbox| 中。
% 这部分代码附加到边框和内部%interior
% \emph{之后}、文本内容绘制\emph{之前}。 因此,你可以自行扩展装饰 |tcolorbox| 环境。常见的一个特定情况是使用overlays实现\emph{水印}。
% 如果你想添加\emph{水印},见\pageref{subsec:watermarks}页的\ref{subsec:watermarks}。



% \begin{marker}
%   % If you use the core package only, the \meta{graphical code} has to be |pgf| code
%   % and there is not much assistance for positioning.
%   % Therefore, the usage of the \refKey{/tcb/enhanced} mode from the library skins
%   % is recommended which allows |tikz| code and gives access to
%   % \refKey{/tcb/geometry nodes} for positioning.
  
%   如果您仅使用核心包, \meta{graphical code} 必须是 |pgf| 代码,而且也没有太多的定位辅助。因此, 推荐使用 |skins| 库的 \refKey{/tcb/enhanced} 模式,它不仅允许 |tikz| 代码,还允许用 \refKey{/tcb/geometry nodes} 进行定位。
%   \end{marker}
 
   
%   \begin{docTcbKey}{overlay}{=\meta{graphical code}}{no default, initially unset}
%     % Adds \meta{graphical code} to the box drawing process. This \meta{graphical code}
%     % is drawn \emph{after} the frame and interior and \emph{before} the text content.
  
%   将 \meta{graphical code} 添加到盒子的绘制过程中。\meta{graphical code}将附加到边框和内部的绘制\emph{之后}、文本内容\emph{之前}。
  
%   \begin{exdispExample}{overlay_1}
%   % \tcbuselibrary{skins} % preamble
%   \tcbset{frogbox/.style={enhanced,colback=green!10,colframe=green!65!black,
%     enlarge top by=5.5mm,
%     overlay={\foreach \x in {2cm,3.5cm} {
%       \begin{scope}[shift={([xshift=\x]frame.north west)}]
%         \path[draw=green!65!black,fill=green!10,line width=1mm] (0,0) arc (0:180:5mm);
%         \path[fill=black] (-0.2,0) arc (0:180:1mm);
%       \end{scope}}}}}
  
%   \begin{tcolorbox}[frogbox,title=My title]
%   This is a \textbf{tcolorbox}.
%   \end{tcolorbox}
%   \end{exdispExample}
  
%   \enlargethispage*{5mm}
%   \begin{exdispExample}{overlay_2}
%   % \usetikzlibrary{patterns} % preamble
%   % \tcbuselibrary{skins}     % preamble
%   \tcbset{ribbonbox/.style={enhanced,colback=red!5!white,colframe=red!75!black,
%     fonttitle=\bfseries,
%     overlay={\path[fill=blue!75!white,draw=blue,double=white!85!blue,
%       preaction={opacity=0.6,fill=blue!75!white},
%       line width=0.1mm,double distance=0.2mm,
%       pattern=fivepointed stars,pattern color=white!75!blue]
%       ([xshift=-0.2mm,yshift=-1.02cm]frame.north east)
%       -- ++(-1,1) -- ++(-0.5,0) -- ++(1.5,-1.5) -- cycle;}}}
  
%   \begin{tcolorbox}[ribbonbox,title=My title]
%   This is a \textbf{tcolorbox}.
%   \tcblower
%   This is the lower part.
%   \end{tcolorbox}
%   \end{exdispExample}
%   \end{docTcbKey}




% % \clearpage
% \begin{docTcbKey}{no overlay}{}{style, no default, initially set}
%   % Removes the overlay if set before.

% 移除覆盖层。
% \end{docTcbKey}

% \begin{docTcbKey}{overlay broken}{=\meta{graphical code}}{no default, initially unset}
%   % If the box is set to be \refKey{/tcb/breakable} and \emph{is} broken actually,
%   % then the \meta{graphical code} is added to the box drawing process.
%   % \refKey{/tcb/overlay} overwrites this key.

% 如果盒子设置为\refKey{/tcb/breakable}且\emph{实际上}分页了,那么\meta{graphical code}会被添加到盒子的绘制过程中。\refKey{/tcb/overlay}会覆盖此项设置。\footnote{译注:即实际上分页时才添加覆盖层。}
% \end{docTcbKey}

% \begin{docTcbKey}{overlay unbroken}{=\meta{graphical code}}{no default, initially unset}
%   % If the box is set to be \refKey{/tcb/breakable} but \emph{is not} broken actually
%   % or if the box is set to be \refKey{/tcb/unbreakable},
%   % then the \meta{graphical code} is added to the box drawing process.
%   % \refKey{/tcb/overlay} overwrites this key.

% 如果盒子设置了\refKey{/tcb/breakable}但\emph{实际上没}分页,或盒子设置为\refKey{/tcb/unbreakable},那么\meta{graphical code}会被添加到盒子的绘制过程中。\refKey{/tcb/overlay}会覆盖此项设置。\footnote{译注:即实际上没有分页时才添加覆盖层。}
% \end{docTcbKey}



% \begin{docTcbKey}{overlay first}{=\meta{graphical code}}{no default, initially unset}
%   % If the box is set to be \refKey{/tcb/breakable} and \emph{is} broken actually,
%   % then the \meta{graphical code} is added to the box drawing process for
%   % the \emph{first} part of the break sequence.
%   % \refKey{/tcb/overlay} overwrites this key.

% 如果盒子设置了\refKey{/tcb/breakable}且\emph{事实上}分页了,
% 那么\meta{graphical code}会被添加到盒子的分开的\emph{第一}部分的绘制过程中。
% \refKey{/tcb/overlay}会覆盖此项设置。
% \end{docTcbKey}

% \begin{docTcbKey}{overlay middle}{=\meta{graphical code}}{no default, initially unset}
%   % If the box is set to be \refKey{/tcb/breakable} and \emph{is} broken actually,
%   % then the \meta{graphical code} is added to the box drawing process for
%   % the \emph{middle} parts (if any) of the break sequence.
%   % \refKey{/tcb/overlay} overwrites this key.

% 如果盒子设置了\refKey{/tcb/breakable}且\emph{事实上}分页了,
% 那么\meta{graphical code}会被添加到盒子的分开的序列中的\emph{中间}部分(如果有)。 \refKey{/tcb/overlay}会覆盖此项设置。
% \end{docTcbKey}

% \begin{docTcbKey}{overlay last}{=\meta{graphical code}}{no default, initially unset}
%   % If the box is set to be \refKey{/tcb/breakable} and \emph{is} broken actually,
%   % then the \meta{graphical code} is added to the box drawing process for
%   % the \emph{last} part of the break sequence.
%   % \refKey{/tcb/overlay} overwrites this key.

% 如果盒子设置了\refKey{/tcb/breakable}且\emph{事实上}分页了,
% 那么\meta{graphical code}会被添加到盒子的分开的\emph{最后}一部分的绘制过程中。
% \refKey{/tcb/overlay}会覆盖此项设置。
% \end{docTcbKey}




% \begin{docTcbKey}{overlay unbroken and first}{=\meta{graphical code}}{no default, initially unset}
%   % This is an optimized abbreviation for setting
%   % \refKey{/tcb/overlay unbroken} and
%   % \refKey{/tcb/overlay first} together.
%   % \refKey{/tcb/overlay} overwrites this key.

% 这是同时设置\refKey{/tcb/overlay unbroken}和\refKey{/tcb/overlay first}的优化缩写。
% \refKey{/tcb/overlay}会覆盖此项设置。
% \end{docTcbKey}

% \begin{docTcbKey}{overlay middle and last}{=\meta{graphical code}}{no default, initially unset}
%   % This is an optimized abbreviation for setting
%   % \refKey{/tcb/overlay middle} and
%   % \refKey{/tcb/overlay last} together.
%   % \refKey{/tcb/overlay} overwrites this key.

% 这是同时设置\refKey{/tcb/overlay middle}和\refKey{/tcb/overlay last}的优化缩写。
% \refKey{/tcb/overlay}会覆盖此项设置。
% \end{docTcbKey}

% \begin{docTcbKey}{overlay unbroken and last}{=\meta{graphical code}}{no default, initially unset}
%   % This is an optimized abbreviation for setting
%   % \refKey{/tcb/overlay unbroken} and
%   % \refKey{/tcb/overlay last} together.
%   % \refKey{/tcb/overlay} overwrites this key.

% 这是同时设置\refKey{/tcb/overlay unbroken}和\refKey{/tcb/overlay last}的优化缩写。
% \refKey{/tcb/overlay}会覆盖此项设置。
% \end{docTcbKey}





% \begin{docTcbKey}[][doc new=2014-09-19]{overlay first and middle}{=\meta{graphical code}}{no default, initially unset}
%   % This is an optimized abbreviation for setting
%   % \refKey{/tcb/overlay first} and \refKey{/tcb/overlay middle} together.
%   % \refKey{/tcb/overlay} overwrites this key.

% 这是同时设置\refKey{/tcb/overlay first}和\refKey{/tcb/overlay middle}的优化缩写。
% \refKey{/tcb/overlay}会覆盖此项设置。
% \end{docTcbKey}



% % todo 再看
% \begin{dispListing*}{breakable,vfill before first,before upper={This example demonstrates
%   the application of break sequence specific overlay options.
%   Here, we define an environment |myexample| based
%   on |tcolorbox| where the visible drawing is done totally by overlay keys.\par
%   Here, the first application of |myexample| produces an unbroken |tcolorbox|.
%   The frame is drawn by the code given with \refKey{/tcb/overlay unbroken}.\par
%   The second application of |myexample| is broken into several parts which
%   are drawn by the codes given with
%   \refKey{/tcb/overlay first}, \refKey{/tcb/overlay middle}, and
%   \refKey{/tcb/overlay last}.
%   \par\bigskip
%   }}
%   % Preamble:
%   %\usepackage{tikz,lipsum}
%   %\tcbuselibrary{skins,breakable}
%   %\newcounter{example}
%   \colorlet{colexam}{red!75!black}
%   \newtcolorbox[use counter=example]{myexample}{%
%     empty,title={Example \thetcbcounter},attach boxed title to top left,
%     boxed title style={empty,size=minimal,toprule=2pt,top=4pt,
%       overlay={\draw[colexam,line width=2pt]
%         ([yshift=-1pt]frame.north west)--([yshift=-1pt]frame.north east);}},
%     coltitle=colexam,fonttitle=\Large\bfseries,
%     before=\par\medskip\noindent,parbox=false,boxsep=0pt,left=0pt,right=3mm,top=4pt,
%     breakable,pad at break*=0mm,vfill before first,
%     overlay unbroken={\draw[colexam,line width=1pt]
%       ([yshift=-1pt]title.north east)--([xshift=-0.5pt,yshift=-1pt]title.north-|frame.east)
%       --([xshift=-0.5pt]frame.south east)--(frame.south west); },
%     overlay first={\draw[colexam,line width=1pt]
%       ([yshift=-1pt]title.north east)--([xshift=-0.5pt,yshift=-1pt]title.north-|frame.east)
%       --([xshift=-0.5pt]frame.south east); },
%     overlay middle={\draw[colexam,line width=1pt] ([xshift=-0.5pt]frame.north east)
%       --([xshift=-0.5pt]frame.south east); },
%     overlay last={\draw[colexam,line width=1pt] ([xshift=-0.5pt]frame.north east)
%       --([xshift=-0.5pt]frame.south east)--(frame.south west);},%
%   }
  
%   \begin{myexample}
%   \lipsum[1]
%   \end{myexample}
  
%   \begin{myexample}
%   \lipsum[2-11]
%   \end{myexample}
  
%   \lipsum[12]% following text
%   \end{dispListing*}
%   {\tcbusetemp}
  
  
%   %\begin{dispExample}
%   %% \tcbuselibrary{skins}
%   %% \newcounter{example}
%   %\newtcolorbox[use counter=example]{FancyTitle}[3][]{%
%   %  enhanced,colback=blue!10!white,colframe=orange,top=4mm,
%   %  enlarge top by=\baselineskip/2+1mm,
%   %  enlarge top at break by=0mm,pad at break=2mm,
%   %  fontupper=\normalsize,label={#3},
%   %  overlay unbroken and first={%
%   %    \node[rectangle,rounded corners,draw=black,fill=blue!20!white,
%   %      inner sep=1mm,anchor=west,font=\small]
%   %      at ([xshift=4.5mm]frame.north west)
%   %         {\strut\textbf{Example \thetcbcounter: #2}};},
%   %  #1}%
  
%   %\begin{FancyTitle}{My fancy title}{fancy:title}
%   %  \lipsum[1]
%   %\end{FancyTitle}
%   %\end{dispExample}




%   % \clearpage
%   % Floating Objects\hfill 
%   \subsection{浮动对象}
%   \begin{docTcbKey}{floatplacement}{=\meta{values}}{no default, initially \texttt{htb}}
%     % Sets \meta{values} as default values for the usage of \refKey{/tcb/float}
%     % and \refKey{/tcb/float*}.
%     % Feasible are the usual parameters for floating objects.
  
%   设置\meta{values}为\refKey{/tcb/float}和\refKey{/tcb/float*}的默认值。可选值是浮动对象的常用参数。
%   \begin{dispListing}
%   \tcbset{enhanced,colback=red!5!white,colframe=red!75!black,
%       watermark color=red!15!white}
  
%   \begin{tcolorbox}[floatplacement=t,float,
%                     title=Floating box from |floatplacement|,
%                     watermark text={I am floating}]
%     This floating box is placed at the top of a page.
%   \end{tcolorbox}
%   \end{dispListing}
%   \end{docTcbKey}
%   {\tcbusetemp}
  
  
%   \begin{docTcbKey}{float}{\colOpt{=\meta{values}}}{default from \texttt{floatplacement}}
%     % Turns the box to a floating object where \meta{values} are the
%     % usual parameters for such floating objects.
%     % If they are not used, the placement uses the default values given by
%     % |floatplacement|.
  
%   将盒子转为浮动对象,\meta{values}是其浮动位置的参数。如果没有指定,那么将使用 |floatplacement| 设置的默认值。
%   \begin{dispListing}
%   \begin{tcolorbox}[float, title=Floating box from |float|,
%       enhanced,watermark text={I'm also floating}]
%     This box floats to a feasible place automatically. You do not have to
%     use a numbering for this floating object.
%   \end{tcolorbox}
%   \end{dispListing}
%   \end{docTcbKey}
%   {\tcbusetemp}
  
  
%   \begin{docTcbKey}{float*}{\colOpt{=\meta{values}}}{default from \texttt{floatplacement}}
%     % Identical to \refKey{/tcb/float}, but for wide boxes spanning the whole page
%     % width of two column documents or in conjunction with the packages
%     % |multicol| or |paracol|. Note that you have to set |width=\textwidth|
%     % additionally, if the box should span the whole page width in these cases!
  
%   同\refKey{/tcb/float}一样,但用在|multicol|或|paracol|的双栏中排版横跨页面的盒子。 
%   注意,你需要额外设置 |width=\textwidth|, 如果需要让盒子在这些情况下横跨页面的话!
%   \begin{dispListing}
%   \begin{tcolorbox}[float*=b, title=Floating box from |float*|,width=\textwidth,
%       enhanced,watermark text={I'm also floating}]
%     In this single column document, you will see no difference to |float|.
%   \end{tcolorbox}
%   \end{dispListing}
%   \end{docTcbKey}
%   {\tcbusetemp}




% \begin{docTcbKey}{nofloat}{}{style, initially set}
%   % Turns the floating behavior off.

% 关闭浮动行为。
% \end{docTcbKey}


% \begin{docTcbKey}[][doc new=2014-09-19]{every float}{=\meta{code}}{no default, initially empty}
%   % For floating objects, the \refKey{/tcb/before} and \refKey{/tcb/after}
%   % settings are ignored. Instead, the given \meta{code} is inserted before
%   % a floating box. If the box is \refKey{/tcb/breakable}, the given \meta{code} is
%   % inserted before every part of the break sequence.
%   % The most common use case is |every float=\centering|.

% 浮动对象会忽略\refKey{/tcb/before}和\refKey{/tcb/after}的设置。
% 取而代之的是, \meta{code}会被插入到浮动盒子之前。
% 如果盒子设置为\refKey{/tcb/breakable}, 那么\meta{code}会被插入每个分开的部分的前部。
% 最常见的用例是|every float=\centering|.

% \begin{dispListing}
% \tcbox[float=htb,title={Floating box},every float=\centering,
%   colback=blue!50!black,colframe=blue!50!white,colbacktitle=blue!10!white,
%   coltitle=black,center title]
%   {\includegraphics[height=6cm]{lichtspiel.jpg}}
% \end{dispListing}
% {\tcbset{reset}\tcbusetemp}

% \end{docTcbKey}

% % \clearpage
% % Embedding into the Surroundings\hfill 
% \subsection{嵌入周围}\label{subsec:surroundings}
% % Typically, but not necessarily, a |tcolorbox| is put inside a separate paragraph and has some vertical space before and after it.
% % This behavior is controlled by the keys \refKey{/tcb/before} and \refKey{/tcb/after}.

% 通常情况下,但不是必须地,一个|tcolorbox|盒子被放置在单独的段落中,在它%之前和之后
% 的前后有一些垂直空间。%
% 这种行为是由\refKey{/tcb/before}和\refKey{/tcb/after}控制。

% \begin{marker}
% % Before version 4.40, the default setting for \refKey{/tcb/before}
% % and \refKey{/tcb/after} was given by \refKey{/tcb/autoparskip}.
% % Starting with version 4.40, the default setting is given by
% % \refKey{/tcb/before skip balanced} and \refKey{/tcb/after skip balanced}.\par
% % Note that old documents may need adaptions of page breaks.\par
% % Alternatively, the old default setting can be restored by using

% 版本4.40之前,\refKey{/tcb/before}和\refKey{/tcb/after}的默认设置由\refKey{/tcb/autoparskip}控制。%
% 从4.40开始, 默认设置由\refKey{/tcb/before skip balanced}和\refKey{/tcb/after skip balanced}控制。\par
% 请注意,旧文档可能需要调整分页。\par
% 或者,可以使用以下命令恢复旧的默认设置
% \begin{dispListing}
% \tcbsetforeverylayer{autoparskip}
% \end{dispListing}
% % inside the document preamble.

% 在文档导言区中。
% \end{marker}

% \begin{docTcbKey}{before}{=\meta{code}}{no default, initially see \refKey{/tcb/before skip balanced}}
%   % Sets the \meta{code} which is executed before the colored box.
%   % It is not used for floating boxes.
%   % Also, it is not used, if the box follows a heading immediately
%   % and \refKey{/tcb/ignore nobreak} is set to \docValue{false}.

% \meta{code}将在盒子绘制前执行。此项不用于浮动盒子。另外一个不生效的情况:如果盒子之前紧跟着标题且\refKey{/tcb/ignore nobreak}设为\docValue{false}.
% \end{docTcbKey}

% \begin{docTcbKey}{after}{=\meta{code}}{no default, initially see \refKey{/tcb/after skip balanced}}
%   % Sets the \meta{code} which is executed after the colored box.
%   % It is not used for floating boxes.

% \meta{code}将在盒子绘制后执行。此项不用于浮动盒子。
% \end{docTcbKey}


% \begin{docTcbKey}{nobeforeafter}{}{style, no value}
%   % Abbreviation for clearing the keys |before| and |after|. The colored box
%   % is not put into a paragraph and there is no space before or after the box.

% 清除|before|和|after|的简写形式。盒子没有放入段落中,盒子之前或之后没有空间。\footnote{译注:用上后,多个盒子就不会分行了}
% \begin{exdispExample}{nobeforeafter}
% \tcbset{myone/.style={colback=LightGreen,colframe=DarkGreen,
%   equal height group=nobefaf,width=\linewidth/4,nobeforeafter}}
% \begin{tcolorbox}[myone,title=Box 1]Box 1\end{tcolorbox}%
% \begin{tcolorbox}[myone,title=Box 2]Box 2\end{tcolorbox}%
% \begin{tcolorbox}[myone,title=Box 3]Box 3\end{tcolorbox}%
% \begin{tcolorbox}[myone,title=Box 4]Box 4\end{tcolorbox}
% \end{exdispExample}
% \end{docTcbKey}



% \begin{docTcbKey}{force nobeforeafter}{}{style, no value}
%   % Forces the setting of \refKey{/tcb/nobeforeafter} even if
%   % \refKey{/tcb/before} and \refKey{/tcb/after} are set to other values
%   % later. Do not use this option globally unless you \emph{really} know what you do.
%   % Note that embedded boxes do not inherit this forced clearance.

% 强制设置\refKey{/tcb/nobeforeafter},甚至\refKey{/tcb/before}和\refKey{/tcb/after}是在这个选项设置之后又设置的。不要全局使用此选项,除非您\emph{真}的知道在做什么。注意,嵌套的盒子不会继承此项设置。
% \end{docTcbKey}





% % \clearpage

% \begin{docTcbKey}[][doc new={2020-09-25}]{before skip balanced}{=\meta{glue}}{no default, initially |0.5\textbackslash baselineskip plus 2pt|}
%   % Inserts some vertical space before the colored box. This style sets \refKey{/tcb/before}.\par
%   % If the depth of the
%   % preceeding \TeX\ box is between |0pt| and |0.3\baselineskip|,
%   % the distance between the \emph{baseline} of the preceeding \TeX\ box and the tcolorbox
%   % ist set to \meta{glue}$+$|0.3\baselineskip|.\par
%   % If the depth is larger, the distance of the preceeding \TeX\ box and the tcolorbox
%   % ist set to \meta{glue}.\par
%   % Alternatively, see \refKey{/tcb/before skip} which ignores the \emph{baseline}.

% 在 tcolorbox 盒子之前,插入一些竖直的空白。此项也会设置 \refKey{/tcb/before}。\par
% 如果要处理的 \TeX\ 盒子的深度介于 |0pt| 到 |0.3\baselineskip|,
% \TeX\ 盒子的 \emph{基线} 到 tcolorbox 盒子间的空白高为 \meta{glue}$+$|0.3\baselineskip|.\par
% 如果深度更大些, 则 \TeX\ 盒子和 tcolorbox 盒子间的空白设为 \meta{glue}.\par
% 也可以参阅 \refKey{/tcb/before skip},它忽略 \emph{baseline}.

% \begin{exdispExample*}{before_skip_balanced}{sbs,lefthand ratio=0.6}
% Some text.
% \begin{tcolorbox}[before skip balanced=1cm,
%     colframe=red!50!white]
%   This is a \textbf{tcolorbox}.
% \end{tcolorbox}
% \end{exdispExample*}
% \end{docTcbKey}


% \begin{docTcbKey}[][doc new={2020-09-25}]{after skip balanced}{=\meta{glue}}{no default, initially |0.5\textbackslash baselineskip plus 2pt|}
%   % Inserts some vertical space of the given \meta{glue} after the colored box.
%   % This style sets \refKey{/tcb/after}.
%   % Additionally, |\prevdepth| is set to |0.3\baselineskip|. The following
%   % \TeX\ box may enlarge the space by further glue to adjust its \emph{baseline}.
%   % Alternatively, see \refKey{/tcb/after skip} which ignores the \emph{baseline}.

% 插入指定的竖直空白到盒子后,高度为 \meta{glue}。%
% 此项会设置 \refKey{/tcb/after}。%
% 此外, |\prevdepth| 设为 |0.3\baselineskip|。下面的 \TeX\ 盒子可以通过弹性%进一步弹性来
% 扩大空间以调整其\emph{基线}。%
% 另见 \refKey{/tcb/after skip},它忽略 \emph{baseline}。

% \begin{exdispExample*}{after_skip_balanced}{sbs,lefthand ratio=0.6}
% \begin{tcolorbox}[after skip balanced=1cm,
%     colframe=red!50!white]
%   This is a \textbf{tcolorbox}.
% \end{tcolorbox}
% Some text.
% \end{exdispExample*}
% \end{docTcbKey}



% \begin{docTcbKey}[][doc new={2020-09-25}]{beforeafter skip balanced}{=\meta{glue}}{no default, initially |0.5\textbackslash baselineskip plus 2pt|}
%   插入一些指定高度为 \meta{glue} 的垂直空间到盒子的前面和后面。
%   此项会同时设置 \refKey{/tcb/before skip balanced} 和 \refKey{/tcb/after skip balanced}。
%   % todo tikzpicture
%   \begin{exdispExample*}{beforeafter_skip_balanced}{sbs,lefthand ratio=0.6}
%   \newtcolorbox{doubleline}[1][]{
%     beforeafter skip balanced=0pt,
%     height=1.8\baselineskip,
%     enlarge top by=.1\baselineskip,
%     enlarge bottom by=.1\baselineskip,
%     colframe=blue!20,colback=blue!5,
%     size=small,valign upper=center,#1 }
  
%   \noindent\begin{tikzpicture}
%   \path[use as bounding box] (0,0)
%     rectangle (0.1,0.1);
%   \foreach \y in {0,1,...,9}  {
%     \draw[very thin,red]
%       (-0.2,-\y*\baselineskip) --
%       (\linewidth+0.2cm,-\y*\baselineskip); }
%   \end{tikzpicture}
%   line 1\par
%   \begin{doubleline}  Abc  \end{doubleline}
%   \begin{doubleline}  Def  \end{doubleline}
%   line 2g\par
%   \begin{doubleline}  Ghi  \end{doubleline}
%   line 3\par
%   line 4 g
%   \end{exdispExample*}
%   \end{docTcbKey}

  

% % \clearpage

% \begin{docTcbKey}[][doc new and updated={2020-09-25}{2015-03-16}]{before skip}{=\meta{glue}}{style, no default}
%   在盒子之前插入指定高度\meta{glue}的垂直空间。%
%   此项会设置 \refKey{/tcb/before}。%
%   同 \refKey{/tcb/before skip balanced} 相比, 这个 \meta{glue} upper 部分的边缘,并不是到基线位置。
%   \begin{exdispExample*}{before_skip}{sbs,lefthand ratio=0.6}
%   Some text.
%   \begin{tcolorbox}[before skip=1cm,
%       colframe=red!50!white]
%     This is a \textbf{tcolorbox}.
%   \end{tcolorbox}
  
%   Some text.
%   \begin{tcolorbox}[before skip=0cm,
%     colframe=red!50!white]
%   This is a \textbf{tcolorbox}.
%   \end{tcolorbox}
%   \end{exdispExample*}
%   \end{docTcbKey}
  
%   \begin{docTcbKey}[][doc new and updated={2020-09-25}{2017-02-01}]{after skip}{=\meta{glue}}{style, no default}
%   在盒子之后插入指定高度\meta{glue}的垂直空间。%
%   此项会设置 \refKey{/tcb/after}.
%   同 \refKey{/tcb/after skip balanced} 相比, %
%   这个 \meta{glue} 是相对于 lower 部分的边缘,并不是到基线位置。%\footnote{译注:后面有空想下英文原文是否有误?}
%   \begin{exdispExample*}{after_skip}{sbs,lefthand ratio=0.6}
%   \begin{tcolorbox}[after skip=1cm,
%       colframe=red!50!white]
%     This is a \textbf{tcolorbox}.
%   \end{tcolorbox}
%   Some text.
%   \end{exdispExample*}
%   \end{docTcbKey}
  
%   \begin{docTcbKey}[][doc new=2014-10-10]{beforeafter skip}{=\meta{glue}}{style, no default}
%     % Inserts some vertical space of the given \meta{glue} before \emph{and} after the colored box.
%     % This style sets \refKey{/tcb/before skip} and \refKey{/tcb/after skip}.
  
%   在盒子的前后插入指定高度\meta{glue}的垂直空间。%
%   此项会设置 \refKey{/tcb/before skip} 和 \refKey{/tcb/after skip}。
  
  
%     \begin{exdispExample*}{beforeafter_skip}{sbs,lefthand ratio=0.6}
%   \tcbset{beforeafter skip=0pt,
%     colframe=red!50!white}
  
%   text before
%   \begin{tcolorbox}
%     This is a \textbf{tcolorbox}.
%   \end{tcolorbox}
%   \begin{tcolorbox}
%     Second box.
%   \end{tcolorbox}
%   text after
%   \end{exdispExample*}
%   \end{docTcbKey}
 
  

% % \clearpage

% \begin{docTcbKey}[][doc new=2014-11-07]{left skip}{=\meta{length}}{style, no default, initially |0mm|}
%   % Inserts some horizontal space of the given \meta{length} before the colored box.
%   % This style sets \refKey{/tcb/grow to left by} with the negated \meta{length},
%   % i.e. the bounding box and box width are changed.
  
% 在盒子前插入给定 \meta{length} 的水平空间。
% 此项会设定 \refKey{/tcb/grow to left by} 为 \meta{length} 的相反数,
%   i.e. 边界框和盒子宽度已更改。
% \begin{exdispExample*}{left_skip}{sbs,lefthand ratio=0.6}
% \noindent\rule{\linewidth}{2pt}

% \begin{tcolorbox}[left skip=1cm,
%     colframe=red!50!white]
%   This is a \textbf{tcolorbox}.
% \end{tcolorbox}
% \end{exdispExample*}
% \end{docTcbKey}

% \begin{docTcbKey}[][doc new=2014-11-07]{right skip}{=\meta{length}}{style, no default, initially |0mm|}
%   % Inserts some horizontal space of the given \meta{length} after the colored box.
%   % This style sets \refKey{/tcb/grow to right by} with the negated \meta{length},
%   % i.e. the bounding box and box width are changed.

% 在盒子{\bf 后}插入给定 \meta{length} 的水平空间。%
% 此项会设定  \refKey{/tcb/grow to right by} 为 \meta{length} 的相反数,
%   i.e. 边界框和盒子宽度已更改。
% \begin{exdispExample*}{right_skip}{sbs,lefthand ratio=0.6}
% \noindent\rule{\linewidth}{2pt}

% \begin{tcolorbox}[right skip=1cm,
%     colframe=red!50!white]
%   This is a \textbf{tcolorbox}.
% \end{tcolorbox}
% \end{exdispExample*}
% \end{docTcbKey}



% \begin{docTcbKey}[][doc new=2014-10-10]{leftright skip}{=\meta{length}}{style, no default}
%   % Inserts some horizontal space of the given \meta{length} before \emph{and} after the colored box.
%   % This style changes the bounding box and the box width.

%   在盒子前后插入给定长度为 \meta{length} 的水平空间。
%   此样式更改边界盒子和盒子宽度。

%   \begin{exdispExample*}{leftright_skip}{sbs,lefthand ratio=0.6}
% \noindent\rule{\linewidth}{2pt}

% \begin{tcolorbox}[leftright skip=1cm,
%     colframe=red!50!white]
%   This is a \textbf{tcolorbox}.
% \end{tcolorbox}
% \end{exdispExample*}
% \end{docTcbKey}


% % \clearpage

% \begin{docTcbKey}[][doc updated=2017-02-01]{parskip}{}{style, no value}
%   % This options is considered to be superseded by
%   % \refKey{/tcb/before skip balanced} and \refKey{/tcb/after skip balanced}
%   % (see note on page~\pageref{subsec:surroundings}).\par
%   % Sets the keys |before| and |after| to values which are
%   % recommended, if the package |parskip| \emph{is} used and there is no better
%   % idea for |before| and |after|. This is similar to:

%   此项有考虑使用 \refKey{/tcb/before skip balanced} 和 \refKey{/tcb/after skip balanced} 取代。
%   (见~\pageref{subsec:surroundings}页).\par
%   将 |before| 和 |after| 的值设置为推荐的值,如果{使用}了 |parskip| 包且 |before| 和 |after| 的值没有更好的主意。效果类似于:
% \begin{dispListing}
% \tcbset{parskip/.style={before={\par\pagebreak[0]\parindent=0pt},
%                         after={\par}}}
% \end{dispListing}
% \end{docTcbKey}

% \begin{docTcbKey}[][doc updated=2017-02-01]{noparskip}{}{style, no value}
%   % This options is considered to be superseded by
%   % \refKey{/tcb/before skip balanced} and \refKey{/tcb/after skip balanced}
%   % (see note on page~\pageref{subsec:surroundings}).\par
%   % Sets the keys |before| and |after| to values which are
%   % recommended, if the package |parskip| is \emph{not} used and there is no better
%   % idea for |before| and |after|. This is similar to:

%   此项有考虑使用 \refKey{/tcb/before skip balanced} 和 \refKey{/tcb/after skip balanced} 取代。
%   (见~\pageref{subsec:surroundings}页).\par
%   将 |before| 和 |after| 的值设置为推荐的值,如果{使用}了 |parskip| 包且 |before| 和 |after| 的值没有更好的主意。效果类似于:
% \begin{dispListing}
% \tcbset{noparskip/.style={before={\par\pagebreak[0]\smallskip\parindent=0pt},
%                           after={\par\smallskip}}}
% \end{dispListing}
% \end{docTcbKey}



% \begin{docTcbKey}{autoparskip}{}{style, no value}
%   % This options is considered to be superseded by
%   % \refKey{/tcb/before skip balanced} and \refKey{/tcb/after skip balanced}
%   % (see note on page~\pageref{subsec:surroundings}).\par
%   % Tries to detect the usage of the package |parskip| and sets
%   % the keys |before| and |after| accordingly. Actually, the following is done:

%   这项可以考虑改用 \refKey{/tcb/before skip balanced} 和 \refKey{/tcb/after skip balanced} 替换(另见~\pageref{subsec:surroundings}~页)。\par
%   尝试检测 |parskip| 的使用情况, 并相应的设置 |before| 和 |after|。 实际上,完成了以下操作:

%   \begin{itemize}
%   \item 
%   % If the length of |\parskip| is greater than |0pt| at the beginning of the document,
%   %   \refKey{/tcb/parskip} is executed. Here, the usage of package |parskip| is \emph{assumed}.

%   % todo
% 如果在文档的开头 |\parskip| 的值大于|0pt|,%
% \refKey{/tcb/parskip} 将会执行。 这里, 假定 |parskip| 引入\footnote{Here, the usage of package |parskip| is \emph{assumed}}。

%   \item 
%   % Otherwise, if the length of |\parskip| is not greater than |0pt| at the beginning of the document,
%   %   \refKey{/tcb/noparskip} is executed. Here, the absence of package |parskip| is \emph{assumed}.
% 另外,如果在文档的开头 |\parskip| 的值不大于|0pt|,%
% \refKey{/tcb/noparskip} 将会执行。 这里, 假定 |parskip| 没有使用\footnote{Here, the absence of package |parskip| is \emph{assumed}}。
%   \end{itemize}
% \end{docTcbKey}




% % \clearpage

% \begin{docTcbKey}{baseline}{=\meta{length}}{no default, initially |0pt|}
%   % Used to set the |\pgfsetbaseline| value of the resulting |tcolorbox|.
% 设置结果盒子的%,同其他 \TeX\ 对象对齐用的
% 基线的值(|\pgfsetbaseline|)。

% \begin{exdispExample}{baseline}
% \tcbset{colframe=red!50!white,width=4cm,nobeforeafter}
% Some text\dotfill
% \begin{tcolorbox}[baseline=3mm]
% 第一行
% \end{tcolorbox}
% \begin{tcolorbox}[baseline=3mm]
% 第一行\\第二行
% \end{tcolorbox}
% \begin{tcolorbox}[baseline=4mm]
% 第一行\\第二行\\第三行
%   \end{tcolorbox}
% \end{exdispExample}
% \end{docTcbKey}




% \begin{docTcbKey}[][doc new=2014-10-10]{box align}{=\meta{alignment}}{style, no default, initially |bottom|}
%   % Used to set the \refKey{/tcb/baseline} value of the resulting |tcolorbox|.
%   % Feasible values for \meta{alignment} are:

%   Used to set the \refKey{/tcb/baseline} value of the resulting |tcolorbox|.
%   Feasible values for \meta{alignment} are:  
%   \begin{itemize}
%   \item\docValue{bottom}: %alignment with the box bottom,
%   与盒子底部对齐,
%   \item\docValue{top}: %alignment with the box top,
%   与盒子顶部对齐,
%   \item\docValue{center}: %alignment with the box center,
%   与盒子中心对齐,
%   \item\docValue{base}: 
%   % alignment with the box content base. This option
%   %   is not applicable for a \refEnv{tcolorbox} but for a \refCom{tcbox} only.
%   %   It is an alias for \refKey{/tcb/tcbox raise base}.
%   与盒子内容的基线对齐。此项不在 \refEnv{tcolorbox} 使用,仅在 \refCom{tcbox} 生效。
% 这是 \refKey{/tcb/tcbox raise base} 的别名。
%   \end{itemize}

% \begin{exdispExample}{box_align_1}
% \tcbset{colframe=red!50!white,width=4cm,nobeforeafter}
% Some text\dotfill
% \begin{tcolorbox}[box align=bottom]
%   bottom
% \end{tcolorbox}
% \begin{tcolorbox}[box align=bottom]
%   bottom\\bottom
% \end{tcolorbox}
% \begin{tcolorbox}
% 第一行\\第二行\\第三行
% \end{tcolorbox}
% \end{exdispExample}

% \begin{exdispExample}{box_align_2}
% \tcbset{colframe=red!50!white,width=4cm,nobeforeafter}
% Some text\dotfill
% \begin{tcolorbox}[box align=top]
%   top
% \end{tcolorbox}
% \begin{tcolorbox}[box align=top]
%   top\\top
% \end{tcolorbox}
% \begin{tcolorbox}
% 第一行\\第二行\\第三行
% \end{tcolorbox}
% \end{exdispExample}

% \begin{exdispExample}{box_align_3}
% \tcbset{colframe=red!50!white,width=4cm,nobeforeafter}
% Some text\dotfill
% \begin{tcolorbox}[box align=center]
%   center
% \end{tcolorbox}
% \begin{tcolorbox}[box align=center]
%   center\\center
% \end{tcolorbox}
% \begin{tcolorbox}
%   第一行\\第二行\\第三行
%     \end{tcolorbox}
% \end{exdispExample}

% \begin{exdispExample}{box_align_4}
% \tcbset{colframe=red!50!white,nobeforeafter}
% Some text\dotfill
% \tcbox[nobeforeafter,box align=base]{base}
% \tcbox[nobeforeafter,box align=base,size=fbox]{base}
% \tcbox[nobeforeafter]{未设置}
% \end{exdispExample}
% \end{docTcbKey}





% \begin{docTcbKey}[][doc new=2014-12-11]{ignore nobreak}{\colOpt{=true\textbar false}}{default |true|, initially |false|}
%   % After a heading, \LaTeX\ tries to avoid a break by setting a |nobreak| boolean value.
%   % Starting from version |3.33|, the \refKey{/tcb/before} respectively \refKey{/tcb/before skip}
%   % settings are not used after a heading if \refKey{/tcb/ignore nobreak} is
%   % set to \docValue{false}. For an unbreakable box, \refKey{/tcb/before nobreak} is used instead.
%   % Further, a \refKey{/tcb/breakable} box will also try to
%   % avoid a break between a heading and a directly following first part of a
%   % break sequence.
  
%   在标题之后, 通过设置 |nobreak| ,\LaTeX\ 会尝试避免分页。
%   从版本 |3.33| 开始, 如果 \refKey{/tcb/ignore nobreak} 设置为 \docValue{false}, 那么 \refKey{/tcb/before} 和 \refKey{/tcb/before skip}
%   的设置在标题之后将不生效。%
%   对于一个不可分的盒子, 将使用 \refKey{/tcb/before nobreak} 替代使用。
%   将来, 一个设置了 \refKey{/tcb/breakable} 的盒子,将会尝试避免在标题和紧随其后的中断序列的第一部分之间中断。
  
%   % Set \refKey{/tcb/ignore nobreak} to \docValue{true}, if |nobreak| should be
%   % ignored as prior to version |3.33|. Also, such a setting may be used locally to
%   % enforce the \refKey{/tcb/before} setting.
%   在版本 |3.33|,如果需要保留这个忽略 |nobreak| 的效果,将 \refKey{/tcb/ignore nobreak} 设置为 \docValue{true}。 此外,这样的设置可以在 locally 使用以强制设置 \refKey{/tcb/before} 。
%   \end{docTcbKey}
  
%   \begin{docTcbKey}[][doc new=2014-12-16]{before nobreak}{=\meta{code}}{no default, initially \cs{noindent}}
%     % Sets the \meta{code} which is executed before the colored box if it
%     % is unbreakable, if \refKey{/tcb/ignore nobreak} is not set, and if
%     % the box follows a heading.
  
%   如果是不可以分的,设置 \meta{code} 在盒子之前执行, 如果没有设置 \refKey{/tcb/ignore nobreak} , 或如果盒子是跟随在标题之后。
%   \end{docTcbKey}

  

% \begin{docTcbKey}[][doc new=2017-02-23]{parfillskip restore}{\colOpt{=true\textbar false}}{default |true|, initially |true|}
%   % If this option is set to be |true|, the minimum value of |\parfillskip| is
%   % tested at specific spots, if it is greater than |0pt|.
%   % If so, |\parfillskip| is restored to |\@flushglue| which happens to be
%   % the default value.

% 如果此项设置为 |true|, 则在特定点测试 |\parfillskip| 的最小值, 如果它大于 |0pt|.
% 如果是这样,|\parfillskip| 恢复到 |\@flushglue|, 这恰好是默认值。

%   % These tests are executed for
% 这些判断将会在以下位置执行:
%   \refKey{/tcb/parskip},
%   \refKey{/tcb/noparskip},
%   \refKey{/tcb/after skip},
%   \refKey{/tcb/breakable}, and
%   \refEnv{tcbraster}.

%   % This option was created to automatically
%   % avoid overfull box warnings with |\parfillskip| changing packages.

% 创建此选项是为了自动避免 |\parfillskip| 改变包裹带来的 |overfull box| 警告 。
% \end{docTcbKey}




% % \clearpage
% % Bounding Box
% \subsection{边界盒子}
% % Normally, every |tcolorbox| has a bounding box which fits exactly to the
% % dimensions of the outer frame. Therefore, \LaTeX\ reserves exactly the space
% % needed for the box.
% % This behavior can be changed by enlarging (or shrinking) the bounding box.
% % If the bounding box is enlarged, the |tcolorbox| will get some clearance
% % around it. If the bounding box is shrunk, i.\,e.\ enlarged with negative
% % values, the |tcolorbox| will overlap to other parts of the page.
% % For example, the |tcolorbox| could be stretched into the page margin.

% 通常,每个 |tcolorbox| 盒子有一个与其外框严丝合缝的边界盒子。%
% 因此,\LaTeX\ 保留了盒子所需的空间。%
% 可以通过扩大(或缩小)边界盒子来更改此空间。%
% 如果边界盒子被放大, 那么 |tcolorbox| 的周围将多出一些空间。 如果边界框缩小, i.\,e.\ 扩大一个负值, |tcolorbox| 将重叠到页面的其他部分。%
% 例如,|tcolorbox| 可能会突到页边距去。

% \begin{marker}
% % The following examples use \refKey{/tcb/show bounding box} to display the actual bounding box. For this, the library \mylib{skins} has to be included and \refKey{/tcb/enhanced} has to be set.

% 以下示例使用 \refKey{/tcb/show bounding box} 来显示实际的边界框。为此,必须包含库 \mylib{skins} 并且必须设置 \refKey{/tcb/enhanced}。
% \end{marker}


% % Shifting Bounding Box Borders
% \subsubsection{移动边界盒子的边框}

% \begin{docTcbKey}{enlarge top initially by}{=\meta{length}}{no default, initially |0mm|}
%   % Enlarges the bounding box distance to the top of the box by \meta{length}.
%   % If the box is \emph{breakable}, only the first box of the break sequence
%   % gets enlarged. \refKey{/tcb/enlarge top by} overwrites this key.

% 扩大边界盒子同 |tcolorbox| 盒子的顶部的距离 \meta{length}。
% 如果 |tcolorbox| 盒子是\emph{可分的}, 只有中断序列的第一个盒子的扩大会生效。 
% \refKey{/tcb/enlarge top by} 会覆盖这个设置。
% \begin{exdispExample}{enlarge_top_initially_by}
% \tcbset{colframe=blue!75!black,colback=white}

% \begin{tcolorbox}[enlarge top initially by=-5mm]
% This is a \textbf{tcolorbox}.
% \end{tcolorbox}
% \begin{tcolorbox}[enlarge top initially by=5mm,enhanced,show bounding box]
% This is a \textbf{tcolorbox}.
% \end{tcolorbox}
% \end{exdispExample}
% \end{docTcbKey}

% % \begin{tcolorbox}[nobeforeafter]
% % \textbf{tcolorbox} a.
% % \end{tcolorbox}
% % \begin{tcolorbox}[nobeforeafter]
% % \textbf{tcolorbox} b.
% % \end{tcolorbox}




% \begin{docTcbKey}{enlarge bottom finally by}{=\meta{length}}{no default, initially |0mm|}
%   % Enlarges the bounding box distance to the bottom of the box by \meta{length}.
%   % If the box is \emph{breakable}, only the last box of the break sequence
%   % gets enlarged. \refKey{/tcb/enlarge bottom by} overwrites this key.
% 扩大边界盒子同 |tcolorbox| 盒子的底部的距离 \meta{length}。%
% 如果盒子是 \emph{可中断的}, 只有中断序列的最后一部分得到扩大。
% \refKey{/tcb/enlarge bottom by} 会覆盖这个设置。
% \begin{exdispExample}{enlarge_bottom_finally_by}
% \tcbset{colframe=blue!75!black,colback=white}

% \begin{tcolorbox}[enlarge bottom finally by=5mm]
% This is a \textbf{tcolorbox}.
% \end{tcolorbox}
% \begin{tcolorbox}[enlarge bottom finally by=-5mm,enhanced,show bounding box]
% This is a \textbf{tcolorbox}.
% \end{tcolorbox}
% \end{exdispExample}
% \end{docTcbKey}

% % \clearpage



 
% \begin{docTcbKey}{enlarge top at break by}{=\meta{length}}{no default, initially \texttt{0mm}}
%   % Enlarges the bounding box distance to the top of the box by \meta{length},
%   % \emph{if} the box is \refKey{/tcb/breakable}.
%   % In this case, it is applied to \emph{middle} and \emph{last} parts in a
%   % break sequence.
%   % \refKey{/tcb/enlarge top by} overwrites this key.

% \emph{如果}盒子是 \refKey{/tcb/breakable}的,扩大边界盒子同 |tcolorbox| 盒子的顶部的距离 \meta{length}。这种情况下, 它在 \emph{中间}和\emph{最后}的中断序列部分生效。 \refKey{/tcb/enlarge top by} 会覆盖此项设置。
% \end{docTcbKey}


% \begin{docTcbKey}{enlarge bottom at break by}{=\meta{length}}{no default, initially \texttt{0mm}}
%   % Enlarges the bounding box distance to the bottom of the box by \meta{length},
%   % \emph{if} the box is \refKey{/tcb/breakable}.
%   % In this case, it is applied to \emph{first} and \emph{middle} parts in a
%   % break sequence. \refKey{/tcb/enlarge bottom by} overwrites this key.

% \emph{如果}盒子是 \refKey{/tcb/breakable}的,扩大边界盒子同 |tcolorbox| 盒子的底部的距离 \meta{length}。
% 这种情况下, 它在 \emph{首个}和\emph{中间}的中断序列部分生效。
% \refKey{/tcb/enlarge bottom by} 会覆盖此项设置。
% \end{docTcbKey}




% \begin{docTcbKey}{enlarge top by}{=\meta{length}}{no default, initially |0mm|}

%   % Enlarges the bounding box distance to the top of the box by \meta{length}.
%   % \refKey{/tcb/enlarge top initially by} and
%   % \refKey{/tcb/enlarge top at break by} are set to \meta{length}.

%   扩大边界盒子同 |tcolorbox| 盒子的{\bf 顶}部的距离 \meta{length}。
%   \refKey{/tcb/enlarge top initially by} 和
%   \refKey{/tcb/enlarge top at break by} 也会被设置为 \meta{length}。
% \end{docTcbKey}


% \begin{docTcbKey}{enlarge bottom by}{=\meta{length}}{no default, initially |0mm|}
%   % Enlarges the bounding box distance to the bottom of the box by \meta{length}.
%   % \refKey{/tcb/enlarge bottom finally by} and
%   % \refKey{/tcb/enlarge bottom at break by} are set to \meta{length}.

%   扩大边界盒子同 |tcolorbox| 盒子的{\bf 底}部的距离 \meta{length}。
%   \refKey{/tcb/enlarge bottom finally by} 和
%   \refKey{/tcb/enlarge bottom at break by} 也会被设为 \meta{length}.
% \end{docTcbKey}



% \begin{docTcbKey}{enlarge left by}{=\meta{length}}{no default, initially |0mm|}
%   % Enlarges the bounding box distance to the left side of the box by \meta{length}.
% 扩大边界盒子同 |tcolorbox| 盒子的{\bf 左侧}的距离 \meta{length}。
% \begin{exdispExample}[safety=2cm]{enlarge_left_by}
% \tcbset{colframe=blue!75!black,colback=white}

% \begin{tcolorbox}[enlarge left by=2cm,width=5cm,enhanced,show bounding box]
% This is a \textbf{tcolorbox}.
% \end{tcolorbox}
% \begin{tcolorbox}[enlarge left by=-2cm,width=\linewidth+2cm]
% This is a \textbf{tcolorbox}.
% \end{tcolorbox}
% \end{exdispExample}
% \end{docTcbKey}

% \begin{docTcbKey}{enlarge right by}{=\meta{length}}{no default, initially |0mm|}
%   % Enlarges the bounding box distance to the right side of the box by \meta{length}.
%   扩大边界盒子同 |tcolorbox| 盒子的{\bf 右侧}的距离 \meta{length}。
% \begin{exdispExample}[safety=2cm]{enlarge_right_by}
% \tcbset{colframe=blue!75!black,colback=white}

% \begin{tcolorbox}[enlarge right by=-2cm,width=\linewidth+2cm,
%   enhanced,show bounding box]
% \textbf{tcolorbox}缩小了同右侧的距离到负数,就突出到右边了.
% \end{tcolorbox}
% \begin{tcolorbox}[enlarge right by=2cm,width=\linewidth-2cm]
% This is a \textbf{tcolorbox}.
% \end{tcolorbox}
% \end{exdispExample}
% \end{docTcbKey}




% % \clearpage
% \begin{docTcbKey}{enlarge by}{=\meta{length}}{no default, initially |0mm|}
%   % Enlarges the bounding box distance to all sides of the box by \meta{length}.

%   扩大边界盒子同 |tcolorbox| 盒子的{\bf 四侧}的距离 \meta{length}。
% \begin{exdispExample}{enlarge_by}
% \tcbset{colframe=blue!75!black,colback=white,width=5cm,nobeforeafter}

% \begin{tcolorbox}
% This is a \textbf{tcolorbox}.
% \end{tcolorbox}
% \begin{tcolorbox}[enlarge by=5mm,enhanced,show bounding box]
% This is a \textbf{tcolorbox}.
% \end{tcolorbox}
% \end{exdispExample}
% \end{docTcbKey}





% \begin{docTcbKey}{grow to left by}{=\meta{length}}{no default, initially |0mm|}
%   % Enlarges the current box width by \meta{length} and
%   % enlarges (shrinks) the bounding box distance to the left side of the box by
%   % $-$\meta{length}. Also see \refKey{/tcb/left skip}.

% 扩大当前盒子的宽度\meta{length},并扩大边界盒子到左侧的距离为
%   $-$\meta{length}。\footnote{突出到左侧了} 另见 \refKey{/tcb/left skip}。
% \begin{exdispExample}[safety=2cm]{grow_to_left_by}
% \tcbset{colframe=blue!75!black,colback=white}

% \begin{tcolorbox}[width=5cm,grow to left by=2cm,enhanced,show bounding box]
% This is a \textbf{tcolorbox} with a width of 7cm.
% \end{tcolorbox}
% \end{exdispExample}
% \end{docTcbKey}

% \begin{docTcbKey}{grow to right by}{=\meta{length}}{no default, initially |0mm|}
%   % Enlarges the current box width by \meta{length} and
%   % enlarges (shrinks) the bounding box distance to the right side of the box by
%   % $-$\meta{length}. Also see \refKey{/tcb/right skip}.

%   扩大当前盒子的宽度\meta{length},并扩大边界盒子到右侧的距离为
%   $-$\meta{length}。\footnote{突出到右侧了} 另见 \refKey{/tcb/right skip}。
% \begin{exdispExample}[safety=2cm]{grow_to_right_by}
% \tcbset{colframe=blue!75!black,colback=white}

% \begin{tcolorbox}[grow to right by=2cm,enhanced,show bounding box]
% This is a \textbf{tcolorbox}.
% \end{tcolorbox}

% \bigskip

% \begin{tcolorbox}[grow to right by=2cm,grow to left by=1cm,
%   enhanced,show bounding box]
% This is a \textbf{tcolorbox}.
% \end{tcolorbox}
% \end{exdispExample}
% \end{docTcbKey}

% % \clearpage


% \begin{docTcbKey}[][doc new=2018-03-22]{grow sidewards by}{=\meta{length}}{no default, initially |0mm|}
%   % Shortcut for setting \refKey{/tcb/grow to left by} and \refKey{/tcb/grow to right by}
%   % to \meta{length}. Also see \refKey{/tcb/oversize} and \refKey{/tcb/spread sidewards}.

% 同时设置 \refKey{/tcb/grow to left by} 和 \refKey{/tcb/grow to right by} 到\meta{length} 的简写形式。另见 \refKey{/tcb/oversize} 和 \refKey{/tcb/spread sidewards}.
% \begin{exdispExample}[safety=2cm]{grow_sidewards_by}
% \tcbset{colframe=blue!75!black,colback=white}

% \begin{tcolorbox}[grow sidewards by=2cm,enhanced,show bounding box]
% This is a \textbf{tcolorbox}.
% \end{tcolorbox}
% \end{exdispExample}
% \end{docTcbKey}



% %  Box Alignment
% \subsubsection{盒子的对齐}

% \begin{docTcbKey}[][doc new=2015-11-20]{flush left}{}{style, no value}
%   % Enlarges the bounding box to the right side to fill the line completely.

% 左对齐效果,扩大边界盒子完全填充到右侧。
% \begin{exdispExample}{flush_left}
% \tcbset{colframe=blue!75!black,colback=white}

% \begin{tcolorbox}[flush left,width=5cm,enhanced,show bounding box]
% This is a \textbf{tcolorbox}.
% \end{tcolorbox}
% \end{exdispExample}
% \end{docTcbKey}


% \begin{docTcbKey}[][doc new=2015-11-20]{flush right}{}{style, no value}
%   % Enlarges the bounding box to the left side to fill the line completely.

%   右对齐效果,扩大边界盒子完全填充到左侧。
% \begin{exdispExample}{flush_right}
% \tcbset{colframe=blue!75!black,colback=white}

% \begin{tcolorbox}[flush right,width=5cm,enhanced,show bounding box]
% This is a \textbf{tcolorbox}.
% \end{tcolorbox}
% \end{exdispExample}
% \end{docTcbKey}


% \begin{docTcbKey}[][doc new=2015-11-20]{center}{}{style, no value}
%   % Enlarges the bounding box equally to both sides to fill the line completely.

%   居中对齐效果,扩大边界盒子完全填充到两侧。
% \begin{exdispExample}{center}
% \tcbset{colframe=blue!75!black,colback=white}

% \begin{tcolorbox}[center,width=5cm,enhanced,show bounding box]
% This is a \textbf{tcolorbox}.
% \end{tcolorbox}
% \end{exdispExample}
% \end{docTcbKey}




% % \clearpage

% \subsubsection{Toggle Enlargements}

% \begin{docTcbKey}[][doc updated=2015-11-13]{toggle enlargement}{=\meta{toggle preset}}{默认 |evenpage|(偶), initially |none|}
%   % According to the \meta{toggle preset}, the left and the right enlargements of
%   % the bounding box are switched or not. Feasible values are:

% 依据 \meta{toggle preset} 的值, 对边界盒子的左边和右边的增加空间的设置进行交换或不交换。可设的值有:
%   \begin{itemize}
%   \item\docValue{none}: %no switching.
%   不切换。
%   \item\docValue{forced}: %the values of the left and right enlargement are switched.
%   强制将边界盒子的左边和右边的增加空间的设置进行交换。
%   \item\docValue{evenpage}: 
%   % if the page is an even page, the values of the left and    right enlargement are switched. This value also sets    \refKey{/tcb/check odd page} to |true|.
% 如果当前页是偶数页, 将边界盒子的左边和右边的增加空间的设置进行交换。这项值也会将  \refKey{/tcb/check odd page} 设置为 |true|.
%   \end{itemize}
% \begin{marker}
% % See \refKey{/tcb/toggle left and right} to toggle geometry settings.
% 见 \refKey{/tcb/toggle left and right} 来切换几何设置。
% \end{marker}

% \begin{dispExample}
% \tcbset{colframe=blue!75!black,colback=white,
%   grow to left by=20mm,%突出左侧
%   grow to right by=-5mm%右侧凹着
%   }

% \begin{tcolorbox}[toggle enlargement=none
%   ,enhanced,show bounding box]
% 设置为 |toggle enlargement=none|,不切换
% \end{tcolorbox}
% \begin{tcolorbox}[toggle enlargement=forced]
%   设置为 |toggle enlargement=forced|,强制切换
% \end{tcolorbox}
% \begin{tcolorbox}[toggle enlargement=evenpage]
% 设置为 |toggle enlargement=evenpage|,偶数页才切换。当前页是 \tcbifoddpage{奇}{偶} 数页。因此, 左边的增加空间的设置 \tcbifoddpage{不会}{会}切换。
% \end{tcolorbox}
% \end{dispExample}

% \begin{dispListing}
% \begin{tcolorbox}[colframe=red!60!black,colback=red!15!white,
%   fonttitle=\bfseries,title=Floating box from \texttt{toggle enlargement},
%   width=\textwidth
% ,grow to right by=2cm%突出到右边
% ,toggle enlargement%默认是偶数页
% ,float=t]
% 当前页是\tcbifoddpage{奇}{偶}数页。%
% 因此, 左边的增加空间的设置 \tcbifoddpage{不会}{会}切换。%
% 这个盒子,在奇数页是突出到右边,在偶数页是突出到左边。%
% 本文档是one-sided文档 -- 这项特性只在two-sided%
% \footnote{译注:即双面打印,%
% 奇数和偶数页的文档内容的左右边距是不同的,以用于装订。}%
% 文档中生效。
% \end{tcolorbox}
% \end{dispListing}
% \tcbusetemp
% \end{docTcbKey}





% % \clearpage
% % Spread Box to Page Borders
% \subsubsection{盒子扩张到页面边缘}

% \begin{marker} 
% % The following border options are \emph{not} applicable to nested boxes, boxes insides tables, etc.
% % For boxes inside lists, the options \emph{may} work, but not necessarily.
% % Also, boxes should be set with |\noindent| and full width.

% 以下的 border 选项对嵌套的盒子, 在表格中的盒子\emph{不}起作用, etc.
% 对于列表中的盒子,这些选项\emph{可以}工作, 但没必要。
% 另外,盒子需要设置 |\noindent| 来达到全宽。
% \end{marker}

% \begin{docTcbKey}[][doc new=2017-02-13]{spread inwards}{\colOpt{=\meta{length}}}{default |0pt|, initially unset}
% % Enlarges the current box width to match the inner page border (left-handed side for one-sided
% % documents). If the optional \meta{length} is greater than |0pt|, the box
% % grows over the border, if \meta{length} is lower than |0pt|, there is a
% % margin between box and page border.
% % \refKey{/tcb/toggle enlargement} is set automatically.

% 扩张当前盒子的宽度到书本的内页边缘(对单面文档是在左侧).如果选项的值\meta{length}是大于 |0pt|, 那么盒子将穿过页面的边缘, 如果\meta{length}是小于|0pt|,那么在盒子和页面边缘就有一段面边空白。会自动设置 \refKey{/tcb/toggle enlargement} 。
% \begin{dispListing}
% \begin{tcolorbox}[enhanced,spread inwards,
%   colframe=blue!75!black,colback=white,show bounding box]
% 扩张当前盒子的宽度到书本的内页边缘 (对单面文档是在左侧).(|spread inwards|)
% \end{tcolorbox}

% \begin{tcolorbox}[enhanced,spread inwards=2em,
%   colframe=blue!75!black,colback=white,show bounding box]
% 前面的内容穿过边缘了(|spread inwards=2em,|)。
% \end{tcolorbox}

% \begin{tcolorbox}[enhanced,spread inwards=-2em,
%   colframe=blue!75!black,colback=white,show bounding box]
% 在盒子和页面边缘就有一段面边空白。(|spread inwards=-2em,|)。
% \end{tcolorbox}
% \end{dispListing}
% {\tcbusetemp}
% \end{docTcbKey}



% \begin{docTcbKey}[][doc new=2017-02-13]{spread outwards}{\colOpt{=\meta{length}}}{default |0pt|, initially unset}
%   % Enlarges the current box width to match the outer page border (right-handed side for one-sided
%   % documents). If the optional \meta{length} is greater than |0pt|, the box
%   % grows over the border, if \meta{length} is lower than |0pt|, there is a
%   % margin between box and page border.
%   % \refKey{/tcb/toggle enlargement} is set automatically.
  
%   扩张当前盒子的宽度到书本的内页边缘(对单面文档是在右侧)。如果选项的值\meta{length}是大于 |0pt|, 那么盒子将穿过页面的边缘, 如果\meta{length}是小于|0pt|,那么在盒子和页面边缘就有一段空白。会自动设置 \refKey{/tcb/toggle enlargement} 。
  
%   \begin{dispListing}
%   \begin{tcolorbox}[enhanced,spread outwards,
%     colframe=blue!75!black,colback=white,show bounding box]
%   This is a \textbf{tcolorbox}.
%   \end{tcolorbox}
%   \end{dispListing}
%   {\tcbusetemp}
%   \end{docTcbKey}


%   \begin{docTcbKey}[][doc new=2017-02-13]{move upwards}{\colOpt{=\meta{length}}}{default |0pt|, initially unset}
%     % Starts a new page with the box at the very top page border.
%     % If the optional \meta{length} is greater than |0pt|, the box
%     % moves over the border, if \meta{length} is lower than |0pt|, there is a
%     % margin between box and page border.
%     新起一页,将盒子放在新页面的最顶部。%
%     如果选项的值\meta{length}是大于 |0pt|, 那么盒子将穿过页面的边缘, 如果\meta{length}是小于|0pt|,那么在盒子和页面边缘就有一段空白。
%     \end{docTcbKey}
    
    
%     \begin{docTcbKey}[][doc new=2017-02-13]{move upwards*}{\colOpt{=\meta{length}}}{default |0pt|, initially unset}
%     同\refKey{/tcb/move upwards}一样,但少了新起一页的操作。
%     \end{docTcbKey}



 
% \begin{docTcbKey}[][doc new=2017-02-13]{fill downwards}{\colOpt{=\meta{length}}}{default |0pt|, initially unset}
%   % Enlarges the height of the box until the very bottom page border.
%   % The library \mylib{breakable} has to be loaded, and
%   % \refKey{/tcb/height fill} is set automatically.
%   % If the optional \meta{length} is greater than |0pt|, the box
%   % moves over the border, if \meta{length} is lower than |0pt|, there is a
%   % margin between box and page border.
%   扩张当前盒子的宽度到书本的底部边缘。%
%   需要加载 \mylib{breakable} 库, 且会自动设置\refKey{/tcb/height fill} 。%
%   如果选项的值\meta{length}是大于 |0pt|, 那么盒子将穿过页面的边缘, 如果\meta{length}是小于|0pt|,那么在盒子和页面边缘就有一段空白。
%   \begin{dispListing}
%   \begin{tcolorbox}[enhanced,fill downwards,
%     colframe=blue!75!black,colback=white,show bounding box]
%   扩张当前盒子的宽度到书本的底部边缘。
%   \end{tcolorbox}
%   \end{dispListing}
%   {\tcbusetemp}
%   \end{docTcbKey}


% \begin{tcolorbox}[enhanced,spread upwards,sharp corners=north,height=3cm,
%   colframe=blue!75!black,interior style={top color=blue!50,bottom color=white}]
% 这是\enquote{spread upwards}的例子。 
% \end{tcolorbox}
% \begin{docTcbKey}[][doc new=2017-02-13]{spread upwards}{\colOpt{=\meta{length}}}{default |0pt|, initially unset}
% 组合,同时将\meta{length}设到
% \refKey{/tcb/move upwards}, \refKey{/tcb/spread inwards}, 和 \refKey{/tcb/spread outwards}.
% \begin{dispListing}
% \begin{tcolorbox}[enhanced,spread upwards,sharp corners=north,height=3cm,
%   colframe=blue!75!black,interior style={top color=blue!50,bottom color=white}]
% 这是 \enquote{spread upwards} 的例子。
% \end{tcolorbox}
% \end{dispListing}
% \end{docTcbKey}


% \begin{docTcbKey}[][doc new=2017-02-13]{spread upwards*}{\colOpt{=\meta{length}}}{default |0pt|, initially unset}
% 同\refKey{/tcb/move upwards}一样,但不会新起一页。
% \end{docTcbKey}



 
% \begin{docTcbKey}[][doc new=2017-02-13]{spread sidewards}{\colOpt{=\meta{length}}}{default |0pt|, initially unset}
%   % Combination of \refKey{/tcb/spread inwards} and \refKey{/tcb/spread outwards}.
%   % The optional \meta{length} is used for all these keys.
%   % Also see \refKey{/tcb/oversize} and \refKey{/tcb/grow sidewards by}.
  
%   \meta{length}被同时设置到 \refKey{/tcb/spread inwards} 和 \refKey{/tcb/spread outwards}。另见 \refKey{/tcb/oversize} 和 \refKey{/tcb/grow sidewards by}.
%   \begin{dispListing}
%   \begin{tcolorbox}[enhanced,spread sidewards,
%     colframe=blue!75!black,colback=white,show bounding box]
%   向左右两侧突出了。
%   \end{tcolorbox}
%   \end{dispListing}
%   {\tcbusetemp}
%   \end{docTcbKey}
  
  
%   \begin{docTcbKey}[][doc new=2017-02-13]{spread}{\colOpt{=\meta{length}}}{default |0pt|, initially unset}
%   % Combination of
%   % \refKey{/tcb/move upwards}, \refKey{/tcb/fill downwards}, \refKey{/tcb/spread inwards},
%   % and \refKey{/tcb/spread outwards}.
%   % Such, the box fills the whole page.
%   % The optional \meta{length} is used for all these keys.
  
%   组合了 \refKey{/tcb/move upwards}, \refKey{/tcb/fill downwards}, \refKey{/tcb/spread inwards},和 \refKey{/tcb/spread outwards}。
%   这样,盒子就填满了整个页面。
%   \meta{length} 被同时设置到这些选项。
%   \end{docTcbKey}




% \begin{docTcbKey}[][doc new=2017-02-13]{spread downwards}{\colOpt{=\meta{length}}}{default |0pt|, initially unset}
%   % Combination of
%   % \refKey{/tcb/fill downwards}, \refKey{/tcb/spread inwards}, and \refKey{/tcb/spread outwards}.
%   % The optional \meta{length} is used for all these keys.
  
%   组合使用 \refKey{/tcb/fill downwards}, \refKey{/tcb/spread inwards}, 和 \refKey{/tcb/spread outwards}.
%   \meta{length} 被同时设置到这些选项。
%   \begin{dispListing}
%   \begin{tcolorbox}[enhanced,spread downwards,sharp corners=south,
%     colframe=red!75!black,interior style={top color=white,bottom color=red!50}]
%   This is an example for \enquote{spread downwards}.
%   \end{tcolorbox}
%   \end{dispListing}
%   \end{docTcbKey}
%   \begin{tcolorbox}[enhanced,spread downwards,sharp corners=south,
%     colframe=red!75!black,interior style={top color=white,bottom color=red!50}]
%   This is an example for \enquote{spread downwards}.
%   \end{tcolorbox}
  





% % \clearpage
% % Box Extrusion
% \subsubsection{挤压盒子}

% \begin{marker}
% % The following keys should not be used with breakable boxes or boxes with a
% % lower part.

% 以下选项不应在可分盒子或带有lower部分的盒子内使用。
% \end{marker}

% \begin{docTcbKey}{shrink tight}{}{style, no value, initially unset}
%   % The total colored box is shrunk to the dimensions of the upper
%   % part. There should be no lower part and no title.
%   % This style sets the \refKey{/tcb/boxsep} to |0pt| and other geometry keys
%   % to fitting values. This option is likely to be used with the following
%   % extrusion keys.

% 整个盒子缩小到upper部分的尺寸。不应有lower部分和标题。
% 此样式会将 \refKey{/tcb/boxsep} 设置为 |0pt|,以及一些其他几何设置。此选项可能与以下挤压键一起使用。
% \begin{exdispExample}{shrink_tight}
% \tcbset{colframe=blue!75!black,colback=white,arc=0mm,boxrule=0.4pt,
%         nobeforeafter,tcbox raise base,shrink tight}

% \begin{tcolorbox}
% This is a \textbf{tcolorbox}.
% \end{tcolorbox}

% Lorem \tcbox{ipsum} dolor sit amet, consectetuer adipiscing elit.
% \end{exdispExample}

% \begin{exdispExample}{shrink_tight2}
%   \tcbset{colframe=blue!75!black,colback=white,arc=0mm,boxrule=0.4pt,
%           shrink tight}
  
%   \begin{tcolorbox}
%   This is a \textbf{tcolorbox}.
%   \end{tcolorbox}
  
%   Lorem \tcbox{ipsum} dolor sit amet, consectetuer adipiscing elit.
%   \end{exdispExample}
% \end{docTcbKey}  

% % extrude
% % ① (force out) 挤出 jǐchū ‹toothpaste, glue, icing›; 压出 yāchū ‹pasta›
% % ② (shape) 压制 yāzhì ‹plastic, metal›
% \begin{docTcbKey}[][doc updated=2014-09-19]{extrude left by}{=\meta{length}}{style, no default, initially unset}
%   % The (upper part of the) colored box is extruded by the given \meta{length} to the left side.
%   % The inner width and the bounding box is kept unchanged and the operation
%   % is additive!

%   (upper部分)盒子向左挤出 \meta{length} 空间\footnote{译注:这部分有点像零宽的盒子效果。}。内部宽度和边界盒子保持不变,挤出是额外的!
% \begin{exdispExample}{extrude_left_by}
% \tcbset{enhanced,colframe=red,colback=yellow!25!white,
%   frame style={opacity=0.25},interior style={opacity=0.5},
%   nobeforeafter,tcbox raise base,shrink tight,extrude by=2mm}

% Lorem ipsum dolor sit amet, consectetuer adipiscing elit. Ut purus elit,
% vestibulum ut, placerat ac, adipiscing vitae, felis.
% \tcbox[extrude left by=1cm]{Curabitur} dictum gravida mauris.
% Nam arcu libero, nonummy eget, consectetuer id, vulputate a, magna.
% \end{exdispExample}

% \begin{exdispExample}{extrude_left_by2}
%   \tcbset{enhanced,colframe=red,colback=yellow!25!white,
%     frame style={opacity=0.25},interior style={opacity=0.5},
%     nobeforeafter,tcbox raise base,shrink tight,extrude by=2mm}
  
%   Lorem ipsum dolor sit amet, consectetuer adipiscing elit. Ut purus elit,
%   vestibulum ut, placerat ac, adipiscing vitae, felis.
%   \tcbox[extrude left by=1cm,,show bounding box]{Curabitur} dictum gravida mauris.
%   Nam arcu libero, nonummy eget, consectetuer id, vulputate a, magna.
%   \end{exdispExample}

% \end{docTcbKey}

% \begin{docTcbKey}[][doc updated=2014-09-19]{extrude right by}{=\meta{length}}{style, no default, initially unset}
%   % The (upper part of the) colored box is extruded by the given \meta{length} to the right side.
%   % The inner width and the bounding box is kept unchanged and the operation
%   % is additive!

%   (upper部分)盒子向{\bf 右}挤出 \meta{length} 空间%\footnote{译注:这部分有点像零宽的盒子效果。}
%   。内部宽度和边界盒子保持不变,挤出是额外的!
% \begin{exdispExample}{extrude_right_by}
% \tcbset{enhanced,colframe=red,colback=yellow!25!white,
%   frame style={opacity=0.25},interior style={opacity=0.5},
%   nobeforeafter,tcbox raise base,shrink tight,extrude by=2mm}

% Lorem ipsum dolor sit amet, consectetuer adipiscing elit. Ut purus elit,
% vestibulum ut, placerat ac, adipiscing vitae, felis.
% \tcbox[extrude right by=1cm]{Curabitur} dictum gravida mauris.
% Nam arcu libero, nonummy eget, consectetuer id, vulputate a, magna.
% \end{exdispExample}
% \end{docTcbKey}

% % \clearpage
% \begin{docTcbKey}{extrude top by}{=\meta{length}}{style, no default, initially unset}
%   % The (upper part of the) colored box is extruded by the given \meta{length} to the top side.
%   % The inner width and the bounding box is kept unchanged and the operation
%   % is additive!

%   (upper部分)盒子向{\bf 上}挤出 \meta{length} 空间%\footnote{译注:这部分有点像零宽的盒子效果。}
%   。内部宽度和边界盒子保持不变,挤出是额外的!
% \begin{exdispExample}{extrude_top_by}
% \tcbset{enhanced,colframe=red,colback=yellow!25!white,
%   frame style={opacity=0.25},interior style={opacity=0.5},
%   nobeforeafter,tcbox raise base,shrink tight,extrude by=2mm}

% Lorem ipsum dolor sit amet, consectetuer adipiscing elit. Ut purus elit,
% vestibulum ut, placerat ac, adipiscing vitae, felis.
% \tcbox[extrude top by=1cm]{Curabitur} dictum gravida mauris.
% Nam arcu libero, nonummy eget, consectetuer id, vulputate a, magna.
% \end{exdispExample}
% \end{docTcbKey}

% \begin{docTcbKey}{extrude bottom by}{=\meta{length}}{style, no default, initially unset}
%   % The (upper part of the) colored box is extruded by the given \meta{length} to the bottom side.
%   % The inner width and the bounding box is kept unchanged and the operation
%   % is additive!
%   (upper部分)盒子向{\bf 下}挤出 \meta{length} 空间%\footnote{译注:这部分有点像零宽的盒子效果。}
%   。内部宽度和边界盒子保持不变,挤出是额外的!
% \begin{exdispExample}[safety=1cm]{extrude_bottom_by}
% \tcbset{enhanced,colframe=red,colback=yellow!25!white,
%   frame style={opacity=0.25},interior style={opacity=0.5},
%   nobeforeafter,tcbox raise base,shrink tight,extrude by=2mm}

% Lorem ipsum dolor sit amet, consectetuer adipiscing elit. Ut purus elit,
% vestibulum ut, placerat ac, adipiscing vitae, felis.
% \tcbox[extrude bottom by=1cm]{Curabitur} dictum gravida mauris.
% Nam arcu libero, nonummy eget, consectetuer id, vulputate a, magna.
% \end{exdispExample}
% \end{docTcbKey}



% \begin{docTcbKey}{extrude by}{=\meta{length}}{style, no default, initially unset}
%   % The (upper part of the) colored box is extruded by the given \meta{length} to all sides.
%   % The inner width and the bounding box is kept unchanged and the operation
%   % is additive!

%   (upper部分)盒子向{\bf 四周}都挤出 \meta{length} 空间%\footnote{译注:这部分有点像零宽的盒子效果。}
%   。内部宽度和边界盒子保持不变,挤出是额外的!
% \begin{exdispExample}{extrude_by}
% \tcbset{enhanced,colframe=red,colback=yellow!25!white,
%   frame style={opacity=0.25},interior style={opacity=0.5},
%   nobeforeafter,tcbox raise base,shrink tight,extrude by=2mm}

% Lorem ipsum dolor sit amet, consectetuer adipiscing elit. Ut purus elit,
% vestibulum ut, placerat ac, adipiscing vitae, felis. \tcbox{Curabitur} dictum
% gravida mauris. \tcbox[colframe=Green,interior style={opacity=0.0}]{Nam}
% arcu libero, nonummy eget, consectetuer id, \tcbox{vulputate} a, magna. Donec
% vehicula augue eu neque. Pellentesque habitant morbi tristique senectus et netus
% et malesuada fames ac turpis egestas. \tcbox{Mauris ut leo.}
% \end{exdispExample}
% \end{docTcbKey}



% % \clearpage
% % Layered Boxes and Every Box Settings
% \subsection{分层盒子和所有的盒子设置}\label{subsec:everybox}
% % A |tcolorbox| may contain another |tcolorbox| and so on. The package
% % takes track of the nesting level using a counter |tcblayer|. Counter values
% % may be used for doing some fancy things, but you should never change
% % the counter value yourself.

% 一个 |tcolorbox| 盒子可能包含着另一个 |tcolorbox| 盒子,像俄罗斯套娃。本包使用计数器 |tcblayer| 记录盒子是处于嵌套的第几层。 可以用这个计数器值来做一些花哨的事情, 但你永远不应该改变这个计数器的值。

% % The package takes special care for the first four layers or nesting levels,
% % called managed layers.
% % Here, footnote texts are administrated to find their intended place
% % and specific layer dependent options may be set by changing
% % \refKey{/tcb/every box on layer n}.
% % If needed, the number of managed layers can be increased by setting
% % \refCom{tcbsetmanagedlayers} to a higher value than~4.

% %todo 再次翻译
% 该包对前四层或嵌套层会特殊处理, 称为管理层。%
% 在这些层,脚注文本 are administrated to find their 预期位置 specific layer dependent options 可以通过 更改 \refKey{/tcb/every box on layer n} 来设置。
% 如果需要,可以通过将 \refCom{tcbsetmanagedlayers} 设置为高于~4 的值来增加管理层的数量。


% % The following styles have a considerable influence on how layered boxes
% % are processed. Note especially that nested boxes are getting a
% % \refKey{/tcb/reset} by default. You can change this, but be prepared for
% % surprises if you do.

% 以下样式对多层盒子的处理方式有相当大的影响。特别注意,嵌套的盒子默认会被设置 \refKey{/tcb/reset} 。您可以更改此设置,但如果你这样做,要做好惊讶的准备。

% % If the defaults are \emph{not changed}, a |tcolorbox| gets its options
% % in the following order. Following options overwrite preceding options.

% 如果默认值\emph{没有被改变}, 一个|tcolorbox|按以下顺序获取其选项。后出现的选项会覆盖前面的选项:


% \begin{enumerate}
%   \item %On package load, all options are set to default values.
%   在包加载时,所有选项都设置为默认值。
%   \item %Every \refCom{tcbset} command adds or changes options for the following boxes inside the current \TeX\ group.
%   每个 \refCom{tcbset} 命令添加或更改当前 \TeX\ 组中后续盒子的选项。
%   \item 
%   %While entering a |tcolorbox|, a \refKey{/tcb/every box on layer n} or  \refKey{/tcb/every box on higher layers} option list is applied.  With default settings this means:
  
%   进入一个 |tcolorbox| 盒子, 会应用 \refKey{/tcb/every box on layer n} 或  \refKey{/tcb/every box on higher layers} 的选项列表。使用默认设置,这意味着:
%     \begin{itemize}
%     \item %
%     % For layer 1 (lowest layer), the \refKey{/tcb/every box} option list is applied.
%     %   Not overwritten options given by a preceding \refCom{tcbset} survive.
%   对于第 1 层(最低层), 会应用 \refKey{/tcb/every box} 的选项列表。%
%   未被 \refCom{tcbset} 覆盖的选项仍然存在。
%     \item 
%     % For layer 2 and above (nested boxes), a \refKey{/tcb/reset} followed by \refKey{/tcb/every box} option list is applied.  Every resettable options given by a preceding \refCom{tcbset} and by the sourrounding box(es) are reset.
%     % todo 重新翻译
%     对于第 2 层及以上层(嵌套盒子),在 \refKey{/tcb/every box} 选项列表之后会应用 \refKey{/tcb/reset}。 所有能被重置的,由 \refCom{tcbset} 以及外层盒子给出的选项,会被重置。
%     \end{itemize}
%   \item 
%   % The \meta{options} given to the |tcolorbox| are applied.
%   %   Or, if the box was generated by \refCom{newtcolorbox} or friends,
%   %   the \meta{options} given there are applied.
%   直接在 |tcolorbox| 环境参数中设置的 \meta{选项} 被设置。或者,如果盒子生成是由 \refCom{newtcolorbox} 或类似命令, 那边给出的 \meta{选项} 被设置。
%   \item 
%   % If the box was generated by \refCom{newtcolorbox} or friends,  some automated options are applied.
%   如果盒子生成是由 \refCom{newtcolorbox} 或类似命令, 会自动被设置一些选项。
%   \end{enumerate}
  
  
% \begin{docTcbKey}{every box}{}{style}
%   % By default, this style is empty.
%   默认情况下,此样式为空。
%   \begin{dispListing}
%   % default setting:
%   \tcbset{every box/.style={}}
%   \end{dispListing}
%   % It may be changed by redefining this style.
%   可以通过重新定义此样式来更改它。
%   \begin{dispListing}
%   % setting all boxes to be enhanced:
%   \tcbset{every box/.style={enhanced}}
%   \end{dispListing}
  
%   \medskip
%   \begin{marker}
%   % The alternative for setting something for every box (on every layer) is\\
%   为每个盒子(在每一层)设置一些东西的替代方法是\\
%    \refCom{tcbsetforeverylayer}:
%   \begin{dispListing}
%   % setting all boxes to be enhanced:
%   \tcbsetforeverylayer{enhanced}
%   \end{dispListing}
%   \end{marker}
%   \end{docTcbKey}




% % \clearpage
% \begin{docTcbKey}{every box on layer n}{}{style}
%   % Here, |n| has to be replaced by a number ranging from 1 to the highest
%   % managed layer number (4 by default).
%   在这里,|n|为从 1 到最高的托管层编号数字(默认为 4)。
%   \begin{dispListing}
%   % default settings:
%   \tcbset{
%     every box on layer 1/.style={every box},
%     every box on layer 2/.style={reset,every box},
%     every box on layer 3/.style={reset,every box},
%     every box on layer 4/.style={reset,every box},
%     }
%   \end{dispListing}
%   \end{docTcbKey}
  
  
%   \begin{docTcbKey}{every box on higher layers}{}{style}
%   % Higher layers are layers above the highest
%   % managed layer number (4 by default).
  
%   更高层是最高托管层数(默认为 4)之上的层。
%   \begin{dispListing}
%   \tcbset{every box on higher layers/.style={reset,every box}}
%   \end{dispListing}
%   \end{docTcbKey}

  


% \begin{docCommand}{tcbsetmanagedlayers}{\marg{number}}
%   % Replaces the highest managed layer number by \meta{number} where 4 is
%   % the default. This macro can only be used inside the preamble.
%   % Using a \meta{number} lower than 4 typically makes no sense, but is
%   % not forbidden.
  
%   用 \meta{number} 替换最高管理层编号,其中 4 是默认值。该宏只能在序言内使用。使用小于 4 的 \meta{number} 通常没有意义,但是不禁止。
%   \end{docCommand}
  
%   \begin{tcboutputlisting}
%   % \usepackage{lipsum}
%   % \tcbuselibrary{skins,breakable}
%   \tcbset{colframe=red!75!black,fonttitle=\bfseries,
%     colback=red!5!white,
%     every box/.style={enhanced,watermark text=\thetcblayer,
%       before=\par\smallskip,after=\par\smallskip},
%     every box on layer 2/.style={reset,every box,colback=yellow!10!white,
%       drop fuzzy shadow}}
%   \begin{tcolorbox}[enhanced jigsaw,breakable,title=第1层盒子]
%     这里有一个脚注\footnote{第1层的脚注}。
%   \lipsum[2]
%     \begin{tcolorbox}[title=第2层盒子]
%     abc\footnote{第2层的脚注}
%     \end{tcolorbox}
%     \begin{tcolorbox}[title=Another Box,ams equation]
%       \tcbhighmath{\sum\limits_{n=1}^{\infty} \frac{1}{n}} = \infty.
%     \end{tcolorbox}
%   Some text\footnote{Footnote from some text}.
%     \begin{tcolorbox}[title=Yet Another Box]%第2层
%       第2层
%       \tcboxfit[height=2cm]{\lipsum[1]}
%       \begin{tcolorbox}
%         第3层\footnote{第3层的脚注}. \lipsum[3]
%         \begin{tcolorbox}[title=Layer 4,colframe=blue,colback=white]
%           Layer 4\footnote{第4层}
%         \end{tcolorbox}
%         The End\footnote{第4层的脚注}.
%       \end{tcolorbox}
%     \end{tcolorbox}
%   \end{tcolorbox}
%   \end{tcboutputlisting}
  
%   \tcbinputlisting{base example,listing only,listing style=mydocumentation}
  
%   {\tcbuselistingtext}
  

  
% % \clearpage
% % figurative (record) 刻画 kèhuà
% % (take by force) 俘获 fúhuò ‹person›; 捕获 bǔhuò ‹animal›; 占领 zhànlǐng ‹place›
% \subsection{Capture Mode}\label{subsec:capture}

% \begin{docTcbKey}{capture}{=\meta{mode}}{no default, initially |minipage|}
%   % The capture \meta{mode} defines how the box content is processed.
% capture \meta{mode} 定义如何处理盒子内容。

  

% % Feasible values for \meta{mode} are:
% \meta{mode} 的可选值是:
% \begin{itemize}
% \item\docValue{minipage}:\\
%   % This is the default \meta{mode} for \refEnv{tcolorbox}.
%   % The content may have an upper and a lower part. 
%   %Optionally, the box
%   % can be \refKey{/tcb/breakable}. The box content is put into a
%   % minipage or into something similar to a minipage.
% 这是 \refEnv{tcolorbox} 的默认 \meta{mode} 。%
% 内容可能有一个 |upper|和一个|低|部分。%
% 可选地,盒子可以是 \refKey{/tcb/breakable} 可分的。
% 盒子内容被放入一个 |minipage| 或类似于 |minipage| 的东西。
% \item\docValue{hbox}:\\
%   % This is the default \meta{mode} for \refCom{tcbox}. The content cannot have
%   % a lower part and cannot be broken. The colored box is sized according
%   % to the dimensions of the content.
%   % A shortcut to set this mode is \refKey{/tcb/hbox}.
% 这是 \refCom{tcbox} 的默认 \meta{mode} 。%
% 内容将没有lower部分,也不可分。%
% 盒子的大小根据内容的尺寸而定。%
% 设置此模式的快捷方式是 \refKey{/tcb/hbox}.
% \item\docValue{fitbox}:%
% %  (needs the \mylib{fitting} library)\\
%  (需要启用 \mylib{fitting} 库)\\
%   % This is the default \meta{mode} for \refCom{tcboxfit}. The content cannot have
%   % a lower part and cannot be broken.
%   % The content is sized according to the dimensions of the colored box.
%   % A shortcut to set this mode is \refKey{/tcb/fit}.
% 这是 \refCom{tcboxfit} 的默认 \meta{mode}。 %
% 盒子内没有lower部分,也不可分。
% 盒子的大小根据内容的尺寸而定。%
% 设置此模式的快捷方式是 \refKey{/tcb/fit}.
% \end{itemize}

% \begin{exdispExample}{capture}
% \tcbset{colframe=blue!75!black,colback=white}

% \begin{tcolorbox}[capture=minipage]
% |capture=minipage|\\
% 这是 \refEnv{tcolorbox} 的默认 \meta{mode} 。%
% 内容可能有一个 |upper|和一个|低|部分。%
% \tcblower
% 可选地,盒子可以是 \refKey{/tcb/breakable} 可分的。
% 盒子内容被放入一个 |minipage| 或类似于 |minipage| 的东西。
% \end{tcolorbox}

% \begin{tcolorbox}[capture=hbox]
% |capture=hbox|\\
% 内容将没有lower部分,也不可分。%
% % \tcblower 使用会报错
% 盒子的大小根据内容的尺寸而定。而定而定而定而定而定%
% \end{tcolorbox}

% \begin{tcolorbox}[capture=fitbox,height=9mm]% needs the `fitting' library
% |capture=fitbox|\\
% 内容将没有lower部分,也不可分。
% %\footnote{译注,看这效果,是内容适应指定的高度等}%
% \end{tcolorbox}
% \end{exdispExample}
% \end{docTcbKey}




% \begin{docTcbKey}{hbox}{}{style, no default}
%   % Shortcut for |capture=hbox|.
% |capture=hbox| 的快捷方式。
% \begin{exdispExample}{hbox}
% \tcbset{colframe=blue!75!black,colback=white}

% \begin{tcolorbox}[hbox]
% This is a tcolorbox.
% \end{tcolorbox}
% \end{exdispExample}
% \end{docTcbKey}


% \begin{docTcbKey}{minipage}{}{style, no default}
%   % Shortcut for |capture=minipage|.
%   |capture=minipage| 的快捷方式。
% \end{docTcbKey}




% % \clearpage
% % Text Characteristics
% % ① (trait) (of person) 特征 tèzhēng ; (of place, work) 特性 tèxìng
% % ▸ a family characteristic
% % 家族特征
% % ② Mathematics [对数的] 首数 shǒushù
% % 典型的 diǎnxíng de ‹feature›; 独特的 dútè de ‹behaviour, appearance, quality›
% \subsection{Text Characteristics}
% \begin{docTcbKey}[][doc updated=2015-10-14]{parbox}{\colOpt{=true\textbar false}}{default |true|, initially |true|}
%   % The text inside a |tcolorbox| is formatted using a \LaTeX\ |minipage|
%   % if the box is unbreakable. 
%   % If breakable, the box tries a mimicry of a |minipage|. 
%   % In a |minipage| or |parbox|, paragraphs are formatted slightly different
%   % as the main text. If the key value is set to |false|, the normal main text
%   % behavior is restored. In some situations, this has some unwanted side
%   % effects. It is recommended that you use this experimental setting only
%   % where you really want to have this feature.

% 如果盒子是不可分的,|tcolorbox| 中的文本使用 \LaTeX\ |minipage| 格式化。
% 如果是可分的, 盒子试图模仿一个 |minipage|。%
% 文本在 |minipage| 和 |parbox| 中的格式处理会有略微的不同。%
% 如果这项值设为 |false|, 将恢复正常的主文本行为。%
% 在某些情况下,这会产生一些不必要的副作用。%
% 建议您只在真正希望具有此特性的地方使用此实验性设置。
% \end{docTcbKey}

% \begin{dispListing}
% % \usepackage{lipsum}  % preamble
% \tcbset{width=(\linewidth-2mm)/2,nobeforeafter,arc=1mm,
%   colframe=blue!75!black,colback=white,fonttitle=\bfseries,fontupper=\small,
%   left=2mm,right=2mm,top=1mm,bottom=1mm,equal height group=parbox}

% \begin{tcolorbox}[parbox,adjusted title={parbox=true (normal)}]
%   \lipsum[1-2]
% \end{tcolorbox}\hfill%
% \begin{tcolorbox}[parbox=false,adjusted title={parbox=false}]
%   \lipsum[1-2]
% \end{tcolorbox}%
% \end{dispListing}
% {\tcbusetemp}




% % \clearpage
% \begin{docTcbKey}{hyphenationfix}{\colOpt{=true\textbar false}}{default |true|, initially |false|}
%   % Long words at the beginning of paragraphs in very narrow boxes
%   % will not be hyphenated using |pdflatex|. This problem is circumvented by
%   % applying the |hyphenationfix| option.

% 使用|pdflatex|时,在非常狭窄的盒子中,段落开头的长单词,不会用连字符。%
% 通过应用|hyphenationfix|选项,可以规避此问题。
% \begin{exdispExample*}{hyphenationfix}{sbs,lefthand ratio=0.6}
% \tcbset{colframe=blue!75!black,
%   fontupper=\normalsize,
%   colback=blue!5!white,width=4cm}

% \begin{tcolorbox}
% Rechnungsadjunktentochter.\par
% Statthaltereikonzipist.
% \end{tcolorbox}

% \begin{tcolorbox}[hyphenationfix]
% Rechnungsadjunktentochter.\par
% Statthaltereikonzipist.
% \end{tcolorbox}
% \end{exdispExample*}

% \smallskip
% \begin{marker}
% % |parbox=false| and |hyphenationfix| should not be used together. 
% % They are targeting different box types and they do not blend very well.

% |parbox=false| 和 |hyphenationfix| 不应该一起使用。%
% 他们的目标是不同的盒子类型。%, 他们不能很好地融合。
% \end{marker}
% \end{docTcbKey}


% % Files
% \subsection{文件}
% \begin{docTcbKey}{tempfile}{=\meta{file name}}{no default, initially \cs{jobname.tcbtemp}}
%   % Sets \meta{file name} as name for the temporary file which is used inside
%   % \refEnv{tcbwritetemp} and \refCom{tcbusetemp} implicitely.
% 隐式地将 \meta{file name}  设置为在 \refEnv{tcbwritetemp} 和 \refCom{tcbusetemp} 中使用的临时文件的名称
% \end{docTcbKey}

% % applicable
% % 美 [əˈplɪkəb(ə)l]
% % 英 [əˈplɪkəb(ə)l]
% % adj.适用;合适
% % 网络可应用的;适当的;适用的
% \subsection{\texttt{\textbackslash tcbox} Specials}
% % The following options are applicable for \refCom{tcbox} and \refCom{tcboxmath}
% % only.

% 以下选项仅适用于 \refCom{tcbox} 和 \refCom{tcboxmath}。
% \begin{docTcbKey}{tcbox raise}{=\meta{length}}{no default, initially \texttt{0pt}}
%   % Raises the \refCom{tcbox} by the given \meta{length}.
%   将 \refCom{tcbox} 上移指定的高度 \meta{length}。
%   % Sets the line width of the right rule to \meta{length}.
% \begin{exdispExample}{tcbox_raise}
% \tcbset{colframe=blue!50!black,colback=white,colupper=red!50!black,
%         fonttitle=\bfseries,nobeforeafter,center title}

% Test\dotfill
% \tcbox[tcbox raise base]{Hello World 1}\dotfill
% \tcbox{Hello World 2}\dotfill
% \tcbox[tcbox raise=5mm]{上移5mm}
% \end{exdispExample}
% \end{docTcbKey}



% \begin{docTcbKey}{tcbox raise base}{}{style, no value, initially unset}
%   % Raises the \refCom{tcbox} such that the base of its content matches
%   % the base of the environmental line; see example above.

% 上移 \refCom{tcbox} ,使盒子内容的基线匹配所在环境的基线对齐;请参见上面的示例。
%   % 与环境行的基础相匹配; 
% \end{docTcbKey}

% \begin{docTcbKey}{on line}{}{style, no value, initially unset}
%   % Combines \refKey{/tcb/tcbox raise base} with \refKey{/tcb/nobeforeafter}.
%   % The resulting box behaves analogue to |\fbox|.

% %   analogue
% % 美 ['ænə.lɔɡ]
% % 英 ['ænə.lɒɡ]
% % adj.模拟的;指针式的
% % n.相似物;类似事情
% % 网络类似物;类比;同源语
% 组合 \refKey{/tcb/tcbox raise base} 和 \refKey{/tcb/nobeforeafter}.
% 得到的盒子的行为类似于|\fbox|.
% \end{docTcbKey}




% % \clearpage
% \begin{docTcbKey}[][doc new=2015-03-23]{tcbox width}{=\meta{mode}}{no default, initially \texttt{auto}}
%   % Controls how \refCom{tcbox} respects a \refKey{/tcb/width} setting.
%   % Feasible values for \meta{mode} are:
  
%   控制\refCom{tcbox}对宽度参数\refKey{/tcb/width}的处理。%
%   \meta{mode}可以选择的值有:
%   \begin{itemize}
%   \item\docValue{auto} 
%   % (initial setting):
%   %   ignore \refKey{/tcb/width} and set box width according to its content.
%   (初始设定) :
%   忽略\refKey{/tcb/width},根据盒子内容设置宽度。
%   \item\docValue{auto limited}:
%     % Set box width according to its content, if it is smaller than \refKey{/tcb/width}.
%     % Otherwise, the content is set like in a \refEnv{tcolorbox} with line breaks.
%   如果盒子内容的宽度小于\refKey{/tcb/width},则据内容设置盒子的宽度。%
%   否则,盒子内的效果类似于可换行的\refEnv{tcolorbox}。
%   \item\docValue{forced center}:
%     % Set box width according to \refKey{/tcb/width}.
%     % The content is centered and may overlap the box borders.
%   将盒子的宽度设置为\refKey{/tcb/width}。%
%   内容居中,可能与盒子两侧重叠。
%   \item\docValue{forced left}:
%     % Set box width according to \refKey{/tcb/width}.
%     % The content is left aligned and may overlap the box borders.
%   将盒子的宽度设置为\refKey{/tcb/width}。%
%   内容居{\bf 左},可能与盒子两侧重叠。
%   \item\docValue{forced right}:
%     % Set box width according to \refKey{/tcb/width}.
%     % The content is right aligned and may overlap the box borders.
%   将盒子的宽度设置为\refKey{/tcb/width}。%
%   内容居{\bf 右},可能与盒子两侧重叠。
%   \item\docValue{minimum center}:
%     % Set box width according to \refKey{/tcb/width}, if the content fits into.
%     % The content is centered and the box width may grow beyond \refKey{/tcb/width}.
%   如果内容合适,将盒子的宽度设置为\refKey{/tcb/width}。%
%   内容是居中的,盒宽可能超出\refKey{/tcb/width}。
%   \item\docValue{minimum left}:
%     % Set box width according to \refKey{/tcb/width}, if the content fits into.
%     % The content is left aligned and the box width may grow beyond \refKey{/tcb/width}.
%   如果内容合适,将盒子的宽度设置为\refKey{/tcb/width}。%
%   内容是居{\bf 左}的,盒宽可能超出\refKey{/tcb/width}。
%   \item\docValue{minimum right}:
%   如果内容合适,将盒子的宽度设置为\refKey{/tcb/width}。%
%   内容是居{\bf 右}的,盒宽可能超出\refKey{/tcb/width}。
%   \end{itemize}

  
% % \enlargethispage*{1cm}

% \begin{exdispExample}{tcbox_width}
%   \tcbset{size=small,on line,before upper=\strut,
%     colframe=blue!75!black,colback=blue!5!white,
%     fontupper=\normalsize,width=4cm}
  
%   \tcbox[tcbox width=auto]{auto}\qquad
%   \tcbox[tcbox width=auto limited]{auto limited}\qquad
%   \tcbox[tcbox width=auto limited]{auto limited遇上长文本}\\
%   \tcbox[tcbox width=forced center]{forced center}\qquad
%   \tcbox[tcbox width=forced center]{forced center with long text}\\
%   \tcbox[tcbox width=forced left]{forced left}\qquad
%   \tcbox[tcbox width=forced left]{forced left with long text}\\
%   \tcbox[tcbox width=forced right]{forced right}\qquad
%   \tcbox[tcbox width=forced right]{forced right with long text}\\
%   \tcbox[tcbox width=minimum center]{minimum center}\qquad
%   \tcbox[tcbox width=minimum center]{minimum center with long text}\\
%   \tcbox[tcbox width=minimum left]{minimum left}\qquad
%   \tcbox[tcbox width=minimum left]{minimum left with long text}\\
%   \tcbox[tcbox width=minimum right]{minimum right}\qquad
%   \tcbox[tcbox width=minimum right]{minimum right with long text}
%   \end{exdispExample}
%   \end{docTcbKey}



% %\subsection{Skins}
% %There are additional option keys which change the appearance of a |tcolorbox|.
% %If only the core package is used, there is only one \emph{skin} and these
% %keys are meaningless.
% %The library \mylib{skins} adds more skins. The appropriate option keys for skins of
% %the core package are therefore described in \Vref{sec:skincorekeys} from
% %page \pageref{sec:skincorekeys}.

% % \clearpage
% % Counters, Labels, and References
% \subsection{计数器、标签和引用}

% \begin{docTcbKey}{phantom}{=\meta{code}}{no default, initially unset}
% % The \meta{code} is put in a box at the upper left corner of the |tcolorbox|.
% % If the |tcolorbox| is breakable, the \meta{code} is executed for the first box of
% % the break sequence only. If there already was some phantom code given, the
% % new \meta{code} is appended.\par
% % The \meta{code} is intended to be used for counter stepping, labelling, and
% % related operations which do not produce visible text.
% \meta{code}被放在|tcolorbox|的左上角的盒子中。%
% 如果 |tcolorbox| 是可分的, \meta{code} 将只会在中断序列的第一部分执行。%
% 如果已经给出了一些phantom代码,新的\meta{code}被追加过去。\par
% \meta{code}旨在用于计数器步进, 标签和一些不会产生可见内容的相关操作。
% \begin{itemize}
% \item 
% % The \meta{code} is executed before the title and box content, i.\,e.\ counter
% %   values are ensured to be increased before usage.
% \meta{code}在标题和盒子内容之前执行, i.\,e.\ 确保计数器的值在使用前增加了。
% \item %Labels are ensured to reference the correct page number.
% 确保标签引用正确的页码。
% \item 
% % The \meta{code} is executed only once even during fitting operations for
% %   title and box content.
% \meta{code}只执行一次,即使是在标题和内容的自适应过程中。
% \item 
% % In combination with the |hyperref| package, the hyper anchor is set
% %   to the upper left corner of the |tcolorbox|, i.\,e.\ 
% % links inside the pdf document   will jump to the box pleasantly.
% %todo 再翻译
% 结合|hyperref|包,超锚被设置为|tcolorbox|的左上角, i.\,e.\ PDF文档中的链接将友好地跳转到相应盒子。

% \item 
% % Since the \meta{code} is executed inside a \TeX\ group, only global
% %   operations can survive this group.
% 由于\meta{code}是在\TeX\ 组中执行的,因此仅是全局的在这个群体中,操作可以存活下来。
% \end{itemize}
% % Examples for the |phantom| usage are given in Section \ref{listing:exercises}
% % from page \pageref{listing:exercises}, e.\,g.\
% % Example \ref{exe:tabular_example} on page \pageref{exe:tabular_example}.
% |phantom| 的使用示例见\pageref{listing:exercises}页的\ref{listing:exercises}小节
% , e.\,g.\ 第 \pageref{exe:tabular_example} 页的 \ref{exe:tabular_example}。
% \end{docTcbKey}


% \begin{docTcbKey}{nophantom}{}{no value, initially set}
% % Removes the phantom code if set before.
% 删除之前设置的phantom代码。
% \end{docTcbKey}
  

% \begin{docTcbKey}{label}{=\meta{marker}}{no default, initially unset}
%   % The \meta{marker} is set as label text for a reference with the |\ref| macro.
%   % Typically, this option is used for numbered boxes, see Subsection \ref{sec:numberedboxes}
%   % from page \pageref{sec:numberedboxes}, e.\,g.\ \refKey{/tcb/new/auto counter}.
  
%   \meta{marker}被设置为|\ref|宏引用的标签文本。%
%   通常,这个选项用于编号的盒子,参见,\pageref{sec:numberedboxes}页的 \ref{sec:numberedboxes} 小节%
%    , e.\,g.\ \refKey{/tcb/new/auto counter}.
%   \end{docTcbKey}
  
%   \begin{docTcbKey}[][doc new=2014-11-28]{phantomlabel}{=\meta{marker}}{no default, initially unset}
%   % Equivalent to \refKey{/tcb/label} for an \emph{unnumbered} box.
%   % A |\phantomsection| from the package |hyperref| \cite{rahtz:hyperref} is used to set a correct
%   % hyperlink target. This is not needed for a numbered box.
%   等效于\emph{未编号}的盒子的\refKey{/tcb/label}。%
%   包|hyperref|中的|\phantomsection|用于设置正确的超链接目标。%
%   对于有编号的盒子,这是不需要的。
%   \end{docTcbKey}



% \begin{docTcbKey}{label type}{=\meta{type}}{no default, initially unset}
%   % This option key can be used only in conjunction with the |cleveref| package
%   % \cite{cubitt:2018a} which has to be loaded separately.
%   % \meta{type} has to be a cross-reference type \emph{known} to |cleveref|
%   % like |theorem|, |algorithm|, |result|, etc. References made with |cleveref|
%   % will use this type. Note that using |label type| will result in compilation
%   % errors, if |cleveref| is not loaded.
%   % For an example, see \Vref{theo:meanvaluetheorem}.
  
%   此选项键只能与|cleveref|包一起使用,|cleveref|包必须单独加载。%
%   \meta{type}必须是|cleveref|的交叉引用类型,如 |theorem|, |algorithm|, |result|, 等。%
%   使用|cleveref|所做的引用将使用此类型。%
%   注意的是,如果|cleveref|未加载, 使用 |label type| 将导致编译错误。例子见 \Vref{theo:meanvaluetheorem}。
%   \end{docTcbKey}
  
%   \begin{docTcbKey}{no label type}{}{no value, initially set}
%   % Removes a \refKey{/tcb/label type}, if set before.
%   删除\refKey{/tcb/label type},如果之前有设置过。
%   \end{docTcbKey}
  
%   \begin{docTcbKey}{step}{=\meta{counter}}{no default, initially unset}
%   % Shortcut for |phantom={\refstepcounter{#1}}|. The given \meta{counter} is
%   % increased and ready for labelling. This option is not needed when
%   % using the convenient automated numbering introduced with version 2.40,
%   % see Subsection \ref{sec:numberedboxes}
%   % from page \pageref{sec:numberedboxes}.
%   |phantom={\refstepcounter{#1}}|的快捷方式。%
%   给定的计数器\meta{counter}被增加并准备好标记。%
%   当使用2.40版本引入的,方便的,自动编号时,不需要这个选项,%
%   见\pageref{sec:numberedboxes}页的\ref{sec:numberedboxes}小节。
%   \end{docTcbKey}



% \begin{docTcbKey}{step and label}{=\marg{counter}\marg{marker}}{no default, initially unset}
%   % Shortcut for using \refKey{/tcb/step} and \refKey{/tcb/label}. This option is not needed when
%   % using the convenient automated numbering introduced with version 2.40,
%   % see Subsection \ref{sec:numberedboxes}
%   % from page \pageref{sec:numberedboxes}.
%   使用\refKey{/tcb/step}和\refKey{/tcb/label}的快捷方式。%
%   当使用2.40版本引入的方便的自动编号时,不需要这个选项,%
%   参见\pageref{sec:numberedboxes}页的\ref{sec:numberedboxes}小节。
%   \end{docTcbKey}
  

% % \clearpage
% \begin{docTcbKey}{list entry}{=\meta{text}}{no default, initially unset}
%   % If the \flqq list of tcolorbox(es)\frqq\ feature described in Subsection
%   % \ref{sec:listsof} from page \pageref{sec:listsof} is used, this key
%   % describes the \meta{text} for an entry into the generated list, e.\,g.
%   如果使用了,在\pageref{sec:listsof}页的\ref{sec:listsof}小节描述的 \flqq tcolorbox(es)列表\frqq\ 特性, 
%   这个选项描述了生成列表中条目的\meta{text}, e.\,g.
%   \begin{dispListing}
%   list entry={\protect\numberline{\thetcbcounter}My beautiful Example}
%   \end{dispListing}
%   完整的例子见 \pageref{listing:exercises} 页的 \ref{listing:exercises} 小节。
%   \end{docTcbKey}

  
% \begin{docTcbKey}[][doc new=2014-09-19]{list text}{=\meta{text}}{style, no default}
%   % This is a shortcut for setting \refKey{/tcb/list entry} to\\
%   % |\protect\numberline{\thetcbcounter}|\meta{text}.
%   % So, the following settings are identical:
%   这是将 \refKey{/tcb/list entry} 设为\\
%   |\protect\numberline{\thetcbcounter}|\meta{text} 的快捷方式。
%   因此,以下设置是相同的:
%   \begin{dispListing}
%   list text={My beautiful Example},
%   list entry={\protect\numberline{\thetcbcounter}My beautiful Example}
%   \end{dispListing}
%   % See Section \ref{listing:exercises} from page \pageref{listing:exercises}
%   % for a complete example.
%   完整的例子见 \pageref{listing:exercises} 页的 \ref{listing:exercises} 小节。
%   \end{docTcbKey}



% \begin{docTcbKey}{add to list}{=\marg{list}\marg{type}}{no default, initially unset}
%   % If the \flqq list of tcolorbox(es)\frqq\ feature described in Subsection
%   % \ref{sec:listsof} from page \pageref{sec:listsof} is used, list entries are
%   % generated automatically. With this key, you can enforce an entry to the
%   % given \meta{list} with the given \meta{type}.
%   % This issues:\\
%   % |\addcontentsline|\marg{list}\marg{type}\marg{entry text}
%   如果使用了,\pageref{sec:listsof}页的\ref{sec:listsof}小节描述的\flqq list of tcolorbox(es)\frqq\ 功能, 列表项会自动生成。使用此键,您可以使用给定的\meta{type}将一个条目强制到给定的\meta{list}。
%   This issues:\\
%   |\addcontentsline|\marg{list}\marg{type}\marg{entry text}
%   \end{docTcbKey}




% \begin{docTcbKey}[][doc new and updated={2016-06-22}{2016-11-18}]{nameref}{=\meta{text}}{no default, initially unset}
%   % If the |nameref| package is loaded, the given \meta{text} is used for
%   % corresponding |\nameref| macros. Typically, the \meta{text} will be chosen
%   % to be identical or nearly identical to the one for \refKey{/tcb/title}.
  
%   如果加载了|nameref|包,%
%   给定的\meta{text}作为|\nameref|宏的参数。%
%   通常,\meta{text}将被选择为与\refKey{/tcb/title}相同或几乎相同。
  
%   \inputpreamblelisting{A}


% \begin{dispExample}
%   \begin{pabox}[label={mynamelabel},nameref={Title or anything else}]{Title text}
%   This is a tcolorbox.
%   \end{pabox}
%   This box is automatically numbered with \ref{mynamelabel} on page
%   \pageref{mynamelabel}.
  
%   The box is titled \enquote{\nameref{mynamelabel}}.
%   \end{dispExample}
  
%   \begin{marker}
%   % \refKey{/tcb/nameref} is used automatically inside \refCom{newtcbtheorem}.
%   \refKey{/tcb/nameref}在\refCom{newtcbtheorem}中自动使用。
%   \end{marker}
  
%   \end{docTcbKey}




% % \clearpage
% \begin{docTcbKey}[][doc new=2017-02-03]{hypertarget}{=\meta{marker}}{no default, initially unset}
%   % A |\hypertarget| from the package |hyperref| \cite{rahtz:hyperref} is used to
%   % create an internal link of an anchor \meta{marker}.
%   % This \meta{marker} can be referenced by |\hyperlink| or
%   % \refKey{/tcb/hyperlink}.
%   包|hyperref|中的|\hypertarget|用于创建一个锚\meta{marker}的内部链接。
%   这个\meta{marker}可以通过|\hyperlink|或\refKey{/tcb/hyperlink}链接引用到。
  
%     \begin{dispExample*}{sbs,lefthand ratio=0.7}
%   % \usepackage{hyperref}%
%   \begin{tcolorbox}[enhanced,
%     colback=red!10,colframe=red!50!black,
%     hypertarget=hypertwinA,
%     hyperlink=hypertwinB,
%     title=Box A]
%   Click me to jump to Box B.
%   \end{tcolorbox}
%     \end{dispExample*}
%   \end{docTcbKey}

%   \begin{docTcbKey}[][doc new=2017-02-10]{bookmark}{=\meta{text}}{no default, initially unset}
%     % Sets a PDF bookmark with the given \meta{text}, if the package |bookmark| \cite{oberdiek:bookmark}
%     % is loaded. This bookmark is set with an automated destination (the current box)
%     % and is set one level below the current bookmark level.
    
%   % Sets a PDF bookmark with the given \meta{text}, if the package |bookmark| is loaded. 
%   如果加载了包|bookmark|,则使用给定的\meta{text}设置PDF书签。%
%   此书签使用自动目标(当前盒子)设置,并设置在当前书签级别以下一级。%
%     \begin{dispExample*}{sbs,lefthand ratio=0.7}
%   % \usepackage{bookmark}%
%   \begin{tcolorbox}[colback=blue!10,colframe=blue!50!black,
%     bookmark=Example for using a bookmark,
%     title=Example for using a bookmark]
%   Open the bookmark view of the previewer
%   to see the bookmark.
%   \end{tcolorbox}
%     \end{dispExample*}
%   \end{docTcbKey}
  
  
%   \begin{docTcbKey}[][doc new=2017-02-10]{bookmark*}{=\marg{options}\marg{text}}{no default, initially unset}
%     % Identical to \refKey{/tcb/bookmark}, but additional \meta{options}
%     % from the package |bookmark| \cite{oberdiek:bookmark} can be given.
%     与\refKey{/tcb/bookmark}相同,但可以从包|bookmark|中给出额外的\meta{options}。
%     \begin{dispExample*}{sbs,lefthand ratio=0.7}
%   % \usepackage{bookmark}%
%   \begin{tcolorbox}[colback=red!10,colframe=red!50!black,
%     bookmark*={color=red,italic,bold}%
%               {Another bookmark example},
%     title=Red and bold bookmark]
%   Open the bookmark view of the previewer
%   to see the bookmark.
%   \end{tcolorbox}
%     \end{dispExample*}
%   \end{docTcbKey}

%   \begin{docTcbKey}[][doc new=2018-07-26]{index}{=\meta{entry}}{no default, initially unset}
%     % Adds an index \meta{entry} for the box. This is a shortcut for
%     % setting |\index|\marg{entry} to \refKey{/tcb/phantom}.
  
%   为盒子添加索引\meta{entry}。 这是一个将|\index|\marg{entry} 设置为 \refKey{/tcb/phantom}的快捷方式。
%   \end{docTcbKey}
  
%   \begin{docTcbKey}[][doc new=2018-07-26]{index*}{=\marg{name}\marg{entry}}{no default, initially unset}
%     % Adds an \meta{entry} to an index with a specific \meta{name}.
%     % This is a shortcut for  setting |\index|\oarg{name}\marg{entry} to \refKey{/tcb/phantom}.
%     % An index extension package like |imakeidx| has to be loaded to use  this option key.
  
%     % Adds an \meta{entry} to an index with a specific \meta{name}.
%     将\meta{entry}添加到具有特定\meta{name}的索引中。
%     这是一个将|\index|\oarg{name}\marg{entry}设置为\refKey{/tcb/phantom}的快捷方式。
%     必须加载像|imakeidx|这样的索引扩展包才能使用此选项键。
%   \end{docTcbKey}
  
%   % \clearpage
% % Even and Odd Pages
% \subsection{偶数页和奇数页}

% \begin{marker}
% % Also see \refKey{/tcb/toggle left and right} and \refKey{/tcb/toggle enlargement} for further even/odd options.
% 也可以参考\refKey{/tcb/toggle left and right}和\refKey{/tcb/toggle enlargement}了解更多的偶数/奇数选项。
% \end{marker}

% \begin{docTcbKey}[][doc updated=2015-11-13]{check odd page}{\colOpt{=true\textbar false}}{default |true|, initially |false|}
% % If set to |true|, a precise even/odd page testing for the current box is applied. 
% % This is done by using labels. If a box moves to another page,
% % the document has to be compiled twice for the correct settings.
% % If set to |false|, even/odd page tests may give wrong results for the first box of a page.


% 如果设置为|true|,则对当前盒子应用精确的偶数/奇数页测试。%
% 这是通过使用标签来实现的。%
% 如果一个盒子移动到另一页,%
% 为了获得正确的设置,必须编译两次文档。%
% 如果设置为|false|,偶数/奇数页测试可能会对页面的第一个盒子给出错误的结果。

% % \refKey{/tcb/toggle left and right},
% % \refKey{/tcb/toggle enlargement}, and
% % \refKey{/tcb/if odd page}
% % automatically set |check odd page|, but for
% % \refCom{tcbifoddpage} this option has to be set explicitely.
% \refKey{/tcb/toggle left and right},
% \refKey{/tcb/toggle enlargement}, 和
% \refKey{/tcb/if odd page}
% 自动设置|check odd page|, 但是对于\refCom{tcbifoddpage},这个选项必须显式设置。
% \end{docTcbKey}


% % \enlargethispage*{1cm}
% \begin{docTcbKey}[][doc new=2015-11-13]{if odd page}{=\marg{odd options}\marg{even options}}{style, no default}
%   % If the current box is on an odd page, the \meta{odd options} are applied.
%   % On an even page, the \meta{even options} are applied.
%   % \refKey{/tcb/check odd page} is automatically set for precise even/odd page testing.
  
%   如果盒子当前位于奇数页,则应用\meta{odd options}。%
%   如果是偶数页, 则应用\meta{even options}。%
%   \refKey{/tcb/check odd page}会自动设置,用于精确的偶数/奇数页测试。
  
%   \begin{dispExample}
%   \begin{tcolorbox}[if odd page={colback=yellow!50}{colback=red!50}]
%   这个盒子在奇数页上是黄色的,在偶数页上是红色的。
%   \end{tcolorbox}
%   \end{dispExample}
  
%   \begin{marker}
%   % If a box is \refKey{/tcb/breakable}, using \refKey{/tcb/if odd page}
%   % only acts upon the \emph{first} box. If the setting should be
%   % repeated for every partial box of the break sequence, the option should be
%   % packed into \refKey{/tcb/extras}. In this case, \refKey{/tcb/check odd page}
%   % has to be set explicitely! Also see \refKey{/tcb/if odd page*}.
  
%   如果盒子设置了 \refKey{/tcb/breakable}, 则 \refKey{/tcb/if odd page} 只在\emph{第一}部分盒子生效。%
%   如果对中断序列的每个部分框都要重复设置,则选项应设到\refKey{/tcb/extras}。%
%   在这种情况下,\refKey{/tcb/check odd page}必须显式设置! 另见 \refKey{/tcb/if odd page*}.
%   \end{marker}
%   \end{docTcbKey}

  


% \begin{docTcbKey}[][doc new=2016-11-18]{if odd page or oneside}{=\marg{odd options}\marg{even options}}{style, no default}
%   % For onesided documents, the \meta{odd options} are applied always.
%   % For twosided documents, this style is identical to \refKey{/tcb/if odd page}.

% 对于单开文档,总是应用\meta{odd选项}。%
% 对于双面文档,此样式与\refKey{/tcb/if odd page}相同。
% \end{docTcbKey}

% % \clearpage
% \begin{docTcbKey}[][doc new=2015-11-13]{if odd page*}{=\marg{odd options}\marg{even options}}{style, no default}
%   \begin{marker}
%   % This option needs the \mylib{breakable} library, see \Fullref{sec:breakable}.
%   这个选项需要\mylib{breakable}库,参见\Fullref{sec:breakable}。
%   \end{marker}
%   % For breakable boxes, if the current partial box is on an odd page, the \meta{odd options} are applied.
%   % On an even page, the \meta{even options} are applied.
%   % \refKey{/tcb/check odd page} is automatically set for precise even/odd page testing.
  
%   对于可分的盒子,如果当前盒子部分位于奇数页,则应用\meta{odd选项}。%
%   在偶数页上,应用\meta{even选项}。%
%   \refKey{/tcb/check odd page}会被自动设置,以用于精确的偶数/奇数页测试。
  
%   % In contrast to \refKey{/tcb/if odd page}, \refKey{/tcb/if odd page*} is used
%   % on \emph{every} partial box of a break sequences and not only on the
%   % \emph{first} box. Another difference is that \refKey{/tcb/if odd page*}
%   % is applied quite \emph{late} during option processing, while
%   % \refKey{/tcb/if odd page} is applied immediately.
%   同\refKey{/tcb/if odd page}对比, \refKey{/tcb/if odd page*} 作用于中断序列的每一个部分的盒子,而不仅仅是对
%   第一个盒子。%
%   另一个区别是, \refKey{/tcb/if odd page*} 在选项处理过程中应用较晚,而 \refKey{/tcb/if odd page} 立即生效。
  
%   % \refKey{/tcb/if odd page*} is implemented as \refKey{/tcb/if odd page}
%   % packed into \refKey{/tcb/extras}.
  
%   \refKey{/tcb/if odd page*} 是在 \refKey{/tcb/extras} 中的一个 \refKey{/tcb/if odd page} 的实现。
  
%   \begin{dispExample}
%   % \tcbuselibrary{breakable}
%   \begin{tcolorbox}[breakable,if odd page*={colback=yellow!50}{colback=red!50}]
%   这个可中断盒子在奇数页上是黄色的,在偶数页上是红色的。%
%   对于每一个部分盒子,测试都会重复执行, i.e. 对于长文本,这会得到黄色,红色,黄色,红色, \ldots\ 这样的序列。
%   \end{tcolorbox}
%   \end{dispExample}
%   \end{docTcbKey}

  


% \begin{docTcbKey}[][doc new=2016-11-18]{if odd page or oneside*}{=\marg{odd options}\marg{even options}}{style, no default}
%   % For onesided documents, the \meta{odd options} are applied always.
%   % For twosided documents, this style is identical to \refKey{/tcb/if odd page*}.

% 对于单开页文档,总是应用\meta{odd选项}。%
% 对于双开页文档,此项与\refKey{/tcb/if odd page*}相同。
% \end{docTcbKey}

  

% % \clearpage
% \begin{docCommand}[doc new=2015-11-13]{tcbifoddpage}{\marg{odd code}\marg{even code}}
%   % If the current box is on an odd page, the \meta{odd code} is executed.
%   % On an even page, the \meta{even code} is executed.
%   % For precise even/odd page testing, the \refKey{/tcb/check odd page} has to be
%   % set manually inside the box options.
  
%   如果当前盒子位于奇数页上,则执行\meta{odd code}。%
%   在偶数页上,执行\meta{even code}。%
%   对于精确的偶数/奇数页测试,\refKey{/tcb/check odd page}必须在盒子的选项中手动设置。
  
%   % The macro \refCom{tcbifoddpage} can be used inside underlay, overlay, or watermark code to
%   % test if the box is on an odd page. This will work also for boxes in a break sequence.
%   宏\refCom{tcbifoddpage}可以在底层、覆盖或水印代码中使用,以测试框是否在奇数页上。这也适用于中断序列中的盒子。
  
%   % The macro can also be used inside the box \textbf{content text}. For unbreakable boxes,
%   % the correct page test is applied.
%   % But for \refKey{/tcb/breakable} boxes, \refCom{tcbifoddpage}
%   % will always give the result for the page of the \emph{first} box inside
%   % the box \textbf{content text}. If needed, the methods from the packages
%   % |changepage| or |ifoddpage| could be used here.
%   宏也可以在盒子的\textbf{内容文本}内使用。对于不可分的盒子,应用正确的页面测试。%
%   但是对于\refKey{/tcb/breakable}的盒子, \refCom{tcbifodpage}将始终给出盒子的内容文本的第一部分所在页面的结果。 如果需要,这里可以使用包|changepage|或|ifoddpage|中的方法。
%   %To mention it again, for overlays, watermarks, etc, \refCom{tcbifoddpage} gives
%   %the correct page test.
  
%   \begin{dispExample}
%   \tcbset{colframe=blue!75!black,colback=white,fonttitle=\bfseries}
  
%   \begin{tcolorbox}[enhanced,check odd page,
%     title={Example for a box on an \tcbifoddpage{odd}{even} page},
%     watermark text={\tcbifoddpage{Odd}{Even} page!}]
%   \lipsum[1]
%   \end{tcolorbox}
%   \end{dispExample}
%   \end{docCommand}

  


% \begin{docCommand}[doc new=2016-11-18]{tcbifoddpageoroneside}{\marg{odd code}\marg{even code}}
%   % For onesided documents, the \meta{odd code} is executed always.
%   % For twosided documents, this macro is identical to \refCom{tcbifoddpage}.
% 对于单开页文档,总是执行\meta{odd code}。%
% 对于双开页文档,这个宏与\refCom{tcbifoddpage}相同。
% \end{docCommand}


% % \clearpage
% \begin{docCommand}[doc new=2015-11-13]{thetcolorboxnumber}{}
%   % This is a unique identifier (arabic number) for a tcolorbox. It is locally
%   % defined inside boxes and has no meaning outside. It is used for
%   % precise even/odd page testing, but may also be valuable for elaborate user
%   % code.
  
%   这是tcolorbox盒子的唯一标识符(阿拉伯数字)。 %
%   它在盒子内局部定义,在盒子外没有意义。%
%   它用于精确的偶数/奇数页测试,但对于复杂的用户代码也很有价值。
  
%   \begin{dispExample}
%   \begin{tcolorbox}[colback=yellow!5,title=Box \thetcolorboxnumber]
%   本盒号\thetcolorboxnumber.
%     \tcbox[on line,size=fbox]{本盒号\thetcolorboxnumber} 然后
%     \tcbox[on line,size=fbox]{本盒号\thetcolorboxnumber}.
%     本盒号 \thetcolorboxnumber.
%   \end{tcolorbox}
%   \end{dispExample}
%   \end{docCommand}

  

% \begin{docCommand}[doc new=2015-11-13]{thetcolorboxpage}{}
%   % This macro contains the expanded arabic page number of the current tcolorbox.
%   % It is locally defined inside boxes and has no meaning outside.
%   % It is precise only, if \refKey{/tcb/check odd page} was set.
  
%   这个宏包含当前tcolorbox盒子所在页的page计数器的阿拉伯数字页码。%
%   它在盒子内局部定义,在盒子外没有意义。%
%   只有当\refKey{/tcb/check odd page}被设置时,它才是精确的。
  
%   \begin{dispExample}
%   \begin{tcolorbox}[colback=yellow!5,check odd page,
%       title=Box on page~\thetcolorboxpage]
%   这个盒子位于~\thetcolorboxpage~页。
%   \end{tcolorbox}
%   \end{dispExample}
%   \end{docCommand}
  

%   % \clearpage
% % Externalization
% \subsection{Externalization}
% \begin{marker}
% % See \Fullref{sec:external} for the \mylib{external} library of |tcolorbox|.
% |tcolorbox|的\mylib{external}库请参见\Fullref{sec:external}。
% \end{marker}

% % If the \emph{externalization} library of the \texttt{tikz} package is used
% % and \refKey{/tcb/graphical environment} is set to |tikzpicture|,
% % a |tcolorbox| could trigger the externalization process which will arise
% % a compilation error.

% 如果使用了\texttt{tikz}包的\emph{externalization}库,%
% 且 \refKey{/tcb/graphical environment}被设置为|tikzpicture|,%
% 一个|tcolorbox|盒子可能触发外部进程,从而就可能产生编译错误。


% % To avoid this, there are two possible strategies:
% 为了避免这种情况,有两种可能的策略:
% \begin{itemize}
% \item 
% % Ensure, that |\tikzexternaldisable| is set before a |tcolorbox| is used.
% %   If you typically use the pattern |\tikzexternalenable| \textit{some picture} |\tikzexternaldisable|,
% %   there is nothing to care about.
% 确保在使用|tcolorbox|之前设置|\tikzexternaldisable|。%
% 如果你经常使用这种 |\tikzexternalenable| \textit{some picture} |\tikzexternaldisable| 模式,
% 没什么好担心的。
% \item 
% % If \emph{externalization} is enabled globally, use \refKey{/tcb/shield externalize} to
% %   shield any |tcolorbox|. The preamble code could look like this:
% 如果\emph{externalization}全局启用, 使用\refKey{/tcb/shield externalize}来保护|tcolorbox|。%
% 导言代码可以是这样的:
% \begin{dispListing}
% \usetikzlibrary{external}
% \tikzexternalize
% \tcbset{shield externalize}
% \end{dispListing}
% \end{itemize}

% \begin{docTcbKey}{shield externalize}{\colOpt{=true\textbar false}}{default |true|, initially |false|}
%   % If set to |true|, the drawing part of the |tcolorbox| is not being externalized
%   % which is a good thing at the current state of art. Nevertheless, if the
%   % |tcolorbox| contains a |tikzpicture|, this picture is still externalized.
%   % Pictures drawn with help of \refKey{/tcb/tikz upper} or alike are \emph{not}
%   % externalized.
  
%   如果设置为|true|, |tcolorbox|的绘图部分不会外部化,就目前的技术水平而言,这是一件好事。%
%   尽管如此,如果|tcolorbox|包含|tikzpicture|,这张图片仍然是外部化的。%
%   借助\refKey{/tcb/tikz upper}或类似工具绘制的图片是不外化的。
%   \end{docTcbKey}
  
%   \begin{marker}
%   % If a |tcolorbox| is used inside a node of an encircling |tikzpicture| which is externalized,
%   % do \emph{not} use |\tikzexternaldisable| in front of the |tcolorbox|.
%   % \refKey{/tcb/shield externalize} is deactivated automatically inside a |tikzpicture|.
%   如果|tcolorbox|在外部化的,被|tikzpicture|包围的节点内使用,%
%   在 |tcolorbox| 之前不要使用 |\tikzexternaldisable| 。%
%   \refKey{/tcb/shield externalize}在|tikzpicture|中自动停用。
%   \end{marker}

  

% \begin{marker}
%   % \refKey{/tcb/shield externalize} is applied for every following |tcolorbox|
%   % inside the current \TeX\ group and is not affected by \refKey{/tcb/reset}.
%   \refKey{/tcb/shield externalize}应用于当前\TeX\ 组内的每个|tcolorbox|,并且不受\refKey{/tcb/reset}的影响。
%   \end{marker}
  
%   \begin{docTcbKey}{external}{=\meta{file name}}{no default, initially unset}
%     % Convenience option which calls |\tikzsetnextfilename|\marg{file name}. Typically,
%     % it may be used inside the option list of a |tcolorbox| to set the
%     % externalization \meta{file name} for the first |tikzpicture| which is discovered
%     % \emph{inside} the box content.
%     % The package |tikz| \cite{tantau:tikz_and_pgf} or the library \mylib{skins} has to be loaded to use this option.
%     % Additionally, |\usetikzlibrary{external}| has to be used.
  
%   方便选项,调用|\tikzsetnextfilename|\marg{file name}。%
%   通常,它可以在|tcolorbox|的选项列表中使用,%
%   以设置在盒子内容中发现的,第一个|tikzpicture|的外部化\meta{file name}。%
  
%     % The package |tikz| \cite{tantau:tikz_and_pgf} or the library \mylib{skins} has to be loaded to use this option.
%     % Additionally, |\usetikzlibrary{external}| has to be used.
  
%   必须加载包|tikz|或库\mylib{skins}才能使用此选项。%
%   此外,必须设置|\usetikzlibrary{external}|。
%   \end{docTcbKey}



% \begin{docTcbKey}{remake}{\colOpt{=true\textbar false}}{default |true|, initially |false|}
%   % Convenience option which calls |/tikz/external/remake next|. Typically,
%   % it may be used inside the option list of a |tcolorbox| to force the remake
%   % of the first |tikzpicture| which is discovered \emph{inside} the box content.
%   % The package |tikz| \cite{tantau:tikz_and_pgf} or the library \mylib{skins} has to be loaded to use this option.
%   % Additionally, |\usetikzlibrary{external}| has to be used.

% 方便选项,调用|/tikz/external/remake next|. %
% 通常,它可以在|tcolorbox|的选项列表中使用,以强制重绘在盒子内容中发现的第一个|tikzpicture|。%
% 必须加载包|tikz|或\mylib{skins}库才能使用此选项。%
% 此外,必须设置|\usetikzlibrary{external}|。
% \end{docTcbKey}




% % \clearpage
% % Miscellaneous
% \subsection{杂项}
% \begin{docTcbKey}{reset}{}{no value, initially set}
% % Sets (nearly) all |tcolorbox| settings (including loaded libraries) back to their default values
% % \emph{plus} any settings given by \refCom{tcbsetforeverylayer}.
% % \refKey{/tcb/savedelimiter}, \refKey{/tcb/capture}, and
% % \refKey{/tcb/shield externalize} keep their values.
% % Also, all raster values (see \Vref{sec:raster}) are not resetted.

% 设置(几乎)所有|tcolorbox|设置(包括已加载的库)回到它们的默认值,\emph{加上} 由\refCom{tcbsetforeverylayer}给出的任何设置.%
% \refKey{/tcb/savedelimiter}, \refKey{/tcb/capture}, 和 \refKey{/tcb/shield externalize} 保持他们的值不变。%
% 此外,所有raster值(见\Vref{sec:raster})都不会重置。

% % This option is useful for boxes in boxes where the inner box should not inherit
% % the settings of the outer box.
% % Note that for boxes inside boxes the |reset| is done automatically, if the
% % standard settings of the package are used (v2.40 and above), see
% % Section \ref{subsec:everybox} from page \pageref{subsec:everybox}.
% 此项在嵌套的盒子中使用,可以重置从外部盒子继承过来的设置。%
% 注意,如果使用包的标准设置(v2.40及以上),嵌套的内部盒子, |reset|会自动完成, 见\pageref{subsec:everybox}页的\ref{subsec:everybox}小节。
% %See \refCom{tcbhighmath} for an example.
% \end{docTcbKey}




% \begin{docTcbKey}{code}{=\meta{code}}{no default, initially unset}
%   % The given \meta{code} is executed immediately. This option is useful
%   % to place some arbitrary code into an option list.
% 给定的\meta{code}立即执行。这个选项很有用,可以将一些任意代码放入选项列表中。
% \begin{exdispExample}{code}
% \tcbset{colback=red!5!white,colframe=red!75!black,
%   code={Useless at this spot but functional.},
%   fonttitle=\bfseries}

% \begin{tcolorbox}[code={\newcommand{\mycommand}{\textit{working}}},
%   title=My \mycommand\ title]
% This is a \textbf{tcolorbox}.
% \end{tcolorbox}
% \end{exdispExample}
% \end{docTcbKey}



% % \clearpage
% \begin{docTcbKey}[][doc new=2016-10-21]{void}{}{no value, initially unset}
%   % Annihilates the current |tcolorbox| as far as possible.
%   % Basically, this comments out the whole |tcolorbox| by using a key.
%   % If the option list of the current |tcolorbox| contains arbitrary code with global
%   % impact (like counter settings), these actions are not undone automatically.
%   % Nevertheless, the effects of \refKey{/tcb/phantom}, \refKey{/tcb/step},
%   % \refKey{/tcb/new/auto counter}, etc., are removed by \refKey{/tcb/void}.
% %TODO 再翻译
% 尽可能消灭当前|tcolorbox|。%
% 基本上,就是一个设置,注释了整个|tcolorbox|盒子。%
% 如果当前|tcolorbox|的选项列表包含具有全局影响的任意代码(如计数器设置),这些操作不会自动撤消。%
% 不过,\refKey{/tcb/phantom}, \refKey{/tcb/step},   \refKey{/tcb/new/auto counter}等的效果会被\refKey{/tcb/void}删除。

% \begin{exdispExample}{void}
% A%
%   \begin{tcolorbox}[
%       title=This box is completely removed by the following key,
%       void
%     ]
%   This is a \textbf{tcolorbox}.
%   \end{tcolorbox}
% B
% \end{exdispExample}

% \begin{marker}
%   % This option key cannot be applied for every situation.
%   % For example, if several box environments with the same environment name
%   % are nested, for the outer environment \refKey{/tcb/void} cannot be used,
%   % since the end of the inner environment will be misinterpreted as
%   % end of the outer environment. Also, \refKey{/tcb/void} cannot be used
%   % for environments wrapped with \refCom{tcolorboxenvironment}.

% 此选项键不能应用于每种情况。%
% 例如,如果嵌套了几个具有相同环境名称的盒子环境,外部环境不能使用\refKey{/tcb/void}。%
% 因为内部环境的结束会被误认为外部环境的终结。%
% 同样,\refKey{/tcb/void}不能用于由\refCom{tcolorboxenvironment}包装的环境。
% \end{marker}
% \end{docTcbKey}



% % nirvana | BrE nɪəˈvɑːnə, AmE nərˈvɑnə,nɪrˈvɑnə |
% % noun
% % ① uncountable Religion 涅槃 nièpán
% % ② countable (idyllic place) 极乐世界 jílè shìjiè ; (state) 无忧无虑的境界 wú yōu wú lǜ de jìngjiè

% \begin{docTcbKey}[][doc new=2019-03-01]{nirvana}{}{no value, initially unset}
%   % The contents of the current |tcolorbox| are processed including counter
%   % settings, but the box is just not drawn.
%   % Therefore, \refKey{/tcb/nirvana} is less radical than \refKey{/tcb/void}
%   % and several box environments can be nested without problems.

% 当前|tcolorbox|的内容被处理,包括计数器设置,只有盒子没有绘制。%
% 因此, \refKey{/tcb/nirvana} 没有 \refKey{/tcb/void} 那么彻底,%
% 而且应用在嵌套的多个盒子环境上,不出问题。
% \begin{exdispExample}{nirvana}
% A%
%   \begin{tcolorbox}[
%       title=This box is completely removed by the following key,
%       nirvana
%     ]
%   This is a \textbf{tcolorbox}.
%     \begin{tcolorbox}
%     Nested Box
%     \end{tcolorbox}
%   \end{tcolorbox}%
% B
% \end{exdispExample}
% \end{docTcbKey}
%译 done v3 2023.3.12  111页
% \setcounter{section}{4}
% \setcounter{subsection}{24}
% \setcounter{subsubsection}{0}
% % !TeX root = tcolorbox.tex
% include file of tcolorbox.tex (manual of the LaTeX package tcolorbox)
% \clearpage

% Initialization Option Keys
\section{初始化选项}\label{sec:initkeys}%
\tcbset{external/prefix=external/initoptions_}%
\begin{stripedbox}
The \emph{initialization} options are only applicable for the generation
of new environments and commands based on |tcolorbox| and friends.
Particularly, they can be used for
\tcblower
\emph{initialization}选项只适用于生成的新环境和基于|tcolorbox|之类的命令。%
特别是,它们可以用于
\end{stripedbox}

\begin{itemize}
\item\refCom{newtcolorbox},
\item\refCom{newtcbox},
\item\refCom{newtcblisting},
\item\refCom{newtcbinputlisting},
\item\refCom{newtcbtheorem}, and
\item\refCom{newtcboxfit}.
\end{itemize}

\bigskip
\begin{marker}
\begin{stripedbox}[blank]
Typically, these options may generate counters and alike.
It is \textbf{strongly} recommended that you use initialization options inside the preamble only. 
Otherwise, you may get trouble when using \LaTeX's |\include| features.
Also, it is recommended to generate new environments and commands with these
options \emph{after} |hyperref| is loaded to avoid warnings about \emph{duplicate identifiers}.
\tcblower
通常,这些选项可能用于生成计数器等。%
\textbf{强烈}建议只在序言中使用初始化选项。%
否则,在使用\LaTeX 的|\include|特性时可能会遇到麻烦。
此外,建议在 |hyperref| 加载\emph{之后}才使用这些选项来生成新的环境和命令,以避免关于\emph{duplicate identifiers}的警告。
\end{stripedbox}
\end{marker}


% Numbered Boxes
\subsection{为盒子编号}\label{sec:numberedboxes}
\begin{stripedbox}
Counters assigned using the initialization options are administrated automatically.
Especially, they are increased for each new box.
Independent from the real counter name, 
the counter value can be referenced by \docAuxCommand{thetcbcounter}, e.\,g.\ 
inside the title of the box. 
The real counter name is stored inside \docAuxCommand{tcbcounter}.
\tcblower
使用初始化选项分配的计数器将被自动管理。%
特别是,每增加一个新盒子,计数器的值就会增加。%
不依赖于具体的真实的计数器名,盒子的计数器值可以由\docAuxCommand{thetcbcounter}取得, e.\,g.\ %
在盒子的标题里面。%
真正的计数器名称存储在\docAuxCommand{tcbcounter}中。
\end{stripedbox}


\begin{newTcbKey}{auto counter}{}{no value, initially unset}
\begin{stripedbox}
Creates a new counter automatically.%
With \refKey{/tcb/new/number format} and \refKey{/tcb/new/number within}, 
the appearance and behavior of the counter can be changed. 
The counter value is referenced by \docAuxCommand{thetcbcounter}.
\tcblower
自动创建一个新的计数器。%
使用 \refKey{/tcb/new/number format} 和 \refKey{/tcb/new/number within}, %
可以更改计数器的外观和行为。%
计数器值可以由\docAuxCommand{thetcbcounter}取得。
\end{stripedbox}


\inputpreamblelisting{A}

% This box is automatically numbered with \ref{myautocounter} on page \pageref{myautocounter}. 
% Inside the box, the \thetcbcounter\ can also be referenced by |\thetcbcounter|.
% The real counter name is \texttt{\tcbcounter}.
\begin{dispExample}
\begin{pabox}[label={myautocounter}]{带编号的标题}
这个盒子是自动编号的,来自第 \pageref{myautocounter} 页的 \ref{myautocounter}。%
在盒子内部,引用 \thetcbcounter\ 可以使用 |\thetcbcounter| 生成。
真正的计数器名称是\texttt{\tcbcounter}。
\end{pabox}
\end{dispExample}
\end{newTcbKey}


% \clearpage
\begin{newTcbKey}{use counter from}{=\meta{tcolorbox}}{no default, initially unset}
\begin{stripedbox}
Here, a counter from another \meta{tcolorbox} is reused.
Note that the settings for 
\refKey{/tcb/new/number format} and \refKey{/tcb/new/number within} 
are inherited and cannot be changed.
The counter value is referenced by \docAuxCommand{thetcbcounter}.
\tcblower
这里,来自另一个\meta{tcolorbox}的计数器被重用。%
注意,\refKey{/tcb/new/number format} 和 \refKey{/tcb/new/number within} 的设置是继承的,不能更改。%
计数器值可以从 \docAuxCommand{thetcbcounter} 取得。
\end{stripedbox}

\begin{dispExample}
\newtcolorbox[use counter from=pabox]{mybox}[2][]{%
colback=blue!5!white,colframe=blue!75!black,fonttitle=\bfseries,
title=Some Box \thetcbcounter: #2,#1}
% 第一个默认参数为空,用于指定额外的选项
% 第二个参数用于指定标题内容
\begin{mybox}[label={myusecounterfrom}]{Title with continued number}
这个盒子是自动编号的,来自第 \pageref{myusecounterfrom} 页的 \ref{myusecounterfrom} 小节。
在盒子内部,引用 \thetcbcounter\ 可以使用 |\thetcbcounter| 生成。%
真正的计数器名称是 \texttt{\tcbcounter}.
\end{mybox}
\end{dispExample}
\end{newTcbKey}

\begin{newTcbKey}{use counter}{=\meta{counter}}{no default, initially unset}
\begin{stripedbox}
Here, an ordinary existing \LaTeX\ \meta{counter} is used for numbering.
With \refKey{/tcb/new/number format} and \refKey{/tcb/new/number within}, 
the appearance and behavior of the counter can be changed. 
The counter value is referenced by \docAuxCommand{thetcbcounter}.
\tcblower
%这里,
使用已经存在的 \LaTeX\ 计数器 \meta{counter} 来编号。%
使用 \refKey{/tcb/new/number format} 和 \refKey{/tcb/new/number within}, %
可以更改计数器的外观和行为。%
计数器值可以由\docAuxCommand{thetcbcounter}取得。
\end{stripedbox}    


\begin{dispExample}
% \newcounter{myexample}%  preamble
\newtcolorbox[use counter=myexample,number format=\Alph]{mybox}[2][]{%
colback=green!5!white,colframe=green!55!black,fonttitle=\bfseries,
title=Some Box \thetcbcounter: #2,#1}

\begin{mybox}[label={myusecounter}]{Title with \LaTeX\ number}
这个盒子自动编号,位于 \pageref{myusecounter} 页第 \ref{myusecounter} 小节。%
在盒子内部,引用 \thetcbcounter\ 可以使用 |\thetcbcounter| 生成。%
真正的计数器名称是\texttt{\tcbcounter}。
\end{mybox}
\end{dispExample}
\end{newTcbKey}


\begin{newTcbKey}[][doc new=2014-09-19]{use counter*}{=\meta{counter}}{no default, initially unset}
\begin{stripedbox}
An existing \LaTeX\ \meta{counter} is used for numbering. 
In contrast to \refKey{/tcb/new/use counter}, the options \refKey{/tcb/new/number format} and
\refKey{/tcb/new/number within} are ignored. 
Use this for counters which are already configured outside the |tcolorbox| package, e.\,g.\ the standard |figure| counter.
\tcblower
使用已经存在的 \LaTeX\ 计数器 \meta{counter} 来编号。%
同 \refKey{/tcb/new/use counter} 相反, 选项 \refKey{/tcb/new/number format} 和 \refKey{/tcb/new/number within} 被忽略。%
这用于使用在 |tcolorbox| 包外定义的计数器, e.\,g.\ 标准的 |figure| 计数器。
\end{stripedbox}
\end{newTcbKey}

\begin{newTcbKey}{no counter}{}{no value, initially set}
\begin{stripedbox}
The created boxes are not numbered. This is the default. 
The option may be used to overrule a previous option.
\tcblower
创建的盒子没有编号。这是默认值。%
该选项可用于否决先前的选项。
\end{stripedbox}
\end{newTcbKey}

% \enlargethispage*{1cm}

\begin{newTcbKey}[][doc new=2019-10-18]{reset counter on overlays}{\colOpt{=true\textbar false}}{default |true|, initially |false|}
\begin{stripedbox}
For |beamer| slides, this invokes the |\resetcounteronoverlays| command for the box counter. 
The counter is automatically reset on subsequent overlay slides of a frame.
Thereby, the counter will be the same on all slides of every frame.
\tcblower
%to 再翻译
对于|beamer|幻灯片,这为盒子计数器调用|\ resetcounteronoverlay|命令。
计数器在帧的后续覆盖幻灯片上自动重置。
因此,计数器在每一帧的所有幻灯片上都是相同的。
\end{stripedbox}
\end{newTcbKey}

% \clearpage
\begin{newTcbKey}{number within}{=\meta{counter}}{no default, initially unset}
\begin{stripedbox}
The automatic counter is set to zero, if \meta{counter} is increased.
Additionally, during output, 
the value of \meta{counter} is prepended to the value of the automatic counter.\par
To prepend the automatic counter with the chapter number and to reset it with every new chapter, use:
\tcblower
如果\meta{counter}增加,则自动计数器被设置为零\footnote{译注,类似于section计数器在chapter计数器增加时会重置的效果一样。}。
\meta{counter}的值被加在盒子自动计数器的值之前。\par
要在自动计数器前加上章节号,并在每一个新的章节中重置它,请使用:
\end{stripedbox}

\begin{dispListing}
number within=chapter
\end{dispListing}
\begin{stripedbox}
See \refKey{/tcb/new/use counter} for a complete example.
\tcblower
完整的示例请参见\refKey{/tcb/new/use counter}。
\end{stripedbox}
\end{newTcbKey}


\begin{newTcbKey}{number format}{=\meta{format macro}}{no default, initially \texttt{\textbackslash arabic}}
\begin{stripedbox}
Declares the format of the automatic counter. 
The \meta{format macro} can be any valid \LaTeX\ number formatting macro like |\arabic|, |\roman|, etc.\par
To display the counter value in large roman numbers, use:
\tcblower
声明自动计数器的格式。%
\meta{format macro}可惜是任何合法的 \LaTeX\ 数字格式化宏命令,比如像 |\arabic|, |\roman|, 等等。\par
要以大罗马数字显示计数器值,请使用:
\end{stripedbox}

\begin{dispListing}
number format=\Roman
\end{dispListing}
\begin{stripedbox}
See \refKey{/tcb/new/auto counter} for a complete example.
\tcblower
完整的示例请参见\refKey{/tcb/new/auto counter}。
\end{stripedbox}
\end{newTcbKey}

\begin{newTcbKey}{number freestyle}{=\meta{code}}{no default, initially unset}
\begin{stripedbox}
Allows advanced control over the complete number format.%
This option overrules the format given by \refKey{/tcb/new/number within} and \refKey{/tcb/new/number format}.
Nevertheless, you can combine it with \refKey{/tcb/new/number within} to get the desired reset property.\par
The \meta{code} is some formatting code which should contain |\tcbcounter| to reference the automated counter.
Since this \meta{code} is expanded, you have to secure each macro with |\noexpand| with \emph{exception} of |\tcbcounter|.
\tcblower
允许对完整的数字格式进行高级控制。%
该选项会推翻\refKey{/tcb/new/number within}和\refKey{/tcb/new/number format}给出的格式。
不过,你可以将它与\refKey{/tcb/new/number within}结合起来,以获得所需的重置属性。\par
\meta{code}是一些格式化代码,可以包含|\tcbcounter|来引用自动计数器。
由于这个\meta{code}是展开的,\emph{除了} |\tcbcounter| 外,你必须用 |\noexpand| 来保护每个命令。
\end{stripedbox}

\inputpreamblelisting{H}

\begin{dispExample}
\begin{phbox}[label={myfreestyle}]{编号是myfreestyle的标题}
这个盒子自动编号,位于 \pageref{myfreestyle} 页 \ref{myfreestyle}。%
在盒子内部,引用 \thetcbcounter\ 可以使用 |\thetcbcounter| 生成。%
真正的计数器名称是\texttt{\tcbcounter}。
\end{phbox}
\end{dispExample}
\end{newTcbKey}

% \clearpage
\begin{marker}
\begin{stripedbox}[blank]
The following options \refKey{/tcb/new/crefname} and \refKey{/tcb/new/Crefname}
need to be set inside the preamble.
\tcblower
以下选项\refKey{/tcb/new/crefname}和\refKey{/tcb/new/crefname}需要设置在导言区内。
\end{stripedbox}
\end{marker}

\begin{newTcbKey}[][doc updated=2014-12-01]{crefname}{=\marg{singular}\marg{plural}}{no default, initially unset}
\begin{stripedbox}
This option key can be used only in conjunction with the |cleveref| package
%\cite{cubitt:2018a} 
which has to be loaded separately.
It creates a cross-reference type for the new |tcolorbox|'es, where the
lowercase \meta{singular} and \meta{plural} forms of the cross-reference are given.
This type is the environment or macro name and \refKey{/tcb/label type} is set automatically.
See \refKey{/tcb/label type} and \cite{cubitt:2018a} 
for more information.
\tcblower
此选项只能与~|cleveref|~包结合使用。%,必须分开装载。
%\cite{cubitt:2018a} 
它为新的 |tcolorbox|'es 创建了一个交叉引用类型, %
其中给出了交叉引用的小写的 \meta{singular}(单数) 和 \meta{plural}(复数)形式。%
该类型是环境或宏名称,会自动设置\refKey{/tcb/label type}。%
更多信息请参见 \refKey{/tcb/label type} 和 \cite{cubitt:2018a}。
\end{stripedbox}
\end{newTcbKey}

\begin{newTcbKey}[][doc updated=2014-12-01]{Crefname}{=\marg{singular}\marg{plural}}{no default, initially unset}
\begin{stripedbox}
This option key can be used only in conjunction with the |cleveref| package
\cite{cubitt:2018a} which has to be loaded separately.
It creates a cross-reference type for the new |tcolorbox|'es, where the
uppercase \meta{singular} and \meta{plural} forms of the cross-reference are given.
This type is the environment or macro name and \refKey{/tcb/label type} is set automatically.
See \refKey{/tcb/label type} and \cite{cubitt:2018a} for more information.
\tcblower
此选项只能与~|cleveref|~包结合使用。%,必须分开装载。
它为新的 |tcolorbox|'es 创建了一个交叉引用类型, %, where the
其中给出了交叉引用的大写的 \meta{singular}(单数) 和 \meta{plural}(复数)形式。%
该类型是环境或宏名称,会自动设置\refKey{/tcb/label type}。%
更多信息请参见 \refKey{/tcb/label type} 和 \cite{cubitt:2018a}。
\end{stripedbox}
\end{newTcbKey}

\inputpreamblelisting{I}
\begin{dispExample}
% \usepackage{varioref}
% \usepackage{cleveref}
\begin{mybluebox}[label={myreference}]{我的标题}
This is an example.
\end{mybluebox}

大写的引用类型名加计数引用,|\Cref{myreference}|:\Cref{myreference},\\ 
小写的引用类型名加计数引用,|\cref{myreference}|:\cref{myreference}.\\

大写的前缀加页数计数器值,|\Cpageref{myreference}|:\Cpageref{myreference}\\
小写的前缀加页数计数器值,|\cpageref{myreference}|:\cpageref{myreference} .\\

大写的引用类型,|\nameCref{myreference}|:\nameCref{myreference}\\
小写的引用类型,|\namecref{myreference}|:\namecref{myreference}\\

计数引用,|\labelcref{myreference}|:\labelcref{myreference}\\
页数,|\labelcpageref{myreference}|:\labelcpageref{myreference}\\

With \texttt{varioref}:\\
大写的引用类型名加计数引用,|\Vref{myreference}|:\Vref{myreference}\\
小写的引用类型名加计数引用,|\vref{myreference}|:\vref{myreference}\\

大写的引用类型名加计数引用,|\Vref*{myreference}|:\Vref*{myreference}\\
小写的引用类型名加计数引用,|\vref*{myreference}|:\vref*{myreference}.
\end{dispExample}

% \clearpage

\begin{newTcbKey}[][doc new=2014-09-19]{blend into}{=\meta{name}}{style, no default, initially unset}
\begin{stripedbox}
Used to comfortably blend into an existing schema of naming and numbering for
some selected cases. For example, a |tcolorbox| can be used to display
and entitle an image pretending to be a standard |figure| environment.
Here, \refKey{/tcb/title} is used instead of the standard |\caption|
and \refKey{/tcb/list text} can be used instead of the optional parameter
of the standard |\caption|.
\tcblower
用于对某些选定的情况, 轻松地混合到现有的命名和编号模式中。%
例如,|tcolorbox| 可用于显示并赋予图像标题,假装是标准~|figure|~环境。%
这里使用了\refKey{/tcb/title}来代替标准的|\caption|,%
而\refKey{/tcb/list text}可以用来代替标准 |\caption| 的可选参数。%
\end{stripedbox}

\begin{stripedbox}
Feasible values for \meta{name} are:
\tcblower
\meta{name}的可行值是:
\end{stripedbox}
\begin{itemize}
\item\docValue{figures}: 
\begin{stripedbox}%[width=\linewidth]
blend into the standard |figure| environment.
\tcblower
融入标准 |figure| 环境。
\end{stripedbox}

\item\docValue{tables}:
\begin{stripedbox}
blend into the standard |table| environment.
\tcblower
融入标准 |table| 环境。
\end{stripedbox}
\item\docValue{listings}: 
\begin{stripedbox}
blend into the standard |lstlisting| environment
of the package |listings| %\cite{hoffmann:listings}
.
\tcblower
融入 |listings| 包的 |lstlisting| 环境。
\end{stripedbox}

\begin{marker}
\begin{stripedbox}[blank]
Note that |blend into=listings| can only be used in the document content
or, preferably, inside a |\AtBeginDocument| clause! Using it without
|\AtBeginDocument| inside the preamble does not work since the |listings|
packages initializes its counter also inside |\AtBeginDocument|.
\tcblower
注意,|blend into=listings| 只能在文档内容中使用,%
或者,最好是在 |\AtBeginDocument| 中!%
在导言区中而不是在 |\AtBeginDocument| 中的话,是不工作的,%
因为 |listings| 包初始化它的计数器也在 |\AtBeginDocument| 里。
\end{stripedbox}
\end{marker}
\end{itemize}
\end{newTcbKey}

% \enlargethispage*{5cm}
\begin{dispListing}
\begin{figure}[htb]
  \centering\includegraphics[height=4cm]{lichtspiel.jpg}
  \caption{标准的figure}
\end{figure}

\newtcolorbox[blend into=figures]{myfigure}[2][]{float=htb,capture=hbox,
  title={#2},every float=\centering,#1}

\begin{myfigure}{tcolorbox的figure}
  \includegraphics[height=4cm]{lichtspiel.jpg}
\end{myfigure}
\end{dispListing}
{\tcbusetemp}

% \clearpage
\begin{docTcbKey}[][doc new=2015-03-13]{blend before title}{=\meta{value}}{no default, initially \docValue{colon}}
\begin{stripedbox}
This option formats the title output of \refKey{/tcb/new/blend into}.
Note that this is a common |tcolorbox| option which should be set
globally or in the normal option part of \refCom{newtcolorbox}.
\tcblower
该选项用于格式化\refKey{/tcb/new/blend into}的标题输出。%
注意,这是一个常见的 |tcolorbox| 选项,应该全局设置, 或在 \refCom{newtcolorbox} 的普通选项部分。\end{stripedbox}

\begin{stripedbox}
Feasible values for \meta{value} are:
\tcblower
\meta{value}的可行值是:
\end{stripedbox}
\begin{itemize}
\item\docValue{colon}: 
\begin{stripedbox}
use name/number plus colon.
\tcblower
使用 |名字|/|编号| 加上 |冒号|
\end{stripedbox}

\item\docValue{dash}: 
\begin{stripedbox}
use name/number plus dash.
\tcblower
使用 |名字|/|编号| 加上 |破折号|
\end{stripedbox}

\item\docValue{colon hang}: 
\begin{stripedbox}
use name/number plus colon with hanging indent.
\tcblower
使用 |名字|/|编号| 加上 |冒号|,并悬挂式缩进。
\end{stripedbox}

\item\docValue{dash hang}: 
\begin{stripedbox}
use name/number plus dash with hanging indent.
\tcblower
使用 |名字|/|编号| 加上 |破折号|,并悬挂式缩进。
\end{stripedbox}
\end{itemize}

\begin{dispListing}
\newtcolorbox[blend into=figures]{myfigure}[2][]{float=htb,capture=hbox,
  blend before title=dash hang,title={#2},every float=\centering,#1}

\begin{myfigure}{一个标题相当长的tcolorbox图}
  \includegraphics[height=5cm]{lichtspiel.jpg}
\end{myfigure}
\end{dispListing}
{\tcbusetemp}
\end{docTcbKey}

% \clearpage
\begin{docTcbKey}[][doc new=2015-03-13]{blend before title code}{=\meta{code}}{no default}
\begin{stripedbox}
This option formats the title output of \refKey{/tcb/new/blend into}.
The \meta{code} takes one parameter, the name/number.
Use this, if \refKey{/tcb/blend before title} is not flexible enough.
\tcblower
该选项格式化\refKey{/tcb/new/blend into}的标题输出。%
\meta{code}接受一个参数,即 name/number。%
如果\refKey{/tcb/blend before title}不够灵活,可以使用这个。
\end{stripedbox}

\begin{dispListing}
\newtcolorbox[blend into=figures]{myfigure}[2][]{float=htb,capture=hbox,
  blend before title code={\fbox{##1}\ },title={#2},every float=\centering,#1}

\begin{myfigure}{A tcolorbox figure}
  \includegraphics[height=6cm]{lichtspiel.jpg}
\end{myfigure}
\end{dispListing}
{\tcbusetemp}
\end{docTcbKey}

% \clearpage
% Lists of \texttt{tcolorbox}es
\subsection{\texttt{tcolorbox}盒子目录}\label{sec:listsof}
\begin{stripedbox}
For figures and tables, \LaTeX\ provides the |\listoffigures| and
|\listoftables| commands to create lists of these numbered entities.
Also, a |tcolorbox| can be part of such a kind of list.
\tcblower
对于图形和表格,\LaTeX\ 提供了 |\listfigures| 和 |\listoftables| 命令创建这些编号实体的列表。%
同样,|tcolorbox|可以是这种列表的一部分。
\end{stripedbox}

\begin{enumerate}
\item 
\begin{stripedbox}
Assign a list \meta{name} by the \emph{initialization} option
\refKey{/tcb/new/list inside}.
\tcblower
通过\emph{初始化}选项\refKey{/tcb/new/list inside}分配一个列表\meta{name}。
\end{stripedbox}

\item 
\begin{stripedbox}
Optionally, a new \meta{type} for list entries may be assigned
by the \emph{initialization} option \refKey{/tcb/new/list type}.
\tcblower
可选地,通过\emph{初始化}选项\refKey{/tcb/new/list type},%
可以为列表条目分配一个新的\meta{type}。
\end{stripedbox}

\item 
\begin{stripedbox}
List entries a generated automatically within each new |tcolorbox|
using the above initialization.
\tcblower
列出使用上述初始化在每个新|tcolorbox|中自动生成的条目。
\end{stripedbox}
    \begin{itemize}
    \item 
\begin{stripedbox}
If \refKey{/tcb/list entry} is set, the entry is generated with it.
\tcblower
如果设置了\refKey{/tcb/list entry},则生成该条目。
\end{stripedbox}
    
    \item 
\begin{stripedbox}
Otherwise, if \refKey{/tcb/title} is set, the entry is generated with it.
\tcblower
否则,如果设置了\refKey{/tcb/title},则生成该条目。
\end{stripedbox}
    
    \item 
\begin{stripedbox}
Otherwise, the entry is generated with the current number and the environment name.
\tcblower
否则,将使用当前编号和环境名称生成条目。
\end{stripedbox}
    
    \end{itemize}
\item 
\begin{stripedbox}
The generated list is displayed by \refCom{tcblistof}.
\tcblower
生成的列表由\refCom{tcblistof}显示。
\end{stripedbox}
\end{enumerate}

\begin{newTcbKey}{list inside}{=\meta{name}}{no default, initially unset}
\begin{stripedbox}
Assigns a list or contents file to the generated |tcolorbox|es.
Entries to this list are saved to a file which gets the \meta{name} as
file name extension. The list is referenced by this name in
\refCom{tcblistof}.
For example:
\tcblower
为生成的 |tcolorbox| 盒子们分配一个列表或内容文件。%
此列表中的条目保存到一个文件中,该文件以\meta{name}作为文件扩展名。%
该列表由\refCom{tcblistof}引用。%
例如:
\end{stripedbox}

\begin{dispListing}
list inside=exam
\end{dispListing}
\begin{stripedbox}
See Section \ref{listing:exercises} from page \pageref{listing:exercises}
for a complete example.
\tcblower
一个完整的例子参见 \pageref{listing:exercises} 页的 \ref{listing:exercises} 小节。
\end{stripedbox}
\end{newTcbKey}

\begin{newTcbKey}{list type}{=\meta{type}}{no default, initially |tcolorbox|}
\begin{stripedbox}
Optionally, some \meta{type} can be assigned to the list entries.
For a new \meta{type}, a macro |\l@|\meta{type} has to exist which controls
the format of the list entry. The default type is defined by
\tcblower
可选地,可以将列表条目设置为一些\meta{type}。%
对于一个新的\meta{type}, 存在宏命令 |\l@|\meta{type} 用于控制列表条目的格式。%
默认类型定义为:
\end{stripedbox}

\begin{dispListing}
\newcommand*\l@tcolorbox{\@dottedtocline{1}{1.5em}{2.3em}}
\end{dispListing}
\begin{stripedbox}
This is identical to the |\l@section| setting of \LaTeX. |\l@tcolorbox| can
be redefined or a new \meta{type} can be assigned.
\tcblower
这同 \LaTeX 的 |\l@section| 的定义是相同的。
|\l@tcolorbox| 可以重新定义或分配一个新的\meta{type}。
\end{stripedbox}
\end{newTcbKey}


% \clearpage
\begin{docCommand}[doc updated=2021-05-20]{tcblistof}{\oarg{macro}\marg{name}\oarg{short}\marg{title text}}
\begin{stripedbox}
Displays the generated list of |tcolorbox|es with the given \meta{name}.
The heading is generated by \meta{macro}\oarg{short}\marg{title text} where \texttt{\textbackslash section}
is the default setting for \meta{macro}.
Here, as usual, \meta{title text} is the title of the section or chapter
while \meta{short} is a shorter title for headings and table of contents.
\tcblower
用给定的\meta{name}显示生成的|tcolorbox|es列表。%
标题由\meta{macro}\oarg{short}\marg{title text}生成,其中\texttt{\textbackslash section} 是\meta{macro}的默认设置。%
这里,像往常一样,\meta{title text}是章或节的标题, \meta{short}是章或节的短标题.
\end{stripedbox}

\begin{itemize}
\item 
\begin{stripedbox}
If \meta{macro} ends with a |*|, \refCom{tcblistof} mimics the behavior of
|\listoffigures| from the standard \LaTeX\ classes and adds the title
to the left and right mark for headings.
\tcblower
如果\meta{macro}以|*|结尾, \refCom{tcblistof} 模拟标准 \LaTeX\ 的 |\listoffigures| 的行为,%
并将标题添加到页眉的左右标记。
\end{stripedbox}

\item 
\begin{stripedbox}
If \meta{macro} starts with |\chapter|, a possible two column document setting
is restored to one column (as standard \LaTeX\ classes do for |\listoffigures|).
\tcblower
如果\meta{macro}以|\chapter|开始, %
有一些的两列文档设置将恢复为一列 (像标准 \LaTeX\ 的 |\listoffigures| 做的那样).
\end{stripedbox}
\end{itemize}

\medskip
\begin{stripedbox}
To display the list inside a subsection, use for example:
\tcblower
要在subsection中显示列表,例如:
\end{stripedbox}
\begin{dispListing}
\tcblistof[\subsection]{exam}{List of Exercises}
\end{dispListing}
\begin{stripedbox}
The result of the example is found as Subsection \ref{listofexercises} on
page \pageref{listofexercises}.
\tcblower
示例结果可以参见在 \pageref{listofexercises} 页的 \ref{listofexercises} 小节。
\end{stripedbox}


\medskip
\begin{stripedbox}
To apply the list similar to |\listoffigures| for a report or book, use for example:
\tcblower
要将类似 |\listoffigures| 的列表应用于报告或书籍,请使用例如:
\end{stripedbox}
\begin{dispListing}
\tcblistof[\chapter*]{exam}{List of Exercises}
\end{dispListing}

\medskip
\begin{stripedbox}
To set a short title for headings with the default |\section| setting, use for example:
\tcblower
%todo 是页眉么?
使用默认|\section|设置,为页眉设置一个短标题,例如:
\end{stripedbox}
\begin{dispListing}
\tcblistof{exam}[List of Exercises]{Elaborate List of Fine Exercises
                                    for all Students of my Course}
\end{dispListing}

\medskip
\begin{marker}
\begin{stripedbox}
The core of the list is generated by |\@starttoc|\marg{name} which
can be wrapped into an own macro.
\tcblower
列表的核心是由|\@starttoc|\marg{name}生成的,%
可以包装到已有的宏。
\end{stripedbox}
\end{marker}
\end{docCommand}
%译 done 2023.1.23  5-initoptions  12页  后面将中英的分开
% \setcounter{section}{5}
% \setcounter{subsection}{2}
% % !TeX root = tcolorbox.tex
% include file of tcolorbox.tex (manual of the LaTeX package tcolorbox)
% \clearpage
% Side by Side

\setcounter{section}{5}
\setcounter{subsection}{2}

\section{并排}\label{sec:sidebyside}%
\tcbset{external/prefix=external/sidebyside_}%

A \csh渐变盒子{side by side} box is a special \refEnvLe{tcolorbox} where
the upper and lower part of the box are set side by side.
All boxes of this kind are unbreakable.


一个\emph{并排}的盒子是一个特殊的\refEnvLe{tcolorbox},其中盒子的上部和下部并排设置。这种类型的所有盒子都是\csh渐变盒子[red]{不可分割的}。


\begin{marker}
% \begin{stripedbox}
Further side by side options for code examples are
% \tcblower

进一步的排版代码的 side by side 选项的例子有
% \end{stripedbox}

\refKeyLe{/tcb/listing side text},
\refKeyLe{/tcb/text side listing},
\refKeyLe{/tcb/listing outside text}, 和
\refKeyLe{/tcb/text outside listing}.
\end{marker}

% Basic Settings \hfill 
\subsection{基本设置}\label{subsec:sidebyside_basic}

\begin{docTcbKey}{sidebyside}{=\colOpt{true\textbar false}}{默认值|true|,初始设置为|false|}
% \begin{docTcbKey}{sidebyside}{\colOpt{=true\textbar false}}{default |true|, initially |false|}
% \begin{stripedbox}
Normally, the upper part and the lower part of the box have their positions
as their names suggest. If |sidebyside| is set to |true|, the upper part
is drawn \emph{left-handed} and the lower part is drawn \emph{right-handed}.
Both parts are drawn together with the geometry settings of the upper part but the
space is divided horizontally according to the following options.
Colors, fonts, and box content additions are used individually.
The resulting box is unbreakable.

% Normally, the upper part and the lower part of the box have their positions as their names suggest. 
% 通常,盒子的upper部分和lower部分的位置,同它们的名字一致。%
% If |sidebyside| is set to |true|, the upper part
% is drawn \emph{left-handed} and the lower part is drawn \emph{right-handed}.
% 如果 |sidebyside| 设置为 true, 则upper部分会放置在左边,lower部分放置在右边。%

% Both parts are drawn together with the geometry settings of the upper part but the
% space is divided horizontally according to the following options.
% 两个部分都使用上部分的几何设置进行绘制,但空间按照以下选项水平分割。
% Colors, fonts, and box content additions are used individually.
% The resulting box is unbreakable.
% 颜色、字体和盒子内容的添加都是独立使用的。生成的盒子是\csh渐变盒子V[red]{不可分割}的。


通常,盒子的upper部分和lower部分的位置,同它们的名字一致。
如果 sidebyside 设置为 true, 则upper部分会放置在左边,lower部分放置在右边。
两个部分都使用上部分的几何设置进行绘制,但空间按照以下选项水平分割。颜色、字体和盒子内容的添加都是独立使用的。生成的盒子是\csh渐变盒子V[red]{不可分割}的。


\begin{dispExample}
\tcbset{colback=red!5!white%
,colframe=red!75!black%
,fonttitle=\bfseries}

\begin{tcolorbox}[title=我的标题,sidebyside]
这是 upper (\textit{左侧}) 部分。
\tcblower
这是 lower (\textit{右侧}) 部分。
\end{tcolorbox}
\end{dispExample}


%todo bicolor 是啥
\begin{dispExample}
% \usepackage{lipsum}
% \tcbuselibrary{skins}
\begin{tcolorbox}[bicolor% 表示创建一个双色盒子,即上下两部分的颜色可以不同。
,sidebyside%并排
,righthand width=3cm%指定右侧的宽度
,sharp corners,boxrule=.4pt,colback=green!5,colbacklower=green!50!black!50]
\lipsum[2]
\tcblower
\includegraphics[width=\linewidth]{goldshade}%
\end{tcolorbox}
\end{dispExample}
\end{docTcbKey}%
% \clearpage
\begin{docTcbKey}[][doc updated=2015-02-06]{sidebyside align}{=\meta{alignment}}{no default, initially |center|}

Sets the vertical \meta{alignment} for the left-handed and right-handed part.%
Feasible values for \meta{alignment} are:

设置左侧和右侧的垂直 \meta{alignment} 方式。%
可选的 \meta{对齐} 值有:

\begin{tcolorbox}[title=\docValue{center},sidebyside,sidebyside align=center]
与 |minipage|环境的选项 |c| 相同。
\tcblower
identical to |minipage| option |c|.
\end{tcolorbox}
 
\begin{tcolorbox}[title=\docValue{top},sidebyside,sidebyside align=top]
与 |minipage| 选项 |t| 相同(根据基线对齐左侧和右侧的顶部行)。
\tcblower
identical to |minipage| option |t| (aligns the top lines of the left-handed and right-handed side according to their baselines).
\end{tcolorbox}

\begin{tcolorbox}[title=\docValue{bottom},sidebyside,sidebyside align=bottom]
与 |minipage| 选项 |b| 相同(根据基线对齐左侧和右侧的底部行)。
\tcblower
identical to |minipage| option |b| (aligns the bottom lines of the left-handed and right-handed side according to their baselines).
\end{tcolorbox}


\begin{tcolorbox}[title=\docValue{center seam},sidebyside,sidebyside align=center seam]
将左侧和右侧的中心对齐。
\tcblower
aligns the center of the left-handed and right-handed side.
\end{tcolorbox}

\begin{tcolorbox}[title=\docValue{top seam},sidebyside,sidebyside align=top seam]
将左侧和右侧的顶部接缝对齐。
\tcblower
aligns the very top seam of the left-handed and right-handed side.
\end{tcolorbox}


\begin{tcolorbox}[title=\docValue{bottom seam},sidebyside,sidebyside align=bottom seam]
将左侧和右侧的底部接缝对齐。
\tcblower
aligns the very bottom seam of the left-handed and right-handed side.
\end{tcolorbox}


% \begin{exdispExample}{sidebyside_align}
% \tcbset{colback=red!5!white,colframe=red!75!black,fonttitle=\bfseries,nobeforeafter,
%   left=2mm,right=2mm,sidebyside,sidebyside gap=6mm,width=(\linewidth-2mm)/3}

% \begin{tcolorbox}[adjusted title=center,sidebyside align=center]
% 这段文字太长了太长了太长了太长了,一行写不完。
% \tcblower
% 简短文字。
% \end{tcolorbox}\hfill
% \begin{tcolorbox}[adjusted title=top,sidebyside align=top]
% 这段文字太长了太长了太长了太长了,一行写不完。
% \tcblower
% 简短文字。
% \end{tcolorbox}\hfill
% \begin{tcolorbox}[adjusted title=bottom,sidebyside align=bottom]
% 这段文字太长了太长了太长了太长了,一行写不完。
% \tcblower
% 简短文字。
% \end{tcolorbox}
% \end{exdispExample}


% \begin{stripedbox}
\docValue{center}, \docValue{top}, and \docValue{bottom} are identical
to the known corresponding |minipage| options.
While this is the preferred approach for text content, the result for
boxed content like tables or images may not be as expected.
% \tcblower

\docValue{center}、\docValue{top}和\docValue{bottom}与 |minipage| 的选项相同。%
% 虽然这是文本内容的首选方法,%
% 对于盒子中是表格或图像等内容,结果可能与预期不同。
虽然这是文本内容的首选方法,但对于表格或图像等盒子内容的结果可能不如预期。
% \end{stripedbox}

% \begin{stripedbox}
For such content, one may use \docValue{center seam}, \docValue{top seam},
and \docValue{bottom seam}. For example, \docValue{top seam} aligns
the very top seam of the left-handed and right-handed side.
% \tcblower

% 对于这样的内容,可以使用\docValue{center seam}、\docValue{top seam}和\docValue{bottom seam},
% 例如,\docValue{top seam}对齐左和右边的最上面的缝。
% \end{stripedbox}

对于这样的内容,可以使用\docValue{center seam}、\docValue{top seam}和\docValue{bottom seam}。例如,\docValue{top seam}将左侧和右侧的顶部接缝对齐。


\begin{dispExample}
\tcbset{colback=red!5!white,colframe=red!75!black,fonttitle=\bfseries,
  size=small,righthand width=4cm,sidebyside,sidebyside gap=6mm,lower separated=false}

\begin{tcolorbox}[adjusted title=center seam,sidebyside align=center seam]
This is my description text for the pictures displayed on the right-handed side.
\tcblower
\includegraphics[width=\linewidth/2]{goldshade}%
\includegraphics[width=\linewidth/2]{blueshade}
\end{tcolorbox}

\begin{tcolorbox}[adjusted title=top seam,sidebyside align=top seam]
  This is my description text for the pictures displayed on the right-handed side.
  \tcblower
  \includegraphics[width=\linewidth/2]{goldshade}%
  \includegraphics[width=\linewidth/2]{blueshade}
\end{tcolorbox}

\begin{tcolorbox}[adjusted title=bottom seam,sidebyside align=bottom seam]
  This is my description text for the pictures displayed on the right-handed side.
  \tcblower
  \includegraphics[width=\linewidth/2]{goldshade}%
  \includegraphics[width=\linewidth/2]{blueshade}
\end{tcolorbox}
\end{dispExample}
\end{docTcbKey}%
% \clearpage
\begin{docTcbKey}{sidebyside gap}{=\meta{length}}{no default, initially |10mm|}
% \begin{stripedbox}
Sets the horizontal distance between the left-handed and right-handed part to \meta{length}.
% \tcblower

将左和右部分之间的水平距离设置为\meta{length}。
% \end{stripedbox}

\begin{dispExample}
\tcbset{colback=red!5!white,colframe=red!75!black,fonttitle=\bfseries,nobeforeafter,
  sidebyside,width=(\linewidth-2mm)/2}

\begin{tcolorbox}[adjusted title=宽gap,sidebyside gap=30mm]
这段文字太长了太长了太长了太长了,一行写不完。
\tcblower
简短文字。
\end{tcolorbox}\hfill
\begin{tcolorbox}[adjusted title=窄gap,sidebyside gap=1mm]
这段文字太长了太长了太长了太长了,一行写不完。
\tcblower
简短文字
\end{tcolorbox}
\end{dispExample}
\end{docTcbKey}%
\begin{docTcbKey}{lefthand width}{=\meta{length}}{no default, initially unset}

Sets the width of the left-handed part to the given \meta{length}.

将左部的宽度设置为给定的\meta{length}。


\begin{dispExample}
\tcbset{colback=red!5!white,colframe=red!75!black,fonttitle=\bfseries}

\begin{tcolorbox}[title=My title,sidebyside,lefthand width=3cm]
这是 upper (\textit{左}) 部分.
\tcblower
这是 lower (\textit{右}) 部分.
\end{tcolorbox}
\end{dispExample}
\end{docTcbKey}

\enlargethispage*{1cm}
\begin{docTcbKey}{righthand width}{=\meta{length}}{no default, initially unset}

Sets the width of the right-handed part to the given \meta{length}.

将右部的宽度设置为给定的\meta{length}。


\begin{dispExample}
\tcbset{colback=red!5!white,colframe=red!75!black,fonttitle=\bfseries}

\begin{tcolorbox}[title=My title,sidebyside,righthand width=3cm]
这是 upper (\textit{左}) 部分.
\tcblower
这是 lower (\textit{右}) 部分.
\end{tcolorbox}
\end{dispExample}
\end{docTcbKey}

% \clearpage
\begin{docTcbKey}{lefthand ratio}{=\meta{fraction}}{no default, initially |0.5|}

Sets the width of the left-handed part to the given \meta{fraction} of
the available space. \meta{fraction} is a value between |0| and |1|.

设置左侧的宽度为所有可用宽度的比例 \meta{fraction}。
\meta{fraction} 取值在 |0| 到 |1|。


\begin{dispExample}
\tcbset{colback=red!5!white,colframe=red!75!black,fonttitle=\bfseries}

\begin{tcolorbox}[title=My title,sidebyside,lefthand ratio=0.25]
这是 upper (\textit{左}) 部分.
\tcblower
这是 lower (\textit{右}) 部分.
\end{tcolorbox}
\end{dispExample}
\end{docTcbKey}


\begin{docTcbKey}{righthand ratio}{=\meta{fraction}}{no default, initially |0.5|}

Sets the width of the right-handed part to the given \meta{fraction} of
the available space. \meta{fraction} is a value between |0| and |1|.

设置右侧的宽度为所有可用宽度的比例 \meta{fraction}。
\meta{fraction} 取值在 |0| 到 |1|。


\begin{dispExample}
\tcbset{colback=red!5!white,colframe=red!75!black,fonttitle=\bfseries}
\begin{tcolorbox}[title=My title,sidebyside,righthand ratio=0.25]
这是 upper (\textit{左}) 部分.
\tcblower
这是 lower (\textit{右}) 部分.
\end{tcolorbox}
\end{dispExample}
\end{docTcbKey}


% \clearpage

If one side of a side-by-side box should be adapted to the width of its content, 
this width has to be computed beforehand.
The following example uses a savebox |\mysavebox| to store the picture to determine its width. 
A more convenient way to handle this task is to use the methods from \Fullref{subsec:sidebyside_xparse}.

如果一个并排的盒子的一边要适应其内容的宽度, %
这个宽度必须事先计算。%
下面的例子使用一个存储盒子 |\mysavebox| 来存储图片以确定其宽度。%
处理这个任务更方便的方法是使用来自 \Fullref{subsec:sidebyside_xparse} 的方法。

\begin{引述之言}{virhuiai}
可以在code中处理,见下面的例子。
\end{引述之言}
% code={\sbox{\mysavebox}{#2}},
% lefthand width=\wd\mysavebox,

\begin{dispExample}
% \tcbuselibrary{skins,xparse}
% \usepackage{lipsum}
% \newsavebox\mysavebox  % preamble
\DeclareTotalTColorBox{\mysidebox}{ O{} +m +m }{
  bicolor,colback=white,colbacklower=yellow!10,
  fonttitle=\bfseries,center title,
  sidebyside,
  code={\sbox{\mysavebox}{#2}},
  lefthand width=\wd\mysavebox,
  drop lifted shadow,
  #1
}
{\usebox{\mysavebox}\tcblower#3}

\mysidebox[title=The Triangle]{%
  \begin{tikzpicture}
    \path[fill=red!20,draw=red!50!black]
      (0,0) node[below]{A} -- (3,1) node[right]{B}
      -- (1,4) node[above]{C} -- cycle;
  \end{tikzpicture}%
}{%
  \lipsum[1]
} 
\end{dispExample}% % code={\sbox{\mysavebox}{#2}}, % lefthand width=\wd\mysavebox, 学习了
% \clearpage
% Advanced Settings from the \mylib{xparse} Library
\subsection{来自\mylib{xparse}库的高级设置}\label{subsec:sidebyside_xparse}

\begin{marker}

All following macros and options need the \mylib{xparse} library to be
loaded, see \Fullref{sec:xparse}.

所有下面的宏和选项都需要加载 \mylib{xparse}库,见\Fullref{sec:xparse}。

\end{marker}


\begin{docCommand}[doc new=2015-11-20]{tcbsidebyside}{\oarg{options}\marg{left-handed content}\marg{right-handed content}}

Creates a colored box using more or less arbitrary \meta{options} for a \refEnvLe{tcolorbox}.
The \refKeyLe{/tcb/sidebyside} option is set to |true| and the \meta{left-handed content} and \meta{right-handed content} is filled into the box appropriately.
The resulting box is unbreakable.

% 为\refEnvLe{tcolorbox}指定任意的选项\meta{options}创建一个 |tcolorbox| 盒子。%
% \refKeyLe{/tcb/sidebyside}选项被设置为|true|, \meta{left-handed content}和\meta{right-handed content}被适当地填充到盒子中。%
% 这样的盒子是不可分页的。

使用更多或更少任意的\meta{选项}创建一个带有\refEnvLe{tcolorbox}的盒子。
\refKey{/tcb/sidebyside}选项设置为|true|,\meta{左侧内容}和\meta{右侧内容}被适当地填充到盒子中。
生成的盒子不可分割。



\refComLe{tcbsidebyside} is not only a shortcut for using a normal \refEnvLe{tcolorbox} with \refKeyLe{/tcb/sidebyside}, 
but allows setting further options like \refKeyLe{/tcb/sidebyside adapt} and \refKeyLe{/tcb/sidebyside switch}.

% \refComLe{tcbsidebyside}不只是使用普通的\refEnvLe{tcolorbox}指定\refKeyLe{/tcb/sidebyside}的快捷方式, 
% 还允许设置更多的选项,如\refKeyLe{/tcb/sidebyside adapt}和\refKeyLe{/tcb/sidebyside switch}。
\refComLe{tcbsidebyside}不仅是使用普通\refEnvLe{tcolorbox}与\refKeyLe{/tcb/sidebyside}的快捷方式,还允许设置其他选项,如\refKeyLe{/tcb/sidebyside adapt}和\refKeyLe{/tcb/sidebyside switch}。

\begin{dispExample}
% \tcbuselibrary{skins,xparse}
% \usepackage{lipsum}
\tcbsidebyside[title=The Triangle,
  sidebyside adapt=left,
  bicolor,colback=white,colbacklower=yellow!10,
  fonttitle=\bfseries,center title,drop lifted shadow,
]{%
  \begin{tikzpicture}
    \path[fill=red!20,draw=red!50!black]
      (0,0) node[below]{A} -- (3,1) node[right]{B}
      -- (1,4) node[above]{C} -- cycle;
  \end{tikzpicture}%
}{%
  \lipsum[1]
}
\end{dispExample}

\begin{dispExample*}{}
\tcbsidebyside[title={sidebyside adapt=left \hfill 左侧变成能不换行的了---virhuiai},
sidebyside adapt=left,
bicolor,colback=white,colbacklower=yellow!10,
fonttitle=\bfseries,center title,drop lifted shadow,
]{农夫倦步长道回家,

仅余我与暮色平分此世界。}{The ploughman

homeward plods his weary way,

And leaves the world

to darkness and to me.
}
\end{dispExample*}
\end{docCommand}%\subsection{来自\mylib{xparse}库的高级设置}
 

% \clearpage
\begin{docTcbKey}[][doc new=2015-11-20]{sidebyside adapt}{=\meta{side(s)}}{no default, initially |none|}

The option allows the left-handed and/or right-handed side to determine the dimensions of the box. 
This option is only valid inside \refComLe{tcbsidebyside}.

该选项允许根据左边和/右边内容确定盒子的尺寸。
此选项仅在\refComLe{tcbsidebyside}内有效。


Feasible values for \meta{side(s)} are:
\begin{itemize}
    \item\docValue{none}: 

no measurement of left-handed and right-handed side.

不测量左边和右边。

    \item\docValue{left}:

the actual width of the left-handed content is used to set \refKeyLe{/tcb/lefthand width}.

左侧内容侧实际宽度用于设置 \refKeyLe{/tcb/lefthand width}。

    \item\docValue{right}:

the actual width of the right-handed content is used to set \refKeyLe{/tcb/righthand width}.

右侧内容的实际宽度用于设置\refKeyLe{/tcb/右手宽度}。


    \item\docValue{both}:

the actual width of the left-handed and right-handed content is used to set \refKeyLe{/tcb/lefthand width},  \refKeyLe{/tcb/righthand width}, and the overall \refKeyLe{/tcb/width}.

左侧和右侧内容的实际宽度用于设置 \refKeyLe{/tcb/lefthand width} 和 \refKeyLe{/tcb/righthand width},%
和整个\refKeyLe{/tcb/width}。


\end{itemize}

\begin{dispExample}
% \tcbuselibrary{skins,xparse}
\tcbsidebyside[sidebyside adapt=left,
  title=非常重要的表格,
  beamer,colframe=blue!50!black,colback=blue!10
  ,lower separated=false%无分隔线了
  ,sidebyside gap=5mm
]{%
  \begin{tabular}{|l|c|r|}\hline
    left & center & right\\\hline
    A & B & C\\\hline
    D & E & F\\\hline
  \end{tabular}
}{%
此表包含所有未来操作的最重要数据。
你可能注意到B跟在A后面,C跟在B后面,以此类推。
}
\end{dispExample}



\begin{dispExample}
% \tcbuselibrary{skins,xparse}
\tcbsidebyside[sidebyside adapt=right,
  blanker,sidebyside gap=5mm
]{%
  \lipsum[2]
}{%
\begin{tikzpicture}
  \path[fill=yellow,draw=yellow!75!red] (0,0) circle (1cm);
  \fill[red] (45:5mm) circle (1mm);
  \fill[red] (135:5mm) circle (1mm);
  \draw[line width=1mm,red] (215:5mm) arc (215:325:5mm);
\end{tikzpicture}
}
\end{dispExample}


\begin{dispExample}
% \tcbuselibrary{skins,xparse}
\tcbsidebyside[sidebyside adapt=both,
  enhanced,center,
  title=Both sides adapted,
  attach boxed title to top center={yshift=-2mm},
  coltitle=black,boxed title style={colback=red!25},
  segmentation style=solid,colback=red!5,colframe=red!50
]{%
  \begin{tabular}{|l|c|r|}\hline
    left & center & right\\\hline
    A & B & C\\\hline
    D & E & F\\\hline
  \end{tabular}
}{%
\begin{tikzpicture}
  \path[fill=yellow,draw=yellow!75!red] (0,0) circle (1cm);
  \fill[red] (45:5mm) circle (1mm);
  \fill[red] (135:5mm) circle (1mm);
  \draw[line width=1mm,red] (215:5mm) arc (215:325:5mm);
\end{tikzpicture}
}
\end{dispExample}
\end{docTcbKey}



% \clearpage
\begin{docTcbKey}[][doc new=2015-11-20]{sidebyside switch}{\colOpt{=true\textbar false}}{default |true|, initially |false|}

If set to |true|, the
\meta{left-handed content} and \meta{right-handed content}
of \refComLe{tcbsidebyside} are switched.
Obviously, this option is only valid inside \refComLe{tcbsidebyside}.

如果设置为|true|,%
\refComLe{tcbsidebyside}的 \meta{left-handed content} 和 \meta{right-handed content} 会被切换。%
显然,这个选项只在\refComLe{tcbsidebyside}内有效。



The side switching can be made even/odd page sensitive, 
if used inside \refKeyLe{/tcb/if odd page}.

如果指定了了 \refKeyLe{/tcb/if odd page},两侧切换对奇/偶页敏感。



\begin{dispExample}
% \tcbuselibrary{skins,xparse}
\tcbsidebyside{Left}{Right}

\tcbsidebyside[sidebyside switch]{Left}{Right}

\tcbsidebyside[title=Very important table,
  if odd page={sidebyside switch,sidebyside adapt=right,flushright title}%
              {sidebyside adapt=left},
  beamer,colframe=blue!50!black,colback=blue!10,
  lower separated=false,sidebyside gap=5mm
]{%
  \begin{tabular}{|l|c|r|}\hline
    left & center & right\\\hline
    A & B & C\\\hline
    D & E & F\\\hline
  \end{tabular}
}{%
  This table contains the most important figures for
  all future actions. You may notice that B follows A,
  C follows B, and so on.
}
\end{dispExample}


\end{docTcbKey} 

%译 done 2023.1.23 lefthand width=\wd\mysavebox
% \setcounter{section}{6}
% \setcounter{subsection}{2}
% \setcounter{subsubsection}{0}
% \include*{tcolorbox.doc.verbatim}%译 done 2023.1.23|v2 20230325
% \setcounter{section}{7}
% \setcounter{subsection}{0}
% \setcounter{subsubsection}{0}
% \include*{tcolorbox.doc.recording}%%译 done 2023.1.24|v2 20230325 v3 2023 0924
% \setcounter{section}{8}
% \setcounter{subsection}{4}
% \setcounter{subsubsection}{0}
% \include*{tcolorbox.doc.technical}%todo 不必要,如何只是使用
\setcounter{section}{9}
\setcounter{subsection}{1}
\setcounter{subsubsection}{0}
 
\include*{tcolorbox.doc.skins}%todo 翻译了一些 TODo

% \setcounter{section}{10}
% \input{tcolorbox.doc.skincatalog}%todo
% \setcounter{section}{11}
% \input{tcolorbox.doc.graphics}%todo 199 lines
% \setcounter{section}{12}
% \input{tcolorbox.doc.filling}%todo
% \setcounter{section}{13}
% \input{tcolorbox.doc.beamer}%todo


% \include*{tcolorbox.doc.vignette}%v1 2023 0325


% % !TeX root = tcolorbox.tex
% include file of tcolorbox.tex (manual of the LaTeX package tcolorbox)
% \clearpage
\setcounter{section}{15}
\section{Library \mylib{raster}}\label{sec:raster}%
\tcbset{external/prefix=external/raster_}%
The library is loaded by a package option or inside the preamble by:

该库可以通过包选项或在导言中加载:
\begin{dispListing}
\tcbuselibrary{raster}
\end{dispListing}
%This also loads the package |xparse| \cite{latexproject:xparse}.
%\\%这也会加载包|xparse| \cite{latexproject:xparse}。

%The purpose of this library is to give comfortable access to the
%powerful document command production with |xparse| for |tcolorbox|.
%See the |xparse| package documentation \cite{latexproject:xparse}
%for details about the argument \meta{specification} used in this section.
%\\该库的目的是为了方便地访问使用|xparse|为|tcolorbox|生成强大文档命令。有关本节中使用的参数\meta{specification}的详细信息,请参阅|xparse|包文档 \cite{latexproject:xparse}。

% \include*{raster/栅格的概念}
% \include*{raster/库的宏}
% \include*{raster/库的选项键}
% \include*{raster/为特定盒子添加样式}
\include*{raster/合并列或行}
\end{document}

\subsection{Rasters inside Rasters}\label{subsec:raster_insideraster}

A \emph{raster} inside a \emph{raster} cannot be used directly, because
a \emph{raster} can only contain a \emph{tcolorbox} or something
derived from a \emph{tcolorbox}. So, a \emph{raster} can be put inside
a \emph{tcolorbox} inside a \emph{raster}.

Some examples for such constructions can be found at \refEnvLe{tcboxedraster},
\refKeyLe{/tcb/raster multicolumn},
\refKeyLe{/tcb/raster multirow}.


\subsubsection{Raster Setup}
The intermediating \refEnvLe{tcolorbox} can be made invisible by using
\refKeyLe{/tcb/blankest}.

\begin{dispExample}
\begin{tcbraster}[raster equal height=rows,
  raster every box/.style={colframe=red!50!black,colback=red!10!white}]
  \begin{tcolorbox}[blankest]
    \begin{tcbraster}[raster columns=1]
      \begin{tcolorbox}One\end{tcolorbox}
      \begin{tcolorbox}Two\end{tcolorbox}
    \end{tcbraster}
  \end{tcolorbox}
  \begin{tcolorbox}raster+tcolorbox+raster\end{tcolorbox}
\end{tcbraster}
\end{dispExample}

\enlargethispage*{1cm}
\begin{dispExample}
\begin{tcbraster}[raster equal height=rows,
  raster every box/.style={colframe=red!50!black,colback=red!10!white}]
  \begin{tcboxedraster}[raster columns=1]{blankest}
    \begin{tcolorbox}One\end{tcolorbox}
    \begin{tcolorbox}Two\end{tcolorbox}
  \end{tcboxedraster}
  \begin{tcolorbox}raster+tcboxedraster\end{tcolorbox}
\end{tcbraster}
\end{dispExample}


\begin{dispExample}
\begin{tcbitemize}[raster equal height=rows,
  raster every box/.style={colframe=red!50!black,colback=red!10!white}]
  \tcbitem[blankest]
    \begin{tcbitemize}[raster columns=1]
      \tcbitem One
      \tcbitem Two
    \end{tcbitemize}
  \tcbitem tcbitemize+tcbitem+tcbitemize
\end{tcbitemize}
\end{dispExample}


\subsubsection{Placing Spaces}
If the heights of boxes inside staggered rasters should be matched, the
space has to be distributed accordingly.

\begin{itemize}
\item For fixed height boxes/rasters using \refKeyLe{/tcb/raster height},
  the height of boxes is available by \refComLe{tcbtextheight}. This can be
  used to size deeper layered boxes/rasters.
\item For boxes/rasters layed out using \refKeyLe{/tcb/raster equal height},
  space can be distributed by \refKeyLe{/tcb/space to}.
  It can take several compilations until all spaces are distributed correctly.
\end{itemize}


\begin{dispExample}
\begin{tcbitemize}[raster rows=2,raster height=6cm,
  raster every box/.style={colframe=red!50!black,colback=red!10!white}]
  \tcbitem[blankest]
    \begin{tcbitemize}[raster columns=1,raster rows=2,raster height=\tcbtextheight]
      \tcbitem One
      \tcbitem Two
    \end{tcbitemize}
  \tcbitem This is a fixed height box.
  \tcbitem Three
  \tcbitem Four
\end{tcbitemize}
\end{dispExample}


\begin{dispExample}
\begin{tcbitemize}[raster columns=4,raster rows=4,raster height=0.8\linewidth,
  raster every box/.style={size=small,beamer,
    colframe=blue!75!yellow,colback=red!75!yellow!20,
    center title,title=Box}]
  \tcbitem One
  \tcbitem Two
  \tcbitem Three
  \tcbitem Four
  \tcbitem[raster multirow=2,blankest]
    \begin{tcbitemize}[raster columns=1,raster rows=2,raster height=\tcbtextheight]
      \tcbitem Twelve
      \tcbitem Eleven
    \end{tcbitemize}
  \tcbitem[raster multirow=2,raster multicolumn=2,
      colframe=red!75!yellow,colback=blue!75!yellow!20]
    This is an example with fixed height boxes.
  \tcbitem[raster multirow=2,blankest]
    \begin{tcbitemize}[raster columns=1,raster rows=2,raster height=\tcbtextheight]
      \tcbitem Five
      \tcbitem Six
    \end{tcbitemize}
  \tcbitem Ten
  \tcbitem Nine
  \tcbitem Eight
  \tcbitem Seven
\end{tcbitemize}
\end{dispExample}


\begin{dispExample}
\begin{tcbitemize}[raster equal height=rows,
  raster every box/.style={colframe=red!50!black,colback=red!10!white}]
  \tcbitem[blankest,space to=\myspace]
    \begin{tcbitemize}[raster columns=1]
      \tcbitem One
      \tcbitem[add to natural height=\myspace]
        This box will adapt its height.
    \end{tcbitemize}
  \tcbitem This is a flexible height box.
  \tcbitem \lipsum[4]
  \tcbitem[blankest,space to=\myspace]
    \begin{tcbitemize}[raster columns=1]
      \tcbitem One
      \tcbitem[add to natural height=\myspace]
        This box will adapt its height.
    \end{tcbitemize}
\end{tcbitemize}
\end{dispExample}



\begin{dispExample}
\begin{tcbitemize}[raster equal height=rows,
  raster every box/.style={colframe=red!50!black,colback=red!10!white}]
  \tcbitem[blankest,space to=\myspace]
    \begin{tcbitemize}[raster columns=1]
      \tcbitem One
      \tcbitem[add to natural height=\myspace]
        This box will adapt its height.
      \tcbitem \lipsum[4]
    \end{tcbitemize}
  \tcbitem[blankest,space to=\myspace]
    \begin{tcbitemize}[raster columns=1]
      \tcbitem[blankest]\includegraphics[width=\linewidth]{goldshade.png}
      \tcbitem[add to natural height=\myspace]
        This box will adapt its height.
    \end{tcbitemize}
\end{tcbitemize}
\end{dispExample}



% v1 20230328
% \setcounter{section}{16}
% % % !TeX root = tcolorbox.tex
% % include file of tcolorbox.tex (manual of the LaTeX package tcolorbox)
% \clearpage
\section{Libraries
  \mylib{listings},
  \mylib{listingsutf8}, and
  \mylib{minted}}\label{sec:listings}%
\tcbset{external/prefix=external/listings_}%

% \subsection{Loading the Libraries\\加载库}
% \subsection{Loading the Libraries\\加载库}

In contrast to other |tcolorbox| libraries, the libraries
\mylib{listings}, \mylib{listingsutf8}, and \mylib{minted} are concurrent in the sense that
they all do the same thing, i.\,e.\ displaying listings with or without typesetting
the listing in \LaTeX\ parallel.
The difference is the underlying \LaTeX\ package which does the core job for
displaying a listing. So, typically, you need just \emph{one} of these
libraries. If you do not have a clue which one of them you should use
and you are using |pdflatex|, you should take \mylib{listingsutf8}.
If you are using |xelatex| or |lualatex|, you should take \mylib{listings}
as |xelatex| and |lualatex| are not compatible with \mylib{listingsutf8}.

与其他 |tcolorbox| 库不同的是,库 \mylib{listings}、\mylib{listingsutf8} 和 \mylib{minted} 在并行方面是相互的,也就是说,它们都可以在 \LaTeX\ 中并行地显示代码,无论是否排版代码。 它们的区别在于执行显示代码的基础 \LaTeX\ 包。因此,通常情况下,您只需要使用其中的 \emph{一个} 库。如果您不知道该使用哪个库,并且您使用的是 |pdflatex|,那么您应该选择 \mylib{listingsutf8}。如果您使用的是 |xelatex| 或 |lualatex|,则应该选择 \mylib{listings},因为 |xelatex| 和 |lualatex| 与 \mylib{listingsutf8} 不兼容。



\begin{marker}
The order in which the libraries are included influences the default settings and
the \refKeyLe{/tcb/reset} behavior. The settings of a later loaded library overwrite
the settings of a previous loaded library. A library is never loaded twice.

库的包含顺序影响默认设置和\refKeyLe{/tcb/reset}行为。后加载的库的设置会覆盖先前加载的库的设置。库永远{\bf 不会被重复加载}。
\end{marker}

% \subsubsection{Loading \mylib{listings}\\加载 \mylib{listings}}

This library uses the package |listings| \cite{hoffmann:listings} to typeset
listings. It is loaded by a package option or inside the preamble by:

这个库使用 |listings| \cite{hoffmann:listings} 包来排版代码清单。可以通过包选项或在导言区内加载它:
\begin{dispListing}
\tcbuselibrary{listings}
\end{dispListing}
This also loads the package |listings| \cite{hoffmann:listings}.

这也加载了 |listings| 包 \cite{hoffmann:listings}。

The \refKey{/tcb/listing engine} is set to |listings| by the library.
To reactivate this setting, if overwritten by other libraries, use

库将\refKey{/tcb/listing engine}设置为|listings|。 如果被其他库覆盖并想要重新激活此设置,请使用
\begin{dispListing}
\tcbset{listing engine=listings}
\end{dispListing}

\subsubsection{Loading \mylib{listingsutf8}\\加载 \mylib{listingsutf8}}
\begin{marker}
This library is not needed (and troublesome) when using Xe\LaTeX\ or Lua\LaTeX.
Therefore, loading this library is automatically replaced by loading
\mylib{listings} only, if pdf\LaTeX\ is \emph{not} used.

当使用 Xe\LaTeX\ 或 Lua\LaTeX\ 时,这个库是不必要的(而且麻烦的)。 因此,如果不使用 pdf\LaTeX,自动将加载这个库的操作替换为仅加载 \mylib{listings}。
\end{marker}
To extend |listings| for UTF-8 encoded sources, you can use the support from
the package |listingsutf8| \cite{oberdiek:listingsutf8} by loading the library
variant \mylib{listingsutf8}.

要扩展 |listings| 以支持 UTF-8 编码的源代码,可以使用包 |listingsutf8| 的支持 \cite{oberdiek:listingsutf8},并加载库变体 \mylib{listingsutf8}。
\begin{dispListing}
\tcbuselibrary{listingsutf8}
\tcbset{listing utf8=latin1}% optional; `latin1' is the default.
\end{dispListing}
This also loads the library \mylib{listings}
and the packages |listings| \cite{hoffmann:listings}
and |listingsutf8| \cite{oberdiek:listingsutf8}.

这也会加载库\mylib{listings}以及包|listings|\cite{hoffmann:listings}和|listingsutf8|\cite{oberdiek:listingsutf8}。

The \refKey{/tcb/listing engine} is set to |listings| by the library.
To reactivate this setting, if overwritten by other libraries, use

库将\refKey{/tcb/listing engine}设置为|listings|。 如果被其他库覆盖,请使用以下方法重新激活此设置:
\begin{dispListing}
\tcbset{listing engine=listings}
\end{dispListing}


%% \clearpage
\subsubsection{Loading \mylib{minted}\\加载\mylib{minted}}
This library uses the package |minted| \cite{poore:minted} to typeset
listings. It is loaded by a package option or inside the preamble by:

本库使用|minted|\cite{poore:minted}包来排版代码。可以通过包选项或在导言区中加载该包:
\begin{dispListing}
\tcbuselibrary{minted}
\end{dispListing}
This also loads the package |minted| \cite{poore:minted}.

这也会加载 |minted| 包 \cite{poore:minted}。
\begin{marker}
The |minted| package uses the external tool |Pygments| \cite{pygments:web}
to apply syntax highlighting. It has to be installed and set up, before the
library can be used, see \cite{poore:minted} and \cite{pygments:web}.
The |tcolorbox| library \mylib{minted} does not work, if the package
|minted| \cite{poore:minted} does not work.

|minted|宏包使用外部工具|Pygments|\cite{pygments:web}来进行语法高亮。在使用该库之前,必须先安装和设置它,参见\cite{poore:minted}和\cite{pygments:web}。如果|minted|\cite{poore:minted}宏包无法正常工作,则\mylib{minted} |tcolorbox|库也无法正常工作。
\end{marker}

The \refKey{/tcb/listing engine} is set to |minted| by the library.
To reactivate this setting, if overwritten by other libraries, use

该库将\refKey{/tcb/listing engine}设置为|minted|。 如果被其他库覆盖,要重新激活此设置,请使用:
\begin{dispListing}
\tcbset{listing engine=minted}
\end{dispListing}
% \subsection{Common Macros of the Libraries\\库的常用宏}
% \begin{docEnvironment}{tcblisting}{\marg{options}}
Creates a colored box based on a \refEnv{tcolorbox}.
Controlled by the given \meta{options}, the
environment content is typeset normally and/or as a listing.
Furthermore, the \meta{options} control appearance and functions of
the |tcolorbox|.
By default, the listing is interpreted as a \LaTeX\ listing.

创建一个基于\refEnv{tcolorbox}的彩色框。 由给定的\meta{options}控制,环境内容可以正常排版或作为一个列表显示。 此外,\meta{options}还控制|tcolorbox|的外观和功能。 默认情况下,列表被解释为一个\LaTeX\ 列表。
\begin{dispExample}
\begin{tcblisting}{colback=red!5!white,colframe=red!75!black}
This is a \LaTeX\ example which displays the text as source code
and in compiled form.

这是一个 \LaTeX\ 示例,它以源代码形式和编译后的形式显示文本。
\end{tcblisting}
\end{dispExample}

% %\clearpage

\begin{dispExample}
% \tcbuselibrary{listings} /or/ \tcbuselibrary{listingsutf8}
\begin{tcblisting}{colback=yellow!5,colframe=yellow!50!black,listing only,
title=This is source code in another language (XML), fonttitle=\bfseries,
listing options={language=XML,columns=fullflexible,keywordstyle=\color{red}}}
<?xml version="1.0"?>
<project name="Package tcolorbox" default="documentation" basedir=".">
<description>
Apache Ant build file (http://ant.apache.org/)
</description>
</project>
\end{tcblisting}
\end{dispExample}

% \enlargethispage*{10mm}

\begin{dispExample}
% \tcbuselibrary{minted}
\begin{tcblisting}{colback=yellow!5,colframe=yellow!50!black,listing only,
title=This is source code in another language (XML), fonttitle=\bfseries,
listing engine=minted,minted language=xml}
<?xml version="1.0"?>
<project name="Package tcolorbox" default="documentation" basedir=".">
<description>
Apache Ant build file (http://ant.apache.org/)
</description>
</project>
\end{tcblisting}
\end{dispExample}



\begin{dispExample}
% This box is as wide as needed (listing only !!)
% \tcbuselibrary{skins}
\begin{tcblisting}{colback=green!5!white,colframe=green!50!black,listing only,
hbox,enhanced,drop fuzzy shadow,before=\begin{center},after=\end{center}}
\begin{tikzpicture}
\fill[red] (0,0) rectangle (1,1);
\end{tikzpicture}
\end{tcblisting}
\end{dispExample}
\end{docEnvironment}


% \clearpage
\begin{docEnvironment}{tcboutputlisting}{}
Saves the environment content to a file which is named by the key value of
|listing file|. Later, this file can be loaded by
|\tcbinputlisting| or |\tcbuselistingtext| or |\tcbuselistinglisting|.

将环境内容保存到一个文件中,文件名由 |listing file| 的键值命名。稍后,此文件可以由 |\tcbinputlisting|、|\tcbuselistingtext| 或 |\tcbuselistinglisting| 加载。
\begin{dispListing}
\begin{tcboutputlisting}
This \textbf{text} is written to a standardized file for later usage.

这个\textbf{text}是写入一个标准化文件以便以后使用。
\end{tcboutputlisting}
\end{dispListing}
\end{docEnvironment}


\begin{docCommand}{tcbinputlisting}{\marg{options}}
Creates a colored boxed based on a |tcolorbox|. The text content is read
from a file named by the key value of |listing file|. Apart from that,
the function is equal to that of \refEnv{tcblisting}.

根据 |tcolorbox| 创建一个有颜色的框。文本内容从名为 |listing file| 的键值对应的文件中读取。除此之外,该函数与 \refEnv{tcblisting} 的功能相同。
\begin{dispExample}
\tcbinputlisting{colback=red!5!white,colframe=red!75!black,text only}
\tcbinputlisting{colback=green!5,colframe=green!75!black,listing only}
\end{dispExample}
\end{docCommand}

\begin{docCommand}{tcbuselistingtext}{}
Loads text from a file named by the key value of |listing file|.

从名为|listing file|的键值指定的文件中加载文本。
\begin{dispExample}
\tcbuselistingtext
\end{dispExample}
\end{docCommand}


\begin{docCommand}{tcbuselistinglisting}{}
Typesets text as listing from a file named by the key value of |listing file|.

将文本设置为列表,列表文件的名称由|listing file|的键值命名。
\begin{dispExample}
\tcbuselistinglisting
\end{dispExample}
\end{docCommand}

%\enlargethispage*{5mm}
\begin{docCommand}{tcbusetemplisting}{}
Typesets text as listing from a temporary file which was written by

将文本设置为列表,该列表来自由\refEnv{tcbwritetemp}写入的临时文件。
\refEnv{tcbwritetemp}.
\end{docCommand}


% \clearpage
\begin{marker}
See \Vref{subsec:xparse_listing} and \Vref{subsec:xparse_inputlisting} for more
elaborate methods to create new environments and commands.

请参见 \Vref{subsec:xparse_listing} 和 \Vref{subsec:xparse_inputlisting},了解更详细的创建新环境和命令的方法。
\end{marker}
\begin{marker}
If a new sort of |tcblisting| environments should be created with
one optional argument only, one is highly recommended to use
\refCom{DeclareTCBListing} or \refCom{NewTCBListing}
instead of \refCom{newtcblisting} to
avoid content scanning problems.

如果要创建一种只有一个可选参数的新 |tcblisting| 环境,强烈建议使用 \refCom{DeclareTCBListing} 或 \refCom{NewTCBListing} 而不是 \refCom{newtcblisting},以避免内容扫描问题。
\end{marker}

\begin{docCommand}{newtcblisting}{\oarg{init options}\marg{name}\oarg{number}\oarg{default}\marg{options}}
Creates a new environment \meta{name} based on \refEnv{tcblisting}.
Basically, |\newtcblisting| operates like |\newenvironment|. This means,
the new environment \meta{name} optionally takes \meta{number} arguments, where
\meta{default} is the default value for the optional first argument.
The \meta{options} are given to the underlying |tcblisting|.
Note that \refKey{/tcb/savedelimiter} is set to the given \meta{name}
automatically.
The \meta{init options} allow setting up automatic numbering,
see Section \ref{sec:initkeys} from page \pageref{sec:initkeys}.

创建基于\refEnv{tcblisting}的新环境\meta{name}。 基本上,|\newtcblisting|的操作类似于|\newenvironment|。这意味着,新环境\meta{name}可以选择性地接受\meta{number}个参数,其中\meta{default}是可选第一个参数的默认值。 \meta{options}应用于底层的|tcblisting|。请注意,\refKey{/tcb/savedelimiter}会自动设置为给定的\meta{name}。 \meta{init options}允许设置自动编号,参见第\pageref{sec:initkeys}页的第\ref{sec:initkeys}节。
\begin{dispExample*}{sbs,lefthand ratio=0.5}
\newtcblisting{mybox}{%
colback=red!5!white,
colframe=red!75!black}

\begin{mybox}
This is my \LaTeX\ box.
\end{mybox}
\end{dispExample*}

\begin{dispExample*}{sbs,lefthand ratio=0.5}
\newtcblisting{mybox}[1]{%
colback=red!5!white,
colframe=red!75!black,
fonttitle=\bfseries,
title={#1}}

\begin{mybox}{Listing Box}
This is my \LaTeX\ box.
\end{mybox}
\end{dispExample*}

\begin{dispExample*}{sbs,lefthand ratio=0.5}
\newtcblisting{mybox}[2][]{%
colback=red!5!white,
colframe=red!75!black,
fonttitle=\bfseries,
title={#2},#1}

\begin{mybox}[listing only]
{Listing Box}
This is my \LaTeX\ box.
\end{mybox}
\bigskip

\begin{mybox}[listing side text]
{Listing Box}
This is my
\LaTeX\ box.
\end{mybox}
\end{dispExample*}

% \clearpage
\inputpreamblelisting{C}

\begin{dispExample*}{sbs,lefthand ratio=0.5}
\begin{mycbox}{Listing Box}
This is my \LaTeX\ box.
\end{mycbox}
\end{dispExample*}
\end{docCommand}


%\enlargethispage*{1cm}
\begin{docCommand}{renewtcblisting}{\oarg{init options}\marg{name}\oarg{number}\oarg{default}\marg{options}}
Operates like \refCom{newtcblisting}, but based on |\renewenvironment| instead of |\newenvironment|.
An existing environment is redefined.

类似于 \refCom{newtcblisting},但是基于 |\renewenvironment| 而非 |\newenvironment| 运行。 已有的环境将被重新定义。
\end{docCommand}


% \clearpage
\begin{docCommand}{newtcbinputlisting}{\oarg{init options}\brackets{\texttt{\textbackslash}\rmfamily\meta{name}}\oarg{number}\oarg{default}\marg{options}}
Creates a new macro \texttt{\textbackslash}\meta{name} based on \refCom{tcbinputlisting}.
Basically, |\newtcbinputlisting| operates like |\newcommand|.
The new macro \texttt{\textbackslash}\meta{name} optionally takes \meta{number} arguments, where
\meta{default} is the default value for the optional first argument.
The \meta{options} are given to the underlying |tcbinputlisting|.
The \meta{init options} allow setting up automatic numbering,
see Section \ref{sec:initkeys} from page \pageref{sec:initkeys}.

基于\refCom{tcbinputlisting}创建一个新的宏\texttt{\textbackslash}\meta{name}。 基本上,|\newtcbinputlisting| 的操作类似于 |\newcommand|。 新的宏\texttt{\textbackslash}\meta{name}可以选择性地带有\meta{number}个参数,其中\meta{default}是可选第一个参数的默认值。 \meta{options}被赋予基础的|tcbinputlisting|。 \meta{init options}允许设置自动编号,参见第\pageref{sec:initkeys}页的第\ref{sec:initkeys}节。
\begin{dispExample}
\newtcbinputlisting[use counter from=mycbox]{\mylisting}[2][]{%
listing file={#2},
title=Listing (\thetcbcounter) of \texttt{#2},
colback=red!5!white,colframe=red!75!black,fonttitle=\bfseries,
listing only,breakable,#1}

\mylisting[before upper=\textit{This is the included file content:}]
        {\jobname.tcbtemp}
\end{dispExample}

\begin{dispExample}
\newtcbinputlisting[use counter from=mycbox]{\mylisting}[2][]{%
listing engine=minted,minted language=latex,minted style=colorful,
listing file={#2},
title=Listing (\thetcbcounter) of \texttt{#2},
colback=red!5!white,colframe=red!75!black,fonttitle=\bfseries,
listing only,breakable,#1}

\mylisting[before upper=\textit{This is the included file content:}]
        {\jobname.tcbtemp}
\end{dispExample}
\end{docCommand}


\begin{docCommand}{renewtcbinputlisting}{\oarg{init options}\brackets{\texttt{\textbackslash}\rmfamily\meta{name}}\oarg{number}\oarg{default}\marg{options}}
Operates like \refCom{newtcbinputlisting}, but based on |\renewcommand| instead of |\newcommand|.
An existing macro is redefined.

类似于 \refCom{newtcbinputlisting},但是基于 |\renewcommand| 而不是 |\newcommand|。 现有的宏被重新定义。
\end{docCommand}


%% \clearpage
% \subsection{Option Keys of the \mylib{listings} Library\\listings库的选项键}\label{sec:speclistingkeys}
% \subsection{Option Keys of the \mylib{listings} Library\\listings库的选项键}\label{sec:speclistingkeys}

\begin{docTcbKey}{listing options}{=\meta{key list}}{no default, initially |style=tcblatex|}
Sets the options from the package |listings| 
which are used during typesetting of the listing.
For \LaTeX\ listings, there is a predefined |listings| style named |tcblatex|
which can be used.

设置来自 |listings| 包的选项,这些选项在排版清单时使用。 对于 \LaTeX\ 清单,有一个预定义的 |listings| 样式名为 |tcblatex|,可以使用。
\begin{dispExample}
\begin{tcblisting}{colback=red!5!white,colframe=red!25,left=6mm,
listing options={style=tcblatex,numbers=left,numberstyle=\tiny\color{red!75!black}}}
This is a \LaTeX\ example which displays the text as source code
and in compiled form. Additionally, we use line numbers here.

这是一个 \LaTeX\ 的例子,它将文本显示为源代码和编译形式。此外,我们在这里使用行号。
\end{tcblisting}
\end{dispExample}
\end{docTcbKey}

\begin{docTcbKey}{no listing options}{}{no value, initially unset}
Abbreviation for |listing options={}|.
This removes all options for the |listings| package.
This includes the |tcblisting| standard style |tcblatex| and the encoding presets.
Use this option, if you want to set the |listings| options outside of |tcblisting|, e.\,g.\ globally in
the preamble.

|listing options={} |的缩写。 这将删除|listings|包的所有选项。 这包括|tcblisting|标准样式|tcblatex|和编码预设。 如果您想在|tcblisting|之外设置|listings|选项,例如在导言中全局使用,请使用此选项。
\begin{dispExample}
\begin{tcblisting}{no listing options}
All \textit{listings} options removed.

所有“列表选项”已被移除。
\end{tcblisting}
\end{dispExample}
\end{docTcbKey}


\begin{docTcbKey}{listing style}{=\meta{style}}{no default, initially |tcblatex|}
Abbreviation for |listing options={style=...}|. This key sets a \meta{style}
for the |listings| package, see .
For \LaTeX, there is a predefined style named |tcblatex|.

|listing options={style=...}|的缩写。此键设置|listings|包的\meta{style},请参见。对于\LaTeX,有一个预定义的样式名为|tcblatex|。
\begin{dispExample}
\begin{tcblisting}{colback=red!5!white,colframe=red!75!black,
listing style=tcblatex}
Here, we use the predefined style.
\end{tcblisting}
\end{dispExample}
\end{docTcbKey}

%% \clearpage
\begin{docTcbKey}{listing inputencoding}{=\meta{encoding}}{no default, initially \texttt{\cs inputencodingname}}
Sets the input encoding value for the predefined listing style |tcblatex|
and |tcbdocumentation| from the library \mylib{documentation}.
The initial value is derived from the package |inputenc| if used.

为预定义的列表样式 |tcblatex| 和 |tcbdocumentation| 设置输入编码值,来自库 \mylib{documentation}。初始值是从使用的包 |inputenc| 中得出的。
\end{docTcbKey}

\begin{docTcbKey}{listing remove caption}{\colOpt{=true\textbar false}}{default |true|, initially |true|}
If set to |true|, some part of the caption building code of the |listings| package
is silenced to prevent some unwanted interaction with the |hyperref| package resulting
in additional vertical space.
If set to |false|, the |listings| package code is kept unchanged.
Note that listings outside \refEnvLe{tcblisting} and
\refComLe{tcbinputlisting} are always processed normally.
Typically, a user is not expected to use this key at all.

如果将其设置为|true|,则会禁用列表|listings|包的一些标题构建代码,以防止与|hyperref|包产生一些不需要的交互,导致额外的垂直空间。 如果设置为|false|,则|listings|包的代码保持不变。 请注意,位于\refEnvLe{tcblisting}和\refComLe{tcbinputlisting}之外的列表始终以正常方式处理。 通常,用户不需要使用此设置。
\end{docTcbKey}

\begin{docTcbKey}{every listing line}{=\meta{text}}{no default, initially unset/empty}
Inserts some \meta{text} to the begin of every line of a listing.
Note that this a hack of the |listings| package code. This may become unusable
or superfluous in the future.

在代码清单的每一行开头插入一些\meta{text}。 请注意,这是|listings|包代码的一个hack。这可能在未来变得无用或多余。
\begin{dispExample}
\newtcblisting{commandshell}{colback=black,colupper=white,colframe=yellow!75!black,
listing only,listing options={style=tcblatex,language=sh},
every listing line={\textcolor{red}{\small\ttfamily\bfseries root \$> }}}

\begin{commandshell}
ls -al
cd /usr/lib
\end{commandshell}
\end{dispExample}
\end{docTcbKey}


\begin{docTcbKey}{every listing line*}{=\meta{text}}{no default, initially unset/empty}
Identical to \refKeyLe{/tcb/every listing line} plus additional enlargement
of \refKeyLe{/tcb/rightupper} by the width of \meta{text}. Therefore, this
option has to be used after the geometry settings are done.
This option is intended to be used in conjunction with \refKeyLe{/tcb/hbox}.

与\refKeyLe{/tcb/every listing line}完全相同,加上\meta{text}的宽度对\refKeyLe{/tcb/rightupper}进行额外放大。因此,必须在几何设置完成后使用此选项。此选项旨在与\refKeyLe{/tcb/hbox}一起使用。
\begin{dispExample}
\newtcblisting{commandshell}{colback=black,colupper=white,colframe=yellow!75!black,
listing only,listing options={style=tcblatex,language=sh},hbox,
every listing line*={\textcolor{red}{\small\ttfamily\bfseries root \$> }}}

\begin{commandshell}
ls -al
cd /usr/lib
\end{commandshell}
\end{dispExample}

\begin{dispExample}
\newtcblisting{commandshell}%
{colback=black,colupper=white,colframe=yellow!75!black%
,listing only,listing options={style=tcblatex,language=sh},hbox,
every listing line={\textcolor{green}{\small\ttfamily\bfseries virhuiai \textasciitilde{} \$> }}}

\begin{commandshell}
ls -al
cd /usr/lib
\end{commandshell}
\end{dispExample}
\end{docTcbKey}

See further options in \Vref{sec:commonlistingkeys}.

请参见 \Vref{sec:commonlistingkeys} 中的更多选项。
\begin{marker}
For an combined example of using |\lstinline| inside a |tcolorbox|, see
\refComLe{DeclareTotalTCBox}.

有关在|tcolorbox|中使用|\lstinline|的组合示例,请参见\refComLe{DeclareTotalTCBox}。
\end{marker}

%% \clearpage
% \subsection{Option Keys of the \mylib{listingsutf8} Library\\listingsutf8库的选项键}
% \begin{marker}
The \mylib{listingsutf8} library is not needed (and troublesome) when using Xe\LaTeX\ or Lua\LaTeX.
Therefore, loading this library is automatically replaced by loading
\mylib{listings} only, if pdf\LaTeX\ is \emph{not} used.

当使用 Xe\LaTeX\ 或 Lua\LaTeX 时,不需要(而且可能会有问题)加载 \mylib{listingsutf8} 库。 因此,如果不使用 pdf\LaTeX,加载此库会自动替换为仅加载 \mylib{listings}。
\end{marker}

The \mylib{listingsutf8} library is an extension of the
\mylib{listings} library, so
all options from \Vref{sec:speclistingkeys} are applicable.

\mylib{listingsutf8} 库是 \mylib{listings} 库的扩展,因此 \Vref{sec:speclistingkeys} 中的所有选项都适用。
\begin{docTcbKey}{listing utf8}{=\meta{one-byte-encoding}}{style, no default, initially |latin1|}
Abbreviation for using \refKeyLe{/tcb/listing inputencoding}
together with UTF-8 support from the package |listingsutf8| .
This option is available only for the library variant \mylib{listingsutf8}.
The \meta{one-byte-encoding} is one of
the applicable encodings from , e.\,g.\ |latin1|
which is the default.\par

% 使用\refKeyLe{/tcb/listing inputencoding}的缩写,结合来自包|listingsutf8|的UTF-8支持。此选项仅适用于库变体\mylib{listingsutf8}。\meta{one-byte-encoding}是来自的适用编码之一,例如|latin1|是默认值。

使用\refKeyLe{/tcb/listing inputencoding}和来自|listingsutf8|包的UTF-8支持的缩写。此选项仅适用于库变体\mylib{listingsutf8}。 \meta{one-byte-encoding} 是可适用的编码之一,例如|latin1|是默认值。

Be aware that this means restriction to this specific \meta{one-byte-encoding}:
e.\,g.\ |latin1| comprises umlauts and other accented characters, but not
the Euro sign. If you want to use the |listings| package \emph{and} \flqq real\frqq\
UTF-8 source code, then do \emph{not} use \mylib{listingsutf8} but \mylib{listings}
with
\refKeyLe{/tcb/listing inputencoding}|=utf8|
\emph{and} with specific manual hacks for specific UTF-8-encoded characters.

% 请注意,这意味着限制在特定的\meta{one-byte-encoding}上:例如|latin1|包括umlauts和其他重音字符,但不包括欧元符号。如果您想同时使用|listings|包和“真正的”UTF-8源代码,则不要使用\mylib{listingsutf8},而是使用\mylib{listings},并使用\refKeyLe{/tcb/listing inputencoding}|=utf8|以及针对特定UTF-8编码字符的手动修补。

请注意,这意味着限制于特定的 \meta{单字节编码}: 例如,|latin1| 包含变音符和其他有重音符号的字符,但不包括欧元符号。如果您想使用 |listings| 宏包并使用“真正”的 UTF-8 源代码,则不要使用 \mylib{listingsutf8},而应该使用带有 \refKeyLe{/tcb/listing inputencoding}|=utf8| 的 \mylib{listings},并且使用特定的手动修改来处理特定的 UTF-8 编码字符。
\end{docTcbKey}

See further options in \Vref{sec:commonlistingkeys}.

请参阅 \Vref{sec:commonlistingkeys} 中的其他选项。
%% \clearpage
% \subsection{Option Keys of the \mylib{minted} Library\\minted库的选项键}
% \begin{docTcbKey}{minted language}{=\meta{programming language}}{no default, initially |latex|}
Sets a \meta{programming language} known to |Pygments| \cite{pygments:web}.

设置一个被 |Pygments| \cite{pygments:web} 所知的 \meta{编程语言}。
\begin{dispExample}
\begin{tcblisting}{listing engine=minted,minted style=trac,
minted language=java,
colback=red!5!white,colframe=red!75!black,listing only}
public class HelloWorld {
    // A `Hello World' in Java
    public static void main(String[] args) {
    System.out.println("Hello World!");
    }
}
\end{tcblisting}
\end{dispExample}
\end{docTcbKey}


\begin{docTcbKey}[][doc updated={2021-12-15}]{minted options}{=\meta{key list}}{no default, initially
\linebreak see \refKey{/tcb/default minted options}}
Sets the options from the package |minted| \cite{poore:minted}
which are used during typesetting of the listing.
Also see \refKey{/tcb/minted options app} and \refKey{/tcb/minted options pre}.

设置使用 |minted| \cite{poore:minted} 包在列表排版期间使用的选项。另请参见 \refKey{/tcb/minted options app} 和 \refKey{/tcb/minted options pre}。
\begin{dispExample}
% \tcbuselibrary{skins}
\newtcblisting{myjava}{listing engine=minted,
minted style=colorful,
minted language=java,
minted options={fontsize=\small,breaklines,autogobble,linenos,numbersep=3mm},
colback=blue!5!white,colframe=blue!75!black,listing only,
left=5mm,enhanced,
overlay={\begin{tcbclipinterior}\fill[red!20!blue!20!white] (frame.south west)
rectangle ([xshift=5mm]frame.north west);\end{tcbclipinterior}}}

\begin{myjava}
public class HelloWorld {
// A `Hello World' in Java
public static void main(String[] args) {
    System.out.println("Hello World!");
}
}
\end{myjava}
\end{dispExample}
\end{docTcbKey}


% \clearpage
\begin{docTcbKey}[][doc new={2021-12-15}]{default minted options}{=\meta{key list}}{no default, initially
|tabsize=2,fontsize=\textbackslash small,|\linebreak|breaklines,autogobble|}
Sets the options from the package |minted| \cite{poore:minted}
which are used during typesetting of the listing, if
\refKey{/tcb/minted options} are \emph{not} used. The intended use is
inside the preamble to change the default behavior.
Note that setting \refKey{/tcb/default minted options} also resets \refKey{/tcb/minted options}.

如果未使用\refKey{/tcb/minted options},则从|minted|\cite{poore:minted}包设置选项,用于列表的排版。预期的用途是在导言中更改默认行为。请注意,设置\refKey{/tcb/default minted options}也会重置\refKey{/tcb/minted options}。
\begin{dispListing}
% inside the preamble
\tcbset{%
default minted options={tabsize=4,fontsize=\normalsize},
}
\end{dispListing}
\end{docTcbKey}



\begin{docTcbKey}{minted style}{=\meta{style}}{no default, initially unset}
Sets a \meta{style} known to |Pygments| \cite{pygments:web}. This is
independent from \refKey{/tcb/minted options}. Note that styles are always
applied globally; all following examples will be set in the given \meta{style}
until a new style is set. Also note that
setting |\usemintedstyle|\marg{style} only once per document is more economic, if
all styles in a document are the same.
For examples of different styles, see
\refKey{/tcb/minted language} and \refKey{/tcb/minted options}.

设置一个已知于 |Pygments| \cite{pygments:web} 的 \meta{style}。这与 \refKey{/tcb/minted options} 是独立的。请注意,样式始终是全局应用的;所有后续的示例都将设置在给定的 \meta{style} 中,直到设置新样式。还请注意,如果文档中的所有样式都相同,则每个文档仅需设置一次 |\usemintedstyle|\marg{style} 更为经济。有关不同样式的示例,请参见 \refKey{/tcb/minted language} 和 \refKey{/tcb/minted options}。
\end{docTcbKey}

See further options in \Vref{sec:commonlistingkeys}.

请参见\ref{sec:commonlistingkeys} 中的更多选项。
%% \clearpage
% \subsection{Common Option Keys of all Libraries\\所有库的常见选项键}\label{sec:commonlistingkeys}
% \subsection{Common Option Keys of all Libraries\\所有库的常见选项键}\label{sec:commonlistingkeys}

For the \meta{options} in \refEnvLe{tcblisting} respectively \refComLe{tcbinputlisting}
the following |pgf| keys can be applied. The key tree path |/tcb/| is not to
be used inside these macros.

对于\refEnvLe{tcblisting}或\refComLe{tcbinputlisting}中的\meta{options},可以应用以下|pgf|键。不应在这些宏中使用键树路径|/tcb/|。
\begin{docTcbKey}{listing engine}{=\meta{engine}}{no default}
Sets the \meta{engine} which typesets the listings. Feasible values are

设置排版代码的\meta{engine}。可行的值为: %\begin{itemize} \item 如果加载了库\mylib{listings}或\mylib{listingsutf8},则为\docValue{listings}。 \item 如果加载了库\mylib{minted},则为\docValue{minted}。 \end{itemize}
\begin{itemize}
\item\docValue{listings}, if library \mylib{listings} or
\mylib{listingsutf8} is loaded.
\item\docValue{minted}, if library \mylib{minted} is loaded.
\end{itemize}
\end{docTcbKey}

\begin{docTcbKey}{listing file}{=\meta{file name}}{no default, initially \cs{jobname.listing}}
Sets the \meta{file name} of the file which is used to save listings.

设置用于保存代码清单的文件的\meta{文件名}。
\end{docTcbKey}


\begin{docTcbKey}{listing and text}{}{no value, initially set}
Typesets the environment content as listing in the upper part and
as compiled text in the lower part.

将环境内容分别作为源码清单显示在上部和编译后的文本显示在下部。
\begin{dispExample}
\begin{tcblisting}{colback=red!5!white,colframe=red!75!black,listing and text}
This is a \LaTeX\ example.
\end{tcblisting}
\end{dispExample}
\end{docTcbKey}


\begin{docTcbKey}{text and listing}{}{no value}
Typesets the environment content as compiled text in the upper part and
as listing in the lower part.

将环境内容分别作为源码清单显示在上部和编译后的文本显示在下部。
\begin{dispExample}
\begin{tcblisting}{colback=red!5!white,colframe=red!75!black,text and listing}
This is a \LaTeX\ example.
\end{tcblisting}
\end{dispExample}
\end{docTcbKey}


\begin{docTcbKey}{listing only}{}{no value}
Typesets the environment content as listing.

将环境内容排版为清单形式。
\begin{dispExample}
\begin{tcblisting}{colback=red!5!white,colframe=red!75!black,listing only}
This is a \LaTeX\ example.
\end{tcblisting}
\end{dispExample}
\end{docTcbKey}


%%\clearpage
\begin{docTcbKey}{text only}{}{no value}
Typesets the environment content as compiled text.

将环境内容排版为已编译的文本。
\begin{dispExample}
\begin{tcblisting}{colback=red!5!white,colframe=red!75!black,text only}
This is a \LaTeX\ example.
\end{tcblisting}
\end{dispExample}
\end{docTcbKey}



\begin{docTcbKey}{comment}{=\meta{text}}{no default, initially empty}
Records a comment with \meta{text} as content. The comment is displayed
e.\,g.\ in conjunction with \refKeyLe{/tcb/listing and comment}
and \refKeyLe{/tcb/comment and listing}.

记录一个以 \meta{text} 为内容的注释。该注释通常与 \refKeyLe{/tcb/listing and comment} 和 \refKeyLe{/tcb/comment and listing} 一起显示。
\begin{dispExample}
\begin{tcblisting}{comment={This comment is really only a comment},
colback=red!5!white,colframe=red!75!black}
This is a \textbf{tcolorbox}.
\end{tcblisting}
\end{dispExample}
\end{docTcbKey}


\begin{docTcbKey}[][doc new=2014-11-17]{comment only}{}{no value}
Typesets the environment content with the comment text.

使用注释文本对环境内容进行排版。
\begin{dispExample}
\begin{tcblisting}{comment only,
comment={This is a comment.},
colback=red!5!white,colframe=red!75!black}
This is a \textbf{tcolorbox}.
\end{tcblisting}
\end{dispExample}
\end{docTcbKey}



\begin{docTcbKey}{image comment}{=\marg{options}\marg{filename}}{style, no default, initially unset}
Uses an image denoted by \meta{filename} as \textit{comment} for the listing.
The image is included by the standard |\includegraphics| macro with
given \meta{options}.

使用由\meta{filename}表示的图像作为源码清单的\textit{注释}。 该图像通过标准的|\includegraphics|宏和给定的\meta{options}被包含进来。
\begin{dispExample}
\begin{tcblisting}{colback=red!5!white,colframe=red!75!black,listing side comment,
image comment={width=2.5cm}{example-image-a.pdf},center lower}
This is a \LaTeX\ example.
\end{tcblisting}
\end{dispExample}
\end{docTcbKey}


%%\clearpage
\begin{docTcbKey}[][doc new=2014-11-14]{tcbimage comment}{=\meta{filename}}{style, no default, initially unset}
Uses an image denoted by \meta{filename} as \textit{comment} for the listing.
The image is included by the \refComLe{tcbincludegraphics} macro.
The inclusion can be customized by \refKeyLe{/tcb/comment style}.

使用由\meta{filename}指定的图像作为源码清单的\textit{注释}。 该图像由\refComLe{tcbincludegraphics}宏包含。 包含可以通过\refKeyLe{/tcb/comment style}进行自定义。
\begin{marker}
The library \mylib{skins} is needed to apply this option.

需要使用库\mylib{skins}来应用此选项。
\end{marker}
\medskip
\begin{dispExample}
% \tcbuselibrary{skins}
\begin{tcblisting}{colback=red!5!white,colframe=red!75!black,listing side comment,
righthand width=3cm,lower separated=false,
tcbimage comment={example-image-a.pdf},
comment style={size=fbox,colframe=blue,colback=blue!50,sharp corners,
drop fuzzy shadow}}
This is a \LaTeX\ example.
\end{tcblisting}
\end{dispExample}
\end{docTcbKey}

%%\clearpage

\begin{docTcbKey}[][doc new=2014-11-14]{pdf comment}{\colOpt{=\meta{filename}}}{style, default listing file, initially unset}
Uses a PDF file denoted by \meta{filename} as \textit{comment} for the listing.
The image is included by \refComLe{tcbincludepdf} inside a \refEnvLe{tcbraster}.
The inclusion can be customized by \refKeyLe{/tcb/comment style}.

使用由\meta{filename}表示的PDF文件作为清单的\textit{注释}。 图像由\refComLe{tcbincludepdf}包含在\refEnvLe{tcbraster}中。 可以通过\refKeyLe{/tcb/comment style}自定义包含。
\begin{marker}
The libraries \mylib{skins} and \mylib{raster} are needed to apply this option.

应用此选项需要 \mylib{skins} 和 \mylib{raster} 库。
\end{marker}
\medskip
\begin{dispExample}
% \tcbuselibrary{skins,raster}
\begin{tcblisting}{colback=red!5!white,colframe=red!75!black,listing and comment,
righthand width=3cm,lower separated=false,middle=1mm,
pdf comment={tcolorbox-example.pdf},
comment style={raster columns=3,graphics pages={1,2,3},
colframe=blue,drop fuzzy shadow}}
This is a \LaTeX\ example.
\end{tcblisting}
\end{dispExample}
\end{docTcbKey}

%%\clearpage


\begin{docTcbKey}[][doc new=2014-11-14]{pdf extension}{=\meta{extension}}{no default, initially |pdf|}
Sets the PDF file name extension for \refKeyLe{/tcb/pdf comment} to \meta{extension}.
Note that \meta{extension} always overwrites any actual extension given
inside \refKeyLe{/tcb/pdf comment}.

将 \refKeyLe{/tcb/pdf comment} 的 PDF 文件名扩展名设置为 \meta{extension}。 请注意,\meta{extension} 总是覆盖 \refKeyLe{/tcb/pdf comment} 中给定的实际扩展名。
\end{docTcbKey}


\begin{docTcbKey}[][doc new=2014-11-14]{comment style}{=\meta{options}}{no default, initially empty}
Sets the \meta{options} for \refKeyLe{/tcb/tcbimage comment} and \refKeyLe{/tcb/pdf comment}.
These are |tcolorbox| options to customize the colored box drawn around the
image(s), also image options encapsulated by \refKeyLe{/tcb/graphics options},
and \refEnvLe{tcbraster} options for \refKeyLe{/tcb/pdf comment}.

设置\refKeyLe{/tcb/tcbimage comment}和\refKeyLe{/tcb/pdf comment}的\meta{options}。 这些都是用于自定义围绕图像绘制的彩色框的|tcolorbox|选项, 还包括由\refKeyLe{/tcb/graphics options}封装的图像选项, 以及用于\refKeyLe{/tcb/pdf comment}的\refEnvLe{tcbraster}选项。
\end{docTcbKey}


\begin{docTcbKey}{listing and comment}{}{no value}
Typesets the environment content as listing in the upper part and
a given comment in the lower part.

将环境内容排版为上部的源码清单形式,下部为给定的注释。
\begin{dispExample}
\begin{tcblisting}{colback=red!5!white,colframe=red!75!black,listing and comment,
comment={This is my comment. It may contain line breaks.\par
It can even use the environment content
\flqq\ignorespaces\tcbuselistingtext\unskip\frqq}}
This is a \LaTeX\ example.
\end{tcblisting}
\end{dispExample}
\end{docTcbKey}

\enlargethispage*{10mm}
\begin{docTcbKey}{comment and listing}{}{no value}
Typesets a given comment in the upper part and
the environment content as listing in the lower part.

将给定的注释排版在上部,将环境内容作为源码清单排列在下部。
\begin{dispExample}
\begin{tcblisting}{colback=red!5!white,colframe=red!75!black,comment and listing,
comment={This is my comment.}}
This is a \LaTeX\ example.
\end{tcblisting}
\end{dispExample}
\end{docTcbKey}


%%\clearpage
\begin{docTcbKey}{listing side text}{}{style, no value}
Typesets the environment content side by side as listing in the left (upper)
part and as compiled text in the right (lower) part.
This is a shortcut for setting \refKeyLe{/tcb/listing and text} and \refKeyLe{/tcb/sidebyside}.

将环境内容并排设置为左侧(上部)的清单和右侧(下部)已编译文本。 这是设置 \refKeyLe{/tcb/listing and text} 和 \refKeyLe{/tcb/sidebyside} 的快捷方式。
\begin{dispExample}
\begin{tcblisting}{colback=red!5!white,colframe=red!75!black,listing side text}
This is a \LaTeX\ example.
\end{tcblisting}
\end{dispExample}
\begin{marker}
Note that |sidebyside=false| has to be added, if the setting of
\refKeyLe{/tcb/listing side text} is to be annihilated.

请注意,如果要取消\refKeyLe{/tcb/listing side text}的设置,则必须添加|sidebyside=false|。
\end{marker}
\end{docTcbKey}


\begin{docTcbKey}{text side listing}{}{style, no value}
Typesets the environment content side by side as compiled text in the left (upper)
part and as listing in the right (lower) part.
This is a shortcut for setting \refKeyLe{/tcb/text and listing} and \refKeyLe{/tcb/sidebyside}.

将环境内容排列在左侧(上部)以编译文本的形式,右侧(下部)以清单的形式并排显示。 这是一种快捷方式,用于设置 \refKeyLe{/tcb/text and listing} 和 \refKeyLe{/tcb/sidebyside}。
\begin{dispExample}
\begin{tcblisting}{colback=red!5!white,colframe=red!75!black,text side listing}
This is a \LaTeX\ example.
\end{tcblisting}
\end{dispExample}
\end{docTcbKey}


\begin{docTcbKey}{listing outside text}{}{no value}
Typesets the environment content side by side as listing in a |tcolorbox|
and as compiled text outside the box in the right part of the page.
Nevertheless, the outside text is treated as \emph{lower} part of the
|tcolorbox| and can be formatted with all lower part options.
The space partitioning is done with the side by side options from
\Fullref{sec:sidebyside}.

在 |tcolorbox| 中,将环境内容并排列出来,作为清单,并将编译文本放在页面右侧的盒子外部。尽管如此,外部文本被视为 |tcolorbox| 的 \emph{lower部分},可以使用所有lower部分选项进行格式化。空间分区使用来自\Fullref{sec:sidebyside}的并排选项进行。
\begin{dispExample}
\begin{tcblisting}{colback=red!5!white,colframe=red!75!black,listing outside text}
This is a \LaTeX\ example.
\end{tcblisting}
\end{dispExample}
\end{docTcbKey}

%%\clearpage

\begin{docTcbKey}{text outside listing}{}{no value}
Typesets the environment content side by side as listing in a |tcolorbox|
and as compiled text outside the box in the left part of the page.
Nevertheless, the outside text is treated as \emph{lower} part of the
|tcolorbox| and can be formatted with all lower part options.
The space partitioning is done with the side by side options from
\Fullref{sec:sidebyside}.

将环境内容在|tcolorbox|中侧面列出,并将已编译的文本放在页面左侧的框外。尽管如此,外部文本被视为|tcolorbox|的\emph{下部},并且可以使用所有下部选项进行格式化。空间分区是使用\Fullref{sec:sidebyside}中的并排选项完成的。
\begin{dispExample}
\begin{tcblisting}{colback=red!5!white,colframe=red!75!black,text outside listing}
This is a \LaTeX\ example.
\end{tcblisting}
\end{dispExample}
\end{docTcbKey}



\begin{docTcbKey}{listing side comment}{}{style, no value}
Typesets the environment content side by side as listing in the left (upper)
part and a given comment in the right (lower) part.
This is a shortcut for setting \refKeyLe{/tcb/listing and comment} and \refKeyLe{/tcb/sidebyside}.

将环境内容排列在左侧(上部)的清单中,给定的注释排列在右侧(下部)。 这是设置\refKeyLe{/tcb/listing and comment}和\refKeyLe{/tcb/sidebyside}的快捷方式。
\begin{dispExample}
\begin{tcblisting}{colback=red!5!white,colframe=red!75!black,listing side comment,
righthand width=1.5cm,image comment={width=1.5cm}{example-image-a.pdf}}
This is a \LaTeX\ example.
\end{tcblisting}
\end{dispExample}
\end{docTcbKey}


\begin{docTcbKey}{comment side listing}{}{style, no value}
Typesets the environment content side by side with a given comment in the left (upper)
part and as listing in the right (lower) part.
This is a shortcut for setting \refKeyLe{/tcb/comment and listing} and \refKeyLe{/tcb/sidebyside}.

将环境内容与给定注释一起排列在左侧(上部),并将源码清单排列在右侧(下部)。 这是设置\refKeyLe{/tcb/comment and listing}和\refKeyLe{/tcb/sidebyside}的快捷方式。
\begin{dispExample}
\begin{tcblisting}{colback=red!5!white,colframe=red!75!black,comment side listing,
lefthand width=1.5cm,image comment={width=1.5cm}{example-image-a.pdf}}
This is a \LaTeX\ example.
\end{tcblisting}
\end{dispExample}
\end{docTcbKey}

%%\clearpage

\begin{docTcbKey}{listing outside comment}{}{no value}
Typesets the environment content side by side as listing in a |tcolorbox|
and a given comment outside the box in the right part of the page.
Nevertheless, the outside text is treated as \emph{lower} part of the
|tcolorbox| and can be formatted with all lower part options.
The space partitioning is done with the side by side options from
\Fullref{sec:sidebyside}.

将环境内容与给定评论并排作为源码清单在 |tcolorbox| 中排列,并在页面右侧的框外给出评论。然而,外部文本被视为 |tcolorbox| 的\emph{lower部分},可以使用所有lower部分选项进行格式化。空间划分是使用 \Fullref{sec:sidebyside} 中的并排选项完成的。
\begin{dispExample}
\begin{tcblisting}{colback=red!5!white,colframe=red!75!black,listing outside comment,
righthand width=1.5cm,image comment={width=1.5cm}{example-image-a.pdf}}
This is a \LaTeX\ example.
\end{tcblisting}
\end{dispExample}
\end{docTcbKey}


\begin{docTcbKey}{comment outside listing}{}{no value}
Typesets the environment content side by side as listing in a |tcolorbox|
and a given comment outside the box in the left part of the page.
Nevertheless, the outside text is treated as \emph{lower} part of the
|tcolorbox| and can be formatted with all lower part options.
The space partitioning is done with the side by side options from
\Fullref{sec:sidebyside}.

在 |tcolorbox| 中将环境内容并排设置为源码清单,并在页面左侧给出给定的注释。然而,外部文本被视为 |tcolorbox| 的 \emph{下部} 部分,并且可以使用所有下部选项进行格式化。空间划分是通过从 \Fullref{sec:sidebyside} 中的并排选项完成的。
\begin{dispExample}
\begin{tcblisting}{colback=red!5!white,colframe=red!75!black,comment outside listing,
lefthand width=1.5cm,image comment={width=1.5cm}{example-image-a.pdf}}
This is a \LaTeX\ example.
\end{tcblisting}
\end{dispExample}
\end{docTcbKey}



\begin{docTcbKey}{listing above text}{}{no value}
Typesets the environment content as listing in a |tcolorbox|
and as compiled text outside and below the box.
The outside text is treated as \emph{lower} part of the
|tcolorbox| and can be formatted with all lower part options.
The distance between box and text is controlled by \refKeyLe{/tcb/middle}.

将环境内容排版为 |tcolorbox| 中的清单,并将已编译的文本放在盒子之外和下方。盒子外的文本被视为 |tcolorbox| 的 \emph{下部},可以使用所有下部选项进行格式化。盒子和文本之间的距离由 \refKeyLe{/tcb/middle} 控制。
\begin{dispExample}
\begin{tcblisting}{colback=red!5!white,colframe=red!75!black,listing above text}
This is a \LaTeX\ example.
\end{tcblisting}
\end{dispExample}
\end{docTcbKey}


\begin{docTcbKey}[][doc new=2014-11-07]{listing above* text}{}{no value}
Widely equal to \refKeyLe{/tcb/listing above text}, but the outside text is
not formatted with the lower part options.
Also, it is not put into a minipage and it may span several pages.
The distance between box and text is controlled by \refKeyLe{/tcb/after}.

与\refKeyLe{/tcb/listing above text}基本相同,但外部文本不使用底部选项格式化。此外,它不会放入一个小页中,可能跨越多个页面。盒子和文本之间的距离由\refKeyLe{/tcb/after}控制。
\end{docTcbKey}

%%\clearpage

\begin{docTcbKey}{text above listing}{}{no value}
Typesets the environment content as listing in a |tcolorbox|
and as compiled text outside and above the box.
The outside text is treated as \emph{lower} part of the
|tcolorbox| and can be formatted with all lower part options.
The distance between box and text is controlled by \refKeyLe{/tcb/middle}.

将环境内容排版为 |tcolorbox| 中的源码清单,并编译成文本在盒子外部和上方。外部文本被视为 |tcolorbox| 的 \emph{下部} 部分,并可以使用所有下部选项进行格式化。盒子和文本之间的距离由 \refKeyLe{/tcb/middle} 控制。
\begin{dispExample}
\begin{tcblisting}{colback=red!5!white,colframe=red!75!black,text above listing}
This is a \LaTeX\ example.
\end{tcblisting}
\end{dispExample}
\end{docTcbKey}


\begin{docTcbKey}[][doc new=2014-11-07]{text above* listing}{}{no value}
Widely equal to \refKeyLe{/tcb/text above listing}, but the outside text is
not formatted with the lower part options.
Also, it is not put into a minipage and it may span several pages.
The distance between box and text is controlled by \refKeyLe{/tcb/before}.

与 \refKeyLe{/tcb/text above listing} 相似,但是外部文本不使用下部选项进行格式化。此外,它不会被放入一个小页中,也可能跨越多个页面。盒子和文本之间的距离由 \refKeyLe{/tcb/before} 控制。
\end{docTcbKey}



\begin{docTcbKey}{listing above comment}{}{no value}
Typesets the environment content as listing in a |tcolorbox|
and a given comment outside and below the box.
The outside text is treated as \emph{lower} part of the
|tcolorbox| and can be formatted with all lower part options.
The distance between box and comment is controlled by \refKeyLe{/tcb/middle}.

将环境内容排版为源码清单形式,放置在一个 |tcolorbox| 中,并在盒子外部和下方给出注释。外部文本被视为 |tcolorbox| 的 \emph{下部} 部分,可以使用所有下部选项进行格式化。盒子和注释之间的距离由 \refKeyLe{/tcb/middle} 控制。
\begin{dispExample}
\begin{tcblisting}{colback=red!5!white,colframe=red!75!black,listing above comment,
center lower,image comment={width=3cm}{example-image-a.pdf}}
This is a \LaTeX\ example.
\end{tcblisting}
\end{dispExample}
\end{docTcbKey}


\begin{docTcbKey}[][doc new=2014-11-07]{listing above* comment}{}{no value}
Widely equal to \refKeyLe{/tcb/listing above comment}, but the outside comment is
not formatted with the lower part options.
Also, it is not put into a minipage and it may span several pages.
The distance between box and comment is controlled by \refKeyLe{/tcb/after}.

与 \refKeyLe{/tcb/listing above comment} 大致相同,但外部注释不使用底部选项格式化。此外,它不会放置在一个小页中,可能跨越多个页面。盒子与注释之间的距离由 \refKeyLe{/tcb/after} 控制。
\end{docTcbKey}

%%%%\clearpage

\begin{docTcbKey}{comment above listing}{}{no value}
Typesets the environment content as listing in a |tcolorbox|
and a given comment outside and above the box.
The outside text is treated as \emph{lower} part of the
|tcolorbox| and can be formatted with all lower part options.
The distance between box and comment is controlled by \refKeyLe{/tcb/middle}.

将环境内容设置为 |tcolorbox| 中的源码清单,并在框外和上方给出注释。外部文本被视为 |tcolorbox| 的 \emph{下部} 部分,并可以使用所有下部选项进行格式化。框和注释之间的距离由 \refKeyLe{/tcb/middle} 控制。
\begin{dispExample}
\begin{tcblisting}{colback=red!5!white,colframe=red!75!black,comment above listing,
center lower,image comment={width=3cm}{example-image-a.pdf}}
This is a \LaTeX\ example.
\end{tcblisting}
\end{dispExample}
\end{docTcbKey}


\begin{docTcbKey}[][doc new=2014-11-07]{comment above* listing}{}{no value}
Widely equal to \refKeyLe{/tcb/comment above listing}, but the outside comment is
not formatted with the lower part options.
Also, it is not put into a minipage and it may span several pages.
The distance between box and comment is controlled by \refKeyLe{/tcb/before}.

与\refKeyLe{/tcb/comment above listing}大致相同,但外部注释没有使用底部选项进行格式化。并且,它没有放置在一个小页面中,可能跨越多个页面。盒子和注释之间的距离由\refKeyLe{/tcb/before}控制。
\end{docTcbKey}

%% \clearpage
% \subsection{Option Keys for Processing and Full Document Examples\\处理和完整文档示例的选项键}\label{sec:proclistingkeys}
% \subsection{Option Keys for Processing and Full Document Examples\\处理选项键和完整文档示例}\label{sec:proclistingkeys}
\include*{listings/处理和完整文档示例的选项键_处理完整的LaTeX文档有两种方法}
\include*{listings/处理和完整文档示例的选项键_包含源文件和生成的PDF文件}
\include*{listings/处理和完整文档示例的选项键_process_code}
% \clearpage
\include*{listings/处理和完整文档示例的选项键_run_system_command}
\include*{listings/处理和完整文档示例的选项键_compilable_listing}
\include*{listings/处理和完整文档示例的选项键_运行常用命令}
\include*{listings/处理和完整文档示例的选项键_冻结缓存}

%% \clearpage
% \subsection{Creation of \LaTeX\ Tutorials\\\LaTeX\ 教程的创建}\label{sec:latextutorial}
% The following source code gives a guideline for the creation of \LaTeX\ tutorials.
In the next section, a framework for \LaTeX\ exercises is described.
All examples shall be numbered optionally.

以下源代码提供了创建 \LaTeX\ 教程的指南。 在下一节中,描述了 \LaTeX\ 练习的框架。 所有示例均可选择编号。

Firstly, some additional |tcb| keys are defined for the appearance.
For the examples, three environments |texexp|, |texexptitled|,
and |texexptitledspec| are defined with automatic numbering.

首先,为了外观方面定义了一些额外的 |tcb| 键。为了举例,定义了三个环境,即带自动编号的 |texexp|、|texexptitled| 和 |texexptitledspec|。
\begin{itemize}
\item |texexp| is used for untitled examples,
\\|texexp| 用于无标题的例子,
\item |texexptitled| is used for titled examples,
\\|texexptitled| 用于带标题的例子,
\item |texexptitledspec| is used for titled examples with special treatment.
\\|texexptitledspec| 用于带特殊处理的标题例子。
\end{itemize}

\inputpreamblelisting{D}

\begin{dispExample}
\begin{tcblisting}{texexp}
This is a \LaTeX\ example which displays the text as source code
and in compiled form.

这是一个展示文本源代码和编译后形式的 \LaTeX\ 示例。
\end{tcblisting}
\end{dispExample}


\begin{dispExample}
\begin{texexptitled}{First example with a title line}{firstExample}
Here, we use Example \ref{firstExample} with a title line.

在这里,我们使用带有标题行的示例\ref{firstExample}。
\end{texexptitled}
\end{dispExample}


\begin{dispExample}
\begin{texexp}{}
This is a \LaTeX\ example which displays the text as source code
and in compiled form.

这是一个 \LaTeX\ 的示例,它可以将文本显示为源代码和编译后的形式。
\end{texexp}
\end{dispExample}


\begin{dispExample}
\begin{texexp}{text and listing}
This is a \LaTeX\ example which displays the text as source code
and in compiled form.

这是一个 \LaTeX\ 的示例,它展示了文本的源代码和编译后的形式。
\end{texexp}
\end{dispExample}


\begin{dispExample}
\begin{texexp}{listing only}
This is a \LaTeX\ example which displays the text as source code only.

这是一个 \LaTeX\ 的例子,仅以源代码形式显示文本。
\end{texexp}
\end{dispExample}


\begin{dispExample}
\begin{texexp}{text only}
This is a \LaTeX\ example which displays the text in compiled form only.

这是一个 \LaTeX\ 示例,仅以编译后的形式展示文本。
\end{texexp}
\end{dispExample}


\begin{dispExample}
\begin{texexptitled}{An Example with a Heading}{heading1}
This is a \LaTeX\ example with a numbered heading line
which can be referred to.

这是一个带有编号标题行的 \LaTeX\ 示例,可以进行引用。
\end{texexptitled}
Here, we see Example \ref{heading1}.

在这里,我们看到示例\ref{heading1}。
\end{dispExample}


\begin{dispExample}
\begin{texexptitled}[listing only]{Another Example with a Heading}{heading2}
The keys can be used in combination. Here, an example with a heading line
and source code only is given.

这些键可以组合使用。下面是仅包含标题和源代码的示例。
\end{texexptitled}
Here, we see Example \ref{heading2}.

在这里,我们看到示例 \ref{heading2}。
\end{dispExample}


\begin{dispListing}
\begin{texexptitled}[float]{A floating Example with a Heading}{heading3}
This is another \LaTeX\ example with numbered heading line.
But now, the box is a floating object.

这是另一个带有编号标题行的 \LaTeX\ 示例。 但现在,这个框是一个浮动对象。
\end{texexptitled}
\end{dispListing}
\tcbusetemp

\begin{dispExample}
The floating box of the last example is seen as Example \ref{heading3}
on page \pageref{heading3}.

上一个例子中的浮动框在第\pageref{heading3}页被视为第\ref{heading3}个例子。
\end{dispExample}


\begin{dispExample}
\begin{texexptitledspec}{Special application}{texexpbox1}
\begin{lstlisting}[style=tcblatex]
Some \LaTeX\ source code.

一些 \LaTeX\ 源代码。
\end{lstlisting}
\tcblower
For special cases, the environment |texexptitledspec| with style
|example| can be used directly. As one can see, the upper and the lower
part of the box can be used uncoupled also.

对于特殊情况,可以直接使用样式为|example|的环境|texexptitledspec|。可以看到,盒子的上部和下部也可以分开使用。
\end{texexptitledspec}
\end{dispExample}


The following series of examples demonstrate the application of
\refEnv{tcolorbox} options for diversification.

以下一系列例子展示了 \refEnv{tcolorbox} 选项的多样化应用。
\begin{dispExample}
\begin{texexptitled}{How to use options (1):\par The basic example}{options1}
\begin{tikzpicture}
\path[fill=yellow!50!white] (0,0) circle (11mm);
\path[fill=white] (0,0) circle (9mm);
\foreach \w/\c in {90/red,210/green,330/blue}
{\path[shading=ball,ball color=\c] (\w:1cm) circle (7mm);}
\end{tikzpicture}
\end{texexptitled}
\end{dispExample}


\begin{dispExample}
\begin{texexptitled}[center lower,enhanced,segmentation hidden,middle=0mm]
{How to use options (2):\par The text output is centered and the
segmentation line has vanished.}{options2}
\begin{tikzpicture}
\path[fill=yellow!50!white] (0,0) circle (11mm);
\path[fill=white] (0,0) circle (9mm);
\foreach \w/\c in {90/red,210/green,330/blue}
{\path[shading=ball,ball color=\c] (\w:1cm) circle (7mm);}
\end{tikzpicture}
\end{texexptitled}
\end{dispExample}

\begin{dispExample}
\begin{texexptitled}[tikz lower,bicolor,colbacklower=white]
{How to use options (3):\par Here, the |tikzpicture| is totally hidden.
The |bicolor| skin highlights the output.}{options3}
\path[fill=yellow!50!white] (0,0) circle (11mm);
\path[fill=white] (0,0) circle (9mm);
\foreach \w/\c in {90/red,210/green,330/blue}
{\path[shading=ball,ball color=\c] (\w:1cm) circle (7mm);}
\end{texexptitled}
\end{dispExample}

\begin{dispExample}
\begin{texexptitled}[center lower,listing side text,righthand width=3.5cm,
bicolor,colbacklower=white]
{How to use options (4):\par The |bicolor| skin also works with side
by side mode}{options4}
\begin{tikzpicture}
\path[fill=yellow!50!white] (0,0) circle (11mm);
\path[fill=white] (0,0) circle (9mm);
\foreach \w/\c in {90/red,210/green,330/blue}
{\path[shading=ball,ball color=\c]
(\w:1cm) circle (7mm);}
\end{tikzpicture}
\end{texexptitled}
\end{dispExample}


\begin{dispExample}
\begin{texexptitled}[center lower,listing outside text,righthand width=3.5cm]
{How to use options (5):\par Putting our picture outside is just
a matter of one word.}{options5}
\begin{tikzpicture}
\path[fill=yellow!50!white] (0,0) circle (11mm);
\path[fill=white] (0,0) circle (9mm);
\foreach \w/\c in {90/red,210/green,330/blue}
{\path[shading=ball,ball color=\c]
(\w:1cm) circle (7mm);}
\end{tikzpicture}
\end{texexptitled}
\end{dispExample}


\begin{dispExample}
\begin{texexptitled}[center lower,text above listing]
{How to use options (6):\par The picture may also be put above
the listing box.}{options6}
\begin{tikzpicture}
\path[fill=yellow!50!white] (0,0) circle (11mm);
\path[fill=white] (0,0) circle (9mm);
\foreach \w/\c in {90/red,210/green,330/blue}
{\path[shading=ball,ball color=\c]
(\w:1cm) circle (7mm);}
\end{tikzpicture}
\end{texexptitled}
\end{dispExample}


\begin{dispExample}
\begin{texexptitled}[beamer,center lower,text outside listing,lefthand width=3.5cm]
{How to use options (7):\par Our style is easily transformed into
a beamerish one.}{options7}
\begin{tikzpicture}
\path[fill=yellow!50!white] (0,0) circle (11mm);
\path[fill=white] (0,0) circle (9mm);
\foreach \w/\c in {90/red,210/green,330/blue}
{\path[shading=ball,ball color=\c]
(\w:1cm) circle (7mm);}
\end{tikzpicture}
\end{texexptitled}
\end{dispExample}
%% \clearpage
\subsection{Creation of \LaTeX\ Exercises\\\LaTeX\ 练习的创建}\label{listing:exercises}
\subsection{Creation of \LaTeX\ Exercises\\\LaTeX\ 练习的创建}\label{listing:exercises}

In the following, a guideline is given for the creation of \LaTeX\ exercises
with solutions. These solutions are saved to disk for application at a place of
choice.
Therefore, all used exercises are logged to a file |\jobname.records| for automatic
processing. The solution contents themselves are saved to a subdirectory named
|solutions|. Also see \Vref{sec:recording}.

下面提供了一个指南,用于创建带有答案的 \LaTeX\ 练习。这些答案被保存到磁盘上,以便在需要的地方应用。因此,所有使用的练习都被记录在一个名为 |\jobname.records| 的文件中\footnote{比如tcolorbox.records。}%
,以便自动处理。解答内容本身保存在一个名为 |solutions| 的子目录中。请参见 \Vref{sec:recording}。
\begin{itemize}
\item Before the first exercise is given,
\refComLe{tcbstartrecording} has to be called to start recording.
\\在给出第一个练习之前,必须调用 \refComLe{tcbstartrecording} 开始录制。
\item The solution is given as content of a \refEnvLe{tcboutputlisting}
environment. Note, that you can use this content also inside the
exercise with \refComLe{tcbuselistingtext} in compiled form.
\\解决方案作为 \refEnvLe{tcboutputlisting} 环境的内容给出。请注意,您可以使用编译后的形式在练习中使用 \refComLe{tcbuselistingtext}。
\item After the last exercise is given (and before using the solutions),
\refComLe{tcbstoprecording} has to be called to stop recording.
\\在给出最后一个练习(且在使用解决方案之前),必须调用 \refComLe{tcbstoprecording} 停止录制。
\item The solutions are loaded by \refComLe{tcbinputrecords}.
\\解决方案通过 \refComLe{tcbinputrecords} 加载。
\end{itemize}

Inside the exercise text, there may be text parts which are needed as
\LaTeX\ source code and as compiled text as well. These parts can be
saved by \refEnvLe{tcbwritetemp} and used in compiled form by \refComLe{tcbusetemp}
or as source code by \refComLe{tcbusetemplisting}.

在练习文本中,可能存在需要作为\LaTeX 源代码和编译后文本的文本部分。这些部分可以通过\refEnvLe{tcbwritetemp}保存,并通过\refComLe{tcbusetemp}以编译形式使用,或通过\refComLe{tcbusetemplisting}作为源代码使用。

At first, we generate some a common style for the exercises and the
solutions. Further, since exercises and solutions should
be numbered, we force to use a label \meta{marker}.
Automatically, the label |exe:|\meta{marker} is used to mark the exercise
and the label |sol:|\meta{marker} is used to mark the solution.

首先,我们为练习和解决方案生成了一种常见的样式。另外,由于练习和解决方案需要编号,我们强制使用标签\meta{marker}。自动地,标签|exe:|\meta{marker}用于标记练习,而标签|sol:|\meta{marker}用于标记解决方案。

\begin{dispListing}
\tcbset{texercisestyle/.style={arc=0.5mm, colframe=blue!25!yellow!90!white,
colback=blue!25!yellow!5!white, coltitle=blue!25!yellow!40!black,
fonttitle=\small\sffamily\bfseries, fontupper=\small, fontlower=\small,
listing options={style=tcblatex,texcsstyle=*\color{red!40!black}},
}}
\end{dispListing}
\tcbusetemp

With these preparations, the kernel environment |texercise| for our
exercises is created quickly:

有了这些准备,我们的练习内核环境 |texercise| 就能够快速创建:

\inputpreamblelisting{E}

%% \clearpage
The following examples demonstrate the application.

以下示例演示了应用程序。

\begin{dispListing}
\tcbstartrecording
\end{dispListing}
\tcbusetemp


\begin{dispExample}
\begin{texercise}{tabular_example}
\textit{Create the following table:}\par\smallskip%
\begin{tcboutputlisting}
\begin{tabular}{|p{3cm}|p{3cm}|p{3cm}|p{3cm}|}\hline
\multicolumn{4}{|c|}{\bfseries\itshape Das alte Italien}\\\hline
\multicolumn{2}{|c|}{\bfseries Antike} &
\multicolumn{2}{c|}{\bfseries Mittelalter}\\\hline
\multicolumn{1}{|c|}{\itshape Republik}&
\multicolumn{1}{c|}{\itshape Kaiserreich}&
\multicolumn{1}{c|}{\itshape Franken}&
\multicolumn{1}{c|}{\itshape Teilstaaten}\\\hline
In den Zeiten der r\"{o}mischen Republik standen dem Staat jeweils zwei
Konsuln vor, deren Machtbefugnisse identisch waren. &
Das r\"{o}mische Kaiserreich wurde von einem Alleinherrscher, dem Kaiser,
regiert.
& In der V\"{o}lkerwanderungszeit \"{u}bernahmen die Goten und sp\"{a}ter die
Franken die Vorherrschaft.
& Im sp\"{a}teren Mittelalter regierten F\"{u}rsten einen Fleckenteppich
von Einzelstaaten.\\\hline
\end{tabular}
\end{tcboutputlisting}
\tcbuselistingtext%
\end{texercise}
\end{dispExample}


\begin{dispExample}
\begin{texercise}{macro_oneparam}
\begin{tcboutputlisting}
\newcommand{\headingline}[1]{%
\begin{center}\Large\bfseries #1\end{center}}
\end{tcboutputlisting}
\tcbuselistingtext%

Create a new macro \verb+\headingline+ which produces the
following output:

创建一个新的宏\verb+\headingline+,它会产生以下输出:
\par\smallskip
\begin{tcbwritetemp}
\headingline{Very important heading}
\end{tcbwritetemp}
\tcbusetemplisting\tcbusetemp%
\end{texercise}
\end{dispExample}



\begin{dispExample}
\begin{texercise}{macro_twoparam}
\begin{tcboutputlisting}
\newcommand{\minitable}[2]{%
\begin{center}\begin{tabular}{p{10cm}}\hline%
\multicolumn{1}{c}{\bfseries#1}\\\hline%
#2\\\hline%
\end{tabular}\end{center}}
\end{tcboutputlisting}
\tcbuselistingtext%
Create a new macro \verb+\minitable+ which produces the
following output:

创建一个名为\verb+\minitable+的新宏,它会生成以下输出\par\smallskip
\begin{tcbwritetemp}
\minitable{My heading}{In this tiny tabular, there is only a heading
and some text below which has a width of ten centimeters.}
\end{tcbwritetemp}
\tcbusetemplisting\par\smallskip\tcbusetemp%
\end{texercise}
\end{dispExample}


\begin{dispExample}
\begin{texercise}{macro_threeparam}
\begin{tcboutputlisting}
\newcommand{\synop}[3]{%
\begin{tabular}{@{}p{(\linewidth-\tabcolsep*2-\arrayrulewidth)/2}|%
p{(\linewidth-\tabcolsep*2-\arrayrulewidth)/2}@{}}\hline
\multicolumn{2}{c}{\bfseries #1}\\\hline
\multicolumn{1}{c|}{\itshape English}&
\multicolumn{1}{c}{\itshape German}\\\hline
#2 & #3
\end{tabular}}
\end{tcboutputlisting}
\tcbuselistingtext%
Create a new macro \verb+\synop+ which typesets a synoptic text according
to the following example. Base your macro on a tabular which takes the
total line width.

创建一个新的宏\verb+\synop+,根据以下示例排版综合文本。基于一个接受总行宽的表格来创建你的宏。\par\smallskip
\begin{tcbwritetemp}
\synop{Neil Armstrong}%
{That's one small step for a man, one giant leap for mankind.}%
{Das ist ein kleiner Schritt f\"{u}r einen Mann,
ein riesiger Sprung f\"{u}r die Menschheit.}
\end{tcbwritetemp}
\tcbusetemplisting\par\smallskip\tcbusetemp%
\end{texercise}
\end{dispExample}
%\closeoutsol

\begin{dispListing}
\tcbstoprecording
\end{dispListing}
\tcbusetemp

\bigskip

Now, we give a list of all exercises with:

现在,我们列出了所有练习的清单,包括:
\begin{dispListing}
\tcblistof[\subsection]{exam}{List of Exercises%
\label{listofexercises}}
\end{dispListing}
\tcbusetemp



% \clearpage
% \subsection{Solutions for the given \LaTeX\ Exercises}

% For all solutions, a macro |\processsol| was written to the file |\jobname.records|.
% Now, we need a definition for this macro to use the solutions.

% \begin{dispListing}
% % \usepackage{hyperref} % for phantomlabel
% \newtcbinputlisting{\processsol}[2]{%
%   texercisestyle,
%   listing only,
%   listing file={#1},
%   phantomlabel={sol:#2},%
%   title={Solution for Exercise \ref{exe:#2} on page \pageref{exe:#2}},
% }
% \end{dispListing}
% \tcbusetemp

% The loading of all solutions is done by:

% \begin{dispListing}
% \tcbinputrecords
% \end{dispListing}

% With this, we get:

% \tcbusetemp



% v1 2023 0314  v2 2023 0325

% \include*{tcolorbox.doc.theorems}%v1 2023 0328

% % !TeX root = tcolorbox.tex
% include file of tcolorbox.tex (manual of the LaTeX package tcolorbox)
\setcounter{section}{18}
\section{Library \mylib{breakable}}\label{sec:breakable}%
\tcbset{external/prefix=external/breakable_}%
The library is loaded by a package option or inside the preamble by:

该库可以通过包选项或在导言区中加载:
\begin{dispListing}
\tcbuselibrary{breakable}
\end{dispListing}
This also loads the package |pdfcol|.

这还会加载 |pdfcol| 包。

% \include*{breakable/技术概述}
% \include*{breakable/限制和已知问题}
\include*{breakable/主要选项键}

\end{document}

\subsection{Option Keys for the Break Appearance}

\begin{docTcbKey}{toprule at break}{=\meta{length}}{no default, initially \texttt{0.5mm}}
  Sets the line width of the top rule to \meta{length} \emph{if} the box is \refKeyLe{/tcb/breakable}.
  In this case, it is applied to \emph{middle} and \emph{last} parts in a
  break sequence. Note that \refKeyLe{/tcb/toprule} overwrites this value
  if used afterwards.
\end{docTcbKey}


\begin{docTcbKey}{bottomrule at break}{=\meta{length}}{no default, initially \texttt{0.5mm}}
  Sets the line width of the bottom rule to \meta{length} \emph{if} the box is \refKeyLe{/tcb/breakable}.
  In this case, it is applied to \emph{first} and \emph{middle} parts in a
  break sequence. Note that \refKeyLe{/tcb/bottomrule} overwrites this value
  if used afterwards.
\end{docTcbKey}


\begin{docTcbKey}{topsep at break}{=\meta{length}}{no default, initially \texttt{0mm}}
  Additional vertical space of \meta{length} which is added at the top of
  \emph{middle} and \emph{last} parts in a break sequence. In general,
  it is not advisable to change this value if these parts start with a rule or a title.
\end{docTcbKey}

\begin{docTcbKey}{bottomsep at break}{=\meta{length}}{no default, initially \texttt{0mm}}
  Additional vertical space of \meta{length} which is added at the bottom of
  \emph{first} and \emph{middle} parts in a break sequence.
  In general, it is not advisable to change this value if these parts end with a rule.
\end{docTcbKey}

\begin{docTcbKey}{pad before break}{=\meta{length}}{style, no default, initially \texttt{3.5mm}}
  Sets the total amount of vertical space after the text content and before the
  break point to \meta{length}. This style sets \refKeyLe{/tcb/toprule at break} to |0pt|
  and changes \refKeyLe{/tcb/topsep at break} as required.
  In general, it is not advisable to change this value if the
  \emph{middle} and \emph{last} parts in a break sequence start with a rule or a title.
\end{docTcbKey}

\begin{docTcbKey}{pad before break*}{=\meta{length}}{style, no default}
  Sets \refKeyLe{/tcb/pad before break} to \meta{length} and
  \refKeyLe{/tcb/enlargepage flexible} to an appropriate value such that
  empty closing frames are avoided.
\end{docTcbKey}

\begin{docTcbKey}{pad after break}{=\meta{length}}{style, no default, initially \texttt{3.5mm}}
  Sets the total amount of vertical space after the break point and before the
  text content to \meta{length}. This style sets \refKeyLe{/tcb/bottomrule at break} to |0pt|
  and changes \refKeyLe{/tcb/bottomsep at break} as required.
  In general, it is not advisable to change this value if the
  \emph{first} and \emph{middle} parts in a break sequence end with a rule.
\end{docTcbKey}

\begin{docTcbKey}{pad at break}{=\meta{length}}{style, no default, initially \texttt{3.5mm}}
  Abbreviation for setting \meta{length} to \refKeyLe{/tcb/pad before break}
  and \refKeyLe{/tcb/pad after break}.
\end{docTcbKey}

\enlargethispage*{5mm}

\begin{docTcbKey}{pad at break*}{=\meta{length}}{style, no default}
  Sets \refKeyLe{/tcb/pad at break} to \meta{length} and
  \refKeyLe{/tcb/enlargepage flexible} to an appropriate value such that
  empty closing frames are avoided.
\end{docTcbKey}

\begin{dispListing}
% \usepackage{lipsum}  % preamble
\tcbset{colback=red!5!white,colframe=red!75!black,fonttitle=\bfseries}

\begin{tcolorbox}[enhanced jigsaw,breakable,pad at break*=0mm,
  title={For this box, the pad space at the break point is set to 0mm}]
  \lipsum[1-2]
\end{tcolorbox}
\end{dispListing}
{\tcbusetemp}


\begin{marker}
\refKeyLe{/tcb/pad at break} or \refKeyLe{/tcb/pad at break*}
should be used as very last option in an option list, because
they adapt other settings.
\end{marker}


\begin{marker}
Also see \refKeyLe{/tcb/enlarge top at break by}
and \refKeyLe{/tcb/enlarge bottom at break by}.
\end{marker}


\begin{docTcbKey}{height fixed for}{=\meta{part}}{no default, initially |none|}
  When certain amount of space is available for a partial box of a
  break sequence, the partial box typically is smaller than this space
  (depending on the box content). For given \meta{part}(s), the height can be
  set to all available space.
  \begin{itemize}
  \item\docValue{none}: Every partial |tcolorbox| is set with its natural height.
  \item\docValue{first}: The \emph{first} partial box is set to a height which matches the available space.
  \item\docValue{middle}: All \emph{middle} partial boxes are set to a height which matches the available space.
  \item\docValue{last}: The \emph{last} partial box is set to a height which matches
    the available space.
  \item\docValue{first and middle}: The \emph{first} and
    all \emph{middle} partial boxes are set to a height which matches the available space.
  \item\docValue{middle and last}: All \emph{middle} partial boxes and the \emph{last} partial box
    are set to a height which matches the available space.
  \item\docValue{all}: All partial boxes are set to a height which matches the available space.
  \end{itemize}
\begin{marker}
  If the box keeps unbroken, this option is not applied.
  See \refKeyLe{/tcb/height} for setting a fixed height for unbroken boxes.
  See \refKeyLe{/tcb/height fill} for giving unbroken boxes maximum height.
\end{marker}
\end{docTcbKey}


\begin{docTcbKey}{vfill before first}{\colOpt{=true\textbar false}}{default |true|, initially |false|}
  Inserts a |\vfill| at the begin of the \emph{first} partial box to move this
  partial box to the end of the current page. This may be used as an alternative
  to \refKeyLe{/tcb/height fixed for}|=|\docValue{first} to get justified
  columns or pages. The |\vfill| is not inserted, if the box gets not
  actually broken.
\end{docTcbKey}


\begin{docTcbKey}[][doc new=2017-03-20]{segmentation at break}{\colOpt{=true\textbar false}}{default |true|, initially |true|}
  If a breakable box contains an \emph{upper part} and a \emph{lower part} and
  the break happens at the \emph{segmentation} between both parts, then
  \begin{itemize}
  \item the segmenation line (or similar) is drawn as first element of the
    partial box containing the \emph{lower part}, if \refKeyLe{/tcb/segmentation at break}
    is set to be |true|.
  \item the segmenation line (or similar) is not drawn at all, if
    \refKeyLe{/tcb/segmentation at break} is set to be |false|.
    This may be preferable for skins like \refSkinLe{bicolor}, \refSkinLe{tile},
    or \refSkinLe{beamer}.
  \end{itemize}
\end{docTcbKey}


\clearpage
\subsection{Extra Options for Partial Boxes}\label{subsec:extras}


\begin{docTcbKey}[][doc new=2015-07-16]{extras}{=\marg{options}}{no default, initially unset}
  Adds |tcolorbox| \meta{options} to every box of a \emph{break sequence}
  after skin settings are done. This is quite late in box processing.
  Geometry and break settings should \emph{not be used} here, because they
  will either be ignored or have unexpected negative results. But it is possible
  to change most colors, skin effects, shadows, borders, frame code, etc.
  Note that using \refKeyLe{/tcb/extras} for every box is very seldom an
  advantage over setting the options directly. Usually, \refKeyLe{/tcb/extras first},
  \refKeyLe{/tcb/extras middle}, etc.\ are sensible to apply.
\end{docTcbKey}


\begin{docTcbKey}[][doc new=2015-07-16]{no extras}{}{style, no default, initially set}
  Removes all extras if set before.
\end{docTcbKey}


\begin{docTcbKey}[][doc new=2015-07-16]{extras broken}{=\marg{options}}{no default, initially unset}
  If the box is set to be \refKeyLe{/tcb/breakable} and \emph{is} broken actually,
  then the \meta{options} are added to every box of the \emph{break sequence}.
  \refKeyLe{/tcb/extras} overwrites this key.
\end{docTcbKey}

\begin{docTcbKey}[][doc new=2015-07-16]{extras unbroken}{=\marg{options}}{no default, initially unset}
  If the box is set to be \refKeyLe{/tcb/breakable} but \emph{is not} broken actually
  or if the box is set to be \refKeyLe{/tcb/unbreakable},
  then the \meta{options} are added to the box.
  \refKeyLe{/tcb/extras} overwrites this key.
\end{docTcbKey}

\begin{docTcbKey}[][doc new=2015-07-16]{no extras unbroken}{}{style, no default, initially set}
  Removes the unbroken extras if set before.
\end{docTcbKey}

\begin{docTcbKey}[][doc new=2015-07-16]{extras first}{=\marg{options}}{no default, initially unset}
  If the box is set to be \refKeyLe{/tcb/breakable} and \emph{is} broken actually,
  then the \meta{options} are added to the \emph{first} box of the break sequence.
  \refKeyLe{/tcb/extras} overwrites this key.
\end{docTcbKey}

\begin{docTcbKey}[][doc new=2015-07-16]{no extras first}{}{style, no default, initially set}
  Removes the first extras if set before.
\end{docTcbKey}

\begin{docTcbKey}[][doc new=2015-07-16]{extras middle}{=\marg{options}}{no default, initially unset}
  If the box is set to be \refKeyLe{/tcb/breakable} and \emph{is} broken actually,
  then the \meta{options} are added to every \emph{middle} box (if any) of the break sequence.
  \refKeyLe{/tcb/extras} overwrites this key.
\end{docTcbKey}

\begin{docTcbKey}[][doc new=2015-07-16]{no extras middle}{}{style, no default, initially set}
  Removes the middle extras if set before.
\end{docTcbKey}

\begin{docTcbKey}[][doc new=2015-07-16]{extras last}{=\marg{options}}{no default, initially unset}
  If the box is set to be \refKeyLe{/tcb/breakable} and \emph{is} broken actually,
  then the \meta{options} are added to the \emph{last} box of the break sequence.
  \refKeyLe{/tcb/extras} overwrites this key.
\end{docTcbKey}

\begin{docTcbKey}[][doc new=2015-07-16]{no extras last}{}{style, no default, initially set}
  Removes the last extras if set before.
\end{docTcbKey}

\begin{docTcbKey}[][doc new=2015-07-16]{extras unbroken and first}{=\marg{options}}{no default, initially unset}
  This is an abbreviation for setting
  \refKeyLe{/tcb/extras unbroken} and
  \refKeyLe{/tcb/extras first} together.
  \refKeyLe{/tcb/extras} overwrites this key.
\end{docTcbKey}

\begin{docTcbKey}[][doc new=2015-07-16]{extras middle and last}{=\marg{options}}{no default, initially unset}
  This is an abbreviation for setting
  \refKeyLe{/tcb/extras middle} and
  \refKeyLe{/tcb/extras last} together.
  \refKeyLe{/tcb/extras} overwrites this key.
\end{docTcbKey}

\begin{docTcbKey}[][doc new=2015-07-16]{extras unbroken and last}{=\marg{options}}{no default, initially unset}
  This is an abbreviation for setting
  \refKeyLe{/tcb/extras unbroken} and
  \refKeyLe{/tcb/extras last} together.
  \refKeyLe{/tcb/extras} overwrites this key.
\end{docTcbKey}

\clearpage

\begin{docTcbKey}[][doc new=2015-07-16]{extras first and middle}{=\marg{options}}{no default, initially unset}
  This is an abbreviation for setting
  \refKeyLe{/tcb/extras first} and
  \refKeyLe{/tcb/extras middle} together.
  \refKeyLe{/tcb/extras} overwrites this key.
\end{docTcbKey}


\begin{docTcbKey}[][doc new=2018-07-26]{extras title after break}{=\marg{options}}{no default, initially unset}
  If the box has a \refKeyLe{/tcb/title after break}, then the \meta{options}
  are added for all titles after the first break, i.e.\ all middle and last.
  The color, font, and alignment of titles after break can be adapted choosing
  \meta{options}, e.g.\ by \refKeyLe{/tcb/coltitle}, \refKeyLe{/tcb/fonttitle},
  \refKeyLe{/tcb/halign title}.
  Note that \refKeyLe{/tcb/colbacktitle} has to be placed into
  \refKeyLe{/tcb/extras middle and last}.
\end{docTcbKey}

\begin{docTcbKey}[][doc new=2018-07-26]{no extras title after break}{}{style, no default, initially set}
  Removes the title after break extras if set before.
\end{docTcbKey}

\bigskip

\begin{exdispExample}{extras}
% \usepackage{lipsum,multicol}
% \usetikzlibrary{decorations.pathmorphing}
% \tcbuselibrary{skins}
\newtcolorbox{mybox}[1][]{
  tile,
  colback=green!7,coltitle=blue!50!black,colbacktitle=blue!5,
  center title,
  toprule=1.25mm,bottomrule=1.25mm,
  extras unbroken and first={
    borderline north={0.25mm}{0.5mm}{blue,decoration={zigzag,amplitude=0.5mm},decorate}},
  extras unbroken and last={
    borderline south={0.25mm}{0.5mm}{blue,decoration={zigzag,amplitude=0.5mm},decorate}},
  #1
}

\begin{mybox}[title=My unbroken box]
\lipsum[1]
\end{mybox}

\begin{multicols}{3}
  \begin{mybox}[title=My broken box,
    enforce breakable,% use only breakable in the real world!
    break at=4.2cm,pad at break=2mm,
    height fixed for=first and middle,  ]
  \lipsum[2]
  \end{mybox}
\end{multicols}
\end{exdispExample}




\clearpage
\subsection{Breakable boxes and the \texttt{multicol} package}\label{subsec:multicol}
\begin{marker}
With version 4.10, the algorithm for detecting the available height
for a |tcolorbox| inside a |multicol| environment was improved with help
of Frank Mittelbach. This change \emph{may} impact existing user
code which \emph{may} have to be adapted.
\end{marker}

\begin{multicols}{2}
\begin{tcolorbox}[enhanced jigsaw,size=small,breakable,colback=yellow!10!white,
  colframe=red!50!white,break at=3cm,height fixed for=all]
Unbreakable |tcolorbox|es can be used without special care inside a
|multicols| environment from the |multicol| package \cite{mittelbach:multicol}.

Since version 3.10, a breakable |tcolorbox| detects, if it is used inside
a |multicols| environment. But choosing break points for a breakable box
cannot be done by the balancing routine of |multicols|. By default, boxes
will break at maximum column height. To get pleasant results, use the
\refKeyLe{/tcb/break at} and \refKeyLe{/tcb/height fixed for} options.
\end{tcolorbox}
\end{multicols}

\enlargethispage{\baselineskip}
\begin{dispListing}
% \usepackage{lipsum,multicol}  % preamble
\footnotesize
\begin{multicols}{2}
  \lipsum[1]
  \begin{tcolorbox}[enhanced jigsaw,breakable,size=title,
    colback=red!5!white,colframe=red!75!black,fonttitle=\bfseries,
    title=My breakable box,pad at break=1mm, break at=-\baselineskip/0pt ]
  \lipsum[2-4]
  \end{tcolorbox}
  \lipsum[4]
\end{multicols}
\end{dispListing}
{\tcbusetemp}

\clearpage

\begin{multicols}{2}
\small
This example is already set inside a |multicols| environment.
This time, a \emph{middle} part has full column height (here |\textheight|).
\refKeyLe{/tcb/height fixed for} is used to spread this box part over the full
height to align with neighboring columns.
\begin{dispListing}
% \usepackage{lipsum,multicol}
\lipsum[1]
\begin{tcolorbox}[enhanced jigsaw,
  breakable,
  size=title,
  colback=red!5!white,
  colframe=red!75!black,
  fonttitle=\bfseries,
  title=My breakable box,
  pad at break=2mm,
  break at=-\baselineskip/0pt,
  height fixed for=middle ]
\lipsum[2-7]
\end{tcolorbox}
\lipsum[8]
\end{dispListing}
{\tcbusetemp}
\end{multicols}


The following example has a |\tcolorbox| which fills the |\multicols|
environment completely. Here, \refKeyLe{/tcb/height fixed for} is used
to give all three columns the full height.
Note that the appropriate \refKeyLe{/tcb/break at} value is not computed
automatically but set manually.

\begin{dispListing}
% \usepackage{lipsum,multicol}  % preamble
\small
\begin{multicols}{3}
  \begin{tcolorbox}[enhanced jigsaw,breakable,size=small,
    colback=red!5!white,colframe=red!75!black,fonttitle=\bfseries,
    title=My breakable box,pad at break=2mm,drop fuzzy shadow,
    height fixed for=all, break at=11.4cm ]
  \lipsum[1-3]
  \end{tcolorbox}
\end{multicols}
\end{dispListing}
{\tcbusetemp}

\clearpage
\subsection{Break Point Insertion}\label{subsec:breakpoints}

\begin{docCommand}[doc new=2017-07-05]{tcbbreak}{}
  A \emph{breakable} box is not broken, if there is enough
  space on the current page or column.
  Therefore, typical penalty insertion with
  |\break|, |\pagebreak|, |\columnbreak|, \ldots \emph{may} only work as
  expected, if the box is broken at least into two parts
  \emph{without} inserting the penalties.\par\smallskip
  To \emph{force} a page or column break, \refComLe{tcbbreak}
  starts a new paragraph and inserts an insane tall rule which causes a
  break and which is immediately discarded. You may ignore this technical
  information and just use it as you would use |\pagebreak|.\par\smallskip
  For an \emph{unbreakable box}, \refComLe{tcbbreak} is identical to insert |\par|,
  i.e.\ it just starts a new paragraph.\par\smallskip
  Also see \refKeyLe{/tcb/break at} for defining height dependend breaks.

\begin{dispListing}
% \usepackage{lipsum,multicol}  % preamble
\begin{multicols}{3}
  \begin{tcolorbox}[breakable,enhanced jigsaw,size=small,
    colback=red!5!white,colframe=red!75!black,fonttitle=\bfseries,
    title=Break into parts
  ]
  First part\tcbbreak
  Second part\tcbbreak
  Third part
  \end{tcolorbox}
\end{multicols}

\begin{multicols}{3}
  \begin{tcolorbox}[enhanced jigsaw,size=small,
    colback=red!5!white,colframe=red!75!black,fonttitle=\bfseries,
    title=You shall not break
  ]
  First part\tcbbreak
  Second part\tcbbreak
  Third part
  \end{tcolorbox}
\end{multicols}

\end{dispListing}
{\tcbusetemp}

\end{docCommand}



\clearpage
\subsection{Break Sequence for the Skins}\label{subsec:breaksequence}
The following diagrams document the \emph{break sequence} for different
skins. Depending on the main skin of a |tcolorbox|, the actual skins of
the \emph{break sequence} parts are displayed.

\def\tcbbreakskininto#1#2#3#4#5{%
\begin{center}\begin{tikzpicture}
\tcbset{width=7cm,colframe=Navy,colback=AliceBlue,fonttitle=\bfseries,
  watermark color=AliceBlue!85!Navy,#5
  }
\node[above] (unbroken) at (0,0) {\begin{tcolorbox}[title=Unbroken Box,skin=#1,watermark text=unbroken,height=3.8cm]
\texttt{skin=#1}
\end{tcolorbox}};
\node[above] (first) at (8.7,2.4) {\begin{tcolorbox}[title=Broken Boxes,skin=#2,watermark text=first,height=1.4cm]
\texttt{skin=#2}
\end{tcolorbox}};
\node[above] (middle) at (8.7,1.2) {\begin{tcolorbox}[skin=#3,watermark text=middle,height=1cm]
\texttt{skin=#3}
\end{tcolorbox}};
\node[above] (last) at (8.7,0) {\begin{tcolorbox}[skin=#4,watermark text=last,height=1cm]
\texttt{skin=#4}
\end{tcolorbox}};
\path[draw=FireBrick,line width=2pt,->] (unbroken) edge (first.west) edge (middle.west) edge (last.west);
\end{tikzpicture}\end{center}}

\tcbbreakskininto{standard}{standard}{standard}{standard}{watermark text/.style={}}
\tcbbreakskininto{standard jigsaw}{standard jigsaw}{standard jigsaw}{standard jigsaw}{watermark text/.style={}}
\tcbbreakskininto{spartan}{spartan}{spartan}{spartan}{}
\clearpage
\tcbbreakskininto{enhanced}{enhancedfirst}{enhancedmiddle}{enhancedlast}{}
\tcbbreakskininto{enhancedfirst}{enhancedfirst}{enhancedmiddle}{enhancedmiddle}{}
\tcbbreakskininto{enhancedmiddle}{enhancedmiddle}{enhancedmiddle}{enhancedmiddle}{}
\tcbbreakskininto{enhancedlast}{enhancedmiddle}{enhancedmiddle}{enhancedlast}{}
\clearpage
\tcbbreakskininto{enhanced jigsaw}{enhancedfirst jigsaw}{enhancedmiddle jigsaw}{enhancedlast jigsaw}{}
\tcbbreakskininto{enhancedfirst jigsaw}{enhancedfirst jigsaw}{enhancedmiddle jigsaw}{enhancedmiddle jigsaw}{}
\tcbbreakskininto{enhancedmiddle jigsaw}{enhancedmiddle jigsaw}{enhancedmiddle jigsaw}{enhancedmiddle jigsaw}{}
\tcbbreakskininto{enhancedlast jigsaw}{enhancedmiddle jigsaw}{enhancedmiddle jigsaw}{enhancedlast jigsaw}{}
\clearpage
{\tcbset{borderline={2pt}{0pt}{black!10!white}}%
\tcbbreakskininto{empty}{emptyfirst}{emptymiddle}{emptylast}{}
\tcbbreakskininto{emptyfirst}{emptyfirst}{emptymiddle}{emptymiddle}{}
\tcbbreakskininto{emptymiddle}{emptymiddle}{emptymiddle}{emptymiddle}{}
\tcbbreakskininto{emptylast}{emptymiddle}{emptymiddle}{emptylast}{}
}
\clearpage
\tcbbreakskininto{bicolor}{bicolorfirst}{bicolormiddle}{bicolorlast}{bicolor}
\tcbbreakskininto{bicolorfirst}{bicolorfirst}{bicolormiddle}{bicolormiddle}{bicolor}
\tcbbreakskininto{bicolormiddle}{bicolormiddle}{bicolormiddle}{bicolormiddle}{bicolor}
\tcbbreakskininto{bicolorlast}{bicolormiddle}{bicolormiddle}{bicolorlast}{bicolor}
\clearpage
\tcbbreakskininto{bicolor jigsaw}{bicolorfirst jigsaw}{bicolormiddle jigsaw}{bicolorlast jigsaw}{bicolor jigsaw}
\tcbbreakskininto{bicolorfirst jigsaw}{bicolorfirst jigsaw}{bicolormiddle jigsaw}{bicolormiddle jigsaw}{bicolor jigsaw}
\tcbbreakskininto{bicolormiddle jigsaw}{bicolormiddle jigsaw}{bicolormiddle jigsaw}{bicolormiddle jigsaw}{bicolor jigsaw}
\tcbbreakskininto{bicolorlast jigsaw}{bicolormiddle jigsaw}{bicolormiddle jigsaw}{bicolorlast jigsaw}{bicolor jigsaw}
\clearpage
\tcbbreakskininto{tile}{tilefirst}{tilemiddle}{tilelast}{tile,colbacktitle=Navy}
\tcbbreakskininto{tilefirst}{tilefirst}{tilemiddle}{tilemiddle}{tile,colbacktitle=Navy}
\tcbbreakskininto{tilemiddle}{tilemiddle}{tilemiddle}{tilemiddle}{tile,colbacktitle=Navy}
\tcbbreakskininto{tilelast}{tilemiddle}{tilemiddle}{tilelast}{tile,colbacktitle=Navy}
\clearpage
\tcbbreakskininto{beamer}{beamerfirst}{beamermiddle}{beamerlast}{beamer}
\tcbbreakskininto{beamerfirst}{beamerfirst}{beamermiddle}{beamermiddle}{beamer}
\tcbbreakskininto{beamermiddle}{beamermiddle}{beamermiddle}{beamermiddle}{beamer}
\tcbbreakskininto{beamerlast}{beamermiddle}{beamermiddle}{beamerlast}{beamer}
\clearpage
\tcbbreakskininto{widget}{widgetfirst}{widgetmiddle}{widgetlast}{widget}
\tcbbreakskininto{widgetfirst}{widgetfirst}{widgetmiddle}{widgetmiddle}{widget}
\tcbbreakskininto{widgetmiddle}{widgetmiddle}{widgetmiddle}{widgetmiddle}{widget}
\tcbbreakskininto{widgetlast}{widgetmiddle}{widgetmiddle}{widgetlast}{widget}
\tcbbreakskininto{draft}{draft}{draft}{draft}{draft}
\clearpage
\tcbbreakskininto{freelance}{freelancefirst}{freelancemiddle}{freelancelast}{}
\tcbbreakskininto{freelancefirst}{freelancefirst}{freelancemiddle}{freelancemiddle}{}
\tcbbreakskininto{freelancemiddle}{freelancemiddle}{freelancemiddle}{freelancemiddle}{}
\tcbbreakskininto{freelancelast}{freelancemiddle}{freelancemiddle}{freelancelast}{}




\clearpage
\subsection{Break by Hand (Faked Break)}

\begin{marker}
See \Vref{subsec:multicol} for \emph{real} column breaks.
\end{marker}

Since the appearance of broken boxes is done by skins, it is quite easy
to 'fake a break'. For this, you actually don't need the
\mylib{breakable} library at
all.

\begin{dispExample}
\tcbset{enhanced,equal height group=fakedbreak,
  colback=LightGreen,colframe=DarkGreen,
  width=(\linewidth-6mm)/3,nobeforeafter,
  left=1mm,right=1mm,top=1mm,bottom=1mm,middle=1mm}
%
\begin{tcolorbox}[title=My broken box,skin=enhancedfirst]
This is a box which breaks from one column to another
\end{tcolorbox}\hfill
\begin{tcolorbox}[skin=enhancedmiddle]
column. I am sorry to say that this is a trick.
Nevertheless, you may use this trick for your
\end{tcolorbox}\hfill
\begin{tcolorbox}[skin=enhancedlast]
own purposes.
\end{tcolorbox}
\end{dispExample}


%todo
% \setcounter{section}{19}
% % !TeX root = tcolorbox.tex
% include file of tcolorbox.tex (manual of the LaTeX package tcolorbox)
% \clearpage
\section{Library \mylib{magazine}}\label{sec:magazine}%
\tcbset{external/prefix=external/magazine_}%
\newboxarray{myarticle}

\begin{tcolorbox}[
enhanced jigsaw,
size=small,width=6cm,
title=文章示例,
fonttitle=\bfseries,center title,
fontupper=\small,
%height fixed for=first and middle,
watermark text=\arabic{tcbbreakpart},
breakable,
break at=8.5\baselineskip/9\baselineskip,
reset and store to box array=myarticle,
colframe=green!50!black,
colback=green!10,
pad at break=5mm,
]
This is an example for an article which starts right here and
is continued to the following pages.
The body text for the article is written inside a single
|tcolorbox|. This box is split into parts using the tools from
this section, namely \refKeyLe{/tcb/reset and store to box array}
with a new box array |myarticle| which was created by
|\newboxarray{myarticle}|.

这是一个从这里开始并继续到下一页的文章示例。
文章正文写在一个单独的 |tcolorbox| 中,这个盒子使用了本节的工具进行拆分,
即使用了 \refKeyLe{/tcb/reset and store to box array} 与新的盒子数组 |myarticle|,
该数组是由 |\newboxarray{myarticle}| 创建的。

The resulting parts are distributed throughout this
\Fullref{sec:magazine} using \refComLe{consumetcboxarray}
at the appropriate places you see.
The linking texts like \emph{continued on page x} are created
by \refKeyLe{/tcb/finish} commands for the embedding \refComLe{tcbox}.
To label the box parts, \refKeyLe{/tcb/phantomlabel} is used.

使用 \refComLe{consumetcboxarray} 将得到的部分分别分布到本章适当的位置。
像“在第 x 页继续”这样的链接文本是由嵌套的 \refComLe{tcbox} 的 \refKeyLe{/tcb/finish} 命令创建的。
使用 \refKeyLe{/tcb/phantomlabel} 标记盒子部分。

These quite small partial boxes are for demonstration purposes.
With the tools of this section, a magazine type document could be
created, but this still needs a lot of manual control.

这些非常小的部分盒子仅用于演示目的。使用本节的工具可以创建杂志类型的文档,但这仍需要很多手动控制。
\end{tcolorbox}

\newtcolorbox{articleside}[1][]{blanker,sidebyside,sidebyside gap=5mm,sidebyside align=top seam,
parbox=false,righthand width=6cm,
goto/.style={finish={\node[above=-2pt,color=green!50!black] at (frame.south)
  {\slshape\scriptsize --- continued on page~\hypersetup{linkcolor=green!50!black}\pageref{myarticle-##1}\ \textcolor{green!50!black}{---}};}},
from/.style={finish={\node[below=-1pt,color=green!50!black] at (frame.north)
  {\slshape\scriptsize --- continued from page~\hypersetup{linkcolor=green!50!black}\pageref{myarticle-##1}\ \textcolor{green!50!black}{---}};}},
#1}

\begin{articleside}[after skip=6pt]
The main purpose of this library is to store a |tcolorbox| into an array
of box registers for later usage.

这个库的主要目的是将 |tcolorbox| 存储到一个盒子寄存器数组中以供以后使用。

If the |tcolorbox| is not breakable, there is not much add-on
compared to usual \TeX/\LaTeX\ box storage and usage (and you do not really need this
library for that use case).
For a breakable |tcolorbox|, this library allows to capture all partial boxes
into a sequence of registers. The partial boxes can be used anywhere in
arbitrary order.

如果 |tcolorbox| 不可分割,则与通常的 \TeX/\LaTeX\ 盒子存储和使用相比,没有太多的补充(在这种情况下,您实际上不需要使用此库)。
对于可分割的 |tcolorbox|,此库允许捕获所有部分盒子到一系列寄存器中。部分盒子可以以任意顺序在任何地方使用。

\tcblower\consumetcboxarray[myarticle]{1}{blanker,nobeforeafter,phantomlabel=myarticle-one,goto=two}
\end{articleside}

The name of this library indicates \emph{magazine} in the sense of storage,
but also in the sense of a journal where an article often is \emph{continued on page x}.
An example for this kind of application is given throughout this section starting
on the right hand side. The creation of this library was motivated by
Ulrike Fischer and Steven B.~Segletes.

这个库的名称表明其用途是存储,也可以看作一种期刊,文章通常会在“第x页续上”。“magazine”的意思是杂志。
这个库的创建是由 Ulrike Fischer 和 Steven B. Segletes 提出的。

The library is loaded by a package option or inside the preamble by:

通过包选项或在导言区内使用如下命令载入库:
\begin{dispListing}
\tcbuselibrary{magazine}
\end{dispListing}
This also loads the library \mylib{breakable}, see \Fullref{sec:breakable}.

这也会载入 \mylib{breakable} 库,参见 \Fullref{sec:breakable}。
\begin{marker}
The box register operations of this library are global. \TeX\ grouping will
not clear the registers when leaving the current group. Also be aware that
extensive use of large box arrays may eat up \TeX's available memory and
registers.

这个库中的盒子寄存器操作是全局的,\TeX 的分组机制不能在离开当前组时清除寄存器。还要注意,大量使用大型盒子数组可能会耗尽 \TeX 的可用内存和寄存器。
\end{marker}

\subsection{Creation and Resetting of Box Arrays\\创建和重置盒子数组}\label{subsec:magazine_creation}

\begin{docCommand}[doc new=2015-07-13]{newboxarray}{\marg{name}}
This creates a new box array called \meta{name}. There already is a
box array available with name |default| which can be used directly.
Note that the creation is a global operation.

这将创建一个名为 \meta{name} 的新盒子数组。有一个名为 |default| 的盒子数组可供直接使用。注意,创建是全局操作。
\begin{dispListing}
\newboxarray{myarray}
\end{dispListing}
\end{docCommand}


\begin{docCommand}[doc new=2015-07-13]{boxarrayreset}{\oarg{name}}
Resets the size counter of a box array \meta{name} to zero.
If \meta{name} is not provided, |default| is used as name.
Use this or \refKeyLe{/tcb/reset box array} before
you apply \refKeyLe{/tcb/store to box array}. Otherwise, all boxes would
be appended to the already existing boxes.
This command does not clear box registers.

重置一个名为\meta{name}的盒子数组的尺寸计数器为0。如果没有提供\meta{name},则使用 |default| 作为名称。
在使用\refKeyLe{/tcb/store to box array}之前,使用此命令或\refKeyLe{/tcb/reset box array}。否则,所有的盒子都会被追加到已经存在的盒子中。
此命令不清除盒子寄存器。
\begin{dispListing}
\boxarrayreset            % resets `default'
\boxarrayreset{myarray}   % resets `myarray'
\end{dispListing}
\end{docCommand}

% \enlargethispage*{20mm}

\begin{docTcbKey}[][doc new=2015-07-13]{reset box array}{\colOpt{=\meta{name}}}{default |default|, initially unset}
Resets the size counter of a box array \meta{name} to zero.
Use this or \refComLe{boxarrayreset} (which does the same) before
you apply \refKeyLe{/tcb/store to box array}.

重置盒子数组 \meta{name} 的大小计数器为零。如果未提供 \meta{name},则默认使用 |default|。在应用 \refKeyLe{/tcb/store to box array} 之前使用此命令或 \refComLe{boxarrayreset}(执行相同操作)。否则,所有盒子都将附加到已存在的盒子上。该命令不清除盒子寄存器。
\begin{dispListing}
\tcbset{
reset box array,         % resets `default'
reset box array=myarray, % resets `myarray'
}
\end{dispListing}
\end{docTcbKey}

% \clearpage
\begin{docCommand}[doc new=2015-07-13]{boxarrayclear}{\oarg{name}}
Works like \refComLe{boxarrayreset} to reset the size counter of a
box array \meta{name} to zero. Additionally, all allocated box registers
of the box array are cleared of their content.
Note that the allocated box registers stay allocated. So, this may be
useful to clear memory, but not to free registers for other applications.
If \refComLe{consumeboxarray} or \refComLe{consumetcboxarray} was used to
apply the stored boxes, there is no advantage in using \refComLe{boxarrayclear}.

该命令与\refComLe{boxarrayreset}类似,将盒子阵列\meta{name}的大小计数器重置为零。此外,盒子阵列的所有已分配盒子寄存器都会清除其内容。请注意,分配的盒子寄存器仍保持分配状态。因此,这可能有助于清除内存,但不能释放盒子寄存器用于其他应用程序。如果使用\refComLe{consumeboxarray}或\refComLe{consumetcboxarray}应用存储的盒子,则使用\refComLe{boxarrayclear}没有任何优势。
\begin{dispListing}
\boxarrayclear            % clears `default'
\boxarrayclear{myarray}   % clears `myarray'
\end{dispListing}
\end{docCommand}



\subsection{Storing Content}\label{subsec:magazine_storing}

\begin{docTcbKey}[][doc new=2015-07-13]{store to box array}{\colOpt{=\meta{name}}}{default |default|, initially unset}
Stores a |tcolorbox| or all parts of a break sequence of a |tcolorbox| into
a box array \meta{name}. If no \meta{name} is given, the already existing |default|
box array is used. Otherwise, the box array has to be created beforehand
with \refComLe{newboxarray}. Note that the box has to be \refKeyLe{/tcb/breakable},
if the box shall break into several parts.
Typically, manual break points are additionally defined by \refKeyLe{/tcb/break at}.
Otherwise, the box parts will have a length of about |\textheight|.
For most use cases, a \refKeyLe{/tcb/reset box array} should be applied
to reset the box array counter.\enlargethispage*{2cm}


将 |tcolorbox| 或者 |tcolorbox| 的所有分段存储到盒子数组 \meta{name} 中。如果没有提供 \meta{name},则使用已经存在的 |default| 盒子数组。否则,需要事先用 \refComLe{newboxarray} 创建盒子数组。注意,如果要使盒子分成几个部分,则盒子必须是可分的(即具有 \refKeyLe{/tcb/breakable} 选项)。通常情况下,手动断点需要通过 \refKeyLe{/tcb/break at} 来添加。否则,盒子的每个部分将具有大约 |\textheight| 的长度。对于大多数用例,应该使用 \refKeyLe{/tcb/reset box array} 来重置盒子数组计数器。
\begin{exdispExample}{storetoboxarray_1}
% \usepackage{lipsum}
\begin{tcolorbox}[enhanced jigsaw,size=fbox,width=4cm,
colback=yellow!10,colframe=yellow!10!black,
enforce breakable,% use only breakable in the real world!
break at=7cm/4cm,
height fixed for=all,
watermark text=\arabic{tcbbreakpart},
reset box array,
store to box array
]
\lipsum[1]
\end{tcolorbox}

\useboxarray{1}\hfill
\begin{tabular}[b]{cc}
\multicolumn{2}{c}{\includegraphics[width=7cm]{/Users/virhuiai/hlProjects/%
Latex-Typesetting-Hub/宏包文档翻译/tcolorbox/Basilica_5.png}}\\
\useboxarray{2} & \useboxarray{3}
\end{tabular}
\end{exdispExample}

% \clearpage
If the first box part should fill the rest of the available space of
the current page, you can use |\pagegoal-\pagetotal| minus some distance for
the first element of \refKeyLe{/tcb/break at}. You may want to have some
additional distance to the preceeding text.

如果第一个盒子部分应该填充当前页面的所有剩余空间,您可以使用 $|\pagegoal-\pagetotal|$ 减去一些距离作为 \refKeyLe{/tcb/break at} 的第一个元素。您可能需要在前面的文本中添加一些额外的距离。
\begin{dispListing}
% \usepackage{lipsum}
\begin{tcolorbox}[enhanced,breakable,
reset box array,
store to box array,
break at=\pagegoal-\pagetotal-5mm/0pt,
height fixed for=first and middle]
\lipsum[1-15]
\end{tcolorbox}%
%
\consumetcboxarray{1}{blanker,before=\par\vfill\noindent}
\end{dispListing}


\begin{exdispExample}{storetoboxarray_2}
\begin{tcolorbox}[blanker,width=4cm,
fontupper=\footnotesize,
enforce breakable,% use only breakable in the real world!
break at=4cm,
height fixed for=all,
watermark text=\arabic{tcbbreakpart},
reset box array,
store to box array
]
\includegraphics[width=\linewidth]{/Users/virhuiai/hlProjects/%
Latex-Typesetting-Hub/宏包文档翻译/tcolorbox/Basilica_5.png}\par
\lipsum[1-2]
\end{tcolorbox}

\begin{tcbitemize}[raster columns=3,raster equal height,
size=small,halign=center,sharp corners,colback=blue!5]
\tcbitem\consumeboxarray{5}
\tcbitem\consumeboxarray{6}
\tcbitem\consumeboxarray{1}
\tcbitem\consumeboxarray{2}
\tcbitem\consumeboxarray{3}
\tcbitem\consumeboxarray{4}
\end{tcbitemize}
\end{exdispExample}
\end{docTcbKey}


\begin{docTcbKey}[][doc new=2015-07-13]{reset and store to box array}{\colOpt{=\meta{name}}}{style, default |default|, initially unset}
Combination of \refKeyLe{/tcb/reset box array} and \refKeyLe{/tcb/store to box array}.

\refKeyLe{/tcb/reset box array} 和 \refKeyLe{/tcb/store to box array} 的组合。
\end{docTcbKey}



\begin{docTcbKey}[][doc new=2015-07-13]{do not store to box array}{}{style, no default, initially set}
Disables the \refKeyLe{/tcb/store to box array} option, if set before.

禁用 \refKeyLe{/tcb/store to box array} 选项,如果之前已设置。
\end{docTcbKey}


\begin{docEnvironment}[doc new=2015-07-13]{boxarraystore}{\marg{name}}
Stores the environment content into a box array \meta{name}.
This corresponds to the standard \LaTeX\ environment |lrbox|, but
the storage operation is global. As long as \refComLe{boxarrayreset} is
not used, every new \refEnvLe{boxarraystore} adds a further box to
the array.

将环境内容存储到一个名为 \meta{name} 的盒子数组中。这对应于标准的 \LaTeX\ 环境 |lrbox|,但存储操作是全局的。只要不使用 \refComLe{boxarrayreset},每个新的 \refEnvLe{boxarraystore} 都会向数组中添加一个新的盒子。
\begin{dispExample}
\boxarrayreset
\begin{boxarraystore}{default}\fbox{Mary}\end{boxarraystore}
\begin{boxarraystore}{default}\fbox{Had}\end{boxarraystore}
\begin{boxarraystore}{default}\fbox{a}\end{boxarraystore}
\begin{boxarraystore}{default}\fbox{Little}\end{boxarraystore}
\begin{boxarraystore}{default}\fbox{Lamb}\end{boxarraystore}
\useboxarray{5}\useboxarray{4}\useboxarray{3}\useboxarray{2}\useboxarray{1}\hfill
\useboxarray{1}\useboxarray{5}
\end{dispExample}
\end{docEnvironment}

\subsection{Retrieving Content}\label{subsec:magazine_retrieve}

\begin{docCommand}[doc new=2015-07-13]{boxarraygetsize}{\oarg{name}\marg{macro}}
\begin{articleside}[before skip=5pt]
Stores the current size of a box array \meta{name} into a given \meta{macro}.
If no \meta{name} is given, the already existing |default| box array is used.

将盒子数组\meta{name}的当前大小存储到给定的\meta{macro}中。如果没有给出\meta{name},则使用已经存在的|default|盒子数组。
\begin{dispExample}
\boxarraygetsize{\mysize}
默认盒子数组的当前大小:
\mysize.
\end{dispExample}
\tcblower\consumetcboxarray[myarticle]{2}{blanker,nobeforeafter,phantomlabel=myarticle-two,from=one,goto=three}
\end{articleside}
\end{docCommand}

\begin{docCommand}[doc new=2015-07-13]{useboxarray}{\oarg{name}\marg{index}}
Typesets the box with the given \meta{index} number from the box array \meta{name}.
If no \meta{name} is given, the already existing |default| box array is used.
It is considered an error, if a not existing box array \meta{name} is used.
It is silently ignored, if the \meta{index} is out of range.
Note that \refComLe{useboxarray} corresponds to the standard |\usebox| macro,
respectively, |\copy|.

从盒子数组\meta{name}中获取给定\meta{index}号的盒子并排版。如果没有给出\meta{name},则使用已经存在的|default|盒子数组。如果使用不存在的盒子数组\meta{name},则视为错误。如果\meta{index}超出范围,则静默忽略。请注意,\refComLe{useboxarray}对应于标准的|\usebox|宏,即|\copy|。
\begin{dispExample}
\boxarraygetsize{\mysize}
\foreach \n in  {1,...,\mysize} { \useboxarray{\n} }
\end{dispExample}
\end{docCommand}

% \clearpage
\begin{docCommand}[doc new=2015-07-13]{usetcboxarray}{\oarg{name}\marg{index}\marg{options}}
Typesets the box with the given \meta{index} number from the box array \meta{name}
using \refComLe{useboxarray} as content of a \refComLe{tcbox}.
If no \meta{name} is given, the already existing |default| box array is used.
It is considered an error, if a not existing box array \meta{name} is used.
It is silently ignored, if the \meta{index} is out of range.
The \refComLe{tcbox} can be customized by |tcolorbox| \meta{options}.

从盒子数组\meta{name}中获取给定\meta{index}号的盒子并使用\refComLe{useboxarray}作为\refComLe{tcbox}的内容进行排版。如果没有给出\meta{name},则使用已经存在的|default|盒子数组。如果使用不存在的盒子数组\meta{name},则视为错误。如果\meta{index}超出范围,则静默忽略。可以通过|tcolorbox| \meta{options}来定制\refComLe{tcbox}。
\begin{dispExample}
\boxarraygetsize{\mysize}
\foreach \n in  {1,...,\mysize} { \usetcboxarray{\n}{on line,colframe=yellow,
colback=yellow!10} }
\end{dispExample}
\end{docCommand}


\begin{docCommand}[doc new=2015-07-13]{consumeboxarray}{\oarg{name}\marg{index}}
Typesets the box with the given \meta{index} number from the box array \meta{name}.
If no \meta{name} is given, the already existing |default| box array is used.
It is considered an error, if a not existing box array \meta{name} is used.
It is silently ignored, if the \meta{index} is out of range.
In contrast to \refComLe{useboxarray},
\refComLe{consumeboxarray} corresponds to the standard |\box| macro, i.e.
after typesetting the box register is cleared and cannot be used again.

从盒子数组\meta{name}中获取给定\meta{index}号的盒子并进行排版。如果没有给出\meta{name},则使用已经存在的|default|盒子数组。如果使用不存在的盒子数组\meta{name},则视为错误。如果\meta{index}超出范围,则静默忽略。与\refComLe{useboxarray}不同,\refComLe{consumeboxarray}对应于标准的|\box|宏,即在排版盒子后,盒子寄存器被清空,不能再次使用。
\begin{dispExample}
\boxarraygetsize{\mysize}
First run: \foreach \n in  {1,...,\mysize} { \consumeboxarray{\n} }
\par
Second run: \foreach \n in  {1,...,\mysize} { \consumeboxarray{\n} }
\end{dispExample}
\end{docCommand}


\begin{docCommand}[doc new=2015-07-13]{consumetcboxarray}{\oarg{name}\marg{index}\marg{options}}
\begin{articleside}[before skip=5pt]
Typesets the box with the given \meta{index} number from the box array \meta{name}
using \refComLe{consumeboxarray} as content of a \refComLe{tcbox}.
If no \meta{name} is given, the already existing |default| box array is used.
It is considered an error, if a not existing box array \meta{name} is used.
It is silently ignored, if the \meta{index} is out of range.
The \refComLe{tcbox} can be customized by |tcolorbox| \meta{options}.
After typesetting the box register is cleared and cannot be used again.

使用给定的索引号从盒子数组\meta{name}中提取盒子,并使用\refComLe{consumeboxarray}作为\refComLe{tcbox}的内容进行排版。如果没有给出\meta{name},则使用已存在的|default|盒子数组。如果使用了不存在的盒子数组\meta{name},则会发生错误。如果\meta{index}超出范围,则会被静默忽略。可以通过|tcolorbox| \meta{options}自定义\refComLe{tcbox}。在排版盒子后,盒子寄存器将被清空,不能再次使用。
\tcblower\consumetcboxarray[myarticle]{3}{blanker,nobeforeafter,phantomlabel=myarticle-three,,from=two,goto=four}
\end{articleside}
\begin{exdispExample}{consumetcboxarray}
% \usepackage{lipsum}
\begin{tcolorbox}[enhanced jigsaw,size=fbox,width=6cm,
colback=yellow!10,colframe=yellow!10!black,
enforce breakable,% use only breakable in the real world!
break at=5cm,
watermark text=\arabic{tcbbreakpart},
reset and store to box array
]
\lipsum[1]
\end{tcolorbox}

\consumeboxarray{2} \hfill \consumeboxarray{1} \hfill \consumeboxarray{1}
\end{exdispExample}
\end{docCommand}


\begin{docCommand}[doc new=2015-07-13]{boxarraygetbox}{\oarg{name}\marg{macro}\marg{index}}
Assigns the box with the given \meta{index} number from the box array \meta{name}
to a \meta{macro}.
If no \meta{name} is given, the already existing |default| box array is used.
It is considered an error, if a not existing box array \meta{name} is used.
If the \meta{index} is out of range, the \meta{macro} will be undefined.

将来自盒子阵列 \meta{name} 的具有给定 \meta{index} 编号的盒子分配给 \meta{macro}。
如果未给出 \meta{name},则使用已经存在的 |default| 盒子阵列。
如果使用不存在的盒子阵列 \meta{name},则会出现错误。
如果 \meta{index} 超出范围,则 \meta{macro} 将未定义。
\begin{exdispExample}{boxarraygetbox}
\tcbox[size=small,colframe=blue!20,colback=yellow!5,on line,
reset and store to box array]{Test}

\boxarraygetsize{\mysize} Array size: \mysize

\boxarraygetbox{\mybox}{1}
Box width: \the\wd\mybox
\quad\usebox{\mybox}
\end{exdispExample}
\end{docCommand}


\begin{docCommand}[doc new=2017-06-27]{ifboxarrayempty}{\oarg{name}\marg{index}\marg{true}\marg{false}}
Tests the box with the given \meta{index} number from the box array \meta{name}
for emptiness be empty and executes \meta{true} if it is empty, and \meta{false} otherwise.
If no \meta{name} is given, the already existing |default| box array is used.
It is considered an error, if a not existing box array \meta{name} is used.

测试来自盒子数组 \meta{name} 的具有给定 \meta{index} 编号的盒子是否为空,如果为空则执行 \meta{true},否则执行 \meta{false}。如果没有给出 \meta{name},则使用已经存在的 |default| 盒子数组。如果使用不存在的盒子数组 \meta{name},则会产生错误。
\begin{exdispExample}{ifboxarrayempty}
\tcbox[size=small,colframe=blue!20,colback=yellow!5,on line,
reset and store to box array]{Test}

\ifboxarrayempty{1}{no Box~1}{Box~1: \useboxarray{1}},
\ifboxarrayempty{2}{no Box~2}{Box~2: \useboxarray{2}}
\end{exdispExample}
\end{docCommand}


% \clearpage
\subsection{Box Dimensions}\label{subsec:magazine_dimensions}

\begin{docCommand}[doc new=2015-07-13]{boxarraygetwidth}{\oarg{name}\marg{macro}\marg{index}}
Assigns the width of the box with the given \meta{index} number from the box array \meta{name}
to a \meta{macro}.
If no \meta{name} is given, the already existing |default| box array is used.
It is considered an error, if a not existing box array \meta{name} is used.
If the \meta{index} is out of range, the \meta{macro} will be set to |0pt|.

将来自盒子数组\meta{name}中给定\meta{index}编号的盒子的宽度赋值给\meta{macro}。如果没有给出\meta{name},则使用已存在的 |default| 盒子数组。如果使用了不存在的盒子数组\meta{name},则会发生错误。如果\meta{index}超出范围,则\meta{macro}将被设置为|0pt|。
\begin{exdispExample}{boxarraygetwidth}
\tcbox[size=small,colframe=blue!20,colback=yellow!5,on line,
reset and store to box array]{Test}

\begin{tabular}{ll}
\useboxarray{1} & width of box 1: \boxarraygetwidth{\mylen}{1} \mylen\\
\useboxarray{2} & width of box 2: \boxarraygetwidth{\mylen}{2} \mylen
\end{tabular}
\end{exdispExample}
\end{docCommand}


\begin{docCommand}[doc new=2015-07-13]{boxarraygetheight}{\oarg{name}\marg{macro}\marg{index}}
Assigns the height of the box with the given \meta{index} number from the box array \meta{name}
to a \meta{macro}.
If no \meta{name} is given, the already existing |default| box array is used.
It is considered an error, if a not existing box array \meta{name} is used.
If the \meta{index} is out of range, the \meta{macro} will be set to |0pt|.

将来自盒子数组\meta{name}中具有给定\meta{index}编号的盒子的高度分配给\meta{macro}。
如果未给出\meta{name},则使用已经存在的|default|盒子数组。
如果使用不存在的盒子数组\meta{name},则会出错。
如果\meta{index}超出范围,则将\meta{macro}设置为|0pt|。
\begin{exdispExample}{boxarraygetheight}
\tcbox[size=small,colframe=blue!20,colback=yellow!5,on line,
reset and store to box array]{Test}

\begin{tabular}{ll}
\useboxarray{1} & height of box 1: \boxarraygetheight{\mylen}{1} \mylen\\
\useboxarray{2} & height of box 2: \boxarraygetheight{\mylen}{2} \mylen
\end{tabular}
\end{exdispExample}
\end{docCommand}


\begin{docCommand}[doc new=2015-07-13]{boxarraygetdepth}{\oarg{name}\marg{macro}\marg{index}}
Assigns the depth of the box with the given \meta{index} number from the box array \meta{name}
to a \meta{macro}.
If no \meta{name} is given, the already existing |default| box array is used.
It is considered an error, if a not existing box array \meta{name} is used.
If the \meta{index} is out of range, the \meta{macro} will be set to |0pt|.

将来自盒子数组\meta{name}中具有给定\meta{index}编号的盒子的深度分配给\meta{macro}。
如果未给出\meta{name},则使用已经存在的|default|盒子数组。
如果使用不存在的盒子数组\meta{name},则会出错。
如果\meta{index}超出范围,则将\meta{macro}设置为|0pt|。
\begin{exdispExample}{boxarraygetdepth}
\tcbox[size=small,colframe=blue!20,colback=yellow!5,on line,
reset and store to box array]{Test}

\begin{tabular}{ll}
\useboxarray{1} & depth of box 1: \boxarraygetdepth{\mylen}{1} \mylen\\
\useboxarray{2} & depth of box 2: \boxarraygetdepth{\mylen}{2} \mylen
\end{tabular}
\end{exdispExample}
\end{docCommand}


% \clearpage
\begin{docCommand}[doc new=2015-07-13]{boxarraygettotalheight}{\oarg{name}\marg{macro}\marg{index}}
\begin{articleside}[before skip=5pt]
Assigns the total height of the box with the given \meta{index} number from the box array \meta{name}
to a \meta{macro}.
If no \meta{name} is given, the already existing |default| box array is used.
It is considered an error, if a not existing box array \meta{name} is used.
If the \meta{index} is out of range, the \meta{macro} will be set to |0pt|.

将来自盒子数组 \meta{name} 中给定 \meta{index} 编号的盒子的总高度赋值给 \meta{macro}。
如果未给出 \meta{name},则使用已经存在的 |default| 盒子数组。
如果使用不存在的盒子数组 \meta{name},则被视为错误。
如果 \meta{index} 超出范围,则将 \meta{macro} 设置为 |0pt|。
\tcblower\consumetcboxarray[myarticle]{4}{blanker,nobeforeafter,phantomlabel=myarticle-four,from=three}
\end{articleside}
\begin{exdispExample}{boxarraygettotalheight}
\boxarrayreset
\tcbox[size=small,colframe=blue!20,colback=yellow!5,on line,
store to box array]{Test}

\begin{tabular}{ll}
\useboxarray{1} & total height of box 1: \boxarraygettotalheight{\mylen}{1} \mylen\\
\useboxarray{2} & total height of box 2: \boxarraygettotalheight{\mylen}{2} \mylen
\end{tabular}
\end{exdispExample}
\end{docCommand}

% \clearpage
\subsection{Leaflet Example}
The following full application example can be used to create leaflets.
Obviously, the code can be adapted and customized in many ways.

以下完整的应用示例可以用于创建传单。
显然,这段代码可以根据需要进行调整和定制。
\begin{tcblisting}{
enhanced jigsaw,lower separated=false,breakable,
listing style=mydocumentation,base example,
listing and comment,
pdf comment,freeze pdf,
comment style={raster columns=1},
compilable listing,
run pdflatex}
\documentclass[a4paper,landscape]{article}
\usepackage[noheadfoot,margin=0pt]{geometry}
\usepackage[skins,raster,magazine]{tcolorbox}
\usepackage{lipsum}

\newenvironment{leaflet}[1][]{%
\begin{tcolorbox}[nobeforeafter,empty,colback=white,
  sharp corners,size=minimal,left=10mm,right=10mm,top=10mm,bottom=10mm,
  width=\textwidth/3,
  breakable,
  break at=\textheight,
  height fixed for=all,
  reset box array,
  store to box array,#1]}
{\end{tcolorbox}%
  \begin{tcbitemize}[raster columns=3,raster equal skip=0pt,blankest]
    \tcbitem\consumeboxarray{5}
    \tcbitem\consumeboxarray{6}
    \tcbitem\consumeboxarray{1}
    \tcbitem\consumeboxarray{2}
    \tcbitem\consumeboxarray{3}
    \tcbitem\consumeboxarray{4}
  \end{tcbitemize}%
}

\pagestyle{empty}
\begin{document}

\begin{leaflet}[underlay={\node[above=5mm,font=\footnotesize]
  at (frame.south) {- \arabic{tcbbreakpart} -};}]
\includegraphics[width=\linewidth]{/Users/virhuiai/hlProjects/%
Latex-Typesetting-Hub/宏包文档翻译/tcolorbox/Basilica_5.png}
\begin{center}
\bfseries\LARGE Example
\end{center}

\section{Introduction}
\lipsum[1]

\section{Main Part A}
\lipsum[2-8]

\section{Main Part B}
\lipsum[9-15]

\section{Conclusion}
\lipsum[16-18]
\end{leaflet}

\end{document}
\end{tcblisting}


%todo
% \setcounter{section}{20}
% % !TeX root = tcolorbox.tex
% include file of tcolorbox.tex (manual of the LaTeX package tcolorbox)
% %
\section{Library \mylib{poster}}\label{sec:poster}%
\tcbset{external/prefix=external/poster_}%

The main purpose of this library is to support creation of single page posters
with |tcolorbox|es.

这个库的主要目的是支持使用|tcolorbox|创建单页海报。

A \refEnvLe{tcbposter} is a |tikzpicture| where |tcolorbox|es can be
placed in a column oriented manner using \refComLe{posterbox} commands.
This base concept is more or less copied from the great |baposter| package.

\refEnvLe{tcbposter}是一个|tikzpicture|,可以使用\refComLe{posterbox}命令以列为导向的方式将|tcolorbox|放置在其中。
这个基本概念或多或少地从伟大的|baposter|包中复制而来。

The \mylib{raster} library, see \Fullref{sec:raster}, can produce
similar looking results and may be more appropriate
depending on the actual project.

\mylib{raster}库(见\Fullref{sec:raster})可以产生类似的外观效果,并且可能更适合实际项目。
\begin{itemize}
\item The \mylib{raster} library has a flow oriented concept, just like a
  convential text flow. The text flow (box flow) is a merely endless ribbon
  which gets broken into lines (and paragraphs) and the lines are broken
  into pages. \mylib{raster} shapes the boxes to convenient sizes to fill
  lines and pages in a pleasant way.

  \mylib{raster}库具有一种流式概念,就像传统文本流一样。文本流(框流)仅是一个无限长的带子,被分成行(和段落),行被分成页面。\mylib{raster}将框体塑造成便于填充行和页面的方便尺寸。
\item The \mylib{tcbposter} library supports a quite free placement of
  boxes inside a page.
  Basically, boxes are placed like |node|s are placed inside a |tikzpicture|.
  In contrast to \mylib{raster}, this is a \emph{single} page
  and not a flow of pages.
  The poster is divided into columns and rows.
  There is a more or less gentle force to use the columns (or spans of columns)
  for positioning and sizing while the row placement is completely optional.

\mylib{tcbposter}库支持相当自由的框的位置放置。基本上,框像|tikzpicture|中的|node|一样放置。与\mylib{raster}相比,这是一个\emph{单}页而不是页面流。海报被分成列和行。有一种或多种强制使用列(或列的跨度)进行定位和调整大小,而行定位完全是可选的。
\end{itemize}
The creation of this library was motivated by Ignasi.

创建这个库的灵感来自于Ignasi。
\begin{marker}
Inside a |tikzpicture| there should be no embedded |tikzpicture|s.
This rule is violated by the \mylib{poster} library. Be aware that there
may be some unwanted interactions between the main |tikzpicture| and
the embedded ones inside the |tcolorbox|es.

在|tikzpicture|中,不应该有嵌入式|tikzpicture|。这个规则被\mylib{poster}库违反了。请注意,主|tikzpicture|和嵌入在|tcolorbox|中的那些之间可能存在一些不需要的相互作用。
\end{marker}

The library is loaded by a package option or inside the preamble by:

该库可以通过包选项或在导言区内加载:
\begin{dispListing}
\tcbuselibrary{poster}
\end{dispListing}
This also loads the libraries
\mylib{skins}, see \Fullref{sec:skins},
\mylib{breakable}, see \Fullref{sec:breakable},
\mylib{magazine}, see \Fullref{sec:magazine}, and
\mylib{fitting}, see \Fullref{sec:fitting}.

这也加载了库\mylib{skins}(见\Fullref{sec:skins})、\mylib{breakable}(见\Fullref{sec:breakable})、\mylib{magazine}(见\Fullref{sec:magazine})和\mylib{fitting}(见\Fullref{sec:fitting})。
%--------------------------
\subsection{Overview\\概述}\label{subsec:poster_overview}


\begin{tcolorbox}[base example,hyperurl={tcolorbox-tutorial-poster.pdf},title=Click me to see the tutorial]
You get the best overview of the \mylib{poster} library and its facilities,
if you look at the \textbf{Poster Tutorial} which is part of the |tcolorbox|
documentation:\par
\texttt{tcolorbox-tutorial-poster.pdf}

要了解\mylib{poster}库及其功能的最佳概述,请参阅\textbf{Poster Tutorial},它是|tcolorbox|文档的一部分:\par
\texttt{tcolorbox-tutorial-poster.pdf}
\end{tcolorbox}



%
%--------------------------
\subsection{Main Poster Environment\\主要海报环境}\label{subsec:poster_environment}

\begin{docEnvironment}[doc new=2017-07-03]{tcbposter}{\oarg{options}}
This creates a |tikzpicture| environment with suitable additional
settings defined by the given \meta{options}.
Basically, \refComLe{posterbox} and \refEnvLe{posterboxenv} are
used to place |tcolorboxes| as nodes into the environment,
but additional \tikzname\ code can also be used.
As \meta{options} all |/tcb/posterset/| keys may be applied, namely:

这将创建一个|tikzpicture|环境,并使用给定的\meta{options}定义适当的附加设置。
基本上,使用\refComLe{posterbox}和\refEnvLe{posterboxenv}将|tcolorbox|作为节点放置在环境中,
但也可以使用其他\tikzname\ 代码。
作为\meta{options},可以应用所有|/tcb/posterset/|关键字,即:
\begin{itemize}
\item\refKeyLe{/tcb/posterset/poster}: poster settings like columns, rows, sizes\ldots

海报设置,如列、行、大小等。
\item\refKeyLe{/tcb/posterset/coverage} and \refKeyLe{/tcb/posterset/no coverage}:
  settings for a surrounding |tcolorbox| for background and margins.

\refKeyLe{/tcb/posterset/coverage}和\refKeyLe{/tcb/posterset/no coverage}:
用于背景和边距的周围|tcolorbox|的设置。
\item\refKeyLe{/tcb/posterset/boxes}: style of the |tcolorbox|es used for the poster.

用于海报的|tcolorbox|的样式。
\item\refKeyLe{/tcb/posterset/fontsize}: scaling of used fonts.

所使用字体的缩放。
\end{itemize}

\begin{exdispExample}{tcbposter}
  \begin{tcbposter}[
    poster = {showframe,height=10cm,spacing=2mm},
    boxes  = {beamer,colframe=blue!50!black,colback=blue!50,colupper=yellow!50},
  ]
  \posterbox{name=A,column=3,row=2}{My first box}
  \posterbox[adjusted title=Second box]
            {name=B,column=2,span=2,below=A}{My second box}
  \posterbox[adjusted title=Third box]
            {name=C,column=2,between=B and bottom}{My third box}
  \end{tcbposter}
\end{exdispExample}
\end{docEnvironment}

%
Inside \refEnvLe{tcbposter}, there are several predefined \tikzname\ nodes.
These nodes share a common \refKeyLe{/tcb/poster/prefix} which is
|TCBPOSTER@| by default. This prefix is used to discriminate the
poster nodes from local nodes of any embedded |tikzpicture| environment.
You will never need this prefix using \refComLe{posterbox} and its
placement options, but if you want to refer to a predefined node using
pure \tikzname\ code.
The predefined nodes (shown without prefix) are:
  
在\refEnvLe{tcbposter}内,有几个预定义的\tikzname\ 节点。
这些节点共享一个通用的\refKeyLe{/tcb/poster/prefix},默认情况下为|TCBPOSTER@|。
此前缀用于将海报节点与任何嵌入的|tikzpicture|环境的本地节点区分开来。
使用\refComLe{posterbox}和其放置选项时,您永远不需要此前缀,
但是如果要使用纯\tikzname\ 代码引用预定义的节点。
预定义节点(不带前缀)为:
\begin{itemize}
\item|poster|: defines the bounding box of the poster (without the coverage).

定义海报的边界框(不包括边距)。
\item|top|: top position plus row spacing

顶部位置加上行间距。
\item|bottom|: bottom position minus row spacing

底部位置减去行间距。
\item|middle|: vertical middle position

竖直中间位置。
\item|col1|, |col2|, \ldots: bounding box of column~1, column~2, \ldots

列1、列2、$\ldots$的边界框。
\item|row1|, |row2|, \ldots: bounding box of row~1, row~2, \ldots

行1、行2、$\ldots$的边界框。
\end{itemize}
Further nodes are defined using the \refKeyLe{/tcb/posterloc/name} option.

使用\refKeyLe{/tcb/posterloc/name}选项可以定义更多的节点。
\begin{marker}
Never use a \refEnvLe{tcbposter} inside a \refEnvLe{tcbposter}.
But, if you do anyway, use a different \refKeyLe{/tcb/poster/prefix} for
the embedded poster or you surely get a total mess.

永远不要在\refEnvLe{tcbposter}内部使用另一个\refEnvLe{tcbposter}。
但是,如果您确实这么做了,请为嵌套海报使用不同的\refKeyLe{/tcb/poster/prefix},否则您肯定会陷入糟糕的境地。
\end{marker}

There are several properties inside a \refEnvLe{tcbposter} which may be useful
for advanced code (skip the following on first reading):

在\refEnvLe{tcbposter}内部有几个属性对于高级代码可能有用(第一次阅读时请跳过以下内容):
\begin{itemize}
\item\docAuxCommand{tcbposterwidth}: Width of the poster (without margins).

海报的宽度(不包括边距)。
\item\docAuxCommand{tcbposterheight}: Height of the poster (without margins).

海报的高度(不包括边距)。
\item\docAuxCommand{tcbpostercolspacing}: Column distance.

列距离。
\item\docAuxCommand{tcbposterrowspacing}: Row distance.

行距离。
\item\docAuxCommand{tcbpostercolumns}: Column quantity.

列数。
\item\docAuxCommand{tcbposterrows}: Row quantity.

行数。
\item\docAuxCommand{tcbpostercolwidth}: Width of a column.

列的宽度。
\item\docAuxCommand{tcbposterrowheight}: Height of a row.

行的高度。
\end{itemize}

% \medskip
\begin{docCommand}[doc new=2017-07-03]{tcbposterset}{\marg{options}}
Sets options for every following \refEnvLe{tcbposter} inside the current \TeX\ group.
For example, the numbers for rows and columns may be defined for the whole document by this:

设置当前\TeX\ 组内后续所有\refEnvLe{tcbposter}的选项。
例如,可以通过以下方式为整个文档定义行数和列数:
\begin{dispListing}
\tcbposterset{poster={columns=2,rows=3}}
\end{dispListing}
See \refEnvLe{tcbposter} for all feasible options.

有关所有可行选项,请参见\refEnvLe{tcbposter}。
\end{docCommand}


%
%--------------------------
\subsection{Poster Settings\\海报设置}\label{subsec:poster_settings}

\begin{postersetTcbKey}[][doc new=2017-07-03]{poster}{=\marg{option list}}{style, no default}
This option can be applied inside \refEnvLe{tcbposter} and \refComLe{tcbposterset}
to set the given poster \meta{option list}, e.g.

可以在\refEnvLe{tcbposter}和\refComLe{tcbposterset}内应用此选项以设置给定的海报\meta{option list},例如
\begin{dispListing}
\tcbposterset{poster={width=20cm,height=15cm}}
\end{dispListing}
For the \meta{option list}, see the following keys.

有关\meta{option list},请参见以下键。
\end{postersetTcbKey}


\begin{posterTcbKey}[][doc new=2017-07-03]{columns}{=\meta{number}}{no default, initially |3|}
Sets the \meta{number} of columns for a |tcbposter|.

设置 |tcbposter| 的列数为\meta{number}。
\begin{exdispExample}{columns}
  \begin{tcbposter}[
    poster = {showframe,columns=5,rows=2,spacing=1mm,height=4cm},
  ]
  \end{tcbposter}
\end{exdispExample}
\end{posterTcbKey}

\begin{posterTcbKey}[][doc new=2017-07-03]{rows}{=\meta{number}}{no default, initially |4|}
Sets the \meta{number} of rows for a |tcbposter|.

设置 |tcbposter| 的行数为\meta{number}。
\end{posterTcbKey}


\begin{posterTcbKey}[][doc new=2017-07-03]{colspacing}{=\meta{length}}{no default, initially |4mm|}
Sets \meta{length} as distance between columns.

设置列之间的距离为\meta{length}。
\end{posterTcbKey}

\begin{posterTcbKey}[][doc new=2017-07-03]{rowspacing}{=\meta{length}}{no default, initially |4mm|}
Sets \meta{length} as distance between rows.

设置行之间的距离为\meta{length}。
\end{posterTcbKey}

\begin{posterTcbKey}[][doc new=2017-07-03]{spacing}{=\meta{length}}{style, no default, initially |4mm|}
Sets \meta{length} as distance between columns and rows.

设置行和列之间的距离为\meta{length}。
\end{posterTcbKey}


\begin{posterTcbKey}[][doc new=2017-07-03]{showframe}{\colOpt{=true\textbar false}}{default |true|, initially |false|}
Displays a red auxiliary mesh as optical support during poster creation.
Also, every \refKeyLe{/tcb/posterloc/name} is displayed.

在海报创建过程中显示红色辅助网格作为视觉支持。还会显示每个\refKeyLe{/tcb/posterloc/name}。
\end{posterTcbKey}


\begin{posterTcbKey}[][doc new=2017-07-03]{width}{=\meta{length}}{no default, initially \cs{linewidth}}
Sets \meta{length} as width of the poster. For a typical poster, this has not
to be set manually. Especially, if \refKeyLe{/tcb/posterset/coverage} is present,
use |coverage={width=|\meta{length}|}| instead to change the overall width.

将\meta{length}设置为海报的宽度。对于典型的海报,不需要手动设置该值。特别是,如果存在\refKeyLe{/tcb/posterset/coverage},请改用|coverage={width=|\meta{length}|}|来更改整体宽度。
\end{posterTcbKey}

% \enlargethispage*{1cm}

\begin{posterTcbKey}[][doc new=2017-07-03]{height}{=\meta{length}}{no default, initially unset}
Sets \meta{length} as height of the poster. For a typical poster, this has not
to be set manually, but is set automatically to an appropriate value.
If \refKeyLe{/tcb/posterset/coverage} is present, use only one if any option
|coverage={height=|\meta{length}|}| or |poster={height=|\meta{length}|}|.

将 \meta{length} 设置为海报的高度。对于典型的海报,这不需要手动设置,但会自动设置为适当的值。如果存在 \refKeyLe{/tcb/posterset/coverage},则仅使用其中一个选项 |coverage={height=|\meta{length}|}| 或 |poster={height=|\meta{length}|}|。
\end{posterTcbKey}


\begin{posterTcbKey}[][doc new=2017-07-03]{prefix}{=\meta{name}}{no default, initially |TCBPOSTER@|}
\meta{name} is set as prefix for any \tikzname\ node which is generated
automatically by the \mylib{poster} library. This encompasses predefined
nodes like |top|, |bottom|, \ldots, and nodes defined by using
\refKeyLe{/tcb/posterloc/name}. Also, see~\Fullref{subsec:poster_environment}.
For a typical poster, this value can stay as it is.

将 \meta{name} 设置为 \tikzname\ 生成的任何节点的前缀,该节点是由 \mylib{poster} 库自动生成的。这包括预定义的节点,如 |top|,|bottom| 等,以及使用 \refKeyLe{/tcb/posterloc/name} 定义的节点。另请参见~\Fullref{subsec:poster_environment}。对于典型的海报,此值可以保持不变。
\end{posterTcbKey}


%--------------------------
\subsection{Coverage\\覆盖}\label{subsec:poster_coverage}

\begin{postersetTcbKey}[][doc new=2017-07-03]{coverage}{=\marg{option list}}{style, no default}
This option can be applied inside \refEnvLe{tcbposter} and \refComLe{tcbposterset}
and it adds an optional coverage for the poster which is a surrounding |tcolorbox|
with the given \meta{option list}. Here, margins and background settings
for the poster can be given.
The \emph{coverage} has several default |tcolorbox| settings
suitable for the purpose:

可以在 \refEnvLe{tcbposter} 和 \refComLe{tcbposterset} 中应用此选项,并将其添加为海报的可选覆盖层,这是一个带有给定 \meta{option list} 的环绕 |tcolorbox|。在此处,可以给出海报的边距和背景设置。 \emph{覆盖} 具有适用于此目的的多个默认 |tcolorbox| 设置:
\begin{dispListing}
enhanced, frame hidden, sharp corners, boxsep=0pt, boxrule=0pt,
top=4mm, bottom=4mm, left=4mm, right=4mm,
toptitle=2mm, bottomtitle=2mm, colback=white
\end{dispListing}

The \meta{option list} can contain any |tcolorbox| option.

\meta{option list} 可以包含任何 |tcolorbox| 选项。
\begin{exdispExample}{coverage}
\begin{tcbposter}[
  poster   = {showframe,spacing=1mm},
  coverage = {height=5cm,
              interior style={top color=yellow,bottom color=yellow!50!red},
              watermark text={My Poster},watermark color=white,
             },
]
\end{tcbposter}
\end{exdispExample}

\begin{itemize}
\item For a typical poster, the option \refKeyLe{/tcb/spread} will use the
whole page for the poster coverage.

对于典型的海报,选项 \refKeyLe{/tcb/spread} 将使用整个页面来覆盖海报。
\item Poster margins can be adapted by \refKeyLe{/tcb/left}, \refKeyLe{/tcb/right},
\refKeyLe{/tcb/top}, \refKeyLe{/tcb/bottom}.

海报边距可以通过 \refKeyLe{/tcb/left},\refKeyLe{/tcb/right},\refKeyLe{/tcb/top},\refKeyLe{/tcb/bottom} 调整。
\item Poster background can be changed by \refKeyLe{/tcb/colback},
\refKeyLe{/tcb/interior style}, \refKeyLe{/tcb/interior style image}, etc.

海报的背景可以通过\refKeyLe{/tcb/colback}、\refKeyLe{/tcb/interior style}、\refKeyLe{/tcb/interior style image}等方式进行更改。
\item Do not use \refKeyLe{/tcb/poster/width} and \refKeyLe{/tcb/poster/height}
  in combination with a \emph{coverage}. Note that you may use
  \refKeyLe{/tcb/width} and \refKeyLe{/tcb/height} inside
  the \emph{coverage} \meta{option list}. Note that this also is not
  necessary when \refKeyLe{/tcb/spread} is applied.

不要在使用\emph{coverage}时结合使用\refKeyLe{/tcb/poster/width}和\refKeyLe{/tcb/poster/height}。请注意,您可以在\emph{coverage} \meta{option list}内使用\refKeyLe{/tcb/width}和\refKeyLe{/tcb/height}。请注意,当应用\refKeyLe{/tcb/spread}时,这也是不必要的。
\end{itemize}
\end{postersetTcbKey}


\begin{postersetTcbKey}[][doc new=2017-07-03]{no coverage}{}{style, no value, initially set}
Removes the surrounding |tcolorbox| completely.

完全删除周围的|tcolorbox|。
\end{postersetTcbKey}

%
%--------------------------
\subsection{Common Box Settings\\通用框设置}\label{subsec:poster_boxsettings}


\begin{postersetTcbKey}[][doc new=2017-07-03]{boxes}{=\marg{option list}}{style, no default}
This option can be applied inside \refEnvLe{tcbposter} and \refComLe{tcbposterset}
and it is used to set up the style of the |tcolorbox|es inside the poster.
The \meta{option list} can contain any |tcolorbox| option, but box size
options are not assumed to be useful here, because the size will be
determined by the placement options.

此选项可以在\refEnvLe{tcbposter}和\refComLe{tcbposterset}内应用,用于设置海报内|tcolorbox|的样式。 \meta{option list}可以包含任何|tcolorbox|选项,但是框的大小选项不被认为在此有用,因为大小将由放置选项确定。
\begin{exdispExample}{boxes}
\begin{tcbposter}[
  poster   = {spacing=2mm,columns=3,rows=2},
  coverage = {height=5cm,
              interior style={top color=yellow,bottom color=yellow!50!red},
             },
  boxes    = {sharp corners=downhill,arc=3mm,boxrule=1mm,
              colback=white,colframe=cyan,
              title style={left color=black,right color=cyan},
              fonttitle=\bfseries}
]
  \posterbox[adjusted title=First]{column=1,row=1,span=2}{First box}
  \posterbox[adjusted title=Second]{column=1,row=2,span=2}{Second box}
  \posterbox[adjusted title=Third]{column=3,row=1,rowspan=2}{Third box}
\end{tcbposter}
\end{exdispExample}

\end{postersetTcbKey}


%--------------------------
\subsection{Font Scaling\\字体缩放}\label{subsec:poster_fontsize}

\begin{postersetTcbKey}[][doc new=2017-07-03]{fontsize}{=\meta{length}}{style, no default, initially unset}
This option can be applied inside \refEnvLe{tcbposter} and \refComLe{tcbposterset}.
It uses \refKeyLe{/tcb/fit basedim} and \refKeyLe{/tcb/fit fontsize macros}
to redefine |\normalsize| to \meta{length} and all other standard
font size macros like |\small| and |\large| accordingly.\par
This needs a freely scalable font family like |lmodern| to work.
If \refKeyLe{/tcb/posterset/fontsize} is not applied, there standard
font size macros are not changed in any way.

此选项可以在\refEnvLe{tcbposter}和\refComLe{tcbposterset}内应用。它使用\refKeyLe{/tcb/fit basedim}和\refKeyLe{/tcb/fit fontsize macros}重新定义|\normalsize|为\meta{length},并相应地重新定义所有其他标准字体大小宏,如|\small|和|\large|。\par
这需要一个自由缩放的字体系列,如|lmodern|才能工作。
如果未应用\refKeyLe{/tcb/posterset/fontsize},则标准字体大小宏不会以任何方式更改。
\begin{dispListing}
\begin{tcbposter}[
  poster   = {spacing=2mm,columns=3,rows=2},
  coverage = {height=5cm,
              interior style={top color=yellow,bottom color=yellow!50!red},
             },
  fontsize = 15pt,   % <--- \normalsize is now 15pt
]
...
\end{dispListing}
\end{postersetTcbKey}


%
%--------------------------
\subsection{Box Placement\\盒子放置}\label{subsec:poster_boxplacement}

\begin{docCommand}[doc new=2017-07-03]{posterbox}{\oarg{options}\marg{placement}\marg{box content}}
Inside a \refEnvLe{tcbposter} environment, this places a |tcolorbox| with
additional |tcolorbox| \meta{options} and the given \meta{box content}
at a place determined by \meta{placement}.
All \meta{placement} options are described in the following.
Note that \meta{box content} cannot contain \emph{verbatim} material,
see \refEnvLe{posterboxenv}.

在\refEnvLe{tcbposter}环境中,此命令可以将带有附加|tcolorbox| \meta{options}和给定的\meta{box content}的|tcolorbox|放置在由\meta{placement}确定的位置。所有\meta{placement}选项如下所述。请注意,\meta{box content}不能包含\emph{verbatim}材料,请参见\refEnvLe{posterboxenv}。
\begin{exdispExample}{posterbox}
\begin{tcbposter}[
  poster = {showframe,height=4cm,spacing=2mm,rows=2},
  boxes  = {beamer,colframe=blue!50!black,colback=blue!50,colupper=yellow!50},
]
\posterbox[title=My title]{name=A,column=2,row=2}{My first box}
\end{tcbposter}
\end{exdispExample}
\end{docCommand}

\begin{docEnvironment}[doc new=2017-07-03]{posterboxenv}{\oarg{options}\marg{placement}}
This is the environment version of \refComLe{posterbox}, i.e.\ inside a
\refEnvLe{tcbposter} environment, this places a |tcolorbox| with
additional |tcolorbox| \meta{options} and the given \meta{environment content}
at a place determined by \meta{placement}.
In contrast to \refComLe{posterbox}, the \meta{environment content} is
allowed to contain \emph{verbatim} material. Note that the implementation
of \refComLe{posterbox} is more efficient than the implementation of \refEnvLe{posterboxenv}.

这是\refComLe{posterbox}的环境版本,即在\refEnvLe{tcbposter}环境中,此命令可以将带有附加|tcolorbox| \meta{options}和给定的\meta{environment content}的|tcolorbox|放置在由\meta{placement}确定的位置。与\refComLe{posterbox}相比,\meta{environment content}允许包含\emph{verbatim}材料。请注意,\refComLe{posterbox}的实现比\refEnvLe{posterboxenv}的实现更高效。
% \enlargethispage*{1cm}
\begin{exdispExample}{posterboxenv}
\begin{tcbposter}[
  poster = {showframe,height=4cm,spacing=2mm,rows=2},
  boxes  = {size=small,beamer,
            colframe=blue!50!black,colback=blue!50,colupper=yellow!50},
]
\begin{posterboxenv}[title=My title]{name=A,column=2,between=top and bottom}
  My first box.
  \begin{tcblisting}{size=small,colback=yellow!10}
My \textbf{first}
poster listing.
  \end{tcblisting}
\end{posterboxenv}
\end{tcbposter}
\end{exdispExample}

\end{docEnvironment}


%
\begin{posterlocTcbKey}[][doc new=2017-07-03]{name}{=\meta{name}}{no default, initially |@|}
Sets \meta{name} as reference for the current \refComLe{posterbox} or
\refEnvLe{posterboxenv}.
A \tikzname\ shape name is constructed automatically as combination
of \refKeyLe{/tcb/poster/prefix} and \meta{name}.

将当前的 \refComLe{posterbox} 或 \refEnvLe{posterboxenv} 设置为 \meta{name} 的参考。
一个 \tikzname 的形状名称会自动构建为 \refKeyLe{/tcb/poster/prefix} 和 \meta{name} 的组合。
\begin{exdispExample}{name}
\begin{tcbposter}[
  poster = {showframe,height=2.5cm,spacing=2mm,rows=2},
  boxes  = {beamer,colframe=blue!50!black,colback=blue!50,colupper=yellow!50},
]
\posterbox{name=A,column=2,row=2}{My first box}
\node[below right=4mm,fill=yellow] (X) at (TCBPOSTER@poster.north west) {Example A};
\draw[blue,very thick,->] (X) |- (TCBPOSTER@A);
\end{tcbposter}
\end{exdispExample}
\end{posterlocTcbKey}


\begin{posterlocTcbKey}[][doc new=2017-07-03]{column}{=\meta{number}}{no default, initially |1|}
Places the box at the column denoted by \meta{number}. If \refKeyLe{/tcb/posterloc/span}
is not |1|, the box is aligned to the left side of column \meta{number}.

将盒子放置在第 \meta{number} 列。如果 \refKeyLe{/tcb/posterloc/span} 不是 |1|,则盒子对齐到第 \meta{number} 列的左侧。
\begin{exdispExample}{column}
\begin{tcbposter}[
  poster = {showframe,height=2.5cm,spacing=2mm,rows=2},
  boxes  = {beamer,colframe=blue!50!black,colback=blue!50,colupper=yellow!50},
]
\posterbox{row=1,column=2,span=2}{First box}
\posterbox{row=2,column=2,span=0.8}{Second box}
\end{tcbposter}
\end{exdispExample}
\end{posterlocTcbKey}

\enlargethispage*{1cm}
\begin{posterlocTcbKey}[][doc new=2017-07-03]{column*}{=\meta{number}}{no default, initially unset}
Places the box at the column denoted by \meta{number}. If \refKeyLe{/tcb/posterloc/span}
is not |1|, the box is aligned to the right side of column \meta{number}.

将盒子放置在第 \meta{number} 列。如果 \refKeyLe{/tcb/posterloc/span} 不是 |1|,则盒子对齐到第 \meta{number} 列的右侧。
\begin{exdispExample}{columnstar}
\begin{tcbposter}[
  poster = {showframe,height=2.5cm,spacing=2mm,rows=2},
  boxes  = {beamer,colframe=blue!50!black,colback=blue!50,colupper=yellow!50},
]
\posterbox{row=1,column*=2,span=2}{First box}
\posterbox{row=2,column*=2,span=0.8}{Second box}
\end{tcbposter}
\end{exdispExample}
\end{posterlocTcbKey}


%
\begin{posterlocTcbKey}[][doc new=2017-07-03]{span}{=\meta{number}}{no default, initially |1|}
Sets the width of the current box to span \meta{number} columns.
\meta{number} is also allowed to be a real number like |0.5| or |1.7|.
See \refKeyLe{/tcb/posterloc/column} and \refKeyLe{/tcb/posterloc/column*}
for examples.

将当前盒子的高度设置为跨越 \meta{number} 行。
\meta{number} 也可以是实数,如 |0.5| 或 |1.7|。
请参见 \refKeyLe{/tcb/posterloc/column} 和 \refKeyLe{/tcb/posterloc/column*} 的示例。
\end{posterlocTcbKey}

\begin{posterlocTcbKey}[][doc new=2017-07-03]{row}{=\meta{number}}{no default, initially unset}
If this option is applied, the box is placed at the row denoted by \meta{number}.
Also, the height is set as fixed according to \refKeyLe{/tcb/posterloc/rowspan}.

如果应用了此选项,则盒子放置在由 \meta{number} 指定的行上。
此外,高度根据 \refKeyLe{/tcb/posterloc/rowspan} 设置为固定值。
\begin{exdispExample}{row}
\begin{tcbposter}[
poster = {showframe,height=2.5cm,spacing=2mm,rows=2},
boxes  = {beamer,colframe=blue!50!black,colback=blue!50,colupper=yellow!50},
]
\posterbox{row=1,column=1}{First box}
\posterbox{row=1,column=2,rowspan=2}{Second box}
\posterbox[natural height]{row=1,column=3}{Third box}
\end{tcbposter}
\end{exdispExample}
\end{posterlocTcbKey}

\begin{posterlocTcbKey}[][doc new=2017-07-03]{rowspan}{=\meta{number}}{no default, initially |1|}
Sets the height of the current box to span \meta{number} rows.
\meta{number} is also allowed to be a real number like |0.5| or |1.7|.

将当前盒子的高度设置为跨越 \meta{number} 行。\meta{number} 也可以是实数,如 |0.5| 或 |1.7|。
\begin{exdispExample}{rowspan}
\begin{tcbposter}[
poster = {showframe,height=2.5cm,spacing=2mm,rows=2},
boxes  = {beamer,colframe=blue!50!black,colback=blue!50,colupper=yellow!50},
]
\posterbox{row=1,column=1,rowspan=0.9}{First box}
\posterbox{row=1,column=2,rowspan=1.5}{Second box}
\posterbox{row=1,column=3,rowspan=2}{Third box}
\end{tcbposter}
\end{exdispExample}
\end{posterlocTcbKey}

\begin{posterlocTcbKey}[][doc new=2017-07-03]{fixed height}{}{no value, initially |0pt|}
Sets the height of the current box span rows as denoted by
\refKeyLe{/tcb/posterloc/rowspan}.
This can be used, if not \refKeyLe{/tcb/posterloc/row}, but another
height placement option is applied.

此选项将当前盒子的高度设置为由 \refKeyLe{/tcb/posterloc/rowspan} 指定的跨越行数。如果不使用 \refKeyLe{/tcb/posterloc/row},而使用其他高度定位选项,则可以使用此选项。
\end{posterlocTcbKey}


%
\begin{posterlocTcbKey}[][doc new=2017-07-03]{below}{=\meta{name}}{no default, initially |top|}
The box is placed below another box with the given \meta{name}. Also,
\meta{name} can be a predefined node, see \Fullref{subsec:poster_environment}.

该选项将在给定的名称 \meta{name} 的盒子下方放置盒子。此外,\meta{name} 也可以是预定义节点,参见 \Fullref{subsec:poster_environment}。
\begin{exdispExample}{below}
\begin{tcbposter}[
poster = {showframe,height=3cm,spacing=2mm,rows=2},
boxes  = {beamer,colframe=blue!50!black,colback=blue!50,colupper=yellow!50},
]
\posterbox{name=A,column=1,below=top}{First box}
\posterbox{name=B,column=1,below=A}{Second box}
\posterbox{name=C,column=2,below=B}{Third box}
\posterbox{name=D,column=3,below=row1}{Fourth box}
\end{tcbposter}
\end{exdispExample}
\end{posterlocTcbKey}


\begin{posterlocTcbKey}[][doc new=2017-07-03]{above}{=\meta{name}}{no default, initially unset}
The box is placed above another box with the given \meta{name}. Also,
\meta{name} can be a predefined node, see \Fullref{subsec:poster_environment}.

这个框被放置在另一个具有给定\meta{name}的框的上方。同时,\meta{name}可以是预定义的节点,见\Fullref{subsec:poster_environment}。
\begin{exdispExample}{above}
\begin{tcbposter}[
poster = {showframe,height=3cm,spacing=2mm,rows=2},
boxes  = {beamer,colframe=blue!50!black,colback=blue!50,colupper=yellow!50},
]
\posterbox{name=A,column=1,above=bottom}{First box}
\posterbox{name=B,column=1,above=A}{Second box}
\posterbox{name=C,column=2,above=B}{Third box}
\posterbox{name=D,column=3,above=row2}{Fourth box}
\end{tcbposter}
\end{exdispExample}
\end{posterlocTcbKey}


%
\begin{posterlocTcbKey}[][doc new=2017-07-03]{at}{=\meta{name}}{no default, initially unset}
The box is placed at the position with the given \meta{name}. This is
quite likely a predefined node, see \Fullref{subsec:poster_environment}.

该框被放置在给定的 \meta{name} 位置。这很可能是一个预定义的节点,参见\Fullref{subsec:poster_environment}。
\begin{exdispExample}{at}
\begin{tcbposter}[
  poster = {showframe,height=3cm,spacing=2mm,rows=2},
  boxes  = {beamer,colframe=blue!50!black,colback=blue!50,colupper=yellow!50},
]
\posterbox{name=A,column=1,at=middle}{First box}
\posterbox{name=B,column=2,at=row1}{Second box}
\end{tcbposter}
\end{exdispExample}
\end{posterlocTcbKey}


\begin{posterlocTcbKey}[][doc new=2017-07-03]{between}{=\meta{name1} and \meta{name2}}{no default, initially unset}
The box is placed below a box \meta{name1} and above another box \meta{name2}. Also,
\meta{name1} and \meta{name2} can be predefined nodes, see \Fullref{subsec:poster_environment}.

此选项将盒子放置在给定的 \meta{name1} 盒子下面且给定的 \meta{name2} 盒子上面。此外,\meta{name1} 和 \meta{name2} 可以是预定义的节点,详见\Fullref{subsec:poster_environment}。
\begin{exdispExample}{between}
\begin{tcbposter}[
  poster = {showframe,height=3cm,spacing=2mm,rows=2},
  boxes  = {beamer,colframe=blue!50!black,colback=blue!50,colupper=yellow!50},
]
\posterbox{name=A,column=1,below=top}{First box}
\posterbox{name=B,column=1,between=A and bottom}{Second box}
\posterbox{name=C,column=2,above=bottom}{Third box}
\posterbox{name=D,column=2,between=top and C,span=2}{Fourth box}
\posterbox{name=E,column=3,between=D and bottom}{Fifth box}
\end{tcbposter}
\end{exdispExample}
\end{posterlocTcbKey}


%
\begin{posterlocTcbKey}[][doc new=2017-07-03]{sequence}{=\meta{sequence}}{no default, initially unset}
The box is broken into partial boxes. These partial boxes are placed
following the given \meta{sequence} of placements.
The feasible syntax for the \meta{sequence} is:\par\medskip
\meta{column a} |between| \meta{name a1} |and| \meta{name a2} |then|\\
\meta{column b} |between| \meta{name b1} |and| \meta{name b2} |then|\\
\meta{column c} |between| \meta{name c1} |and| \meta{name c2} |then|\ldots\par\medskip
Obviously, this places the first part box at \meta{column a} between
\meta{name a2} and \meta{name a2}. The second box part is placed
at \meta{column b} between
\meta{name b2} and \meta{name b2}, and so on.

该选项将框架分成多个部分,并按照给定的\meta{sequence}序列放置。
\meta{sequence}的语法格式如下:\par\medskip
\meta{column a} |between| \meta{name a1} |and| \meta{name a2} |then|\
\meta{column b} |between| \meta{name b1} |and| \meta{name b2} |then|\
\meta{column c} |between| \meta{name c1} |and| \meta{name c2} |then|\ldots\par\medskip
显然,这将第一个部分框放置在\meta{column a}中,位于\meta{name a2}和\meta{name a2}之间。第二个部分框被放置在\meta{column b}中,位于\meta{name b2}和\meta{name b2}之间,以此类推。
\begin{exdispExample}{sequence}
\begin{tcbposter}[
  poster = {showframe,height=6cm,spacing=2mm,rows=2},
  boxes  = {beamer,colframe=blue!50!black,colback=blue!50,colupper=yellow!50},
]
\posterbox[adjusted title=A]{name=A,column=1,below=top,span=2}{First box}
\posterbox{name=B,column=2,above=bottom,span=2}{Second box}
\posterbox[adjusted title=C,colframe=red!50!black,colback=red!50]{
  name=C, sequence=1 between A and bottom then
                   2 between A and B then
                   3 between top and B
  }{\lipsum[2]}
\end{tcbposter}
\end{exdispExample}
\end{posterlocTcbKey}

%

\begin{docTcbKey}[][doc new=2017-07-03]{placeholder}{}{style, no value}
If the box content of a \refKeyLe{/tcb/posterloc/sequence} is too short
to fill all reserved box parts, the empty boxes are drawn with
the \refKeyLe{/tcb/placeholder} style. This style can be redefined, e.g.
to \refKeyLe{/tcb/blankest}, if nothing should be drawn for empty boxes.

如果一个 \refKeyLe{/tcb/posterloc/sequence} 的盒子内容太短而不能填满所有保留的盒子部分,那么空的盒子部分将用 \refKeyLe{/tcb/placeholder} 样式绘制。如果不希望绘制空的盒子,则可以重新定义此样式,例如使用 \refKeyLe{/tcb/blankest}。
\begin{exdispExample}{placeholder}
\begin{tcbposter}[
  poster = {showframe,height=2.5cm,spacing=2mm,rows=2},
  boxes  = {beamer,colframe=blue!50!black,colback=blue!50,colupper=yellow!50},
]
\posterbox{name=A,column=1,below=top,span=2}{First box}
\posterbox[colframe=red!50!black,colback=red!50]{
  name=B, sequence=1 between A and bottom then
                   2 between A and bottom then
                   3 between top and bottom
  }{Second box followed by placeholder boxes}
\end{tcbposter}
\end{exdispExample}
\end{docTcbKey}



\begin{posterlocTcbKey}[][doc new=2017-07-03]{xshift}{=\meta{length}}{no default, initially |0pt|}
Horizontal shift of a box by \meta{length}.

将一个框向水平方向移动\meta{length}长度。
\begin{exdispExample}{xshift}
\begin{tcbposter}[
  poster = {showframe,height=3cm,spacing=2mm,rows=2},
  boxes  = {beamer,colframe=blue!50!black,colback=blue!50,colupper=yellow!50},
]
\posterbox{name=A,column=1,row=1,xshift=6mm}{First box}
\posterbox{name=B,column=2,row=2,xshift=-6mm}{Second box}
\end{tcbposter}
\end{exdispExample}
\end{posterlocTcbKey}

%

\begin{posterlocTcbKey}[][doc new=2017-07-03]{yshift}{=\meta{length}}{no default, initially |0pt|}
Vertical shift of a box by \meta{length}.

将一个盒子垂直移动 \meta{length} 的距离。
\begin{exdispExample}{yshift}
\begin{tcbposter}[
  poster = {showframe,height=3cm,spacing=2mm,rows=2},
  boxes  = {beamer,colframe=blue!50!black,colback=blue!50,colupper=yellow!50},
]
\posterbox{name=A,column=1,row=1,yshift=-4mm}{First box}
\posterbox{name=B,column=2,row=2,yshift=4mm}{Second box}
\end{tcbposter}
\end{exdispExample}
\end{posterlocTcbKey}

%todo 译但没看内容
% \setcounter{section}{21}
% \include*{tcolorbox.doc.fitting}% v1 2023 04 07
% \setcounter{section}{22}
% \include*{tcolorbox.doc.hooks}%v1 2023 04 08
% \setcounter{section}{23}
% % !TeX root = tcolorbox.tex
% include file of tcolorbox.tex (manual of the LaTeX package tcolorbox)
% \clearpage
\section{Library \mylib{xparse}}\label{sec:xparse}%
\tcbset{external/prefix=external/xparse_}%
\begin{stripedbox}
The library is loaded by a package option or inside the preamble by:
\tcblower
库通过包选项或在序言中加载:
\end{stripedbox}

\begin{dispListing}
\tcbuselibrary{xparse}
\end{dispListing}

\begin{stripedbox}
This also loads the package |xparse| %\cite{latexproject:xparse}
.
\tcblower
这还会加载包|xparse|。
\end{stripedbox}

\begin{stripedbox}
The purpose of this library is to give comfortable access to the powerful document command production with |xparse| for |tcolorbox|.
See the |xparse| package documentation %\cite{latexproject:xparse}
for details about the argument \meta{specification} used in this section.
\tcblower
这个库的目的是通过 |xparse| 给 |tcolorbox| 提供方便、功能强大的文档命令.
有关本节中使用的参数 \meta{specification} 的详细信息。 参见|xparse|包文档。
\end{stripedbox}

%\subsection{Producing Document Commands With \texttt{xparse}}
% Option Keys
\subsection{可选择的选项}\label{subsec:xparse_options}

\begin{docTcbKey}{verbatim}{}{style, no value}
\begin{stripedbox}
Sets options for a \textit{verbatim} style \refComLe{tcbox}.
Since the indented boxes may contain only very few words, 
the dimensions are made smaller and \refKeyLe{/tcb/nobeforeafter} and \refKeyLe{/tcb/tcbox raise base} are set.
\tcblower
设置\refComLe{tcbox}拥有\textit{verbatim}样式的选项。%
由于缩进的盒子可能只包含很少的单词,%
尺寸被设置变小,还指定\refKeyLe{/tcb/nobeforeafter}和\refKeyLe{/tcb/tcbox raise base}选项。
\end{stripedbox}

\begin{dispExample*}{sbs,lefthand ratio=0.6}
\DeclareTotalTCBox{\myverb}{ v }{verbatim,
  colframe=red!75!black,colupper=blue}{#1}

\myverb{\textbf} is a \myverb{\LaTeX} command.
\end{dispExample*}
\end{docTcbKey}

\begin{docTcbKeys}[doc description = {no default}]
  {
    {
      doc name        = IfNoValueTF,
      doc parameter   = {=\marg{argument}\marg{true options}\marg{false options}},
    },
    {
      doc name        = IfNoValueT,
      doc parameter   = {=\marg{argument}\marg{true options}},
      doc new         = 2020-09-16,
    },
    {
      doc name        = IfNoValueF,
      doc parameter   = {=\marg{argument}\marg{false options}},
      doc new         = 2020-09-16,
    }
  }
\begin{stripedbox}
Wraps the |\IfNoValue(TF)| command(s) of |xparse| for option setting.
If the \meta{argument} has no value, the \meta{true options} are set.
Otherwise, the \meta{false options} are set.
\tcblower
包装 |xparse| 的 |\IfNoValue(TF)| 命令用于选项设置。%
如果\meta{argument}没有值,则设置\meta{true options}。%
否则,将设置\meta{false options}。
\end{stripedbox}
\begin{dispExample}
\DeclareTColorBox{mybox}{ o }{colframe=red!75!black,
  IfNoValueTF={#1}{colback=red!5!white}{enhanced,interior style image=#1}}

\begin{mybox}
This is a tcolorbox.
\end{mybox}

\begin{mybox}[goldshade.png]
This is a tcolorbox.
\end{mybox}
\end{dispExample}
\end{docTcbKeys}

\clearpage
\begin{docTcbKeys}[doc description = {no default}]
  {
    {
      doc name        = IfValueTF,
      doc parameter   = {=\marg{argument}\marg{true options}\marg{false options}},
    },
    {
      doc name        = IfValueT,
      doc parameter   = {=\marg{argument}\marg{true options}},
      doc new         = 2020-09-16,
    },
    {
      doc name        = IfValueF,
      doc parameter   = {=\marg{argument}\marg{false options}},
      doc new         = 2020-09-16,
    }
  }
\begin{stripedbox}
Wraps the |\IfValue(TF)| command(s) of |xparse| for option setting.
If the \meta{argument} has a value, the \meta{true options} are set.
Otherwise, the \meta{false options} are set.
\tcblower
包装 |xparse| 的|\IfValue(TF)|命令,用于选项设置。%
如果\meta{argument}有值,则设置\meta{true options}。%
否则,将设置\meta{false options}。
\end{stripedbox}

\begin{dispExample}
\DeclareTColorBox{mybox}{ o }{colframe=red!75!black,colback=red!5!white,
  IfValueT={#1}{title={\flqq #1\frqq},fonttitle=\bfseries}}

\begin{mybox}
This is a tcolorbox.
\end{mybox}

\begin{mybox}[My title]
This is a tcolorbox.
\end{mybox}
\end{dispExample}
\end{docTcbKeys}

\medskip

\begin{docTcbKeys}[doc description = {no default}]
  {
    {
      doc name        = IfBooleanTF,
      doc parameter   = {=\marg{argument}\marg{true options}\marg{false options}},
    },
    {
      doc name        = IfBooleanT,
      doc parameter   = {=\marg{argument}\marg{true options}},
      doc new         = 2020-09-16,
    },
    {
      doc name        = IfBooleanF,
      doc parameter   = {=\marg{argument}\marg{false options}},
      doc new         = 2020-09-16,
    }
  }
\begin{stripedbox}
Wraps the |\IfBoolean(TF)| command(s) of |xparse| for option setting.
If the \meta{argument} is |\BooleanTrue|, the \meta{true options} are set.
If the \meta{argument} is |\BooleanFalse|, the \meta{false options} are set.
\tcblower
包装|xparse|的|\IfBoolean(TF)|命令用于选项设置。
如果\meta{argument}是|\BooleanTrue|,则设置\meta{true options}。
如果\meta{argument}是|\BooleanFalse|,则设置\meta{false options}。
\end{stripedbox}
  

\begin{dispExample}
\DeclareTColorBox{mybox}{ s }{colframe=red!75!black,
  IfBooleanTF={#1}{colback=yellow!50!red}{colback=red!5!white}}

\begin{mybox}
This is a tcolorbox.
\end{mybox}

\begin{mybox}*
This is a tcolorbox.
\end{mybox}
\end{dispExample}
\end{docTcbKeys}

% \clearpage
% Producing \texttt{tcolorbox} Environments and Commands
\subsection{生成\texttt{tcolorbox}环境和命令}\label{subsec:xparse_tcolorbox}

\begin{docCommand}{DeclareTColorBox}{\oarg{init options}\marg{name}\marg{specification}\marg{options}}
\begin{stripedbox}
Creates a new environment \meta{name} based on \refEnvLe{tcolorbox}.
\tcblower
基于\refEnvLe{tcolorbox}创建一个新的环境\meta{name}。
\end{stripedbox}

\begin{stripedbox}
Basically, |\DeclareTColorBox| operates like |\DeclareDocumentEnvironment|. This means,
the new environment \meta{name} is constructed with the given argument \meta{specification}.
The \meta{options} are given to the underlying \refEnvLe{tcolorbox}.
\tcblower
基本上,|\DeclareTColorBox|操作类似|\DeclareDocumentEnvironment|。%
这意味着,%
新的环境\meta{name}是用给定的参数\meta{specification}构造的。%
\meta{options}被赋予底层的\refEnvLe{tcolorbox}。
\end{stripedbox}

\begin{stripedbox}
Note that \refKeyLe{/tcb/savedelimiter} is set to the given \meta{name}
automatically.
\tcblower
注意,\refKeyLe{/tcb/savedelimiter}自动被设置为给定的\meta{name}。
\end{stripedbox}

\begin{stripedbox}
The \meta{init options} allow setting up automatic numbering,
see Section \ref{sec:initkeys} from page \pageref{sec:initkeys}.
\tcblower
\meta{init options}允许设置自动编号,%
请参见 \pageref{sec:initkeys}页的章节\ref{sec:initkeys}。
\end{stripedbox}

\begin{stripedbox}
The new environment is always created, irrespective of an already existing
environment with the same name.
\tcblower
新环境总是被创建,而不考虑已经存在的具有相同名称的环境。
\end{stripedbox}

\begin{dispExample}
% counter from previous example
\DeclareTColorBox[use counter from=pabox]{mybox}{ O{red} m d"" !O{} }
  {enhanced,colframe=#1!75!black,colback=#1!5!white,
   fonttitle=\bfseries,title={\thetcbcounter~#2},
   IfValueT={#3}{watermark text={#3}},#4}

\begin{mybox}{My title}
This is a tcolorbox.
\end{mybox}

\begin{mybox}[blue]{My title}
This is a tcolorbox.
\end{mybox}

\begin{mybox}[green]{My title}"My Watermark"
This is a tcolorbox.
\end{mybox}

\begin{mybox}[yellow]{My title}[colbacktitle=yellow!50!white,coltitle=black]
This is a tcolorbox.
\end{mybox}

\begin{mybox}[purple]{My title}"All together"[coltitle=yellow]
This is a tcolorbox.
\end{mybox}
\end{dispExample}
\end{docCommand}

% \clearpage
\begin{docCommand}{NewTColorBox}{\oarg{init options}\marg{name}\marg{specification}\marg{options}}
\begin{stripedbox}
Operates like \refComLe{DeclareTColorBox}, 
but based on |\NewDocumentEnvironment| instead of |\DeclareDocumentEnvironment|.
An error is issued if \meta{name} has already been defined.
\tcblower
操作方式类似\refComLe{DeclareTColorBox},%
但是基于|\NewDocumentEnvironment|而不是|\DeclareDocumentEnvironment|。%
如果已经定义了\meta{name},则会发出错误。
\end{stripedbox}
\end{docCommand}

\begin{docCommand}{RenewTColorBox}{\oarg{init options}\marg{name}\marg{specification}\marg{options}}
\begin{stripedbox}
Operates like \refComLe{DeclareTColorBox}, but based on |\RenewDocumentEnvironment| instead of |\DeclareDocumentEnvironment|.
An existing environment is redefined.
\tcblower
操作类似\refComLe{DeclareTColorBox},%
但是基于|\RenewDocumentEnvironment|而不是|\DeclareDocumentEnvironment|。%
重新定义现有环境。
\end{stripedbox}
\end{docCommand}

\begin{docCommand}{ProvideTColorBox}{\oarg{init options}\marg{name}\marg{specification}\marg{options}}
\begin{stripedbox}
Operates like \refComLe{DeclareTColorBox}, but based on |\ProvideDocumentEnvironment| instead of |\DeclareDocumentEnvironment|.
The environment \meta{name} is only created if it is not already defined.
\tcblower
操作类似\refComLe{DeclareTColorBox},但基于|\ProvideDocumentEnvironment|而不是|\DeclareDocumentEnvironment|。
环境\meta{name}只有在尚未定义时才会被创建。
\end{stripedbox}
\end{docCommand}

% \clearpage
\begin{docCommand}{DeclareTotalTColorBox}{\oarg{init options}\brackets{\texttt{\textbackslash}\meta{name}}\marg{specification}\marg{options}\marg{content}}
\begin{stripedbox}
Creates a new command \texttt{\textbackslash}\meta{name} based on \refEnvLe{tcolorbox}.
In contrast to \refComLe{DeclareTColorBox}, also the \meta{content} of the |tcolorbox| is specified.
\tcblower
基于\refEnvLe{tcolorbox}创建一个新命令\texttt{\textbackslash}\meta{name}。%
与\refComLe{DeclareTColorBox}相反,|tcolorbox|的\meta{content}也被指定。
\end{stripedbox}
  
  
\begin{stripedbox}
Basically, |\DeclareTotalTColorBox| operates like |\DeclareDocumentCommand|. This means,
the new command \texttt{\textbackslash}\meta{name} is constructed with the given argument \meta{specification}.
The \meta{options} are given to the underlying \refEnvLe{tcolorbox} which is filled with
the specified \meta{content}.
\tcblower
基本上,|\DeclareTotalTColorBox|操作类似|\DeclareDocumentCommand|。%
这意味着,新命令\texttt{\textbackslash}\meta{name}是用给定的参数\meta{specification}构造的。%
\meta{options}被设置到底层的\refEnvLe{tcolorbox},该\refEnvLe{tcolorbox}被指定的\meta{content}填充。
\end{stripedbox}
  
\begin{stripedbox}
Note that \refKeyLe{/tcb/savedelimiter} is set to the given \meta{name}
automatically.
\tcblower
注意\refKeyLe{/tcb/savedelimiter}自动被设置为给定的\meta{name}
\end{stripedbox}

\begin{stripedbox}
The \meta{init options} allow setting up automatic numbering,
see Section \ref{sec:initkeys} from page \pageref{sec:initkeys}.
\tcblower
\meta{init选项}允许设置自动编号,
请参见 \pageref{sec:initkeys}页的\ref{sec:initkeys}章节。
\end{stripedbox}

\begin{stripedbox}
The new command is always created, irrespective of an already existing
command with the same name.
\tcblower
新命令总是被创建,而不考虑已经存在的同名命令。
\end{stripedbox}

\begin{dispExample}
\DeclareTotalTColorBox{\diabox}{ O{} v  m }
  { bicolor,nobeforeafter,equal height group=diabox,width=5.7cm,
    fonttitle=\bfseries\ttfamily,adjusted title={#2},center title,
    colframe=blue!20!black,leftupper=0mm,rightupper=0mm,colback=black!75!white,#1}
  { \tikz\path[fill zoom image={#2}] (0,0) rectangle (\linewidth,4cm);%
    \tcblower#3}

\diabox{blueshade.png}{Created with |GIMP|.\\\url{http://www.gimp.org}}
\diabox{goldshade.png}{Created with |GIMP|.\\\url{http://www.gimp.org}}

\end{dispExample}
\end{docCommand}

\begin{docCommand}{NewTotalTColorBox}{\oarg{init options}\brackets{\texttt{\textbackslash}\meta{name}}\marg{specification}\marg{options}\marg{content}}
\begin{stripedbox}
Operates like \refComLe{DeclareTotalTColorBox}, but based on |\NewDocumentCommand| instead of |\DeclareDocumentCommand|.
An error is issued if \texttt{\textbackslash}\meta{name} has already been defined.
\tcblower
操作类似\refComLe{DeclareTotalTColorBox},但基于|\NewDocumentCommand|而不是|\DeclareDocumentCommand|
如果已经定义了\texttt{\textbackslash}\meta{name},则会发出错误。
\end{stripedbox}
\end{docCommand}

\begin{docCommand}{RenewTotalTColorBox}{\oarg{init options}\brackets{\texttt{\textbackslash}\meta{name}}\marg{specification}\marg{options}\marg{content}}
\begin{stripedbox}
Operates like \refComLe{DeclareTotalTColorBox}, but based on |\RenewDocumentCommand| instead of |\DeclareDocumentCommand|.
An existing command is redefined.
\tcblower
操作类似\refComLe{DeclareTotalTColorBox},但基于|\RenewDocumentCommand|而不是|\DeclareDocumentCommand|。
重新定义现有命令。
\end{stripedbox}
\end{docCommand}

\begin{docCommand}{ProvideTotalTColorBox}{\oarg{init options}\brackets{\texttt{\textbackslash}\meta{name}}\marg{specification}\marg{options}\marg{content}}
\begin{stripedbox}
Operates like \refComLe{DeclareTotalTColorBox}, but based on |\ProvideDocumentCommand| instead of |\DeclareDocumentCommand|.
The command \texttt{\textbackslash}\meta{name} is only created if it is not already defined.
\tcblower
操作类似\refComLe{DeclareTotalTColorBox},但基于|\ProvideDocumentCommand|而不是|\DeclareDocumentCommand|。命令\texttt{\textbackslash}\meta{name}只有在尚未定义时才会被创建。
\end{stripedbox}  
\end{docCommand}

% \clearpage
% Producing \texttt{tcbox} Commands
\subsection{生成\texttt{tcbox}命令}\label{subsec:xparse_tcbox}

\begin{docCommand}{DeclareTCBox}{\oarg{init options}\brackets{\texttt{\textbackslash}\meta{name}}\marg{specification}\marg{options}}
\begin{stripedbox}
Creates a new command \texttt{\textbackslash}\meta{name} based on \refComLe{tcbox}.
Basically, |\DeclareTCBox| operates like |\DeclareDocumentCommand|. This means,
the new command \texttt{\textbackslash}\meta{name} is constructed with the given argument \meta{specification}.
The \meta{options} are given to the underlying \refComLe{tcbox}.
\tcblower
基于\refComLe{tcbox}创建一个新命令\texttt{\textbackslash}\meta{name}。%
基本上,|\DeclareTCBox|操作类似|\DeclareDocumentCommand|。这意味着,%
新命令\texttt{\textbackslash}\meta{name}是用给定的参数\meta{specification}构造的。\meta{options}被赋给底层的\refComLe{tcbox}。
\end{stripedbox}


\begin{stripedbox}
Note that \refKeyLe{/tcb/savedelimiter} is set to the given \meta{name}
automatically.
\tcblower
注意\refKeyLe{/tcb/savedelimiter}自动被设置为给定的\meta{name}。
\end{stripedbox}

\begin{stripedbox}
The \meta{init options} allow setting up automatic numbering,
see Section \ref{sec:initkeys} from page \pageref{sec:initkeys}.
\tcblower
\meta{init选项}允许设置自动编号,
请参见 \pageref{sec:initkeys} 页中的 \ref{sec:initkeys} 章节。
\end{stripedbox}

\begin{stripedbox}
The new command is always created, irrespective of an already existing
command with the same name.
\tcblower
新命令总是被创建,而不考虑已经存在的同名命令。
\end{stripedbox}
  

\begin{dispExample}
% counter from previous example
\DeclareTCBox[use counter from=pabox]{\mybox}{ s m s }
{ nobeforeafter,colback=red!5!white,colframe=red!75!black,
  title={#2 (Box \thetcbcounter)},fonttitle=\bfseries,
  IfBooleanT={#1}{enhanced,drop shadow},
  IfBooleanT={#3}{colbacktitle=red!50!white} }

\mybox{Bird}{This is my first box.}
  \hfill
\mybox*{Tree}{This is my second box.}
  \par\bigskip
\mybox{Bike}*{This is my third box.}
  \hfill
\mybox*{City}*{This is my fourth box.}
\end{dispExample}
\end{docCommand}

\begin{docCommand}{NewTCBox}{\oarg{init options}\brackets{\texttt{\textbackslash}\meta{name}}\marg{specification}\marg{options}}
\begin{stripedbox}
Operates like \refComLe{DeclareTCBox}, but based on |\NewDocumentCommand| instead of |\DeclareDocumentCommand|.
An error is issued if \texttt{\textbackslash}\meta{name} has already been defined.
\tcblower
操作类似\refComLe{DeclareTCBox},但基于|\NewDocumentCommand|而不是|\DeclareDocumentCommand|。
如果已经定义了\texttt{\textbackslash}\meta{name},则会发出错误。
\end{stripedbox}
\end{docCommand}

\begin{docCommand}{RenewTCBox}{\oarg{init options}\brackets{\texttt{\textbackslash}\meta{name}}\marg{specification}\marg{options}}
\begin{stripedbox}
Operates like \refComLe{DeclareTCBox}, but based on |\RenewDocumentCommand| instead of |\DeclareDocumentCommand|.
An existing command is redefined.
\tcblower
操作类似\refComLe{DeclareTCBox},但基于|\RenewDocumentCommand|而不是|\DeclareDocumentCommand|。%
重新定义现有命令。
\end{stripedbox}
\end{docCommand}

\begin{docCommand}{ProvideTCBox}{\oarg{init options}\brackets{\texttt{\textbackslash}\meta{name}}\marg{specification}\marg{options}}
\begin{stripedbox}
Operates like \refComLe{DeclareTCBox}, but based on |\ProvideDocumentCommand| instead of |\DeclareDocumentCommand|.
The command \texttt{\textbackslash}\meta{name} is only created if it is not already defined.
\tcblower
操作类似\refComLe{DeclareTCBox},但基于|\ProvideDocumentCommand|而不是|\DeclareDocumentCommand|。
命令 \texttt{\textbackslash}\meta{name} 只有在尚未定义时才会被创建。
\end{stripedbox}
\end{docCommand}

% \clearpage

\begin{docCommand}{DeclareTotalTCBox}{\oarg{init options}\brackets{\texttt{\textbackslash}\meta{name}}\marg{specification}\marg{options}\marg{content}}

\begin{stripedbox}
Creates a new command \texttt{\textbackslash}\meta{name} based on \refComLe{tcbox}.
In contrast to \refComLe{DeclareTCBox}, also the \meta{content} of the |tcbox| is specified.
\tcblower
基于\refComLe{tcbox}创建一个新命令\texttt{\textbackslash}\meta{name}。%
与\refComLe{DeclareTCBox}相反,|tcbox|的\meta{content}也被指定。
\end{stripedbox}

\begin{stripedbox}
Basically, |\DeclareTotalTCBox| operates like |\DeclareDocumentCommand|. This means,
the new command \texttt{\textbackslash}\meta{name} is constructed with the given argument \meta{specification}.
The \meta{options} are given to the underlying \refComLe{tcbox} which is filled with
the specified \meta{content}.
\tcblower
基本上,|\DeclareTotalTCBox|操作类似|\DeclareDocumentCommand|。这意味着,
新命令\texttt{\textbackslash}\meta{name}是用给定的参数\meta{specification}构造的。
\meta{options}被赋予底层的\refComLe{tcbox},该\refComLe{tcbox}被填充
指定的\meta{content}。
\end{stripedbox}
  

\begin{stripedbox}
Note that \refKeyLe{/tcb/savedelimiter} is set to the given \meta{name}
automatically.
\tcblower
注意\refKeyLe{/tcb/savedelimiter}被自动设置为给定的\meta{name}。
\end{stripedbox}


\begin{stripedbox}
The \meta{init options} allow setting up automatic numbering,
see Section \ref{sec:initkeys} from page \pageref{sec:initkeys}.
\tcblower
\meta{init选项}允许设置自动编号,
请参见 \pageref{sec:initkeys} 页中的 \ref{sec:initkeys} 章节。
\end{stripedbox}


\begin{stripedbox}
The new command is always created, irrespective of an already existing
command with the same name.
\tcblower
新命令总是被创建,而不考虑已经存在的同名命令。
\end{stripedbox}

\begin{dispExample}
\DeclareTotalTCBox{\myverb}{ O{red} v !O{} }
{ fontupper=\ttfamily,nobeforeafter,tcbox raise base,arc=0pt,outer arc=0pt,
  top=0pt,bottom=0pt,left=0mm,right=0mm,
  leftrule=0pt,rightrule=0pt,toprule=0.3mm,bottomrule=0.3mm,boxsep=0.5mm,
  colback=#1!10!white,colframe=#1!50!black,#3}{#2}

To set a word \textbf{bold} in \myverb{\LaTeX}, use
\myverb[green]{\textbf{bold}}. Alternatively, write
\myverb[yellow]{{\bfseries bold}}.
In \myverb[blue]{\LaTeX}[enhanced,fuzzy halo], other font settings are
done in the same way, e.\,g. \myverb{\textit}, \myverb{\itshape}\\
or \myverb[brown]{\texttt}, \myverb[brown]{\ttfamily}.
\end{dispExample}

\begin{stripedbox}
The next example uses |\lstinline| from the |listings| package to
typeset the verbatim content.
\tcblower
下面的例子使用 |listings| 包的 |\lstinline| 逐字排版内容。
\end{stripedbox}


\begin{dispExample}
% \usepackage{listings} or \tcbuselibrary{listings}
\DeclareTotalTCBox{\commandbox}{ s v }
{verbatim,colupper=white,colback=black!75!white,colframe=black}
{\IfBooleanT{#1}{\textcolor{red}{\ttfamily\bfseries > }}%
  \lstinline[language=command.com,keywordstyle=\color{blue!35!white}\bfseries]^#2^}

\commandbox*{cd "My Documents"} changes to directory \commandbox{My Documents}.

\commandbox*{dir /A} lists the directory content.

\commandbox*{copy example.txt d:\target} copies \commandbox{example.txt} to
  \commandbox{d:\target}.
\end{dispExample}
\end{docCommand}

% \clearpage
\begin{docCommand}{NewTotalTCBox}{\oarg{init options}\brackets{\texttt{\textbackslash}\meta{name}}\marg{specification}\marg{options}\marg{content}}
\begin{stripedbox}
Operates like \refComLe{DeclareTotalTCBox}, but based on |\NewDocumentCommand| instead of |\DeclareDocumentCommand|.
An error is issued if \texttt{\textbackslash}\meta{name} has already been defined.
\tcblower
操作类似\refComLe{DeclareTotalTCBox},但基于|\NewDocumentCommand|而不是|\DeclareDocumentCommand|。%
如果已经定义了\texttt{\textbackslash}\meta{name},则会发出错误。
\end{stripedbox}
\end{docCommand}

\begin{docCommand}{RenewTotalTCBox}{\oarg{init options}\brackets{\texttt{\textbackslash}\meta{name}}\marg{specification}\marg{options}\marg{content}}
\begin{stripedbox}
Operates like \refComLe{DeclareTotalTCBox}, but based on |\RenewDocumentCommand| instead of |\DeclareDocumentCommand|.
An existing command is redefined.
\tcblower
操作类似\refComLe{DeclareTotalTCBox},但基于|\RenewDocumentCommand|而不是|\DeclareDocumentCommand|。重新定义现有命令。
\end{stripedbox}
\end{docCommand}

\begin{docCommand}{ProvideTotalTCBox}{\oarg{init options}\brackets{\texttt{\textbackslash}\meta{name}}\marg{specification}\marg{options}\marg{content}}
\begin{stripedbox}
Operates like \refComLe{DeclareTotalTCBox}, but based on |\ProvideDocumentCommand| instead of |\DeclareDocumentCommand|.
The command \texttt{\textbackslash}\meta{name} is only created if it is not already defined.
\tcblower
操作类似\refComLe{DeclareTotalTCBox},但基于|\ProvideDocumentCommand|而不是|\DeclareDocumentCommand|。命令\texttt{\textbackslash}\meta{name}只有在尚未定义时才会被创建。
\end{stripedbox}
\end{docCommand}


\begin{docCommand}{tcboxverb}{\oarg{options}\marg{verbatim box content}}
\begin{stripedbox}
Creates a colored box based on \refComLe{tcbox} which is fitted to the width of the given
\meta{verbatim box content}.
The underlying \refComLe{tcbox} is styled with
\refKeyLe{/tcb/verbatim} plus the given \meta{options}.
The difference to \refComLe{tcbox} is that the \meta{verbatim box content} is
interpreted \textit{verbatim}. Therefore, |\tcboxverb| acts similar to |\verb|.
\tcblower
基于\refComLe{tcbox}创建一个与给定\meta{verbatim box content}宽度相匹配的彩色盒子。
底层的\refComLe{tcbox}的样式使用\refKeyLe{/tcb/verbatim}加上给定的\meta{options}进行设置。
\refComLe{tcbox}的不同之处在于\meta{verbatim box content}被\textit{逐个输出}。因此,|\tcboxverb|的行为类似于|\verb|。
\end{stripedbox}


\begin{dispExample}
\tcboxverb{\LaTeX}, \tcboxverb[colback=blue!10!white,colupper=blue]{\LaTeX},
\tcboxverb[blank,fuzzy halo]{\LaTeX}, \tcboxverb[beamer]{\LaTeX},
\tcboxverb[enhanced,skin=enhancedmiddle jigsaw,colframe=red]{\LaTeX}.
\end{dispExample}
\end{docCommand}

% \clearpage
% Producing \texttt{tcblisting} Environments
\subsection{生成\texttt{tcblisting}环境}\label{subsec:xparse_listing}
\begin{marker}
\begin{stripedbox}
Besides \mylib{xparse}, the following commands also need the \mylib{listings} library to be included.
\tcblower
除了\mylib{xparse},下面的命令还需要包含\mylib{listings}库。
\end{stripedbox}
\end{marker}

\begin{docCommand}{DeclareTCBListing}{\oarg{init options}\marg{name}\marg{specification}\marg{options}}
  
\begin{stripedbox}
Creates a new environment \meta{name} based on \refEnvLe{tcblisting}.
\tcblower
基于\refEnvLe{tcblisting}创建一个新环境\meta{name}。
\end{stripedbox}


\begin{stripedbox}
Basically, |\DeclareTCBListing| operates like |\DeclareDocumentEnvironment|. This means,
the new environment \meta{name} is constructed with the given argument \meta{specification}.
The \meta{options} are given to the underlying \refEnvLe{tcblisting}.
\tcblower
基本上,|\DeclareTCBListing|操作类似|\DeclareDocumentEnvironment|。这意味着,
新的环境\meta{name}是用给定的参数\meta{specification}构造的。
\meta{options}被设置到底层的\refEnvLe{tcblisting}。
\end{stripedbox}


\begin{stripedbox}
Note that \refKeyLe{/tcb/savedelimiter} is set to the given \meta{name}
automatically.
\tcblower
注意\refKeyLe{/tcb/savedelimiter}被自动设置为给定的\meta{name}。
\end{stripedbox}

\begin{stripedbox}
The \meta{init options} allow setting up automatic numbering,
see Section \ref{sec:initkeys} from page \pageref{sec:initkeys}.
\tcblower
\meta{init options}允许设置自动编号,%
请参见 \pageref{sec:initkeys} 页中的 \ref{sec:initkeys}章节。
\end{stripedbox}

\begin{stripedbox}
The new environment is always created, irrespective of an already existing
environment with the same name.
\tcblower
新环境总是被创建,而不考虑已经存在的具有相同名称的环境。
\end{stripedbox}

\begin{dispExample*}{sbs,lefthand ratio=0.5}
\DeclareTCBListing{mybox}{ s O{} m }{%
  colback=red!5!white,
  colframe=red!75!black,
  fonttitle=\bfseries,
  IfBooleanTF={#1}
    {listing side text}
    {text side listing},
  title={#3},#2}

\begin{mybox}{Listing Box}
This is my
\LaTeX\ box.
\end{mybox}
\bigskip

\begin{mybox}*{Listing Box}
This is my
\LaTeX\ box.
\end{mybox}
\bigskip

\begin{mybox}[colback=yellow]
  {Listing Box}
This is my
\LaTeX\ box.
\end{mybox}
\end{dispExample*}
\end{docCommand}

\begin{docCommand}{NewTCBListing}{\oarg{init options}\marg{name}\marg{specification}\marg{options}}
\begin{stripedbox}
Operates like \refComLe{DeclareTCBListing}, but based on |\NewDocumentEnvironment| instead of |\DeclareDocumentEnvironment|.
An error is issued if \meta{name} has already been defined.
\tcblower
操作类似\refComLe{DeclareTCBListing},但基于|\NewDocumentEnvironment|而不是|\DeclareDocumentEnvironment|。
如果已经定义了\meta{name},则会发出错误。
\end{stripedbox}
\end{docCommand}

\begin{docCommand}{RenewTCBListing}{\oarg{init options}\marg{name}\marg{specification}\marg{options}}
\begin{stripedbox}
Operates like \refComLe{DeclareTCBListing}, but based on |\RenewDocumentEnvironment| instead of |\DeclareDocumentEnvironment|.
An existing environment is redefined.
\tcblower
操作类似\refComLe{DeclareTCBListing},但基于|\RenewDocumentEnvironment|而不是|\DeclareDocumentEnvironment|。
重新定义现有环境。
\end{stripedbox}
\end{docCommand}

\begin{docCommand}{ProvideTCBListing}{\oarg{init options}\marg{name}\marg{specification}\marg{options}}
\begin{stripedbox}
Operates like \refComLe{DeclareTCBListing}, but based on |\ProvideDocumentEnvironment| instead of |\DeclareDocumentEnvironment|.
The environment \meta{name} is only created if it is not already defined.
\tcblower
操作类似\refComLe{DeclareTCBListing},但基于|\ProvideDocumentEnvironment|而不是|\DeclareDocumentEnvironment|。
环境\meta{name}只有在尚未定义时才会被创建。
\end{stripedbox}    
\end{docCommand}

% \clearpage
\begin{marker}
\begin{stripedbox}
With date of 2018-05-12, the |xparse| %\cite{latexproject:xparse} 
package changed the argument collection process.
Now, spaces are ignored which leads to a serious change for listing environments
ending with an optional argument like \verb+O{}+.
The former behavior of respecting spaces can be preserved by adding a \flqq\verb+!+\frqq.
Note that the following code uses \verb+!O{}+ now.
\tcblower
从2018-05-12开始,|xparse|包改变了参数收集过程。%
现在,空格被忽略了,这导致以 \verb+O{}+ 这样的可选参数结尾的环境列表发生了重大变化。%
通过添加 \flqq\verb+!+\frqq ,可以保留以前空格的行为。%
注意,下面的代码使用了 \verb+!O{}+ 了。
\end{stripedbox}
\begin{itemize}
\item For older |xparse| versions, the following code is correct when using \verb+O{}+.
\item For |xparse| of 2018-05-12, only the first two examples of
  the following code using \verb+O{}+ are really \flqq good\frqq\ -- all others do not work.
\item For |xparse| of 2018-05-12 and later, the following code is correct when using \verb+!O{}+.
\end{itemize}
\end{marker}





\begin{dispListing*}{title={Caveats of using an environment ending with an
  optional argument},fonttitle=\bfseries}
\DeclareTCBListing{mybox}{ !O{} }{listing only,#1}

\begin{mybox}[colframe=red]
\good
\end{mybox}

\begin{mybox}[colframe=red]\good\end{mybox}

\begin{mybox}
\good
\end{mybox}

\begin{mybox} \good\end{mybox}

\begin{mybox}\bad!\end{mybox}

\begin{mybox}
[\good]
\end{mybox}

\begin{mybox} [\good]\end{mybox}

\begin{mybox}[\bad!]\end{mybox}
\end{dispListing*}

% \clearpage
% Producing \texttt{tcbinputlisting} Commands
\subsection{生成\texttt{tcbinputlisting}命令}\label{subsec:xparse_inputlisting}
\begin{marker}
\begin{stripedbox}
The following commands need the \mylib{listings} library to be included.
\tcblower
下面的命令需要包含\mylib{listings}库。
\end{stripedbox}
\end{marker}


\begin{docCommand}{DeclareTCBInputListing}{\oarg{init options}\brackets{\texttt{\textbackslash}\meta{name}}\marg{specification}\marg{options}}
\begin{stripedbox}
Creates a new command \texttt{\textbackslash}\meta{name} based on \refComLe{tcbinputlisting}.
Basically, |\DeclareTCBInputListing| operates like |\DeclareDocumentCommand|. This means,
the new command \texttt{\textbackslash}\meta{name} is constructed with the given argument \meta{specification}.
The \meta{options} are given to the underlying \refComLe{tcbinputlisting}.\\
The \meta{init options} allow setting up automatic numbering,
see Section \ref{sec:initkeys} from page \pageref{sec:initkeys}.\\
The new command is always created, irrespective of an already existing
command with the same name.
\tcblower
基于\refComLe{tcbinputlisting}创建一个新命令\texttt{\textbackslash}\meta{name}。%
基本上,|\DeclareTCBInputListing|操作类似|\DeclareDocumentCommand|。这意味着,新命令\texttt{\textbackslash}\meta{name}是用给定的参数\meta{specification}构造的。%
\meta{options}被赋给底层的\refComLe{tcbinputlisting}。\\%
\meta{init选项}允许设置自动编号,
请参见 \pageref{sec:initkeys} 页的 \ref{sec:initkeys} 章节。\\
新命令总是被创建,不管是否已经存在同名命令。
\end{stripedbox}

\begin{dispExample}
% counter from previous example
\DeclareTCBInputListing[use counter from=pabox]{\mylisting}{ O{} O{red} m }{%
  listing file={#3},title=Listing~\thetcbcounter,
  colback=#2!5!white,colframe=#2!50!black,colbacktitle=#2!75!black,
  fonttitle=\bfseries,listing only,#1}

\mylisting[before upper=\textit{This is the included file content:}]
  [blue]{\jobname.tcbtemp}
\end{dispExample}
  \end{docCommand}

\begin{docCommand}{NewTCBInputListing}{\oarg{init options}\brackets{\texttt{\textbackslash}\meta{name}}\marg{specification}\marg{options}}
\begin{stripedbox}
Operates like \refComLe{DeclareTCBInputListing}, but based on |\NewDocumentCommand| instead of |\DeclareDocumentCommand|.
An error is issued if \texttt{\textbackslash}\meta{name} has already been defined.
\tcblower
操作类似\refComLe{DeclareTCBInputListing},但基于|\NewDocumentCommand|而不是|\DeclareDocumentCommand|。
如果已经定义了\texttt{\textbackslash}\meta{name},则会发出错误。
\end{stripedbox}
\end{docCommand}

\begin{docCommand}{RenewTCBInputListing}{\oarg{init options}\brackets{\texttt{\textbackslash}\meta{name}}\marg{specification}\marg{options}}
\begin{stripedbox}
Operates like \refComLe{DeclareTCBInputListing}, but based on |\RenewDocumentCommand| instead of |\DeclareDocumentCommand|.
An existing command is redefined.
\tcblower
操作类似\refComLe{DeclareTCBInputListing},但基于|\RenewDocumentCommand|而不是|\DeclareDocumentCommand|。
重新定义现有命令。
\end{stripedbox}
\end{docCommand}

\begin{docCommand}{ProvideTCBInputListing}{\oarg{init options}\brackets{\texttt{\textbackslash}\meta{name}}\marg{specification}\marg{options}}
\begin{stripedbox}
Operates like \refComLe{DeclareTCBInputListing}, but based on |\ProvideDocumentCommand| instead of |\DeclareDocumentCommand|.
The command \texttt{\textbackslash}\meta{name} is only created if it is not already defined.
\tcblower
作类似\refComLe{DeclareTCBInputListing},但基于|\ProvideDocumentCommand|而不是|\DeclareDocumentCommand|。
命令\texttt{\textbackslash}\meta{name}只有在尚未定义时才会被创建。
\end{stripedbox}
\end{docCommand} 

% \clearpage
% Producing \texttt{tboxfit} Commands
\subsection{生成\texttt{tboxfit}命令}\label{subsec:xparse_tcboxfit}
\begin{marker}
\begin{stripedbox}
The following commands need the \mylib{fitting} library to be included.
\tcblower
下面的命令需要包含\mylib{fitting}库。
\end{stripedbox}
\end{marker}

\begin{docCommand}{DeclareTCBoxFit}{\oarg{init options}\brackets{\texttt{\textbackslash}\meta{name}}\marg{specification}\marg{options}}
\begin{stripedbox}
Creates a new command \texttt{\textbackslash}\meta{name} based on \refComLe{tcboxfit}.
Basically, |\DeclareTCBoxFit| operates like |\DeclareDocumentCommand|. This means,
the new command \texttt{\textbackslash}\meta{name} is constructed with the given argument \meta{specification}.
The \meta{options} are given to the underlying \refComLe{tcboxfit}.\\
Note that \refKeyLe{/tcb/savedelimiter} is set to the given \meta{name}
automatically.\\
The \meta{init options} allow setting up automatic numbering,
see Section \ref{sec:initkeys} from page \pageref{sec:initkeys}.\\
The new command is always created, irrespective of an already existing
command with the same name.
\tcblower
基于\refComLe{tcboxfit}创建一个新命令\texttt{\textbackslash}\meta{name}。%
基本上,|\DeclareTCBoxFit|操作类似|\DeclareDocumentCommand|。这意味着,新命令\texttt{\textbackslash}\meta{name}是用给定的参数\meta{specification}构造的。
\meta{options}被赋给底层的\refComLe{tcboxfit}。\\
注意\refKeyLe{/tcb/savedelimiter}自动被设置为给定的\meta{name}。\\
\meta{init选项}允许设置自动编号,%
请参见 \pageref{sec:initkeys} 页的 \ref{sec:initkeys} 章节。\\
新命令总是被创建,不管是否已经存在同名的命令。
\end{stripedbox}
\begin{dispExample*}{sbs,lefthand ratio=0.6}
% \usepackage{lipsum}

\DeclareTCBoxFit{\mybox}{ O{} m !o }
 {colback=red!5!white,
  colframe=red!75!black,
  width=#2,height=#2/3*2,
  IfValueT={#3}{height=#3},
  #1}

\mybox[colback=yellow]{5cm}%
  {\lipsum[2]}

\mybox[colback=yellow]{5cm}[4cm]{\lipsum[2]}
\end{dispExample*}
\end{docCommand}

\begin{docCommand}{NewTCBoxFit}{\oarg{init options}\brackets{\texttt{\textbackslash}\meta{name}}\marg{specification}\marg{options}}
\begin{stripedbox}
Operates like \refComLe{DeclareTCBoxFit}, but based on |\NewDocumentCommand| instead of |\DeclareDocumentCommand|.
An error is issued if \texttt{\textbackslash}\meta{name} has already been defined.
\tcblower
操作类似\refComLe{DeclareTCBoxFit},但基于|\NewDocumentCommand|而不是|\DeclareDocumentCommand|。
如果已经定义了\texttt{\textbackslash}\meta{name},则会发出错误。
\end{stripedbox}
\end{docCommand}

\begin{docCommand}{RenewTCBoxFit}{\oarg{init options}\brackets{\texttt{\textbackslash}\meta{name}}\marg{specification}\marg{options}}
\begin{stripedbox}
Operates like \refComLe{DeclareTCBoxFit}, but based on |\RenewDocumentCommand| instead of |\DeclareDocumentCommand|.
An existing command is redefined.
\tcblower
操作类似\refComLe{DeclareTCBoxFit},但基于|\RenewDocumentCommand|而不是|\DeclareDocumentCommand|。
重新定义现有命令。
\end{stripedbox}
\end{docCommand}

\begin{docCommand}{ProvideTCBoxFit}{\oarg{init options}\brackets{\texttt{\textbackslash}\meta{name}}\marg{specification}\marg{options}}
\begin{stripedbox}
Operates like \refComLe{DeclareTCBoxFit}, but based on |\ProvideDocumentCommand| instead of |\DeclareDocumentCommand|.
The command \texttt{\textbackslash}\meta{name} is only created if it is not already defined.
\tcblower
操作类似\refComLe{DeclareTCBoxFit},但基于|\ProvideDocumentCommand|而不是|\DeclareDocumentCommand|。
命令\texttt{\textbackslash}\meta{name}只有在尚未定义时才会被创建。
\end{stripedbox}
\end{docCommand}

% \clearpage

\begin{docCommand}{DeclareTotalTCBoxFit}{\oarg{init options}\brackets{\texttt{\textbackslash}\meta{name}}\marg{specification}\marg{options}\marg{content}}
\begin{stripedbox}
Creates a new command \texttt{\textbackslash}\meta{name} based on \refComLe{tcboxfit}.
In contrast to \refComLe{DeclareTCBoxFit}, also the \meta{content} of the |tcboxfit| is specified.\\
Basically, |\DeclareTotalTCBoxFit| operates like |\DeclareDocumentCommand|. This means,
the new command \texttt{\textbackslash}\meta{name} is constructed with the given argument \meta{specification}.
The \meta{options} are given to the underlying \refComLe{tcboxfit} which is filled with
the specified \meta{content}.\\
Note that \refKeyLe{/tcb/savedelimiter} is set to the given \meta{name}
automatically.\\
The \meta{init options} allow setting up automatic numbering,
see Section \ref{sec:initkeys} from page \pageref{sec:initkeys}.\\
The new command is always created, irrespective of an already existing
command with the same name.
\tcblower
基于\refComLe{tcboxfit}创建一个新命令\texttt{\textbackslash}\meta{name}。
与\refComLe{DeclareTCBoxFit}不同,|tcboxfit|的\meta{content}也被指定。\\
基本上,|\DeclareTotalTCBoxFit|操作类似|\DeclareDocumentCommand|。这意味着,
新命令\texttt{\textbackslash}\meta{name}是用给定的参数\meta{specification}构造的。
\meta{options}被设置到底层的\refComLe{tcboxfit},该\refComLe{tcboxfit}被填充指定的\meta{content}。\\
注意\refKeyLe{/tcb/savedelimiter}被自动设置为给定的\meta{name}。\\
\meta{init选项}允许设置自动编号,
请参见 \pageref{sec:initkeys} 页的 \ref{sec:initkeys} 章节。\\
新命令总是被创建,不管是否已经存在同名命令。
\end{stripedbox}

\begin{dispExample}
% \usepackage{lipsum}

\DeclareTotalTCBoxFit{\multibox}{ O{} m O{10} m }
 {nobeforeafter,colback=red!5!white,colframe=red!75!black,width=#2,height=#2/3*2,
  valign=center,#1}
 { \foreach \n in {1,...,#3} { #4} }

\multibox{5cm}{I shall not repeat.}
\multibox[colframe=blue!75!white]{5cm}[20]{I shall not repeat.}\\
\multibox[colback=yellow,height=5cm]{14cm}[100]{I shall not repeat.}
\end{dispExample}
\end{docCommand}

\begin{docCommand}{NewTotalTCBoxFit}{\oarg{init options}\brackets{\texttt{\textbackslash}\meta{name}}\marg{specification}\marg{options}\marg{content}}
\begin{stripedbox}
Operates like \refComLe{DeclareTotalTCBoxFit}, but based on |\NewDocumentCommand| instead of |\DeclareDocumentCommand|.
An error is issued if \texttt{\textbackslash}\meta{name} has already been defined.
\tcblower
操作类似\refComLe{DeclareTotalTCBoxFit},但基于|\NewDocumentCommand|而不是|\DeclareDocumentCommand|。
如果已经定义了\texttt{\textbackslash}\meta{name},则会发出错误。
\end{stripedbox}
\end{docCommand}


\begin{docCommand}{RenewTotalTCBoxFit}{\oarg{init options}\brackets{\texttt{\textbackslash}\meta{name}}\marg{specification}\marg{options}\marg{content}}
\begin{stripedbox}
Operates like \refComLe{DeclareTotalTCBoxFit}, but based on |\RenewDocumentCommand| instead of |\DeclareDocumentCommand|.
An existing command is redefined.
\tcblower
操作类似\refComLe{DeclareTotalTCBoxFit},但基于|\RenewDocumentCommand|而不是|\DeclareDocumentCommand|。
重新定义现有命令。
\end{stripedbox}
\end{docCommand}

\begin{docCommand}{ProvideTotalTCBoxFit}{\oarg{init options}\brackets{\texttt{\textbackslash}\meta{name}}\marg{specification}\marg{options}\marg{content}}
\begin{stripedbox}
Operates like \refComLe{DeclareTotalTCBoxFit}, but based on |\ProvideDocumentCommand| instead of |\DeclareDocumentCommand|.
The command \texttt{\textbackslash}\meta{name} is only created if it is not already defined.
\tcblower
操作类似\refComLe{DeclareTotalTCBoxFit},但基于|\ProvideDocumentCommand|而不是|\DeclareDocumentCommand|。
命令\texttt{\textbackslash}\meta{name}只有在尚未定义时才会被创建。
\end{stripedbox}
\end{docCommand}% 译 done 2023.1.26
% \setcounter{section}{24}
% \include*{tcolorbox.doc.external}%v1 2023 04 13  有一个有些问题
% \setcounter{section}{25}

% % % !TeX root = tcolorbox.tex
% % include file of tcolorbox.tex (manual of the LaTeX package tcolorbox)
% \clearpage
\section{Library \mylib{documentation}}\label{sec:documentation}%
\tcbset{external/prefix=external/documentation_}%
% This library has the single purpose to support \LaTeX\ package documentations
% like this one. Actually, the visual nature follows the approach from
% Till Tantau's |pgf| \cite{tantau:tikz_and_pgf} documentation.
% Typically, this library is assumed to be used in conjunction with the
% class |ltxdoc| or alike.
% Denis Bitouz\'e, Muzimuzhi, and many others provided very valuable input for this library.

% 这个库的唯一目的是支持像这个一样的\LaTeX\ 包文档。实际上,其视觉风格遵循了Till Tantau的|pgf|\cite{tantau:tikz_and_pgf}文档的方法。通常,这个库被认为是与类|ltxdoc|或类似的类一起使用的。Denis Bitouz'e,Muzimuzhi和许多其他人为这个库提供了非常有价值的输入。

% The library is loaded by a package option or inside the preamble by:

% 该库可以通过软件包选项或者在导言部分中被加载:
% \begin{dispListing}
%   \tcbuselibrary{documentation}
% \end{dispListing}
% This also loads
% the library \mylib{skins}, see \Vref{sec:skins},
% the library \mylib{raster}, see \Vref{sec:raster},
% the library \mylib{listings}, see \Vref{sec:listings},
% the library \mylib{xparse}, see \Vref{sec:xparse},
% and a bunch of packages, namely
% |makeidx|, |marginnote|, |refcount|, and |hyperref|.
% The packages |pifont| and |marvosym| should be installed for some symbols, but
% need not to be loaded.

% 这也会加载库\mylib{skins},参见\Vref{sec:skins}, 库\mylib{raster},参见\Vref{sec:raster}, 库\mylib{listings},参见\Vref{sec:listings}, 库\mylib{xparse},参见\Vref{sec:xparse}, 以及一堆包,即 |makeidx|,|marginnote|,|refcount|和|hyperref|。 对于一些符号,应安装包|pifont|和|marvosym|,但不需要加载。

% \begin{marker}
% The package |makeidx| is loaded only, if \docAuxCommand*{printindex} is
% \emph{not} already defined. Therefore, one can include an alternative to |makeidx| like
% |imakeidx| \emph{before} the library |documentation| is used.

% 只有当\docAuxCommand*{printindex}没有被定义时,才会加载|makeidx|包。因此,在使用库|documentation|之前,可以在其之前包含一个替代|makeidx|的包,例如|imakeidx|。
% \end{marker}
% \begin{marker}
% The package |marginnote| is loaded only, if \docAuxCommand*{marginnote} is
% \emph{not} already defined.

% 只有当 \docAuxCommand*{marginnote} 没有被定义时,才会加载 |marginnote| 包。
% \end{marker}
% \begin{marker}
% In contrast to other |tcolorbox| options, the option
% settings for \mylib{documentation} are typically not
% getting reset by \refKey{/tcb/reset}, i.e. they keep their
% values for embedded boxes.

% 与其他 |tcolorbox| 选项不同的是,\mylib{documentation} 的选项设置通常不会被 \refKey{/tcb/reset} 重置,即它们会保持嵌入箱子的值不变。
% \end{marker}
% \begin{marker}
% In combination with DocStrip, \refKey{/tcb/verbatim ignore percent} may be helpful.

% 结合使用DocStrip,\refKey{/tcb/verbatim ignore percent}可能会有所帮助。
% \end{marker}

% For UTF-8 support load (ignore this when using Xe\LaTeX):

% 对于UTF-8支持加载(当使用Xe\LaTeX 时请忽略此项):
% \begin{dispListing}
%   \tcbuselibrary{listingsutf8,documentation}
% \end{dispListing}

% For |minted| \cite{poore:minted} support, load:

% 要使用 |minted| \cite{poore:minted},请加载以下内容:
% \begin{dispListing}
%   \tcbuselibrary{documentation,minted}
%   \tcbset{listing engine=minted}
% \end{dispListing}


%%% todo 以上后面放开

% %\clearpage
% %-------------------------------------------------------------------------------
\subsection{Macros of the Library\\库的宏}
% \begin{docEnvironment}[doclang/environment content=command description,doc updated=2020-04-22]
  {docCommand}{\oarg{options}\marg{name}\marg{parameters}}
Documents a \LaTeX\ macro with given \meta{name} where \meta{name} is
written without backslash. The given \meta{options} are set with \refCom{tcbset}.
This macro takes mandatory or optional \meta{parameters}.
It is automatically indexed and can be referenced with
\refCom{refCom}\marg{name}.

记录一个具有给定 \meta{name} 的 \LaTeX\ 宏,其中 \meta{name} 不带反斜杠。给定的 \meta{options} 通过 \refCom{tcbset} 设置。这个宏需要强制或可选的 \meta{parameters}。它会自动索引,并可以通过 \refCom{refCom}\marg{name} 引用。
\begin{dispExample}
\begin{docCommand}{foomakedocSubKey}{\marg{name}\marg{key path}}
Creates a new environment \meta{name} based on \refEnv{docKey} for the
documentation of keys with the given \meta{key path}.

基于\refEnv{docKey},为给定的\meta{key path}创建一个名为\meta{name}的新环境,用于记录关键字的文档。
\end{docCommand}
\end{dispExample}
\begin{dispExample}
\begin{docCommand}[color definition=blue]{foomakedocSubKey*}%
  {\marg{name}\marg{key path}}
Creates a new environment \meta{name} based on \refEnv{docKey} for the
documentation of keys with the given \meta{key path}.

基于\refEnv{docKey},为给定的\meta{key path}创建一个名为\meta{name}的新环境,用于键的文档化。
\end{docCommand}
\end{dispExample}
\end{docEnvironment}

\begin{docEnvironment}[doclang/environment content=command description,doc updated=2020-04-22]
  {docCommand*}{\oarg{options}\marg{name}\marg{parameters}}
Identical to \refEnv{docCommand}, but without index entry.

与 |docCommand| 相同,但不包含索引条目。
\end{docEnvironment}

\begin{docEnvironment}[doclang/environment content=command description,doc new=2020-04-22]
  {docCommands}{\oarg{options}\brackets{\marg{variant1},\marg{variant2},...}}
Documents several (similar) \LaTeX\ macro variants simultaneously.
The given \meta{options} are set with \refCom{tcbset} and are valid for
all variants and the documentation text.
Every variant is described by an option set \meta{variant1}, \meta{variant2}, and so on.
The most crucial options are \refKeyLe{/tcb/doc name} and \refKeyLe{/tcb/doc parameter}.

同时记录了几个类似的 \LaTeX\ 宏变量。 给定的 \meta{选项} 是通过 |tcbset| 设置的,对于所有变体和文档文本都有效。 每个变体都由选项集 \meta{variant1}、\meta{variant2} 等描述。 最关键的选项是 |/tcb/doc name| 和 |/tcb/doc parameter|。
\begin{dispExample}
\begin{docCommands}[
  doc no index,  %  no index entries for this example
  doc name      = newtheorem,
  doc parameter = \marg{envname},
]
{
  {  },
  { doc parameter = \marg{envname}\oarg{numbered within} },
  { doc parameter = \oarg{numbered like}\marg{envname} },
  { doc name      = newtheorem* },
}
example
\end{docCommands}
\end{dispExample}
\end{docEnvironment}
% % \clearpage
{\let\xdocEnvironment\docEnvironment
\let\endxdocEnvironment\enddocEnvironment
\begin{xdocEnvironment}[doclang/environment content=environment description,doc updated=2020-04-22]
  {docEnvironment}{\oarg{options}\marg{name}\marg{parameters}}
Documents a \LaTeX\ environment with given \meta{name}.
The given \meta{options} are set with \refComLe{tcbset}.
This environment takes mandatory or optional \meta{parameters}.
It is automatically indexed and can be referenced with
\refComLe{refEnv}\marg{name}.

记录带有给定名称 \meta{name} 的 \LaTeX\ 环境。 给定的 \meta{options} 通过 |tcbset| 进行设置。 该环境可以带有强制或可选的 \meta{parameters}。 它会自动编制索引,并可以通过 |\refEnv|\marg{name} 进行引用。

\begin{dispExample}
\begin{docEnvironment}{foocolorbox}{\oarg{options}}
This is the main environment to create an accentuated colored text box with
rounded corners and, optionally, two parts.

这是创建带有突出颜色的圆角文本框的主要环境,还可以选择性地分为两个部分。
\end{docEnvironment}
\end{dispExample}
\begin{dispExample}
\begin{docEnvironment}%
  [doclang/environment content=My content text]%
  {foocolorbox*}{\oarg{options}}
This is the main environment to create an accentuated colored text box with
rounded corners and, optionally, two parts.

这是创建一个强调着色的文本框的主要环境,它具有圆角,并可选地分为两个部分。
\end{docEnvironment}
\end{dispExample}
\end{xdocEnvironment}}

{\let\xdocEnvironment\docEnvironment
\let\endxdocEnvironment\enddocEnvironment
\begin{xdocEnvironment}[doclang/environment content=environment description,doc updated=2020-04-22]
  {docEnvironment*}{\oarg{options}\marg{name}\marg{parameters}}
Identical to \refEnvLe{docEnvironment}, but without index entry.

与\refEnvLe{docEnvironment}相同,但没有索引条目。
\end{xdocEnvironment}}

%% \clearpage
{\let\xdocEnvironment\docEnvironment
\let\endxdocEnvironment\enddocEnvironment
\begin{xdocEnvironment}[doclang/environment content=environment description,doc new=2020-04-22]
  {docEnvironments}{\oarg{options}\brackets{\marg{variant1},\marg{variant2},...}}
Documents several (similar) \LaTeX\ environment variants simultaneously.
The given \meta{options} are set with \refComLe{tcbset} and are valid for
all variants and the documentation text.
Every variant is described by an option set \meta{variant1}, \meta{variant2}, and so on.
The most crucial options are \refKeyLe{/tcb/doc name} and \refKeyLe{/tcb/doc parameter}.

同时记录了几个(相似的)\LaTeX\ 环境变体。 给定的 \meta{options} 是通过 |\tcbset| 设置的,对于所有变体和文档文本都有效。 每个变体都由选项集 \meta{variant1}、\meta{variant2} 等描述。 最关键的选项是 |/tcb/doc name| 和 |/tcb/doc parameter|。
\begin{dispExample}
\begin{docEnvironments}[
  doc no index,   %  no index entries for this example
  doc parameter = \oarg{options}\marg{title},
  doclang/environment content = box content,
]
{
  {
    doc name        = redbox,
    doc description = a red colored box,
  },
  {
    doc name        = greenbox,
    doc description = a green colored box,
  },
  {
    doc name        = bluebox,
    doc description = a blue colored box,
  },
  {
    doc name        = custombox,
    doc parameter   = \oarg{options}\marg{color}\marg{title},
    doc description = a colored box,
  },
}
example
\end{docEnvironments}
\end{dispExample}
\end{xdocEnvironment}}
% %% \clearpage
\begin{docEnvironment}[doclang/environment content=key description,doc updated=2020-04-22]
  {docKey}{\oarg{key path}\oarg{options}\marg{name}\marg{parameters}\marg{description}}
Documents a key with given \meta{name} and an optional \meta{key path}.
The given \meta{options} are set with \refCom{tcbset}.
This key takes mandatory or optional \meta{parameters} as value
with a short \meta{description}.
It is automatically indexed and can be referenced with
\refCom{refKey}\marg{name}.

记录一个带有给定的 \meta{name} 和可选的 \meta{key path} 的关键字。 给定的 \meta{options} 由 |\tcbset| 设置。 该关键字以强制或可选的 \meta{parameters} 作为值, 并带有简短的 \meta{description}。 它会自动索引,并可以通过 |\refKey|\marg{name} 引用。
\begin{dispExample}
\begin{docKey}[foo]{footitle}{=\meta{text}}{no default, initially empty}
Creates a heading line with \meta{text} as content.
\end{docKey}
\end{dispExample}
\end{docEnvironment}


\begin{docEnvironment}[doclang/environment content=key description,doc updated=2020-04-22]
  {docKey*}{\oarg{key path}\oarg{options}\marg{name}\marg{parameters}\marg{description}}
Identical to \refEnv{docKey}, but without index entry.

与 \refEnv{docKey} 相同,但没有索引条目。
\end{docEnvironment}

\begin{docEnvironment}[doclang/environment content=key description,doc new=2020-04-22]
  {docKeys}{\oarg{options}\brackets{\marg{variant1},\marg{variant2},...}}
Documents several (similar) key variants simultaneously.
The given \meta{options} are set with \refCom{tcbset} and are valid for
all variants and the documentation text.
Every variant is described by an option set \meta{variant1}, \meta{variant2}, and so on.
The most crucial options are
\refKeyLe{/tcb/doc keypath}, \refKeyLe{/tcb/doc name}, \refKeyLe{/tcb/doc parameter},
and \refKeyLe{/tcb/doc description}.

同时记录几个(相似的)关键变量。 给定的\meta{options}是通过|\tcbset|设置的,对所有变量和文档文本有效。 每个变量由一个选项集\meta{variant1},\meta{variant2}等描述。 最关键的选项是|/tcb/doc keypath|,|/tcb/doc name|,|/tcb/doc parameter|和|/tcb/doc description|。
\begin{dispExample}
\begin{docKeys}[
  doc no index,   %  no index entries for this example
  doc keypath   = mykeyroot,
  doc parameter = {=\meta{length}},
]
{
  {
    doc name        = width,
    doc description = initially \texttt{10cm},
  },
  {
    doc name        = height,
    doc description = initially \texttt{7cm},
  },
}
example
\end{docKeys}
\end{dispExample}
\end{docEnvironment}


% %% \clearpage
\begin{docEnvironment}[doclang/environment content=operation description,
  doc new and updated={2019-09-18}{2020-04-22}]{docPathOperation}{\oarg{options}\marg{name}\marg{parameters}}
Documents a \tikzname\ path operation with given \meta{name}.
The given \meta{options} are set with \refCom{tcbset}.
This \tikzname\ path operation takes mandatory or optional \meta{parameters}.
It is automatically indexed and can be referenced with
\refCom{refPathOperation}\marg{name}.

使用给定的\meta{name}记录\tikzname\ 路径操作。给定的\meta{options}使用|\tcbset|设置。这个\tikzname\ 路径操作接受必选或可选的\meta{parameters}。它会自动索引,并可以通过|\refPathOperation|\marg{name}进行引用。
\begin{dispExample}
\begin{docPathOperation}{fooop}{\oarg{opt}(\meta{name})\colOpt{at(\meta{coord})}}
Imaginary path operation for illustration.
\end{docPathOperation}
\end{dispExample}
\end{docEnvironment}


\begin{docEnvironment}[doclang/environment content=command description,
  doc new and updated={2019-09-18}{2020-04-22}]{docPathOperation*}{\oarg{options}\marg{name}\marg{parameters}}
Identical to \refEnv{docPathOperation}, but without index entry.

\refEnv{docPathOperation} 的副本,但不含索引条目。
\end{docEnvironment}


\begin{docEnvironment}[doclang/environment content=command description,
  doc new={2020-04-22}]{docPathOperations}{\oarg{options}\brackets{\marg{variant1},\marg{variant2},...}}
Documents several (similar) \tikzname\ path operation variants simultaneously.
The given \meta{options} are set with \refCom{tcbset} and are valid for
all variants and the documentation text.
Every variant is described by an option set \meta{variant1}, \meta{variant2}, and so on.
The most crucial options are \refKey{/tcb/doc name} and \refKey{/tcb/doc parameter}.

同时记录了几个(类似的)\tikzname\ 路径操作变体。 给定的\meta{选项}通过|\tcbset|进行设置,并适用于所有变体和文档文本。 每个变体由一个选项集\meta{variant1}、\meta{variant2}等描述。 最关键的选项是|/tcb/doc name|和|/tcb/doc parameter|。

\begin{dispExample}
\begin{docPathOperations}[
  doc no index,   %  no index entries for this example
]
{
  {
    doc name      = rectangle,
    doc parameter = \meta{corner or cycle},
  },
  {
    doc name      = circle,
    doc parameter = \oarg{options},
  },
  {
    doc name      = ellipse,
    doc parameter = \oarg{options},
  },
}
example
\end{docPathOperations}
\end{dispExample}
\end{docEnvironment}
% % \clearpage
\begin{docCommands}[doc parameter=\oarg{options}\marg{name}]
{
  {
    doc name = docValue,
    doc updated=2020-04-23,
  },
  {
    doc name = docValue*,
  },
}
Documents a value with given \meta{name}. Typically, this is a value for a key.
The given \meta{options} are set with \refComLe{tcbset}.
This value is automatically indexed for \refComLe{docValue}
and has no index entry for \refComLe{docValue*}.

使用给定的 \meta{name} 记录一个值。通常,这是一个键的值。 给定的 \meta{options} 使用 |\tcbset| 进行设置。 此值会自动索引到 |docValue| 中,但不会在 |docValue*| 中有索引条目。
\begin{dispExample}
A feasible value for \refKeyLe{/foo/footitle} is \docValue*{foovalue}.
\end{dispExample}
\end{docCommands}
% \begin{docCommands}[doc parameter=\oarg{options}\marg{name}]
{
  {
    doc name    = docAuxCommand,
    doc updated = 2020-04-23,
  },
  {
    doc name = docAuxCommand*,
  },
}
Documents an auxiliary or minor \LaTeX\ macro with given \meta{name}
where \meta{name} is written without backslash.
The given \meta{options} are set with \refComLe{tcbset}.
This macro is automatically indexed for \refComLe{docAuxCommand}
and has no index entry for \refComLe{docAuxCommand*}.

记录一个带有给定\meta{name}的辅助或次要\LaTeX\ 宏,\meta{name}不带反斜杠。给定的\meta{options}由\refComLe{tcbset}设置。此宏将自动索引为\refComLe{docAuxCommand},而\refComLe{docAuxCommand*}没有索引条目。
\begin{dispExample}
The macro \docAuxCommand{fooaux} holds some interesting data.
\end{dispExample}
\end{docCommands}


\begin{docCommands}[doc parameter=\oarg{options}\marg{name}]
{
  {
    doc name    = docAuxEnvironment,
    doc updated = 2020-04-23,
  },
  {
    doc name = docAuxEnvironment*,
  },
}
Documents an auxiliary or minor \LaTeX\ environment with given \meta{name}.
The given \meta{options} are set with \refComLe{tcbset}.
This macro is automatically indexed indexed for \refComLe{docAuxEnvironment}
and has no index entry for \refComLe{docAuxEnvironment*}.

记录一个带有给定\meta{name}的辅助或次要的\LaTeX 环境。 给定的\meta{options}使用\refComLe{tcbset}设置。 此宏自动索引索引用于\refComLe{docAuxEnvironment},并且对于\refComLe{docAuxEnvironment*}没有索引条目。
\begin{dispExample}
The environment \docAuxEnvironment{fooauxenv} holds some interesting data.
\end{dispExample}
\end{docCommands}


\begin{docCommands}[doc parameter=\oarg{key path}\oarg{options}\marg{name}]
{
  {
    doc name    = docAuxKey,
    doc updated = 2020-04-23,
  },
  {
    doc name = docAuxKey*,
  },
}
Documents an auxiliary key with given \meta{name} and an optional \meta{key path}.
The given \meta{options} are set with \refComLe{tcbset}.
It is automatically indexed for \refComLe{docAuxKey}
and has no index entry for \refComLe{docAuxKey*}.

记录带有给定的 \meta{name} 和可选的 \meta{key path} 的辅助键。给定的 \meta{options} 通过 \refComLe{tcbset} 设置。它会自动被 \refComLe{docAuxKey} 索引,但不会有 \refComLe{docAuxKey*} 的索引条目。
\begin{dispExample}
The key \docAuxKey[foo]{fooaux} holds some interesting data.
\end{dispExample}
\end{docCommands}


\begin{docCommands}[doc parameter=\oarg{options}\marg{name}]
{
  {
    doc name    = docCounter,
    doc updated = 2020-04-23,
  },
  {
    doc name = docCounter*,
  },
}
Documents a counter with given \meta{name}.
The given \meta{options} are set with \refCom{tcbset}.
The counter is automatically indexed for \refCom{docCounter}
and has no index entry for \refCom{docCounter*}.

记录一个给定\meta{name}的计数器。 给定的\meta{options}由\refCom{tcbset}设置。 计数器会自动为\refCom{docCounter}索引,但对于\refCom{docCounter*}没有索引条目。
\begin{dispExample}
The counter \docCounter{foocounter} can be used for computation.
\end{dispExample}
\end{docCommands}


%% \clearpage
\begin{docCommands}[doc parameter=\oarg{options}\marg{name}]
{
  {
    doc name    = docLength,
    doc updated = 2020-04-23,
  },
  {
    doc name = docLength*,
  },
}
Documents a length with given \meta{name}.
The given \meta{options} are set with \refCom{tcbset}.
The length is automatically indexed for \refCom{docLength}
and has no index entry for \refCom{docLength*}.

使用给定的\meta{name}记录长度。 使用\refCom{tcbset}设置给定的\meta{options}。 该长度将自动索引\refCom{docLength}, 并且\refCom{docLength*}没有索引条目。
\begin{dispExample}
The length \docLength{foolength} can be used for computation.
\end{dispExample}
\end{docCommands}


\begin{docCommands}[doc parameter=\oarg{options}\marg{name}]
{
  {
    doc name    = docColor,
    doc updated = 2020-04-23,
  },
  {
    doc name = docColor*,
  },
}
Documents a color with given \meta{name}.
The given \meta{options} are set with \refCom{tcbset}.
The color is automatically indexed for \refCom{docColor}
and has no index entry for \refCom{docColor*}.

记录给定\meta{name}的颜色。 使用\refCom{tcbset}设置给定的\meta{options}。 该颜色将自动为\refCom{docColor}建立索引, 但对于\refCom{docColor*}没有索引条目。
\begin{dispExample}
The color \docColor{foocolor} is available.
\end{dispExample}
\end{docCommands}



\begin{docCommand}{cs}{\marg{name}}
Macro from |ltxdoc| \cite{carlisle:ltxdoc} to typeset a command word \meta{name}
where the backslash is prefixed. The library overwrites the original macro.

从 |ltxdoc| \cite{carlisle:ltxdoc} 中的宏,用于排版以反斜杠为前缀的命令词 \meta{name}。该库会覆盖原始宏。
\begin{dispExample}
This is a \cs{foocommand}.
\end{dispExample}
\end{docCommand}

\begin{docCommand}{meta}{\marg{text}}
Macro from |doc| \cite{mittelbach:2011a} to typeset a meta \meta{text}.
The library overwrites the original macro.

从 |doc| \cite{mittelbach:2011a} 到排版元 \meta{text} 的宏。该库覆盖了原始的宏。
\begin{dispExample}
This is a \meta{text}.
\end{dispExample}
\end{docCommand}


\begin{docCommand}{marg}{\marg{text}}
Macro from |ltxdoc| \cite{carlisle:ltxdoc} to typeset a \meta{text} with
curly brackets as a mandatory argument. The library overwrites the original macro.

从 |ltxdoc| \cite{carlisle:ltxdoc} 中获取宏,用花括号作为强制参数来排版 \meta{text}。该库覆盖了原始宏。
\begin{dispExample}
This is a mandatory \marg{argument}.
\end{dispExample}
\end{docCommand}

\begin{docCommand}{oarg}{\marg{text}}
Macro from |ltxdoc| \cite{carlisle:ltxdoc} to typeset a \meta{text} with
square brackets as an optional argument. The library overwrites the original macro.

从 |ltxdoc| \cite{carlisle:ltxdoc} 中的宏开始,使用方括号作为可选参数排版 \meta{text}。该库会覆盖原始宏。
\begin{dispExample}
This is an optional \oarg{argument}.
\end{dispExample}
\end{docCommand}

%% \clearpage

\begin{docCommand}{brackets}{\marg{text}}
Sets the given \meta{text} with curly brackets.

将给定的 \meta{text} 使用花括号设置。
\begin{dispExample}
Here we use \brackets{some text}.
\end{dispExample}
\end{docCommand}


{\let\xdispExample\dispExample
\let\endxdispExample\enddispExample
\begin{docEnvironment}[doc updated=2014-10-10]{dispExample}{}
Creates a colored box based on a \refEnv{tcolorbox}.
It displays the environment content as source code in the upper part
and as compiled text in the lower part of the box.
The appearance is controlled by \refKey{/tcb/documentation listing style}
and the style \refKey{/tcb/docexample}. It may be
changed by redefining this style.

基于\refEnv{tcolorbox}创建一个带颜色的框。它将环境内容分别显示在框的上部源代码和下部编译文本中。外观由\refKey{/tcb/documentation listing style} 和样式\refKey{/tcb/docexample}控制。可以通过重新定义此样式来改变它。
{
%\tcbset{before lower app={\tcbset{docexample/.style={docexample original}}}}
%\tcbset{docexample/.style={docexample original}}%
\begin{xdispExample}
\begin{dispExample}
This is a \LaTeX\ example.
\end{dispExample}
\end{xdispExample}
}
\end{docEnvironment}}


{\let\xdispExample\dispExample
\let\endxdispExample\enddispExample
\begin{docEnvironment}[doc updated=2014-10-10]{dispExample*}{\marg{options}}
The starred version of \refEnv{dispExample} takes \refEnv{tcolorbox} \meta{options}
as parameter. These \meta{options} are executed after \refKey{/tcb/docexample}.

\refEnv{dispExample} 的带星号版本接受 \refEnv{tcolorbox} 的 \meta{选项} 作为参数。这些 \meta{选项} 在 \refKey{/tcb/docexample} 之后执行。
\begin{xdispExample}
\begin{dispExample*}{sidebyside}
This is a \LaTeX\ example.
\end{dispExample*}
\end{xdispExample}
\end{docEnvironment}}


%%\clearpage
\begin{docEnvironment}{dispListing}{}
Creates a colored box based on a \refEnv{tcolorbox}.
It displays the environment content as source code.
The appearance is controlled by \refKey{/tcb/documentation listing style}
and the style \refKey{/tcb/docexample}. It may be
changed by redefining this style.

基于 \refEnv{tcolorbox} 创建一个有颜色的盒子。它将环境内容以源代码形式显示出来。外观由 \refKey{/tcb/documentation listing style} 和样式 \refKey{/tcb/docexample} 控制。可以通过重新定义此样式来进行更改。
\begin{dispExample}
\begin{dispListing}
This is a \LaTeX\ example.
\end{dispListing}
\end{dispExample}
\end{docEnvironment}

\begin{docEnvironment}{dispListing*}{\marg{options}}
The starred version of \refEnv{dispListing} takes \refEnv{tcolorbox} \meta{options}
as parameter. These \meta{options} are executed after \refKey{/tcb/docexample}.

\refEnv{dispListing} 的加星版本以 \refEnv{tcolorbox} 的 \meta{选项} 作为参数。这些 \meta{选项} 在 \refKey{/tcb/docexample} 之后执行。
\begin{dispExample}
\begin{dispListing*}{title=My listing}
This is a \LaTeX\ example.
\end{dispListing*}
\end{dispExample}
\end{docEnvironment}


% \begin{docEnvironment}{absquote}{}
% Used to typeset an abstract as quoted and small text.
% \begin{dispExample}
% \begin{absquote}
% |tcolorbox| provides an environment for colored and framed text boxes with a
% heading line. Optionally, such a box can be split in an upper and a lower part.
% \end{absquote}
% \end{dispExample}
% \end{docEnvironment}

% \clearpage
% \begin{docCommand}[doc updated=2020-04-22]{tcbmakedocSubKey}{\marg{name}\marg{key path}}
% Creates a new environment \meta{name} based on \refEnv{docKey} for the
% documentation of keys with the given \meta{key path} as root.
% The new environment \meta{name} takes the same para\-meters as \refEnv{docKey} itself.
% A second starred environment \meta{name} is also created, which is identical
% to \meta{name} but without index entry.
% \begin{dispExample}
% \tcbmakedocSubKey{docFooKey}{foo}

% \begin{docFooKey}{foodummy}{=\meta{nothing}}{no default, initially empty}
% Some key.
% \end{docFooKey}

% \begin{docFooKey*}{foo another dummy}{=\meta{nothing}}{no default, initially empty}
% Some key (not indexed).
% \end{docFooKey*}
% \end{dispExample}
% \end{docCommand}


% \begin{docCommand}[doc new=2020-04-22]{tcbmakedocSubKeys}{\marg{name}\marg{key path}}
% Creates a new environment \meta{name} based on \refEnv{docKeys} for the
% documentation of keys with the given \meta{key path} as root.
% The new environment \meta{name} takes the same para\-meters as \refEnv{docKeys} itself.
% \begin{dispExample}
% \tcbmakedocSubKeys{docFooKeys}{foo}

% \begin{docFooKeys}[
%   doc parameter   = {=\meta{nothing}},
%   doc description = {no default, initially empty},
% ]
% {
%   {
%     doc name = foodummy 2,
%   },
%   {
%     doc name = foo another dummy 2,
%     doc no index,
%   }
% }
% Some description.
% \end{docFooKeys}
% \end{dispExample}
% \end{docCommand}


% \clearpage

% \begin{docCommand}{refCom}{\marg{name}}
% References a documented \LaTeX\ macro with given \meta{name} where \meta{name} is
% written without backslash. The page reference is suppressed if it links
% to the same page.
% \begin{dispExample}
% We have created \refCom{foomakedocSubKey} as an example.
% \end{dispExample}
% \end{docCommand}

% \begin{docCommand}{refCom*}{\marg{name}}
% References a documented \LaTeX\ macro with given \meta{name} where \meta{name} is
% written without backslash. There is no page reference.
% \begin{dispExample}
% We have created \refCom*{foomakedocSubKey} as an example.
% \end{dispExample}
% \end{docCommand}


% \begin{docCommand}{refEnv}{\marg{name}}
% References a documented \LaTeX\ environment with given \meta{name}.
% The page reference is suppressed if it links to the same page.
% \begin{dispExample}
% We have created \refEnv{foocolorbox} as an example.
% \end{dispExample}
% \end{docCommand}

% \begin{docCommand}{refEnv*}{\marg{name}}
% References a documented \LaTeX\ environment with given \meta{name}.
% There is no page reference.
% \begin{dispExample}
% We have created \refEnv*{foocolorbox} as an example.
% \end{dispExample}
% \end{docCommand}


% \begin{docCommand}{refKey}{\marg{name}}
% References a documented key with given \meta{name} where \meta{name}
% is the full path name of the key.
% The page reference is suppressed if it links to the same page.
% \begin{dispExample}
% We have created \refKey{/foo/footitle} as an example.
% \end{dispExample}
% \end{docCommand}

% \begin{docCommand}{refKey*}{\marg{name}}
% References a documented key with given \meta{name} where \meta{name}
% is the full path name of the key.
% There is no page reference.
% \begin{dispExample}
% We have created \refKey*{/foo/footitle} as an example.
% \end{dispExample}
% \end{docCommand}

% \clearpage

% \begin{docCommand}[doc new=2019-09-17]{refPathOperation}{\marg{name}}
% References a documented \tikzname\ path operation with given \meta{name}.
% The page reference is suppressed if it links to the same page.
% \begin{dispExample}
% We have created \refPathOperation{fooop} as an example.
% \end{dispExample}
% \end{docCommand}

% \begin{docCommand}[doc new=2019-09-17]{refPathOperation*}{\marg{name}}
% References a documented \tikzname\ path operation with given \meta{name}.
% There is no page reference.
% \begin{dispExample}
% We have created \refPathOperation*{fooop} as an example.
% \end{dispExample}
% \end{docCommand}



% \begin{docCommand}[doc updated=2020-02-11]{refAux}{\marg{name}}
% References some auxiliary environment, key, value, or color.
% The \meta{name} is colored according to \refKey{/tcb/color hyperlink},
% if |hyperref| colorlinks are set, but there is no real link.
% \begin{dispExample}
% Some pages back, one can see \refAux{/foo/footitle} as an example.
% \end{dispExample}
% \end{docCommand}

% \begin{docCommand}[doc updated=2020-02-11]{refAuxcs}{\marg{name}}
% References some auxiliary macro \meta{name} where \meta{name} is
% written without backslash.
% The \meta{name} is colored according to \refKey{/tcb/color hyperlink},
% if |hyperref| colorlinks are set, but there is no real link.
% \begin{dispExample}
% Some pages back, one can see \refAuxcs{fooaux} as an example.
% \end{dispExample}
% \end{docCommand}


% \begin{docCommand}{colDef}{\marg{text}}
% Sets \meta{text} with the command color, see \refKey{/tcb/color command}.
% \begin{dispExample}
% This is my \colDef{text}.
% \end{dispExample}
% \end{docCommand}

% \begin{docCommand}{colOpt}{\marg{text}}
% Sets \meta{text} with the option color, see \refKey{/tcb/color option}.
% \begin{dispExample}
% This is my \colOpt{text}.
% \end{dispExample}
% \end{docCommand}

% \clearpage

% \begin{docCommand}[doc new=2019-09-18]{colFade}{\marg{text}}
% Sets \meta{text} with the fade color, see \refKey{/tcb/color fade}.
% \begin{dispExample}
% This is my \colFade{text}.
% \end{dispExample}
% \end{docCommand}


% \begin{docCommand}[doc new=2014-09-19]{tcbdocmarginnote}{\oarg{options}\marg{text}}
% Creates a |tcolorbox| note with the given \meta{text} inside the margin using
% the |marginnote| package. The style of the |tcolorbox| is predefined and can be
% altered by \refKey{/tcb/doc marginnote} and the given \meta{options}.
% \begin{dispExample}
% Some text\tcbdocmarginnote{Note A}
% which is commented by a note inside the margin.
% Alternatively to |\tcbdocmarginnote|, you can always use
% |\marginnote| with a |tcolorbox| directly.\par
% This is further text%
% \tcbdocmarginnote[colframe=blue!50!white,colback=blue!5!white]{Note B}
% with another note.
% \end{dispExample}
% \end{docCommand}

% \begin{docCommand}[doc new=2014-09-19]{tcbdocnew}{\marg{date}}
% Auxiliary macro which typesets the \refKey{/tcb/doclang/new} text with
% the given \meta{date}. It may be redefined for customization.
% \makeatletter\renewcommand*{\tcbdocnew}[1]{\kvtcb@text@new: #1}\makeatother%
% \begin{dispExample*}{sidebyside}
% \tcbdocnew{1981-10-29}.
% % Next one is displayed in the margin:
% \tcbdocmarginnote{\tcbdocnew{1978-02-09}}
% \end{dispExample*}
% \end{docCommand}

% \begin{docCommand}[doc new=2014-09-19]{tcbdocupdated}{\marg{date}}
% Auxiliary macro which typesets the \refKey{/tcb/doclang/updated} text with
% the given \meta{date}. It may be redefined for customization.
% \makeatletter\renewcommand*{\tcbdocupdated}[1]{\kvtcb@text@updated: #1}\makeatother%
% \begin{dispExample*}{sidebyside}
% \tcbdocupdated{2014-09-19}.
% \end{dispExample*}
% \end{docCommand}





% \clearpage
% %-------------------------------------------------------------------------------
% \subsection{Entry Content Option Keys}


% \begin{docTcbKey}[][doc new={2020-04-22}]{doc name}{=\meta{name}}{no default, initially empty}
%   Sets the \meta{name} of the entry to document, i.e. the \meta{name} of the
%   command, environment, key, etc. For \refEnv{docCommand}, \refEnv{docEnvironment}, etc.
%   the \meta{name} is set by a mandatory parameter, but can also be set
%   by \refKey{/tcb/doc name}.
%   \refKey{/tcb/doc name} also sets \meta{name} to
%   \refKey{/tcb/doc label}, \refKey{/tcb/doc index},
%   and \refKey{/tcb/doc sort index}.
% \begin{dispExample}
% \begin{docCommands}[
%     doc no index,  %  no index entries for this example
%     doc name      = bfseries,
%   ] {}
%   Font setting to bold face.
% \end{docCommands}
% \end{dispExample}
% \end{docTcbKey}


% \begin{docTcbKey}[][doc new={2020-04-22}]{doc parameter}{=\meta{parameters}}{no default, initially empty}
%   Sets the \meta{parameters} of the entry to document, i.e. the \meta{parameters} of the
%   command, environment, key, etc. For \refEnv{docCommand}, \refEnv{docEnvironment}, etc.
%   the \meta{parameters} is set by a mandatory option, but can also be set
%   by \refKey{/tcb/doc parameter}.
% \begin{dispExample}
% \begin{docCommands}[
%     doc no index,  %  no index entries for this example
%     doc name      = textbf,
%     doc parameter = \marg{text},
%   ] {}
%   Sets \meta{text} in bold face.
% \end{docCommands}
% \end{dispExample}
% \end{docTcbKey}



% \begin{docTcbKey}[][doc new={2020-04-22}]{doc keypath}{=\meta{key path}}{no default, initially empty}
%   Sets the \meta{key path} of the key to document. For \refEnv{docKey}
%   and \refEnv{docKey*} the \meta{key path}  is set by a specialized option,
%   but can also be set by \refKey{/tcb/doc keypath}.
% \begin{dispExample}
% \begin{docKeys}[
%     doc no index,  %  no index entries for this example
%     doc keypath     = tikz,
%     doc name        = fill,
%     doc parameter   = \colOpt{=\meta{color}},
%     doc description = default is scope's color setting,
%   ] {}
%   This option causes the path to be filled.
% \end{docKeys}
% \end{dispExample}
% \end{docTcbKey}

% \clearpage

% \begin{docTcbKey}{doc description}{=\meta{description}}{no default, initially empty}
%   Sets a (short!) additional \meta{description} for
%   \refEnv{docCommand}, \refEnv{docEnvironment}, or \refEnv{docPathOperation}.
%   Such a description is
%   mandatory for \refEnv{docKey}.
% \begin{dispExample}
% \begin{docCommand*}[doc description=my description]{myCommandF}{\marg{argument}}
%   This is the documentation of \refCom{myCommandF} which takes one \meta{argument}.
%   \refCom{myCommandF} does some funny things with its \meta{argument}.
% \end{docCommand*}
% \end{dispExample}
% \begin{marker}
% Note that the description \meta{text} may overlap with the text on the left
% hand side if too long. Linebreaks can be used inside the \meta{text}.
% \end{marker}
% \end{docTcbKey}


% \begin{docTcbKey}[][doc new={2019-09-18}]{doc label}{=\meta{text}}{no default, initially unset}
%   If used inside the option list of \refEnv{docCommand}, \refEnv{docEnvironment},
%   \refEnv{docKey}, etc, then \meta{text} is used
%   for labeling instead of the name of the definition.
% \begin{dispExample}
% \begin{docPathOperation*}[doc label=pathline]{-{}-}{\meta{coordinate or cycle}}
%   This is the documentation of \refPathOperation{pathline}.
% \end{docPathOperation*}
% \end{dispExample}
% \end{docTcbKey}

% \begin{docTcbKey}[][doc new={2020-01-07}]{doc index}{=\meta{text}}{no default, initially unset}
%   If used inside the option list of \refEnv{docCommand}, \refEnv{docEnvironment},
%   \refEnv{docKey}, etc, then \meta{text} is used
%   for the index instead of the name of the definition.
% \begin{dispExample}
% \begin{docPathOperation}[doc index=foo path (horizontal then vertical),
%     doc label=pathline2]{-\textbar}{\meta{coordinate or cycle}}
%   This is the documentation of \refPathOperation{pathline2}.
% \end{docPathOperation}
% \end{dispExample}
% \end{docTcbKey}


% \begin{docTcbKey}[][doc new={2020-04-23}]{doc sort index}{=\meta{text}}{no default, initially unset}
%   If used inside the option list of \refEnv{docCommand}, \refEnv{docEnvironment},
%   \refEnv{docKey}, etc, then \meta{text} is used
%   for as sort key for the index instead of the name of the definition.
% \begin{dispListing}
% \begin{docCommands}[
%     doc name        = l_tcobox_example_tl,
%     doc sort index  = example_tl,  % sorted unter e like example
%   ]{}
% \end{docCommands}
% \end{dispListing}
% \end{docTcbKey}

% \clearpage

% \begin{docTcbKey}{doc into index}{\colOpt{=true\textbar false}}{default |true|, initially |true|}
%   If set to |false|, no index entries are written for the main documentation
%   environments. The same effect is achieved by using e.\,g.\ \refEnv{docCommand*}
%   instead of \refEnv{docCommand}.
% \end{docTcbKey}


% \begin{docTcbKey}[][doc new={2020-04-22}]{doc no index}{}{style, initially unset}
%   If set, no index entries are written for the main documentation
%   environments. This is a shortcut for using \refKey{/tcb/doc into index}|=false|.
% \end{docTcbKey}



% \begin{docTcbKey}[][doc new=2014-09-19]{doc marginnote}{=\meta{options}}{no default, initially empty}
%   Sets style \meta{options} for the displayed box of the \refCom{tcbdocmarginnote} command.
% \begin{dispExample}
% \tcbset{doc marginnote={colframe=blue!50!white,colback=blue!5!white}}%
% This is some text\tcbdocmarginnote{Note A}
% which is commented by a note inside the margin.
% \end{dispExample}
% \end{docTcbKey}

% \begin{docTcbKey}[][doc new=2014-09-19]{doc new}{=\meta{date}}{style, no default}
%   Adds a a marginnote with a \enquote{New: \meta{date}} message at the beginning of
%   the upper box part. The intended use is inside the option list of
%   \refEnv{docCommand}, \refEnv{docEnvironment}, etc.
%   \makeatletter\renewcommand*{\tcbdocnew}[1]{\kvtcb@text@new: #1}\makeatother%
% \begin{dispExample}
% \begin{docCommand}[doc new=2000-01-01]{foosomething}{\marg{text}}
% Some command for something.
% \end{docCommand}
% \end{dispExample}
% \end{docTcbKey}


% \begin{docTcbKey}[][doc new=2014-09-19]{doc updated}{=\meta{date}}{style, no default}
%   Adds a marginnote with a \enquote{Updated: \meta{date}} message at the beginning of
%   the upper box part. See \refKey{/tcb/doc new}.
% \end{docTcbKey}


% \begin{docTcbKey}[][doc new=2014-09-19]{doc new and updated}{=\marg{new date}\marg{update date}}{style, no default}
%   Adds a marginnote with \enquote{New: \meta{new date}} and \enquote{Updated: \meta{update date}} messages at the beginning of
%   the upper box part. See \refKey{/tcb/doc new}.
% \end{docTcbKey}



% \clearpage
% %-------------------------------------------------------------------------------
% \subsection{Entry Customization Option Keys}


% \begin{docTcbKey}{doc left}{=\meta{length}}{no default, initially |2em|}
%   Sets the left hand offset of the documentation texts from
%   \refEnv{docCommand}, \refEnv{docEnvironment}, \refEnv{docKey}, etc, to \meta{length}.
% \begin{dispExample}
% \begin{docCommand*}[doc left=2cm,doc left indent=-2cm]{myCommandA}{\marg{argument}}
%   This is the documentation of \refCom{myCommandA} which takes one \meta{argument}.
%   \refCom{myCommandA} does some funny things with its \meta{argument}.
% \end{docCommand*}
% \end{dispExample}
% \end{docTcbKey}

% \begin{docTcbKey}{doc right}{=\meta{length}}{no default, initially |0em|}
%   Sets the right hand offset of the documentation texts from
%   \refEnv{docCommand}, \refEnv{docEnvironment}, \refEnv{docKey}, etc, to \meta{length}.
% \begin{dispExample}
% \begin{docCommand*}[doc right=2cm]{myCommandB}{\marg{argument}}
%   This is the documentation of \refCom{myCommandB} which takes one \meta{argument}.
%   \refCom{myCommandB} does some funny things with its \meta{argument}.
% \end{docCommand*}
% \end{dispExample}
% \end{docTcbKey}

% \begin{docTcbKey}{doc left indent}{=\meta{length}}{no default, initially |-2em|}
%   Sets the left hand indent of documentation heads from
%   \refEnv{docCommand}, \refEnv{docEnvironment}, \refEnv{docKey}, etc, to \meta{length}.
% \begin{dispExample}
% \begin{docCommand*}[doc left indent=2cm]{myCommandC}{\marg{argument}}
%   This is the documentation of \refCom{myCommandC} which takes one \meta{argument}.
%   \refCom{myCommandC} does some funny things with its \meta{argument}.
% \end{docCommand*}
% \end{dispExample}
% \end{docTcbKey}


% \begin{docTcbKey}{doc right indent}{=\meta{length}}{no default, initially |0pt|}
%   Sets the right hand indent of documentation heads from
%   \refEnv{docCommand}, \refEnv{docEnvironment}, \refEnv{docKey}, etc, to \meta{length}.
% \begin{dispExample}
% \begin{docCommand*}[doc right indent=-10mm,doc right=10mm,
%     doc description=test value]{myCommandD}{\marg{argument}}
%   This is the documentation of \refCom{myCommandD} which takes one \meta{argument}.
%   \refCom{myCommandD} does some funny things with its \meta{argument}.
% \end{docCommand*}
% \end{dispExample}
% \end{docTcbKey}

% \clearpage
% The head lines of the main documentation environments \refEnv{docCommand},
% \refEnv{docEnvironment}, \refEnv{docKey}, etc, are |tcolorbox|es inside a
% \refEnv{tcbraster}.
% Options to the surrounding |tcbraster|s and the embedded
% |tcolorbox|es can be given using the following keys.


% \begin{docTcbKeys}[
%   doc name        = doc raster command,
%   doc parameter   = {=\meta{options}},
%   doc description = {no default, initially empty},
%   doc new         = 2020-04-24,
% ]{}
%   Sets \meta{options} for the surrounding \refEnv{tcbraster} of\\
%   \refEnv{docCommand}, \refEnv{docCommand*}, and \refEnv{docCommands}.

% \begin{dispExample}
% \tcbset{doc raster command={raster before skip=7mm,raster after skip=0mm}}

% The is an example text.

% \begin{docCommand*}{myCommandI}{\marg{argument}}
%   This is the documentation of \refCom{myCommandI} which takes one \meta{argument}.
%   \refCom{myCommandI} does some funny things with its \meta{argument}.
% \end{docCommand*}
% \end{dispExample}

% \end{docTcbKeys}


% \begin{docTcbKeys}[
%   doc name        = doc raster environment,
%   doc parameter   = {=\meta{options}},
%   doc description = {no default, initially empty},
%   doc new         = 2020-04-24,
% ]{}
%   Sets \meta{options} for the surrounding \refEnv{tcbraster} of\\
%   \refEnv{docEnvironment}, \refEnv{docEnvironment*}, and \refEnv{docEnvironments}.
% \end{docTcbKeys}


% \begin{docTcbKeys}[
%   doc name        = doc raster key,
%   doc parameter   = {=\meta{options}},
%   doc description = {no default, initially empty},
%   doc new         = 2020-04-24,
% ]{}
%   Sets \meta{options} for the surrounding \refEnv{tcbraster} of\\
%   \refEnv{docKey}, \refEnv{docKey*}, and \refEnv{docKeys}.
% \end{docTcbKeys}


% \begin{docTcbKeys}[
%   doc name        = doc raster path,
%   doc parameter   = {=\meta{options}},
%   doc description = {no default, initially empty},
%   doc new         = 2020-04-24,
% ]{}
%   Sets \meta{options} for the surrounding \refEnv{tcbraster} of\\
%   \refEnv{docPathOperation}, \refEnv{docPathOperation*}, and \refEnv{docPathOperations}.
% \end{docTcbKeys}


% \begin{docTcbKeys}[
%   doc name        = doc raster,
%   doc parameter   = {=\meta{options}},
%   doc description = {no default, initially empty},
%   doc new         = 2020-04-24,
% ]{}
%   Shortcut for setting the same \meta{options} for
%   \refKey{/tcb/doc raster command}, \refKey{/tcb/doc raster environment},
%   \refKey{/tcb/doc raster key}, and \refKey{/tcb/doc raster path}.
% \end{docTcbKeys}


% \begin{docTcbKey}{doc head command}{=\meta{options}}{no default, initially empty}
%   Sets \meta{options} for the head line of
%   \refEnv{docCommand}, \refEnv{docCommand*}, and \refEnv{docCommands}.
% \begin{dispExample}
% \tcbset{doc head command={interior style={fill,left color=red!20!white,
%   right color=blue!20!white}}}

% \begin{docCommand*}{myCommandE}{\marg{argument}}
%   This is the documentation of \refCom{myCommandE} which takes one \meta{argument}.
%   \refCom{myCommandE} does some funny things with its \meta{argument}.
% \end{docCommand*}
% \end{dispExample}
% \end{docTcbKey}


% \clearpage

% \begin{docTcbKey}{doc head environment}{=\meta{options}}{no default, initially empty}
%   Sets \meta{options} for the head line of
%   \refEnv{docEnvironment}, \refEnv{docEnvironment*}, and \refEnv{docEnvironments}.
% \begin{dispExample}
% \tcbset{doc head environment={beamer,boxsep=2pt,arc=2pt,colback=green!20!white}}

% \begin{docEnvironment*}{myEnvironment}{\marg{argument}}
%   This is the documentation of \refEnv{myEnvironment} which
%   takes one \meta{argument}.
% \end{docEnvironment*}
% \end{dispExample}
% \end{docTcbKey}

% \begin{docTcbKey}{doc head key}{=\meta{options}}{no default, initially empty}
%   Sets \meta{options} for the head line of
%   \refEnv{docKey}, \refEnv{docKey*}, and \refEnv{docKeys}.
% \begin{dispExample}
% \tcbset{doc head key={boxsep=4pt,arc=4pt,boxrule=0.6pt,
%   frame style=fill,interior style=fill,colframe=green!50!black}}

% \begin{docKey}{/foo/myKey}{}{no value}
%   This is the documentation of \refKey{/foo/myKey}.
% \end{docKey}
% \end{dispExample}
% \end{docTcbKey}


% \begin{docTcbKey}[][doc new=2019-09-18]{doc head path}{=\meta{options}}{no default, initially empty}
%   Sets \meta{options} for the head line of
%   \refEnv{docPathOperation}, \refEnv{docPathOperation*}, and \refEnv{docPathOperations}.
% \begin{dispExample}
% \tcbset{doc head command={interior style={fill,left color=red!7!white,
%   right color=blue!7!white}}}

% \begin{docPathOperation*}{-{}-}{\meta{coordinate or cycle}}
%   This is the documentation of \refPathOperation{-{}-}.
% \end{docPathOperation*}
% \end{dispExample}
% \end{docTcbKey}


% \begin{docTcbKey}[][doc updated=2019-09-18]{doc head}{=\meta{options}}{no default, initially empty}
%   Shortcut for setting the same \meta{options} for
%   \refKey{/tcb/doc head command}, \refKey{/tcb/doc head environment},
%   \refKey{/tcb/doc head key}, and \refKey{/tcb/doc head path}.
% \end{docTcbKey}


% \clearpage

% The description texts of the main documentation environments \refEnv{docCommand},
% \refEnv{docEnvironment}, \refEnv{docKey}, etc, are set in a compact form without
% indention and |parskip=0pt|. This settings can overruled by using the following
% keys to insert code before (or after) the description texts.

% \begin{docTcbKey}[][doc new=2015-10-09]{before doc body command}{=\meta{code}}{no default, initially empty}
%   Executes \meta{code} before the description texts
%   of \refEnv{docCommand} and \refEnv{docCommand*}.
% \begin{dispExample}
% \tcbset{before doc body command={%
%     \setlength{\parindent}{2.5em}%
%     \setlength{\parskip}{1ex plus 0.75ex minus 0.25ex}%
% }}

% \begin{docCommand*}{myCommandG}{\marg{argument}}
%   This is the documentation of \refCom{myCommandG} which takes one \meta{argument}.
%   \refCom{myCommandG} does some funny things with its \meta{argument}.
% \end{docCommand*}
% \end{dispExample}
% \end{docTcbKey}


% \begin{docTcbKey}[][doc new=2015-10-09]{after doc body command}{=\meta{code}}{no default, initially empty}
%   Executes \meta{code} after the description texts
%   of \refEnv{docCommand} and \refEnv{docCommand*}.
% \begin{dispExample}
% \tcbset{after doc body command={%
%     \hfill\nolinebreak[1]\hspace*{\fill}\textcolor{red}{$\diamondsuit$}%
% }}

% \begin{docCommand*}{myCommandH}{\marg{argument}}
%   This is the documentation of \refCom{myCommandH} which takes one \meta{argument}.
%   \refCom{myCommandH} does some funny things with its \meta{argument}.
% \end{docCommand*}
% \end{dispExample}
% \end{docTcbKey}


% \begin{docTcbKey}[][doc new=2015-10-09]{before doc body environment}{=\meta{code}}{no default, initially empty}
%   Executes \meta{code} before the description texts
%   of \refEnv{docEnvironment} and \refEnv{docEnvironment*}.
% \end{docTcbKey}

% \begin{docTcbKey}[][doc new=2015-10-09]{after doc body environment}{=\meta{code}}{no default, initially empty}
%   Executes \meta{code} after the description texts
%   of \refEnv{docEnvironment} and \refEnv{docEnvironment*}.
% \end{docTcbKey}


% \begin{docTcbKey}[][doc new=2015-10-09]{before doc body key}{=\meta{code}}{no default, initially empty}
%   Executes \meta{code} before the description texts
%   of \refEnv{docKey} and \refEnv{docKey*}.
% \end{docTcbKey}

% \begin{docTcbKey}[][doc new=2015-10-09]{after doc body key}{=\meta{code}}{no default, initially empty}
%   Executes \meta{code} after the description texts
%   of \refEnv{docKey} and \refEnv{docKey*}.
% \end{docTcbKey}

% \clearpage

% \begin{docTcbKey}[][doc new=2019-09-18]{before doc body path}{=\meta{code}}{no default, initially empty}
%   Executes \meta{code} before the description texts
%   of \refEnv{docPathOperation} and \refEnv{docPathOperation*}.
% \end{docTcbKey}

% \begin{docTcbKey}[][doc new=2019-09-18]{after doc body path}{=\meta{code}}{no default, initially empty}
%   Executes \meta{code} after the description texts
%   of \refEnv{docPathOperation} and \refEnv{docPathOperation*}.
% \end{docTcbKey}


% \begin{docTcbKey}[][doc new and updated={2015-10-09}{2019-09-18}]{before doc body}{=\meta{options}}{no default, initially empty}
%   Shortcut for setting the same \meta{options} for
%   \refKey{/tcb/before doc body command}, \refKey{/tcb/before doc body environment},
%   \refKey{/tcb/before doc body key}, and \refKey{/tcb/before doc body path}.
% \end{docTcbKey}

% \begin{docTcbKey}[][doc new and updated={2015-10-09}{2019-09-18}]{after doc body}{=\meta{options}}{no default, initially empty}
%   Shortcut for setting the same \meta{options} for
%   \refKey{/tcb/after doc body command}, \refKey{/tcb/after doc body environment},
%   \refKey{/tcb/after doc body key}, and \refKey{/tcb/after doc body path}.
% \end{docTcbKey}







% \clearpage
% \subsection{General Customization Option Keys}

% \begin{docTcbKey}[][doc updated=2015-03-16]{docexample}{}{style, no value}
%   Sets the style for \refEnv{dispExample} and \refEnv{dispListing}
%   with the colors |ExampleBack| and |ExampleFrame|.
%   To change the appearance of the examples, this style can be
%   redefined.
% \begin{dispListing}
% % Predefined style:
% \tcbset{
%   docexample/.style={colframe=ExampleFrame,colback=ExampleBack,
%     before skip=\medskipamount,after skip=\medskipamount,
%     fontlower=\footnotesize}
% }
% \end{dispListing}
% \end{docTcbKey}

% \begin{docTcbKey}{documentation listing options}{=\meta{key list}}{no default,\\\hspace*{\fill} initially |style=tcbdocumentation|}
%   Sets the options from the package |listings| \cite{hoffmann:listings}.
%   They are used inside \refEnv{dispExample} and \refEnv{dispListing} to typeset
%   the listings. Note that this is not identical to the key
%   \refKey{/tcb/listing options} which is used for \enquote{normal} listings.\\
%   Used for \refKey{/tcb/listing engine}|=listings| only.
% \end{docTcbKey}

% \begin{docTcbKey}{documentation listing style}{=\meta{listing style}}{no default, initially |tcbdocumentation|}
%   Abbreviation for |documentation listing options={style=...}|.
%   This key sets a \meta{style}
%   for the |listings| package, see \cite{hoffmann:listings}.
%   Note that this is not identical to the key
%   \refKey{/tcb/listing style} which is used for \enquote{normal} listings.\\
%   Used for \refKey{/tcb/listing engine}|=listings| only.
% \end{docTcbKey}

% \begin{docTcbKey}{documentation minted options}{=\meta{key list}}{no default,\\\hspace*{\fill} initially |tabsize=2,fontsize=\textbackslash small|}
%   Sets the options from the package |minted| \cite{poore:minted}
%   which are used during typesetting of the listing, if used.
%   Note that this is not identical to the key
%   \refKey{/tcb/minted options} which is used for \enquote{normal} listings.\\
%   Used for \refKey{/tcb/listing engine}|=minted| only.
% \end{docTcbKey}

% \begin{docTcbKey}{documentation minted style}{=\meta{key list}}{no default, initially unset}
%   Sets a \meta{style} known to |Pygments| \cite{pygments:web} for
%   the package |minted| \cite{poore:minted}, if used.
%   Note that this is not identical to the key
%   \refKey{/tcb/minted style} which is used for \enquote{normal} listings.\\
%   Used for \refKey{/tcb/listing engine}|=minted| only.
% \end{docTcbKey}

% \begin{docTcbKey}[][doc new=2017-04-24]{documentation minted language}{=\meta{programming language}}{no default, initially |latex|}
%   Sets a \meta{programming language} known to |Pygments| \cite{pygments:web}
%   for the package |minted| \cite{poore:minted}, if used.
%   Note that this is not identical to the key
%   \refKey{/tcb/minted language} which is used for \enquote{normal} listings.\\
%   Used for \refKey{/tcb/listing engine}|=minted| only.
% \end{docTcbKey}


% \begin{marker}
% The following two keys are deprecated and without function (v3.50 and above).
% Use \refKey{/tcb/before} and \refKey{/tcb/after} with appropriate values
% instead. Also see \refKey{/tcb/docexample}.
% \end{marker}

% \begin{docTcbKey}[][doc updated=2015-03-16]{before example}{=\meta{macros}}{no default, initially empty}
% \smallskip\begin{deprecated}
%   Sets the \meta{macros} which are executed before \refEnv{dispExample} and \refEnv{dispListing}
%   additional to \refKey{/tcb/before}.
% \end{deprecated}
% \end{docTcbKey}

% \enlargethispage*{1cm}

% \begin{docTcbKey}{after example}{=\meta{macros}}{no default, initially empty}
% \smallskip\begin{deprecated}
%   Sets the \meta{macros} which are executed after \refEnv{dispExample} and \refEnv{dispListing}
%   additional to \refKey{/tcb/after}.
% \end{deprecated}
% \end{docTcbKey}

% \clearpage
% \begin{docTcbKey}[][doc new=2017-04-25]{keywords bold}{\colOpt{=true\textbar false}}{default |true|, initially |true|}
%   Keyword used in \refEnv{docEnvironment}, \refEnv{docCommand}, etc. are printed
%   boldface (or not). Since the typewriter font is used, the effect may be
%   invisible with Computer Modern fonts or similar which do not
%   have a bold variant. Note that references to keywords are not printed boldface at all.
% \begin{dispExample*}{sidebyside}
% \LARGE
% \docAuxCommand{fooaux}, \refCom{tcbset}

% \tcbset{keywords bold=false}
% \docAuxCommand{fooaux}, \refCom{tcbset}
% \end{dispExample*}
% \end{docTcbKey}



% \begin{docTcbKey}[][doc new=2015-01-09]{index command}{=\meta{macro}}{no default, initially \cs{index}}
%   Replaces the internally used \cs{index} macro by the given \meta{macro}.
%   The \meta{macro} has to take one mandatory argument like \cs{index}.
%   This option is mutually exclusive with \refKey{/tcb/index command name}.
% \begin{dispListing}
% \tcbset{index command=\myindexcommand}
% \end{dispListing}
% \end{docTcbKey}


% \begin{docTcbKey}[][doc new=2015-01-09]{index command name}{=\meta{name}}{no default, initially unset}
%   Replaces the internally used \cs{index} macro by
%   \mbox{\cs{index}\texttt{[\meta{name}]}}, i.e.\ 
%   \mbox{\cs{index}\texttt{\textbraceleft\ldots\textbraceright}} is replaced by
%   \mbox{\cs{index}\texttt{[\meta{name}]\textbraceleft\ldots\textbraceright}}.
%   This option is intended to be used with |imakeidx| and is
%   mutually exclusive with \refKey{/tcb/index command}.
% \begin{dispListing}
% \tcbset{index command name=mydoc}
% \end{dispListing}
% \end{docTcbKey}



% \begin{docTcbKey}{index format}{=\meta{format}}{no default, initially |pgf|}
%   Determines the basic \meta{format} of the generated index.
%   Feasible values are:
%   \begin{itemize}
%   \item\docValue{pgfsection}: The index is formatted like in the |pgf| documentation (as a section).
%   \item\docValue{pgfchapter}: The index is formatted like in the |pgf| documentation (as a chapter).
%   \item\docValue{pgf}: Alias for |pgfsection|.
%   \item\docValue{doc}: The index is assumed to be formatted by |doc| or |ltxdoc|. The usage of |makeindex|
%     with |-s gind.ist| is assumed. The package |hypdoc| has to be loaded
%     \emph{before} |tcolorbox|. Only a limited set of customizations will
%     work! This option cannot be unset when used!
%   \item\docValue{off}: The index is not formatted by |tcolorbox|. Use this, if
%     the index is formatted by other package like |imakeidx|.
%   \end{itemize}
% \end{docTcbKey}


% \begin{docTcbKey}{index actual}{=\meta{character}}{no default, initially |@|}
%   Sets the character for \enquote{actual} in automatic indexing.
% \end{docTcbKey}

% \begin{docTcbKey}{index quote}{=\meta{character}}{no default, initially |"|}
%   Sets the character for \enquote{quote} in automatic indexing.
% \end{docTcbKey}

% \begin{docTcbKey}{index level}{=\meta{character}}{no default, initially |!|}
%   Sets the character for \enquote{level} in automatic indexing.
% \end{docTcbKey}

% \begin{docTcbKey}{index default settings}{}{style, no value}
%   Sets the |makeindex| default values for
%   \refKey{/tcb/index actual},
%   \refKey{/tcb/index quote}, and
%   \refKey{/tcb/index level}.
% \end{docTcbKey}

% \enlargethispage*{1cm}

% \begin{docTcbKey}{index german settings}{}{style, no value}
%   Sets the |makeindex| values recommended for German language texts.
%   This is identical to setting the following:
% \begin{dispListing}
% \tcbset{index actual={=},index quote={!},index level={>}}
% \end{dispListing}
% \end{docTcbKey}

% \clearpage

% \begin{docTcbKey}{index annotate}{\colOpt{=true\textbar false}}{default |true|, initially |true|}
%   If set to |true|, the index entries are annotated with short descriptions
%   given by \refKey{/tcb/doclang/environment}, \refKey{/tcb/doclang/key},
%   and others.
% \end{docTcbKey}

% \begin{docTcbKey}{index colorize}{\colOpt{=true\textbar false}}{default |true|, initially |false|}
%   If set to |true|, the index entries colorized according to the color
%   settings given by \refKey{/tcb/color environment}, \refKey{/tcb/color key},
%   and others.
% \end{docTcbKey}


% \begin{docTcbKey}{color command}{=\meta{color}}{no default, initially |Definition|}
%   Sets the highlight color used by macro definitions.
% \end{docTcbKey}

% \begin{docTcbKey}{color environment}{=\meta{color}}{no default, initially |Definition|}
%   Sets the highlight color used by environment definitions.
% \end{docTcbKey}

% \begin{docTcbKey}{color key}{=\meta{color}}{no default, initially |Definition|}
%   Sets the highlight color used by key definitions.
% \end{docTcbKey}

% \begin{docTcbKey}[][doc new={2019-09-18}]{color path}{=\meta{color}}{no default, initially |Definition|}
%   Sets the highlight color used by \tikzname\ path operation definitions.
% \end{docTcbKey}

% \begin{docTcbKey}{color value}{=\meta{color}}{no default, initially |Definition|}
%   Sets the highlight color used by value definitions.
% \end{docTcbKey}

% \begin{docTcbKey}[][doc new={2015-01-08}]{color counter}{=\meta{color}}{no default, initially |Definition|}
%   Sets the highlight color used by counter definitions.
% \end{docTcbKey}

% \begin{docTcbKey}[][doc new={2015-01-08}]{color length}{=\meta{color}}{no default, initially |Definition|}
%   Sets the highlight color used by length definitions.
% \end{docTcbKey}

% \begin{docTcbKey}{color color}{=\meta{color}}{no default, initially |Definition|}
%   Sets the highlight color used by color definitions.
% \end{docTcbKey}

% \begin{docTcbKey}[][doc updated={2019-09-18}]{color definition}{=\meta{color}}{no default, initially |Definition|}
%   Sets the highlight color for \refKey{/tcb/color command}, \refKey{/tcb/color environment},
%   \refKey{/tcb/color key}, \refKey{/tcb/color path}, \refKey{/tcb/color value}, \refKey{/tcb/color counter},
%   \refKey{/tcb/color length}, and \refKey{/tcb/color color}.
% \end{docTcbKey}

% \begin{docTcbKey}{color option}{=\meta{color}}{no default, initially |Option|}
%   Sets the color used for optional arguments.
% \end{docTcbKey}

% \begin{docTcbKey}{color fade}{=\meta{color}}{no default, initially |Fade|}
%   Sets the color used for faded text like \colFade{\textbackslash path}
%   in \refEnv{docPathOperation}.
% \end{docTcbKey}


% \begin{docTcbKey}{color hyperlink}{=\meta{color}}{no default, initially |Hyperlink|}
%   Sets the color for all hyper-links, i.\,e. all internal and external links.
% \end{docTcbKey}


% \clearpage
% %-------------------------------------------------------------------------------
% \subsection{Language Option Keys}

% The following keys are provided for language specific settings.
% The English language is predefined.

% \begin{docTcbKey}{english language}{}{style, no value}
%   Sets all language specific settings to English.
% \end{docTcbKey}

% \begin{langTcbKey}{color}{=\meta{text}}{no default, initially |color|}
%   Text used in the index for colors.
% \end{langTcbKey}

% \begin{langTcbKey}{colors}{=\meta{text}}{no default, initially |Colors|}
%   Heading text in the index for colors.
% \end{langTcbKey}

% \begin{langTcbKey}[][doc new={2015-01-08}]{counter}{=\meta{text}}{no default, initially |counter|}
%   Text used in the index for counters.
% \end{langTcbKey}

% \begin{langTcbKey}[][doc new={2015-01-08}]{counters}{=\meta{text}}{no default, initially |Counters|}
%   Heading text in the index for counters.
% \end{langTcbKey}

% \begin{langTcbKey}{environment}{=\meta{text}}{no default, initially |environment|}
%   Text used in the index for environments.
% \end{langTcbKey}

% \begin{langTcbKey}{environments}{=\meta{text}}{no default, initially |Environments|}
%   Heading text in the index for environments.
% \end{langTcbKey}

% \begin{langTcbKey}{environment content}{=\meta{text}}{no default, initially |environment content|}
%   Text used in \refEnv{docEnvironment}.
% \end{langTcbKey}

% \begin{langTcbKey}{index}{=\meta{text}}{no default, initially |Index|}
%   Heading text for the index.
% \end{langTcbKey}

% \begin{langTcbKey}{key}{=\meta{text}}{no default, initially |key|}
%   Text used in the index for keys.
% \end{langTcbKey}

% \begin{langTcbKey}{keys}{=\meta{text}}{no default, initially |Keys|}
%   Heading text used in the index for keys.
% \end{langTcbKey}

% \begin{langTcbKey}[][doc new={2015-01-08}]{length}{=\meta{text}}{no default, initially |length|}
%   Text used in the index for lengths.
% \end{langTcbKey}

% \begin{langTcbKey}[][doc new={2015-01-08}]{lengths}{=\meta{text}}{no default, initially |Lengths|}
%   Heading text in the index for lengths.
% \end{langTcbKey}

% \begin{langTcbKey}[][doc new={2014-09-19}]{new}{=\meta{text}}{no default, initially |New|}
%   Announcement text for new content.
% \end{langTcbKey}

% \begin{langTcbKey}[][doc new={2019-09-18}]{path}{=\meta{text}}{no default, initially |path operation|}
%   Text used in the index for path operations.
% \end{langTcbKey}

% \begin{langTcbKey}[][doc new={2019-09-18}]{paths}{=\meta{text}}{no default, initially |Path operations|}
%   Heading text in the index for path operations.
% \end{langTcbKey}

% \begin{langTcbKey}{pageshort}{=\meta{text}}{no default, initially |P.|}
%   Short text for page references.
% \end{langTcbKey}

% \begin{langTcbKey}[][doc new={2014-09-19}]{updated}{=\meta{text}}{no default, initially |Updated|}
%   Announcement text for updated content.
% \end{langTcbKey}

% \begin{langTcbKey}{value}{=\meta{text}}{no default, initially |value|}
%   Text used in the index for values.
% \end{langTcbKey}

% \begin{langTcbKey}{values}{=\meta{text}}{no default, initially |Values|}
%   Heading text in the index for values.
% \end{langTcbKey}






% \clearpage



% \subsection{Predefined Colors of the Library}\tcbdocmarginnote{\tcbdocupdated{2019-09-18}}
% The following colors are predefined. They are used as default colors
% in some library commands.

% \def\dispColor#1{\docColor{#1}~\tikz[baseline=1mm]\path[fill=#1,draw] (0,0) rectangle (0.4,0.4);~}

% \dispColor{Option},
% \dispColor{Definition},
% \dispColor{ExampleFrame},
% \dispColor{ExampleBack},
% \dispColor{Hyperlink},
% \dispColor{Fade}.


% v1 2023 0314
% \appendix
% \input{tcolorbox.doc.picturecredits}
% \input{tcolorbox.doc.references}
% \input{tcolorbox.doc.index}

% \mshowc{chapter}
\mshowc{section}
\mshowc{subsection}
\mshowc{subsubsection}
% \mshowc{page}

\end{document}
% cd /Volumes/RamDisk/ &&  xelatex --output-directory=/Volumes/RamDisk/ -synctex=1 -shell-escape /Users/virhuiai/hlProjects/Latex-Typesetting-Hub/宏包文档翻译/tcolorbox/tcolorbox.tex

(\\refCom|\\refEnv|\\refKey|\\refSkin)(\{)

$1Le$2


\\cite\{[^\}]+\}