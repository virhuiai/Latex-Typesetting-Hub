\section{Library \mylib{external}}\label{sec:external}%
\tcbset{external/prefix=external/external_}%
The library is loaded by a package option or inside the preamble by:

该库可以通过包选项或在导言区中加载,如下所示:
\begin{dispListing}
\tcbuselibrary{external}
\end{dispListing}

The purpose of this library is to support externalization of document
snippets like graphics or boxes which can be compiled stand-alone.
These snippets are written to external files, compiled and the resulting
pdf files are included to the main document as images.
The whole procedure saves compilation time, if such a snippet is costly to
compile but needs to compile just once or very seldom.

该库的目的是支持文档片段的外部化,例如图形或框,这些片段可以单独编译。这些片段会被写入外部文件,编译,然后将生成的 PDF 文件作为图像包含到主文档中。如果这样的片段编译成本很高,但只需要编译一次或很少编译,整个过程可以节省编译时间。

There are very good alternatives to this library. One should consider
the |standalone| package or the \tikzname\ externalization library instead.
The \mylib{external} library is something in between and can be seen as
poor man variant of the \tikzname\ externalization library.

虽然 \mylib{external} 库是一种不错的选择,但还有其他非常好的替代方案。例如可以考虑使用 |standalone| 宏包或者 \tikzname\ 外部化库。与 \tikzname\ 外部化相比,\mylib{external} 库介于二者之间,可以看作是一种较为简单的外部化方案。


The main differences between \tikzname\ externalization and \mylib{external} are:

\tikzname\ 外部化和 \mylib{external} 之间的主要区别包括:
\begin{itemize}
\item\tikzname\ |external| compiles the whole original document in a sophisticated
  way while \mylib{external} uses only the preamble or a part of the preamble
  of the original document.

\tikzname\ 的 |external| 在复杂的文档中会以一种复杂的方式编译整个文档,而 \mylib{external} 仅使用原始文档的导言区或导言区的一部分进行编译。
\item\tikzname\ |external| can automatically externalize all |tikzpicture|
  environments while \mylib{external} externalizes marked snippets only.

\tikzname\ 的 |external| 可以自动外部化所有的 |tikzpicture| 环境,而 \mylib{external} 仅外部化标记的片段。
  \item Code snippets to be externalized by \mylib{external} are not restricted to
  |tikzpicture| environments. But these snippets have to be stand-alone without
  dependencies to the rest of the document.

由 \mylib{external} 外部化的代码片段不限于 |tikzpicture| 环境。但是这些片段必须是独立的,不依赖于文档的其他部分。
\end{itemize}
Why should somebody use \mylib{external} instead of the more powerful \tikzname\ |external|?
One reason could be compilation speed, but the main reason for creating the
library at all was that \tikzname\ |external| tends to choke on complicated
documents where the sophisticated mechanism stumbles. Since \mylib{external} does
not use the original document body for compilation, this cannot happen.

为什么有人要使用 \mylib{external} 而不是更强大的 \tikzname\ |external|?一个原因可能是编译速度,但创建该库的主要原因是 \tikzname\ |external| 倾向于在复杂的文档中出现故障,这种复杂机制容易出现问题。由于 \mylib{external} 不使用原始文档体进行编译,因此不会出现这种情况。
\begin{marker}
Source snippets are compiled, if their |md5| checksum has changed.
They are not compiled automatically, if option settings are changed or
anything outside the snippet is changed.
Use \refKeyLe{/tcb/external/force remake} to force compilation in this case
or simply delete the externalized pdf oder md5 files.

如果源片段的 |md5| 校验和已更改,则会编译这些源片段。如果更改了选项设置或片段外部的任何内容,则不会自动编译。在这种情况下,可以使用 \refKeyLe{/tcb/external/force remake} 强制重新编译,或者直接删除外部化的 PDF 或 md5 文件。
\end{marker}

\begin{marker}
To use the externalization options, the compiler has to be called with the
|-shell-escape| permission to authorize potentially dangerous system calls.
Be warned that this is a security risk.

为了使用外部化选项,编译器必须使用 |-shell-escape| 权限调用,以授权可能危险的系统调用。请注意,这可能存在安全风险。
\end{marker}