% \clearpage
\subsection{Preparation of a Document for Externalization\\准备文档以进行外部化}\label{subsec:external_preparation}

The preamble of the main document has to contain the \refComLe{tcbEXTERNALIZE}
command. Without this command, no externalization operation will be
executed.

主文档的导言部分必须包含\refComLe{tcbEXTERNALIZE}命令。如果没有此命令,将不会执行任何外部化操作。
\begin{docCommand}[doc new=2015-03-11]{tcbEXTERNALIZE}{}
It is mandatory for externalization that this command is used once
in the preamble of the main document. Every setting \emph{before}
\refComLe{tcbEXTERNALIZE} will also be used for compiling an external
snippet. Every setting \emph{after} \refComLe{tcbEXTERNALIZE} will be
ignored for compiling an external snippet.
Place this command right before |\begin{document}|, if you are not
absolutely sure about another place.

对于外部化而言,必须在主文档的导言部分使用此命令。在\refComLe{tcbEXTERNALIZE}之前的每个设置也将用于编译外部代码片段。在\refComLe{tcbEXTERNALIZE}之后的每个设置将被忽略以编译外部代码片段。
如果您不确定其他位置,请在|\begin{document}|之前放置此命令。

The main document has to look like the following:

主文档应该像下面这样:
\begin{dispListing}
\documentclass[a4paper]{book}%   for example
\usepackage{...}%                anything
% ...
% Tpyically, all or the very most settings for the document.

\tcbEXTERNALIZE% Typically, just before \begin{document}

% Additional settings which are ABSOLUTELY irrelevant for the
% stand-alone snippets.
%
\begin{document}
  % The document.
  % This also contains the marked snippets for externalization.
\end{document}
\end{dispListing}
\end{docCommand}

During compilation, a \refKeyLe{/tcb/external/runner} file
is dynamically created (several times). This is the actual main file for
compiling an externalized snippet.

在编译过程中,动态创建了一个\refKeyLe{/tcb/external/runner}文件(多次)。这是编译外部代码片段的实际主文件。
\begin{extTcbKey}[][doc new=2015-03-11]{runner}{=\meta{file name}}{no default,
  initially \texttt{\cs{jobname}\detokenize{_run.tex}}}
Sets the \meta{file name} for dynamically created |runner| file.
This is the actual main file for a document snippet.
Typically, the initial setting is not needed to be changed.

设置动态创建的 |runner| 文件的 \meta{file name}。这是一个文档片段的实际主文件。通常,初始设置不需要更改。
\begin{dispListing}
\tcbset{external/runner=myrunner.tex}
\end{dispListing}
\end{extTcbKey}

\begin{extTcbKey}[][doc new=2015-03-11]{prefix}{=\meta{text}}{no default,
  initially \texttt{external/}}
The \meta{text} is prefixed to any \refKeyLe{/tcb/external/name} for an
externalization snippet. The initial setting implies saving all snippets
into an |external/| subdirectory. Depending on the operation system,
the subdirectory may have to be created manually once.

\meta{text} 会添加到任何外部化片段的\refKeyLe{/tcb/external/name}。初始设置意味着将所有片段保存到一个 |external/| 子目录中。根据操作系统,可能需要手动创建子目录一次。
\begin{dispListing}
% Use a 'real' prefix instead of writing into a subdirectory:
\tcbset{external/prefix=ext_}
\end{dispListing}
\end{extTcbKey}


\begin{extTcbKey}[][doc new=2015-03-11]{externalize}{\colOpt{=true\textbar false}}{default |true|,
  initially |true|}
If set to |true|, the marked snippets are compiled if necessary.
If set to |false|, the marked snippets are not compiled but included as text.
\refKeyLe{/tcb/external/externalize} can only be used after \refComLe{tcbEXTERNALIZE}.

如果设置为 |true|,则编译标记的片段(如果有必要)。如果设置为 |false|,则标记的片段不会被编译,而是作为文本包含在内。\refKeyLe{/tcb/external/externalize}只能在\refComLe{tcbEXTERNALIZE}之后使用。
\end{extTcbKey}

\begin{extTcbKey}[][doc new=2015-03-11]{force remake}{\colOpt{=true\textbar false}}{default |true|,
  initially |false|}
If set to |true|, the marked snippets are always compiled.
If set to |true|, the marked snippets are compiled only if necessary.
The necessity is given, if a compiled pdf file is missing or the
|md5| checksum of the source snippet has changed.

如果设置为 |true|,则标记的片段总是会被编译。如果设置为 |false|,则只有在必要时才编译标记的片段。必要性在编译生成的pdf文件丢失或源片段的|md5|校验和发生变化时被认为是必要的。
\end{extTcbKey}

\enlargethispage*{1cm}
\begin{extTcbKey}[][doc new and updated={2015-03-11}{2017-02-24}]{\tcbexclamation}{}{style}
Shortcut for setting \refKeyLe{/tcb/external/force remake} to |true|.

这是设置 \refKeyLe{/tcb/external/force remake} 为 |true| 的快捷方式。
\end{extTcbKey}

\begin{extTcbKey}[][doc new and updated={2015-06-12}{2017-02-24}]{-}{}{style}
Shortcut for setting \refKeyLe{/tcb/external/externalize} to |false|.

这是设置 \refKeyLe{/tcb/external/externalize} 为 |false| 的快捷方式
\end{extTcbKey}
