% \clearpage
\subsection{Customization\\自定义}\label{subsec:external_custom}

\begin{extTcbKey}[][doc new=2015-03-11]{safety}{=\meta{length}}{no default,
  initially |2mm|}
The snippet box is surrounded with a safety border with a thickness of
\meta{length}. This border is automatically trimmed during picture inclusion.
The reason for this mechanism is to catch  box content which
extrudes over the bounding box. For example, shadows of a |tcolorbox| are
painted outside the bounding box and would be lost otherwise.

代码片段的周围用一条厚度为\meta{length}的安全边框包围。此边框在图片被包含时会自动修剪。这种机制的原因是为了捕获超出边界框的盒子内容。例如,一个|tcolorbox|的阴影会绘制在边界框之外,否则就会丢失。
\end{extTcbKey}

\begin{extTcbKey}[][doc new=2015-03-11]{environment}{=\meta{env}}{no default, initially unset}
Surrounds the exported snippet text with an environment \meta{env} without
parameters.
Note that this option is ignored for \refKeyLe{/tcb/externalize listing}.

用没有参数的环境\meta{env}包围导出的代码片段文本。请注意,对于\refKeyLe{/tcb/externalize listing},此选项将被忽略。
\end{extTcbKey}

\begin{extTcbKey}[][doc new=2015-05-05]{environment with percent}{\colOpt{=true\textbar false}}{default |true|,
  initially |true|}
If set to |true|, the |\begin| and |\end| code of \refKeyLe{/tcb/external/environment}
is appended with a percent sign. For verbatim environments, this option
typically has to be se to |false|.

如果设置为|true|,则在\refKeyLe{/tcb/external/environment}的|\begin|和|\end|代码后附加一个百分号。对于抄录环境,通常必须将此选项设置为|false|。
\end{extTcbKey}


\begin{extTcbKey}[][doc new=2015-03-11]{minipage}{\colOpt{=\meta{length}}}{default \texttt{\cs{linewidth}},
  initially unset}
Surrounds the exported snippet text with a minipage. The optional \meta{length}
parameter sets the width of the minipage. Note that the default width is the
current line width of the main document.
See \refEnvLe{tcbexternal} for examples.
Note that this option is ignored for \refKeyLe{/tcb/externalize listing}.

用一个minipage包围导出的代码片段文本。可选的\meta{length}参数设置minipage的宽度。请注意, 默认宽度为主文档的当前行宽度。 有关示例,请参见\refEnvLe{tcbexternal}。请注意,对于\refKeyLe{/tcb/externalize listing},此选项将被忽略。
\end{extTcbKey}


\begin{extTcbKey}[][doc new=2015-03-11]{plain}{}{no value, initially set}
Removes any text which was set to surround the snippet.
This removes the setting of  \refKeyLe{/tcb/external/minipage}, but is
independent of \refKeyLe{/tcb/external/safety}.

删除设置为包围片段的任何文本。这将删除\refKeyLe{/tcb/external/minipage}的设置,但与\refKeyLe{/tcb/external/safety}无关。
\end{extTcbKey}


\begin{extTcbKey}[][doc new=2015-03-11]{compiler}{=\meta{text}}{no default,
initially \texttt{pdflatex}}
Sets the name of the compiler for the snippets. Note that this compiler
has to support the |\pdfmdfivesum| primitive e.g. using the
|pdftexcmds| package. This should work for |xelatex| and |lualatex|.

设置片段的编译器名称。请注意,此编译器必须支持|\pdfmdfivesum|原语,例如使用|pdftexcmds|包。这对于|xelatex|和|lualatex|应该是有效的。
\end{extTcbKey}

\begin{extTcbKey}[][doc new=2015-03-11]{runs}{=\meta{number}}{no default,
  initially |1|}
Sets the number of compiler runs for the snippet.

设置片段的编译次数为 \meta{number}。
\begin{dispExample}
\begin{tcbexternal}[minipage,runs=2]{example_raster}
  \begin{tcbitemize}[raster equal height,
      size=small,colframe=red!50!black,colback=red!10!white]
    \tcbitem One
    \tcbitem \Huge Two
    \tcbitem Three
    \tcbitem Four
  \end{tcbitemize}
\end{tcbexternal}
\end{dispExample}
\end{extTcbKey}

\enlargethispage*{1cm}
\begin{extTcbKey}[][doc new=2015-03-11]{input source on error}{\colOpt{=true\textbar false}}{default |true|,
  initially |true|}
If set to |true|, the source code of the snippet is loaded instead of
the failed pdf picture. Typically, this will lead to an error stop at the
faulty place of the source and such helps detecting the cause.
If the source input compiles without error, the document setup
may be incorrect, see \Fullref{subsec:external_preparation}.
Maybe, the |external/| subdirectory has to be created manually in this case,
see \refKeyLe{/tcb/external/prefix}.\par
If the option is set to |false|, the compilation stops immediately on an error.
The log file of the external snippet has to be consulted for error messages
in this case.

如果设置为 |true|,则会加载片段的源代码而非失败的 PDF 图片。通常,这将导致在源代码的错误位置停止,并有助于检测问题的原因。如果源输入没有错误,则可能是文档设置有问题,请参阅 \Fullref{subsec:external_preparation}。在这种情况下,可能需要手动创建 |external/| 子目录,参见 \refKeyLe{/tcb/external/prefix}。
\end{extTcbKey}


% \clearpage

\begin{extTcbKey}[][doc new=2015-05-05]{preclass}{=\meta{code}}{no default,
  initially unset}
The given \meta{code} is added before the snippet document.
Typically, this means before |\documentclass|.
This is not used for compilation of the main document.

如果将此选项设置为 |false|,则会立即在错误时停止编译。在这种情况下,必须查看外部片段的日志文件以获取错误消息。
\end{extTcbKey}


\begin{extTcbKey}[][doc new=2015-05-05]{PassOptionsToPackage}{=\marg{options}\marg{package}}{no default,
  initially unset}
The given \meta{options} are passed to the given \meta{package} for
the snippet document. This is a shortcut for using \refKeyLe{/tcb/external/preclass}
with |\PassOptionsToPackage|.
This not used for compilation of the main document.

在片段文档之前添加给定的 \meta{code}。通常,这意味着在 |\documentclass| 之前。这不用于编译主文档。
\end{extTcbKey}


\begin{extTcbKey}[][doc new=2015-05-05]{PassOptionsToClass}{=\marg{options}\marg{class}}{no default,
initially unset}
The given \meta{options} are passed to the given \meta{class} for
the snippet document. This is a shortcut for using \refKeyLe{/tcb/external/preclass}
with |\PassOptionsToClass|.
This not used for compilation of the main document.

将给定的 \meta{options} 传递给给定的 \meta{package} 的片段文档。这是使用 |\PassOptionsToPackage| 和 \refKeyLe{/tcb/external/preclass} 的快捷方式。这不用于编译主文档。
\end{extTcbKey}


\begin{extTcbKey}[][doc new=2015-05-05]{clear preclass}{}{no value}
Removes all additional \refKeyLe{/tcb/external/preclass} settings.

清除所有额外的 \refKeyLe{/tcb/external/preclass} 设置。
\end{extTcbKey}



\begin{extTcbKey}[][doc new=2015-03-11]{preamble}{=\meta{code}}{no default,
initially unset}
The given \meta{code} is added to the preamble of the snippet document.
This is not used for compilation of the main document.

将给定的 \meta{代码} 添加到片段文档的导言区。这不用于主文档的编译。
\end{extTcbKey}


\begin{extTcbKey}[][doc new=2015-05-05]{preamble tcbset}{=\meta{options}}{no default,
initially unset}
The given \meta{options} are added as parameter for \refComLe{tcbset}
to the preamble of the snippet document.
This are not used for compilation of the main document.

将给定的 \meta{选项} 作为参数添加到片段文档的导言区的 \refCom{tcbset} 命令中。
这不用于主文档的编译。
\end{extTcbKey}


\begin{extTcbKey}[][doc new=2015-03-16]{clear preamble}{}{no value}
Removes all additional \refKeyLe{/tcb/external/preamble} settings.

清除所有额外的 \refKeyLe{/tcb/external/preamble} 设置。
\end{extTcbKey}



\begin{docCommand}[doc new=2015-03-11]{tcbifexternal}{\marg{true}\marg{false}}
Expands to \meta{true}, if executed during snippet compilation,
and to \meta{false}, if executed during main document compilation.
This can be used \emph{before} \refComLe{tcbEXTERNALIZE} to
give different setting to snippet and main document.

如果在片段编译期间执行,则展开为 \meta{true},如果在主文档编译期间执行,则展开为 \meta{false}。
这可以在 \refCom{tcbEXTERNALIZE} 前使用,为片段和主文档赋予不同的设置。
\begin{dispListing}
\tcbifexternal{
  \usepackage{onlyforexternal}
}{
  \usepackage{onlyformain}
}
\end{dispListing}
\end{docCommand}


% \clearpage
\begin{docCommand}[doc new=2015-03-11]{newtcbexternalizeenvironment}{\marg{newenv}\marg{env}\marg{options}\marg{begin}\marg{end}}
Creates a new environment \meta{newenv} which is based on
\refEnvLe{tcbexternal}. This enviroment takes \emph{at least}
one optional parameter and one mandatory parameter.
These two parameters are passed to \refEnvLe{tcbexternal}.
Further, the given \meta{options} are always added to the option list of \refEnvLe{tcbexternal}.\par
The environment content is externalized and the external snippet is surrounded
by an environment \meta{env}. All further parameters of \meta{newenv}
are given to \meta{env} as parameters.\par
The included image is prepended by \meta{begin} and appended by \meta{end}.\par
\refEnvLe{extikzpicture} is an example application
for \refComLe{newtcbexternalizeenvironment}.

创建一个基于\refEnvLe{tcbexternal}的新环境\meta{newenv}。此环境至少需要一个可选参数和一个必选参数,这两个参数将传递给\refEnvLe{tcbexternal}。此外,给定的\meta{options}将始终添加到\refEnvLe{tcbexternal}的选项列表中。环境内容被外部化,外部片段由一个名为\meta{env}的环境包围。\meta{newenv}的所有其他参数都作为参数传递给\meta{env}。图像前面添加了\meta{begin},后面添加了\meta{end}。\refEnvLe{extikzpicture}是使用\refComLe{newtcbexternalizeenvironment}的一个示例应用。
\begin{dispExample}
\newtcbexternalizeenvironment{extabular}{tabular}{}{\par\centering}{\par}

\begin{extabular}{example_tabular}{|l|p{6cm}|r|}\hline
A & B & C\\\hline
a & This table is externalized as snippet. Obviously,
  this only makes sense for highly complex tables.
& b\\\hline
\end{extabular}
\end{dispExample}
\end{docCommand}


\begin{docCommand}[doc new=2015-03-11]{renewtcbexternalizeenvironment}{\marg{newenv}\marg{env}\marg{options}\marg{begin}\marg{end}}
Identical to \refComLe{newtcbexternalizeenvironment}, but the environment \meta{newenv}
is created by |\renewenvironment| instead of |\newenvironment|.

与\refComLe{newtcbexternalizeenvironment}相同,但是环境\meta{newenv}是由|\renewenvironment|而不是|\newenvironment|创建的。
\end{docCommand}


\begin{docCommand}[doc new=2015-03-11]{newtcbexternalizetcolorbox}{\marg{newenv}\marg{env}\marg{options}\marg{begin end options}}
  Creates a new environment \meta{newenv} which is based on
  \refEnvLe{tcbexternal}. This enviroment takes \emph{at least}
  one optional parameter and one mandatory parameter.
  These two parameters are passed to \refEnvLe{tcbexternal}.
  Further, the given \meta{options} are always added to the option list of \refEnvLe{tcbexternal}.\par
  The environment content is externalized and the external snippet is surrounded
  by an environment \meta{env}. All further parameters of \meta{newenv}
  are given to \meta{env} as parameters.
  \textbf{In contrast to \refComLe{newtcbexternalizeenvironment}, the
  environment \meta{env} is intended to be based on \refEnvLe{tcolorbox}
  or \refEnvLe{tcblisting}.}\par
  The \meta{begin end options} are options for settings the space before
  and after the included image using \refKeyLe{/tcb/before}, \refKeyLe{/tcb/before skip},
  \refKeyLe{/tcb/after}, or \refKeyLe{/tcb/after skip}.

  创建一个新的环境 \meta{newenv},它基于 \refEnvLe{tcbexternal}。该环境至少需要一个可选参数和一个必需参数。这两个参数将传递给 \refEnvLe{tcbexternal}。另外,给定的 \meta{options} 总是添加到 \refEnvLe{tcbexternal} 的选项列表中。环境的内容会被外部化,并且外部片段由环境 \meta{env} 包围。\meta{newenv} 的所有其他参数将作为参数传递给 \meta{env}。

  \textbf{与 \refComLe{newtcbexternalizeenvironment} 不同,环境 \meta{env} 旨在基于 \refEnvLe{tcolorbox} 或 \refEnvLe{tcblisting}。}

  \meta{begin end options} 是设置包含图像前后的空间的选项,可以使用 \refKeyLe{/tcb/before}、\refKeyLe{/tcb/before skip}、\refKeyLe{/tcb/after} 或 \refKeyLe{/tcb/after skip}。


  \begin{marker}
  Use the exact identical values for \refKeyLe{/tcb/before} and \refKeyLe{/tcb/after}
  inside \meta{begin end options} as they where used for definition of
  \meta{env}! Otherwise, externalized and non-externalized version will have
  different spacings.

  在 \meta{begin end options} 中,使用与定义 \meta{env} 时完全相同的 \refKeyLe{/tcb/before} 和 \refKeyLe{/tcb/after} 值!否则,外部化和非外部化版本将具有不同的间距。
  \end{marker}
  \refEnvLe{extcolorbox} is an example application for \refComLe{newtcbexternalizetcolorbox}.

  \refEnvLe{extcolorbox} 是使用 \refComLe{newtcbexternalizetcolorbox} 的一个示例应用。

\inputpreamblelisting{M}

{
\tcbset{external/preamble={\input{tcolorbox_preamble_M.tex}}}
\begin{dispExample}
\begin{exmyownlisting}{example_mylisting}% <- name for the external file
  {My externalized example box}
This is my \LaTeX\ box.
\end{exmyownlisting}
\end{dispExample}
}
\end{docCommand}


\begin{docCommand}[doc new=2015-03-11]{renewtcbexternalizetcolorbox}{\marg{newenv}\marg{env}\marg{options}\marg{begin end options}}
Identical to \refComLe{newtcbexternalizetcolorbox}, but the environment \meta{newenv}
is created by |\renewenvironment| instead of |\newenvironment|.

与\refComLe{newtcbexternalizetcolorbox}相同,但环境\meta{newenv}是通过|\renewenvironment|而不是|\newenvironment|创建的。
\end{docCommand}


\begin{docCommand}[doc new=2016-07-14]{tcbiffileprocess}{\marg{condition}\marg{source}\marg{md5-file}\marg{target}\marg{true}\marg{false}}
This is a low-level macro which is internally used.
The MD5 digest of a \meta{source} file is compared with
a stored MD5 digest from an auxiliary \meta{md5-file}.
If they are not equal, the auxiliary \meta{md5-file} is updated to
store the current MD5 digest. Further,

这是一个内部使用的低级宏。将\meta{source}文件的MD5摘要与来自辅助\meta{md5-file}的存储MD5摘要进行比较。如果它们不相等,则更新辅助\meta{md5-file}以存储当前的MD5摘要。进一步地,
\begin{itemize}
\item if \meta{condition} equals |0|, \meta{true} is executed.

如果\meta{condition}等于|0|,则执行\meta{true}。
\item if \meta{condition} equals |1|:\\
If the current and stored MD5 digests were different, \meta{true} is executed.\\
Otherwise, if the \meta{target} file is not existing, \meta{true} is executed.\\
Otherwise, if the \meta{target} file is older than the \meta{md5-file}, \meta{true} is executed.\\
Otherwise, \meta{false} is executed.

如果\meta{condition}等于|1|:\\
如果当前和存储的MD5摘要不同,则执行\meta{true}。\\
否则,如果\meta{target}文件不存在,则执行\meta{true}。\\
否则,如果\meta{target}文件比\meta{md5-file}旧,则执行\meta{true}。\\
否则,执行\meta{false}。
\item if \meta{condition} equals |2|, \meta{false} is executed.

如果\meta{condition}等于|2|,则执行\meta{false}。
\end{itemize}
The intended processing purpose of the \meta{true} code is to produce a \meta{target}
file from the given \meta{source} file.

\meta{true}代码的预期处理目的是从给定的\meta{source}文件生成\meta{target}文件。
\end{docCommand}
