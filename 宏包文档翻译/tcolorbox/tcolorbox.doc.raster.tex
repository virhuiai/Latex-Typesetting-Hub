% !TeX root = tcolorbox.tex
% include file of tcolorbox.tex (manual of the LaTeX package tcolorbox)
% \clearpage
\setcounter{section}{15}
\section{Library \mylib{raster}}\label{sec:raster}%
\tcbset{external/prefix=external/raster_}%
The library is loaded by a package option or inside the preamble by:

该库可以通过包选项或在导言中加载:
\begin{dispListing}
\tcbuselibrary{raster}
\end{dispListing}
%This also loads the package |xparse| \cite{latexproject:xparse}.
%\\%这也会加载包|xparse| \cite{latexproject:xparse}。

%The purpose of this library is to give comfortable access to the
%powerful document command production with |xparse| for |tcolorbox|.
%See the |xparse| package documentation \cite{latexproject:xparse}
%for details about the argument \meta{specification} used in this section.
%\\该库的目的是为了方便地访问使用|xparse|为|tcolorbox|生成强大文档命令。有关本节中使用的参数\meta{specification}的详细信息,请参阅|xparse|包文档 \cite{latexproject:xparse}。

\include*{raster/栅格的概念}
\end{document}

\subsection{Macros of the Library\\库的宏}\label{subsec:raster_macros}

\begin{docEnvironment}[doc new and updated={2014-11-10}{2017-02-01}]{tcbraster}{\oarg{options}}
A raster arranges enclosed boxes in a regular way, mainly into rows and
columns. The \meta{options} are used to control the raster
parameters and to set the properties for the enclosed boxes.
\begin{itemize}
\item The \emph{raster} is only allowed to contain a series of
  \refEnvLe{tcolorbox} environments or derived constructs.
  With some small restrictions, boxes created with \refComLe{tcboxfit} can also be added.
  Boxes created with \refComLe{tcbox} are not reasonable here, but may be
  used to a certain degree.
\item Do not add anything else between the boxes inside the raster with
  exception of white\-space. Especially, do not use |\\| or |\par| to end
  a row; row breaks are done automatically.
\item The boxes inside a raster are numbered automatically.
  \docAuxCommand{thetcbrasternum} may be used inside a box to access
  this number.
  The \LaTeX\ counter \docCounter{tcbrastercolumn} holds the current column,
  the counter \docCounter{tcbrasterrow} holds the current row,
  and the counter \docCounter{tcbrasternum} holds the current box number.
\end{itemize}

\enlargethispage*{1cm}

\begin{dispExample}
\begin{tcbraster}[raster columns=3, raster equal height,
  size=small,colframe=red!50!black,colback=red!10!white,colbacktitle=red!50!white,
  title={Box \# \thetcbrasternum}]
  \begin{tcolorbox}First box\end{tcolorbox}
  \begin{tcolorbox}Second box\end{tcolorbox}
  \begin{tcolorbox}This is a box\\with a second line\end{tcolorbox}
  \begin{tcolorbox}Another box\end{tcolorbox}
  \begin{tcolorbox}A box again\end{tcolorbox}
\end{tcbraster}
\end{dispExample}

\begin{dispExample}
\begin{tcbraster}[raster columns=2, raster equal height=rows,
  size=small,colframe=red!50!black,colback=red!10!white,colbacktitle=red!50!white,
  title={Box \# \thetcbrasternum}]
  \begin{tcolorbox}First box\end{tcolorbox}
  \begin{tcolorbox}Second box\end{tcolorbox}
  \begin{tcolorbox}This is a box\\with a second line\end{tcolorbox}
  \begin{tcolorbox}Another box\end{tcolorbox}
  \begin{tcolorbox}A box again\end{tcolorbox}
\end{tcbraster}
\end{dispExample}
\end{docEnvironment}


\clearpage

\begin{docEnvironment}[doc new=2014-11-10]{tcbitemize}{\oarg{options}}
This is a special case of a \refEnvLe{tcbraster} with the given \meta{options}.
\begin{itemize}
\item Here, the enclosed boxes are created using \refComLe{tcbitem}.
\item There has to be at least one \refComLe{tcbitem}.
\item One cannot use anything else than \refComLe{tcbitem} to add something
  to the \emph{raster}.
\end{itemize}
This leads to a very compact syntax.

\begin{dispExample}
\begin{tcbitemize}[raster columns=2, raster equal height=rows,
  size=small,colframe=red!50!black,colback=red!10!white,colbacktitle=red!50!white,
  title={Box \# \thetcbrasternum}]
  \tcbitem First box
  \tcbitem Second box
  \tcbitem This is a box\\with a second line
  \tcbitem[colback=yellow,colbacktitle=yellow!50!black] Another box
  \tcbitem A box again
\end{tcbitemize}
\end{dispExample}

\begin{marker}
\refEnvLe{tcbitemize} has more restrictions than \refEnvLe{tcbraster}.
Especially, the \refKeyLe{/tcb/capture} mode has to be \docValue{minipage}.
For example, \refKeyLe{/tcb/fit} cannot be used safely.
If \refKeyLe{/tcb/fit} should be used, turn over to \refEnvLe{tcbraster}.
\end{marker}
\end{docEnvironment}


\begin{docCommand}[doc new=2014-11-10]{tcbitem}{\oarg{options}}
  Used inside \refEnvLe{tcbitemize} to create a new \refEnvLe{tcolorbox}
  with the given \meta{options}.
\end{docCommand}


\clearpage
\begin{docEnvironment}[doc new=2016-02-19]{tcboxedraster}{\oarg{raster options}\marg{box options}}
  This is a convenience environment which combines a \refEnvLe{tcolorbox} with
  an embedded \refEnvLe{tcbraster}. The \meta{box options} are given to the
  outer \refEnvLe{tcolorbox}, while the \meta{raster options} are given to the
  embedded \refEnvLe{tcbraster}.
  This environment is especially useful for rasters inside rasters.

\begin{dispExample}
\begin{tcboxedraster}[raster columns=3, raster equal height,
  size=small,colframe=red!50!black,colback=red!10!white,colbacktitle=red!50!white,
  title={Box \# \thetcbrasternum}]
  {colback=yellow!10,fonttitle=\bfseries,title=Boxed Raster}
  \begin{tcolorbox}First box\end{tcolorbox}
  \begin{tcolorbox}Second box\end{tcolorbox}
  \begin{tcolorbox}This is a box\\with a second line\end{tcolorbox}
  \begin{tcolorbox}Another box\end{tcolorbox}
  \begin{tcolorbox}A box again\end{tcolorbox}
\end{tcboxedraster}
\end{dispExample}

\begin{dispExample}
% \tcbuselibrary{skins}
\begin{tcbraster}[raster columns=2, raster equal height,
  raster every box/.style={size=small,colframe=red!50!black,colback=red!10!white,
    valign=center,halign=center}]
  \begin{tcolorbox}One\end{tcolorbox}
  \begin{tcolorbox}Two\end{tcolorbox}
    \begin{tcboxedraster}{blankest}
      \begin{tcolorbox}Three\end{tcolorbox}
      \begin{tcolorbox}Four\end{tcolorbox}
      \begin{tcolorbox}Five\end{tcolorbox}
      \begin{tcolorbox}Six\end{tcolorbox}
    \end{tcboxedraster}
    \begin{tcolorbox}Seven\end{tcolorbox}
\end{tcbraster}
\end{dispExample}
\end{docEnvironment}


\clearpage
\begin{docEnvironment}[doc new=2016-04-27]{tcboxeditemize}{\oarg{raster options}\marg{box options}}
  This is a convenience environment which combines a \refEnvLe{tcolorbox} with
  an embedded \refEnvLe{tcbitemize}. The \meta{box options} are given to the
  outer \refEnvLe{tcolorbox}, while the \meta{raster options} are given to the
  embedded \refEnvLe{tcbitemize}.
  This environment is especially useful for rasters inside rasters.

\begin{dispExample}
\begin{tcboxeditemize}[raster columns=3, raster equal height,
  size=small,colframe=red!50!black,colback=red!10!white,colbacktitle=red!50!white,
  title={Box \# \thetcbrasternum}]
  {colback=yellow!10,fonttitle=\bfseries,title=Boxed Itemize}
  \tcbitem First box
  \tcbitem Second box
  \tcbitem This is a box\\with a second line
  \tcbitem Another box
  \tcbitem A box again
\end{tcboxeditemize}
\end{dispExample}
\end{docEnvironment}


\clearpage
\subsection{Option Keys of the Library}\label{subsec:raster_options}

\begin{docTcbKey}[][doc new=2014-11-10]{raster columns}{=\meta{number}}{no default, initially |2|}
  Sets the \meta{number} of columns for a \emph{raster}.
\begin{dispExample}
\begin{tcbitemize}[raster columns=3,
  size=small,colframe=red!50!black,colback=red!10!white]
  \tcbitem One
  \tcbitem Two
  \tcbitem Three
  \tcbitem Four
\end{tcbitemize}
\begin{tcbitemize}[raster columns=4,
  size=small,colframe=blue!50!black,colback=blue!10!white]
  \tcbitem One
  \tcbitem Two
  \tcbitem Three
  \tcbitem Four
\end{tcbitemize}
\end{dispExample}
\end{docTcbKey}

\begin{docTcbKey}[][doc new=2014-11-10]{raster rows}{=\meta{number}}{no default, initially |2|}
  Sets the \meta{number} of rows for a \emph{raster}.
  Note that this is only relevant in connection with setting \refKeyLe{/tcb/raster height}
  to a value greater than |0pt|. Then, it defines the number of rows \emph{per} given
  height.
\end{docTcbKey}


\begin{docTcbKey}[][doc new=2014-11-10]{raster width}{=\meta{length}}{no default, initially \cs{linewidth}}
  Sets the total raster width to the given \meta{length}.
  \refKeyLe{/tcb/raster left skip} and \refKeyLe{/tcb/raster right skip} are part
  of the total width. Note that both skip values are not changed by this option.
\begin{dispExample}
\begin{tcbitemize}[raster width=\linewidth/2,
  size=small,colframe=red!50!black,colback=red!10!white]
  \tcbitem One
  \tcbitem Two
  \tcbitem Three
  \tcbitem Four
\end{tcbitemize}
\end{dispExample}
\end{docTcbKey}


\clearpage
\begin{docTcbKey}[][doc new=2018-11-30]{raster width flush left}{=\meta{length}}{style, no default}
  Sets the total \refKeyLe{/tcb/raster width} to \cs{linewidth} and adapts
  \refKeyLe{/tcb/raster left skip} and \refKeyLe{/tcb/raster right skip} to
  place the raster on the left hand side
  with a visual width of the given \meta{length}.
\begin{dispExample}
\begin{tcbitemize}[raster width flush left=\linewidth/2,
  size=small,colframe=red!50!black,colback=red!10!white]
  \tcbitem One
  \tcbitem Two
  \tcbitem Three
  \tcbitem Four
\end{tcbitemize}
\end{dispExample}
Note that the results of \refKeyLe{/tcb/raster width} and \refKeyLe{/tcb/raster width flush left}
look identical, but differ on technical side since the later always fills
the available \cs{linewidth}.
\end{docTcbKey}


\begin{docTcbKey}[][doc new=2018-11-30]{raster width center}{=\meta{length}}{style, no default}
  Sets the total \refKeyLe{/tcb/raster width} to \cs{linewidth} and adapts
  \refKeyLe{/tcb/raster left skip} and \refKeyLe{/tcb/raster right skip} to center
  the raster with a visual width of the given \meta{length}.
\begin{dispExample}
\begin{tcbitemize}[raster width center=\linewidth/2,
  size=small,colframe=red!50!black,colback=red!10!white]
  \tcbitem One
  \tcbitem Two
  \tcbitem Three
  \tcbitem Four
\end{tcbitemize}
\end{dispExample}
\end{docTcbKey}


\begin{docTcbKey}[][doc new=2018-11-30]{raster width flush right}{=\meta{length}}{style, no default}
  Sets the total \refKeyLe{/tcb/raster width} to \cs{linewidth} and adapts
  \refKeyLe{/tcb/raster left skip} and \refKeyLe{/tcb/raster right skip} to
  place the raster on the right hand side
  with a visual width of the given \meta{length}.
\begin{dispExample}
\begin{tcbitemize}[raster width flush right=\linewidth/2,
  size=small,colframe=red!50!black,colback=red!10!white]
  \tcbitem One
  \tcbitem Two
  \tcbitem Three
  \tcbitem Four
\end{tcbitemize}
\end{dispExample}
\end{docTcbKey}


\clearpage
\begin{docTcbKey}[][doc new=2014-11-10]{raster height}{=\meta{length}}{no default, initially |0pt|}
  Sets the raster height \emph{per} \refKeyLe{/tcb/raster rows} to the given \meta{length}.
  This forces an appropriate height for the enclosed boxes.
  \refKeyLe{/tcb/raster before skip} and \refKeyLe{/tcb/raster after skip}
  are not part of this calculation.
  If the \meta{length} is set to |0pt|, this feature is deactivated.
\begin{dispExample}
\begin{tcbitemize}[raster height=4cm, raster rows=2,
  size=small,colframe=red!50!black,colback=red!10!white]
  \tcbitem One
  \tcbitem Two
  \tcbitem[enhanced,
    finish={\draw[blue,very thick,<->] (frame.south)
      -- node[right,pos=.75]{4cm} +(0,4); }]
    Three
  \tcbitem Four
  \tcbitem Five
\end{tcbitemize}
\end{dispExample}
\end{docTcbKey}


\begin{docTcbKey}[][doc new and updated={2014-11-10}{2014-12-16}]{raster before skip}{=\meta{glue}}{no default, initially |2mm|}
  Space of the given \meta{glue} is inserted vertically before the \emph{raster}.
  This space is discardable.
\end{docTcbKey}

\begin{docTcbKey}[][doc new and updated={2014-11-10}{2014-12-16}]{raster after skip}{=\meta{glue}}{no default, initially |2mm|}
  Space of the given \meta{glue} is inserted vertically after the \emph{raster}.
  This space is discardable.
\end{docTcbKey}


\begin{docTcbKey}[][doc new=2015-01-08]{raster equal skip}{=\meta{length}}{style, no default}
  Shortcut to set
  \refKeyLe{/tcb/raster before skip},
  \refKeyLe{/tcb/raster after skip},
  \refKeyLe{/tcb/raster column skip}, and
  \refKeyLe{/tcb/raster row skip}
  to the same \meta{length} value.
\begin{dispExample}
\begin{tcbitemize}[raster equal skip=4mm,
  size=small,colframe=red!50!black,colback=red!10!white]
  \tcbitem One
  \tcbitem Two
  \tcbitem Three
  \tcbitem Four
\end{tcbitemize}
\end{dispExample}
\end{docTcbKey}


\clearpage

\begin{docTcbKey}[][doc new=2014-11-10]{raster left skip}{=\meta{length}}{no default, initially |0pt|}
  Space of the given \meta{length} is inserted horizontally left of the \emph{raster}.
\begin{dispExample}
\begin{tcbitemize}[raster left skip=2cm,
  size=small,colframe=red!50!black,colback=red!10!white]
  \tcbitem One
  \tcbitem Two
  \tcbitem Three
  \tcbitem Four
\end{tcbitemize}
\end{dispExample}
\end{docTcbKey}


\begin{docTcbKey}[][doc new=2014-11-10]{raster right skip}{=\meta{length}}{no default, initially |0pt|}
  Space of the given \meta{length} is inserted horizontally right of the \emph{raster}.
\begin{dispExample}
\begin{tcbitemize}[raster right skip=2cm,
  size=small,colframe=red!50!black,colback=red!10!white]
  \tcbitem One
  \tcbitem Two
  \tcbitem Three
  \tcbitem Four
\end{tcbitemize}
\end{dispExample}
\end{docTcbKey}

\enlargethispage*{1cm}

\begin{docTcbKey}[][doc new=2014-11-10]{raster column skip}{=\meta{length}}{no default, initially |2mm|}
 Space of the given \meta{length} is inserted horizontally between the columns.
\begin{dispExample}
\begin{tcbitemize}[raster column skip=2cm,
  size=small,colframe=red!50!black,colback=red!10!white]
  \tcbitem One
  \tcbitem Two
  \tcbitem Three
  \tcbitem Four
\end{tcbitemize}
\end{dispExample}
\end{docTcbKey}

\begin{docTcbKey}[][doc new=2014-11-10]{raster row skip}{=\meta{length}}{no default, initially |2mm|}
 Space of the given \meta{length} is inserted vertically between the rows.
\begin{dispExample}
\begin{tcbitemize}[raster row skip=0pt,
  size=small,colframe=red!50!black,colback=red!10!white]
  \tcbitem One
  \tcbitem Two
  \tcbitem Three
  \tcbitem Four
\end{tcbitemize}
\end{dispExample}
\end{docTcbKey}

\clearpage

\begin{docTcbKey}[][doc new=2014-11-10]{raster halign}{=\meta{alignment}}{no default, initially \docValue{left}}
  Defines the horizontal alignment for the boxes of the rows of a \emph{raster},
  if these rows are not completely filled (mainly: the last one).

  Feasible values for \meta{alignment} are:
  \begin{itemize}
  \item\docValue{left}: align to the left side,
  \item\docValue{center}: align to the center,
  \item\docValue{right}: align to the right side.
  \end{itemize}

\begin{dispExample}
\begin{tcbitemize}[raster halign=center,
  size=small,colframe=red!50!black,colback=red!10!white]
  \tcbitem One
  \tcbitem Two
  \tcbitem Three
\end{tcbitemize}
\end{dispExample}
\end{docTcbKey}


\begin{docTcbKey}[][doc new=2014-11-10]{raster valign}{=\meta{alignment}}{no default, initially \docValue{center}}
  Defines the vertical alignment for the boxes of a row,
  if the boxes do not have equal height. This sets the
  \refKeyLe{/tcb/box align} option.

  Feasible values for \meta{alignment} are:
  \begin{itemize}
  \item\docValue{top}: align to the top side,
  \item\docValue{center}: align to the center,
  \item\docValue{bottom}: align to the bottom side.
  \end{itemize}

\begin{dispExample}
\begin{tcbitemize}[raster valign=top, raster columns=3,
  size=small,colframe=red!50!black,colback=red!10!white]
  \tcbitem \Huge One
  \tcbitem \Large Two
  \tcbitem Three
\end{tcbitemize}
\begin{tcbitemize}[raster valign=center, raster columns=3,
  size=small,colframe=blue!50!black,colback=blue!10!white]
  \tcbitem \Huge One
  \tcbitem \Large Two
  \tcbitem Three
\end{tcbitemize}
\begin{tcbitemize}[raster valign=bottom, raster columns=3,
  size=small,colframe=green!50!black,colback=green!10!white]
  \tcbitem \Huge One
  \tcbitem \Large Two
  \tcbitem Three
\end{tcbitemize}
\end{dispExample}
\end{docTcbKey}


\clearpage
\begin{docTcbKey}[][doc new and updated={2014-11-10}{2017-02-28}]{raster equal height}{=\meta{type}}{default \docValue{all}, initially \docValue{none}}
  Puts the enclosed boxes into a common \refKeyLe{/tcb/equal height group}.
  The \meta{id} of the equal height group is chosen automatically, but
  it may be set manually by \refKeyLe{/tcb/raster equal height group}.
  Also see \refKeyLe{/tcb/minimum for current equal height group}.

  Feasible values for \meta{type} are:
  \begin{itemize}
  \item\docValue{none}: no equal height setting,
  \item\docValue{rows}: all boxes in a row are set to equal height,
  \item\docValue{all}: all boxes in the raster are set to equal height.
  \end{itemize}
  Note that you have to compile twice to see changes.

\begin{dispExample}
\begin{tcbitemize}[raster equal height=rows,
  size=small,colframe=red!50!black,colback=red!10!white]
  \tcbitem One
  \tcbitem \Huge Two
  \tcbitem Three
  \tcbitem Four
\end{tcbitemize}
\end{dispExample}
\begin{dispExample}
\begin{tcbitemize}[raster equal height,
  size=small,colframe=red!50!black,colback=red!10!white]
  \tcbitem One
  \tcbitem \Huge Two
  \tcbitem Three
  \tcbitem Four
\end{tcbitemize}
\end{dispExample}
\end{docTcbKey}


\begin{docTcbKey}[][doc new=2014-11-10]{raster equal height group}{=\meta{id}}{no default}
  Overwrites the automatically chosen id with the given \meta{id}.
  If this is used to share a common height between the \emph{raster} and
  another raster or box, the \refKeyLe{/tcb/raster equal height} option
  should be set to \docValue{all}.
\begin{dispExample}
\tcbset{size=small,colframe=red!50!black,colback=red!10!white}
\begin{tcolorbox}[equal height group=raster-manual-id]
  A single box
\end{tcolorbox}
\begin{tcbitemize}[raster equal height,raster equal height group=raster-manual-id]
  \tcbitem One
  \tcbitem \Huge Two
\end{tcbitemize}
\end{dispExample}
\end{docTcbKey}


\clearpage

\begin{docTcbKey}[][doc new=2014-11-10]{raster force size}{\colOpt{=true\textbar false}}{default |true|, initially |true|}
  Enforces the raster size computations onto the enclosed boxes.
  If set to \docValue{false}, individual settings can be used (for the better or worse).
\begin{dispExample}
\begin{tcbitemize}[raster force size=false, raster halign=center,
  size=small,colframe=red!50!black,colback=red!10!white]
  \tcbitem One
  \tcbitem Two
  \tcbitem[add to width=-3cm] Three
  \tcbitem[add to width=-3cm] Four
  \tcbitem[add to width=-3cm] Five
  \tcbitem[add to width=3cm] Six
\end{tcbitemize}
\end{dispExample}
\end{docTcbKey}



\begin{docTcbKey}[][doc new=2014-11-10]{raster reset}{}{no value}
  Sets all raster settings back to their default values.
  Note that \refKeyLe{/tcb/reset} does not execute this option.
  Style settings like \refKeyLe{/tcb/raster odd column} etc. are not
  touched by \refKeyLe{/tcb/raster reset}.
\end{docTcbKey}



\subsection{Adding Styles for Specific Boxes}\label{subsec:raster_styles}

The following styles can be defined to address certain boxes inside
a \emph{raster}. Note that such style definitions are not removed by
\refKeyLe{/tcb/reset} or \refKeyLe{/tcb/raster reset}.
The style definitions are used in the order given below.

\begin{docTcbKey}[][doc new=2014-11-24]{raster every box}{}{style}
This style is used for every box.
\end{docTcbKey}

\begin{docTcbKey}[][doc new=2014-11-10]{raster odd column}{}{style}
This style is used for every box in an odd column.
\begin{dispExample}
\begin{tcbitemize}[size=small,colframe=red!50!black,colback=red!10!white,
  raster odd column/.style={colframe=blue!50!black,colback=blue!10!white}]
  \tcbitem One
  \tcbitem Two
  \tcbitem Three
  \tcbitem Four
\end{tcbitemize}
\end{dispExample}
\end{docTcbKey}

\begin{docTcbKey}[][doc new=2014-11-10]{raster even column}{}{style}
This style is used for every box in an even column.
\end{docTcbKey}


\begin{docTcbKey}[][doc new=2014-11-10]{raster column n}{}{style}
This style is used for every box in the |n|-th column.
|n| has to be replaced by a number.
\end{docTcbKey}


\begin{docTcbKey}[][doc new=2014-11-10]{raster odd row}{}{style}
This style is used for every box in an odd row.
\end{docTcbKey}

\begin{docTcbKey}[][doc new=2014-11-10]{raster even row}{}{style}
This style is used for every box in an even row.
\end{docTcbKey}


\begin{docTcbKey}[][doc new=2014-11-10]{raster row m}{}{style}
This style is used for every box in the |m|-th row.
|m| has to be replaced by a number.
\begin{dispExample}
\begin{tcbitemize}[size=small,colframe=red!50!black,colback=red!10!white,
  raster row 2/.style={colframe=blue!50!black,colback=blue!10!white}]
  \tcbitem One
  \tcbitem Two
  \tcbitem Three
  \tcbitem Four
\end{tcbitemize}
\end{dispExample}
\end{docTcbKey}

\begin{docTcbKey}[][doc new=2014-11-10]{raster odd number}{}{style}
This style is used for every box with an odd number.
\end{docTcbKey}

\begin{docTcbKey}[][doc new=2014-11-10]{raster even number}{}{style}
This style is used for every box with an even number.
\begin{dispExample}
\begin{tcbitemize}[size=small,colframe=red!50!black,colback=red!10!white,
  raster columns=3,
  raster even number/.style={colframe=blue!50!black,colback=blue!10!white}]
  \tcbitem One
  \tcbitem Two
  \tcbitem Three
  \tcbitem Four
  \tcbitem Five
  \tcbitem Six
\end{tcbitemize}
\end{dispExample}
\end{docTcbKey}


\begin{docTcbKey}[][doc new=2014-11-10]{raster row m column n}{}{style}
This style is used for the box in the
|m|-th row and |n|-th column.
|m| and |n| have to be replaced by numbers.
\end{docTcbKey}


\begin{docTcbKey}[][doc new=2014-11-10]{raster number n}{}{style}
This style is used for the box with number |n|.
|n| has to be replaced by a number.
\begin{dispExample}
\begin{tcbitemize}[size=small,colframe=red!50!black,colback=red!10!white,
  raster number 4/.style={colframe=blue!50!black,colback=blue!10!white}]
  \tcbitem One
  \tcbitem Two
  \tcbitem Three
  \tcbitem Four
\end{tcbitemize}
\end{dispExample}
\end{docTcbKey}


\clearpage
\subsection{Combining Columns or Rows}\label{subsec:raster_multicolrow}

\begin{docTcbKey}[][doc new=2016-02-19]{raster multicolumn}{=\meta{number}}{no default, initially unset}
  This option has to be set inside the option list of a \refEnvLe{tcolorbox}
  inside a \refEnvLe{tcbraster} or inside \refComLe{tcbitem} inside \refEnvLe{tcbitemize}.
  It merges the given \meta{number} of boxes into one single box
  on the same line. The resulting box gets the \docAuxCommand{thetcbrasternum}
  of the first box.
  If there are not enough boxes available on the current line, this option
  is ignored and a warning is given.

\begin{dispExample}
\begin{tcbitemize}[raster equal height=rows,raster columns=3,
  title=\thetcbrasternum,colframe=red!50!black,colback=red!10!white]
\tcbitem[colframe=blue!50!black,colback=blue!10!white,raster multicolumn=1]
  multicolumn=1
\tcbitem
\tcbitem
\tcbitem[colframe=blue!50!black,colback=blue!10!white,raster multicolumn=2]
  multicolumn=2
\tcbitem
\tcbitem[colframe=blue!50!black,colback=blue!10!white,raster multicolumn=3]
  multicolumn=3
\tcbitem
\tcbitem[colframe=blue!50!black,colback=blue!10!white,raster multicolumn=2]
  multicolumn=2
\end{tcbitemize}
\end{dispExample}
\end{docTcbKey}


\clearpage
\begin{docTcbKey}[][doc new=2016-02-19]{raster multirow}{=\meta{number}}{no default, initially unset}
  This option has to be set inside the option list of a \refEnvLe{tcolorbox}
  inside a \refEnvLe{tcbraster} or inside \refComLe{tcbitem} inside \refEnvLe{tcbitemize}.
  This option not really merges boxes, but simply sizes the current box to fit
  the space of \meta{number} rows.

  \begin{marker}
    \refKeyLe{/tcb/raster multirow} needs \refKeyLe{/tcb/raster height} to be set.
    How to achieve a similar result for boxes without fixed \refKeyLe{/tcb/raster height}
    is shown afterwards.
  \end{marker}

\begin{dispExample}
\begin{tcbitemize}[raster rows=3,raster columns=3,raster height=6cm,
  raster every box/.style={colframe=red!50!black,colback=red!10!white}]
\tcbitem
\tcbitem
\tcbitem
\tcbitem[colframe=blue!50!black,colback=blue!10!white,raster multirow=2]
  multirow=2
\tcbitem[raster multicolumn=2,raster multirow=2,blankest]
  \begin{tcbitemize}[raster rows=2,raster columns=2,raster height=\tcbtextheight]
  \tcbitem
  \tcbitem
  \tcbitem
  \tcbitem
  \end{tcbitemize}
\end{tcbitemize}
\end{dispExample}


\clearpage
For rasters without fixed \refKeyLe{/tcb/raster height}, \refKeyLe{/tcb/raster multirow}
cannot be used. Note that \refComLe{tcbtextheight} also cannot be used like
in the previous example.

But, with combination of \refKeyLe{/tcb/raster equal height} and
\refKeyLe{/tcb/space to}, a similar effect can be created:

\begin{dispExample}
\begin{tcbitemize}[raster columns=3,raster equal height=rows,
  raster every box/.style={colframe=red!50!black,colback=red!10!white}]
\tcbitem
\tcbitem
\tcbitem
\tcbitem[colframe=blue!50!black,colback=blue!10!white]
 \lipsum[2]
\tcbitem[raster multicolumn=2,blankest,space to=\myspace]
  \begin{tcbitemize}[raster columns=2]
  \tcbitem This is a box of the inner raster.
  \tcbitem
  \tcbitem[height=\myspace]
  \tcbitem[height=\myspace]
  \end{tcbitemize}
\end{tcbitemize}
\end{dispExample}

\end{docTcbKey}



\clearpage
\subsection{Rasters inside Rasters}\label{subsec:raster_insideraster}

A \emph{raster} inside a \emph{raster} cannot be used directly, because
a \emph{raster} can only contain a \emph{tcolorbox} or something
derived from a \emph{tcolorbox}. So, a \emph{raster} can be put inside
a \emph{tcolorbox} inside a \emph{raster}.

Some examples for such constructions can be found at \refEnvLe{tcboxedraster},
\refKeyLe{/tcb/raster multicolumn},
\refKeyLe{/tcb/raster multirow}.


\subsubsection{Raster Setup}
The intermediating \refEnvLe{tcolorbox} can be made invisible by using
\refKeyLe{/tcb/blankest}.

\begin{dispExample}
\begin{tcbraster}[raster equal height=rows,
  raster every box/.style={colframe=red!50!black,colback=red!10!white}]
  \begin{tcolorbox}[blankest]
    \begin{tcbraster}[raster columns=1]
      \begin{tcolorbox}One\end{tcolorbox}
      \begin{tcolorbox}Two\end{tcolorbox}
    \end{tcbraster}
  \end{tcolorbox}
  \begin{tcolorbox}raster+tcolorbox+raster\end{tcolorbox}
\end{tcbraster}
\end{dispExample}

\enlargethispage*{1cm}
\begin{dispExample}
\begin{tcbraster}[raster equal height=rows,
  raster every box/.style={colframe=red!50!black,colback=red!10!white}]
  \begin{tcboxedraster}[raster columns=1]{blankest}
    \begin{tcolorbox}One\end{tcolorbox}
    \begin{tcolorbox}Two\end{tcolorbox}
  \end{tcboxedraster}
  \begin{tcolorbox}raster+tcboxedraster\end{tcolorbox}
\end{tcbraster}
\end{dispExample}


\begin{dispExample}
\begin{tcbitemize}[raster equal height=rows,
  raster every box/.style={colframe=red!50!black,colback=red!10!white}]
  \tcbitem[blankest]
    \begin{tcbitemize}[raster columns=1]
      \tcbitem One
      \tcbitem Two
    \end{tcbitemize}
  \tcbitem tcbitemize+tcbitem+tcbitemize
\end{tcbitemize}
\end{dispExample}


\subsubsection{Placing Spaces}
If the heights of boxes inside staggered rasters should be matched, the
space has to be distributed accordingly.

\begin{itemize}
\item For fixed height boxes/rasters using \refKeyLe{/tcb/raster height},
  the height of boxes is available by \refComLe{tcbtextheight}. This can be
  used to size deeper layered boxes/rasters.
\item For boxes/rasters layed out using \refKeyLe{/tcb/raster equal height},
  space can be distributed by \refKeyLe{/tcb/space to}.
  It can take several compilations until all spaces are distributed correctly.
\end{itemize}


\begin{dispExample}
\begin{tcbitemize}[raster rows=2,raster height=6cm,
  raster every box/.style={colframe=red!50!black,colback=red!10!white}]
  \tcbitem[blankest]
    \begin{tcbitemize}[raster columns=1,raster rows=2,raster height=\tcbtextheight]
      \tcbitem One
      \tcbitem Two
    \end{tcbitemize}
  \tcbitem This is a fixed height box.
  \tcbitem Three
  \tcbitem Four
\end{tcbitemize}
\end{dispExample}


\begin{dispExample}
\begin{tcbitemize}[raster columns=4,raster rows=4,raster height=0.8\linewidth,
  raster every box/.style={size=small,beamer,
    colframe=blue!75!yellow,colback=red!75!yellow!20,
    center title,title=Box}]
  \tcbitem One
  \tcbitem Two
  \tcbitem Three
  \tcbitem Four
  \tcbitem[raster multirow=2,blankest]
    \begin{tcbitemize}[raster columns=1,raster rows=2,raster height=\tcbtextheight]
      \tcbitem Twelve
      \tcbitem Eleven
    \end{tcbitemize}
  \tcbitem[raster multirow=2,raster multicolumn=2,
      colframe=red!75!yellow,colback=blue!75!yellow!20]
    This is an example with fixed height boxes.
  \tcbitem[raster multirow=2,blankest]
    \begin{tcbitemize}[raster columns=1,raster rows=2,raster height=\tcbtextheight]
      \tcbitem Five
      \tcbitem Six
    \end{tcbitemize}
  \tcbitem Ten
  \tcbitem Nine
  \tcbitem Eight
  \tcbitem Seven
\end{tcbitemize}
\end{dispExample}


\begin{dispExample}
\begin{tcbitemize}[raster equal height=rows,
  raster every box/.style={colframe=red!50!black,colback=red!10!white}]
  \tcbitem[blankest,space to=\myspace]
    \begin{tcbitemize}[raster columns=1]
      \tcbitem One
      \tcbitem[add to natural height=\myspace]
        This box will adapt its height.
    \end{tcbitemize}
  \tcbitem This is a flexible height box.
  \tcbitem \lipsum[4]
  \tcbitem[blankest,space to=\myspace]
    \begin{tcbitemize}[raster columns=1]
      \tcbitem One
      \tcbitem[add to natural height=\myspace]
        This box will adapt its height.
    \end{tcbitemize}
\end{tcbitemize}
\end{dispExample}



\begin{dispExample}
\begin{tcbitemize}[raster equal height=rows,
  raster every box/.style={colframe=red!50!black,colback=red!10!white}]
  \tcbitem[blankest,space to=\myspace]
    \begin{tcbitemize}[raster columns=1]
      \tcbitem One
      \tcbitem[add to natural height=\myspace]
        This box will adapt its height.
      \tcbitem \lipsum[4]
    \end{tcbitemize}
  \tcbitem[blankest,space to=\myspace]
    \begin{tcbitemize}[raster columns=1]
      \tcbitem[blankest]\includegraphics[width=\linewidth]{goldshade.png}
      \tcbitem[add to natural height=\myspace]
        This box will adapt its height.
    \end{tcbitemize}
\end{tcbitemize}
\end{dispExample}



