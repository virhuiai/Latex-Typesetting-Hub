% !TeX root = tcolorbox.tex
% include file of tcolorbox.tex (manual of the LaTeX package tcolorbox)
% \clearpage
\setcounter{section}{15}
\section{Library \mylib{raster}}\label{sec:raster}%
\tcbset{external/prefix=external/raster_}%
The library is loaded by a package option or inside the preamble by:

该库可以通过包选项或在导言中加载:
\begin{dispListing}
\tcbuselibrary{raster}
\end{dispListing}
%This also loads the package |xparse| \cite{latexproject:xparse}.
%\\%这也会加载包|xparse| \cite{latexproject:xparse}。

%The purpose of this library is to give comfortable access to the
%powerful document command production with |xparse| for |tcolorbox|.
%See the |xparse| package documentation \cite{latexproject:xparse}
%for details about the argument \meta{specification} used in this section.
%\\该库的目的是为了方便地访问使用|xparse|为|tcolorbox|生成强大文档命令。有关本节中使用的参数\meta{specification}的详细信息,请参阅|xparse|包文档 \cite{latexproject:xparse}。

% \include*{raster/栅格的概念}
% \include*{raster/库的宏}
\include*{raster/库的选项键}
\end{document}


\subsection{Adding Styles for Specific Boxes}\label{subsec:raster_styles}

The following styles can be defined to address certain boxes inside
a \emph{raster}. Note that such style definitions are not removed by
\refKeyLe{/tcb/reset} or \refKeyLe{/tcb/raster reset}.
The style definitions are used in the order given below.

\begin{docTcbKey}[][doc new=2014-11-24]{raster every box}{}{style}
This style is used for every box.
\end{docTcbKey}

\begin{docTcbKey}[][doc new=2014-11-10]{raster odd column}{}{style}
This style is used for every box in an odd column.
\begin{dispExample}
\begin{tcbitemize}[size=small,colframe=red!50!black,colback=red!10!white,
  raster odd column/.style={colframe=blue!50!black,colback=blue!10!white}]
  \tcbitem One
  \tcbitem Two
  \tcbitem Three
  \tcbitem Four
\end{tcbitemize}
\end{dispExample}
\end{docTcbKey}

\begin{docTcbKey}[][doc new=2014-11-10]{raster even column}{}{style}
This style is used for every box in an even column.
\end{docTcbKey}


\begin{docTcbKey}[][doc new=2014-11-10]{raster column n}{}{style}
This style is used for every box in the |n|-th column.
|n| has to be replaced by a number.
\end{docTcbKey}


\begin{docTcbKey}[][doc new=2014-11-10]{raster odd row}{}{style}
This style is used for every box in an odd row.
\end{docTcbKey}

\begin{docTcbKey}[][doc new=2014-11-10]{raster even row}{}{style}
This style is used for every box in an even row.
\end{docTcbKey}


\begin{docTcbKey}[][doc new=2014-11-10]{raster row m}{}{style}
This style is used for every box in the |m|-th row.
|m| has to be replaced by a number.
\begin{dispExample}
\begin{tcbitemize}[size=small,colframe=red!50!black,colback=red!10!white,
  raster row 2/.style={colframe=blue!50!black,colback=blue!10!white}]
  \tcbitem One
  \tcbitem Two
  \tcbitem Three
  \tcbitem Four
\end{tcbitemize}
\end{dispExample}
\end{docTcbKey}

\begin{docTcbKey}[][doc new=2014-11-10]{raster odd number}{}{style}
This style is used for every box with an odd number.
\end{docTcbKey}

\begin{docTcbKey}[][doc new=2014-11-10]{raster even number}{}{style}
This style is used for every box with an even number.
\begin{dispExample}
\begin{tcbitemize}[size=small,colframe=red!50!black,colback=red!10!white,
  raster columns=3,
  raster even number/.style={colframe=blue!50!black,colback=blue!10!white}]
  \tcbitem One
  \tcbitem Two
  \tcbitem Three
  \tcbitem Four
  \tcbitem Five
  \tcbitem Six
\end{tcbitemize}
\end{dispExample}
\end{docTcbKey}


\begin{docTcbKey}[][doc new=2014-11-10]{raster row m column n}{}{style}
This style is used for the box in the
|m|-th row and |n|-th column.
|m| and |n| have to be replaced by numbers.
\end{docTcbKey}


\begin{docTcbKey}[][doc new=2014-11-10]{raster number n}{}{style}
This style is used for the box with number |n|.
|n| has to be replaced by a number.
\begin{dispExample}
\begin{tcbitemize}[size=small,colframe=red!50!black,colback=red!10!white,
  raster number 4/.style={colframe=blue!50!black,colback=blue!10!white}]
  \tcbitem One
  \tcbitem Two
  \tcbitem Three
  \tcbitem Four
\end{tcbitemize}
\end{dispExample}
\end{docTcbKey}


\clearpage
\subsection{Combining Columns or Rows}\label{subsec:raster_multicolrow}

\begin{docTcbKey}[][doc new=2016-02-19]{raster multicolumn}{=\meta{number}}{no default, initially unset}
  This option has to be set inside the option list of a \refEnvLe{tcolorbox}
  inside a \refEnvLe{tcbraster} or inside \refComLe{tcbitem} inside \refEnvLe{tcbitemize}.
  It merges the given \meta{number} of boxes into one single box
  on the same line. The resulting box gets the \docAuxCommand{thetcbrasternum}
  of the first box.
  If there are not enough boxes available on the current line, this option
  is ignored and a warning is given.

\begin{dispExample}
\begin{tcbitemize}[raster equal height=rows,raster columns=3,
  title=\thetcbrasternum,colframe=red!50!black,colback=red!10!white]
\tcbitem[colframe=blue!50!black,colback=blue!10!white,raster multicolumn=1]
  multicolumn=1
\tcbitem
\tcbitem
\tcbitem[colframe=blue!50!black,colback=blue!10!white,raster multicolumn=2]
  multicolumn=2
\tcbitem
\tcbitem[colframe=blue!50!black,colback=blue!10!white,raster multicolumn=3]
  multicolumn=3
\tcbitem
\tcbitem[colframe=blue!50!black,colback=blue!10!white,raster multicolumn=2]
  multicolumn=2
\end{tcbitemize}
\end{dispExample}
\end{docTcbKey}


\clearpage
\begin{docTcbKey}[][doc new=2016-02-19]{raster multirow}{=\meta{number}}{no default, initially unset}
  This option has to be set inside the option list of a \refEnvLe{tcolorbox}
  inside a \refEnvLe{tcbraster} or inside \refComLe{tcbitem} inside \refEnvLe{tcbitemize}.
  This option not really merges boxes, but simply sizes the current box to fit
  the space of \meta{number} rows.

  \begin{marker}
    \refKeyLe{/tcb/raster multirow} needs \refKeyLe{/tcb/raster height} to be set.
    How to achieve a similar result for boxes without fixed \refKeyLe{/tcb/raster height}
    is shown afterwards.
  \end{marker}

\begin{dispExample}
\begin{tcbitemize}[raster rows=3,raster columns=3,raster height=6cm,
  raster every box/.style={colframe=red!50!black,colback=red!10!white}]
\tcbitem
\tcbitem
\tcbitem
\tcbitem[colframe=blue!50!black,colback=blue!10!white,raster multirow=2]
  multirow=2
\tcbitem[raster multicolumn=2,raster multirow=2,blankest]
  \begin{tcbitemize}[raster rows=2,raster columns=2,raster height=\tcbtextheight]
  \tcbitem
  \tcbitem
  \tcbitem
  \tcbitem
  \end{tcbitemize}
\end{tcbitemize}
\end{dispExample}


\clearpage
For rasters without fixed \refKeyLe{/tcb/raster height}, \refKeyLe{/tcb/raster multirow}
cannot be used. Note that \refComLe{tcbtextheight} also cannot be used like
in the previous example.

But, with combination of \refKeyLe{/tcb/raster equal height} and
\refKeyLe{/tcb/space to}, a similar effect can be created:

\begin{dispExample}
\begin{tcbitemize}[raster columns=3,raster equal height=rows,
  raster every box/.style={colframe=red!50!black,colback=red!10!white}]
\tcbitem
\tcbitem
\tcbitem
\tcbitem[colframe=blue!50!black,colback=blue!10!white]
 \lipsum[2]
\tcbitem[raster multicolumn=2,blankest,space to=\myspace]
  \begin{tcbitemize}[raster columns=2]
  \tcbitem This is a box of the inner raster.
  \tcbitem
  \tcbitem[height=\myspace]
  \tcbitem[height=\myspace]
  \end{tcbitemize}
\end{tcbitemize}
\end{dispExample}

\end{docTcbKey}



\clearpage
\subsection{Rasters inside Rasters}\label{subsec:raster_insideraster}

A \emph{raster} inside a \emph{raster} cannot be used directly, because
a \emph{raster} can only contain a \emph{tcolorbox} or something
derived from a \emph{tcolorbox}. So, a \emph{raster} can be put inside
a \emph{tcolorbox} inside a \emph{raster}.

Some examples for such constructions can be found at \refEnvLe{tcboxedraster},
\refKeyLe{/tcb/raster multicolumn},
\refKeyLe{/tcb/raster multirow}.


\subsubsection{Raster Setup}
The intermediating \refEnvLe{tcolorbox} can be made invisible by using
\refKeyLe{/tcb/blankest}.

\begin{dispExample}
\begin{tcbraster}[raster equal height=rows,
  raster every box/.style={colframe=red!50!black,colback=red!10!white}]
  \begin{tcolorbox}[blankest]
    \begin{tcbraster}[raster columns=1]
      \begin{tcolorbox}One\end{tcolorbox}
      \begin{tcolorbox}Two\end{tcolorbox}
    \end{tcbraster}
  \end{tcolorbox}
  \begin{tcolorbox}raster+tcolorbox+raster\end{tcolorbox}
\end{tcbraster}
\end{dispExample}

\enlargethispage*{1cm}
\begin{dispExample}
\begin{tcbraster}[raster equal height=rows,
  raster every box/.style={colframe=red!50!black,colback=red!10!white}]
  \begin{tcboxedraster}[raster columns=1]{blankest}
    \begin{tcolorbox}One\end{tcolorbox}
    \begin{tcolorbox}Two\end{tcolorbox}
  \end{tcboxedraster}
  \begin{tcolorbox}raster+tcboxedraster\end{tcolorbox}
\end{tcbraster}
\end{dispExample}


\begin{dispExample}
\begin{tcbitemize}[raster equal height=rows,
  raster every box/.style={colframe=red!50!black,colback=red!10!white}]
  \tcbitem[blankest]
    \begin{tcbitemize}[raster columns=1]
      \tcbitem One
      \tcbitem Two
    \end{tcbitemize}
  \tcbitem tcbitemize+tcbitem+tcbitemize
\end{tcbitemize}
\end{dispExample}


\subsubsection{Placing Spaces}
If the heights of boxes inside staggered rasters should be matched, the
space has to be distributed accordingly.

\begin{itemize}
\item For fixed height boxes/rasters using \refKeyLe{/tcb/raster height},
  the height of boxes is available by \refComLe{tcbtextheight}. This can be
  used to size deeper layered boxes/rasters.
\item For boxes/rasters layed out using \refKeyLe{/tcb/raster equal height},
  space can be distributed by \refKeyLe{/tcb/space to}.
  It can take several compilations until all spaces are distributed correctly.
\end{itemize}


\begin{dispExample}
\begin{tcbitemize}[raster rows=2,raster height=6cm,
  raster every box/.style={colframe=red!50!black,colback=red!10!white}]
  \tcbitem[blankest]
    \begin{tcbitemize}[raster columns=1,raster rows=2,raster height=\tcbtextheight]
      \tcbitem One
      \tcbitem Two
    \end{tcbitemize}
  \tcbitem This is a fixed height box.
  \tcbitem Three
  \tcbitem Four
\end{tcbitemize}
\end{dispExample}


\begin{dispExample}
\begin{tcbitemize}[raster columns=4,raster rows=4,raster height=0.8\linewidth,
  raster every box/.style={size=small,beamer,
    colframe=blue!75!yellow,colback=red!75!yellow!20,
    center title,title=Box}]
  \tcbitem One
  \tcbitem Two
  \tcbitem Three
  \tcbitem Four
  \tcbitem[raster multirow=2,blankest]
    \begin{tcbitemize}[raster columns=1,raster rows=2,raster height=\tcbtextheight]
      \tcbitem Twelve
      \tcbitem Eleven
    \end{tcbitemize}
  \tcbitem[raster multirow=2,raster multicolumn=2,
      colframe=red!75!yellow,colback=blue!75!yellow!20]
    This is an example with fixed height boxes.
  \tcbitem[raster multirow=2,blankest]
    \begin{tcbitemize}[raster columns=1,raster rows=2,raster height=\tcbtextheight]
      \tcbitem Five
      \tcbitem Six
    \end{tcbitemize}
  \tcbitem Ten
  \tcbitem Nine
  \tcbitem Eight
  \tcbitem Seven
\end{tcbitemize}
\end{dispExample}


\begin{dispExample}
\begin{tcbitemize}[raster equal height=rows,
  raster every box/.style={colframe=red!50!black,colback=red!10!white}]
  \tcbitem[blankest,space to=\myspace]
    \begin{tcbitemize}[raster columns=1]
      \tcbitem One
      \tcbitem[add to natural height=\myspace]
        This box will adapt its height.
    \end{tcbitemize}
  \tcbitem This is a flexible height box.
  \tcbitem \lipsum[4]
  \tcbitem[blankest,space to=\myspace]
    \begin{tcbitemize}[raster columns=1]
      \tcbitem One
      \tcbitem[add to natural height=\myspace]
        This box will adapt its height.
    \end{tcbitemize}
\end{tcbitemize}
\end{dispExample}



\begin{dispExample}
\begin{tcbitemize}[raster equal height=rows,
  raster every box/.style={colframe=red!50!black,colback=red!10!white}]
  \tcbitem[blankest,space to=\myspace]
    \begin{tcbitemize}[raster columns=1]
      \tcbitem One
      \tcbitem[add to natural height=\myspace]
        This box will adapt its height.
      \tcbitem \lipsum[4]
    \end{tcbitemize}
  \tcbitem[blankest,space to=\myspace]
    \begin{tcbitemize}[raster columns=1]
      \tcbitem[blankest]\includegraphics[width=\linewidth]{goldshade.png}
      \tcbitem[add to natural height=\myspace]
        This box will adapt its height.
    \end{tcbitemize}
\end{tcbitemize}
\end{dispExample}



