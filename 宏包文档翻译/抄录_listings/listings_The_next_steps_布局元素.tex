\columnratio{0.55}
\begin{paracol}{2}
\subsection{Layout elements}
\switchcolumn
\subsection{布局元素}
\switchcolumn[0]*%%%%%%%%%%%%
It's always a good idea to structure the layout by vertical space,
horizontal lines, or different type sizes and typefaces. The best to stress
whole listings are---not all at once---colours, frames, vertical space, and
captions. The latter are also good to refer to listings, of course.
\switchcolumn
在布局中使用垂直空间、水平线条或不同的字体大小和字体类型是一个很好的主意。最好强调整个代码列表的方法是——不是一次性地——使用颜色、框架、垂直空间和标题。当然,标题也是指向代码列表的好方法。
\switchcolumn[0]*%%%%%%%%%%%%
\textbf{Vertical space}
The keys {\rstyle\ikeyname{aboveskip}} and {\rstyle\ikeyname{belowskip}}
control the vertical space above and below displayed listings. Both keys get
a dimension or skip as value and are initialized to |\medskipamount|.
\switchcolumn
\textbf{垂直空间}
关键字{\rstyle\ikeyname{aboveskip}}和{\rstyle\ikeyname{belowskip}}控制显示的代码列表上方和下方的垂直空间。这两个关键字的值可以是尺寸或间距,默认值为 |\medskipamount|。
\switchcolumn[0]*%%%%%%%%%%%%
\textbf{Frames}
The key \ikeyname{frame} takes the verbose values \keyvalue{none},
\keyvalue{leftline}, \keyvalue{topline}, \keyvalue{bottomline},
\keyvalue{lines} (top and bottom), \keyvalue{single} for single frames, or
\keyvalue{shadowbox}.
\switchcolumn
\textbf{边框}
关键字\ikeyname{frame}接受以下值:\keyvalue{none}、\keyvalue{leftline}、\keyvalue{topline}、\keyvalue{bottomline}、\keyvalue{lines}(顶部和底部)、\keyvalue{single}(单边框)或\keyvalue{shadowbox}。
\begin{lstsample}[frame]{}{}
        \begin{lstlisting}[frame=single]
        for i:=maxint to 0 do
        begin
            { do nothing }
        end;
        \end{lstlisting}
\end{lstsample}
\switchcolumn[0]*%%%%%%%%%%%%
\begin{advise}
\item The rules aren't aligned.
     \advisespace
     This could be a bug of this package or a problem with your
     \texttt{.dvi} driver. \emph{Before} sending a bug report to the package
     author, modify the parameters described in section \ref{rFrames}
     heavily. And do this step by step!
     For example, begin with `|framerule=10mm|'. If the rules are
     misaligned by the same (small) amount as before, the problem does not
     come from the rule width. So continue with the next parameter.  Also,
     Adobe Acrobat sometimes has single-pixel rounding errors which can
     cause small misalignments at the corners when PDF files are displayed
     on screen; these are unfortunately normal.
\end{advise}
\switchcolumn
\begin{advise}
    \item 这些线条没有对齐。
    \advisespace
    这可能是此宏包的一个错误,或者与您的\texttt{.dvi}驱动程序有关的问题。\emph{在}向宏包作者发送错误报告之前,强烈建议您根据第\ref{rFrames}节所述的参数进行修改,并逐步进行。例如,从|framerule=10mm|'开始。如果线条的偏移量与之前相同(小),那么问题就不是由线条宽度引起的。因此,继续下一个参数。此外,Adobe Acrobat有时会出现单像素的舍入误差,当在屏幕上显示PDF文件时,这些误差会导致角落处的小偏差;这是正常现象。 
\end{advise}
\switchcolumn[0]*%%%%%%%%%%%%
Alternatively you can control the rules at the \texttt{t}op, \texttt{r}ight,
\texttt{b}ottom, and \texttt{l}eft directly by using the four initial letters
for single rules and their upper case versions for double rules.
\switchcolumn
或者,您可以直接使用四个首字母来控制\texttt{t}op、\texttt{r}ight、\texttt{b}ottom和\texttt{l}eft的线条。对于单线条,使用四个首字母的小写形式,对于双线条,使用四个首字母的大写形式。
\begin{lstsample}[frame]{}{}
        \begin{lstlisting}[frame=trBL]
        for i:=maxint to 0 do
        begin
            { do nothing }
        end;
        \end{lstlisting}
\end{lstsample}
\switchcolumn[0]*%%%%%%%%%%%%
Note that a corner is drawn if and only if both adjacent rules are requested.
You might think that the lines should be drawn up to the edge, but what's
about round corners? The key \ikeyname{frameround} must get exactly four
characters as value. The first character is attached to the upper right
corner and it continues clockwise. `\texttt{t}' as character makes the
corresponding corner round.
\switchcolumn
请注意,只有在请求了两个相邻线条时才会绘制一个角落。您可能认为线条应该绘制到边缘,但是圆角呢?关键字\ikeyname{frameround}的值必须是四个字符。第一个字符与右上角相连,然后按顺时针方向继续。字符\texttt{t}'使相应的角变为圆角。 
\begin{lstsample}[frameround]{}{}
        \lstset{frameround=fttt}
        \begin{lstlisting}[frame=trBL]
        for i:=maxint to 0 do
        begin
            { do nothing }
        end;
        \end{lstlisting}
\end{lstsample}
Note that \ikeyname{frameround} has been used together with |\lstset| and thus
the value affects all following listings in the same group or environment.
Since the listing is inside a \texttt{minipage} here, this is no problem.
\begin{advise}
\item Don't use frames all the time, and in particular not with short listings.
     This would emphasize nothing. Use frames for $10\%$ or even less of
     your listings, for your most important ones.
\item If you use frames on floating listings, do you really want frames?
     \advisespace
     No, I want to separate floats from text.
     \advisespace
     Then it is better to redefine \LaTeX's `|\topfigrule|' and
     `|\botfigrule|'. For example, you could write
     `|\renewcommand*\topfigrule{\hrule\kern-0.4pt\relax}|' and make the
     same definition for |\botfigrule|.
\end{advise}

\textbf{Captions}
Now we come to \ikeyname{caption} and \ikeyname{label}. You might guess
(correctly) that they can be used in the same manner as \LaTeX's |\caption|
and |\label| commands, although here it is also possible to have a caption
regardless of whether or not the listing is in a float:
\begin{lstsample}[caption,label]{\lstset{xleftmargin=.05\linewidth}}{}
  \begin{lstlisting}[caption={Useless code},label=useless]
  for i:=maxint to 0 do
  begin
      { do nothing }
  end;
  \end{lstlisting}
\end{lstsample}
Afterwards you could refer to the listing via |\ref{useless}|. By default
such a listing gets an entry in the list of listings, which can be printed
with the command {\rstyle\icmdname\lstlistoflistings}. The key
{\rstyle\ikeyname{nolol}} suppresses an entry for both the environment or
the input command. Moreover, you can specify a short caption for the list
of listings:
  \keyname{caption}|={|\oarg{short}\meta{long}|}|.
Note that the whole value is enclosed in braces since an optional value is
used in an optional argument.

If you don't want the label \texttt{\lstlistingname} plus number, you should
use \ikeyname{title}:
\begin{lstsample}[title]{\lstset{xleftmargin=.05\linewidth}}{}
  \begin{lstlisting}[title={`Caption' without label}]
  for i:=maxint to 0 do
  begin
      { do nothing }
  end;
  \end{lstlisting}
\end{lstsample}
\begin{advise}
\item Something goes wrong with `\keyname{title}' in my document: in front of
     the title is a delimiter.
     \advisespace
     The result depends on the document class; some are not compatible.
     Contact the package author for a work-around.
\end{advise}

\textbf{Colours}
One more element. You need the \packagename{color} package and can then
request coloured background via
\ikeyname{backgroundcolor}|=|\meta{color command}.
\begin{advise}
\item Great! I love colours.
     \advisespace
     Fine, yes, really. And I like to remind you of the warning about
     striking styles on page \pageref{wStrikingStyles}.
\end{advise}
\ifcolor
\begin{lstxsample}[backgroundcolor]
  \lstset{backgroundcolor=\color{yellow}}
\end{lstxsample}
\else
\begin{verbatim}
  color package not installed\end{verbatim}
\fi
\begin{lstsample}{}{}
  \begin{lstlisting}[frame=single,
                     framerule=0pt]
  for i:=maxint to 0 do
  begin
      j:=square(root(i));
  end;
  \end{lstlisting}
\end{lstsample}
The example also shows how to get coloured space around the whole listing:
use a frame whose rules have no width.


\end{paracol}