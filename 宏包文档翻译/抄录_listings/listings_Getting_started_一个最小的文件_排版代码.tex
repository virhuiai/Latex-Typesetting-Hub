\columnratio{0.55}
\begin{paracol}{2}
\subsection{Typesetting listings}
\switchcolumn
\subsection{排版代码}
\switchcolumn[0]*%%%%%%%%%%%%
Three types of source codes are supported: code snippets, code segments, and
listings of stand alone files.  Snippets are placed inside paragraphs and the
others as separate paragraphs---the difference is the same as between text
style and display style formulas.
\switchcolumn
支持三种类型的源代码:代码片段、代码段和独立文件的列表。片段放置在段落内,其他两种放置为独立段落---区别与文本样式和展示样式公式之间的区别。
\end{paracol}
\begin{advise}
\columnratio{0.55}
\begin{paracol}{2}

\item No matter what kind of source you have, if a listing contains national
    characters like \'e, \L, \"a, or whatever, you must tell the
    package about it! Section \lstref{uSpecialCharacters} discusses this issue.
    \switchcolumn
    \item 不管您有什么样的源代码,如果列表中包含如'e、\L、"a等国际字符,您必须告诉宏包!请参阅\lstref{uSpecialCharacters}节讨论此问题。      
\end{paracol}
\end{advise}

\begin{paracol}{2}
\switchcolumn[0]*%%%%%%%%%%%%
\textbf{Code snippets}
The well-known \LaTeX\ command |\verb| typesets code snippets verbatim.
The new command |\lstinline| pretty-prints the code, for example
`\lstinline!var i:integer;!' is typeset by
`{\rstyle|\lstinline|}|!var i:integer;!|'. The exclamation marks delimit
the code and can be replaced by any character not in the code;
|\lstinline$var i:integer;$| gives the same result.
\switchcolumn
\textbf{代码片段}
众所周知,\LaTeX 命令|\verb|以逐字方式排版代码片段。新命令 |\lstinline| 将代码漂亮地打印出来,\\例如 `\lstinline!var i:integer;!' 通过 `{\rstyle|\lstinline|}|!var i:integer;!|' 排版。感叹号用于界定代码,并且可以用代码中不存在的任何字符替换;|\lstinline$var i:integer;$|会得到相同的结果。% todo 后面重新调整,用tcolor的来排
\switchcolumn[0]*%%%%%%%%%%%%
\textbf{Displayed code}
The \texttt{lstlisting} environment typesets the enclosed source code. Like
most examples, the following one shows verbatim \LaTeX\ code on the right
and the result on the left. You might take the right-hand side, put it into
the minimal file, and run it through \LaTeX.
\switchcolumn
\textbf{显示的代码}
\texttt{lstlisting}环境用于排版所包含的源代码。像大多数示例一样,下面的示例在右边展示了逐字排版的\LaTeX 代码,左边展示了结果。您可以将右侧的代码放入最小文件中,然后通过\LaTeX 运行它。
% \switchcolumn[0]*%%%%%%%%%%%%
\begin{lstsample}[lstlisting]{}{}
        \begin{lstlisting}
        for i:=maxint to 0 do
        begin
            { do nothing }
        end;

        Write('Case insensitive ');
        WritE('Pascal keywords.');
        \end{lstlisting}
\end{lstsample}
\switchcolumn[0]*%%%%%%%%%%%%
It can't be easier.
\switchcolumn
这是再容易不过了。
\end{paracol}

\begin{advise}
\begin{paracol}{2}    
\item That's not true. The name `\texttt{listing}' is shorter.
\advisespace
Indeed. But other packages already define environments with that name.
To be compatible with such packages, all commands and environments of
the \packagename{listings} package use the prefix `\texttt{lst}'.
\switchcolumn
\item 这不对。名字 `\texttt{listing}' 更短。 \advisespace 确实如此。但是,其他包已经定义了同名的环境。 为了与这些包兼容,\packagename{listings}包的所有命令和环境都使用前缀\texttt{lst}'。
\end{paracol}
\end{advise}

\begin{paracol}{2}
The environment provides an optional argument. It tells the package to
perform special tasks, for example, to print only the lines 2--5:
\switchcolumn
该环境提供了一个可选参数。它告诉包执行特殊任务,例如只打印2到5行:
\begin{lstsample}{\lstset{frame=trbl,framesep=0pt}\label{gFirstKey=ValueList}}{}
        \begin{lstlisting}[firstline=2,
                            lastline=5]
        for i:=maxint to 0 do
        begin
            { do nothing }
        end;

        Write('Case insensitive ');
        WritE('Pascal keywords.');
        \end{lstlisting}
\end{lstsample}
\end{paracol}


\begin{advise}
    \begin{paracol}{2}
\item Hold on! Where comes the frame from and what is it good for?
\advisespace
You can put frames around all listings except code snippets.
You will learn how later. The frame shows that empty lines at the end
of listings aren't printed. This is line 5 in the example.
\switchcolumn
\item 等等!这个框从哪里来的?它有什么作用?
\advisespace
除了代码片段外,你可以给所有的代码片段加上框。
稍后你会了解如何做到这一点。框表明列表末尾的空行不会被打印出来。在示例中,这是第5行。
\switchcolumn[0]*
\item Hey, you can't drop my empty lines!
\advisespace
You can tell the package not to drop them:
The key `\ikeyname{showlines}' controls these empty lines and is
described in section \ref{rTypesettingListings}. Warning: First
read ahead on how to use keys in general.
\switchcolumn
\item 嘿,你不能删除我的空行!
\advisespace
你可以告诉包不要删除它们:
键 `\ikeyname{showlines}' 控制这些空行,将在第\ref{rTypesettingListings}节中进行描述。警告:首先请提前阅读如何使用键。
\switchcolumn[0]*      
\item I get obscure error messages when using `\ikeyname{firstline}'.
\advisespace
That shouldn't happen. Make a bug report as described in section
\lstref{uTroubleshooting}.
\switchcolumn      
\item 当我使用 `\ikeyname{firstline}' 时,我收到了晦涩的错误消息。
\advisespace
不应该发生这种情况。请按照第\lstref{uTroubleshooting}节中描述的方法报告错误。
\end{paracol}
\end{advise}

\begin{paracol}{2}
\textbf{Stand alone files}
Finally we come to |\lstinputlisting|, the command used to pretty-print
stand alone files. It has one optional and one file name argument.
Note that you possibly need to specify the relative path to the file.
Here now the result is printed below the verbatim code since both together
don't fit the text width.
\switchcolumn
\textbf{独立文件}
最后,我们来看看 |\lstinputlisting| 命令,它用于漂亮地显示独立文件。它有一个可选参数和一个文件名参数。请注意,您可能需要指定文件的相对路径。现在,结果被打印在等宽代码的下方,因为它们两者加起来超过了文本宽度。
\begin{lstsample}{\lstset{comment=[l]\%,columns=fullflexible}}^^A
            {\lstset{alsoletter=\\,emph=\\lstinputlisting,emphstyle=\rstyle}^^A
            \lstaspectindex{\lstinputlisting}{}}
        \lstinputlisting[lastline=4]{listings.sty}
\end{lstsample}
% gpt 注释
% comment=[l]\% 表示注释是由 % 符号开始的,并且是行注释([l] 代表“line”)
% columns=fullflexible 使得字符间距是可变的,这样代码的排版会更接近源文件的实际显示。
% alsoletter=\\ 表示反斜线 \ 也被视作字母的一部分,这通常用于正确地高亮 TeX 命令。
% emph=\\lstinputlisting 指定 lstinputlisting 命令应该被强调显示,而 emphstyle=\rstyle 定义了强调样式,其中 \rstyle 应该在其他地方定义。
% \lstaspectindex{\lstinputlisting}{}: 这似乎是自定义命令,可能是用来将 \lstinputlisting 添加到某个索引或目录中,但这不是 listings 宏包的标准命令。

% \lstinputlisting[lastline=4]{listings.sty}: 这行代码用于实际导入并显示 listings.sty 文件的前四行。lastline=4 选项告诉 listings 宏包仅包含从文件开始到第四行的内容。
\end{paracol}

\begin{advise}
    \begin{paracol}{2}
\item The spacing is different in this example.
      \advisespace
      Yes. The two previous examples have aligned columns, i.e.~columns with
      identical numbers have the same horizontal position---this package
      makes small adjustments only. The columns in the example here are not
      aligned. This is explained in section \ref{uFixedAndFlexibleColumns}
      (keyword: full flexible column format).
\switchcolumn
\item 这个例子中的间距不同。
\advisespace
是的。前面两个例子有对齐的列,即具有相同编号的列具有相同的水平位置---这个包只进行了小的调整。此处示例中的列没有对齐。这在第\ref{uFixedAndFlexibleColumns}节中进行了解释(关键词:全灵活列格式)。(\verb|columns=fullflexible|)
    \end{paracol}
\end{advise}

\begin{paracol}{2}
\switchcolumn[0]*%%%%%%%%%%%%
Now you know all pretty-printing commands and environments. It remains
to learn the parameters which control the work of the \packagename{listings}
package. This is, however, the main task. Here are some of them.
\switchcolumn
现在你已经了解了所有的漂亮打印命令和环境。剩下的就是学习控制\packagename{listings}包工作的参数。然而,这是主要的任务。下面是其中的一些。
\end{paracol}