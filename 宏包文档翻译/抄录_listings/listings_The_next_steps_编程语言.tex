\columnratio{0.55}
\begin{paracol}{2}
\subsection{Programming languages}\label{uProgrammingLanguages}
\switchcolumn
\subsection{编程语言}
\switchcolumn[0]*%%%%%%%%%%%%
You already know how to activate programming languages---at least Pascal.
An optional parameter selects particular dialects of a language. For example,
|language=[77]Fortran| selects Fortran 77 and |language=[XSC]Pascal| does the
same for Pascal XSC. The general form is
   {\rstyle\ikeyname{language}}|=|\oarg{dialect}\meta{language}.
If you want to get rid of keyword, comment, and string detection, use
|language={}| as an argument to |\lstset| or as optional argument.
\switchcolumn
您已经知道如何激活编程语言了——至少是Pascal。可选参数选择特定方言的语言。例如,|language=[77]Fortran|选择Fortran 77,|language=[XSC]Pascal|选择Pascal XSC。一般形式是
{\rstyle\ikeyname{language}}|=|\oarg{方言}\meta{语言}。
如果您想去除关键词、注释和字符串的检测,请使用|language={}|作为|\lstset|的参数或可选参数。
\switchcolumn[0]*%%%%%%%%%%%%
Table \ref{uPredefinedLanguages} shows all predefined languages and dialects.
Use the listed names as \meta{language} and \meta{dialect}, respectively. If
no dialect or `empty' is given in the table, just don't specify a dialect.
Each underlined dialect is default; it is selected if you leave out
the optional argument. The predefined defaults are the newest language
versions or standard dialects.
^^A
^^A  Make table of predefined languages.
^^A
\switchcolumn
表\ref{uPredefinedLanguages}展示了所有预定义的语言和方言。将列出的名称用作\meta{语言}和\meta{方言}。如果表中没有给出方言或“empty”,则不指定方言。
每个带有下划线的方言是默认方言;如果省略可选参数,将选择该方言。预定义的默认值是最新的语言版本或标准方言。
% /usr/local/texlive/2022/texmf-dist/tex/latex/listings
% lstlang3.sty 
% lstlang2.sty 
% lstlang1.sty
^^A
^^A 生成预定义语言的表格。
^^A
% \switchcolumn[0]*%%%%%%%%%%%%
\let\lstlanguages\empty
\makeatletter
\@for\lst@temp:={lstlang1.sty,lstlang2.sty,lstlang3.sty}\do
   {\IfFileExists\lst@temp{}{\let\lstlanguages\relax}}
\makeatother
\ifx\lstlanguages\relax
   \PackageWarningNoLine{Listings}
       {Standard drivers not available.\MessageBreak
        Please check your installation.\MessageBreak
        Compilation aborted}
   \csname @@end\expandafter\endcsname
\fi
\lstscanlanguages\lstlanguages{lstlang1.sty,lstlang2.sty,lstlang3.sty}{}^^A
\def\topfigrule{\hrule\kern-0.4pt\relax}^^A
\let\botfigrule\topfigrule
\belowcaptionskip=\smallskipamount
\begin{table}[tbhp]
\small
\caption{%
% Predefined languages.
%          Note that some definitions are preliminary, for example HTML and XML.
%          Each underlined dialect is the default dialect.
         预定义的语言。
请注意,某些定义是初步的,例如HTML和XML。
每个带有下划线的方言是默认方言。}^^A
         \label{uPredefinedLanguages}^^A
\makeatletter
\setbox\@tempboxa\hbox{^^A
   \InputIfFileExists{listings.cfg}{\lst@InputCatcodes}{}}^^A
\lstprintlanguages\lstlanguages
\end{table}
^^A
^^A end of table
^^A
\lstset{defaultdialect=[doc]Pascal}^^A restore
\end{paracol}

\begin{advise}
\columnratio{0.55}
\begin{paracol}{2}
\item How can I define default dialects?
      \advisespace
      Check section \ref{rLanguagesAndStyles} for `\keyname{defaultdialect}'.
\switchcolumn
\item 如何定义默认方言?
\advisespace
请参阅第 \ref{rLanguagesAndStyles} 节中的 `\keyname{defaultdialect}'。
\switchcolumn[0]*%%%%%%%%%%%%
\item I have C code mixed with assembler lines. Can \packagename{listings}
      pretty-print such source code, i.e.~highlight keywords and comments of
      both languages?
      \advisespace
      `\ikeyname{alsolanguage}|=|\oarg{dialect}\meta{language}' selects a
      language additionally to the active one. So you only have to write a
      language definition for your assembler dialect, which doesn't interfere
      with the definition of C, say. Moreover you might want to use the key
      `\keyname{classoffset}' described in section \ref{rLanguagesAndStyles}.
\switchcolumn
\item 我的C代码中混有汇编行。能否使用\packagename{listings}美化这样的源代码,即突出显示两种语言的关键词和注释?
\advisespace
`\ikeyname{alsolanguage}|=|\oarg{方言}\meta{语言}' 除了活动语言外,还选择另一种语言。因此,您只需为汇编方言编写语言定义即可,该定义不会与C的定义相冲突。此外,您可能还想使用第\ref{rLanguagesAndStyles}节中描述的 `\keyname{classoffset}' 键。
\switchcolumn[0]*%%%%%%%%%%%%
\item How can I define my own language?
      \advisespace
      This is discussed in section \ref{rLanguageDefinitions}. And if you
      think that other people could benefit by your definition, you might
      want to send it to the address in section \ref{uSoftwareLicense}.
      Then it will be published under the \LaTeX\ Project Public License.
\switchcolumn
\item 如何定义自己的语言?
\advisespace
这在第\ref{rLanguageDefinitions}节中讨论。如果您认为其他人可以从您的定义中受益,您可以将其发送到第\ref{uSoftwareLicense}节中的地址。然后它将在 \LaTeX\ 项目公共许可证下发布。
\end{paracol}
\end{advise}
\columnratio{0.55}
\begin{paracol}{2}
Note that the arguments \meta{language} and \meta{dialect} are case
insensitive and that spaces have no effect.

There is at least one language (VDM, Vienna Development Language,
\url{http://www.vdmportal.org}) which is not directly supported by the
\packagename{listings} package. It needs a package for its own:
\packagename{vdmlisting}. On the other hand \packagename{vdmlisting} uses
the \packagename{listings} package and so it should be mentioned in this
context.


\subsubsection{Preferences}\label{uPreferences}

Sometimes authors of language support provide their own configuration
preferences. These may come either from their personal experience or
from the settings in an IDE and can be defined as a \packagename{listings}
style. From version 1.5b of the \packagename{listings} package on these
styles are provided as files with the name
|listings-|\meta{language}|.prf|, \meta{language} is the name of the
supported programming language in lowercase letters.

So if an user of the \packagename{listings} package wants to use these
preferences, she/he can say for example when using Python
\begin{quote}
    |\input{listings-python.prf}|
\end{quote}
at the end of her/his |listings.cfg| configuration file as long as the
file |listings-python.prf| resides in the \TeX{} search path. Of course
that file can be changed according to the user's preferences.

At the moment there are five such preferences files:
\begin{enumerate}
  \item |listings-acm.prf|
  \item |listings-bash.prf|
  \item |listings-fortran.prf|
  \item |listings-lua.prf|
  \item |listings-python.prf|
\end{enumerate}
All contributors are invited to supply more personal preferences.

\end{paracol}