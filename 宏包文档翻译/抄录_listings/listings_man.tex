\PassOptionsToPackage{no-math}{fontspec}%禁用了使用fontspec宏包中的数学字体功能。
\PassOptionsToPackage{AutoFakeBold=true,AutoFakeSlant=true}{xeCJK}%让xeCJK宏包自动产生伪粗体和伪斜体效果。

\documentclass[a4paper]{ltxdoc}
\usepackage[heading=true
,scheme=chinese%中文方案
,fontset=none%不使用默认的字体设置
,space=auto%自动调整中英文间距
]{ctex}
\setCJKmainfont{FangZhengShuSong-GBK-1.ttf}[Path=/Users/virhuiai/hlProjects/Latex-Typesetting-Hub/font/方正/]%设置文本的中文有衬线字体
\setCJKsansfont{FangZhengHeiTi-GBK-1.ttf}[Path=/Users/virhuiai/hlProjects/Latex-Typesetting-Hub/font/方正/]%设置文本的中文无衬线字体为
\setCJKmonofont{FangZhengFangSong-GBK-1.ttf}[Path=/Users/virhuiai/hlProjects/Latex-Typesetting-Hub/font/方正/] %设置文本的中文等宽字体 
\usepackage[a3paper,landscape]{geometry}
\usepackage{paracol}

\DisableCrossrefs
\OnlyDescription

\usepackage{lstdoc,textcomp}
\usepackage{mdframed}           % frames for external files
\usepackage{moreverb}           % writing external files
\usepackage{xcolor}             % because of colouring the background
 
\makeindex

\begin{document}
\DocInput{listings.dtx.pre}

^^A
^^A  The long awaited beginning of documentation
^^A =============================================
^^A
% \newbox\abstractbox
% \setbox\abstractbox=\vbox{
% 	\begin{abstract}
% 	The \packagename{listings} package is a source code printer for \LaTeX.
% 	You can typeset stand alone files as well as listings with an environment
%   similar to \texttt{verbatim} as well as you can print code snippets using
%   a command similar to |\verb|.
% 	Many parameters control the output and if your preferred programming
%   language isn't already supported, you can make your own definition.
% 	\end{abstract}}

\title{\vspace*{-2\baselineskip}The \textsf{Listings} Package}
\author{Copyright 1996--2004, Carsten Heinz%
   \\ Copyright 2006--2007, Brooks Moses
   \\ Copyright 2013--, Jobst Hoffmann
   \\ Maintainer: Jobst Hoffmann\thanks{Jobst %
      Hoffmann became the maintainer of the \packagename{listings}
      package in 2013; see the Preface for details.}~ %
   \textless\lstemail\textgreater\and 翻译:virhuiai}
\date{2020/03/24\enspace\enspace Version 1.8d\ %\box\abstractbox
}
\ifhyper
  \hypersetup{pdftitle=The Listings Package,
              pdfsubject=Package guide,
              pdfauthor=Jobst Hoffmann <j.hoffmann(at)fh-aachen.de>,%
              pdfkeywords={source code formatter, programming languages}}
\fi

% \csname @twocolumntrue\endcsname
\maketitle

\begin{paracol}{2}
\renewcommand{\abstractname}{Abstract}
\begin{abstract}
The \packagename{listings} package is a source code printer for \LaTeX.
You can typeset stand alone files as well as listings with an environment
similar to \texttt{verbatim} as well as you can print code snippets using
a command similar to |\verb|.
Many parameters control the output and if your preferred programming
language isn't already supported, you can make your own definition.
\end{abstract}
\switchcolumn
\renewcommand{\abstractname}{摘要}
\begin{abstract}
\texttt{listings}宏包是一个用于排版\LaTeX 的源代码排版工具。您可以排版独立的文件,也可以使用类似于\texttt{verbatim}的环境排版代码列表,还可以使用类似于 |\verb| 的命令打印代码片段。许多参数可以控制输出,如果您的首选编程语言尚未支持,您可以自定义定义。
\end{abstract}

\end{paracol}


^^A \enlargethispage{2\baselineskip}
\csname @starttoc\endcsname{toc}
% \onecolumn




    % \DocInput{listings.dtx}
\end{document}