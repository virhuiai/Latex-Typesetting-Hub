\PassOptionsToPackage{no-math}{fontspec}%禁用了使用fontspec宏包中的数学字体功能。
\PassOptionsToPackage{AutoFakeBold=true,AutoFakeSlant=true}{xeCJK}%让xeCJK宏包自动产生伪粗体和伪斜体效果。

\documentclass[a4paper]{ltxdoc}
\usepackage[heading=true
,scheme=chinese%中文方案
,fontset=none%不使用默认的字体设置
,space=auto%自动调整中英文间距
]{ctex}
\setCJKmainfont{FangZhengShuSong-GBK-1.ttf}[Path=/Users/virhuiai/hlProjects/Latex-Typesetting-Hub/font/方正/]%设置文本的中文有衬线字体
\setCJKsansfont{FangZhengHeiTi-GBK-1.ttf}[Path=/Users/virhuiai/hlProjects/Latex-Typesetting-Hub/font/方正/]%设置文本的中文无衬线字体为
\setCJKmonofont{FangZhengFangSong-GBK-1.ttf}[Path=/Users/virhuiai/hlProjects/Latex-Typesetting-Hub/font/方正/] %设置文本的中文等宽字体 
\usepackage[a3paper,landscape]{geometry}
\usepackage{paracol}
\columnsep=2em 


\DisableCrossrefs
\OnlyDescription

\usepackage{lstdoc,textcomp}
\usepackage{mdframed}           % frames for external files
\usepackage{moreverb}           % writing external files
\usepackage{xcolor}             % because of colouring the background
 
\makeindex

\usepackage{multicol} 
\begin{document}
\def\lstemail{\href{mailto:j.hoffmann@fh-aachen.de}{\texttt{j.hoffmann(at)fh-aachen.de}}}

\DocInput{listings.dtx.pre}
% 
^^A
^^A  The long awaited beginning of documentation
^^A =============================================
^^A
% \newbox\abstractbox
% \setbox\abstractbox=\vbox{
% 	\begin{abstract}
% 	The \packagename{listings} package is a source code printer for \LaTeX.
% 	You can typeset stand alone files as well as listings with an environment
%   similar to \texttt{verbatim} as well as you can print code snippets using
%   a command similar to |\verb|.
% 	Many parameters control the output and if your preferred programming
%   language isn't already supported, you can make your own definition.
% 	\end{abstract}}

\title{\vspace*{-2\baselineskip}The \textsf{Listings} Package}
\author{Copyright 1996--2004, Carsten Heinz%
   \\ Copyright 2006--2007, Brooks Moses
   \\ Copyright 2013--, Jobst Hoffmann
   \\ Maintainer: Jobst Hoffmann\thanks{Jobst %
      Hoffmann became the maintainer of the \packagename{listings}
      package in 2013; see the Preface for details.}~ %
   \textless\lstemail\textgreater\and 翻译:virhuiai}
\date{2020/03/24\enspace\enspace Version 1.8d\ %\box\abstractbox
}
\ifhyper
  \hypersetup{pdftitle=The Listings Package,
              pdfsubject=Package guide,
              pdfauthor=Jobst Hoffmann <j.hoffmann(at)fh-aachen.de>,%
              pdfkeywords={source code formatter, programming languages}}
\fi

% \csname @twocolumntrue\endcsname
\maketitle

\begin{paracol}{2}
\renewcommand{\abstractname}{Abstract}
\begin{abstract}
The \packagename{listings} package is a source code printer for \LaTeX.
You can typeset stand alone files as well as listings with an environment
similar to \texttt{verbatim} as well as you can print code snippets using
a command similar to |\verb|.
Many parameters control the output and if your preferred programming
language isn't already supported, you can make your own definition.
\end{abstract}
\switchcolumn
\renewcommand{\abstractname}{摘要}
\begin{abstract}
\texttt{listings}宏包是一个用于排版\LaTeX 的源代码排版工具。您可以排版独立的文件,也可以使用类似于\texttt{verbatim}的环境排版代码列表,还可以使用类似于 |\verb| 的命令打印代码片段。许多参数可以控制输出,如果您的首选编程语言尚未支持,您可以自定义定义。
\end{abstract}

\end{paracol}


^^A \enlargethispage{2\baselineskip}
\csname @starttoc\endcsname{toc}
% \onecolumn


% \columnratio{0.55}
\begin{paracol}{2}
\section*{Preface}
\switchcolumn
\section*{前言}
\switchcolumn[0]*%%%%%%%%%%%%
\textbf{Transition of package maintenance}
The \TeX\ world lost contact with Carsten Heinz in late 2004, shortly after
he released version 1.3b of the \packagename{listings} package.  After many
attempts to reach him had failed, Hendri Adriaens took over maintenance of
the package in accordance with the LPPL's procedure for abandoned packages.
He then passed the maintainership of the package to Brooks Moses, who had
volunteered for the position while this procedure was going through. The
result is known as listings version 1.4.
\switchcolumn
\textbf{包维护的转变}
在2004年底,\TeX\ 领域失去了与Carsten Heinz的联系,他在发布了\packagename{listings}宏包的1.3b版本后不久便失去了联系。在多次联系失败后,Hendri Adriaens根据LPPL对废弃包的处理程序接管了该包的维护工作。然后,他将包的维护权移交给了Brooks Moses,在此过程中,Brooks Moses自愿担任了这个职位。结果就是现在的listings 1.4版本。
\switchcolumn[0]*%%%%%%%%%%%%
This release, version 1.5, is a minor maintenance release since
I accepted maintainership of the package.  I would like to thank Stephan
Hennig who supported the Lua language definitions. He is the one who
asked for the integration of a new language and gave the impetus to me to
become the maintainer of this package.
\switchcolumn
这个发布的版本,1.5版,是我接手维护该宏包之后的一个小型维护版本。我要感谢Stephan Hennig对Lua语言定义的支持。他是那个提出整合新语言并给我成为这个宏包的维护者提供动力的人。
\switchcolumn[0]*%%%%%%%%%%%%
\textbf{News and changes}
Version 1.5 is the fifth bugfix release.  There are no changes
in this version, but two extensions: support of modern Fortran (2003,
2008) and Lua.
\switchcolumn
\textbf{新闻和变化}
1.5版本是第五个修复错误版本。这个版本没有任何变化,只有两个扩展:对现代Fortran(2003、2008)和Lua的支持。
\switchcolumn[0]*%%%%%%%%%%%%
\vfill
\textbf{Thanks}
There are many people I have to thank for fruitful communication, posting
their ideas, giving error reports, adding programming languages to
\texttt{lstdrvrs.dtx}, and so on. Their names are listed in section
\ref{uClosingAndCredits}.
\switchcolumn
\textbf{感谢}
有很多人我需要感谢他们的积极交流、发表意见、提供错误报告、添加编程语言到\texttt{lstdrvrs.dtx}等等。他们的名字将在\ref{uClosingAndCredits}节中列出。
\switchcolumn[0]*%%%%%%%%%%%%
\textbf{Trademarks}
Trademarks appear throughout this documentation without any trademark
symbol; they are the property of their respective trademark owner.
There is no intention of infringement; the usage is to the benefit of the
trademark owner.
\switchcolumn
\textbf{商标}
本文档中的商标没有使用任何商标符号,它们是其各自商标所有者的财产。没有侵权的意图;使用是为了商标所有者的利益。
\end{paracol}
% \begin{paracol}{2}
\part{User's guide}
\switchcolumn
\part{用户指南}
\switchcolumn[0]*%%%%%%%%%%%%
\section{Getting started}\label{uGettingStarted}
\switchcolumn
\section{开始使用}
\end{paracol}
% \columnratio{0.55}
\begin{paracol}{2}
\subsection{A minimal file}\label{uAMinimalFile}
\switchcolumn
\subsection{一个最小的文件}
\switchcolumn[0]*%%%%%%%%%%%%
Before using the \packagename{listings} package, you should be familiar with
the \LaTeX\ typesetting system. You need not to be an expert.
Here is a minimal file for \packagename{listings}.
\switchcolumn
在使用\packagename{listings}宏包之前,您应该熟悉\LaTeX 排版系统。您不需要成为专家。下面是一个\packagename{listings}宏包的最小文件示例。
\switchcolumn[0]*%%%%%%%%%%%%
\begin{verbatim}
    % \documentclass{article}
    % \usepackage{listings}

    % \begin{document}
    % \lstset{language=Pascal}

    %   % Insert Pascal examples here.

    % \end{document}\end{verbatim}
\switchcolumn
\begin{verbatim}
    % \documentclass{article}
    % \usepackage{listings}

    % \begin{document}
    % \lstset{language=Pascal}

    % % 在此处插入Pascal示例代码。

    % \end{document}\end{verbatim}
\switchcolumn[0]*%%%%%%%%%%%%
Now type in this first example and run it through \LaTeX.
\switchcolumn
现在输入这个第一个示例并运行它通过\LaTeX。
\end{paracol}




\begin{advise}
\begin{paracol}{2}    
\item Must I do that really?
    \advisespace
    Yes and no. Some books about programming say this is good.
    What a mistake! Typing takes time---which is wasted if the code is clear to
    you. And if you need that time to understand what is going on, the
    author of the book should reconsider the concept of presenting the
    crucial things---you might want to say that about this guide even---or
    you're simply inexperienced with programming. If only the latter case
    applies, you should spend more time on reading (good) books about
    programming, (good) documentations, and (good) source code from other
    people. Of course you should also make your own experiments.
    You will learn a lot. However, running the example through \LaTeX\
    shows whether the \packagename{listings} package is installed correctly.
    \switchcolumn
    \item 我真的需要这样做吗?
    \advisespace
    既是又非。有些关于编程的书籍说这样做是好的。这是一个错误!输入需要时间,如果代码对您来说很清晰,那么这些时间就是浪费的。如果您需要时间来理解正在发生的事情,那么书的作者应该重新考虑介绍关键内容的概念---甚至可能会对这个指南说出这样的话---或者您对编程经验不够丰富。如果只是后一种情况,您应该花更多时间阅读(好的)关于编程的书籍,(好的)文档和其他人的(好的)源代码。当然,您也应该进行自己的实验。您将学到很多。然而,通过\LaTeX 运行示例可以显示\packagename{listings}宏包是否正确安装。
\switchcolumn[0]*
\item The example doesn't work.
    \advisespace
    Are the two packages \packagename{listings} and \packagename{keyval}
    installed on your system? Consult the administration tool of your
    \TeX\ distribution, your system administrator, the local \TeX\ and
    \LaTeX\ guides, a \TeX\ FAQ, and section \ref{rInstallation}---in
    that order. If you've checked \emph{all} these sources and are
    still helpless, you might want to write a post to a \TeX\ newsgroup
    like \texttt{comp.text.tex}.
    \switchcolumn
    \item 这个示例不起作用。
    \advisespace
    您的系统上是否安装了两个包\packagename{listings}和\packagename{keyval}?请查阅您的\TeX 发行版的管理工具、系统管理员、本地\TeX 和\LaTeX 指南、\TeX 常见问题解答,并按照这个顺序查找。如果您已经检查了\emph{所有}这些来源,仍然无助,您可能想在\TeX 新闻组(如\texttt{comp.text.tex})上发布一篇帖子。
    \switchcolumn[0]*    
\item Should I read the software license before using the package?
    \advisespace
    Yes, but read this \emph{Getting started} section first to decide
    whether you are willing to use the package.^^A ;-)
    \switchcolumn
    \item 在使用该宏包之前,我应该阅读软件许可证吗?
\advisespace
是的,但在使用该宏包之前,请先阅读本“入门”部分,以决定您是否愿意使用该宏包。^^A ;-)
\end{paracol}
\end{advise}
% \columnratio{0.55}
\begin{paracol}{2}
\subsection{Typesetting listings}
\switchcolumn
\subsection{排版代码}
\switchcolumn[0]*%%%%%%%%%%%%
Three types of source codes are supported: code snippets, code segments, and
listings of stand alone files.  Snippets are placed inside paragraphs and the
others as separate paragraphs---the difference is the same as between text
style and display style formulas.
\switchcolumn
支持三种类型的源代码:代码片段、代码段和独立文件的列表。片段放置在段落内,其他两种放置为独立段落---区别与文本样式和展示样式公式之间的区别。
\end{paracol}
\begin{advise}
\columnratio{0.55}
\begin{paracol}{2}

\item No matter what kind of source you have, if a listing contains national
    characters like \'e, \L, \"a, or whatever, you must tell the
    package about it! Section \lstref{uSpecialCharacters} discusses this issue.
    \switchcolumn
    \item 不管您有什么样的源代码,如果列表中包含如'e、\L、"a等国际字符,您必须告诉宏包!请参阅\lstref{uSpecialCharacters}节讨论此问题。      
\end{paracol}
\end{advise}

\begin{paracol}{2}
\switchcolumn[0]*%%%%%%%%%%%%
\textbf{Code snippets}
The well-known \LaTeX\ command |\verb| typesets code snippets verbatim.
The new command |\lstinline| pretty-prints the code, for example
`\lstinline!var i:integer;!' is typeset by
`{\rstyle|\lstinline|}|!var i:integer;!|'. The exclamation marks delimit
the code and can be replaced by any character not in the code;
|\lstinline$var i:integer;$| gives the same result.
\switchcolumn
\textbf{代码片段}
众所周知,\LaTeX 命令|\verb|以逐字方式排版代码片段。新命令 |\lstinline| 将代码漂亮地打印出来,\\例如 `\lstinline!var i:integer;!' 通过 `{\rstyle|\lstinline|}|!var i:integer;!|' 排版。感叹号用于界定代码,并且可以用代码中不存在的任何字符替换;|\lstinline$var i:integer;$|会得到相同的结果。% todo 后面重新调整,用tcolor的来排
\switchcolumn[0]*%%%%%%%%%%%%
\textbf{Displayed code}
The \texttt{lstlisting} environment typesets the enclosed source code. Like
most examples, the following one shows verbatim \LaTeX\ code on the right
and the result on the left. You might take the right-hand side, put it into
the minimal file, and run it through \LaTeX.
\switchcolumn
\textbf{显示的代码}
\texttt{lstlisting}环境用于排版所包含的源代码。像大多数示例一样,下面的示例在右边展示了逐字排版的\LaTeX 代码,左边展示了结果。您可以将右侧的代码放入最小文件中,然后通过\LaTeX 运行它。
% \switchcolumn[0]*%%%%%%%%%%%%
\begin{lstsample}[lstlisting]{}{}
        \begin{lstlisting}
        for i:=maxint to 0 do
        begin
            { do nothing }
        end;

        Write('Case insensitive ');
        WritE('Pascal keywords.');
        \end{lstlisting}
\end{lstsample}
\switchcolumn[0]*%%%%%%%%%%%%
It can't be easier.
\switchcolumn
这是再容易不过了。
\end{paracol}

\begin{advise}
\begin{paracol}{2}    
\item That's not true. The name `\texttt{listing}' is shorter.
\advisespace
Indeed. But other packages already define environments with that name.
To be compatible with such packages, all commands and environments of
the \packagename{listings} package use the prefix `\texttt{lst}'.
\switchcolumn
\item 这不对。名字 `\texttt{listing}' 更短。 \advisespace 确实如此。但是,其他包已经定义了同名的环境。 为了与这些包兼容,\packagename{listings}包的所有命令和环境都使用前缀\texttt{lst}'。
\end{paracol}
\end{advise}

\begin{paracol}{2}
The environment provides an optional argument. It tells the package to
perform special tasks, for example, to print only the lines 2--5:
\switchcolumn
该环境提供了一个可选参数。它告诉包执行特殊任务,例如只打印2到5行:
\begin{lstsample}{\lstset{frame=trbl,framesep=0pt}\label{gFirstKey=ValueList}}{}
        \begin{lstlisting}[firstline=2,
                            lastline=5]
        for i:=maxint to 0 do
        begin
            { do nothing }
        end;

        Write('Case insensitive ');
        WritE('Pascal keywords.');
        \end{lstlisting}
\end{lstsample}
\end{paracol}


\begin{advise}
    \begin{paracol}{2}
\item Hold on! Where comes the frame from and what is it good for?
\advisespace
You can put frames around all listings except code snippets.
You will learn how later. The frame shows that empty lines at the end
of listings aren't printed. This is line 5 in the example.
\switchcolumn
\item 等等!这个框从哪里来的?它有什么作用?
\advisespace
除了代码片段外,你可以给所有的代码片段加上框。
稍后你会了解如何做到这一点。框表明列表末尾的空行不会被打印出来。在示例中,这是第5行。
\switchcolumn[0]*
\item Hey, you can't drop my empty lines!
\advisespace
You can tell the package not to drop them:
The key `\ikeyname{showlines}' controls these empty lines and is
described in section \ref{rTypesettingListings}. Warning: First
read ahead on how to use keys in general.
\switchcolumn
\item 嘿,你不能删除我的空行!
\advisespace
你可以告诉包不要删除它们:
键 `\ikeyname{showlines}' 控制这些空行,将在第\ref{rTypesettingListings}节中进行描述。警告:首先请提前阅读如何使用键。
\switchcolumn[0]*      
\item I get obscure error messages when using `\ikeyname{firstline}'.
\advisespace
That shouldn't happen. Make a bug report as described in section
\lstref{uTroubleshooting}.
\switchcolumn      
\item 当我使用 `\ikeyname{firstline}' 时,我收到了晦涩的错误消息。
\advisespace
不应该发生这种情况。请按照第\lstref{uTroubleshooting}节中描述的方法报告错误。
\end{paracol}
\end{advise}

\begin{paracol}{2}
\textbf{Stand alone files}
Finally we come to |\lstinputlisting|, the command used to pretty-print
stand alone files. It has one optional and one file name argument.
Note that you possibly need to specify the relative path to the file.
Here now the result is printed below the verbatim code since both together
don't fit the text width.
\switchcolumn
\textbf{独立文件}
最后,我们来看看 |\lstinputlisting| 命令,它用于漂亮地显示独立文件。它有一个可选参数和一个文件名参数。请注意,您可能需要指定文件的相对路径。现在,结果被打印在等宽代码的下方,因为它们两者加起来超过了文本宽度。
\begin{lstsample}{\lstset{comment=[l]\%,columns=fullflexible}}^^A
            {\lstset{alsoletter=\\,emph=\\lstinputlisting,emphstyle=\rstyle}^^A
            \lstaspectindex{\lstinputlisting}{}}
        \lstinputlisting[lastline=4]{listings.sty}
\end{lstsample}
% gpt 注释
% comment=[l]\% 表示注释是由 % 符号开始的,并且是行注释([l] 代表“line”)
% columns=fullflexible 使得字符间距是可变的,这样代码的排版会更接近源文件的实际显示。
% alsoletter=\\ 表示反斜线 \ 也被视作字母的一部分,这通常用于正确地高亮 TeX 命令。
% emph=\\lstinputlisting 指定 lstinputlisting 命令应该被强调显示,而 emphstyle=\rstyle 定义了强调样式,其中 \rstyle 应该在其他地方定义。
% \lstaspectindex{\lstinputlisting}{}: 这似乎是自定义命令,可能是用来将 \lstinputlisting 添加到某个索引或目录中,但这不是 listings 宏包的标准命令。

% \lstinputlisting[lastline=4]{listings.sty}: 这行代码用于实际导入并显示 listings.sty 文件的前四行。lastline=4 选项告诉 listings 宏包仅包含从文件开始到第四行的内容。
\end{paracol}

\begin{advise}
    \begin{paracol}{2}
\item The spacing is different in this example.
      \advisespace
      Yes. The two previous examples have aligned columns, i.e.~columns with
      identical numbers have the same horizontal position---this package
      makes small adjustments only. The columns in the example here are not
      aligned. This is explained in section \ref{uFixedAndFlexibleColumns}
      (keyword: full flexible column format).
\switchcolumn
\item 这个例子中的间距不同。
\advisespace
是的。前面两个例子有对齐的列,即具有相同编号的列具有相同的水平位置---这个包只进行了小的调整。此处示例中的列没有对齐。这在第\ref{uFixedAndFlexibleColumns}节中进行了解释(关键词:全灵活列格式)。(\verb|columns=fullflexible|)
    \end{paracol}
\end{advise}

\begin{paracol}{2}
\switchcolumn[0]*%%%%%%%%%%%%
Now you know all pretty-printing commands and environments. It remains
to learn the parameters which control the work of the \packagename{listings}
package. This is, however, the main task. Here are some of them.
\switchcolumn
现在你已经了解了所有的漂亮打印命令和环境。剩下的就是学习控制\packagename{listings}包工作的参数。然而,这是主要的任务。下面是其中的一些。
\end{paracol}
% \columnratio{0.55}
\begin{paracol}{2}
\subsection{Figure out the appearance}\label{gFigureOutTheAppearance}
\switchcolumn
\subsection{确定外观}
\switchcolumn[0]*%%%%%%%%%%%%
Keywords are typeset bold, comments in italic shape, and spaces in strings
appear as \textvisiblespace. You don't like these settings? Look at this:
\switchcolumn
关键字以粗体字显示,注释以斜体显示,字符串中的空格显示为 \textvisiblespace 。你不喜欢这些设置?看看这个:

\ifcolor
\begin{lstxsample}[basicstyle,keywordstyle,identifierstyle,commentstyle,^^A
      stringstyle,showstringspaces]
            \lstset{% general command to set parameter(s)
                basicstyle=\small,          % print whole listing small
                keywordstyle=\color{black}\bfseries\underbar,
                                            % underlined bold black keywords
                identifierstyle=,           % nothing happens
                commentstyle=\color{white}, % white comments
                stringstyle=\ttfamily,      % typewriter type for strings
                showstringspaces=false}     % no special string spaces
\end{lstxsample}
\else
\begin{lstxsample}[basicstyle,keywordstyle,identifierstyle,commentstyle,^^A
      stringstyle,showstringspaces]
            \lstset{% general command to set parameter(s)
            basicstyle=\small,          % print whole listing small
            keywordstyle=\bfseries\underbar,
                                        % underlined bold keywords
            identifierstyle=,           % nothing happens
            commentstyle=\itshape,      % default
            stringstyle=\ttfamily,      % typewriter type for strings
            showstringspaces=false}     % no special string spaces
\end{lstxsample}
\fi
\begin{lstsample}{}{}
        \begin{lstlisting}
        for i:=maxint to 0 do
        begin
            { do nothing }
        end;

        Write('Case insensitive ');
        WritE('Pascal keywords.');
        \end{lstlisting}
\end{lstsample}
\end{paracol}

% \ifcolor
\begin{advise}
\columnratio{0.55}
\begin{paracol}{2}
\item You've requested white coloured comments, but I can see the comment
      on the left side.
\switchcolumn
\item 您要求白色的注释,但我可以看到左侧的注释。

      \advisespace
\switchcolumn[0]*
      There are a couple of possible reasons:
      (1) You've printed the documentation on nonwhite paper.
      (2) If you are viewing this documentation as a \texttt{.dvi}-file, your
          viewer seems to have problems with colour specials. Try to print
          the page on white paper.
      (3) If a printout on white paper shows the comment, the colour
          specials aren't suitable for your printer or printer driver.
          Recreate the documentation and try it again---and ensure that
          the \packagename{color} package is well-configured.
\switchcolumn
有几个可能的原因:
(1)您将文档打印在非白色纸上。
(2)如果您将此文档视为.dvi文件查看,您的查看器似乎无法处理颜色特殊字符。尝试在白纸上打印页面。
(3)如果白纸上的打印显示注释,则颜色特殊字符不适用于您的打印机或打印机驱动程序。重新创建文档并重试-确保\packagename{color}包配置正确。
\end{paracol}
\end{advise}
% \fi

\columnratio{0.55}
\begin{paracol}{2}
The styles use two different kinds of commands. |\ttfamily| and |\bfseries|
both take no arguments but |\underbar| does; it underlines the following
argument. In general, the \emph{very last} command may read exactly one
argument, namely some material the package typesets. There's one exception.
The last command of \ikeyname{basicstyle} \emph{must not} read any
tokens---or you will get deep in trouble.
\switchcolumn
样式使用两种不同类型的命令。|\ttfamily| 和 |\bfseries| 都不需要参数,但 |\underbar| 需要一个参数;它会给下一个参数加下划线。一般来说,\emph{最后一个}命令可能读取一个参数,即该包排版的一些内容。有一个例外。\ikeyname{basicstyle}的最后一个命令\emph{不能}读取任何记号,否则会遇到麻烦。
\end{paracol}

\begin{advise}
\begin{paracol}{2} 
\item `|basicstyle=\small|' looks fine, but comments look really bad with
      `|commentstyle=\tiny|' and empty basic style, say.
      \advisespace
      Don't use different font sizes in a single listing.
\switchcolumn
\item `|basicstyle=\small|' 看起来不错,但是 `|commentstyle=\tiny|' 和空的基本样式的注释看起来真的很糟糕。
\advisespace
在单个源码列表中不要使用不同的字体大小。
\switchcolumn[0]*
\item But I really want it!
\switchcolumn
\item 但我真的想要它!
      \advisespace
\switchcolumn[0]*
      No, you don't.
^^A       The package adjusts internal data after selecting the basic style at
^^A       the beginning of each listing. This is a problem if you change the
^^A       font size for comments or strings, for example.
^^A       Section \ref{rColumnAlignment} shows how to overcome this.
^^A       But once again: Don't use different font sizes in a single listing
^^A       unless you really know what you are doing.
\switchcolumn
不,你不需要。
^^A 每个列表开始时,该包在选择基本样式后会调整内部数据。如果您更改了注释或字符串的字体大小,这是一个问题,例如。
^^A 第\ref{rColumnAlignment}节介绍了如何解决此问题。
^^A 但再次强调:在单个列表中不要使用不同的字体大小,除非您确切知道自己在做什么。
\end{paracol}
\end{advise}


\columnratio{0.55}
\begin{paracol}{2}
\textbf{Warning}\label{wStrikingStyles}
You should be very careful with striking styles; the recent example is rather
moderate---it can get horrible. \emph{Always use decent highlighting.}
Unfortunately it is difficult to give more recommendations since they depend
on the type of document you're creating. Slides or other presentations often
require more striking styles than books, for example.
In the end, it's \emph{you} who have to find the golden mean!
\switchcolumn
\textbf{警告}\label{wStrikingStyles}
使用引人注目的样式时要非常小心;最近的例子比较温和-它可能变得非常糟糕。\emph{始终使用得体的高亮显示}。不幸的是,很难给出更多建议,因为它们取决于您创建的文档类型。幻灯片或其他演示文稿通常需要比书籍更引人注目的样式。最终,是\emph{您}必须找到中庸之道!
\switchcolumn[0]*%%%%%%%%%%%%
\subsection{Seduce to use}\label{gSeduceToUse}
\switchcolumn
\subsection{引诱使用}
\switchcolumn[0]*%%%%%%%%%%%%
You know all pretty-printing commands and some main parameters. Here now
comes a small and incomplete overview of other features. The table of
contents and the index also provide information.
\switchcolumn
您已经了解了所有漂亮的抄录命令和一些主要参数。现在,这里是其他功能的简要和不完整的概述。目录和索引也提供了信息。
\switchcolumn[0]*%%%%%%%%%%%%
\textbf{Line numbers}
are available for all displayed listings, e.g.~tiny numbers on the left, each
second line, with 5pt distance to the listing:
\switchcolumn
\textbf{行号}
对所有显示的列表都可用,例如:左侧的微小数字,每隔两行,与列表之间的5pt距离:
\begin{lstxsample}[numbers,numberstyle,stepnumber,numbersep]
        \lstset{numbers=left, numberstyle=\tiny, stepnumber=2, numbersep=5pt}
\end{lstxsample}
\begin{lstsample}{}{}
        \begin{lstlisting}
        for i:=maxint to 0 do
        begin
            { do nothing }
        end;

        Write('Case insensitive ');
        WritE('Pascal keywords.');
        \end{lstlisting}
\end{lstsample}
\end{paracol}

\begin{advise}
\begin{paracol}{2}    
\item I can't get rid of line numbers in subsequent listings.
\advisespace
`|numbers=none|' turns them off.
\switchcolumn
\item 我无法在后续的列表中去掉行号。
\advisespace
`|numbers=none|' 可将它们关闭。
\switchcolumn[0]*
\item Can I use these keys in the optional arguments?
\advisespace
Of course. Note that optional arguments modify values for one
particular listing only: you change the appearance, step or distance
of line numbers for a single listing. The previous values are
restored afterwards.
\switchcolumn
\item 我可以在可选参数中使用这些键吗?
\advisespace
当然可以。请注意,可选参数仅修改一个特定列表的值:您可以更改单个列表的行号外观、步长或距离。之前的值在此之后会恢复。
\end{paracol}
\end{advise}

\columnratio{0.55}
\begin{paracol}{2}
The environment allows you to interrupt your listings: you can end a listing
and continue it later with the correct line number even if there are other
listings in between. Read section \ref{uLineNumbers} for a thorough
discussion.
\switchcolumn
该环境允许您中断列表:即使中间有其他列表,您也可以结束一个列表并在以后继续,而且行号也是正确的。详细讨论请参阅第\ref{uLineNumbers}节。
\switchcolumn[0]*%%%%%%%%%%%%
\textbf{Floating listings}
Displayed listings may float:
\switchcolumn
\textbf{浮动列表}
显示的列表可能浮动:
\begin{lstsample}{\lstset{frame=tb}}{}
        \begin{lstlisting}[float,caption=A floating example]
        for i:=maxint to 0 do
        begin
            { do nothing }
        end;

        Write('Case insensitive ');
        WritE('Pascal keywords.');
        \end{lstlisting}
\end{lstsample}
\switchcolumn[0]*%%%%%%%%%%%%
Don't care about the parameter \ikeyname{caption} now. And if you put the
example into the minimal file and run it through \LaTeX, please don't wonder:
you'll miss the horizontal rules since they are described elsewhere.
\switchcolumn
现在不要关心参数\ikeyname{caption}。如果将此示例放入最小文件中并通过\LaTeX 运行它,请不要奇怪:您将缺少水平线,因为它们在其他地方被描述。
\end{paracol}

\begin{advise}
\begin{paracol}{2}    
\item \LaTeX's float mechanism allows one to determine the placement of floats.
      How can I do that with these?
      \advisespace
      You can write `|float=tp|', for example.
\switchcolumn
\item \LaTeX 的浮动机制允许确定浮动位置。我如何使用这些?
\advisespace
例如,您可以编写 `|float=tp|'。
\end{paracol}
\end{advise}

\begin{paracol}{2}
\textbf{Other features}
There are still features not mentioned so far: automatic breaking of long
lines, the possibility to use \LaTeX\ code in listings, automated indexing,
or personal language definitions.
One more little teaser? Here you are. But note that the result is not
produced by the \LaTeX\ code on the right alone. The main parameter is
hidden.
\switchcolumn
\textbf{其他功能}
到目前为止,还有一些未提及的功能:自动换行长行,使用列表中的 \LaTeX 代码的可能性,自动索引或个人语言定义。
还想知道更多?这里有一个小小的引子。但请注意,结果不仅仅由右侧的 \LaTeX 代码生成。主要参数被隐藏了。

\begin{lstsample}{^^A
        \lstset{literate={:=}{{$\gets$}}1 {<=}{{$\leq$}}1 {>=}{{$\geq$}}1 ^^A
            {<>}{{$\neq$}}1}}{}
        \begin{lstlisting}
        if (i<=0) then i := 1;
        if (i>=0) then i := 0;
        if (i<>0) then i := 0;
        \end{lstlisting}
\end{lstsample} 


\switchcolumn[0]*%%%%%%%%%%%%
You're not sure whether you should use \packagename{listings}?
Read the next section!
\switchcolumn
您不确定是否应该使用 \packagename{listings} ?请阅读下一节!
\switchcolumn[0]*%%%%%%%%%%%%
\subsection{Alternatives}
\switchcolumn
\subsection{替代方案}
\end{paracol}

\begin{advise}
\begin{paracol}{2}
\item Why do you list alternatives?
      \advisespace
      Well, it's always good to know the competitors.^^A :-)
\switchcolumn
\item 为什么要列出替代方案?
\advisespace
嗯,了解竞争对手总是有好处。^^A :-)
\switchcolumn[0]*%%%%%%%%%%%%
\item I've read the descriptions below and the \packagename{listings} package
      seems to incorporate all the features. Why should I use one of the
      other programs?
      \advisespace
      Firstly, the descriptions give a taste and not a complete overview,
      secondly, \packagename{listings} lacks some properties, and, ultimately,
      you should use the program matching your needs most precisely.
\switchcolumn
\item 我已经阅读了下面的描述和 \packagename{listings} 宏包,它似乎包含了所有的功能。为什么我要使用其他程序?
\advisespace
首先,描述只是提供了一个概述,其次,\packagename{listings} 缺少一些特性,最后,您应该使用最符合您需求的程序。
\end{paracol}
\end{advise}

\begin{paracol}{2}
This package is certainly not the final utility for typesetting source code.
Other programs do their job very well, if you are not satisfied with
\packagename{listings}. Some are independent of \LaTeX, others come as
separate program plus \LaTeX\ package, and others are packages which
don't pretty-print the source code. The second type includes converters,
cross compilers, and preprocessors. Such programs create \LaTeX\ files
you can use in your document or stand alone ready-to-run \LaTeX\ files.
\switchcolumn
如果您对 \packagename{listings} 不满意,那么这个宏包当然不是排版源代码的最终工具。其他程序也能很好地完成其任务。
其中一些独立于 \LaTeX,其他作为独立的程序加上 \LaTeX\ 宏包,还有一些是不美化源代码的宏包。第二种类型包括转换器、交叉编译器和预处理器。
这些程序会创建 \LaTeX\ 文件,您可以将其用于您的文档或作为独立的、可直接运行的 \LaTeX\ 文件。
\switchcolumn[0]*%%%%%%%%%%%%
Note that I'm not dealing with any literate programming tools here, which
could also be alternatives. However, you should have heard of the
\texttt{WEB} system, the tool Prof.~Donald E.~Knuth developed and made use
of to document and implement \TeX.
\switchcolumn
请注意,我在这里没有处理任何文学编程工具,它们也可以作为替代方案。然而,您应该听说过 \texttt{WEB}系统,这是Donald E. Knuth教授开发和使用的工具,用于记录和实现 \TeX。
\switchcolumn[0]*%%%%%%%%%%%%
\textbf{\href{http://www.infres.enst.fr/~demaille/a2ps}{\packagename{a2ps}}}
started as `ASCII to PostScript' converter, but today you can invoke the
program with \texttt{--pretty-print=}\meta{language} option. If your
favourite programming language is not already supported, you can write your
own so-called style sheet. You can request line numbers, borders, headers,
multiple pages per sheet, and many more. You can even print symbols like
$\forall$ or $\alpha$ instead of their verbose forms. If you just want
program listings and not a document with some listings, this is the best
choice.
\switchcolumn
\textbf{\href{http://www.infres.enst.fr/~demaille/a2ps}{\packagename{a2ps}}}
最初是一个 `ASCII到PostScript' 的转换器,但今天你可以使用选项 \texttt{--pretty-print=}\meta{language} 来调用该程序。
如果你喜欢的编程语言还没有被支持,你可以编写自己的所谓样式表。您可以请求行号、边框、页眉、每张纸上的多个页面等等。
您甚至可以打印出符号,比如 $\forall$ 或 $\alpha$,而不是它们的冗长形式。如果您只想要程序清单而不是带有一些清单的文档,这是最佳选择。

\switchcolumn[0]*%%%%%%%%%%%%
\textbf{\href{http://mirror.ctan.org/support/lgrind}{\packagename{LGrind}}}
is a cross compiler and comes with many predefined programming languages.
For example, you can put the code on the right in your document, invoke
\packagename{LGrind} with \texttt{-e} option (and file names), and run the
created file through \LaTeX. You should get a result similar to the
left-hand side:
\switchcolumn
\textbf{\href{http://mirror.ctan.org/support/lgrind}{\packagename{LGrind}}}
是一个交叉编译器,带有许多预定义的编程语言。例如,您可以在您的文档中放入右侧的代码,在该文档中使用 \texttt{-e} 选项(以及文件名)调用 \packagename{LGrind},然后运行
创建的文件通过 \LaTeX。您应该得到类似于左侧的结果:

\begin{center}
\begin{minipage}{0.45\linewidth}
\iflgrind
   \LGindent=0pt
   \LGinlinefalse\LGbegin\lgrinde
   \L{\LB{\K{for}_\V{i}:=\V{maxint}_\K{to}_\N{0}_\K{do}}}
   \L{\LB{\K{begin}}}
   \L{\LB{____\C{}\{_do_nothing_\}\CE{}}}
   \L{\LB{\K{end};}}
   \L{\LB{}}
   \L{\LB{\V{Write}(\S{}{'}Case_insensitive_{'}\SE{});}}
   \L{\LB{\V{WritE}(\S{}{'}Pascal_keywords.{'}\SE{});}}
   \endlgrinde\LGend
\else
   \packagename{LGrind} not installed.
\fi
\end{minipage}
\begin{minipage}{0.45\linewidth}
\begin{verbatim}
%[
for i:=maxint to 0 do
begin
    { do nothing }
end;
\switchcolumn

\switchcolumn[0]*%%%%%%%%%%%%
Write('Case insensitive ');
WritE('Pascal keywords.');
%]\end{verbatim}
\end{minipage}
\end{center}

\switchcolumn[0]*%%%%%%%%%%%%
If you use |%(| and |%)| instead of |%[| and |%]|, you get a code snippet
instead of a displayed listing. Moreover you can get line numbers to the
left or right, use arbitrary \LaTeX\ code in the source code, print symbols
instead of verbose names, make font setup, and more. You will (have to)
like it (if you don't like \packagename{listings}).
\switchcolumn
如果您使用的是 |%(| 和 |%)| 而不是 |%[| 和 |%]| 
 ,则会得到一个代码片段而不是显示的清单。此外,您可以在源代码中使用任意的 \LaTeX\ 代码,将符号打印出来而不是冗长的名称,进行字体设置等等。如果您不喜欢 \packagename{listings},您应该会喜欢它。

\switchcolumn[0]*%%%%%%%%%%%%
Note that \packagename{LGrind} contains code with a no-sell license and is
thus nonfree software.
\switchcolumn
请注意,\packagename{LGrind} 包含具有不可销售许可的代码,因此属于非自由软件。
\switchcolumn[0]*%%%%%%%%%%%%
\textbf{\href{ftp://axp3.sv.fh-mannheim.de/cvt2latex}{\packagename{cvt2ltx}}}
is a family of `source code to \LaTeX' converters for C, Objective C, \Cpp,
IDL and Perl. Different styles, line numbers and other qualifiers can be
chosen by command-line option. Unfortunately it isn't documented how other
programming languages can be added.
\switchcolumn
\textbf{\href{ftp://axp3.sv.fh-mannheim.de/cvt2latex}{\packagename{cvt2ltx}}}是一个用于C、Objective C、\Cpp、IDL和Perl的源代码转换为 \LaTeX 的转换器系列。可以通过命令行选项选择不同的样式、行号和其他限定符。不幸的是,没有文档说明如何添加其他编程语言。
\switchcolumn[0]*%%%%%%%%%%%%
\textbf{\href{http://mirror.ctan.org/support/C++2LaTeX-1_1pl1}{\packagename{\Cpp2\LaTeX}}}
is a C/\Cpp\ to \LaTeX\ converter. You can specify the fonts for comments,
directives, keywords, and strings, or the size of a tabulator. But as far as
I know you can't number lines.
\switchcolumn
\textbf{\href{http://mirror.ctan.org/support/C++2LaTeX-1_1pl1}{\packagename{\Cpp2\LaTeX}}}是一个C/\Cpp 到 \LaTeX\ 的转换器。您可以指定注释、指令、关键字和字符串的字体,或者制表符的大小。但据我所知,您无法对行进行编号。
\switchcolumn[0]*%%%%%%%%%%%%
\textbf{\href{http://mirror.ctan.org/support/slatex}{\packagename{S\LaTeX}}}
is a pretty-printing Scheme program (which invokes \LaTeX\ automatically)
especially designed for Scheme and other Lisp dialects. It supports stand
alone files, text and display listings, and you can even nest the
commands/environments if you use \LaTeX\ code in comments, for example.
Keywords, constants, variables, and symbols are definable and use of
different styles is possible. No line numbers.
\switchcolumn
\textbf{\href{http://mirror.ctan.org/support/slatex}{\packagename{S\LaTeX}}} 是一个漂亮的Scheme程序(自动调用\LaTeX),专门为Scheme和其他Lisp方言设计。它支持独立文件、文本和显示清单,如果在注释中使用\LaTeX\ 代码,甚至可以嵌套命令/环境。关键字、常量、变量和符号是可定义的,并且可以使用不同的样式。没有行号。
\switchcolumn[0]*%%%%%%%%%%%%
\textbf{\href{http://mirror.ctan.org/support/tiny_c2l}^^A
  {\packagename{tiny\textunderscore c2ltx}}}
is a C/\Cpp/Java to \LaTeX\ converter based on \packagename{cvt2ltx} (or the
other way round?). It supports line numbers, block comments, \LaTeX\ code
in/as comments, and smart line breaking. Font selection and tabulators are
hard-coded, i.e.~you have to rebuild the program if you want to change the
appearance.
\switchcolumn
\textbf{\href{http://mirror.ctan.org/support/tiny_c2l}^^A
{\packagename{tiny\textunderscore c2ltx}}}是一个基于 \packagename{cvt2ltx} 的C/\Cpp/Java到 \LaTeX\ 的转换器(或者反之?)。它支持行号、块注释、在注释中使用 \LaTeX\ 代码以及智能换行。字体选择和制表符是硬编码的,即如果要更改外观,您必须重新构建程序。
\switchcolumn[0]*%%%%%%%%%%%%
\textbf{\href{http://mirror.ctan.org/macros/latex/contrib/misc}{\packagename{listing}}}
---note the missing \packagename{s}---is not a pretty-printer and the
aphorism about documentation at the end of \texttt{listing.sty} is not
true.\space ^^A :-)
It defines |\listoflistings| and a nonfloating environment for listings.
All font selection and indention must be done by hand. However, it's
useful if you have another tool doing that work, e.g.~\packagename{LGrind}.
\switchcolumn
\textbf{\href{http://mirror.ctan.org/macros/latex/contrib/misc}{\packagename{listing}}}——请注意缺少\packagename{s}——不是一个漂亮打印程序,并且\texttt{listing.sty}末尾关于文档的格言是不正确的。它定义了|\listoflistings|和一个非浮动的列表环境。所有的字体选择和缩进都必须手动完成。但是,如果您有另一个工具来完成这项工作,例如\packagename{LGrind},它会很有用。
\switchcolumn[0]*%%%%%%%%%%%%
\textbf{\href{http://mirror.ctan.org/macros/latex/contrib/alg}{\packagename{alg}}}
provides essentially the same functionality as \packagename{algorithms}.
So read the next paragraph and note that the syntax will be different.
\switchcolumn
\textbf{\href{http://mirror.ctan.org/macros/latex/contrib/alg}{\packagename{alg}}}提供了与 \packagename{algorithms} 基本相同的功能。因此,请阅读下一段并注意语法将不同。
\switchcolumn[0]*%%%%%%%%%%%%
\textbf{\href{http://mirror.ctan.org/macros/latex/contrib/algorithms}^^A
  {\packagename{algorithms}}}
goes a quite different way. You describe an algorithm and the package
formats it, for example
\switchcolumn
\textbf{\href{http://mirror.ctan.org/macros/latex/contrib/algorithms}^^A
{\packagename{algorithms}}}采取了完全不同的方式。您描述一个算法,然后该包对其进行格式化,例如:

\begin{center}
\begin{minipage}{0.45\linewidth}
\ifalgorithmicpkg
   \begin{algorithmic}
   \IF {$i\leq0$}
   \STATE $i\gets1$
   \ELSE\IF {$i\geq0$}
   \STATE $i\gets0$
   \ENDIF\ENDIF
   \end{algorithmic}
\else
   \packagename{algorithms} not installed.
\fi
\end{minipage}
\begin{minipage}{0.45\linewidth}
\begin{verbatim}
\begin{algorithmic}
\IF{$i\leq0$}
\STATE $i\gets1$
\ELSE\IF{$i\geq0$}
\STATE $i\gets0$
\ENDIF\ENDIF
\end{algorithmic}\end{verbatim}
\end{minipage}
\end{center}

\switchcolumn[0]*%%%%%%%%%%%%
As this example shows, you get a good looking algorithm even from a bad
looking input. The package provides a lot more constructs like |for|-loops,
|while|-loops, or comments. You can request line numbers, `ruled', `boxed'
and floating algorithms, a list of algorithms, and you can customize the
terms \textbf{if}, \textbf{then}, and so on.
\switchcolumn
正如这个例子所展示的那样,即使输入看起来很糟糕,你也可以得到一个漂亮的算法。该宏包提供了更多的结构,如 |for| 循环、|while| 循环或注释。你可以请求行号,使用`ruled'、`boxed'和浮动算法,获取算法列表,并可以自定义诸如\textbf{if}、\textbf{then}等术语。
\switchcolumn[0]*%%%%%%%%%%%%
\textbf{\href{http://www.mimuw.edu.pl/~wolinski/pretprin.html}^^A
  {\packagename{pretprin}}}
is a package for pretty-printing texts in formal languages---as the title
in TUGboat, Volume 19 (1998), No.~3 states. It provides environments which
pretty-print \emph{and} format the source code. Analyzers for Pascal and
Prolog are defined; adding other languages is easy---if you are or get a bit
familiar with automatons and formal languages.
\switchcolumn
\textbf{\href{http://www.mimuw.edu.pl/~wolinski/pretprin.html}{\packagename{pretprin}}}是一个用于漂亮打印形式语言文本的宏包,正如TUGboat,第19卷(1998年),第3期的标题所述。它提供了能够漂亮打印和格式化源代码的环境。它定义了Pascal和Prolog的分析器;如果您对自动机和形式语言有一点了解,添加其他语言也很容易。
\switchcolumn[0]*%%%%%%%%%%%%
\textbf{\packagename{alltt}}
defines an environment similar to \texttt{verbatim} except that |\|, |{| and
|}| have their usual meanings. This means that you can use commands in the
verbatims, e.g.~select different fonts or enter math mode.
\switchcolumn
\textbf{\packagename{alltt}}定义了一个类似于\texttt{verbatim}的环境,除了||、|{|和|}|具有它们的正常含义。这意味着你可以在verbatim中使用命令,例如选择不同的字体或进入数学模式。
\switchcolumn[0]*%%%%%%%%%%%%
\textbf{\href{http://mirror.ctan.org/macros/latex/contrib/moreverb}^^A
  {\packagename{moreverb}}}
requires \packagename{verbatim} and provides verbatim output to a file,
`boxed' verbatims and line numbers.
\switchcolumn
\textbf{\href{http://mirror.ctan.org/macros/latex/contrib/moreverb}{\packagename{moreverb}}}需要\packagename{verbatim},并提供将输出输出到文件的verbatim、带框的verbatim和行号。
\switchcolumn[0]*%%%%%%%%%%%%
\textbf{\packagename{verbatim}}
defines an improved version of the standard \texttt{verbatim} environment and
a command to input files verbatim.
\switchcolumn
\textbf{\packagename{verbatim}}定义了改进的标准\texttt{verbatim}环境和一个以verbatim方式输入文件的命令。
\switchcolumn[0]*%%%%%%%%%%%%
\textbf{\href{http://mirror.ctan.org/macros/latex/contrib/fancyvrb}^^A
  {\packagename{fancyvrb}}}
is, roughly speaking, a superset of \packagename{alltt},
\packagename{moreverb}, and \packagename{verbatim}, but many more parameters
control the output. The package provides frames, line numbers on the left or
on the right, automatic line breaking (difficult), and more. For example, an
interface to \packagename{listings} exists, i.e.~you can pretty-print source
code automatically.
The package \packagename{fvrb-ex} builds on \packagename{fancyvrb} and
defines environments to present examples similar to the ones in this guide.
\switchcolumn
\textbf{\href{http://mirror.ctan.org/macros/latex/contrib/fancyvrb}{\packagename{fancyvrb}}}大致上是\packagename{alltt}、\packagename{moreverb}和\packagename{verbatim}的超集,但是有更多的参数来控制输出。该宏包提供了框架、左侧或右侧的行号、自动换行(困难)等功能。例如,它提供了与\packagename{listings}的接口,即可以自动漂亮打印源代码。宏包\packagename{fvrb-ex}是在\packagename{fancyvrb}的基础上构建的,它定义了类似本指南中示例的环境。
\switchcolumn[0]*%%%%%%%%%%%%
\end{paracol} 

% \columnratio{0.55}
\begin{paracol}{2}
\section{The next steps}\label{uTheNextSteps}
\switchcolumn
\section{下一步}
\switchcolumn[0]*%%%%%%%%%%%%
Now, before actually using the \packagename{listings} package, you should
\emph{really} read the software license. It does not cost much time and
provides information you probably need to know.
\switchcolumn
在实际使用 \packagename{listings} 宏包之前,您应该\emph{真正}阅读软件许可协议。这不会花费太多时间,但提供了您可能需要了解的信息。
\switchcolumn[0]*%%%%%%%%%%%%
\subsection{Software license}\label{uSoftwareLicense}
\switchcolumn
\subsection{软件许可协议}
\switchcolumn[0]*%%%%%%%%%%%%
The files \texttt{listings.dtx} and \texttt{listings.ins} and all
files generated from only these two files are referred to as `the
\packagename{listings} package' or simply `the package'.
\texttt{lstdrvrs.dtx} and the files generated from that file are
`drivers'.
\switchcolumn
文件\texttt{listings.dtx}和\texttt{listings.ins}以及仅由这两个文件生成的所有文件被称为“\packagename{listings}宏包”或简称为“宏包”。\texttt{lstdrvrs.dtx}和由该文件生成的文件称为“驱动程序”。
\switchcolumn[0]*%%%%%%%%%%%%
\textbf{Copyright}
  The \packagename{listings} package is copyright 1996--2004 Carsten Heinz,
  and copyright 2006 Brooks Moses.  The drivers are copyright any individual
  author listed in the driver files.
\switchcolumn
\textbf{版权}
\packagename{listings}宏包版权归Carsten Heinz(1996-2004)和Brooks Moses(2006)所有。驱动程序的版权归驱动程序文件中列出的每个个人作者所有。
\switchcolumn[0]*%%%%%%%%%%%%
\textbf{Distribution and modification}
  The \packagename{listings} package and its drivers may be distributed
  and/or modified under the conditions of the LaTeX Project Public License,
  either version 1.3c of this license or (at your option) any later version.
  The latest version of this license is in
     \href{http://www.latex-project.org/lppl.txt}{http://www.latex-project.org/lppl.txt}
  and version 1.3c or later is part of all distributions of LaTeX version
 2003/12/01 or later.
\switchcolumn
\textbf{分发和修改}
\packagename{listings}宏包及其驱动程序可以根据LaTeX项目公共许可证的条件进行分发和/或修改,可以选择该许可证的版本1.3c或(根据您的选择)任何更新的版本。此许可证的最新版本位于\href{http://www.latex-project.org/lppl.txt}{http://www.latex-project.org/lppl.txt},1.3c或更高版本是所有LaTeX 2003/12/01版本或更高版本的一部分。
\switchcolumn[0]*%%%%%%%%%%%%
\textbf{Contacts}
  Read section \lstref{uTroubleshooting} on how to submit a bug report.
  Send all other comments, ideas, and additional programming languages to
  \lstemail\ using \texttt{listings} as part of the subject.%todo 
\switchcolumn
\textbf{联系方式}
请阅读\lstref{uTroubleshooting}节以了解如何提交错误报告。将所有其他评论、想法和额外的编程语言发送至\lstemail,并在主题中使用\texttt{listings}。
\end{paracol}

% 
\columnratio{0.55}
\begin{paracol}{2}


\subsection{Package loading}\label{uPackageLoading}
\switchcolumn
\subsection{加载宏包}
\switchcolumn[0]*%%%%%%%%%%%%
As usual in \LaTeX, the package is loaded by\\
   |\usepackage[|\meta{options}|]{listings}|\\,
where |[|\meta{options}|]| is optional and gives a comma separated list of
options. Each either loads an additional \packagename{listings} aspect, or
changes default properties. Usually you don't have to take care of such
options. But in some cases it could be necessary: if you want to compile
documents created with an earlier version of this package or if you use
special features. Here's an incomplete list of possible options.
\switchcolumn
如同在\LaTeX 中一样,可以通过以下方式加载宏包:\\
|\usepackage[|\meta{options}|]{listings}|\\,
其中 |[|\meta{options}|]| 是可选的,并且是一个逗号分隔的选项列表。每个选项可以加载额外的\packagename{listings}模块,或者更改默认属性。通常情况下,您不需要关注这些选项。但在某些情况下可能是必要的:例如,如果您想编译使用早期版本的此宏包创建的文档,或者您使用特殊功能。以下是可能的选项的不完整列表。
\end{paracol}
\begin{advise}
    \columnratio{0.55}
    \begin{paracol}{2}
\item Where is a list of all of the options?
      \advisespace
      In the developer's guide since they were introduced to debug the
      package more easily. Read section \ref{uHowTos} on how to get that
      guide.
\switchcolumn
\item 是否有所有选项的列表?
\advisespace
在开发人员指南中,因为它们被引入以便更容易调试宏包。请阅读第\ref{uHowTos}节,了解如何获取该指南。
    \end{paracol}

\end{advise}

\begin{description}
\columnratio{0.55}
\begin{paracol}{2}
\item[\normalfont\texttt{0.21}]\leavevmode
      invokes a compatibility mode for compiling documents written for
      \packagename{listings} version 0.21.
      \switchcolumn
\item[\normalfont\texttt{0.21}]\leavevmode
启用兼容模式,用于编译针对\packagename{listings} 0.21版本编写的文档。
      \switchcolumn[0]*%%%%%%%%%%%%
\item[\normalfont\texttt{draft}]\leavevmode
      The package prints no stand alone files, but shows the captions and
      defines the corresponding labels.
      Note that a global |\documentclass|-option \texttt{draft} is
      recognized, so you don't need to repeat it as a package option.
      \switchcolumn
\item[\normalfont\texttt{draft}]\leavevmode
该宏包不会生成独立的文件,但会显示标题并定义相应的标签。
请注意,全局|\documentclass|选项\texttt{draft}也会被识别,因此您无需将其作为宏包选项重复指定。
      \switchcolumn[0]*%%%%%%%%%%%%
\item[\normalfont\texttt{final}]\leavevmode\label{uoption:final}
      Overwrites a global \texttt{draft} option.
      \switchcolumn
\item[\normalfont\texttt{final}]\leavevmode\label{uoption:final}
覆盖全局的\texttt{draft}选项。
      \switchcolumn[0]*%%%%%%%%%%%%
\item[\normalfont\texttt{savemem}]\leavevmode

      tries to save some of \TeX's memory. If you switch between languages
      often, it could also reduce compile time. But all this depends on the
      particular document and its listings.
\switchcolumn
\item[\normalfont\texttt{savemem}]\leavevmode
尝试节省一些\TeX 的内存。如果您经常切换语言,这也可能减少编译时间。但这一切都取决于特定的文档及其代码片段。
    \end{paracol}
\end{description}
\columnratio{0.55}
\begin{paracol}{2}
Note that various experimental features also need explicit loading via
options. Read the respective lines in section \ref{rExperimentalFeatures}.
\switchcolumn
请注意,各种实验性功能也需要通过选项显式加载。请阅读第\ref{rExperimentalFeatures}节中的相应行。
\switchcolumn[0]*%%%%%%%%%%%%
\medbreak
After package loading it is recommend to load all used dialects of programming
languages with the following command. It is faster to load several languages
with one command than loading each language on demand.
\switchcolumn
\medbreak
在加载宏包后,建议使用以下命令加载所有使用的编程语言的方言。使用一个命令加载多个语言比按需加载每个语言更快。
\end{paracol}
\begin{syntax}
    \columnratio{0.55}
    \begin{paracol}{2}    
\item {\rstyle\icmdname\lstloadlanguages}\marg{comma separated list of languages}
\switchcolumn
\item {\rstyle\icmdname\lstloadlanguages}\marg{逗号分隔的语言列表}
\switchcolumn[0]*%%%%%%%%%%%%

      Each language is of the form \oarg{dialect}\meta{language}. Without
      the optional \oarg{dialect} the package loads a default dialect. So
      write `|[Visual]C++|' if you want Visual \Cpp\ and `|[ISO]C++|' for
      ISO \Cpp. Both together can be loaded by the command\\
      |\lstloadlanguages{[Visual]C++,[ISO]C++}|.
      \switchcolumn
      每种语言的格式为 \oarg{方言}\meta{语言}。如果没有可选的\oarg{方言},宏包将加载默认方言。因此,如果要加载Visual \Cpp,则写为`|[Visual]C++|',如果要加载ISO \Cpp,则写为`|[ISO]C++|'。同时加载两者可以使用命令\\|\lstloadlanguages{[Visual]C++,[ISO]C++}|。
      \switchcolumn[0]*%%%%%%%%%%%%
      Table \ref{uPredefinedLanguages} on page \pageref{uPredefinedLanguages}
      shows all defined languages and their dialects.
      \switchcolumn
      表\ref{uPredefinedLanguages}在第\pageref{uPredefinedLanguages}页显示了所有定义的语言及其方言。
      \switchcolumn[0]*%%%%%%%%%%%%
    \end{paracol}      
\end{syntax}
\columnratio{0.55}
\begin{paracol}{2}
^^A After or even before language loading, you might want to define default
^^A dialects---just to be independent of configuration files.
\switchcolumn
^^A在加载语言之后,甚至在之前,您可能希望定义默认方言,以便独立于配置文件。
\switchcolumn[0]*%%%%%%%%%%%%
\end{paracol} 
% \columnratio{0.55}
\begin{paracol}{2}
\subsection{The key=value interface}\label{uTheKey=ValueInterface}
\switchcolumn
\subsection{键值对接口}
\switchcolumn[0]*%%%%%%%%%%%%
This package uses the \packagename{keyval} package from the
\packagename{graphics} bundle by David Carlisle. Each parameter is
controlled by an associated key and a user supplied value. For example,
\ikeyname{firstline} is a key and |2| a valid value for this key.
\switchcolumn
该包使用了David Carlisle的\packagename{graphics}包中的\packagename{keyval}包。每个参数都由一个相关的键和用户提供的值控制。例如,\ikeyname{firstline}是一个键,|2|是该键的有效值。
\switchcolumn[0]*%%%%%%%%%%%%
The command {\rstyle\icmdname\lstset} gets a comma separated list of
``key|=|value'' pairs. The first list with more than a single entry is on
page \pageref{gFirstKey=ValueList}: |firstline=2,lastline=5|.
\switchcolumn
命令{\rstyle\icmdname\lstset}接收一个逗号分隔的“键|=|值”列表。第一个具有多个条目的列表在第\pageref{gFirstKey=ValueList}页上:|firstline=2,lastline=5|。
\end{paracol}
\begin{advise}
    \columnratio{0.55}
    \begin{paracol}{2}
\item So I can write `|\lstset{firstline=2,lastline=5}|' once for all?
      \advisespace
      No. `\ikeyname{firstline}' and `\ikeyname{lastline}' belong to a small
      set of
      keys which are only used on individual listings. However, your command is
      not illegal---it has no effect. You have to use these keys inside the
      optional argument of the environment or input command.
      \switchcolumn
      \item 所以我可以一次写好|\lstset{firstline=2,lastline=5}|'就行了吗? \advisespace 不行。\ikeyname{firstline}'和\ikeyname{lastline}'属于只用于个别代码清单的一小组键。然而,你的命令并不违法,只是没有效果。你必须在环境的可选参数或输入命令中使用这些键。
      \switchcolumn[0]*%%%%%%%%%%%%
\item What's about a better example of a key|=|value list?
      \advisespace
      There is one in section \ref{gFigureOutTheAppearance}.
      \switchcolumn
      \item 那么有一个关于键|=|值列表的更好的例子吗? \advisespace 在第\ref{gFigureOutTheAppearance}节中有一个例子。
\switchcolumn[0]*%%%%%%%%%%%%
\item `|language=[77]Fortran|' does not work inside an optional argument.
      \advisespace
      You must put braces around the value if a value with optional argument
      is used inside an optional argument. In the case here write
      `|language={[77]Fortran}|' to select Fortran 77.
      \switchcolumn
      \item 在可选参数中使用|language=[77]Fortran|'无效。
      \advisespace
      如果在可选参数中使用有可选参数的值,必须在值周围加上大括号。在这种情况下,写成|language={[77]Fortran}|'以选择Fortran 77。
\switchcolumn[0]*%%%%%%%%%%%%
\item If I use the `\ikeyname{language}' key inside an optional argument, the
      language isn't active when I typeset the next listing.
      \advisespace
      All parameters set via `|\lstset|' keep their values up to the end of
      the current environment or group. Afterwards the previous values are
      restored. The optional parameters of the two pretty-printing commands
      and the `\texttt{lstlisting}' environment take effect on the particular
      listing only, i.e.~values are restored immediately. For example, you
      can select a main language and change it for special listings.
      \switchcolumn
      \item 如果我在可选参数中使用\ikeyname{language}'键,则在排版下一个代码清单时,该语言不会生效。
      \advisespace
      通过|\lstset|'设置的所有参数都会保持它们的值,直到当前环境或组的结束。之后,先前的值将被恢复。两个漂亮排版命令和\texttt{lstlisting}'环境的可选参数仅对特定的代码清单生效,即值会立即恢复。例如,您可以选择一个主要语言,并为特殊的代码清单修改它。
\switchcolumn[0]*%%%%%%%%%%%%
\item \icmdname\lstinline\ has an optional argument?
      \advisespace
      Yes. And from this fact comes a limitation: you can't use the left
      bracket `|[|' as delimiter unless you specify at least an empty
      optional argument as in `|\lstinline[][var i:integer;[|'.
      If you forget this, you will either get a ``runaway argument'' error
      from \TeX, or an error message from the \packagename{keyval} package.
      \switchcolumn
      \item \icmdname\lstinline\ 有一个可选参数吗?
      \advisespace
      有的。而且由此带来了一个限制:除非您至少指定一个空的可选参数,否则您不能使用左方括号 '|[|' 作为分隔符,如 '|\lstinline[][var i:integer;[|'。如果忘记了这一点,您将要么从\TeX 中得到“runaway argument”错误,要么从\packagename{keyval}包中得到错误消息。
    \end{paracol}
\end{advise}


% \columnratio{0.55}
\begin{paracol}{2}
\subsection{Programming languages}\label{uProgrammingLanguages}
\switchcolumn
\subsection{编程语言}
\switchcolumn[0]*%%%%%%%%%%%%
You already know how to activate programming languages---at least Pascal.
An optional parameter selects particular dialects of a language. For example,
|language=[77]Fortran| selects Fortran 77 and |language=[XSC]Pascal| does the
same for Pascal XSC. The general form is
   {\rstyle\ikeyname{language}}|=|\oarg{dialect}\meta{language}.
If you want to get rid of keyword, comment, and string detection, use
|language={}| as an argument to |\lstset| or as optional argument.
\switchcolumn
您已经知道如何激活编程语言了——至少是Pascal。可选参数选择特定方言的语言。例如,|language=[77]Fortran|选择Fortran 77,|language=[XSC]Pascal|选择Pascal XSC。一般形式是
{\rstyle\ikeyname{language}}|=|\oarg{方言}\meta{语言}。
如果您想去除关键词、注释和字符串的检测,请使用|language={}|作为|\lstset|的参数或可选参数。
\switchcolumn[0]*%%%%%%%%%%%%
Table \ref{uPredefinedLanguages} shows all predefined languages and dialects.
Use the listed names as \meta{language} and \meta{dialect}, respectively. If
no dialect or `empty' is given in the table, just don't specify a dialect.
Each underlined dialect is default; it is selected if you leave out
the optional argument. The predefined defaults are the newest language
versions or standard dialects.
^^A
^^A  Make table of predefined languages.
^^A
\switchcolumn
表\ref{uPredefinedLanguages}展示了所有预定义的语言和方言。将列出的名称用作\meta{语言}和\meta{方言}。如果表中没有给出方言或“empty”,则不指定方言。
每个带有下划线的方言是默认方言;如果省略可选参数,将选择该方言。预定义的默认值是最新的语言版本或标准方言。
% /usr/local/texlive/2022/texmf-dist/tex/latex/listings
% lstlang3.sty 
% lstlang2.sty 
% lstlang1.sty
^^A
^^A 生成预定义语言的表格。
^^A
% \switchcolumn[0]*%%%%%%%%%%%%
\let\lstlanguages\empty
\makeatletter
\@for\lst@temp:={lstlang1.sty,lstlang2.sty,lstlang3.sty}\do
   {\IfFileExists\lst@temp{}{\let\lstlanguages\relax}}
\makeatother
\ifx\lstlanguages\relax
   \PackageWarningNoLine{Listings}
       {Standard drivers not available.\MessageBreak
        Please check your installation.\MessageBreak
        Compilation aborted}
   \csname @@end\expandafter\endcsname
\fi
\lstscanlanguages\lstlanguages{lstlang1.sty,lstlang2.sty,lstlang3.sty}{}^^A
\def\topfigrule{\hrule\kern-0.4pt\relax}^^A
\let\botfigrule\topfigrule
\belowcaptionskip=\smallskipamount
\begin{table}[tbhp]
\small
\caption{%
% Predefined languages.
%          Note that some definitions are preliminary, for example HTML and XML.
%          Each underlined dialect is the default dialect.
         预定义的语言。
请注意,某些定义是初步的,例如HTML和XML。
每个带有下划线的方言是默认方言。}^^A
         \label{uPredefinedLanguages}^^A
\makeatletter
\setbox\@tempboxa\hbox{^^A
   \InputIfFileExists{listings.cfg}{\lst@InputCatcodes}{}}^^A
\lstprintlanguages\lstlanguages
\end{table}
^^A
^^A end of table
^^A
\lstset{defaultdialect=[doc]Pascal}^^A restore
\end{paracol}

\begin{advise}
\columnratio{0.55}
\begin{paracol}{2}
\item How can I define default dialects?
      \advisespace
      Check section \ref{rLanguagesAndStyles} for `\keyname{defaultdialect}'.
\switchcolumn
\item 如何定义默认方言?
\advisespace
请参阅第 \ref{rLanguagesAndStyles} 节中的 `\keyname{defaultdialect}'。
\switchcolumn[0]*%%%%%%%%%%%%
\item I have C code mixed with assembler lines. Can \packagename{listings}
      pretty-print such source code, i.e.~highlight keywords and comments of
      both languages?
      \advisespace
      `\ikeyname{alsolanguage}|=|\oarg{dialect}\meta{language}' selects a
      language additionally to the active one. So you only have to write a
      language definition for your assembler dialect, which doesn't interfere
      with the definition of C, say. Moreover you might want to use the key
      `\keyname{classoffset}' described in section \ref{rLanguagesAndStyles}.
\switchcolumn
\item 我的C代码中混有汇编行。能否使用\packagename{listings}美化这样的源代码,即突出显示两种语言的关键词和注释?
\advisespace
`\ikeyname{alsolanguage}|=|\oarg{方言}\meta{语言}' 除了活动语言外,还选择另一种语言。因此,您只需为汇编方言编写语言定义即可,该定义不会与C的定义相冲突。此外,您可能还想使用第\ref{rLanguagesAndStyles}节中描述的 `\keyname{classoffset}' 键。
\switchcolumn[0]*%%%%%%%%%%%%
\item How can I define my own language?
      \advisespace
      This is discussed in section \ref{rLanguageDefinitions}. And if you
      think that other people could benefit by your definition, you might
      want to send it to the address in section \ref{uSoftwareLicense}.
      Then it will be published under the \LaTeX\ Project Public License.
\switchcolumn
\item 如何定义自己的语言?
\advisespace
这在第\ref{rLanguageDefinitions}节中讨论。如果您认为其他人可以从您的定义中受益,您可以将其发送到第\ref{uSoftwareLicense}节中的地址。然后它将在 \LaTeX\ 项目公共许可证下发布。
\end{paracol}
\end{advise}
\columnratio{0.55}
\begin{paracol}{2}
Note that the arguments \meta{language} and \meta{dialect} are case
insensitive and that spaces have no effect.
\switchcolumn
请注意,参数\meta{语言}和\meta{方言}不区分大小写,空格不起作用。
\switchcolumn[0]*%%%%%%%%%%%%
There is at least one language (VDM, Vienna Development Language\footnote{\url{http://www.vdmportal.org}}) which is not directly supported by the
\packagename{listings} package. It needs a package for its own:
\packagename{vdmlisting}. On the other hand \packagename{vdmlisting} uses
the \packagename{listings} package and so it should be mentioned in this
context.
\switchcolumn
至少有一种语言(VDM,Vienna Development Language\footnote{\url{http://www.vdmportal.org}})不直接受\packagename{listings}包支持。它需要一个专门的包:\packagename{vdmlisting}。另一方面,\packagename{vdmlisting}使用\packagename{listings}包,因此在这个上下文中应该提到它。
\switchcolumn[0]*%%%%%%%%%%%%
\subsubsection{Preferences}\label{uPreferences}
\switchcolumn
\subsubsection{首选项}
\switchcolumn[0]*%%%%%%%%%%%%
Sometimes authors of language support provide their own configuration
preferences. These may come either from their personal experience or
from the settings in an IDE and can be defined as a \packagename{listings}
style. From version 1.5b of the \packagename{listings} package on these
styles are provided as files with the name
|listings-|\meta{language}|.prf|, \meta{language} is the name of the
supported programming language in lowercase letters.
\switchcolumn
有时,语言支持的作者会提供自己的配置首选项。这些首选项可能来自个人经验或IDE中的设置,并可以定义为\packagename{listings}样式。从\packagename{listings}包的1.5b版本开始,这些样式被提供为以小写字母命名的文件,文件名为|listings-|\meta{语言}|.prf|,其中\meta{语言}是所支持的编程语言的名称。
\switchcolumn[0]*%%%%%%%%%%%%
So if an user of the \packagename{listings} package wants to use these
preferences, she/he can say for example when using Python
\begin{quote}
    |\input{listings-python.prf}|
\end{quote}
\switchcolumn
因此,如果\packagename{listings}包的用户想要使用这些首选项,例如在使用Python时,可以在她/他的|listings.cfg|配置文件末尾写入以下内容:
\begin{quote}
|\input{listings-python.prf}|
\end{quote}
\switchcolumn[0]*%%%%%%%%%%%%
at the end of her/his |listings.cfg| configuration file as long as the
file |listings-python.prf| resides in the \TeX{} search path. Of course
that file can be changed according to the user's preferences.
\switchcolumn
只要文件|listings-python.prf|位于\TeX{}搜索路径中。当然,该文件可以根据用户的偏好进行更改。
\switchcolumn[0]*%%%%%%%%%%%%
At the moment there are five such preferences files:
\switchcolumn
目前有五个这样的首选项文件:
\begin{enumerate}
  \item |listings-acm.prf|
  \item |listings-bash.prf|
  \item |listings-fortran.prf|
  \item |listings-lua.prf|
  \item |listings-python.prf|
\end{enumerate}
\switchcolumn[0]*%%%%%%%%%%%%
All contributors are invited to supply more personal preferences.
\switchcolumn
欢迎所有贡献者提供更多个人偏好。
\end{paracol} 
% \columnratio{0.55}
\begin{paracol}{2}
\subsection{Special characters}\label{uSpecialCharacters}
\switchcolumn
\subsection{特殊字符}
\switchcolumn[0]*%%%%%%%%%%%%
\textbf{Tabulators}
You might get unexpected output if your sources contain tabulators.
The package assumes tabulator stops at columns 9, 17, 25, 33, and so on.
This is predefined via |tabsize=8|. If you change the eight to the number
$n$, you will get tabulator stops at columns $n+1,2n+1,3n+1,$ and so on.
\switchcolumn
\textbf{制表符}
如果你的源文件包含制表符,可能会得到意外的输出结果。
该宏包默认将制表符停在第9、17、25、33等列。
这是通过 |tabsize=8| 预定义的。如果你将8更改为数字$n$,则将在第$n+1,2n+1,3n+1$等列停止。
\begin{lstsample}[tabsize]{}{}
        \lstset{tabsize=2}
        \begin{lstlisting}
        123456789
            { one tabulator }
                { two tabs }
        123     { 123 + two tabs }
        \end{lstlisting}
\end{lstsample}
\switchcolumn[0]*%%%%%%%%%%%%
For better illustration, the left-hand side uses |tabsize=2| but the verbatim
code |tabsize=4|. Note that |\lstset| modifies the values for all following
listings in the same environment or group. This is no problem here since the
examples are typeset inside minipages. If you want to change settings for a
single listing, use the optional argument.
\switchcolumn
为了更好地说明问题,左侧使用 |tabsize=2|,但原始代码使用 |tabsize=4|。
请注意,|\lstset| 修改了同一环境或组中所有后续列表的值。在这里没有问题,因为示例是在minipage内排版的。
如果要更改单个列表的设置,请使用可选参数。
\switchcolumn[0]*%%%%%%%%%%%%
\textbf{Visible tabulators and spaces}
One can make spaces and tabulators visible:
\switchcolumn
\textbf{可见制表符和空格}
可以使空格和制表符可见:
\begin{lstsample}[showspaces,showtabs,tab]{}{}
        \lstset{showspaces=true,
                showtabs=true,
                tab=\rightarrowfill}
        \begin{lstlisting}
            for i:=maxint to 0 do
            begin
            { do nothing }
            end;
        \end{lstlisting}
\end{lstsample}

\switchcolumn[0]*%%%%%%%%%%%%
If you request \ikeyname{showspaces} but no \ikeyname{showtabs},
tabulators are converted to visible spaces.
The default definition of \ikeyname{tab} produces a `wide visible space'
\lstinline[showtabs]!	!. So you might want to use |$\to$|, |$\dashv$|
or something else instead.
\switchcolumn
如果请求 \ikeyname{showspaces} 但没有 \ikeyname{showtabs},则制表符将转换为可见空格。
\ikeyname{tab} 的默认定义产生一个“宽可见空格”\lstinline[showtabs]! !。因此,你可能想使用 |$\to$|、|$\dashv$| 或其他替代品。
\end{paracol}
\begin{advise}
    \columnratio{0.55}
\begin{paracol}{2}
\item Some sort of advice: (1) You should really indent lines of source code
      to make listings more readable. (2) Don't indent some lines with
      spaces and others via tabulators. Changing the tabulator size (of your
      editor or pretty-printing tool) completely disturbs the columns.
      (3) As a consequence, never share your files with differently tab sized
      people!^^A true only if you use tabulators, just :-)
\switchcolumn \item 一些建议:(1)为了使代码清晰易读,应该缩进源代码的行。(2)不要使用空格缩进某些行,使用制表符缩进其他行。更改制表符大小(编辑器或美化工具的制表符大小)会完全扰乱列。(3)因此,永远不要与制表符大小不同的人共享文件!^^A只有当你使用制表符时才是真的 :-)
\switchcolumn[0]*%%%%%%%%%%%%\switchcolumn[0]*%%%%%%%%%%%%
\item To make the \LaTeX\ code more readable, I indent the environments'
      program listings. How can I remove that indention in the output?
      \advisespace
      Read `How to gobble characters' in section \ref{uHowTos}.
\switchcolumn
\item 为了使 \LaTeX\ 代码更易读,我缩进了环境中的程序列表。如何在输出中删除缩进?
\advisespace
请阅读“如何吞掉字符”一节中的“gobble characters”。
    \end{paracol}
\end{advise}

\columnratio{0.55}
\begin{paracol}{2}
\textbf{Form feeds}
Another special character is a form feed causing an empty line by default.
{\rstyle\ikeyname{formfeed}}|=\newpage| would result in a new page every
form feed. Please note that such definitions (even the default) might get
in conflict with frames.
\switchcolumn
\textbf{换页符}
另一个特殊字符是默认情况下导致空行的换页符。
{\rstyle\ikeyname{formfeed}}|=\newpage| 将导致每个换页符都换到新的一页。
请注意,这样的定义(甚至默认定义)可能与框架冲突。
\switchcolumn[0]*%%%%%%%%%%%%

\textbf{National characters}
If you type in such characters directly as characters of codes 128--255 and
use them also in listings, let the package know it---or you'll get really
funny results. {\rstyle\ikeyname{extendedchars}}|=true| allows and
|extendedchars=false| prohibits \packagename{listings} from handling
extended characters in listings. If you use them, you should load
\packagename{fontenc}, \packagename{inputenc} and/or
any other package which defines the characters.
\switchcolumn
\textbf{国际字符}
如果直接输入代码128--255的字符并在列表中使用它们,应该让宏包知道---否则你将得到非常奇怪的结果。
\packagename{listings} 允许 \ikeyname{extendedchars}=true,禁止 \ikeyname{extendedchars}=false 处理列表中的扩展字符。
如果使用这些字符,应该加载 \packagename{fontenc}、\packagename{inputenc} 或其他定义字符的包。
\switchcolumn[0]*%%%%%%%%%%%%
\begin{advise}
\item I have problems using \packagename{inputenc} together with
      \packagename{listings}.
      \advisespace
      This could be a compatibility problem. Make a bug report as described
      in section \lstref{uTroubleshooting}.
\end{advise}
\switchcolumn
\begin{advise}
\item 我在使用 \packagename{inputenc} 和 \packagename{listings} 时遇到了问题。
\advisespace
这可能是一个兼容性问题。请按照第\lstref{uTroubleshooting}节中的说明提交错误报告。
\end{advise}
\switchcolumn[0]*%%%%%%%%%%%%
The extended characters don't cover Arabic, Chinese, Hebrew, Japanese, and so
on---specifically, any encoding which uses multiple bytes per character.
\switchcolumn
扩展字符不包括阿拉伯语、中文、希伯来语、日语等使用多字节字符的编码。
\switchcolumn[0]*%%%%%%%%%%%%
Thus, if you use the a package that supports multibyte characters, such as
the \packagename{CJK} or \packagename {ucs} packages for Chinese and
UTF-8 characters, you must avoid letting \packagename{listings}
process the extended characters.  It is generally best to also specify
|extendedchars=false| to avoid having \packagename{listings} get entangled
in the other package's extended-character treatment.
\switchcolumn
因此,如果使用支持多字节字符的包,例如用于中文和UTF-8字符的\packagename{CJK}或\packagename{ucs}包,必须避免让\packagename{listings}处理扩展字符。
最好也指定 |extendedchars=false|,以避免\packagename{listings}与其他包的扩展字符处理发生混淆。
\switchcolumn[0]*%%%%%%%%%%%%
If you do have a listing contained within a CJK environment, and want to have
CJK characters inside the listing, you can place them within a comment that
escapes to \LaTeX -- see section \ref{rEscapingToLaTeX} for how to do that.
(If the listing is not inside a CJK environment, you can simply put a small
CJK environment within the escaped-to-\LaTeX portion of the comment.)
\switchcolumn
如果你在CJK环境中包含一个列表,并且希望在列表中使用CJK字符,可以在转义到\LaTeX 中的注释中放置它们。有关如何执行此操作,请参见第\ref{rEscapingToLaTeX}节。
(如果列表不在CJK环境中,可以在注释的转义到\LaTeX 部分中简单地放置一个小的CJK环境。)
\switchcolumn[0]*%%%%%%%%%%%%
Similarly, if you are using UTF-8 extended characters in a listing, they must
be placed within an escape to \LaTeX.
\switchcolumn
类似地,如果在列表中使用UTF-8扩展字符,则必须将它们放在转义到\LaTeX 中。
\switchcolumn[0]*%%%%%%%%%%%%
Also, section \ref{uNationalCharacters} has a few details on how to work with
extended characters in the context of $\Lambda$.
\switchcolumn
此外,在第\ref{uNationalCharacters}节中介绍了如何在$\Lambda$的上下文中处理扩展字符的一些细节。
\end{paracol}
% \columnratio{0.55}
\begin{paracol}{2}
\subsection{Line numbers}\label{uLineNumbers}
\switchcolumn
\subsection{行号}
\switchcolumn[0]*%%%%%%%%%%%%
You already know the keys \ikeyname{numbers}, \ikeyname{numberstyle},
\ikeyname{stepnumber}, and \ikeyname{numbersep} from section
\ref{gSeduceToUse}. Here now we deal with continued listings.
You have two options to get consistent line numbering across listings.
\switchcolumn
你已经了解到了 \ikeyname{numbers}、\ikeyname{numberstyle}、\ikeyname{stepnumber} 和 \ikeyname{numbersep} 这些键,它们来自于第 \ref{gSeduceToUse} 节。现在我们来处理连续的代码列表。你有两种选择来实现代码列表之间的连续行号。

\begin{lstsample}[firstnumber]{\lstset{numbers=left,numberstyle=\tiny,^^A
        stepnumber=2,numbersep=5pt}}{}
        \begin{lstlisting}[firstnumber=100]
        for i:=maxint to 0 do
        begin
            { do nothing }
        end;

        \end{lstlisting}
        And we continue the listing:
        \begin{lstlisting}[firstnumber=last]
        Write('Case insensitive ');
        WritE('Pascal keywords.');
        \end{lstlisting}
\end{lstsample}
\switchcolumn[0]*%%%%%%%%%%%%
In the example, \ikeyname{firstnumber} is initially set to 100; some lines
later the value is \texttt{last}, which continues the numbering of the last
listing. Note that the empty line at the end of the first part is not printed
here, but it counts for line numbering. You should also notice that you can
write |\lstset{firstnumber=last}| once and get consecutively numbered code
lines---except you specify something different for a particular listing.
\switchcolumn
在这个例子中,\ikeyname{firstnumber} 的初始值设置为100;几行之后,值为 \texttt{last},它将继续上一个代码列表的编号。请注意,第一部分末尾的空行在这里没有打印出来,但它计入行号。你还应该注意到,你可以写 |\lstset{firstnumber=last}| 一次,然后连续地获得编码行的编号,除非你为特定的代码列表指定了其他值。
\switchcolumn[0]*%%%%%%%%%%%%
On the other hand you can use |firstnumber=auto| and name your listings.
Listings with identical names (case sensitive!) share a line counter.
\switchcolumn
另一方面,你可以使用 |firstnumber=auto| 并为代码列表命名。具有相同名称(区分大小写!)的代码列表共享行号计数器。

\begin{lstsample}[name]{\lstset{numbers=left,numberstyle=\tiny,stepnumber=2,numbersep=5pt}}{}
        \begin{lstlisting}[name=Test]
        for i:=maxint to 0 do
        begin
            { do nothing }
        end;

        \end{lstlisting}
        And we continue the listing:
        \begin{lstlisting}[name=Test]
        Write('Case insensitive ');
        WritE('Pascal keywords.');
        \end{lstlisting}
\end{lstsample}
\switchcolumn[0]*%%%%%%%%%%%%
The next |Test| listing goes on with line number {\makeatletter\lstno@Test},
no matter whether there are other listings in between.
\switchcolumn
下一个名为 |Test| 的代码列表使用的行号是 {\makeatletter\lstno@Test},不管中间是否有其他代码列表。
\switchcolumn[0]*%%%%%%%%%%%%
You can also select the lines to be printed, the options
`\ikeyname{linerange}' and `\ikeyname{consecutivenumbers}' are your
friend. In a presentation for example you don't need comments for your
programs, so you prefer the line numbers being consecutively numbered,
but the results should reflect the behaviour of the program---you omit
parts of the lengthy output. So
you may have the following program and its results.
\switchcolumn
你还可以选择要打印的行,选项 \ikeyname{linerange} 和 \ikeyname{consecutivenumbers} 是你的朋友。例如,在演示中,你不需要程序的注释,所以你更喜欢行号连续编号,但结果应该反映程序的行为——你忽略了冗长输出的部分。因此,你可以有以下程序及其结果。

% \begin{lstsample}[name]{\lstset{numbers=left,numberstyle=\tiny,stepnumber=1,numbersep=5pt}}{}
% \begin{lstlisting}[name=Test,
%     language={[ansi]C},
%     linerange={1-4,6-7,10-14,
%     17-19,21-22},
%     firstnumber=1]
% #include <stdio.h>
% #include <stdlib.h>

% int main(int argc,char* argv[]){
%     /* declaring variables */
%     int i;
%     int limit;

%     /* checking arguments */
%     if ( argc > 1 ) {
%     limit = atoi(argv[1]);
%     } else {
%     limit = 100;
%     }

%     /* counting lines */
%     for (i = 1;i <= limit;i++) {
%     printf("Line no. %3.0d\n", i);
%     }

%     return 0;
% }

% \end{lstlisting}
% And these are the results:
% \begin{lstlisting}[language={},
%     linerange={1-2,6-7},
%     consecutivenumbers=false]
% Line no.   1
% Line no.   2
% Line no.   3
% Line no.   4
% Line no.   5
% Line no.   6
% Line no.   7
% \end{lstlisting}
% \end{lstsample}

% \begin{advise}
% \item Okay. And how can I get decreasing line numbers?
%     \advisespace
%     Sorry, what?
%     \advisespace
%     Decreasing line numbers as on page \pageref{rDecreasingLabels}.
%     \advisespace
%     May I suggest to demonstrate your individuality by other means?
%     If you differ, you should try a negative `\ikeyname{stepnumber}'
%     (together with `\ikeyname{firstnumber}').
% \end{advise}

% Read section \ref{uHowTos} on how to reference line numbers.


\end{paracol} 
\columnratio{0.55}
\begin{paracol}{2}
\subsection{Layout elements}
\switchcolumn
\subsection{布局元素}
\switchcolumn[0]*%%%%%%%%%%%%
It's always a good idea to structure the layout by vertical space,
horizontal lines, or different type sizes and typefaces. The best to stress
whole listings are---not all at once---colours, frames, vertical space, and
captions. The latter are also good to refer to listings, of course.
\switchcolumn
在布局中使用垂直空间、水平线条或不同的字体大小和字体类型是一个很好的主意。最好强调整个代码列表的方法是——不是一次性地——使用颜色、框架、垂直空间和标题。当然,标题也是指向代码列表的好方法。
\switchcolumn[0]*%%%%%%%%%%%%
\textbf{Vertical space}
The keys {\rstyle\ikeyname{aboveskip}} and {\rstyle\ikeyname{belowskip}}
control the vertical space above and below displayed listings. Both keys get
a dimension or skip as value and are initialized to |\medskipamount|.
\switchcolumn
\textbf{垂直空间}
关键字{\rstyle\ikeyname{aboveskip}}和{\rstyle\ikeyname{belowskip}}控制显示的代码列表上方和下方的垂直空间。这两个关键字的值可以是尺寸或间距,默认值为 |\medskipamount|。
\switchcolumn[0]*%%%%%%%%%%%%
\textbf{Frames}
The key \ikeyname{frame} takes the verbose values \keyvalue{none},
\keyvalue{leftline}, \keyvalue{topline}, \keyvalue{bottomline},
\keyvalue{lines} (top and bottom), \keyvalue{single} for single frames, or
\keyvalue{shadowbox}.
\switchcolumn
\textbf{边框}
关键字\ikeyname{frame}接受以下值:\keyvalue{none}、\keyvalue{leftline}、\keyvalue{topline}、\keyvalue{bottomline}、\keyvalue{lines}(顶部和底部)、\keyvalue{single}(单边框)或\keyvalue{shadowbox}。
\begin{lstsample}[frame]{}{}
        \begin{lstlisting}[frame=single]
        for i:=maxint to 0 do
        begin
            { do nothing }
        end;
        \end{lstlisting}
\end{lstsample}
\switchcolumn[0]*%%%%%%%%%%%%
\begin{advise}
\item The rules aren't aligned.
     \advisespace
     This could be a bug of this package or a problem with your
     \texttt{.dvi} driver. \emph{Before} sending a bug report to the package
     author, modify the parameters described in section \ref{rFrames}
     heavily. And do this step by step!
     For example, begin with `|framerule=10mm|'. If the rules are
     misaligned by the same (small) amount as before, the problem does not
     come from the rule width. So continue with the next parameter.  Also,
     Adobe Acrobat sometimes has single-pixel rounding errors which can
     cause small misalignments at the corners when PDF files are displayed
     on screen; these are unfortunately normal.
\end{advise}
\switchcolumn
\begin{advise}
    \item 这些线条没有对齐。
    \advisespace
    这可能是此宏包的一个错误,或者与您的\texttt{.dvi}驱动程序有关的问题。\emph{在}向宏包作者发送错误报告之前,强烈建议您根据第\ref{rFrames}节所述的参数进行修改,并逐步进行。例如,从|framerule=10mm|'开始。如果线条的偏移量与之前相同(小),那么问题就不是由线条宽度引起的。因此,继续下一个参数。此外,Adobe Acrobat有时会出现单像素的舍入误差,当在屏幕上显示PDF文件时,这些误差会导致角落处的小偏差;这是正常现象。 
\end{advise}
\switchcolumn[0]*%%%%%%%%%%%%
Alternatively you can control the rules at the \texttt{t}op, \texttt{r}ight,
\texttt{b}ottom, and \texttt{l}eft directly by using the four initial letters
for single rules and their upper case versions for double rules.
\switchcolumn
或者,您可以直接使用四个首字母来控制\texttt{t}op、\texttt{r}ight、\texttt{b}ottom和\texttt{l}eft的线条。对于单线条,使用四个首字母的小写形式,对于双线条,使用四个首字母的大写形式。
\begin{lstsample}[frame]{}{}
        \begin{lstlisting}[frame=trBL]
        for i:=maxint to 0 do
        begin
            { do nothing }
        end;
        \end{lstlisting}
\end{lstsample}
\switchcolumn[0]*%%%%%%%%%%%%
Note that a corner is drawn if and only if both adjacent rules are requested.
You might think that the lines should be drawn up to the edge, but what's
about round corners? The key \ikeyname{frameround} must get exactly four
characters as value. The first character is attached to the upper right
corner and it continues clockwise. `\texttt{t}' as character makes the
corresponding corner round.
\switchcolumn
请注意,只有在请求了两个相邻线条时才会绘制一个角落。您可能认为线条应该绘制到边缘,但是圆角呢?关键字\ikeyname{frameround}的值必须是四个字符。第一个字符与右上角相连,然后按顺时针方向继续。字符\texttt{t}'使相应的角变为圆角。 
\begin{lstsample}[frameround]{}{}
        \lstset{frameround=fttt}
        \begin{lstlisting}[frame=trBL]
        for i:=maxint to 0 do
        begin
            { do nothing }
        end;
        \end{lstlisting}
\end{lstsample}
Note that \ikeyname{frameround} has been used together with |\lstset| and thus
the value affects all following listings in the same group or environment.
Since the listing is inside a \texttt{minipage} here, this is no problem.
\begin{advise}
\item Don't use frames all the time, and in particular not with short listings.
     This would emphasize nothing. Use frames for $10\%$ or even less of
     your listings, for your most important ones.
\item If you use frames on floating listings, do you really want frames?
     \advisespace
     No, I want to separate floats from text.
     \advisespace
     Then it is better to redefine \LaTeX's `|\topfigrule|' and
     `|\botfigrule|'. For example, you could write
     `|\renewcommand*\topfigrule{\hrule\kern-0.4pt\relax}|' and make the
     same definition for |\botfigrule|.
\end{advise}

\textbf{Captions}
Now we come to \ikeyname{caption} and \ikeyname{label}. You might guess
(correctly) that they can be used in the same manner as \LaTeX's |\caption|
and |\label| commands, although here it is also possible to have a caption
regardless of whether or not the listing is in a float:
\begin{lstsample}[caption,label]{\lstset{xleftmargin=.05\linewidth}}{}
  \begin{lstlisting}[caption={Useless code},label=useless]
  for i:=maxint to 0 do
  begin
      { do nothing }
  end;
  \end{lstlisting}
\end{lstsample}
Afterwards you could refer to the listing via |\ref{useless}|. By default
such a listing gets an entry in the list of listings, which can be printed
with the command {\rstyle\icmdname\lstlistoflistings}. The key
{\rstyle\ikeyname{nolol}} suppresses an entry for both the environment or
the input command. Moreover, you can specify a short caption for the list
of listings:
  \keyname{caption}|={|\oarg{short}\meta{long}|}|.
Note that the whole value is enclosed in braces since an optional value is
used in an optional argument.

If you don't want the label \texttt{\lstlistingname} plus number, you should
use \ikeyname{title}:
\begin{lstsample}[title]{\lstset{xleftmargin=.05\linewidth}}{}
  \begin{lstlisting}[title={`Caption' without label}]
  for i:=maxint to 0 do
  begin
      { do nothing }
  end;
  \end{lstlisting}
\end{lstsample}
\begin{advise}
\item Something goes wrong with `\keyname{title}' in my document: in front of
     the title is a delimiter.
     \advisespace
     The result depends on the document class; some are not compatible.
     Contact the package author for a work-around.
\end{advise}

\textbf{Colours}
One more element. You need the \packagename{color} package and can then
request coloured background via
\ikeyname{backgroundcolor}|=|\meta{color command}.
\begin{advise}
\item Great! I love colours.
     \advisespace
     Fine, yes, really. And I like to remind you of the warning about
     striking styles on page \pageref{wStrikingStyles}.
\end{advise}
\ifcolor
\begin{lstxsample}[backgroundcolor]
  \lstset{backgroundcolor=\color{yellow}}
\end{lstxsample}
\else
\begin{verbatim}
  color package not installed\end{verbatim}
\fi
\begin{lstsample}{}{}
  \begin{lstlisting}[frame=single,
                     framerule=0pt]
  for i:=maxint to 0 do
  begin
      j:=square(root(i));
  end;
  \end{lstlisting}
\end{lstsample}
The example also shows how to get coloured space around the whole listing:
use a frame whose rules have no width.


\end{paracol} 

% 
% \section{The next steps}\label{uTheNextSteps}
%
% Now, before actually using the \packagename{listings} package, you should
% \emph{really} read the software license. It does not cost much time and
% provides information you probably need to know.
%
%
% \subsection{Software license}\label{uSoftwareLicense}
%
% The files \texttt{listings.dtx} and \texttt{listings.ins} and all
% files generated from only these two files are referred to as `the
% \packagename{listings} package' or simply `the package'.
% \texttt{lstdrvrs.dtx} and the files generated from that file are
% `drivers'.
%
% \paragraph{Copyright}
%   The \packagename{listings} package is copyright 1996--2004 Carsten Heinz,
%   and copyright 2006 Brooks Moses.  The drivers are copyright any individual
%   author listed in the driver files.
%
% \paragraph{Distribution and modification}
%   The \packagename{listings} package and its drivers may be distributed
%   and/or modified under the conditions of the LaTeX Project Public License,
%   either version 1.3c of this license or (at your option) any later version.
%   The latest version of this license is in
%      \href{http://www.latex-project.org/lppl.txt}{http://www.latex-project.org/lppl.txt}
%   and version 1.3c or later is part of all distributions of LaTeX version
%  2003/12/01 or later.
%
% \paragraph{Contacts}
%   Read section \lstref{uTroubleshooting} on how to submit a bug report.
%   Send all other comments, ideas, and additional programming languages to
%   \lstemail\ using \texttt{listings} as part of the subject.
%
%
% \subsection{Package loading}\label{uPackageLoading}
%
% As usual in \LaTeX, the package is loaded by
%    |\usepackage[|\meta{options}|]{listings}|,
% where |[|\meta{options}|]| is optional and gives a comma separated list of
% options. Each either loads an additional \packagename{listings} aspect, or
% changes default properties. Usually you don't have to take care of such
% options. But in some cases it could be necessary: if you want to compile
% documents created with an earlier version of this package or if you use
% special features. Here's an incomplete list of possible options.
% \begin{advise}
% \item Where is a list of all of the options?
%       \advisespace
%       In the developer's guide since they were introduced to debug the
%       package more easily. Read section \ref{uHowTos} on how to get that
%       guide.
% \end{advise}
% \begin{description}
% \item[\normalfont\texttt{0.21}]\leavevmode
%
%       invokes a compatibility mode for compiling documents written for
%       \packagename{listings} version 0.21.
%
% \item[\normalfont\texttt{draft}]\leavevmode
%
%       The package prints no stand alone files, but shows the captions and
%       defines the corresponding labels.
%       Note that a global |\documentclass|-option \texttt{draft} is
%       recognized, so you don't need to repeat it as a package option.
%
% \item[\normalfont\texttt{final}]\leavevmode\label{uoption:final}
%
%       Overwrites a global \texttt{draft} option.
%
% \item[\normalfont\texttt{savemem}]\leavevmode
%
%       tries to save some of \TeX's memory. If you switch between languages
%       often, it could also reduce compile time. But all this depends on the
%       particular document and its listings.
% \end{description}
% Note that various experimental features also need explicit loading via
% options. Read the respective lines in section \ref{rExperimentalFeatures}.
%
% \medbreak
% After package loading it is recommend to load all used dialects of programming
% languages with the following command. It is faster to load several languages
% with one command than loading each language on demand.
% \begin{syntax}
% \item {\rstyle\icmdname\lstloadlanguages}\marg{comma separated list of languages}
%
%       Each language is of the form \oarg{dialect}\meta{language}. Without
%       the optional \oarg{dialect} the package loads a default dialect. So
%       write `|[Visual]C++|' if you want Visual \Cpp\ and `|[ISO]C++|' for
%       ISO \Cpp. Both together can be loaded by the command
%       |\lstloadlanguages{[Visual]C++,[ISO]C++}|.
%
%       Table \ref{uPredefinedLanguages} on page \pageref{uPredefinedLanguages}
%       shows all defined languages and their dialects.
% \end{syntax}
%^^A After or even before language loading, you might want to define default
%^^A dialects---just to be independent of configuration files.
%
%
% \subsection{The key=value interface}\label{uTheKey=ValueInterface}
%
% This package uses the \packagename{keyval} package from the
% \packagename{graphics} bundle by David Carlisle. Each parameter is
% controlled by an associated key and a user supplied value. For example,
% \ikeyname{firstline} is a key and |2| a valid value for this key.
%
% The command {\rstyle\icmdname\lstset} gets a comma separated list of
% ``key|=|value'' pairs. The first list with more than a single entry is on
% page \pageref{gFirstKey=ValueList}: |firstline=2,lastline=5|.
% \begin{advise}
% \item So I can write `|\lstset{firstline=2,lastline=5}|' once for all?
%       \advisespace
%       No. `\ikeyname{firstline}' and `\ikeyname{lastline}' belong to a small
%       set of
%       keys which are only used on individual listings. However, your command is
%       not illegal---it has no effect. You have to use these keys inside the
%       optional argument of the environment or input command.
% \item What's about a better example of a key|=|value list?
%       \advisespace
%       There is one in section \ref{gFigureOutTheAppearance}.
% \item `|language=[77]Fortran|' does not work inside an optional argument.
%       \advisespace
%       You must put braces around the value if a value with optional argument
%       is used inside an optional argument. In the case here write
%       `|language={[77]Fortran}|' to select Fortran 77.
% \item If I use the `\ikeyname{language}' key inside an optional argument, the
%       language isn't active when I typeset the next listing.
%       \advisespace
%       All parameters set via `|\lstset|' keep their values up to the end of
%       the current environment or group. Afterwards the previous values are
%       restored. The optional parameters of the two pretty-printing commands
%       and the `\texttt{lstlisting}' environment take effect on the particular
%       listing only, i.e.~values are restored immediately. For example, you
%       can select a main language and change it for special listings.
% \item \icmdname\lstinline\ has an optional argument?
%       \advisespace
%       Yes. And from this fact comes a limitation: you can't use the left
%       bracket `|[|' as delimiter unless you specify at least an empty
%       optional argument as in `|\lstinline[][var i:integer;[|'.
%       If you forget this, you will either get a ``runaway argument'' error
%       from \TeX, or an error message from the \packagename{keyval} package.
% \end{advise}
%
%
% \subsection{Programming languages}\label{uProgrammingLanguages}
%
% You already know how to activate programming languages---at least Pascal.
% An optional parameter selects particular dialects of a language. For example,
% |language=[77]Fortran| selects Fortran 77 and |language=[XSC]Pascal| does the
% same for Pascal XSC. The general form is
%    {\rstyle\ikeyname{language}}|=|\oarg{dialect}\meta{language}.
% If you want to get rid of keyword, comment, and string detection, use
% |language={}| as an argument to |\lstset| or as optional argument.
%
% Table \ref{uPredefinedLanguages} shows all predefined languages and dialects.
% Use the listed names as \meta{language} and \meta{dialect}, respectively. If
% no dialect or `empty' is given in the table, just don't specify a dialect.
% Each underlined dialect is default; it is selected if you leave out
% the optional argument. The predefined defaults are the newest language
% versions or standard dialects.
%^^A
%^^A  Make table of predefined languages.
%^^A
%\let\lstlanguages\empty
%\makeatletter
%\@for\lst@temp:={lstlang1.sty,lstlang2.sty,lstlang3.sty}\do
%    {\IfFileExists\lst@temp{}{\let\lstlanguages\relax}}
%\makeatother
%\ifx\lstlanguages\relax
%    \PackageWarningNoLine{Listings}
%        {Standard drivers not available.\MessageBreak
%         Please check your installation.\MessageBreak
%         Compilation aborted}
%    \csname @@end\expandafter\endcsname
%\fi
%\lstscanlanguages\lstlanguages{lstlang1.sty,lstlang2.sty,lstlang3.sty}{}^^A
%\def\topfigrule{\hrule\kern-0.4pt\relax}^^A
%\let\botfigrule\topfigrule
%\belowcaptionskip=\smallskipamount
% \begin{table}[tbhp]
% \small
% \caption{Predefined languages.
%          Note that some definitions are preliminary, for example HTML and XML.
%          Each underlined dialect is the default dialect.}^^A
%          \label{uPredefinedLanguages}^^A
% \makeatletter
% \setbox\@tempboxa\hbox{^^A
%    \InputIfFileExists{listings.cfg}{\lst@InputCatcodes}{}}^^A
% \lstprintlanguages\lstlanguages
% \end{table}
%^^A
%^^A end of table
%^^A
%\lstset{defaultdialect=[doc]Pascal}^^A restore
% \begin{advise}
% \item How can I define default dialects?
%       \advisespace
%       Check section \ref{rLanguagesAndStyles} for `\keyname{defaultdialect}'.
% \item I have C code mixed with assembler lines. Can \packagename{listings}
%       pretty-print such source code, i.e.~highlight keywords and comments of
%       both languages?
%       \advisespace
%       `\ikeyname{alsolanguage}|=|\oarg{dialect}\meta{language}' selects a
%       language additionally to the active one. So you only have to write a
%       language definition for your assembler dialect, which doesn't interfere
%       with the definition of C, say. Moreover you might want to use the key
%       `\keyname{classoffset}' described in section \ref{rLanguagesAndStyles}.
% \item How can I define my own language?
%       \advisespace
%       This is discussed in section \ref{rLanguageDefinitions}. And if you
%       think that other people could benefit by your definition, you might
%       want to send it to the address in section \ref{uSoftwareLicense}.
%       Then it will be published under the \LaTeX\ Project Public License.
% \end{advise}
% Note that the arguments \meta{language} and \meta{dialect} are case
% insensitive and that spaces have no effect.
%
% There is at least one language (VDM, Vienna Development Language,
% \url{http://www.vdmportal.org}) which is not directly supported by the
% \packagename{listings} package. It needs a package for its own:
% \packagename{vdmlisting}. On the other hand \packagename{vdmlisting} uses
% the \packagename{listings} package and so it should be mentioned in this
% context.
%
%
% \subsubsection{Preferences}\label{uPreferences}
%
% Sometimes authors of language support provide their own configuration
% preferences. These may come either from their personal experience or
% from the settings in an IDE and can be defined as a \packagename{listings}
% style. From version 1.5b of the \packagename{listings} package on these
% styles are provided as files with the name
% |listings-|\meta{language}|.prf|, \meta{language} is the name of the
% supported programming language in lowercase letters.
%
% So if an user of the \packagename{listings} package wants to use these
% preferences, she/he can say for example when using Python
% \begin{quote}
%     |\input{listings-python.prf}|
% \end{quote}
% at the end of her/his |listings.cfg| configuration file as long as the
% file |listings-python.prf| resides in the \TeX{} search path. Of course
% that file can be changed according to the user's preferences.
%
% At the moment there are five such preferences files:
% \begin{enumerate}
%   \item |listings-acm.prf|
%   \item |listings-bash.prf|
%   \item |listings-fortran.prf|
%   \item |listings-lua.prf|
%   \item |listings-python.prf|
% \end{enumerate}
% All contributors are invited to supply more personal preferences.
%
%
% \subsection{Special characters}\label{uSpecialCharacters}
%
%
% \paragraph{Tabulators}
% You might get unexpected output if your sources contain tabulators.
% The package assumes tabulator stops at columns 9, 17, 25, 33, and so on.
% This is predefined via |tabsize=8|. If you change the eight to the number
% $n$, you will get tabulator stops at columns $n+1,2n+1,3n+1,$ and so on.
% \begin{lstsample}[tabsize]{}{}
%    \lstset{tabsize=2}
%    \begin{lstlisting}
%    123456789
%    	{ one tabulator }
%    		{ two tabs }
%    123		{ 123 + two tabs }
%    \end{lstlisting}
% \end{lstsample}
% For better illustration, the left-hand side uses |tabsize=2| but the verbatim
% code |tabsize=4|. Note that |\lstset| modifies the values for all following
% listings in the same environment or group. This is no problem here since the
% examples are typeset inside minipages. If you want to change settings for a
% single listing, use the optional argument.
%
%
% \paragraph{Visible tabulators and spaces}
% One can make spaces and tabulators visible:
% \begin{lstsample}[showspaces,showtabs,tab]{}{}
%    \lstset{showspaces=true,
%            showtabs=true,
%            tab=\rightarrowfill}
%    \begin{lstlisting}
%        for i:=maxint to 0 do
%        begin
%    	{ do nothing }
%        end;
%    \end{lstlisting}
% \end{lstsample}
% If you request \ikeyname{showspaces} but no \ikeyname{showtabs},
% tabulators are converted to visible spaces.
% The default definition of \ikeyname{tab} produces a `wide visible space'
% \lstinline[showtabs]!	!. So you might want to use |$\to$|, |$\dashv$|
% or something else instead.
% \begin{advise}
% \item Some sort of advice: (1) You should really indent lines of source code
%       to make listings more readable. (2) Don't indent some lines with
%       spaces and others via tabulators. Changing the tabulator size (of your
%       editor or pretty-printing tool) completely disturbs the columns.
%       (3) As a consequence, never share your files with differently tab sized
%       people!^^A true only if you use tabulators, just :-)
% \item To make the \LaTeX\ code more readable, I indent the environments'
%       program listings. How can I remove that indention in the output?
%       \advisespace
%       Read `How to gobble characters' in section \ref{uHowTos}.
% \end{advise}
%
%
% \paragraph{Form feeds}
% Another special character is a form feed causing an empty line by default.
% {\rstyle\ikeyname{formfeed}}|=\newpage| would result in a new page every
% form feed. Please note that such definitions (even the default) might get
% in conflict with frames.
%
%
% \paragraph{National characters}
% If you type in such characters directly as characters of codes 128--255 and
% use them also in listings, let the package know it---or you'll get really
% funny results. {\rstyle\ikeyname{extendedchars}}|=true| allows and
% |extendedchars=false| prohibits \packagename{listings} from handling
% extended characters in listings. If you use them, you should load
% \packagename{fontenc}, \packagename{inputenc} and/or
% any other package which defines the characters.
% \begin{advise}
% \item I have problems using \packagename{inputenc} together with
%       \packagename{listings}.
%       \advisespace
%       This could be a compatibility problem. Make a bug report as described
%       in section \lstref{uTroubleshooting}.
% \end{advise}
% The extended characters don't cover Arabic, Chinese, Hebrew, Japanese, and so
% on---specifically, any encoding which uses multiple bytes per character.
%
% Thus, if you use the a package that supports multibyte characters, such as
% the \packagename{CJK} or \packagename {ucs} packages for Chinese and
% UTF-8 characters, you must avoid letting \packagename{listings}
% process the extended characters.  It is generally best to also specify
% |extendedchars=false| to avoid having \packagename{listings} get entangled
% in the other package's extended-character treatment.
%
% If you do have a listing contained within a CJK environment, and want to have
% CJK characters inside the listing, you can place them within a comment that
% escapes to \LaTeX -- see section \ref{rEscapingToLaTeX} for how to do that.
% (If the listing is not inside a CJK environment, you can simply put a small
% CJK environment within the escaped-to-\LaTeX portion of the comment.)
%
% Similarly, if you are using UTF-8 extended characters in a listing, they must
% be placed within an escape to \LaTeX.
%
% Also, section \ref{uNationalCharacters} has a few details on how to work with
% extended characters in the context of $\Lambda$.
%
%
% \subsection{Line numbers}\label{uLineNumbers}
%
% You already know the keys \ikeyname{numbers}, \ikeyname{numberstyle},
% \ikeyname{stepnumber}, and \ikeyname{numbersep} from section
% \ref{gSeduceToUse}. Here now we deal with continued listings.
% You have two options to get consistent line numbering across listings.
%
% \begin{lstsample}[firstnumber]{\lstset{numbers=left,numberstyle=\tiny,^^A
%       stepnumber=2,numbersep=5pt}}{}
%    \begin{lstlisting}[firstnumber=100]
%    for i:=maxint to 0 do
%    begin
%        { do nothing }
%    end;
%
%    \end{lstlisting}
%    And we continue the listing:
%    \begin{lstlisting}[firstnumber=last]
%    Write('Case insensitive ');
%    WritE('Pascal keywords.');
%    \end{lstlisting}
% \end{lstsample}
% In the example, \ikeyname{firstnumber} is initially set to 100; some lines
% later the value is \texttt{last}, which continues the numbering of the last
% listing. Note that the empty line at the end of the first part is not printed
% here, but it counts for line numbering. You should also notice that you can
% write |\lstset{firstnumber=last}| once and get consecutively numbered code
% lines---except you specify something different for a particular listing.
%
% On the other hand you can use |firstnumber=auto| and name your listings.
% Listings with identical names (case sensitive!) share a line counter.
% \begin{lstsample}[name]{\lstset{numbers=left,numberstyle=\tiny,stepnumber=2,numbersep=5pt}}{}
%    \begin{lstlisting}[name=Test]
%    for i:=maxint to 0 do
%    begin
%        { do nothing }
%    end;
%
%    \end{lstlisting}
%    And we continue the listing:
%    \begin{lstlisting}[name=Test]
%    Write('Case insensitive ');
%    WritE('Pascal keywords.');
%    \end{lstlisting}
% \end{lstsample}
% The next |Test| listing goes on with line number {\makeatletter\lstno@Test},
% no matter whether there are other listings in between.
%
% You can also select the lines to be printed, the options
% `\ikeyname{linerange}' and `\ikeyname{consecutivenumbers}' are your
% friend. In a presentation for example you don't need comments for your
% programs, so you prefer the line numbers being consecutively numbered,
% but the results should reflect the behaviour of the program---you omit
% parts of the lengthy output. So
% you may have the following program and its results.
% \begin{lstsample}[name]{\lstset{numbers=left,numberstyle=\tiny,stepnumber=1,numbersep=5pt}}{}
%    \begin{lstlisting}[name=Test,
%      language={[ansi]C},
%      linerange={1-4,6-7,10-14,
%        17-19,21-22},
%      firstnumber=1]
%    #include <stdio.h>
%    #include <stdlib.h>
%
%    int main(int argc,char* argv[]){
%      /* declaring variables */
%      int i;
%      int limit;
%
%      /* checking arguments */
%      if ( argc > 1 ) {
%        limit = atoi(argv[1]);
%      } else {
%        limit = 100;
%      }
%
%      /* counting lines */
%      for (i = 1;i <= limit;i++) {
%        printf("Line no. %3.0d\n", i);
%      }
%
%      return 0;
%    }
%
%    \end{lstlisting}
%    And these are the results:
%    \begin{lstlisting}[language={},
%      linerange={1-2,6-7},
%      consecutivenumbers=false]
%    Line no.   1
%    Line no.   2
%    Line no.   3
%    Line no.   4
%    Line no.   5
%    Line no.   6
%    Line no.   7
%    \end{lstlisting}
% \end{lstsample}
%
% \begin{advise}
% \item Okay. And how can I get decreasing line numbers?
%       \advisespace
%       Sorry, what?
%       \advisespace
%       Decreasing line numbers as on page \pageref{rDecreasingLabels}.
%       \advisespace
%       May I suggest to demonstrate your individuality by other means?
%       If you differ, you should try a negative `\ikeyname{stepnumber}'
%       (together with `\ikeyname{firstnumber}').
% \end{advise}
%
% Read section \ref{uHowTos} on how to reference line numbers.
%
%
% \subsection{Layout elements}
%
% It's always a good idea to structure the layout by vertical space,
% horizontal lines, or different type sizes and typefaces. The best to stress
% whole listings are---not all at once---colours, frames, vertical space, and
% captions. The latter are also good to refer to listings, of course.
%
% \paragraph{Vertical space}
% The keys {\rstyle\ikeyname{aboveskip}} and {\rstyle\ikeyname{belowskip}}
% control the vertical space above and below displayed listings. Both keys get
% a dimension or skip as value and are initialized to |\medskipamount|.
%
% \paragraph{Frames}
% The key \ikeyname{frame} takes the verbose values \keyvalue{none},
% \keyvalue{leftline}, \keyvalue{topline}, \keyvalue{bottomline},
% \keyvalue{lines} (top and bottom), \keyvalue{single} for single frames, or
% \keyvalue{shadowbox}.
% \begin{lstsample}[frame]{}{}
%    \begin{lstlisting}[frame=single]
%    for i:=maxint to 0 do
%    begin
%        { do nothing }
%    end;
%    \end{lstlisting}
% \end{lstsample}
% \begin{advise}
% \item The rules aren't aligned.
%       \advisespace
%       This could be a bug of this package or a problem with your
%       \texttt{.dvi} driver. \emph{Before} sending a bug report to the package
%       author, modify the parameters described in section \ref{rFrames}
%       heavily. And do this step by step!
%       For example, begin with `|framerule=10mm|'. If the rules are
%       misaligned by the same (small) amount as before, the problem does not
%       come from the rule width. So continue with the next parameter.  Also,
%       Adobe Acrobat sometimes has single-pixel rounding errors which can
%       cause small misalignments at the corners when PDF files are displayed
%       on screen; these are unfortunately normal.
% \end{advise}
% Alternatively you can control the rules at the \texttt{t}op, \texttt{r}ight,
% \texttt{b}ottom, and \texttt{l}eft directly by using the four initial letters
% for single rules and their upper case versions for double rules.
% \begin{lstsample}[frame]{}{}
%    \begin{lstlisting}[frame=trBL]
%    for i:=maxint to 0 do
%    begin
%        { do nothing }
%    end;
%    \end{lstlisting}
% \end{lstsample}
% Note that a corner is drawn if and only if both adjacent rules are requested.
% You might think that the lines should be drawn up to the edge, but what's
% about round corners? The key \ikeyname{frameround} must get exactly four
% characters as value. The first character is attached to the upper right
% corner and it continues clockwise. `\texttt{t}' as character makes the
% corresponding corner round.
% \begin{lstsample}[frameround]{}{}
%    \lstset{frameround=fttt}
%    \begin{lstlisting}[frame=trBL]
%    for i:=maxint to 0 do
%    begin
%        { do nothing }
%    end;
%    \end{lstlisting}
% \end{lstsample}
% Note that \ikeyname{frameround} has been used together with |\lstset| and thus
% the value affects all following listings in the same group or environment.
% Since the listing is inside a \texttt{minipage} here, this is no problem.
% \begin{advise}
% \item Don't use frames all the time, and in particular not with short listings.
%       This would emphasize nothing. Use frames for $10\%$ or even less of
%       your listings, for your most important ones.
% \item If you use frames on floating listings, do you really want frames?
%       \advisespace
%       No, I want to separate floats from text.
%       \advisespace
%       Then it is better to redefine \LaTeX's `|\topfigrule|' and
%       `|\botfigrule|'. For example, you could write
%       `|\renewcommand*\topfigrule{\hrule\kern-0.4pt\relax}|' and make the
%       same definition for |\botfigrule|.
% \end{advise}
%
% \paragraph{Captions}
% Now we come to \ikeyname{caption} and \ikeyname{label}. You might guess
% (correctly) that they can be used in the same manner as \LaTeX's |\caption|
% and |\label| commands, although here it is also possible to have a caption
% regardless of whether or not the listing is in a float:
% \begin{lstsample}[caption,label]{\lstset{xleftmargin=.05\linewidth}}{}
%    \begin{lstlisting}[caption={Useless code},label=useless]
%    for i:=maxint to 0 do
%    begin
%        { do nothing }
%    end;
%    \end{lstlisting}
% \end{lstsample}
% Afterwards you could refer to the listing via |\ref{useless}|. By default
% such a listing gets an entry in the list of listings, which can be printed
% with the command {\rstyle\icmdname\lstlistoflistings}. The key
% {\rstyle\ikeyname{nolol}} suppresses an entry for both the environment or
% the input command. Moreover, you can specify a short caption for the list
% of listings:
%    \keyname{caption}|={|\oarg{short}\meta{long}|}|.
% Note that the whole value is enclosed in braces since an optional value is
% used in an optional argument.
%
% If you don't want the label \texttt{\lstlistingname} plus number, you should
% use \ikeyname{title}:
% \begin{lstsample}[title]{\lstset{xleftmargin=.05\linewidth}}{}
%    \begin{lstlisting}[title={`Caption' without label}]
%    for i:=maxint to 0 do
%    begin
%        { do nothing }
%    end;
%    \end{lstlisting}
% \end{lstsample}
% \begin{advise}
% \item Something goes wrong with `\keyname{title}' in my document: in front of
%       the title is a delimiter.
%       \advisespace
%       The result depends on the document class; some are not compatible.
%       Contact the package author for a work-around.
% \end{advise}
%
% \paragraph{Colours}
% One more element. You need the \packagename{color} package and can then
% request coloured background via
% \ikeyname{backgroundcolor}|=|\meta{color command}.
% \begin{advise}
% \item Great! I love colours.
%       \advisespace
%       Fine, yes, really. And I like to remind you of the warning about
%       striking styles on page \pageref{wStrikingStyles}.
% \end{advise}
%\ifcolor
% \begin{lstxsample}[backgroundcolor]
%    \lstset{backgroundcolor=\color{yellow}}
% \end{lstxsample}
%\else
% \begin{verbatim}
%    color package not installed\end{verbatim}
%\fi
% \begin{lstsample}{}{}
%    \begin{lstlisting}[frame=single,
%                       framerule=0pt]
%    for i:=maxint to 0 do
%    begin
%        j:=square(root(i));
%    end;
%    \end{lstlisting}
% \end{lstsample}
% The example also shows how to get coloured space around the whole listing:
% use a frame whose rules have no width.
%
%
% \subsection{Emphasize identifiers}\label{uEmphasizeIdentifiers}
%
% Recall the pretty-printing commands and environment. |\lstinline| prints
% code snippets, |\lstinputlisting| whole files, and \texttt{lstlisting}
% pieces of code which reside in the \LaTeX\ file. And what are these
% different `types' of source code good for? Well, it just happens that a
% sentence contains a code fragment. Whole files are typically included in or
% as an appendix. Nevertheless some books about programming also include such
% listings in normal text sections---to increase the number of pages.
% Nowadays source code should be shipped on disk or CD-ROM and only the main
% header or interface files should be typeset for reference. So, please, don't
% misuse the \packagename{listings} package. But let's get back to the topic.
%
% Obviously `\texttt{lstlisting} source code' isn't used to make an executable
% program from. Such source code has some kind of educational purpose or even
% didactic.
% \begin{advise}
% \item What's the difference between educational and didactic?
%       \advisespace
%       Something educational can be good or bad, true or false.
%       Didactic is true by definition.^^A :-)
% \end{advise}
% Usually \emph{keywords} are highlighted when the package typesets a piece of
% source code. This isn't necessary for readers who know the programming
% language well. The main matter is the presentation of interface, library or
% other functions or variables. If this is your concern, here come the right
% keys. Let's say, you want to emphasize the functions |square| and |root|,
% for example, by underlining them. Then you could do it like this:
% \begin{lstxsample}[emph,emphstyle]
%    \lstset{emph={square,root},emphstyle=\underbar}
% \end{lstxsample}
% \begin{lstsample}{}{}
%    \begin{lstlisting}
%    for i:=maxint to 0 do
%    begin
%        j:=square(root(i));
%    end;
%    \end{lstlisting}
% \end{lstsample}
% \begin{advise}
% \item Note that the list of identifiers |{square,root}| is enclosed in
%       braces. Otherwise the \packagename{keyval} package would complain
%       about an undefined key \keyname{root} since the comma finishes the
%       key=value pair.
%       Note also that you \emph{must} put braces around the value if you
%       use an optional argument of a key inside an optional argument of a
%       pretty-printing command. Though it is not necessary, the following
%       example uses these braces. They are typically forgotten when they
%       become necessary,
% \end{advise}
%
% Both keys have an optional \meta{class number} argument for multiple
% identifier lists:
%\ifcolor
% \begin{lstxsample}[emph,emphstyle]
%    \lstset{emph={square},      emphstyle=\color{red},
%            emph={[2]root,base},emphstyle={[2]\color{blue}}}
% \end{lstxsample}
%\else
% \begin{lstxsample}[emph,emphstyle]
%    \lstset{emph={square},      emphstyle=\underbar,
%            emph={[2]root,base},emphstyle={[2]\fbox}}
% \end{lstxsample}
%\fi
% \begin{lstsample}{}{}
%    \begin{lstlisting}
%    for i:=maxint to 0 do
%    begin
%        j:=square(root(i));
%    end;
%    \end{lstlisting}
% \end{lstsample}
% \begin{advise}
% \item What is the maximal \meta{class number}?
%       \advisespace
%       $2^{31}-1=2\,147\,483\,647$. But \TeX's memory will exceed before you
%       can define so many different classes.
% \end{advise}
%
% One final hint: Keep the lists of identifiers disjoint. Never use a keyword
% in an `emphasize' list or one name in two different lists. Even if your
% source code is highlighted as expected, there is no guarantee that it is
% still the case if you change the order of your listings or if you use the
% next release of this package.
%
%
%\iffalse
% \subsection{*Listing alignment}\label{uListingAlignment}
%
% The examples are typeset with centered \texttt{minipage}s. That's the reason
% why you can't see that line numbers are printed in the margin. Now we
% separate the minipage margin and the minipage by a vertical rule:
% \begin{lstsample}{\lstset{frame=l,framesep=0pt,numberstyle=\tiny,stepnumber=2,^^A
%     numbersep=5pt}}{}
%    Some text before
%    \begin{lstlisting}
%    for i:=maxint to 0 do
%    begin
%        { do nothing }
%    end;
%    \end{lstlisting}
% \end{lstsample}
% The listing is lined up with the normal text. The parameter \ikeyname{xleftmargin}
% moves the listing to the right (or left if the dimension is negative).
% \begin{lstsample}{\lstset{frame=l,framesep=0pt,numberstyle=\tiny,stepnumber=2,^^A
%     numbersep=5pt}}{}
%    Some text before
%    \begin{lstlisting}[xleftmargin=15pt]
%    for i:=maxint to 0 do
%    begin
%        { do nothing }
%    end;
%    \end{lstlisting}
%
%    \begin{lstlisting}{ }
%    Write('Insensitive');
%    WritE('keywords.');
%    \end{lstlisting}
% \end{lstsample}
% Note again that optional arguments change settings for single listings.
%
% If you use environments like \texttt{itemize} or \texttt{enumerate}, there
% is `natural' indention coming from these environments. By default the
% \packagename{listings} package respects this. But you might use
% |resetmargins=true| (or |false|) to make your own decision. You can use it
% together with |xleftmargin|, of course.
% \begin{advise}
% \item I get heavy overfull |\hbox|es from some listings.
%       \advisespace
%       This comes from long lines in your listings. You have some options
%       to get rid of the overful |\hbox|es. Firstly I recommend to typeset
%       listings in smaller fonts than the surrounding text, for example
%       `|basicstyle=\small|'. Secondly you might want to use the flexible
%       column format. Thirdly you can increase the line width or set it
%       explicitly, refer section \ref{rMarginsAndLineShape}.
%       If all this doesn't help, you might want to change
%       `\ikeyname{basewidth}', but be careful! The two unknown items are
%       explained in the next section.
% \end{advise}
%
% You might need to control the vertical position of listings with the
% \ikeyname{boxpos} key, for example, if you use them in \texttt{minipage} or
% \texttt{tabular} environments. Here `listings' means \texttt{lstlisting} or
% |\lstinputlisting|. As the following example shows, you can even place such
% listings inside paragraphs, but you must force the package to do this by
% enclosing the listing in |\hbox{| and |}|.
% \begin{advise}
% \item Is it good form to use the \TeX-primitive `|\hbox|' in a \LaTeX\
%       document?
%       \advisespace
%       No, it's not. But \LaTeX's `|\mbox|' does not work in this example:
% \end{advise}
% \begin{lstsample}{}{}
%    Here are some multi-line listings inside a paragraph.
%    The `boxpos' key controls their vertical alignment:
%    \hbox{\begin{lstlisting}[boxpos=c]
%    center
%    center
%    \end{lstlisting}}
%    \hbox{\begin{lstlisting}[boxpos=b]
%    bottom baseline
%    bottom baseline
%    \end{lstlisting}}
%    \hbox{\begin{lstlisting}[boxpos=t]
%    top baseline
%    top baseline
%    \end{lstlisting}}
% \end{lstsample}
%\fi
%
%
% \subsection{Indexing}\label{uIndexing}
%
% Indexing is just like emphasizing identifiers---I mean the usage:
% \begin{lstxsample}[index]
%    \lstset{index={square},index={[2]root}}
% \end{lstxsample}
% \begin{lstsample}{}{}
%    \begin{lstlisting}
%    for i:=maxint to 0 do
%    begin
%        j:=square(root(i));
%    end;
%    \end{lstlisting}
% \end{lstsample}
% Of course, you can't see anything here. You will have to look at the index.
% \begin{advise}
% \item Why is the `\ikeyname{index}' key able to work with multiple identifier
%       lists?
%       \advisespace
%       This question is strongly related to the `{\rstyle\ikeyname{indexstyle}}'
%       key. Someone might want to create multiple indexes or want to insert
%       prefixes like `|constants|', `|functions|', `|keywords|', and so on.
%       The `\ikeyname{indexstyle}' key works like the other style keys except
%       that the last token \emph{must} take an argument, namely the
%       (printable form of the) current identifier.
%
%       You can define `|\newcommand\indexkeywords[1]{\index{keywords, #1}}|'
%       and make similar definitions for constant or function names. Then
%       `|indexstyle=[1]\indexkeywords|' might meet your purpose. This becomes
%       easier if you want to create multiple indexes with the
%       \href{http://mirror.ctan.org/macros/latex/contrib/camel}
%       {\packagename{index}} package.
%       If you have defined appropriate new indexes, it is possible to write
%       `|indexstyle=\index[keywords]|', for example.
%
% \item Let's say, I want to index all keywords. It would be annoying to
%       type in all the keywords again, specifically if the used programming
%       language changes frequently.
%       \advisespace
%       Just read ahead.
% \end{advise}
% The \ikeyname{index} key has in fact two optional arguments. The first is the
% well-known \meta{class number}, the second is a comma separated list of other
% keyword classes whose identifiers are indexed. The indexed identifiers then
% change automatically with the defined keywords---not automagically, it's not
% an illusion.^^A :-)
%
% Eventually you need to know the names of the keyword classes. It's usually
% the key name followed by a class number, for example, |emph2|, |emph3|,
% \ldots, |keywords2| or |index5|. But there is no number for the first order
% classes |keywords|, |emph|, |directives|, and so on.
% \begin{advise}
% \item `|index=[keywords]|' does not work.
%       \advisespace
%       The package can't guess which optional argument you mean. Hence you
%       must specify both if you want to use the second one. You should try
%       `|index=[1][keywords]|'.
% \end{advise}
%
%
% \subsection{Fixed and flexible columns}\label{uFixedAndFlexibleColumns}
%
% The first thing a reader notices---except different styles for keywords,
% etc.---is the column alignment. Arne John Glenstrup invented the flexible
% column format in 1997. Since then some efforts were made to develop this
% branch farther. Currently four column formats are provided: fixed, flexible,
% space-flexible, and full flexible. Take a close look at the following
% examples.
% \begin{center}
% \lstset{style={},language={}}
% \def\sample{\begin{lstlisting}^^J WOMEN\ \ are^^A
%                               ^^J \ \ \ \ \ \ \ MEN^^A
%                               ^^J WOMEN are^^A
%                               ^^J better MEN^^J \end{lstlisting}}
% \begin{tabular}{@{}c@{\qquad\quad}c@{\qquad\quad}c@{\qquad\quad}c@{}}
% {\rstyle\ikeyname{columns}}|=| & \texttt{fixed} & \texttt{flexible} & \texttt{fullflexible}\\
%          & (at {\makeatletter\lst@widthfixed})
%          & (at {\makeatletter\lst@widthflexible})
%          & (at {\makeatletter\lst@widthflexible})\\
% \noalign{\medskip}
%   \lstset{basicstyle=\ttfamily,basewidth=0.51em}\sample
% & \lstset{columns=fixed}\sample
% & \lstset{columns=flexible}\sample
% & \lstset{columns=fullflexible}\sample
% \end{tabular}
% \end{center}
% \begin{advise}
% \item Why are women better men?
%       \advisespace
%       Do you want to philosophize? Well, have I ever said that the
%       statement ``women are better men'' is true? I can't even remember this
%       about ``women are men'' \ldots . ^^A ;-)
% \end{advise}
% In the abstract one can say: The fixed column format ruins the spacing
% intended by the font designer, while the flexible formats ruin the column
% alignment (possibly) intended by the programmer. Common to all is that the
% input characters are translated into a sequence of basic output units like
% \begingroup \lstset{gobble=6,xleftmargin=\leftmargini}
% \makeatletter
%^^A  Make \fbox around each output unit.
% \fboxsep=0pt
% \def\lst@alloverstyle#1{\fbox{\kern-\fboxrule\strut#1}\kern-\fboxrule}
% \begin{lstlisting}[basewidth=1em]
%     if x=y then write('align')
%            else print('align');
% \end{lstlisting}
% Now, the fixed format puts $n$ characters into a box of width $n\times{}
% $`base width', where the base width is {\makeatletter\lst@widthfixed} in the
% example. The format shrinks and stretches the space between the characters
% to make them fit the box. As shown in the example, some character strings look
%    \hbox to 2em{b\hss a\hss d}
% or
%    \hbox to 2em{w\hss o\hss r\hss s\hss e},
% but the output is vertically aligned.
% \endgroup
%
% If you don't need or like this, you should use a flexible format. All
% characters are typeset at their natural width. In particular, they never
% overlap. If a word requires more space than reserved, the rest of the line
% simply moves to the right. The difference between the three formats is that
% the full flexible format cares about nothing else, while the normal flexible
% and space-flexible formats try to fix the column alignment if a character
% string needs less space than `reserved'.  The normal flexible format will
% insert make-up space to fix the alignment at spaces, before and after
% identifiers, and before and after sequences of other characters; the
% space-flexible format will only insert make-up space by stretching
% existing spaces.  In the flexible example above, the two MENs are vertically
% aligned since some space has been inserted in the fourth line to fix the
% alignment. In the full flexible format, the two MENs are not aligned.
%
% Note that both flexible modes printed the two blanks in the first line as a
% single blank, but for different reasons: the normal flexible format fixes
% the column alignment (as would the space-flexible format), and the full
% flexible format doesn't care about the second space.
%
%
 

    % \DocInput{listings.dtx}
\end{document}