\PassOptionsToPackage{no-math}{fontspec}%禁用了使用fontspec宏包中的数学字体功能。
\PassOptionsToPackage{AutoFakeBold=true,AutoFakeSlant=true}{xeCJK}%让xeCJK宏包自动产生伪粗体和伪斜体效果。

\documentclass[a4paper]{ltxdoc}
\usepackage[heading=true
,scheme=chinese%中文方案
,fontset=none%不使用默认的字体设置
,space=auto%自动调整中英文间距
]{ctex}
\setCJKmainfont{FangZhengShuSong-GBK-1.ttf}[Path=/Users/virhuiai/hlProjects/Latex-Typesetting-Hub/font/方正/]%设置文本的中文有衬线字体
\setCJKsansfont{FangZhengHeiTi-GBK-1.ttf}[Path=/Users/virhuiai/hlProjects/Latex-Typesetting-Hub/font/方正/]%设置文本的中文无衬线字体为
\setCJKmonofont{FangZhengFangSong-GBK-1.ttf}[Path=/Users/virhuiai/hlProjects/Latex-Typesetting-Hub/font/方正/] %设置文本的中文等宽字体 
\usepackage[a3paper,landscape]{geometry}
\usepackage{paracol}
\columnsep=2em 


\DisableCrossrefs
\OnlyDescription

\usepackage{lstdoc,textcomp}
\usepackage{mdframed}           % frames for external files
\usepackage{moreverb}           % writing external files
\usepackage{xcolor}             % because of colouring the background
 
\makeindex

\begin{document}
\DocInput{listings.dtx.pre}
% 
^^A
^^A  The long awaited beginning of documentation
^^A =============================================
^^A
% \newbox\abstractbox
% \setbox\abstractbox=\vbox{
% 	\begin{abstract}
% 	The \packagename{listings} package is a source code printer for \LaTeX.
% 	You can typeset stand alone files as well as listings with an environment
%   similar to \texttt{verbatim} as well as you can print code snippets using
%   a command similar to |\verb|.
% 	Many parameters control the output and if your preferred programming
%   language isn't already supported, you can make your own definition.
% 	\end{abstract}}

\title{\vspace*{-2\baselineskip}The \textsf{Listings} Package}
\author{Copyright 1996--2004, Carsten Heinz%
   \\ Copyright 2006--2007, Brooks Moses
   \\ Copyright 2013--, Jobst Hoffmann
   \\ Maintainer: Jobst Hoffmann\thanks{Jobst %
      Hoffmann became the maintainer of the \packagename{listings}
      package in 2013; see the Preface for details.}~ %
   \textless\lstemail\textgreater\and 翻译:virhuiai}
\date{2020/03/24\enspace\enspace Version 1.8d\ %\box\abstractbox
}
\ifhyper
  \hypersetup{pdftitle=The Listings Package,
              pdfsubject=Package guide,
              pdfauthor=Jobst Hoffmann <j.hoffmann(at)fh-aachen.de>,%
              pdfkeywords={source code formatter, programming languages}}
\fi

% \csname @twocolumntrue\endcsname
\maketitle

\begin{paracol}{2}
\renewcommand{\abstractname}{Abstract}
\begin{abstract}
The \packagename{listings} package is a source code printer for \LaTeX.
You can typeset stand alone files as well as listings with an environment
similar to \texttt{verbatim} as well as you can print code snippets using
a command similar to |\verb|.
Many parameters control the output and if your preferred programming
language isn't already supported, you can make your own definition.
\end{abstract}
\switchcolumn
\renewcommand{\abstractname}{摘要}
\begin{abstract}
\texttt{listings}宏包是一个用于排版\LaTeX 的源代码排版工具。您可以排版独立的文件,也可以使用类似于\texttt{verbatim}的环境排版代码列表,还可以使用类似于 |\verb| 的命令打印代码片段。许多参数可以控制输出,如果您的首选编程语言尚未支持,您可以自定义定义。
\end{abstract}

\end{paracol}


^^A \enlargethispage{2\baselineskip}
\csname @starttoc\endcsname{toc}
% \onecolumn


% \columnratio{0.55}
\begin{paracol}{2}
\section*{Preface}
\switchcolumn
\section*{前言}
\switchcolumn[0]*%%%%%%%%%%%%
\textbf{Transition of package maintenance}
The \TeX\ world lost contact with Carsten Heinz in late 2004, shortly after
he released version 1.3b of the \packagename{listings} package.  After many
attempts to reach him had failed, Hendri Adriaens took over maintenance of
the package in accordance with the LPPL's procedure for abandoned packages.
He then passed the maintainership of the package to Brooks Moses, who had
volunteered for the position while this procedure was going through. The
result is known as listings version 1.4.
\switchcolumn
\textbf{包维护的转变}
在2004年底,\TeX\ 领域失去了与Carsten Heinz的联系,他在发布了\packagename{listings}宏包的1.3b版本后不久便失去了联系。在多次联系失败后,Hendri Adriaens根据LPPL对废弃包的处理程序接管了该包的维护工作。然后,他将包的维护权移交给了Brooks Moses,在此过程中,Brooks Moses自愿担任了这个职位。结果就是现在的listings 1.4版本。
\switchcolumn[0]*%%%%%%%%%%%%
This release, version 1.5, is a minor maintenance release since
I accepted maintainership of the package.  I would like to thank Stephan
Hennig who supported the Lua language definitions. He is the one who
asked for the integration of a new language and gave the impetus to me to
become the maintainer of this package.
\switchcolumn
这个发布的版本,1.5版,是我接手维护该宏包之后的一个小型维护版本。我要感谢Stephan Hennig对Lua语言定义的支持。他是那个提出整合新语言并给我成为这个宏包的维护者提供动力的人。
\switchcolumn[0]*%%%%%%%%%%%%
\textbf{News and changes}
Version 1.5 is the fifth bugfix release.  There are no changes
in this version, but two extensions: support of modern Fortran (2003,
2008) and Lua.
\switchcolumn
\textbf{新闻和变化}
1.5版本是第五个修复错误版本。这个版本没有任何变化,只有两个扩展:对现代Fortran(2003、2008)和Lua的支持。
\switchcolumn[0]*%%%%%%%%%%%%
\vfill
\textbf{Thanks}
There are many people I have to thank for fruitful communication, posting
their ideas, giving error reports, adding programming languages to
\texttt{lstdrvrs.dtx}, and so on. Their names are listed in section
\ref{uClosingAndCredits}.
\switchcolumn
\textbf{感谢}
有很多人我需要感谢他们的积极交流、发表意见、提供错误报告、添加编程语言到\texttt{lstdrvrs.dtx}等等。他们的名字将在\ref{uClosingAndCredits}节中列出。
\switchcolumn[0]*%%%%%%%%%%%%
\textbf{Trademarks}
Trademarks appear throughout this documentation without any trademark
symbol; they are the property of their respective trademark owner.
There is no intention of infringement; the usage is to the benefit of the
trademark owner.
\switchcolumn
\textbf{商标}
本文档中的商标没有使用任何商标符号,它们是其各自商标所有者的财产。没有侵权的意图;使用是为了商标所有者的利益。
\end{paracol}
% \begin{paracol}{2}
\part{User's guide}
\switchcolumn
\part{用户指南}
\switchcolumn[0]*%%%%%%%%%%%%
\section{Getting started}\label{uGettingStarted}
\switchcolumn
\section{开始使用}
\end{paracol}
% \columnratio{0.55}
\begin{paracol}{2}
\subsection{A minimal file}\label{uAMinimalFile}
\switchcolumn
\subsection{一个最小的文件}
\switchcolumn[0]*%%%%%%%%%%%%
Before using the \packagename{listings} package, you should be familiar with
the \LaTeX\ typesetting system. You need not to be an expert.
Here is a minimal file for \packagename{listings}.
\switchcolumn
在使用\packagename{listings}宏包之前,您应该熟悉\LaTeX 排版系统。您不需要成为专家。下面是一个\packagename{listings}宏包的最小文件示例。
\switchcolumn[0]*%%%%%%%%%%%%
\begin{verbatim}
    % \documentclass{article}
    % \usepackage{listings}

    % \begin{document}
    % \lstset{language=Pascal}

    %   % Insert Pascal examples here.

    % \end{document}\end{verbatim}
\switchcolumn
\begin{verbatim}
    % \documentclass{article}
    % \usepackage{listings}

    % \begin{document}
    % \lstset{language=Pascal}

    % % 在此处插入Pascal示例代码。

    % \end{document}\end{verbatim}
\switchcolumn[0]*%%%%%%%%%%%%
Now type in this first example and run it through \LaTeX.
\switchcolumn
现在输入这个第一个示例并运行它通过\LaTeX。
\end{paracol}




\begin{advise}
\begin{paracol}{2}    
\item Must I do that really?
    \advisespace
    Yes and no. Some books about programming say this is good.
    What a mistake! Typing takes time---which is wasted if the code is clear to
    you. And if you need that time to understand what is going on, the
    author of the book should reconsider the concept of presenting the
    crucial things---you might want to say that about this guide even---or
    you're simply inexperienced with programming. If only the latter case
    applies, you should spend more time on reading (good) books about
    programming, (good) documentations, and (good) source code from other
    people. Of course you should also make your own experiments.
    You will learn a lot. However, running the example through \LaTeX\
    shows whether the \packagename{listings} package is installed correctly.
    \switchcolumn
    \item 我真的需要这样做吗?
    \advisespace
    既是又非。有些关于编程的书籍说这样做是好的。这是一个错误!输入需要时间,如果代码对您来说很清晰,那么这些时间就是浪费的。如果您需要时间来理解正在发生的事情,那么书的作者应该重新考虑介绍关键内容的概念---甚至可能会对这个指南说出这样的话---或者您对编程经验不够丰富。如果只是后一种情况,您应该花更多时间阅读(好的)关于编程的书籍,(好的)文档和其他人的(好的)源代码。当然,您也应该进行自己的实验。您将学到很多。然而,通过\LaTeX 运行示例可以显示\packagename{listings}宏包是否正确安装。
\switchcolumn[0]*
\item The example doesn't work.
    \advisespace
    Are the two packages \packagename{listings} and \packagename{keyval}
    installed on your system? Consult the administration tool of your
    \TeX\ distribution, your system administrator, the local \TeX\ and
    \LaTeX\ guides, a \TeX\ FAQ, and section \ref{rInstallation}---in
    that order. If you've checked \emph{all} these sources and are
    still helpless, you might want to write a post to a \TeX\ newsgroup
    like \texttt{comp.text.tex}.
    \switchcolumn
    \item 这个示例不起作用。
    \advisespace
    您的系统上是否安装了两个包\packagename{listings}和\packagename{keyval}?请查阅您的\TeX 发行版的管理工具、系统管理员、本地\TeX 和\LaTeX 指南、\TeX 常见问题解答,并按照这个顺序查找。如果您已经检查了\emph{所有}这些来源,仍然无助,您可能想在\TeX 新闻组(如\texttt{comp.text.tex})上发布一篇帖子。
    \switchcolumn[0]*    
\item Should I read the software license before using the package?
    \advisespace
    Yes, but read this \emph{Getting started} section first to decide
    whether you are willing to use the package.^^A ;-)
    \switchcolumn
    \item 在使用该宏包之前,我应该阅读软件许可证吗?
\advisespace
是的,但在使用该宏包之前,请先阅读本“入门”部分,以决定您是否愿意使用该宏包。^^A ;-)
\end{paracol}
\end{advise}
% \columnratio{0.55}
\begin{paracol}{2}
\subsection{Typesetting listings}
\switchcolumn
\subsection{排版代码}
\switchcolumn[0]*%%%%%%%%%%%%
Three types of source codes are supported: code snippets, code segments, and
listings of stand alone files.  Snippets are placed inside paragraphs and the
others as separate paragraphs---the difference is the same as between text
style and display style formulas.
\switchcolumn
支持三种类型的源代码:代码片段、代码段和独立文件的列表。片段放置在段落内,其他两种放置为独立段落---区别与文本样式和展示样式公式之间的区别。
\end{paracol}
\begin{advise}
\columnratio{0.55}
\begin{paracol}{2}

\item No matter what kind of source you have, if a listing contains national
    characters like \'e, \L, \"a, or whatever, you must tell the
    package about it! Section \lstref{uSpecialCharacters} discusses this issue.
    \switchcolumn
    \item 不管您有什么样的源代码,如果列表中包含如'e、\L、"a等国际字符,您必须告诉宏包!请参阅\lstref{uSpecialCharacters}节讨论此问题。      
\end{paracol}
\end{advise}

\begin{paracol}{2}
\switchcolumn[0]*%%%%%%%%%%%%
\textbf{Code snippets}
The well-known \LaTeX\ command |\verb| typesets code snippets verbatim.
The new command |\lstinline| pretty-prints the code, for example
`\lstinline!var i:integer;!' is typeset by
`{\rstyle|\lstinline|}|!var i:integer;!|'. The exclamation marks delimit
the code and can be replaced by any character not in the code;
|\lstinline$var i:integer;$| gives the same result.
\switchcolumn
\textbf{代码片段}
众所周知,\LaTeX 命令|\verb|以逐字方式排版代码片段。新命令 |\lstinline| 将代码漂亮地打印出来,\\例如 `\lstinline!var i:integer;!' 通过 `{\rstyle|\lstinline|}|!var i:integer;!|' 排版。感叹号用于界定代码,并且可以用代码中不存在的任何字符替换;|\lstinline$var i:integer;$|会得到相同的结果。% todo 后面重新调整,用tcolor的来排
\switchcolumn[0]*%%%%%%%%%%%%
\textbf{Displayed code}
The \texttt{lstlisting} environment typesets the enclosed source code. Like
most examples, the following one shows verbatim \LaTeX\ code on the right
and the result on the left. You might take the right-hand side, put it into
the minimal file, and run it through \LaTeX.
\switchcolumn
\textbf{显示的代码}
\texttt{lstlisting}环境用于排版所包含的源代码。像大多数示例一样,下面的示例在右边展示了逐字排版的\LaTeX 代码,左边展示了结果。您可以将右侧的代码放入最小文件中,然后通过\LaTeX 运行它。
% \switchcolumn[0]*%%%%%%%%%%%%
\begin{lstsample}[lstlisting]{}{}
        \begin{lstlisting}
        for i:=maxint to 0 do
        begin
            { do nothing }
        end;

        Write('Case insensitive ');
        WritE('Pascal keywords.');
        \end{lstlisting}
\end{lstsample}
\switchcolumn[0]*%%%%%%%%%%%%
It can't be easier.
\switchcolumn
这是再容易不过了。
\end{paracol}

\begin{advise}
\begin{paracol}{2}    
\item That's not true. The name `\texttt{listing}' is shorter.
\advisespace
Indeed. But other packages already define environments with that name.
To be compatible with such packages, all commands and environments of
the \packagename{listings} package use the prefix `\texttt{lst}'.
\switchcolumn
\item 这不对。名字 `\texttt{listing}' 更短。 \advisespace 确实如此。但是,其他包已经定义了同名的环境。 为了与这些包兼容,\packagename{listings}包的所有命令和环境都使用前缀\texttt{lst}'。
\end{paracol}
\end{advise}

\begin{paracol}{2}
The environment provides an optional argument. It tells the package to
perform special tasks, for example, to print only the lines 2--5:
\switchcolumn
该环境提供了一个可选参数。它告诉包执行特殊任务,例如只打印2到5行:
\begin{lstsample}{\lstset{frame=trbl,framesep=0pt}\label{gFirstKey=ValueList}}{}
        \begin{lstlisting}[firstline=2,
                            lastline=5]
        for i:=maxint to 0 do
        begin
            { do nothing }
        end;

        Write('Case insensitive ');
        WritE('Pascal keywords.');
        \end{lstlisting}
\end{lstsample}
\end{paracol}


\begin{advise}
    \begin{paracol}{2}
\item Hold on! Where comes the frame from and what is it good for?
\advisespace
You can put frames around all listings except code snippets.
You will learn how later. The frame shows that empty lines at the end
of listings aren't printed. This is line 5 in the example.
\switchcolumn
\item 等等!这个框从哪里来的?它有什么作用?
\advisespace
除了代码片段外,你可以给所有的代码片段加上框。
稍后你会了解如何做到这一点。框表明列表末尾的空行不会被打印出来。在示例中,这是第5行。
\switchcolumn[0]*
\item Hey, you can't drop my empty lines!
\advisespace
You can tell the package not to drop them:
The key `\ikeyname{showlines}' controls these empty lines and is
described in section \ref{rTypesettingListings}. Warning: First
read ahead on how to use keys in general.
\switchcolumn
\item 嘿,你不能删除我的空行!
\advisespace
你可以告诉包不要删除它们:
键 `\ikeyname{showlines}' 控制这些空行,将在第\ref{rTypesettingListings}节中进行描述。警告:首先请提前阅读如何使用键。
\switchcolumn[0]*      
\item I get obscure error messages when using `\ikeyname{firstline}'.
\advisespace
That shouldn't happen. Make a bug report as described in section
\lstref{uTroubleshooting}.
\switchcolumn      
\item 当我使用 `\ikeyname{firstline}' 时,我收到了晦涩的错误消息。
\advisespace
不应该发生这种情况。请按照第\lstref{uTroubleshooting}节中描述的方法报告错误。
\end{paracol}
\end{advise}

\begin{paracol}{2}
\textbf{Stand alone files}
Finally we come to |\lstinputlisting|, the command used to pretty-print
stand alone files. It has one optional and one file name argument.
Note that you possibly need to specify the relative path to the file.
Here now the result is printed below the verbatim code since both together
don't fit the text width.
\switchcolumn
\textbf{独立文件}
最后,我们来看看 |\lstinputlisting| 命令,它用于漂亮地显示独立文件。它有一个可选参数和一个文件名参数。请注意,您可能需要指定文件的相对路径。现在,结果被打印在等宽代码的下方,因为它们两者加起来超过了文本宽度。
\begin{lstsample}{\lstset{comment=[l]\%,columns=fullflexible}}^^A
            {\lstset{alsoletter=\\,emph=\\lstinputlisting,emphstyle=\rstyle}^^A
            \lstaspectindex{\lstinputlisting}{}}
        \lstinputlisting[lastline=4]{listings.sty}
\end{lstsample}
% gpt 注释
% comment=[l]\% 表示注释是由 % 符号开始的,并且是行注释([l] 代表“line”)
% columns=fullflexible 使得字符间距是可变的,这样代码的排版会更接近源文件的实际显示。
% alsoletter=\\ 表示反斜线 \ 也被视作字母的一部分,这通常用于正确地高亮 TeX 命令。
% emph=\\lstinputlisting 指定 lstinputlisting 命令应该被强调显示,而 emphstyle=\rstyle 定义了强调样式,其中 \rstyle 应该在其他地方定义。
% \lstaspectindex{\lstinputlisting}{}: 这似乎是自定义命令,可能是用来将 \lstinputlisting 添加到某个索引或目录中,但这不是 listings 宏包的标准命令。

% \lstinputlisting[lastline=4]{listings.sty}: 这行代码用于实际导入并显示 listings.sty 文件的前四行。lastline=4 选项告诉 listings 宏包仅包含从文件开始到第四行的内容。
\end{paracol}

\begin{advise}
    \begin{paracol}{2}
\item The spacing is different in this example.
      \advisespace
      Yes. The two previous examples have aligned columns, i.e.~columns with
      identical numbers have the same horizontal position---this package
      makes small adjustments only. The columns in the example here are not
      aligned. This is explained in section \ref{uFixedAndFlexibleColumns}
      (keyword: full flexible column format).
\switchcolumn
\item 这个例子中的间距不同。
\advisespace
是的。前面两个例子有对齐的列,即具有相同编号的列具有相同的水平位置---这个包只进行了小的调整。此处示例中的列没有对齐。这在第\ref{uFixedAndFlexibleColumns}节中进行了解释(关键词:全灵活列格式)。(\verb|columns=fullflexible|)
    \end{paracol}
\end{advise}

\begin{paracol}{2}
\switchcolumn[0]*%%%%%%%%%%%%
Now you know all pretty-printing commands and environments. It remains
to learn the parameters which control the work of the \packagename{listings}
package. This is, however, the main task. Here are some of them.
\switchcolumn
现在你已经了解了所有的漂亮打印命令和环境。剩下的就是学习控制\packagename{listings}包工作的参数。然而,这是主要的任务。下面是其中的一些。
\end{paracol}
% \input{listings_Getting_started_确定外观} 

  

    % \DocInput{listings.dtx}
\end{document}