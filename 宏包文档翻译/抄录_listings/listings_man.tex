\PassOptionsToPackage{no-math}{fontspec}%禁用了使用fontspec宏包中的数学字体功能。
\PassOptionsToPackage{AutoFakeBold=true,AutoFakeSlant=true}{xeCJK}%让xeCJK宏包自动产生伪粗体和伪斜体效果。

\documentclass[a4paper]{ltxdoc}
\usepackage[heading=true
,scheme=chinese%中文方案
,fontset=none%不使用默认的字体设置
,space=auto%自动调整中英文间距
]{ctex}
\setCJKmainfont{FangZhengShuSong-GBK-1.ttf}[Path=/Users/virhuiai/hlProjects/Latex-Typesetting-Hub/font/方正/]%设置文本的中文有衬线字体
\setCJKsansfont{FangZhengHeiTi-GBK-1.ttf}[Path=/Users/virhuiai/hlProjects/Latex-Typesetting-Hub/font/方正/]%设置文本的中文无衬线字体为
\setCJKmonofont{FangZhengFangSong-GBK-1.ttf}[Path=/Users/virhuiai/hlProjects/Latex-Typesetting-Hub/font/方正/] %设置文本的中文等宽字体 
\usepackage[a3paper,landscape]{geometry}
\usepackage{paracol}

\DisableCrossrefs
\OnlyDescription

\usepackage{lstdoc,textcomp}
\usepackage{mdframed}           % frames for external files
\usepackage{moreverb}           % writing external files
\usepackage{xcolor}             % because of colouring the background
 
\makeindex

\begin{document}
\DocInput{listings.dtx.pre}
% 
^^A
^^A  The long awaited beginning of documentation
^^A =============================================
^^A
% \newbox\abstractbox
% \setbox\abstractbox=\vbox{
% 	\begin{abstract}
% 	The \packagename{listings} package is a source code printer for \LaTeX.
% 	You can typeset stand alone files as well as listings with an environment
%   similar to \texttt{verbatim} as well as you can print code snippets using
%   a command similar to |\verb|.
% 	Many parameters control the output and if your preferred programming
%   language isn't already supported, you can make your own definition.
% 	\end{abstract}}

\title{\vspace*{-2\baselineskip}The \textsf{Listings} Package}
\author{Copyright 1996--2004, Carsten Heinz%
   \\ Copyright 2006--2007, Brooks Moses
   \\ Copyright 2013--, Jobst Hoffmann
   \\ Maintainer: Jobst Hoffmann\thanks{Jobst %
      Hoffmann became the maintainer of the \packagename{listings}
      package in 2013; see the Preface for details.}~ %
   \textless\lstemail\textgreater\and 翻译:virhuiai}
\date{2020/03/24\enspace\enspace Version 1.8d\ %\box\abstractbox
}
\ifhyper
  \hypersetup{pdftitle=The Listings Package,
              pdfsubject=Package guide,
              pdfauthor=Jobst Hoffmann <j.hoffmann(at)fh-aachen.de>,%
              pdfkeywords={source code formatter, programming languages}}
\fi

% \csname @twocolumntrue\endcsname
\maketitle

\begin{paracol}{2}
\renewcommand{\abstractname}{Abstract}
\begin{abstract}
The \packagename{listings} package is a source code printer for \LaTeX.
You can typeset stand alone files as well as listings with an environment
similar to \texttt{verbatim} as well as you can print code snippets using
a command similar to |\verb|.
Many parameters control the output and if your preferred programming
language isn't already supported, you can make your own definition.
\end{abstract}
\switchcolumn
\renewcommand{\abstractname}{摘要}
\begin{abstract}
\texttt{listings}宏包是一个用于排版\LaTeX 的源代码排版工具。您可以排版独立的文件,也可以使用类似于\texttt{verbatim}的环境排版代码列表,还可以使用类似于 |\verb| 的命令打印代码片段。许多参数可以控制输出,如果您的首选编程语言尚未支持,您可以自定义定义。
\end{abstract}

\end{paracol}


^^A \enlargethispage{2\baselineskip}
\csname @starttoc\endcsname{toc}
% \onecolumn


\columnratio{0.55}
\begin{paracol}{2}
\section*{Preface}
\switchcolumn
\section*{前言}
\switchcolumn[0]*%%%%%%%%%%%%
\textbf{Transition of package maintenance}
The \TeX\ world lost contact with Carsten Heinz in late 2004, shortly after
he released version 1.3b of the \packagename{listings} package.  After many
attempts to reach him had failed, Hendri Adriaens took over maintenance of
the package in accordance with the LPPL's procedure for abandoned packages.
He then passed the maintainership of the package to Brooks Moses, who had
volunteered for the position while this procedure was going through. The
result is known as listings version 1.4.
\switchcolumn
\textbf{包维护的转变}
在2004年底,\TeX\ 领域失去了与Carsten Heinz的联系,他在发布了\packagename{listings}宏包的1.3b版本后不久便失去了联系。在多次联系失败后,Hendri Adriaens根据LPPL对废弃包的处理程序接管了该包的维护工作。然后,他将包的维护权移交给了Brooks Moses,在此过程中,Brooks Moses自愿担任了这个职位。结果就是现在的listings 1.4版本。
\switchcolumn[0]*%%%%%%%%%%%%
This release, version 1.5, is a minor maintenance release since
I accepted maintainership of the package.  I would like to thank Stephan
Hennig who supported the Lua language definitions. He is the one who
asked for the integration of a new language and gave the impetus to me to
become the maintainer of this package.
\switchcolumn
这个发布的版本,1.5版,是我接手维护该宏包之后的一个小型维护版本。我要感谢Stephan Hennig对Lua语言定义的支持。他是那个提出整合新语言并给我成为这个宏包的维护者提供动力的人。
\switchcolumn[0]*%%%%%%%%%%%%
\textbf{News and changes}
Version 1.5 is the fifth bugfix release.  There are no changes
in this version, but two extensions: support of modern Fortran (2003,
2008) and Lua.
\switchcolumn
\textbf{新闻和变化}
1.5版本是第五个修复错误版本。这个版本没有任何变化,只有两个扩展:对现代Fortran(2003、2008)和Lua的支持。
\switchcolumn[0]*%%%%%%%%%%%%
\vfill
\textbf{Thanks}
There are many people I have to thank for fruitful communication, posting
their ideas, giving error reports, adding programming languages to
\texttt{lstdrvrs.dtx}, and so on. Their names are listed in section
\ref{uClosingAndCredits}.
\switchcolumn
\textbf{感谢}
有很多人我需要感谢他们的积极交流、发表意见、提供错误报告、添加编程语言到\texttt{lstdrvrs.dtx}等等。他们的名字将在\ref{uClosingAndCredits}节中列出。
\switchcolumn[0]*%%%%%%%%%%%%
\textbf{Trademarks}
Trademarks appear throughout this documentation without any trademark
symbol; they are the property of their respective trademark owner.
There is no intention of infringement; the usage is to the benefit of the
trademark owner.
\switchcolumn
\textbf{商标}
本文档中的商标没有使用任何商标符号,它们是其各自商标所有者的财产。没有侵权的意图;使用是为了商标所有者的利益。
\end{paracol}
 

    % \DocInput{listings.dtx}
\end{document}