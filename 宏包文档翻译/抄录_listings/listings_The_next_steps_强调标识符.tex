\columnratio{0.55}
\begin{paracol}{2}
\subsection{Emphasize identifiers}\label{uEmphasizeIdentifiers}
\switchcolumn
\subsection{强调标识符}
\switchcolumn[0]*%%%%%%%%%%%%
 Recall the pretty-printing commands and environment. |\lstinline| prints
 code snippets, |\lstinputlisting| whole files, and \texttt{lstlisting}
 pieces of code which reside in the \LaTeX\ file. And what are these
 different `types' of source code good for? Well, it just happens that a
 sentence contains a code fragment. Whole files are typically included in or
 as an appendix. Nevertheless some books about programming also include such
 listings in normal text sections---to increase the number of pages.
 Nowadays source code should be shipped on disk or CD-ROM and only the main
 header or interface files should be typeset for reference. So, please, don't
 misuse the \packagename{listings} package. But let's get back to the topic.
 \switchcolumn
%  pretty-printing
 回想一下相关命令和环境。|\lstinline|用于打印代码片段,\texttt{lstlisting}用于\LaTeX 源文件中的代码片段,|\lstinputlisting|用于整个文件。那么这些不同类型的源代码有什么用?嗯,碰巧的是,一个句子中包含一个代码片段。整个文件通常包含在附录中。然而,一些关于编程的书籍也会在正文中包含这样的代码段,从而增加页面数。现在,源代码应该通过磁盘或CD-ROM传送,并且只应该排版主头文件或接口文件供参考。所以,请不要滥用\packagename{listings}宏包。但是,让我们回到主题。
 \switchcolumn[0]*%%%%%%%%%%%%
 Obviously `\texttt{lstlisting} source code' isn't used to make an executable
 program from. Such source code has some kind of educational purpose or even
 didactic.
 \switchcolumn
显然,`\texttt{lstlisting}源代码' 并不是用来制作可执行程序的。这样的源代码具有某种教育目的或甚至是教学目的。
\switchcolumn[0]*%%%%%%%%%%%%
 \begin{advise}
 \item What's the difference between educational and didactic?
       \advisespace
       Something educational can be good or bad, true or false.
       Didactic is true by definition.^^A :-)
 \end{advise}
 \switchcolumn
 \begin{advise}
\item 教育和教学有什么区别?
\advisespace
教育可以是好的或坏的,真实的或虚假的。
教学从定义上来说是正确的。^^A :-)
\end{advise}
 \switchcolumn[0]*%%%%%%%%%%%%
 Usually \emph{keywords} are highlighted when the package typesets a piece of
 source code. This isn't necessary for readers who know the programming
 language well. The main matter is the presentation of interface, library or
 other functions or variables. If this is your concern, here come the right
 keys. Let's say, you want to emphasize the functions |square| and |root|,
 for example, by underlining them. Then you could do it like this:
 \switchcolumn
 通常,在宏包排版源代码时会突出显示\emph{关键字}。对于熟悉编程语言的读者来说,这并不是必需的。主要问题是界面、库或其他函数或变量的表达。如果这是你关注的问题,下面是正确的键。假设你想强调函数|square|和|root|,例如通过下划线标记它们。你可以这样做:
% \switchcolumn[0]*%%%%%%%%%%%%
 \begin{lstxsample}[emph,emphstyle]
           \lstset{emph={square,root},emphstyle=\underbar}
 \end{lstxsample}
 \begin{lstsample}{}{}
            \begin{lstlisting}
            for i:=maxint to 0 do
            begin
                j:=square(root(i));
            end;
            \end{lstlisting}
 \end{lstsample}
\switchcolumn[0]*%%%%%%%%%%%%
 \begin{advise}
 \item Note that the list of identifiers |{square,root}| is enclosed in
       braces. Otherwise the \packagename{keyval} package would complain
       about an undefined key \keyname{root} since the comma finishes the
       key=value pair.
       Note also that you \emph{must} put braces around the value if you
       use an optional argument of a key inside an optional argument of a
       pretty-printing command. Though it is not necessary, the following
       example uses these braces. They are typically forgotten when they
       become necessary,
 \end{advise}
 \switchcolumn
 \begin{advise}
    \item 注意,标识符列表|{square,root}|被括在花括号中。否则,\packagename{keyval}宏包会抱怨未定义的键\keyname{root},因为逗号结束了键=值对。
    还要注意,如果在漂亮排版命令的可选参数中使用键的可选参数,\emph{必须}将值放在花括号中。尽管这并不是必需的,下面的示例使用了这些花括号。当需要时,它们通常会被忘记。
    \end{advise}
\switchcolumn[0]*%%%%%%%%%%%%

 Both keys have an optional \meta{class number} argument for multiple
 identifier lists:
 \switchcolumn
 这两个键都有一个可选的\meta{class number}参数,用于多个标识符列表:
\switchcolumn[0]*%%%%%%%%%%%%
\ifcolor
 \begin{lstxsample}[emph,emphstyle]
            \lstset{emph={square},      emphstyle=\color{red},
                    emph={[2]root,base},emphstyle={[2]\color{blue}}}
 \end{lstxsample}
\else
 \begin{lstxsample}[emph,emphstyle]
    \lstset{emph={square},      emphstyle=\underbar,
            emph={[2]root,base},emphstyle={[2]\fbox}}
 \end{lstxsample}
\fi
 \begin{lstsample}{}{}
    \begin{lstlisting}
    for i:=maxint to 0 do
    begin
        j:=square(root(i));
    end;
    \end{lstlisting}
 \end{lstsample}
 \begin{advise}
 \item What is the maximal \meta{class number}?
       \advisespace
       $2^{31}-1=2\,147\,483\,647$. But \TeX's memory will exceed before you
       can define so many different classes.
 \end{advise}

 One final hint: Keep the lists of identifiers disjoint. Never use a keyword
 in an `emphasize' list or one name in two different lists. Even if your
 source code is highlighted as expected, there is no guarantee that it is
 still the case if you change the order of your listings or if you use the
 next release of this package.



\end{paracol}