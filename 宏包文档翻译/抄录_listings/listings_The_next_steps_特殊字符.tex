\columnratio{0.55}
\begin{paracol}{2}
\subsection{Special characters}\label{uSpecialCharacters}
\switchcolumn
\subsection{特殊字符}
\switchcolumn[0]*%%%%%%%%%%%%
\textbf{Tabulators}
You might get unexpected output if your sources contain tabulators.
The package assumes tabulator stops at columns 9, 17, 25, 33, and so on.
This is predefined via |tabsize=8|. If you change the eight to the number
$n$, you will get tabulator stops at columns $n+1,2n+1,3n+1,$ and so on.
\switchcolumn
\textbf{制表符}
如果你的源文件包含制表符,可能会得到意外的输出结果。
该宏包默认将制表符停在第9、17、25、33等列。
这是通过 |tabsize=8| 预定义的。如果你将8更改为数字$n$,则将在第$n+1,2n+1,3n+1$等列停止。
\begin{lstsample}[tabsize]{}{}
        \lstset{tabsize=2}
        \begin{lstlisting}
        123456789
            { one tabulator }
                { two tabs }
        123     { 123 + two tabs }
        \end{lstlisting}
\end{lstsample}
\switchcolumn[0]*%%%%%%%%%%%%
For better illustration, the left-hand side uses |tabsize=2| but the verbatim
code |tabsize=4|. Note that |\lstset| modifies the values for all following
listings in the same environment or group. This is no problem here since the
examples are typeset inside minipages. If you want to change settings for a
single listing, use the optional argument.
\switchcolumn
为了更好地说明问题,左侧使用 |tabsize=2|,但原始代码使用 |tabsize=4|。
请注意,|\lstset| 修改了同一环境或组中所有后续列表的值。在这里没有问题,因为示例是在minipage内排版的。
如果要更改单个列表的设置,请使用可选参数。
\switchcolumn[0]*%%%%%%%%%%%%
\textbf{Visible tabulators and spaces}
One can make spaces and tabulators visible:
\switchcolumn
\textbf{可见制表符和空格}
可以使空格和制表符可见:
\begin{lstsample}[showspaces,showtabs,tab]{}{}
        \lstset{showspaces=true,
                showtabs=true,
                tab=\rightarrowfill}
        \begin{lstlisting}
            for i:=maxint to 0 do
            begin
            { do nothing }
            end;
        \end{lstlisting}
\end{lstsample}

\switchcolumn[0]*%%%%%%%%%%%%
If you request \ikeyname{showspaces} but no \ikeyname{showtabs},
tabulators are converted to visible spaces.
The default definition of \ikeyname{tab} produces a `wide visible space'
\lstinline[showtabs]!	!. So you might want to use |$\to$|, |$\dashv$|
or something else instead.
\switchcolumn
如果请求 \ikeyname{showspaces} 但没有 \ikeyname{showtabs},则制表符将转换为可见空格。
\ikeyname{tab} 的默认定义产生一个“宽可见空格”\lstinline[showtabs]! !。因此,你可能想使用 |$\to$|、|$\dashv$| 或其他替代品。
\end{paracol}
\begin{advise}
    \columnratio{0.55}
\begin{paracol}{2}
\item Some sort of advice: (1) You should really indent lines of source code
      to make listings more readable. (2) Don't indent some lines with
      spaces and others via tabulators. Changing the tabulator size (of your
      editor or pretty-printing tool) completely disturbs the columns.
      (3) As a consequence, never share your files with differently tab sized
      people!^^A true only if you use tabulators, just :-)
\switchcolumn \item 一些建议:(1)为了使代码清晰易读,应该缩进源代码的行。(2)不要使用空格缩进某些行,使用制表符缩进其他行。更改制表符大小(编辑器或美化工具的制表符大小)会完全扰乱列。(3)因此,永远不要与制表符大小不同的人共享文件!^^A只有当你使用制表符时才是真的 :-)
\switchcolumn[0]*%%%%%%%%%%%%\switchcolumn[0]*%%%%%%%%%%%%
\item To make the \LaTeX\ code more readable, I indent the environments'
      program listings. How can I remove that indention in the output?
      \advisespace
      Read `How to gobble characters' in section \ref{uHowTos}.
\switchcolumn
\item 为了使 \LaTeX\ 代码更易读,我缩进了环境中的程序列表。如何在输出中删除缩进?
\advisespace
请阅读“如何吞掉字符”一节中的“gobble characters”。
    \end{paracol}
\end{advise}

\columnratio{0.55}
\begin{paracol}{2}
\textbf{Form feeds}
Another special character is a form feed causing an empty line by default.
{\rstyle\ikeyname{formfeed}}|=\newpage| would result in a new page every
form feed. Please note that such definitions (even the default) might get
in conflict with frames.
\switchcolumn
\textbf{换页符}
另一个特殊字符是默认情况下导致空行的换页符。
{\rstyle\ikeyname{formfeed}}|=\newpage| 将导致每个换页符都换到新的一页。
请注意,这样的定义(甚至默认定义)可能与框架冲突。
\switchcolumn[0]*%%%%%%%%%%%%

\textbf{National characters}
If you type in such characters directly as characters of codes 128--255 and
use them also in listings, let the package know it---or you'll get really
funny results. {\rstyle\ikeyname{extendedchars}}|=true| allows and
|extendedchars=false| prohibits \packagename{listings} from handling
extended characters in listings. If you use them, you should load
\packagename{fontenc}, \packagename{inputenc} and/or
any other package which defines the characters.
\switchcolumn
\textbf{国际字符}
如果直接输入代码128--255的字符并在列表中使用它们,应该让宏包知道---否则你将得到非常奇怪的结果。
\packagename{listings} 允许 \ikeyname{extendedchars}=true,禁止 \ikeyname{extendedchars}=false 处理列表中的扩展字符。
如果使用这些字符,应该加载 \packagename{fontenc}、\packagename{inputenc} 或其他定义字符的包。
\switchcolumn[0]*%%%%%%%%%%%%
\begin{advise}
\item I have problems using \packagename{inputenc} together with
      \packagename{listings}.
      \advisespace
      This could be a compatibility problem. Make a bug report as described
      in section \lstref{uTroubleshooting}.
\end{advise}
\switchcolumn
\begin{advise}
\item 我在使用 \packagename{inputenc} 和 \packagename{listings} 时遇到了问题。
\advisespace
这可能是一个兼容性问题。请按照第\lstref{uTroubleshooting}节中的说明提交错误报告。
\end{advise}
\switchcolumn[0]*%%%%%%%%%%%%
The extended characters don't cover Arabic, Chinese, Hebrew, Japanese, and so
on---specifically, any encoding which uses multiple bytes per character.
\switchcolumn
扩展字符不包括阿拉伯语、中文、希伯来语、日语等使用多字节字符的编码。
\switchcolumn[0]*%%%%%%%%%%%%
Thus, if you use the a package that supports multibyte characters, such as
the \packagename{CJK} or \packagename {ucs} packages for Chinese and
UTF-8 characters, you must avoid letting \packagename{listings}
process the extended characters.  It is generally best to also specify
|extendedchars=false| to avoid having \packagename{listings} get entangled
in the other package's extended-character treatment.
\switchcolumn
因此,如果使用支持多字节字符的包,例如用于中文和UTF-8字符的\packagename{CJK}或\packagename{ucs}包,必须避免让\packagename{listings}处理扩展字符。
最好也指定 |extendedchars=false|,以避免\packagename{listings}与其他包的扩展字符处理发生混淆。
\switchcolumn[0]*%%%%%%%%%%%%
If you do have a listing contained within a CJK environment, and want to have
CJK characters inside the listing, you can place them within a comment that
escapes to \LaTeX -- see section \ref{rEscapingToLaTeX} for how to do that.
(If the listing is not inside a CJK environment, you can simply put a small
CJK environment within the escaped-to-\LaTeX portion of the comment.)
\switchcolumn
如果你在CJK环境中包含一个列表,并且希望在列表中使用CJK字符,可以在转义到\LaTeX 中的注释中放置它们。有关如何执行此操作,请参见第\ref{rEscapingToLaTeX}节。
(如果列表不在CJK环境中,可以在注释的转义到\LaTeX 部分中简单地放置一个小的CJK环境。)
\switchcolumn[0]*%%%%%%%%%%%%
Similarly, if you are using UTF-8 extended characters in a listing, they must
be placed within an escape to \LaTeX.
\switchcolumn
类似地,如果在列表中使用UTF-8扩展字符,则必须将它们放在转义到\LaTeX 中。
\switchcolumn[0]*%%%%%%%%%%%%
Also, section \ref{uNationalCharacters} has a few details on how to work with
extended characters in the context of $\Lambda$.
\switchcolumn
此外,在第\ref{uNationalCharacters}节中介绍了如何在$\Lambda$的上下文中处理扩展字符的一些细节。
\end{paracol}