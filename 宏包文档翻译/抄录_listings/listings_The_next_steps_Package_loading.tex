
\columnratio{0.55}
\begin{paracol}{2}


\subsection{Package loading}\label{uPackageLoading}
\switchcolumn
\subsection{加载宏包}
\switchcolumn[0]*%%%%%%%%%%%%
As usual in \LaTeX, the package is loaded by\\
   |\usepackage[|\meta{options}|]{listings}|\\,
where |[|\meta{options}|]| is optional and gives a comma separated list of
options. Each either loads an additional \packagename{listings} aspect, or
changes default properties. Usually you don't have to take care of such
options. But in some cases it could be necessary: if you want to compile
documents created with an earlier version of this package or if you use
special features. Here's an incomplete list of possible options.
\switchcolumn
如同在\LaTeX 中一样,可以通过以下方式加载宏包:\\
|\usepackage[|\meta{options}|]{listings}|\\,
其中 |[|\meta{options}|]| 是可选的,并且是一个逗号分隔的选项列表。每个选项可以加载额外的\packagename{listings}模块,或者更改默认属性。通常情况下,您不需要关注这些选项。但在某些情况下可能是必要的:例如,如果您想编译使用早期版本的此宏包创建的文档,或者您使用特殊功能。以下是可能的选项的不完整列表。
\end{paracol}
\begin{advise}
    \columnratio{0.55}
    \begin{paracol}{2}
\item Where is a list of all of the options?
      \advisespace
      In the developer's guide since they were introduced to debug the
      package more easily. Read section \ref{uHowTos} on how to get that
      guide.
\switchcolumn
\item 是否有所有选项的列表?
\advisespace
在开发人员指南中,因为它们被引入以便更容易调试宏包。请阅读第\ref{uHowTos}节,了解如何获取该指南。
    \end{paracol}

\end{advise}

\begin{description}
\columnratio{0.55}
\begin{paracol}{2}
\item[\normalfont\texttt{0.21}]\leavevmode
      invokes a compatibility mode for compiling documents written for
      \packagename{listings} version 0.21.
      \switchcolumn
\item[\normalfont\texttt{0.21}]\leavevmode
启用兼容模式,用于编译针对\packagename{listings} 0.21版本编写的文档。
      \switchcolumn[0]*%%%%%%%%%%%%
\item[\normalfont\texttt{draft}]\leavevmode
      The package prints no stand alone files, but shows the captions and
      defines the corresponding labels.
      Note that a global |\documentclass|-option \texttt{draft} is
      recognized, so you don't need to repeat it as a package option.
      \switchcolumn
\item[\normalfont\texttt{draft}]\leavevmode
该宏包不会生成独立的文件,但会显示标题并定义相应的标签。
请注意,全局|\documentclass|选项\texttt{draft}也会被识别,因此您无需将其作为宏包选项重复指定。
      \switchcolumn[0]*%%%%%%%%%%%%
\item[\normalfont\texttt{final}]\leavevmode\label{uoption:final}
      Overwrites a global \texttt{draft} option.
      \switchcolumn
\item[\normalfont\texttt{final}]\leavevmode\label{uoption:final}
覆盖全局的\texttt{draft}选项。
      \switchcolumn[0]*%%%%%%%%%%%%
\item[\normalfont\texttt{savemem}]\leavevmode

      tries to save some of \TeX's memory. If you switch between languages
      often, it could also reduce compile time. But all this depends on the
      particular document and its listings.
\switchcolumn
\item[\normalfont\texttt{savemem}]\leavevmode
尝试节省一些\TeX 的内存。如果您经常切换语言,这也可能减少编译时间。但这一切都取决于特定的文档及其代码片段。
    \end{paracol}
\end{description}
\columnratio{0.55}
\begin{paracol}{2}
Note that various experimental features also need explicit loading via
options. Read the respective lines in section \ref{rExperimentalFeatures}.
\switchcolumn
请注意,各种实验性功能也需要通过选项显式加载。请阅读第\ref{rExperimentalFeatures}节中的相应行。
\switchcolumn[0]*%%%%%%%%%%%%
\medbreak
After package loading it is recommend to load all used dialects of programming
languages with the following command. It is faster to load several languages
with one command than loading each language on demand.
\switchcolumn
\medbreak
在加载宏包后,建议使用以下命令加载所有使用的编程语言的方言。使用一个命令加载多个语言比按需加载每个语言更快。
\end{paracol}
\begin{syntax}
    \columnratio{0.55}
    \begin{paracol}{2}    
\item {\rstyle\icmdname\lstloadlanguages}\marg{comma separated list of languages}
\switchcolumn
\item {\rstyle\icmdname\lstloadlanguages}\marg{逗号分隔的语言列表}
\switchcolumn[0]*%%%%%%%%%%%%

      Each language is of the form \oarg{dialect}\meta{language}. Without
      the optional \oarg{dialect} the package loads a default dialect. So
      write `|[Visual]C++|' if you want Visual \Cpp\ and `|[ISO]C++|' for
      ISO \Cpp. Both together can be loaded by the command\\
      |\lstloadlanguages{[Visual]C++,[ISO]C++}|.
      \switchcolumn
      每种语言的格式为 \oarg{方言}\meta{语言}。如果没有可选的\oarg{方言},宏包将加载默认方言。因此,如果要加载Visual \Cpp,则写为`|[Visual]C++|',如果要加载ISO \Cpp,则写为`|[ISO]C++|'。同时加载两者可以使用命令\\|\lstloadlanguages{[Visual]C++,[ISO]C++}|。
      \switchcolumn[0]*%%%%%%%%%%%%
      Table \ref{uPredefinedLanguages} on page \pageref{uPredefinedLanguages}
      shows all defined languages and their dialects.
      \switchcolumn
      表\ref{uPredefinedLanguages}在第\pageref{uPredefinedLanguages}页显示了所有定义的语言及其方言。
      \switchcolumn[0]*%%%%%%%%%%%%
    \end{paracol}      
\end{syntax}
\columnratio{0.55}
\begin{paracol}{2}
^^A After or even before language loading, you might want to define default
^^A dialects---just to be independent of configuration files.
\switchcolumn
^^A在加载语言之后,甚至在之前,您可能希望定义默认方言,以便独立于配置文件。
\switchcolumn[0]*%%%%%%%%%%%%
\end{paracol}