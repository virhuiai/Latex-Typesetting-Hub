
\title{Package \textsf{paracol}:\\
	Yet Another Multi-Column Package to Typeset Columns in
	\textit{Parallel}}

\author{Hiroshi Nakashima\\(Kyoto University) \and 翻译:virhuiai\\(福建师范大学)}
\date{version \expandafter\@gobble\fileversion: \filedate}
\setbox0\vbox{\vskip\topskip\maketitle\vskip0pt}\unitlength\ht0\copy0

\ifx\ONLYDESCRIPTION\undefined
\tableofcontents
\newpage \ifodd\c@page\else \hbox{}\newpage \fi
\vbox to\unitlength{\unvbox0\vfil}
\part{User's Manual}\label{part:man}
\let\MidxSave\Midx \let\Midx\Uidx
\fi

\begin{abstract}
\columnratio{0.55}
\begin{paracol}{2}
\noindent
This package provides a \LaTeX{} environment named |paracol| in which you
may {\em switch} and {\em synchronize} columns by a command
|\switchcolumn| and by internal environments |column|, |nthcolumn|,
|leftcolumn| and |rightcolumn|.
\ifx\ONLYDESCRIPTION\undefined\else
See p.\Tie\pageref{page:toc} for the table of contents of this manual.
\fi
\switchcolumn 
\noindent
本宏包提供了一个名为 |paracol| 的 \LaTeX{} 环境,在其中你可以通过命令 |\switchcolumn| 和内部环境 |column|、|nthcolumn|、|leftcolumn| 和 |rightcolumn| 来{\fontKai 切换}和{\fontKai 同步}列。
\ifx\ONLYDESCRIPTION\undefined\else
请参考第 \pageref{page:toc} 页的本手册目录。
\fi
\end{paracol}
\end{abstract} 
% 
% \tracingpages1 \tracingoutput1 \showboxdepth10000 \showboxbreadth10000
% % \begingroup
% \hbadness9000 \hfuzz6pt



% 
% \begin{paracol}{2}[\section{介绍\\Introduction}]
% This document describes the usage of yet another multi-column package named
% \textsf{paracol}.  The unique feature of the package is that columns are
% typeset {\em in parallel.}
% \switchcolumn
% 本文档介绍了另一个名为 \textsf{paracol} 的多栏排版宏包的使用方法。该宏包的独特特点是可以将栏以{\em 并行}的方式排版。 
% 
% \switchcolumn*
% Suppose you are writing a bilingual document whose left column is written in
% a language, say English, and right column has the translation of the left
% column in another language, e.g., Japanese.  With the \textsf{paracol}
% package you may write an English part of arbitrarily length and then {\em
% switch} to its Japanese counterpart to place both parts side by side.  Of
% course you may return to the English writing similarly.
% \switchcolumn
% 假设你正在撰写一份双语文档,左栏使用一种语言(如英语),右栏则是左栏的另一种语言(如日语)的翻译。使用 \textsf{paracol} 宏包,你可以先写任意长度的英文部分,然后{\em 切换}到对应的日文部分,将两部分并排放置在一起。当然,你也可以类似地返回到英文撰写。
% \switchcolumn*
% The {\em\Uidx\cswitch} is always allowed when you complete an outermost
% level paragraph.  You may be unaware whether a column is broken into
% multiple pages before switching because the package automatically goes
% back and forward to the correct page and vertical position when you switch
% the column.  Moreover, you may {\em\Uidx\sync{}e} columns so that the tops
% of the first paragraphs after switching in all columns are vertically
% aligned.  At a \sync{}ation point, you may give a single-column text,
% for example a common section header, optionally.  You may also switch
% single-column and multi-column in a page arbitrary.
% \switchcolumn
% 在外层段落完成后,总是允许使用 {\em\Uidx\cswitch} 命令。在切换之前,你可能不知道栏是否被分成多个页面,因为当你切换栏时,宏包会自动回到正确的页面和垂直位置。此外,你可以通过 {\em\Uidx\sync{}e} 命令来使列对齐,这样在切换后,所有列中第一个段落的顶部会垂直对齐。在 \sync{}ation 点,你可以选择给出单栏文本,例如一个公共的章节标题。你还可以随意在页面上切换单栏和多栏排版。
% \switchcolumn*
% This manual itself is an example of two-column documents typeset by
% \textsf{paracol}.  
% \switchcolumn
% 本手册本身就是使用 \textsf{paracol} 宏包排版的两栏文档的一个示例。
%
% \end{paracol}

%\begin{Verbatim}
%\begin{paracol}{2}[\section{Introduction}]
%\hbadness5000
%en.....
%\switchcolumn
%中文....

%\switchcolumn*
%en.....
%\switchcolumn
%中文....
%\switchcolumn*[\section{Basic Usage}]....
%\end{paracol}
% \end{Verbatim}



% \begin{paracol}{2}[\section{Basic Usage\hfill 基本用法}]
% Loading the package is very simple.  What you have to do is
% \!\usepackage!|{|\Uidx{\env{paracol}}|}| in the preamble.  Note that
% \textsf{paracol} can be used with \LaTeXe{} and does not work with
% \LaTeX{} 2.09.
% \switchcolumn
% 加载该宏包非常简单。在导言区使用\\ \!\usepackage!|{|\Uidx{\env{paracol}}|}| 命令即可。请注意,\textsf{paracol} 可以与 \LaTeXe{} 一起使用,不支持 \LaTeX{} 2.09。
% 
% \switchcolumn*
% The fundamental means of parallel-column typesetting are the environment
% \env{paracol} and the command \Uidx{\!\switchcolumn!}.  The \env{paracol}
% environment needs an argument to specify the number of columns.  Thus the
% following is the basic construct for two-parallel-column documents.
% \switchcolumn
% 并列栏排版的基本手段是使用 \env{paracol} 环境和命令 \Uidx{\!\switchcolumn!}。\env{paracol} 环境需要一个参数来指定栏的数量。因此,以下是两栏并列文档的基本结构。
%\switchcolumn*
% \begin{quote}
% \!\begin!|{|\env{paracol}|}{2}|\\
% \textit{left column text}\\
% \!\switchcolumn!\\
% \textit{right column text}\\
% \!\switchcolumn!\\
% \textit{left column text}\\
% \!\switchcolumn!\\
% \textit{right column text}\\
% \!\switchcolumn!\\
% \mbox{\hspace{4em}}$\vdots$\\
% \!\end!|{|\env{paracol}|}|
% \end{quote}
% \switchcolumn
% \begin{quote}
% \!\begin!|{|\env{paracol}|}{2}|\\
% \textit{左栏文本}\\
% \!\switchcolumn!\\
% \textit{右栏文本}\\
% \!\switchcolumn!\\
% \textit{左栏文本}\\
% \!\switchcolumn!\\
% \textit{右栏文本}\\
% \!\switchcolumn!\\
% \mbox{\hspace{4em}}$\vdots$\\
% \!\end!|{|\env{paracol}|}|
% \end{quote}

% \switchcolumn*
% The \!\switchcolumn! command may have an optional argument to specify the
% column number (zero origin) to start.  That is, \!\switchcolumn!|[0]|
% means to switch to the leftmost column, |\switchcolumn[1]| is to start the
% second column and so on.  Thus the |\switchcolumn| without the optional
% argument may be considered as \!\switchcolumn!|[|$i+1\bmod{n}$|]| where
% $i$ is the ordinal of the column you are leaving from and $n$ is the
% number of columns given to \env{paracol} environment.
% \switchcolumn
% \!\switchcolumn! 命令可以带有可选参数来指定从第几栏(从零开始计数)开始切换。也就是说,\!\switchcolumn!|[0]| 表示切换到最左边的栏,|\switchcolumn[1]| 表示从第二栏开始,依此类推。因此,不带可选参数的 |\switchcolumn| 可以视为 \!\switchcolumn!|[|$i+1\bmod{n}$|]|,其中 $i$ 是你离开的栏的序号,$n$ 是给定给 \env{paracol} 环境的栏数。
% 
% 
% \switchcolumn[0]*[\section{Column Synchronization\\栏同步}\label{sec:sync}]
% The \!\switchcolumn! command may also be followed by a `|*|' to
% {\em\Uidx\sync{}e} columns.  After you switch from a column to another by
% \!\switchcolumn!|*| (or \!\switchcolumn!|[|$i$|]*|), all the columns are
% vertically aligned at the bottom of the {\em deepest} one preceding the
% command.  For example, the previous section has three \!\switchcolumn!|*|
% commands at which left and right columns are vertically aligned.
% \switchcolumn
% \!\switchcolumn! 命令后面可以加上 `|*|',用来{\em 同步}栏。当你使用 \!\switchcolumn!|*|(或 \!\switchcolumn!|[|$i$|]*|)从一栏切换到另一栏时,所有栏都会垂直对齐在该命令之前最{\em 深}的栏的底部。例如,前一节使用了三个 \!\switchcolumn!|*| 命令,使左右两栏垂直对齐。
% \switchcolumn*
% The {\em starred} version of \!\switchcolumn! may have an optional
% argument to specify a single-column {\em\Uidx\mctext} whose bottom is the
% vertical alignment point of columns.  For example, \!\section!
% commands in this manual are given as optional arguments
% of \!\switchcolumn!|*| like;
%\begin{Verbatim}
%  \switchcolumn*[\section{Basic Usage}]
%\end{Verbatim}
% The \env{paracol} environment may also start with a \mctext{} by
% specifying it as the optional argument of \!\begin!|{|\env{paracol}|}|.
% For example, at the beginning of this document, the author put;
%\begin{Verbatim}
%  \begin{paracol}{2}[\section{Introduction}]
%\end{Verbatim}
% 
% \switchcolumn
%\begin{Verbatim}
%\switchcolumn[0]*[%
%  \section{Column Synchronization}
%  \label{sec:sync}]
%The |\switchcolumn| command may also be
%followed by a `|*|' to {\em synchronize}
%columns. ...
% 
%The {\em starred} version of
%|\switchcolumn| may have an optional
%argument to specify a multi-column text
%whose bottom is the vertical alignment
%points of the columns.  ...
%\switchcolumn
%\end{Verbatim}
% \textit{source}
% 
% 
% 
% \begin{column*}[\section{Environments for Columns}\label{sec:env}]
% \Uidx{\Index{column-switching environment}}
% \subsection{Environment \texttt{column}}
% The \!\switchcolumn! is simple but you may prefer to pack the contents of a
% column in an environment.  The \Uidx{\env{column}} environment is
% available for this well-structuralization of \LaTeX{} sources for
% parallel-columned documents. A construct;
% \begin{quote}
% \!\begin!|{|\env{column}|}|\\
% \textit{text for a column}\\
% \!\end!|{|\env{column}|}|
% \end{quote}
% is (almost) equivalent to;
% \begin{quote}
% \!\switchcolumn!\\
% \textit{text for a column}
% \end{quote}
% The \Uidx{\env{column*}} environment is also available for the column
% \sync{}ation and may have an optional argument for \mctext.
% \end{column*}
% 
% \begin{column}
% \subsection{\ttfamily Environment column}
%\begin{Verbatim}
%\begin{column*}[%
%  \section{Environments for Columns}
%  \label{sec:env}]
%\subsection{Environment \texttt{column}}
%The |\switchcolumn| is simple but you may
%prefer to pack the contents of a column in
%an environment.  ...
%\end{column*}
%\begin{column}
%\end{Verbatim}
% \textit{source}\\
% |\end{column}|
% \end{column}
% 
% \begin{nthcolumn*}{1}
% \subsection{\ttfamily Environment nthcolumn}
%\begin{Verbatim}
%\begin{nthcolumn*}{1}
%\end{Verbatim}
% \textit{source}
%\begin{Verbatim}
%\end{nthcolumn*}
% 
%\begin{nthcolumn}{0}
%\subsection{Environment \texttt{nthcolumn}}
%The |\switchcolumn| can start an
%arbitrarily specified column with the
%column number given through its optional
%argument, but the |column| environment
%cannot do it. ...
%\end{nthcolumn}
%\end{Verbatim}
% \end{nthcolumn*}
% \begin{nthcolumn}{0}
% \subsection{Environment \texttt{nthcolumn}}
% The \!\switchcolumn! can start an arbitrarily specified column with the
% column number given through its optional argument, but the \env{column}
% environment cannot do it.  If you want to start $i$-th column, you have to
% do \!\begin!|{|\Uidx{\env{nthcolumn}}|}{|$i$|}| (or
% \Uidx{\env{nthcolumn*}} with an optional argument to \sync{}e).
% \end{nthcolumn}
% 
% 
% 
% \begin{leftcolumn*}
% \subsection[Environments \texttt{leftcolumn} and \texttt{rightcolumn}]
%     {Environments \texttt{leftcolumn} and\\\texttt{rightcolumn}}
% The environments \Uidx{\env{leftcolumn}} and \Uidx{\env{rightcolumn}} (and
% their starred versions with an optional argument) are available as more
% convenient means than saying \!\begin!|{|\env{nthcolumn}|}{0}| to switch
% to the left(most) column and
% \!\begin!|{|\env{nthcolumn}|}{1}| to the right (but may not be rightmost)
% one.
% 
% \Uidx{\EnvIndex{leftcolumn*}}\Uidx{\EnvIndex{rightcolumn*}}
% 
% \begin{figure*}\nosv
% \def\arraystretch{0.8}
% \centerline{\begin{tabular}[b]{|c|}\hline
%     \hbox to.9\textwidth{}\\
%     double-column figure \#1\\
%     \\\hline
%     \end{tabular}}
% \caption{A Double-Column Figure}
% \end{figure*}
% \begin{figure}[t]\nosv
% \def\arraystretch{0.8}
% \centerline{\begin{tabular}[b]{|c|}\hline
%     \hbox to.9\columnwidth{}\\\\
%     single-column figure \#1\\
%     \\\\\hline
%     \end{tabular}}
% \caption{A Single-Column Figure}
% \end{figure}
% \end{leftcolumn*}
% 
% \begin{rightcolumn}
% \subsection{\ttfamily Environment leftcolumn and\\rightcolumn}
%\begin{Verbatim}
%\begin{leftcolumn*}
%\subsection{%
%  Environments \texttt{leftcolumn} and\\
%  \texttt{rightcolumn}}
%The environments |leftcolumn| and
%|rightcolumn| (and their starred versions
%with an optional argument) are available as
%more convenient means than saying
%|\begin{nthcolumn}{0}| to switch to the
%left(most) column and ...
%\begin{figure*}...\end{figure*}
%\begin{figure}[t]...\end{figure}
%\end{leftcolumn*}
%\begin{rightcolumn}
%\end{Verbatim}
% \textit{source and a \texttt{figure} env}\\
% |\end{rightcolumn}|
% \begin{figure}[t]\nosv
% \def\arraystretch{0.8}
% \centerline{\begin{tabular}[b]{|c|}\hline
%     \hbox to.9\columnwidth{}\\
%     \ttfamily single-column figure \#2\\
%     \\\hline
%     \end{tabular}}
% \caption{\ttfamily Another Single-Column Figure}
% \end{figure}
% \end{rightcolumn}
% 
% 
% 
% \begin{leftcolumn*}[\section{Floats, Footnotes and Counters}
%     \label{sec:float}]
% \changes{v1.2-7}{2013/05/11}
% 	{Remove \cs{nosv} from verbatim example of Table~1 shown in the right
%	 column.}
% \changes{v1.32-3}{2015/10/10}
% 	{Add footnote to mention the page-wise float problem.}
% \begin{table}[b]\nosv
% \caption{A Single-Column Table}
% \centerline{\begin{tabular}[t]{|l|c|r|}\hline
%   An&example&of\\\hline
%   single&column&table\\\hline
%   \end{tabular}}
% \end{table}
% \subsection{Figures and Tables}
% As shown in this page, double-column figures\slash tables (or those
% spanned multiple columns if you have three or more) may be placed by
% \env{figure*} and \env{table*} environments as usual\footnote{
% 
% See Section~\ref{sec:problem} for the appearance order issue of
% double-column floats.}.
% 
% A single-column figure\slash table will be placed in the column in which
% you put \env{figure} and \env{table}.  For example, the body of a
% \env{figure} environment in a \env{leftcolumn} environment is
% \emph{always} placed in a left column.  That is, even if the column of the
% \emph{current} page does not have enough room to place the figure, it will
% not be thrown to the right column but will be placed in the left column of
% the next page\footnote{
% 
% Or some farther page if \LaTeX{} cannot solve the placement problem wisely.}.
% 
% Another caution about float placement is that you have to be careful when
% you try to put a top-float explicitly with |t|-option or implicitly without
% placement option (i.e., |tbp| in most classes) and to \sync{}e columns.
% The rule is as follows; after you \sync{}e columns in a page, the page
% cannot have top-floats any more.  When you \sync{}e columns,
% \textsf{paracol} fixes a virtual horizontal line in the page as the
% \sync{}ation barrier.  Thus no top-floats cannot be added above the
% line\footnote{
% 
% Even if you have enough space above, sorry.}.
% 
% Therefore, the author put two \env{figure} environments for the figures
% shown in this page into the \env{leftcolumn*} and \env{rightcolumn}
% environment for the previous section.
% 
% \subsection{Footnotes and Marginal Notes}
% \changes{v1.2-2}{2013/05/11}
% 	{Add a footnote mentioning page-wise footnotes.}
%
% Footnotes are also put at the bottom of the column in which \!\footnote!
% commands and their references reside (like this\footnote{
% 
% Unless you specify to make footnotes {\em page-wise} as explained in
% Section \ref{sec:ref-scfnote} and \ref{sec:fnnp}.}),
% 
% as shown in page~\pageref{fn:first} and this page.  Marginal
% notes behave similarly like what you are seeing in the left margin of this
% sentence\marginpar{
% 
% \raggedright An example of marginal note.}
% 
% and the right marginal note in this page\footnote{
% 
% If you have three or more columns, marginal notes of the second or
% succeeding columns are placed in the right margin in default setting.  The
% \textsf{paracol} package solves the placement problem of marginal notes
% from two or more columns sharing a side margin by moving some of them down
% if they conflict over the space with each other.}.
% 
% \subsection{Local and Global Counters}
% \UsageIndex{local counter}
% \UsageIndex{global counter}
% 
% You probably found that the numbering of figures and tables is \emph{global}
% while that of footnotes are \emph{local}.  That is, the figure in the right
% column of the previous page has number~3 following its left-column
% counterpart Figure~2.  The tables in the page are also numbered as 1 and 2
% crossing the column boundary.  However, the footnotes in each column have
% their own numbering sequence.  Moreover, the footnote numbers in left
% columns are typeset in roman font while those in right columns have italic
% shapes.  Similarly, subsection numbering is local and the headings in right
% columns have typewriter-face numbers.
% 
% This happens because the author declared the counters \counter{figure} and
% \counter{table} are \emph{global} in the preamble of this document by
% saying;
% \begin{itemize}\item[]
% \Uidx{\!\globalcounter!}|{figure}|\\
% \!\globalcounter!|{table}|
% \end{itemize}
% and do nothing about \counter{footnote} and \counter{subsection} counters.
% By default, all the counters except for |page| are local to columns.  The
% value of a \lcounter{} of a column is saved somewhere when you leave the
% column, and it is restored when you revisit the column.  The initial values
% of the \lcounter{}s are the values they have at
% \!\begin!|{|\env{paracol}|}|.  After you close the \env{paracol}
% environment, the values of the leftmost column are used for the rest of
% your document until you start new \env{paracol} environment.  On a
% restart, \lcounter{}s in a column have the values they had at the last
% \Endparacol, except for those which have been modified outside the
% environment because the modifications are \emph{broadcasted} to
% \lcounter{}s in all columns.  You will see the effect of this
% inter-environment counter value conservation in the footnote numbers in
% the right column in page~\pageref{fn:right3} and \pageref{fn:right4}.
% 
% This broadcasting of a \lcounter{} value can be done explicitly in
% \env{paracol} environments by a command $\Uidx{\!\synccounter!}\Arg{ctr}$.
% This command makes $\mathit{ctr}$ in all columns have the value of that in
% the column in which the command appears.  In addition, another command
% \Uidx{\!\syncallcounters!} performs this broadcasting for all \lcounter{}s.
% 
% If you make a counter global by the command \!\globalcounter!, the
% save/restore operations are not performed to the counter and thus it is
% globally incremented by |\[ref]|\AB|stepcounter|
% 
% \SpecialIndex{\refstepcounter}\SpecialIndex{\stepcounter}
% 
% or commands such as \!\caption! and \!\section!.  Note that the value of a
% \gcounter{} depends on the place where it is incremented (or set) in
% the \emph{source code} rather than where it appears in the output.  Thus
% if the author put a \env{table} environment here to increment \env{table}
% counter, the right-column table at the bottom of page~\pageref{tab:right}
% would be Table~3 because its \env{table} environment does not appear yet
% in the source code.  Note that, however, though the counter \counter{page}
% is global as expected, its numbering is consistent among all columns as
% far as you refer to the value by $\!\pageref!\Arg{label}$ and/or see the
% values in table of contents, etc.
% 
% Another counter which the author made global in this document is
% \counter{section}.  As explained in Section~\ref{sec:sync}, an optional
% \mctext{} of \cswitch{} is considered as in the leftmost column.  Since
% \!\section! commands in this document are always given in \mctext{}s, so
% far, it seems unnecessary to make \counter{section} global because it is
% incremented correctly in the leftmost column.  However, the stepping
% \counter{section} has a side effect to reset its descendent counter
% \counter{subsection} and referred to from \!\thesubsection! command.  Thus
% if \counter{section} were local, the right-column subsections in
% Section~\ref{sec:env} would be numbered as ``0.1'', ``0.2'' and ``0.3''
% because the local value of \counter{section} would be zero.  Moreover, the
% right-column subsections of this section would be ``0.4'', ``0.5'' and
% ``0.6'' because stepping \counter{section} local to the left column would
% not reset \counter{subsection} local to the right column.
% 
% You may give a local appearance to a counter \textit{ctr} for the $i$-th
% column (zero origin) by a command;
% \begin{itemize}\item[]
% \Uidx{\!\definethecounter!}|{|\textit{ctr}|}{|$i$|}{|\textit{def}|}|
% \end{itemize}
% where \textit{def} is to be the body of the local definition of
% |\the|\textit{ctr}.  For example, the preamble of this document has the
% following to give non-default defitions to \!\thefootnote! and
% \!\thesubsection! for right columns.
% 
%\begin{Verbatim}
%  \definethecounter{footnote}{1}{%
%    \textit{\arabic{footnote}}}
%  \definethecounter{subsection}{1}{%
%    \texttt{%
%      \arabic{section}.\arabic{subsection}}}
%\end{Verbatim}
%\end{leftcolumn*}
% 
% \begin{rightcolumn}
% \begin{table}[b]\nosv
% \caption{\ttfamily Another Single-Column Table}
% \label{tab:right}
% \centerline{\ttfamily \begin{tabular}[t]{|l|r|}\hline
%   Another&example\\\hline
%   of&single\\\hline
%   column&table\\\hline
%   \end{tabular}}
% \end{table}
% \subsection{\ttfamily Figures and Tables}
%\begin{Verbatim}
%\begin{leftcolumn*}[\section{%
%   Floats, Footnotes and Counters}]
%\begin{table}[b]
%\caption{A Single-Column Table}
%\centerline{\begin{tabular}[t]{|l|c|r|}
%  \hline
%  An&example&of\\\hline
%  single&column&table\\\hline
%  \end{tabular}}
%\end{table}
%\subsection{Figures and Tables}
%As shown in this page, double-column
%figures\slash tables (or those spanned
%multiple columns if you have three or more
%columns) may be placed by |figure*| and
%\end{Verbatim}
% {\ttfamily |table*| environments as usual\footnote{Another example
% of footnote.\label{fn:right3}}. ...}
% 
% \subsection{\ttfamily Footnotes and Marginal Notes}
%\begin{Verbatim}
%Footnotes are also put at the bottom of the
%column in which |\footnote| commands and
%their references reside (like
%this\footnote{...}), as shown in page~2 and
%this page. Marginal notes behave similarly
%like what you are seeing in the left margin
%of this sentense\marginpar{\raggedright 
%An example of marginal note.} and the right
%marginal note in this
%\end{Verbatim}
% |page\footnote{...}. ...|
% \marginpar{\raggedright\ttfamily
% Another example of marginal note.}
% 
% \begin{figure}[t]\nosv
% \centerline{\begin{tabular}[b]{|c|}\hline
%     \hbox to.9\columnwidth{}\\\\\\\\
%     \ttfamily another figure with [t] option\\
%     \ttfamily to fill space
%     \\\\\\\\\hline
%     \end{tabular}}
% \caption{\ttfamily
% Another Figure with [t] Option}
% \end{figure}
%
% \begin{figure}[b]\nosv
% \centerline{\begin{tabular}[b]{|c|}\hline
%     \hbox to.9\columnwidth{}\\
%     \ttfamily a figure with [b] option\\
%     \ttfamily to fill space\\
%     \\\hline
%     \end{tabular}}
% \caption{\ttfamily
% A Figure with [b] Option}
% \end{figure}
% 
% \subsection{\ttfamily Local and Global Counters}
%\begin{Verbatim}
%You probably found that the numbering of
%figures and tables is \emph{global} while
%that of footnotes are \emph{local}. ...
%\end{leftcolumn*}
%\begin{rightcolumn}
%\end{Verbatim}
% \textit{source}.\\
% |\end{rightcolumn}|
%
% \begin{figure}[p]\nosv
% \centerline{\begin{tabular}[b]{|c|}\hline
%     \hbox to.9\columnwidth{}\\\\\\\\
%     \ttfamily a figure with [p] option\\
%     \ttfamily to fill space
%     \\\\\\\\\hline
%     \end{tabular}}
% \caption{\ttfamily
% A Figure with [p] Option}
% \end{figure}
% 
% \begin{figure}[p]\nosv
% \centerline{\begin{tabular}[b]{|c|}\hline
%     \hbox to.9\columnwidth{}\\\\\\\\
%     \ttfamily another figure with [p] option\\
%     \ttfamily to fill space
%     \\\\\\\\\hline
%     \end{tabular}}
% \caption{\ttfamily
% Another Figure with [p] Option}
% \end{figure}
% 
% \begin{figure}[p]\nosv
% \centerline{\begin{tabular}[b]{|c|}\hline
%     \hbox to.9\columnwidth{}\\\\\\\\
%     \ttfamily yet another figure with [p]\\
%     \ttfamily option to fill space
%     \\\\\\\\\hline
%     \end{tabular}}
% \caption{\ttfamily
% Yet Another Figure with [p] Option}
% \end{figure}
% 
% \begin{figure}[p]\nosv
% \centerline{\begin{tabular}[b]{|c|}\hline
%     \hbox to.9\columnwidth{}\\\\\\\\
%     \ttfamily fourth figure with [p]\\
%     \ttfamily option to fill space
%     \\\\\\\\\hline
%     \end{tabular}}
% \caption{\ttfamily
% Forth Figure with [p] Option}
% \end{figure}
% 
% \begin{figure}[t]\nosv
% \centerline{\begin{tabular}[b]{|c|}\hline
%     \hbox to.9\columnwidth{}\\\\\\
%     \ttfamily yet another figure with [t]\\
%     \ttfamily option to fill space
%     \\\\\\\hline
%     \end{tabular}}
% \caption{\ttfamily
% Yet Another Figure with [t] Option}
% \end{figure}
% \end{rightcolumn}  
% \switchcolumn*
% \flushpage
% \end{paracol}
% 
% 
% 
% \section{Closing \texttt{paracol} Environment and Page Flushing}
% \label{sec:man-close}
% The final example shown here is this single-column text which the author put
% after the \env{paracol} environment above is closed.  As you are seeing, a
% \env{paracol} environment can be finished at any vertical position in a
% page and can be followed by ordinary single column texts.
% 
% \begin{paracol}{2}
% \begin{leftcolumn}
% The environment may also be restarted anywhere you like as shown here.
% 
% The last issue is to flush a page.  The ordinary \!\newpage! command works
% as you expect.  If you say \!\newpage! in the left column in a page, the
% contents following it will appear in the left column in the next page.  Note
% that this does not affect the layout of the right column.
% 
% To flush all columns in a page, a command \Uidx{\!\flushpage!} is
% available.  This command in $i$-th column is almost equivalent to;
% \begin{itemize}\item[]
% \!\switchcolumn!|[|$i$|]*[|\!\newpage!|]|
% \end{itemize}
% but more robust\footnotemark\label{fn:flush}.
% The ordinary page breaking command \Uidx{\!\clearpage!} may also be used
% to flush all columns and to start a fresh page, but it has a side effect
% to put all figures and tables which are not yet output.
% \end{leftcolumn}
% 
% \begin{rightcolumn}
%\begin{Verbatim}
%\begin{paracol}{2}
%\begin{leftcolumn}
%The environment may also be restarted
%anywhere you like as shown here. ...
%\end{leftcolumn}
%\begin{rightcolumn}
%\end{Verbatim}
% \textit{source}\\
% |\end{rightcolumn}|\\
% |\end{paracol}|\\
% |Now the aurthor will do ...|
% \end{rightcolumn}
% \end{paracol}
% 
% \changes{v1.1}{2012/05/11}
% 	{Add \cs{columnratio}\texttt{\char`\{0.6\char`\}} and a phrase
%	 for it.}
% Now the author will do |\flushpage| shortly to start a real binlingual
% example from the next page, after showing another example of closing 
% \env{paracol} environments in this sentence and of restarting in the next
% one, in which {\em unbalanced column width} is demonstrated using
% \Uidx{\!\columnratio!} command shown in Section~\ref{sec:ref-colwidth}.
% 
% \columnratio{0.6}
% \begin{paracol}{2}
% \begin{leftcolumn}
% O.K., we have restarted \env{paracol} environment and we will see the
% effect of \!\flushpage! now!!\footnotetext{
% 
% For example \texttt{\string\switchcolumn*} may flush a page for the
% \sync{}ation and thus \texttt{\string\newpage} may leave an empty page.}
% 
% \end{leftcolumn}
% \begin{rightcolumn}
%\begin{Verbatim}
%\columnratio{0.6}
%\begin{paracol}{2}
%\begin{leftcolumn}
%O.K., ...
%\end{leftcolumn}
%\end{Verbatim}
% |\begin{rightcolumn}| \textit{source}\\
% |\end{rightcolumn}|
% \flushpage
% \end{rightcolumn}
% 
% 
% 
% \changes{v1.2-7}{2013/05/11}
% 	{Correct a few words in German and English libretti.}
% \newenvironment{Gverse}{\ensurevspace{2\baselineskip}\begin{leftcolumn*}
% 	\begin{myverse}}
%   {\end{myverse}\end{leftcolumn*}}
% \newenvironment{Everse}{\begin{rightcolumn}\begin{myverse}}
%   {\end{myverse}\end{rightcolumn}}
% \makeatletter
% \newenvironment{myverse}{\leftmargini0pt\partopsep0pt\verse}{\endverse}
% 
% \begin{leftcolumn*}[
% \centerline{\Large An Die Freude/To Joy}\label{page:bfreude}\smallskip
% \centerline{\large Friedrich Schiller}\smallskip
% The following is the libretto of the fourth movement of Beethoven's Ninth
% Symphony, his adaptation of Schiller's ode ``An Die Freude'' (or ``To Joy'' in
% English). Beethoven's additions and revisions are indicated in italics.]
% \end{leftcolumn*}
% \begin{Gverse}
% \itshape O Freunde, nicht diese T\"one! \\
% Sondern la{\ss}t uns angenehmere anstimmen und freu\-denvollere
% \footnote{If I had been a good student in my German class, I could find
% the German translation of the right column footnote \ref{fn:right4} is
% ``Dieser Teil wurde van Beethoven hinzugef\"ugt'' by myself without
% the kind help from a user.}.
% \end{Gverse}
% \begin{Everse}
% \itshape Oh friends, no more of these sad tones!\\
% Let us rather raise our voices together\\
% In more pleasant and joyful tones
% \footnote{This part was added by Beethoven.\label{fn:right4}}.
% \end{Everse}
% \begin{Gverse}
% Freude!\\
% Freude, sch\"oner G\"otterfunken
% Tochter aus Elysium,\\
% Wir betreten feuertrunken,
% Himmlische, dein Heiligtum!\\
% Deine Zauber binden wieder,
% {\itshape Was die Mode streng geteilt;\\
% Alle Menschen werden Br\"uder\footnote{
% Original: Was der Mode Schwert geteilt;\\
% Bettler werden F\"urstenbr\"uder,},}
% Wo dein sanfter Fl\"u\-gel weilt
% \end{Gverse}
% \begin{Everse}
% Joy! \\
% Joy, thou shining spark of God,\\
% Daughter of Elysium,\\
% With fiery rapture, goddess,\\
% We approach thy shrine.\\
% Your magic reunites\\
% {\itshape That which stern custom has parted;\\
% All humans will become brothers\footnote{
% Original: 
% What custom's sword has parted;\\
% Beggars become princes' brothers}}\\
% Under your protective wing.
% \end{Everse}
% \begin{Gverse}
% Wem der gro{\ss}e Wurf gelungen,
% eines Freundes Freund zu sein;\\
% Wer ein holdes Weib errungen,
% mische seinen Jubel ein!\\
% Ja, wer auch nur eine Seele 
% sein nennt auf dem Erdenrund!\\
% Und wer's nie gekonnt, der stehle 
% weinend sich aus diesem Bund! 
% \end{Gverse}
% \begin{Everse}
% Let the man who has had the fortune\\
% To be a helper to his friend,\\
% And the man who has won a noble woman,\\
% Join in our chorus of jubilation!\\
% Yes, even if he holds but one soul\\
% As his own in all the world!\\
% But let the man who knows nothing of this\\
% Steal away alone and in sorrow.
% \end{Everse}
% \begin{Gverse}
% Freude trinken alle Wesen
% an den Br\"usten der Natur;\\
% Alle Guten, alle B\"osen
% folgen ihrer Rosenspur.\\
% K\"usse gab sie uns und Reben,
% einen Freund, gepr\"uft im Tod;\\
% Wollust ward dem Wurm gegeben,
% und der Cherub steht vor Gott. 
% \end{Gverse}
% \begin{Everse}
% All the world's creatures drink\\
% From the breasts of nature;\\
% Both the good and the evil\\
% Follow her trail of roses.\\
% She gave us kisses and wine\\
% And a friend loyal unto death;\\
% She gave the joy of life to the lowliest,\\
% And to the angels who dwell with God. 
% \end{Everse}
% \begin{Gverse}
% Froh, wie seine Sonnen fliegen 
% durch des Himmels pr\"acht'gen Plan,\\
% Laufet, Br\"uder, eure Bahn,
% freudig, wie ein Held zum Siegen. 
% \end{Gverse}
% \begin{Everse}
% Joyous, as his suns speed\\
% Through the glorious order of Heaven,\\
% Hasten, brothers, on your way,\\
% Joyful as a hero to victory.
% \end{Everse}
% \begin{Gverse}
% Seid umschlungen, Millionen! 
% Diesen Ku{\ss} der ganzen Welt!\\
% Br\"uder, \"uber'm Sternenzelt
% mu{\ss} ein lieber Vater woh\-nen. 
% \end{Gverse}
% \begin{Everse}
% Be embraced, all ye millions!\\
% With a kiss for all the world!\\
% Brothers, beyond the stars\\
% Surely dwells a loving Father. 
% \end{Everse}
% \begin{Gverse}
% Ihr st\"urzt nieder, Millionen?
% Ahnest du den Sch\"opfer, Welt?\\
% Such'ihn \"uberm Sternenzelt! 
% \"Uber Sternen mu{\ss} er wohnen.
% \end{Gverse}
% \begin{Everse}
% Do you kneel before him, oh millions?\\
% Do you sense the Creator's presence?\\
% Seek him beyond the stars!\\
% He must dwell beyond the stars.
% \end{Everse}
% \end{paracol}
% \label{page:efreude}
% \endgroup
\endinput

% \tracingpages0 \tracingoutput0
% \newpage
% % \footnotelayout{m}
% \columnratio{}
% \section{Reference Manual\\参考手册}
% \label{sec:ref}


% 
% \subsection{计数器的命令\hfill Commands for Counters}
% \label{sec:ref-counter}
% 
% \begin{description}
% \item[\Midx{\!\globalcounter!}\marg{ctr}]\mbox{}
% \Item[\Midx{\!\globalcounter!}\texttt{*}]\mbox{}\par
% \changes{v1.32-1}{2015/10/10}
% 	{Add descriptions of \cs{globalcounter*}.}
% 
% The command \!\globalcounter!\marg{ctr} declares that the counter
% \meta{ctr} is global to all columns, while \!\globalcounter!|*| does so
% for all counters.  An update of a \Uidx\gcounter{} in a column is seen by
% any other columns.

% 命令 \!\globalcounter!\marg{ctr} 声明计数器\meta{ctr}在所有列中是全局的,而 \!\globalcounter!|*| 则对所有计数器都是如此。在某列中更新\Uidx\gcounter{}会被其他列看到。
% \begin{itemize}
% \item
% All column-local values of a descendant \lcounter{} of a \gcounter{} are
% zero-cleared when the \gcounter{} is explicitly stepped by \!\stepcounter!
% or \!\refstepcounter!, or implicitly by a sectioning command and so on.
% 
% 当一个 \gcounter{} 被 \!\stepcounter! 或 \!\refstepcounter! 显式步进,或者通过节标题命令等隐式步进时,其子孙 \lcounter{} 的所有列局部值都会被清零。
% \item
% The counter \counter{page} is always global but an explicit update of it
% by e.g., \!\setcounter! in a non-leftmost column is not seen by other
% columns and is canceled even for the column itself after a \cswitch{} or a
% page break in the column.  Therefore, if you want to make a \emph{jump} of
% \counter{page}, it must be done in the leftmost column 0.  Note that a
% jump from a page $p$ to $q$ can be seen in other columns even if they have
% gone beyond $p$ \emph{before} the column 0 makes the jump, as far as
% \counter{page} having $q$ (or its successor) is referred to by \!\pageref!
% or through \emph{contents} files such as |.toc|\footnote{
% Direct reference to \counter{page} may give an inconsistent result, as you
% might have in ordinary \LaTeX{} documents.}.

% 计数器\counter{page}始终是全局的,但是在非最左列中通过 \!\setcounter! 进行的显式更新在其他列中是不可见的,并且在该列进行\cswitch{}或页面断页后,甚至对于该列本身也会被取消。因此,如果要进行\emph{jump}(即跳转)\counter{page},必须在最左列0中进行。请注意,即使其他列在列0进行跳转\emph{之前}已经超过了页面$p$,只要\counter{page}具有$q$(或其后继者)的值,并且通过 \!\pageref! 或通过\emph{contents}文件(如|.toc|)进行引用,其他列仍然可以看到从页面$p$跳转到$q$。\footnote{直接引用 \counter{page} 可能会导致不一致的结果,就像在普通的 \LaTeX{}文档中可能遇到的那样。}
% 
% \item
% All counters except for \counter{page} are local by default.  This feature
% may cause a problem with some packages including \textsf{marginnote} and
% \textsf{(auto-)pst-pdf} having their own counters which must be global.
% Since it is tough to find the name of such counters from package sources,
% if you have something wrong with these (or other) packages, try to put
% \!\globalcounter!|*| in your preamble and use \!\localcounter! shown below
% to localize specific counters which you need to be local.
% 
除了\counter{page}计数器外,默认情况下所有计数器都是局部的。这一特性可能会导致一些包(包括\textsf{marginnote}和\textsf{(auto-)pst-pdf})出现问题,这些包具有必须是全局的计数器。由于很难从包的源代码中找到这些计数器的名称,如果您在使用这些(或其他)包时遇到问题,请尝试在导言区中使用 \!\globalcounter!|*| 命令,并使用下面显示的 \!\localcounter! 命令将需要局部化的特定计数器局部化。
% \item
% Globalizing a \meta{ctr} being already global is just ignored without any
% complaints.

如果一个已经是全局的\meta{ctr}被再次全局化,它会被静默地忽略,而不会有任何警告。
% \end{itemize}
% 
% 
% 
% \item[\Midx{\!\localcounter!}\marg{ctr}]\mbox{}\par
% The command declares that the counter \meta{ctr} is local for each column.
% \begin{itemize}
% \item
% Though this command is intended for localizing a \meta{ctr} which is once
% globalized, localizing a local counter does not causes any error but is
% just ignored.  Localizing the permanently global \counter{page} is also
% just ignored without any complaints.
% \end{itemize}
% 
% 
% 
% \item[\Midx{\!\definethecounter!}\marg{ctr}\marg{col}\marg{rep}]\mbox{}\par
% The command defines |\the|\meta{ctr} being \marg{rep} for the local use in
% the column \meta{col}.  That is, |\the|\meta{ctr} in the column \meta{col}
% acts as if it is defined by
% \!\renewcommand!\Arg{\cs{the}\meta{ctr}}\Arg{\meta{rep}}.
% 
% 
% 
% \item[\Midx{\!\synccounter!}\marg{ctr}]\mbox{}\par
% The command \emph{broadcasts} the value of the \lcounter{} \meta{ctr} in
% the column in which the command appears to the values in all other columns.
% 
% \item[\Midx{\!\syncallcounters!}]\mbox{}\par
% The command broadcasts the values of all \lcounter{}s in the column in
% which the command appears to the values in all other columns.
% \end{description}
% 
% 
% 
% \subsection{Page-Wise Footnotes}
% \label{sec:ref-scfnote}
% \changes{v1.2-2}{2013/05/11}
% 	{Add the sub-section ``Single-Columned Footnotes'' to describe newly
%	 introducerd commands for page-wise footnotes.}
% \changes{v1.3-5}{2013/09/17}
%	{Rename the sub-section title from ``Single-Columned Footnotes'' to
% 	``Page-Wise Footnotes'' following new naming.}
% 
% \begin{description}
% \item[\Midx{\!\footnotelayout!}\marg{layout}]\mbox{}\par
% The command specifies the \meta{layout}${}\in\{|c|,|p|,|m|\}$ of footnotes
% in \env{paracol} environments as follows.
% 
% \begin{description}
% \item[|c|\rm(\textit{olumn})] makes footnotes {\em\Uidx\mcfnote} (aka
% multi-columned) being default to place footnotes in each column at the
% bottom of the column and separating them from \Preenv{} and \Postenv{}
% footnotes.
% 
% \item[|p|\rm(\textit{age})] makes footnotes {\em\Uidx\scfnote} (aka
% single-columned) so that footnotes in all columns are gathered, typeset
% spanning all columns, and placed at the bottom of the page in which they
% appear or at the end of the \env{paracol} environment they belong to, so
% that they are separated from \Preenv{} and \Postenv{} footnotes.
% 
% \item[|m|\rm(\textit{erge})] makes \Scfnote{}s {\em\Uidx\mgfnote} with
% footnotes in outside of the environment but in the same page, i.e., those
% in \Preenv{} and \postenv.
% \end{description}
% 
% 
% \begin{itemize}
% \item
% An example of \Mgfnote{} is found in p.~\pageref{sec:ref-paracol} while
% you will see many of them in Section~\ref{sec:fnnp}\footnote{
% 
% The left-column footnote \ref{fn:flush} in p.~\pageref{fn:flush} looks like
% a merged footnote because it is at the bottom of the page and the marked
% text is above the single-column text.  However, it is an ordinary
% \mcfnote{} one produced by a trick with \cs{footnotemark} and
% \cs{footnotetext} in different \env{paracol} environments.}.
% 
% \item
% In any layouts, a footnote cannot have page breaks in it, i.e., a footnote
% is always put in a page as a whole.  This makes it impossible to have a
% footnote taller than \!\textheight! and thus you will see a warning
% message if you give a very long footnote which will be printed intruding
% into the area for page footer (or out of the paper bound).
% 
% \item
% Choosing the layout |p|age-wise or |m|erged makes \counter{footnote}
% counter global and \!\fncounteradjustment!  shown below performed inside
% \!\footnotelayout!.  Choosing |c|olumn-wise let the command do the
% operations oppositely, i.e., localizes \counter{footnote} and does
% \!\nofncounteradjustment!.  Though these settings are usually appropriate
% for each footnote layout but you can override them by explicitly using
% commands like \!\localcounter!|{footnote}|.
% 
% \item
% The command has to be outside of \env{paracol} environments to decide the
% action in the environments following them.  If it appears in a
% \env{paracol} environment, you will have a warning message saying it is
% ignored.
% 
% \item
% \changes{v1.3-5}{2013/09/17}
% 	{Remove description of \cs{multicolumnfootnotes},
%	 \cs{singlecolumnfootnotes}, \cs{mergedfootnotes} but mention they
%	 are still available.}
% 
% In old versions of \Paracol, namely 1.2 and its minor revisions 1.2x,
% footnote layout was controlled by a set of lengthy commands
% \Midx{\!\multicolumnfootnotes!} for |c|, \Midx{\!\singlecolumnfootnotes!}
% for |p|, and \Midx{\!\mergedfootnotes!} for |m|.
% Though they are still available and will be so forever for backward
% compatibility, it is recommended to use \!\footnotelayout!\footnote{
% 
% Not only for type saving but also for being familiar with this command
% which could have some advanced feature, for example to put gathered
% footnotes into a specific column, someday.}.
% 
% % \item
% It must be $\meta{layout}\in\{|c|,|p|,|m|\}$, or you will have an error
% message of illegal layout specifier.
% \end{itemize}
% 
% 
% 
% \KeepSpace{5}
% \item[\Midx{\!\footnote!}\texttt{*}\oarg{num}\marg{text}]\mbox{}
% \Item[\Midx{\!\footnotemark!}\texttt{*}\oarg{num}]\mbox{}
% \Item[\Midx{\!\footnotetext!}\texttt{*}\oarg{num}\marg{text}]\mbox{}\par
% The starred version of \!\footnote!, \!\footnotemark! and \!\footnotetext!
% are for the adjustment of the footnote numbering, the order of footnote
% marks in main texts, and the stacking order of footnotes at page
% bottom.  Their usages with various examples are given in
% Section~\ref{sec:fnnp}.
% 
% 
% 
% \KeepSpace{3}
% \item[\Midx{\!\fncounteradjustment!}]\mbox{}
% \Item[\Midx{\!\nofncounteradjustment!}]\mbox{}\par
% The maintenance of \counter{footnote} with the starred footnote commands
% such as \!\footnote!|*| shown above causes out-of-order progress of the
% counter to make it hard to have a consistent counter value at
% \Endparacol.  The command \!\fncounteradjustment! is to let \Endparacol{}
% adjust the value of the counter based on its value at
% \beginparacol{} and the number of footnote commands in the environment.
% The command \!\nofncounteradjustment! is to tell \Endparacol{} to do
% nothing as in default.
% 
% \begin{itemize}
% \item
% Though \!\footnotelayout! with |p|(age-wise) or |m|(erged) argument does
% \!\fncounteradjustment! while that with |c|(olumn) does
% \!\nofncounteradjustment! inside of it, you can override these settings by
% explicitly putting a counter adjustment command after \!\footnotelayout!.
% 
% \item
% The effect of \!\fncounteradjustment! is shown in Section~\ref{sec:fnnp}.
% \end{itemize}
% 
% 
% \item[\Midx{\!\belowfootnoteskip!}]\mbox{}\par
% \changes{v1.35-4}{2018/12/31}
% 	{Add description of \cs{belowfootnoteskip}.}
% The typesetting parameter specifies the amount of the space inserted below
% footnotes of single-column \preenv{} if it does not have bottom floats.  The
% default amount is 0\,pt, i.e., no space is added.
% 
% \end{description}
% 
% 
% \KeepSpace{6}
% \subsection{Commands for Coloring Texts and Column-Separating Rules}
% \label{sec:ref-tcolor}
% \changes{v1.3-3}{2013/09/17}
%	{Rename the sub-sectoin title from ``Commands for Text Coloring'' to
%	 ``Commands for Coloring Texts and Column-Separating Rules'' to
%	 add description of the rule coloring together with the rule
%	 drawing itself.}
% 
% \begin{description}
% \item[\Midx{\!\columncolor!}\oarg{mode}\marg{color}\oarg{col}]\mbox{}
% \Item[\Midx{\!\normalcolumncolor!}\oarg{col}]\mbox{}\par
% \changes{v1.2-1}{2013/05/11}
%	{Add description of \cs{columncolor} and \cs{normlcolumncolor}.}
% 
% The command \!\columncolor! declares that the \emph{default color} of a
% column is \meta{color} or what it specifies by the combination with the
% optional \meta{mode}.  The command \!\normalcolumncolor! declares the
% default color is what \!\normalcolor! specifies, i.e., black usually.  The
% target column of these commands is that in which the commands reside, or
% \meta{col} if it specified.
% 
% \begin{itemize}
% \item
% The command may be outside of \env{paracol} environment.  If so and
% \meta{col} is not provided, the target column is the leftmost 0.
% 
% \item
% The default color declaration is \emph{global}.  Therefore, even if the
% command appears in a \env{paracol} environment (and even in some grouping
% structure in it), the declaration will be kept effective after
% \Endparacol{} to determine the default color of the specified column in
% succeeding \env{paracol} environments.
% 
% \item
% To give a color to texts (and maybe other stuff) in a column correctly,
% you need to load \textsf{color} package or its relative (e.g.,
% \textsf{xcolor}) which the implementation of coloring in \textsf{paracol}
% relies on.
% 
% \item
% Coloring with \!\color!\oarg{mode}\marg{color} and other coloring commands
% in \env{paracol} environments is of course allowed.  One caution is that
% the \!\color! decides the color for following texts until other
% specification is given or the group surrounding the command is closed.
% Therefore, \!\switchcolumn! does not affect the coloring but a color given
% to the texts in a column is also applied to the texts in the column to be
% switched to.  This irrelativeness of coloring and \cswitch{} is shown in
% the example below.
% 
% \twosided[]\columnratio{0.5}\columnsep0pt
% \tolerance5000\hbadness5000
% \begin{paracol}{2}
% \columncolor{blue}
% This column is colored blue because\\
% \mbox{}\qquad \!\columncolor!|{blue}|\\
% is specfied.  Here we have a \!\switchcolumn!.
% \switchcolumn
% \columncolor{red}
% This column is colored red because\\
% \mbox{}\qquad\!\columncolor!|{red}|\\
% is specified.
% 
% \begin{color}{green}
% Now the color of the right column is changed to green because\\
% \mbox{}\qquad\!\begin!|{color}{green}|\\
% is given prior to this paragraph.  Now we have another \!\switchcolumn! to
% go back to the left.
% \switchcolumn
% The color of this paragraph is green because we are still in the
% environment of green coloring, which we are now closing.\par
% \end{color}%
% 
% Since the coloring environment has been closed, the color of this
% paragraph is the default blue.  Now we have yet another and the last
% \!\switchcolumn! to the right.
% \switchcolumn
% Since this paragraph is outside of the coloring environment, its color is
% the default red.
% \end{paracol}
% 
% \normalcolumncolor[0]\normalcolumncolor[1]
% The default coloring of columns does not affect anything outside of
% \env{paracol} environment of course, and thus this sentence is not
% colored\footnote{
% 
% Or colored black as \cs{normalcolor} specifies.}.
% \end{itemize}
% 
% 
% 
% \KeepSpace{4}
% \item[\Midx{\!\coloredwordhyphenated!}]\mbox{}
% \Item[\Midx{\!\nocoloredwordhyphenated!}]\mbox{}\par
% \changes{v1.3-3}{2013/09/17}
%	{Add description of \cs{coloredwordhyphenated} and
%	 \cs{nocoloredwordhyphenated}.}
% 
% The command \!\coloredwordhyphenated! allows the first word following a
% coloring command such as \!\color! to be hyphenated, but at the same time
% make it possible that a line is broken before the word.  The command
% \!\nocoloredwordhyphenated! acts oppositely and thus line breaking before
% the first word and hyphenating it are inhibited.  By default,
% \!\coloredwordhyphenated! is effective.
% 
% \begin{itemize}
% \item
% The implementation of \textsf{color} package and its relatives makes it
% impossible that \meta{word} is hyphenated when it appears like
% |{|\!\color!|{red}|\meta{word} \ldots|}| or
% \!\textcolor!|{|\meta{word} \ldots|}|.  This inhibition of the hyphenation
% is sometimes annoying especially when the document is multi-columned and
% thus a line is narrow and a column is written in a language having long
% words such as German.  Therefore in \Paracol{} package, a trick is used to
% allow the \meta{word} is hyphenated.  However this trick being insertion
% of a null horizontal space has a side effect that the word can have a line
% break before it.  Though this line break is usually unharmful, in a
% special occasion the break is undesirable and
% in\textcolor{red}{appropriate} by making it possible that the
% \emph{half-colored} word `inappropriate' is broken between `in' and
% `appropriate' without hyphenation.  Therefore, if you find such a
% inappropriate break, use \!\nocoloredwordhyphenated! as follows, for example.
% \begin{quote}
% |{\nocoloredwordhyphenated in\textcolor{red}{appropriate}}|
% \end{quote}
% \end{itemize}
% 
% 
% \KeepSpace{4}
% \item[\Midx{\!\colseprulecolor!}\oarg{mode}\marg{color}\oarg{col}]\mbox{}
% \Item[\Midx{\!\normalcolseprulecolor!}\oarg{col}]\mbox{}\par
% \changes{v1.3-3}{2013/09/17}
%	{Add description of \cs{colseprulecolor} and
%	 \cs{normalcolseprulecolor}.}
% 
% The command \!\colseprulecolor! declares the color for
% {\em\Uidx\cseprule{}s}, being the vertical rules drawn at the center of
% gaps between columns, is \meta{color} or what it specifies by the
% combination with the optional \meta{mode}.  The command
% \!\normalcolseprulecolor! declares the color of rules is what
% \!\normalcolor! specifies, i.e., black usually.  If the optional argument
% \meta{col} is given, these commands specifies the color of the rule in the
% gap following the column whose ordinal is \meta{col}, rather than all rules.
% 
% \begin{itemize}
% \item
% The rules are drawn if \LaTeX's typesetting parameter \!\columnseprule!
% for the rule width has non-zero value, e.g., 0.4\,|pt| to obey the
% standard rule thickness.  The rules are \emph{not} drawn on \pwstuff{},
% i.e., \Preenv{} and \postenv, \pwise{} floats or (\mgfnote{} or
% non-merged) \Scfnote{}s of course but also \mctext{}s.  Therefore, if a
% page has \mctext{}s, the rules are {\em broken} by them as shown in the
% red rule example below.
% \global\unitlength\@totalleftmargin
% \end{itemize}
% \end{description}
% 
% \columnseprule0.4pt\colseprulecolor{red}[1]\colseprulecolor{white}[0]
% \setcolumnwidth{\unitlength/0pt}
% \begin{paracol}{3}\switchcolumn\noindent 
% This is a left column paragraph preceding a \mctext.  Of cource the rule
% separating this and the next column starts from the top of this paragraph.
% \switchcolumn\noindent
% This is a right column paragraph preceding a \mctext{} given by the
% \!\switchcolumn!|*| at its end.
% \switchcolumn[1]*[\subsubsection*{\hbox to\unitlength{}
% An Example of Spanning Text Given by \cs{subsubsection}|*| Command}]
% Since we have a \mctext{} above, the red rule separating this and the next
% column is broken by the text.
% \switchcolumn
% It is also natural that the rule separating this and the previous column is
% terminated at the end of this \env{paracol} environment.
% \end{paracol}
% \columnseprule0pt\columnratio{}
% 
% \begin{description}
% \Item[]\mbox{}
% \begin{itemize}
% \Item
% To give a color to rules correctly, you need to load \textsf{color}
% package or its relative (e.g., \textsf{xcolor}) which the implementation
% of coloring in \textsf{paracol} relies on.
% 
% \item
% Once you give a color to rules in a specific gap with the optional
% \meta{col}, another \!\colseprulecolor! or \!\normalcolseprulecolor!
% without \meta{col} does \emph{not} change the color of the rule in the
% gap.
% \end{itemize}
% \end{description}
% 
% 
% 
% \KeepSpace{7}
% \subsection{Commands for Background Painting}
% \label{sec:ref-bgpaint}
% \changes{v1.3-3}{2013/09/17}
%	{Add the sub-section ``Commands for Background Painting.}
% 
% \begin{description}
% \item[\Midx{\!\backgroundcolor!}\marg{region}\oarg{mode}\marg{color}]
%     \mbox{}\par
% \Item[\Midx{\!\backgroundcolor!}
%     \Arg{\meta{region}$|(|x_0|,|y_0|)|$}\oarg{mode}\marg{color}]
%     \mbox{}\par
% \Item[\Midx{\!\backgroundcolor!}
%     \Arg{\meta{region}$|(|x_0|,|y_0|)||(|x_1|,|y_1|)|$}
%     \oarg{mode}\marg{color}]
%     \mbox{}\par
% \changes{v1.3-3}{2013/09/17}
%	{Add description of \cs{backgroundcolor}.}
% 
% The command declares that {\em\Uidx\bgpaint} of \meta{region} is performed
% with \meta{color} or what it specifies by the combination of the optional
% \meta{mode}.  The \meta{region} whose \bground{} is painted is one of the
% following.
% 
% \begin{description}
% \item[|c|\rm(\textit{olumn})] for all columns, or particular one if
% \meta{region} is |c|\oarg{col} to specify its ordinal \meta{col}.
% 
% \item[|g|\rm(\textit{ap})] for all gaps between columns, or particular one
% if \meta{region} is |g|\oarg{col} to specify the ordinal \meta{col} of the
% column preceding the gap.
% 
% \item[|s|\rm(\textit{panning})] for \mctext{}s.
% 
% \item[|f|\rm(\textit{loat})] for \pwise{} floats.
% 
% \item[|n|\rm(\textit{ote})] for (\mgfnote{} or non-merged) \Scfnote{}s.
% 
% \item[|p|\rm(\textit{re/post})] for \Preenv{} and \postenv.
% 
% \item[|t|\rm(\textit{op})] for top margin.
% 
% \item[|b|\rm(\textit{ottom})] for bottom margin.
% 
% \item[|l|\rm(\textit{eft})] for left margin.
% 
% \item[|r|\rm(\textit{ight})] for right margin.
% \end{description}
% 
% In addition, capitals of the keys above, i.e., |C|, |G|, \ldots, |L|, are
% also legitimate for {\em under painting}.  For example, you may specify to
% paint the \bground{} of a region, say top margin, by two
% \!\backgroundcolor! with |t| and |T| and with different color arranging the
% size of the region of either |t| or |T| (or both of them) by the
% \emph{\bgext} option shown below.
% 
% The optional $|(|x_0|,|y_0|)|$ is to enlarge the region to be painted
% shifting its left-top and right-bottom corner outside by
% the dimension $x_0$ horizontally and $y_0$ vertically, or to shrink it
% with negative dimensions.  This {\em\Uidx\bgext} can be asymmetric giving
% another optional $|(|x_1|,|y_1|)|$ so that it acts on the right-bottom
% corner while let $|(|x_0|,|y_0|)|$ shift only the left-top corner.
% Moreover, you may make each \bgext{} {\em infinite} by giving 10000\,|pt|
% (about 3.5\,m) to $x_0$, $y_0$, $x_1$ and/or $y_1$ so that the
% corresponding region edge is shifted to the paper edge.  Furthermore, this
% {\em\Uidx\bginfext{}} can be terminated at the point $\alpha$ inside the
% corresponding paper edge by giving $10000\,|pt|-\alpha$
% ($\alpha\leq1000\,|pt|$) to an extension parameter $x_0$, etc.
% 
% \begin{itemize}
% \item
% A region whose color is not specified is not painted and thus left blank
% (or kept as painted by \!\pagecolor! if you specify it).
% 
% \item
% Under-painting of columns and gaps by |C| and |G| is made for regions
% different from those over-painting |c| and |g|.  That is, under-painting
% is done ignoring all \pwstuff{} and thus the height of the regions is
% always $\!\textheight!+\!\maxdepth!$.  On the other hand, over-painting is
% only for chunks shrunk or separated by \pwstuff.
% 
% \item
% You may exploit the following painting order, where $x_i$
% is the $i$-th \mctext{} ($x\in\{|s|,|S|\}$) or $i$-th chunk followed by
% the $i$-th \mctext, $m$ and $n$ is the number of \mctext{}s and columns in
% a page respectively, to overlay a preceding region with a succeeding
% region, if your \emph{printer} allows overlaid color painting.
% 
% \begin{eqnarray*}
% |T|&\to&|B|\to|L|\to|R|
%     \to|G[|0|]|\to\cdots\to|G[|n{-}1|]|\to|C[|0|]|\to\cdots\to|C[|n{-}1|]|\\
% &\to&|t|\to|b|\to|l|\to|r|\to|N|\to|n|\to\{|F|,|P|\}\to\{|f|,|p|\}
%  \to|S|_1\to\cdots\to|S|_m\\
% &\to&|g|_1|[|0|]|\to\cdots|g|_1|[|n{-}2|]|\to
%      |c|_1|[|0|]|\to\cdots|c|_1|[|n{-}1|]|\to|s|_1\\
% &\to&\cdots\\
% &\to&|g|_m|[|0|]|\to\cdots|g|_m|[|n{-}2|]|\to
%      |c|_m|[|0|]|\to\cdots|c|_m|[|n{-}1|]|\to|s|_m\\
% &\to&|g|_{m+1}|[|0|]|\to\cdots|g|_{m+1}|[|n{-}2|]|\to
%      |c|_{m+1}|[|0|]|\to\cdots|c|_m|[|n{-}1|]|
% \end{eqnarray*}
% 
% \item
% If you specify |b| feature by \!\twosided!, \bgpaint{} is
% {\em\Uidx\mirror{}ed} in even-numbered pages so that |l| and |L| mean
% right margin, |r| and |R| mean left margin, and asymmetric extensions are
% applied to right-top and left-bottom corners.
% 
% \item
% To give a color for \bgpaint{} correctly, you need to load \textsf{color}
% package or its relative (e.g., \textsf{xcolor}) which the implementation
% of coloring in \textsf{paracol} relies on.
% 
% \item
% To paint margins and regions having infinite extension correctly, the
% parameters \!\paperwidth! and \!\paperheight! should be set properly by,
% for example, a paper selection option of \!\documentclass!.
% 
% \item
% Section~\ref{sec:bgpaint} shows examples of \bgpaint{} to give you more
% intutive explanations of \!\backgroundcolor! and its region specifications.
% \end{itemize}
% 
% 
% 
% \item[\Midx{\!\nobackgroundcolor!}\marg{region}]\mbox{}
% \Item[\Midx{\!\resetbackgroundcolor!}]\mbox{}\par
% \changes{v1.3-3}{2013/09/17}
%	{Add description of \cs{nobackgroundcolor} and
%	 \cs{resetbackgroundcolor}.}
% 
% The command \!\nobackgroundcolor! declares that the \bground{} of
% \meta{region} is not painted, where \meta{region} is one of legitimate
% region specifiers of \!\backgroundcolor!.  The command
% \!\resetbackgroundcolor! declares no regions are painted and thus gives
% you the default state.
% 
% \begin{itemize}
% \item
% If you specified the \bgpaint{} of |c|\oarg{col} or |g|\oarg{col} by
% \!\backgroundcolor!, the painting is \emph{not} canceled by
% \!\nobackgroundcolor! with |c| or |g| but without \oarg{col}.  Similarly,
% once you made declarations of \bgpaint{} of both |c| and |c|\oarg{col}
% (resp.\ |g| and |g|\oarg{col}), \!\nobackgroundcolor! with |c|\oarg{col}
% (resp.\ |g|\oarg{col}) cancels the painting of |c|\oarg{col} (resp.\
% |g|\oarg{col}) but the region will still be painted by the color you gave
% to |c| (resp.\ |g|).
% \end{itemize}
% 
% 
% 
% \item[\Midx{\!\pagerim!}]\mbox{}\par
% \changes{v1.3-3}{2013/09/17}
%	{Add description of \cs{pagerim}.}
% 
% This is a (kind of) \emph{length command}\footnote{
% 
% In reality, it is a \cs{dimen} register rather than a \cs{skip} register.}
% 
% to have the width of the \emph{rim} area placed at each paper edge to 
% inhibit \bgpaint{} in the area.  That is, the inner edges of the area are
% considered as virtual paper edges to block painting of all margins and
% regions having \bginfext{} to the edges, for example in order to
% avoid printing troubles caused by painting the rim area too close to the
% real paper edges.  The default value of \!\pagerim! is 0 to allow paint
% anywhere in a paper.
% \end{description}
% 
% 
% 
% \subsection{Control of Contents Output}
% \label{sec:ref-contents}
% 
% \begin{description}
% \item[\Midx{\!\addcontentsonly!}\marg{file}\marg{col}]\mbox{}\par
% The command inhibits the output of contents information to
% $\meta{file}\in\{|toc|,|lof|,|lot|\}$ from columns other than \meta{col}.
% 
% \begin{itemize}
% \item
% For example, this manual has \!\addcontentsonly!|{toc}{0}| to
% inhibit the contents information output from \!\subsection! commands
% in the right column in Section~\ref{sec:env} and~\ref{sec:float},
% or the table should have duplicated entries of sub-sections.
% 
% \item
% It must be $\meta{file}\in\{|toc|,|lof|,|lot|\}$, or you will have an
% error message of illegal type of contents file.
% \end{itemize}
% \end{description}
% 
% 
% 
% \subsection{Page Flushing Commands}
% \label{sec:ref-flush}
% 
% \begin{description}
% \item[\Midx{\!\flushpage!}]\mbox{}\par
% The command flushes pages up to the {\em\Uidx\tpage} in which the \lcolumn{}
% resides.  Deferred floats which can be put in the pages up to the \tpage{}
% are also flushed.
% 
% \item[\Midx{\!\clearpage!}]\mbox{}\par
% The command does what \!\flushpage! does and then flushes all floats still
% deferred if any.  The deferred float flushing beyond the \tpage{} takes
% place at first for \cwise{} ones creating \fcolumn{}s for them, and
% then for \pwise{} ones creating {\em\Uidx\fpage{}s} only with
% \pwise{} floats, as \LaTeX's \!\clearpage! does outside \env{paracol}
% environment.
% 
% \item[\Midx{\!\cleardoublepage!}]\mbox{}\par
% \changes{v1.3-5}{2013/09/17}
%	{Add description of \cs{cleardoublepage}.}
% The command does what \LaTeX's \!\cleardoublepage! does outside
% \env{paracol}.  That is, it does \!\clearpage! always and then leaves a
% blank page if it is even-numbered and two-sided |p|(age) feature is
% enabled by |twoside| option of \!\documentclass! or \Paracol's own
% \!\twosided! command shown in Section~\ref{sec:ref-twoside}.
% 
% \begin{itemize}
% \item
% This command is equivalent to \!\clearpage! in \env{paracol} environments
% for \npaired{} \parapag{}ing because \!\clearpage! flushes \emph{both}
% left and right \parapag{}es.
% \end{itemize}
% \end{description}
\endinput

% \newpage
% % \def\Dotquad{\leavevmode\cleaders\hbox to.44em{\hss.\hss}%
%   \hskip\parindent\kern0pt}
% \def\fnpar#1#2{#1 paragraph\Dotfill\\\Dotfill with a footnote#2\Dotfill
%   in it.\par}
% \def\Fnpar#1#2#3{#1 paragraph\Dotfill\\\Dotfill with a footnote#2\Dotfill
%   in it.\\#3\par}
% 

 
% \subsection{Commands \cs{footnote*} and Relatives\hfill \cs{footnote*} 命令及相关命令}
% \label{sec:fnnp-starred}
% 
% \begin{description}
% \item[\Midx{\!\footnote!}\texttt{*}\oarg{|+|disp}\marg{text}]\mbox{}
% \Item[\Midx{\!\footnote!}\texttt{*}\oarg{|-|disp}\marg{text}]\mbox{}
% \Item[\Midx{\!\footnote!}\texttt{*}\oarg{disp}\marg{text}]\mbox{}\par
% The command is similar to its non-starred counterpart but the explicit
% numbering with the optional argument is done in \emph{self-relative} or
% \emph{base-displacement} style.  That is, if the optional argument has a
% leading `|+|' or `|-|',  the number given to the footnote is
% $f+\meta{disp}$ or $f-\meta{disp}$ respectively where $f$ is the value of
% \counter{footnote} counter, or in other words the number given to the last
% footnote\footnote{%
% If it is put by the ordinary \cs{footnote}.}.

% 该命令与其非星号版本类似,但是使用可选参数进行的显式编号是以\emph{自相对}或\emph{基准位移}的方式进行的。也就是说,如果可选参数以`|+|'或`|-|'开头,给予脚注的编号分别为$f+\meta{disp}$或$f-\meta{disp}$,其中$f$是\counter{footnote}计数器的值,或者换句话说,是给予最后一个脚注的编号\footnote{如果它是由普通的\cs{footnote}命令放置的。}。

% Otherwise, i.e., the optional argument is a number without |+|/|-| sign,
% the number given to the footnote is $b+\meta{disp}$ where $b$ is the base
% value of \counter{footnote} counter at \beginparacol{} for the environment
% in which the command appears, or in other words the number given to the
% last \Preenv{} footnote\footnote{
% 
% Or the last footnote in the previous \env{paracol} environment,
% etc.\label{fn:4L0}}.

否则,即可选参数是一个没有 |+|/|-| 符号的数字,则给定的脚注编号是 $b+\meta{disp}$,其中 $b$ 是 \beginparacol{} 处的 \counter{footnote} 计数器的基础值,用于包含该命令的环境,或者换句话说,给定的是最后一个 \Preenv{} 脚注\footnote{或者是前一个 \env{paracol} 环境中的最后一个脚注,等等。\label{fn:4L0}}。
% 
% In addition, unlike the non-starred version, this command updates
% \counter{footnote} counter with the number given to the footnote, i.e.,
% $f\gets f+\meta{disp}$, $f\gets f-\meta{disp}$ or $f\gets b+\meta{disp}$
% is performed, so that following \!\footnote! without explicit numbering
% option have numbers $f+1$, $f+2$ and so on with new $f$.

此外,与非星号版本不同,该命令使用给定的脚注编号更新\counter{footnote}计数器,即执行$f\gets f+\meta{disp}$、$f\gets f-\meta{disp}$或$f\gets b+\meta{disp}$,以便在没有显式编号选项的情况下,后续的 \!\footnote!命令具有编号$f+1$、$f+2$等,并更新$f$的值。
% \begin{itemize}
% \item
% If the optional argument is not provided, it is assumed that |[+1]| is
% given and thus \!\footnote!|*|\marg{text} acts as \!\footnote!\marg{text}.

如果没有提供可选参数,则假定提供了|[+1]|,因此 \!\footnote!|*|\marg{text}的作用等同于 \!\footnote!\marg{text}。
% \end{itemize}
% 
% \item[\Midx{\!\footnotemark!}\rm|*[|{[|+-|]}\meta{disp}{|]|}]\mbox{}\par
% This command is a mixture of its non-starred counterpart and
% \!\footnote!|*|.  That is the number for the footnote mark is calculated
% in the way of \!\footnote!|*| and \counter{footnote} counter is updated.
% 
这个命令是它的非星号版本和 \!\footnote!||的混合体。即脚注标记的编号是根据 \!\footnote!|*|的方式计算的,并且\counter{footnote}计数器会被更新。
% \item[\Midx{\!\footnotetext!}\rm|*[|{[|+-|]}\meta{disp}{|]|}\marg{text}]
% \mbox{}\par
% Without the optional argument |[|[|+-|]\meta{disp}|]|, this command does what
% \!\footnotetext!\marg{text} does but in addition increments
% \counter{footnote} counter before that.  With the optional argument, on
% the other hand, the number given to the footnote \meta{text} is calculated
% as done in \!\footnote!, but the \counter{footnote} counter is not
% updated.

如果没有提供可选参数 |[|[|+-|]\meta{disp}|]|,则此命令的作用与 \!\footnotetext!\marg{text} 相同,但在此之前会增加\counter{footnote}计数器的值。另一方面,如果提供了可选参数,那么给定给脚注\meta{text}的编号将按照 \!\footnote! 的方式计算,但\counter{footnote}计数器不会被更新。
% \end{description}
% 
% With these starred commands, you can produce the following using the
% base-displacement mechanism without worrying about the absolute value of
% \!\footnote! counter and its change.
% 
使用这些带星号的命令,您可以使用基础位移机制生成以下内容,而无需担心 \!\footnote! 计数器的绝对值及其变化。
% \Hrule
% \begin{paracol}{2}
% \tolerance5000\hbadness5000
% \fnpar{First left-column}{\footnote{First left-column
% footnote.\label{fn:4L1}}}
% \Fnpar{Second left-column}{\footnote{Second left-column
% footnote.\label{fn:4L2}}}{
% It is followed by \cs{footnotetext}|*[3]|\marg{text} and a
% \cs{switchcolumn}.}
% \footnotetext*[3]{Third left-column footnote given by
% \cs{footnotetext}|*[3]|\marg{text} placed at the end of
% the second left-column paragraph to have
% $\ref{fn:4L3}=\ref{fn:4L0}+3$.\label{fn:4L3}}
% \switchcolumn
% \Fnpar{First right-column}{\footnote*[4]{First right-column
% footnote whose number \ref{fn:4R1} is given by
% \cs{footnote}|*[4]|\marg{text} because
% $\ref{fn:4R1}=\ref{fn:4L0}+4$.\label{fn:4R1}}}{It is followed by a
% \cs{switchcolumn*}.}
% \switchcolumn*
% Third and synchronized left-column paragraph\Dotfill\\
% \Dotfill with a footnote whose mark
% here\footnotemark*[3]\Dotfill\\
% is given by \!\footnotemark!|*[3]| because $\ref{fn:4L3}=\ref{fn:4L0}+3$.
% It is followed by a \!\switchcolumn!.
% \switchcolumn
% \fnpar{Second and synchronized right-column}{\footnote*[5]{Second right-column
% footnote produced by \cs{footnote}|*[5]|\marg{text} because
% $\ref{fn:4R2}=\ref{fn:4L0}+5$.\label{fn:4R2}}}
% \fnpar{Third right-column}{\footnote{Third right-column
% footnote produced by \cs{footnote}\marg{text} because
% $\ref{fn:4R3}=\ref{fn:4R2}+1$.\label{fn:4R3}}}
% \end{paracol}
% \newpage
% 
% The other way to produce the same result except for the absolute footnote
% numbers is to use the self-relative mechanism and to exploit the progress
% of \counter{footnote} counter as follows.
% 
另一种产生相同结果的方法(除了绝对脚注编号)是使用自相对机制,并利用 \counter{footnote} 计数器的进展,方法如下:
% \Hrule
% \begin{paracol}{2}
% \tolerance5000\hbadness5000
% \fnpar{First left-column}{\footnote{First left-column
% footnote.\label{fn:5L1}}}
% \Fnpar{Second left-column}{\footnote{Second left-column
% footnote.\label{fn:5L2}}}{
% It is followed by \cs{footnotetext}|*|\marg{text} and a
% \cs{switchcolumn}.}
% \footnotetext*{Third left-column footnote given by
% \cs{footnotetext}|*|\marg{text} placed at the end of
% the second left-column paragraph because it follows the second footnote
% \ref{fn:5L2}.\label{fn:5L3}}
% \switchcolumn
% \Fnpar{First right-column}{\footnote{First right-column
% footnote whose number \ref{fn:5R1} is given by
% \cs{footnote}\marg{text} because
% $\ref{fn:5R1}=\ref{fn:5L3}+1$ and \cs{footnotetext*} for \ref{fn:5L3} lets
% \counter{footnote} have the value.\label{fn:5R1}}}{It is followed by a
% \cs{switchcolumn*}.}
% \switchcolumn*
% Third and synchronized left-column paragraph\Dotfill\\
% \Dotfill with a footnote whose mark
% here\footnotemark*[-1]\Dotfill\\
% is given by \!\footnotemark!|*[-1]| because $\ref{fn:5L3}=\ref{fn:5R1}-1$.
% It is followed by a \!\switchcolumn!.
% \switchcolumn
% \fnpar{Second and synchronized right-column}{\footnote*[+2]{Second
% right-column footnote produced by \cs{footnote}|*[+2]|\marg{text} because
% $\ref{fn:5R2}=\ref{fn:5L3}+2$.\label{fn:5R2}}}
% \fnpar{Third right-column}{\footnote{Third right-column
% footnote produced by \cs{footnote}\marg{text} because
% $\ref{fn:5R3}=\ref{fn:5R2}+1$.\label{fn:5R3}}}
% \end{paracol}
% \Hrule
% 
% It depends on the structure of your document which of the
% base-displacement and self-relative is better.  If your document has
% frequent switching between single- and multi-column text typesetting and
% thus the contents of a \env{paracol} environment is relatively small, the
% base-displacement is a good choice because you may concentrate on one
% base value of \counter{footnote} counter.  Otherwise, especially when your
% document consists of one single and large \env{paracol} environment, the
% base-displacement is almost equivalent to maintaining absolute values and
% thus the self-relative should be preferred.
% 

% 这取决于你的文档结构,基准位移和自相对哪个更好。如果你的文档经常在单列和多列文本排版之间切换,因此\env{paracol}环境的内容相对较小,那么基准位移是一个不错的选择,因为你可以专注于\counter{footnote}计数器的一个基准值。否则,特别是当你的文档由一个单独且较大的\env{paracol}环境组成时,基准位移几乎等同于维护绝对值,因此应该优先选择自相对方式。

% Note that if the last \!\footnote! or \!\footnotemark! in a \env{paracol}
% environment is starred, the command lets \counter{footnote} counter have
% some value smaller than that for the last stacked footnote.  For example, 
% if the second and third right-column footnotes \ref{fn:5R2} and
% \ref{fn:5R3} are omitted from the example above, the last footnote-related
% command will be \!\footnotemark!|*[-1]| which makes the counter at
% \Endparacol{} \ref{fn:5L3} rather than \ref{fn:5R1}.  You may not worry
% about this problem, however, because \Endparacol{} automatically maintains
% the counter letting it have $b+n$ where $n$ is the number of \!\footnote!
% and \!\footnotemark! in the environment, if the maintenance is ordered by
% the command \!\fncounteradjustment! which is automatically executed by
% \!\footnotelayout! with the argument |p| or |m|.
% 
请注意,如果在 \env{paracol} 环境中的最后一个 \!\footnote! 或 \!\footnotemark! 带有星号,那么该命令会使 \counter{footnote} 计数器的值小于最后一个堆叠脚注的值。例如,如果上面的示例中省略了第二个和第三个右列脚注 \ref{fn:5R2} 和 \ref{fn:5R3},那么最后一个与脚注相关的命令将是 \!\footnotemark!|*[-1]|,它使得在 \Endparacol{} 处的计数器为 \ref{fn:5L3} 而不是 \ref{fn:5R1}。然而,您可能不必担心这个问题,因为 \Endparacol{} 会自动维护计数器,使其为 $b+n$,其中 $n$ 是环境中 \!\footnote! 和  \!\footnotemark! 的数量,如果维护是由命令 \!\fncounteradjustment! 规定的,该命令会在 \!\footnotelayout! 中使用参数 |p| 或 |m| 自动执行。

% 
% \subsection{Page Break}
% \label{sec:fnnp-pbreak}
% 
% When a \env{paracol} environment with footnotes lays across a page boundary,
% you could have some weird result even if the environment have just one
% \!\switchcolumn! as shown below.
% 
% \Hrule
% \begin{paracol}{2}
% First left-column paragraph \Dotfill\\
% \Dotfill with a footnote\footnote{First left-column
% footnote.\label{fn:6L1}}\Dotfill\\
% \Dotfill\\ \Dotfill\\ \Dotfill\\ \Dotfill\\ \Dotfill\\
% \Dotfill in it.
% \par
% Second left-column paragraph \Dotfill\\
% \Dotfill with a footnote\footnote{Second left-column
% footnote.\label{fn:6L2}}
% \Dotfill in it.
% \switchcolumn
% First right-column paragraph \Dotfill\\
% \Dotfill with a footnote\footnote{First right-column
% footnote weirdly placed here while the footnoted main text is in the
% previous page.\label{fn:6R1}}\Dotfill\\
% \Dotfill\\ \Dotfill\\ \Dotfill\\ \Dotfill\\ \Dotfill\\
% \Dotfill in it.
% \par
% Second right-column paragraph \Dotfill\\
% \Dotfill with a footnote\footnote{Second right-column
% footnote whose mark in the main text gives impression that footnote
% numbering jumps from \ref{fn:6L2} to \ref{fn:6R2}.\label{fn:6R2}}
% \Dotfill in it.
% \end{paracol}
% \Hrule
% 
% Since the part of the source |.tex| for this example above is
% fundamentally same as that in p.~\pageref{sec:fnnp} at the beginning of
% this Section~\ref{sec:fnnp}, footnotes are simply numbered in
% left-column-first manner without any tricks.  However it results in
% giving an impression that two paragraphs in each of both columns at the
% bottom of the last page have footnote marks of inconsecutive numbers
% \ref{fn:6L1} and \ref{fn:6R1} due to the second left-column paragraph and
% the footnote \ref{fn:6L2} in it.  More weirdly, the first right-column
% footnote \ref{fn:6R1} is not put in the last page where its mark is shown
% but is stacked below \ref{fn:6L2} in this page.
% 
% The reason why this happens is that a footnote is not immediately put to
% the bottom of the page where its mark resides but to the page constructing
% at the time when the footnote is processed at the end of the paragraph in
% which the corresponding \!\footnote! (or \!\footnotetext!)
% occurs\footnote{
% 
% More accurately, the footnote is kept in a place in \TeX{} together with
% other preceding but still unprocessed footnotes and then \TeX{} examines
% them at the end of a paragraph in which a page break is found to decide
% whether each of them is included in the page just being completed.}.
% 
% Therefore, it may happen even in an ordinary single-column document or a
% \env{paracol}ed multi-column one with \Mcfnote{}s that a
% footnote is thrown to the page $p+1$ next to the page $p$ in which its
% mark is left, when the mark is placed around the bottom of the page
% $p$.
% 
% This footnote placement mechanism becomes clearly visible in the example
% above in which the footnote \ref{fn:6R1} is processed {\em after} the
% second left-column paragraph is processed to complete the last page giving
% no chance to the footnote placed in the page\footnote{%
% 
% In fact, even \cs{footnote} for the footnote is processed after the page
% break in this case.}.
% 
% Therefore, the solution of this placement problem is to let the first
% right-column footnote processed {\em before} the page is broken by the
% progress of the left-column.  That is, in the solution shown below the
% author inserted \!\switchcolumn! after the first left-column paragraph to
% let the first right-column paragraph and its footnote are processed, and
% then did \!\switchcolumn! again after the right-column paragraph to go
% back to the left-column.
% 
% \Hrule
% \begin{paracol}{2}
% First left-column paragraph \Dotfill\\
% \Dotfill with a footnote\footnote{First left-column
% footnote.\label{fn:7L1}}\Dotfill\\
% \Dotfill\\ \Dotfill\\ \Dotfill\\ \Dotfill\\ \Dotfill\\ \Dotfill\\ \Dotfill\\
% \Dotfill\\ \Dotfill\\ \Dotfill\\ \Dotfill\\ \Dotfill\\ \Dotfill\\ \Dotfill\\
% \Dotfill\\
% \Dotfill in it.\\
% It is followed by a \!\switchcolumn!.
% \par\switchcolumn
% First right-column paragraph \Dotfill\\
% \Dotfill with a footnote\footnote{First right-column
% footnote which is now placed in this page where its mark \ref{fn:7R1}
% resides.\label{fn:7R1}}\Dotfill\\
% \Dotfill\\ \Dotfill\\ \Dotfill\\ \Dotfill\\ \Dotfill\\ \Dotfill\\ \Dotfill\\
% \Dotfill\\ \Dotfill\\ \Dotfill\\ \Dotfill\\ \Dotfill\\ \Dotfill\\ \Dotfill\\
% \Dotfill in it.\\
% It is followed by a \!\switchcolumn! to go back to the left column.
% \par\newpage\switchcolumn
% Second left-column paragraph \Dotfill\\
% \Dotfill with a footnote\footnote{Second left-column
% footnote whose number \ref{fn:7L2} follows the right-column footnote
% \ref{fn:7R1} in the last page.\label{fn:7L2}}
% \Dotfill in it.\\
% It is also followed by a \!\switchcolumn!.
% \switchcolumn
% Second right-column paragraph \Dotfill\\
% \Dotfill with a footnote\footnote{Second right-column
% footnote whose number \ref{fn:7R2} follows the left-column footnote
% \ref{fn:7L2}.\label{fn:7R2}}
% \Dotfill in it.
% \end{paracol}
% \Hrule
% 
% Unfortunately, this tactics does not always solve the problem.  If a
% left-column paragraph has a page break in it and a footnote before the
% break, doing \!\switchcolumn! after the paragraph is too late to let
% right-column footnotes reside in the page just having been broken, while
% inserting \!\switchcolumn! before the paragraph should cause incorrect
% stacking order.
% 
% The remedy for this problem is similar to that shown in
% Section~\ref{sec:fnnp-multsc} to cope with multiple \!\switchcolumn! in a
% \env{paracol} environment.  Here it is shown a little bit more formally.
% Suppose we have a page in a \env{paracol} environment in which a page
% break occurs in $p_l$-th and $p_r$-th paragraphs in the left and right
% columns respectively.  Thus we have $p_l-1$ and $p_r-1$ completed
% paragraphs in each of both columns.  Let $n_l$ (resp.\ $n_r$) be the
% number of footnotes in the pre-break left-column (resp.\ right-column)
% paragraphs, and $m_l$ (resp.\ $m_r$) be the number of pre-break footnotes
% in the $p_l$-th (resp.\ $p_r$-th) paragraph.  Thus we have $n_l+m_l$
% (resp.\ $n_r+m_r$) footnotes in the left (resp.\ right) column of the page
% before the break.  The following construct assures that those footnotes
% are properly numbered and stacked at the bottom of the page.
% 
% \begin{list}{}{\rightmargin\leftmargin \itemindent-.5\leftmargin
% \listparindent\itemindent \leftmargin1.5\leftmargin \parsep0pt}\it\item
% First to $(p_l-1)$-th paragraphs with $n_l$ footnotes in total given by
% {\rm\!\footnote!\marg{text}}.\par
% {\rm\!\footnotetext!|*{|{\it 1st footnote in $p_l$-th paragraph}|}|}\par
% \mbox{\qquad}\ldots\par
% {\rm\!\footnotetext!|*{|{\it$m_l$-th footnote in $p_l$-th paragraph}|}|}\par
% \!\switchcolumn!\par
% First to $(p_r-1)$-th paragraphs with $n_r$ footnotes in total given by
% {\rm\!\footnote!\marg{text}}.\par
% {\rm\!\footnotetext!|*{|{\it 1st footnote in $p_r$-th paragraph}|}|}\par
% \mbox{\qquad}\ldots\par
% {\rm\!\footnotetext!|*{|{\it$m_r$-th footnote in $p_r$-th paragraph}|}|}\par
% \!\switchcolumn!\par
% $p_l$-th paragraph whose first footnote mark is given by
% {\rm\!\footnotemark!|*[-|$(m_l{+}n_r{+}m_r{-1})$|]|}, while second to
% $m_l$-th ones are given by \!\footnotemark! without {\rm|*|} nor optional
% {\rm\oarg{num}}.  The first subsequent footnotes beyond the page break, if
% any, is given by {\rm\!\footnote!|*[+|$(n_r{+}m_r{+1})$|]|\marg{text}}
% while further subsequent ones are given by
% {\rm\!\footnote!\marg{text}}.\par
% \!\switchcolumn!\par
% $p_r$-th paragraph whose first footnote mark is given by
% {\rm\!\footnotemark!|*[-|$(m_r{+}k_l{-1})$|]|} where $k_l$ is the number
% of left-column footnotes beyond the break, while second to $m_r$-th ones
% are given by \!\footnotemark!.  The first subsequent footnotes beyond the
% page break, if any, is given by
% {\rm\!\footnote!|*[+|$(k_l{+1})$|]|\marg{text}}, while further subsequent
% ones are given by {\rm\!\footnote!\marg{text}}.
% \end{list}
% %
% The example shown in the next two pages is for the case of
% $p_l=p_r=n_l=n_r=m_l=m_r=k_l=2$.
% 
% \newpage
% \Hrule
% \begin{paracol}{2}
% First left-column paragraph with two footnotes\break
% \mbox{}\Dotquad here\footnote{First left-column footnote given by
% \cs{footnote}\marg{text}.\label{fn;8L1}} by
% \!\footnote!\marg{text}\Dotfill\\
% \Dotquad and here\footnote{Second left-column footnote also given by
% \cs{footnote}\marg{text}.\label{fn:8L2}} also by
% \!\footnote!\marg{text}\Dotfill\\
% \Dotfill\\\Dotfill\\\Dotfill\\\Dotfill\\\Dotfill\\
% \Dotfill\\\Dotfill\\\Dotfill\\\Dotfill\\\Dotfill\\
% \Dotfill\\\Dotfill\\\Dotfill\\\Dotfill\\\Dotfill\\
% followed by a series of \!\footnotetext!|*|\marg{text} and then a
% \!\switchcolumn!.
% \footnotetext*{Third left-column footnote given by
% \cs{footnotetext*}\marg{text}.\label{fn:8L3}}
% \footnotetext*{Fourth left-column footnote given by
% \cs{footnotetext*}\marg{text}.\label{fn:8L4}}
% 
% \switchcolumn
% First right-column paragraph with two footnotes\break
% \mbox{}\Dotquad here\footnote{First right-column footnote given by
% \cs{footnote}\marg{text}.\label{fn;8R1}} by
% \!\footnote!\marg{text}\Dotfill\\
% \Dotquad and here\footnote{Second right-column footnote also given by
% \cs{footnote}\marg{text}.\label{fn:8R2}} also by
% \!\footnote!\marg{text}\Dotfill\\
% \Dotfill\\\Dotfill\\\Dotfill\\\Dotfill\\\Dotfill\\
% \Dotfill\\\Dotfill\\\Dotfill\\\Dotfill\\\Dotfill\\
% \Dotfill\\\Dotfill\\\Dotfill\\\Dotfill\\\Dotfill\\
% followed by a series of \!\footnotetext!|*|\marg{text} and then a
% \!\switchcolumn!.
% \footnotetext*{Third right-column footnote given by
% \cs{footnotetext*}\marg{text}.\label{fn:8R3}}
% \footnotetext*{Fourth right-column footnote given by
% \cs{footnotetext*}\marg{text}.\label{fn:8R4}}
% 
% \switchcolumn
% Second left-column paragraph across two pages\break
% \mbox{}\Dotquad with two pre-break footnotes\Dotfill\\
% \Dotquad here\footnotemark*[-5] by \!\footnotemark!|*[-5]|
% because $m_l+n_r+m_r-1=2+2+2-1=5$ and thus
% $\ref{fn:8L3}=\ref{fn:8R4}-5$\Dotfill\\
% \Dotquad and here\footnotemark{} by \!\footnotemark!\Dotfill\\
% \Dotfill\\\Dotfill\\\Dotfill\\\Dotfill\\\Dotfill\\
% \Dotfill\\\Dotfill\\\Dotfill\\\Dotfill\\\Dotfill\\
% \Dotfill\\\Dotfill\\\Dotfill\\\Dotfill\\\Dotfill\\
% \Dotfill\\\Dotfill\\
% \Dotquad and two post-break footnotes\Dotfill\\
% \Dotquad here\footnote*[+5]{Fifth left-column footnote given by
% \cs{footnote}|*[+5]| because $n_r+m_r+1=2+2+1=5$ and thus
% $\ref{fn:8L5}=\ref{fn:8L4}+5$.\label{fn:8L5}} by
% \!\footnote!|*[+5]|\marg{text}\Dotfill\\
% \Dotquad and here\footnote{Sixth left-column foootnote given by
% \cs{footnote}\marg{text}.\label{fn:8L6}} by
% \!\footnote!\marg{text}\Dotfill\\
% followed by a \!\switchcolumn!.
% 
% \switchcolumn
% Second right-column paragraph across two pages\break
% \mbox{}\Dotquad with two pre-break footnotes\Dotfill\\
% \Dotquad here\footnotemark*[-3] by \!\footnotemark!|*[-3]|
% because $m_r+k_l-1=2+2-1=3$ and thus
% $\ref{fn:8R3}=\ref{fn:8L6}-3$\Dotfill\\
% \Dotquad and here\footnotemark{} by \!\footnotemark!\Dotfill\\
% \Dotfill\\\Dotfill\\\Dotfill\\\Dotfill\\\Dotfill\\
% \Dotfill\\\Dotfill\\\Dotfill\\\Dotfill\\\Dotfill\\
% \Dotfill\\\Dotfill\\\Dotfill\\\Dotfill\\\Dotfill\\
% \Dotfill\\\Dotfill\\
% \Dotquad and two post-break footnotes\Dotfill\\
% \Dotquad here\footnote*[+3]{Fifth right-column footnote given by
% \cs{footnote}|*[+3]| because $k_l+1=3$ and thus
% $\ref{fn:8R5}=\ref{fn:8R4}+3$.\label{fn:8R5}} by
% \!\footnote!|*[+3]|\marg{text}\Dotfill\\
% \Dotquad and here\footnote{Sixth right-column foootnote given by
% \cs{footnote}\marg{text}.\label{fn:8R6}} by
% \!\footnote!\marg{text}\Dotfill.
% \end{paracol}
% \Hrule
% 
% Note that though the remedy works well as shown above, it is not a good
% idea to do that when you are writing draft versions of your document
% because page break points go up and down by your modifications to the
% document.  Therefore, it is recommended to put all footnotes by
% non-starred \!\footnote! until your document becomes perfect except for
% footnote numbering and placement and then to adjust them by the techique
% described in this section.
% 
% \endinput

% \newpage
% \twosided
% \edef\OddSideMargin{\number\oddsidemargin}
% \advance\oddsidemargin1in
% \evensidemargin1.25\oddsidemargin \advance\evensidemargin-1in
% \oddsidemargin.75\oddsidemargin \advance\oddsidemargin-1in
% % \def\oddeven#1#2{\ifodd\value{page}#1\else#2\fi}
% \marginparpush5pt \@mparswitchtrue
% \section{Two-Sided Typesetting and Parallel-Paging\hfill 双面排版和并列分页}
% \label{sec:ppts}
% \changes{v1.3-2}{2013/09/17}
%	{Add the section ``Two-Sided Typesetting and Parallel-Paging''.}
% \changes{v1.3-4}{2013/09/17}
%	{Add the section ``Two-Sided Typesetting and Parallel-Paging''.}
% \changes{v1.3-5}{2013/09/17}
%	{Add the section ``Two-Sided Typesetting and Parallel-Paging''.}
% 
% This and the next section are typeset with \Uidx{\!\twosided!} enabling
% features |p|, |c| and |m| and also |b| for a part of the next section.
% The effect of |p| feature can be seen by the \oddeven{left}{right}, or in
% other word inside, margin of this \oddeven{odd}{even}-numbered page is
% narrower than that of the previous pages because the author reduced the
% effective \oddeven{left}{right} side margin being calculated from
% \oddeven{\cs{oddsidemargin}}{\cs{evesidemargin}}

% 这一节和下一节使用 \Uidx{\!\twosided!} 启用特性|p|、|c|和|m|,以及部分下一节的特性|b|进行排版。通过查看此\oddeven{奇数}{偶数}页的内边距,可以看到|p|特性的效果,即比前面的页面的内边距更窄,因为作者减小了从 \oddeven{\cs{oddsidemargin}}{\cs{evesidemargin}} 计算出的有效 \oddeven{left}{right} 边距。

% 
% \SpecialIndex{\oddsidemargin}
% \SpecialIndex{\evensidemargin}
% 
% by 75\,\%\footnote{
% This document itself does not have |twoside| option in its
% \!\documentclass! but the inconsistency between the option and
% \!\twosided! is not visible because \!\pagestyle! is |plain|.}.
% This setting makes the \oddeven{right}{left} side or outside margin of
% this page enlarged by 125\,\%, as well as the \oddeven{left}{right} side
% and outside margin of the next \oddeven{even}{odd}-numbered page specified
% by \oddeven{\cs{evensidemargin}}{\cs{oddsidemargin}}.

% 由于75,%的设置\footnote{%
% 此文档本身在其 \!\documentclass! 中没有 |twoside| 选项,但是选项与 \!\twosided! 之间的不一致之处并不可见,因为 \!\pagestyle! 是 |plain|。}。
% 这个设置使得本页的\oddeven{右}{左}边缘或外侧边缘增大了125,%,以及下一页的\oddeven{左}{右}边缘和外侧边缘,由 \oddeven{\cs{evensidemargin}}{\cs{oddsidemargin}} 指定的\oddeven{偶数}{奇数}页。

% Next, we see the effects of |c| and |m| features by the \env{paracol}
% environment below for which \Uidx{\!\columnratio!}|{0.6}| and
% \Uidx{\!\marginparthreshold!}|{0}| are declared to make the \emph{inside}
% columns (\oddeven{left}{right} ones in \oddeven{odd}{even}-numbered pages)
% are wider than the \emph{outside} ones and all marginal notes go to
% outside (\oddeven{right}{left} in \oddeven{odd}{even}-numbered pages)
% margins.

% 接下来,我们看到以下 \env{paracol} 环境通过 \Uidx{\!\columnratio!}|{0.6}| 和  \Uidx{\!\marginparthreshold!}|{0}| 来实现 |c| 和 |m| 特性的效果,使得\emph{内部}列(在\oddeven{奇数}{偶数}页中的\oddeven{左}{右}列)比\emph{外部}列更宽,并且所有的边注都放在外侧边缘(在\oddeven{奇数}{偶数}页中的\oddeven{右}{左}边缘)。

% \columnratio{0.6}\marginparthreshold{0}
% 
% \par\Hrule
% \begin{paracol}{2}
% \switchcolumn
% \footnotetext*{Since the author is temporarily disabling the warning from
% marginal note placement mechanism of \LaTeX, pushing down the second
% marginal note from column-1 is silently performed when you process this
% document.}
% \switchcolumn
% This line\Marginpar{First marginal note from column-0.} of the first
% paragraph of the inside column-0 has a marginal note.  Now the author puts
% a few dummy lines to keep a space below the marginal note.\\
% \Dotfill\\ \Dotfill\\ \Dotfill\\ \Dotfill\\ \Dotfill\\ \Dotfill\\
% \Dotfill\\ \Dotfill\\ \Dotfill\\ \Dotfill\\ \Dotfill\\ \Dotfill\par
% 
% This line\Marginpar{Second marginal note from column-0.} of the second
% paragraph of the inside column-0 also has a marginal note.  Now the author
% puts a few dummy lines again but this time to go down to the bottom of the
% page.\\
% \Dotfill\\ \Dotfill\\ \Dotfill\\ \Dotfill\\ \Dotfill\\ \Dotfill\\
% \Dotfill\\ \Dotfill\\ \Dotfill\\ \Dotfill\\ \Dotfill\\ \Dotfill\\
% \Dotfill\\ \Dotfill\par
% 
% This is the third paragraph of the inside column-0 having a page break in
% it.  Since shortly we will be in an \oddeven{even}{odd}-numbered page
% \pageref{page:ppts2} (now), this wider column\Marginpar{Third marginal
% note from column-0} is now \oddeven{right}{left} one keeping it
% inside, while the marginal note given in the first line of this page goes
% to \oddeven{left}{right} and outside.  Now we will have a \!\switchcolumn!
% below this paragraph to go to the column-1 and back to the previous page
% \pageref{sec:ppts}.\label{page:ppts2}
% \switchcolumn
% \it
% This is the first paragraph in the narrower, italicized and outside
% column-1.  In this paragraph, we shortly have a marginal note, italicized
% too, which goes to the outside margin shared by all marginal notes from
% both columns.\Marginpar{\it First marginal note from column-1.}  The
% marginal note given here is placed its natural position and its first line
% is aligned to the first line of the second sentence of this paragraph by
% exploitation of the space between two marginal notes from the column-0,
% though we already have had three notes from the column.
% 
% Now\Marginpar{\it Second marginal note from column-1.} the author puts
% another marginal note whose first line would be aligned to that of this
% paragraph, but it is pushed down below the second marginal note from the
% column-0 because two notes conflict with each other over the
% space\footnotemark*[+0].  Note that since the note from this column is given
% \emph{after} that from the column-0 was given, the conflict is solved
% pushing the note from this column down rather than that from the
% column-0.  Now the author puts a few dummy lines to go to the second last
% line of this page.\\
% \Dotfill\\ \Dotfill\\ \Dotfill\\ \Dotfill\\ \Dotfill\\
% \Dotfill\par
% 
% This is the third paragraph of the outside column-1, which becomes
% \oddeven{left}{right} shortly by the page break.\Marginpar{\it Third
% marginal note from column-1.}  The third marginal note is given in the
% first line of this page, but it is pushed down again due to the conflict
% with the note from the column-0.
% \end{paracol}
% \Hrule
% 
% Note that the position of the last marginal note in the \env{paracol}
% \Marginpar{Marginal note given after \env{paracol} environment is closed.}
% environment which we just have closed affects the marginal note placement
% in \postenv.  For example, the marginal note given in the first line of
% this paragraph is pushed down.

% 请注意,在我们刚刚关闭的\env{paracol}环境中,最后一个边注的位置会影响\postenv 中的边注位置。例如,给出在本段落第一行的边注会被推下去。

% \ifodd\value{page}
% We will see a few examples of \parapag{}ing shortly, but before that we
% will have an intentional black page to make the first page of the example
% odd-numbered to avoid you have an impression that its layout is
% incorrect\footnote{%
% At least the author himself had such impression without the blank page.}
% because if it were in an even page you would see the {\em outside\/} third
% and fourth supplementary {\em columns\/} at first.

% \ifodd\value{page}
% 不久我们将看到一些\parapag{}的例子,但在此之前,我们将有一个有意留白的页面,使示例的第一页成为奇数页,以避免给您一种布局错误的印象\footnote{至少在没有空白页面的情况下,作者本人也有这样的印象。},因为如果它在偶数页,您将首先看到第三和第四个辅助{\em 列}的{\em 外侧}。

% \newpage\vspace*{\fill}\centerline{(intentionally blanked page)}\vfill
% 
% \else
% From the next page, we will see a few examples of \parapag{}ing.

从下一页开始,我们将看到一些\parapag{}的例子。
% \fi
% 
% 
% 
% \newpage
% \subsection{Example of Paired Parallel-Paging}
% \label{sec:ppts-paired}
% 
% Shortly we will start a \env{paracol} environment by \beginparacol|[2]{4}|
% having four columns but two for each of left and right \paired{}
% \parapag{}es.  Since the author declares \!\columnratio!|{0.6}[0.5]|, the
% columns in left pages are made unbalanced while those in right pages are
% balanced.
% 
% \columnratio{0.6}[0.5]
% \par\Hrule
% \begin{paracol}[2]{4}
% This is the first paragraph of the leftmost column-0,
% \Marginpar{Marginal note from column-0.}
% whose first line has a marginal note placed in the right margin because
% the setting of \!\marginparthreshold! being 0 is still effective and we
% are in the odd-numbered page \pageref{sec:ppts-paired}.  Now we
% have a \!\switchcolumn! to the next column-1.
% 
% \switchcolumn
% \begin{Hfuzz}{1.1pt}\it
% This is the first paragraph of the second and right column-1 in the left
% \parapag{}e.  We shortly give an italicized mar\-gin\-al note carefully, so
% that it does not conflict with the marginal note from the column-0.
% \Marginpar{\it Marginal note from column-1.}
% That is, now the author puts the note.  Now we
% have a \!\switchcolumn! to the next column-2.
% \end{Hfuzz}
% \footnotetext*{This footnote is put in the left \parapag{}e together with
% another footnote below given in the column-2 in the right \parapag{}e.
% \label{fn:ppts-paired1}}
% 
% \switchcolumn
% \begingroup\sf
% This is the first paragraph of the column-2 being the left column of the
% right \parapag{}e.  Though we are in a page different from that column-0
% and 1 reside in, this page is still numbered \pageref{sec:ppts-paired}
% because the left and right page is \paired.  Therefore, the left margin of
% this page is narrower than the right margin because the page number is
% odd.
% 
% \footnotetext*{This footnote is \emph{not} put in the right \parapag{}e
% though it is given in the column-2 in the right \parapag{}e and thus its
% reference is in the column, of course.\label{fn:ppts-paired2}}
% 
% You have to notice
% \Marginpar{\sf Marginal note from column-2.}
% the first paragraph does not start from the page top
% but above it we have some space of exactly same size as the \preenv{}
% shown in the left \parapag{}e.  Therefore, the top of the first paragraphs
% in all columns are aligned.  The marginal note given in the first line of
% this paragraph goes to the right margin of this page because of the
% \!\marginparthreshold! setting and the parity of this page.  Now we have a
% \!\switchcolumn! to the next column-3.
% \par\endgroup
% \begin{figure*}\nosv
% \def\arraystretch{0.8}
% \centerline{\begin{tabular}[b]{|c|}\hline
%     \hbox to.9\textwidth{}\\
%     \sf page-wise figure given in column-2\\
%     \\\hline
%     \end{tabular}}
% \caption{A Page-Wise Figure}
% \end{figure*}
% 
% \switchcolumn
% \begingroup\sl
% This is the first paragraph
% \Marginpar{\sl Marginal note from column-3.}
% in the last rightmost column-3 whose width is equal to that of the column-2.
% The marginal note given in the first line goes to right and does not
% conflict with that from the column-2.  We are now going back to the
% column-0 by a {\rm\!\switchcolumn!|*|} with a \mctext.
% \endgroup
% 
% \switchcolumn*[\subsection*{A Spanning Text: though this is wider than the
% page width, this text does not span the boundary between the left and
% right parallel-pages.}]
% 
% We have come back to this column-0.  The space above the \mctext{} is due
% to the \sync{}ation because two paragraphs in the column-2 are
% significantly taller in total than the paragraphs in other columns.  As
% the spanning text itself says, it cannot extend to the right \parapag{}e.
% The author puts dummy lines to go to the page bottom.\\
% \Dotfill\\ \Dotfill\\ \Dotfill\\ \Dotfill\\ \Dotfill\\ \Dotfill\\
% \Dotfill\\ \Dotfill\\ \Dotfill\\ \Dotfill\\ \Dotfill\par
% 
% Now we will have a page break shortly.  You could be surprised by seeing
% this column is not in the left \parapag{}e after the break but in the
% right one.  This is because the feature |c| is enabled to swap not only
% columns in a page but also the left and right \paired{} \parapag{}es when
% they are even-numbered.  The other feature |p| makes the left outside
% margins of this right and the previous left pages wider than the right
% inside margins.\label{page:ppts-paired2}
% 
% \switchcolumn
% \begingroup\it
% We have restarted this column-1.  This paragraph has a
% footnote\footnotemark*[-1] as shown below.\\
% \Dotfill\\ \Dotfill\\ \Dotfill\\ \Dotfill\\ \Dotfill\\ \Dotfill\\
% \Dotfill\\ \Dotfill\\ \Dotfill\\ \Dotfill\\ \Dotfill\\ \Dotfill\\
% \Dotfill\\ \Dotfill\par
% 
% After the page break below, this column also goes to the right page
% together with the column-0
% \Marginpar{\it Another marginal note from column-1.}
% and is placed outside (left) in the page, as well as the marginal note
% in this right page but in the outside margin.
% \par\endgroup
% 
% \switchcolumn
% \begingroup\sf
% We have a few other materials not shown in right \parapag{}es.  The space
% above this paragraph is for the \mctext{} placed in the left \parapag{}e.
% The \Scfnote{} given here\footnotemark{} is also not in this page but in
% the left.  Finally, the author has put a page-wise figure spanning columns
% just before \!\switchcolumn! by which we left this column, but it will be
% in the right page \pageref{page:ppts-paired2} together with column-0 and
% 1.\\
% \Dotfill\\ \Dotfill\\ \Dotfill\\ \Dotfill\\ \Dotfill\\ \Dotfill\\
% \Dotfill\\ \Dotfill\\ \Dotfill\par
% 
% Though the footnote numbered \ref{fn:ppts-paired2} goes to the left page,
% its space and that of \ref{fn:ppts-paired1} make this and the next columns
% shorter in the previous page.  Similarly, we have a space above for the
% page-wise figure shown in the right page.
% \par\endgroup
% 
% \switchcolumn
% \begingroup\sl
% As expected, this line is aligned to the first line of the paragraph in
% the column-2 as well as those in column-0 and 1.  It is also consistent
% the first lines including that of this paragraph are not indented because
% the \mctext{} is given by {\rm\!\subsection!|*|} which makes first
% paragraphs unindented.\\
% \Dotfill\\ \Dotfill\\ \Dotfill\\ \Dotfill\\ \Dotfill\\ \Dotfill\\
% \Dotfill\\ \Dotfill\\ \Dotfill\\ \Dotfill\\ \Dotfill\par
%
% After the page break we will have shortly, this column becomes the
% leftmost in the left \parapag{}e, as you are seeing now,
% \Marginpar{\sl Another marginal note from column-3.}
% but still outermost as well as the marginal note in the outside left
% margin.
% \endgroup
% \end{paracol}
% \Hrule
% 
% Now you are seeing yet another material placed only in the page in which
% the column-0 resides and thus being the right page now, i.e., this
% paragraph and the next one in the \postenv.  You might be disappointed by
% the fact the \emph{outside} pages, i.e., left in this page
% \pageref{page:ppts-paired2} and right in the previous page
% \pageref{sec:ppts-paired}, cannot have \pwstuff{} but it is what the
% author can do now for the version 1.3 and thus you have to wait some
% future versions in which the author could devise a mechanism to exploit
% the corresponding space in the pages\footnote{
% 
% You might complain the immaturity of \parapag{}ing and might claim that it
% should be included in \Paracol{} after the author implements the
% mechanism.  In fact the author himself is frustrated current features of
% \parapag{}ing but he dared to release the version 1.3 knowing that there
% are people who happily typeset their \parapag{}ed documents with the
% current limited features.}.
% 
% In addition, you might think it is weird that the |c| feature of
% \!\twosided! swaps columns \emph{and} paired pages.  However this swapping
% is a natural consequence of the combination of \cswap{} and \paired{}
% \parapag{}ing.  Therefore, you can simply disable the |c| feature (maybe
% together with other features) to have more intuitive results.
% 
% In the next Section~\ref{sec:ppts-npaired}, you will see another kind of
% \parapag{}ing namely \npaired{} one.  Before that, we need a blank page to
% let the \npaired{} \parapag{}ing start from an even-numbered page so that
% a left and right page pair comprises a double spread.  A short remark on
% the blank next page is that it does not have a right counterpart
% \parapag{}e because the page is outside \env{paracol} environments and does
% not have any portion from the environments\footnote{
% 
% To illustrate this fact, the author dares to put a real blank page rather
% than stepping the \counter{page} counter.}.
% 
% \newpage\vspace*{\fill}\centerline{(intentionally blanked page)}\vfill
% 
% 
% 
% \newpage
% \subsection{Example of Non-Paired Parallel-Paging}
% \label{sec:ppts-npaired}
% 
% This and following three pages are to show an example of \npaired{}
% \parapag{}ing, in which the author keeps the setting of \!\twosided!,
% \!\columnratio! and \!\marginparthreshold! unchanged.
% The arguments of \beginparacol{} for column population are also unchanged
% to have $2+2$ configuration, but the first argument is followed by |*| for
% \npaired{} typesetting.  That is, the environment below starts by
% \beginparacol|[2]*{4}|.  The contents of the environment is also almost
% same as the previous Section~\ref{sec:ppts-paired}, while
% \Emph{bold-faced} words show the difference from the \paired{}
% typesetting.
% 
% \columnratio{0.6}[0.5]
% \par\Hrule
% \begin{paracol}[2]*{4}
% This is the first paragraph of the leftmost column-0,
% \Marginpar{Marginal note from column-0.}
% whose first line has a marginal note placed in the \Emph{left} margin
% because the setting of \!\marginparthreshold! being 0 is still effective
% and we are in the \Emph{even}-numbered page
% \Emph{\pageref{sec:ppts-npaired}}.  Now we have a \!\switchcolumn! to the
% next column-1.
% 
% \switchcolumn
% \begingroup\it
% This is the first paragraph of the second and right column-1 in the left
% \parapag{}e.  We shortly give an italicized mar\-gin\-al note carefully, so
% that it does not conflict with the marginal note from the column-0.
% \Marginpar{\it Marginal note from column-1.}
% That is, now the author puts the note.  Now we
% have a \!\switchcolumn! to the next column-2.
% \par\endgroup
% \footnotetext*{This footnote is put in the left \parapag{}e together with
% another footnote below given in the column-2 in the right \parapag{}e.
% \label{fn:ppts-npaired1}}
% 
% \switchcolumn
% \begingroup\sf\label{page:ppts-npaired1r}
% This is the first paragraph of the column-2 being the left column of the
% right \parapag{}e.  \Emph{Since we are in the page next to} that column-0
% and 1 reside in, this page is numbered \Emph{\pageref{page:ppts-npaired1r}}
% because the left and right page is \Emph{\npaired}.  Therefore, the left
% margin of this page is narrower than the right margin because the page
% number is odd.
% 
% \footnotetext*{This footnote is \emph{not} put in the right \parapag{}e
% though it is given in the column-2 in the right \parapag{}e and thus its
% reference is in the column, of course.\label{fn:ppts-npaired2}}
% 
% You have to notice
% \Marginpar{\sf Marginal note from column-2.}
% the first paragraph does not start from the page top
% but above it we have some space of exactly same size as the \preenv{}
% shown in the left \parapag{}e.  Therefore, the top of the first paragraphs
% in all columns are aligned.  The marginal note given in the first line of
% this paragraph goes to the right margin of this page because of the
% \!\marginparthreshold! setting and the parity of this page.  Now we have a
% \!\switchcolumn! to the next column-3.
% \par\endgroup
% \begin{figure*}\nosv
% \def\arraystretch{0.8}
% \centerline{\begin{tabular}[b]{|c|}\hline
%     \hbox to.9\textwidth{}\\
%     \sf page-wise figure given in column-2\\
%     \\\hline
%     \end{tabular}}
% \caption{A Page-Wise Figure}
% \end{figure*}
% 
% \switchcolumn
% \begingroup\sl
% This is the first paragraph
% \Marginpar{\sl Marginal note from column-3.}
% in the last rightmost column-3 whose width is equal to that of the column-2.
% The marginal note given in the first line goes to right and does not
% conflict with that from the column-2.  We are now going back to the
% column-0 by a {\rm\!\switchcolumn!|*|} with a \mctext.
% \endgroup
% 
% \switchcolumn*[\subsection*{A Spanning Text: though this is wider than the
% page width, this text does not span the boundary between the left and
% right parallel-pages.}]
% 
% We have come back to this column-0.  The space above the \mctext{} is due
% to the \sync{}ation because two paragraphs in the column-2 are
% significantly taller in total than the paragraphs in other columns.  As
% the spanning text itself says, it cannot extend to the right \parapag{}e.
% The author puts dummy lines to go to the page bottom.\\
% \Dotfill\\ \Dotfill\\ \Dotfill\\ \Dotfill\\ \Dotfill\\ \Dotfill\\
% \Dotfill\\ \Dotfill\par
% 
% Now we will have a page break shortly.  You \Emph{will not} be surprised
% by seeing this column \Emph{is still in the left \parapag{}e after the
% break.}  This is because the feature |c| is \Emph{not effective in
% \npaired{} \parapag{}ing.}  The other feature |p| \Emph{consistently makes
% the left outside margins of this and the previous page in which this
% column resides} wider than the right inside margins.
% \label{page:ppts-npaired2}
% 
% \switchcolumn
% \begingroup\it
% We have restarted this column-1.  This paragraph has a
% footnote\footnotemark*[-1] as shown below.\\
% \Dotfill\\ \Dotfill\\ \Dotfill\\ \Dotfill\\ \Dotfill\\ \Dotfill\\
% \Dotfill\\ \Dotfill\\ \Dotfill\\ \Dotfill\\ \Dotfill\par
% 
% After the page break below, this column also \Emph{stays in the left page}
% together with the column-0
% \Marginpar{\it Another marginal note from column-1.}
% and is placed \Emph{inside (right)} in the page, as well as the marginal
% note in this \Emph{left} page \Emph{still} in the outside margin.
% \par\endgroup
% 
% \switchcolumn
% \begingroup\sf
% We have a few other materials not shown in right \parapag{}es.  The space
% above this paragraph is for the \mctext{} placed in the left \parapag{}e.
% The \Scfnote{} given here\footnotemark{} is also not in this page but in
% the left.  Finally, the author has put a page-wise figure spanning columns
% just before \!\switchcolumn! by which we left this column, but it will be
% in the \Emph{left} page \Emph{\pageref{page:ppts-npaired2}} together with
% column-0 and 1.\\
% \Dotfill\\ \Dotfill\\ \Dotfill\\ \Dotfill\\ \Dotfill\\ \Dotfill\par
% 
% Though the footnote numbered \Emph{\ref{fn:ppts-npaired2}} goes to the
% left page, its space and that of \Emph{\ref{fn:ppts-npaired1}} make this
% and the next columns shorter in the previous page.  Similarly, we have a
% space above for the page-wise figure shown in the \Emph{left} page.
% \par\endgroup
% 
% \switchcolumn
% \begingroup\sl
% As expected, this line is aligned to the first line of the paragraph in
% the column-2 as well as those in column-0 and 1.  It is also consistent
% the first lines including that of this paragraph are not indented because
% the \mctext{} is given by {\rm\!\subsection!|*|} which makes first
% paragraphs unindented.\\
% \Dotfill\\ \Dotfill\\ \Dotfill\\ \Dotfill\\ \Dotfill\\ \Dotfill\\
% \Dotfill\\ \Dotfill\par
%
% After the page break we will have shortly, this column \Emph{is kept being
% the rightmost in the right \parapag{}e}, as you are seeing now,
% \Marginpar{\sl Another marginal note from column-3.}
% \Emph{and} still outermost as well as the marginal note in the outside
% \Emph{right} margin.
% \endgroup
% \end{paracol}
% \Hrule
% 
% As the \postenv{} in Section~\ref{sec:ppts-paired} is, this paragraph
% being the \postenv{} of the \npaired{} \parapag{}es appears only in the
% \parapag{}e in which the column-0 belongs to, and thus in the left
% \parapag{}e in this case.
% \endinput

% \newpage
% % \backgroundcolor{t}[rgb]{0.7,0,0}
% \backgroundcolor{b}[rgb]{0.8,0.6,0}
% \backgroundcolor{l}[rgb]{0,0,0.7}
% \backgroundcolor{r}[rgb]{0,0.7,0}
% \backgroundcolor{c[0]}[rgb]{1,0.8,1}
% \backgroundcolor{c[1]}[rgb]{1,1,0.8}
% \backgroundcolor{g}[rgb]{0.8,1,1}
% \backgroundcolor{f}[rgb]{0.8,0,1}
% \backgroundcolor{n}[rgb]{0.8,0.6,1}
% \backgroundcolor{p}[rgb]{0.8,1,0.6}
% \backgroundcolor{s}[rgb]{0.8,0.8,0.8}
% \pagerim5pt
% 
% \section{Examples of Background Painting}
% \label{sec:bgpaint}
% \subsection{Fundamental Painting}
% \label{sec:bgpaint-fund}
% \twosided[pcm]
% 
% As you undoubtedly notice, this page and a few pages following it are
% colorfully painted.  For this and the next three pages, the author
% declared the \bground{} color of each region as follows.
% 
% \begin{itemize}\item[]
% |\backgroundcolor{t}[rgb]{0.7,0,0}       % dark red for top margin|\\
% |\backgroundcolor{b}[rgb]{0.8,0.6,0}     % dark orange for bottom margin|\\
% |\backgroundcolor{l}[rgb]{0,0,0.7}       % dark blue for left margin|\\
% |\backgroundcolor{r}[rgb]{0,0.7,0}       % dark green for right margin|\\
% |\backgroundcolor{c[0]}[rgb]{1,0.8,1}    % pink for colunmn-0|\\
% |\backgroundcolor{c[1]}[rgb]{1,1,0.8}    % cream yellow for column-1|\\
% |\backgroundcolor{g}[rgb]{0.8,1,1}       % light blue for the gap|\\
% |\backgroundcolor{f}[rgb]{0.8,0,1}       % purple for page-wise floats|\\
% |\backgroundcolor{n}[rgb]{0.8,0.6,1}     |
%     |% light purple for page-wise footnotes|\\
% |\backgroundcolor{p}[rgb]{0.8,1,0.6}     |
%     |% pale green for pre/post-environment|\\
% |\backgroundcolor{s}[rgb]{0.8,0.8,0.8}   % light gray for spanning texts|
% \end{itemize}
% 
% \SpecialUsageIndex{\backgroundcolor}
% 
% Therefore, the \bground{} of this |p|re-environment paragraph and other
% stuff above is painted by pale green.
% 
% \Index{pre-environment stuff}
% 
% Since the author set \Uidx{\!\pagerim!} to be 5\,|pt|, you will see
% unpainted strips of 5\,|pt| wide at all paper edges surrounding painted
% regions.  For this and the next three pages, \Uidx{\!\twosided!}|[pcm]| is
% declared to enable |p|, |c| and |m| features but to disable the |b|
% feature.  Therefore, though this page \pageref{sec:bgpaint} is even and
% thus the left outside margin is wider than the right inside one, the
% \bground{}s of |l|(eft) and |r|(ight) margins are painted by dark blue and
% dark green respectively.
% \par\bigskip
% 
% \begin{paracol}{2}
% This column-0 is now right and inside because of the |c| feature of
% \!\twosided! is enabled.  On the other hand, the \bground{} is this column
% is painted by pink because \!\backgroundcolor! for |c[0]| specifies so.
% That is, the column ordinals optionally given to |c|(olumn) (and |g|(ap))
% regions are \emph{logical} ones not always corresponding to their
% \emph{physical} positions in a page.
% 
% \switchcolumn
% \begingroup\it
% As explained in the right column-0, the \bground{} of this left and
% outside column-1 is painted by cream yellow as
% {\rm\!\backgroundcolor!|{c[1]}|} specifies.  Now we have a
% {\rm\!\switchcolumn!|*|} with a \mctext{} to show the \bgpaint{} for
% it\footnote{
% 
% Since the footnotes in this \env{paracol} environment are \scfnote{} and
% \mgfnote{}, and \!\backgroundcolor!\texttt{\char`\{n\char`\}} specifies
% light purple, the \bground{} of this (foot)|n|(ote) region is painted by
% the color.}.
% 
% \par\endgroup
% \switchcolumn*[\subsection*{The background of this |s|(panning text)
% region is painted by light gray}\medskip]
% 
% \begin{figure*}\nosv
% \def\arraystretch{0.8}
% \centerline{\begin{tabular}[b]{|c|}\hline
%     \hbox to.9\textwidth{}\\
%     \texttt{f}(loat) region for this page-wise figure is painted by purple\\ 
%     \\\hline
%     \end{tabular}}
% \caption{A Page-Wise Figure}
% \end{figure*}
% 
% This paragraph is to show how the first line of a paragraph just below a
% \mctext{} is placed in the painted region.
% \par\vfill
% 
% \switchcolumn
% \begingroup\it
% See the right column for the reason why this paragraph is here.
% \par\vfill
% 
% See the right column for what we are now doing.
% \par\endgroup
% \switchcolumn
% 
% Now we have a \!\flushpage! to see the \bgpaint{} for a material not shown
% in the page, i.e., a page-wise float.
% \flushpage
% 
% Since we are now in an odd-numbered page \pageref{page:bgpaint2}, this
% column-0 is now a left one and is still painted by pink of course.
% \par\vfill\label{page:bgpaint2}
% 
% This paragraph is to show how the last line of a page without \Scfnote{}s
% is placed in the painted region.
% \par\newpage
% 
% This page is to show how the page without any \pwstuff{} looks like.
% \par\vfill
% 
% Shortly we will close this \env{paracol} environment in the next page.
% \par\newpage
% 
% Now we are closing this \env{paracol} environment to show how its
% \postenv{} is painted.
% 
% \switchcolumn
% \begingroup\it
% As expected, the \bground{} of this column-1 is still painted by cream
% yellow.
% \par\vfill
% 
% See the comment in the left column.
% \par\newpage
% 
% See the right column for the reason why we have this almost blank page.
% \par\vfill
% 
% See the right column for what will happen shortly.
% \par\newpage
% 
% See the left column for the reason why we are now closing the environment.
% \endgroup
% \end{paracol}
% \bigskip
% 
% The \bground{} of this paragraph in |p|(ost-environment) region is also
% painted by pale green, because \postenv{} can be \preenv{} at the same
% time as we see shortly.  \par\bigskip
% 
% \begin{paracol}{2}
% This short \env{paracol} environment illustrates how the \preenv{} of this
% environment, or the \postenv{} of the last environment in other words, is
% painted.
% 
% \switchcolumn
% \begingroup\it
% Therefore, the author does not have much to say in this column, except for
% giving a footnote here\footnote{
% 
% Since this footnote is \mgfnote{} with that in the \postenv{}, it is
% considered as a part of \postenv{} and thus painted by pale green rather
% than light purple.\label{fn:bgpaint1}}.
% \endgroup
% \end{paracol}
% \bigskip
% 
% Before moving to the next example, one caution is given for \bgpaint{} of
% \Mgfnote{}s.  As the footnote \ref{fn:bgpaint1} itself says, \Mgfnote{}s
% given in the \lpage{} of a \env{paracol} environment are considered as
% belonging to \postenv{}.  Therefore, the footnote \ref{fn:bgpaint1} is
% painted by pale green as well as another footnote given now\footnote{
% 
% Since this footnote really belongs to \postenv{}, its \bground{} is painted
% by pale green naturally.}.
% \par\label{page:bgpaint4}
% 
% 
% 
% \newpage
% \backgroundcolor{t(0pt,0pt)(0pt,-4pt)}[rgb]{0.7,0,0}
% \backgroundcolor{b(0pt,-4pt)(0pt,0pt)}[rgb]{0.8,0.6,0}
% \backgroundcolor{l(0pt,4pt)(-4pt,4pt)}[rgb]{0,0,0.7}
% \backgroundcolor{r(-4pt,4pt)(0pt,4pt)}[rgb]{0,0.7,0}
% \backgroundcolor{c[0](4pt,4pt)}[rgb]{1,0.8,1}
% \backgroundcolor{c[1](4pt,4pt)}[rgb]{1,1,0.8}
% \backgroundcolor{g(-4pt,4pt)}[rgb]{0.8,1,1}
% \backgroundcolor{f(4pt,4pt)(4pt,-4pt)}[rgb]{0.8,0,1}
% \backgroundcolor{n(4pt,-4pt)(4pt,4pt)}[rgb]{0.8,0.6,1}
% \backgroundcolor{p(4pt,4pt)}[rgb]{0.8,1,0.6}
% \backgroundcolor{s(4pt,-4pt)}[rgb]{0.8,0.8,0.8}
% 
% \subsection{Mirrored Painting and Enlarging/Shrinking/Shifting Regions}
% \label{sec:bgpaint-me}
% \twosided
% 
% At a glance, this and the next three pages look painted similarly to
% previous four pages, but by a careful examination you should notice
% two important differences.  The first one is found in the colors
% of left and right margins.  As the author enabled all features of
% \Uidx{\!\twosided!} including |b| for \mirror{}ing and we are now in an
% even-numbered page \pageref{sec:bgpaint-me}, the left and outside margin
% is painted by dark green for the region |r|(ight margin), while the right
% and inside one is painted by dark blue for |l|(eft margin).
% 
% The other is that regions are enlarged, shrunk or shifted by 4\,|pt| by
% the following \!\backgroundcolor! commands with extensions.
% 
% \begin{itemize}\item[]
% |\backgroundcolor{t(0pt,0pt)(0pt,-4pt)}[rgb]{0.7,0,0}   |
%     |% B up|\\
% |\backgroundcolor{b(0pt,-4pt)(0pt,0pt)}[rgb]{0.8,0.6,0} |
%     |% T down|\\
% |\backgroundcolor{l(0pt,4pt)(-4pt,4pt)}[rgb]{0,0,0.7}   |
%     |% R left T/B outside|\\
% |\backgroundcolor{r(-4pt,4pt)(0pt,4pt)}[rgb]{0,0.7,0}   |
%     |% L right T/B outside|\\
% |\backgroundcolor{c[0](4pt,4pt)}[rgb]{1,0.8,1}          |
%     |% all edges outside|\\
% |\backgroundcolor{c[1](4pt,4pt)}[rgb]{1,1,0.8}          |
%     |% all edges outside|\\
% |\backgroundcolor{g(-4pt,4pt)}[rgb]{0.8,1,1}            |
%     |% L/R inside & T/B outside|\\
% |\backgroundcolor{f(4pt,4pt)(4pt,-4pt)}[rgb]{0.8,0,1}   |
%     |% L/R outside & T/B up|\\
% |\backgroundcolor{n(4pt,-4pt)(4pt,4pt)}[rgb]{0.8,0.6,1} |
%     |% L/R outside & T/B down|\\
% |\backgroundcolor{p(4pt,4pt)}[rgb]{0.8,1,0.6}           |
%     |% all edges outside|\\
% |\backgroundcolor{s(4pt,-4pt)}[rgb]{0.8,0.8,0.8}        |
%     |% L/R outside & T/B inside|
% \end{itemize}
% 
% \SpecialUsageIndex{\backgroundcolor}
% 
% In the comments above, |L|(eft), |R|(ight), |T|(op) and |B|(ottom) mean
% edges moved by a given extension.  Therefore, for example,
% ``|L/R outside & T/B up|'' for |f|(loat) region means it is enlarged
% horizontally and shifted up vertically by the asymmetric extension
% |(4pt,4pt)(4pt,-4pt)|.  These a little bit complicated setting of
% extensions are to solve the problems in the fundamental example shown in
% previous four pages, namely too strict definition of the regions to be
% painted.  That is, both vertical edges of regions having texts, e.g.,
% |c|(olumn) regions, should look too close to the first and last letters.
% Similarly both horizontal edges of those regions seem too close especially
% when the first line is tall (e.g., the section title in
% p.\Tie\pageref{sec:bgpaint} and the page-wise figure in
% p.\Tie\pageref{page:bgpaint2}) and the last line of a column is followed by
% \mctext{} or \postenv.  Therefore, the author made fine tuning moving
% inside edges of margins outside, and so on.  We will come back this issue
% after exemplifying the effect of the tuning.
% \par\bigskip
% 
% \advance\skip\footins4pt\relax
% \begin{paracol}{2}
% By the tuning to enlarge this |c|(olumn) region, this paragraph has
% comfortable spaces above and below it, as well as at the both side edges.
% 
% \switchcolumn
% \begingroup\it
% This paragraph is surrounded by spaces of a small but comfortable amount as
% well.\footnote{
% 
% Shifting this (foot)|n|(ote) region down a little bit, the space below this
% footnote and above the top edge of the |b|(ottom margin) region is enlarged.}.
% 
% \par\endgroup
% \switchcolumn*[\subsection*{The background of this |s|(panning text)
% region is painted by light gray and enlarged horizontally but shrunk
% vertically}\par\medskip]
% 
% \begin{figure*}\nosv
% \def\arraystretch{0.8}
% \centerline{\begin{tabular}[b]{|c|}\hline
%     \hbox to.9\textwidth{}\\
%     shifting up this \texttt{f}(loat) region gives us a small space above
%     the top edge of the rectangle\\
%     \\\hline
%     \end{tabular}}
% \caption{A Page-Wise Figure}
% \end{figure*}
% 
% This paragraph is to show how well the first line of a paragraph just below a
% \mctext{} is separated from the boundary of two painted regions.
% \par\vfill
% 
% \switchcolumn
% \begingroup\it
% See the right column for the reason why this paragraph is here.
% \par\vfill
% 
% See the right column for what we are now doing.
% \par\endgroup
% \switchcolumn
% 
% By enlarging this |c|(olumn) region and shift the (foot)|n|(ote) region
% down, this paragraph has a comfortable amount of space below it.
% \flushpage
% 
% Similarly to other paragraphs below \pwstuff, this paragraph is well
% separated from the bottom edge of the |f|(loat) region above.
% 
% \par\vfill\label{page:bgpaint-me2}
% 
% As in the case of the line above \Scfnote{}s, the last line of this
% paragraph has a sufficient space separating it from the top edge of the
% |b|(ottom margin) region.
% \par\newpage
% 
% This page is to show how the page without any \pwstuff{} looks like.  As
% you are seeing, the space above this paragraph is sufficient and
% comfortable.
% \par\vfill
% 
% Shortly we will close this \env{paracol} environment in the next page.
% \par\newpage
% 
% Now we are closing this \env{paracol} environment to show how this
% paragraph is separated from the boundary of |c|(olumn) and
% |p|(ost-environment) regions.
% 
% \switchcolumn
% \begingroup\it
% See the comment in the left column for the intention of placing this
% paragraph here.
% \par\vfill
% 
% See the comment in the left column, too.
% \par\newpage
% 
% See the right column for the reason why we have this almost blank page.
% \par\vfill
% 
% See the right column for what will happen shortly.
% \par\newpage
% 
% See the left column for the reason why we are now closing the environment.
% \endgroup
% \end{paracol}
% \bigskip
% 
% The \bground{} of this paragraph in |p|(ost-environment) region is
% painted by pale green as done in p.\Tie\pageref{page:bgpaint4}, but its top
% and bottom edges \emph{look} shifted down and up to give spaces below and
% above the last and first paragraphs in \env{paracol} environments,
% respectively.
% \par\bigskip
% 
% \begin{paracol}{2}
% This short \env{paracol} environment illustrates how the \preenv{} of this
% environment, or the \postenv{} of the last environment in other words, is
% painted.
% 
% \switchcolumn
% \begingroup\it
% Therefore, the author does not have much to say in this column, except for
% giving a footnote here\footnote{
% 
% As the footnote \ref{fn:bgpaint1} in p.\Tie\pageref{fn:bgpaint1}, this
% \Mgfnote{} is a part of \postenv{} and thus painted by pale green rather
% than light purple.\label{fn:bgpaint-me1}}.
% \endgroup
% \end{paracol}
% \bigskip
% 
% In the setting with \!\backgroundcolor! commands in
% p.\Tie\pageref{sec:bgpaint-me}, the author carefully moved contacting edges
% of regions.  For example, to enlarge |c|(olumn) regions, the inside edges
% of |l|(eft margin) and |r|(ight margin) regions are moved outside, and both
% vertical edges of the |g|(ap) region shifted toward its inside.  As for
% the horizontal edges, the bottom edges of |t|(op margin) and |f|(loat)
% regions are moved up, the top edges of |b|(ottom margin) and
% (foot)|n|(ote) regions are moved down, and both top and bottom edges of
% the |s|(panning text) region are shifted toward its inside.
% 
% These edge shifting could make a region too narrow or too much shifted
% resulting in a material in it overreaching its boundary, especially in
% vertical shifting of horizontal edges.  However we can exploit some large
% space automatically or manually inserted above and/or below the material
% to avoid overreaching.  That is the author exploited the following spaces;
% \!\headsep! below the page head (though it is empty in this document);
% \!\dbltextfloatsep! below the bottom-most page-wise float; spaces that
% \!\subsection!|*| inserts above and below it together with manually
% inserted \!\medskip! below it; \!\skip!\!\footins!\footnote{
% 
% This is a kind of ``length command'' maybe not widely known.}
% 
% above the first footnote which the author enlarged by 4\,|pt| temporarily
% for this and the next subsections; and \!\footskip! from the bottom edge
% of text area to that of the page number.
% 
% Now you might notice that the explanation above does not mention the |p|
% region for \Preenv{} and \postenv.  As you should find in the settings,
% this region is enlarged horizontally \emph{and vertically} so that its top
% and bottom edges are moved up and down when the region is at the top or
% bottom of a page, as you are seeing now and find in
% p.\Tie\pageref{sec:bgpaint-me}.  However, this enlargement of course has a
% side effect that the region collides against |c|(olumn) and |g|(ap) regions
% also enlarged vertically making them overlapped.  This overlap will be
% invisible with most of \emph{printers} because, as shown in
% Section\Tie\ref{sec:ref-bgpaint}, |p| region is painted \emph{before} |c|
% and |g| regions are painted.  In addition, since relatively large spaces
% of \!\bigskip! are manually inserted before each \beginparacol{} and after
% each \Endparacol{}, texts in \Preenv{} and \postenv{} are well separated
% from region boundaries.
% 
% This overlay painting |c| and |g| over |p|, however, might produce an
% unexpected result with some printer with which, for example, two colors
% are \emph{blended} in the thin overlapped strip\footnote{
% 
% For example, a dvi previewer |dviout| produces such a blended result with
% the default setting of coloring.}.
% 
% Unfortunately, this overlay painting is inevitable in the current version
% 1.3, but in a future version, hopefully 1.4, more sophisticated
% \emph{position-dependent} region definition, for example, to shift the top
% edge of |p| region only when the region is at the top of page, could be
% introduced.
% 
% Another remark is that the \mirror{}ing specified by the |b| feature of
% \!\twosided! works not only on the colors of side margins but also on
% their asymmetric shrinkage.  That is, the asymmetric shifts of vertical
% edges of |l| and |r| regions correctly performed irrespective of their
% physical positions, i.e., even when the |l| (resp.\ |r|) region is at
% the right (resp.\ left) margin and the edge to be shift is the left
% (resp.\ right) one rather than right (resp.\ left).
% 
% 
% 
% \newpage \suppressfloats
% \nobackgroundcolor{t}
% \nobackgroundcolor{b}
% \nobackgroundcolor{l}
% \nobackgroundcolor{r}
% \nobackgroundcolor{g}
% \backgroundcolor{c[0](4pt,4pt)(0.5\columnsep,4pt)}[rgb]{1,0.8,1}
% \backgroundcolor{c[1](0.5\columnsep,4pt)(4pt,4pt)}[rgb]{1,1,0.8}
% \backgroundcolor{C[0](10000pt,10000pt)(0.5\columnsep,10000pt)}[rgb]{1,0.8,1}
% \backgroundcolor{C[1](0.5\columnsep,10000pt)(10000pt,10000pt)}[rgb]{1,1,0.8}
% 
% \subsection{Regions with Infinite Extensions}
% \label{sec:bgpaint-inf}
% 
% You are now seeing another \bgpaint{} much different from previous two
% examples.  That is, after disabling painting of |t|, |b|, |l|, |r| and |g|
% regions by \Uidx{\!\nobackgroundcolor!}, the author gave the followings
% for painting this and the next pages.
% 
% \begin{itemize}\item[]
% |\backgroundcolor|
%     |{c[0](4pt,4pt)(0.5\columnsep,4pt)}[rgb]{1,0.8,1}|\\
% |\backgroundcolor|
%     |{c[1](0.5\columnsep,4pt)(4pt,4pt)}[rgb]{1,1,0.8}|\\
% |\backgroundcolor|
%     |{C[0](10000pt,10000pt)(0.5\columnsep,10000pt)}[rgb]{1,0.8,1}|\\
% |\backgroundcolor|
%     |{C[1](0.5\columnsep,10000pt)(10000pt,10000pt)}[rgb]{1,1,0.8}|
% \end{itemize}
% 
% \SpecialUsageIndex{\backgroundcolor}
% 
% The first two lines above is different from the previous declaration
% because inside edges of |c[0]| and |c[1]| regions are shifted toward
% outside of them and thus inside of unpainted |g| region so that the edges
% are contacted.  On the other hand, the last two lines are for
% \emph{under-painting} of columns and has \emph{\bginfext} to make top,
% bottom and outside edges of |C| regions reaching to the corresponding
% paper edges.  Since this under-painting is done with colors same as those
% of over-painting of |c| regions, you will have an impression that the
% paper is two-toned and \pwstuff{} are pasted on the paper\footnote{
% 
% This footnote is given outside \env{paracol} environment but its
% \bground{} is painted by light purple because it is merged with the
% footnote \ref{fn:bgpaint-inf2}.\label{fn:bgpaint-inf1}}.
% 
% \par\bigskip
% 
% \begin{figure}\nosv
% \def\arraystretch{0.8}
% \centerline{\begin{tabular}[b]{|c|}\hline
%     \hbox to.9\textwidth{}\\
%     \parbox{.8\textwidth}{
% 	This \texttt{f}(loat) region could be extended to both side edges
%	and the top edge of the paper if its extension were
%	\texttt{(10000pt,10000pt)(10000pt,-4pt)}.}\\
%     \\\hline
%     \end{tabular}}
% \caption{A Page-Wise Figure \emph{Imported} from Pre-Environment}
% \label{fig:bgpaint-inf}
% \end{figure}
% 
% \begin{paracol}{2}
% Though you cannot see, the right edge of this over-painted |c[0]| region
% is shifted right by 4\,|pt| to hide the small patch at the right bottom
% corner of the |p| region above by overlaying.
% 
% \switchcolumn
% \begingroup\it
% As explained in the right column, this {\rm|c[1]|} region also has an
% invisible left edge shifted left by {\rm4\,|pt|}\footnote{
% 
% This (foot)|n|(ote) region could be extended to both side edges and the
% bottom edge of the paper if its extension were
% \texttt{(10000pt,-4pt)(10000pt,10000pt)}.\label{fn:bgpaint-inf2}}.
% \endgroup
% 
% \switchcolumn*[\subsection*{This \texttt{s}(panning text) region could be
% extended to both side edges of the paper if its extension were
% \texttt{(10000pt,-4pt)}.}\par\medskip]
% 
% The author does not have much to say now for this column chunk.
% \par\vfill
% 
% Still nothing to say particular to the page break we will have shortly.
% \par\newpage
% 
% This paragraph is just for keeping the \env{paracol} environment alive in
% this page.
% \switchcolumn
% 
% \begingroup\it
% Little to say as well.
% \par\vfill
% 
% Nothing to say as well.
% \par\newpage
% 
% This paragraph is not necessary for keeping alive the environment but is
% given for consistent view.
% \endgroup
% 
% \begin{figure*}\nosv
% \def\arraystretch{0.8}
% \centerline{\begin{tabular}[b]{|c|}\hline
%     \hbox to.9\textwidth{}\\
%     \parbox{.8\textwidth}{
% 	This figure is given in the \env{paracol} environment closed in the
%	previous page but its background is not painted.}\\
%     \\\hline
%     \end{tabular}}
% \caption{A Page-Wise Figure \emph{Exported} to Post-Environment}
% \label{fig:bgpaint-inf2}
% \end{figure*}
% \end{paracol}
% \bigskip
% 
% Note that overlay painting is inevitable for two-toned page painting, as
% far as you want to paint \bground{} of \pwstuff.
% 
% The last issue of \bgpaint{} is about painting materials given outside
% \env{paracol}.  As you have seen, \Preenv{} and \postenv{} are painted but
% it is done only when they reside in a page having a portion of a
% \env{paracol} environment (maybe) of course.  Therefore, the next page is
% \emph{not} painted because the page does not have any parallel-columned
% stuff.  Therefore, even if you wish to paint the whole of your document
% including pages without \env{paracol} stuff, you cannot do it just with
% \Paracol{} package, at least so far.
% 
% On the other hand, some materials given outside \env{paracol} environments
% are painted as if they are given in the environment when they are
% \emph{imported} into the environment.  One category has footnotes given in
% \preenv{} when \!\footnotelayout!|{m}| is specified for merging, as
% exemplified by the footnote \ref{fn:bgpaint-inf1} in the previous page.
% Note that such a footnote is painted by the color for |n| region rather
% than |p| region even when there are no footnotes in the \env{paracol}
% environment.  The other category has ordinary floats given by \env{figure}
% and/or \env{table}
% (i.e., neither \env{figure*} nor \env{table*}) environments outside
% \env{paracol} and then \emph{deferred} to a page having (a portion of)
% stuff produced by \env{paracol}.  Since such a float, e.g.,
% Figure\Tie\ref{fig:bgpaint-inf} in this page, is considered as a page-wise
% float given in the \env{paracol} environment in this section, its
% background is painted by the color for the |f| region, rather than that
% for the |p| region which would be used if the float were is placed in the
% previous page.  Note that such a deferred float import could occur not
% only from the page having \beginparacol{} but also from pages preceding
% it.  For example, if you have three \env{figure} environments in a page
% $p-1$ just preceding the page $p$ in which you start a \env{paracol}
% environment, it could happen that first one is placed in $p-1$ without
% painting, the second is placed in $p$ and painted by the color for |p|,
% and the third is placed in $p+1$ and painted by the color for |f|.
% 
% Finally some materials \emph{exported} from a \env{paracol} environment
% are painted as if they are in \postenv.  In previous two subsections, we
% saw \Mgfnote{}s (e.g., \ref{fn:bgpaint1} in p.\Tie\pageref{fn:bgpaint1}
% and \ref{fn:bgpaint-me1} in p.\Tie\pageref{fn:bgpaint-me1}) are painted by
% the color of |p| rather than |n|.  The other kind of exportation is of
% page-wise floats given in a \env{paracol} environment but deferred to the
% page next to the page having \Endparacol, or further.  For example,
% Figure~\ref{fig:bgpaint-inf2} is given in the \env{paracol} environment
% above in this page, but its \bground{} is not painted because the next page
% in which the figure is placed does not have any parallel-columned
% stuff\footnote{
% 
% If it has, the background is painted by the color for |p|.}.
% 
% \newpage\vspace*{\fill}
% \centerline{(intentionally blanked page to show this page is \emph{not}
% painted)}
% \vfill
% \advance\skip\footins-4pt\relax
% \endinput
 
% \newpage
% \twosided[]\oddsidemargin\OddSideMargin sp
% \resetbackgroundcolor \columnratio{}
% % \section{Known and Unknown Problems\hfill 已知和未知的问题}
% \label{sec:problem}
% \changes{v1.2-7}{2013/05/11}
% 	{Add the section ``Known and Unknown Problems'' to summarize a few
% 	 typesetting issues and warn users of the possiblity of bugs.}
% \changes{v1.3-1}{2013/09/17}
%	{Remove the problem description of the placement of page-crossing
%	 spanning texts because it has been solved.}
% \changes{v1.3-3}{2013/09/17}
%	{Remove the problem description of the lack of column-separating rule
%	 drawing because it has been implmented.}
% 
% Here a few problems you could face in the use of \textsf{paracol} are
% summarized.

% 在使用 \textsf{paracol} 时可能遇到的一些问题总结如下。
% \begin{itemize}
% \item
% If your (e.g.,) left column goes ahead too much farther than the right
% column, \LaTeX{} could stop with the following error message.

% 如果你的(例如)左列比右列前进得更远,\LaTeX{}可能会停止,并显示以下错误消息。
% \begin{quote}
% |! Package paracol Error: Too many unprocessed columns/floats.|
% \end{quote}

% This usually means that the internal space to keep materials in the left
% column is exhausted.  More specifically, suppose at some point in your
% |.tex| the left column is in the page $p$ while the right is in $q<p$.
% We need $(p-q)$ \emph{boxes} to keep the left column contents in the pages
% $q$, $q+1$, \ldots, $p-1$ because these pages cannot be \emph{printed} yet
% until the right column fills them.  In addition, we also need two boxes
% for the left column in $p$ and the right column in $q$ so that you make
% \cswitch{} between them keeping unprinted contents in them.  Therefore, at
% least we need to have $(p-q)+2$ boxes, while the number of them provided
% by \LaTeX{} is only 18\footnote{%
% Readers who are acquainted with \LaTeX{} implementation will understand
% that 18 is the cardinality of the set $\{\cs{bx@A},\ldots,\cs{bx@R}\}$ for
% floats acquired by \cs{newinsert}.  Those who are more familiar with that
% might know that most \LaTeX{}, based on e-\TeX{} or others having similar
% extensions, now have 52 \cs{insert}s
% $\{\cs{bx@A},\ldots,\cs{bx@Z},\cs{bx@AA},\ldots,\cs{bx@ZZ}\}$ for floats
% and materials of \Paracol, since 2015}.

这通常意味着左列中保存材料的内部空间已经用尽。更具体地说,假设在您的|.tex|文件的某个点,左列位于页面$p$,而右列位于$p$之前的页面$q$中。我们需要$(p-q)$个\emph{盒子},以将左列的内容保存在页面$q$,$q+1$,……,$p-1$中,因为在右列填充它们之前,这些页面不能被\emph{打印}出来。此外,我们还需要两个盒子,分别用于页面$p$中的左列和页面$q$中的右列,以便您可以在它们之间进行\cswitch{},保持其中的未打印内容。因此,至少我们需要有$(p-q)+2$个盒子,而\LaTeX{}提供的盒子数量只有18个\footnote{熟悉\LaTeX{}实现的读者会明白,18是由\cs{newinsert}获取的浮动体集合${\cs{bx@A},\ldots,\cs{bx@R}}$的基数。那些更熟悉的人可能会知道,大多数基于e-\TeX{}或其他具有类似扩展的\LaTeX{}版本,自2015年以来都有52个\cs{insert},分别用于浮动体和\Paracol{}的材料${\cs{bx@A},\ldots,\cs{bx@Z},\cs{bx@AA},\ldots,\cs{bx@ZZ}}$。}。

% Therefore, \textsf{paracol} cannot continue its work if $(p-q)$ reaches
% 17.  Furthermore, other stuff also consumes the boxes as follows.

因此,如果$(p-q)$达到17,\textsf{paracol}将无法继续工作。此外,其他内容也会按照以下方式消耗盒子。
% \begin{itemize}
% \item
% If there are $n$ pages in $q$, $q+1$, \ldots, $p$ having \preenv{} or
% page-wise floats, $n$ boxes are consumed by them.  Similarly, if $m$ pages
% in them have \Scfnote{}s, $m$ boxes are given to them.
% 
如果在$q$,$q+1$,\ldots,$p$中有$n$页具有\preenv{}或按页的浮动体,那么它们会消耗$n$个盒子。同样,如果其中$m$页具有\Scfnote{},那么它们会获得$m$个盒子。
% \item
% If the left (resp.\ right) column has \Mcfnote{}s in $p$ (resp.\ $q$), a box
% is used for them.
% 
如果左(右)栏在$p$($q$)中有\Mcfnote{},则会为它们使用一个盒子。
% \item
% If the left (resp.\ right) column has $k$ floats to be placed in $p$
% (resp.\ $q$) or to be deferred to $p+1$ (resp.\ $q+1$) or a succeeding page,
% $k$ boxes are reserved for them.

如果左(右)栏在$p$($q$)中有$k$个浮动体需要放置,或者延迟到$p+1$($q+1$)或后续页面,则会为它们保留$k$个盒子。
% \end{itemize}

% Therefore, it should be safe to keep $(p-q)$ from exceeding 10 or so
% placing \!\switchcolumn! in both columns fairly frequently.

因此,在两个列中频繁地使用 \!\switchcolumn!,将$(p-q)$保持在不超过10左右应该是安全的。
% \item
% As discussed in Section~\ref{sec:ref-switchcolumn}, setting a \sync{}ation
% point in a page brings the following side effects.

如第~\ref{sec:ref-switchcolumn}节所讨论的,将同步点设置在页面中会产生以下副作用。
% \begin{itemize}
% \item
% Stretch and shrink factors of all vertical skips in the page are
% nullified.  The nullification of stretch factors could make a sparse
% column in the page have a vertical space at its bottom as if
% \!\raggedbottom! setting is in effect even with \!\flushbottom! one,
% rather than distributing the amount of the space to the skips so that the
% bottom line is aligned at the page bottom.  As for the nullification of
% shrink factors, it makes the page have lines a little bit less than that
% it would have without \sync{}ation because lines above the (last)
% \sync{}ation point cannot be compressed.  The other effect is a little bit
% subtle because the shrink factors below the last \sync{}ation point are
% taken care of by \TeX{}'s page builder when it examine the appropriateness
% of each breakable point, but they are nullified when the page is printed.
% That is, if \TeX{} finds a good break point which needs that the stuff
% between the \sync{}ation and break points is compressed a little bit, the
% stuff is printed without compression making its bottom edge a little
% bit below the page bottom.
% 
页面中所有垂直间距的伸缩因子被设为零。伸缩因子的设为零可能会导致页面中的稀疏列在底部具有垂直间距,就好像使用了 \raggedbottom 设置一样,即使实际使用的是 \flushbottom 设置,而不是将空间的量分布到间距中,使得底线与页面底部对齐。至于收缩因子的设为零,这使得页面的行数比没有同步化时少一点,因为同步化点上方的行不能被压缩。另一个效果稍微微妙一些,因为当 \TeX{} 检查可断点的合适性时,位于最后一个同步化点以下的收缩因子由 \TeX{} 的页面构建器处理,但在打印页面时,它们被设为零。也就是说,如果 \TeX{} 找到一个需要将同步化点和断点之间的内容稍微压缩一点的良好断点,那么该内容将以无压缩方式打印,使得其底边略微低于页面底部。
% \item
% After a \sync{}ation point is set, columns in the page cannot have top
% floats any more even if a column has space above the \sync{}ation point
% and large enough to place the float.  Therefore, if you like to exploit
% the space, you have to place the \env{figure} or \env{table} environment
% in question prior to the \cswitch{} command or environment for the
% \sync{}ation.

在设置了同步化点之后,即使某一列在同步化点上方有足够的空间放置浮动体,该列也无法再放置顶部浮动体。因此,如果想要利用这个空间,必须在进行同步化之前将相关的 \env{figure} 或 \env{table} 环境放置在 \cswitch{} 命令或环境之前。
% \end{itemize}
% 
% \item
% \changes{v1.3-6}{2013/09/17}
%	{Add comments about usage of \cs{paragraph} etc.\ in spanning texts.}
% \begingroup\parskip\z@
% As the author did for Section\Tie1 to \ref{sec:float}, sometimes you will
% make a section header spanning all columns by giving a sectioning command
% such as \!\section!, \!\subsection! and \!\subsubsection! to the optional
% argument of \!\switchcolumn!|*| or \!\begin! of a \sync{}ing \csenv.
% These three commands work well and you will have what you intend to have,
% but you have to be careful with lower-level commands \!\paragraph! and
% \!\subparagraph!.  Unlike higher-level relatives, these lower-level
% commands does \emph{not} put the header \emph{immediately} but keep it
% somewhere\footnote{%
% For people familiar to \TeX's \emph{dangerous bends}, the header is kept in
% \!\everypar!.}

% 就像作者在第\ref{sec:float}节中所做的那样,有时你会通过将诸如 \!\section!、\!\subsection! 和 \!\subsubsection! 之类的节标题命令放在 \!\switchcolumn!|*| 或 \!\begin! 的可选参数中,来使一个节标题跨越所有列。这三个命令可以很好地工作,你会得到你想要的效果,但是你必须小心使用低级命令 \!\paragraph! 和 \!\subparagraph!。与高级命令不同,这些低级命令并\emph{不会}立即放置标题,而是将其保存在某个地方\footnote{%
% 对于熟悉\TeX 的\emph{危险弯曲}的人来说,标题保存在 \!\everypar! 中。}。

% so that when the paragraph following the command starts it is put as the
% leading part of the paragraph.  Therefore if the \mctext{} has (e.g.)
% \!\paragraph! only, the header is not put as a \mctext{} but at the head of
% the first paragraph of the column to which you switch, leaving an
% empty \mctext{} with some large space as follows.

这样,当命令后面的段落开始时,它就会作为段落的开头部分放置。因此,如果\mctext{}只有(例如)\!\paragraph!,则标题不会作为\mctext{}放置,而是放置在您切换到的列的第一个段落的开头,留下一个空的\mctext{},其中包含一些大空间,如下所示。

% \par\leavevmode\Hrule
% \columnsep\z@
% \begin{paracol}{2}
% This left-column paragraph precedes a \sync{}ed \cswitch.
% \switchcolumn
% This right-column paragraph precedes a \sync{}ed \cswitch.
% \switchcolumn*[\paragraph{A Spanning Text Given by \cs{paragraph}}]
% This left-column paragraph follows the \sync{}ation but is led by
% \!\paragraph! given to the optional argument of \!\switchcolumn!|*| for
% \mctext.
% \switchcolumn
% This right-column paragraph follows the \sync{}ation with an empty \mctext.
% \end{paracol}
% \par\leavevmode\Hrule\par

% Therefore, unless this is what you intend to do, you have to give some
% paragraph together with \!\paragraph! to the optional argument for
% \mctext.  For example, \!\mbox!|{}| is a good candidate as the paragraph
% following \!\paragraph! because it produces (almost) nothing.  By using
% this technique the example above becomes the followings.

因此,除非这是您打算做的事情,否则您必须将一些段落与 \!\paragraph! 一起提供给\mctext 的可选参数。例如,\!\mbox!|{}| 是作为 \!\paragraph! 之后的段落的一个好选择,因为它几乎不产生任何内容。通过使用这种技术,上面的示例将变成以下内容。
% \par\leavevmode\Hrule
% \begin{paracol}{2}
% This left-column paragraph precedes a \sync{}ed \cswitch.
% \switchcolumn
% This right-column paragraph precedes a \sync{}ed \cswitch.
% \switchcolumn*[\paragraph{A Spanning Text Given by \cs{paragraph} Followed
% by \cs{mbox}\texttt{\char`\{\char`\}}}\mbox{}]
% This left-column paragraph follows the \mctext{} above.
% \switchcolumn
% This right-column paragraph follows the \mctext{} above.
% \end{paracol}
% \par\leavevmode\Hrule
% \endgroup
% 
% \item
% As shown in Section~\ref{sec:fnnp}, it is not easy to have good numbering
% and stacking order of \Scfnote{}s even with the supports from
% \!\footnote!|*| and its relatives.  In addition, a footnote in a
% \env{paracol} environment cannot be broken into two (or more) pages.
% 
正如第~\ref{sec:fnnp}节所示,即使使用 \!\footnote!|*| 及其相关命令的支持,也很难获得良好的\Scfnote{}编号和堆叠顺序。此外,在\env{paracol}环境中的脚注不能分为两页(或更多页)。
% \item
% \changes{v1.3-2}{2013/09/17}
%	{Add comments about the limitaion of parallel-paging.}

% As the author confessed in Section\Tie\ref{sec:ppts-paired}, right
% \parapag{}es cannot have \pwstuff{} but have blank spaces in the
% corresponding region for them.  The author will try to remove this
% limitation from a future version of \Paracol, in the version 1.4 hopefully.
%
正如作者在第\ref{sec:ppts-paired}节中承认的那样,右侧的\parapag{}不能有\pwstuff{},但在相应的区域中有空白。作者将努力在未来的\Paracol{}版本中消除这个限制,希望是1.4版本。
% \item
% \changes{v1.3-3}{2013/09/17}
%	{Add comments about the imperfectnss of extention of backgrond
%	 painting regions.}
% \begin{Hfuzz}{1.5pt}
% As discussed in Section\Tie\ref{sec:bgpaint-me}, it is desirable that
% \bgpaint{} region definition in \!\backgroundcolor! has position dependent
% extensions.  The author is fairly optimistic about the incorporation of
% this advanced feature in the version 1.4.
% \end{Hfuzz}
% 
% \item
% \changes{v1.32-3}{2015/10/10}
%	{Add comments about the out-of-order appearance of page-wise floats
%	 even with \protect\LaTeX-2015/01/10 or later.}
% In the release dated 2015/01/10, \LaTeX{} changed its mechanism of the
% placement of double-column floats (or in our terminology, \pwise{}
% floats) to avoid {\em out-of-order\/} appearance of them.  That is, until
% the release on 2014/05/01 a double-column float (e.g., \env{figure*}) can
% be overtaken by a single-column float of the same category (e.g.,
% \env{figure}) when they cannot be put into the page in which texts around
% them are put.  In order to cope with the problem, the new version merged
% two lists to keep {\em deferred\/} double- and single-column floats into
% one so that the appearance order of them is determined by their order in
% the single list.  Though this change should have made people happy when
% they typeset {\em ordinary\/} two-column (or multiple-column) documents,
% the new feature might not be welcomed by \Paracol{} users because your
% parallel-columns have their own {\em streams\/} of floats to be put in the
% corresponding columns.  Therefore, and for the sake of simplicity of
% \Paracol's implementation, the author decided to nullify this new feature
% in \env{paracol} environments.  That is, even with new releases of
% \LaTeX{}, your \pwise{} floats given in a \env{paracol} environment can be
% overtaken by \cwise{} floats.
% \end{itemize}
% 
% In addition to the problems above known to the author, there may be (or
% should be, honestly speaking) other unknown problems in \textsf{paracol}
% because it cannot be perfect though the author has made his best effort
% for testing and debugging it.  Particularly, sometimes it is very tough,
% if not impossible, to make \textsf{paracol} compatible with other
% packages, especially with those having dark magic as \textsf{paracol} has
% in it\footnote{
% 
% For example, the author knows it is almost impossible to make
% \textsf{paracol} compatible with one of the author's own package available
% in CTAN.}.
% 
% Therefore, though reporting incompatibleness with a package you use is
% very welcome\footnote{
% 
% For example, \textsf{paracol} is now compatible with \textsf{color}
% package thanks to a report from a user.},
% 
% you should kindly understand the toughness of the compatibility issue.
% 
% Furthermore, even without such problematic packages, \textsf{paracol}
% might produce weird results due to its bug.  If your document has
% something to make unknown bugs visible, you might have one (or more) of
% the followings which the author encountered in his debugging work.
% 
% \begin{itemize}
% \item
% A page, a column, a footnote and/or a float disappears\footnote{
% 
% In fact, a bug fixed in version 1.2 caused page losing though it happens
% very very rarely but an unlucky user encountered it.}.
% 
% \item
% A page, a column, a footnote and/or a float is duplicated.
% 
% \item
% A message like ``{\tt Overfull |\vbox| (1.23456pt too high) has occurred
% while |\ouptut| is active}'' is shown.
% 
% \item
% A message ``{\tt Underfull |\vbox| (badness 10000) has occurred while
% |\ouptut| is active}'' is shown.  This message, however, does not always
% mean a bug but may just be a complaint that a column or a page is too
% sparse to meet your request to align the bottom of all columns and pages
% by \!\flushbottom! setting.  Therefore, if you have this message and you
% cannot be sure whether it means a bug or not, try \!\raggedbottom! setting
% to see if you still have the message, before sending a bug report to the
% author.
% \end{itemize}
% 
% If you encounter anything like them (or whatever you cannot solve by
% yourself), don't hesitate to report it to the author with minimum source
% file to produce the problem\footnote{
% 
% And with patience because your problem might not be solved quickly.}.
% 
% \endinput

% 
% \IndexPrologue{\newpage\section*{Index}
% Underlined number refers to the page where the specification of
% corresponding entry is described, while italicized number is for the page
% in which the usage of the entry is explained.}
% 
% \StopEventually{\ifx\ONLYDESCRIPTION\undefined\newpage\fi
% \section*{Acknowledgments\hfill 致谢}
% \addcontentsline{toc}{part}{\protect\numberline{}{Acknowledgments}}

% The author thanks to Yacine Daddi Addoun who gave the author the
% motivation to write the style for his bilingual document.  He also thanks
% to the following people;

作者感谢Yacine Daddi Addoun给予作者编写双语文档样式的动力。他还感谢以下的人;

% Robin Fairbairns who kindly invited the style to CTAN after the author's
% lazy six years failing to upload the style;

Robin Fairbairns亲切地邀请了作者将这个样式上传到CTAN,这是在作者懒散六年、未能上传该样式之后的事情。

% Joseph
% G.\ Rosenstein and Dieter K\"ohler who suggested the author adding the
% function of unbalanced column width incorporated in version 1.1;

Joseph G.\ Rosenstein和Dieter K"ohler建议作者在1.1版本中添加了不平衡列宽的功能;

% Joaqu\'in Blas who motivated the author to challenge page-wise footnotes;

Joaqu'in Blas激励了作者挑战按页脚注的能力;

% Olivier Vogel who pointed out the compatibility problem with coloring
% packages;

Olivier Vogel指出了与着色宏包的兼容性问题;

% Heiner Richter who asked for the possibility of swapping unbalanced
% columns, revealed two bugs in version 1.22 related to coloring and
% float pages, showed the necessity of \!\coloredwordhyphenated!, and finally
% found the necessity of \!\globalcounter!|*|;

Heiner Richter提出了交换不平衡列的可能性,并在1.22版本中发现了与着色和浮动页面相关的两个错误,展示了 \!\coloredwordhyphenated! 的必要性,并最终发现了 \!\globalcounter!|*| 的必要性。

% an anonymous user who reported a very rare-case but severe bug in the
% version~1.1 by which a page can be lost (whoops!);

一个匿名用户在1.1版本中报告了一个非常罕见但严重的错误,导致页面丢失(哎呀!)。


% Olivier Gerard who found another terrible bug fixed in version 1.21 but
% hidden in \textsf{paracol} for two years by which a column disappears or
% moves to a wrong page (another whoops!\@), suggested to implement
% \!\setcolumnwidth!, \!\marginparthreshold! and \!\thecolumn!  introduced
% in version 1.3, and kindly proofread this manual;

Olivier Gerard发现了另一个可怕的错误,在1.21版本中得到修复,但在\textsf{paracol}中隐藏了两年,导致列消失或移动到错误的页面(另一个哎呀!),他建议实现 \!\setcolumnwidth!、\!\marginparthreshold! 和 \!\thecolumn!,这些功能在1.3版本中引入,并且还对本手册进行了校对。

% George Kamel who let the author know the coloring function newborn in
% version 1.2 had a bug fixed in version 1.22 to which he also made a great
% contribution testing many tentative versions with his own colored
% documents;

George Kamel让作者知道在1.2版本中新出现的着色功能存在一个错误,在1.22版本中得到修复,他还用自己的着色文档测试了许多尝试性版本,对此做出了巨大的贡献。

% another anonymous user who pointed out version 1.22 had yet another
% coloring bug fixed in version 1.24;

另一个匿名用户指出1.22版本中还有另一个着色错误,在1.24版本中得到修复。

% Jean Druel who motivated the author to implement an advanced functionality
% parallel-paging;

Jean Druel激励作者实现了高级功能并行分页。

% Tilo Arens and other patient users who had wished \Paracol{} would have the
% capability of rule drawing in the gaps separating columns and painting
% backgrounds of columns and so on;

 Tilo Arens和其他耐心的用户希望\Paracol{}能够具有在分隔列之间绘制规则和绘制列背景等功能。

% Michael Bolin who gave the author motivated examples showing the
% necessity of \!\ensurevspace!.

Michael Bolin给出了作者有动机的例子,显示了 \!\ensurevspace! 的必要性。

% Tigran Aivazian who reported a memory leak problem fixed in version 1.32;

Tigran Aivazian报告了一个在1.32版本中修复的内存泄漏问题。

% Marcus Zelezny and Touhami Mamouni who found an incompatibility with
% \LaTeX{} itself (2015/01/10 or later) and enlighten the author on the cause
% of the problem;

Marcus Zelezny和Touhami Mamouni发现了与\LaTeX{}本身(2015/01/10或之后的版本)的不兼容性,并向作者解释了问题的原因。


% Manuel Kuehner who reported a bug in text coloring which had hidden
% for five years until the version 1.34 was released;

Manuel Kuehner报告了一个文本着色的错误,在1.34版本发布之前隐藏了五年。

% ZongXian Wang who found that the paracol misbehaves when an environment
% starts with an unusually tall item;

ZongXian Wang发现当一个环境以一个异常高的项目开始时,\Paracol{}的行为不正常。

% and Frank Mittelbach who pointed out bugs in \cs{marginpar} implementation
% and vertical spacing with \cs{trivlist}-like environments, and suggested
% new functionality with \cs{marginnote}, \cs{belowfootnoteskip} and
% \cs{definecolumnpreamble}.
% 
感谢 Frank Mittelbach 指出了 \cs{marginpar} 实现中的错误,以及与 \cs{trivlist}-like 环境的垂直间距问题,并提出了关于 \cs{marginnote}、\cs{belowfootnoteskip} 和 \cs{definecolumnpreamble} 的新功能建议。

% For the implementation of the style file, the author referred to the base
% implementations of \cs{output} and othe many macros of \LaTeXe{} written
% by Leslie Lamport, Johannes Braams and other authors.  The author also
% referred to \textsf{color} written by David Carlisle and
% \textsf{marginnote} written by Markus Kohm to make the package working
% well with them.

在实现样式文件时,作者参考了由 Leslie Lamport、Johannes Braams 和其他作者编写的 \LaTeXe{} 的基本实现中的 \cs{output} 和其他许多宏。作者还参考了 David Carlisle 编写的 \textsf{color} 和 Markus Kohm 编写的 \textsf{marginnote},以使该包能够与它们很好地配合使用。

% \ifx\ONLYDESCRIPTION\undefined\else
% \newpage\label{page:toc} \tableofcontents
% \fi
% 
% \PrintIndex}
% 
% \newpage
% \addtocounter{page}{2}
% \let\Midx\MidxSave
% \advance\oddsidemargin1in\evensidemargin\oddsidemargin
% \advance\textwidth-1in\columnwidth\textwidth
% \hsize\textwidth \linewidth\textwidth
% \part{Implementation}\label{part:impl}
% % \section{Overview}
% \label{sec:imp-ovv}
% 
% \subsection{Column-Pages}
% \label{sec:imp-ovv-colpage}
% 
% In our parallel multi-column typesetting, a column may grow independently
% of other columns and may cross its page boundary asynchronously with
% others.  Therefore, we cannot throw away the contents of a column in a
% page, or a {\em\Uidx\colpage} in short, when a page break occurs in the
% column.  Instead, we have to keep \colpage{}s until all columns are
% {\em\Uidx\sync{}ed} implicitly or explicitly.
% 
% An {\em\Uidx\imsync} takes place when all columns in a page see
% page-breaks to let the page is shipped out.  In general, all columns but
% the last one which arrives the page-break have completed \colpage{}s in the
% page in question and some of them may have succeeding \colpage{}s.
% Therefore, we maintain the list of completed \colpage{}s
% $\Uidx\S_c=|\pcol@shipped|{\cdot}c$
% 
% \SpecialArrayMainIndex{c}{\pcol@shipped}
% 
% for each column $c\In0\C$, where $\Uidx\C=\!\pcol@ncol!$ is the number of
% columns given through the argument of \env{paracol} environment, and the
% set of them $\Uidx\SS=\Set{\S_c}{c\In0\C}$.
% 
% Each element $\Uidx\s_c(p)$ of a list $\S_c$ is an \!\insert! whose
% \!\vbox! contains the 
% $p$-th completed \colpage\footnote{
% 
% Other registers such as \cs{count} are not used.},
% 
% where $p=0$ for the first \colpage{} produced in \env{paracol}
% environment or that following a page flushing macro \!\flushpage!,
% \!\clearpage! or \!\cleardoublepage!.  That is, $\S_c$ is defined as
% follows, where $\Uidx\pbase=\!\pcol@basepage!$ is the zero-origin ordinal
% of the {\em\Uidx\bpage} being the oldest page not shipped out yet.
% 
% \begin{eqnarray*}
% \S_c&=&(\s_c(\pbase),\s_c(\pbase{+}1),\ldots,\s_c(\pbase{+}k{-}1))\\
% &=&\!\@elt!\,\s_c(\pbase)\;\!\@elt!\,\s_c(\pbase{+}1)\;\cdots\;
% 	\!\@elt!\,\s_c(\pbase{+}k{-}1)
% \end{eqnarray*}
% 
% Note that a list $\S_c$ can be empty and all members in $\SS$ may be empty.
% 
% The other type of \sync{}ation, {\em\Uidx\exsync}, takes place by
% \!\switchcolumn!|*| or the beginning of starred \csenv{}s, by
% \Endparacol, or by one of page flushing macros \!\flushpage!, 
% \!\clearpage! and \!\cleardoublepage!.  A flushing \exsync{} ships out the
% pages from $\pbase$ to $\Uidx\ptop=\!\pcol@toppage!$ being the ordinal of
% the {\em\Uidx\tpage} to which the most advanced {\em\Uidx\lcolumn} has
% reached.  On the other hand, other non-flushing \exsync{} keeps the page
% $\ptop$ from being shipped out because the \colpage{}s in it or the page
% itself will grow further.
% 
% 
% 
% \subsection{Current Column-Pages and Their Contexts}
% \label{sec:imp-ovv-ccol}
% \changes{v1.3-4}{2013/09/17}
%	{Remove $\mu$ for \cs{@mparbottom} from column-context because it is
%	 now in page context.}
% 
% We also have to maintain another type of \colpage{}s which are currently
% built, or {\em\Uidx\ccolpage{}s} in short, to switch from a column to
% another.  Since each column may have its own {\em context} for the
% typsetting of it, or {\em\Uidx\cctext} in short, it were perfect to save
% the context when we leave from a 
% column and to restore that when we revisit the column if we could.
% However, \TeX{} and \LaTeX{} has a tremendously large number of context
% variables and the number becomes virtualy boundless when we take variables
% defined in various styles and by users themselves into account.
% Therefore, we had to abandon to keep the whole context of the column but
% carefully chose a small subset comprising variables automatically modified
% outside of users' control.  That is, the \cctext{}
% $\Uidx\cc_c=|\pcol@col|{\cdot}c$
% 
% \SpecialArrayMainIndex{c}{\pcol@col}
% 
% of a column $c$ consists of the following elements, each of which named
% $e$ is referred to as $\cc_c(e)$ hereafter.
% 
% \begin{itemize}
% \item $\Uidx\vb$
% represents $\!\insert!{\cdot}\vb$ containing the followings.
% 
% \begin{itemize}
% \item
% $\vb^b=\!\box!{\cdot}\vb=\!\@holdpg!$ is the \!\vbox! containing the main
% vertical list which has already contributed to the \ccolpage{}.
% 
% \item
% $\vb^p=\!\count!{\cdot}\vb=\!\pcol@page!$ means the \ccolpage{} belongs to
% the page $\vb^p$.
% 
% \item
% $\vb^r=\!\dimen!{\cdot}\vb=\!\@colroom!$ is the room of the column.
% \end{itemize}
% 
% \item $\Uidx\ft=\Midx{\!\pcol@currfoot!}$
% is the \!\insert! containing the footnotes added in the \ccolpage, if
% \Mcfnote{} typesetting is in effect.  Its constituent \!\box!, \!\count!,
% \!\dimen! and \!\skip! are denoted as $\ft^b$, $\ft^c$, $\ft^d$ and
% $\ft^s$ respectively.  On the other hand, if \Scfnote{} typesetting is in
% effect, $\ft$ is always empty\footnote{
% 
% But the macro \cs{pcol@currfoot} is used to keep \Scfnote{}s temporarily.}.
% 
% \item $\Uidx\pd=\!\pcol@prevdepth!$
% is the depth of the last vertical item in $\vb^b$ obtained by
% \!\prevdepth!.
% 
% \item $\Uidx\tl=\!\@toplist!$
% is the list of top floats inserted in the \ccolpage.
% 
% \item $\Uidx\ml=\!\@midlist!$
% is the list of mid floats inserted in the \ccolpage.
% 
% \item $\Uidx\bl=\!\@botlist!$
% is the list of bottom floats inserted in the \ccolpage.
% 
% \item $\Uidx\dl=\!\@deferlist!$
% is the list of \cwise{} floats deferred to the next \colpage.
% 
% \item $\Uidx\tf=\!\pcol@textfloatsep!$
% is the vertical skip used instead of \!\textfloatsep! for top floats in the
% \ccolpage{} if it has \sync{}ation points, or $\infty$ otherwise.
% 
% \item $\Uidx\fh=\!\@textfloatsheight!$
% is the total height of mid floats and their separators in the \ccolpage.
% 
% \item $\Uidx\tn=\!\@topnum!$
% is the maximum number of top floats which the \ccolpage{} can accommodate
% further.
% 
% \item $\Uidx\tr=\!\@toproom!$
% is the room for top floats in the \ccolpage.
% 
% \item $\Uidx\bn=\!\@botnum!$
% is the maximum number of bottom floats which the \ccolpage{} can accommodate
% further.
% 
% \item $\Uidx\br=\!\@botroom!$
% is the room for bottom floats in the \ccolpage.
%
% \item $\Uidx\cn=\!\@colnum!$
% is the maximum total number of floats which the \ccolpage{} can accommodate
% further.
% 
% \item $\Uidx\sw$
% is the following encoding of \CSIndex{if@nobreak} and
% \CSIndex{if@afterindent} at the time we left from the column $c$.
% $$
% \sw=\cases{0&$\CSIndex{if@nobreak}=\false$\cr
%            1&$\CSIndex{if@nobreak}=\true\;\land\;
%               \CSIndex{if@afterindent}=\true$\cr
%            2&$\CSIndex{if@nobreak}=\true\;\land\;
%               \CSIndex{if@afterindent}=\false$}
% $$
% Note that we have only three states because \CSIndex{if@afterindent} is
% meaningful only when $\CSIndex{if@nobreak}=\true$\footnote{
% 
% If only with the standard \LaTeX{} and so far.}.
% 
% \item $\Uidx\ep=\!\everypar!$
% is the tokens stored in \!\everypar! at the time we left from the column
% $c$.
% \end{itemize}
% 
% \changes{v1.1}{2012/05/11}
% 	{Add description of $w_c\EQ\cs{pcol@columnwidth}{\cdot}c$.}
% 
% In addition, we have special context variables
% $\Uidx\w_c=|\pcol@columnwidth|{\cdot}c$
% 
% \SpecialArrayMainIndex{c}{\pcol@columnwidth}
% 
% in which we keep \!\columnwidth! for the column $c$.
% 
% Note that we could add other variables to the saved context and/or provide
% some API macro to define them by users, but abandon them because it should
% be too complicated for users\footnote{
% 
% And for the author if we include save/restore of macros, though it could
% be done with a \cs{toks} containing the \cs{def}initions of macros.}.
% 
% Also note that we provide a save/restore mechanism for \lcounter{}s as
% discussed in \secref{sec:imp-ovv-counter}.
% 
% 
% 
% \subsection{Pages and Their Contexts}
% \label{sec:imp-ovv-page}
% \changes{v1.2-2}{2013/05/11}
%	{Redesign page context and its implementation.}
% \changes{v1.3-3}{2013/09/17}
%	{Add $\pi^s(p)$ to the page context of $p$ for column-separating
%	 rule drawing and background painting.}
% \changes{v1.3-4}{2013/09/17}
%	{Add $\pi^m(p)$ to the page context of $p$ for marginal note
%	 placement.}
% 
% Besides the \colpage{}s, we have to keep track each whole page not yet
% shipped out but has some complete or incomplete (i.e., current) \colpage{}s.
% We maintain the list;
% 
% \SpecialMainIndex{\pcol@pages}
% 
% \begin{eqnarray*}
% \Uidx\PP
% &=&\!\pcol@pages!=(\Uidx\pp(\pbase),\pp(\pbase{+}1),\ldots\pp(\ptop{-}1))\\
% &=&\!\@elt!\,\pp(\pbase)\;\!\@elt!\,\pp(\pbase{+}1)\;\cdots\;
%    \!\@elt!\,\pp(\ptop{-}1)\\
% \pp(p)&=&|{|\pp^p(p)|}|\pp^i(p)\pp^f(p)|{|\pp^s(p)|}||{|\pp^m(p)|}|
% \end{eqnarray*}
% 
% where $\pp(p)$ is the {\em\Uidx\pctext} of $p$ and its elements $\pp^p(p)$,
% $\pp^i(p)$, $\pp^f(p)$, $\pp^s(p)$ and $\pp^m(p)$ have the followings.
% 
% \begin{itemize}
% \item
% $\pp^p(p)=\Uidx\page(p)$ is the value of the counter \counter{page}
% (i.e. \!\c@page!) for the page $p$.
% 
% \item
% Iff $\pp^i(p)\neq\bot$, the page $p$ has \pwise{} floats or the
% single-column {\em\Uidx\preenv} preceeding \beginparacol{} in the
% {\em\Uidx\spage} where it resides and spannig all columns.  In this case
% $\pp^i(p)=i$ represents $\!\insert!{\cdot}i$, often {\em cached}
% in the macro \!\pcol@spanning!, for such {\em\Uidx\spanning}
% whose registers have the followings.
% 
% \begin{itemize}
% \item
% $\pp^b(p)=\!\box!{\cdot}i$ contains the \spanning.
% 
% \item
% $\pp^h(p)=\!\dimen!{\cdot}i=\!\@colht!$ if positive for the height of
% columns shrunk by the \spanning.  If negative, the page is only for
% the \spanning, i.e. a {\em\Uidx\fpage}.  We use the notation $\pp^h(p)$
% for the pages $\pp^i(p)=\bot$ to mean \!\textheight!.
% 
% \item
% $\pp^t(p)=\!\skip!{\cdot}i=\!\pcol@topskip!$ being the value of
% \!\topskip! at \beginparacol{} to be inserted at the top of each column in
% each non-first page.  Otherwise, i.e., for the columns in the \spage{}
% following the \preenv{}, it has 0 to prevent the \!\topskip! insertion.
% We use the notation $\pp^t(p)$ for the pages $\pp^i(p)=\bot$ to
% mean \!\pcol@topskip!.
% \end{itemize}
% 
% \item
% Iff $\pp^f(p)\neq\bot$, \Scfnote{} typesetting, discussed in
% \secref{sec:imp-ovv-scfnote}, is in effect and the page $p$ has some
% footnotes in $\!\box!\cdot\pp^f(p)$.  This element is often {\em cached}
% in the macro \Midx{\!\pcol@footins!}.
% 
% \SpecialMainIndex{\pcol@pages}
% 
% \item
% $\pp^s(p)=(\Uidx\spt(H_1,h_1),\ldots,\spt(H_n,h_n))=
% \!\@elt!\Arg{H_1,h_1}\ldots\!\@elt!\Arg{H_n,h_n}$ is the list of \mctext{}s
% in the page $p$, where $i$-th one's top edge is at $H_i$ from the top of
% the page (excluding \spanning) and its height-plus-depth is $h_i$, where
% $H_i$ and $h_i$ are represented in the form of integers.  Therefore, it is
% emptied by \!\pcol@startpage!, and then the elements are added by
% \!\pcol@makecol! (only for the last one) and \!\pcol@output@switch!
% whenever they find a \mctext{} completes.  The element is often {\em
% cached} in the macro \!\pcol@sptextlist! and is referred to by
% \!\pcol@buildcolseprule! to draw \cseprule{} and to paint columns and
% \csepgap{} leaving spaces for \mctext{}s.  The usage of this element is
% discussed in \secref{sec:imp-ovv-cswap} a little bit more detailedly.
% 
% \item
% $\pp^m(p)=\Arg{\Uidx\mpb_L^l}\Arg{\mpb_L^r}\Arg{\mpb_R^l}\Arg{\mpb_R^r}$
% is the set of lists of marginal notes in the left ($l$) and right
% ($r$) margins and in the left ($L$) and right ($R$) \parapag{}es.  The
% words left and right of margins mean physical left and right, while left
% and right of \parapag{}es mean the logical ones, i.e., the page where the
% column-0 resides is left.  Each element $\mpb_{\{L,R\}}^{\{l,r\}}$ has a
% list $(\Uidx\mpar(t_1,b_1),\ldots,\mpar(t_n,b_n))=
% \!\@elt!\Arg{t_1}\Arg{b_1}\ldots\!\@elt!\Arg{t_n}\Arg{b_n}$ of marginal
% notes whose top and bottom are at $t_i$ and $b_i$ from the top of the
% column area, where $t_i$ and $b_i$ are represented in the form of
% integers.  Each element can be empty of course, and $\pp^m(p)$ itself can
% be so as well to mean all elements are empty\footnote{
% 
% To minimize the possibility of miscoding for emptying and save a small
% amount of memory for pages having no marginal notes.}.
% 
% Therefore, $\pp^m(p)$ is emptied by
% \!\pcol@startpage!, and then examined and modified by
% \!\pcol@addmarginpar! when it adds a marginal note through macros
% \!\pcol@getmparbottom! and \!\pcol@setmpbelt!.  Another modifier
% \!\pcol@output@start! initializes one of the element $\mpb_L^{\{l,r\}}$
% with the value representing the last marginal note in \preenv{}, while
% another examiner \!\pcol@output@end! lets the outside \!\@mparbottom! have
% a value based on $b_n$ of one of the element, according to \LaTeX's
% setting of marginal note placement.  The whole elemnt $\pp^m(p)$ is often
% {\em cached} in the macro \!\pcol@mparbottom!.  The usage of this
% element is discussed in \secref{sec:imp-ovv-cswap} a little bit more
% detailedly.
% \end{itemize}
% 
% Note that even in \parapag{}ing and in \npaired{} one in particular,
% a page $p$ consists of all columns $c\In0\C$.  Therefore, the term {\em
% left\slash right \parapag{}e} $p$ always mean the left and right component
% of a \parapag{}e (pair) $p$.
% 
% The reason why we keep track of $\page(p)$ is that page numbering is not
% necessrary to be consecutive.  If such a {\em jump} occurs randomly in any
% columns explicitly updating \counter{page}, it is very tough to give a
% consistent view of the page number of a specific page to all columns.
% Therefore we suppose jumps occur only in the leftmost column 0\footnote{
% 
% But we neither inhibit nor nullify a jump in non-leftmost column and thus
% the update can be seen referring to \texttt{page} counter explicitly.}
% 
% which controls the page numbering, while non-leftmost columns are expected
% to refer the \counter{page} passively.
%
% This page numbering is implemented as follows.  Each time a \colpage{} at
% $p$ of the leftmost column is completed to start a new \colpage{},
% $\page(p)$ is fixed to be the value of \counter{page} and
% $\page(q)=\pp^p(q)$ for all $q\in[p,\ptop]$ are let be $\page(p)+(q-p)$ in
% usual cases but $\page(p)+2(q-p)$ in \npaired{} \parapag{}ing.
% This update also takes place on \cswitch{} from the leftmost \colpage{} at
% $p$ to another column so that a jump happning before the switching is
% notified to other columns.  On the other hand, starting or \cswitch{} to a
% non-leftmost \colpage{} at $p$ lets \counter{page} have $\page(p)$
% referring to $\pp(p)$, unless the column starts the most advanced \tpage.
% In this new \tpage{} case, $\pp(\ptop{+}1)$ is added to $\PP$ with the
% temporary setting $\pp^p(\ptop+1)=\page(\ptop+1)=\page(\ptop)+1$ usually but
% $\pp^p(\ptop+1)=\page(\ptop+1)=\page(\ptop)+2$ in \npaired{}
% \parapag{}ing, and $\ptop$ is incremented.
% 
% Note that this management is imperfect because direct references of
% \counter{page} in non-leftmost columns can give inconsistent results if 
% \counter{page} is modified in a non-leftmost column or the reference occurs
% in a page $p$ after that the leftmost column modifies \counter{page} in a
% page $q$ such that $q\leq p$.  In addition to them, this mechanism in
% \npaired{} \parapag{}ing always gives incorrect page number to the
% columns in a right \parapag{}e because $\pp^p(p)$ always has $\page(p)$
% for the left \parapag{}e.  However, it is expected that the progress of
% the leftmost column usually preceeds other columns to give consistent
% \counter{page} reference even with jumps, unless the reference is made by
% a column in a right \npaired{} \parapag{}e.  More importantly, it is
% assured that indirect refereces through |.aux| records and page numbers
% recorded in |.toc|, |.idx|, and so on are always consistent because of the
% lazy evaluation of $\counter{page}=\page(p)$ at ship-out of an ordinary
% page $p$ or a left \parapag{}e $p$, while the counter is let have
% $\page(p)+1$ when a right \npaired{} \parapag{}e $p$ is shipped out.
% 
% Also note that we also keep $\pp(\ptop)$ in \Midx{\!\pcol@currpage!}
% which is initialized by \!\pcol@output@start! to let $\pp^i(\ptop)$ have
% the \preenv.  Then the macro is redefined to have the value representing
% the new page possibly with $\pp^i(\ptop)$ for \pwise{} floats in
% \!\pcol@startpage! by the macro \!\pcol@defcurrpage!.  Another
% \!\def!inition is done in \!\pcol@output@switch! also with
% \!\pcol@defcurrpage! to let $\pp^f(\ptop)$ have \Scfnote{}s built in
% \!\footins! if \scfnote{} typsetting is in effect and the \cswitch{}
% leaves the column in $\ptop$\footnote{
% 
% The \!\def!inition of \!\pcol@currpage! in \!\pcol@setpnoelt!, and
% emptying it in \!\pcol@output@start! and \!\pcol@freshpage! are for coding
% trick and thus not for giving a really new \!\def!initions.}.
% 
% We denote the concatenation of $\PP$ and $\pp(\ptop)$ as $\Uidx\PPP$ to
% represent all pages {\em on-the-fly}.
% 
% 
% 
% \subsection{Counters}
% \label{sec:imp-ovv-counter}
% 
% Besides the context variables discussed in \secref{sec:imp-ovv-ccol}, we
% need to make counters local to each column except for those declared to be
% global by \!\globalcounter!.  Let $\Uidx\CC$ be the set (list) of all
% counters declared before \beginparacol, i.e., $\CC=\!\cl@@ckpt!$, and
% $$
% \Uidx\CG=\!\pcol@gcounters!=\{\Uidx\cg_1,\ldots\}=\!\@elt!|{|\cg_1|}|\cdots
% $$
% be the set of {\em\Uidx\gcounter{}s} which have declared so by
% \!\globalcounter!|{|$\cg_i$|}|.  Then the set of {\em\Uidx\lcounter{}s}
% $\Uidx\CTL$ is defined as follows.
% 
% \SpecialMainIndex{\pcol@counters}
% $$
% \CTL=\CC-\CG=\!\pcol@counters!=\{\Uidx\cl_1,\ldots\}=\!\@elt!|{|\cl_1|}|\cdots
% $$
% 
% Since each column has its own values in \lcounter{}s, we have to keep the
% set of counter\slash value pairs
% 
% \SpecialArrayMainIndex{c}{\pcol@counters}
% $$
% \Uidx\Cc_c=\!\pcol@counters!{\cdot}c=\{\<\cl_1,\val_c(\cl_1)\>,\ldots\}
% =\!\@elt!|{|\cl_1|}||{|\val_c(\cl_1)|}|\cdots
% $$
% for each column $c$, where $\Uidx\val_c(\cl_i)$ is the value of a counter
% $\cl_i$ local to $c$.  That is, whenever we switch from a column $c$ to
% $d$, we save $\<\cl_i,\val_c(\cl_i)\>$ in $\Cc_c$ and restore $\cl_i$ for
% $d$ by letting it have $\val_d(\cl_i)$ in $\Cc_d$, for all $\cl_i\in\CTL$.
% 
% A \gcounter{} is free from these save\slash restore operations but needs
% another special operation when it is incremented by \!\stepcounter!.  That
% is, the invocation of \!\stepcounter! for a \gcounter{} $\cg_i$ may clear
% \lcounter{}s in its set of descendant counters
% $\Uidx\clist(\cg_i)=|\pcol@cl@|{\cdot}\cg_i$
% 
% \SpecialArrayMainIndex{\theta}{\pcol@cl@}
% 
% and this clearing must be performed on the all instances of
% $\cl_j\in\clist(\cg_i)$ saved in $\Cc_c$ for all $c\In0\C$.  Therefore, on
% the \!\stepcounter!, we do the followings for all $c\In0\C$; temporarily
% restore all $\cl_k\in\CTL$ from $\Cc_c$; clear all $\cl_j\in\clist(\cg_i)$;
% and then save $\<\cl_k,\val_c(\cl_k)\>$ back to $\Cc_c$.
% 
% The other item we maintain for a \lcounter{} $\cl$ is its {\em\Uidx\lrep}
% $\arg{rep}$ in a column $c$ defined by
% $\!\definethecounter!\<\cl\>\<c\>\arg{rep}$.  The \lrep{} $\arg{rep}$ is
% kept in $|\pcol@thectr@|{\cdot}\cl{\cdot}c$ and is made \!\let!-equal to
% $|\the|{\cdot}\cl$ when the column $c$ is visted.
% 
% \SpecialArrayMainIndex{\theta{\cdot}c}{\pcol@thectr@}
% \SpecialArrayIndex{\theta}{\the}
% 
% 
% 
% \subsection{Page-Wise and Merged Footnotes}
% \label{sec:imp-ovv-scfnote}
% \changes{v1.2-2}{2013/05/11}
%	{Add the subsection ``Single-Columned and Merged Footnotes''.}
% \changes{v1.3-6}{2013/09/17}
%	{Change the title from ``Single-Columned and Merged Footnotes'' to
%	 ``Page-Wise and Merged Footnotes'' according to the new naming.}
% 
% \Uidx{\Index{page-wise footnote}}{\em Page-wise footnote}
% typesetting is completely different from ordinary {\em\Mcfnote}
% typesetting.
% 
% When a \colpage{} in the \tpage{} is built, \!\footins! keeps all footnotes
% \!\insert!ed by \!\footnote! or \!\footnotetext! in {\em any} columns in
% the page.  Therefore, \!\footnote! and \!\footnotetext! in the \tpage{}
% act as usual to add the footnote to \!\footins!.  Then if a \cswitch{}
% takes place to leave the column, \!\footins! is saved into $\pp^f(\ptop)$
% by \!\pcol@output@switch!, so that $\pp^f(\ptop)$ is \!\insert!ed to
% \!\footins! again by \!\pcol@restartcolumn! when it visits a column in
% $\ptop$, or by \!\pcol@startcolumn! when it finds a column proceeds to
% $\ptop$.
% 
% Then, when a \colpage{} in the \tpage{} completes advancing $\ptop$,
% \!\footins! is kept in $\pp^f(\ptop{-}1)$ by \!\pcol@startpage!, rather
% than being combined with the \colpage{}.  This saving into
% $\pp^f(\ptop{-}1)$ {\em fixes} the footnotes in $\ptop{-}1$ so that
% $\pp^f(\ptop{-}1)$ is combined with other materials in the page by
% \!\pcol@outputelt! or \!\pcol@makeflushedpage! through \!\pcol@putfootins!
% when the page is shipped out.
% 
% Fixing $\pp^f(p)$ for $p<\ptop$ makes it impossible to add footnotes in a
% column in the page $p$ not only to $\pp^f(p)$ but also to \!\footins!
% for the page $p$ becaues we have at least one fixed \colpage{} $\s_c(p)$
% unable to shrink to have such additional footnotes in $p$\footnote{
% 
% The \colpage{} $\s_c(p)$ could have some space at its bottom produced by,
% for example, \cs{newpage}, but exploitation of such space is extremely
% hard.}.
% 
% Therefore, such a footnote addition is {\em deferred} and is thrown into
% $\pp^f(\ptop)$ through a list;
% $$
% \Uidx\df=\!\pcol@topfnotes!=
% (f_1,f_2,\ldots,f_n)=\!\vbox!|{|f_1\;f_2\;\cdots\;f_n|}|
% $$
% where $f_i$ is a \!\vbox! containing the deferred footnote preceded by
% \!\penalty!\!\interlinepenalty! to allow \TeX{} to break footnotes to place
% them in two (or more) pages.  That is, \!\footnote! or \!\footnotetext! in
% $p<\ptop$ adds an element for the footnote to $\df$, then all the elements
% 
% \footnote{More accurately, some trailing elements may be left in $\df$ if
% its total height is too large, as discussed in
% \secref{sec:imp-sout-scfnote}.}
% 
% are \!\insert!ed to \!\footins! by \!\pcol@deferredfootins! invoked in
% \!\pcol@restartcolumn! when it visits a column in $\ptop$, or in
% \!\pcol@startcolumn! when it starts a \colpage{} in $\ptop$.  The macro
% \!\pcol@output@end! also do the \!\insert!ion by itself with \Mgfnote{}
% typesetting to let deferred footnotes be a part of \postenv.
% 
% The reference to $\pp^f(p)$ for $p<\ptop$ is also made in
% \!\pcol@restartcolumn! and \!\pcol@flushcolumn!.  The former \!\insert!s
% $\pp^f(p)$ to \!\footins! so that the \colpage{} which the macro restarts
% is built as if it has the footnotes in $\pp^f(p)$ to make the \colpage{}
% broken leaving the space for the footnotes.  However, \!\footins! is never
% grown because it has been fixed and thus additional footnotes will go
% to $\df$ as discussed above.  Then \!\footins! is discarded by
% \!\pcol@makecol! when the \colpage{} completes, or by
% \!\pcol@output@switch! when it leaves the column.
% 
% The reference to $\pp^f(p)$ by the latter macro \!\pcol@flushcolumn! is to
% build the ship-out image of the \colpage{} to be flushed.  When this macro
% and other macros, namely \!\pcol@makecol! and \!\pcol@makeflushedpage!,
% build the ship-out image in a page $p$ having $\pp^f(p)$ using
% \!\@makecol!, we have to be careful of the fact that the \colpage{} has
% been build as if it has footnotes in $\pp^f(p)$ but the footnotes are not
% included in its ship-out image but that of the page.  Therefore,
% \!\@colht! referred in \!\@makecol! should be shrunk by the sum of
% height and depth of $\pp^f(p)$ and $\!\skip!\cdot\pp^f(p)$ by
% \!\pcol@shrinkcolbyfn!.  Other and more subtle adjustment is to add the
% stretch and shrink factors of $\!\skip!\cdot\pp^f(p)$ at the tail of the
% \colpage{} by \!\pcol@unvbox@cclv!.  This is necessary because \TeX{} has
% broken the \colpage{} taking account of the stretch and, more essentially,
% shrink factors, and thus without the factors the main vertical list in the
% \colpage{} could be a little bit taller than \!\@colht! causing overfull.
% 
% The feature gathering footnotes in all columns in a page brings a problem
% to \exsync{}, because a column whose contents fit the \tpage{} at the
% last visit may be too tall on the \sync{}ation because other columns have
% put some footnotes after the last visit.  That is, we cannot simply build
% the \tpage{} combining $\s_c(\ptop)$ for all $c\In0\C$ and $\pp^f(\ptop)$
% because there could be $\s_c(\ptop)$ too tall to reside in $\ptop$ with
% $\pp^f(\ptop)$.
% 
% To solve this problem, we perform the following operations prior to fix
% the contents of $\ptop$ having an \exsync{} point in it.  First one is
% {\em\Uidx\cscan} to visit all columns by \cswitch{} prior to the
% \sync{}ation so that \TeX's page builder has opportunities to break too
% tall \colpage{}s.  Since this scan could merely break footnotes rather
% than the main vertical lists in the \colpage{}s and the broken footnotes
% will be reconnected when the \!\output!-routine is invoked for the
% \sync{}ation, we then examine if all $\s_c(\ptop)$ are accommodated in
% $\ptop$ with $\pp^f(p)$.
% 
% This examination for a \sync{}ation by \!\switchcolumn!|*| or its
% relatives is done as a part of the inherent \sync{}ing procedure to see if
% the combination of the tallest {\em top} items, i.e., top floats and the
% main vertical list, and the tallest {\em bottom} items, i.e., bottom
% floats and \Mcfnote{}s, is too large causing page flushing.  As for page
% flushing and environment closing, this {\em\Uidx\pfcheck} requires a
% special kind of \sync{}ed \cswitch{} by which we flush pages up to
% $\ptop-1$ and examine if there is a too tall column.
% 
% Then if too tall columns are found, in either cases, we move to the {\em
% tallest} column to force a page break in the column so that we have a new
% page with shorter columns and shorter \Scfnote{}s as well.  In the
% \sync{}ation by \!\switchcolumn!|*| or its relatives, this forced page
% break is then applied to all other columns so that new \colpage{}s have
% top floats, if any, below which we should place the \sync{}ation point.
% This examination and forced page break is repeated until we have a page
% without any too tall columns, because a page break may bring deferred
% floats and footnotes which may result in a too tall column.
% 
% 
% 
% \subsection{Text Coloring}
% \label{sec:imp-ovv-color}
% \changes{v1.2-1}{2013/05/11}
%	{Add the subsection ``Coloring''.}
% \changes{v1.3-5}{2013/09/17}
%	{Change the subsection title from ``Coloring'' to ``Text Coloring''
%	 to distinguish it from background painting clearly.}
% 
% \subsubsection{Fundamental Mechanism}
% \label{sec:imp-ovv-color-fundamental}
% \changes{v1.34}{2018/05/07}
% 	{Revise the description of \Sec1.6.1 according to the new
%	 implementation with \cs{insert}.}
% 
% Text coloring done by \textsf{color} package and its relatives using
% \!\special! stands on the fact that the main vertical list is {\em
% printed} in the order of occurrence in the source |.tex|.  That is, a
% command such as \!\color!|{red}| puts
% \!\special!|{color push [1 0 0]}|\footnote{
% 
% If \texttt{.dvi} is processed by \texttt{dvips}, or other
% printer-dependent command corresponding to it.}
% 
% into |.dvi| to make all stuff in the main vertical list colored red until
% other coloring \!\special! inserted by other coloring macro appears in
% |.dvi|.  This simple mechanism works well even when the pair of coloring
% \!\special!s are in different pages and/or columns because, with respect
% to the main vertical list, everything between them in |.tex| is also
% surrounded by the \!\special! pair in |.dvi|.  As for other stuff such as
% header, footer, floats and footnotes, \LaTeX{} surrounds them by
% \!\color@begingroup! and \!\color@endgroup! or other similar constructs so
% that they are colored without interference with the coloring of the main
% vertical list.
% 
% In \env{paracol} environment, however, the orders of the main vertical list
% in |.tex| and |.dvi| are not always same.  When a column encounters a page
% break, in |.dvi| the other column should intervene between the stuff in
% the broken pair of \colpage{}s possibly changing the color of the second
% \colpage.  A \cswitch{} from $c_1$ to $c_2$ also makes the main vertical
% list out-of-order to cause another unexpected coloring because a coloring
% command in $c_2$ will have no effect when $c_1$ is revisited after that
% following its pre-switching stuff in |.dvi| which was put before the
% coloring.  Therefore, we have to make {\em\Uidx\colorctext{}s} in both
% |.tex| and |.dvi| coherent inserting appropriate \!\special!s into |.dvi|
% whenever an out-of-order {\em jump} occurs in |.dvi| by a page break or in
% |.tex| by a \cswitch{}.
%
% The \textsf{color} package and its relatives\footnote{
% 
% And all other coloring mechanism compliant with \LaTeXe{}, hopefully and
% believingly.}
% 
% assume that {\em printers} have a stack for coloring and thus a coloring
% \!\special! pushes the new color into the stack while it is popped by
% another \!\special! which will be inserted by \!\aftergroup!  mechanism
% when a group surrounding the coloring \!\special! is closed.  Therefore we
% have to keep track of the \colorctext{} with {\em\Uidx\colorstack}
% $$
% \Uidx\cst=
% (\Uidx\celt_1,\celt_1,\ldots,\celt_n)=
% \!\vbox!|{|\celt_1\;\celt_2\;\ldots\;\celt_n|}|
% $$
% where $\celt_i$ is a \!\vbox! of 1\,|sp| tall, 0 deep and 0 wide containing
% a coloring \!\special! which \!\set@color! puts into the main vertical
% list.  That is, when \!\set@color! is invoked we push $\celt$ to the tail
% of $\cst$, while when the corresponding \!\reset@color! appears we pop it
% from $\cst$\@.  Then when we encounter an out-of-order jump, at first we
% rewind the \colorstack{} in |.dvi| by putting \!\special!s which
% \!\reset@color! would put, and then reestablish the \colorstack{} by
% putting \!\special!s in $\celt_i$ as if \!\set@color! for it is invoked
% for all $\celt_i\in\cst$.  Therefore from the viewpoint of a {\em
% printer}, it will see stack-rewinding at the end of each \colpage{} and
% the leaving points of \cswitch{}, while the beginning of each \colpage{}
% and the entry points of \cswitch{} should have the sequence of coloring
% \!\special!s to regain the \colorstack{} which the {\em printer} must have
% at each of the points.
% 
% \SpecialArrayMainIndex{c}{\pcol@columncolor@box}
% 
% In addition, for each column $c$ we keep
% $\Uidx\Celt^c=|\pcol@columncolor@box|\cdot c$ as the {\em default} color
% of the column $c$, optionally given by the API macro \!\columncolor! or
% \!\normalcolumncolor!.  If given for $c$, it is assumed to be at the bottom
% of the \colorstack{} denoted by
% $\Uidx\CST^c=(\Celt^c,\celt_1,\ldots,\celt_n)$ which we rewind\slash
% reestablish at each out-of-order jump in the column $c$.
% 
% 
% 
% \subsubsection{Coloring in Horizontal Mode}
% \label{sec:imp-ovv-color-hmode}
% \changes{v1.22}{2013/06/30}
%	{Add the subsection ``Coloring in Horizontal Mode''.}
% \changes{v1.34}{2018/05/07}
% 	{Revise the description of \Sec1.6.2 according to the new
%	 implementation with \cs{insert}.}
% 
% We have to pay attention to the fact a coloring command can appear in
% horizontal mode of course, and thus push/pop operations in a \colpage{}
% would be done {\em before} the \colpage{} starts when \!\set@color! or
% \!\reset@color! is in the second half of a page-crossing paragraph and if
% we immediately performed push\slash pop of the \colorstack{} in these
% macros.  In addition, even in vertical mode these macros can appear before
% \TeX{} finds a page break after which they must be in effect, if they are
% preceded by a sequece of non-breakable vertical items by which \TeX{}'s
% examination of the page break is {\em delayed} as well as the invocation
% of \!\output! at the break.
% 
% In order to solve the problem of push/pop timing, we perform push\slash
% pop operations through \!\insert! to our own register set
% \!\pcol@colorins!.  That is, we \!\insert! $\celt$ to \!\pcol@colorins!
% when we encounter a \!\set@color! for $\celt$, while its corresponding
% \!\reset@color! also \!\insert!s another \!\vbox! $\Uidx\celtpop$ of
% null-height\slash depth\slash width having a \!\special! which the
% \!\reset@color! puts into the main vertical list.  Since we let
% $\!\count!\!\pcol@colorins!=0$ and $\!\skip!\!\pcol@colorins!=0$ to keep
% the \!\insert!iion from affecting the growth of \!\pagetotal!, it is
% guaranteed that an inserted $\celt$ or $\celtpop$ is given to \!\output!
% through \!\pcol@colorins! together with |\box255| containing the
% corresponding \!\special!.
% 
% When \!\output! is invoked, \!\pcol@colorins! has $\Uidx\cstraw$
% containing $\celt_i$ and possibly its corresponding $\celtpop_i$.
% Therefore, if \!\output! is for a page break or a \cswitch, we remove all
% pairs of $\celt_i$ and $\celtpop_i$ from \!\pcol@colorins! to let it have
% $\cst$ only with $\celt_j$ whose corresponding $\celt_j^-$ is not in
% $\cstraw$.  For this removal, we scan $\cstraw$ from its tail
% incrementing\slash decrementing a counter $\Uidx\npop$ which we initialize
% to 0 before scanning.  In the scan, we remove all $\celtpop$
% unconditionally incrementing $\npop$, and $\celt$ such that $\npop>0$ on
% the encounter with it decrementing $\npop$.  This scan is done by
% \!\pcol@clearcolorstack!, invoked from \!\pcol@opcol! for a page break and
% \!\pcol@output@switch! for a \cswitch{} through \!\pcol@clearcst@unvbox!,
% and is for rewinding the \colorstack{}
% $(\Celt^c,\cstraw)=\Uidx\CSTraw^c$.  Therefore, for each $\celt$ to be kept
% because of $\npop=0$ on the encoutner with it we put \!\special! for
% \!\reset@color!.  Note that on another scan for stack reestablishment,
% \!\pcol@colorins! has $\cst$ and is kept unchanged.  Also note that other
% \!\output!  invocations such as that for floats do not touch $\cstraw$ to
% allow it grows with $\celt$ and $\celtpop$ corresponding to \!\set@color!
% and \!\reset@color! in the \colpage{} in which the invocation
% happens\footnote{
% 
% Unlike \cs{footins} which becomes void by putting its contents back to the
% main vertical list to reexamine the footnote placement possibly with
% splitting.}.
% 
% The mechanism above especially for horizontal mode has subtle issues as
% follows.
% 
% \begin{itemize}
% \item
% If \!\set@color! appears in a \!\vbox!, the \!\insert!ion for pushing
% is not effective but corresponding \!\reset@color! can be outside of the
% \!\vbox! to make pushes and pops unbalanced because \!\aftergroup! for it
% inserts it just after the closing of the \!\vbox! if \!\set@color! is not
% surrounded by an inner group.
% 
% \item
% If we are in vertical mode, we can know if we are in a \!\vbox! by
% \CSIndex{ifinner}.  However, in horizontal or math mode, \CSIndex{ifinner}
% cannot help us because it is true iff we are in a \!\hbox! or in an
% in-text math.  In short, \TeX{} does not provide us with any convenient
% means to know if we are in a \!\vbox!.
% \end{itemize}
% 
% To solve the problem above, we introduced a trick with \!\everyvbox! to
% turn a switch $\CSIndex{ifpcol@inner}=\true$ at the beginning of every
% \!\vbox! in a \env{paracol} environment, by which we supress the
% \!\insert!ion for \!\set@color! because a \!\vbox! cannot cross a
% page boundary.  As for that of \!\reset@color!, we suppress it by not
% reserving our own macro \!\pcol@reset@color@pop! for the \!\insert!ion by
% \!\aftergroup!.  That is, we reserve both \!\reset@color! and
% \!\pcol@reset@color@pop! with \!\aftergroup! if we are outside of any
% \!\vbox!es, while does the former only otherwise.  By the same reason, we
% supress the \!\insert!ion if we are in restricted horizontal mode, i.e.,
% if both \CSIndex{ifhmode} and \CSIndex{ifinner} are true.  On the other
% hand, we cannot supress the \!\insert!ion when we are in an in-text math
% because it can cross a page boundary\footnote{
% 
% If an in-text math is in a \cs{hbox}, \cs{insert}ion is not
% necessary because the math cannot cross a page boundary.  Though we can
% detect it by a trick with \!\everyhbox!, we abandon this idea because the
% request is not harmful.  Another and more serious issue of coloring in
% math mode will be discussed shortly.}.
% 
% Note that the detailed implementation shown in \secref{sec:imp-startenv}
% does not interfere the use of \!\everyvbox! inside/outside of
% \env{paracol} environments or is not affected by the use.
% 
% Another attention we should pay is that \!\color! will leave \!\aftergroup! 
% tokens of \!\reset@color! and thus they are invoked just after
% \Endparacol.  However, since we have completed all \colpage{}s in the
% \lpage, the \colorstack{} in |.dvi| should be empty.  Therefore to avoid
% stack underflow, we should reestablish $\cst$ (not $\CST^c$) so that
% elements in the stack are popped by \!\reset@color! invoked with the
% \!\aftergroup! mechanism.  We also take care of our own \colorstack{}
% popper \!\pcol@reset@color@pop! which must do nothing, i.e., must not make
% an \!\insert!ion, after we completed the \lpage, i.e., if
% \CSIndex{ifpcol@output} is $\false$.
% 
% 
% 
% \subsubsection{Changing Default Column Color}
% \label{sec:imp-ovv-color-colcolor}
% \changes{v1.34}{2018/05/07}
% 	{Split the description of \cs{columncolor} from \Sec1.6.2 to have
%	 new \Sec1.6.3 ``Changing Default Column Color'' because we have
%	 serveral new issues in the new implementation with \cs{insert}.}
% 
% The implemntation of \!\columncolor! and \!\normalcolumncolor! is
% relatively easy for the cases that they appear outside
% \env{paracol} environment or they define the default color of a
% column different from the current column.  That is, for the default color
% of a column $c$ we simply \!\def!ine
% ${\Uidx\Celtshadow}^c=|pcol@columncolor|\cdot c$ to let it have what
% \!\current@color! has for the color.  Then, in \beginparacol{} in the
% former case or immediately in the latter, we let
% $\Celt^c=|pcol@coloumncolor@box|\cdot c$ have the coloring \!\special! for
% the color acquiring an \!\insert! from \!\@freelist! if the box is
% $\bot$.
% 
% \SpecialArrayMainIndex{c}{\pcol@columncolor}
% \SpecialArrayIndex{c}{\pcol@columncolor@box}
% 
% On the other hand, when the API commands are to define the default color
% of the current column $c$, we need to place the coloring at the bottom of
% \colorstack{}s in terms of |.tex| and |.dvi|.  That is, for the former we
% have to rewind and reestablish the stack which can be different from
% $\CST^c$ because the API command can follow a page break which \TeX{} does
% not yet find.  Therefore, we maintain a {\em shadow} of $\cst$ namely;
% $$
% {\Uidx\cstshadow}={\Midx{\!\pcol@colorstack@shadow!}}
% =({\Uidx\celtshadow}_1,\celtshadow_2,\ldots,\celtshadow_n)
% =\!\@elt!\Arg{\celtshadow_1}\;\!\@elt!\Arg{\celtshadow_2}\;\cdots\;
%  \!\@elt!\Arg{\celtshadow_n}
% $$
% to which our version of \!\set@color! pushes $\celtshadow_i$ being
% \!\current@color! which the original one defines, while popping is done
% automatically by \TeX's grouping mechanism because pushes are done by
% \!\edef! rather than \!\xdef!.  Then before we \!\def!ine $\Celtshadow^c$
% we rewind
% ${\Uidx\CSTshadow}^c=(\Celtshadow^c,\celtshadow_1,\ldots,\celtshadow_n)$
% putting \!\special! for pop to the main vertical list for each elements,
% and then after the \!\def!initon of $\Celtshadow^c$ we reestablish
% $\CSTshadow^c$ putting coloring \!\special! for each element.
% 
% As for placing $\Celt^c$ at the bottom of $\CST^c$, we must ensure that
% the placement is done for the \colpage{} in which the API command
% belongs to, as we did in ordinary push\slash pop of the \colorstack.
% Therefore the API command \!\insert!s $\Celt^c$ to $\cstraw$ in the form
% of a \!\vbox!, whose height and depth are 1\,|sp| and width is 0,
% containing the coloring \!\special! for $\Celt^c$.  Then when $\cstraw$ is
% scanned for rewinding in \!\output!, this \!\vbox! is found to let
% $\Celt^c$ have the \!\special! acquiring an \!\insert! from \!\@freelist!
% it was $\bot$.  Note that $\cstraw$ may have multiple \!\vbox!es to update
% $\Celt^c$ and if so the last one is effective.
% 
% 
% 
% \subsubsection{Coloring in Math Mode}
% \label{sec:imp-ovv-color-mmode}
% \changes{v1.24}{2013/07/27}
%	{Add the subsection ``Coloring in Math Mode''.}
% \changes{v1.34}{2018/05/07}
% 	{Revise the description of \Sec1.6.4 according to the new
%	 implementation with \cs{insert}.}
% 
% Unfortunately the solution above is imperfect because \TeX{} builds an
% implicit \!\hbox! for a |{|\textit{math stuff}|}| construct in math mode
% and an \!\insert! in the construct does not contribute to
% the main vertical list at all\footnote{
% 
% The contents is not thrown away but \cs{insert}ion itself is added
% to the list rather than given to \cs{output}.}.
% 
% Since the implicit \!\hbox! does not care about \!\everyhbox!, we cannot
% use the trick similar to that with \!\everyvbox!.  Another bad news is
% that built-in \cs{if}s for mode checking cannot help us because we always
% have $\CSIndex{ifvmode}=\CSIndex{ifhmode}=\false$ and
% $\CSIndex{ifmmode}=\true$ while $\CSIndex{ifinner}$ is $\true$ or $\false$
% when we are in in-text or displayed math mode respectively.  Therefore, we
% have to take care of the potential loss of \!\insert!ion for pushes and
% thus unmatched pops in $\cstraw$.
% 
% For example, we have to remember that, in the cases like
% |${|\!\color!$\Arg{c}\textit{text}$|}$| or
% |$|\!\textcolor!$\~\Arg{c}\ARg{text}$|$| expanded to the former, the
% \!\insert!ion for push is lost while its counterpart for pop survives
% making it necessary to check the existance of pushing counterpart for each
% pop in $\cstraw$\footnote{
% 
% Since a pop is always in a group one level outer from its push
% counterpart, the pop request should be presented if the push does.}.
% 
% Note that the fact that the pop in the examples is in the in-text math
% does not help us, because the pop in
% |$|\!\begingroup!\!\color!\Arg{c}\textit{text}\!\endgroup!|$| is also in
% the in-text math while its pushing counterpart performs an effective
% \!\insert!ion, and two \!\insert!ions must be presented in $\cstraw$
% because we can have a page-break in \textit{text}.  Therefore, we have to
% find a means to examine whether a pop $\celtpop_i$ has its counterpart
% $\celt_i$ in $\cstraw$ to remove $\celt_i$ from $\cstraw$ if exists or to
% igonore $\celtpop_i$ otherwise.  That is, we have to attach an identifier
% $m$ to $\celt_i$ and $\celtpop_i$, i.e., to make them $\Uidx\mcelt_{i,m}$
% and $\Uidx\mceltpop_{i,m}$.
% 
% Since the only means we have for the communication with \!\output! routine
% is what we \!\insert! to $\cstraw$, the \!\insert!ed \!\vbox! must carry an
% identifier $m$ for a push\slash pop in math mode.  To do that, we make
% \!\vbox! $m$\,|sp| wide ($m>0$) if our version of \!\set@color! is in math
% mode to represent $\mcelt_{i,m}$ and $\mceltpop_{i,m}$, while the width is
% 0 otherwise as described in \secref{sec:imp-ovv-color-hmode}.  Then in the
% scan of $\cstraw$ for rewinding in \!\output!, we supress
% incrementing\slash decrementing $\npop$ for $\mcelt_{i,m}$ and
% $\mceltpop_{i,m}$, but remove $\mcelt_{i,m}$ if $\mceltpop_{i,m}$ is in
% $\cstraw$ as a successor while we keep it in $\cstraw$ otherwise putting a
% \!\special! of pop for it to the main vertical list.
% 
% To ensure that $\mcelt_{i,m}$ has its counterpart $\mceltpop_{i,m}$ in
% $\cstraw$ iff the push and pop are in a \colpage, we maintain the counter
% \!\pcol@mcid! incremented before (the attempt of) the \!\insert!ion of
% $\mcelt_{i,m}$ with $m=\!\pcol@mcid!$ and the \!\aftergroup! reservaion
% for that of $\mceltpop_{i,m}$.  Then the counter is zero-cleared by
% \!\output! routine in order to keep it less than
% $\!\pcol@mcpushlimit!=1000$ unless, roughly speaking, a \colpage{} has a
% unexpectedly large number of math constructs having coloring commands in
% them.  Note that this zero-clearing does not ensure that an identifier $m$
% is unique in $\cstraw$.  That is, it can happen that $\cstraw$ has
% $\mcelt_{i,m}$, $\mceltpop_{i,m}$, $\mcelt_{j,m}$ and/or $\mceltpop_{j,m}$
% in this order for $i<j$, when two math constructs with coloring for $i$
% and $j$ are in different paragraphs and \!\output! is invoked at or after
% the end of the paragraph with the math for $i$.  This potential
% duplication is, however, unharmful because of the following.
% 
% \begin{itemize}
% \item
% Since a math construct cannot have immediate \!\output! invocations in it,
% the order of the elements in $\cstraw$ must be $\mcelt_{i,m}$,
% $\mceltpop_{i,m}$, $\mcelt_{j,m}$ and $\mceltpop_{j,m}$ from its bottom to
% top, though some of them could be missing.  Therefore, if
% $\mceltpop_{i,m}$ is in $\cstraw$, $\mcelt_{j,m}$ must follow it if
% exsits not causing accidental matching with $\mceltpop_{i,m}$.
% 
% \item
% If $\mcelt_{i,m}$ is in $\cstraw$ but $\mceltpop_{i,m}$ is not, it means
% we have a page break between vertical items corresponding to
% $\mcelt_{i,m}$ and $\mceltpop_{i,m}$ to keep the \!\insert!ion of
% $\mceltpop_{i,m}$ and anything following it from appended into $\cstraw$.
% Therefore, $\cstraw$ cannot have $\mceltpop_{j,m}$ not causing accidental
% matching with $\mcelt_{i,m}$.
% \end{itemize}
% 
% 
% 
% \subsubsection{Emptiness of a Column-Page}
% \label{sec:imp-ovv-color-emptycol}
% 
% The mechanism above works well with respect to coloring, but it has a
% problem that a \colpage{} created by, for example, a forced page break may
% not be perfectly empty but can have some coloring \!\special!s for
% \colorstack{} reestablishing and rewinding.  They are of course invisible
% but affect the examination of \colpage{} emptiness for \exsync.  That is,
% we examine if a \colpage{} does not have anything by a tricky way by
% \!\pcol@ifempty! but the existence of coloring \!\special!s makes the
% examination failed even if no other ordinary stuff such as boxes and skips
% are in the \colpage.
% 
% Therefore we need to put coloring \!\special!s for \colorstack{}
% establishing and rewinding a little bit more carefully to avoid empty
% \colpage{}s just having such \!\special!s as follows.  When we start a new
% \colpage{}, we don't put \!\special!s for establishing immediately but
% save the \colorstack{} $\CST^c$ into
% $\Uidx\csts=\!\pcol@colorstack@saved!$.  Then when we leave the
% \colpage{} by switching or page breaking, we examine the emptiness of the
% \colpage{} and if so we do nothing, while otherwise we put the
% \!\special!s for reestablishing $\csts$ at the top of the \colpage{} and
% those for rewinding $\cstraw$ at the bottom.  Similary, when we revisit a
% \colpage, we examine its emptiness and if so we save $\CST^c$ into
% $\csts$, while otherwise we put \!\special!s for reestablishing $\CST^c$
% and nullify $\csts$ so that nothing will be put at the top of the
% \colpage{} when we leave it.  By these mechanisms, an empty \colpage{}
% should not have coloring \!\special!s, while non-empty ones should have a
% sequence of triples; reestablishing \!\special!s; ordinary main vertical
% list items including coloring \!\special!s inserted by \!\color! etc.; and
% then rewinding \!\special!s.
% 
% 
% 
% \subsection
% [Parallel-Paging, Column-Swapping, Column-Separating Rule
%   Drawing and Background Painting]
% {Parallel-Paging, Column-Swapping, Column-Separating\\Rule
%   Drawing and Background Painting}
% \label{sec:imp-ovv-cswap}
% \changes{v1.2-4}{2013/05/11}
%	{Add the subsection ``Column-Swapping''}
% \changes{v1.3-1}{2013/09/17}
%	{Change the section title from ``Column-Swapping'' to
%	 ``Parallel-Paging, Column-Swapping, Column-Separating Rule Drawing
%	 and Background Painting'' to discuss related issues together''.}
% \changes{v1.3-2}{2013/09/17}
%	{Add overview description of parallel-paging.}
% \changes{v1.3-3}{2013/09/17}
%	{Add overview description of column-separating rule drawing and
%	 background painting.}
% 
% We have the following four extensions, which are correlated to each other,
% from the basic parallel-columning.
% 
% \paragraph{\em Parallel-paging}\UsageIndex{parallel-paging}\hskip-.5em
% to extend the concept of parallel-columning in a page to a pair of
% adjacent pages.  A {\em left} \parapag{}e starts from column-0, has
% $\Uidx\CL$ columns where $\CL$ is given by the first optional argument of
% \beginparacol, while a {\em right} \parapag{}e starts from column-$\CL$
% and has $\C-\CL$ columns.  Since we let $\CL=\C$ when \parapag{}ing is not
% in effect, we may ship out columns $c\In0\CL$ always and then, if
% $\CL<\C$, ship out columns $c\In\CL\C$ as a right \parapag{}e.
% 
% The pair of \parapag{}es can be {\em\Uidx\paired} to comprise a virtual
% page $p$ and thus has common page number $\page(p)$, while
% {\em\Uidx\npaired} \parapag{}ing produces two individual pages from a
% internal page $p$ (i.e., set of all columns $\Set{c}{c\In0\C}$) whose
% left and right components have page numbers $\page(p)$ and $\page(p)+1$
% respectively.  Since a page $p$ is internally considered as the set of all
% columns $c\In0\C$ always, regardless of \paired{} or \npaired{}
% \parapag{}ing, the difference between them arises only in two-sided
% ship-out process in which the header, footer and left-margin are common
% for left\slash right \paired{} \parapag{}s while they have to depend on
% the parity of the number of each \npaired{} \parapag{}e.  Note that
% \textsf{paracol} does not specify the parity of a left \npaired{}
% \parapag{}e number, but the number is decided by the page from which a
% \parapag{}ed \env{paracol} environment starts.
% 
% In ship-out process, we build the ship-out image of a right \parapag{}e in
% our own \!\box! register \!\pcol@rightpage! instead of the usual
% \!\@outputbox!.  The register, however, must {\em survive} after
% \Endparacol{} to keep the columns in the last right \parapag{}e,
% 
% \Index{last page}
% 
% so that it is shipped out when the whole of last page including \postenv{}
% is shipped out, or, more complicatedly, to be passed to the next
% \env{paracol} environment as a part of its \preenv.
% 
% {\em Page-wise stuff}
% \UsageIndex{page-wise stuff}
% spanning all columns, i.e., \spanning{} being \preenv{} or \pwise{}
% floats, \mctext{}s, \scfnote{} footnotes and \postenv{} are always placed
% in a left \parapag{}e, while the corresponding regions for them in a right
% \parapag{}e are always blank\footnote{
% 
% So far.  In some future, we could implement a special setting to let
% \preenv{}, \postenv{} and \scfnote{} footnotes are split into both
% \parapag{}es, and to make it possible that a \pwise{} float or a
% \mctext{} has its counterpart placed in the corresponding right
% \parapag{}e.}
% 
% unless \preenv{} has the \lpage{} of the previous \env{paracol}
% environment.
%
% \paragraph{\em Column-swapping}\UsageIndex{column-swapping}\hskip-.5em
% to reverse the order of columns in even numbered pages from left-to-right
% to right-to-left.  It is enabled by the specifier `|c|' of
% \!\twosided!\footnote{
% 
% Or the backward compatible macro \!\swapcolumninevenpages!.}.
% 
% Though it is fundamentally simple because we just need to reverse the
% scanning order of columns from left-to-right (i.e., 0 to $\C-1$) to
% right-to-left (i.e., $\C-1$ to 0) in the ship-out process of an even
% numbered page, there are a few complications in the implementation of
% related functionalities.
% 
% First, a \paired{} \parapag{}e should also be swapped so that a
% \emph{physical} left (resp.\ right) \parapag{}e has columns $\C-1$ to
% $\CL$ (resp.\ $\CL-1$ to 0) in this order.  Note that this \parapag{}e
% swapping also swaps the page in which \pwstuff{} are placed.  That is, if
% both \paired{} \parapag{}ing and \cswap{} are in effect, \pwstuff{} are
% placed in the physical right \parapag{}e, or in other words they always
% placed in the page in which column-0 resides.  Note that since \cswap{}
% with \npaired{} \parapag{}ing is meaningless and thus \cswap{} is
% disabled.
% 
% Second, the side margin to which a marginal note goes can be swapped but
% enabling this swap is independent of \cswap{} and done by the specifier
% `|m|' of \!\twosided!, though almost all users will specify both swapping
% consistently.  Since the side margin for a marginal note is decided in
% \!\output! routine by \!\pcol@addmarginpar! being our own version of
% \LaTeX's macro for maginal notes \!\@addmarginpar!, the page in which the
% margial note resides has been fixed.  However, the number of the page and
% thus its parity may not have been fixed yet due to the possible jump in
% column-0 taking place afterward, unlike \cswap{} for which the page number
% has been fixed because it is performed in ship-out process.  Since it is
% too costly to avoid this possibly wrong placement, we have to accept the
% possibility as \LaTeX{} itself does.  Also unlike \cswap, the swapping of
% marginal notes is not disabled in \npaired{} \parapag{}ing because it is
% meaningful.
% 
% Another remark for marginal notes is that two ore more columns may {\em
% share} a margin, inevitably if a (parallel) page has three or more columns
% or intentionally with a setting of \!\marginparthreshold!.  Therefore, the
% context of marginal notes cannot be in \cctext{} but should be in \pctext,
% or cannot simply give the bottom of the last marginal note (i.e., \LaTeX's
% \!\@mparbottom!) but should show all marginal notes in margins in a
% page\footnote{
% 
% Before version 1.3, we have \!\@mparbottom! in \cctext{} because a column
% has its own area for marginal notes, which can be the gap between columns
% rather than a margin of a page.}.
% 
% Therefore, each \pctext{} has all marginal notes in the form of lists of
% their top and bottom positions in all margins as $\pp^m(p)$, so that we
% find a space for a marginal note in a column to add it to not only to the
% bottom but also into a space between two marginal notes having already
% been put by other columns.
% 
% Third and finally, we have to take care the placement of \mctext{}s.  In
% version 1.2 to which \cswap{} is introduced, we let a \mctext{} belong to
% column-$(\C-1)$ instead usual column-0 so that its left edge is aligned to
% the left edge of the leftmost column, i.e., that of the text area.
% However this simple solution has a severe problem that, if a \mctext{} is
% broken into two pages, its second half should be put in the rightmost
% column.  In addition, even when a \mctext{} does not have page break in
% it, such wrong placement may happen if the text is followed by \!\nobreak!
% and thus a page break is made above the text but {\em after} the text is
% processed.
% 
% In version 1.3, this problem is solved by capturing the first half of a
% \mctext{} in \!\output! routine for the page break in the text, and the
% second half or the whole of it in that for \sync{}ed \cswitch{} to close
% the text.  Since an invocation of \!\output! routine means that it has been
% fixed which page the \mctext{} or its part resides in, we can place the
% text much more reliably expecting the parity of the page number has also
% been fixed.  In addition, this decision making in \!\output! routine allows
% (or forces) us to let \mctext{}s always belong to column-0 preserving the
% consistency of, for example, local counter values referred to in them,
% while we need to shift a text to the left edge of the text area if it
% resides in an even numbered page.  Furhtermore, this \mctext{} capturing
% enables to measure the vertical size of the captured text together with the
% vertical position of its top edge to record them in the list $\pp^s(p)$,
% so that we draw \cseprule{}s skipping the text and painting its \bground{}
% with a specific color different from colors of columns and \csepgap{}s, as
% discussed shortly.
% 
% \paragraph{\em Column-separating rule drawing}
% \UsageIndex{column-separating rule}\hskip-.5em
% to draw a vertical rule in {\em\Uidx\csepgap} is correlated with a part of
% {\em\bgpaint} to paint each region in a page with a color specific to the
% region.  Thanks to the list of \mctext{}s $\pp^s(p)=(\spt(H_i,h_i))_i^n$, we
% can draw \cseprule{}s skipping \mctext{}s in the page $p$ as the sequence
% of;
% 
% \begingroup
% \def\RULE{\mathit{rule}}\def\GAP{\mathit{gap}}
% \begin{eqnarray*}
% &&\RULE(H'_1),\GAP(h_1),\ldots,\RULE(H'_n),\GAP(H'_n),\RULE(H'_{n+1})\\
% &&H'_i=H_i-(H_{i-1}+h_{i-1})
% \qquad H_0=h_0=0
% \qquad H_{n+1}=\pp^h(p)
% \end{eqnarray*}
% \endgroup
% 
% where $\mathit{rule}(H')$ is a vertical rule of $H'$ high and
% $\mathit{gap}(h)$ is a vertical space of $h$.  A rule may be colored with
% the color specified by \!\colseprulecolor! for each \csepgap{} or all
% of them.  Note that if \cswap{} is in effect, a column $c$ is {\em
% preceded} by $c$-th \csepgap{} which may have its own width and color for
% its rule, rather than being followed by it.
% 
% \paragraph{\em Background painting}\UsageIndex{background painting}\hskip-.5em
% also uses the list $\pp^s(p)$ to paint the \bground{} of each column-$c$
% with the color $\Uidx\bgc_c^c$, each \csepgap{} following the column-$c$
% with $\bgc_g^c$, and \mctext{}s with $\bgc_s$ and $\bgc_S$, where
% $\bgc_a^{[c]}$ is specified by the second argument of
% $\!\backgroundcolor!\Arg{\hbox{\ttfamily\itshape a}|[|c|]|}\~\Arg{color}$
% ($\hbox{\ttfamily\itshape a}\in\{|c|,|g|,|s|,|S|\}$) and kept in the macro
% $|\pcol@bg@color@|{\cdot}a[|@|{\cdot}c]$.
% 
% \SpecialArrayMainIndex{a}{\pcol@bg@color@}
% \SpecialArrayMainIndex{a{\cdot}\string\texttt{@}{\cdot}c}{\pcol@bg@color@}
% 
% The region to be painted for each item is as follows where
% $[(x_0,y_0)(x_1,y_1)]$ means the region
% $\Set{(x,y)}{x\In{x_0}{x_1},\;y\In{y_0}{y_1}}$ of the top-down
% $xy$-coordinate whose origin is at the left-top corner of the leftmost
% column.
% 
% \begingroup\def\sS{{\{s,S\}}}
% \begin{eqnarray*}
% \Uidx\bgr_c^c(i)&=&[(\Uidx\W_c,\;H_{i-1}+h_{i-1})\;
%   (\W_c+w_c,\;H_i+d_{c/g})]\\
% \bgr_g^c(i)&=&[(\W_c+w_c,\;H_{i-1}+h_{i-1})\;(\W_{c+1},\;H_i+d_{c/g})]\\
% \bgr_\sS(i)&=&[(0,\;H_i)\;(\Uidx\WT,\;H_i+h_i+d_s)]\\
% \W_{c}&=&\sum_{d=c_0}^{c-1}(w_c+g_c)\qquad
%     c_0=\cases{0&$c<\CL$\cr
%		 \CL&$c\geq\CL$}\qquad
%     \WT=\!\textwidth!\\
% d_{c/g}&=&\cases{\!\maxdepth!&$i=n+1\;\land\;H'_{n+1}>0\;\land\;$non-\lpage\cr
%                0&otherwise}\\[1ex]
% d_s&=&\cases{H_{n+1}-(H_n+h_n)+\!\maxdepth!&
%                $i=n\;\land\;H'_{n+1}=0\;\land\;$non-\lpage\cr
%              0&otherwise}
% \end{eqnarray*}
% \endgroup
% 
% In the specifications above, $\w_c$ and $\Uidx\gap_c$ is the width of the
% column $c$ and that of the \csepgap{} following it, defined by
% \!\columnratio! or \!\setcolumnwidth! and stored in
% $|\pcol@columnwidth|{\cdot}c$ and $|\pcol@columnsep|{\cdot}c$
% respectively.
% 
% \SpecialArrayIndex{c}{\pcol@columnwidth}
% \SpecialArrayMainIndex{c}{\pcol@columnsep}
% 
% The additions of $d_{c/g}$ and $d_s$ are to extend the bottom edge of each
% region down to the bottom of text area.  In addition, for each
% $\bgr_a^{[c]}=[(x_0,y_0)(x_1,y_1)]$, {\em\Uidx\bgext}s
% $\Uidx{\bge}_a^{[c]}(\{x,y\}^\pm)$ can be specified to shift the base
% points $x_0$, $y_0$, $x_1$ and $y_1$ left ($x^-$), right ($x^+$), upward
% ($y^-$) and downward ($y^+$) respectively.  That is, a region is defined
% as;
% $$
% \bgr_a^{[c]}=[(x_0-\bge_a^{[c]}(x^-),\;y_0-\bge_a^{[c]}(y^-))\;
%               (x_1+\bge_a^{[c]}(x^+),\;y_1+\bge_a^{[c]}(y^+))]
% $$
% \begingroup\par\hfuzz0.9pt\noindent
% with the optional shifts specified by the first argument of
% \!\backgroundcolor!  as
% \Arg{\hbox{\ttfamily\itshape{a}}\oarg{c}$|(|x^\pm|,|y^\pm|)|$} (for both
% $x^-$/$y^-$ and $x^+$/$y^+$) or \Arg{\hbox{\ttfamily\itshape
% a}\oarg{c}$|(|x^-|,|y^-|)||(|x^+|,|y^+|)|$} and kept in macros
% $|\pcol@bg@ext@|{\cdot}\~d{\cdot}|@|{\cdot}a[{\cdot}|@|{\cdot}c]$
% 
% \SpecialArrayIndex{d{\cdot}\string\texttt{@}{\cdot}a}{\pcol@bg@ext@}
% \SpecialArrayIndex
%   {d{\cdot}\string\texttt{@}{\cdot}a{\cdot}\string\texttt{@}{\cdot}c}
%   {\pcol@bg@ext@}
% 
% where $d\in\{|l|,|r|,|t|,|b|\}$ for $x^-$ (|l|), $x^+$ (|r|), $y^-$ (|t|)
% and $y^+$ (|b|).  Note that $\bge_a^{[c]}(\{x,y\}^\pm)$ can be extremely
% large, namely greater than or equal to 9000pt, to mean the region is
% extended to a border near by the corresponding paper edge.  More
% specifically, by this {\em\Uidx\bginfext}, each $xy$ coordinate in
% $[(x_0,y_0)(x_1,y_1)]$ is defined as follows to represent a coordinate
% being $10000\PT-\bge_a^{[c]}(\{x,y\}^\pm)+\!\pagerim!$ inside from the page
% edge;
% \par\endgroup
% 
% \begingroup \def\feven{f_{\mathit{even}}}
% \begin{eqnarray*}
% x_0&=&-\WM+(10000\PT-\bge_a^{[c]}(x^-)+\WR)\\
% y_0&=&-(\HS+\HM)+(10000\PT-\bge_a^{[c]}(y^-)+\HR)\\
% x_1&=&(\WP-\WM)-(10000\PT-\bge_a^{[c]}(x^+)+\WR)\\
% y_1&=&(\HP-\HS-\HM)-(10000\PT-\bge_a^{[c]}(y^+)+\HR)\\
% \Uidx\WP&=&\!\paperwidth!\\
% \Uidx\WM&=&1|in|+\cases{\!\oddsidemargin!&$\lnot\feven$\cr
%                         \!\evensidemargin!&$\feven$}\\
% \Uidx\WR&=&\Uidx\HR=\!\pagerim!\\
% \Uidx\HP&=&\!\paperheight!\\
% \Uidx\HM&=&\!\headsep!+\!\headheight!+\!\topmargin!+1|in|\\
% \Uidx\HS&=&\mathit{height}(\pp^b(p))+\mathit{depth}(\pp^b(p))
% \end{eqnarray*}
% \endgroup
% 
% where $f_{\mathit{even}}$ is $\true$ iff we are in an even numbered page
% and two-sided typesetting is specified by the optional argument of
% \!\documentclass! or by the specifier `|p|' of \!\twosided! explicitly or
% implicitly.
% 
% Another remark is that \cswap{} affects $\bgr_c^c(i)$ and $\bgr_s^c(i)$ to
% {\em\Uidx\mirror} the region making a reflection-symmetric transformation
% on it using a vertical edge of a page as the axis.  That is,
% $[(x_0,y_0)(x_1,y_1)]$ for a region is transformed to
% $[(\WT-x_1,y_0)(\WT-x_0,y_1)]$ if $x_{\{0,1\}}$ is not extended infinitely.
% With \bginfext{} on the other hand, before this transformation $x_0$ and/or
% $x_1$ are calculated by the rule above replacing $\WM$ with
% $\WP-(\WM+\WT)$ to represent the width of the right margin rather than the
% left.
% 
% On the other hand, the \mirror{}ing of $\bgr_{\{s,S\}}$ is enabled by the
% specifier `|b|' of \!\twosided!, together with other regions being top
% margin ($t,T$), bottom margin ($b,B$), left margin ($l,L$), right margin
% ($r,R$), \pwise{} floats ($f,F$) and \scfnote{} footnotes ($n,N$).  The
% geometical specifications $\bgr_a$ for those regions are given as follows,
% but the coordinate origin is at the top-left corner of text area (rather
% than the leftmost column).
% 
% \begin{eqnarray*}
% \bgr_{\{t,T\}}&=&[(-\WM+\WR,\;-\HM+\HR),\;(\WP-\WM-\WR,\;0)]\\
% \bgr_{\{b,B\}}&=&[(-\WM+\WR,\;\HT-\HM),\;(\WP-\WM-\WR,\;\HP-\HM-\HR)]\\
% \bgr_{\{l,L\}}&=&[(-\WM+\WR,\;0),\;(0,\HT)]\\
% \bgr_{\{r,R\}}&=&[(\WT,\;0),\;(\WP-\WM-\WR,\;\HT)]\\
% \bgr_{\{f,F\}}&=&[(0.\;0),\;(\WT,\;\HS)]\\
% \bgr_{\{n,N\}}&=&[(0,\;\HT-\HN)\;(\WT,\;\HT)]\\
% \Uidx\HT&=&\!\textheight!+\!\maxdepth!\\
% \Uidx\HN&=&\!\skip!\!\footins!+\!\ht!\!\footins!+\!\dp!\!\footins!
% \end{eqnarray*}
% 
% Note that, since we use text area coordinates, in the calculation of
% \bginfext{} $\HS$ is let be 0.
% 
% We have other regions for columns and \csepgap{}s, namely $\bgr_C^c$ and
% $\bgr_G^c$, which vertically span all over text area regardless exsistance
% of any \pwstuff.  Therefore, their geometical specifications are as
% follows with text area coordinates.
% 
% \begin{eqnarray*}
% \bgr_C^c&=&[(\W_{c-1},\;0),\:(\W_{c-1}+w_c,\;\HT)]\\
% \bgr_G^c&=&[(\W_{c-1}+w_c,\:0),\;(\W_c,\;\HT)]
% \end{eqnarray*}
% 
% In addition, we have to paint \preenv{} and \postenv{} with color
% $\bgc_{\{p,P\}}$.  The region $\bgr_{\{p,P\}}$ for them is defined as
% follows with text area coordinates where $\Uidx\HB$ is the $y$-coordinate
% of the bottom of previous \env{paracol} environment if any, or 0
% otherwise.
% $$
% \bgr_{\{p,P\}}=\cases{
%	[(0,\;\HB)\;(\WT,\;\HS)]&	\preenv\cr
% 	[(0,\;\HB)\;(\WT,\;\HT)]&	\postenv}
% $$
% Note that paiting of \postenv{} is done {\em outside} \env{paracol}
% environment when the \postenv{} encounters a page break, unless another
% \env{paracol} environment starts in the page and thus the \postenv{}
% becomes \preenv{} of the second (or subsequent) environment.
% 
% Finally, we define the order of \bgpaint{} as follows, where $a$, $a(i)$,
% $a^c$ and $a^c(i)$ mean $\bgr_a$, $\bgr_a(i)$, $\bgr_a^c$ and
% $\bgr_a^c(i)$ respectively, so that a succeding region is {\em overlaid}
% on preceding regions.
% 
% \begin{eqnarray*}
% T&\to&B\to L\to R\\
% &\to&G^0\to\cdots\to G^{\C-2}\to C^0\to\cdots\to C^{\C-1}\\
% &\to&t\to b\to l\to r\to N\to n\to\{F,P\}\to\{f,p\}\footnotemark\\
% &\to&S(1)\to\cdots\to\S(n)\\
% &\to&g^0(1)\to\cdots\to g^{\C-2}(1)\to c^0(1)\to\cdots\to c^{\C-1}(1)\to
% 	s(1)\\
% &\to&\cdots\\
% &\to&g^0(n)\to\cdots\to g^{\C-2}(n)\to c^0(n)\to\cdots\to c^{\C-1}(n)\to
% 	s(n)\\
% &\to&g^0(n{+}1)\to\cdots\to g^{\C-2}(n{+}1)
%  \to c^0(n{+}1)\to\cdots\to c^{\C-1}(n{+}1)
% \end{eqnarray*}
% 
% \footnotetext{In column flushing, the order is $\{F,P\}\to\{f,p\}\to N\to
% n$ but this reversion should have no effect (almost always).}
%
% 
% 
% \subsection{Page-wise Float Placement}
% \label{sec:imp-ovv-float}
% \changes{v1.32-3}{2015/10/10}
% 	{Add the section ``Page-wise Float Placement'' to discuss the
%	 page-wise float problem.} 
% 
% In the release on 2015/01/10, \LaTeX{}'s float placement mechanism was
% drastically changed to avoid {\em out-of-order\/} appearance of \pwise{}
% floats as follows.  Since the cause of overtaking of a \pwise{} float by a
% \cwise{} float is that they are in two separated lists \!\@dbldeferlist!
% for the former and \!\@deferlist! for the latter, in the new implemenation
% the two lists are merged to let all floats go to \!\@deferlist!.  To
% distinguish \pwise{} and \cwise{} floats in the list, \!\end@dblfloat!
% lets the \pwise{} float processed by the macro have a special depth of
% |1sp|, while depth of \cwise{} floats are 0 since \!\@endfloatbox! add a
% \!\vskip! of 0 at the end of the box of floats.
% 
% Then all float placement macros invoked in \!\output!-routine examine the
% depth of floats in the list they are working on against a newly introduced
% macro \!\f@depth! by also newly introduced \!\@testwrongwidth!, so that
% they process only floats of a page/column category specified by
% \!\f@depth!, while those not matching to \!\f@depth! are let go to
% \!\@deferlist! to inhibit succeeding floats of the same type from
% overtaking them.  The \!\def!inition of \!\f@depth! is done only by
% modified \!\@dblfloatplacement!, always invoked in a group, to let it have
% |1sp| so that float placement macros usually work on \cwise{} ones with
% the default setting of $\!\f@depth!=\!\z@!$ except for special occasions
% in which the placement of \pwise{} floats is tried.
% 
% Though the mechanism should work well with {\em ordinary\/} multi-columned
% documents, it is incompatible with \Paracol{} almost inherently.  That is,
% in the first place we have to separate float-related lists into the sets
% of them corresponding to columns as we do\footnote{
% 
% If counters \counter{figure} and \counter{table} are global and we have to
% avoid inter-column overtaking with respect to, for example, the partial
% ordering rooted by the top-left corner, merging \cwise{} lists together
% with the merge of \cs{@deferlist} and \cs{@dbldeferlist} might be a
% solution to let the depth of a \cwise{} float be $c\,\texttt{sp}$ while
% that of \pwise{} is $\C$\,\texttt{sp}.  However such implementation is not
% only tough but also doubtful to be worthwhile.}.
% 
% Therefore, it is obviously nonsense to merge the list for  \pwise{} floats,
% i.e., \!\@dbldeferlist!, to \!\@deferlist! of a particular column, and
% thus we have to stick with the conventional implementation to process
% \pwise{} and \cwise{} floats separately as follows.
% 
% \begin{enumerate}\def\labelenumi{(\arabic{enumi})}
% \item
% \label{item:ovv-float-end@dblfloat}
% We \!\def!ine our own \!\end@dblfloat! namely \!\pcol@end@dblfloat!
% whose definition is exactly same as the old version of \!\end@dblfloat!,
% and replace the new version with it by \!\let!ting them equivalent in
% \!\pcol@zparacol! by which start \env{paracol} environments.  Therefore,
% \pwise{} floats composed in a \env{paracol} environment is processed in
% the traditional way, i.e., being inclueded in \!\@dbldeferlist! rather than
% \!\@deferlist! and having ordinary depth 0.
% 
% \item
% \label{item:ovv-float-@dblfloatplacement}
% Each invocation of \!\@dblfloatplacement! in our own \!\output!-routine is
% followed by a \!\let! to nullify the setting of $\!\f@depth!=|1sp|$ done
% by \!\@dblfloatplacement! by doing \!\f@depth!=\!\z@!.  By this setting,
% \!\@tryfcolumn! in \!\pcol@startpage!  and \!\@makefcolumn! in
% \!\pcol@output@clear! work on their argument \!\@dbldeferlist! in the way
% exactly same as in 2014 or before.
% 
% \item
% \label{item:ovv-float-@addtodblcol}
% Among \LaTeX's macros in its \!\output!-routine which we use in our own
% one as well, only \!\@addtodblcol! changed its target from
% \!\@dbldeferlist! to \!\@deferlist!.  That is, if the macro fails to put a
% \pwise{} float to the page we just have started by \!\pcol@startpage!,
% the float is added to \!\@deferlist! rather than \!\@dbldeferlist!.
% Therefore, when we apply \!\@sdblcolelt! to the copy of \!\@dbldeferlist!
% to invoke \!\@addtodblcol! for each of its element float, we have to save
% \!\@deferlist! somewhere, to \!\reserved@c! in reality, and clear it prior
% to the application.  Then after all elements are processed, we have to let
% \!\@dbldeferlist! have what \!\@deferlist! have, while \!\@deferlist!
% should regain its origial contents from the saved place.  A subtle issue
% is that we might work with \LaTeX{} of 2014 or older in which the floats
% are returned to \!\@dbldeferlist!.  Therefore to make \Paracol{}
% compatible with both of new and old versions, we have to {\em add}
% \!\@deferlist! to \!\@dbldeferlist! rather than replacing
% \!\@dbldeferlist! with \!\@deferlist!.  This addition should work well,
% because we clear both lists before the application of \!\@sdblcolelt!  and
% then one of them will have the still-deferred floats after the application
% while the other remains empty.
% 
% \item
% \label{item:ovv-float-@output@start}
% We convert \!\@deferlist! to \!\@dbldeferlist! in \!\pcol@output@start! to
% start a \env{paracol} environment, and perform the reverse operation in
% \!\pcol@output@end! to close the environment.  Though it is very unlikely
% (or maybe impossible) that the \!\@deferlist! {\em imported\/} in the
% former operation has \LaTeX's (i.e., not \Paracol's) double-column floats
% of |1sp| deep, we make such floats old-fashioned making their depth 0 so
% that they can be put in a page built in the \env{paracol} environment.  On
% the other hand, the latter cannot {\em export\/} a list having floats of
% |1sp| deep because they have been produced in the closing \env{paracol}
% environment or have passed our {\em custom\/} \!\pcol@output@start! when
% they were imported\footnote{
% 
% Therefore, if one try to {\em smuggle\/} a double-column float of the new
% scheme into a \env{paracol} and to pass it through the environment to
% another double-column world, the float will become a single-column one.
% Even if such guy a really exists and complains abount this transformation,
% however, we have good right to say ``don't do that''.}.
% \end{enumerate}
% 
% Note that the operations (1), (2) and (4) are fully compatible with 2014
% or older version of \LaTeX{}, because with the old version; (1) 
% \!\pcol@end@dblfloat! is equivalent to \!\end@dblfloat!; (2) modification
% of \!\f@depth! cannot be seen because it does not exist;  and (4) we
% virtually do nothing in the importation.  As for (3), we explicitly take
% care of the compatibility as shown above.
% 
% 
% 
% \section{Interaction with \TeX{} and \LaTeX{}}
% \label{sec:imp-tex}
% 
% The macros of \Paracol{} interacts with \TeX{} and \LaTeX{} through
% various registers and macros as discussed in this section.
% 
% \subsection{Registers}
% \label{sec:imp-tex-reg}
% 
% \subsubsection{Insertion Registers}
% 
% \begin{description}
% \item[\Uidx{\!\footins!}]
% is used to \!\insert! footnotes through it by \!\footnote! and
% \!\footnotetext!, and then in \!\output! routine the footnotes
% \!\insert!ed in a page is presented in the register.  The register is
% referred to by the following macros.
% 
% \begin{itemize}
% \item
% \!\pcol@makecol! examines if the register has \Scfnote{}s and, if so,
% saves it into $\pp^f(p)$ if $p=\ptop$ or discards it otherwise.
% 
% \item
% \!\pcol@startcolumn! \!\insert!s $\pp^f(p)$ into the \colpage{} to be
% created through the register.
% 
% \item
% \!\pcol@specialoutput! logs the contents of the register for debugging.
% 
% \item
% \!\pcol@output@start! examines if the register has footnotes to be merged
% with those in \env{paracol} environment, refers to its height plus depth to
% calculate effective \!\@colht!, and/or \!\insert!s its contents through
% itself to the main vertical list as the first part of \Mgfnote{}s.
% 
% \item
% \!\pcol@makenormalcol! combines footnotes in the register to other
% \preenv{} to make a \spanning, or makes save\slash restore of the register
% to/from \!\@tempboxa! to exclude footnotes from \spanning{} when
% \mgfnote{}-footnote typesetting is specified.
% 
% \item
% \!\pcol@output@switch! saves the register into $\pp^f(p)$ or $\cc_c(\ft)$,
% or discards its contents, when we leave from the column $c$ with footnotes.
% 
% \item
% \!\pcol@restartcolumn! restores $\cc_d(\ft)$ or $\pp^f(p)$ to the register
% and then \!\insert!s the contents of \!\box!\!\footins! into itself so
% that it contributes to the main vertical list to be rebuilt for the column
% $d$.
% 
% \item
% \!\pcol@getcurrfoot! for column $d$ lets the register have $\cc_d(\ft)$.
% 
% \item
% \!\pcol@savefootins! saves the register into an \!\insert! for $\pp^f(p)$
% or $\cc_c(\ft)$.
% 
% \item
% \!\pcol@deferredfootins! refers the \!\skip! component of the register to
% have the vertical skip above \Scfnote{}s and \!\insert!s deferred
% footnotes through the register.
% 
% \item
% \!\pcol@output@end! \!\insert!s $\pp^f(p)$ into the \lpage{} through the
% register.
% 
% \item
% \!\pcol@fntexttop!\marg{text} \!\insert!s the footnote $\arg{text}$ and a
% penalty through the register.
% 
% \item
% \!\pcol@fntextbody!\marg{text} refers to the \!\skip! component of the
% register to cap the height of the footnote $\arg{text}$.
% \end{itemize}
% 
% \item[\rm\Uidx{\!\bx@A!}, \ldots, \Uidx{\!\bx@R!}]
% have floats created by \!\@xfloat! in the ordinary usage of {\em
% fundamental\/} \LaTeX{} of 2014 or earlier or that without the extension
% of e-\TeX{} or its relatives.  On the other hand, in \LaTeX{} of 2015 or
% later and with e-\TeX{} or its relatives, the set is \!\bx@A!, \ldots,
% \Uidx{\!\bx@Z!}, \Uidx{\!\bx@AA!}, \ldots, \Uidx{\!\bx@ZZ!}\@.  In addtion
% to the use in \LaTeX, we use these registers for completed \colpage{}s
% $s_c(p)$ (\!\pcol@opcol!, \!\pcol@flushcolumn!), main vertical list
% $\cc_c(\vb)$ (\!\pcol@output@start!, \!\pcol@output@switch!,
% \!\pcol@flushcolumn!)  and \Mcfnote{}s $\cc_c(\ft)$
% (\!\pcol@output@switch!) in \ccolpage{}s, \spanning{} including \preenv{}
% $\pp^i(p)$ (\!\pcol@startpage!, \!\pcol@output@start!) and \Scfnote{}s
% $\pp^f(p)$ (\!\pcol@makecol!, \!\pcol@output@switch!) in pages,
% \mvlfloat{}s for main vertical lists in \sync{}ed pages
% (\!\pcol@synccolumn!), and \pwise{} floats deferred from \env{paracol} to
% \postenv{} (\!\pcol@output@end!).
% \end{description}
% 
% 
% 
% \subsubsection{Integer Registers}
% 
% \begin{description}
% \item[\Uidx{\!\deadcycles!}]
% is \TeX's primitive register to count the number of \!\output! requests
% made between two \!\shipout! operatios.  It is zero-cleared by
% \!\pcol@invokeoutput! because it can have a large number in a
% \env{paracol} environment.
% 
% \item[\Uidx{\!\outputpenalty!}]
% is \TeX's primitive register to have the page-break penalty with which
% \!\output! routine is invoked.  It is referred to by \!\pcol@output! to
% know whether it has special code less than $-10000$, and by
% \!\pcol@specialoutput! in detail for the dispatch according to the code.
% The register is also used for the communication from the latter, which lets
% it be $-10000$ for our own special \!\output! routines, to the former to
% determine \!\vsize! according to if the register has a value greater than
% $-10004$ or not.
% 
% \item[\Uidx{\!\interlinepenalty!}]
% is \TeX's primitive register to have the page-break penalty inserted
% between two lines.  The register is referred to in the following macros.
% 
% \begin{itemize}
% \item
% \!\pcol@output@start!  to make \Preenv{} \Mgfnote{}s followed by this
% \!\penalty! on the \!\insert!ion, and to insert it to start the first
% \colpage{} allowing page-break before the start unless it is inhibited
% by $\CSIndex{if@nobreak}=\true$.
% 
% \item
% \!\pcol@restartcolumn! to insert this \!\penalty! to resume a \colpage{}
% allowing page-break if $\CSIndex{if@nobreak}=\false$.
% 
% \item
% \!\pcol@deferredfootins!  to let the secnod half of split $\df$ have this
% \!\penalty! as the very first element.
% 
% \item
% \!\pcol@fntexttop!\marg{text} to make the footnote $\arg{text}$ followed
% by this \!\penalty! on the \!\insert!ion.
% 
% \item
% \!\pcol@fntextother!\marg{text} to make the footnote $\arg{text}$ preceded
% by this \!\penalty! in $\df$.
% 
% \item
% \!\pcol@fntextbody! to let the register have \!\interfootnotelinepenalty!.
% \end{itemize}
% 
% \item[\Uidx{\!\floatingpenalty!}]
% is \TeX's primitive register to have the page-break penalty added to
% \!\insertpenalties! if an \!\insert! is moved to the page next to the page
% in which the line having the \!\insert! resides.  It is let have 20000 in
% \!\pcol@fntextbody! for footnote typesetting.
% 
% \item[\Uidx{\!\vbadness!}]
% is \TeX's primitive register to have the threshold of the badness of
% \!\vbox! construction with underfull messages.  That is, if the badness
% exceeds the threshold on a \!\vbox! construction, \TeX{} will complain
% showing an underfull message. In \!\pcol@makenormalcol!  and
% \!\pcol@deferredfootins!, the register is temporarily let have 10000 to
% avoid that \!\@makecol! invoked in the former and \!\vsplit! done in the
% latter causes the message with inevitable underfull.
% 
% \item[\Uidx{\!\showboxdepth!}]
% is \TeX's primitive register to determine the maximum depth of box
% structure to be shown in logging etc.  The register is let have 10000 in
% \!\pcol@ShowBox!  for full logging.
% 
% \item[\Uidx{\!\showboxbreadth!}]
% is \TeX's primitive register to determine the maximum breadth of box
% structure to be shown in logging etc.  The register is let have 10000 in
% \!\pcol@ShowBox!  for full logging.
% 
% \item[\Uidx{\!\interfootnotelinepenalty!}]
% is an API \!\count! register to have \!\interlinepenalty! for footnotes.
% It is used in \!\pcol@fntextbody! to let \!\interlinepenalty! have it.
% 
% \item[\Uidx{\!\@ne!}]
% is a \!\chardef! register to have 1.  The register is referred to by the
% following macros mainly for incrementing another register.
% 
% \begin{quote}\raggedright
% \!\pcol@F@count!,
% \!\pcol@output!,
% \!\pcol@opcol!,
% \!\pcol@setpnoelt!,
% \!\pcol@nextpage!,
% \!\pcol@nextpelt!,
% \!\pcol@startpage!,
% \!\pcol@checkshipped!,
% \!\pcol@outputelt!,
% \!\pcol@ioutputelt!,
% \!\@outputpage!,
% \!\pcol@bg@paint@ii!,
% \!\pcol@output@start!,
% \!\pcol@makenormalcol!,
% \!\pcol@output@switch!,
% \!\pcol@setcurrcol!,
% \!\pcol@iscancst!,
% \!\pcol@addmarginpar!,
% \!\pcol@do@mpbout@i!,
% \!\pcol@sync!,
% \!\pcol@flushcolumn!,
% \!\pcol@measurecolumn!,
% \!\pcol@synccolumn!,
% \!\pcol@makeflushedpage!,
% \!\pcol@imakeflushedpage!,
% \!\pcol@iflushfloats!,
% \!\pcol@freshpage!,
% \!\pcol@output@end!,
% \!\pcol@invokeoutput!,
% \!\pcol@zparacol!,
% \!\pcol@setcolwidth@r!,
% \!\pcol@setcolwidth@s!,
% \!\pcol@setcw@scan!,
% \!\pcol@setcw@calcf!,
% \!\pcol@synccounter!,
% \!\pcol@com@syncallcounters!,
% \!\pcol@stepcounter!,
% \!\pcol@com@switchcolumn!,
% \!\pcol@sptext!,
% \!\pcol@visitallcols!,
% \!\pcol@ifootnote!,
% \!\pcol@ifootnotemark!.
% \!\pcol@swapcolumn!,
% \!\pcol@set@color@push!,
% \end{quote}
% 
% \item[\Uidx{\!\tw@!}]
% is a \!\chardef! register to have 2.  It is used in \!\pcol@setcurrcol! to
% let $\cc_c(\sw)=2$ when $\CSIndex{if@nobreak}=\true$ but
% $\CSIndex{if@afterindent}=\false$, in
% $\!\pcol@setcw@calcf!\<x\>\<y\>\<z\>$ to calculate $x\cdot2^k$, $y/2^k$
% and $(x/y)\cdot2^k$ with various $k$, and in \!\pcol@swapcolumn! to
% calculate $\Cto-(c'-\Cfrom)-2=c-1=c^g$ for the \csepgap{} ordinal $c^g$
% physically following the column $c$ at the position $c'$.
% 
% \item[\Uidx{\!\m@ne!}]
% is a \!\count! register to have $-1$.  It is used in the following macros.
% 
% \begin{itemize}
% \item
% \!\pcol@setpnoelt!, \!\pcol@nextpelt!, \!\pcol@getpelt! and
% \!\pcol@setmpbelt! to decrement
% \!\@tempcnta! which initially has $p-\pbase$ for a page $p$.
% 
% \item
% \!\pcol@bg@paint@i! to decrement $\CBto$ by one locally to have the column
% scanning range $\LBRP\CBfrom{\CBto{-}1}$.
% 
% \item
% \!\pcol@iscancst! to decrement $\npop$ by one.
% 
% \item
% \!\pcol@do@mpbout@i! to let \!\@tempcnta! have it to indicate left margin.
% 
% \item
% \!\pcol@setcolwidth@r! to calculate $\Cto-\Cfrom-1$.
% 
% \item
% $\!\pcol@setcw@calcf!\<x\>\<y\>\<z\>$ in \!\@whilenum! loops to
% calculate $z'/2^k$ and $z'/2{k-16}$ where $z'/2^k\approx x/y$.
% 
% \item
% \!\pcol@iadjustfnctr! to decrement \!\c@footnote!.
% \end{itemize}
% 
% \item[\Uidx{\!\sixt@@n!}]
% is a \!\chardef! register to have 16.  It is used in
% $\!\pcol@setcw@calcf!\<x\>\<y\>\<z\>$ to calculate
% $Z=z\times1\,|pt|=z'\cdot2^{16-k}$ where $z'/2^k\approx x/y$.
% 
% \item[\Uidx{\!\@m!}]
% is a \!\mathchardef! register to have 1000.  It is used in
% \!\pcol@synccolumn! and \!\pcol@output@end! to let
% $\!\prevdepth!=1000\,|pt|$ on a \sync{}ation or the closing \env{paracol}
% environment with an empty main vertical list, and in
% \!\pcol@setcw@getspec@i! to add 1000\,|pt| to strech and shrink
% compontents of \!\@tempskipa! having a column\slash gap sepcification to
% make it sure the skip register has those components.
% 
% \item[\Uidx{\!\@M!}]
% is a \!\mathchardef! register to have 10000.  It is used in the following
% macros
% 
% \begin{itemize}
% \item
% \!\pcol@ShowBox! to let \!\showboxdepth! and \!\showboxbreadth!  have
% 10000 for full logging of a box.
% 
% \item
% \!\pcol@output! to examine if $\!\outputpenalty!<-10000$ to mean a special
% \!\output! request.
% 
% \item
% \!\pcol@specialoutput! to let $\!\outputpenalty!=-10000$ to tell
% \!\pcol@output! that the special \!\output! request is our own.
% 
% \item
% \!\pcol@makenormalcol! and \!\pcol@deferredfootins! to let \!\vbadness!
% have 10000 temporarily to avoid underfull messages.
% 
% \item
% \!\pcol@synccolumn!  to bias \!\pcol@textfloatsep! by 10000\,|pt| to
% indicate a \colpage{} has an \mvlfloat{} and in \!\pcol@cflt! and
% \!\pcol@addflhd! to remove the bias.
% 
% \item
% \!\pcol@switchcol! and \!\pcol@flushclear! to put \!\penalty!|-10000| for
% forced page break.
% 
% \item
% $\!\pcol@setcw@calcf!\<x\>\<y\>\<z\>$ to let $Z=z\times1\,|pt|=10000\,|pt|$
% if $x/y$ is too large.
% \end{itemize}
% 
% \item[\Uidx{\!\@Mii!}]
% is a \!\mathchardef! register to have 10002.  It is used in
% \!\pcol@end@dblfloat! to examine if $\!\@floatpenalty!=-10002$ to mean the
% float environment to be closed is given in horizontal mode.
% 
% \item[\Uidx{\!\@Miv!}]
% is a \!\mathchardef! register to have 10004.  It is used in \!\pcol@output!
% to examine if $\!\outputpenalty!=-10004$ for a dummy \!\output! request
% made by \LaTeX's float-related macros and our \!\pcol@invokeoutput! to
% ensure the real request is not eliminated when it is made at the very
% beginning of a page or a \colpage.  It is also used in
% \!\pcol@specialoutput! for footnote logging when
% $\!\outputpenalty!=-10004$.
% 
% \item[\Uidx{\!\@MM!}]
% is a \!\mathchardef! register to have 20000.  It is used in
% \!\pcol@fntextbody! to let \!\floatingpenalty! have it for footnote
% typesetting.
% 
% \item[\Uidx{\!\@beginparpenalty!}]
% is a \!\count! register to have the page-break penalty inserted before the
% first \!\item! of each \env{list}-like environment.  The penalty is
% determined in class files and is, for example, $-\!\@lowpenalty!=-51$ with
% \textsf{article.cls}.  It is referred to and inserted by \!\pcol@zparacol!
% when it finds the \env{paracol} environment to start is at the very
% beginning of a \env{list}-like environment.
% 
% \item[\Uidx{\!\@floatpenalty!}]
% is a \!\count! register to have the penalty code -10002 or -10003 given by
% \!\@xfloat! at the beginning of a float environment according to the
% environment is in horizontal or vertical mode respectively, or by
% \!\marginpar! for a marginal note in the same meaning.  It is
% referred to by \!\pcol@end@dblfloat! to insert the penalty, and by
% \!\pcol@xympar! to confirm \!\marginpar! is error free.
% 
% \item[\Uidx{\!\@topnum!}]
% is a \!\count! register to have the maximum number of top floats which the
% \ccolpage{} can accept further.  It is used in \!\pcol@setcurrcol!
% and \!\pcol@iigetcurrcol! to save\slash restore it into\slash from
% $\cc_c(\tn)$.  The macro \!\pcol@synccolumn! also lets $\!\@topnum!=0$ to
% inhibit top-float insertions in the \ccolpage{} any more after a
% \sync{}ation.
% 
% \item[\Uidx{\!\@botnum!}]
% is a \!\count! register to have the maximum number of bottom floats which
% the \ccolpage{} can accept further.  It is used in \!\pcol@setcurrcol!
% and \!\pcol@iigetcurrcol! to save\slash restore it into\slash from
% $\cc_c(\bn)$.
% 
% \item[\Uidx{\!\@colnum!}]
% is a \!\count! register to have the maximum total number of floats which
% the \ccolpage{} can accept further.  It is used in \!\pcol@setcurrcol!
% and \!\pcol@iigetcurrcol! to save\slash restore it into\slash from
% $\cc_c(\cn)$.
% 
% \item[\Uidx{\!\col@number!}]
% is a \!\count! register to have the number of columns.  It is let have 1
% by \!\pcol@zparacol! and \!\pcol@sptext! regardless the real number of
% columns $\C$ in order to keep \!\maketitle! from putting the title by
% \!\twocolumn!.
% 
% \item[\Uidx{\!\c@page!}]
% is \LaTeX's counter \Uidx{\counter{page}} being a \!\count! register to
% have the page number.  It is referred to by \!\pcol@setpnoelt!, and
% \!\pcol@output@start! to let $\pp^p(p)=\page(p)$.  The macro
% \!\pcol@startpage! reload the register from $\pp^p(\ptop{-}1)$ and then
% increment it by one usually but two in \npaired{} \parapag{}ing, and
% repeat $\pp^p(\ptop)=\page(\ptop)$ and incrementing $\page(p)$ for each
% \fpage{}s of \pwise{} floats.  Reloading $\page(p)$ to the register from
% $\pp^p(p)$ is also done by \!\pcol@getpelt! for macros using
% \!\pcol@getcurrpage!, and by \!\pcol@outputelt!, \!\pcol@sync! and
% \!\pcol@makeflushedpage! by \!\pcol@getcurrpinfo!.  Then the register is
% referred to by the following macros to exmine its parity.
% 
% \begin{itemize}
% \item
% Our own \!\@outputpage! to give $\page(p)$ or $\page(p)+1$ to
% \!\pcol@outputpage@l! and \!\pcol@outputpage@r! which finally let the
% register have the value to be referred to by \!\pcol@@outputpage! being
% \LaTeX's \!\@outputpage!.
% 
% \item
% \begin{Sloppy}{2800}%
% \!\pcol@bg@swappage! to determine the values of \!\pcol@bg@leftmargin!
% and \CSIndex{ifpcol@bg@@swap} with other factors.
% \end{Sloppy}
% 
% \item
% \!\pcol@shiftspanning! to decide the necessity of shifting \mctext{} left
% with \cswap{}, examining {\em raw} \!\c@page! at the \!\output! request to
% close the \mctext{} rather than $\pp^p(\ptop)$ which will have the correct
% value with respect to possible jump {\em after} the macro completes its
% work.
% 
% \item
% \!\pcol@addmarginpar! to determine the margin to which a marginal note
% goes.
% 
% \item
% \!\pcol@do@mpbout@i! to determine which of $\mpb_L^l$ or $\mpb_L^r$ is the
% target of the operation specified by \!\pcol@do@mpbout@elem!.
% 
% \item
% $\!\pcol@swapcolumn!\<c_1\>\<c_2\>\<\Cfrom\>\<\Cto\>$ to determine $c_2$
% for $c_1$ if \cswap{} is in effect.
% \end{itemize}
% 
% In addition, to do the parity examination in \!\pcol@bg@swappage! above
% correctly, the macros \!\pcol@ioutputelt!, \!\pcol@makeflushedpage!,
% \!\pcol@imakeflushedpage!, \!\pcol@iflushfloats! and \!\pcol@output@end!
% temporarily increment the register by one when they are working on a right
% \npaired{} \parapag{}e.
% 
% The other users are \!\localcounter!\marg{ctr} to check
% $\arg{ctr}\neq\counter{page}$, \!\pcol@remctrelt! to let
% $|\cl@|{\cdot}\theta=\!\pcol@stepcounter!\Arg{\theta}$
% 
% \SpecialArrayIndex{\theta}{\cl@}
% 
% unless $\theta=\counter{page}$, and \!\pcol@FF! to write it to a log file
% as a part the logging information of memory leak debugging.
% 
% \item[\Uidx{\!\c@footnote!}]
% is \LaTeX's counter \Uidx{\counter{footnote}} being a \!\count! register
% to have the footnote number.  It is referred to by \!\pcol@zparacol! to
% let $\bf=\!\pcol@footnotebase!$ have its value, by \!\pcol@calcfnctr! to
% calculate the footnote ordinal which the one of its invoker
% \!\pcol@iadjustfnctr! sets into the register, and by \!\endparacol! to let
% the register have $\bf+\nf=\!\pcol@footnotebase!+\!\pcol@nfootnotes!$.
% 
% \item[$\cs{c@}{\cdot}\theta$]
% \SpecialArrayUsageIndex{\theta}{\c@}
% 
% is a \!\count! register being \LaTeX's counter $\theta$.  It is referred
% to by \!\pcol@savectrelt! to let $\val_c(\theta)=\Val(\theta)$ for saving
% it in $\Cc_c$, by \!\pcol@cmpctrelt! to examine if
% $\val_0(\theta)=\Val(\theta)$ to detect a modification outside
% \env{paracol} environment, and by \!\pcol@syncctrelt! to let
% $\val_c(\theta)=\Val(\theta)$ for all $c$ for \csync{}.  It is also
% referred to by \!\pcol@remctrelt! and \!\localcounter! to examine if
% $\theta=\counter{page}$.  The macros \!\pcol@setctrelt! and
% \!\pcol@stpldelt! restore the value of the counter from $\val_c(\theta)$,
% while \!\pcol@stpclelt! lets $\Val(\theta)=0$ for zero-clearing of
% descendent counters.
% 
% \item[\Uidx{\!\count@!}]
% is a \!\count! register for temporary use.  It is used in
% \!\pcol@iscancst! to have $m$ of $\mcelt_{i,m}$,
% \!\pcol@addmarginpar! to have the physical column position of the current
% column $c$ in which \!\marginpar! is given, and in
% $\!\pcol@extract@fil@i!\<s\>\!\@nil!$ to exract the unit of the stretch
% component $s$ of a glue.
% 
% \item[\Uidx{\!\@tempcnta!}]
% is a \!\count! register for temporary use.  The usages of this register
% are as follows.
% 
% \begin{itemize}
% \item
% In \!\pcol@F@count!, it is used to measure the cardinality of
% \!\@freelist!.
% 
% \item
% In \!\pcol@makecol!, it is used to keep $\page(\ptop)$ gotten by
% \!\pcol@getcurrpinfo! until we store it back by \!\pcol@defcurrpage!.
% 
% \item\begin{Sloppy}{3700}
% In \!\pcol@setpageno!, \!\pcol@setpnoelt!, \!\pcol@nextpage!,
% \!\pcol@nextpelt!, \!\pcol@getcurrpage!, \!\pcol@getpelt!,
% \!\pcol@addmarginpar! and \!\pcol@setmpbelt!, it has $p-q$ when we scan
% $\pp(q)$ for all $q\in[\pbase,\ptop]$ and the \ccolpage{} is at $p$.
% \end{Sloppy}
% 
% \item
% In \!\pcol@checkshipped!, it has $c$ when we scan $\S_c$ for all
% $c\In0\C$ to examine if all of them are not empty and thus we have pages
% to be shipped out.
% 
% \item
% In \!\pcol@ioutputelt!, it has $c'$ when we scan $\s_c(q)$ for all
% $c\In0\CL$ or $c\In\CL\C$ to build the shipping out image of a page $q$,
% where $c=c'$ or $c=\Cto-1-(c'-\Cfrom)$ where
% $(\Cfrom,\Cto)\in\{(0,\CL),(\CL,\C)\}$.
% 
% \item
% In \!\@outputpage!, it has $\page(p)$ or $\page(p)+1$ to be given to
% \!\pcol@outputpage@l! or \!\pcol@outputpage@r! as their argument when they
% are used to ship out the second (not always right) component of a
% \parapag{}e pair.
% 
% \item
% In \!\pcol@bg@columnleft!, it has a value in $\LBRP\CBfrom{c}$ to sum up
% the width of columns and \csepgap{}s in the range.
% 
% \item
% In \!\pcol@output@switch!, it is used to have $\page(p)$ obtained by
% \!\pcol@getcurrp~info! and simply to store the value in $\pp^p(p)$ by
% \!\pcol@defcurrpage!  when we use these macros to add an element to
% $\pp^s(p)$ and/or let $\pp^f(p)=\!\footins!$.
% 
% \item
% In \!\pcol@setcurrcol!, it has the code calculated from
% \CSIndex{if@nobreak} and |\if@after|\~|indent| to be saved in
% $\cc_c(\sw)$.
% 
% \SpecialIndex{\if@afterindent}
% 
% \item
% In $\!\pcol@scancst!\arg{box}$, it is let have
% $\arg{box}\in\{\!\pcol@colorins!,|\pcol@color|\~|stack@saved|\}$ and then is
% referred to in \!\pcol@iscancst!.
% \SpecialIndex{\pcol@colorstack@saved}
% 
% \item
% In \!\pcol@addmarginpar! besides the page scan shown above, it is used to
% scan all columns whose physical position is left from the current
% colum $c$ to measure the distance between the left edges of the leftmost
% column and $c$.
% 
% \item
% In \!\pcol@do@mpbout@i!, it has $\pm1$ according to the margin ($\rm
% left=-1$) which marginal notes outside \env{paracol} environments goes
% to.
% 
% \item
% In \!\pcol@flushcolumn!, it is used to throw $\page(\ptop)$ away when we
% get it by \!\pcol@getcurrpinfo! because we don't need it.
% 
% \item
% In \!\pcol@setcolwidth@r!, it has $c$ to scan fractions $r_d$ where
% $d=c-\Cfrom$ in its argument $\arg{ratio}$ specified by \!\columnratio!,
% and then to distribute the unspecifed portion of \!\textwidth! evenly to
% $\w_c$ for all $c\In{\Cfrom{+}k{+}1}\Cto$, where $k$ is the number of
% fractions and $(\Cfrom,\Cto)\in\{(0,\CL),(\CL,\C)\}$.
% 
% \item
% In $\!\pcol@setcw@scan!\<\Cfrom\>\<\Cto\>\Arg{spec}$, it has $c$ for two
% loops for $c\In\Cfrom\Cto$ to add `|,|' to the tail of $\arg{spec}$ as
% many as $k=\Cto-\Cfrom$ and then to process first $k$ elements in
% $\arg{spec}$, and is referred to by \!\pcol@setcw@set! invoked in the
% second loop.
% 
% \item
% In $\!\pcol@setcw@calcf!\<x\>\<y\>\<z\>$, it used to calculate and to have
% $k$ such that $z'/2^k\approx x/y$.
% 
% \item
% In \!\pcol@cmpctrelt!, it has $\Val(\theta)$ of a counter $\theta$ to be
% compared with $\val_0(\theta)$.
% 
% \item
% In \!\pcol@com@switchcolumn!, it has $(c+1)\bmod\C$ being the target of
% \cswitch{}.
% 
% \item
% In \!\pcol@sptext!, it temporarily has $d$ being the target of \cswitch{}
% during we let \!\pcol@nextcol! have 0 to visit the leftmost column
% to put a \mctext{}.
% 
% \item
% In \!\pcol@visitallcols!, it has $d\In0\C-\{c\}$ being the columns to be
% visited for \cscan{}ning.
% 
% \item
% In \!\definecolumnpreamble!\marg{c}\marg{pream}, $c$ is assigned to the
% register to ensure $c$ is a number.
% 
% \item
% In \!\pcol@calcfnctr!, it has the footnote ordinal calculated by the
% macro to be referred to by the invokers \!\pcol@iadjustfnctr! and
% \!\pcol@iifootnotetext!.
% 
% \item
% In $\!\marginparthreshold!\Arg{t_l}|[|t_r|]|$ it is let have $t_l$, while
% in the related macro $\!\pcol@marginparthreshold!|[|t_r|]|$ it is let have
% $t_r$.
% \end{itemize}
% 
% \item[\Uidx{\!\@tempcntb!}]
% is a \!\count! register for temporary use.  It is used in the following
% macros.
% 
% \begin{itemize}
% \item
% In \!\pcol@ioutputelt! it has $c$, while in \!\pcol@imakeflushedpage! and
% \!\pcol@iflushfloats! it has $c'$, to have $c=c'$ or $c=\Cto-1-(c'-\Cfrom)$
% according to \cswap{} for $c'$-th iteration of column scanning loop for
% $c'\In\Cfrom\Cto$, where $(\Cfrom,\Cto)\in\{(0,\CL),(\CL,\C)\}$.
% 
% \item
% In \!\pcol@scancst! and \!\pcol@iscancst!, it has $\npop$.
% 
% \item
% In \!\pcol@addmarginpar!, it is let have the column number $d$ whose
% phisical position is left from the current colum $c$ to measure the
% distance between the left edges of the leftmost column and $c$.
% 
% \item
% In \!\pcol@sync! and \!\pcol@measurecolumn!, it has the (so-far) tallest
% column which gives $\VP$.
% 
% \item
% In \!\pcol@setcolwidth@r!, it has $\Cto-\Cfrom-1$ then $\Cto-1$ and
% finally $\Cto-\min(\Cfrom{+}k,\,\Cto{-}1)$, where $k$ is the number of
% fractions given in the argument of \!\columnratio! and
% $(\Cfrom,\Cto)\in\{(0,\CL),(\CL,\C)\}$.
% 
% \item
% In $\!\pcol@setcw@calcf!\<x\>\<y\>\<z\>$, at first it is used to calculate
% $z'/2^k\approx x/y$ and then to calculate
% $Z=z\times1\,|pt|=z'\cdot2^{16-k}$.
% 
% \item
% In \!\pcol@visitallcols!, it has $c=\!\pcol@currcol!$ to exclude it from
% the \cscan{} targets.
% \end{itemize}
% \end{description}
% 
% 
% 
% \subsubsection{Dimension Registers}
% 
% \begin{description}
% \item[\Uidx{\!\vsize!}]
% is \TeX's primitive register to have the height of a page or a \colpage{}
% being built.  It is let be \!\@colroom! or \!\maxdimen! by
% \!\pcol@output!.
% 
% \item[\Uidx{\!\hsize!}]
% is \TeX's primitive register to have the width of a page or a \colpage{}
% being built.  It is let be $\w_c$ by \!\pcol@invokeoutput! to restart (or
% stay in) the column $c$, be \!\textwidth! by \!\pcol@sptext! for \mctext,
% and be either of \!\textwidth! or $\w_c$ by \!\pcol@fntextbody! according
% to the footnote typesetting being \scfnote{} or \mcfnote{} resepectively.
% 
% \item[\Uidx{\!\maxdepth!}]
% is \TeX's primitive register to have the maximum depth of the page being
% built.  In \LaTeX, it is assumed that its value is fixed at
% \beginenv{document}, in which the value is saved into \!\@maxdepth!, for
% the typesetting throughout the document, unless a bottom float is added to
% a page in which the register is let have 0 until it is shipped out.  This
% temporary setting for a page with bottom floats has some reasonability but
% its implementation for \env{paracol} environments having \cswitch{}
% from/to a \colpage{} with bottom floats to/from another one without
% them is too costly\footnote{
% 
% That is, we would need to incorporate \cs{maxdepth} as a member of
% \cctext, but we don't because the idea of temporary setting itself is too
% vague to pay the effort and a precious membership in \cctext.}.
% 
% Therefore, we let the register have \!\@maxdepth! in \!\pcol@output! and
% \!\pcol@combinefloats! to cancel the temporary setting done in
% \!\@addtobot! for the references by \TeX's page builder and \!\@cflt!
% respectively.
% 
% \item[\Uidx{\!\boxmaxdepth!}]
% is \TeX's primitive register to have the maximum depth of \!\vbox!es.
% The macros \!\pcol@cflt!, \!\pcol@opcol!,
% \!\pcol@ioutputelt!, \!\pcol@combine~floats!, \!\pcol@output@flush! and
% \!\pcol@output@clear! let it be \!\@maxdepth! for the boxes having a
% completed \colpage{} or page to cap the depth of the box.  The macro
% $\!\pcol@@makecol!\<d\>$ also do that when $d=\!\@maxdepth!$ but it can be
% $d=0$ when it is invoked to build \lpage{}.
% 
% \item[\Uidx{\!\splitmaxdepth!}]
% is \TeX's primitive register to have the maximum depth of the \!\vbox!
% being the first half of a box being split.  It is used in
% \!\pcol@deferredfootins! to cap the depth of the first half of deferred
% footnotes split from $\df$, and in \!\pcol@fntextbody! to let it have the
% depth of \!\strutbox!.
% 
% \item[\Uidx{\!\prevdepth!}]
% is \TeX's primitive register to have the depth of the box which just has
% been added to a vertical list, or to be given to \TeX's vertical list
% builder for the calculation of the vertical skip inserted below the last
% box.  The macro \!\pcol@invokeoutput! refers to it to save its value in
% \!\pcol@prevdepth! before putting a dummy \!\vbox! and making a \!\output!
% request, and then let it have \!\pcol@prevdepth! which is given by
% \!\output! routine for the column that we resume.  The macro
% \!\pcol@zparacol!  also refers to it to save it in \!\pcol@firstprevdepth!
% for the \preenv.
% 
% \item[\Uidx{\!\vfuzz!}]
% is \TeX's primitive register to have the threshold height of overfull
% messaging.  It is set to 0 in $\!\pcol@ShowBox!\arg{b}$ to ensure overfull
% for any \!\box! $b$ of non-null height.
% 
% \item[\Uidx{\!\maxdimen!}]
% is a \!\dimen! register to have $16383.99999\,|pt|$ being the largest
% legal dimensional value.  The usages of this register are as follows.
% 
% \begin{itemize}
% \item
% For the \!\insert! register set $\!\pcol@colorins!=\cstraw$,
% \!\dimen!\!\pcol@colorins! is let be \!\maxdimen! for the consistency with
% our intention that a \colpage{} can have virtually infinite number of
% \!\insert!ions for text coloring.
% 
% \item
% In \!\pcol@output!, it is set into \!\vsize! when $\!\outputpenalty!=-10004$
% for the dummy \!\output! request so that no page break should occur
% between the dummy and real requests.
% 
% \item
% Our own \!\dimen! register \!\pcol@textfloatsep! has \!\maxdimen! if a
% \colpage{} does not have \sync{}ation points to let top floats are
% inserted in usual way.  Therefore, \!\pcol@floatplacement! and
% \!\pcol@zparacol!  let the register have \!\maxdimen! as the initial
% value.  Then the macros \!\pcol@makecol! and \!\pcol@combinefloats!
% examine if $\!\pcol@textfloatsep!=\!\maxdimen!$ to determine the operation
% type of top float insertion, while \!\pcol@synccolumn! does that to know
% if it is flushing a \colpage{} with a \sync{}ation point or is setting the
% first \sync{}ation point in the \colpage{}.  The macro \!\pcol@addflhd!
% also examines it for the measurement of the combined height of top and
% bottom floats, while \!\pcol@measurecolumn! gives it \!\maxdimen! as its
% third argument for bottom floats.
% 
% \item
% The \pctext{} $\pp^h(p)$ has $-\!\maxdimen!$ if the page is a \fpage{}.
% The macro \!\pcol@startpage! makes that when it creates such a page.
% 
% \item
% Our own \!\pcol@prevdepth! and that saved in $\cc_c(\pd)$ have
% \!\maxdimen! if the main vertical list is empty at a \sync{}ation.  The
% macro \!\pcol@measurecolumn! makes that when it finds an empty list.
% 
% \item
% The \cctext{} $\cc_c(\tr)$ may have \!\maxdimen! if the column $c$ has a
% \fcolumn{} in the \lpage{} of a \env{paracol} environment and the floats in
% it can be put as top floats.  The macro \!\pcol@makefcolumn! makes this
% special assignment, and \!\pcol@flushcolumn! and \!\pcol@imakeflushedpage!
% examine it.
% 
% \item
% The macro \!\pcol@makefcolelt! let the room for floats in a \fpage{} be
% $-\!\maxdimen!$ if it finds no futher floats can be added to the page.
% 
% \item
% At a \sync{}ation, we measure the maximum combined size of top floats and
% the main vertical list $\VT$, that of footnotes and bottom floats $\VB$,
% that of four items $\VP$, and $\VPP$ being $\VP$ or $\VP+\!\textfloatsep!$
% according to the existence of bottom floats.  We also let $\DT$ and $\DP$
% be the minimum depth of the \colpage{}s which gives $\VT$ and $\VPP$
% respectively.  For the measurement, the macro \!\pcol@sync! lets
% $\VT=\VB=\VP=\VPP=-\!\maxdimen!$ and $\DT=\DP=\!\maxdimen!$ as their
% initial values.  Then the macro \!\pcol@synccolumn! examines if
% $\DT=\!\maxdimen!$ to mean the \sync{}ation point is set just below the
% top floats of a column whose main vertical list is empty.  On the other
% hand, \!\pcol@makeflushedpage! and \!\pcol@output@end! examine if
% $\VPP=-\!\maxdimen!$ to mean the last \colpage{}s are empty.
% \end{itemize}
% 
% \item[\Uidx{\!\linewidth!}]
% is a \!\dimen! register for an API parameter of \LaTeX{} to have the width
% of a line possibly shorter than \!\columnwidth! in \env{list}-like
% environments.  It is let have $\w_c-\lrm$ by
% \!\pcol@invokeoutput! for the outermost paragraphs in \env{paracol}
% environment, where $\lrm=\!\pcol@lrmargin!=\!\textwidth!-l$ which
% \!\pcol@zparacol! lets have to represent the left-plus-right margin of the
% \env{list}-like environment, whose \!\linewidth! is $l$, enclosing
% \env{paracol} if any.  The macro \!\pcol@sptext! also sets the register
% temporarily for \mctext{}s but letting it have $\!\textwidth!-\lrm$.
% 
% \item[\Uidx{\!\footnotesep!}]
% is a \!\dimen! register for an API parameter of \LaTeX{} to have the
% vertical space inserted in a footnote when it is split into two or more
% pages.  It is uesed in \!\pcol@fntextbody! to \!\splittopskip! have it,
% and to make the first line of the footnote is at least as tall as the
% amount in the register.
% 
% \item[\Uidx{\!\topmargin!}]
% is a \!\dimen! register for an API parameter of \LaTeX{} to have the width
% (height) of the top margin minus 1\,inch.  The register is used as an
% element of \!\pcol@bg@pagetop! to calculate the distance from the origin
% at the left-top corner of text area to the top edge of a page.  The
% other users are \!\pcol@ioutputelt! and \!\pcol@makeflushedpage!
% temporarily add the height-plus-depth of $\pp^b(p)$ to the register to
% make the calculation biased shifting the origin to the left-top corner of
% column area, i.e., below $\pp^b(p)$.  The macro \!\@outputpage! also
% refers to the register together with \!\headheight! and \!\headsep! to
% calculate the distance from the page top (ingoring 1\,inch shift) to the
% text area top.
% 
% \item[\Uidx{\!\oddsidemargin!}]
% is a \!\dimen! register for an API parameter of \LaTeX{} to have the width
% of the left margin minus 1\,inch for two-sided odd-numbered pages and
% all single-sided pages.  The register is used together with
% \!\evensidemargin! in \!\pcol@outputpage@l!, \!\pcol@outputpage@r! and
% \!\pcol@bg@swappage! to decide the left margin of the page they are
% working on.
% 
% \item[\Uidx{\!\evensidemargin!}]
% is a \!\dimen! register for an API parameter of \LaTeX{} to have the width
% of the left margin minus 1\,inch for two-sided even-numbered pages.  The
% register is used together with \!\oddsidemargin! in \!\pcol@outputpage@l!,
% \!\pcol@outputpage@r! and \!\pcol@bg@swappage! to decide the left margin
% of the page they are working on.
% 
% \item[\Uidx{\!\headheight!}]
% is a \!\dimen! register for an API parameter of \LaTeX{} to have the
% height of page headers.  The register is used together with \!\topmargin!
% and \!\headsep! as an element of \!\pcol@bg@pagetop! and in \!\@outputpage!
% to calculate the distance from the real and imaginary page top to the text
% area top respectively.
% 
% \item[\Uidx{\!\headsep!}]
% is a \!\dimen! register for an API parameter of \LaTeX{} to have the
% vertical distance from the bottom of a page header to the text area top.
% The register is used together with \!\topmargin!  and \!\headheight! as an
% element of \!\pcol@bg@pagetop! and in \!\@outputpage!  to calculate the
% distance from the real and imaginary page top to the text area top
% respectively.
% 
% \item[\Uidx{\!\textheight!}]
% is a \!\dimen! register for an API parameter of \LaTeX{} to have the
% height of text area in a page.  The register is referred to by
% \!\pcol@output! to examine if a page is very short, by \!\pcol@getpinfo!,
% \!\pcol@startpage!, \!\pcol@flushfloats! and \!\pcol@output@end! to let
% \!\@colht! be it for a page without \spanning{} (so far), by
% \!\pcol@outputelt!, \!\@outputpage!, \!\pcol@output@flush! and
% \!\pcol@output@clear! to build a page to be shipped out, by
% \!\pcol@bg@textheight! to calculate $\HT$, and by \!\pcol@fntextbody! to
% cap the height of a footnote.
% 
% \item[\Uidx{\!\textwidth!}]
% is a \!\dimen! register for an API parameter of \LaTeX{} to have the width
% of a page, which we occasionally refer to as $\WT$.  The register is
% referred to by \!\pcol@ioutputelt!, \!\pcol@imakeflushedpage! and
% \!\pcol@iflushfloats! to build a \!\hbox! of \!\textwidth! wide to have
% all columns (in a left or right \parapag{}e).  It also referred to by
% following macros; \!\pcol@bg@swappage! to calculate the right margin for
% \mirror{}ed \bgpaint; \!\pcol@bg@@r!, \!\pcol@bg@@f!, \!\pcol@bg@@n!,
% \!\pcol@bg@@p!  and \!\pcol@bg@@s! to specify the width of \bground{} of
% \pwstuff{} to be painted; \!\pcol@shiftspanning! to calculate the
% left-shift amount of a \mctext{} in \cswap; \!\pcol@addmarginpar! to
% measure the distance between the right edges of the rightmost and current
% columns; \!\pcol@zparacol! for the calculation of
% $\lrm=\!\pcol@lrmargin!$; \!\pcol@setcolwidth@r! for the calculation of
% $\w_c$ for all $c\In0\CL$ or $c\In\CL\C$; \!\pcol@setcw@calcfactors! for
% the calculation of $\WT/W$ and $(\WT-W)/F$; \!\pcol@sptext!,
% \!\footnoterule!  of \env{paracol}'s local and \!\endparacol!  to set it
% in \!\columnwidth!; and \!\pcol@fntextbody!  to set it in \!\hsize! if
% \Scfnote{} typesetting is in effect.
% 
% \item[\Uidx{\!\columnwidth!}]
% is a \!\dimen! register for an API parameter of \LaTeX{} to have the width
% of a column.  The register is let have $\w_c$ by \!\pcol@getcurrcol! for
% the column $c$, then is referred to by the following macros;
% \!\pcol@shiftspanning! to calculate the left-shift amount of a \mctext{} in
% \cswap; \!\pcol@addmarginpar! to measure the distance between the right
% edges of the rightmost and current columns; \!\pcol@imakeflushedpage! and
% \!\pcol@iflushfloats! to put each \colpage{} into a \!\hbox! of $\w_c$ wide
% for shipping a page out; \!\pcol@invokeoutput! to let \!\linewidth! and
% \!\hsize!  have the value of or based on it; and \!\pcol@fntextbody! to
% let \!\hsize! have it for \Mcfnote{} typesetting.  The register is also
% let have \!\textwidth! by \!\footnoterule! of \env{paracol}'s local
% defined in \!\pcol@zparacol! if \Scfnote{} typesetting is in effect, by
% \!\pcol@sptext!  for \mctext{}s, and by \!\endparacol! for \postenv.
% 
% \item[\Uidx{\!\columnsep!}]
% is a \!\dimen! register for an API parameter of \LaTeX{} to have the width
% of \csepgap{}s.  It is referred to by \!\pcol@setcolwidth@r! to
% calculate $\w_c$ for all $c\In0\CL$ or $c\In\CL\C$, and by
% \!\pcol@setcw@getspec! as the default width of \csepgap{}s.
% 
% \item[\Uidx{\!\columnseprule!}]
% is a \!\dimen! register for an API parameter of \LaTeX{} to have the width
% of the rules to be drawn in \csepgap{}s. It is referred to by
% \!\pcol@buildcolseprule! and \!\pcol@buildcselt! to draw the rule, and by
% \!\pcol@hfil! to examine if it is positive to mean the rule is really
% drawn and if so to add skips of $-\!\columnseprule!/2$ to surround the
% rule to nullify the width of the rule.
% 
% \item[\Uidx{\!\marginparwidth!}]
% is a \!\dimen! register for an API parameter of \LaTeX{} to have the width
% of a marginal note.  It is temporarily modified by \!\pcol@addmarginpar!
% so that a left marginal note is placed appropriately.
% 
% \item[\Uidx{\!\marginparsep!}]
% is a \!\dimen! register for an API parameter of \LaTeX{} to have the
% distance between a marginal note and text area.  It is temporarily
% modified by \!\pcol@addmarginpar!  so that a right marginal note is placed
% appropriately.
% 
% \item[\Uidx{\!\marginparpush!}]
% is a \!\dimen! register for an API parameter of \LaTeX{} to have the
% minimum vertical distance between two marginal notes.  It is referred to
% by \!\pcol@addmarginpar! to find a place for a marginal note and remember
% the place in $\pp^m(p)$.
% 
% \item[\Uidx{\!\paperheight!}]
% is a \!\dimen! register for an API parameter of \LaTeX{} to have the height
% of physical pages $\HP$.  It is referred to by \!\pcol@bg@paperheight!
% to calculate $\HP-2\WR$.
% 
% \item[\Uidx{\!\paperwidth!}]
% is a \!\dimen! register for an API parameter of \LaTeX{} to have the width
% of physical pages $\WP$.  It is referred to by \!\pcol@bg@swappage! to
% calculate the right margin for \mirror{}ed \bgpaint, and by
% \!\pcol@bg@paperwidth!  to calculate $\WP-2\WR$.
% 
% \item[\Uidx{\!\z@!}]
% is a \!\dimen! register to have 0\,|pt| to initialize \!\pagerim!,
% \!\belowfootnoteskip! and \!\skip!\!\pcol@colorins! at their declarations,
% and is used in the following macros.
% 
% \begin{quote}\raggedright
% \!\pcol@ShowBox!,
% \!\pcol@makecol!,
% \!\pcol@combinefloats!,
% \!\pcol@nextpelt!,
% \!\pcol@floatplacement!,
% \!\pcol@startpage!,
% \!\pcol@restartcolumn!,
% \!\pcol@outputelt!,
% \!\pcol@ioutputelt!,
% \!\pcol@buildcolseprule!,
% \!\pcol@buildcselt@S!,
% \!\pcol@buildcselt!,
% \!\@outputpage!,
% \!\pcol@startcolumn!,
% \!\pcol@bg@paint@i!,
% \!\pcol@bg@paintregion!,
% \!\pcol@output@start!,
% \!\pcol@putbackmvl!,
% \!\pcol@iscancst!,
% \!\pcol@deferredfootins!,
% \!\pcol@combinefootins!,
% \!\pcol@addmarginpar!,
% \!\pcol@getmparbottom!,
% \!\pcol@sync!,
% \!\pcol@measurecolumn!,
% \!\pcol@synccolumn!,
% \!\pcol@makeflushedpage!,
% \!\pcol@imakeflushedpage!,
% \!\pcol@output@end!,
% \!\pcol@invokeoutput!,
% \!\pcol@setcolwidth@s!,
% \!\pcol@setcw@calcfactors!,
% \!\pcol@setcw@calcf!,
% \!\pcol@extract@fil@ii!,
% \!\pcol@sptext!,
% \!\pcol@fntextbody!.
% \!\pcol@marginpar!,
% \!\pcol@icolumncolor!,
% \!\pcol@set@color@push!,
% \!\pcol@reset@color@pop!,
% \!\pcol@reset@color@mpop!,
% \!\pcol@backgroundcolor@iii!.
% \end{quote}
% 
% It is also used to give the number 0 for the initializations of
% \!\pcol@currcol!, \!\pcol@ncol!, \!\pcol@ncolleft! and
% \!\count!\!\pcol@colorins! at their declarations, and in the following
% macros.
% 
% \begin{quote}\raggedright
% \!\pcol@ShowBox!,
% \!\pcol@F@count!,
% \!\pcol@makecol!,
% \!\pcol@opcol!,
% \!\pcol@setpnoelt!,
% \!\pcol@nextpelt!,
% \!\pcol@checkshipped!,
% \!\pcol@getpelt!,
% \!\pcol@outputelt!,
% \!\pcol@ioutputelt!,
% \!\@outputpage!,
% \!\pcol@startcolumn!,
% \!\pcol@output@start!,
% \!\pcol@output@switch!,
% \!\pcol@setcurrcol!,
% \!\pcol@iscancst!,
% \!\pcol@addmarginpar!,
% \!\pcol@setmpbelt!,
% \!\pcol@do@mpbout@i!,
% \!\pcol@sync!,
% \!\pcol@synccolumn!,
% \!\pcol@makeflushedpage!,
% \!\pcol@imakeflushedpage!,
% \!\pcol@flushfloats!,
% \!\pcol@iflushfloats!,
% \!\pcol@freshpage!,
% \!\pcol@output@end!,
% \!\pcol@zparacol!,
% \!\pcol@setcolwidth@r!,
% \!\pcol@setcw@calcf!,
% \!\pcol@synccounter!,
% \!\pcol@com@syncallcounters!,
% \!\pcol@stepcounter!,
% \!\pcol@stpclelt!,
% \!\pcol@com@switchcolumn!,
% \!\pcol@switchcolumn!,
% \!\pcol@sptext!,
% \!\pcol@switchcol!,
% \!\pcol@visitallcols!,
% \!\pcol@xympar!,
% \!\endparacol!.
% \end{quote}
% 
% \item[\Uidx{\!\p@!}]
% \begin{Sloppy}{3950}%
% is a \!\dimen! register to have 1\,|pt|.  It is used in \!\pcol@ShowBox!,
% \!\pcol@cflt!, \!\pcol@addflhd!, \!\pcol@synccolumn!, \!\pcol@output@end!,
% \!\pcol@setcolwidth@s!, \!\pcol@setcw@getspec@i!, \!\pcol@setcw@fill! and
% \!\pcol@setcw@calcf!, and the top level assignment to \!\@tempskipa! for
% the invocation of \!\pcol@defkw!, as the shorthand of |pt|.
% \end{Sloppy}
% 
% \item[\Uidx{\!\@totalleftmargin!}]
% is a \!\dimen! register to have the total size of the left margins of a
% \env{list}-like environment and those surrounding it.  It is given to
% \!\parshape! by \!\pcol@invokeoutput! and \!\pcol@sptext! if \env{paracol}
% is enclosed in a \env{list}-like environment.
% 
% \item[\Uidx{\!\@themargin!}]
% is a control sequence \!\let!-equal to \!\evensidemargin! for two-sided
% even numbered pages or \!\oddsidemargin! for others.  In
% \!\pcol@outputpage@l! and \!\pcol@outputpage@r! it is bound to one of
% \!\dimen! registers for the references in \!\pcol@outputpage@ev!\footnote{
% 
% The reference in \CSIndex{pcol@@outputpage} being \LaTeX's
% \CSIndex{@outputpage} is done after the macro itself makes the assignment,
% which is of course consistent with those in our macros.}.
% 
% \item[\Uidx{\!\@maxdepth!}]
% is a \!\dimen! register to have \!\maxdepth! at \beginenv{document} to
% recover the temporary update of \!\maxdepth! with 0 by \!\@addtobot! for
% bottom float incorporation in a page.  As discussed in the explanation of
% \!\maxdepth!, in \env{paracol} environments \!\maxdepth! is let have
% \!\@maxdepth! always by the assignments in \!\pcol@output! and
% \!\pcol@combinefloats!.  Other users, \!\pcol@cflt!, \!\pcol@opcol!,
% \!\pcol@ioutputelt!, \!\pcol@combinefootins!, \!\pcol@output@flush! and
% \!\pcol@output@clear!, let \!\boxmaxdepth! have \!\@maxdepth! so as to
% limit the depth of boxes for a completed \colpage{} or page to the value
% for page typesetting, while \!\pcol@flushcolumn! and
% \!\pcol@imakeflushedpage! do that by \!\pcol@@makecol! giving the register
% to it.  The other usage of the register is to calculate \bgpaint{}
% parameter $\HT$ by \!\pcol@bg@textheight!, and to determine the bottom
% edge of the \bground{}s of columns and \csepgap{}s through the argument of
% \!\pcol@buildcolseprule! given by \!\pcol@ioutputelt!,
% \!\pcol@imakeflushedpage! and \!\pcol@iflushfloats!.  The register is also
% referred to by \!\pcol@unvbox@cclv! to go back the last baseline of the
% main vertical list in \!\box!|255|, and by \!\pcol@deferredfootins! to let
% \!\splitmaxdepth! have its value to cap the depth of the first half of
% footnotes split from $\df$.
% 
% \item[\Uidx{\!\@colht!}]
% is a \!\dimen! register to have the height of columns in a page possibly
% shrunk from \!\textheight! by \spanning.  The usages of the register
% are as folllows.
% 
% \begin{itemize}
% \item\begin{Sloppy}{1600}
% In \!\pcol@startpage!, \!\pcol@output@start!, \!\pcol@flushfloats!  and
% \!\pcol@output@end!, it is initialized to \!\textheight!.  In first two,
% the value of the register is reduced to reflect \spanning{} if exists and
% then set into $\pp^h(p)$, while the setting by the third is referred to by
% its callee \!\pcol@iflushfloats!.
% \end{Sloppy}
% 
% \item
% In \!\pcol@getpelt!, \!\pcol@sync!, \!\pcol@flushcolumn!,
% \!\pcol@makeflushedpage! and \!\pcol@imakeflushedpage!, it is let have
% $\pp^h(p)$.  In addition \!\pcol@sync! examines if
% $\!\@colht!<\VT+\VB+v(f)$, and \!\pcol@makefcolumn! uses it to initialize
% the room of a \fcolumn{} as well as the height of $\cc_c(\vb)$ for it.
% 
% \item
% In \!\pcol@opcol!, it is used to add \!\pcol@clearcolorstack! to the
% bottom of $\cc_c(\vb)$ whose height is \!\@colht!.
% 
% \item
% In \!\pcol@startcolumn!(*), \!\pcol@flushcolumn!(*) and
% \!\pcol@freshpage!, it is used to let \!\@colroom! have it.
% 
% \item
% In \!\pcol@restartcolumn!(*), it is saved and restored for the use as the
% height cap of deferred footnote \!\insert!ion in \!\pcol@deferredfootins!
% because it can be shrunk by the non-deferred \Scfnote{}s.
% 
% \item
% In \!\pcol@output@flush! and \!\pcol@output@clear!, it is given to
% \!\pcol@make~flushedpage! as its argument.  The macro
% \!\pcol@makeflushedpage!(*) lets \!\@colht! be the argument if it is less
% than \!\@colht! and thus is given by \!\pcol@output@end!.
% \end{itemize}
% 
% In addition, in the macros with `(*)' above and \!\pcol@makecol!, the
% register is passed to \!\pcol@shrinkcolbyfn! to shrink the height in it
% temporarily to keep the space required to put \Scfnote{}s in the page they
% are working on, for the reference by starred macros themselves or
% \!\@makecol! invoked in \!\pcol@makecol!.
% 
% \item[\Uidx{\!\@colroom!}]
% is a \!\dimen! register to have the height of a column possibly shrunk
% from \!\@colht! by top and bottom floats.  The register is initialized to
% have \!\@colht! by \!\pcol@startcolumn!, \!\pcol@output@start!,
% \!\pcol@flushcolumn! and \!\pcol@freshpage!, the last three of which also
% save it into $\cc_c(\vb^r)$.  This save operation is also done by
% \!\pcol@output@switch! while restoring from it done by
% \!\pcol@restartcolumn!, but the latter macro may shrink the amount in its
% callee \!\pcol@putbackmvl! to capture a \mctext{} while the former cancel
% this shrinkage.  The macros \!\pcol@output! and \!\pcol@output@start! also
% refer to this register to let \!\vsize! have it in the former and to
% calculate the room for each \colpage{} in the \spage{} in the latter.  The
% macro \!\pcol@output@end!  lets the register have \!\textheight! for the
% \postenv{} because the \colpage{}s above it simply precedes the stuff in
% the main vertical list.  The other users \!\pcol@makefcolumn! and
% \!\pcol@makefcolelt! use this register to accumulate the total size of
% floats to be put in a \fcolumn{} temporarily.
% 
% \item[\Uidx{\!\@pageht!}]
% is a \!\dimen! register to be used in \LaTeX's \!\@specialoutput! to have
% the height of \!\@holdpg!.  It is referred to by \!\pcol@addmarginpar! to
% deteremine the position at which a marginal note is placed.
% We also use it as a scratchpad to have $\VP$ in \!\pcol@sync! and
% \!\pcol@measurecolumn!, and to save $\pp^h(\ptop)$ in \!\pcol@flushcolumn!
% for the reference in itself, and to do so in \!\pcol@makeflushedpage! for
% \!\pcol@imakeflushedpage!.
% 
% \item[\Uidx{\!\@pagedp!}]
% is a \!\dimen! register to be used in \LaTeX's \!\@specialoutput! to have
% the depth of \!\@holdpg!.  However, we use it as a scratchpad in
% \!\pcol@sync! and \!\pcol@measurecolumn! to have $\DP$, and in
% \!\pcol@output@end! to have the value to be set in \!\pcol@prevdepth!.
% 
% \item[\Uidx{\!\@toproom!}]
% is a \!\dimen! register to have the room for top floats.  The register is
% saved in $\cc_c(\tr)$ by \!\pcol@setcurrcol! and restored from it by
% \!\pcol@iigetcurrcol!.  The macro \!\pcol@makefcolumn! uses this register
% as a flag to indicate that $\cc_c(\tl)$ of the column $c$ having
% $\cc_c(\tr)=\infty$ contains floats to be put in its last \fcolumn{}
% possibly as top floats so that it is examined by \!\pcol@flushcolumn! and
% \!\pcol@imakeflushedpage!, the former of which then lets $\cc_c(\tr)=0$ to
% mean the floats are put in a \fcolumn{} in a non-\lpage{} by the macro.
% 
% \item[\Uidx{\!\@botroom!}]
% is a \!\dimen! register to have the room for bottom floats.  The register
% is saved in $\cc_c(\br)$ by \!\pcol@setcurrcol! and restored from it by
% \!\pcol@iigetcurrcol!.
% 
% \item[\Uidx{\!\@fpmin!}]
% is a \!\dimen! register to have $\!\floatpagefraction!\times\!\@colht!$
% being the minimum total size of floats for which an ordinary (not flushed)
% \fcolumn{} can be build.  It is referred to by \!\pcol@makefcolumn! as the
% threshhold below which floats in the last \fcolumn{} can be put as top
% floats.
% 
% \item[\Uidx{\!\@mparbottom!}]
% is a \!\dimen! register to have the bottom position of the last
% \!\marginpar! stuff.  Its value at \beginparacol{} is referred to by
% \!\pcol@output@start! to let $\mpb_L^l$ or $\mpb_L^r$ of $\pp^m(0)$ has an
% element based on it, while the tail of one of the lists in $\pp^m(\ptop)$
% defines the value at \Endparacol{} which \!\pcol@output@end! lets the
% register have.  The register is also updated by \!\pcol@getmparbottom! and
% \!\pcol@getmpbelt! to let \!\pcol@@addmarginpar! being \LaTeX's original
% \!\@addmarginpar! know the uppermost available position for the marginal
% note to be added.  This update is, however, just for communicatoin betwee
% these macros and thus is ineffective for typesetting posterior to that, as
% well as the update in \!\pcol@@addmarginpar!, because whole information
% for marginal note placement is kept in $\pp^m(p)$ in $\PPP$.
% 
% \item[\Uidx{\!\@textfloatsheight!}]
% is a \!\dimen! register to have the combined height of mid floats and
% their separators.  It is initialize to be 0 by \!\pcol@floatplacement!,
% saved in $\cc_c(\fh)$ by \!\pcol@setcurrcol!, and restored from it by
% \!\pcol@iigetcurrcol!.
% 
% \item[\Uidx{\!\dimen@!}]
% is a \!\dimen! register for temporary use.  It is used in the following
% macros.
% 
% \begin{itemize}
% \item
% \!\pcol@buildcolseprule!, \!\pcol@buildcselt@S! and \!\pcol@buildcselt! to
% have the argument $d\in\{\!\@maxdepth!,0\}$ of the first macro.
% 
% \item
% \!\pcol@bg@paintregion@i! to have $y_1$ of $\bgr_a^{[c]}$.
% 
% \item
% $\!\pcol@bias@mpbout@i!\Arg{y}\Arg{t}\Arg{b}$ to have $t$ and then $t+y$.
% 
% \item
% \!\pcol@output@switch! to have the height of \prespan{} in
% \!\pcol@prespan!, or 0 if it is $\bot$.
% 
% \item
% \!\pcol@sync! to have $V$ or $V-\DT+\VE$.
% 
% \item
% \!\pcol@addflhd! and \!\pcol@hdflelt! to measure the height of top and
% bottom floats, \!\pcol@makecol! and \!\pcol@output@switch! to measure
% the height of \prespan{} including the top floats, and
% \!\pcol@measurecolumn! for top and bottom floats and $\VT$, $\VB$ and
% $\VP$.
% 
% \item
% \!\pcol@setcolwidth@s! and \!\pcol@setcw@accumwd! to accumulate $W$ being
% the sum of natural widths of column\slash gap specifications, and then
% used by \!\pcol@setcw@calcfactors! to calculate $W/\WT$ and $W-\WT$.
% \end{itemize}
% 
% \item[\Uidx{\!\dimen@ii!}]
% is a \!\dimen! register for temporary use.  It is used in the following
% macros.
% 
% \begin{itemize}
% \item
% \!\pcol@makecol! to measure the total height of top floats by
% \!\pcol@addflhd!.
% 
% \item
% \!\pcol@bg@addext! to have
% $e=|pcol@bg@ext@|{\cdot}d{\cdot}|@|{\cdot}\{a{\cdot}|@|{\cdot}c,a\}$
% 
% \SpecialArrayIndex{d{\cdot}\string\texttt{@}
%     {\cdot}a{\cdot}\string\texttt{@}{\cdot}c}{\pcol@bg@ext@}
% \SpecialArrayIndex{d{\cdot}\string\texttt{@}{\cdot}a}{\pcol@bg@ext@}
% 
% and then $10000\PT-e$ to calculate an \bgext{} of \bgpaint.
% 
% \item
% $\!\pcol@bias@mpbout@i!\Arg{y}\Arg{t}\Arg{b}$ to have $b$ and then $b+y$.
% 
% \item
% \!\pcol@measurecolumn! to measure $\VT$, $\VP$ and $\DP$.
% 
% \item
% \!\pcol@setcolwidth@s! and \!\pcol@setcw@accumwd! to accumulate $F$ being
% the sum of infinite stretch factors in column\slash gap specifications with
% the unit of pt, and then used by \!\pcol@setcw@calcfactors! to calculate
% $(W-\WT)/F$, where $W$ is the sum of natural widths.
% 
% \item
% \!\pcol@setcw@calcfactors! to have $(W-\WT)/F$ above or 0 to be used in
% \!\pcol@def@extract@fil@iii!  through \!\pcol@setcw@filunit!  made
% \!\let!-equal to the register by \!\pcol@setcolwidth@s!.
% \end{itemize}
% 
% \item[\Uidx{\!\@tempdima!}]
% is a \!\dimen! register for temporary use.  The usages of this register
% are as follows.
% 
% \begin{itemize}
% \item
% In \!\pcol@makecol! and \!\pcol@startpage!, it is used to throw
% $\pp^h(\ptop)$ away when we get it by \!\pcol@getcurrpinfo! because we
% don't need it.
% 
% \item
% In \!\pcol@outputelt!, it has $\pp^h(p)$ to examine if $p$ is a \fpage.
% 
% \item
% In \!\pcol@ioutputelt!, it has $\pp^h(p)$ possibly shrunk by \Scfnote{}s
% to know the \bground{}s to be painted for columns etc.  After that it has
% $\w_c$ being the width of each \!\hbox! into which the \colpage{} of each
% column $c$ is put.
% 
% \item
% In \!\pcol@buildcolseprule! and its callees \!\pcol@buildcselt@S! and
% \!\pcol@buildcselt!, the register has the first argment $H=\pp^h(p)$ of
% the caller macro.
% 
% \item
% In \!\pcol@hfil!$\<c\>$, it has $\gap_c=|\pcol@columnsep|{\cdot}c$.
% 
% \SpecialArrayIndex{c}{\pcol@columnsep}
% 
% \item
% In \!\@outputpage!, it has the sum of \!\topmargin!, \!\headheight! and
% \!\headsep! being the distance between tops of imaginary page and its text
% area.
% 
% \item
% In \!\pcol@startcolumn!, it is used to save \!\@colht! which can be shrunk
% temporarily by \Scfnote{}s.
% 
% \item
% In \!\pcol@bg@paintregion@i!, it is let have $x_0$ of $\bgr_a^{[c]}$.
% 
% \item
% In \!\pcol@output@start!, it is used to have the room for each \colpage{}
% in the \spage, and then the height-plus-depth of the \preenv.
% 
% \item
% In \!\pcol@output@switch!, it is used to throw $\pp^h(p)$ away when
% we get it by \!\pcol@getcurrpinfo! because we don't need it.
% 
% \item
% In \!\pcol@shiftspanning!, it is let have the left-shift amount of a
% \mctext{} in \cswap.
% 
% \item
% In \!\pcol@restartcolumn!, it is used to save \!\@colht! which can be
% shrunk temporarily by \Scfnote{}s.
% 
% \item
% In \!\pcol@unvbox@cclv!$\arg{ins}$, it has the depth of \!\box!|255| for
% going back to the baseline of the box, and then has the natural component
% of $\!\skip!\arg{ins}$ to add its stretch and shrink components only.
% 
% \item
% In \!\pcol@addmarginpar!, at first it has the distance between left edges
% of the leftmost and current columns.  Then it is let have the distance
% between top edges of the column and the marginal text to be put.
% 
% \item
% In $\!\pcol@getmparbottom!\<t\>\<h\>$ and
% $\!\pcol@getmpbelt!\<t_i\>\<b_i\>$, it at first has $t$ and then is
% let have $b_i$ when the marginal note cannot be put at $t$.
% 
% \item
% In \!\pcol@sync!, \!\pcol@measurecolumn! and \!\pcol@synccolumn!, it has
% $\VT$ being the maximum combined height of top floats and the main
% vertical list.
% 
% \item
% In \!\pcol@makefcolumn! and \!\pcol@makefcolelt!, it has the room for
% floats to be put in a \fcolumn{}.
% 
% \item
% In \!\pcol@makeflushedpage!, it has the height-plus-depth of \spanning in
% $\pp^i(\ptop)$.
% 
% \item
% In \!\pcol@output@end!, at first it is let have $\VPP-H$, where $H$ is the
% height(-plus-depth) of \!\@outputbox! having the ship-out image of the
% \lpage, being the negative counterpart of the height-plus-depth of
% \spanning{} in the \lpage{} for setting $\mpbout$, and then have $H$ to be
% set in \!\pcol@bg@preposttop! for the \bgpaint{} of \postenv.
% 
% \item
% In \!\pcol@setcolwidth@r!, it has
% $\!\textwidth!-(\Cto-\Cfrom-1)\!\columnsep!$ being the base of $\w_c$.
% 
% \item
% In \!\pcol@setcw@getspec@i!, it is let have the natural width of a
% column\slash gap specification, to be used in \!\pcol@setcolwidth@s!,
% \!\pcol@setcw@accumwd! and \!\pcol@setcw@set!, while in the last of them
% it finally has $\w_c$ or $\gap_c$.
% 
% \item
% In $\!\pcol@setcw@calcf!\<x\>\<y\>\<z\>$, at first it has $y$, then
% $y/2^{k_2}$, and then $\lceil y/2^{k_2+k_3}\rceil$, where $k_2$ and $k_3$
% are scaling parameters for good approximation.
% 
% \item
% In \!\pcol@switchcol!, it is let have what \!\pcol@ensurevspace! has so
% that the dimensional expression in it is evaluated in the macro and the
% evaluateion result is assigned to $\VE=\!\pcol@@ensurevspace!$.
% 
% \item
% In $\!\ensurevspace!\ARg{space}$, it is let have $\arg{space}$ to ensure
% the argument is a dimension including forced one.
% 
% \item
% In \!\pcol@fntextbody!, it has the height-plus-depth of the \!\vbox! in
% which the footnote is encapsulated.
% 
% \item
% In \!\pcol@set@color@push!, it has the width of the \!\vbox! to be
% \!\insert!ed, which is $m$\,|sp| for a math-mode push of $\mcelt_{i,m}$ or
% 0 for a non-math one $\celt_i$.
% 
% \item
% In $\!\pcol@bg@defext!\Arg{d}\Arg{e}$, it is let have $e$ to confirm $e$
% is a proper dimension.
% \end{itemize}
% 
% \item[\Uidx{\!\@tempdimb!}]
% is a \!\dimen! register for temporary use.  The usages of this register
% are as follows.
% 
% \begin{itemize}
% \item
% In \!\pcol@makecol!, it is used to measure the height-plus-depth $h_i$ of
% decapsulated \!\box!|255| and its original form to add an element
% $\spt(H_i,h_i)$ to $\pp^s(\ptop)$ for a \mctext{} captured in the box.
% 
% \item
% In \!\pcol@ioutputelt!, if has the height-plus-depth of \spanning{}
% $\pp^b(q)$ to be temporarily added to \!\topmargin!.
% 
% \item
% In \!\pcol@buildcolseprule! it has $H_0+h_0$, while in its callee
% $\!\pcol@buildcselt!\<H_i\>\~\<h_i\>$ it has $H_{i-1}+h_{i-1}$ and then
% $H_i+h_i$ where $\spt(H_i,h_i)\in\pp^s(p)$.
% 
% \item
% In \!\pcol@bg@paintregion@i!, it is let have $y_0$ of $\bgr_a^{[c]}$.
% 
% \item
% In \!\pcol@output@switch!, it is let have the height-plus-depth $h_i$ of
% \!\@holdpg! having a \mctext{} to add an element $\spt(H_i,h_i)$ to
% $\pp^s(\ptop)$.
% 
% \item
% In \!\pcol@shrinkcolbyfn!, it is let have the inverse of the \!\skip!
% component of the argument \!\insert! register of the macro, so that in
% \!\pcol@startcolumn! and \!\pcol@restartcolumn! it has that of $\pp^f(p)$
% if $p$ has \Scfnote{}s, or 0 otherwise, and then is referred to by
% \!\pcol@deferredfootins! which then lets the register have the height cap
% of $\df$ splitting.
% 
% \item
% In \!\pcol@addmarginpar!, \!\pcol@getmparbottom! and \!\pcol@getmpbelt!,
% it is let have the vertical space to be occupied
% by the marginal text to be put, being the second argument of
% \!\pcol@getmparbottom!.
% 
% \item
% In \!\pcol@sync! and \!\pcol@measurecolumn!, it has $\VB$ and then, in the
% former, it has $\VP+v(f)$, $\VT$ or $\VT+\VB+v(f)$ according to the
% contents of the page to be \sync{}ed.
% 
% \item
% In \!\pcol@makefcolelt!, it has the size of vertical space consumed by a
% float.
% 
% \item
% In \!\pcol@synccolumn!, it has $\VT-\vc(t)$ being the vertical space from
% the bottom of $\cc_c(\vb^b)$ to the \sync{}ation point.  If the
% \sync{}ation point is defined by a column without main vertical list but
% with top floats, then the register is let have
% $\VT-\vc(t)+\!\textfloatsep!-\!\floatsep!+10000\,|pt|$ to be set in
% $\cc_c(\tf)=\!\pcol@textfloatsep!$ as the space below the \mvlfloat{}
% biased by 10000\,|pt| to indicate the last float is the \mvlfloat{}.
% 
% \item
% In \!\pcol@setcolwidth@r!, it has
% $\!\textwidth!-(\Cto-\Cfrom-1)\!\columnsep!-\sum_{d=0}^{k-1}w_d$ being the
% base of $\w_c$ for $c\In{\Cfrom{+}k}{\,\Cto}$ where $k$ is the number of
% fractions given in the argument of \!\columnratio!.
% 
% \item
% In $\!\pcol@setcw@calcf!\<x\>\<y\>\<z\>$, at first it is let have $x$,
% then $x\cdot2^{k_1}$, then
% $z'=\break\lfloor(x\cdot2^{k_1})/\lceil{}y/2^{k_2+k_3}\rceil\rfloor$, and
% finally
% $Z=z\times1\,|pt|=z'\cdot2^{16-k}$ referred to by
% \!\pcol@setcw@calcfactors! as $\phi_f=(\WT-W)/F$, where $k_1$, $k_2$ and
% $k_3$ are scaling parameters for good approximation and $k=k_1+k_2+k_3$.
% 
% \item
% In \!\pcol@extract@fil@ii! and \!\pcol@extract@fil@iii!, it is let have
% the infinite stretch factor of a column\slash gap specification
% represented with the unit \!\pcol@setcw@filunit!, to be used in
% \!\pcol@setcolwidth@s!, \!\pcol@setcw@accumwd!, and \!\pcol@setcw@set!.
% 
% \item
% In \!\pcol@fntextbody!, it has $\!\textheight!-\!\skip!\!\footins!$ as the
% cap of the footnote.
% \end{itemize}
% 
% 
% \item[\Uidx{\!\@tempdimc!}]
% is a \!\dimen! register for temporary use.  It is let have values as
% follows.
% 
% \begin{itemize}
% \item
% $H_i-(H_{i-1}+h_{i-1})$ in $\!\pcol@buildcselt!\<H_i\>\<h_i\>$.
% 
% \item
% $x_1$ of $\bgr_a^{[c]}$ in \!\pcol@bg@paintregion@i!.
% 
% \item
% $t+h$ in $\!\pcol@getmparbottom!\<t\>\<h\>$.
% 
% \item
% $\max(t,b_{i-1})+h$ in $\!\pcol@getmpbelt!\<t_i\>\<b_i\>$ invoked from
% $\!\pcol@getmparbottom!\<t\>\~\<h\>$.
% 
% \item
% $\DT$ in \!\pcol@sync!, \!\pcol@measurecolumn! and \!\pcol@synccolumn!.
% 
% \item
% \!\floatsep! or \!\@fpsep! in \!\pcol@makefcolumn! and
% \!\pcol@makefcolelt!.
% 
% \item
% $\w_c$ being the width of column $c$ in \!\pcol@setcolwidth@r!.
% 
% \item
% $\WT-W$ in \!\pcol@setcw@calcfactors!.
% 
% \item
% At first for calculation of $y/2^{k_2}$ and then $z'/2^k\approx x/y$
% in $\!\pcol@setcw@calcf!\<x\>\<y\>\~\<z\>$ where $k_2$ and $k$ are scaling
% parameters for good approximation.
% \end{itemize}
% \end{description}
% 
% 
% 
% \subsubsection{Skip Registers}
% 
% \begin{description}
% \item[\Uidx{\!\baselineskip!}]
% is \TeX's primitive register to have the vertical skip to separate
% adjacent baselines.  It is referred to by \!\pcol@output! and
% \!\pcol@output@start! to examine if \!\@colroom! is unexpectedly small,
% and by \!\pcol@switchcol! to give it to \!\ensurevspace! to let
% \!\pcol@ensurevspace! have the default value.
% 
% \item[\Uidx{\!\topskip!}]
% is \TeX's primitive register to have the vertical skip from the top edge
% of a page to the baseline of its first vertical item.  It is let be 0 by
% \!\pcol@output@start! if we have \preenv{} and is saved in $\pp^t(0)$,
% while \!\pcol@startpage! lets it be \!\pcol@topskip!, into which
% \!\pcol@zparacol! saves the value outside \env{paracol} environment,
% saving the value in $\pp^t(p)$.  Then the register is restored from
% $\pp^t(p)$ by \!\pcol@getpelt! and \!\pcol@sync!, while
% \!\pcol@synccolumn! refers to the value restored by the latter to adjust a
% \sync{}ation point.  The macro \!\pcol@putbackmvl! lets the register have
% 0 when it starts a \mctext{} because it originally follows \prespan{} in
% the \colpage{} to be restarted rather than at the page top.  The macro
% \!\pcol@output@end!  temporarily lets the register have 0 if we have
% non-empty columns in the last page, while \!\endparacol! restores it from
% \!\pcol@topskip! for the pages outside \env{paracol} environmnet.
% 
% \item[\Uidx{\!\splittopskip!}]
% is \TeX's primitive register to have the vertical skip inserted at the
% beginning of the second half of the box split by \!\vsplit! or \TeX's
% internal operation for splitting an \!\insert! at a page break.  The
% register is temporarily let have 0 by \!\pcol@deferredfootins! when it
% splits $\df$ so that the second half does not have any skip at the top.
% The register is also let have \!\footnotesep! in \!\pcol@fntextbody! for
% footnote typesetting.
% 
% \item[\Uidx{\!\parskip!}]
% is \TeX's primitive register to have the vertical skip inserted above each
% paragraph.  It is referred to by \!\pcol@zparacol! to nullify the
% insertion going to be made by the first \!\item! of a \env{list}-like
% environment, when the macro finds the \env{paracol} environment to start
% is at the very beginning of a \env{list}-like environment.
% 
% \item[\Uidx{\!\fill!}]
% is an API \!\skip! register to have a skip 0\,|pt| |plus| 1\,|fill|.  In
% our macros, however, it is used as a keyword in \!\pcol@setcw@getspec!,
% \!\pcol@setcw@getspec@i! and \!\pcol@setcw@fill! to extract the infinite
% stretch factor $f$ given as $f\!\fill!$ in the specification.
% 
% \item[\Uidx{\!\itemsep!}]
% is an API \!\skip! register to have the vertical skip inserted above each
% non-first \!\item! in \env{list}-like environments.  It is referred to by
% \!\pcol@zparacol! to nullify the insertion going to be made by the first
% \!\item! of a \env{list}-like environment, when the macro finds the
% \env{paracol} environment to start is at the very beginning of a
% \env{list}-like environment.
% 
% \item[\Uidx{\!\floatsep!}]
% is an API \!\skip! register to have the vertical skip between adjacent floats.
% It is referred to by \!\pcol@cflt! to cancel the skip following the last
% float, by \!\pcol@makefcolumn! to let \!\pcol@makefcolelt! examine the
% capacity of a \fcolumn{} in the \lpage, by \!\pcol@addflhd! and
% \!\pcol@hdflelt!  to measure the total height of top and bottom floats,
% and by \!\pcol@sync! to calculate the space below the \mvlfloat.
% 
% \item[\Uidx{\!\textfloatsep!}]
% is an API \!\skip! register to have the vertical skip between the main
% vertical list and top\slash bottom floats.  It is referred to by
% \!\pcol@output@start! to calculate the room for each \colpage{} in the
% \spage{} with bottom floats in the \preenv{}, by \!\pcol@combinefloats! to
% insert a skip below the bottom floats in the \preenv{} and \lpage{},
% by \!\pcol@measurecolumn! to take this skip into account in the
% calculation of $\VPP$, by \!\pcol@addflhd!  to measure the vertical space
% for top and bottom floats, and by \!\pcol@synccolumn! to calculate the
% \sync{}ation point for columns with top floats.
% 
% \item[\Uidx{\!\dblfloatsep!}]
% is an API \!\skip! register to have the vertical skip between adjacent
% \pwise{} floats, and is used in \!\pcol@startpage! to cancel the skip
% below the last float.
% 
% \item[\Uidx{\!\dbltextfloatsep!}]
% is an API \!\skip! register to have the vertical skip between the last
% \pwise{} float and the top of columns, and is used in
% \!\pcol@startpage! to put the skip.
% 
% \item[\Uidx{\!\@topsep!}]
% is a \!\skip! register to have the vertical skip inserted above the first
% \!\item! of a \env{list}-like environment.  The actual value is determined
% by \!\@trivlist! from API parameters \!\topsep!, \!\partopsep! and
% \!\parskip! depending on how the environment appears.  The skip in the
% register is inserted by \!\pcol@zparacol! when it finds the \env{paracol}
% environment to start is at the very beginning of a \env{list}-like
% environment.
% 
% \item[\Uidx{\!\@fptop!}]
% is a \!\skip! register to have the vertical skip inserted at the top of
% a \fcolumn, and is used in \!\pcol@makefcolpage!.
% 
% \item[\Uidx{\!\@fpsep!}]
% is a \!\skip! register to have the vertical skip between adjacent floats
% in a \fcolumn, and is used in \!\pcol@makefcolpage!.
% 
% \item[\Uidx{\!\@fpbot!}]
% is a \!\skip! register to have the vertical skip inserted at the bottom of
% a \fcolumn, and is used in \!\pcol@makefcolpage!.
% 
% \item[\Uidx{\!\@tempskipa!}]
% is a \!\skip! register for temporary use.  It is used in the following
% macros.
% 
% \begin{itemize}
% \item
% \begin{Sloppy}{2400}%
% \!\pcol@makecol!, \!\pcol@startpage!, \!\pcol@outputelt!,
% \!\pcol@output@switch!, \!\pcol@flushcolumn! and \!\pcol@makeflushedpage!
% to throw $\pp^t(p)$ away because we don't need it.
% \end{Sloppy}
% 
% \item
% \!\pcol@output@start! to determine \!\topskip! for the \spage.
% 
% \item
% \!\pcol@setcw@getspec@i! and \!\pcol@setcw@fill! to have the width
% specification of a column or gap.
% \end{itemize}
% It is also used in the top level invocation of \!\pcol@defkw! with a glue
% of $0\,|pt|\ |plus|\ 1\,|fil|\~\ |minus|\ 1\,|fil|$.
% \end{description}
% 
% 
% 
% \subsubsection{Box Registers}
% 
% \begin{description}
% \item[\Uidx{\!\strutbox!}]
% is an API \!\box! register to have the strut for the current font size.
% It is used in \!\pcol@fntextbody!\marg{text} to let \!\splitmaxdepth! have
% its depth, and to let the last line of the footnote $\arg{text}$ have its
% depth at shallowest.
% 
% \item[\Uidx{\!\@cclv!}] 
% is a \!\box! register but \TeX{} defines that it has the main vertical
% list when \!\output! routine is invoked.  It is referred to by
% \!\pcol@makecol! when it has a broken \mctext{} to measure its
% height-plus-depth for the element to be added to $\pp^s(\ptop)$ and to
% update it combining its contents with \prespan{} optionally shifting it
% left by passing the register to \!\pcol@shiftspanning!.  The macro also
% uses the register together with its callee \!\pcol@unvbox@cclv! to add
% stretch\slash shrink factor of \!\skip!\!\footins! at its bottom for a
% \colpage{} in a page having \Scfnote{}s.  The macro \!\pcol@specialoutput!
% examines the register to discard the dummy \!\vbox! inserted in it by
% \!\pcol@invokeoutput!.  The other users \!\pcol@output@start!,
% \!\pcol@makenormalcol!, \!\pcol@flushcolumn! and \!\pcol@imakeflushedpage!
% let the register have the main vertical list of \preenv{} or a column to
% be passed to \!\@makecol!, and \!\pcol@flushcolumn! also takes care the
% skip above \Scfnote{}s.
% 
% \item[\Uidx{\!\voidb@x!}]
% is a \!\box! register to be void ($\bot$) always.  It is used to
% initialize \!\pcol@prespan! and \!\pcol@rightpage! at their declaration,
% and is referred to by the following macros.
% 
% \begin{itemize}
% \item
% \!\pcol@makecol! to make \!\pcol@currfoot! void unless \Scfnote{}s in
% \!\foot~ins! is saved into $\pp^f(p)$.
% 
% \item
% \!\pcol@startpage! to let
% $\pp^i(\ptop)=\bot$ if the new \tpage{} does not have \spanning{} and
% $\pp^f(\ptop)=\bot$ for all \fpage{}s and the new \tpage.
% 
% \item
% \!\pcol@outputelt! to initialize \!\@outputbox!.
% 
% \item
% \!\pcol@ioutputelt! to examine if $\S_c$ is empty.
% 
% \item
% \!\pcol@output@start! to let $\pp^f(0)=\bot$, and $\Celt^c=\bot$ if
% $\Celtshadow^c$ is undefined.
% 
% \item
% \!\pcol@output@switch! to let $\cc_c(\ft^b)=\bot$ if the column does
% not have \Mcfnote{}s.
% 
% \item
% \!\pcol@getcurrfoot! to let \!\footins! be void if so.
% 
% \item
% \!\pcol@setcurrcolnf! to let $\cc_c(\ft^b)=\bot$ because the column $c$
% does not have \Mcfnote{}s.
% 
% \item
% \!\pcol@putbackmvl! to let $\!\pcol@prespan!=\bot$ if a \mctext{} really
% starts from the top of a \colpage, and $\csts=\bot$ if the \colpage{}
% $\cc_c(\vb)$ to be restarted is non-empty.
% 
% \item
% \!\pcol@savecolorstack! to let $\csts=\bot$ or its first item be $\bot$ if
% $\CST^c$ or $\Celt^c$ is $\bot$, resepectively.
% 
% \item
% \!\pcol@savefootins! to let its argument macro have a void box if
% \!\@freelist! is exhausted.
% 
% \item
% \!\pcol@makeflushedpage! to initialize \!\@outputbox! and
% \!\pcol@rightpage! to be $\bot$ if the flushed page does not have
% \spanning{}, and to let $\pp^f(\ptop)=\bot$ after putting it in the
% \lpage{} so that \!\pcol@output@end! will be unaware of the \scfnote{} and
% non-\Mgfnote{}s.
% 
% \item
% \!\pcol@flushfloats! to let $\!\pcol@rightpage!=\bot$ if \parapag{}ing is
% not in effect.
% 
% \item
% \!\pcol@output@end! to let $\!\pcol@rightpage!=\bot$ if
% the last page has nothing other than \spanning{} being \pwise{} floats
% and thus we don't have the right \parapag{}e.  The macro also lets
% $\Celt^c=\bot$ for all $c$ and $\cst=\bot$.
%
% \item
% \!\pcol@com@flushpage! and \!\pcol@com@clearpage! gives the void box to
% \!\pcol@flushclear! as its argument to mean these macros are only aware
% of \CSIndex{ifpcol@flush} as the result of \pfcheck.  The macro
% \!\endparacol! also does that if the footnote typesetting is \mgfnote.
% \end{itemize}
% 
% \item[\Uidx{\!\@holdpg!}]
% is a \!\box! register to have the main vertical list when \!\output! is
% invoked with a special \!\penalty! code.  It is let have that by
% \!\pcol@specialoutput!, and is referred to by \!\pcol@output@start!  and
% \!\pcol@makenormalcol! for \preenv{}, and by \!\pcol@output@switch! for the
% column from which we are leaving.
% 
% \item[\Uidx{\!\@outputbox!}]
% is a \!\box! register to have a partial or the complete ship-out image
% of a column or a page.  The usages of the register are as follows.
% 
% \begin{itemize}
% \item
% In \!\pcol@@makecol!, it has a \colpage{} made by \!\@makecol! for
% \!\pcol@flush~column! and \!\pcol@imakeflushedpage!.
% 
% \item
% In \!\pcol@combinefloats!, it has a \colpage{} to which top and bottom
% floats are combined.
% 
% \item
% In \!\pcol@cflt!, it has a \colpage{} to which top floats are combined.
% 
% \item
% In \!\pcol@opcol!, it has the complete \colpage{} built by \!\@makecol!.
% 
% \item
% In \!\pcol@startpage!, it has the complete \fpage{} built by \!\@tryfcolumn!.
% 
% \item
% In \!\pcol@outputelt!, it has the complete (left parallel-) page to be
% shipped out by \!\@outputpage!.
% \Index{parallel-paging}
% 
% \item
% In \!\pcol@outputpage@r!, it is temporarily made \!\let!-equal to
% \!\pcol@rightpage! so that the box is shipped out by \!\pcol@@outputpage!
% being \LaTeX's \!\@outputpage! instead of the real \!\@outputbox!.
% 
% \item\begin{Sloppy}{1700}
% In \!\pcol@output@start!, it has the \preenv{} built by
% \!\pcol@makenormalcol!.
% \end{Sloppy}
% 
% \item
% In \!\pcol@combinefootins!, it is let have the \preenv{} with footnotes.
% 
% \item
% In \!\pcol@flushcolumn!, it has a flushed \colpage{} built by \!\@makecol!
% or a \fcolumn{} built by \!\@makefcolumn!.
% 
% \item
% In \!\pcol@output@flush! and \!\pcol@output@clear!, it has a flushed page
% built by \!\pcol@makeflushedpage! and \!\pcol@imakeflushedpage! in which
% it has each of flushed \colpage{} built by \!\@makecol!.
% 
% \item
% In \!\pcol@flushfloats!, it has the complete (left parallel) page for
% flushed \fcolumn{}s.
% 
% \Index{parallel-paging}
% 
% \item
% In \!\pcol@iflushfloats!, it has a \fcolumn{} built by
% \!\pcol@makefcolumn!.
% 
% \item
% In \!\pcol@output@end!, it has the ship-out image of the \lpage{} of a
% \env{paracol} environment built by \!\pcol@makeflushedpage! and
% \!\pcol@imakeflushedpage!.
% \end{itemize}
% 
% \item[\Uidx{\!\@tempboxa!}]
% is a \!\box! register for temporary use.  The usages of the register are
% as follows.
% 
% \begin{itemize}
% \item
% In \!\pcol@makecol!, it is used to decapsulate \!\box!|255| containing a
% broken \mctext{}.  In the macro and \!\pcol@output@switch!, it is also
% used as a waste bascket to make \!\footins! void when it contains
% \Scfnote{}s in a non-\tpage.
% 
% \item\begin{Sloppy}{2100}
% In \!\pcol@cflt! and \!\pcol@startpage!, it has top column-\slash
% \pwise{} floats combined by the application of
% \!\@comflelt!\slash\!\@comdblflelt! to \!\@toplist!\slash\!\@dbltoplist!
% respectively.
% \end{Sloppy}
% 
% \item
% In \!\pcol@phantom!$\<b\>$, it has an empty box whose height and depth are
% equal to those of the argument box $b$.
% 
% \item
% By \!\pcol@buildcolseprule! and its callees \!\pcol@buildcselt@S! and
% \!\pcol@buildcselt!, it is let have the painted \bground{}s for columns,
% \csepgap{}s and \mctext{}s in a page, and then is put into the ship-out
% image of the page by \!\pcol@ioutputelt!, \!\pcol@imakeflushedpage! or
% \!\pcol@iflushfloats!.
% 
% \item
% In \!\@outputpage!, it is let have the painted \bground{} of the right
% page referred to by its callee \!\pcol@outputpage@r!.
% 
% \item
% In $\!\pcol@bg@paint@i!\Arg{body}$, it is let have painted \bground{}s
% built by $\arg{body}$.
% 
% \item
% In $\!\pcol@bg@paintregion!\arg{a}\arg{c}$, it is let have painted
% \bground{} of the region $\bgr_a^{[c]}$.
% 
% \item
% In \!\pcol@specialoutput!, it is used to discard the dummy \!\vbox! put by
% \!\pcol@invokeoutput!.
% 
% \item
% In \!\pcol@makenormalcol!, it is used to save \!\footins! into it and make
% it $\bot$ temporarily to exclude \mgfnote{} footnotes from \spanning{} for
% \preenv.
% 
% \item
% In \!\pcol@ifempty!$\arg{box}\arg{then}\arg{else}$, it is used to examine
% if $\arg{box}$ is empty.
% 
% \item
% In $\!\pcol@scancst!\arg{box}$ and \!\pcol@iscancst!, it is used to have
% what $\cst$ or $\csts$ has after the scan of $\arg{box}\in\{\cst,\csts\}$.
% 
% \item
% In \!\pcol@savecolorstack!, it is used to have the \!\vbox! for $\Celt^c$
% to be placed at the top of $\csts$.
% 
% \item
% In \!\pcol@deferredfootins!, it is used to have the first half split from
% $\df$ being the deferred footnotes to be \!\insert!ed.
% 
% \item
% In \!\pcol@fntextbody!\marg{text}, it is used to encapsulate $\arg{text}$
% in it.
% 
% \item
% In \!\pcol@icolumncolor!, it is used to have a \!\vbox! to be \!\insert!ed
% for the update of $\Celt^c$.
% 
% \item
% In \!\pcol@set@color@push!, it is used to have a \!\vbox! to be \!\insert!ed
% to push $\celt_i$ or $\mcelt_{i,m}$ to $\cstraw$.
% 
% \item
% In \!\pcol@reset@color@pop! and \!\pcol@reset@color@mpop!, it is used to
% have a \!\vbox! to be \!\insert!ed to add $\celtpop_i$ or
% $\mceltpop_{i,m}$ to $\cstraw$.
% \end{itemize}
% \end{description}
% 
% 
% 
% \subsubsection{Token Registers}
% 
% \begin{description}
% \item[\Uidx{\!\output!}]
% is \TeX's primitive to have \!\output! routine.  It is let have
% \!\pcol@output! as its sole token by \!\pcol@zparacol!.
% 
% \item[\Uidx{\!\everypar!}]
% is \TeX's primitive to have tokens inserted at the beginning of each
% paragarph.  In \!\pcol@sptext! and \!\pcol@com@endcolumn!, it is
% \!\global!ized to keep its contents after the end of a group.  In
% \!\pcol@output@switch!, its contents are broadcasted to $\cc_c(\ep)$ for
% all $c\In0\C$ if columns are \sync{}ed with a \mctext.  Then these values
% or that simply given in a column are saved into $\cc_c(\ep)$ by
% \!\pcol@setcurrcol!, and then restored from it by \!\pcol@iigetcurrcol!.
% 
% \item[\Uidx{\!\everyvbox!}]
% is \TeX's primitive to have tokens inserted at the beginning of each
% \!\vbox!.  In \!\pcol@zparacol!, after tokens in it are saved into
% \!\pcol@everyvbox!, it is let have a \!\the!-reference to
% \!\pcol@everyvbox! and |\pcol@innertrue| to turn
% $\CSIndex{ifpcol@inner}=\true$, and then the register itself is made
% \!\let!-equal to \!\pcol@everyvbox!.  In addition, it is let have tokens
% in \!\pcol@everyvbox! if a \!\global! assignment to the register is made
% in the \env{paracol} just having been closed.  Another usage of this
% register is to insert a painted page \bground{} to the \!\vbox! to be
% \!\shipout! by \!\pcol@@outputpage! being \LaTeX's \!\@outputpage!, and is
% used for this purpose by \!\pcol@outputpage@l! and \!\pcol@outputpage@r!,
% and by \!\pcol@outputpage@ev! to nullify this special function for other
% inside \!\vbox!es.
% 
% \item[\Uidx{\!\@temptokena!}]
% is a \!\toks! register for temporary use.  It is used in
% \!\pcol@output@switch! to broadcast \!\everypar! to $\cc_c(\ep)$ for all
% $c\In0\C$.
% 
% \end{description}
% 
% 
% 
% \subsubsection{Switches}
% 
% \begin{description}
% \item[\Uidx{\CSIndex{if@twocolumn}}]
% is a switch to be $\true$ iff multi-column pages are being typeset.  It is
% turned $\true$ by \!\pcol@zparacol!, and then turned $\false$ by
% \!\endparacol!.  In addition, it is turned $\false$ when \!\pcol@output!
% finds that the \!\output! request for a page break outside \env{paracol}
% is sneaked into our own \!\output! routine, in order to avoid that
% \LaTeX's original \!\output! routine misunderstands it is working on a
% two-columned document.  The switch is examined by \LaTeX's own macros
% including old \!\end@dblfloat! kept in our own \!\pcol@end@dblfloat!.  It
% is also examined by \!\pcol@zparacol! before being turned $\true$ to
% ensure it is $\false$ or to complain about the inappropriateness otherwise.
% 
% \item[\Uidx{\CSIndex{if@firstcolumn}}]
% is a switch to be $\true$ iff the first column is being typeset.  Its
% truth value is determined by \!\pcol@addmarginpar! to tell
% \!\pcol@@addmarginpar!, \!\pcol@getmparbottom@i! and \!\pcol@setmpbelt@i!
% the margin which a marginal note goes to.
% 
% \item[\Uidx{\CSIndex{if@twoside}}]
% is a switch to be $\true$ iff two-sided page typesetting is in effect and
% thus even numbered page may have their own left margins, headers and
% footers different from those for odd numbered pages.  Besides the
% initialization by the main class file such as \textsf{article.cls}
% according to the class option |twoside|, the switch is \!\global!ly turned
% $\false$ by \!\pcol@twosided! for the case in which API macro \!\twosided!
% does not have `|p|' in its optional argument, and then \!\global!ly turned
% $\true$ by \!\pcol@twosided@p! which is invoked when the argument contains
% `|p|', or the API macro does not have the argument at all.  Then the
% switch is referred to by \!\pcol@outputpage@l!, \!\pcol@outputpage@r! and
% \!\pcol@bg@swappage! to decide the left margin of even numbered pages,
% i.e., \!\evensidemargin! if the switch is $\true$ or \!\oddsidemargin!
% otherwise.  The switch is also referred to by \!\pcol@com@cleardoublepage!
% to have a blank page if the switch is $\true$ and the command
% \!\cleardoublepage! is used in an odd-numbered page.
% 
% \item[\Uidx{\CSIndex{if@reversemargin}}]
% is a switch to be $\true$ iff \!\reversemarginpar! is specified to reverse
% the side which marginal notes go to.  It is examined by
% \!\pcol@addmarginpar! as a factor to decide the margin which a marginal
% note goes to, and by \!\pcol@do@mpbout@i! for the same purpose but for
% marginal notes in pre-environment or \postenv.
% 
% \Index{pre-environment stuff}
% 
% \item[\Uidx{\CSIndex{if@mparswitch}}]
% is a switch to be $\true$ iff it is specified by, for example,
% \texttt{twoside} option of a class such as \textsf{article}, that marginal
% notes in even numbered pages go to thier left margin.  It is examined by
% \!\pcol@do@mpbout@i! as a factor to decide the margin which a maginal note
% goes to in pre-environment or \postenv{}.
% 
% \Index{pre-environment stuff}
% 
% \item[\Uidx{\CSIndex{if@nobreak}}]
% is a switch to be $\true$ iff the last paragraph is for a sectioning
% command.  The switch is saved into $\cc_c(\sw)$ together with
% \CSIndex{if@afterindent} by \!\pcol@setcurrcol!, and then restored from
% it by \!\pcol@iigetcurrcol!.  The macro \!\pcol@output@switch! refers to
% it to broadcast its value set by a \mctext{} to $\cc_c(\sw)$ for all
% $c\In0\C$, while \!\pcol@output@start! and \!\pcol@restartcolumn! insert
% $\!\penalty!=10000$ by \!\nobreak! if the switch is $\true$.  This
% conditional \!\nobreak!  is also done by \!\pcol@icolumncolor!,
% \!\pcol@set@color@push!, \!\pcol@reset@color@pop! and
% \!\pcol@reset@color@mpop! to avoid a break after an \!\insert!.  The macro
% \!\pcol@zparacol! also exmaines the switch, but with the truth value in it
% given outside \env{paracol} environment, to invoke \!\@nbitem! if $\true$
% when the macro finds the \env{paracol} environment to start is at the very
% beginning of a \env{list}-like environment.
% 
% \item[\Uidx{\CSIndex{if@newlist}}]
% is a switch to be $\true$ in the duration after a \env{list}-like
% environment starts and until its first \!\item! appears.  The switch is
% examined by \!\pcol@zparacol! to know if the \env{paracol} environment to
% start is at the very beginning of a \env{list}-like environment and, if
% so, is turned $\false$ by the macro after it inserts vertical skips
% pretending the first \!\item! is given.
% 
% \item[\Uidx{\CSIndex{if@inlabel}}]
% is a switch to be $\true$ in the duration after an \!\item! appears and
% until its first paragraph is given.  The switch is examined by
% \!\pcol@zparacol! together with \CSIndex{if@newlist} to know if the
% \env{paracol} environment to start is at the very beginning of a
% \env{list}-like environment ($\false$) and not \env{trivlist}-like one
% ($\true$).
% 
% \item[\Uidx{\CSIndex{if@afterindent}}]
% is a switch to be $\true$ iff a sectioning commmand tells that the fisrt
% paragraph following it is to be indented.  The switch is saved into
% $\cc_c(\sw)$ together with \CSIndex{if@nobreak} by \!\pcol@setcurrcol!,
% and then restored from it by \!\pcol@iigetcurrcol!.  The macro
% \!\pcol@output@switch! refers to it to broadcast its value set by a
% \mctext{} to $\cc_c(\sw)$ for all $c\In0\C$.
% 
% \item[\Uidx{\CSIndex{if@fcolmade}}]
% is a switch to be $\true$ iff a \fcolumn{} or \fpage{} is built by
% \!\@tryfcolumn! or \!\@makefcolumn!.  The value is set by \!\@tryfcolumn!
% for $\cc_c(\dl)$ is referred to by \!\pcol@output!, \!\pcol@startcolumn!
% and \!\pcol@freshpage!, while that for \!\@dbldeferlist! is referred to by
% \!\pcol@startpage!.  The value set by \!\@makefcolumn! for $\cc_c(\dl)$ is
% referred to by \!\pcol@flushcolumn!, while that for \!\@dbldeferlist! is
% referred to by \!\pcol@output@clear!.  The macros \!\pcol@flushfloats! and
% \!\pcol@iflushfloats! also refer to the switch to build pages having only
% \fcolumn{}s and turn the switch $\true$ or $\false$ by themselves to know
% such paegs are still to be built or not.  The macro \!\pcol@output@end!
% also turns the switch $\true$ if a \lpage{} will be followed by page(s)
% having \fcolumn{}s to tell that to \!\pcol@flushfloats!.
% 
% \item[\Uidx{\CSIndex{if@tempswa}}]
% is a switch for temporary use.  The usages of the switch are as follows.
% 
% \begin{itemize}
% \item
% In \!\pcol@checkshipped!, it is turned $\true$ iff $\S_c$ for all
% $c\In0\C$ have \colpage{}s to be shipped out, and then it is examined by 
% \!\pcol@opcol!.
% 
% \item
% In \!\pcol@nextpage! and \!\pcol@nextpelt!, it is $\true$ until
% \!\pcol@nextpelt! finds the first $q$ such that $q>p$ and
% $\pp^h(q)\geq0$ to mean $q$ is not for a \fpage{}, so that we let $p=q$
% to skip \fpage{}s following to the old $p$ if any.
% 
% \item
% In \!\pcol@outputcolumns! and \!\pcol@outputelt!, it is $\true$ until
% \!\pcol@outputelt! finds the first $q$ such that $q\geq\pbase$ and
% $\pp^h(q)\geq0$ to mean $q$ is not for a \fpage{}, and the arguemnt of
% \!\pcol@outputcolumns! is 0 to mean that it is not for page flushing, so that
% we ship out $q$ and all \fpage{}s following it if any.
% 
% \item
% In \!\@outputpage! it is let have the value of \CSIndex{ifpcol@bg@painted}
% indicating if \bgpaint{} for the left page is done, and then it is
% examined by \!\pcol@outputpage@l! to determine whether the \bground{} is
% put into the final ship-out image.
% 
% \item
% In \!\pcol@makenormalcol!, it is $\true$ iff the footnotes in \preenv{} is
% included in \!\@outputbox! which the macro builds.
% 
% \item
% In \!\pcol@output@switch!, at first it holds \CSIndex{if@nobreak} of the
% \mctext{} if columns are \sync{}ed with it to broadcast
% \CSIndex{if@nobreak} to all $\cc_c(\sw)$.  Then it is turned $\true$ iff
% $\CSIndex{ifpcol@sync}=\true$ for \sync{}ation or
% $\CSIndex{ifpcol@clear}={\true}$ for flushing, so as to invoke
% \!\pcol@sync!.  And finally, it is turned $\true$ iff
% $\CSIndex{ifpcol@clear}={\false}$ or $\CSIndex{ifpcol@sync}=\true$, so as
% to invoke \!\pcol@restartcolumn!.
% 
% \item
% In \!\pcol@restartcolumn!, it is turned $\true$ iff footnote typesetting
% is \scfnote{} and $p<\ptop$.
% 
% \item
% In \!\pcol@scancst!, it is initialized to be $\true$.  Then it is referred
% to by \!\pcol@iscancst! for each $\celt\in\cstraw$ to update $\Celt^c$ and
% then turned $\false$ when the first one is found.
% 
% \item
% In \!\pcol@savecolorstack!, it is $\true$ iff either $\cst\neq\bot$ or
% $\Celt^c\neq\bot$, i.e., $\CST^c$ to be saved is not $\bot$.
% 
% \item
% In \!\pcol@getmparbottom!, it is intialized to be $\false$ and then may be
% turned $\true$ by \!\pcol@getmpbelt! if it finds a gap between two marginal
% notes to accommodate that to be added, and then examined by
% \!\pcol@getmparbottom! to know the fact.
% 
% \item
% In \!\pcol@sync!, it is turned $\true$ iff the \sync{}ed or flushed page
% can be built by \!\pcol@synccolumn!.
% 
% \item
% In \!\pcol@makefcolumn! having non-empty $\cc_c(\tl)$, it is turned
% $\false$ iff the macro is acting on a column in the \lpage{}, $\cc_c(\dl)$
% is emptied by the macro itself, and the total size of the floats to be put
% in the \fcolumn{} being built by the macro is less than \!\@fpmin!, to
% mean it is possible that the floats in $\cc_c(\tl)$ is put in the
% \fcolumn{} as top floats.
% 
% \item
% In \!\pcol@measurecolumn! and \!\pcol@addflhd!, it is set to be $\false$
% iff both top floats and the main vertical list are empty, so that
% \!\pcol@measureupdate! examines it for the update of $\VT$ and $\DT$.
% Then it is kept $\false$ iff both of footnotes and bottom floats are
% empty, so that \!\pcol@measurecolumn! examines it for the update of $\VP$
% and \!\pcol@measureupdate! does for $\VPP$ and $\DP$.
% 
% \item
% In \!\pcol@makeflushedpage!, it is made $\false$ iff $\ptop$ is the
% \lpage{}, $\VPP=-\infty$ to mean all columns are empty and
% $\pp^f(\ptop)=\bot$, so as to make the \spanning{} in $\pp^i(\ptop)$ a
% float in \postenv{} if \CSIndex{ifpcol@dfloats} also $\false$.  Then it is
% kept $\false$ if $\CSIndex{ifpcol@dfloats}=\false$ or $\pp^i(\ptop)=\bot$
% to mean nothing is shipped out for \lpage{}.  Then it is made $\false$ iff
% $\ptop$ is the \lpage{} without deferred floats and \Mgfnote{}
% typesetting is in effect, i.e., the switch is $\true$ iff \Scfnote{}s are
% put in the page to be flushed.
% 
% \item
% In \!\pcol@imakeflushedpage!, it is turned $\true$ iff $\cc_c(\tr)=\infty$
% and $\VPP=\pp^h(\ptop)$ to mean the floats in $\cc_c(\tl)$ should be put in
% a \fcolumn{} in the \lpage{} as usual.
% 
% \item
% In \!\pcol@iflushfloats!, it is turned $\true$ iff one or more columns have
% non-empty $\cc_c(\dl)$ after shipping a page for \fcolumn{}s out, so that
% \CSIndex{if@fcolmade} is let have its value after scanning all columns.
% 
% \item
% In \!\pcol@output@end!, it is turned $\true$ iff we built \fcolumn{}s, or
% the main vertical list in the \lpage{} is empty and the page is not the
% \spage, so that we create a new page for the \postenv.
% 
% \item
% In \!\globalcounter!\marg{ctr}, it is turned $\true$ iff $\arg{ctr}\in\CG$
% already.
% 
% \item
% In $\!\pcol@cmpctrelt!\arg{\theta}$, it is turned $\true$ iff $\theta$ is
% not in $\Cc_0$ or $\Val(\theta)\neq\val_0(\theta)$, so that $\theta$ is
% added to \!\@gtempa! being the list of \lcounter{}s to be synchronized.
% 
% \item
% In \!\pcol@switchcolumn!$|[|d|]|$, it is turned $\false$ iff $0\leq d<\C$
% so that we complain $c$ is invalid if the switch is $\true$.
% 
% \item
% In $\!\pcol@ac@caption@def!\arg{s}\arg{t}$, \CSIndex{@tempswatrue} or
% \CSIndex{@tempswafalse} is given as its first argument $s$ by
% $\!\pcol@ac@caption@enable!$ or $\!\pcol@ac@caption@disable!$ respectively,
% so that $|\if@ac@caption@if@|{\cdot}t$ is made \!\let!-equal to $s$ and
% \!\pcol@ac@caption! examines it for enabling\slash disabling
% \!\addcontentsline! respectively.  The macros \!\pcol@ac@caption@if@lof!
% and \!\pcol@ac@caption@if@lot! are initialized to be \!\let!-equal to
% \!\@tempswatrue! as the default.
% 
% \item
% In \!\pcol@icolumncolor!, it is turned $\true$ iff we are in a \!\vbox! or
% in restricted horizontal or math mode.
% 
% \item
% In \!\pcol@backgroundcolor@i!, it is exmained if the root of the
% invocation chain is \!\backgroundcolor! which turns the switch $\true$, or
% \!\nobackgroundcolor! which turns it $\false$, to determine whether the
% \bground{} of a region is painted or not.
% \end{itemize}
% \end{description}
% 
% 
% 
% \subsection{Macros}
% \label{sec:imp-tex-macro}
% 
% \subsubsection{Procedural Macros}
% 
% \begin{description}
% \item[\Uidx{\CSIndex{par}}]
% is \TeX's primitive to end\slash start paragraphs, but may be modified by
% \LaTeX{} to have some special functionality occasionally.  The macro
% \!\pcol@output! makes it \!\let!-equal to \hbox{\!\@@par!} in which the
% \TeX's orinigal definition is kept, while \!\pcol@zparacol! and
% \!\pcol@par! use it as is.
% 
% \item[\Uidx{\!\space!}]
% is an API macro to have a space token.  It is used in \!\pcol@output!,
% \!\pcol@icolumncolor!, \!\pcol@defcseprulecolor@i! and
% \!\pcol@backgroundcolor@ii! for warning messages, and in
% \!\pcol@def@extract@fil! to \!\def!ine the macro \!\pcol@extract@fil!
% having spaces in its argument specification.
% 
% \item[\Uidx{\!\nointerlineskip!}]
% is an API macro to let $\!\prevdepth!=-1000$\,pt to inhibit \TeX's
% baseline progress mechanism.  It is used in \!\pcol@ioutputelt!,
% \!\pcol@makeflushedpage! and \!\pcol@imakeflushedpage! to joint boxes
% without \!\baselineskip! between them, in \!\pcol@outputpage@ev! to
% suppress the \!\baselineskip! insertion after the first box of painted
% \bground{} in the final ship-out image, and in \!\pcol@bg@paint@i!
% for the same purpose for the box having painted \bground{}s.
% 
% \item[\Uidx{\!\offinterlineskip!}]
% is an API macro to let $\!\baselineskip!=-1000$\,pt, $\!\lineskip!=0$ and
% $\!\lineskiplimit!=\!\maxdimen!$ to supress \!\baselineskip! insertion for
% all boxes following this macro.  It is used in \!\pcol@bg@paint@i! to do
% that in the box in which painted \bground{}s are built.
% 
% \item[\Uidx{\!\thepage!}]
% is an API macro to have the representation of the counter \counter{page}.
% It is used in \!\pcol@output! for a warning message.
% 
% \item[\Uidx{$\cs{the}{\cdot}\theta$}]
% \SpecialArrayIndex{\theta}{\the}
% 
% is an API macro to have the representation of the counter $\theta$.  The
% macro is kept in $|\pcol@thectr@|{\cdot}\theta$
% 
% \SpecialArrayIndex{\theta}{\pcol@thectr@}
% 
% by \!\pcol@thectrelt! which also makes the macro \!\let!-equal to
% $|\pcol@thectr@|{\cdot}\theta{\cdot}0$
% 
% \SpecialArrayIndex{\theta{\cdot}c}{\pcol@thectr@}
% 
% if the \lrep{} of $\theta$ in the column 0 is defined by
% \!\definethecounter!.  The macro \!\pcol@setctrelt! also makes this
% overriding for the column $c$ to which the macro belongs if
% $|\pcol@thectr@|{\cdot}\theta{\cdot}c$
% 
% \SpecialArrayIndex{\theta{\cdot}c}{\pcol@thectr@}
% 
% being the \lrep{} for $c$ is defined, while it makes $|\the|{\cdot}\theta$
% 
% \SpecialArrayIndex{\theta}{\the}
% 
% \!\let!-equal to $|\pcol@thectr@|{\cdot}\theta$
% 
% \SpecialArrayIndex{\theta}{\pcol@thectr@}
% 
% otherwise to give it its original definition.
% 
% \item[\Uidx{$\!\stepcounter!\<\theta\>$}]
% is an API macro to increment the counter $\theta$ and zero-clear its
% descendant counters.  It is used in \!\pcol@startpage! for the counter
% \counter{page}, and in \!\pcol@iifootnotetext! for \counter{footnote}.
% 
% \item[\Uidx{\!\nobreak!}]
% is an API macro to insert $\!\penalty!=10000$ to inhibit line or page
% breaks.  It is used in \!\pcol@output@start!, \!\pcol@restartcolumn!,
% \!\pcol@icolumncolor!, \!\pcol@set@color@push!, \!\pcol@reset@color@pop!
% and \!\pcol@reset@color@mpop!  to meet the page-break inhibition request
% made by $\CSIndex{if@nobreak}=\true$.
% 
% \item[\Uidx{$\!\addvspace!$\marg{skip}}]
% is an API macro to insert a vertical $\arg{skip}$ if
% $\!\lastskip!<\arg{skip}$, or a skip of $\arg{skip}+\!\lastskip!$
% otherwise.  The macro is used in \!\pcol@zparacol!  when it finds the
% \env{paracol} environment to start is at the very beginning of a
% \env{list}-like environment, to insert \!\@topsep! instead of
% $\!\parskip!+\!\itemsep!$ going to be inserted by the first \!\item!.
% 
% \item[\Uidx{$\!\addpenalty!\arg{pen}$}]
% is an API macro to insert a page break $\!\penalty!=\arg{pen}$ if
% $\CSIndex{if@nobreak}={\false}$.  The \!\penalty! is inserted removing the
% last vertical skip which is reinserted after the \!\penalty!.  The macro
% is used in \!\pcol@output@start! and \!\pcol@restartcolumn! to insert
% \!\interlinepenalty! if $\CSIndex{if@nobreak}=\false$, while
% \!\pcol@zparacol! uses it to insert \!\@beginparpenalty! when it finds the
% \env{paracol} environment to start is at the very beginning of a
% \env{list}-like environment.
% 
% \item[\Uidx{\!\footnotesize!}]
% is an API macro to set the font size for footnotes.  It is used in
% \!\pcol@fntextbody! for footnote typesetting.
% 
% \item[\Uidx{\!\rule!\oarg{\mbox{$r$}}$\Arg{w}\Arg{h}$}]
% is an API macro to draw a vertical rule of $w$ width and $h$ tall,
% optionally raised by $r$.  It is used in \!\pcol@fntextbody! to have the
% rule of $w=r=0$ and $h=\!\footnotesep!$ to make the first line of the
% footnote is at least as tall as \!\footnotesep!.
% 
% \item[\Uidx{$\!\addcontentsline!\arg{file}\arg{sec}\arg{entry}$}]
% is an API macro to put \!\addtocontents! for the arguments to |.aux| file.
% The orignial definition of the macro is kept in
% \!\pcol@addcontentsline! so that \!\pcol@ac@disable@toc! and
% \!\pcol@ac@caption! make the macro regain its original definition after
% temporarily disabling its function by making it \!\let!-equal to
% \!\pcol@gobblethree!.
% 
% \item[\Uidx{\!\marginpar!}\oarg{left}\marg{right}]
% is an API macro to put merginal note $\arg{left}$ or $\arg{right}$ to the
% left or right margin.  In \!\pcol@zparacol! it is made \!\let!-equal to
% \!\pcol@marginpar! for the emulation of \!\marginnote!, while its original
% version is kept in \!\pcol@@marginpar!.
% 
% \item[\Uidx{\!\footnote!}\oarg{num}\marg{text}]
% is an API macro to give a footnote $\arg{text}$ optionally with its number
% $\arg{num}$.  In \!\pcol@zparacol! it is made \!\let!-equal to
% \!\pcol@footnote!  to implement its starred version and the adjustment of
% \counter{footnote} at \Endparacol, while its original version is kept in
% \!\pcol@@footnote!.
% 
% \item[\Uidx{\!\footnotemark!}\oarg{num}]
% is an API macro to give a footnote mark optionally with the number
% $\arg{num}$ which the mark represents.  In \!\pcol@zparacol! it is made
% \!\let!-equal to \!\pcol@footnotemark! to implement its starred version
% and the adjustment of \counter{footnote} at \Endparacol, while its
% original version is kept in \!\pcol@@footnotemark!.
% 
% \item[\Uidx{\!\footnotetext!}\oarg{num}\marg{text}]
% is an API macro to give a footnote $\arg{text}$ optionally with its number
% $\arg{num}$ but without putting the mark in the footnoted text.  In
% \!\pcol@zparacol!  it is made \!\let!-equal to \!\pcol@footnotetext! to
% implement its starred version, while its original version is kept in
% \!\pcol@@footnotetext!.
% 
% \item[\Uidx{\!\footnoterule!}]
% is an API macro to draw a horizontal line above footnotes, or to insert
% whatever it has above them.  With \Scfnote{} typesetting, it is redefined
% in \!\pcol@zparacol! so that it refers to \!\textwidth! instead of
% \!\columnwidth! for drawing the horizontal line or whatever defined, while
% the original version is kept in \!\pcol@footnoterule!.  Then it is used in
% \!\pcol@putfootins! to separate footnotes from the stuff above them, with
% the original or modified definition.
% 
% \item[\Uidx{\!\newpage!}]
% is an API macro to break a page.  It is used in \!\pcol@switchcol! as the
% argument of \!\pcol@visitallcols! to break the \colpage{}s visited in the
% \cscan{} when the \sync{}ed \cswitch{} requires explicit page breaks.
% 
% \item[\Uidx{\!\dblfigrule!}]
% is an API macro to draw a horizontal line between the last \pwise{}
% floats and the main vertical list, or to insert whatever it has between
% them.  The macro is used in \!\pcol@startpage! to build \spanning{} in
% the page $p$ in $\pp^b(p)$.
% 
% \item[\Uidx{\!\topfigrule!}]
% is an API macro to draw a horizontal line between the last \cwise{}
% top float and the main vertical list, or to insert whatever it has between
% them.  The macro is used in \!\pcol@cflt! and \!\pcol@synccolumn! to
% insert it below the last (real) top float.  It is also made \!\let!-equal
% to \!\relax! temporarily by \!\pcol@imakeflushedpage! when it put floats in
% a \fcolumn{} as top floats.  Note that the macro and its bottom
% counterpart \!\botfigrule! should produce a vertical list whose total
% height and depth is 0, because \LaTeX's float mechanism and thus our
% macros believe so.
% 
% \item[\Uidx{\!\normalcolor!}]
% is an API macro to have color specification stuff for normal coloring.
% The macro is used in \!\pcol@putfootins! to specify the color of footnotes
% to be put in \!\@outputbox!, in \!\normalcolumncolor!\oarg{c} to
% define that the default color of the column $c$ is the normal color,
% in \!\normalcolseprulecolor! to specify that the color for \cseprule{}s is
% \!\normalcolor!, and in the initial definition of \!\pcol@colseprulecolor!
% to give the default color for \cseprule{}s.
% 
% \item[\Uidx{\!\color!}\oarg{mode}\marg{color}]
% is an API macro defined in coloring pacakges to start text coloring with
% $\arg{color}$ optionally with $\arg{mode}$.  The macro is used in
% \!\pcol@xcolumncolor!\oarg{mode}\~\marg{color}\oarg{c} and
% \!\pcol@ycolumncolor!\marg{color}\oarg{c} to define the default color of
% the column $c$ is $\arg{color}$ optionally with $\arg{mode}$, in
% \!\pcol@defcseprulecolor@x! and \!\pcol@defcseprulecolor@y! to define the
% color of \cseprule{}s, and in \!\pcol@backgroundcolor@x! to define the
% color for \bgpaint{} of a region.
% 
% \item[\Uidx{\!\pfmtname!}]
% is an API macro defined in p\LaTeX{} to have its format name |pLaTeX2e|
% (so far).  It is used in the top level assignment of the constant switch
% \CSIndex{ifpcol@bfbottom}.
% 
% \item[\Uidx{$\!\PackageError!\arg{pkg}\arg{msg}\arg{hlp}$}]
% is a developer's API macro to stop the exectution showing the error
% message $\arg{msg}$ with the package identification $\arg{pkg}$ and the
% help mssage $\arg{hlp}$.  The macro is used in the following macros;
% \!\pcol@ovf! on \!\@freelist! shortage; \!\pcol@set@color@push! on too
% many math-mode colorings; in \!\pcol@zparacol! on two-column
% typesetting outside \env{paracol} and illegal nesting of \env{paracol};
% $\!\pcol@setcw@calcf!\<x\>\<y\>\<z\>$ on too large $x/y$;
% \!\pcol@switchcolumn! on invalid target column; \!\pcol@switchenv! on
% illegal \cswitch{} commands\slash environments in a \csenv;
% \!\addcontentsonly! on unknown contents type; \!\footnotelayout! on
% unknown layout;  \!\pcol@twosided! on unknown two-sided typesetting
% feature;  \!\pcol@backgroundcolor! on unknown region of \bgpaint;  and in 
% \!\pcol@backgroundcolor@i! on a region not being a column or \csepgap{}
% but its ordinal being specified.
% 
% \item[\Uidx{$\!\PackageWarning!\arg{pkg}\arg{msg}$}]
% is a developer's API macro to report a warning message $\arg{msg}$ with
% the package identification $\arg{pkg}$.  The macro is used in
% \!\pcol@ignore! to complain an API macro appears in \env{paracol}
% inappropriately, in \!\pcol@fntextbody!  if the footnote is taller than
% $\!\textheight!-\!\skip!\!\footins!$, in \!\pcol@mn@warning! to show
% \!\marginnote! is emulated, and in \!\pcol@icolumncolor!,
% \!\pcol@defcseprulecolor@i! and \!\pcol@backgroundcolor@ii! to complain
% \!\columncolor!\slash\!\normalcolumncolor!,
% \!\colseprulecolor!\slash\!\normalcolseprulecolor! or \!\background~color!
% is used without coloring packages respectively.
% 
% \item[\Uidx{\!\@@par!}]
% is an internal macro to keep \TeX's original primitive \CSIndex{par} in it.
% The macro is used in \!\pcol@output! to let \CSIndex{par} act with its
% original definition, and in \!\pcol@switchcol! and \!\pcol@flushclear! as
% the argument given to \!\pcol@visitallcols! to give \TeX's page builder
% the chance of page break in \cscan{}ning.
% 
% \item[\Uidx{\!\@height!}]
% is an internal macro having the keyword |height|.  It is used in
% \!\pcol@buildcolsep~rule!, \!\pcol@buildcselt!, \!\pcol@bg@paintregion@i!,
% \!\pcol@output@start!, and \!\pcol@putbackmvl! to draw a \!\hrule! for
% \cseprule{} in the first two, a \!\vrule! to be painted in the third, and
% an invisible \!\hrule! in the fourth and last.
% 
% \item[\Uidx{\!\@width!}]
% is an internal macro having the keyword |width|.  It is used in
% \!\pcol@buildcolsep~rule!, \!\pcol@buildcselt!, \!\pcol@bg@paintregion@i!,
% \!\pcol@output@start!, and \!\pcol@putbackmvl! to draw a \!\hrule! for
% \cseprule{} in the first two, a \!\vrule! to be painted in the third, and
% an invisible \!\hrule! in the fourth and last.
% 
% \item[\Uidx{\!\@plus!}]
% is an internal macro having the keyword |plus|.  It is used in the
% following macros.
% 
% \begin{itemize}
% \item
% \!\pcol@makecol! to \!\def!ine \!\@textbottom! with the body of a vertical
% skip with small infinite stretch and shrink.
% 
% \item
% \!\pcol@combinefloats! for a skip of the same amount in \!\@textbottom!
% above and that of negative amount to {\em move} the effect.
% 
% \item
% \!\pcol@hfil! for skips having 1\,|fil| infinite stretch with $\gap_c$ or
% $\gap_c/2$ to make it sure the series of columns and \csepgap{}s does not
% cause underfull.
% 
% \item
% \!\pcol@synccolumn! to put a $1\,|fil|$ infinte stretch below the
% main vertical list together with a small infinte shrink in the \colpage{}
% being flushed and having a \sync{}ation point, and a vertical skip with a
% small infinite stretch to push up the main vertical list above a
% \sync{}ation point.
% 
% \item
% \!\pcol@setcw@getspec@i! to add $0\,|pt|\ |plus|\ 1000\,|pt|\ |minus|\
% 1000\,|pt|$ to \!\@tempskipa! to ensure the register have stretch and
% shrink components.
% 
% \item
% \!\pcol@setcw@fill! to let $\!\@tempskipa!=0\,|pt|\ |plus|\ f\,|fil|$ as
% the infinite stretch factor of $f$.
% \end{itemize}
% 
% It is also used in the top level assignment of $0\,|pt|\ |plus|\ 1\,|fil|\
% |minus|\ 1\,|fil|$ to \!\@tempskipa! for the invocatoin of \!\pcol@defkw!.
% 
% \item[\Uidx{\!\@minus!}]
% is an internal macro having the keyword |minus|.  It is used in the
% following macros.
% 
% \begin{itemize}
% \item
% \!\pcol@makecol! to \!\def!ine \!\@textbottom! with the body of a vertical
% skip with small infinite stretch and shrink.
% 
% \item
% \!\pcol@combinefloats!  for a skip of the same amount and that of negative
% amount.
% 
% \item
% \!\pcol@synccolumn! to put a small infinite shrink together with a
% stretch of $1\,|fil|$ at the bottom of the main vertical list in a
% \colpage{} being flused and having a \sync{}ation point.
%
% \item
% \!\pcol@setcw@getspec@i! to add $0\,|pt|\ |plus|\ 1000\,|pt|\ |minus|\
% 1000\,|pt|$ to \!\@tempskipa! to ensure the register have stretch and
% shrink components.
% \end{itemize}
% 
% It is also used in the top level assignment of $0\,|pt|\ |plus|\ 1\,|fil|\
% |minus|\ 1\,|fil|$ to \!\@tempskipa! for the invocatoin of \!\pcol@defkw!.
% 
% \item[\Uidx{\!\hb@xt@!}]
% is an internal macro having the sequence ``\!\hbox!| to|''.  It is used in 
% \!\pcol@ioutputelt!, \!\pcol@imakeflushedpage! and \!\pcol@iflushfloats! to
% put each \colpage{} in a \!\hbox! of \!\columnwidth! wide and to enclose
% all of them in a \!\hbox! of \!\textwidth! wide.
% 
% \item[\Uidx{$\!\@namedef!\arg{cs}\arg{body}$}]
% is an internal macro to do $\!\def!|\|\arg{cs}\Arg{body}$.  It is used in
% the following macros.
% 
% \begin{itemize}
% \item
% \!\pcol@zparacol! for \!\column*! and \!\pcol@com@column*!.
% 
% \item
% $\!\pcol@remctrelt!\<\theta\>$ for $|\cl@|{\cdot}\theta$.
% 
% \SpecialArrayIndex{\theta}{\cl@}
% 
% \item
% $\!\definethecounter!\<\theta\>\<c\>\arg{rep}$ for
% $|\pcol@thectr@|{\cdot}\theta{\cdot}c$,
% 
% \SpecialArrayIndex{\theta{\cdot}c}{\pcol@thectr@}
% 
% \item
% $\!\pcol@loadctrelt!\<\theta\>\<\val_c(\theta)\>$ for
% $|\pcol@ctr@|{\cdot}\theta$.
% 
% \SpecialArrayIndex{\theta}{\pcol@ctr@}
% 
% \item
% \!\pcol@defcolumn! for \!\pcol@com@column*!.
% 
% \item
% \!\pcol@defcseprulecolor@i! for $\!\pcol@colseprulecolor![{\cdot}c]$.
% 
% \SpecialArrayIndex{c}{\pcol@colseprulecolor}
% 
% \end{itemize}
% 
% We also use this macro in top level \!\def!initions of
% \!\pcol@com@nthcolumn*!, \!\pcol@com@leftcolumn*! and
% \!\pcol@com@rightcolumn*! for the starter of the environments
% \env{nthcolumn*}, \env{leftcolumn*} and \env{rightcolumn*}.
% 
% \item[\Uidx{$\!\@nameuse!\arg{cs}$}]
% is an internal macro to do $|\|\arg{cs}$.  It is used in the following
% macros.
% 
% \begin{itemize}
% \item
% $\!\pcol@bg@addext!\arg{z}\Arg{s}\Arg{d}$ for
% $|\pcol@bg@ext@|{\cdot}d{\cdot}|@|{\cdot}\{a{\cdot}|@|{\cdot}c,a\}$.
% 
% \SpecialArrayIndex{d{\cdot}\string\texttt{@}
%     {\cdot}a{\cdot}\string\texttt{@}{\cdot}c}{\pcol@bg@ext@}
% \SpecialArrayIndex{d{\cdot}\string\texttt{@}{\cdot}a}{\pcol@bg@ext@}
% 
% \item
% \!\pcol@bg@columnleft! for $|\pcol@columnwidth|{\cdot}c$ and
% $|\pcol@columnsep|{\cdot}c$.
% 
% \SpecialArrayIndex{c}{\pcol@columnwidth}
% \SpecialArrayIndex{c}{\pcol@columnsep}
% 
% \item
% \!\pcol@bg@columnwidth! for $|\pcol@columnwidth|{\cdot}c$.
% 
% \SpecialArrayIndex{c}{\pcol@columnwidth}
% 
% \item
% \!\pcol@bg@columnsep! for $|\pcol@columnsep|{\cdot}c$.
% 
% \SpecialArrayIndex{c}{\pcol@columnsep}
% 
% \item
% \!\pcol@ccuse! for
% $\Celt^c=|\pcol@columncolor@box|{\cdot}c$ or
% $\Celtshadow^c=|\pcol@columncolor|{\cdot}c$.
% 
% \SpecialArrayIndex{c}{\pcol@columncolor}
% \SpecialArrayIndex{c}{\pcol@columncolor@box}
% 
% \item
% \!\column*! for \!\pcol@com@column*!.
% 
% \item
% \!\pcol@zparacol! for $|\pcol@colpream|{\cdot}0$.
% 
% \SpecialArrayIndex{c}{\pcol@colpream}
% 
% \item
% $\!\pcol@storectrelt!\<\theta\>$ for $|\pcol@ctr@|{\cdot}\theta$.
% 
% \SpecialArrayIndex{\theta}{\pcol@ctr@}
% 
% \item
% $\!\pcol@cmpctrelt!\<\theta\>$ for $|\c@|{\cdot}\theta$
% 
% \SpecialArrayIndex{\theta}{\c@}
% 
% and $|\pcol@ctr@|{\cdot}\theta$.
% 
% \SpecialArrayIndex{\theta}{\pcol@ctr@}
% 
% \item
% \!\pcol@synccounter! for $|\pcol@counters|{\cdot}c$ for the column $c$.
% 
% \SpecialArrayIndex{c}{\pcol@counters}
% 
% \item
% $\!\pcol@syncctrelt!\<\theta\>$ for $|\c@|{\cdot}\theta$.
% 
% \SpecialArrayIndex{\theta}{\c@}
% 
% \item
% $\!\pcol@stepcounter!\<\theta\>$ for $|\pcol@counters|{\cdot}c$ for the
% column $c$,
% 
% \SpecialArrayIndex{c}{\pcol@counters}
% 
% and for $|\cl@|{\cdot}\theta$.
% 
% \SpecialArrayIndex{\theta}{\cl@}
% 
% \item
% \!\pcol@switchcol! to the column $c$ for $|\pcol@colpream|{\cdot}c$.
% 
% \SpecialArrayIndex{c}{\pcol@colpream}
% 
% \item
% $\!\pcol@aconlyelt!\<t\>\<c\>$ for $|\pcol@ac@def@|{\cdot}t$.
% 
% \SpecialIndex{\pcol@ac@def@lof}\SpecialIndex{\pcol@ac@def@lot}
% 
% \item
% $\!\pcol@ac@def@lof!\arg{eord}$ and $\!\pcol@ac@def@lot!\arg{eord}$
% for $|\pcol@ac@caption@|{\cdot}\~\arg{eord}$.
% 
% \SpecialIndex{\pcol@ac@caption@enable}
% \SpecialIndex{\pcol@ac@caption@disable}
% 
% \item
% $\!\pcol@ac@caption!\arg{type}|[|\arg{lcap}|]|\arg{cap}$ for
% $|\pcol@ac@caption@if@|{\cdot}t$
% 
% \SpecialIndex{\pcol@ac@caption@if@lof}
% \SpecialIndex{\pcol@ac@caption@if@lot}
% 
% and for $|\ext@|{\cdot}\~\arg{type}$.
% 
% \SpecialIndex{\ext@figure}\SpecialIndex{\ext@table}
% 
% \item
% $\!\footnotelayout!\Arg{l}$ for $|pcol@fnlayout@|{\cdot}l$.
% 
% \SpecialArrayIndex{l}{\pcol@fnlayout@}
% \SpecialIndex{\pcol@fnlayout@c}
% \SpecialIndex{\pcol@fnlayout@p}
% \SpecialIndex{\pcol@fnlayout@m}
% 
% \item
% $\!\pcol@twosided!|[|T|]|$ for $|pcol@twosided@|{\cdot}t$ where $t\in T$.
% 
% \SpecialArrayIndex{t}{\pcol@twosided@}
% \SpecialIndex{\pcol@twosided@p}
% \SpecialIndex{\pcol@twosided@c}
% \SpecialIndex{\pcol@twosided@m}
% \SpecialIndex{\pcol@twosided@b}
% \end{itemize}
% 
% \item[\Uidx{$\!\@gobble!\arg{arg}$}]
% discards its argument $\arg{arg}$.  It is used in \!\pcol@output@start!,
% \!\pcol@icolumncolor! and \!\pcol@set@color@push! for temporarily letting
% \!\aftergroup! be \!\@gobble! to nullify \!\aftergroup! with
% \!\reset@color! invoked in \!\pcol@set@color!, being the original version
% of \!\set@color!, and in \!\pcol@zparacol! to make \!\pcol@bg@paintbox!
% \!\let!-equal to \!\@gobble! to nullify it if any coloring packages have
% not been loaded.  In addition, the macros \!\pcol@F! and \!\pcol@Fe! for
% logging are made \!\let!-equial to \!\@gobble! at the top level to nullify
% them.
% 
% \item[\Uidx{$\!\@ifundefined!\arg{cs}\arg{then}\arg{else}$}]
% \begin{Hfuzz}{0.8pt}%
% is an internal macro to do $\arg{then}$ or $\arg{else}$ if $\arg{cs}$ is
% undefined or defined respectively.
% It is used in the following macros;
% $\!\pcol@bg@paintregion!\arg{a}\arg{c}$ to examine if either
% $|\pcol@bg@color@|{\cdot}a{\cdot}|@|{\cdot}c$ or
% $|\pcol@bg@color@|{\cdot}a$ is defined;
% 
% \SpecialArrayIndex{a{\cdot}\string\texttt{@}{\cdot}c}{\pcol@bg@color@}
% \SpecialArrayIndex{a}{\pcol@bg@color@}
% 
% \!\add~contentsonly!$\<t\>\<c\>$ to stop the execution if
% $|\pcol@ac@def@|{\cdot}t$ is not defined;
% 
% \SpecialArrayIndex{t}{\pcol@ac@def@}
% 
% \!\footnote~layout!$\Arg{l}$ to stop the execution if
% $|\pcol@fnlayout@|{\cdot}l$ is not defined;
% 
% \SpecialArrayIndex{l}{\pcol@fnlayout@}
% 
% $\!\pcol@twosided!|[|T|]|$ to stop the execution if $T$ has $t$
% such that $|\pcol@twosided@|{\cdot}t$ is not defined;
% 
% \SpecialArrayIndex{t}{\pcol@twosided@}
% 
% \!\pcol@backgroundcolor! to stop the execution if $|\pcol@bg@@|{\cdot}a$
% is not defined for a region $a$;
% 
% \SpecialArrayIndex{a}{\pcol@bg@@}
% 
% and \!\pcol@backgroundcolor@i! to stop the execution if
% $|\pcol@bg@mayhavecol|{\cdot}a$ is not defined for a region $a$.
% \end{Hfuzz}
% 
% \item[\Uidx{$\!\@ifnextchar!\arg{char}\arg{then}\arg{else}$}]
% is an internal macro to do $\arg{then}$ or $\arg{else}$ if the character
% following to the macro is $\arg{char}$ or not respectively.  It is used in
% the following macros to examine if they are followed by a `\texttt{\LB}'.
% 
% \begin{quote}\raggedright
% \!\paracol!,
% \!\pcol@zparacol!,
% \!\columnratio!,
% \!\pcol@com@column*!\hbox{ (initial definition)},
% \!\pcol@com@switchcolumn!,
% \!\pcol@iswitchcolumn!,
% \!\pcol@adjustfnctr!,
% \!\pcol@ifootnotetext!,
% \!\twosided!,
% \!\marginparthreshold!,
% \!\columncolor!,
% \!\pcol@columncolor!,
% \!\normalcolumncolor!.
% \!\colseprulecolor!,
% \!\pcol@defcseprulecolor!,
% \!\normalcolseprulecolor!,
% \!\pcol@backgroundcolor!,
% \!\pcol@backgroundcolor@w!.
% \end{quote}
% It is also used in \!\pcol@backgroundcolor@iii! and
% \!\pcol@backgroundcolor@iv! if they are followed by a `\texttt{\LP}'.
% 
% \item[\Uidx{$\!\@ifstar!\arg{then}\arg{else}$}]
% is an internal macro to do $\arg{then}$ or $\arg{else}$ if the character
% following to the macro is `|*|'.  It is used in \!\pcol@yparacol!,
% \!\globalcounter!, \!\pcol@switchcolumn!, \!\pcol@footnote!,
% \!\pcol@footnotemark! and \!\pcol@footnotetext! to examine if the optional
% `|*|' is specified.
% 
% \item[\Uidx{$\!\@whilesw!\arg{sw}\cs{fi}\arg{body}$}]
% is an internal macro to iterate $\arg{body}$ while the switch $\arg{sw}$
% is $\true$.  It is used in \!\pcol@output!, \!\pcol@startpage!,
% \!\pcol@output@clear!, \!\pcol@flushfloats!, \!\pcol@freshpage! and
% \!\pcol@output@end! to iterate building process of \fcolumn{}s or
% \fpage{}s while $\CSIndex{if@fcolmade}=\true$, and in \!\pcol@switchcol!
% and \!\pcol@flushclear! to iterate the height check of \sync{}ed or
% flushed pages while $\CSIndex{ifpcol@flush}=\true$.
% 
% \item[\Uidx{$\!\@whilenum!\arg{ifnum}\cs{do}\arg{body}$}]
% is an internal macro to interate $\arg{body}$ while the integer comparison
% expression $\arg{ifnum}$ is $\true$.  The macro is used in the following
% macros to iterate their own procedures for all columns $c\In0\C$.
% 
% \begin{quote}\raggedright
% \!\pcol@checkshipped!,
% \!\pcol@output@start!,
% \!\pcol@output@switch!,
% \!\pcol@sync!,
% \!\pcol@makeflushedpage!,
% \!\pcol@freshpage!,
% \!\pcol@output@end!,
% \!\pcol@synccounter!,
% \!\pcol@com@syncallcounters!,
% \!\pcol@stepcounter!,
% \!\pcol@visitallcols!.
% \end{quote}
% 
% The macro is also used in the following macros for the ranges following
% macro name, where $(\Cfrom,\Cto)\in\{(0,\CL),(\CL,\C)\}$ and $c$ is the
% column they are working on.
% 
% \begin{quote}
% \!\pcol@ioutputelt! $\LBRP\Cfrom\Cto$\\
% \!\pcol@bg@paint@ii! $\LBRP\CBfrom{\Cto{-}1}$\\
% \!\pcol@bg@columnleft! $\LBRP\CBfrom{c}$\\
% \!\pcol@addmarginpar! $\LBRP\Cfrom{c'}$ ($c'\in\{c,\;\Cto-(c-\Cfrom)-1\}$)\\
% \!\pcol@imakeflushedpage! $\LBRP\Cfrom\Cto$\\
% \!\pcol@iflushfloats! $\LBRP\Cfrom\Cto$\\
% \!\pcol@setcw@scan! $\LBRP\Cfrom\Cto$
% \end{quote}
% 
% The other users have a little bit more complicated range as follows.
% \begin{itemize}
% \item
% \!\pcol@flushcolumn! to iterate \fcolumn{} building for a column $c$ in
% pages $q$ such that $q\in(\cc_c(\vb^p),\ptop)$.
% 
% \item
% \!\pcol@setcolwidth@r! to make assignment of $\w_c$ for
% $c\In{\min(\Cfrom{+}k,\Cto{-}1)}{\,\Cto}$ where $k$ is the number of
% fractions given as the first or second argument of
% \!\columnratio! and kept in \!\pcol@columnratioleft! or
% \!\pcol@columnratioright!, respectively.
% 
% \item
% $\!\pcol@setcw@calcf!\<x\>\<y\>\<z\>$ to calculate
% $\lceil{y/2^{k_2+k_3}}\rceil$ finding $k_3$ by iterating $y/2$ until the
% result becomes less than $2^{15}$, to calculate $z'/2^k$ with the range
% $\LBRP0k$, to calculate $z'/2^{k-16}$ with the range $\LBRP0{\,k{-}16}$,
% and to calculate $z'\cdot2^{16-k}$ with the range $\LBRP0{\,16{-}k}$,
% where $z'\cdot2^{16-k}=Z=z\times1\,|pt|$ and $k_2$, $k_3$ and $k$ are
% scaling parameters for good approximation.
% \end{itemize}
% 
% \item[\Uidx{$\!\@whiledim!\arg{ifdim}\cs{do}\arg{body}$}]
% is an internal macro to interate $\arg{body}$ while the dimensional
% comparison expression $\arg{ifdim}$ is $\true$.  The macro is used in
% $\!\pcol@setcw@calcf!\<x\>\<y\>\<z\>$ twice, at first to find
% $k_1=\min\Set{k}{x\cdot2^k\geq2^{13}\,|pt|}$ and to have $x\cdot2^{k_1}$,
% and then to find $k_2=\max\Set{k}{y\bmod2^k=0}$ and to have $y/2^{k_2}$.
% 
% \item[$\Uidx{\!\@for!}\arg{cs}\texttt{:=}\arg{list}\cs{do}\arg{body}$]
% is an internal macro to iterate $\arg{body}$ for each element of the
% comma-separated $\arg{list}$ letting $\arg{cs}$ have the element.  The
% macro is used in \!\pcol@setcolwidth@r! to scan its argument
% $\arg{ratio}$ defined by \!\columnratio!, and in \!\pcol@setcw@scan! to
% scan its argment $\arg{spec}$ defined by \!\setcolumnwidth!.
% 
% \item[$\Uidx{\!\@tfor!}\arg{cs}|:=|\arg{list}\cs{do}\arg{body}$]
% is an internal macro to iterate $\arg{body}$ for each non-space token in
% $\arg{list}$ letting $\arg{cs}$ have the token.  The macro is used in
% \!\pcol@bg@paint@ii!, \!\pcol@setcw@getspec! and \!\pcol@twosided! to scan
% their arguments, in \!\pcol@setcw@getspec@i! to scan a column\slash
% gap specification to remove spaces from it, and in
% $\!\pcol@twosided!|[|T|]|$ to scan all tokens being two-sided typesetting
% features in $T$.
% 
% \item[\Uidx{$\!\@next!\arg{elm}\arg{lst}\arg{suc}\arg{fail}$}]
% is an internal macro to remove the first element from $\arg{lst}$,
% \!\def!ine $\arg{elm}$ to have the first element, and then do $\arg{suc}$,
% if $\arg{lst}$ is not empty.  Otherwise, it performs $\arg{fail}$.  The
% macro is used in the following macros to obtain an \!\insert! from
% \!\@freelist!.
% 
% \begin{itemize}
% \item
% \!\pcol@opcol! for the completed \colpage{}.
% 
% \item
% \!\pcol@startpage! for \fpage{}s and \spanning{} for \pwise{} top
% floats.
% 
% \item
% \!\pcol@output@start! for the \preenv{}, and \colpage{}s and $\Celt^c$ of
% all columns.
% 
% \item
% \!\pcol@output@switch! for the \colpage{} from which we are leaving.
% 
% \item
% \!\pcol@iscancst! for $\Celt^c$.
% 
% \item
% \!\pcol@savefootins! for footnotes.
% 
% \item
% \!\pcol@flushcolumn! for \fcolumn{}s and the empty \colpage{} in $\ptop$.
% 
% \item
% \!\pcol@synccolumn! for an \mvlfloat{} on a \sync{}ation if its point
% defined by a column whose main vertical list is empty.
% 
% \item
% \!\pcol@output@end! for the \pwise{} floats in the \lpage{} if the
% main vertical list of the page is empty.
% 
% \item
% \!\pcol@icolumncolor! for $\Celt^c$.
% \end{itemize}
% 
% The macro is also used in \!\pcol@ioutputelt! to obtain completed
% \colpage{}s from $\S_c$.
% 
% \item[\Uidx{%
%	$\!\@xnext!\cs{@elt}\arg{car}\arg{cdr}\cs{@@}\arg{first}\arg{rest}$}]
% is an internal macro to remove the first element $|\@elt|\arg{car}$ from a
% list in the form of $|\@elt|\,e_1\,\cdots\,|\@elt|\,e_n$ where
% $\arg{cdr}=|\@elt|\,e_2\,\cdots\~\,|\@elt|\,e_n$ and |\def|ine
% $\arg{first}$ as $\arg{car}$ and globally |\def|ine $\arg{rest}$ as
% $\arg{cdr}$.  It is used in \!\pcol@addmarginpar! to get the first element
% of \!\@currlist! being a \!\insert! for a right marginal note without
% modifying \!\@currlist!.
% 
% \item[\Uidx{$\!\@cons!\arg{lst}\arg{elm}$}]
% is an internal macro to add $\!\@elt!\arg{elm}$ to the tail of $\arg{lst}$.
% 
% \begin{itemize}
% \item
% \!\pcol@makecol! to add $\spt(H,h)$ to the tail of $\pp^s(\ptop)$.
% 
% \item
% \!\pcol@opcol! to add the completed \ccolpage{} $\cc_c(\vb)$ to $\S_c$.
% 
% \item
% \!\pcol@startpage! to add $\pp(\ptop-1)$ and \fpage{}s to $\PP$.
% 
% \item
% \!\pcol@outputelt! to return \spanning{} $\pp^i(q)$ in a shipped-out \fpage{}
% $q$ to \!\@freelist!, or to add $\pp(q)$ to $\PP$ if the page $q$ is kept.
% 
% \item
% \!\pcol@ioutputelt! to return \spanning{} $\pp^i(q)$, \Scfnote{}s
% $\pp^f(q)$ and/or \colpage{}s $\s_c(q)$ for all $c\In0C$ in a shipped-out
% page $q$ to \!\@freelist!.
% 
% \item
% \!\pcol@output@start! to return the \ccolpage{} $\cc_0(\vb)$ to
% \!\@freelist!.
% 
% \item
% \!\pcol@output@switch! to add $\spt(H,h)$ to the tail of $\pp^s(\ptop)$.
% 
% \item
% \!\pcol@restartcolumn! to return the \ccolpage{} $\cc_c(\vb)$ to be resumed,
% and its footnotes $\cc_c(\ft)$ if any, to \!\@freelist!.
% 
% \item
% $\!\pcol@getmparbottom!\<t\>\<h\>$ to add
% $\mpar(\max(t,b_n),\max(t,b_n){+}h)$ to the tail of the list
% $\mpb_{\{L,R\}}^{\{l,r\}}$, and its callee
% $\!\pcol@getmpbelt!\<t_i\>\<b_i\>$ to add $\mpar(t_i,b_i)$ or
% $\mpar(\max(t,b_{i-1}),\max(t,b_{i-1}){+}h)$ to the so-far tail of the
% list in rebuilding.
% 
% \item
% \!\pcol@flushcolumn! to return footnotes $\cc_c(\ft)$ in the \ccolpage{}
% of $c$ to be flushed to \!\@freelist! if any, and to add the flushed
% \colpage{} and \fcolumn{}s to $\S_c$.
% 
% \item
% \!\pcol@makefcolelt! to add a float to \!\@toplist! or $\cc_c(\dl)$.
% 
% \item
% \!\pcol@synccolumn! to add an \mvlfloat{} for \sync{}ation to
% $\cc_c(\tl)$.
% 
% \item
% \!\pcol@makeflushedpage! to return \spanning{} $\pp^i(\ptop)$ and/or
% \Scfnote{}s $\pp^f(\ptop)$ in the top page to be flushed to \!\@freelist!
% if any.
% 
% \item
% \!\pcol@imakeflushedpage! to return \Mcfnote{}s in $\cc_c(\ft)$ s.t.\
% $\cc_c(\vb^p)=\ptop$ to \!\@freelist! if any.
% 
% \item
% \!\pcol@output@end! to return $\pp^f(\ptop)$, all \ccolpage{}s
% $\cc_c(\vb)$, and all $\Celt^c\neq\bot$ to \!\@freelist!.
% 
% \item
% \!\pcol@end@dblfloat! to add a \pwise{} float in \!\@currbox! to
% \!\@dbldeferlist!.
% 
% \item
% $\!\globalcounter!\<\theta\>$ to add a \gcounter{} $\theta$ to $\CG$.
% 
% \item
% $\!\pcol@iremctrelt!\<\theta\>$ to add a \lcounter{} $\theta$ to $\CTL$.
% 
% \item
% $\!\pcol@storectrelt!\<\theta\>$ to add a pair $\<\theta,\val_c(\theta)\>$
% to $\Cc_c$ for a column $c$.
% 
% \item
% $\!\pcol@savectrelt!\<\theta\>$ to add a pair $\<\theta,\val(\theta)\>$ to
% $\Cc_c$ for a column $c$.
% 
% \item
% $\!\pcol@cmpctrelt!\<\theta\>$ to add a counter $\theta$ to the list of
% \lcounter{}s to be synchronized.
% 
% \item
% $\!\addcontentsonly!\<t\>\<c\>$ to add a pair $\<t,c\>$ to $\T$.
% 
% \item
% \!\pcol@backgroundcolor@ii! to add a region whose \bground{} is painted to
% \!\pcol@bg@defined!.
% \end{itemize}
% 
% \item[\Uidx{$\!\@cdr!\<a_1\>\<a_2\>\cdots\<a_n\>\cs{@nil}$}]
% is an internal macro to be expanded to $\<a_2\>\cdots\<a_n\>$. It is used
% in $\!\pcol@getcurrpinfo!\arg{cs}$ to extract $\pp(\ptop)$ from
% $\!\pcol@currpage!=\!\@elt!\arg{\pp(\ptop)}$ and to \!\def!ine $\arg{cs}$
% letting it have $\pp(\ptop)$.
% 
% \item[\Uidx{$\!\protected@edef!\arg{macro}$\marg{body}}]
% is an internal macro to do \!\edef!$\arg{macro}$\marg{body} with the
% \!\protect!ion so that \!\protect!$\arg{cs}$ is kept in the expansion.  It
% is used in \!\pcol@fntextbody! to \!\edef!ine \!\@currentlabel!.
% 
% \item[\Uidx{$\!\@latex@warning@no@line!\arg{msg}$}]
% is an internal macro to report a warning message $\arg{msg}$ without the
% line number in which the cause lies.  It is used in \!\pcol@output! if
% a page with floats and very short main vertical list is built.
% 
% \item[\Uidx{\!\@eha!}]
% is an internal macro having a help message saying the command causing an
% error is ignored.  It is used in \!\pcol@zparacol!, \!\pcol@setcw@calcf!,
% \!\pcol@switchcolumn!, \!\pcol@switchenv!, and \!\addcontentsonly! as the
% argument of \!\PackageError!.
% 
% \item[\Uidx{\!\@ehb!}]
% is an internal macro having a help message saying the error causes a
% serious problem.  It is used in \!\pcol@ovf!, \!\pcol@zparacol! and
% \!\pcol@set@color@push! as the argument of \!\PackageError!.
% 
% \item[\Uidx{\!\@parmoderr!}]
% is an internal macro to complain about misplacement of a macro or
% environment which is expected to appear in ``outer par mode''.
% It is used in \!\pcol@zparacol! if it finds $\CSIndex{ifinner}=\true$.
% 
% \item[\Uidx{\!\@Esphack!}]
% is an internal macro to put back the horizontal skip and space factor
% saved by \!\@bsphack! at the end of an environment.  It is used in
% \!\pcol@end@dblfloat!.
% 
% \item[\Uidx{\!\reset@font!}]
% is an internal macro \!\let!-equal to \!\normalfont! to use a starndard
% font.  It is used in \!\pcol@fntextbody! for footnote typesetting.
% 
% \item[\Uidx{\!\set@color!}]
% is an internal macro to start coloring of texts following it.  By default
% it is \!\relax! but may have a definition to put a \!\special! for
% coloring with the color in \!\current@color!.  In the following macros, it
% is examined if $\!\set@color!=\!\relax!$ and/or some local definition is
% given to \!\set@color!.
% 
% \begin{itemize}
% \item
% \!\pcol@output! lets $\!\set@color!=\!\pcol@set@color!$ i.e., lets it
% regain its original definition because we don't need any special
% operations in \!\output! routine.
% 
% \item
% \!\@outputpage! performs \bgpaint{} if $\!\set@color!\neq\!\relax!$.
% 
% \item
% \!\pcol@zparacol! performs set-up operations for text coloring, including
% making \!\set@color! \!\let!-equal to \!\pcol@set@color@push! saving
% its original definition into \!\pcol@set@color!, and enabling \bgpaint{}
% macros if $\!\set@color!\neq\!\relax!$, while these \bgpaint{} macros are
% nullified otherwise.
% 
% \item
% \!\pcol@icolumncolor! complains that no color packages have been loaded if
% \!\set@color!${}={}$\!\relax!, and then otherwise temporarily lets it be
% the original saved in \!\pcol@set@color! to \!\insert! a \!\vbox! to
% update $\Celt^c$ or to do the update immediately.
% 
% \item
% \!\pcol@iicolumncolor! temporarily lets $\!\set@color!=\!\relax!$ so that
% \!\color! or \!\normalcolor! invoked in the macro just defines
% \!\current@color! to be set into $\Celtshadow^c$ without doing any other
% coloring operations.
% 
% \item
% \!\pcol@defcseprulecolor@i! complains that no color packages have been
% loaded if $\!\set@color!=\!\relax!$, while otherwise the macro temporarily
% lets $\!\set@color!=\!\relax!$ to invoke \!\color! (or \!\normalcolor!) to
% check if its arguments are properly given.
% 
% \item
% \!\pcol@backgroundcolor@ii! complains that no color packages have been
% loaded if $\!\set@color!=\!\relax!$, while otherwise its descendent 
% \!\pcol@backgroundcolor@x! temporarily lets
% $\!\set@color!=\!\pcol@backgroundcolor@y!$ to \!\def!ine
% $|\pcol@bg@|\~|color@|a[|@|{\cdot}c]$ to be \!\current@color!.
% 
% \SpecialArrayIndex{a}{\pcol@bg@color@}
% \SpecialArrayIndex{a{\cdot}\string\texttt{@}{\cdot}c}{\pcol@bg@color@}
% \end{itemize}
% 
% \item[\Uidx{\!\reset@color!}]
% is an internal macro to finish text coloring started by \!\set@color!.  By
% default it is undefined but may have some definition to put a \!\special!
% to finish coloring.  It is used in \!\pcol@clearcolorstack! so as to apply
% it to all elements in $\CST^c$, in \!\pcol@iscancst! to put it to the
% main vertical list in the case that $\Celt^c$ was $\bot$ and then updated,
% in \!\pcol@icolumncolor! to apply it to all elements in $\CSTshadow^c$,
% and in \!\pcol@reset@color@pop! and \!\pcol@reset@color@mpop! to have an
% uncoloring \!\special! in the \!\vbox! for $\celt_i$ and $\mcelt_{i,m}$.
% 
% \item[\Uidx{\!\color@begingroup!}]
% is an internal macro to open a group in which a color is specified.  It is
% used in \!\pcol@putfootins! to enclose footnotes with \!\normalcolor!, and
% in \!\pcol@fntextbody!\marg{text} to enclose the coloring in the footnote
% $\arg{text}$.
% 
% \item[\Uidx{\!\color@endgroup!}]
% is an internal macro to close a group in which a color is specified.  It is
% used in \!\pcol@putfootins! to enclose footnotes with \!\normalcolor!, and
% in \!\pcol@fntextbody!\marg{text} to enclose the coloring in the footnote
% $\arg{text}$.
% 
% \item[\Uidx{$\!\@stpelt!\<\theta\>$}]
% is an internal macro to zero-clear the counter $\theta$ for the
% implementation of \!\stepcounter!.  It is used in \!\pcol@stepcounter! to
% clear the descendent counters of a \gcounter{} $\cg$ listed in
% $|\pcol@cl@|{\cdot}\cg$.
% 
% \SpecialArrayIndex{\theta}{\pcol@cl@}
% 
% \item[\Uidx{\!\@nbitem!}]
% is an internal macro to insert a vertical skip of
% $\!\@outerparskip!-\!\parskip!$ above the first \!\item! of a
% \env{list}-like environment when $\CSIndex{if@nobreak}=\true$.  It is used
% by \!\pcol@zparacol! when it finds the \env{paracol} environment to start
% is at the very beginning of a \env{list}-like environment and
% $\CSIndex{if@nobreak}=\true$.
% 
% \item[\Uidx{\!\@parboxrestore!}]
% is an internal macro to set up typesetting paramenters for paragraphs
% encapsulated in a box.  It is used in \!\pcol@fntextbody! for paragraphs
% in a footnonte.
% 
% \item[\Uidx{\!\@finalstrut!}$\arg{box}$]
% is an internal macro to add an invisible vertical rule whose depth is that
% of a $\arg{box}$.  It is used in \!\pcol@fntextbody! to make the last line
% of the footnotes is as deep as \!\strutbox! at shallowest.
% 
% \item[\Uidx{\!\@sect!}]
% is an internal macro to implement sectioning commands.  The original
% definition of the macro is kept in \!\pcol@ac@enable@toc!, while
% \!\pcol@ac@def@toc! makes it \!\let!-equal to \!\pcol@ac@enable@toc! or
% \!\pcol@ac@disable@toc!, the latter of which uses the original version
% temporarily disabling \!\addcontentsline!.
% 
% \item[\Uidx{\!\@svsechd!}]
% is an internal macro (locally) \!\def!ined in \!\@sect! and \!\@ssect! to
% keep the section header of a sectioning command such as \!\paragraph!
% which puts the header as the leading text of the paragraph following the
% command rather than putting it as an individual paragraph.  The macro is
% \!\global!ized in \!\pcol@sptext! so that it is properly referred to in
% \!\everypar! for the paragraph led by the text in case that a \mctext{}
% has the sectioning command only and thus the \!\def!inition of the macro
% must survive after we close the group in which the \mctext{} is put.
% 
% \item[\Uidx{\!\@svsec!}]
% is an internal macro (locally) \!\def!ined in \!\@sect! to keep
% \!\thesection!  etc.\ to be displayed as the leading part of the section
% header.  The macro is \!\global!ized in \!\pcol@sptext! together with
% \!\@svsechd!  because it is in the body of \!\@svsechd!.
% 
% \item[\Uidx{\!\@caption!}]
% is an internal macro to implement \!\caption!.  The original definition of
% the macro is kept in \!\pcol@ac@caption@latex!, while
% \!\pcol@ac@caption@def! makes it \!\let!-equal to \!\pcol@ac@caption!
% which uses the original version temporarily disabling \!\addcontentsline!
% if necessary.
% 
% \item[\Uidx{\!\end@float!}]
% is an internal macro to close a \cwise{} float environment.  It is used in
% \!\pcol@end@dblfloat! (but never invoked because we havd \cs{if@twocolumn}
% true always).
% 
% \item[\Uidx{\!\end@dblfloat!}]
% is an internal macro to close a \pwise{} float environment.  It is replaced
% with our own \!\pcol@end@dblfloat!.
% 
% \item[\Uidx{\!\@endfloatbox!}]
% is an internal macro to close a \!\vbox! for a float.  It is used in
% \!\pcol@end@dblfloat!.
% 
% \item[\Uidx{\!\@largefloatcheck!}]
% is an internal macro to examine if a float is too large.  It is used in
% \!\pcol@end@dblfloat!.
% 
% \item[\Uidx{\!\@floatplacement!}]
% is an internal macro for \!\output! routine to reinitialize \cwise{}
% float placement parameters.  It is used in our own version of it,
% \!\pcol@floatplacement!.
% 
% \item[\Uidx{\!\@dblfloatplacement!}]
% is an internal macro for \!\output! routine to reinitialize \pwise{}
% float placement paramenters.  It is used in \!\pcol@startpage! and
% \!\pcol@output@clear! prior to processing \pwise{} floats in
% \!\@dbldeferlist!.  As discussed in
% item-(\ref{item:ovv-float-@dblfloatplacement}) of
% \secref{sec:imp-ovv-float}, this macro in 2015 or later version of
% \LaTeX{} lets $\!\f@depth!=|1sp|$.
% 
% \item[\Uidx{\!\@xympar!}]
% is an interal macro to perform the last operations for \!\marginpar!.  In
% \!\pcol@zparacol! it is made \!\let!-equal to \!\pcol@xympar! for the
% emulation of \!\marginnote!, while its original version is kept in
% \!\pcol@@xympar!.
% 
% \item[\Uidx{\!\p@footnote!}]
% is an internal macro to have the prefix to \!\thefootnote! in the printed
% reference of the counter \counter{footnote}.  It is used in
% \!\pcol@fntextbody! to produce \!\@currentlabel!.
% 
% \item[\Uidx{\!\@thefnmark!}]
% is an internal macro to have \!\thefootnote!\footnote{
% 
% Or \cs{thempfootnote} in \env{minipage} environment.}.
% 
% It is used in \!\pcol@fntextbody! to produce \!\@currentlabel!.
% 
% \item[\Uidx{\!\@footnotetext!}\marg{text}]
% is an internal macro to implement \!\footnote! and \!\footnotetext! for
% $\arg{text}$.  In \!\pcol@zparacol!, it is made \!\let!-equal to
% \!\pcol@fntext!.
% 
% \item[\Uidx{\!\@makefntext!}\marg{fn}]
% is an internal macro to typeset the footnote $\arg{fn}$.  It is used in
% \!\pcol@fntextbody!\marg{text} to typeset the footnote $\arg{text}$ with
% some other stuff.
% 
% \item[\Uidx{\!\@emptycol!}]
% is an internal macro for \!\output! routine to put back an empty page to
% the main vertical list.  It is used \!\pcol@output! if a page with floats
% and very short main vertical list is built.
% 
% \item[\Uidx{\!\@specialoutput!}]
% is an internal macro for \!\output! routine to process an \!\output!
% request made by \LaTeX's original \!\clearpage!, $\!\end!\Arg{float}$ and
% \!\marginpar!.  It is used in \!\pcol@specialoutput! to process the
% request for floats or marginal notes.
% 
% \item[\Uidx{\!\@opcol!}]
% is an internal macro for \!\output! routine to output a page or to keep
% the first column until the second one is completed.  This macro is used in
% \!\pcol@output! to process a sneaked \!\output! request from outside of
% \env{paracol}, and in \!\pcol@output@end! for the case \pwise{} floats
% are left at \Endparacol{} and they are put in \fpage{}s.
% 
% \item[\Uidx{\!\@makecol!}]
% is an internal macro for \!\output! routine to build the ship-out
% image of a column in \!\@outputbox! consisting of top floats, main
% vertical list in \!\box!|255|, footnotes in \!\footins!, and bottom
% floats\footnote{
% 
% In p\LaTeX, the order of footnotes and bottom floats are reversed.}.
% 
% It is used in \!\pcol@output! to process a sneaked \!\output!  request
% from outside of \env{paracol}, in \!\pcol@@makecol! for a \colpage{} to be
% flushed by \!\pcol@flushcolumn! and \!\pcol@makeflushedpage!, in
% \!\pcol@makecol! for an ordinary \colpage{}, and in \!\pcol@output@start!
% and \!\pcol@makenormalcol! for \preenv{}.
% 
% \item[\Uidx{\!\@textbottom!}]
% is an internal macro for \!\output! routine to be put at the bottom of
% \!\@outputbox! in which a \colpage{} is stored, by \!\@makecol!.  This
% macro is temporarily re\!\def!ined by \!\pcol@makecol! for a \colpage{}
% having \sync{}ation points so that it has a vertical skip of inifinite
% stretch and shrink to push up/down the stuff below the last \sync{}ation
% point in order to adjust its top to the point.  After that, its original
% definition kept in \!\pcol@textbottom! is restored.  Another modifiers of
% the macro are as follows; \!\pcol@makenormalcol! to make the macro
% \!\let!-equal to \!\relax! temporarily to avoid the insertion of whatever
% the macro has in \!\@makecol!; and \!\pcol@makeflushedpage! to let the
% macro have \!\vfil! temporarily so that empty columns in a \lpage{} are
% made {\em full size} without underfull.
% 
% \item[\Uidx{\!\@outputpage!}]
% is an internal macro for \!\output! routine to output a page kept in
% \!\@outputbox! together with the header and footer.  The original
% definition of this macro is saved in \!\pcol@@outputpage! to be used in
% \!\pcol@outputpage@l! and \!\pcol@outputpage@r! being callees of our own
% revised version of \!\@outputpage!.  Therefore, any \!\output! request to
% result in page ship-out reaches our own \!\@outputpage! and then \LaTeX's
% one after we perform operations for \parapag{}ing and \bgpaint{} onto the
% ship-out image.
% 
% \item[\Uidx{\!\@combinefloats!}]
% is an internal macro for \!\output! routine to combine top and bottom
% floats in \!\@toplist! and \!\@botlist! respectively with \!\@outputbox!
% in which the main vertical list and footnotes\footnote{
% 
% In p\LaTeX, footnotes are not in \cs{@outputbox} because of the reversal of
% footnotes and bottom floats.}
% 
% have been put by \!\@makecol!, and to have the result in \!\@outputbox!
% again.  In \!\pcol@zparacol!, it is made \!\let!-equal to our own
% \!\pcol@combinefloats!  so that \!\@makecol! and \!\pcol@makenormalcol!
% uses it instead of the original one.  However, if \!\pcol@output! finds
% that the \!\output!  request for a page break outside \env{paracol} is
% sneaked into our own \!\output! routine, it makes this macro \!\let!-equal
% to \!\pcol@@combinefloats! in which \LaTeX's version is kept so that
% \LaTeX's original \!\output! routine works perfectly as original.
% 
% \item[\Uidx{\!\@cflt!}]
% is an internal macro for \!\output! routine to put all top floats in
% \!\@toplist! and related stuff such as the vertical skip of
% \!\textfloatsep! into \!\@outputbox!  together with its old contents being
% the main vertical list and footnotes
% 
% \addtocounter{footnote}{-1}\footnotemark.
% 
% It is used in \!\pcol@combinefloats! if the \colpage{} being processed
% does not have \sync{}ation points.
% 
% \item[\Uidx{\!\@cflb!}]
% is an internal macro for \!\output! routine to put all bottom floats in
% \!\@botlist! and related stuff such as the vertical skip of
% \!\textfloatsep! into \!\@outputbox!  together with its old contents being
% top floats, the main vertical list and footnotes
% 
% \addtocounter{footnote}{-1}\footnotemark.
% 
% It is used in \!\pcol@combinefloats!.
% 
% \item[\Uidx{$\!\@comflelt!\arg{flt}$}]
% is an internal macro for \!\output! routine to put $\arg{flt}$ being a
% \cwise{} top or bottom float to the tail of \!\@tempboxa! which
% finally has all top\slash bottom floats in a column.  It is used in
% \!\pcol@cflt! to apply it to each element of \!\@toplist!.
% 
% \item[\Uidx{$\!\@comdblflelt!\arg{flt}$}]
% is an internal macro for \!\output! routine to put $\arg{flt}$ being a
% \pwise{} float to the tail of \!\@tempboxa! which finally has the
% \spanning{} for \pwise{} floats.  It is used in \!\pcol@startpage! to
% apply to each element of \!\@dbltoplist! to have the \spanning.
% 
% \item[\Uidx{\!\@startcolumn!}]
% is an internal macro for \!\output! routine which tries to build a
% \fcolumn{} for the floats in \!\@deferlist! and, if the column
% is not built, tries to move floats to \!\@toplist! and \!\@botlist!.  This
% macro is used in \!\pcol@output! to process a sneaked \!\output! request
% from outside of \env{paracol}, and in \!\pcol@output@end! for the case
% \pwise{} floats are left at \Endparacol{} after which they become
% \cwise{} ones.
% 
% \item[\Uidx{$\!\@tryfcolumn!\arg{lst}$}]
% is an internal macro for \!\output! routine which examines if a \fcolumn{}
% or a \fpage{} can be built with some floats in $\arg{lst}$ and, if so,
% builds the page in \!\@outputbox! removing floats put in the page from
% $\arg{lst}$.  It is used in \!\pcol@startpage! for \pwise{} floats in
% \!\@dbldeferlist!, and in \!\pcol@startcolumn! for \cwise{} floats in
% $\cc_c(\dl)$.
% 
% \item[\Uidx{$\!\@scolelt!\arg{flt}$}]
% is an internal macro for \!\output! routine which examines if a
% \cwise{} float $\arg{flt}$ can be added to \!\@toplist! or
% \!\@botlist! being the list of the floats to be put at the top or bottom
% of a page respectively.  Then, if the examination succeeds, $\arg{flt}$ is
% added to \!\@toplist! or \!\@botlist!, while it is added to
% \!\@deferlist! otherwise.  It is used in \!\pcol@trynextcolumn! to
% apply to each element of (the copy of) $\cc_c(\dl)$.
% 
% \item[\Uidx{$\!\@sdblcolelt!\arg{flt}$}]
% is an internal macro for \!\output! routine which examines if a
% \pwise{} float $\arg{flt}$ can be added to \!\@dbltoplist! being the
% list of the floats to be put at the top of a page, and if so, adds
% $\arg{flt}$ to \!\@dbltoplist!, while it is added to \!\@dbldeferlist! or
% \!\@deferlist! depending on \LaTeX's version otherwise as discussed in
% item-(\ref{item:ovv-float-@addtodblcol}) of \secref{sec:imp-ovv-float}.
% It is used in \!\pcol@startpage! to apply to each element of (the copy of)
% \!\@dbldeferlist!.
% 
% \item[\Uidx{\!\@addmarginpar!}]
% is an internal macro for \!\output! routine to add a marginal note.  Its
% original definition is kept in \!\pcol@@addmarginpar! and used in
% \!\pcol@addmarginpar! being our own \!\@addmarginpar!.
% 
% \item[\Uidx{$\!\@makefcolumn!\arg{lst}$}]
% is an internal macro for \!\output! routine to build a \fcolumn{} with
% some floats at the head of a float list $\arg{lst}$ and to remove the
% floats from the list.  It is used in \!\pcol@flushcolumn! and
% \!\pcol@iflushfloats! for $\cc_c(\dl)$ of \cwise{} floats, and in
% \!\pcol@output@clear! for \!\@dbldeferlist! of \pwise{} floats.
% \end{description}
%
% 
% 
% \subsubsection{Structural Macros}
% 
% \begin{description}
% \item[\Uidx{$\!\@elt!\<a_1\>\cdots\<a_n\>$}]
% is an internal control sequence to represent a list element having $n$
% subelements.  The sequence is often made \!\let!-equal to a macro which
% processes $\<a_1\>\cdots\<a_n\>$ and is applied to all members in a list.
% It is also made \!\let!-equal to \!\relax! on a manipulation of a list,
% such as element addition and concatenation, by \!\edef! or \!\xdef!.  The
% usages of the sequence are as follows.
% 
% \begin{itemize}
% \item
% \!\pcol@F@count! \!\def!ines \!\@elt! as a macro to increment
% \!\@tempcnta! by one to measure the cardinality of \!\@freelist!.
% 
% \item
% \!\pcol@cflt! lets $\!\@elt!=\!\@comflelt!$ for $\cc_c(\tl)$, and then
% $\!\@elt!=\!\relax!$ to concatenate \!\@freelist! and $\cc_c(\tl)$.
% 
% \item
% \!\pcol@setpageno! lets $\!\@elt!=\!\pcol@setpnoelt!$ for $\PP$.
% 
% \item
% \!\pcol@setpnoelt! and \!\pcol@setmpbelt! let $\!\@elt!=\!\relax!$ to add
% $\pp(q)$ to $\PP$.  It also uses \!\@elt! to \!\def!ine \!\pcol@currpage!
% with $\!\@elt!
% \Arg{\pp^p(q)}\<\pp^i(q)\>\<\pp^f(q)\>\~\Arg{\pp^s(p)}\Arg{\pp^m(p)}$.
% 
% \item
% \!\pcol@defcurrpage! lets $\!\@elt!=\!\relax!$ to \!\xdef!ine
% \!\pcol@currpage! with
% $\!\@elt!\Arg{\pp^p(q)}\<\pp^i(q)\>\<\pp^f(q)\>\Arg{\pp^s(p)}\Arg{\pp^m(p)}$.
% 
% \item
% \!\pcol@nextpage! lets $\!\@elt!=\!\pcol@nextpelt!$ for $\PP$.
% 
% \item
% \!\pcol@getcurrpage! lets $\!\@elt!=\!\pcol@getpelt!$ for $\PPP$.
% 
% \item
% \!\pcol@startpage! lets $\!\@elt!=\!\@sdblcolelt!$ for (the copy of)
% \!\@dbldeferlist!, and $\!\@elt!=\!\@comdblflelt!$ for \!\@dbltoplist!.  It
% also lets $\!\@elt!=\!\relax!$ for the concatenation of \!\@dbldeferlist!
% and \!\@deferlist!, and that of \!\@freelist! and \!\@dbltoplist!.
% 
% \item
% \!\pcol@outputcolumns! lets $\!\@elt!=\!\pcol@outputelt!$ with two arguments
% for (the copy of) $\PP$.
% 
% \item
% \begin{Sloppy}{2400}%
% \!\pcol@ioutputelt! \!\def!ines \!\pcol@bg@footnoteheight! and
% \!\pcol@bg@float~height! with \!\@elt! to let painting macros add elements
% in them to have the height of the \bground{} regions $\bgr_{\{n,N\}}$ and
% $\bgr_{\{f,F\}}$ to be painted.
% \end{Sloppy}
% 
% \item
% \!\pcol@buildcolseprule! lets $\!\@elt!=\!\pcol@buildcselt@S!$ and then
% $\!\@elt!=\!\pcol@buildcselt!$ for $\pp^s(p)$, and then \!\def!ines
% \!\pcol@bg@columnheight! with \!\@elt! to add $H'_n$ and $\!\@maxdepth!$
% or 0 for the \bground{} region $\bgr_{\{c,g\}}^c(n+1)$ where
% $n=\Abs{\pp^s(p)}$.
% 
% \item
% \!\pcol@buildcselt@S! \!\def!ines \!\pcol@bg@spanningtop! and
% \!\pcol@bg@spanning~height! with \!\@elt! to define the region
% $\bgr_S(i)$.
% 
% \item
% \begin{Hfuzz}{0.55pt}
% \!\pcol@buildcselt! \!\def!ines \!\pcol@bg@columnheight!,
% \!\pcol@bg@spanningtop! and \!\pcol@bg@spanningheight! with \!\@elt! to
% define regions $\bgr_{\{c,g\}}^c(i)$ and $\bgr_s(i)$.
% \end{Hfuzz}
% 
% \item
% $\!\pcol@bg@calculate!\arg{z}\arg{z_0}\Arg{F}$ lets
% $\!\@elt!=\!\pcol@bg@advance!$ to let $\!\@elt!\arg{f}$ in $F$ do
% $z\gets z+f$.
% 
% \item
% $\!\pcol@bg@negative!\Arg{F^-}$ lets $\!\@elt!=\!\pcol@bg@nadvance!$ to
% let $\!\@elt!\arg{f}$ in $F^-$ do $z\gets z-f$, and then lets
% $\!\@elt!=\!\pcol@bg@advance!$ to go back to addition.
% 
% \item
% \!\pcol@output@start! lets all floats $f$ imported in \!\@dbldeferlist!
% have depth 0 by \!\def!ining $\!\@elt!\arg{f}$ to do it, \!\edef!ines
% \!\pcol@bg@textheight! with \!\@elt!  having height-plus-depth of
% \preenv{} for \bgpaint{} of it, and \!\def!ines $\pp^m(0)$ having one
% element in $\mpb_L^l$ or $\mpb_L^r$ for \!\@mparbottom! in \preenv{}.
% 
% \item
% \!\pcol@makenormalcol! lets $\!\@elt!=\!\relax!$ to concatenate
% \!\@freelist! and \!\@midlist!.
% 
% \item
% \!\pcol@trynextcolumn! lets $\!\@elt!=\!\@scolelt!$ for (the copy of)
% $\cc_c(\dl)$.
% 
% \item
% \!\pcol@setcurrcol! lets $\!\@elt!=\!\relax!$ to \!\xdef!ine $\cc_c$ with
% $\cc_c(\tl)=\!\@toplist!$, $\cc_c(\ml)=\!\@midlist!$,
% $\cc_c(\bl)=\!\@botlist!$  and $\cc_c(\dl)=\!\@deferlist!$.
% 
% \item
% \!\pcol@scancst! and \!\pcol@iscancst! let $\!\@elt!=\!\relax!$ to
% \!\edef!ine the list $M=(m\,\Bar\,\mceltpop_{j,m}\in\cstraw,\;j\geq i)$ for
% $\mceltpop_{i,*}$, and then the latter \!\def!ines \!\@elt! as a macro with
% an argument $m$ to examine $m\in M$ for $\mcelt_{i,m}$.
% 
% \item
% \!\pcol@addmarginpar! lets $\!\@elt!=\!\pcol@setmpbelt!$ for $\PPP$.
% 
% \item
% \!\pcol@getmparbottom! lets $\!\@elt!=\!\pcol@getmpbelt!$ for the list
% $\mpb_{\{L,R\}}^{\{l,r\}}$.  It also lets $\!\@elt!=\!\relax!$ for the
% addition of $\mpar(h_i,t_i)$ to the list by itself and
% \!\pcol@getmpbelt!.
% 
% \item
% \!\pcol@mparbottom@zero! has \!\@elt! in its body to have $\mpar(0,0)$ for
% each $M\in\Set{\mpb_X^x}{X\in\{L,R\},x\in\{l,r\}}$ the macro has.
% 
% \item
% $\!\pcol@bias@mpbout@i!\Arg{y}\!\@elt!\Arg{t}\Arg{b}\!\@nil!$ has \!\@elt!
% in its argument specification, and \!\def!ines \!\reserved@b! with
% \!\@elt! for $\mpar(t+y,b+y)$.
% 
% \item
% \begingroup\def\,{\mskip0mu plus2mu\relax}
% $\!\pcol@getmparbottom@last@i!\,
% \Arg{y}\,\mpar(t_1,b_1)\,\cdots\,\mpar(t_n,b_n)\,\!\@nil!$ at first
% \!\def!\-ines \!\reserved@b! with \!\@elt! for $\mpar(y,y)$ and then
% \!\def!ines \!\@elt!  to do it for $\mpar(t_i,b_i)$ for all $i\in[1,n]$.
% \endgroup
% 
% \item
% \!\pcol@makefcolumn! lets $\!\@elt!=\!\pcol@makefcolelt!$ for (the copy
% of) $\cc_c(\dl)$ to examine if each float in it can be put in a \fcolumn{}
% to be built, and then \!\def!ine it to put floats in $\cc_c(\tl)$ into
% $\cc_c(\vb^b)$.
% 
% \item
% \!\pcol@addflhd! lets $\!\@elt!=\!\pcol@hdflelt!$ for its argument
% $\cc_c(\tl)$ or $\cc_c(\bl)$, and then $\!\@elt!=\!\relax!$ to give the
% default.
% 
% \item
% \!\pcol@makeflushedpage! \!\edef!ines \!\pcol@bg@floatheight! with
% \!\@elt! letting it be \!\relax!.
% 
% \item
% \!\pcol@imakeflushedpage! \!\def!ines \!\pcol@bg@footnoteheight! with
% \!\@elt!.
% 
% \item
% \!\pcol@output@end! uses \!\@elt! in the argument specification of the
% \!\def!inition of \!\pcol@do@mpbout@elem!, lets $\!\@elt!=\!\relax!$ to
% add the \spanning{} of the last page to the head of \!\@dbldeferlist!, and
% uses it in the body of \!\pcol@bg@textheight! and
% \!\pcol@bg@footnoteheight! to be \!\def!ined for \bgpaint{} of
% \Scfnote{}s.
% 
% \item
% \begin{Hfuzz}{0.14pt}%
% \!\pcol@zparacol! lets $\!\@elt!=\!\pcol@remctrelt!$ for $\CG$,
% $\!\@elt!=\!\pcol@thectrelt!$ for $\CTL$, $\!\@elt!=\!\pcol@loadctrelt!$
% for $\Cc_0$, $\!\@elt!=\!\pcol@cmpctrelt!$ for $\CTL$,
% $\!\@elt!=\!\pcol@defcomelt!$ for \!\pcol@localcommands!, and then
% $\!\@elt!=\!\relax!$ to give the default.
% \end{Hfuzz}
% 
% \item
% \!\globalcounter!\marg{ctr} defeines \!\@elt!\marg{\theta} for $\CG$ to
% check if there is $\theta\in\CG$ such that $\theta=\arg{ctr}$.
% 
% \item
% \!\pcol@localcommands! has the sequence of $\!\@elt!\ARg{com}$ for all
% \lcommand{}s $|\|\arg{com}$.
% 
% \item
% \!\pcol@gcounters! has \!\@elt!|{page}| as its initial definition.
% 
% \item
% $\!\pcol@removecounter!\arg{\Theta'}\arg{\theta}$ lets
% $\!\@elt!=\!\pcol@iremctrelt!$ for (the copy of) its argument
% $\mathit{\Theta}'$ to remove $\theta$ from it.
% 
% \item
% $\!\pcol@sscounters!\arg{elt}$ lets $\!\@elt!=\arg{elt}$, where
% $\arg{elt}=\!\pcol@storectrelt!$ or $\arg{elt}=\!\pcol@savectrelt!$, for
% $\CTL$, and then $\!\@elt!=\!\relax!$ to \!\xdef!ine $\Cc_c$.
% 
% \item \begingroup\hfuzz1.4pt
% $\!\pcol@com@synccounter!\<\theta\>$ gives $\!\@elt!\Arg{\theta}$ as the
% argument of |\pcol@synccounter|.
% 
% \SpecialIndex{\pcol@synccounter}
% \par\endgroup
% 
% \item \begingroup\hfuzz0.51pt
% $\!\pcol@synccounter!\arg{lst}$ lets $\!\@elt!=\!\relax!$ to have the list
% $\arg{lst}$ in \!\reserved@a! by \!\edef!, $\!\@elt!=\!\pcol@loadctrelt!$
% for $\CC_c$, and then $\!\@elt!=\!\pcol@syncctrelt!$ for $\arg{lst}$.
% \par\endgroup
% 
% \item
% $\!\pcol@stepcounter!\<\theta\>$ lets $\!\@elt!=\!\pcol@stpldelt!$ for
% $\CTL$, $\!\@elt!=|\pcol@|\~|stpclelt|$
% 
% \SpecialIndex{\pcol@stpclelt}
% 
% for $\clist(\theta)$, and then $\!\@elt!=\!\@stpelt!$ for
% $\clist(\theta)$.
% 
% \item
% \!\pcol@switchcol! lets $\!\@elt!=\!\pcol@setctrelt!$ for $\CC_c$,
% $\!\@elt!=|\pcol@|\~|aconlyelt|$
% 
% \SpecialIndex{\pcol@aconlyelt}
% 
% for $\T$, and then $\!\@elt!=\!\relax!$ to give the default.
% 
% \item
% \!\pcol@icolumncolor! \!\def!ines $\!\@elt!\arg{\celtshadow_i}$ to apply
% \!\reset@color! for rewinding and \!\pcol@set@color! for reestablishing
% to each $\celtshadow_i\in\CSTshadow^c=(\Celtshadow^c,\cstshadow)$
% by \!\pcol@scancst@shadow!, in which \!\@elt! is explicitly applied to
% $\Celtshadow^c$ if it is defined and then implicitly done to
% $\cstshadow=\!\pcol@colorstack@shadow!$.
% 
% \item
% \!\pcol@set@color@push! lets $\!\@elt!=\!\relax!$ to push a color
% information into $\cstshadow$, with save\slash restore of its original
% value.
% 
% \item
% \!\resetbackgroundcolor! lets $\!\@elt!=\!\pcol@resetbackgroundcolor!$ to
% scan \!\pcol@bg@defined! containing $\!\@elt!\Arg{a'_i}$ to let
% $|\pcol@bg@color|{\cdot}a'_i$ for each of $i$.
% 
% \SpecialArrayIndex{a}{\pcol@bg@color@}
% \SpecialArrayIndex{a{\cdot}\string\texttt{@}{\cdot}c}{\pcol@bg@color@}
% \end{itemize}
% 
% \item[\Uidx{\!\@empty!}]
% is a macro always having nothing.  Its major usages are to examine if a
% macro often having a list is empty, and to make such a macro empty.  The
% following macros use \!\@empty! to examine the emptiness of the objects in
% parentheses.
% 
% \begin{quote}
% \!\pcol@combinefloats! ($\cc_c(\tl)$, $\cc_c(\bl)$)\\
% \!\pcol@checkshipped! ($S_c$)\\
% \!\pcol@startpage! ($\pp(\ptop)$, \!\@dbltoplist!)\\
% \!\pcol@makenormalcol! (\!\@botlist! in \preenv)\\
% \!\pcol@getmparbottom! ($\pp^m(p)$)\\
% \!\pcol@getmparbottom@last! ($\pp^m(\ptop)$)\\
% \!\pcol@setmpbelt! ($\pp^m(p)$)\\
% \!\pcol@flushcolumn! ($\cc_c(\dl)$)\\
% \!\pcol@makefcolumn! ($\cc_c(\tl)$, $\cc_c(\dl)$)\\
% \!\pcol@measurecolumn! ($\cc_c(\bl),\cc_c(\dl)$)\\
% \!\pcol@addflhd! ($\cc_c(\tl)$, $\cc_c(\bl)$)\\
% \!\pcol@synccolumn! ($\cc_c(\tl)$)\\
% \!\pcol@makeflushedpage! ($\cc_c(\dl)$)\\
% \!\pcol@imakeflushedpage! ($\cc_c(\dl)$)\\
% \!\pcol@iflushfloats! ($\cc_c(\dl)$)\\
% $\!\pcol@setcw@getspec@i!\arg{default}\<x'_d\>$ ($x'_d$)\\
% $\!\pcol@setcw@fill!\<f\>\!\fill!$ ($f$)
% \end{quote}
% 
% The following macros use \!\@empty! to empty the objects in parentheses.
% 
% \begin{quote}
% \!\pcol@cflt! ($\cc_c(\tl)$)\\
% \!\pcol@setpageno! ($\PP$, $\pp(\ptop)$)\\
% \!\pcol@startpage! ($\cc_c(\dl)$, \!\@dbldeferlist!, \!\@deferlist!,
% 	\!\@dbltoplist!)\\
% \!\pcol@trynextcolumn! ($\cc_c(\dl)$)\\
% \!\pcol@output@start! ($\PP$, $\pp(\ptop)$, $\cc_c(\dl)$)\\
% \!\pcol@makenormalcol! (\!\@midlist! in \preenv)\\
% \!\pcol@addmarginpar! ($\PP$, $\pp(\ptop)$)\\
% \!\pcol@getmparbottom! ($\mpb_{\{L,R\}}^{\{l,r\}}$)\\
% \!\pcol@makefcolumn! ($\cc_c(\dl)$, $\cc_c(\tl)$)\\
% \!\pcol@makeflushedpage! ($\cc_c(\tl)$)\\
% \!\pcol@freshpage! ($\PP$, $\pp(\ptop)$)\\
% \!\pcol@zparacol! (\!\@gtempa!)\\
% \!\pcol@removecounter!$\<\mathit{\Theta}'\>\Arg{\theta}$
% ($\mathit{\Theta}'\in\{\CG,\CL\}$)\\
% \!\pcol@sscounters! ($\Cc_c$)
% \end{quote}
% 
% \item[\Uidx{\!\@currentlabel!}]
% is an internal macro to have the reference to be associated with the
% \!\label! following it.  It is defined in \!\pcol@fntextbody! to have the
% reference to the footnote with \!\@thefnmark!.
% 
% \item[\Uidx{\!\ext@figure!}]
% is an internal macro having ``|lof|'' being the extention of the file for
% list of figures.  It is used in
% $\!\pcol@ac@caption!\arg{type}|[|\arg{lcap}|]|\<cap\>$ to have ``|lof|''
% when $\arg{type}=|figure|$.
% 
% \item[\Uidx{\!\ext@table!}]
% is an internal macro having ``|lot|'' being the extention of the file for
% list of tables.  It is used in
% $\!\pcol@ac@caption!\arg{type}|[|\arg{lcap}|]|\<cap\>$ to have ``|lot|''
% when $\arg{type}=|table|$.
% 
% \item[\Uidx{\!\@currbox!}]
% is an internal macro which conventionally has an \!\insert! for floats,
% etc.  The follwing macros use \!\@currbox! having the objects in
% parentheses.
% 
% \begin{quote}
% \!\pcol@opcol! ($\cc_c(\vb)$)\\
% \!\pcol@startpage! ($\pp^i(\ptop)$)\\
% \!\pcol@ioutputelt! ($\s_c(q)$)\\
% \!\pcol@output@start! ($\pp^i(0)$, $\cc_c(\vb)$, $\Celt^c$)\\
% \!\pcol@output@switch! ($\cc_c(\vb)$)\\
% \!\pcol@restartcolumn! ($\cc_c(\vb)$)\\
% \!\pcol@igetcurrcol! ($\cc_c(\vb)$)\\
% \!\pcol@setcurrcol! ($\cc_c(\vb)$)\\
% \!\pcol@putbackmvl! ($\cc_c(\vb)$)\\
% \!\pcol@iscancst! ($\Celt^c$)\\
% \!\pcol@addmarginpar! (left marginal note)\\
% \!\pcol@flushcolumn! ($\cc_c(\vb), \s_c(q)$)\\
% \!\pcol@makefcolumn! ($\cc_c(\vb)$)\\
% \!\pcol@measurecolumn! ($\cc_c(\vb)$)\\
% \!\pcol@synccolumn! ($\cc_c(\vb)$)\\
% \!\pcol@imakeflushedpage! ($\cc_c(\vb)$)\\
% \!\pcol@freshpage! ($\cc_c(\vb)$)\\
% \!\pcol@output@end! (top float, $\cc_c(\vb)$)\\
% \!\pcol@end@dblfloat! (\pwise{} float)\\
% \!\pcol@icolumncolor! ($\Celt^c$)
% \end{quote}
% 
% \item[\Uidx{\!\@marbox!}]
% is an internal macro which has an \!\insert! for left marginal notes.  It
% is used in \!\pcol@xympar!.
% 
% \item[\Uidx{\!\@currlist!}]
% is an internal macro which has an list of \!\insert!s for floats and
% marginal notes given to \!output!.  It is used in \!\pcol@addmarginpar! to
% get the right marginal note from its head.
% 
% \item[\Uidx{\!\@freelist!}]
% is an internal macro having available \!\insert!s for floats originially,
% but also \colpage{}s, \spanning{}, footnotes and default column-color in
% our usage.  Besides the acquisition of an \!\insert! from it shown in the
% description of \!\@next!, it is used by the following macros to return the
% the objects in parentheses to \!\@freelist!.
% 
% \begin{quote}
% \!\pcol@cflt! ($\cc_c(\tl)$)\\
% \!\pcol@startpage! (\!\@dbltoplist!)\\
% \!\pcol@outputelt! ($\pp^i(q)$)\\
% \!\pcol@ioutputelt! ($\pp^i(q)$, $\pp^f(q)$, $\s_c(q)$)\\
% \!\pcol@startcolumn! ($\pp^f(\ptop)$)\\
% \!\pcol@output@start! ($\cc_0(\vb)$)\\
% \!\pcol@makenormalcol! (\!\@midlist! in \preenv)\\
% \!\pcol@restartcolumn! ($\cc_c(\vb)$, $\pp^f(p)$ or $\cc_c(\ft)$)\\
% \!\pcol@savefootins! ($\pp^f(p)$ or $\cc_c(\ft)$)\\
% \!\pcol@flushcolumn! ($\cc_c(\ft)$)\\
% \!\pcol@makefcolpage! ($\cc_c(\tl)$)\\
% \!\pcol@makeflushedpage! ($\pp^i(\ptop)$, $\pp^f(\ptop)$)\\
% \!\pcol@imakeflushedpage! ($\cc_c(\ft)$)\\
% \!\pcol@output@end! ($\pp^f(\ptop)$, $\cc_c(\vb)$, $\Celt^c$)
% \end{quote}
% 
% In addition \!\pcol@F@count! scans its element to have its cardinality.
% 
% \item[\Uidx{\!\@nil!}]
% is an internal control sequence which is conventionally used to terminate
% a variable length argument.  It is used in the following macros.
% 
% \begin{itemize}
% \item
% \!\pcol@getcurrpinfo! for the invocation of \!\@cdr!.
% 
% \item
% $\!\pcol@bias@mpbout@i!\Arg{y}\!\@elt!\Arg{t}\Arg{b}\!\@nil!$ to capture
% $t$ and $b$ following the convention in \!\pcol@do@mpb@all@ii!.
% 
% \item
% \begingroup\def\,{\mskip0mu plus2mu\relax}
% $\!\pcol@getmparbottom@last@i!\,
% \Arg{y}\,\mpar(t_1,b_1)\,\cdots\,\mpar(t_n,b_n)\,\!\@nil!$ to capture
% $\mpar(t_i,b_i)$ for all $i\in[1,n]$.
% \endgroup
% 
% \item
% $\!\pcol@do@mpb@all@i!
% \Arg{\mpb_L^l}\Arg{\mpb_L^r}\Arg{\mpb_R^l}\Arg{\mpb_R^r}$ to terminate the
% list $M\in\Set{\mpb_X^x}{X\in\{L,R\},x\in\{l,r\}}$ in the invocation of
% \!\pcol@do@mpb@all@ii!.
% 
% \item
% $\!\pcol@do@mpb@all@ii!
% \Arg{y}\mpar(t_1,b_1)\cdots\mpar(t_n,b_n)\!\@nil!$ to capture
% $\mpar(t_i,b_i)$ for all $i\in[1,n]$, and then to terminate them
% passed to \!\pcol@bias@mpbout@i! or \!\pcol@getmparbottom@last!.
% 
% \item
% \!\pcol@setcw@scan! for the invocation of\\\mbox{}\qquad
% $\!\pcol@setcw@getspec!\<w'_d\>|/|\<g'_d\>|/|\arg{garbage}\~\!\@nil!$.
% 
% \item
% \!\pcol@setcw@getspec@i! for the invocation of\\\mbox{}\qquad
% $\!\pcol@extract@fil!\<n\>| plus |\<f\>| minus|\arg{garbage}\!\@nil!$\\
% and thus in the \!\def!inition of it in \!\pcol@def@extract@fil!.
% 
% \item
% \!\pcol@defkw! has \!\@nil! in its argument specification to terminate the
% argument $0\,|pt|\ |plus|\ 1\,|fil|\ |minus|\ 1\,|fil|$.
% 
% \item
% \!\pcol@extract@fil@i!, \!\pcol@extract@fil@ii! and
% \!\pcol@extract@fil@iii! have \!\@nil! in their argument specifications as
% the terminator, and thus \!\@nil! appears in their invocations in
% \!\pcol@extract@fil!, \!\pcol@extract@fil@i! and \!\pcol@extract@fil@ii!
% respectively, and in the \!\def!inition of \!\pcol@extract@fil@iii! in
% \!\pcol@def@extract@fil@iii!.
% 
% \item
% \!\pcol@iadjustfnctr! and \!\pcol@iifootnotetext! to terminate their
% argument |[|$[|+-|]\meta{disp}$|]| passed to \!\pcol@calcfnctr!.
% 
% \item
% \!\backgroundcolor! and \!\nobackgroundcolor! to terminate their first
% argument given to \!\pcol@backgroundcolor! so that its descendants
% \!\pcol@backgroundcolor@x! and \!\pcol@backgroundcolor@z! finally capture
% everything not processed by their ancestors.
% \end{itemize}
% 
% \item[\Uidx{\!\current@color!}]
% is an internal macro having color information to be put into |.dvi| as a
% part of the argument of coloring \!\special!.  It is referred to by
% \!\pcol@bg@paintregion@i!,
% \!\pcol@output@start!,
% \!\pcol@icolumncolor!,
% \!\pcol@iicolumncolor!, and
% \!\pcol@set@color@push!,
% \!\pcol@backgroundcolor@y!.
% 
% \item[\Uidx{\!\@dbldeferlist!}]
% is an internal macro having the list of \pwise{} floats whose page
% appearance are not yet fixed.  It is scanned and then updated in
% \!\pcol@startpage! and \!\pcol@output@clear!, while \!\pcol@output@start!
% lets it have \!\@deferlist! made before \beginparacol, and
% \!\pcol@output@end! adds \pwise{} floats to be put in the empty
% \lpage{} to it and then move the whole of the list to \!\@deferlist!.
% As discussed in \secref{sec:imp-ovv-float}, 2015 or later version of
% \LaTeX{} no longer uses this list, but we stick with it for \pwise{}
% floats produced in \env{paracol} environments and thus have its top level
% definition with empty body in \Paracol.
% 
% \item[\Uidx{\!\@dbltoplist!}]
% is an internal macro having the list of \pwise{} floats which
% \!\@sdblcolelt! decided to be put in the new page.  The macro
% \!\pcol@startpage! scans it to put all floats into the page $\ptop$ as its
% \spanning{} $\pp^i(\ptop)$, and then empties it after returning all floats
% to \!\@freelist!.
% 
% \item[\Uidx{\!\@deferlist!}]
% is an internal macro having the list of \cwise{} floats whose page
% appearance are not yet fixed.  It is scanned and then updated in
% \!\pcol@startcolumn!, \!\pcol@trynextcolumn!, \!\pcol@flushcolumn!,
% \!\pcol@makefcolumn! and \!\pcol@iflushfloats!, while the following macros
% also act on it.
% 
% \begin{itemize}
% \item
% \!\pcol@output@start! moves it to \!\@dbldeferlist! because it is created
% before \beginparacol.
% 
% \item
% \!\pcol@startpage! uses it as the interface with \!\@addtodblcol! of 2015
% or later version of \LaTeX{} as discussed in
% item-(\ref{item:ovv-float-@addtodblcol}) of
% \secref{sec:imp-ovv-float}.
% 
% \item
% \!\pcol@setcurrcol! and \!\pcol@igetcurrcol! saves\slash restores it
% into/from $\cc_c(\dl)$, respectively.
% 
% \item
% \!\pcol@makefcolelt!$\arg{flt}$ returns $\arg{flt}$ to the list if
% $\arg{flt}$ cannot be put in the \fcolumn{} which the macro is working on.
% 
% \item
% \!\pcol@measurecolumn! examines its emptiness to let
% $\CSIndex{ifpcol@dfloats}=\true$ iff not empty.
% 
% \item
% \!\pcol@output@end! lets it have \!\@dbldeferlist! so that it processed as
% \cwise{} floats after \Endparacol.
% \end{itemize}
% 
% \item[\Uidx{\!\@toplist!}]
% is an internal macro having the list of \cwise{} floats which is
% decided to be put at the top of the \ccolpage{} by float environments or by
% \!\pcol@trynextcolumn!.  This list is scanned by \!\pcol@cflt! if its
% invoker \!\pcol@combinefloats! finds the macro is not empty.  The list is
% also scanned by \!\pcol@makecol!, \!\pcol@output@switch! and
% \!\pcol@measurecolumn! using \!\pcol@addflhd! for the measurement of the
% combined size of top floats, while \!\pcol@setcurrcol! and
% \!\pcol@igetcurrcol!  saves\slash restores the list into/from $\cc_c(\tl)$
% respectively.  In addition our macros may add an element or build the
% entire list in the following two cases.  One case is for a \sync{}ation
% for which \!\pcol@synccolumn! lets the main vertical list in
% $\cc_c(\vb^b)$ be a float, namely {\em\mvlfloat} to be added to this list.
% Another case is for a \fcolumn{} in the \lpage{} for which
% \!\pcol@makefcolumn! and \!\pcol@makefcolelt! build this list for deferred
% floats.  In the latter case, the list is scanned by \!\pcol@makefcolpage!
% invoked from \!\pcol@makefcolumn! itself, \!\pcol@flushcolumn! and
% \!\pcol@makeflushedpage!.
% 
% \item[\Uidx{\!\@botlist!}]
% is an internal macro having the list of \cwise{} floats which is
% decided to be put at the bottom of the \ccolpage{} by float environments or
% by \!\pcol@trynextcolumn!.  The emptiness of this list examined by
% \!\pcol@combinefloats! to invoke the scanner \!\@cflb! unless empty,
% by \!\pcol@output@start! to calculate the room for each \colpage{} in the
% \spage, and by \!\pcol@makenormalcol! to decide whether \!\@makecol! is
% used or not for building \preenv.  The list is also scanned by
% \!\pcol@measurecolumn! for the measurement of the combined size of bottom
% floats, while \!\pcol@setcurrcol! and \!\pcol@igetcurrcol! saves\slash
% restores the list into/from $\cc_c(\bl)$ respectively.
% 
% \item[\Uidx{\!\@midlist!}]
% is an internal macro having the list of in-text floats which has already
% been put in the \ccolpage{} but is kept to check the ordering of the
% succeeding floats.  The list is emptied by \!\pcol@makenormalcol! after
% returning all elements in it to \!\@freelist!, while \!\pcol@setcurrcol!
% and \!\pcol@igetcurrcol! saves\slash restores the list into/from
% $\cc_c(\ml)$ respectively.
% 
% \item[\Uidx{\!\f@depth!}]
% is an internal macro having |1sp| or being \!\let!-equal to \!\z@! to
% specify the float category, \pwise{} or \cwise{} respectively, which
% float-related macros work on.  As discussed in
% item-(\ref{item:ovv-float-@dblfloatplacement}) of
% \secref{sec:imp-ovv-float}, this feature introduced in 2015 version
% of \LaTeX{} is nullified in \env{paracol} environments and thus the
% setting with |1sp| done by \!\@dblfloatplacement! is overriden by
% \!\pcol@startpage! and \!\pcol@output@clear! when they invoke the macro.
% 
% \item[\Uidx{\!\cl@@ckpt!}]
% is an internal macro having the list of all counters defined by
% \!\newcounter! and the counter \counter{page}.  The original usage of this
% macro is to log the values of all counters into |.aux| by \!\include!, but
% we use it to obatin all counters in \!\pcol@zparacol! and
% \!\pcol@globalcounter@s!.
% 
% \item[$\cs{cl@}{\cdot}\theta$]
% is an internal macro having the list $\clist(\theta)$ of descendant
% counters of the counter $\theta$ whose increment by \!\stepcounter! lets
% them 0.  The macro \!\pcol@remctrelt! moves it to
% $|\pcol@cl@|{\cdot}\theta$
% 
% \Uidx{\SpecialArrayIndex{\theta}{\cl@}}
% 
% and re\!\def!ines it to have $\!\pcol@stepcounter!\Arg{\theta}$.
% 
% \item[\Uidx{\!\reserved@a!}]
% is an internal macro for temporary use.  Its usages are as follows.
% 
% \begin{itemize}
% \item
% In \!\pcol@Fe!, it is used to keep the cardinality of \!\@freelist! in
% \!\pcol@F@n! to log it by \!\pcol@FF! after \!\pcol@F@n! is let have
% another measurement result.
% 
% \item
% In $\!\pcol@iLogLevel!\<l\>\arg{name}$, it is used to implement
% $\!\let!|\|\arg{name}=|\|\arg{name}{\cdot}l'$ where $l'$ is the roman
% representation of the level $l$.
% 
% \item
% In \!\pcol@setpageno!, it has $\PPP$ so that we update $\PP$ and
% $\pp(\ptop)$ scanning their original contents.
% 
% \item
% In \!\pcol@getcurrpinfo!, it has $\pp(\ptop)$ so that we give its five
% components to \!\pcol@getpinfo! as its first five arguments.
% 
% \item
% In \!\@outputpage!, it has a sequence of \bgpaint{} for both left and
% right \parapag{}es to be shipped out outside \env{paracol} environment.
% 
% \item
% In \!\pcol@outputpage@ev!, it has the expansion result of
% \!\meaning!\!\yoko! to be compared with \!\reserved@b! having
% \!\string!\!\yoko! for examining if \!\yoko! is a primitive of underlying
% \TeX{}.
% 
% \item
% In $\!\pcol@bg@paintregion!\arg{a}\arg{c}$, it is let have
% $a'=a{\cdot}|@|{\cdot}c$ or $a'=a$, and then referred to by
% \!\pcol@bg@paintregion@i! to use $|\pcol@bg@color|{\cdot}a'$, and by
% $\!\pcol@bg@addext!\arg{z}\Arg{s}\Arg{d}$ to use
% $|\pcol@bg@ext@|{\cdot}d{\cdot}|@|{\cdot}a'$.
% 
% \SpecialArrayIndex{a{\cdot}\string\texttt{@}{\cdot}c}{\pcol@bg@color@}
% \SpecialArrayIndex{a}{\pcol@bg@color@}
% \SpecialArrayIndex{d{\cdot}\string\texttt{@}
%     {\cdot}a{\cdot}\string\texttt{@}{\cdot}c}{\pcol@bg@ext@}
% \SpecialArrayIndex{d{\cdot}\string\texttt{@}{\cdot}a}{\pcol@bg@ext@}
% 
% \item
% In \!\pcol@specialoutput!, it is \!\let!-equal to $|\pcol@output@|{\cdot}f$
% corresponding to $\!\outputpenalty!=|\pcol@op@|{\cdot}f$, or
% \!\@specialoutput!.
% 
% \item
% In \!\pcol@output@start!, it is let have a \bgpaint{} macro
% \!\pcol@bg@paintbox! and the definition of a parameter
% \!\pcol@bg@textheight! for it.
% 
% \item
% In \!\pcol@output@switch!, it is \!\let!-equal to \!\@nobreaktrue! or
% \!\@nobreakfalse! according to \CSIndex{if@nobreak} in the leaving column
% to broadcast it to other columns.
% 
% \item
% In \!\pcol@ifempty!$\arg{box}\arg{then}\arg{else}$, it has $\arg{then}$ or
% $\arg{else}$ according to the emptiness of $\arg{box}$.
% 
% \item
% In \!\pcol@clearcolorstack!, it is \!\def!ined to put an uncoloring
% \!\special! by \!\reset@color! for its argument $\celt_i$ in
% \!\pcol@iscancst!.
% 
% \item
% In \!\pcol@restorecst!, it is \!\def!ined to put a coloring \!\special! in
% its argument $\celt_i$ by \!\unvbox! done in \!\pcol@iscancst!.
% 
% \item
% In \!\pcol@addmarginpar!, at first it is made let equal to 0 or $\CL$
% according to $c<\CL$ or not.  Then it is let have $\PPP$ to be scanned to
% find $\pp^m(p)$.
% 
% \item
% In \!\pcol@getmparbottom@i!, it is let have one of
% $\mpb_{\{L,R\}}^{\{l,r\}}$ according to the side margin which the marginal
% note to be added goes to, and then it is referred to by
% \!\pcol@getmparbottom!.
% 
% \item
% In \!\pcol@setmpbelt@i!, it is let have what $\pp^m(p)$ should have after
% the update of a list of marginal notes in it, and then \!\pcol@setmpbelt!
% updates $\pp(p)$ with the new $\pp^m(p)$ in the macro.
% 
% \item
% In $\!\pcol@bias@mpbout!\Arg{y}$ and $\!\pcol@getmparbottom@last!\Arg{y}$,
% it is let have $\!\pcol@bias@mpbout@i!\Arg{y}$ and
% $\!\pcol@getmparbottom@last@i!\Arg{y}$ respectively, so that they are
% invoked in \!\pcol@do@mpb@all@ii!.
% 
% \item
% In \!\pcol@makeflushedpage!, it is let have an invocation of
% \!\pcol@bg@paintbox! for \pwise{} floats in $\pp^i(\ptop)$ together with
% the condition of the \bgpaint.
% 
% \item
% In \!\pcol@output@end!, it is let have the invocation of
% \!\pcol@bg@paintbox!  for \bgpaint{} of \Scfnote{}s with the condition to
% do it and a \!\def!inition of \!\pcol@bg@footnoteheight!.
% 
% \item
% In $\!\pcol@defcomelt!\arg{com}$, it is used to implement
% $\!\let!|\|\arg{com}=|\pcol@com@|{\cdot}\~\arg{com}$.
% 
% \item
% In \!\pcol@setcolumnwidth!, it is made \!\let!-equal to
% \!\pcol@setcolwidth@s! or \!\pcol@setcolwidth@r! according to
% $\!\pcol@columnratioleft!=\!\relax!$ or not.
% 
% \item
% In \!\pcol@setcolwidth@r!, it is used to have the fraction $r_d$ being a
% comma-seperated list element in its argument $\arg{ratio}$ defined by
% \!\columnratio! and scanned by a \!\@for! loop.
% 
% \item
% In $\!\pcol@setcw@scan!\<\Cfrom\>\<\Cto\>\Arg{spec}$, at first it is let
% have $\arg{spec}$, then the result of adding `|,|' as many as
% $\Cto-\Cfrom$ to the tail, and finally each element in the extended
% $\arg{spec}$ in a \!\@for! loop.
% 
% \item
% In $\!\pcol@setcw@getspec@i!\arg{default}\<x'_d\>$, it is let have
% $\<x'_d\>$ from which all space tokens are removed.
% 
% \item
% In \!\pcol@setcw@calcfactors!, it is used as a waste bascket to throw away
% $(\WT-W)/(F\times1\,|pt|)$ calculated by \!\pcol@setcw@calcf!.
% 
% \item
% In $\!\pcol@extract@fil@i!\<n\>|.|\<m{\cdot}\mathit{unit}\>\!\@nil!$, it has
% $\<n\>|.|\<m{\cdot}\mathit{unit}\>$, and is referred to by
% \!\pcol@extract@fil@ii!.
% 
% \item
% In \!\globalcounter!\marg{ctr}, it is used to have $\arg{ctr}$ for the
% \CSIndex{ifx}-comparison with each $\theta\in\CG$.
% 
% \item
% In $\!\pcol@remctrelt!\<\theta\>$, it is used to implement
% $\!\let!|\pcol@cl@|{\cdot}\theta=|\cl@|{\cdot}\theta$,
% 
% \SpecialArrayIndex{\theta}{\pcol@cl@}\SpecialArrayIndex{\theta}{\cl@}
% 
% \item
% In $\!\pcol@removecounter!\<\mathit{\Theta}'\>\Arg{\theta}$, it is used to
% have $\theta$ for the \CSIndex{ifx}-comparison in \!\pcol@iremctrelt!.
% 
% \item
% In $\!\pcol@thectrelt!\<\theta\>$, it is used to implement
% $\!\let!|\pcol@thectr@|{\cdot}\theta=|\the|{\cdot}\theta$,
% 
% \SpecialArrayIndex{\theta}{\pcol@thectr@}\SpecialArrayIndex{\theta}{\the}
% 
% and then is made \!\let!-equal to $|\pcol@thectr@|{\cdot}\theta{\cdot}0$.
% 
% \SpecialArrayIndex{\theta{\cdot}c}{\pcol@thectr@}
% 
% \item
% In $\!\pcol@synccounter!\arg{lst}$, it has $\arg{lst}$.
% 
% \item
% In $\!\pcol@setctrelt!\<\theta\>\<\val_c(\theta)\>$, it is made
% \!\let!-equal to $|\pcol@thectr@|{\cdot}\theta$
% 
% \SpecialArrayIndex{\theta}{\pcol@thectr@}
% 
% or
% $|\pcol@thectr@|{\cdot}\theta{\cdot}c$.
% 
% \SpecialArrayIndex{\theta{\cdot}c}{\pcol@thectr@}
% 
% \item
% In \!\pcol@switchenv!, it is used to save \!\switchcolumn! which is
% redefined in the macro, and then to invoke \!\switchcolumn! with the
% original definition.
% 
% \item
% In \!\pcol@fntext!\marg{text}, it is \!\let!-equal to \!\pcol@fntextother!
% or \!\pcol@fntext~top! according to the footnote $\arg{text}$ is deferred
% or not.
% 
% \item
% \begin{Hfuzz}{0.6pt}%
% In \!\pcol@calcfnctr!$\arg{num}$\!\@nil!, it has the first token of
% $\arg{num}$ for \CSIndex{ifx}-comparison with `|+|' and `|-|'.
% \end{Hfuzz}
% 
% \item
% In $\!\pcol@twosided!|[|T|]|$, it is let have each non-space token in $T$
% given by a \!\@tfor! loop.
% \end{itemize}
% 
% \item[\Uidx{\!\reserved@b!}]
% is an internal macro for temporary use.  It is used in the following
% macros to keep a list shown in paraentheses so that we update the list in
% the scan of list elements. 
% 
% \begin{quote}
% \!\pcol@startpage! (\!\@dbldeferlist!)\\
% \!\pcol@outputcolumns! ($\PP$)\\
% \!\pcol@trynextcolumn! ($\cc_c(\dl)$)\\
% \!\pcol@makefcolumn! ($\cc_c(\dl)$)\\
% \!\pcol@removecounter!$\<\mathit{\Theta}'\>\Arg{\theta}$
% ($\mathit{\Theta}'\in\{\CG,\CC\}$)
% \end{quote}
% 
% In addition, it is used in the following macros.
% 
% \begin{itemize}
% \item
% In \!\pcol@outputpage@ev!, it has the expansion result of
% \!\string!\!\yoko! to be compared with \!\reserved@a! having
% \!\meaning!\!\yoko! for examining if \!\yoko! is a primitive of underlying
% \TeX{}.
% 
% \item
% In \!\pcol@bg@paint@ii!, it has a token in the arguments $K_b$, $K_g$ and
% $K_c$ of the macro scanned by \!\@tfor!.
% 
% \item
% In \!\pcol@output@switch!, it is \!\let!-equal to |\@afterindenttrue| or
% |\@after|\~|indentfalse| according to \CSIndex{if@afterindent} in the
% leaving column to broadcast it to other columns.
% 
% \item
% In \!\pcol@clearcolorstack!, it is \!\def!ined to put an uncoloring
% \!\special! by \!\reset@color! for its argument $\Celt^c$ in
% \!\pcol@scancst!.
% 
% \item
% In \!\pcol@restorecst!, it is \!\def!ined to put a coloring \!\special! in
% its argument $\Celt^c$ by \!\unvcopy! done in \!\pcol@scancst!.
% 
% \item
% In \!\pcol@scancst! and \!\pcol@iscancst!, after the reference for the
% purposes shown in the two items above, it has
% $M=(m\,\Bar\,\mceltpop_{j,m}\in\cstraw,\;j\geq i)$ for $\mceltpop_{i,*}$
% and in the latter is scanned to find $m$ for $\mcelt_{i,m}$ in $M$.
% 
% \item
% In \!\pcol@addmarginpar!, it is made let equal to $\CL$ or $\C$
% according to $c<\CL$ or not.
% 
% \item
% In $\!\pcol@bias@mpbout@i!\Arg{y}\!\@elt!\Arg{t}\Arg{b}\!\@nil!$, it is
% let have $\mpar(t+y,b+y)$, and in $\!\pcol@getmparbottom@last@i!
% \Arg{y}\mpar(t_1,b_1)\cdots\mpar(t_n,b_n)\!\@nil!$ it is let have
% $\mpar(y,y)$ or $\mpar(t_n,b_n)$, so that they are let be a
% $\mpb_{\{L,R\}}^{\{l,r\}}$ by \!\pcol@do@mpb@all@ii!.
% 
% \item
% In $\!\pcol@setcw@getspec@i!\arg{default}\<x'_d\>$, it is let have each
% non-space token in $\<x'_d\>$ to remove space tokens from it.
% 
% \item
% In $\!\pcol@setcw@fill!\<f\>\!\fill!$, it is let have $f$.
% 
% \item
% In $\!\pcol@extract@fil@ii!\arg{unit}\!\@nil!$, it is let have
% $\arg{unit}$.
% 
% \item
% In \!\globalcounter!$\Arg{\theta}$, it has $\cg_i\in\CG$ for
% \CSIndex{ifx}-comparison with $\theta$.
% 
% \item
% In $\!\pcol@iremctrelt!\<\mathit{\Theta}'\>\Arg\theta$, it has $\theta$ for
% \CSIndex{ifx}-comparison with $\theta'$ to be removed from $\CG$ or $\CC$.
% 
% \item
% In \!\pcol@calcfnctr!$\arg{num}$\!\@nil!, it has `|+|' and then `|-|' for
% \CSIndex{ifx}-comparison with the first token of $\arg{num}$.
% 
% \item
% In \!\pcol@backgroundcolor@ii!, it has \!\pcol@backgroundcolor@x! or
% \!\pcol@back~groundcolor@z! according that the region of \bgpaint{} and its
% color is defined or undefined.
% \end{itemize}
% 
% \item[\Uidx{\!\reserved@c!}]
% is an internal macro for temporary use.  It is used in \!\pcol@startpage!
% to save \!\@deferlist! in it and then to restore the list from it, and in
% \!\pcol@iscancst! to have \!\relax! or the macro itself to iterate the
% macro recursively.
% 
% \item[\Uidx{\!\reserved@d!}]
% is an internal macro for temporary use.  It is used in
% \!\pcol@iscancst! as a \!\chardef! register to have 0 if $m$ for
% $\mcelt_{i,m}$ is not in the list
% $M=(n\,\Bar\,\mceltpop_{j,n}\in\cstraw,\;j\geq i)$, or 1 if found.
% 
% \item[\Uidx{\!\@gtempa!}]
% is an internal macro used as a \!\global!ly modifiable scratchpad.  Its
% usages are as follows.
% 
% \begin{itemize}
% \item
% In \!\pcol@ifempty!$\arg{box}\arg{then}\arg{else}$, it has \!\lastpenalty!
% in a \!\vbox! whose value is examined outside the \!\vbox! for the
% emptiness check of $\arg{box}$.
% 
% \item
% In \!\pcol@addmarginpar!, it is given to \!\@xnext! as the target to have
% the second and successive elements of \!\@currlist! which we cannot modify.
% 
% \item
% In \!\pcol@zparacol! and \!\pcol@cmpctrelt!, it has the list of counters
% to be synchronized.
% 
% \item
% In $\!\pcol@setcw@getspec@i!\arg{default}\<x'_d\>$, it is made
% \!\@empty! or \!\relax! according to $x'_d$ has \!\fill! or not.
% 
% \item
% In \!\pcol@storectrelt!, \!\pcol@savectrelt! and \!\pcol@sscounters!, it
% has the new version of $\CC_c$.
% \end{itemize}
% \end{description}
% 
% 
% 
%\iffalse
%<*paracol>
%\fi
% 
% \section{Register Declaration}
% \label{sec:imp-decl}
% 
% \subsection{\cs{count} Registers}
% 
% Here we declare registers and switches.  The first group is for \!\count!
% registers.
% 
% \begin{macro}{\pcol@currcol}
% \changes{v1.2-1}{2013/05/11}
%	{Add initialization to 0 after the declaration.}
% 
% The register \!\pcol@currcol! has the zero-origin ordinal $c$ of the
% column which we were in when \!\output! is invoked.  Therefore, for
% example, in the process of \!\switchcolumn!, the register has $c$ from
% which we are switching to another column.  The register is initialized to
% be 0 by \!\pcol@output@start!, and then set to $\!\pcol@nextcol!=d$ by
% \!\pcol@restartcolumn! to switch to (or stay in) $d$.  Note that these two
% assignments are \!\global! while other macros may {\em locally} use the
% register to, for example, scan all columns $c\In0\C$.  Besides two macros
% above, the following macros refer to the register to know which column we
% are in (or which column is processed by their invokers).
% 
% \begin{quote}\raggedright
% \!\pcol@Log@iii!,
% \!\pcol@Log@ii!,
% \!\pcol@FF!,
% \!\pcol@makecol!,
% \!\pcol@opcol!,
% \!\pcol@bg@columnleft!,
% \!\pcol@bg@columnwidth!,
% \!\pcol@bg@columnsep!,
% \!\pcol@output@switch!,
% \!\pcol@getcurrcol!,
% \!\pcol@setcurrcol!,
% \!\pcol@clearcolorstack!,
% \!\pcol@restorecolorstack!,
% \!\pcol@addmarginpar!,
% \!\pcol@getmparbottom@i!,
% \!\pcol@setmpbelt@i!,
% \!\pcol@invokeoutput!\footnote{
% 
% Only for logging.},
% 
% \!\thecolumn!,
% \!\pcol@sscounters!,
% \!\pcol@setctrelt!,
% \!\pcol@com@switchcolumn!,
% \!\pcol@switchcol!,
% \!\pcol@visitallcols!,
% \!\pcol@aconlyelt!,
% \!\pcol@flushclear!.
% \!\pcol@columncolor!,
% \!\normalcolumncolor!,
% \!\pcol@icolumncolor!,
% \end{quote}
% 
% Among the macros above, \!\columncolor! and \!\normalcolumncolor! could
% refer to the register outside \env{paracol} environment and thus before
% the initialization by \!\pcol@output@start!.  Therefore, the register is
% also initialized to be 0 after its declaration to assure safe reference.
% 
% The following macros use the register for the scan of all $c\In0\C$ by
% themselves or their invokers.
% 
% \begin{quote}\raggedright
% \!\pcol@output@start!,
% \!\pcol@output@switch!,
% \!\pcol@sync!,
% \!\pcol@flushcolumn!,
% \!\pcol@measurecolumn!,
% \!\pcol@synccolumn!,
% \!\pcol@makeflushedpage!,
% \!\pcol@flushfloats!,
% \!\pcol@freshpage!,
% \!\pcol@output@end!,
% \!\pcol@synccounter!,
% \!\pcol@com@syncallcounters!,
% \!\pcol@stepcounter!.
% \end{quote}
% 
% The macros \!\pcol@imakeflushedpage! and \!\pcol@iflushfloats! also use
% the register for scanning but for $\LBRP\Cfrom\Cto$ given by their
% arguments.
% 
% In addition \!\pcol@ccuse!, \!\pcol@ifccdefined! and \!\pcol@ccxdef!
% refer to the register to have the control sequence
% $|\pcol@columncolor|\cdot c=\Celtshadow^c$ or
% $|\pcol@columncolor@box|\cdot c=\Celt^c$ where $c$ is for the current
% column or for all columns depending on their invokers.
% 
% \SpecialArrayIndex{c}{\pcol@columncolor}
% \SpecialArrayIndex{c}{\pcol@columncolor@box}
% \end{macro}
% 
% \begin{macro}{\pcol@nextcol}
% The register \!\pcol@nextcol! has the zero-origin ordinal $d$ of the
% column to which we are switching, or in which we are staying, to be set
% into \!\pcol@currcol! by \!\pcol@restartcolumn!.  The main usage of the
% register is to set the switching target in \!\pcol@switchcolumn!, but
% other macros use it to specify the (temporary) target of
% \!\pcol@switchcol!; the tallest column in \!\pcol@sync!; 0 in
% \!\pcol@zparacol!, \!\pcol@sptext! and \!\endparacol!; all in $\LBRP0\C$
% but $c=\!\pcol@currcol!$ in \!\pcol@visitallcols!; and $c$ in
% \!\pcol@flushclear! to stay in the current column $c$.  The other user of
% this register is \!\pcol@invokeoutput! but only for logging.
% \end{macro}
% 
% \begin{macro}{\pcol@ncol}
% \changes{v1.3-3}{2013/09/17}
%	{Add initial zero-clearing for safe reference in \cs{@outputpage}
%	 invokded prior to the first \string\texttt{paracol}.}
% \begin{macro}{\pcol@ncolleft}
% \changes{v1.3-2}{2013/09/17}
%	{Introduced to specify the number of columns in left
%	 parallel-pages.}
% 
% The register \!\pcol@ncol! has the number of columns $\C$ given as the
% argument of \!\paracol!, whose callee \!\pcol@zparacol! being the sole
% modifier of the register \!\global!ly assigns $C$ to the register to give
% safe reference to \!\@outputbox! invoked after a \env{paracol} is closed.
% In addition for the reference in \!\@outputbox! before the first
% \env{paracol}, the register is initialized with zero after the declaration.
% 
% The following macros refer to the register to scan all columns $c\In0\C$.
% 
% \begin{quote}\raggedright
% \!\pcol@checkshipped!,
% \!\pcol@output@start!,
% \!\pcol@output@switch!,
% \!\pcol@sync!,
% \!\pcol@makeflushedpage!,
% \!\pcol@freshpage!,
% \!\pcol@output@end!,
% \!\pcol@synccounter!,
% \!\pcol@com@syncallcounters!,
% \!\pcol@stepcounter!,
% \!\pcol@visitallcols!.
% \end{quote}
% 
% The register \!\pcol@ncolleft! has $\Uidx\CL$ being the number of columns
% in the left {\em\Uidx\parapag{}e} if \parapag{}ing is in effect, or have
% $\C$ otherwise.  Similar to $\C$, the number $\CL$ is given as the
% optional argument of \!\pcol@zparacol! and is \!\global!ly assigned to the
% register by the sole modifier \!\pcol@zparacol!, unless the optional
% arugment is not less than $\C$ which is assigned to $\CL$ if this
% exception is found.  The reason of the \!\global! assignment and the
% initial zero-clearing after the declaration is same as that for $C$, i.e.,
% for the reference in \!\@outputpage! outside \env{paracol}.
% 
% The following macros examines if $\Uidx\CL<\C$, i.e., if \parapag{}ing is
% in effect.
% 
% \begin{quote}\raggedright
% \!\pcol@outputelt!,
% \!\@outputpage!,
% \!\pcol@output@start!,
% \!\pcol@output@flush!,
% \!\pcol@output@clear!,
% \!\pcol@makeflushedpage!,
% \!\pcol@flushfloats!,
% \!\pcol@output@end!,
% \!\pcol@zparacol!,
% \end{quote}
% 
% \begin{Sloppy}{2350}
% In the macros listed above, \!\pcol@outputelt!, \!\pcol@makeflushedpage!
% and \!\pcol@flushfloats! passes $\LBRP0\CL$ and $\LBRP\CL\C$ to their
% respective callees \!\pcol@ioutputelt!, \!\pcol@imakeflushedpage!  and
% \!\pcol@iflushfloats! as their argument pair
% $\LBRP{\Uidx\Cfrom}{\Uidx\Cto}$ to let them work on the left and right
% \parapag{}es repsectively.  The callees above also pass the pair to
% \!\pcol@swapcolumn! to swap the scanning order of columns if \cswap{} is
% in efect.
% 
% They also pass the pair to \!\pcol@buildcolseprule! which then passes it
% to \!\pcol@bg@paintcolumns!  and \!\pcol@bg@paintbox! by binding it to
% $\LBRP\CBfrom\CBto=\LBRP{\!\pcol@bg@from!}{\~\!\pcol@bg@to!}$ referred to
% by \!\pcol@bg@paint@i! and its callee \!\pcol@bg@paint@ii! to define the
% range of columns to be painted is $\LBRP\CBfrom\CBto$.  Similar passing is
% done by (our own version of) \!\@outputpage!, but it directly uses
% $\LBRP0\CL$ and $\LBRP\CL\C$ as the sources and the target painting macros
% are \!\pcol@bg@paintpage!, \!\pcol@bg@@paintpage! and
% \!\pcol@bg@paintbox!.  Note that $\CBto$ is initialized to be
% \!\let!-equal to $\C$ because it may be referred to without
% binding\footnote{
% 
% This meaningless reference has no harmful side effects.}.
% \end{Sloppy}
% 
% The macro \!\pcol@addmarginpar! also referes to $\CL$ to know if the
% column $c$ on which it is working on is in the left or right \parapag{}e,
% i.e., $c<\CL$ or not, to decide the margin to which a marginal note is put,
% and to pass $\LBRP0\CL$ or $\LBRP\CL\C$ to \!\pcol@swapcolumn! to
% calculate the distance to the left or right margin from the column.  The
% examination of $c<\CL$ is also done in related macros
% \!\pcol@getmparbottom@i! and \!\pcol@setmpbelt@i!.
% 
% Similar column-range specification is done for the argument pair
% $\LBRP\Cfrom\Cto$ of \!\pcol@setcolumnwidth!  invoked from
% \!\pcol@zparacol!.  Then the arguments are passed to the callees
% \!\pcol@setcolwidth@r! or \!\pcol@setcolwidth@s!, the latter of
% which also passes them to its callee \!\pcol@setcw@scan!, to define the
% width of columns in $\LBRP\Cfrom\Cto$ and thier separators.
% 
% The other references to $\C$ are made by \!\pcol@com@switchcolumn! and
% \!\pcol@switchcolumn! to examine $c<\C$, to wraparound $\C-1$ to 0 for the
% former and to complain if $c\geq\C$ for the latter.
% \end{macro}\end{macro}
% 
% \begin{macro}{\pcol@page}
% The register \!\pcol@page! has the zero-orgin ordinal $p$ of the page
% which we are in.  The register is initialized to be 0 not only by
% \!\pcol@output@start! to give the obvious starting point, but also by
% \!\pcol@freshpage! for page flushing which clears $\PP=\!\pcol@pages!$ to
% give us another type of starting point.  Then the register is incremented
% by \!\pcol@nextpage! to advance one page, by \!\pcol@nextpelt! to skip a
% \fpage, and by \!\pcol@startpage! for a \fpage{} to be created.  The other
% type of updates of the register is done by \!\pcol@restartcolumn! which
% lets $p$ be $\cc_c(\vb^p)$ when we revisit the column $c$ belonging to the
% page $p$.  Note that, besides these \!\global!  updates,
% \!\pcol@flushcolumn! locally updates the register to scan
% $\PP=\!\pcol@pages!$, and \!\pcol@freshpage! also performs local updates
% but in more weird manner.  Besides the updates discussed above, the
% macros \!\pcol@Log@iii!, \!\pcol@Log@ii!, \!\pcol@FF!, \!\pcol@makecol!,
% \!\pcol@opcol!, \!\pcol@setpageno!, \!\pcol@getcurrpage!,
% \!\pcol@startcolumn!, \!\pcol@output@switch!, \!\pcol@addmarginpar! and
% \!\pcol@fntext! refer to the register to know which page they are
% operating on.
% \end{macro}
% 
% \begin{macro}{\pcol@basepage}
% The register \!\pcol@basepage! has the ordinal $\pbase$ of the \bpage{}
% being the oldest page not shipped out yet.  The register is initialized to
% be 0 by \!\pcol@output@start! and \!\pcol@freshpage! together with
% \!\pcol@page!, and then incremented by \!\pcol@outputelt! when it ships
% the page $\pbase$ out.  The macros \!\pcol@setpageno!, \!\pcol@nextpage!,
% \!\pcol@getcurrpage! and \!\pcol@addmarginpar! refer to the register in
% their scans of $\PP$ or $\PPP$ to know the zero-origin ordinal of the
% element for the current page $p$ is $p-\pbase$.
% \end{macro}
% 
% \begin{macro}{\pcol@toppage}
% \changes{v1.0}{2011/10/10}
%	{Renamed from \cs{pcol@maxpage}.}
% 
% The register \!\pcol@toppage! has the ordinal $\ptop$ of the \tpage{}
% having the most advanced \colpage{}s, or {\em\Uidx\lcolpage{}s} in short.
% The register is initialized to be 0 by \!\pcol@output@start! and
% \!\pcol@freshpage! together with \!\pcol@page!, and then let be
% $p=\!\pcol@page!$ by \!\pcol@startpage! to start a new page $p$.  The
% macros \!\pcol@makecol!, \!\pcol@opcol!, \!\pcol@startcolumn!,
% \!\pcol@output@switch! and \!\pcol@restartcolumn!  refer to the register
% to examine if they are working on a \colpage{} in the \tpage, while
% \!\pcol@flushcolumn! and \!\pcol@fntext! examines if the \ccolpage{} is
% behind the \tpage.  The register is also referred to by \!\pcol@Log@iii!,
% \!\pcol@Log@ii! and \!\pcol@FF! for logging.
% \end{macro}
% 
% \begin{macro}{\pcol@footnotebase}
% \changes{v1.2-2}{2013/05/11}
%	{Introduced for page-wise footnotes.}
% 
% The register \!\pcol@footnotebase! is let have the value of \!\c@footnote!
% at the start of a \env{paracol} environment by \!\pcol@zparacol! to give
% the base value $\Uidx\bf$ for relative numbering of \counter{footnote}
% done in \!\pcol@calcfnctr! for the starred versions of \!\footnote!,
% \!\footnotemark! and \!\footnotetext!.  The register is also referred to
% by \!\endparacol!  to let \!\c@footnote! have $\bf+\nf$ where
% $\nf=\!\pcol@nfootnotes!$ shown below is the number of footnotes in the
% \env{paracol} environment to be closed.
% \end{macro}
% 
% \begin{macro}{\pcol@nfootnotes}
% \changes{v1.2-2}{2013/05/11}
%	{Introduced for page-wise footnotes.}
% 
% The register \!\pcol@nfootnotes! is to accumulate the number of footnotes
% $\Uidx\nf$ in a \env{paracol} environment.  Therefore, it is zero-cleared
% by \!\pcol@zparacol!, then incremented by \!\pcol@ifootnote! and
% \!\pcol@ifootnotemark! for \!\footnote! and \!\footnotemark!, and finally
% referred to by \!\endparacol! to let $\!\c@footnote!=\bf+\nf$.
% \end{macro}
% 
% \begin{macro}{\pcol@mcid}
% \changes{v1.24}{2013/07/27}
%	{Introduced for coloring specified in math mode.}
% \changes{v1.34}{2018/05/07}
% 	{Change its meaning and operations with it a little bit according to
%	 the new text coloring with \cs{insert}.}
% 
% The register \!\pcol@mcid! has the number of pushes of
% \colorstack{} by coloring commands in math mode between two consecutive
% invocations of \!\output!.  The register is
% zero-cleared by \!\pcol@output! because we are definitely in the main
% vertical mode and thus all pops corresponding to pushes in math mode must
% have been applied to |.tex|'s \colorstack.  Then the register
% is referred to by \!\pcol@set@color@push! when it is invoked in math mode,
% to increment it and then examine if it does not exceed the limit
% \!\pcol@mcpushlimit! to mean the math-mode coloring still can be made.  The
% macro then uses the value of the register as the identifier of the push
% operation given to \!\output! through an \!\insert!ion.
% \end{macro}
%    \begin{macrocode}

%% Register Declaration

\newcount\pcol@currcol \global\pcol@currcol\z@
\newcount\pcol@nextcol
\newcount\pcol@ncol \global\pcol@ncol\z@
\newcount\pcol@ncolleft \global\pcol@ncolleft\z@
\newcount\pcol@page
\newcount\pcol@basepage
\newcount\pcol@toppage
\newcount\pcol@footnotebase
\newcount\pcol@nfootnotes
\newcount\pcol@mcid
%    \end{macrocode}
% 
% 
% 
% \subsection{Switches}
% 
% The second declaration group is for switches.
% 
% \begin{macro}{\ifpcol@output}
% \changes{v1.2-7}{2013/05/11}
% 	{Introduced to solve the \cs{output} request sneaking.}
% \changes{v1.22}{2013/06/30}
%	{Add a user \cs{pcol@reset@color@pop} to inhibit uncoloring if
%	 $\mathit{false}$.}
% \changes{v1.3-2}{2013/09/17}
%	{Add a user \cs{@outputpage} for parallel-paging.}
% \changes{v1.3-3}{2013/09/17}
%	{Add a user \cs{@outputpage} for background-painting.}
% 
% The switch \CSIndex{ifpcol@output} is $\true$ iff \!\pcol@output@start!
% which turns the switch $\true$ has been invoked but \!\pcol@output@end!
% which does $\false$ has not yet.  Then the switch is examined by
% \!\pcol@output! to detect an \!\output! request sneaked from outside of
% the \env{paracol} environment.  The other users \!\@outputpage! and
% \!\pcol@reset@color@pop! examine this switch to know if the macro is
% invoked inside or outside of \env{paracol} environment, while the macro
% \!\pcol@output@start!  temporarily turns the switch $\false$ when it ships
% out a page having \preenv{} only.
% \end{macro}
% 
% \begin{macro}{\ifpcol@nospan}
% \changes{v1.0}{2011/10/10}
%	{Renamed from \cs{pcol@textonly}.}
% 
% The switch |\ifpcol@nospan| is $\true$ iff a page $p$ does not have
% \spanning, i.e., $\pp^i(p)=\bot$.  It is set by by
% \!\pcol@getpinfo! for the examination in \!\pcol@ioutputelt!,
% \!\pcol@makeflushedpage! and \!\pcol@output@end!.
% \end{macro}
% 
% \begin{macro}{\ifpcol@sync}
% \changes{v1.0}{2011/10/10}
%	{Add initialization to be $\string\mathit{false}$.}
% \changes{v1.2-2}{2013/05/11}
%	{Add \cs{pcol@switchcol} and \cs{pcol@flushclear} to the macros
%	 turning the switch $\string\mathit{true}$ due to column scanning
%	 and pre-flushing column height check.}
% 
% The switch |\ifpcol@sync| is $\true$ iff \!\pcol@output@switch! is invoked
% for \sync{}ed \cswitch{} by \!\switchcolumn!|*| or its relative
% enviroment openers, or \pfcheck{} prior to page flushing or environment
% closing.  Therefore, the switch is \!\global!ly turned $\true$ by
% \!\pcol@iswitchcolumn! and \!\pcol@sptext! for the \sync{}ing
% \cswitch{}ing but then temporarily turned $\false$ by \!\pcol@switchcol!
% invoked by them for \cscan{}ning and then turned $\true$ again by the
% macro.  For \pfcheck{}, the macro \!\pcol@flushclear! turns the switch
% $\true$.  The other macro turns this switch is \!\pcol@output@switch! at
% the end of which the switch is turned $\false$ to go back to the default
% state.  The macros examining this switch are \!\pcol@output@switch!,
% \!\pcol@putbackmvl!, \!\pcol@sync!, \!\pcol@invokeoutput! (for logging) and
% \!\pcol@switchcol!.
% \end{macro}
% 
% \begin{macro}{\ifpcol@sptextstart}
% \changes{v1.3-1}{2013/09/17}
%	{Introduced to capture the starting point of a spanning text so that
%	 the text is split from other main vertical list stuff.}
% \begin{macro}{\ifpcol@sptext}
% \changes{v1.0}{2011/10/10}
%	{Introduced to restrict the broadcast of \cs{if@nobreak} and
%	 \cs{everypar} only when a column-switching is accompanied with
%	 spanning text.}
% \changes{v1.3-1}{2013/09/17}
%	{Renamed from \cs{ifpcol@mctext} following the naming convention, and
%	 move the timing of turning $\string\mathit{true}$ from the end
%	 of a spanning text to its beginning.}
% 
% The switch |\ifpcol@sptextstart| is $\true$ iff \!\pcol@output@switch! is
% invoked from \!\pcol@sptext! prior to a \mctext.  That is, the switch is
% \!\global!ly\footnote{
% 
% Not necessary to be \cs{global}ly turned but we dare to do that to clearly
% distinguish that from the local turning in \cs{pcol@putbackmvl}.}
% 
% let $\true$ and then $\false$ by \!\pcol@sptext! before and
% after the invocation of \sync{}ed \!\pcol@switchcol! prior to the
% \mctext.  Then the switch is examined by \!\pcol@putbackmvl! after
% the \sync{}ation to save the {\em\Uidx\prespan}, being all stuff in main
% vertical list prior to the \sync{}ation, so that the \mctext{} is split
% from the stuff and is captured afterward by \!\pcol@makecol! and/or
% \!\pcol@output@switch!.  The macro also locally turns the switch
% $\false$ if it does not follow the \sync{}ation, i.e., its invocation is
% for \cscan{}ning or is caused by pre-\sync{}ation page break, to do the
% saving only when it follows the \sync{}ation.  The switch is also examined
% in \!\pcol@output! to inhibit the warning and forced page break even when
% $\cc_0(\vb^r)=\!\@colroom!<1.5\!\baselineskip!$, because we may let it have
% a small value when the \mctext{} starts near the page bottom to capture
% the text portion in the page by \!\pcol@makecol!.  In addition, it is
% examined by \!\pcol@switchcol! to invoke $|\pcol@colpream|{\cdot}c$,
% where $c=-1$ if $\true$ or $c=\!\pcol@currcol!$ otherwise.
% 
% \SpecialArrayIndex{c}{\pcol@colpream}
% 
% The macro \!\pcol@sptext! then \!\global!ly turns another switch
% \CSIndex{ifpcol@sptext} $\true$ before putting the \mctext{} into the main
% vertical list so that \!\pcol@makecol! for the page break in the text and
% \!\pcol@output@switch! for closing capture the text to place it
% appropriately especially when \cswap{} is in effect.  Then the switch 
% is \!\global!ly turned $\false$ by \!\pcol@output@switch! to give the
% default state after it {\em broadcasts} \CSIndex{if@nobreak},
% \CSIndex{if@afterindent} and \!\everypar! to all columns.
% \end{macro}\end{macro}
% 
% \begin{macro}{\ifpcol@clear}
% \changes{v1.0}{2011/10/10}
%	{Add initialization to be $\string\mathit{false}$.}
% \changes{v1.2-2}{2013/05/11}
%	{Add \cs{pcol@flushclear} to the macros turning the switch
%	 $\string\mathit{true}$ due to the pre-flushing column height
%	 check.}
% 
% The switch |\ifpcol@clear| is $\true$ iff \!\pcol@output@switch! is
% invoked for \pfcheck{}, page flushing or environment closing.  Therefore,
% the switch is turned $\true$ by \!\pcol@flushclear! in the first case, and
% by \!\pcol@makeflushedpage! in the latter two.  These two macros also
% turned the switch $\false$ after the direct\slash indirect invocation of
% \!\pcol@output@switch! to give the default state.  The switch is examined
% by \!\pcol@output@switch! and its descendants \!\pcol@sync!,
% \!\pcol@flushcolumn! and \!\pcol@synccolumn!  for \sync{}ation, and by
% \!\pcol@invokeoutput!  for logging.
% \end{macro}
% 
% \begin{macro}{\ifpcol@flush}
% \changes{v1.2-3}{2013/05/11}
%	{Introduced to fix the problem with a too-tall page at
%	 synchronization.}
% \changes{v1.2-2}{2013/05/11}
%	{Add uses for page-wise footnote functions.}
% 
% The switch |\ifpcol@flush| is turned $\true$ by \!\pcol@sync! iff it finds
% that the page to be \sync{}ed or to be flushed is too tall because the
% sum of the total height of top floats and main text in a column and that
% of bottom floats and footnotes in another column is larger than
% $\pp^h(\ptop)-v^f-\VE$, where $v^f$ is the total heigth-plus-depth of the
% \Scfnote{}s if $\ptop$ has them or 0 otherwise, and $\VE$ is the amount
% given by \!\ensurevspace! in \sync{}ation or 0 in flushing.  Then the
% switch is examined in \!\pcol@sync! itself to restart the tallest column
% if $\true$, in \!\pcol@putbackmvl! to check if a \mctext{} is really to
% start, in \!\pcol@switchcol! to have a explicit page break in each column
% if $\true$, and in \!\pcol@flushclear! also to have a page break if
% $\true$.  The last examiner \!\pcol@flushclear!  may turn the switch
% $\true$ when it is invoked from \!\endparacol! if the last page leaves
% deferred and non-merged \Scfnote{}s for which an explicit page break is
% also required.
% \end{macro}
% 
% \begin{macro}{\ifpcol@outputflt}
% \changes{v1.0}{2011/10/10}
%	{Renamed from \cs{ifpcol@stopoutput} with the reversal.}
% 
% The switch |\ifpcol@outputflt| is used in \!\pcol@outputelt! to know
% whether a \fpage{} is to be shipped out ($\true$) or not.  The switch is
% initialized to be $\true$ by \!\pcol@outputcolumns! which invokes
% \!\pcol@outputelt! for all $\pp(p)\in\PP$.  Then if \!\pcol@outputcolumns! is
% invoked from \!\pcol@opcol! to ship the oldest page $\pbase$ out, the
% switch is turned $\false$ when we visit the second non-\fpage.  That is,
% all \fpage{}s following the first (oldest) page are shipped out but others
% are not.  On the other hand, if \!\pcol@outputcolumns! is invoked from
% \!\pcol@sync! to ship out all pages including \fpage{}s in $\PP$, the
% switch is kept $\true$ throughout all invocations of \!\pcol@outputelt!.
% \end{macro}
% 
% \begin{macro}{\ifpcol@lastpage}
% \changes{v1.0}{2011/10/10}
%	{Introduced for special operations in the last page.}
% \begin{macro}{\ifpcol@lastpagesave}
% \changes{v1.2-7}{2013/05/11}
%	{Introduced to fix the bug that \cs{@makecol} and
%	 \cs{pcol@makefcolumn} in \cs{pcol@flushcolumn} misunderstand that
%	 non-last pages are last.}
% 
% The switch |\ifpcol@lastpage| is used to know whether the following macros
% work on the {\em\Uidx\lpage} of a \env{paracol} environment to do special
% operations if so.
% 
% \begin{itemize}
% \item
% \!\pcol@combinefloats! adds \!\textfloatsep! below bottom floats of each
% column if any so that the floats are well sepearated from the \postenv.
% 
% \item
% \!\pcol@sync! examines $\VPP$ instead of $\VP$ for the \pfcheck.
% 
% \item
% \!\pcol@makefcolumn! trys to make deferred floats as top floats.
% 
% \item
% \!\pcol@makeflushedpage! builds a short page of $\VPP$ tall, leaves
% \spanning{} from ship-out if $\VPP=\!\pcol@colht!=-\infty$ so that it
% becomes a float in \postenv, and leaves \Scfnote{}s untouched if
% \Mgfnote{} typesetting is in effect.  This macro itself turns this switch
% $\false$ if $\CSIndex{ifpcol@dfloats}=\true$ to mean one or more columns
% in the \lpage{} have deferred \cwise{} floats and thus the \lpage{}
% must be {\em full size}.
% 
% \item
% \!\pcol@imakeflushedpage! leaves the \bground{} of \Scfnote{}s unpainted,
% and lets the depth of the \lpage{} be 0 in \bgpaint{} and packing of
% \colpage{}s.
% \end{itemize}
% 
% The switch is initialized to be $\false$ by \!\pcol@zparacol!, then turned
% $\true$ by \!\endparacol!, and then finally turned $\false$ by
% \!\pcol@output@end! again for
% \fpage{}s following the \lpage{} if any.  The macro \!\pcol@flushcolumn!
% saves the switch into \CSIndex{ifpcol@lastpagesave} then turning it
% $\false$ during it works on \colpage{}s in non-top and thus non-\lpage{}s
% to keep \!\@makecol! and \!\pcol@makefcolumn! from misunderstanding the
% pages are last, and then restore the switch when the macro reaches to the
% \tpage.  This saving and temporary turning $\false$ is also done in
% \!\pcol@flushclear! when it forces a page break so that the \!\output!
% routine working on the broken non-\lpage{} correctly recognizes that.
% Another temporary turning is made by \!\pcol@makenormalcol! but in reverse
% to let the switch be $\true$ so that its indirect callee
% \!\pcol@combinefloats!  puts a vertical skip of \!\textfloatsep! below the
% bottom floats in \preenv.
% \end{macro}\end{macro}
% 
% \begin{macro}{\ifpcol@scfnote}
% \changes{v1.2-2}{2013/05/11}
%	{Introduced for page-wise footnote functions.}
% 
% The switch \!\ifpcol@scfnote! is turned $\true$ by \!\pcol@fnlayout@p!
% and \!\pcol@fnlayout@m! (through \!\pcol@fnlayout@p!) to indicate
% footnotes in all columns are merged and \scfnote, while
% \!\pcol@fnlayout@c! turns it $\false$ to make footnote typesetting
% \mcfnote{}, being default as well.  The switch is examined by the
% following macros to do special operations for \Scfnote{}s if
% it is $\true$.
% 
% \begin{itemize}
% \item
% \!\pcol@makecol! shrinks \!\@colht! and put the stretch\slash shrink
% factor of \!\skip!\!\footins! at the bottom of the \colpage{} to be built
% by the macro if the page has footnotes.  If the \colpage{} is in the
% \tpage{} $\ptop$, the macro also saves \!\footins! into \!\pcol@currfoot!
% which then will be saved into $\pp^f(\ptop)$ by \!\pcol@startpage!, or
% \!\footins! is discarded otherwise.
% 
% \item
% \!\pcol@startcolumn!, if it is invoked from \!\pcol@output! to start a
% \colpage{} in a page $p$, \!\insert!s $\pp^f(p)$ through \!\footins!, and
% also the deferred footnotes in $\df$ by \!\pcol@deferredfootins!, if
% $p=\ptop$.
% 
% \item
% \!\pcol@output@switch! saves \!\footins! into $\pp^f(\ptop)$ if the macro
% is to leave a \colpage{} in the \tpage, or discards it otherwise.
% 
% \item
% \!\pcol@restartcolumn! to restart a \colpage{} in a page $p$ \!\insert!s
% $\pp^f(p)$ through \!\footins!, and also the deferred footnotes in $\df$
% by \!\pcol@deferredfootins!, if $p=\ptop$.
% 
% \item
% \!\pcol@sync! examines whether the total height of \Scfnote{}s is too
% large to let them reside in the page to be synchronized or flushed as a
% whole.
% 
% \item
% \!\pcol@zparacol! redefines \!\footnoterule! so that it refers to
% \!\textwidth!  rather than \!\column~width! to determine the width of the
% rule above footnotes.
% 
% \item
% \!\pcol@fntext! invokes \!\pcol@fntextother! to add the footnote given to
% it to $\df$ as a deferred footnote if $p<\ptop$.
% 
% \item
% \!\pcol@fntextbody! lets \!\hsize! be \!\textwidth! rather than
% \!\columnwidth! to typeset the footnote given to it.
% \end{itemize}
% \end{macro}
% 
% \begin{macro}{\ifpcol@mgfnote}
% \changes{v1.2-2}{2013/05/11}
%	{Introduced for page-wise footnote functions.}
% 
% The switch \!\ifpcol@mgfnote! is turned $\true$ by \!\pcol@fnlayout@m!
% to indicate footnotes in the \spage{} and \lpage{} of a \env{paracol}
% environment are merged with those in pre- and \postenv{}, while
% \!\pcol@fnlayout@p! and \!\pcol@fnlayout@c! turn it $\false$ to
% put them above\slash below the columns in the starting\slash\lpage{}
% respectively, being default as well.  The switch is examined by the
% following macros to do special operations for \mgfnote{} \Scfnote{}s
% if it is $\true$.
% 
% \Index{pre-environment stuff}
% \Index{starting page}
% 
% \begin{itemize}
% \item
% \!\pcol@makenormalcol! leaves \!\footins! untouched rather than putting it
% as a part of \preenv.
% 
% \item
% \!\pcol@makeflushedpage! leaves \!\footins! untouched rather than putting it
% as a part of the \lpage{} of \env{paracol} if it works on the page.
% 
% \item
% \!\endparacol! does not let \!\pcol@flushclear! examine the existence of
% deferred footnotes in \pfcheck{} for the \lpage.
% \end{itemize}
% \end{macro}
% 
% \begingroup\let\small\footnotesize
% \begin{macro}{\ifpcol@fncounteradjustment}
% \changes{v1.2-2}{2013/05/11}
%	{Introduced for page-wise footnote functions.}
% 
% The switch \!\ifpcol@fncounteradjustment! is turned $\true$ by the API
% macro \!\fncounter~adjustment!, which is also invoked from
% \!\pcol@fnlayout@p! and \!\pcol@fnlayout@m! (through \!\pcol@fnlayout@p!),
% to let $\!\c@footnote!=\bf+\nf$ by \!\endparacol!.  The macro
% \!\nofncounter~adjustment! turns the switch $\false$ to give the default
% state.
% \end{macro}
% \endgroup
% 
% \begin{macro}{\ifpcol@inner}
% \changes{v1.22}{2013/06/30}
%	{Introduced to know if we are in a \cs{vbox}.}
% 
% The switch \cs{ifpcol@inner} is turned $\false$ by \!\pcol@zparacol! to
% mean we are outside any \!\vbox!es, while the macro also lets
% \!\everyvbox! in the \env{paracol} environment has the operation to turn
% the switch $\true$ so that it is true whenever we are in a \!\vbox!.  The
% switch is examined by \!\pcol@set@color@push! and \!\pcol@icolumncolor!,
% the former of which also turns it $\true$ if we are in restricted
% horizontal mode, to make an \!\output! request for \colorstack{}
% manipuratoin and, in the former, to reserve the stack popper by
% \!\aftergroup!, iff the switch is $\false$.
% \end{macro}
% 
% \begin{macro}{\ifpcol@firstpage}
% \changes{v1.3-3}{2013/09/17}
%	{Introduced to know if a spanning stuff is pre-environment one.}
% 
% The switch \cs{ifpcol@firstpage} is \!\global!ly turned $\true$ or
% $\false$ by \!\pcol@output@start! if it captures \preenv{} as a
% \spanning{} or not because it is too large, respectively.  Then the switch
% is examined by \!\pcol@ioutputelt! or \!\pcol@makeflushedpage! when they
% finds a \spanning{} to be combined to the page to be printed so as to
% paint the \bground{} for the color of \pwise{} floats unless the switch is
% $\true$ to mean the stuff is pre-environment one rather than floats.  Then
% \!\pcol@outputelt! or \!\pcol@makeflushedpage! itself \!\global!ly turns
% the switch $\false$ after printing a page because we no longer have
% \preenv{} in the \env{paracol} environment we are in.
% \end{macro}
% 
% \begin{macro}{\ifpcol@havelastpage}
% \changes{v1.3-3}{2013/09/17}
%	{Introduced to know if a page to be put has the last page of a
%	 \string\texttt{paracol} environment.}
% 
% The switch \cs{ifpcol@havelastpage} is, after intiailized $\false$,
% \!\global!ly turned $\true$ by \!\pcol@output@end! if it finds the
% \lpage{} of the \env{paracol} environment is connected to the \postenv, or
% $\false$ otherwise.  Then the switch is examined by (our own version of)
% \!\@outputpage! which paints the \bground{} of the page to be printed iff
% the switch is $\true$ because a part of the page was prodcued by a
% \env{paracol} environment.  Then the macro \!\global!ly turns the switch
% $\false$ because so far \bgpaint{} should be disabled.
% \end{macro}
% 
% \begin{macro}{\ifpcol@paired}
% \changes{v1.3-2}{2013/09/17}
%	{Introduced for parallel-paging which has paired and non-paired mode.}
% 
% The switch \cs{ifpcol@paired} is $\true$ if the \parapag{}ed typesetting
% should be done in {\em\paired} mode in which the pair of left and right
% \parapag{}es comprises a virtual page, while it is $false$ if
% \npaired{} to treat the left and right pages as individual ones.
% Therefore, the switch is \!\global!ly turned $\false$ by \!\pcol@yparacol!
% when \beginparacol{} has the optional argument for the number of columns
% in the left \parapag{}e followed by |*|, or turned $\true$
% otherwise by \!\paracol! for giving default or by \!\pcol@zparacol! if it
% finds $\CL\geq\C$ to mean \parapag{}ing is not in effect
% in reality\footnote{
% 
% The initialization to let the switch $\false$ is not necessary because it
% is examined only after the first \cs{paracol} even in the
% \cs{@outputpage} outside \texttt{paracol} environment, but we dare to do
% this for the sake of clearity.}.
% 
% Then the switch is examined by \!\pcol@setpnoelt! and \!\pcol@startpage!
% so that, if the switch is $\false$, they let $\page(q)=\page(q-1)+2$ where
% $\cc_0(\vb^p)<q\leq\ptop$ in the former and $q=\ptop$ in the latter,
% instead of $\page(q)=\page(q-1)+1$ in the usual $\true$ case, because the
% left\slash right pair of \parapag{}es is treated as two pages rather than
% one.
% 
% The other macros \!\pcol@ioutputelt!, (our own version of) \!\@outputpage!,
% \!\pcol@output@start!, \!\pcol@imakeflushedpage!, \!\pcol@iflushfloats! and
% \!\pcol@output@end! also refer to the switch so that, if $\false$, they
% temporarily let $\!\c@page!=\page(p)+1$ in building the shipout image of
% the right component of the \parapag{}e pair of the page $p$ in order to
% have the appropriate page number parity for the right component.  Among
% them, \!\@outputpage! has another mode dependant operation, if the switch
% is $\true$, to decrement \!\c@page! by one before shipping out the right
% component to cancel the increment in the ship-out process of the left
% component.  The macro \!\pcol@addmarginpar! also examines the switch to
% decide the margin for marginal notes in \npaired{} \parapag{}es.  Another
% examiner \!\pcol@zparacol! lets $\CSIndex{ifpcol@swapcolumn}=\false$
% only in the \env{paracol} environment to start if the switch is $\true$
% because \cswap{} is meaningless in \npaired{} \parapag{}ing.
% \end{macro}
% 
% \begin{macro}{\ifpcol@swapcolumn}
% \changes{v1.2-4}{2013/05/11}
%	{Introduced for column-swapping in even pages.}
% \begin{macro}{\ifpcol@swapmarginpar}
% \changes{v1.3-4}{2013/09/17}
%	{Introduced for marginal-note-swapping in even pages.}
% \begin{macro}{\ifpcol@bg@swap}
% \changes{v1.3-3}{2013/09/17}
%	{Introduced for mirrored backgournd painting for even numbered
%	 pages.}
% \begin{macro}{\ifpcol@bg@@swap}
% \changes{v1.3-3}{2013/09/17}
%	{Introduced for mirrored backgournd painting for even numbered
%	 pages.} 
% 
% The switch \cs{ifpcol@swapcolumn}, \cs{ifpcol@swapmarginpar} and
% \cs{ifpcol@bg@swap} specify that, if $\true$, columns, marginal notes,
% and \bgpaint{} in even numbered pages are swapped, respectively.
% That is, \cs{ifpcol@swapcolumn} lets columns be put from right to left,
% \cs{ifpcol@swapmarginpar} lets marginal notes go to the opposite side from
% that in odd numbered pages, and \cs{ifpcol@bg@swap} makes \bgpaint{}
% \mirror{}ed.
% 
% Besides the initial setting to let them $\false$ \!\global!ly after the
% declaration, the switches are \!\global!ly turned $\false$ by
% \!\pcol@twosided! for the cases in which API macro \!\twosided! does not
% have `|c|', `|m|' or `|b|' in its optional argument respectively, and then
% \!\global!ly turned $\true$ by $|\pcol@twosided@|{\cdot}k$
% ($k\in\{|c|,|m|,|b|\}$ )
% 
% \SpecialIndex{\pcol@twosided@c}
% \SpecialIndex{\pcol@twosided@m}
% \SpecialIndex{\pcol@twosided@b}
% 
% which is invoked when the argument contains `|c|', `|m|' or `|b|'
% respectively, or the API macro does not have the argument at
% all\footnote{
% 
% \cs{ifpcol@swapcolumn} is also turned $\true$ and $\false$ by backward
% compatible API macros \cs{swapcolumn}\~\texttt{inevenpages} and
% \cs{noswapcolumninevenpages} respectively.}.
% 
% \SpecialIndex{\swapcolumninevenpages}
% \SpecialIndex{\noswapcolumninevenpages}
% 
% The switch \CSIndex{ifpcol@swapcolumn} is also turned $\false$ by
% \!\pcol@zparacol! but locally if \npaired{} \parapag{}ing is specified
% because \cswap{} is meaningless in the environment.  Another modifier is (our
% own version of) \!\@outputpage!, but setting and examining the switch in
% this macro is also local and is to decide the ship-out order of left and
% right \parapag{}es.
% 
% Besides the local use by \!\@outputpage!, \CSIndex{ifpcol@swapcolumn} is
% then examined by the following macros to do special operations if it is
% $\true$ and we are in an even numbered page.
% 
% \begin{itemize}
% \item
% \!\pcol@swapcolumn! to reverse the order of column visiting in
% \!\pcol@ioutputelt!, \!\pcol@imakeflushedpage! and \!\pcol@iflushfloats!.
% 
% \item
% \!\pcol@shiftspanning! to shift a \mctext{} to the left edge of text area.
% 
% \item
% \!\pcol@bg@paint@ii! to \mirror{} the \bgpaint{} of columns and \csepgap{}s.
% \end{itemize}
% 
% On the other hand, examiners of \CSIndex{ifpcol@swapmarginpar} and
% \CSIndex{ifpcol@bg@swap} are sole for each, namely \!\pcol@addmarginpar!
% and \!\pcol@bg@paint@i! respectively.  If each switch is $\true$ and we
% are in an even numbered page, the former reverses \CSIndex{if@firstcolumn}
% from the value having been set for non-swapping case, while the latter
% \mirror{}s the \bgpaint{} of the regions excepting columns and
% \csepgap{}s.
% 
% A related switch \CSIndex{ifpcol@bg@@swap} is let be $\true$ if
% \CSIndex{ifpcol@swapcolumn} or \cs{ifpcol@bg@}\~\texttt{swap}
% 
% \CSINDEX{ifpcol@bg@swap}
% 
% is $\true$ and we are working on a even numbered page by
% \!\pcol@bg@swappage! for \mirror{}ed \bgpaint{} of columns and
% \csepgap{}s, or other regions respectively, and then examined by
% \!\pcol@bg@paintregion@i! for \mirror{}ing.
% \end{macro}\end{macro}\end{macro}\end{macro}
% 
% \begin{macro}{\ifpcol@bg@painted}
% \changes{v1.3-3}{2013/09/17}
%	{Introduced to examine if a set of regions are painted.}
% 
% The switch \cs{ifpcol@bg@painted} is \!\global!ly turned $\false$ at the
% beginning of \!\pcol@bg@paint@i!, and then \!\global!ly turned $\true$ by
% \!\pcol@bg@paintregion! if it paint the region specified by its argument,
% i.e., \!\backgroundcolor! for the region is declared.  Then the switch is
% examined by \!\pcol@bg@paint@i! to combine the painted region with others,
% by (our own version of) \!\@outputpage! and \!\pcol@outputpage@r! to
% incorporate painted regions into ship-out image.
% \end{macro}
% 
% \begin{macro}{\ifpcol@bfbottom}
% \changes{v1.3-6}{2013/09/17}
%	{Introduced to know which column-wise footnotes or bottom floats
%	 are put at the bottom of a column.}
% 
% The switch \cs{ifpcol@bfbottom} is $\true$ if \!\@makecol! puts bottom
% floats at the bottom of a column as done by the macro in \LaTeX's standard
% implementation, or $\false$ otherwise and thus bottom floats can be
% followed by footnotes as done in p\LaTeX.  Since we know only p\LaTeX{} is 
% exceptional, we let the switch $\false$ iff \!\pfmtname! is defined and
% has |pLaTeX2e| in its body.  The switch is examined by
% \!\pcol@measurecolumn! to determine which footnotes or bottom floats
% determine $\DP$ if both of them exist.
% \end{macro}
% 
% \begin{macro}{\ifpcol@dfloats}
% \changes{v1.3-6}{2013/09/17}
%	{Introduced to know if the last page has deferred column-wise floats.}
% 
% The switch \cs{ifpcol@dfloats} is $\true$ iff one or more columns (in a
% \lpage) have deferred \cwise{} floats.  Therefore, it is turned
% $\false$ by \!\pcol@sync! before it invokes \!\pcol@measurecolumn! which
% turns it $\true$ when it finds a column $c$ such that
% $\cc_c(\dl)\neq\emptyset$.  Then the switch is examined by
% \!\pcol@makeflushedpage! to make a \lpage{} {\em full size}, and by
% \!\pcol@output@end! to flush these floats.
% \end{macro}
%    \begin{macrocode}
\newif\ifpcol@output \global\pcol@outputfalse
\newif\ifpcol@nospan
\newif\ifpcol@sync \pcol@syncfalse
\newif\ifpcol@sptextstart \pcol@sptextstartfalse
\newif\ifpcol@sptext \pcol@sptextfalse
\newif\ifpcol@clear \pcol@clearfalse
\newif\ifpcol@flush
\newif\ifpcol@outputflt
\newif\ifpcol@lastpage
\newif\ifpcol@lastpagesave
\newif\ifpcol@scfnote \pcol@scfnotefalse
\newif\ifpcol@mgfnote \pcol@mgfnotefalse
\newif\ifpcol@fncounteradjustment \pcol@fncounteradjustmentfalse
\newif\ifpcol@inner
\newif\ifpcol@firstpage
\newif\ifpcol@havelastpage \global\pcol@havelastpagefalse
\newif\ifpcol@paired \global\pcol@pairedtrue
\newif\ifpcol@swapcolumn \global\pcol@swapcolumnfalse
\newif\ifpcol@swapmarginpar \global\pcol@swapmarginparfalse
\newif\ifpcol@bg@swap \global\pcol@bg@swapfalse
\newif\ifpcol@bg@@swap
\newif\ifpcol@bg@painted
\newif\ifpcol@bfbottom
\def\reserved@a{pLaTeX2e}
\ifx\reserved@a\pfmtname \pcol@bfbottomfalse \else \pcol@bfbottomtrue \fi
\newif\ifpcol@dfloats
%    \end{macrocode}
% 
% 
% 
% \subsection{\cs{dimen} and \cs{skip} Registers}
% 
% The next declaration group is for six \!\dimen! and one \!\skip!
% registers.
% 
% \begin{macro}{\pcol@prevdepth}
% The \!\dimen! register \!\pcol@prevdepth! is set to the depth of the last
% item added to the main vertical list of column $c$ from which we switch to
% another column $d$, i.e., \!\prevdepth! seen in \!\pcol@invokeoutput!
% before \!\output! request.  The value of the register is then  set into
% \!\prevdepth! also by \!\pcol@invokeoutput! after \!\output! for the
% column $d$.  The value of the register is stored in $\cc_c(\pd)$ by
% \!\pcol@setcurrcol! and then restored into the register by
% \!\pcol@igetcurrcol! for the use in \!\pcol@invokeoutput! above and in
% \!\pcol@measurecolumn!, which may let the register and $\cc_c(\pd)$ have
% $\infty$ if the \colpage{} $c$ is empty.  The register is also updated by
% \!\pcol@synccolumn! for empty main vertical list case, and by
% \!\pcol@output@end!  to be set into \!\prevdepth! for the first vertical
% item of \postenv.
% \end{macro}
% 
% \begin{macro}{\pcol@colht}
% The \!\dimen! register \!\pcol@colht! has $\VPP$ being the height of the
% tallest column in the \lpage{} taking \!\textfloatsep! below bottom floats
% into account if any.  The register is initialized to be $-\!\maxdimen!$ by
% \!\pcol@sync! and then is examined and updated in \!\pcol@measurecolumn!
% to find the tallest column.  Besides the internal use of this exploration,
% its result is referred to by \!\pcol@sync! as the threshold of \pfcheck,
% and by \!\pcol@makeflushedpage! through its argument given by
% \!\pcol@output@end! to know the height of multi-column stuff in the
% \lpage{}.  The macro \!\pcol@makeflushedpage! also lets the register have
% 0 if the \lpage{} has nothing but non-\mgfnote{} \Scfnote{}.  The other
% usage of this register is in \!\pcol@freshpage! to keep the value of
% \!\@colht! of the page made by \!\flushpage! or \!\clearpage! so that it
% is given to $\!\@colroom!=\cc_c(\vb^r)$ of each column $c$ in case a
% column $c'$ s.t.\ $c'<c$ made another page for \fcolumn{}s updating
% \!\@colht!.
% \end{macro}
% 
% \begin{macro}{\pcol@textfloatsep}
% \changes{v1.0}{2011/10/10}
%	{Introduced for the bug fix of float space enlargement.}
% 
% The \!\dimen! (not \!\skip!) register \!\pcol@textfloatsep! has
% \!\maxdimen! if a \colpage{} does not have \sync{}ation points, to let top
% floats are inserted in usual way.  If it has, the register may hold the
% vertical space amount inserted after top floats in a \colpage{} instead of
% \!\textfloatsep! so that, if a column only with top floats defines the
% first \sync{}ation point, the space for floats are extended to the
% \sync{}ation point.  In this extension case, the register has the amount
% above biased by 10000\,|pt| to distinguish the case from another case with
% ordinary top floats in which the register has non-biased \!\textfloatsep!.
% In addition, if the register is less than \!\maxdimen! including a value
% equal to \!\textfloatsep!, top floats are packed in a \!\vbox! so that
% stretch\slash shrink factor of \!\floatsep!  cannot move \sync{}ation
% points.  After the default setting to be \!\maxdimen! by \!\pcol@zparacol!
% and \!\pcol@floatplacement!, the value of the register is stored in
% $\cc_c(\tf)$ by \!\pcol@setcurrcol! and then restored into the register by
% \!\pcol@igetcurrcol! for the use in \!\pcol@makecol!,
% \!\pcol@combinefloats!, \!\pcol@cflt!, \!\pcol@output@switch!,
% \!\pcol@measurecolumn!, \!\pcol@addflhd! and \!\pcol@synccolumn!.
% \end{macro}
% 
% \begin{macro}{\pcol@lrmargin}
% \changes{v1.2-5}{2013/05/11}
%	{Introduced to let \cs{linewidth} for each column has the value
%	 according to $w_c$ and the list-like environment surrounding
%	 \string\texttt{paracol} envrionment.}
% \begin{macro}{\pcol@bg@leftmargin}
% \changes{v1.3-3}{2013/09/17}
%	{Introduced for background painting.}
% 
% The \!\dimen! register \!\pcol@lrmargin! is let have
% $\Uidx\lrm=\!\textwidth!-\!\linewidth!$ by \!\pcol@zparacol!, so that
% \!\linewidth! for column $c$ is let have $\w_c-\lrm$ by
% \!\pcol@invokeoutput!, which also sets \!\parshape! if $\lrm>0$.
% 
% The other usage of this register is to have the left or right margin for
% \bgpaint{} in the alias \!\pcol@bg@leftmargin! for strict local use in
% \!\pcol@bg@paint@i! and its descendent macros for \bgpaint.  That is, the
% register is aliased as \!\pcol@bg@leftmargin! by \!\pcol@bg@paint@i!, let
% have left or right margin by \!\pcol@bg@swappage!, and then referred to by
% \!\pcol@bg@pageleft!.
% \end{macro}\end{macro}
% 
% \begin{macro}{\pagerim}
% \changes{v1.3-3}{2013/09/17}
%	{Introduced to specify the page rim size for backgournd coloring.}
% 
% The API \!\dimen! register \!\pagerim! has the size of {\em\Uidx\prim{}s}
% specified by users.  Since the rims are the area for which \bgpaint{}
% is inhibited, the register is used in area specification macros
% \!\pcol@bg@paperwidth!, \!\pcol@bg@paperheight!, \!\pcol@bg@pageleft!  and
% \!\pcol@bg@pagetop!, in which the register has the negative counterpart of
% the specified value set by \!\pcol@bg@paint@i!.
% \end{macro}
% 
% \begin{macro}{\pcol@topskip}
% The \!\skip! register \!\pcol@topskip! keeps the value of \!\topskip! at
% \beginparacol{} for the ordinary usage of \!\topskip! which may have 0 in
% the starting and \lpage{} temporarily.  After the initialization by
% \!\pcol@zparacol!, it is referred to by \!\pcol@getpinfo! for pages without
% \spanning{} and thus \preenv, by \!\pcol@startpage! to let \!\topskip! and
% $\pp^t(p)$ has it for non-\spage{} $p$, by \!\pcol@output@start! for the
% second page if it finds \preenv{} is too large to combine with the
% multi-column stuff, and by \!\endparacol! to recover \!\topskip! for the
% pages following the \lpage.
% \end{macro}
% 
% \begin{macro}{\belowfootnoteskip}
% \changes{v1.35-4}{2018/12/31}
% 	{Introduced to specify the additional space below the non-merged
%	 pre-environment footnotes.}
% 
% The API \!\skip! register \!\belowfootnoteskip! has the amount of the space
% added below non-merged \Preenv{} footnotes.  The register is initialized
% with the default 0\,pt, and then used in \!\pcol@output@start! to measure
% the room in the first page, and in \!\pcol@combinefootins! to add the
% space.
% \end{macro}
% 
%    \begin{macrocode}
\newdimen\pcol@prevdepth
\newdimen\pcol@colht
\newdimen\pcol@textfloatsep
\newdimen\pcol@lrmargin
\newdimen\pagerim \pagerim\z@
\newskip\pcol@topskip
\newskip\belowfootnoteskip \belowfootnoteskip\z@
%    \end{macrocode}
% 
% 
% 
% \subsection{\cs{box} Registers}
% 
% The next declaration group is for the following \!\box! registers.
% 
% \begin{macro}{\pcol@topfnotes}
% \changes{v1.2-2}{2013/05/11}
%	{Introduced for page-wise footnote functions.}
% 
% The \!\box! register \!\pcol@topfnotes! is the implementation of $\df$ to
% have the list of deferred footnotes.  The register is made void by
% \!\pcol@output@start! and then is made grown by \!\pcol@fntextother!  with
% a deferred footnote added by the macro.  The macro
% \!\pcol@deferredfootins! invoked from \!\pcol@startcolumn! and
% \!\pcol@restartcolumn!  tries to \!\insert! the contents of the register
% through \!\footins! but may keep some of trailing ones in it if the total
% height of the footnotes is too large, while \!\pcol@output@end! does the
% \!\insert!ion without height capping.  The macro \!\endparacol! with
% non-\mgfnote{} \Scfnote{} typesetting also refers to the register to pass
% it to \!\pcol@flushclear! as its argument so as to make an explicit page
% break if the register has some deferred footnotes.
% \end{macro}
% 
% \begin{macro}{\pcol@prespan}
% \changes{v1.3-1}{2013/09/17}
%	{Introduced to save pre-spanning-text stuff.}
% 
% The \!\box! register \!\pcol@prespan! keeps the \prespan{} during a
% \mctext{} is processed by \TeX{} and our own \!\output! routine.  That is, 
% the macro \!\pcol@putbackmvl! saves the contents $\cc_0(\vb^b)$ of
% the column 0 to be restarted into the register instead of putting it back
% to the main vertical list, or makes the register $\bot$ if the column has
% nothing, when the restart follows the \sync{}ation with
% $\CSIndex{ifpcol@sptextstart}=\true$.  Then the contents of the register
% is put back to the main vertical list together with the box having
% \mctext{} after its vertical size is registered in the list $\pp^s(p)$ of
% \mctext{} postions and heights, by \!\pcol@makecol! when the text sees a
% page break, or by \!\pcol@output@switch! when the text is completed.
% \end{macro}
% 
% \begin{macro}{\pcol@rightpage}
% \changes{v1.3-2}{2013/09/17}
%	{Introduced to have the ship-out image of a right parallel-page.}
% 
% The \!\box! register \!\pcol@rightpage! is used to build (a part of) the
% ship-out image of a right \parapag{}e in it.  The macros;
% 
% \begin{quote}\raggedright
% \!\pcol@outputelt!,
% \!\pcol@ioutputelt!,
% \!\pcol@makeflushedpage!,
% \!\pcol@flushfloats!,
% \!\pcol@output@flush!, and
% \!\pcol@output@clear!
% \end{quote}
% 
% work on the register together with \!\@outputbox! for
% the left \parapag{}e to pass both of them to (our of version of)
% \!\@outputpage!.  The macro \!\pcol@output@end! also uses the register to
% paint the \bground{} of the empty counterpart of non-merged \Scfnote{}s in
% it, or to make the register $\bot$ when it have an empty \lpage{} but with
% \spanning{} of \pwise{} floats.  After closing a \env{paracol}
% environment, the contents of the register 
% will be shiped out by
% \!\@outputpage! invoked outside \env{paracol} environment when the
% \postenv{} sees a page break, or referred to by \!\pcol@output@start! as
% the \preenv{} in the right \parapag{}e.  This right \preenv{} then will be
% combined with \colpage{}s in the right \parapag{}e by \!\pcol@ioutputelt!
% or \!\pcol@imakeflushedpage! for shipping-out, or by
% the latter indirectly invoked from \!\pcol@output@end!  as the
% last\Index{last page}{} right \parapag{}e again.  Therefore the \!\setbox!
% of the register in \!\pcol@output@start!, \!\pcol@makeflushedpage!,
% \!\pcol@imakeflushedpage! and \!\pcol@output@end!  must be done
% \!\global!ly\footnote{
% 
% The \!\global! setting in \!\pcol@makeflushedpage! and
% \!\pcol@imakeflushedpage!, together with \!\@outputbox! which does not
% need \!\global! assignment, is also required by the sake of simplicity in
% its implementation, incidentally.}.
% \end{macro}
% 
% \begin{macro}{\pcol@colorstack@saved}
% \changes{v1.34}{2018/05/07}
%	{Introduced as $\mathit{\Gamma}_s$ to keep the color stack
%	 $\mathit{\Gamma}^c$ until a column-page of $c$ becomes non-empty.}
% 
% The \!\box! register \!\pcol@colorstack@saved! is $\csts$ to keep the
% \colorctext{} $\CST^c$ of column $c$ until its \ccolpage{} becomes non-empty
% to avoid that the \colpage{} only has coloring \!\special!s for
% \colorstack{} establishing and rewinding to let \!\pcol@ifempty! misjudge
% the \colpage{} is non-empty.  It is let have $\Celt^c$, if defined,
% and $\cst$ by \!\pcol@savecolorstack! invoked from \!\pcol@startcolumn!
% and \!\pcol@output@start!, and from \!\pcol@restartcolumn! through
% \!\pcol@putbackmvl! when we know or find the (re)starting \colpage{} is
% empty.  The macro \!\pcol@putbackmvl! also makes the \!\box! register
% $\bot$ when the restarting \colpage{} is not empty and thus the \colpage{}
% has had coloring \!\special!s for establishing \colorctext{} at its
% beginning.  Then the register is given to \!\pcol@restorecst! by
% \!\pcol@clearcst@unvbox! to put leading coloring \!\special!s for
% establishing of the \colpage{} when we complete it by \!\pcol@opcol!  or
% leave from it by \!\pcol@output@switch!.
% \end{macro}
% 
% \begin{macro}{\pcol@tempboxa}
% \changes{v1.3-3}{2013/09/17}
%	{Introduced to have materials temporarily for column-separatig rule
%	 drawing or background painting.}
% \changes{v1.34}{2018/05/07}
%	{Renamed from \cs{pcol@tempbox} because its relative
%	 \cs{pcol@tempboxb} is introduced.}
% \changes{v1.34}{2018/05/07}
%	{Add ussage in \cs{pcol@scancst} and \cs{pcol@iscancst}.}
% 
% The \!\box! register \!\pcol@tempboxa! is used to have stuff temporarily as
% follows.
% 
% \begin{itemize}
% \item
% The macro \!\pcol@buildcolseprule! and its callee \!\pcol@buildcselt!
% builds the \cseprule{} in the register for a page to be shipped out, while
% its contents is put into each \csepgap{} by \!\pcol@hfil!.
% 
% \item
% In (our own version of) \!\@outputpage!, the register has the \bgpaint{}
% of the (left parallel-) page,
% 
% \Index{parallel-paging}
% 
% which is inserted into the ship-out image by its callee
% \!\pcol@outputpage@l! through \!\everyvbox! and its contents
% \!\pcol@outputpage@ev!.
% 
% \item
% In \!\pcol@scancst! and its callee \!\pcol@iscancst! to scan $\CSTraw^c$,
% $\CST^c$ or $\csts$, the sequence of (un)coloring \!\special!s to be put
% into the main vertical list is build in it.
% \end{itemize}
% \end{macro}
% 
% \begin{macro}{\pcol@tempboxb}
% \changes{v1.34}{2018/05/07}
%	{Introduced to extract the top of color stack $\mathit{\Gamma}$,
%	 $\mathit{\Gamma}_r$ or $\mathit{\Gamma}_s$.}
% 
% The \!\box! register \!\pcol@tempboxb! is used in \!\pcol@iscancst!
% to extract the top (last) element of $\cst$, $\cstraw$ or $\csts$.
% \end{macro}
% 
%    \begin{macrocode}
\newbox\pcol@topfnotes
\newbox\pcol@prespan \setbox\pcol@prespan\box\voidb@x
\newbox\pcol@rightpage \global\setbox\pcol@rightpage\box\voidb@x
\newbox\pcol@colorstack@saved
\newbox\pcol@tempboxa
\newbox\pcol@tempboxb
%    \end{macrocode}
% 
% 
% 
% \subsection{\cs{insert} Register Set}
% \changes{v1.34}{2018/05/07}
%	{Add \Sec3.6 ``\cs{insert} Register Set'' for \cs{pcol@colorins}.}
% 
% The next declaration is for the following \!\insert! register set.
% 
% \begin{macro}{\pcol@colorins}
% \changes{v1.34}{2018/05/07}
%	{Introduced to present text-coloring operations to \cs{output}
%	 synchronously with column-pages.}
% 
% The register set \!\pcol@colorins! is to \!\insert! a \!\vbox! containing a
% (un)coloring \!\special! for color pushing or popping, or the definition
% of a new default color of the current column.  In order to make it sure
% that an \!\insert!ion does not affect \!\pagetotal! and is given to
% \!\output!  with |\box255| containing the corresponding coloring
% \!\special! put in the main vertical list, \!\count!\!\pcol@colorins! and
% \!\skip!\!\pcol@colorins! are let be 0, while \!\dimen!\!\pcol@colorins! is
% let be \!\maxdimen! to allow a \colpage{} to have vitually infinite number
% of \!\insert!ions.
% 
% The \!\insert!ion is done by \!\pcol@icolumncolor! for a default color
% definition, \!\pcol@set@color@push! for color pushing, and
% \!\pcol@reset@color@pop! and \!\pcol@reset@color@mpop! for color popping
% in non-math and math mode respectrively.  Then \!\insert!ed \!\vbox!es are
% packed into \!\box!\!\pcol@colorins! and is given to \!\output! as
% $\cstraw$ to be scanned by \!\pcol@clearcolorstack! to reform it as
% $\cst$, and then scanned by \!\pcol@restorecolorstack! or saved into
% $\csts=\!\pcol@colorstack@saved!$ by \!\pcol@savecolorstack!.  The
% register is also referred to by $\!\pcol@scancst!\arg{box}$ to examine if
% $\arg{box}$ is this register or \!\pcol@colorstack@saved!, and is made
% $\bot$ by \!\pcol@output@end! after the final reestablishment of the
% \colorstack.
% 
%    \begin{macrocode}
\newinsert\pcol@colorins
\count\pcol@colorins\z@ \skip\pcol@colorins\z@ \dimen\pcol@colorins\maxdimen
%    \end{macrocode}
% \end{macro}
% 
% 
% 
% \subsection{\cs{toks} Register}
% 
% The last declaration is for the following \!\toks! register.
% 
% \begin{macro}{\pcol@everyvbox}
% \changes{v1.22}{2013/06/30}
%	{Introduced to keep \cs{everyvbox} work as usual while having
%	 \cs{pcol@innertrue} in it always.}
% 
% The register \!\pcol@everyvbox! acts as \!\everyvbox! in \env{paracol}
% environments.  That is, by \!\pcol@zparacol!, \!\everyvbox! is made
% \!\let!-equal to this register so that updates and references of
% \!\everyvbox! is made to this register, while the real \!\everyvbox! is
% let have the reference to the register and |\pcol@innertrue| to make
% $\CSIndex{ifpcol@inner}=\true$ in every \!\vbox!.  Besides
% \!\pcol@zparacol!, the register is reffered to by
% \!\pcol@restoreeveryvbox! to examine if it has been \!\global!ly updated,
% i.e., its contents is not \!\pcol@dummytoken!.
% \end{macro}
% 
%    \begin{macrocode}
\newtoks\pcol@everyvbox

%    \end{macrocode}
% 
% 
% 
% \section{Logging Tools}
% \label{sec:imp-logging}
% 
% Prior to the \!\def!initions of macros to implement \textsf{paracol}'s
% functions, we define a few macros for debug logging.
% 
% \begin{macro}{\pcol@ShowBox}
% \changes{v1.2-2}{2013/05/11}
%	{Introduced for debugging page-wise footnote functions.}
% \changes{v1.3-6}{2013/09/17}
%	{Change \cs{unvcopy} to \cs{copy} to make sure the argument box
%	 causes overfull if its height is positive and even if it has
%	 nothing.}
% \changes{v1.34}{2018/05/07}
%	{Add messaging {\tt (VOID)} if $\langle b\rangle\EQ\bot$,
%	 $\cs{vfuzz}\gets0$ to ensure overfull, and \cs{vskip} of 1\,{\tt
%	 pt} if $\langle b\rangle$'s height is 0 to ensure overfull too.}
% 
% The macro $\!\pcol@ShowBox!\arg{b}$ puts a logging \!\message! showing the
% height, depth and width of the \!\box! (or \!\insert!) register $b$, or
% ``|(VOID)|'' if $b=\bot$.  Then, if $b\neq\bot$, $b$'s contents is dumped
% into |.log| file making overfull intentionally by putting $b$ into
% \!\box!|0| of null height, together with \!\vskip! of 1\,|pt| if $b$'s
% height is 0, with setting $\!\vfuzz!=0$.
% \end{macro}
% 
% \KeepSpace{13}
% \begin{macro}{\pcol@LogLevel}
% \changes{v1.2-2}{2013/05/11}
%	{Introduced for debugging page-wise footnote functions.}
% \begin{macro}{\pcol@iLogLevel}
% \changes{v1.2-2}{2013/05/11}
%	{Introduced for debugging page-wise footnote functions.}
% \begin{macro}{\pcol@Log}
% \changes{v1.2-2}{2013/05/11}
%	{Introduced for debugging page-wise footnote functions.}
% \begin{macro}{\pcol@Log@iii}
% \changes{v1.2-2}{2013/05/11}
%	{Introduced for debugging page-wise footnote functions.}
% \begin{macro}{\pcol@Log@ii}
% \changes{v1.2-2}{2013/05/11}
%	{Introduced for debugging page-wise footnote functions.}
% \begin{macro}{\pcol@Log@i}
% \changes{v1.2-2}{2013/05/11}
%	{Introduced for debugging page-wise footnote functions.}
% \begin{macro}{\pcol@Logstart}
% \changes{v1.2-2}{2013/05/11}
%	{Introduced for debugging page-wise footnote functions.}
% \begin{macro}{\pcol@Logstart@ii}
% \changes{v1.2-2}{2013/05/11}
%	{Introduced for debugging page-wise footnote functions.}
% \begin{macro}{\pcol@Logstart@i}
% \changes{v1.2-2}{2013/05/11}
%	{Introduced for debugging page-wise footnote functions.}
% \begin{macro}{\pcol@Logend}
% \changes{v1.2-2}{2013/05/11}
%	{Introduced for debugging page-wise footnote functions.}
% \begin{macro}{\pcol@Logend@ii}
% \changes{v1.2-2}{2013/05/11}
%	{Introduced for debugging page-wise footnote functions.}
% \begin{macro}{\pcol@Logend@i}
% \changes{v1.2-2}{2013/05/11}
%	{Introduced for debugging page-wise footnote functions.}
% \begin{macro}{\pcol@Logfn}
% \changes{v1.2-2}{2013/05/11}
%	{Introduced for debugging page-wise footnote functions.}
% \begin{macro}{\pcol@Logfn@ii}
% \changes{v1.2-2}{2013/05/11}
%	{Introduced for debugging page-wise footnote functions.}
% \begin{macro}{\pcol@Logfn@i}
% \changes{v1.2-2}{2013/05/11}
%	{Introduced for debugging page-wise footnote functions.}
% 
% The macro $\!\pcol@LogLevel!\arg{l_1}\arg{l_2}\arg{l_3}$ defines the
% detailedness of logging done by logging macros.  It invokes
% $\!\pcol@iLogLevel!\Arg{l_i}\Arg{cs}$ to let the following $|\|\arg{cs}$
% be $|\|\arg{cs}|@|\arg{l'_i}$ where $l'_i$ is the \!\romannumeral!
% representation of $l_i$.
% 
% \begin{itemize}
% \item
% $\!\pcol@Log!\arg{cs}\Arg{m}\arg{f}$ is to log the contents of the \!\insert!
% register $f$ containing footnotes which is referred to by the macro
% $\arg{cs}$ in a context shown by $m$.  The macro \!\pcol@Log@iii!
% ($l_1=3$) logs the detailed contents of $f$ by \!\pcol@ShowBox!, while
% \!\pcol@Log@ii! ($l_1=2$) just shows the height of $f$ and \!\pcol@Log@i!
% ($l_1=1$) does nothing.
% 
% \item
% $\!\pcol@Logstart!\Arg{m}$ and $\!\pcol@Logend!\Arg{m}$ put logging
% \!\message!s `$|S:|m$' and `$|E:|m$' respectively to show the beginning
% and end of a procedure in the macro whose name is at the head of $m$, if
% $l_2=2$ and thus they are \!\let!-equal to \!\pcol@Logstart@ii! and
% \!\pcol@Logend@ii!.  If $l_2=1$, \!\pcol@Logstart@i! and \!\pcol@Logend@i!
% do nothing.
% 
% \item
% $\!\pcol@Logfn!\Arg{m}$ puts a logging \!\message! $m$ whose head is the
% macro name for footnotes whose information such as the ordinal number of
% the footnote processed by the macro may be shown in $m$ as well, if
% $l_3=2$ and thus it is \!\let!-equal to \!\pcol@Logfn@ii!.  If $l_3=1$,
% \!\pcol@Logfn@i! does nothing.
% \end{itemize}
% \end{macro}\end{macro}
% \end{macro}\end{macro}\end{macro}\end{macro}
% \end{macro}\end{macro}\end{macro}
% \end{macro}\end{macro}\end{macro}
% \end{macro}\end{macro}\end{macro}
% 
%    \begin{macrocode}
%% Logging Tools

\def\pcol@ShowBox#1{%
  \ifvoid#1\message{(VOID)}%
  \else
    \message{(\the\ht#1+\the\dp#1)x(\the\wd#1)}%
    {\vfuzz\z@ \showboxdepth\@M \showboxbreadth\@M
     \setbox\z@\vbox to\z@{\ifdim\ht#1=\z@ \vskip1\p@\fi \copy#1}}%
  \fi}
\def\pcol@LogLevel#1#2#3{%
  \pcol@iLogLevel{#1}{pcol@Log}%
  \pcol@iLogLevel{#2}{pcol@Logstart}%
  \pcol@iLogLevel{#2}{pcol@Logend}%
  \pcol@iLogLevel{#3}{pcol@Logfn}}
\def\pcol@iLogLevel#1#2{%
  \expandafter\let\expandafter\reserved@a
    \csname #2@\romannumeral#1\endcsname
  \expandafter\let\csname #2\endcsname\reserved@a}
\def\pcol@Log@iii#1#2#3{\message{\string#1{#2%
    (\number\pcol@page:\number\pcol@currcol/\number\pcol@toppage)}}%
  \pcol@ShowBox#3\message{end\string#1}}
\def\pcol@Log@ii#1#2#3{\message{\string#1{#2%
    (\number\pcol@page:\number\pcol@currcol/\number\pcol@toppage)}=\the\ht#3}}
\def\pcol@Log@i#1#2#3{}
\def\pcol@Logstart@ii#1{\message{S\string#1}}
\def\pcol@Logend@ii#1{\message{E\string#1}}
\def\pcol@Logstart@i#1{}
\def\pcol@Logend@i#1{}
\def\pcol@Logfn@ii#1{\message{\string#1}}
\def\pcol@Logfn@i#1{}
\pcol@LogLevel111

%    \end{macrocode}
% 
% \begin{macro}{\pcol@F@write}
% \changes{v1.32-2}{2015/10/10}
%	{Introduced for debugging memory leak problems.}
% \begin{macro}{\pcol@F}
% \changes{v1.32-2}{2015/10/10}
%	{Introduced for debugging memory leak problems.}
% \begin{macro}{\pcol@FF}
% \changes{v1.32-2}{2015/10/10}
%	{Introduced for debugging memory leak problems.}
% \begin{macro}{\pcol@F@count}
% \changes{v1.32-2}{2015/10/10}
%	{Introduced for debugging memory leak problems.}
% \begin{macro}{\pcol@F@n}
% \changes{v1.32-2}{2015/10/10}
%	{Introduced for debugging memory leak problems.}
% \begin{macro}{\pcol@Fb}
% \changes{v1.32-2}{2015/10/10}
%	{Introduced for debugging memory leak problems.}
% \begin{macro}{\pcol@Fe}
% \changes{v1.32-2}{2015/10/10}
%	{Introduced for debugging memory leak problems.}
% Another debugging tool is for investigating memory leak problems.  Since
% \Paracol{} uses \!\insert! registers for various purposes, management
% operations of them especially that for proper release of them are crusial
% for the correctness of the implmentation.  A source of toughness in
% debugging {\em memory leak} caused by missing a release of a register back
% to \!\@freelist! is that the resulting shortage is revealed long after the
% leakage to make it hard to find the point of the leakage.
% 
% The set of macros is to help such debugging by logging the aquire and
% release of \!\insert! registers into a file named $\arg{job}|.fls|$
% associated with \!\pcol@F@write! where $\arg{job}$ is given by
% \!\jobname!.  After opened when \Paracol{} is loaded, the file is written by
% $\!\pcol@FF!\Arg{m_a}\Arg{m_b}$ with a line of the following form with
% text messages $m_a$ and $m_b$, where $p=\!\pcol@page!$,
% $c=\!\pcol@currcol!$, $\ptop=\!\pcol@toppage!$, $\pi=\page(p)=\!\c@page!$,
% and $n_b=\!\pcol@F@n!$ is the cardinality of \!\@deferlist! counted by
% \!\pcol@F@count!.
% 
% \begin{itemize}\item[]
% $\arg{m_a}|(|\arg{p}|:|\arg{c}|/|\arg{\ptop}|:|\arg{\pi}|)=|
% \arg{n_b}\arg{m_b}$
% \end{itemize}
% 
% The argument $\arg{m_b}$ is empty when $\!\pcol@FF!$ is invoked from
% $\!\pcol@F!\arg{m_a}$ for snapshot, while
% $\arg{m_b}=\hbox{`|<=|$\arg{n_b}$'}$ when invoked from
% $\!\pcol@Fe!\arg{m_a}$ paired by $\!\pcol@Fb!=\!\pcol@F@count!$ by which
% the cardinality of \!\@freelist! is given to $\arg{n_b}$ through
% \!\pcol@F@n! and then \!\reserved@a!.  Therefore, by the pair of
% \!\pcol@Fb! and \!\pcol@Fe!, the consumption or restitution in a series of
% operations surrounded by the pair is logged in the file.
% 
% In the production version, the logging is disabled of course by \!\let!ting
% \!\pcol@F! and \!\pcol@Fe! be \!\@gobble! and \!\pcol@F! be \!\relax!,
% while the open of \!\pcol@F@write! is disabled as well by a pair of
% \cs{iffalse} and \cs{fi}.
% \end{macro}\end{macro}\end{macro}\end{macro}\end{macro}\end{macro}\end{macro}
% 
%    \begin{macrocode}
\iffalse
\newwrite\pcol@F@write
\immediate\openout\pcol@F@write\jobname.fls
\fi
\def\pcol@F#1{\pcol@FF{#1}{}}
\def\pcol@FF#1#2{\pcol@F@count
  \immediate\write\pcol@F@write{#1(\number\pcol@page:\number\pcol@currcol/%
    \number\pcol@toppage:\number\c@page)=\pcol@F@n #2}}
\def\pcol@F@count{{\@tempcnta\z@
    \def\@elt##1{\advance\@tempcnta\@ne}\@freelist
    \xdef\pcol@F@n{\number\@tempcnta}}}
\let\pcol@Fb\pcol@F@count
\def\pcol@Fe#1{{\let\reserved@a\pcol@F@n \pcol@FF{#1}{<=\reserved@a}}}
\let\pcol@F\@gobble
\let\pcol@Fb\relax
\let\pcol@Fe\@gobble

%    \end{macrocode}
% 
% 
% 
% \section{\cs{output} Routine}
% \label{sec:imp-output}
% 
% \begin{macro}{\pcol@ovf}
% Before giving the definitions of macros in \!\output! routine, we define
% the macro \!\pcol@ovf! invoked if \!\@freelist! is empty on an acquision
% of an \!\insert! by \!\@next! and thus we have to abort the execution by
% \!\PackageError! with a message notifying the shortage.  The additional
% help message is \!\@ehb! as in \!\@fltovf!.  This macro is used in
% \!\pcol@opcol!, \!\pcol@startpage!, \!\pcol@output@start!,
% \!\pcol@output@switch!, \!\pcol@iscancst!, \!\pcol@savefootins!,
% \!\pcol@flushcolumn!, \!\pcol@synccolumn!, \!\pcol@output@end! and
% \!\pcol@icolumncolor!.
% \end{macro}
% 
%    \begin{macrocode}
%% \output Routine

\def\pcol@ovf{%
  \PackageError{paracol}{Too many unprocessed columns/floats}\@ehb}

%    \end{macrocode}
% 
% \begin{macro}{\pcol@output}
% \changes{v1.2-2}{2013/05/11}
%	{Add \cs{pcol@Logstart} and \cs{pcol@Logend}.}
% \changes{v1.2-7}{2013/05/11}
%	{Add the examination of \cs{ifpcol@output} and \protect\LaTeX's
%	 original sequence for \cs{output} request sneaked from outside of
%	 \texttt{paracol} environment.}
% \changes{v1.2-7}{2013/05/11}
%	{Add the assignment of \cs{@maxdepth} to \cs{maxdepth} to nullify
%	 the temproary setting done by \cs{@addtobot}.}
% \changes{v1.33-2}{2016/11/19}
% 	{Add a space after \cs{@opcol} to obey the coding convention.}
% 
% The macro \!\pcol@output! is the \Paracol's version of \!\output! which
% is let have this macro as its sole token by \!\pcol@zparacol!.  The
% structure of this macro is same as that of \LaTeX's \!\output! but the
% following replacements are made\footnote{.
% 
% Besides the logging with \cs{pcol@Logstart} and \cs{pcol@Logend}.}.
% 
% \begin{itemize}
% \item
% $\!\@specialoutput!\to\!\pcol@specialoutput!$ to process \LaTeX's genuine
% functions including the customized marginal note placement, and \Paracol's
% own special output functions; starting first page, \colorctext{} management,
% \cswitch, page flushing with\slash without float flusing, and building the
% multi-column part of the \lpage.
% 
% \item
% \changes{v1.0}{2011/10/10}
%	{Replace \cs{@makecol} with \cs{pcol@makecol} for a special care for
% 	 column-pages with synchronization points.}
% 
% $\!\@makecol!\to\!\pcol@makecol!$ for a special care for \ccolpage{} having
% \sync{}ation point and/or \Scfnote{}s.
% 
% \item
% $\!\@opcol!\to\!\pcol@opcol!$ to hold the \ccolpage{} which just has
% completed.
% 
% \item
% \changes{v1.2-2}{2013/05/11}
%	{Add an argument \cs{@ne} to \cs{pcol@startcolumn} to distinguish it
%	 from the invocation in \cs{pcol@freshpage}.}
% 
% $\!\@startcolumn!\to\!\pcol@startcolumn!$ to creat a new \colpage{}.  The
% argument \!\@ne! is to distinguish the invocation in this macro from
% that in \!\pcol@freshpage! so that the \!\insert!ions of $\pp^f(p)$ and
% $\df$ are done only when a new \colpage{} is created with ordinary page
% break.
% \end{itemize}
% 
% \changes{v1.22}{2013/06/30}
%	{Add reset of \cs{set@color}.}
% \changes{v1.24}{2013/07/27}
%	{Add zero-clear of \cs{pcol@mcid}.}
% \changes{v1.3-1}{2013/09/17}
%	{Add $\cs{ifpcol@sptextstart}\EQ\string\mathit{false}$ to the
%	 condition for the warning of too small \cs{vsize}.}
% 
% In addition, before we start the main body of \!\output! routine, we add
% two operations for coloring.  One is to make
% \!\set@color! \!\let!-equal to \!\pcol@set@color!, i.e, to let it regain
% its original definition throughout \!\output! routine, because
% no manipulation of \colorstack{} is necessary\footnote{
% 
% Though this operation is not necessary because \cs{everyvbox} should work
% for any \cs{set@color} because they should be in a \cs{vbox}, we dare to
% do it for clearity.}.
% 
% The other is to zero-clear the counter \!\pcol@mcid! because we are
% definitely in the main (non-internal) vertical mode and thus all push/pop
% pairs of the coloring in math mode have been processed.
% 
% Further, before we start the sequence for non-special \!\output!  request
% on page breaks, we examine if $\CSIndex{ifpcol@output}=\true$ to mean
% \!\pcol@output@start! has already been invoked in order to cope with
% \!\output! request {\em sneaking}.  This sneaking happens when
% \beginparacol{} is at a crtical position of page breaking at which the
% \preenv{} has already exceeds \!\vsize! but \TeX{} cannot make the
% \!\output! request for the page break at \CSIndex{par} at the beginning of
% \!\pcol@zparacol! because it sees $\!\penalty!=10000$ due to, e.g., a
% sectioning command just preceding \beginparacol.  In this case, the
% request is postponed until \TeX{} see a \!\penalty! less than 10000 and
% thus it is made with some non-special \!\outputpenalty! greather than
% $-10000$ when \TeX{} sees the dummy request of $\!\penalty!=-10004$ in
% \!\pcol@invokeoutput! for \!\pcol@output@start!.  At this timing,
% \!\pcol@zparacol! has already let \!\output! have \!\pcol@output! of
% course but the request must be processed by original \!\output! because it
% is made {\em outside} of the \env{paracol} environment which has just
% started.  Therefore, if $\CSIndex{ifpcol@output}=\false$, we have to
% perform the operation sequence as the original \!\output! does.
% Furthermore, we have to take care of the fact that a few our own settings
% related to \!\output!  routine has already been made in \!\pcol@zparacol!,
% namely $\CSIndex{if@twocolumn}=\true$ and
% $\!\@combinefloats!=\!\pcol@combinefloats!$, which should make the macros
% in the original sequnce confused especially by the former.  Therefore, we
% turns $\CSIndex{if@twocolumn}=\false$ and let \!\@combinefloats! have the
% original definition kept in \!\pcol@@combinefloats!\footnote{
% 
% Though we know \cs{pcol@combinefloats} acts exactly as \cs{@combinefloats}
% because \cs{pcol@zparacol} initializes $\cs{pcol@textfloatsep}=\infty$ and
% $\cs{ifpcol@lastpage}=\false$.  On the other hand, we don't cancel the
% re\cs{def}itnition of \cs{footnoterule} because it should be
% $\cs{textwidth}=\cs{columnwidth}$ outside of \env{paracol} environments.}
% 
% temporarily, i.e., only in the group automatically surrounding the
% invocation of \!\output!.
% 
% Another addition is to assign \!\@maxdepth! to \!\maxdepth! in order to
% nullify the temporary setting to 0 done in \!\@addtobot!.  By this
% assignment, in \env{paracol} environments \TeX's page builder always
% refers to the value in \!\@maxdepth!.  Yet another addition is to add
% $\CSIndex{ifpcol@sptextstart}=\false$ to the condition for the warning of
% too short \!\vsize!, because a \mctext{} can start near the bottom of a
% page with a small \!\@colroom! less than $1.5\times\!\baselineskip!$ and
% thus the warning is unnecessary and inappropriate when
% $\CSIndex{ifpcol@sptextstart}=\true$.
% 
%    \begin{macrocode}
\def\pcol@output{\let\par\@@par \let\set@color\pcol@set@color
  \global\pcol@mcid\z@
  \pcol@Logstart{\pcol@output\number\outputpenalty
    (\number\c@page:\number\pcol@currcol)}%
  \ifnum\outputpenalty<-\@M
    \pcol@specialoutput
  \else\ifpcol@output
    \pcol@makecol
    \pcol@opcol
    \pcol@startcolumn\@ne
    \@whilesw\if@fcolmade\fi{\pcol@opcol \pcol@startcolumn\@ne}%
  \else
    \@twocolumnfalse \let\@combinefloats\pcol@@combinefloats
    \@makecol
    \@opcol
    \@startcolumn
    \@whilesw\if@fcolmade\fi{\@opcol \@startcolumn}%
  \fi\fi
  \global\maxdepth\@maxdepth
  \ifnum\outputpenalty>-\@Miv
    \ifdim\@colroom<1.5\baselineskip
      \ifdim\@colroom<\textheight
        \ifpcol@sptextstart
          \global\vsize\@colroom
        \else
          \@latex@warning@no@line{Text page \thepage\space
                                  contains only floats}%
          \@emptycol
        \fi
      \else
        \global\vsize\@colroom
      \fi
    \else
      \global\vsize\@colroom
    \fi
  \else
    \global\vsize\maxdimen
  \fi
  \pcol@Logend\pcol@output}

%    \end{macrocode}
% \end{macro}
% 
% 
% 
% \section{Completing Column-Page}
% \label{sec:imp-opcol}
% 
% \begin{macro}{\pcol@@makecol}
% \changes{v1.2-7}{2013/05/11}
%	{Introduced to cope with the careless implementaion of \cs{@makecol}
%	 in p\string\LaTeX.}
% \changes{v1.3-6}{2013/09/17}
%	{Add an argument $d$ to be assigned to \cs{boxmaxdepth} to let it
%	 have 0 rather than \cs{@maxdepth} for last page.}
% 
% The macro $\!\pcol@@makecol!\<d\>$ is used in \!\pcol@flushcolumn! and
% \!\pcol@imakeflushedpage! which simply require \LaTeX's original
% \!\@makecol! to build the ship-out image of a \colpage.  The reason why we
% need our own version is that a variation of \LaTeX, namely p\LaTeX{} for
% Japanese, carelessly implements its own \!\@makecol! to make the resulting
% \!\@outputbox! has a depth larger than \!\@maxdepth! if the \colpage{} has
% \Mcfnote{}s whose last line is unusually deep.  To cope with the problem,
% this macro at first invokes \!\@makecol!, and then reshape \!\@outputbox!
% assigning $d=\!\@maxdepth!$ to \!\boxmaxdepth! to cap its depth, unless
% this macro is used for the \lpage{} with $d=0$ because depth of the last
% component of the \!\@outputbox! is incorporated in \!\@colht!.
% 
%    \begin{macrocode}
%% Completing Column-Page

\def\pcol@@makecol#1{\@makecol
  \setbox\@outputbox\vbox to\@colht{\boxmaxdepth#1\unvbox\@outputbox}}
%    \end{macrocode}
% \end{macro}
% 
% \begin{macro}{\pcol@makecol}
% \changes{v1.0}{2011/10/10}
%	{Introduced for special float handling in a column-page with
%	 synchronization points.}
% \changes{v1.2-2}{2013/05/11}
%	{Add save/discard of page-wise footnotes.}
% \changes{v1.2-7}{2013/05/11}
%	{Remove unnecessary check of \cs{ifpcol@lastpage} on the
%	 redefinition of \cs{@textbottom}.}
% \changes{v1.3-1}{2013/09/17}
%	{Add a function to capture a broken spanning text, to combine it
%	 with pre-spanning-text stuff, and to shift it left on
%	 column-swapping.}
% \changes{v1.3-3}{2013/09/17}
%	{Add the addition of the element to $\pi^s(p)$ for a broken
%	 spanning text.}
% \changes{v1.3-4}{2013/09/17}
%	{Add $\pi^m(p)\EQ\cs{pcol@mparbottom}$ to the argument of
%	 \cs{pcol@defcurrpage}.}
% \changes{v1.3-3}{2013/09/17}
%	{Add \cs{@colht} and \cs{relax} as the first and third argument of
%	 \cs{pcol@shrinkcolbyfn}.}
% \changes{v1.32-2}{2015/10/10}
% 	{Add \cs{pcol@Fb}/\cs{pcol@Fe} pair(s).}
% \changes{v1.33-2}{2016/11/19}
% 	{Move down the \cs{def}inition of \cs{pcol@currfoot} with $\bot$ to
%	 place it just before the \cs{ifpcol@scfnote}/\cs{fi} construct to
%	 make it clear how \cs{pcol@currfoot} is \cs{def}ined.}
% 
% The macro \!\pcol@makecol! is invoked solely from \!\pcol@output! to build
% the shipping image of the \ccolpage{} which just has completed in
% \!\@outputbox!.  This macro has two additional functions to its original
% version \!\@makecol!\footnote{
% 
% Not \cs{pcol@@makecol} because the depth capping of \cs{@outputbox} is
% done by \cs{pcol@opcol} when it saves the box into $\cc_c(\vb^b)$.}
% 
% invoked in this macro.
% 
% First, if $\cc_c(\tf)\neq\infty$ to mean the \colpage{} has \sync{}ation
% points, \!\@makecol! is invoked with a special \!\def!inition of
% \!\@textbottom! to put a vertical skip having $1/10000\,|fil|$ as its
% stretch and shrink.  This modification is to nullify not only finite
% stretches (as \!\raggedbottom! does) but also finite shrinks possibly
% inserted just below the last \sync{}ation point to move up the first
% visible item upward a little bit if active.  Therefore, \!\flushbottom!
% setting is nullified for \colpage{}s having \sync{}ation points and a
% small exceess from the bottom of a \colpage{} cannot be absorbed by
% shrinks but visible at the bottom\footnote{
% 
% That is, the author gives higher priority to the perfect alignment of the
% items following a \sync{}ation point.}.
% 
% Note that the original definition of \!\@textbottom! is saved in
% \Midx{\!\pcol@textbottom!} before the invocation of \!\@makecol! and is
% restored after that\footnote{
% 
% This save/restore cannot be done by a grouping because \!\@makecol! builds
% \!\@outputbox! by local assignments.}.
% 
% Second, if $\CSIndex{pcol@sptext}=\true$ and $c=0$ to mean a \mctext{}
% encounters a page break, we have the first half (or second or succeeding
% part if the text lays across three or more pages) of the text in
% \!\box!|255|.  Therefore, we add an element $\spt(H_n,h_n)$ to the tail of
% the list of \mctext{}s $\pp^s(\ptop)=\!\pcol@sptextlist!$, where $H_n$ is
% the height of \prespan{} in \!\pcol@prespan!\footnote{
% 
% Since we have a \sync{}ation point before a \mctext{} always, \prespan{}
% or its sole contents \cs{vbox} has a vertical skip at its tail to make the
% its depth 0 as discussed in \secref{sec:imp-sout-sync}.}
% 
% plus the total height of top floats calculated by \!\pcol@addflhd! with
% $\cc_c(\tl)=\!\@toplist!$, and $h_n$ is the height-plus-depth of
% \!\box!|255|.  Note that $H_n$ and $h_n$ are represented in the form of
% integer and thus we produce them by expansions with \!\number!.
% 
% The addition, however, is not made if $h_n=0$ because painting its
% \bground{} is harmful if an extension is specified to make the region
% visible, while not painting or drawing a segment of \cseprule{} is very
% natural.  Note that this $h_n=0$ case includes that in which \!\box!|255|
% has nothing but its height-plus-depth is non-zero because of discarding
% leading skips of the \mctext{} as pre-break skips.  This special case is
% detected by decapsulating \!\box!|255| by \!\unvcopy!  and examining the
% height-plus-depth of the result\footnote{
% 
% We cannot do
% \cs{setbox}\cs{@cclv}\cs{vbox}\texttt{\Arg{\cs{unvbox}\cs{@cclv}}}
% because it erases the effect of pre-break skip following some visible
% material.}.
% 
% Also note that the list to be added is always for the \tpage{}, i.e.,
% $\pp^s(\ptop)$ and thus we get and update it by \!\pcol@getcurrpinfo! and
% \!\pcol@defcurrpage!, because the \mctext{} immediately follows a
% \sync{}ation point in $\ptop$.  Then we let \!\box!|255| have the
% \prespan{} followed by the \mctext{} being the original contents of
% \!\box!|255|, which may be shifted left by
% $\WT-w_c=\!\textwidth!-\!\columnwidth!$ by the macro
% \!\pcol@shiftspanning! if \cswap{} is in effect so that its left edge is
% aligned to that of the leftmost column, i.e., of the text area.
% 
% The third addition is for \Scfnote{}s.  If they are presented in
% \!\footins!, we shrink \!\@colht! by its height plus depth
% by \!\pcol@shrinkcolbyfn! and put the stretch and shrink factor of
% \!\skip!\!\footins! at the bottom of \!\box!|255| by \!\pcol@unvbox@cclv!
% to {\em remove} footnotes from the \colpage{} but keeping the
% stretch\slash shrink contribution to the page breaking by their existence.
% Then we save \!\footins! into a new \!\insert! to be referred to as
% \!\pcol@currfoot!\footnote{
% 
% Not in \cs{pcol@footins} because it is destroyed in \cs{pcol@startpage}
% just before saving operation into $\pp^f(p)$.}
% 
% by \!\pcol@savefootins! if $p=\ptop$ so that it is saved in $\pp^f(p)$ by
% \!\pcol@startpage! afterward, or simply discard the contents of
% \!\footins! otherwise because $\pp^f(p)$ has already been fixed.  Note
% that these saving\slash discarding make \!\footins! void and thus
% \!\@makecol! will not put footnotes.
% 
% On the other hand, if footnote typesetting is \mcfnote{}, \!\footins! is
% kept unchanged so that its contents will be put by \!\@makecol! if it has
% something.  As for \!\pcol@currfoot!, it should have its default value
% $\!\voidb@x!=\bot$ assigned to it beforehand, so that, if $p=\ptop$,
% \!\pcol@startpage! will make $\pp^f(p)=\bot$ unless \Scfnote{s} are given
% in \!\footins!.
% 
%    \begin{macrocode}
\def\pcol@makecol{\let\pcol@textbottom\@textbottom
  \ifdim\pcol@textfloatsep=\maxdimen\else
    \def\@textbottom{\vskip\z@\@plus.0001fil\@minus.0001fil}\fi
  \ifpcol@sptext \ifnum\pcol@currcol=\z@
    \pcol@getcurrpinfo\@tempcnta\@tempdima\@tempskipa
    \setbox\@tempboxa\vbox{\unvcopy\@cclv}%
    \@tempdimb\ht\@tempboxa \advance\@tempdimb\dp\@tempboxa
    \ifdim\@tempdimb>\z@
      \@tempdimb\ht\@cclv \advance\@tempdimb\dp\@cclv
      \dimen@\ht\pcol@prespan \pcol@addflhd\@toplist\pcol@textfloatsep
      \@cons\pcol@sptextlist{{\number\dimen@}{\number\@tempdimb}}%
    \fi
    \pcol@defcurrpage{\number\@tempcnta}\pcol@spanning\pcol@footins
                     {\pcol@sptextlist}{\pcol@mparbottom}%
    \setbox\@cclv\vbox{\unvbox\pcol@prespan \pcol@shiftspanning\@cclv
                       \unvbox\@cclv}%
  \fi\fi
  \def\pcol@currfoot{\voidb@x}%
  \ifpcol@scfnote \ifvoid\footins\else
    \pcol@shrinkcolbyfn\@colht\footins\relax
    \setbox\@cclv\vbox{\pcol@unvbox@cclv\footins}%
    \ifnum\pcol@page=\pcol@toppage
      \pcol@Log\pcol@makecol{save}\footins
      \pcol@Fb
      \pcol@savefootins\pcol@currfoot
      \pcol@Fe{makecol(pagefn)}%
    \else
      \pcol@Log\pcol@makecol{discard}\footins
      \setbox\@tempboxa\box\footins
    \fi
  \fi\fi
  \pcol@Logstart\pcol@makecol
  \ifvoid\footins\else \pcol@Log\@makecol{put}\footins \fi
  \@makecol
  \pcol@Logend\pcol@makecol
  \let\@textbottom\pcol@textbottom}
%    \end{macrocode}
% \end{macro}
% 
% \KeepSpace{1}
% \begin{macro}{\@combinefloats}
% \begin{macro}{\pcol@combinefloats}
% \changes{v1.0}{2011/10/10}
%	{Introduced for special float handling in a column-page with
%	 synchronization points.}
% \changes{v1.2-2}{2013/05/11}
%	{Remove the shrink of \cs{textfloatsep} because each column in the last
%	 page is now made not taller than \cs{@colht} definitely by the
%	 introduction of the pre-flushing column height check.}
% \changes{v1.2-7}{2013/05/11}
%	{Add the assignment of \cs{@maxdepth} to \cs{maxdepth} to nullify
%	 the temproary setting done by \cs{@addtobot}.}
% \changes{v1.3-6}{2013/09/17}
%	{Add special operations for columns having synchronization point to
%	 move the infinite stretch and shrink to let it follow bottom floats
%	 ratehr than preceding them.}
% \begin{macro}{\pcol@@combinefloats}
% \changes{v1.2-7}{2013/05/11}
%	{Introduced to solve the \cs{output} request sneaking.}
% 
% The macro \!\pcol@combinefloats! being our own version of \!\@combinefloats!
% is used in \LaTeX's \!\@makecol! because the original and our
% own are made \!\let!-equal by \!\pcol@zparacol!, and also used in
% \!\pcol@makenormalcol!  explicitly.  The customization is twofold for both
% of top and bottom floats.
% 
% For the top floats, we invoke the original \!\@cflt! if
% $\cc_c(\tf)=\infty$ to mean the \colpage{} to be shipped out does not have
% \sync{}ation points, or otherwise our own \!\pcol@cflt! which we will
% discuss shortly.  Prior to the invocation of \!\@cflt!, in addition, we
% let $\!\maxdepth!=\!\@maxdepth!$ so that the macro refers to the value used
% throughout a \env{paracol} environment instead of that modified by
% \!\@addtobot! possibly affect the work in \!\@cflt! by the following
% sequence.
% 
% \begin{eqnarray*}
% \!\pcol@flushcolumn!(c)&\to&\!\pcol@trynextcolumn!\to\cdots\to\!\@addtobot!\\
% &\to&\!\pcol@flushcolumn!(c{+}k)\to\!\pcol@@makecol!\to\!\@makecol!\\
% &\to&\!\pcol@combinefloats!\to\!\@cflt!
% \end{eqnarray*}
%
% For the bottom floats, we invoke the original \!\@cflb! always but, if
% the \colpage{} has \sync{}ation points, we insert vertical skips of
% $s=0\,|pt|\ |plus|\ 0.0001\,|fil|\ |minus|\ 0.0001\,|fil|$ and $-s$ before
% and after the invocation respectively.  Since \!\@textbottom! is let have
% the skip of $s$ by \!\pcol@makecol! for a \colpage{} having \sync{}ation
% points and is inserted below bottom floats by \!\@makecol!\footnote{
% 
% The insertion point is common to \LaTeX{} and p\LaTeX{}.},
% 
% the effect of \!\@textbottom! is canceled by the skip of $-s$ but looks
% moved above bottom floats.  Therefore, if the natural height of the
% \colpage{} is smaller than \!\@colht!, bottom floats are flushed to the
% page bottom as if \colpage{} itself is flushed by \!\newpage! etc.  On the
% other hand, if the natural height is larger, more importantly, all shrinks
% below the last \sync{}ation point is canceled by the infinite shrink in
% $s$ above the bottom float but we should have sufficient space for shrinks
% there thanks to \!\textfloatsep! to avoid the interference between the
% bottom text and bottom floats.
% 
% In addition, if $\CSIndex{ifpcol@lastpage}=\true$ to mean the \colpage{} is
% in the \lpage{}, we insert \!\textfloatsep! in \!\@outputbox! below the
% bottom floats so that they are well separated from \postenv.  The switch
% is also $\true$ in the invocation from \!\pcol@makenormalcol! for
% \preenv{}, so that the bottom floats in it are well separated from the top
% of multi-column stuff in the \spage.
% 
% On the other hand, the original \!\@combinefloats! saved in
% \!\pcol@@combinefloats! by \!\pcol@zparacol! is used in \!\pcol@output! to
% restore the original when it finds \!\output! request sneaking.
% 
%    \begin{macrocode}
\def\pcol@combinefloats{%
  \global\maxdepth\@maxdepth
  \ifx\@toplist\@empty\else
    \ifdim\pcol@textfloatsep=\maxdimen \@cflt \else \pcol@cflt \fi
  \fi
  \ifx\@botlist\@empty\else
    \ifdim\pcol@textfloatsep=\maxdimen \@cflb
    \else
      \setbox\@outputbox\vbox{\unvbox\@outputbox
        \vskip\z@\@plus.0001fil\@minus.0001fil}%
      \@cflb
      \setbox\@outputbox\vbox{\unvbox\@outputbox
        \vskip\z@\@plus-.0001fil\@minus-.0001fil}%
    \fi
    \ifpcol@lastpage
      \setbox\@outputbox\vbox{\box\@outputbox \vskip\textfloatsep}%
    \fi
  \fi}
%    \end{macrocode}
% \end{macro}\end{macro}\end{macro}
% 
% \begin{macro}{\pcol@cflt}
% \changes{v1.0}{2011/10/10}
%	{Introduced for special float handling in a column-page with
%	 synchronization points.}
% \changes{v1.2-7}{2013/05/11}
%	{Replace \cs{maxdepth} with \cs{@maxdepth}.}
% \changes{v1.32-2}{2015/10/10}
% 	{Add \cs{pcol@Fb}/\cs{pcol@Fe} pair(s).}
% \changes{v1.33-2}{2016/11/19}
% 	{Add {\tt\%} to the end of the line to open \cs{vbox} for
%	 \cs{@outputbox} to obey the coding convention.}
% 
% The macro \!\pcol@cflt! is invoked solely from \!\pcol@combinefloats! if
% the \colpage{} for which the macro combines the top floats has
% \sync{}ation points.  The macro has the same structure as \LaTeX's version
% \!\@cflt! but has three modifications.  The first one is that the floats
% are packed in a \!\vbox! rather than listed in \!\@outputbox! to nullify
% the stretch and shrink of \!\floatsep! to keep the \sync{}ation point from
% moving by them\footnote{
% 
% Maybe unnecessary because of \!\@textbottom! inserted by \!\pcol@makecol!
% but \ldots}.
% 
% The second is that we use \!\@maxdepth! instead of \!\maxdepth! to make it
% clear we always use the value common throughout a \env{paracol}
% environment.  The third is that the \!\textfloatsep! is replaced with
% $\!\pcol@textfloatsep!=\cc_c(\tf)$ (definitely finite) which can have a
% value different from \!\textfloatsep! when the float space is enlarged for
% \sync{}ation.  If this enlargement is required, $\cc_c(\tf)$ is biased by
% 10000\,|pt| and thus is assuredly\footnote{
% 
% Though not definitely in theoretical sense.}
% 
% larger than 5000\,|pt|.  If so, the insertion of \!\topfigrule! should be
% inhibited because it has already been inserted by \!\pcol@synccolumn! or
% there are no real floats but we only have the float for main vertical list
% prior to the \sync{}ation point, or {\em\Uidx\mvlfloat} in short.
% 
%    \begin{macrocode}
\def\pcol@cflt{%
  \let\@elt\@comflelt
  \setbox\@tempboxa\vbox{}%
  \@toplist
  \setbox\@outputbox\vbox{%
    \boxmaxdepth\@maxdepth
    \box\@tempboxa
    \vskip-\floatsep
    \ifdim\pcol@textfloatsep>5000\p@
      \advance\pcol@textfloatsep-\@M\p@
    \else
      \topfigrule
    \fi
    \vskip\pcol@textfloatsep
    \unvbox\@outputbox}%
  \let\@elt\relax
  \pcol@Fb
  \xdef\@freelist{\@freelist\@toplist}%
  \pcol@Fe{cflt}%
  \global\let\@toplist\@empty}

%    \end{macrocode}
% \end{macro}
% 
% \begin{macro}{\pcol@opcol}
% \changes{v1.0}{2011/10/10}
%	{Remove unnecessary assignment of \cs{@colht}.}
% \changes{v1.0}{2011/10/10}
%	{Rename \cs{pcol@maxpage} as \cs{pcol@toppage}.}
% \changes{v1.2-1}{2013/05/11}
%	{Add \cs{pcol@clearcst@unvbox} to add coloring \cs{special}s at the
%	 top and bottom of the column-page to be shipped out, together with
%	 the setting $\cs{boxmaxdepth}\EQ\cs{@maxdepth}$ for depth capping.}
% \changes{v1.3-2}{2013/09/17}
%	{Rename \cs{pcol@outputpage} as \cs{pcol@outputcolumns}.}
% \changes{v1.32-2}{2015/10/10}
% 	{Add \cs{pcol@Fb}/\cs{pcol@Fe} pair(s).}
% \changes{v1.33-2}{2016/11/19}
% 	{Add {\tt\%} to the end of the line to open \cs{vbox} for
%	 \cs{@currbox} to obey the coding convention.}
% 
% The macro \!\pcol@opcol! is invoked from \!\pcol@output! for the ordinary
% completed \colpage{} built by \!\pcol@makecol!, or from the loop creating
% \fcolumn{}s in \!\pcol@output! or \!\pcol@freshpage!.  At first it
% saves the \colpage{} of column $c$ in \!\@outputbox!, which
% \!\pcol@makecol! or \!\@tryfcolumn! just has built for an ordinary or
% \fcolumn{} respectively, in an \!\insert! acquired from \!\@freelist! by
% \!\@next!, and then adds it to the tail of $\S_c=|\pcol@shipped|{\cdot}c$
% 
% \SpecialArrayIndex{c}{\pcol@shipped}
% 
% by \!\@cons!.  In this saving operation, we add the sequence of uncoloring
% \!\special!s at the bottom to clear \colorstack{} by
% \!\pcol@clearcst@unvbox! giving it \!\@outputbox! to be \!\unvbox!ed and
% possibly coloring \!\special!s for the \colpage{}'s \colorctext{} saved in
% $\csts$ at the top, so that the succeeding \colpage{} in printing order
% starts with its own \colorctext.  For this addition, furthermore, we let
% $\!\boxmaxdepth!=\!\@maxdepth!$ to keep the depth capping made in the box
% builder from nullified\footnote{
% 
% Or to apply the capping dropped from p\LaTeX's \cs{@makecol}, or to do
% nothing for the box made by \cs{@tryfcolumn} and thus being 0 deep.}.
% 
% Then if $c=0$, we fix the page number of the page $p$ having the
% \colpage{} and let $\pp^p(q)$ have $\page(p)+(q-p)$ usually but possibly
% $\page(p)+2(q-p)$ with \npaired{} \parapag{}ing, for all
% $q\in[p{+1},\ptop]$ by \!\pcol@setpageno!.  After that, we invoke
% \!\pcol@nextpage! to let $p=p'$ for the next \colpage{} of $c$, where
% $p'=p+1$ usually but can be $p+k+1$ if we have consecutive $k$ \fpage{}s
% from $p+1$.
% 
% Next, we check if the oldest page $\pbase$ is made ready to be shipped out
% by the participation of the completed \colpage{} by \!\pcol@checkshipped!.
% If so, we invoke \!\pcol@outputcolumns! giving argument 0 to ship out
% $\pbase$ and its successor \fpage{}s.
% 
% Finally we set up the next page $p$ by \!\pcol@startpage! if $p>\ptop$
% meaning it is new one, or by \!\pcol@getcurrpage! otherwise, and
% reinitialize parameters for floats by \!\pcol@floatplacement! before
% returning to the invoker.
% 
%    \begin{macrocode}
\def\pcol@opcol{%
  \pcol@Fb
  \@next\@currbox\@freelist{\global\setbox\@currbox\vbox to\@colht{%
      \boxmaxdepth\@maxdepth
      \pcol@clearcst@unvbox\@outputbox}}\pcol@ovf
  \pcol@Fe{opcol}%
  \expandafter\@cons\csname pcol@shipped\number\pcol@currcol\endcsname\@currbox
  \ifnum\pcol@currcol=\z@ \pcol@setpageno \fi
  \pcol@nextpage
  \pcol@checkshipped
  \if@tempswa \pcol@outputcolumns\z@ \fi
  \ifnum\pcol@page>\pcol@toppage \pcol@startpage
  \else                          \pcol@getcurrpage
  \fi
  \pcol@floatplacement}

%    \end{macrocode}
% \end{macro}
% 
% \begin{macro}{\pcol@setpageno}
% \changes{v1.33-2}{2016/11/19}
% 	{Add \cs{let}\cs{@elt}\cs{relax} before \cs{edef} of \cs{reserved@a}
%	 for the sake of clarity.}
% \begin{macro}{\pcol@setpnoelt}
% \changes{v1.0}{2011/10/10}
%	{Rename \cs{ifpcol@textonly} as \cs{ifpcol@nospan}.}
% \changes{v1.2-2}{2013/05/11}
%	{Completely recode reflecting the redesign of page context.}
% \changes{v1.3-2}{2013/09/17}
%	{Add an operation to increment $\string\mathit{page}(p)$ by two for
%	 non-paired parallel-paging.}
% \changes{v1.3-3}{2013/09/17}
%	{Revise reflecting the new page context element $\pi^s(p)$, and add a
%	 invoker \cs{pcol@makecol}.}
% \changes{v1.3-4}{2013/09/17}
%	{Revise reflecting the new page context element $\pi^m(p)$.}
% 
% The macro \!\pcol@setpageno! is invoked from \!\pcol@opcol! when it
% proceeses the \colpage{} of the first column $c=0$ to fix the page number
% $\page(p)\gets\counter{page}=\!\c@page!$ of the page $p=\!\pcol@page!$
% having the \colpage{}.  It is also invoked from \!\pcol@output@switch!
% when it leaves from the first column to reflect a jump of \counter{page}
% made in the column building.  In both cases, the macro lets $\pp^p(q)$
% have $\page'(q)=\page(p)+(q-p)$, except for the case of \npaired{}
% \parapag{}ing in which $\page'(q)=\page(p)+2(q-p)$ instead, for all
% $q\in[p,\ptop]$.
% 
% Since we possibly have to update $\pp(q)$ such that $q\geq p$, at first we
% temporarily let
% $\Uidx\PPP=(\PP,\pp(\ptop))=\!\pcol@pages!\!\pcol@currpage!$ empty
% after copying its original value into $\mathit{\Pi}'=\!\reserved@a!$.
% Then we scan $\pi'(q)\in\mathit{\Pi}'$ for all $q\in[\pbase,\ptop]$ by
% applying \!\pcol@setpnoelt! to each $\pi'(q)$ giving its five components
% to the macro, so that the macro updates $\pp(q)$ by \!\pcol@defcurrpage!
% letting $\pp^p(q)=\page'(q)$ if $q\geq p$, or equivalently $p-q\leq0$.
% Note that we let \!\c@page!  have $\page'(q)$, but this assignment is
% temporary and \!\c@page! will regain the value $\page(p)$ after
% \!\pcol@setpageno! finishes.
% 
%    \begin{macrocode}
\def\pcol@setpageno{\begingroup
  \@tempcnta\pcol@page \advance\@tempcnta-\pcol@basepage
  \let\@elt\relax \edef\reserved@a{\pcol@pages\pcol@currpage}%
  \global\let\pcol@pages\@empty \global\let\pcol@currpage\@empty
  \let\@elt\pcol@setpnoelt \reserved@a
  \endgroup}
\def\pcol@setpnoelt#1#2#3#4#5{%
  {\let\@elt\relax \xdef\pcol@pages{\pcol@pages\pcol@currpage}}%
  \ifnum\@tempcnta>\z@ \gdef\pcol@currpage{\@elt{#1}#2#3{#4}{#5}}%
  \else \pcol@defcurrpage{\number\c@page}{#2}{#3}{#4}{#5}%
    \advance\c@page\@ne
    \ifpcol@paired\else \advance\c@page\@ne \fi
  \fi
  \advance\@tempcnta\m@ne}
%    \end{macrocode}
% \end{macro}\end{macro}
% 
% \begin{macro}{\pcol@defcurrpage}
% \changes{v1.2-2}{2013/05/11}
%	{Introduced by the redesign of page context, partly replacing
%	 \cs{pcol@setordpage} which once we call \cs{pcol@settextpage}.}
% \changes{v1.3-3}{2013/09/17}
%	{Revise reflecting the new page context element $\pi^s(p)$, and add a
%	 invoker \cs{pcol@makecol}.}
% \changes{v1.3-4}{2013/09/17}
%	{Revise reflecting the new page context element $\pi^m(p)$.}
% 
% The macro
% $\!\pcol@defcurrpage!
% \Arg{\pp^p(p)}\arg{\pp^i(p)}\arg{\pp^f(p)}\Arg{\pp^s(p)}\Arg{\pp^m(p)}$ 
% is invoked from \!\pcol@makecol! to update $\pp^s(\ptop)$,
% \!\pcol@setpnoelt! to update $\pp^p(p)$, \!\pcol@startpage! to initialize
% a newly created page, \!\pcol@output@start! to initialize a \spage{},
% \!\pcol@output@switch! to update $\pp^s(\ptop)$ and/or $\pp^f(\ptop)$, and
% \!\pcol@setmpbelt! to update $\pp^m(p)$.  The macro \!\xdef!ines
% \!\pcol@currpage! letting it have the \pctext{} $\pp(p)$ given by the
% arguments.
% 
%    \begin{macrocode}
\def\pcol@defcurrpage#1#2#3#4#5{{%
  \let\@elt\relax \xdef\pcol@currpage{\@elt{#1}#2#3{#4}{#5}}}}

%    \end{macrocode}
% \end{macro}
% 
% \begin{macro}{\pcol@nextpage}
% \changes{v1.2-7}{2013/05/11}
%	{Remove unnecessary scan of $\pi(p_t)$.}
% \begin{macro}{\pcol@nextpelt}
% \changes{v1.2-2}{2013/05/11}
%	{Revise reflecting the redesign of page context.}
% \changes{v1.3-3}{2013/09/17}
%	{Revise reflecting the new page context element $\pi^s(p)$, and add a
%	 invoker \cs{pcol@makecol}.}
% \changes{v1.3-4}{2013/09/17}
%	{Revise reflecting the new page context element $\pi^m(p)$.}
% \changes{v1.3-6}{2013/09/17}
%	{Fix the bug that $\pi^h(q)$ is not referred correctly.}
% 
% The macro \!\pcol@nextpage! is invoked solely in \!\pcol@opcol! to let
% $p$ be $p+k+1$ where $k$ is the number of \fpage{}s directly following
% $p$, i.e., $k=\Abs{\Set{q>p}{p<\forall q'\leq q:\pp(q')^h<0}}$.  For
% this update, the macro scans $\pp(q)\in\PP$ for all $q\In\pbase\ptop$
% applying \!\pcol@nextpelt! to $\pp(q)$, to perform the following where
% $p_0$ is $p$ before update and $f=\CSIndex{if@tempswa}$ being $\true$ at
% initial, to let $p\gets p+k$, and then increments $p$ to have $p+k+1$.
% $$
% \<p,f\>\gets\cases{
% 	\<p,f\>&	$q\leq p_0$\cr
% 	\<p{+}1,f\>&	$q>p_0\;\land\;f\;\land\;\pp^i(q)\neq\bot\;\land\;
%			\pp^h(q)<0$\cr
%	\<p,\false\>&	otherwise}
% $$
% 
%    \begin{macrocode}
\def\pcol@nextpage{\begingroup
  \@tempcnta\pcol@page \advance\@tempcnta-\pcol@basepage
  \@tempswatrue
  \let\@elt\pcol@nextpelt \pcol@pages
  \global\advance\pcol@page\@ne
  \endgroup}
\def\pcol@nextpelt#1#2#3#4#5{%
  \ifnum\@tempcnta<\z@
    \ifvoid#2\@tempswafalse
    \else\ifdim\dimen#2<\z@
      \if@tempswa \global\advance\pcol@page\@ne \fi
    \else \@tempswafalse
    \fi\fi
  \fi
  \advance\@tempcnta\m@ne}

%    \end{macrocode}
% \end{macro}\end{macro}
% 
% \begin{macro}{\pcol@checkshipped}
% The macro \!\pcol@checkshipped! is invoked solely in \!\pcol@opcol! to let
% \CSIndex{if@tempswa} be $\true$ iff
% $S_c=|\pcol@shipped|{\cdot}c\neq\emptyset$
% 
% \SpecialArrayIndex{c}{\pcol@shipped}
% 
% for all $c\In0\C$ to mean the oldest page $\pbase$ is ready to be shipped
% out.
% 
%    \begin{macrocode}
\def\pcol@checkshipped{\@tempswatrue
  \@tempcnta\z@ \@whilenum\@tempcnta<\pcol@ncol\do{%
    \expandafter\ifx\csname pcol@shipped\number\@tempcnta\endcsname\@empty
      \@tempswafalse \fi
   \advance\@tempcnta\@ne}}

%    \end{macrocode}
% \end{macro}
% 
% \KeepSpace{2}
% \begin{macro}{\pcol@getcurrpage}
% \changes{v1.3-4}{2013/09/17}
%	{Add a user \cs{pcol@addmarginpar}.}
% \begin{macro}{\pcol@getpelt}
% \changes{v1.2-2}{2013/05/11}
%	{Revise reflecting the redesign of page context.}
% \begin{macro}{\pcol@getpinfo}
% \changes{v1.0}{2011/10/10}
%	{Rename \cs{ifpcol@textonly} as \cs{ifpcol@nospan}.}
% \changes{v1.2-2}{2013/05/11}
%	{Revise reflecting the redesign of page context.}
% \begin{macro}{\pcol@getcurrpinfo}
% \changes{v1.2-2}{2013/05/11}
%	{Revise reflecting the redesign of page context.}
% \changes{v1.3-3}{2013/09/17}
%	{Revise reflecting the new page context element $\pi^s(p)$, and add a
%	 invoker \cs{pcol@makecol}.}
% \changes{v1.3-4}{2013/09/17}
%	{Revise reflecting the new page context element $\pi^m(p)$.}
% 
% The macro \!\pcol@getcurrpage! is invoked in \!\pcol@opcol!,
% \!\pcol@restartcolumn!, \!\pcol@addmarginpar!, \!\pcol@flushcolumn! and
% \!\pcol@freshpage! to let;
% 
% \begin{eqnarray*}
% &&\!\c@page!=\pp^p(p)\quad
%   \!\@colht!=\pp^h(p)\quad
%   \!\topskip!=\pp^t(p)\quad
%   \CSIndex{ifpcol@nospan}=(\pp^i(p)=\bot)\\
% &&\Midx{\!\pcol@spanning!}=\pp^i(p)\quad
%   \Midx{\!\pcol@footins!}=\pp^f(p)\quad
%   \Midx{\!\pcol@sptextlist!}=\pp^s(p)\\
% &&\Midx{\!\pcol@mparbottom!}=\pp^m(p)
% \end{eqnarray*}
% 
% for $p=\!\pcol@page!\in[p_b,p_t]$.  To do that, the macro scans all
% $\pp(q)\in\PPP=(\PP,\pp(\ptop))$ applying \!\pcol@getpelt! to
% $\pp(q)=
% \Arg{\pp^p(q)}\arg{\pp^i(q)}\arg{\pp^f(q)}\Arg{\pp^s(q)}\Arg{\pp^m(q)}$
% to invoke
% $$
% \!\pcol@getpinfo!
% \Arg{\pp^p(q)}\arg{\pp^i(q)}\arg{\pp^f(q)}\Arg{\pp^s(q)}\Arg{\pp^m(q)}
% \arg{pg}\arg{ch}\arg{ts}
% $$
% with the following arguments for \!\global! assignments, if $q=p$.
% $$
% \arg{pg}=\!\global!\!\c@page!\quad
% \arg{ch}=\!\global!\!\@colht!\quad\arg{ts}=\!\global!\!\topskip!
% $$
% Then the macro \!\pcol@getpinfo! do the obvious assighments to
% \!\pcol@spanning!, \!\pcol@footins!, \!\pcol@sptextlist!,
% \!\pcol@mparbottom! and $\arg{pg}$, and the following
% condintional assignments.
% 
% \begin{eqnarray*}
% &&\<\arg{ch},\arg{ts},\CSIndex{ifpcol@nospan}\>=
% \left\{\begin{array}{llll}
%	\<\!\textheight!,&\!\pcol@topskip!,&\true\>&\quad\pp^i(q)=\bot\\
%	\<\pp^h(q),&\pp^t(q),&\false\>&\quad\pp^i(q)\neq\bot\end{array}\right.
% \end{eqnarray*}
% 
% \begin{Sloppy}{2800}\noindent
% The other macro $\!\pcol@getcurrpinfo!\arg{pg}\arg{ch}\arg{ts}$ is invoked
% in \!\pcol@makecol!, \!\pcol@startpage!, \!\pcol@output@switch!,
% \!\pcol@sync!, \!\pcol@flushcolumn! and \!\pcol@make~flushedpage! do the
% similar assignments using \!\pcol@getpinfo!, but it is not for $\pp(p)$
% but for $\pp(\ptop)={}$\!\pcol@currpage!.  The macro \!\pcol@getpinfo!
% also has a direct invoker \!\pcol@outputelt!.
% \end{Sloppy}
% 
%    \begin{macrocode}
\def\pcol@getcurrpage{\begingroup
  \@tempcnta\pcol@page \advance\@tempcnta-\pcol@basepage
  \let\@elt\pcol@getpelt \pcol@pages\pcol@currpage
  \endgroup}
\def\pcol@getpelt#1#2#3#4#5{%
  \ifnum\@tempcnta=\z@
    \pcol@getpinfo{#1}#2#3{#4}{#5}%
                  {\global\c@page}{\global\@colht}{\global\topskip}%
  \fi
  \advance\@tempcnta\m@ne}
\def\pcol@getpinfo#1#2#3#4#5#6#7#8{\pcol@nospantrue
  \gdef\pcol@spanning{#2}\gdef\pcol@footins{#3}\gdef\pcol@sptextlist{#4}%
  \gdef\pcol@mparbottom{#5}%
  #6#1\relax
  \ifvoid#2\relax #7\textheight #8\pcol@topskip
  \else #7\dimen#2\relax #8\skip#2\relax \pcol@nospanfalse
  \fi}
\def\pcol@getcurrpinfo{%
  \edef\reserved@a{\expandafter\@cdr\pcol@currpage\@nil}%
  \expandafter\pcol@getpinfo\reserved@a}

%    \end{macrocode}
% \end{macro}\end{macro}\end{macro}\end{macro}
% 
% \begin{macro}{\pcol@floatplacement}
% \changes{v1.0}{2011/10/10}
%	{Add initialization of \cs{pcol@textfloatsep}.}
% \changes{v1.3-4}{2013/09/17}
%	{Remove clearing operation on \cs{@mparbottom} because it is no
%	 longer in column-context.}
% 
% The macro \!\pcol@floatplacement! is invoked from \!\pcol@opcol!,
% \!\pcol@output@start!, \!\pcol@flushcolumn!, \!\pcol@freshpage! and
% \!\pcol@output@end! to reinitialize the parameters of \cwise{} float
% placement at the beginning of a \colpage{} or that of \postenv.  The macro
% lets \!\@textfloatsheight! be 0 and then invokes \!\@floatplacement!, as
% \!\@opcol! does its tail\footnote{
% 
% But $\!\@mparbottom!=0$ is not done because it is meaningless now.}.
% 
% In addition, the macro lets \!\pcol@textfloatsep! be \!\maxdimen! to mean
% the new \colpage{} does not have \sync{}ation point at initial.
% 
%    \begin{macrocode}
\def\pcol@floatplacement{%
  \global\@textfloatsheight\z@ \global\pcol@textfloatsep\maxdimen
  \@floatplacement}

%    \end{macrocode}
% \end{macro}
% 
% 
% 
% \section{Starting New Page}
% \label{sec:imp-startpage}
% 
% \begin{macro}{\pcol@startpage}
% \changes{v1.0}{2011/10/10}
%	{Add assignment of \cs{pcol@firstprevdepth} to be \cs{relax}.}
% \changes{v1.0}{2011/10/10}
%	{Rename \cs{pcol@maxpage} as \cs{pcol@toppage}.}
% \changes{v1.0}{2011/10/10}
%	{Rename \cs{pcol@settextpage} as \cs{pcol@setordpage}.}
% \changes{v1.2-2}{2013/05/11}
%	{Revise reflecting the redesign of page context.}
% \changes{v1.3-2}{2013/09/17}
%	{Duplicate \cs{stepcounter} of \cs{c@page} if non-paired
%	 parallel-paging is in effect.} 
% \changes{v1.3-3}{2013/09/17}
%	{Revise reflecting the new page context element $\pi^s(p)$.}
% \changes{v1.3-4}{2013/09/17}
%	{Revise reflecting the new page context element $\pi^m(p)$.}
% \changes{v1.32-3}{2015/10/10}
% 	{Add $\hbox{\cs{f@depth}}\EQ0$ to override
%	 $\hbox{\cs{f@depth}}\EQ\hbox{\texttt{1sp}}$ done by
%	 \cs{@dblfloatplacement}.}
% \changes{v1.32-3}{2015/10/10}
% 	{Modify the code to apply \cs{@sdblcolelt} to \cs{@dbldeferlist} so
%	 as to work with both 2015 (or newer) and 2014 (or older) versions
%	 of \LaTeX.}
% \changes{v1.32-2}{2015/10/10}
% 	{Add \cs{pcol@Fb}/\cs{pcol@Fe} pair(s).}
% 
% The macro \!\pcol@startpage! is invoked from \!\pcol@opcol! with
% $\!\pcol@currpage!=\pp(p{-}1)$ to start a new page $p=\!\pcol@page!$, or
% from \!\pcol@output@start! with a too large \preenv{} or
% \!\pcol@freshpage! with $\!\pcol@currpage!=|{}|$ and $\!\c@page!=\page(p)$
% to start a new page $p=0$.
% 
% First, we let $\!\pcol@firstprevdepth!=\!\relax!$ to mean we have (had)
% left from the \spage{} so that \!\pcol@output@end! will be informed of
% that.  Next we let $\ptop=p$ and then, if invoked from \!\pcol@opcol!,
% obtain $\pp^p(p{-}1)$ by \!\pcol@getcurrpinfo! to have $\page(p{-}1)$ in
% \!\c@page!, and let $\PP\gets\PPP=(\PP,\pp(p{-}1))$ with
% $\pp^f(p{-}1)=\!\pcol@currfoot!$ into which \!\pcol@makecol! saved
% \Scfnote{}s if any.  Next we lets $\!\c@page!=\page(p{-}1)+1$ unless
% \npaired{} \parapag{}ing is in effect or in other words if
% $\CSIndex{ifpcol@paired}=\true$, or $\!\c@page!=\page(p{-}1)+2$ otherwise,
% by \!\stepcounter!.  Then we let $\!\@colht!=\!\textheight!$ as the base
% value without \spanning, and $\!\topskip!=\!\pcol@topskip!$ because the
% new page is the second or succeeding one built in \env{paracol}
% environment.
% 
%    \begin{macrocode}
%% Starting New Page

\def\pcol@startpage{%
  \global\let\pcol@firstprevdepth\relax
  \global\pcol@toppage\pcol@page
  \ifx\pcol@currpage\@empty\else
    \pcol@getcurrpinfo{\global\c@page}\@tempdima\@tempskipa
    \@cons\pcol@pages
      {{\number\c@page}\pcol@spanning\pcol@currfoot
       {\pcol@sptextlist}{\pcol@mparbottom}}%
    \stepcounter{page}\ifpcol@paired\else \stepcounter{page}\fi
  \fi
  \global\@colht\textheight
  \global\topskip\pcol@topskip
%    \end{macrocode}
% 
% Then, we build \fpage{}s if any as follows.  First we invoke
% \!\@dblfloatplacement! to reinitialize the parameters for \pwise{} float
% placement.  In addition, we let $\!\f@depth!=0$ to nullify the
% setting $\!\f@depth!=|1sp|$ possiblly done by \!\@dblfloatplacement! as
% discussed in the item-(\ref{item:ovv-float-@dblfloatplacement}) of
% \secref{sec:imp-ovv-float}.  Then we repeat \!\@tryfcolumn!  giving it
% \!\@dbldeferlist!  having \pwise{} floats not contributed to previous
% pages yet, while $\CSIndex{if@fcolmade}=\true$ meaning it builds \fpage{}s
% in \!\@outputbox!.  For each \fpage{}, we acquire an \!\insert!  from
% \!\@freelist! by \!\@next! for $\pp^i(\ptop)$ to let it have the
% followings to represent the \fpage{}.
% 
% \begin{eqnarray*}
% \pp^p(\ptop)&=&\!\c@page!\quad
% \pp^b(\ptop)=\!\@outputbox!\quad
% \pp^h(\ptop)=-\!\maxdimen!\quad
% \pp^t(\ptop)=\!\pcol@topskip!\\
% \pp^f(\ptop)&=&\bot\quad
% \pp^s(\ptop)=\emptyset\quad
% \pp^m(\ptop)=\emptyset
% \end{eqnarray*}
% 
% We also increment $p$ and $\ptop$, and also \!\c@page! by one or two
% according to \CSIndex{ifpcol@paired} by \!\stepcounter!, to let them have
% the values for the page following the \fpage{}s.
% 
%    \begin{macrocode}
  \@dblfloatplacement \let\f@depth\z@
  \@tryfcolumn\@dbldeferlist
  \@whilesw\if@fcolmade\fi{%
    \pcol@Fb
    \@next\@currbox\@freelist{%
      \global\setbox\@currbox\box\@outputbox}\pcol@ovf
    \pcol@Fe{startpage(fcol)}%
    \global\dimen\@currbox-\maxdimen
    \global\skip\@currbox\pcol@topskip
    \@cons\pcol@pages{{\number\c@page}\@currbox\voidb@x{}{}}%
    \stepcounter{page}\ifpcol@paired\else \stepcounter{page}\fi
    \global\advance\pcol@page\@ne \global\pcol@toppage\pcol@page
    \@tryfcolumn\@dbldeferlist}%
%    \end{macrocode}
% 
% Next, we copy \!\@dbldeferlist! containing \pwise{} floats which
% could not be included in \fpage{}s to \!\reserved@b!, clear the list, and
% then scan the copied list by applying \!\@sdblcolelt! to each list
% element to invoke \!\@addtodblcol! for adding the element to \!\@dbltoplist!
% or keeping it in \!\@dbldeferlist! or \!\@deferlist! depending on \LaTeX's
% version, as \LaTeX's \!\@startdblcolumn!  does.  In addition, as discussed
% in item-(\ref{item:ovv-float-@addtodblcol}) of
% \secref{sec:imp-ovv-float}, we also clear \!\@deferlist! after saving it in
% \!\reserved@c! prior to the scan, and then after the scan we concatenate
% \!\@dbldeferlist! and \!\@deferlist! to let the former have the result and
% restore \!\@deferlist! from \!\reserved@c!.
% 
% Then If this scan results in empty \!\@dbltoplist! to mean the new page
% does not have any \spanning, we invoke \!\pcol@defcurrpage! with
% $\pp^i(\ptop)=\pp^f(\ptop)=\bot$ and $\pp^s(\ptop)=\pp^m(\ptop)=\emptyset$
% so that $\pp(\ptop)$ reprsents a page perfectly empty.
% 
%    \begin{macrocode}
  \begingroup
    \let\reserved@b\@dbldeferlist \let\reserved@c\@deferlist
    \global\let\@dbldeferlist\@empty \global\let\@deferlist\@empty
    \let\@elt\@sdblcolelt
    \reserved@b
    \let\@elt\relax \xdef\@dbldeferlist{\@dbldeferlist\@deferlist}%
    \global\let\@deferlist\reserved@c
  \endgroup
  \ifx\@dbltoplist\@empty
    \pcol@defcurrpage{\number\c@page}\voidb@x\voidb@x{}{}%
%    \end{macrocode}
% 
% Otherwise, i.e., \!\@dbltoplist! is not empty, we scan all elements in it
% by letting $\!\@elt!=\!\@comdblflelt!$ to have all \pwise{} floats in
% \!\@tempboxa!.  Then, after returning all elements to \!\@freelist!, we
% acquire an \!\insert! from \!\@freelist! to be $\pp^i(\ptop)$ by \!\@next!
% and store the contents of \!\@tempboxa! in $\pp^b(\ptop)$ after removing
% the last vertical skip \!\dblfloatsep! and then adding \!\dblfigrule! and
% the vertical skip \!\dbltextfloatsep!.  The other elements of $\pp(\ptop)$
% are set as follows to represent the page with \spanning{} which makes the
% height of each column \!\@colht! shrunk from its initial value
% \!\textheight!  by the series of \!\@addtodblcol!.
% 
% \begin{eqnarray*}
% \pp^p(\ptop)&=&\!\c@page!=\page(\ptop)\quad
% \pp^h(\ptop)=\!\@colht!\quad
% \pp^t(\ptop)=\!\pcol@topskip!\quad
% \pp^f(\ptop)=\bot\\
% \pp^s(\ptop)&=&\emptyset\quad
% \pp^m(\ptop)=\emptyset\quad
% \end{eqnarray*}
% 
% Finally, regardless of the existence of the \pwise{} floats, we let
% $\!\pcol@footins!=\bot$ to mean the \tpage{} does not have any \Scfnote{}s,
% so far if footnote typesetting is \scfnote, or never otherwise.
% 
%    \begin{macrocode}
  \else
    \setbox\@tempboxa\vbox{}%
    \begingroup
      \let\@elt\@comdblflelt
      \@dbltoplist
      \let\@elt\relax
      \pcol@Fb
      \xdef\@freelist{\@freelist\@dbltoplist}%
      \pcol@Fe{startpage(dbltop)}%
      \global\let\@dbltoplist\@empty
      \pcol@Fb
      \@next\@currbox\@freelist{\global\setbox\@currbox\vbox{%
        \unvbox\@tempboxa \vskip-\dblfloatsep \dblfigrule
        \vskip\dbltextfloatsep}}\pcol@ovf
      \pcol@Fe{startpage(spanning)}%
      \global\dimen\@currbox\@colht
      \global\skip\@currbox\pcol@topskip
      \pcol@defcurrpage{\number\c@page}\@currbox\voidb@x{}{}%
    \endgroup
  \fi
  \gdef\pcol@footins{\voidb@x}}

%    \end{macrocode}
% \end{macro}
% 
% 
% 
% \section{Shipping Page Out}
% \label{sec:imp-outputpage}
% 
% \begin{macro}{\pcol@outputcolumns}
% \changes{v1.0}{2011/10/10}
%	{Rename \cs{ifpcol@stopoutput} as \cs{ifpcol@outputflt}.}
% \changes{v1.3-2}{2013/09/17}
%	{Rename \cs{pcol@outputpage} as \cs{pcol@outputcolumns}.}
% 
% The macro \!\pcol@outputcolumns!$\arg{all}$ is invoked from \!\pcol@opcol!
% with $\arg{all}=0$ to ship out the page $\pbase$ and \fpage{}s following
% it if any, or \!\pcol@sync! with $\arg{all}=1$ to ship out all pages in
% $\PP$.  It copies $\PP=\!\pcol@pages!$ into $\mathit{\Pi}'=\!\reserved@b!$
% and clear $\PP$ once to remove pages shipped out from it.  Then, after
% initializing $f_o=\CSIndex{if@tempswa}=\true$ to ship out (the first)
% ordinary page and $f_f=\CSIndex{ifpcol@outputflt}=\true$ to ship out
% \fpage{}s (following the first page), it scans all
% $\pp(q)\in\mathit{\Pi}'$ applying $\!\pcol@outputelt!\arg{all}$ to
% $\pp(q)$ to ship it out or keep it in $\PP$.
% 
%    \begin{macrocode}
%% Shipping Page Out

\def\pcol@outputcolumns#1{\begingroup
  \def\@elt{\pcol@outputelt#1}\@tempswatrue \pcol@outputflttrue
  \let\reserved@b\pcol@pages \gdef\pcol@pages{}%
  \reserved@b
  \endgroup}
%    \end{macrocode}
% \end{macro}
% 
% \begin{macro}{\pcol@outputelt}
% \changes{v1.2-2}{2013/05/11}
%	{Revise reflecting the redesign of page context.}
% \changes{v1.3-3}{2013/09/17}
%	{Revise reflecting the new page context element $\pi^s(p)$.}
% \changes{v1.3-4}{2013/09/17}
%	{Revise reflecting the new page context element $\pi^m(p)$.}
% 
% The macro $\!\pcol@outputelt!
% \arg{all}\Arg{\pp^p(q)}\arg{\pp^i(q)}\arg{\pp^f(q)}
% \Arg{\pp^s(q)}\Arg{\pp^m(q)}$ ships out the ordinary or \fpage{} $q$ if
% $f_o=\true$ or $f_f=\true$ respectively.  After initializing
% \!\@outputbox! to be $\bot$, we retrive the page $q$'s information by
% \!\pcol@getpinfo! to have $\page(q)$ in \!\c@page! locally, \!\textheight!
% or $\pp^h(q)$ in $h=\!\@tempdima!$,
% $f_{\it{}ns}=\CSIndex{ifpcol@nospan}=(\pp^i(q)=\bot)$, and
% $\!\pcol@footins!={\pp^f(q)}$.
% 
%    \begin{macrocode}
\def\pcol@outputelt#1#2#3#4#5#6{%
  \setbox\@outputbox\box\voidb@x
  \pcol@getpinfo{#2}#3#4{#5}{#6}\c@page\@tempdima\@tempskipa
%    \end{macrocode}
% 
% Then, we do one of the followings according to $h$ and $f_o$.
% 
% \begin{itemize}
% \item $h<0$\\
% It means $q$ is a \fpage{}.  If $f_f=\true$, we let \!\@outputbox! have
% $\pp^b(q)$ to be shipped out, paint its \bground{} with $\bgc_{\{F,f\}}$
% by \!\pcol@bg@paintbox! letting the basic height \!\pcol@bg@floatheight!
% of the paining region 
% $\bgr_{\{F,f\}}$ be
% $\HT=\!\pcol@bg@textheight!=\!\textheight!+\!\maxdepth!$, and return
% $\!\insert!{\cdot}\pp^i(q)$ to \!\@freelist! by \!\@cons! because it is no
% longer necessary.  Then if $\CL<\C$ to mean \parapag{}ing is in effect, we
% let \!\pcol@rightpage! be an empty box but paint its \bground{} too, because
% the right counterpart of left parallel float page should be always blank.
% Note that we temporarily increment $\page(q)$ by one for \npaired{} right
% \parapag{}ee so that the paiting macro performs page-parity dependent
% operations correctly.
% 
% \Index{parallel-paging}
% \Index{float page}
% 
% On the other hand if $f_f=\false$, we simply return $\pp(q)$ back to $\PP$.
% \end{itemize}
% 
% \changes{v1.0}{2011/10/10}
%	{Rename \cs{ifpcol@stopoutput} as \cs{ifpcol@outputflt}.}
% \changes{v1.3-3}{2013/09/17}
%	{Add painting of page-wise float page.}
% \changes{v1.3-2}{2013/09/17}
%	{Add building a empty right parallel float page.}
% \changes{v1.32-2}{2015/10/10}
% 	{Add \cs{pcol@Fb}/\cs{pcol@Fe} pair(s).}
% 
%    \begin{macrocode}
  \ifdim\@tempdima<\z@
    \ifpcol@outputflt
      \def\pcol@bg@floatheight{\pcol@bg@textheight}%
      \setbox\@outputbox\vbox to\textheight{%
        \pcol@bg@paintbox{Ff}\unvbox\pcol@spanning}%
      \pcol@Fb
      \@cons\@freelist\pcol@spanning
      \pcol@Fe{outputelt(spanning)}%
      \ifnum\pcol@ncolleft<\pcol@ncol
        \setbox\pcol@rightpage\vbox to\textheight{%
          \ifpcol@paired\else \advance\c@page\@ne \fi
          \pcol@bg@paintbox{Ff}\vfil}%
      \fi
    \else
      \@cons\pcol@pages{{#2}#3#4{#5}{#6}}%
    \fi
%    \end{macrocode}
% 
% \begin{itemize}
% \item $h\geq0\;\land\;f_o=\true$\\
% It means $q$ is a non-\fpage{} to be shipped out.  If $\arg{all}=0$, we
% let $f_o=\false$ to keep succeeding non-\fpage{}s from being shipped out.
% Then we build the ship-out image of the right \parapag{}e $q$ in
% \!\pcol@rightpage! by \!\pcol@ioutputelt! giving it the box and the column
% range $\LBRP\CL\C$ if $\CL<\C$ to mean \parapag{}ing, and then that of the
% left \parapag{}e in \!\@outputbox! by \!\pcol@ioutputelt! again but giving
% it $\LBRP0\CL$ and \!\@outputbox!.  Note that the right-first left-second
% order is essential, because in the process to build right \parapag{}e we
% have to examine the existence of $\pp^i(q)$ and $\pp^f(q)$ and then refer
% to their height and depth to make the region corresponding to them blank,
% while the boxes of these \!\insert!s are made void in the building process
% of the left \parapag{}e obviously.
% 
% Then after the ship-out image building, we \!\global!ly let
% $\CSIndex{ifpcol@firstpage}=\false$ to tell \!\pcol@ioutputelt! and
% \!\pcol@makeflushedpage! that the pages they build are no longer first of
% a \env{paracol} environment and thus $\pp^i(q)$ should have \pwise{}
% floats rather than \preenv{} hereafter.
% \end{itemize}
% 
% \changes{v1.0}{2011/10/10}
%	{Rename \cs{ifpcol@textonly} as \cs{ifpcol@nospan}.}
% \changes{v1.1}{2012/05/11}
% 	{Use $\cs{pcol@columnwidth}{\cdot}c$ instead of \cs{columnwidth} as
%	 the width of $c$'s \cs{hbox}.}
% \changes{v1.2-4}{2013/05/11}
%	{Add column-swapping for even pages if specified.}
% \changes{v1.2-2}{2013/05/11}
%	{Add ship-out of page-wise footnote.}
% \changes{v1.2-7}{2013/05/11}
%	{Add $\cs{boxmaxdepth}\EQ\cs{@maxdepth}$ for depth capping.}
% \changes{v1.3-2}{2013/09/17}
%	{Move the core of ship-out image building to \cs{pcol@ioutputelt} for
% 	 parallel pageing.}
% 
%    \begin{macrocode}
  \else\if@tempswa
    \ifnum#1=\z@ \@tempswafalse \fi
    \ifnum\pcol@ncolleft<\pcol@ncol
      \pcol@Logstart{\pcol@outputelt{right}}%
      \pcol@ioutputelt\pcol@ncolleft\pcol@ncol\pcol@rightpage
      \pcol@Logend{\pcol@outputelt{right}}%
    \fi
    \pcol@Logstart{\pcol@outputelt{left}}%
    \pcol@ioutputelt\z@\pcol@ncolleft\@outputbox
    \pcol@Logend{\pcol@outputelt{left}}%
    \global\pcol@firstpagefalse
%    \end{macrocode}
% 
% \begin{itemize}
% \item $h\geq0\;\land\;f_o=\false$\\
% It means $q$ is a non-\fpage{} to be kept.  Therefore, we let
% $f_f=\false$ to keep \fpage{}s following it from being shipped out.  Then
% we return $\pp(q)$ to $\PP$ by \!\@cons!.
% \end{itemize}
% 
%    \begin{macrocode}
  \else
    \pcol@outputfltfalse
    \@cons\pcol@pages{{#2}#3#4{#5}{#6}}%
  \fi\fi
%    \end{macrocode}
% 
% Finally, if $\!\@outputbox!\neq\bot$ to mean $\pp(q)$ is to be
% shipped out, we invoke \!\@outputpage! to do it and increment $\pbase$ to
% let it has $q+1$.  Note that since we have let $\!\c@page!=\page(q)$, the
% direct and indirect references to it in \!\@outputpage! are correctly done.
% Also note that the \!\global! increment of it by \!\stepcounter! in
% \!\@outputpage! will be overriden by the \!\global! assignment to it done
% by \!\pcol@startpage! or \!\pcol@getcurrpage! invoked from \!\pcol@opcol!
% if $\arg{all}=0$, or by \!\pcol@getcurrpinfo! invoked from \!\pcol@sync!
% otherwise.
% 
%    \begin{macrocode}
  \ifvoid\@outputbox\else
    \global\advance\pcol@basepage\@ne \@outputpage
  \fi}

%    \end{macrocode}
% \end{macro}
% 
% \begin{macro}{\pcol@ioutputelt}
% \changes{v1.3-2}{2013/09/17}
%	{Introduced for parallel-paging.}
% \changes{v1.3-3}{2013/09/17}
%	{Add column-separating rule drawing and background painting.}
% \changes{v1.3-4}{2013/09/17}
%	{Add a logic to cope with non-uniform column-separating gaps.}
% 
% The macro $\!\pcol@ioutputelt!\<\Cfrom\>\<\Cto\>\arg{b}$ is invoked solely
% in \!\pcol@outputelt! but can be done twice with
% $(\Cfrom,\Cto,b)=(\CL,\C,\!\pcol@rightpage!)$ if \parapag{}ing is in
% effect and with $(\Cfrom,\Cto,b)=(0,\CL,\!\@outputbox!)$ always, to build
% the ship-out image of the right or left \parapag{}e $q$ in the box $b$
% respectively.
% 
% After opening a \!\vbox! of \!\textheight! tall for $b$, at first we
% increment $\page(q)$ by one for right \parapag{}e if it is \npaired, so
% that painting macros perform page-parity dependent operations correctly.
% 
% Next, we put materials to be shipped out in the box $b$ as follows.
% First, if $\pp^f(q)\neq\bot$ to mean the page $q$ has \Scfnote{}, we paint
% their \bground{} with $\bgc_{\{N,n\}}$ by \!\pcol@bg@paintbox! letting the
% basic height \!\pcol@bg@footnoteheight! of the paining region
% $\bgr_{\{N,n\}}$ be the height-plus-depth of $\pp^f(q)$.
% We do the painting at this earliest stage of the image building in order
% to use the left-top corner of the text area where we are now at as the
% origin for painting, and to let the region may be overlaid by those of
% columns and \csepgap{}s.  We also let $h=\pp^h(q)=\!\@tempdima!$ shrunk by
% the height-plus-depth by \!\pcol@shrinkcolbyfn!.
% 
%    \begin{macrocode}
\def\pcol@ioutputelt#1#2#3{\setbox#3\vbox to\textheight{%
  \ifpcol@paired\else\ifnum#1=\z@\else \advance\c@page\@ne \fi\fi
  \ifvoid\pcol@footins\else
    \def\pcol@bg@footnoteheight{\@elt{\ht\pcol@footins}\@elt{\dp\pcol@footins}}%
    \pcol@bg@paintbox{Nn}%
    \pcol@shrinkcolbyfn\@tempdima\pcol@footins\relax
  \fi
%    \end{macrocode}
%
% Second, if $f_{\it ns}=\false$ to mean $\pp(q)$ has \spanning{} in
% $\pp^b(q)$, we do one of the followings.
% 
% \begin{itemize}
% \item
% If $\Cfrom=0$ to mean the target is the left \parapag{}e, $\pp^b(q)$ is
% put by \!\unvbox!, painting its \bground{} with $\bgc_{\{F,f\}}$ by
% \!\pcol@bg@paintbox! letting the basic height \!\pcol@bg@floatheight!  of
% the paining region $\bgr_{\{F,f\}}$ be the height-plus-depth of $\pp^b(q)$
% if $q$ is not the first page and thus $\pp^b(q)$ has \pwise{} floats.  We
% also return the \!\insert!  $\pp^i(q)$ to \!\@freelist! by \!\@cons!.
% 
% \item
% If $\Cfrom\neq0$ to mean right \parapag{}e and $q$ is the first page,
% \!\pcol@rightpage! has \preenv{} in the right \parapag{}e.  Therefore, we
% simply put the box but making its height and depth equal to those of
% $\pp^b(q)$, without painting because the box has already been painted, and
% with \!\nointerlineskip! to prevent baseline-skip insertion below the box.
% 
% \item
% Otherwise, i.e., $\Cfrom\neq0$ and $q$ is not the first page, we put an
% empty box whose height and depth equal to those of $\pp^b(q)$ by
% \!\pcol@phantom!, with painting as done for floats in the left \parapag{}e
% and with \!\nointerlineskip!.
% \end{itemize}
% 
% Note that after putting \spanning{} and/or painting the \bground{}, we
% temporarily increment \!\topmargin! by the height-plus-depth of
% $\pp^b(q)$, so that painting macros for columns, \csepgap{}s and
% \mctext{}s assume the top edge of column area as that of text area when
% they extend the top edges of their regions upward to the page top.
% 
% \Index{extension of background painting region}
% 
% \changes{v1.32-2}{2015/10/10}
% 	{Add \cs{pcol@Fb}/\cs{pcol@Fe} pair(s).}
% 
%    \begin{macrocode}
  \ifpcol@nospan\else
    \def\pcol@bg@floatheight{%
      \@elt{\ht\pcol@spanning}\@elt{\dp\pcol@spanning}}%
    \@tempdimb\ht\pcol@spanning \advance\@tempdimb\dp\pcol@spanning
    \ifnum#1=\z@
      \ifpcol@firstpage\else \pcol@bg@paintbox{Ff}\fi
      \pcol@Fb
      \@cons\@freelist\pcol@spanning \unvbox\pcol@spanning
      \pcol@Fe{ioutputelt(spanning)}%
    \else\ifpcol@firstpage
      \ht\pcol@rightpage\ht\pcol@spanning
      \dp\pcol@rightpage\dp\pcol@spanning
      \box\pcol@rightpage \nointerlineskip
    \else
      \pcol@bg@paintbox{Ff}\pcol@phantom\pcol@spanning \nointerlineskip
    \fi\fi
    \advance\topmargin\@tempdimb
  \fi
%    \end{macrocode}
%
% Third, we invoke \!\pcol@buildcolseprule! giving it $h$ being $\pp^h(q)$
% but possibly shrunk by \Scfnote{}s, the column range $\LBRP\Cfrom\Cto$,
% and \!\@maxdepth! to mean $q$ is non-\lpage, to draw a \cseprule{}
% possibly broken by the speces for \mctext{}s in the box \!\pcol@tempboxa!
% and to paint the \bground{}s of columns, \csepgap{}s and \mctext{} in the
% box \!\@tempboxa! which we put into $b$ immediately.
% 
% Fourth, we put a \!\hbox! of \!\textwidth! wide having \!\hbox!es
% of $\w_c=|\pcol@columnwidth|{\cdot}c$
% 
% \SpecialArrayIndex{c}{\pcol@columnwidth}
% 
% wide containing $\sigma_c$ followed by \!\hss! for all $c\In\Cfrom\Cto$,
% where $\sigma_c=\!\box!{\cdot}s_c(q)$ being the first element removed from
% $\S_c$ by \!\@next! and then returned to \!\@freelist! by \!\@cons! if it
% is not $\bot$, or \!\voidb@x!  otherwise.  We sepearate \!\hbox!es of
% $\sigma_c$ by making each \!\hbox! preceded by \Midx{\!\pcol@@hfil!} being
% \!\relax! for the first one and the macro \!\pcol@hfil! for others giving
% it an argument $c^g=\!\pcol@colsepid!\in\{c,c{-}1\}$ which we discuss
% shortly to put a gap of $\gap_{c^g}=|\pcol@columnsep|{\cdot}c^g$
% 
% \SpecialArrayIndex{c}{\pcol@columnsep}
% 
% wide optionally having a \cseprule{} and being painted.
% 
% Note that the scanning order of $c\In\Cfrom\Cto$ is usually ascending of
% course, but is descending if \cswap{} is specified and $\page(q)\bmod2=0$.
% For this ordering, we invoke
% $\!\pcol@swapcolumn!\arg{c'}\arg{c}\<\Cfrom\>\<\Cto\>$ to have $c$ for
% the $(c'-\Cfrom)$-th iteration of the scanning where $c'=\!\@tempcnta!$
% and $c=\!\@tempcntb!$.  Another operation done by the macro is to let
% $c^g=c-1$ if swapped or $c^g=c$ otherwise, because if swapped the
% column-$c$ is followed by the gap which follows $c-1$ if not swapped.
% 
% \changes{v1.32-2}{2015/10/10}
% 	{Add \cs{pcol@Fb}/\cs{pcol@Fe} pair(s).}
% 
%    \begin{macrocode}
  \pcol@buildcolseprule\@tempdima#1#2\@maxdepth \unvbox\@tempboxa
  \hb@xt@\textwidth{%
    \let\pcol@@hfil\relax
    \@tempcnta#1\relax \@whilenum\@tempcnta<#2\do{%
      \pcol@swapcolumn\@tempcnta\@tempcntb#1#2\relax
      \expandafter\@next\expandafter\@currbox
        \csname pcol@shipped\number\@tempcntb\endcsname
        \relax{\let\@currbox\voidb@x}%
      \ifvoid\@currbox\else
        \pcol@Fb
        \@cons\@freelist\@currbox
        \pcol@Fe{ioutputelt(page)}%
      \fi
      \expandafter\@tempdima
        \csname pcol@columnwidth\number\@tempcntb \endcsname
      \pcol@@hfil \hb@xt@\@tempdima{\box\@currbox\hss}%
      \edef\pcol@@hfil{\noexpand\pcol@hfil{\pcol@colsepid}}%
     \advance\@tempcnta\@ne}}%
%    \end{macrocode}
%
% Fifth, if $\pp^f(q)\neq\bot$ to mean the page $q$ has \Scfnote{}s, we put
% them at the bottom of \!\@outputbox! by \!\pcol@putfootins!, and return
% $\pp^f(q)$ to \!\@freelist!, if $\Cfrom=0$ meaning left \parapag{}e.
% Otherwise for the right \parapag{}e, we simply put an empty box whose
% height and depth equal to those of $\pp^f(q)$ by \!\pcol@phantom!,
% preceded by a vertical skip of $\!\skip!{\cdot}\pp^f(q)$ and then
% \!\nointerlineskip! to inhibit baseline skip insertion above the box, and
% followed by null \!\vskip! as done in \!\pcol@putfootins!.
% 
% Sixth and finally\footnote{
% 
% Not necessary to be finally, but we placed this assignment at the end of
% the box to make it clear the depth capping is only for the box.},
% 
% we let $\!\boxmaxdepth!=\!\@maxdepth!$ to cap the depth of $b$ which we
% are now closing, as done for each \colpage{} and as expected to be applied
% to \Scfnote{}s.
% 
% \changes{v1.32-2}{2015/10/10}
% 	{Add \cs{pcol@Fb}/\cs{pcol@Fe} pair(s).}
% 
%    \begin{macrocode}
  \ifvoid\pcol@footins\else
    \ifnum#1=\z@
      \pcol@Log\pcol@outputelt{output}\pcol@footins
      \pcol@putfootins\pcol@footins
      \pcol@Fb
      \@cons\@freelist\pcol@footins
      \pcol@Fe{ioutputelt(footins)}%
    \else
      \vskip\skip\pcol@footins \nointerlineskip
      \pcol@phantom\pcol@footins \vskip\z@
    \fi
  \fi
  \boxmaxdepth\@maxdepth}}
%    \end{macrocode}
% \end{macro}
% 
% \begin{macro}{\pcol@phantom}
% \changes{v1.3-2}{2013/09/17}
%	{Introduced for parallel-paging.}
% 
% The macro \!\pcol@phantom!$\<b\>$ is used in \!\pcol@ioutputelt!,
% \!\pcol@makeflushedpage! and \!\pcol@output@end! to put an empty box, whose
% height and depth are equal to that of the argument box $b$ being a kind of
% \pwstuff, into \!\pcol@rightpage! for the right \parapag{}e whose left
% counterpart has $b$ in it.  That is, the macro is used to make a region
% corresponding to $b$ blank.  To put the empty box, we locally let
% \!\@tempboxa! have it setting its height and depth to those of $b$ and
% then put it.
% 
%    \begin{macrocode}
\def\pcol@phantom#1{{%
  \setbox\@tempboxa\vbox{}\ht\@tempboxa\ht#1\dp\@tempboxa\dp#1\box\@tempboxa}}

%    \end{macrocode}
% \end{macro}
% 
% \KeepSpace{1}
% \begin{macro}{\pcol@buildcolseprule}
% \changes{v1.3-3}{2013/09/17}
%	{Introduced for column-separating rule drawing and background
%	 painting for columns, column-separating gaps and spanning texts.}
% \changes{v1.3-4}{2013/09/17}
%	{Introduced for non-uniform column-separating gaps.}
% \changes{v1.34}{2018/05/07}
% 	{Rename \cs{pcol@tempbox} as \cs{pcol@tempboxa}.}
% \begin{macro}{\pcol@buildcselt@S}
% \changes{v1.3-3}{2013/09/17}
%	{Introduced for background under-painting for spanning texts.}
% \begin{macro}{\pcol@buildcselt}
% \changes{v1.3-3}{2013/09/17}
%	{Introduced for column-separating rule drawing and background
%	 painting for columns, column-separating gaps and spanning texts.}
% \changes{v1.3-4}{2013/09/17}
%	{Introduced for non-uniform column-separating gaps.}
% \changes{v1.34}{2018/05/07}
% 	{Rename \cs{pcol@tempbox} as \cs{pcol@tempboxa}.}
% 
% The macro $\!\pcol@buildcolseprule!\<H_{n+1}\>\<\Cfrom\>\<\Cto\>\<d\>$ is
% used in \!\pcol@ioutputelt!, \!\pcol@imakeflushedpage! and
% \!\pcol@iflushfloats!  to build a box containing \cseprule{} possibly
% broken by \mctext{}s and to paint \bground{}s of columns and \csepgap{}s
% for $c\In\Cfrom\Cto$ and \mctext{}s in the \lpage{} ($d=0$) or non-last
% page ($d=\!\@maxdepth!$) $p$ having \colpage{}s of $H_{n+1}$ tall where
% $n=\Abs{\pp^s(p)}$.
% 
% For initializing the drawing and painting process, we let
% $\!\@tempdimb!=H_0+h_0=0$,
% $(\!\pcol@bg@from!,\!\pcol@bg@to!)=(\CBfrom,\CBto)=(\Cfrom,\Cto)$, and
% make boxes $b_r=\!\pcol@tempboxa!$ for the rule and $b_b=\!\@tempboxa!$ for
% the \bground{} empty.  Then we apply
% \!\pcol@buildcselt@S!$\<H_i\>\<h_i\>$ to each element $\spt(H_i,h_i)$ of
% $\pp^s(p)$ to under-paint the \bground{} of each \mctext{} by
% \!\pcol@bg@paintbox! defining its region $\bgr_S(i)$ by letting their top
% edge positions $y_0=\!\pcol@bg@spanningtop!=H_i$, and height
% $y_1-y_0=\!\pcol@bg@spanningheight!=h_i$ if $H_i+h_i<H_{n+1}$ to mean the
% \mctext{} is non-last, or $(H_{n+1}-H_i)+d$ if last to fill the narrow
% strip of $d=\!\@maxdepth!$ tall below the text for non-\lpage{}s.
% 
% Then we scan $\pp^s(p)$ again but applying
% $\!\pcol@buildcselt!\~\<H_i\>\<h_i\>$ to each element $\spt(H_i,h_i)$ to
% do the followings.
% 
% \begin{enumerate}
% \item
% To $b_r$, add a vertical rule whose height is $H'_i=H_i-(H_{i-1}+h_{i-1})$
% and width is \!\columnseprule! if $H'_i>0$, and then a vertical skip of
% $h_i$, as the rule segment between $(i{-}1)$-th and $i$-th \mctext{}s.
% Note that $H_i$ and $h_i$ are represented in the form of integers and thus
% we need |sp| to use them as dimensions.
% 
% \item
% To $b_b$, add painted \bground{}s for all columns $c\In\Cfrom\Cto$ and
% \csepgap{}s $c\In\Cfrom{\Cto{-}1}$ by \!\pcol@bg@paintcolumns! defining
% their regions $\bgr_{\{c,g\}}^c(i)$ by letting common top edge position
% $y_0=\!\pcol@bg@columntop!=H_{i-1}+h_{i-1}$ and common height
% $y_1-y_0=\!\pcol@bg@columnheight!=H'_i$, if $H'_i>0$.  Also add painted
% \bground{} for the $i$-th \mctext{} by \!\pcol@bg@paintbox! as we did for
% under-painting but this time the region is $\bgr_s(i)$.
% 
% \item
% Let $\!\@tempdimb!=H_i+h_i$ for the next element $\spt(H_{i+1},h_{i+1})$.
% \end{enumerate}
% 
% Then if $H'_{n+1}>0$, we add the last rule segment of $H'_{n+1}$ tall to
% $b_r$, and add painted \bground{}s for columns and \csepgap{}s as done in
% the step\Tie2 above but letting $y_1-y_0=H'_{n+1}+d$ to let the common
% bottom edge of the their regions reach the bottom of text area for
% non-\lpage{}s.
% 
%    \begin{macrocode}
\def\pcol@buildcolseprule#1#2#3#4{%
  \@tempdima#1\relax \dimen@#4\relax
  \let\pcol@bg@from#2\relax \let\pcol@bg@to#3\relax
  \setbox\pcol@tempboxa\vbox{}\setbox\@tempboxa\vbox{}%
  \let\@elt\pcol@buildcselt@S \pcol@sptextlist
  \@tempdimb\z@ \let\@elt\pcol@buildcselt \pcol@sptextlist
  \let\@elt\relax \advance\@tempdima-\@tempdimb
  \ifdim\@tempdima>\z@
    \setbox\pcol@tempboxa\vbox{\unvbox\pcol@tempboxa
      \hrule\@height\@tempdima\@width\columnseprule}%
    \setbox\@tempboxa\vbox{\unvbox\@tempboxa
      \let\@elt\relax
      \edef\pcol@bg@columntop{\number\@tempdimb sp}%
      \edef\pcol@bg@columnheight{%
        \@elt{\number\@tempdima sp}\@elt{\number\dimen@ sp}}%
      \pcol@bg@paintcolumns}%
  \fi}
\def\pcol@buildcselt@S#1#2{%
  \setbox\@tempboxa\vbox{\unvbox\@tempboxa
    \let\@elt\relax
    \def\pcol@bg@spanningtop{\@elt{#1sp}}%
    \advance\@tempdima-#1sp\relax \advance\@tempdima-#2sp\relax
    \advance\dimen@\@tempdima
    \edef\pcol@bg@spanningheight{\@elt{#2sp}%
      \ifdim\@tempdima>\z@\else \@elt{\number\dimen@ sp}\fi}%
    \pcol@bg@paintbox{S}}}
\def\pcol@buildcselt#1#2{%
  \@tempdimc#1sp \advance\@tempdimc-\@tempdimb
  \setbox\pcol@tempboxa\vbox{\unvbox\pcol@tempboxa
    \ifdim\@tempdimc>\z@ \hrule\@height\@tempdimc\@width\columnseprule \fi
    \vskip#2sp}%
  \setbox\@tempboxa\vbox{\unvbox\@tempboxa
    \let\@elt\relax
    \edef\pcol@bg@columntop{\number\@tempdimb sp}%
    \edef\pcol@bg@columnheight{\@elt{\number\@tempdimc sp}}%
    \ifdim\@tempdimc>\z@ \pcol@bg@paintcolumns \fi
    \def\pcol@bg@spanningtop{\@elt{#1sp}}%
    \advance\@tempdima-#1sp\relax \advance\@tempdima-#2sp\relax
    \advance\dimen@\@tempdima
    \edef\pcol@bg@spanningheight{\@elt{#2sp}%
      \ifdim\@tempdima>\z@\else \@elt{\number\dimen@ sp}\fi}%
    \pcol@bg@paintbox{s}}%
  \@tempdimb#1sp \advance\@tempdimb#2sp\relax}

%    \end{macrocode}
% \end{macro}\end{macro}\end{macro}
% 
% \begin{macro}{\pcol@hfil}
% \changes{v1.3-3}{2013/09/17}
%	{Introduced for column-separating rule drawing.}
% \changes{v1.3-4}{2013/09/17}
%	{Introduced for non-uniform column-separating gaps.}
% \changes{v1.34}{2018/05/07}
% 	{Rename \cs{pcol@tempbox} as \cs{pcol@tempboxa}.}
% 
% The macro $\!\pcol@hfil!\<c\>$ is used in \!\pcol@ioutputelt!,
% \!\pcol@imakeflushedpage! and \!\pcol@iflushfloats! to separate column
% $c{+}1$ and $c$ or $c$ and $c{+}1$ according as the columns are
% swapped or not in the page the caller macros are building.
% 
% \Index{column-swapping}
% 
% If $\!\columnseprule!=r>0$, the macro puts the followings; a horizontal
% space of $\gap_{c}/2=|\pcol@columnsep|{\cdot}c/2$
% 
% \SpecialArrayIndex{c}{\pcol@columnsep}
% 
% followed by a skip $-r/2$ to nullify the width of the rule; the rule in
% \!\pcol@tempboxa! which \!\pcol@buildcolseprule! built, with color
% $|\pcol@colseprulecolor|{\cdot}c$
% 
% \SpecialArrayIndex{c}{\pcol@columnsep}
% 
% or \!\pcol@colseprulecolor! according as the former is defined or not,
% i.e., \!\colseprulecolor!\oarg{mode}\ARg{color}$|[|c|]|$ is declared or
% not; and $\gap_c/2$ again but preceded by $-r/2$.  On the other hand if
% $r=0$, we simply put a space of $\gap_c$.  Note that the skips of
% $\gap_c/2$ and $\gap_c$ are accompanied by 1\,|fil| inifinit stretch to
% avoid underfull when $\sum_{c=\Cfrom}^{\Cto-2}(\w_c+\gap_c)+\w_{\Cto-1}<\WT$
% where $(\Cfrom,\Cto)=\{(0,\CL),(\CL,C)\}$, due to arithmetic errors in
% calculations of $\w_c$ and $\gap_c$\footnote{
% 
% It is assured the sum of $w_c$ and $\gap_c$ cannot exceed $\WT$ even with
% arithmetic errors and thus overfull never occurs.}. 
% 
%    \begin{macrocode}
\def\pcol@hfil#1{{%
  \@tempdima\csname pcol@columnsep#1\endcsname\relax
  \ifdim\columnseprule>\z@
    \hskip.5\@tempdima\@plus1fil\relax
    \hskip-.5\columnseprule
    \@ifundefined{pcol@colseprulecolor#1}%
      {\pcol@colseprulecolor}{\@nameuse{pcol@colseprulecolor#1}}%
    \copy\pcol@tempboxa \hskip-.5\columnseprule
    \hskip.5\@tempdima\@plus1fil\relax
  \else \hskip\@tempdima\@plus1fil\relax
  \fi}}

%    \end{macrocode}
% \end{macro}
% 
% \begin{macro}{\pcol@@outputpage}
% \changes{v1.3-2}{2013/09/17}
%	{Introduced keep the original definition of \cs{@outputpage}.}
% \changes{v1.3-3}{2013/09/17}
%	{Introduced keep the original definition of \cs{@outputpage}.}
% \changes{v1.3-4}{2013/09/17}
%	{Introduced keep the original definition of \cs{@outputpage}.}
% \begin{macro}{\@outputpage}
% \changes{v1.3-2}{2013/09/17}
%	{Redefined for parallel-paging.}
% \changes{v1.3-3}{2013/09/17}
%	{Redefined for background painting.}
% \changes{v1.3-4}{2013/09/17}
%	{Redefined for marginal note placement.}
% \changes{v1.34}{2018/05/07}
% 	{Rename \cs{pcol@tempbox} as \cs{pcol@tempboxa}.}
% 
% The macro \!\@outputpage!, being our own version of \LaTeX's one kept in
% \!\pcol@@outputpage!, ships out a page $p$ or \parapag{}e pair in $p$.
% The reason why we redefine this macro is that we need a few special
% operations for \parapag{}ing and \bgpaint{} {\em outside} of \env{paracol}
% environments.  Therefore, the macro is not only used in our own macros
% \!\pcol@outputelt!, \!\pcol@output@start!, \!\pcol@output@flush!,
% \!\pcol@output@clear!, \!\pcol@flushfloats! and \!\pcol@output@end!, but
% also in \LaTeX's \!\@opcol! and \!\@doclearpage!\footnote{
% 
% And possibly in \CSIndex{@outputdblcol} if double-column typesetting is
% done outside \env{paracol}.}
% 
% invoked from our own or \LaTeX's \!\output! routine.
% 
% First we calculate $H'_M=\!\topmargin!+\!\headheight!+\!\headsep!$ to place
% the origin of \bgpaint{} at the top edge of text area in what \LaTeX{}
% assumes as a page, i.e., shifted 1\,inch down from the real page.  Then if
% $\CSIndex{ifpcol@output}=\true$ to mean this macro is used in a
% \env{paracol} environment, we build the painted \bground{}s of left and
% right \parapag{}es in $\!\pcol@tempboxa!=b_l$ and $\!\@tempboxa!=b_r$ by
% putting a vertical skip of $H'_M$, and invoking \!\pcol@bg@paintpage!
% with the setting $(\!\pcol@bg@from!,\!\pcol@bg@to!)=(\CBfrom,\CBto)$ be
% $(0,\CL)$ and $(\CL,\C)$ respectively.  Note that \!\pcol@bg@paintpage!
% paints \bground{}s of regions $\bgr_a^{[c]}$ for all
% $a\in\{T,B,L,R,C,S,t,b,l,r\}$ and $c\In\CBfrom\CBto$, and for $b_r$ we
% temporarily increment $\page(p)$ by one if \npaired{} \parapag{}ing is in
% effect.
% 
% Otherwise, i.e., if $\CSIndex{ifpcol@output}=\false$ indicating outside
% use, we build the painted \bground{}s in $b_l$ and $b_r$ similarly but
% with the following differences;  \bgpaint{} is done if
% $\CSIndex{ifpcol@havelastpage}=\true$ to mean the page to be shipped has
% the \lpage{} of closed \env{paracol} as its part and
% $\!\set@color!\neq\!\relax!$ to mean some coloring package is loaded;
% page \bground{} painting is done by \!\pcol@bg@@paintpage! because
% \!\pcol@bg@paintpage! is not available outside \env{paracol} environments;
% the \bground{} of \postenv{} is painted by \!\pcol@bg@@paintbox! for the
% region $\bgr_{\{P,p\}}=[(0,\WT)(\HB,\HT)]$ where
% $\HB=\!\pcol@bg@preposttop!
% \in\{\!\pcol@bg@preposttop@left!,\!\pcol@bg@preposttop@right!\}$
% having the bottom edge of the last 
% \env{paracol} environment (having right \parapag{}e for $b_r$).  In
% addition, we examine if $\!\pcol@rightpage!\neq\bot$ to mean the right
% \parapag{}e was built by \!\pcol@output@end! when the last \env{paracol}
% environment was closed and, if so, make the box \!\textheight! tall adding
% \!\vfil! to its bottom.
% 
% Then regardless of \CSIndex{ifpcol@output}, we do the followings; let the
% height and depth of $b_l$ and $b_r$ be 0 because they cannot occupy any
% real spaces in the ship-out image; temporarily let
% $\CSIndex{ifpcol@swapcolumn}=\false$ if $\page(p)$ is odd, $\CL=\C$ to
% mean \parapag{}ing is not in effect\footnote{
% 
% Since the assignments of $\CL$ and $\C$ in \CSIndex{pcol@zparacol} are
% \CSIndex{global} and they are not nodified anywhere else, examining their
% equality outside \env{paracol} environments is safe and meaningful.},
% 
% \parapag{}ing is done in \npaired{} mode\footnote{
% 
% We need this examination because $\CSIndex{ifpcol@swapcolumn}=\false$ for
% \npaired{} \parapag{}ing is made locally by \!\pcol@zparacol!.},
% 
% or we are outside \env{paracol} environments and the page does not have
% anything produced in environments.  That is, we let
% $\CSIndex{ifpcol@swapcolumn}=\true$ if the page has something produced by
% a \env{paracol} environment, \cswap{} and \parapag{}ing are specified for
% the (last) environment\footnote{
% 
% Column-swapping may be enabled {\em after} the last \env{paracol}
% environment was closed but we consider the enablement is effective for the
% page having the environment.},
% 
% and the page number is even.  Note that a page may have two or more
% (\lpage{}s of) \env{paracol} environements whose \parapag{}ing style can
% be inconsistent including the case some of them are not \parapag{}ed.  If
% this inconsistency happens the page is shipped out following the style of
% the last environment.  Also note that even if the last environment is not
% \parapag{}ed, the right \parapag{}e kept in \!\pcol@rightpage! is assuredly
% shipped out.
% 
% Then if \cswap{} is in effect, we ship out the right \parapag{}e at first
% by \!\pcol@outputpage@r! and then the left one by \!\pcol@outputpage@l! to
% swap the left and right.  Otherwise, the ship-out order is normal and thus
% the invocation order is \!\pcol@outputpage@l!  then \!\pcol@outputpage@r!.
% Note that if \npaired{} \parapag{}ing is in effect, the page number to
% given to \!\pcol@outputpage@r! as its argument is $\page(p)+1$ if it is
% the second one, i.e., not swapped, while the argument in other cases and
% of \!\pcol@outputpage@l! are always $\page(p)$.  Then finally, we
% \!\global!ly let $\CSIndex{ifpcol@havelastpage}=\false$ because so far the
% next page does not have \env{paracol}'s \lpage{} especially when we are
% outside it, let \!\pcol@bg@preposttop@left! and
% \!\pcol@bg@preposttop@right! have 0 because, if we are outside, the next
% \preenv{} should start from the top of a page, and let
% $\mpbout=\!\pcol@mparbottom@out!$ be $\mpboutz=\!\pcol@mparbottom@zero!$
% because so far we have no marginal notes given in \env{paracol}
% environments\footnote{
% 
% This assignment in a \env{paracol} environment is meaningless because
% $\mpbout$ is meaningless too, but not harmful.}.
% 
%    \begin{macrocode}
\let\pcol@@outputpage\@outputpage
\def\@outputpage{\begingroup
  \@tempdima\topmargin \advance\@tempdima\headheight \advance\@tempdima\headsep
  \ifpcol@output
    \setbox\pcol@tempboxa\vtop{\vskip\@tempdima \global\pcol@bg@paintedfalse
     \let\pcol@bg@from\z@ \let\pcol@bg@to\pcol@ncolleft
     \pcol@bg@paintpage}%
    \ifpcol@bg@painted \@tempswatrue \else \@tempswafalse \fi
    \setbox\@tempboxa\vtop{\vskip\@tempdima \global\pcol@bg@paintedfalse
      \ifpcol@paired\else \advance\c@page\@ne \fi
      \let\pcol@bg@from\pcol@ncolleft \let\pcol@bg@to\pcol@ncol
      \pcol@bg@paintpage}%
  \else
    \def\reserved@a{\vskip\@tempdima \global\pcol@bg@paintedfalse
      \ifpcol@havelastpage \ifx\set@color\relax\else
        \pcol@bg@@paintpage \pcol@bg@@paintbox{Pp}%
      \fi\fi}%
    \setbox\pcol@tempboxa\vbox{%
      \let\pcol@bg@preposttop\pcol@bg@preposttop@left
      \let\pcol@bg@from\z@ \let\pcol@bg@to\pcol@ncolleft \reserved@a}%
    \ifpcol@bg@painted \@tempswatrue \else \@tempswafalse \fi
    \setbox\@tempboxa\vbox{\ifpcol@paired\else \advance\c@page\@ne \fi
      \let\pcol@bg@preposttop\pcol@bg@preposttop@right
      \let\pcol@bg@from\pcol@ncolleft \let\pcol@bg@to\pcol@ncol
      \reserved@a}%
    \ifvoid\pcol@rightpage\else
      \pcol@Logstart{\@outputpage{rightset}}%
      \setbox\pcol@rightpage\vbox to\textheight{\unvbox\pcol@rightpage \vfil}%
      \pcol@Logend{\@outputpage{rightset}}%
    \fi
  \fi
  \ht\pcol@tempboxa\z@ \dp\pcol@tempboxa\z@
  \ht\@tempboxa\z@ \dp\@tempboxa\z@
  \ifodd\c@page                                 \pcol@swapcolumnfalse \fi
  \ifnum\pcol@ncolleft<\pcol@ncol\else          \pcol@swapcolumnfalse \fi
  \ifpcol@output\else \ifpcol@havelastpage\else \pcol@swapcolumnfalse \fi\fi
  \@tempcnta\c@page
  \ifpcol@paired\else \advance\@tempcnta\@ne    \pcol@swapcolumnfalse \fi
  \ifpcol@swapcolumn \pcol@outputpage@r\c@page \pcol@outputpage@l\@tempcnta
  \else              \pcol@outputpage@l\c@page \pcol@outputpage@r\@tempcnta
  \fi
  \global\pcol@havelastpagefalse \gdef\pcol@bg@preposttop@left{0pt}%
  \global\let\pcol@bg@preposttop@right\pcol@bg@preposttop@left
  \global\let\pcol@mparbottom@out\pcol@mparbottom@zero
  \endgroup}

%    \end{macrocode}
% \end{macro}\end{macro}
% 
% \KeepSpace{1}
% \begin{macro}{\pcol@outputpage@l}
% \changes{v1.3-2}{2013/09/17}
%	{Introduced for shipping out left parallel-pages.}
% \changes{v1.3-3}{2013/09/17}
%	{Introduced for column-separating rule drawing and background
%	 painting in left parallel-pages.}
% \changes{v1.34}{2018/05/07}
% 	{Rename \cs{pcol@tempbox} as \cs{pcol@tempboxa}.}
% \begin{macro}{\pcol@outputpage@r}
% \changes{v1.3-2}{2013/09/17}
%	{Introduced for shipping out right parallel-pages.}
% \changes{v1.3-3}{2013/09/17}
%	{Introduced for column-separating rule drawing and background
%	 painting in right parallel-pages.}
% \begin{macro}{\pcol@outputpage@ev}
% \changes{v1.3-3}{2013/09/17}
%	{Introduced for background painting.}
% 
% The macro $\!\pcol@outputpage@l!\arg{page}$, used solely in our own
% version of \!\@outputpage!, at first lets \!\c@page! have $\arg{page}$
% which definitely has the value that \!\c@page! had when we start
% \!\@outputpage!.  That is, even when this macro is invoked after
% \!\pcol@outputpage@r! due to swapped \parapag{}ing,
% 
% \Index{column-swapping}
% 
% this assignment cancels the increment of \!\c@page! done in \LaTeX's
% \!\@outputpage! or in other word \!\pcol@@outputpage! because in this case
% \parapag{}es are \paired{}.  Then we make \!\@themargin! \!\let!-equal to
% \!\evensidemargin! if two-side typesetting is in effect and \!\c@page! is
% even, or to \!\oddsidemargin! otherwise for the reference in
% \!\pcol@outputpage@ev! as shown shortly.
% 
% Next, if \bgpaint{} took place in \!\@outputpage!, we let \!\everyvbox!
% have the macro invocation $\!\pcol@outputpage@ev!\<b_l\>$ to be expanded
% to the following sequence so that they are the leading materials in the
% \!\vbox! to be \!\shipout!; examination if the document is processed by a
% Japanese \LaTeX{} named p\LaTeX{} and then, if so, a control sequence
% \!\yoko! to put materials naturally; the painted \bground{} $b_l$ shifted
% right by \!\@themargin!; \!\nointerlineskip!  to inhibit \!\baselineskip!
% insertion after $b_l$; emptying \!\everyvbox! to ensure nothing will be
% inserted into internal \!\vbox!es; and the assignment of \!\yoko! to
% \!\let! it be \!\relax! if necessary.  This trick with \!\everyvbox! is
% necessary\footnote{
% 
% Unless we rewrite \CSIndex{@outputpage}.}
% 
% because $b_l$ should be put {\em before} \!\pcol@@outputpage! puts the
% page header, or the header would be overlaid by regions, e.g.,
% $\bgr_{\{t,T\}}$ in natural cases.
% 
% The tricky elements to handle \!\yoko! in the sequence is necessary for
% p\LaTeX{} whose \!\@outputpage! has \!\yoko! as the first element of the
% \!\vbox! to be \!\shipout!, because \!\yoko! must be the first element of
% a box but our \!\everyvbox! to put \bground{} would make it non-first.
% That is by the tricky elements, the \!\vbox! should have \!\yoko! as the
% first element from \!\everyvbox! and then that put by p\LaTeX's
% \!\@outputpage! is nullified by \!\let!\!\yoko!\!\relax! in the
% \!\everyvbox! just for the \!\vbox! to be shipped out.  On the other hand
% in ordinary \LaTeX, \!\yoko! does not appear in the \!\vbox! or is
% modified.  The examination of the use of p\LaTeX{} is also trickily done
% by comparing the expansion results of \!\meaning!\!\yoko! and
% \!\string!\!\yoko!.  Since the former results in the tokens ``\!\yoko!''
% which \!\string!\!\yoko!  gives us iff \!\yoko! is a primitive of
% underlying \TeX{} being p\TeX{} if so, the comparison should give us
% equality iff p\LaTeX{} is in use\footnote{
% 
% Unless some other \TeX{} has a primitive named \CSIndex{yoko}.  This
% examination is more strict than that with \CSIndex{pfmtname} for
% \CSIndex{ifpcol@bfbottom}.}.
% 
% Then we invoke \!\pcol@@outputpage! being (p)\LaTeX's original version of
% \!\@outputpage! to ship out \!\@outputbox! finally.
% 
% The macro \!\pcol@outputpage@r!$\arg{page}$ performs similar operations but
% it does them only when $\!\pcol@rightpage!\neq\bot$ to mean we are in an
% \env{paracol} environment with \parapag{}ing or outside it but in the page
% in which it resides.  Other differences are as follows; $\arg{page}$ can
% be $\page(p)+1$ for \npaired{} right \parapag{}es; \!\@outputbox! is
% locally made \!\let!-equal to \!\pcol@rightpage! prior to the invocation
% of \!\pcol@@outputpage!; and $b_r$ is given to \!\pcol@outputpage@ev! as
% its argument.
% 
%    \begin{macrocode}
\def\pcol@outputpage@l#1{%
  \pcol@Logstart{\@outputpage{left}}%
  \global\c@page#1\relax
  \let\@themargin\oddsidemargin
  \if@twoside\ifodd\c@page\else \let\@themargin\evensidemargin \fi\fi
  \if@tempswa \everyvbox{\pcol@outputpage@ev\pcol@tempboxa}\fi
  \pcol@@outputpage
  \pcol@Logend{\@outputpage{left}}}
\def\pcol@outputpage@r#1{%
  \begingroup
  \ifvoid\pcol@rightpage\else
    \global\c@page#1\relax
    \let\@outputbox\pcol@rightpage
    \pcol@Logstart{\@outputpage{right}}%
    \let\@themargin\oddsidemargin
    \if@twoside\ifodd\c@page\else \let\@themargin\evensidemargin \fi\fi
    \ifpcol@bg@painted \everyvbox{\pcol@outputpage@ev\@tempboxa}\fi
    \pcol@@outputpage
    \pcol@Logend{\@outputpage{right}}%
  \fi
  \endgroup}
\def\pcol@outputpage@ev#1{%
  \edef\reserved@a{\meaning\yoko}\edef\reserved@b{\string\yoko}%
  \ifx\reserved@a\reserved@b \yoko\fi
  \moveright\@themargin\box#1\nointerlineskip \everyvbox{}%
  \ifx\reserved@a\reserved@b \let\yoko\relax \fi}

%    \end{macrocode}
% \end{macro}\end{macro}\end{macro}
% 
% 
% 
% 
% \section{Starting New Column-Page}
% \label{sec:imp-startcolumn}
% 
% \begin{macro}{\pcol@startcolumn}
% \changes{v1.3-3}{2013/09/17}
%	{Add \cs{@colht} and \cs{@tempdimb} as the first and third argument of
%	 \cs{pcol@shrinkcolbyfn}.}
% \changes{v1.32-2}{2015/10/10}
% 	{Fix the memory leak caused by mistakingly preserving $\pi^f(p)$
%	 when $p\EQ p_t$.}
% \changes{v1.32-2}{2015/10/10}
% 	{Add \cs{pcol@Fb}/\cs{pcol@Fe} pair(s).}
% 
% The macro $\!\pcol@startcolumn!\arg{f}$ is invoked from \!\pcol@output!
% with $f=1$ and \!\pcol@freshpage! with $f=0$ to start a new \colpage.
% This macro has two additional functions to \LaTeX's \!\@startcolumn!, one
% for \Scfnote{}s and the other for coloring.
% 
% First, if the page $p$ in which the new \colpage{} resides has \Scfnote{}s
% in $\pp^f(p)=\!\pcol@footins!$ because the column is not the leading one,
% 
% \Index{leading column}
% 
% we temporarily shrink \!\@colht! and \!\@colroom! by the space required to
% put $\pp^f(p)$ by \!\pcol@shrinkcolbyfn! during the trial of deferred
% float placement, remembering the existence of the footnotes by letting
% $\!\@tempdimb!=-\!\skip!\!\pcol@footins!$ which should be 0 otherwise.
% This shrinkage is essentially required when $p<\ptop$ because $\pp^f(p)$
% has been fixed to be a part of $p$ and thus deferred floats cannot push
% footnotes down to succeeding pages.  In the case of $p=\ptop$, the
% shrinkage is also desirable to avoid unnecessary pushing down of footnotes
% which \TeX{} has decided to be in $p$.
% 
% Then after trying put deferred floats in the \colpage{} by \!\@tryfcolumn!
% and \!\pcol@trynextcolumn! as done in \LaTeX's \!\@startcolumn!, we
% \!\insert! $\pp^f(p)$, if it is has some footnoes, by letting \!\footins!
% have it by \!\pcol@getcurrfoot! so that \TeX{} will be aware of the
% footnotes when it examines the page break of the \colpage{}.  That is, if
% $p<\ptop$ the \!\insert!ion is to keep the vertical space for $\pp^f(p)$ in
% the building process of the \colpage{} in $p$ because any \Scfnote{s}
% cannot be added to $p$ any more, and thus $\pp^f(p)$ is preserved until
% the page $p$ is shipped out.  On the other hand if $p=\ptop$, \Scfnote{s}
% in $p$ can grow further and thus \!\insert!ed footnotes will be captured
% again by \!\pcol@output@switch! or \!\pcol@startpage!.  Therefore, if
% $p=\ptop$, we release $\pp^f(p)$ to \!\@freelist!.
% 
% Then if $p=\ptop$, we also \!\insert! deferred footnotes in $\df$ until
% their total height reaches \!\@colht! by \!\pcol@deferredfootins! if $f=1$
% to mean this macro is invoked from \!\pcol@output!\footnote{
% 
% The \cs{insert}ion of $\pp^f(p)$ also requiers $f=1$ but this examination
% is redundant because $\pp^f(p)=\bot$ definately if $f=0$.}.
% 
% Note that the deferred footnote \!\insert!ion in the case of $f=0$ will
% be done afterward when \!\pcol@freshpage! does \!\pcol@restartcolumn! at
% its tail.  Also note that \!\pcol@deferredfootins! examines if
% $\!\@tempdimb!=0$ to mean $\pp^f(p)=\bot$ and thus \!\skip!\!\footins!
% should be taken into account in its extraction of the footnotes from $\df$.
% 
% Then after restoring \!\@colht! and canceling the temporary shrinkage of
% \!\@colroom!, we invoke \!\pcol@savecolorstack! to save \colpage{}'s
% \colorctext{} into $\csts$ so that coloring \!\special!s to reestablish
% $\csts$ will be put at its top if it has something when we leave it.
% 
%    \begin{macrocode}
%% Starting New Column Page

\def\pcol@startcolumn#1{%
  \@tempdima\@colht \@tempdimb\z@
  \ifvoid\pcol@footins\else
    \pcol@shrinkcolbyfn\@colht\pcol@footins\@tempdimb
  \fi
  \global\@colroom\@colht
  \@tryfcolumn\@deferlist
  \if@fcolmade\else
    \pcol@trynextcolumn
    \ifpcol@scfnote \ifnum#1>\z@
      \ifvoid\pcol@footins\else
        \edef\pcol@currfoot{\pcol@footins}%
        \pcol@getcurrfoot\copy
        \pcol@Log\pcol@startcolumn{insert}\footins
        \insert\footins{\unvbox\footins}%
        \ifnum\pcol@page=\pcol@toppage
          \pcol@Fb
          \@cons\@freelist\pcol@footins
          \pcol@Fe{startcolumn(pagefn)}%
        \fi
      \fi
      \ifnum\pcol@page=\pcol@toppage
        \pcol@deferredfootins\pcol@startcolumn \fi
    \fi\fi
  \fi
  \advance\@tempdima-\@colht
  \global\advance\@colroom\@tempdima
  \global\advance\@colht\@tempdima
  \pcol@savecolorstack}
%    \end{macrocode}
% \end{macro}
% 
% \begin{macro}{\pcol@trynextcolumn}
% \changes{v1.2-2}{2013/05/11}
%	{No change in the code itself but its explanation is modified
%	 according to the drastic redesign of \cs{pcol@startcolumn}.}
% 
% The macro \!\pcol@trynextcolumn! is invoked from \!\pcol@startcolumn! and
% \!\pcol@flush~column! to try to move deferred floats in \!\@deferlist!
% into \!\@toplist! or \!\@botlist!.  The body of this macro is perfectly
% equivalent to the \cs{else} part of \CSIndex{if@fcolmade} in \LaTeX's
% \!\@startcolumn!.
% 
%    \begin{macrocode}
\def\pcol@trynextcolumn{\begingroup
  \let\reserved@b\@deferlist
  \global\let\@deferlist\@empty
  \let\@elt\@scolelt
  \reserved@b
  \endgroup}

%    \end{macrocode}
% \end{macro}
% 
% 
% 
% \section{Background Painting}
% \label{sec:imp-bgpaint}
% 
% \begin{macro}{\pcol@bg@from}
% \changes{v1.3-3}{2013/09/17}
%	{Introduced for background painting.}
% \begin{macro}{\pcol@bg@to}
% \changes{v1.3-3}{2013/09/17}
%	{Introduced for background painting.}
% 
% The control sequence pair
% $(\!\pcol@bg@from!,\!\pcol@bg@to!)=(\Uidx\CBfrom,\Uidx\CBto)$ are made
% \!\let!-equal to $(0,\CL)$ or $(\CL,\C)$ by \!\pcol@buildcolseprule! and
% \!\@outputpage!  for \bgpaint{} of columns and \csepgap{}s, and referred
% to by column scanning loops in \!\pcol@bg@paint@ii! and
% \!\pcol@bg@columnleft!.  The control sequence \!\pcol@bg@to! is also
% referred to by \!\pcol@bg@paint@i! to decrement it by one temporarily so
% that the loop in \!\pcol@bg@paint@ii! scans $c\In\CBfrom{\CBto{-}1}$
% rather than $\LBRP\CBfrom\CBto$.  Since this decrement is done whenever a
% painting macro is used regardless some setting of $\CBto$, \!\pcol@bg@to!
% has default setting with $\C$ to avoid unbound reference at the
% decrement.  Note that since this decrement is done in a \!\vbox! and an
% appropriate setting must have been done if $\CBto$ is referred in
% \!\pcol@bg@paint@ii!, this decrement and default setting are safe.
% 
%    \begin{macrocode}
%% Background Painting

\let\pcol@bg@to\pcol@ncol
%    \end{macrocode}
% \end{macro}\end{macro}
% 
% \KeepSpace{4}
% \begin{macro}{\pcol@bg@paintpage}
% \changes{v1.3-3}{2013/09/17}
%	{Introduced for background painting.}
% \begin{macro}{\pcol@bg@@paintpage}
% \changes{v1.3-3}{2013/09/17}
%	{Introduced for background painting.}
% \begin{macro}{\pcol@bg@paintcolumns}
% \changes{v1.3-3}{2013/09/17}
%	{Introduced for background painting.}
% \begin{macro}{\pcol@bg@@paintcolumns}
% \changes{v1.3-3}{2013/09/17}
%	{Introduced for background painting.}
% \begin{macro}{\pcol@bg@paintbox}
% \changes{v1.3-3}{2013/09/17}
%	{Introduced for background painting.}
% \begin{macro}{\pcol@bg@@paintbox}
% \changes{v1.3-3}{2013/09/17}
%	{Introduced for background painting.}
% 
% The macros \!\pcol@bg@@paintpage!, \!\pcol@bg@@paintcolumns! and
% $\!\pcol@bg@@paintbox!\Arg{A}$ are made \!\let!-equal to their interface
% counterparts \!\pcol@bg@paintpage!, \!\pcol@bg@paint~columns! and
% \!\pcol@bg@paintbox! by \!\pcol@zparacol! if some coloring package has
% been loaded.  Otherwise, these interface macros are \!\let!-equal to
% \!\relax! for first two and \!\@gobble! for the last, so that macros in
% \!\output! routine freely use them unaware of coloring capability.  One
% exception is in \!\@outputpage! which uses \!\pcol@bg@@paintpage! and
% \!\pcol@bg@@paintbox! explicitly when it is outside \env{paracol}
% environments, examining the availability of coloring.
% 
% The macro \!\pcol@bg@paintpage! and \!\pcol@bg@@paintpage! are used in
% \!\@outputpage! to paint \bground{}s of regions $\bgr_a^{[c]}$ for all
% $a\in\{T,B,L,R,G,C,t,b,l,r\}$ and $c\In\CBfrom\CBto$ for $a=C$ while
% $c\In\CBfrom{\CBto{-}1}$ for $a=G$.  Therefore, the macro invokes
% \!\pcol@bg@paint@i! with two
% $\!\pcol@bg@paint@ii!\Arg{A_b}\Arg{A_g}\Arg{A_c}$, letting
% $A_b=|TBLR|$, $A_g=|G|$ and $A_c=|C|$ in the first invocation and then
% $A_b=|tblr|$ and $A_g=A_c=\emptyset$ in the second.
% 
% The macro \!\pcol@bg@paintcolumns! is used in \!\pcol@buildcolseprule! and
% \!\pcol@buildcselt! to paint \bground{}s of regions $\bgr_g^c(i)$ for
% $c\In\CBfrom{\CBto{-}1}$ and $\bgr_c^c(i)$ for $c\In\CBfrom\CBto$.
% Therefore, the macro invokes \!\pcol@bg@paint@i! with \!\pcol@bg@paint@ii!
% giving it $A_b=\emptyset$, $A_g=|g|$ and $A_c=|c|$.
% 
% The macro $\!\pcol@bg@paintbox!\Arg{A}$ is used in the following macros
% with $A$ shown in the parantheses to paint the \bground{}s of regions
% $\bgr_{\{a_1,a_2\}}$ where $(a_1,a_2)\in\{(S,s),(F,f),(N,n),\~(P,p)\}$.
% 
% \begin{quote}\raggedright
% \!\pcol@outputelt!\Tie(|Ff|),
% \!\pcol@ioutputelt!\Tie(|Nn|, |Ff|),
% \!\pcol@buildcselt!\Tie(|Ss|),
% \!\pcol@output@start!\Tie(|Pp|),
% \!\pcol@output@clear!\Tie(|Ff|),
% \!\pcol@makeflushedpage!\Tie(|Ff|),
% \!\pcol@imakeflushedpage!\Tie(|Nn|),
% \!\pcol@output@end!\Tie(|Nn|).
% \end{quote}
% 
% The macro \!\@outputpage! also uses the function but \!\pcol@bg@@paintbox!
% explicitly with $A=|Pp|$.  Therefore \!\pcol@bg@@paintbox! invokes
% \!\pcol@bg@paint@i! with \!\pcol@bg@paint@ii!  giving it $A_b=A$ and
% $A_g=A_c=\emptyset$.
% 
%    \begin{macrocode}
\def\pcol@bg@@paintpage{%
  \pcol@bg@paint@i{%
    \pcol@bg@paint@ii{TBLR}{G}{C}\pcol@bg@paint@ii{tblr}{}{}}}
\def\pcol@bg@@paintcolumns{\pcol@bg@paint@i{\pcol@bg@paint@ii{}{g}{c}}}
\def\pcol@bg@@paintbox#1{\pcol@bg@paint@i{\pcol@bg@paint@ii{#1}{}{}}}

%    \end{macrocode}
% \end{macro}\end{macro}\end{macro}\end{macro}\end{macro}\end{macro}
% 
% \begin{macro}{\pcol@bg@paint@i}
% \changes{v1.3-3}{2013/09/17}
%	{Introduced for background painting.}
% 
% The macro $\!\pcol@bg@paint@i!\Arg{body}$ is used in
% \!\pcol@bg@@paintpage!, \!\pcol@bg@@paint~columns! and
% \!\pcol@bg@@paintbox! to paint \bground{}s by a sequence of
% \!\pcol@bg@paint@ii! specified in $\arg{body}$.  The painted \bground{} is
% built in \!\@tempboxa! being a \!\vtop! having a null \!\vskip! as its
% first element so that everything put in the box is below its reference
% point at its top.  Then, before invoking \!\pcol@bg@paint@ii! in
% $\arg{body}$, we do the followings;  \!\global!ly let
% $\CSIndex{ifpcol@bg@painted}=\false$ to indicate so far any painted
% \bground{} are produced;  make \!\pcol@bg@leftmargin! \!\let!-equal
% to \!\pcol@lrmargin! to use this \!\dimen! register locally with the
% more appropriate alias;  negate \!\pagerim! locally to calculate $\HT$
% easily;  decrement $\CBto$ by one locally for the column scanning loop for
% $c\In\CBfrom{\CBto{-}1}$ in \!\pcol@bg@paint@ii!;  and \!\offinterlineskip!
% to inhbit inter-line \!\baselineskip! insertion in the box.  Then after
% the invocation of the sequence of \!\pcol@bg@paint@ii! in $\arg{body}$ and
% closing the box, we let the height, depth and width of the box be 0 so
% that it does not occupy any real space in the outer box in which the box
% is put.  Finally, if $\CSIndex{ifpcol@bg@painted}=\true$ meaning that some
% painted \bground{}s are built in the box, we put the box into the outer
% box surrounding it by \!\nointerlineskip! to inhibit inter-line
% \!\baselineskip! insertion before and after it.
% 
%    \begin{macrocode}
\def\pcol@bg@paint@i#1{%
  \setbox\@tempboxa\vtop{\vskip\z@
    \global\pcol@bg@paintedfalse
    \let\pcol@bg@leftmargin\pcol@lrmargin
    \pagerim-\pagerim \advance\pcol@bg@to\m@ne
    \offinterlineskip #1}%
  \ht\@tempboxa\z@ \dp\@tempboxa\z@ \wd\@tempboxa\z@
  \ifpcol@bg@painted \nointerlineskip \box\@tempboxa \nointerlineskip \fi}
%    \end{macrocode}
% \end{macro}
% 
% \begin{macro}{\pcol@bg@paint@ii}
% \changes{v1.3-3}{2013/09/17}
%	{Introduced for background painting.}
% 
% The macro $\!\pcol@bg@paint@ii!\Arg{A_b}\Arg{A_g}\Arg{A_c}$ appears only
% in the argument of \!\pcol@bg@paint@i! used in
% \!\pcol@bg@@paintpage!, \!\pcol@bg@@paintcolumns! and
% \!\pcol@bg@@paintbox! to paint \bground{}s of regions $\bgr_a$ for
% $a\in{}A_b\subseteq\{T,B,L,R,S,F,N,P,t,b,l,r,s,f,n,p\}$, $\bgr_a^c$ for
% $a\in{}A_g\subseteq\{G,g\}$ and $c\In\CBfrom{\Cto{-}1}$, and $\bgr_a^c$ for
% $a\in A_c\subseteq\{C,c\}$ and $c\In\CBfrom\CBto$.
% 
% First we invoke \!\pcol@bg@swappage! with \CSIndex{ifpcol@bg@swap} to
% let \!\pcol@bg@left~margin! and \CSIndex{ifpcol@bg@@swap} have values
% accorging to \CSIndex{ifpcol@bg@swap}, \CSIndex{if@twoside} and the parity
% of $\page(p)$.  Then we invoke $\!\pcol@bg@paintregion!\arg{a}\arg{c}$ for
% all $a\in A_b$ and $c=-1$ to paint the \bground{} of $\bgr_a$.  Second, we
% invoke \!\pcol@bg@swappage! again but with \CSIndex{ifpcol@swapcolumn}
% instead of \CSIndex{ifpcol@bg@swap}, and \!\pcol@bg@paintregion! as well
% but for $a\in\{A_g,A_c\}$ and $c\In\CBfrom{\CBto{-}1}$.  Third and
% finally, we make yet another invocation of \!\pcol@bg@paintregion! for
% $a\in A_c$ and $c=\CBto-1$.
% 
%    \begin{macrocode}
\def\pcol@bg@paint@ii#1#2#3{%
  \pcol@bg@swappage\ifpcol@bg@swap\fi
  \@tfor\reserved@b:=#1\do{\pcol@bg@paintregion\reserved@b\m@ne}%
  \pcol@bg@swappage\ifpcol@swapcolumn\fi
  \@tfor\reserved@b:=#2#3\do{%
    \pcol@currcol\pcol@bg@from \@whilenum\pcol@currcol<\pcol@bg@to\do{%
      \pcol@bg@paintregion\reserved@b\pcol@currcol
     \advance\pcol@currcol\@ne}}%
  \@tfor\reserved@b:=#3\do{\pcol@bg@paintregion\reserved@b\pcol@currcol}}
%    \end{macrocode}
% \end{macro}
% 
% \begin{macro}{\pcol@bg@swappage}
% \changes{v1.3-3}{2013/09/17}
%	{Introduced for background painting.}
% 
% The macro $\!\pcol@bg@swappage!\arg{if}\cs{fi}$ is used solely in
% \!\pcol@bg@paint@ii! but twice with $\arg{if}=\CSIndex{ifpcol@bg@swap}$
% and then with $\arg{if}=\CSIndex{ifpcol@swapcolumn}$, to let
% \!\pcol@bg@leftmargin! and \CSIndex{ifpcol@bg@@swap} have values for
% \mirror{}ing according to the truth values of $\arg{if}$ and
% \CSIndex{if@twoside} and the parity of $\page(p)$ of the page $p$ for
% which \bgpaint{} is taking place.  That is, they are let have the values
% as follows.
% 
% \begin{eqnarray*}
% W&=&\cases{\CSIndex{oddsidemargin}&
%              $\page(p)\bmod2=1\;\lor\CSIndex{if@twoside}=\false$\cr
%            \CSIndex{evensidemargin}&
%              $\page(p)\bmod2=0\;\land\CSIndex{if@twoside}=\true$}\\
% \rlap{$\displaystyle(\!\pcol@bg@leftmargin!,\CSIndex{ifpcol@bg@@swap})$}
%   \phantom{W}\\
% &=&\cases{(W,\;\false)&$\page(p)\bmod2=1\lor\;\arg{if}=\false$\cr
%           (\WP-(W+\WT+2\,|in|),\;\true)&
%             $\page(p)\bmod2=\land\;\arg{if}=\true$}
% \end{eqnarray*}
% 
% Note that $\WP-(W+\WT+2\,|in|)$ means the right margin width minus 1\,|in|
% with given left margin width $W$, and thus
% $\WM=\!\pcol@bg@leftmargin!+1\,|in|$ gives us the right margin width we
% need in \mirror{}ed \bgpaint.
% 
%    \begin{macrocode}
\def\pcol@bg@swappage#1#2{%
  \pcol@bg@leftmargin\oddsidemargin \pcol@bg@@swapfalse
  \ifodd\c@page\else
    \if@twoside \pcol@bg@leftmargin\evensidemargin \fi
    #1% \ifpcol@{bg@swap,swapcolumn}
    \pcol@bg@@swaptrue
    \advance\pcol@bg@leftmargin\textwidth \advance\pcol@bg@leftmargin2in
    \advance\pcol@bg@leftmargin-\paperwidth
    \pcol@bg@leftmargin-\pcol@bg@leftmargin
    #2% \fi
  \fi}

%    \end{macrocode}
% \end{macro}
% 
% \begin{macro}{\pcol@bg@paintregion}
% \changes{v1.3-3}{2013/09/17}
%	{Introduced for background painting.}
% \begin{macro}{\pcol@bg@paintregion@i}
% \changes{v1.3-3}{2013/09/17}
%	{Introduced for background painting.}
% 
% The macro $\!\pcol@bg@paintregion!\arg{a}\arg{c}$ is used only in
% $\!\pcol@bg@paint@ii!\Arg{A_b}\Arg{A_g}\Arg{A_c}$ but as many times as
% $\Abs{A_b}+\Abs{A_g}(\CBto-\CBfrom-1)+\Abs{A_c}(\CBto-\CBfrom)$ to paint
% \bground{} region $\bgr_a^{[c]}$ specified by $|\pcol@bg@@|{\cdot}a$
% 
% \SpecialArrayIndex{a}{\pcol@bg@@}
% 
% with color $\bgc_a^c=|\pcol@bg@color@|{\cdot}a{\cdot}|@|{\cdot}c$
% 
% \SpecialArrayIndex{a{\cdot}\string\texttt{@}{\cdot}c}{\pcol@bg@color@}
% 
% or, if it is undefined, $\bgc_a=|\pcol@bg@color@|{\cdot}a$.
% 
% \SpecialArrayIndex{a}{\pcol@bg@color@}
%
% If $\bgr_a^c$ or $\bgr_a$ is defined, the painted \bground{} is built in
% \!\@tempboxa! with \!\vtop! having null vertical skip at its top and by
% $\!\pcol@bg@paintregion@i!\Arg{F_x}\Arg{F_y}\Arg{F_w}\Arg{F_h}$ where
% the arguments are defined in the body of the macro $|\pcol@bg@@|{\cdot}a$
% 
% \SpecialArrayIndex{a}{\pcol@bg@@}
% 
% and thus we need triple \!\expandafter! to give them to the macro.  Prior
% to the invocatoin of the macro, we let $\!\reserved@a!=a'$ have
% $a{\cdot}|@|{\cdot}c$ if $\bgc_a^c$ is defined for $a\in\{G,C,g,c\}$, or
% $a$ otherwise definitely for $a\notin\{G,C,g,c\}$.
% 
% Then \!\pcol@bg@paintregion@i! calculates $x_0=\!\@tempdima!$,
% $y_0=\!\@tempdimb!$, $x_1=\!\@tempdimc!$ and $y_1=\!\dimen@!$ of the
% region $\bgr_a^{[c]}$ by $\!\pcol@bg@calculate!\arg{z}\arg{z_0}\Arg{F}$
% giving it
% $(z,z_0,F)\in\{(x_0,0,F_x),(y_0,0,F_y),(x_1,x_0,F_w),(y_1,y_0,F_h)\}$,
% where $(x_0,y_0)$ and $(x_1,y_1)$ is the left-top and right-bottom corner
% of the painting region in the text-area coordinate, i.e., left-right and
% top-down coordinate whose origin is at the left-top corner of the leftmost
% column.  Next we modify $\{x,y\}_{\{0,1\}}$ for \bgext{} by
% $\!\pcol@bg@addext!\arg{z}\Arg{s}\Arg{d}$ with $(z,s,d)\in
% \{(x_0,\hbox{`|-|'},|l|),(y_0,\hbox{`|-|'},|t|),
% (x_1,\emptyset,|r|),\~(y_1,\emptyset,|b|)\}$.
% 
% Now we have $[(x_0,y_0)(x_1,y_1)]$ and thus, if not \mirror{}ed, we place
% $\bgr_a^{[c]}$ at $(x_0,y_0)$ by a vertical skip of $y_0$ and shifting a
% \!\hbox! for the region right by $x_0$ by \!\moveright!, and paint the box
% putting a \!\vrule! of $(x_1-x_0)$ wide and $(y_1-y_0)$ tall, letting
% \!\current@color! have $\bgc_r^{[c]}=|\pcol@bg@colr|{\cdot}a'$ and then
% invoking \!\pcol@set@color! being the orinigal definition of
% \!\set@color!.
% 
% \SpecialArrayIndex{a{\cdot}\string\texttt{@}{\cdot}c}{\pcol@bg@color@}
% \SpecialArrayIndex{a}{\pcol@bg@color@}
% 
% On the other hand if \mirror{}ing is to be done, the region should be
% $[(\WT{-}x_1,y_0)(\WT{-}x_0,y_1)]$ and thus the shift amount for
% \!\moveright! of the \!\hbox! is $(\WT-x_1)$.
% 
% Then after \!\pcol@bg@paintregion@i! finshes its work,
% \!\pcol@bg@paintregion! lets the switch
% \CSIndex{ifpcol@bg@painted}${}=\true$ because we painted $\bgr_a^{[c]}$,
% lets the height, depth and width of \!\@tempboxa!  be 0 to make it a
% phantom, and then put it into the outside box opened by
% \!\pcol@bg@paint@i!.  On the other hand, if neither $\bgc_r^c$ nor
% $\bgc_r$ defined to mean the \bgpaint{} of the region is not specified, we
% do nothing.
% 
%    \begin{macrocode}
\def\pcol@bg@paintregion#1#2{%
  \@ifundefined{pcol@bg@color@#1@\number#2}%
    {\def\reserved@a{#1}}{\edef\reserved@a{#1@\number#2}}%
  \@ifundefined{pcol@bg@color@\reserved@a}\relax
    {\setbox\@tempboxa\vtop{\vskip\z@
       \expandafter\expandafter\expandafter
         \pcol@bg@paintregion@i\csname pcol@bg@@#1\endcsname}%
     \global\pcol@bg@paintedtrue
     \ht\@tempboxa\z@ \dp\@tempboxa\z@ \wd\@tempboxa\z@ \box\@tempboxa}}
\def\pcol@bg@paintregion@i#1#2#3#4{%
  \pcol@bg@calculate\@tempdima\z@{#1}%
  \pcol@bg@calculate\@tempdimb\z@{#2}%
  \pcol@bg@calculate\@tempdimc\@tempdima{#3}%
  \pcol@bg@calculate\dimen@\@tempdimb{#4}%
  \pcol@bg@addext\@tempdima{-}{l}\pcol@bg@addext\@tempdimc{}{r}%
  \pcol@bg@addext\@tempdimb{-}{t}\pcol@bg@addext\dimen@{}{b}%
  \vskip\@tempdimb
  \ifpcol@bg@@swap
    \advance\@tempdima-\@tempdimc \@tempdima-\@tempdima
    \advance\@tempdimc-\textwidth \@tempdimc-\@tempdimc
    \moveright\@tempdimc\hbox{%
      \advance\dimen@-\@tempdimb
      \edef\current@color{\@nameuse{pcol@bg@color@\reserved@a}}\pcol@set@color
      \vrule\@width\@tempdima\@height\dimen@}%
  \else
    \moveright\@tempdima\hbox{%
      \advance\@tempdimc-\@tempdima \advance\dimen@-\@tempdimb
      \edef\current@color{\@nameuse{pcol@bg@color@\reserved@a}}\pcol@set@color
      \vrule\@width\@tempdimc\@height\dimen@}%
  \fi}

%    \end{macrocode}
% \end{macro}\end{macro}
% 
% \KeepSpace{3}
% \begin{macro}{\pcol@bg@calculate}
% \changes{v1.3-3}{2013/09/17}
%	{Introduced for background painting.}
% \begin{macro}{\pcol@bg@advance}
% \changes{v1.3-3}{2013/09/17}
%	{Introduced for background painting.}
% \begin{macro}{\pcol@bg@negative}
% \changes{v1.3-3}{2013/09/17}
%	{Introduced for background painting.}
% \begin{macro}{\pcol@bg@nadvance}
% \changes{v1.3-3}{2013/09/17}
%	{Introduced for background painting.}
% \begin{macro}{\pcol@bg@dimen}
% \changes{v1.3-3}{2013/09/17}
%	{Introduced for background painting.}
% 
% The macro $\!\pcol@bg@calculate!\arg{z}\arg{z_0}\arg{F}$ is used in
% \!\pcol@bg@paintregion@i! and \!\pcol@bg@addext! to accumulate dimensional
% values specified in $F$ into a \!\dimen! register $z$ with initial value
% $z_0$.  The specificatoin $F$ is a sequence of $\!\@elt!\Arg{f}$ to add $f$
% to $z$, $\!\pcol@bg@negative!\Arg{F^-}$ to subtract the amount specified
% by $F^-$ from $z$, or macros expanded to either of them.
% 
% The macro makes \!\pcol@bg@dimen! \!\let!-equal to $z$ and \!\@elt! to
% \!\pcol@bg@advance!, lets $z=z_0$, and then does what is specified in $F$\@.
% Therefore, $\!\@elt!\arg{f}$ appearing directly or indirectly in $F$ does
% $\!\advance!\arg{z}\arg{f}$ for the accumulation.  On the other hand,
% \!\pcol@bg@negative! makes \!\@elt! \!\let!-equal to \!\pcol@bg@nadvance!
% to let $\!\@elt!\arg{f}$ do $\!\advance!\arg{z}{-}\arg{f}$ for
% subtraction, does $F^-$, and then remake $\!\@elt!=\!\pcol@bg@advance!$.
% Note that $f$ may be expanded to a negative amount having `|-|' its
% beginning to results in $\!\@elt!\arg{f}$ expanded to
% $\!\advance!\arg{z}{-}{-}\arg{f'}$ with some positive amount $f'$, but
% this double negation is legitimate in \TeX{} and is equivalent to
% $\!\advance!\arg{z}\arg{f'}$.  The macro \!\pcol@bg@negative! is used in
% the following macros.
% 
% \begin{quote}\raggedright
% \!\pcol@bg@ext@inf@l!,
% \!\pcol@bg@ext@inf@r!,
% \!\pcol@bg@ext@inf@t!,
% \!\pcol@bg@@t!,
% \!\pcol@bg@@b!,
% \!\pcol@bg@@l!,
% \!\pcol@bg@@r!,
% \!\pcol@bg@@n!,
% \!\pcol@bg@@p!.
% \end{quote}
% 
%    \begin{macrocode}
\def\pcol@bg@calculate#1#2#3{\let\pcol@bg@dimen#1\relax
  \let\@elt\pcol@bg@advance \pcol@bg@dimen#2\relax #3}
\def\pcol@bg@negative#1{\let\@elt\pcol@bg@nadvance #1\relax
  \let\@elt\pcol@bg@advance}
\def\pcol@bg@advance#1{\advance\pcol@bg@dimen#1\relax}
\def\pcol@bg@nadvance#1{\advance\pcol@bg@dimen-#1\relax}

%    \end{macrocode}
% \end{macro}\end{macro}\end{macro}\end{macro}\end{macro}
% 
% \KeepSpace{3}
% \begin{macro}{\pcol@bg@addext}
% \changes{v1.3-3}{2013/09/17}
%	{Introduced for background painting.}
% \begin{macro}{\pcol@bg@ext@inf@l}
% \changes{v1.3-3}{2013/09/17}
%	{Introduced for background painting.}
% \begin{macro}{\pcol@bg@ext@inf@r}
% \changes{v1.3-3}{2013/09/17}
%	{Introduced for background painting.}
% \begin{macro}{\pcol@bg@ext@inf@t}
% \changes{v1.3-3}{2013/09/17}
%	{Introduced for background painting.}
% \begin{macro}{\pcol@bg@ext@inf@b}
% \changes{v1.3-3}{2013/09/17}
%	{Introduced for background painting.}
% 
% The macro $\!\pcol@bg@addext!\arg{z}\Arg{s}\Arg{d}$ is used only in
% \!\pcol@bg@paintregion@i! but four times with $(z,s,d)\in
% \{(x_0,\hbox{`|-|'},|l|),(y_0,\hbox{`|-|'},|t|),
% (x_1,\emptyset,|r|),\~(y_1,\emptyset,|b|)\}$, to perform \bgext{} on a
% \!\dimen! register $z$.
% 
% First the macro gets $e=|\pcol@bg@ext@|{\cdot}d{\cdot}|@|{\cdot}a'
% \in e_a^{[c]}(\{x,y\}^{\{{+},{-}\}})$
% 
% \SpecialArrayIndex{d{\cdot}\string\texttt{@}
%     {\cdot}a{\cdot}\string\texttt{@}{\cdot}c}{\pcol@bg@ext@}
% \SpecialArrayIndex{d{\cdot}\string\texttt{@}{\cdot}a}{\pcol@bg@ext@}
% 
% where $a'=\!\reserved@a!\~\in\{a{\cdot}|@|{\cdot}c,a\}$.  Then if
% $e<9000\PT$ being a finite \bgext, we let $z\gets z\pm e$
% according to $s$, i.e. $+$ if $s=\emptyset$ while $-$ if $s={}$`|-|'.
% Otherwise, i.e., $e\geq9000\PT$ for a infinite \bgext, let $e'$
% be the value shown below by invoking $|\pcol@bg@ext@inf@|{\cdot}d$
% 
% \SpecialArrayMainIndex{d}{\pcol@bg@ext@inf@}
% $$
% e'=\cases{-(\WM-\WR)&$d=|l|$\cr
%           \WP-(\WM-\WR)&$d=|r|$\cr
% 	    -(\HM-\HR)&$d=|t|$\cr
% 	    \HP-(\HM-\HR)&$d=|b|$\cr}
% $$
% where $\WM-\WR$ is specified by \!\pcol@bg@pageleft! and $\HM-\HR$ by
% \!\pcol@bg@pagetop!.  Then we let $z=e'\pm(e-10000\PT)$ according to $s$
% again, i.e., move $z$ {\em inside\/} by $(10000\PT-e)$ from $e'$.
% 
%    \begin{macrocode}
\def\pcol@bg@addext#1#2#3{%
  \dimen@ii\@nameuse{pcol@bg@ext@#3@\reserved@a}\relax
  \ifdim\dimen@ii<9000\p@\relax \advance#1#2\dimen@ii
  \else
    \pcol@bg@calculate#1\z@{\@nameuse{pcol@bg@ext@inf@#3}}%
    \advance\dimen@ii-\@M\p@ \advance#1#2\dimen@ii
  \fi}
\def\pcol@bg@ext@inf@l{\pcol@bg@negative\pcol@bg@pageleft}
\def\pcol@bg@ext@inf@r{\pcol@bg@negative\pcol@bg@pageleft
  \pcol@bg@paperwidth}
\def\pcol@bg@ext@inf@t{\pcol@bg@negative\pcol@bg@pagetop}
\def\pcol@bg@ext@inf@b{\pcol@bg@negative\pcol@bg@pagetop
  \pcol@bg@paperheight}

%    \end{macrocode}
% \end{macro}\end{macro}\end{macro}\end{macro}\end{macro}
% 
% \KeepSpace{7}
% \begin{macro}{\pcol@bg@paperwidth}
% \changes{v1.3-3}{2013/09/17}
%	{Introduced for background painting.}
% \begin{macro}{\pcol@bg@paperheight}
% \changes{v1.3-3}{2013/09/17}
%	{Introduced for background painting.}
% \begin{macro}{\pcol@bg@pageleft}
% \changes{v1.3-3}{2013/09/17}
%	{Introduced for background painting.}
% \begin{macro}{\pcol@bg@pagetop}
% \changes{v1.3-3}{2013/09/17}
%	{Introduced for background painting.}
% \begin{macro}{\pcol@bg@textheight}
% \changes{v1.3-3}{2013/09/17}
%	{Introduced for background painting.}
% \begin{macro}{\pcol@bg@columnleft}
% \changes{v1.3-3}{2013/09/17}
%	{Introduced for background painting.}
% \begin{macro}{\pcol@bg@columnright}
% \changes{v1.3-3}{2013/09/17}
%	{Introduced for background painting.}
% \begin{macro}{\pcol@bg@columnwidth}
% \changes{v1.3-3}{2013/09/17}
%	{Introduced for background painting.}
% \begin{macro}{\pcol@bg@columnsep}
% \changes{v1.3-3}{2013/09/17}
%	{Introduced for background painting.}
% 
% The following macros specify the whole or a part of
% $F\in\{F_x,F_y,F_w,F_h\}$ being the body of $|\pcol@bg@@|{\cdot}a$ for
% \bgpaint{} regions $\bgr_a^{[c]}$.
% 
% \begin{eqnarray*}
% \!\pcol@bg@paperwidth!&=&	\WP-2\WR=\!\paperwidth!-2\!\pagerim!
% 	\quad(t,T,b,B,r,R)\\
% \!\pcol@bg@paperheight!&=&	\HP-2\HR=\!\paperheight!-2\!\pagerim!
% 	\quad(b,B)\\
% \!\pcol@bg@pageleft!&=&	\WM-\WR=
% 				(\!\pcol@bg@leftmargin!+1\,|in|)-\!\pagerim!\\
% &	\rlap{$(t,T,b,B,l,L,r,R)$}\\
% \!\pcol@bg@pagetop!&=&	\HM-\WR\\
% 	&=&(\!\topmargin!+\!\headheight!+\!\headsep!+1\,|in|)-\!\pagerim!\\
% &	\rlap{$(t,T,b,B)$}\\
% \!\pcol@bg@textheight!&=&	\HT=\!\textheight!+\!\@maxdepth!\\
% &	\rlap{$(b,B,l,L,r,R,C,G,n,N,p,P)$}\\
% \!\pcol@bg@columnleft!&=&	\W_c=\sum_{d=\CBfrom}^{c-1}(w_c+g_c)
% 	\quad(c,C)\\
% \!\pcol@bg@columnright!&=&	\W_c+w_c=
% 				\!\pcol@bg@columnleft!+\!\pcol@bg@columnwidth!\\
% &	\rlap{$(g,G)$}\\
% \!\pcol@bg@columnwidth!&=&	w_c=|\pcol@columnwidth|{\cdot}c
% 	\quad(c,C)\\
% \!\pcol@bg@columnsep!&=&	g_c=|\pcol@columnsep|{\cdot}c
% 	\quad(g,G)
% \end{eqnarray*}
% 
% \SpecialArrayIndex{c}{\pcol@columnwidth}
% \SpecialArrayIndex{c}{\pcol@columnsep}
% 
% Note that \!\pagerim! in $F$ means $-\!\pagerim!$ because its sign is
% reversed by \!\pcol@bg@paint@i!.  The macros are used in
% $|\pcol@bg@@|{\cdot}a$
% 
% \SpecialArrayIndex{a}{\pcol@bg@@}
% 
% whose region identifier $a$ is shown in paretheses above, but besides them
% \!\pcol@bg@paperwidth! is also used in \!\pcol@bg@ext@inf@r!,
% \!\pcol@bg@paperheight! in \!\pcol@bg@ext@inf@b!, \!\pcol@bg@pageleft! in
% \!\pcol@bg@ext@inf@l! and \!\pcol@bg@ext@inf@r!, and \!\pcol@bg@pagetop!
% in \!\pcol@bg@ext@inf@t!  and \!\pcol@bg@ext@inf@b!.  Also note that
% \!\pcol@bg@textheight! is used in \!\pcol@output@clear! while it is
% temporarily redefined in \!\pcol@output@start! and \!\pcol@output@end!.
% 
%    \begin{macrocode}
\def\pcol@bg@paperwidth{\@elt\paperwidth \@elt{2\pagerim}}
\def\pcol@bg@paperheight{\@elt\paperheight \@elt{2\pagerim}}
\def\pcol@bg@pageleft{\@elt{1in}\@elt\pcol@bg@leftmargin \@elt\pagerim}
\def\pcol@bg@pagetop{\@elt{1in}\@elt\topmargin \@elt\headheight \@elt\headsep
  \@elt\pagerim}
\def\pcol@bg@textheight{\@elt\textheight \@elt\@maxdepth}
\def\pcol@bg@columnleft{%
  \@tempcnta\pcol@bg@from \@whilenum\@tempcnta<\pcol@currcol\do{%
    \@elt{\@nameuse{pcol@columnwidth\number\@tempcnta}}%
    \@elt{\@nameuse{pcol@columnsep\number\@tempcnta}}%
   \advance\@tempcnta\@ne}}
\def\pcol@bg@columnright{\pcol@bg@columnleft \pcol@bg@columnwidth}
\def\pcol@bg@columnwidth{\@elt{\@nameuse{pcol@columnwidth\number\pcol@currcol}}}
\def\pcol@bg@columnsep{\@elt{\@nameuse{pcol@columnsep\number\pcol@currcol}}}
%    \end{macrocode}
% \end{macro}\end{macro}\end{macro}\end{macro}\end{macro}
% \end{macro}\end{macro}\end{macro}\end{macro}
% 
% \KeepSpace{7}
% \begin{macro}{\pcol@bg@preposttop}
% \changes{v1.3-3}{2013/09/17}
%	{Introduced for background painting.}
% \begin{macro}{\pcol@bg@preposttop@left}
% \changes{v1.3-3}{2013/09/17}
%	{Introduced for background painting.}
% \begin{macro}{\pcol@bg@preposttop@right}
% \changes{v1.3-3}{2013/09/17}
%	{Introduced for background painting.}
% \begin{macro}{\pcol@bg@columntop}
% \changes{v1.3-3}{2013/09/17}
%	{Introduced for background painting.}
% \begin{macro}{\pcol@bg@columnheight}
% \changes{v1.3-3}{2013/09/17}
%	{Introduced for background painting.}
% \begin{macro}{\pcol@bg@floatheight}
% \changes{v1.3-3}{2013/09/17}
%	{Introduced for background painting.}
% \begin{macro}{\pcol@bg@footnoteheight}
% \changes{v1.3-3}{2013/09/17}
%	{Introduced for background painting.}
% \begin{macro}{\pcol@bg@spanningtop}
% \changes{v1.3-3}{2013/09/17}
%	{Introduced for background painting.}
% \begin{macro}{\pcol@bg@spanningheight}
% \changes{v1.3-3}{2013/09/17}
%	{Introduced for background painting.}
% 
% Besides the macros shown above, $|\pcol@bg@@|{\cdot}a$ uses the following
% macros defined by macros using \!\pcol@bg@paintpage!,
% \!\pcol@bg@paintcolumns! or \!\pcol@bg@paintbox!.
% 
% \begin{itemize}
% \item
% \!\pcol@bg@preposttop! being \!\pcol@bg@preposttop@left! or
% \!\pcol@bg@preposttop@right! for $a\in\{p,P\}$ by \!\@outputpage! and
% \!\pcol@output@end!, the latter of which may define only the left one if
% the closing environment is not \parapag{}ed.  That is, both of left and
% right macros are usually equivalent, but the right one can be smaller than
% the left if we have two or more (\lpage{}s of) \env{paracol} environments
% in a page and the closing environment is not \parapag{}ed while some
% others are.  In such case, \!\@outputpage! or \!\pcol@output@start!,
% another macro referring to them, must paint the region below
% \!\pcol@bg@preposttop@right! in the right page as a part of \preenv{} or
% \postenv{} by \!\let!ting \!\pcol@bg@preposttop! be
% \!\pcol@bg@preposttop@left! and \!\pcol@bg@preposttop@right! for the left
% and right \parapag{}es respectively.  Both macros have common initial
% value 0.
% 
% \item
% \!\pcol@bg@columntop! and \!\pcol@bg@columnheight! for $a\in\{c,g\}$ by
% \!\pcol@build~colseprule! and \!\pcol@buildcselt!.
% 
% \item
% \!\pcol@bg@spanningtop! and \!\pcol@bg@spanningheight! for $a\in\{s,S\}$ by
% \!\pcol@buildcselt!.
% 
% \item
% \!\pcol@bg@floatheight! for $a\in\{f,F\}$ by \!\pcol@outputelt!,
% \!\pcol@ioutputelt!, \!\pcol@output@clear! and \!\pcol@makeflushedpage!
% 
% \item
% \!\pcol@bg@footnoteheight! for $a\in\{n,N\}$ by \!\pcol@ioutputelt!,
% \!\pcol@imakeflushed~page! and \!\pcol@output@end!.
% \end{itemize}
% 
%    \begin{macrocode}
\def\pcol@bg@preposttop@left{0pt}
\let\pcol@bg@preposttop@right\pcol@bg@preposttop@left

%    \end{macrocode}
% \end{macro}\end{macro}\end{macro}\end{macro}\end{macro}
% \end{macro}\end{macro}\end{macro}\end{macro}
% 
% \KeepSpace{18}
% \begin{macro}{\pcol@bg@@c}
% \changes{v1.3-3}{2013/09/17}{Introduced for background painting.}
% \begin{macro}{\pcol@bg@@C}
% \changes{v1.3-3}{2013/09/17}{Introduced for background painting.}
% \begin{macro}{\pcol@bg@@g}
% \changes{v1.3-3}{2013/09/17}{Introduced for background painting.}
% \begin{macro}{\pcol@bg@@G}
% \changes{v1.3-3}{2013/09/17}{Introduced for background painting.}
% \begin{macro}{\pcol@bg@@s}
% \changes{v1.3-3}{2013/09/17}{Introduced for background painting.}
% \begin{macro}{\pcol@bg@@S}
% \changes{v1.3-3}{2013/09/17}{Introduced for background painting.}
% \begin{macro}{\pcol@bg@@t}
% \changes{v1.3-3}{2013/09/17}{Introduced for background painting.}
% \begin{macro}{\pcol@bg@@T}
% \changes{v1.3-3}{2013/09/17}{Introduced for background painting.}
% \begin{macro}{\pcol@bg@@b}
% \changes{v1.3-3}{2013/09/17}{Introduced for background painting.}
% \begin{macro}{\pcol@bg@@B}
% \changes{v1.3-3}{2013/09/17}{Introduced for background painting.}
% \begin{macro}{\pcol@bg@@l}
% \changes{v1.3-3}{2013/09/17}{Introduced for background painting.}
% \begin{macro}{\pcol@bg@@L}
% \changes{v1.3-3}{2013/09/17}{Introduced for background painting.}
% \begin{macro}{\pcol@bg@@r}
% \changes{v1.3-3}{2013/09/17}{Introduced for background painting.}
% \begin{macro}{\pcol@bg@@R}
% \changes{v1.3-3}{2013/09/17}{Introduced for background painting.}
% \begin{macro}{\pcol@bg@@f}
% \changes{v1.3-3}{2013/09/17}{Introduced for background painting.}
% \begin{macro}{\pcol@bg@@F}
% \changes{v1.3-3}{2013/09/17}{Introduced for background painting.}
% \begin{macro}{\pcol@bg@@n}
% \changes{v1.3-3}{2013/09/17}{Introduced for background painting.}
% \begin{macro}{\pcol@bg@@N}
% \changes{v1.3-3}{2013/09/17}{Introduced for background painting.}
% \begin{macro}{\pcol@bg@@p}
% \changes{v1.3-3}{2013/09/17}{Introduced for background painting.}
% \begin{macro}{\pcol@bg@@P}
% \changes{v1.3-3}{2013/09/17}{Introduced for background painting.}
% 
% The macros $|\pcol@bg@@|{\cdot}a$
% 
% \SpecialArrayMainIndex{a}{\pcol@bg@@}
% 
% define arguments $F_x$, $F_y$, $F_w$ and $F_h$ to be passed to
% \!\pcol@bg@paintregion@i! in their bodies to calculate $(x_0,y_0)$ and
% $(x_1{-}x_0,y_1{-}y_0)$ for regions $\bgr_a^{[c]}$ as shown below in the
% form of $(x_0,y_0)+(x_1{-}x_0,y_1{-}y_0)$ (to have $(x_1,y_1)$), where
% $H^c=\!\pcol@bg@columntop!$, $h^c=\!\pcol@bg@columnheight!$,
% $H^s=\!\pcol@bg@spanningtop!$, $h^s=\!\pcol@bg@spanningheight!$,
% $h^f=\!\pcol@bg@floatheight!$, $h^n=\!\pcol@bg@footnoteheight!$ and
% $H^p=\!\pcol@bg@preposttop!$ calculated by macros which invoke \bgpaint{}
% macros, while $s^n=\!\skip!\!\footins!$.
% 
% \begin{eqnarray*}
% \bgr_c^c&\;:\;&(\W_c\;,\;H^c)\ +\ (w_c\;,\;h^c)\\
% \bgr_C^c&:&(\W_c\;,\;0)\ +\ (w_c\;,\;\HT)\\
% \bgr_g^c&:&((\W_c+w_c)\;,\;H^c)\ +\ (g_c\;,\;h^c)\\
% \bgr_G^c&:&((\W_c+w_c)\;,\;0)\ +\ (g_c\;,\;\HT)\\
% \bgr_{\{s,S\}}&:&(0\;,\;H^s)\ +\ (\WT\;,\;h^s)\\
% \bgr_{\{t,T\}}&:&(-(\WM-\WR)\;,\;-(\HM-\HR))\ +\ ((\WP-2\WR)\;,\;\HM-\HR)\\
% \bgr_{\{b,B\}}&:&(-(\WM-\WR)\;,\;\HT)\ +\ 
%                  ((\WP-2\WR)\;,\;(\HP-2\HR)-((\HM-\HR)+\HT))\\
% \bgr_{\{l,L\}}&:&(-(\WM-\WR)\;,\;0)\ +\ ((\WM-\WR)\;,\;\HT)\\
% \bgr_{\{r,R\}}&:&(\WT\;,\;0)\ +\ ((\WP-2\WR)-((\WM-\WR)+\WT)\;,\;\HT)\\
% \bgr_{\{f,F\}}&:&(0\;,\;0)\ +\ (\WT\;,\;h^f)\\
% \bgr_{\{n,N\}}&:&(0\;,\;\HT-(h^n+s^n))\ +\ (\WT\;,\;h^n+s^n)\\
% \bgr_{\{p,P\}}&:&(0\;,\;H^p)\ +\ (\WT\;,\;\HT-H^p)
% \end{eqnarray*}
% 
%    \begin{macrocode}
\def\pcol@bg@@c{%
  {\pcol@bg@columnleft}%
  {\@elt\pcol@bg@columntop}%
  {\pcol@bg@columnwidth}%
  {\pcol@bg@columnheight}}
\def\pcol@bg@@C{%
  {\pcol@bg@columnleft}%
  {}%
  {\pcol@bg@columnwidth}%
  {\pcol@bg@textheight}}
\def\pcol@bg@@g{%
  {\pcol@bg@columnright}%
  {\@elt\pcol@bg@columntop}%
  {\pcol@bg@columnsep}%
  {\pcol@bg@columnheight}}
\def\pcol@bg@@G{%
  {\pcol@bg@columnright}%
  {}%
  {\pcol@bg@columnsep}%
  {\pcol@bg@textheight}}
\def\pcol@bg@@s{%
  {}%
  {\pcol@bg@spanningtop}%
  {\@elt\textwidth}%
  {\pcol@bg@spanningheight}}
\def\pcol@bg@@t{%
  {\pcol@bg@negative\pcol@bg@pageleft}%
  {\pcol@bg@negative\pcol@bg@pagetop}%
  {\pcol@bg@paperwidth}%
  {\pcol@bg@pagetop}}
\def\pcol@bg@@b{%
  {\pcol@bg@negative\pcol@bg@pageleft}%
  {\pcol@bg@textheight}%
  {\pcol@bg@paperwidth}%
  {\pcol@bg@paperheight
    \pcol@bg@negative{\pcol@bg@pagetop \pcol@bg@textheight}}}
\def\pcol@bg@@l{%
  {\pcol@bg@negative\pcol@bg@pageleft}%
  {}%
  {\pcol@bg@pageleft}%
  {\pcol@bg@textheight}}
\def\pcol@bg@@r{%
  {\@elt\textwidth}%
  {}%
  {\pcol@bg@paperwidth
    \pcol@bg@negative{\pcol@bg@pageleft \@elt\textwidth}}%
  {\pcol@bg@textheight}}
\def\pcol@bg@@f{%
  {}%
  {}%
  {\@elt\textwidth}%
  {\pcol@bg@floatheight}}
\def\pcol@bg@@n{%
  {}%
  {\pcol@bg@textheight
    \pcol@bg@negative{\pcol@bg@footnoteheight \@elt{\skip\footins}}}%
  {\@elt\textwidth}%
  {\pcol@bg@footnoteheight \@elt{\skip\footins}}}
\def\pcol@bg@@p{%
  {}%
  {\@elt\pcol@bg@preposttop}%
  {\@elt\textwidth}%
  {\pcol@bg@textheight \pcol@bg@negative{\@elt\pcol@bg@preposttop}}}
\let\pcol@bg@@S\pcol@bg@@s
\let\pcol@bg@@T\pcol@bg@@t
\let\pcol@bg@@B\pcol@bg@@b
\let\pcol@bg@@L\pcol@bg@@l
\let\pcol@bg@@R\pcol@bg@@r
\let\pcol@bg@@F\pcol@bg@@f
\let\pcol@bg@@N\pcol@bg@@n
\let\pcol@bg@@P\pcol@bg@@p

%    \end{macrocode}
% \end{macro}\end{macro}\end{macro}\end{macro}\end{macro}
% \end{macro}\end{macro}\end{macro}\end{macro}\end{macro}
% \end{macro}\end{macro}\end{macro}\end{macro}\end{macro}
% \end{macro}\end{macro}\end{macro}\end{macro}\end{macro}
% 
% 
% 
% \KeepSpace{10}
% \section{Special Output Routines}
% \label{sec:imp-sout}
% 
% \subsection{Dispatcher}
% \label{sec:imp-sout-disp}
% 
% \changes{v1.22}{2013/06/30}
%	{\cs{pcol@op@cpush} was introduced for output request to push color
%	 stack but removed in v1.34.} 
% \changes{v1.34}{2018/05/07}
%	{\cs{pcol@op@cpush} is removed according to the change of text
%	 coloring from \cs{output} to \cs{insert}.}
% \changes{v1.22}{2013/06/30}
%	{\cs{pcol@op@cpop} was introduced for output request to pop color
%	 stack but removed in v1.34.} 
% \changes{v1.34}{2018/05/07}
%	{\cs{pcol@op@cpop} is removed according to the change of text
%	 coloring from \cs{output} to \cs{insert}.}
% \changes{v1.22}{2013/06/30}
%	{\cs{pcol@op@cset} was introduced for output request to set
%	 $\gamma_0^c$ but removed in v1.34}
% \changes{v1.34}{2018/05/07}
%	{\cs{pcol@op@cpop} is removed according to the change of text
%	 coloring from \cs{output} to \cs{insert}.}
% 
% \begin{macro}{\pcol@op@start}
% \begin{macro}{\pcol@op@switch}
% \begin{macro}{\pcol@op@flush}
% \begin{macro}{\pcol@op@clear}
% \begin{macro}{\pcol@op@end}
% 
% The macro $|\pcol@op@|{\cdot}f$ where $f\in
% F=\{|start|,|switch|,|flush|,|clear|,|end|\}$ has
% our own \!\outputpenalty! code less than $-10000$ to invoke the
% corresponding macro $|\pcol@output@|{\cdot}f$.  The code macros are given
% to \!\pcol@invokeoutput! as its argument by \!\pcol@zparacol! ($f=|start|$),
% \!\pcol@switchcol! ($f=|switch|$), \!\pcol@visitallcols! ($f=|switch|$),
% \!\pcol@com@flushpage! ($f=|flush|$), \!\pcol@com@clearpage!
% ($f=|clear|$), \!\pcol@flushclear! ($f=|switch|$), and \!\endparacol!
% ($f=|end|$) to set one of them into \!\outputpenalty!, so that the other
% user \!\pcol@specialoutput! examines which special function is invoked.
% 
% \SpecialIndex{\pcol@output@start}
% \SpecialIndex{\pcol@output@switch}
% \SpecialIndex{\pcol@output@flush}
% \SpecialIndex{\pcol@output@clear}
% \SpecialIndex{\pcol@output@end}
% 
%    \begin{macrocode}
%% Special Output Routines: Dispatcher

\def\pcol@op@start{-10010}
\def\pcol@op@switch{-10011}
\def\pcol@op@flush{-10012}
\def\pcol@op@clear{-10013}
\def\pcol@op@end{-10014}

%    \end{macrocode}
% \end{macro}\end{macro}\end{macro}\end{macro}\end{macro}
% 
% \changes{v1.24}{2013/07/27}
%	{\cs{pcol@op@mcpush} was introduced for coloring specified in math
%	 mode but removed in v1.34.} 
% \changes{v1.34}{2018/05/07}
%	{\cs{pcol@op@mcpush} is removed according to the change of text
%	 coloring from \cs{output} to \cs{insert}.}
% \changes{v1.24}{2013/07/27}
%	{\cs{pcol@op@mcpush@pone} was introduced for coloring specified in
%	 math mode but removed in v1.34.}
% \changes{v1.34}{2018/05/07}
%	{\cs{pcol@op@mcpush@pone} is removed according to the change of text
%	 coloring from \cs{output} to \cs{insert}.}
% \changes{v1.24}{2013/07/27}
%	{\cs{pcol@op@mcpop} was introduced for coloring specified in math
%	 mode but removed in v1.34.} 
% \changes{v1.34}{2018/05/07}
%	{\cs{pcol@op@mcpop} is removed according to the change of text
%	 coloring from \cs{output} to \cs{insert}.}
% \changes{v1.24}{2013/07/27}
%	{\cs{pcol@op@mcpop@pone} was introduced for coloring specified in
%	 math mode but removed in v1.34.}
% \changes{v1.34}{2018/05/07}
%	{\cs{pcol@op@mcpop@pone} is removed according to the change of text
%	 coloring from \cs{output} to \cs{insert}.}
% 
% \begin{macro}{\pcol@specialoutput}
% \changes{v1.0}{2011/10/10}
%	{Remove unnecessary \cs{pcol@latex@specialoutput}.}
% \changes{v1.2-2}{2013/05/11}
%	{Add footnote logging if $\cs{outputpenalty}\EQ-10004$.}
% \changes{v1.22}{2013/06/30}
%	{Add the invocation of \cs{pcol@output@}${\cdot}f$ for
%	 $f\in\{\string\texttt{cpush},\string\texttt{cpop},
% 	 \string\texttt{cset}\}$}
% \changes{v1.24}{2013/07/27}
%	{Add examination with $P_{\string\rm{push}}$ and $P_{\string\rm{pop}}$
%	 and invocation of \cs{pcol@output@mcpush} and
%	 \cs{pcol@output@mcpop}.} 
% \changes{v1.34}{2018/05/07}
%	{Remove examinations related to $f\in\{{\tt cpush},{\tt cpop},{\tt
%	 mcpush},{\tt mcpop}\}$.}
% 
% The macro \!\pcol@specialoutput! is invoked solely in \!\pcol@output! to
% invoke our own or \LaTeX's special output routine.  It examines if
% $P=\!\outputpenalty!\in\Set{\cs{pcol@op@}{\cdot}f}{f\in F}$ and then, if
% so, before invoking $|\pcol@output@|{\cdot}f$, we rebuild \!\@holdpg!
% removing \!\lastbox! and the last vertical skip as done in \LaTeX's
% \!\@specialoutput!.  We also let $\!\outputpenalty!=-10000$\footnote{
% 
% It can be any value larger than -10004.}
% 
% so that \!\vsize! is correctly set to \!\@colroom! in the second half of
% \!\pcol@output! after this macro finishes.
% 
% \hfuzz9pt
% Otherwise, i.e., if $P\notin\Set{\cs{pcol@op@}{\cdot}f}{f\in{}F}$,
% we simply invokes \LaTeX's \!\@specialoutput!\footnote{
% 
% With footnote logging if $\cs{outputpenalty}=-10004$.}.
% 
%    \begin{macrocode}
\def\pcol@specialoutput{%
  \ifnum\outputpenalty=\pcol@op@start\relax
    \let\reserved@a\pcol@output@start
  \else\ifnum\outputpenalty=\pcol@op@switch\relax
    \let\reserved@a\pcol@output@switch
  \else\ifnum\outputpenalty=\pcol@op@flush\relax
    \let\reserved@a\pcol@output@flush
  \else\ifnum\outputpenalty=\pcol@op@clear\relax
    \let\reserved@a\pcol@output@clear
  \else\ifnum\outputpenalty=\pcol@op@end\relax
    \let\reserved@a\pcol@output@end
  \else \let\reserved@a\@specialoutput
  \fi\fi\fi\fi\fi
  \ifnum\outputpenalty=-\@Miv\relax
    \ifvoid\footins\else \pcol@Log\dummy{dummy}\footins \fi
  \fi
  \ifx\reserved@a\@specialoutput\else
    \global\setbox\@holdpg\vbox{\unvbox\@holdpg \unvbox\@cclv
      \setbox\@tempboxa\lastbox \unskip}%
    \outputpenalty-\@M
  \fi
  \reserved@a}

%    \end{macrocode}
% \end{macro}
% 
% 
% 
% \subsection{Building Starting Page} \label{sec:imp-sout-start}
% 
% \begin{macro}{\pcol@output@start}
% \changes{v1.0}{2011/10/10}
% 	{Change the order of operations.}
% \changes{v1.0}{2011/10/10}
%	{Rename \cs{pcol@maxpage} as \cs{pcol@toppage}.}
% \changes{v1.0}{2011/10/10}
%	{Add special operation in case of too small room for column-pages.}
% \changes{v1.2-2}{2013/05/11}
%	{Add initialization of $\string\mathit{\string\Phi}$ and
%	 page-wise footnote output operations, and revise reflecting
%	 the redesign of page context.} 
% \changes{v1.2-7}{2013/05/11}
%	{Add \cs{pcol@outputtrue} to solve the \cs{output} request sneaking.}
% \changes{v1.2-7}{2013/05/11}
%	{Include the effect of the separation of pre-environment bottom
%	 floats and columns in the starting page into the check of too large
%	 pre-envronment stuff.}
% \changes{v1.3-6}{2013/09/17}
%	{Change the page builder for too tall pre-environment stuff from
%	 \cs{pcol@makenormalcol} to \cs{@makecol} because the page should be
%	 built by the ordinary mechanism.}
% \changes{v1.3-3}{2013/09/17}
%	{Let $\cs{ifpcol@output}\EQ\string\mathit{false}$ temporarily before
%	 the invocation of \cs{@outputpage} for too tall pre-environment
%	 stuff because the page is considered as outside
%	 \string\texttt{paracol} envrionments.}
% \changes{v1.32-3}{2015/10/10}
% 	{Add depth clearing of imported deferred floats in case that some of
%	 them has \texttt{1sp}.}
% \changes{v1.35-4}{2018/12/31}
% 	{Add \cs{belowfootnoteskip} to $H_f$ being the space for the
%	 non-merged pre-environment footnotes.}
% 
% The macro \!\pcol@output@start! is invoked solely from
% \!\pcol@specialoutput! to process the special \!\output! request made in
% \!\pcol@zparacol! and to build the {\em\Uidx\spage} from which parallel
% columns start possibly with the stuff preceding \beginparacol{}, or
% {\em\Uidx\preenv} in short.  First, we turn
% $\CSIndex{ifpcol@output}=\true$ so that \!\output!  requests for page
% breaks are processed by our own macros such as \!\pcol@makecol! hereafter.
% Then we let $p=\pbase={\ptop}=0$ and $\PP=\emptyset$ because we have
% nothing for $q<\ptop=0$.  We also move \!\@deferlist! to \!\@dbldeferlist!
% and then let \!\@deferlist! be empty because all \cwise{} dererred floats
% become \pwise{}.  In this float importation, as discussed in
% item-(\ref{item:ovv-float-@output@start}) of
% \secref{sec:imp-ovv-float}, we force all floats in the list have
% depth 0 to ensure no one has |1sp| to conform our own and old-fashioned
% \pwise{} float placement mechanism\footnote{
% 
% Though having \texttt{1sp} is almost imposiible.}.
% 
% We then and let $\df=\bot$ because we don't have any deferred footnotes.
% 
% Next we caculate $H=H_r-(H_m+H_f+H_b)$ where $H_r=\!\@colroom!$; $H_m$ is
% the height-plus-depth of the main vertical list in \!\@holdpg!; $H_f$ is
% the sum of \!\skip!\!\footins!, the height-plus-depth of \!\footins! and
% \!\belowfootnoteskip!, if \!\footins! is not $\bot$ or 0 otherwise; and
% $H_b=\!\textfloatsep!$ if the \preenv{} has bottom floats or 0 otherwise.
% That is, $H$ is the room for each of \colpage{} in the \spage.  Then we
% examine if $H<1.5\times\!\baselineskip!$ to mean \!\pcol@output!  would
% force a page break with warning.  If so, we assume we have a page break
% before \beginparacol{} to ship out \preenv{} to avoid the warning.
% Therefore, we invoke \LaTeX's \!\@makecol!\footnote{
% 
% \Sloppy{2500}%
% We can be unaware of our customization for \sync{}ation in
% \CSIndex{pcol@combinefloats} because \CSIndex{pcol@textfloatsep} is
% made $\infty$ by \CSIndex{pcol@zparacol}.}
% 
% giving it \!\@holdpg!  through \!\@cclv! to build the ship-out image in
% \!\@outputbox!.  Then the box is passed to \!\@outputbox! for which we
% temporarily let $\CSIndex{ifpcol@output}=\false$ because the page is
% assumed to be outside the \env{paracol} environment having just started.
% 
% After that we invoke \!\pcol@startpage! to let it produce $\pp(\ptop)$ for
% the \spage{} $\ptop=0$ letting $\!\pcol@currpage!$ be empty so that the
% macro will not refer to it.  The page $\pp(0)$ is usually empty but can
% have non-empty $\pp^i(0)$ with imported deferred floats which are now
% \pwise.  Moreover, we can have two or more pages if deferred \pwise{}
% floats produce \fpage{s}.  However, we can be unaware of these effects of
% floats because the resulting $\PP^+$ with them is correct of course.
% 
% Then let $\!\topskip!=\!\pcol@topskip!$ being the value at \beginparacol,
% and $\cs{ifpcol@}\~|firstpage|={\false}$, because we have the \spage{}
% without \preenv{} and thus the first item of each column will be at its
% top.
% 
% \CSINDEX{ifpcol@firstpage}
% 
%    \begin{macrocode}
%% Special Output Routines: Building First Page

\def\pcol@output@start{%
  \global\pcol@outputtrue
  \global\pcol@page\z@ \global\pcol@toppage\z@ \global\pcol@basepage\z@
  \global\let\pcol@pages\@empty
  \global\let\@dbldeferlist\@deferlist \global\let\@deferlist\@empty
  {\def\@elt##1{\global\dp##1\z@}\@dbldeferlist}%
  \setbox\z@\box\pcol@topfnotes
  \@tempdima\@colroom
  \advance\@tempdima-\ht\@holdpg \advance\@tempdima-\dp\@holdpg
  \ifvoid\footins\else
    \advance\@tempdima-\skip\footins
    \advance\@tempdima-\ht\footins \advance\@tempdima-\dp\footins
    \advance\@tempdima-\belowfootnoteskip
  \fi
  \ifx\@botlist\@empty\else \advance\@tempdima-\textfloatsep \fi
  \ifdim\@tempdima<1.5\baselineskip
    \setbox\@cclv\box\@holdpg \@makecol
    \pcol@outputfalse \@outputpage \pcol@outputtrue
    \global\let\pcol@currpage\@empty \pcol@startpage
    \global\topskip\pcol@topskip \global\pcol@firstpagefalse
%    \end{macrocode}
% 
% Otherwise, i.e., if $H\geq1.5\times\!\baselineskip!$, we invoke
% \!\pcol@makenormalcol! to make the \preenv{} as the \spanning{} of the
% \spage.  The macro is different from \!\@makecol! as follows;  the height
% of resulting \!\@outputbox! is natural rather than \!\textheight!;
% \Mgfnote{}s is excluded if any; and the skip of \!\textfloatsep! is added
% below the bottom floats also if any,
% 
% Then we let $h$ be the height-plus-depth of \!\@outputbox! being the
% \spanning{} and shrink \!\@colht! by $h$.  Next if
% $h>\HB=\!\pcol@bg@preposttop!
% \in\{\!\pcol@bg@preposttop@left!,\~\!\pcol@bg@preposttop@right!\}$, being
% the bottom of the previous \env{paracol} environment (having right
% \parapag{}e) or 0 if the curret page does not have it, to mean we have
% ordinary single-columned stuff in \preenv{}, we paint its \bground{} by
% \!\pcol@bg@paintbox! temporarily letting $\!\pcol@bg@textheight!=h$ so
% that $y_0=\HB$ and $y_1=h$ for $\bgr_{\{p,P\}}$.  This \bgpaint{} is not
% only for $\pp^b(0)$ which we acquire from from \!\@freelist! by \!\@next!
% and let have the \spanning, but also for \!\pcol@rightpage! if $\CL<\C$ to
% mean \parapag{}ing for which we temporarily increment \!\c@page! by one if
% \npaired.
% 
% We also let $\pp^h(0)$ be the shrunk \!\@colht!, and $\pp^t(0)$ be
% \!\topskip! if $h=0$ assuming that the page does
% not have any \spanning{}\footnote{
% 
% Checking the emptiness by \cs{pcol@ifempty} does not work well for the
% very first page of a document because it has a \cs{write} as the very
% first item.}
% 
% to typeset \colpage{}s from the top of the page, or otherwise be 0
% together with \!\topskip! to inhibit the ordinary \!\topskip! insertion.
% 
% As for $\pp^m(0)$, we define it as follows, referring to
% $\mpbout=\!\pcol@mparbottom@out!=\{\mpb_L^l,\mpb_L^r,\mpb_R^l,\mpb_R^r\}$,
% where $\mpb_X^x$ has exactly one element $\mpar(h,t)$ which may be the
% position of last marginal notes in the last \env{paracol} environment in
% the page we are working on, or $\mpb_X^x=\{\mpar(0,0)\}$ if such marginal
% note or the environment itself does not exist in the page.  On the other
% hand, $B=\!\@mparbottom!$ may have non-zero for the bottom edge of the
% last marginal note in \preenv{} including the last \env{paracol}
% environment if any.  Therefore, what we need to do is to let
% $\mpb_L^x=\{\mpar(0,B)\}$ to reflect the marginal node whose bottom is at
% $B$ and which can be different from what $\mpb_L^x$ had, where $x$ is the
% target margin in the \preenv{} determined by \CSIndex{if@mparswitch}, the
% parity of $\page(0)$ and \CSIndex{if@reversemargin}.
% 
% The replacement is done by \!\pcol@do@mpbout! which invokes
% $\!\pcol@do@mpbout@whole!\arg{m_L^l}\~\arg{m_L^r}\arg{\mpb_R^l}\arg{\mpb_R^r}$
% where $m_L^x\in\{M_L^x,\,\!\pcol@do@mpbout@elem!\arg{M_L^x}\}$ whose choice
% is made according that $x\in\{l,r\}$ is the target margin (latter) or not
% (former).  Therefore, prior to the invocation of \!\pcol@do@mpbout!, we
% \!\def!fine \!\pcol@do@mpbout@whole! so that it \!\xdef!ines
% $\mpbout=\!\pcol@mparbottom@out!$ with its four arguments, and
% \!\pcol@do@mpbout@elem! to let it be expanded to
% $\!\@elt!\Arg{0}\Arg{B}=\mpar(0,B)$.  After that, we also invoke
% \!\pcol@bias@mpbout! giving it $-h$ to replace $\mpar(t,b)$ being the sole
% element of each $\mpb_{\{L,R\}}^{\{l,r\}}$ in the resulting $\mpbout$ with
% $\mpar(t-h,b-h)$ to have what we give to $\pp^m(0)$.  This replacement
% transforms the coordinates for text area to that for columns, and makes it
% possible for the first marginal note in each margin in the \env{paracol}
% environment we now start to exploit the space for \preenv{} even if it is
% tall extraordinarily.
% 
% Then we let $\pp(0)$ have $\pp^i(0)$ and $\pp^m(0)$ shown above, and
% $\pp^p(0)=\!\c@page!$, $\pp^f(0)=\bot$ and $\pp^s(0)=\emptyset$ by
% \!\pcol@defcurrpage!, and let $\CSIndex{ifpcol@firstpage}=\true$ because
% $\pp^b(0)$ has \preenv.
% 
% \changes{v1.3-6}{2013/09/17}
%	{Delete the argument of \cs{pcol@makenormalcol} because now it is
%	 not used too tall pre-environment stuff.}
% \changes{v1.3-3}{2013/09/17}
%	{Add background painting of pre-environment stuff.}
% \changes{v1.3-4}{2013/09/17}
%	{Add initialization of $\pi^m(0)$.}
% \changes{v1.32-2}{2015/10/10}
% 	{Add \cs{pcol@Fb}/\cs{pcol@Fe} pair(s).}
% 
%    \begin{macrocode}
  \else
    \pcol@makenormalcol
    \@tempdima\ht\@outputbox \advance\@tempdima\dp\@outputbox
    \global\advance\@colht-\@tempdima
    \def\reserved@a{%
      \ifdim\pcol@bg@preposttop=\@tempdima\else
        \edef\pcol@bg@textheight{\@elt{\number\@tempdima sp}}%
        \pcol@bg@paintbox{Pp}%
      \fi}
    \ifnum\pcol@ncolleft<\pcol@ncol
      \global\setbox\pcol@rightpage\vbox{%
        \ifpcol@paired\else \advance\c@page\@ne \fi
        \let\pcol@bg@preposttop\pcol@bg@preposttop@right
        \reserved@a \unvbox\pcol@rightpage}%
    \fi
    \pcol@Fb
    \@next\@currbox\@freelist{\global\setbox\@currbox\vbox{%
      \let\pcol@bg@preposttop\pcol@bg@preposttop@left
      \reserved@a \unvbox\@outputbox}}\pcol@ovf
    \pcol@Fe{output@start(preenv)}%
    \global\dimen\@currbox\@colht
    \ifdim\@tempdima=\z@ \@tempskipa\topskip \else \@tempskipa\z@ \fi
    \global\skip\@currbox\@tempskipa \global\topskip\@tempskipa
    \def\pcol@do@mpbout@whole##1##2##3##4{%
      \xdef\pcol@mparbottom@out{{##1}{##2}{##3}{##4}}}%
    \def\pcol@do@mpbout@elem\@elt##1##2{\@elt{0}{\number\@mparbottom}}%
    \pcol@do@mpbout
    \pcol@bias@mpbout{-\@tempdima}%
    \pcol@defcurrpage{\number\c@page}\@currbox\voidb@x{}{\pcol@mparbottom@out}%
    \global\pcol@firstpagetrue
  \fi
%    \end{macrocode}
% 
% Then regardless of $H$, we do the followings for all columns $c\In0\C$ to
% build $\cc_c$, after initializing \!\@colroom! to be \!\@colht!, and
% invoking \!\pcol@floatplacement! to reinitialize the parameters of
% \cwise{} float placement.
% 
% First, if we have let $\!\topskip!=0$ with the \preenv,
% we let $\cc_c(\vb^b)$ have an invisible \!\hrule! whose
% height and depth are 0 as the very first vertical item of the \colpage.
% When we visit the column $c$ for the first time afterward, we will
% \!\unvbox! the box to let \TeX's page builder have $\!\topskip!=0$ and the
% invisible rule.  Then the first vertical item of the \colpage{} is added
% but it is recognized as non-first by \TeX's page builder and thus it
% inserts \!\baselineskip! referring to \!\prevdepth! as the depth of the
% last item.  The important issue is that the \!\prevdepth! to be referred
% is assured having its value at \beginparacol, which is usually the depth
% of the last item of \spanning, by the following mechanism: (1)
% \!\pcol@invokeoutput! invoked in \!\pcol@zparacol! saves \!\prevdepth! in
% \!\pcol@prevdepth! before the \!\output! request for
% \!\pcol@output@start!; (2) \!\pcol@prevdepth! is saved in $\cc_c(\pd)$ by
% \!\pcol@setcurrcolnf!  invoked from \!\pcol@output@start! as discussed
% afterward; (3) when the column $c$ is visited for the first time, the
% special output routine \!\pcol@output@start! itself ($c=0$) or
% \!\pcol@output@switch! ($c>0$) restores \!\pcol@prevdepth! from
% $\cc_c(\pd)$ by \!\pcol@getcurrcol!; (4) \!\pcol@invokeoutput!  which made
% the \!\output! request for (3) lets \!\prevdepth! have the value of
% \!\pcol@prevdepth! after the request.  Therefore, the baseline progress
% from the last line of the \spanning{} to the first line of each
% \colpage{} should be very natural as we see in the third and fourth lines
% of \secref{sec:man-close} of Part \ref{part:man}.
% 
% Then we invoke \!\pcol@setcurrcolnf! to save the following values for
% $\cc_c(e)$ ($e\neq\vb$); $\cc_c(\ft)=\!\voidb@x!$ because $c$ does not
% have \Mcfnote{}s so far;  $\cc_c(\pd)=\!\prevdepth!$ as discussed above;
% $\cc_c(\tl)=\cc_c(\ml)=\cc_c(\bl)=\emptyset$ because
% \!\pcol@makenormalcol! and \!\@combinefloats! invoked from it emptied
% them; $\cc_c(\dl)=\emptyset$ as discussed above;
% $\cc_c(\tn)=\!\c@topnumber!$, $\cc_c(\bn)=\!\c@botnumber!$ and
% $\cc_c(\tn)=\!\c@totalnumber!$ as initialized by \!\@floatplacement!
% invoked from \!\pcol@floatplacement!;
% $\cc_c(\tr)=\!\topfraction!\times\!\@colht!$ and
% $\cc_c(\br)=\!\bottomfraction!\times\!\@colht!$ as initialized by
% \!\@floatplacement!; $\cc_c(\sw)$ is defined by \CSIndex{if@nobreak} and
% \cs{if@after}\~|indent|
% 
% \expandafter\SpecialIndex\csname if@afterindent\endcsname
% 
% at the time of \beginparacol; and
% $\cc_c(\ep)=\!\everypar!$ at the time of \beginparacol.  We also let
% $\cc_c(\vb^p)=0$ because $p=0$ and $\cc_c(\vb^r)=\!\@colroom!$ defined
% above.  In addition, we let $\S_c=\emptyset$ because we don't have any
% \colpage{} having been completed.
% 
% \SpecialArrayIndex{c}{\pcol@columncolor}
% \SpecialArrayIndex{c}{\pcol@columncolor@box}
% We also examine if $\Celtshadow^c=\cs{pcol@columncolor}\cdot c$ is defined
% and, if so, acquire an \!\insert! from \!\@freelist! to let
% $\Celt^c=\cs{pcol@columncolor@box}\cdot c$ have the coloring \!\special!
% for the color defined in $\Celtshadow^c$ by invoking \!\pcol@set@color!
% being the origina \!\set@color! with nullification of \!\aftergroup!.
% Otherwise, we let $\Celt^c=\bot$ .
% 
% \changes{v1.0}{2011/10/10}
%	{Add clearing of $S_c$.}
% \changes{v1.32-2}{2015/10/10}
% 	{Add \cs{pcol@Fb}/\cs{pcol@Fe} pair(s).}
% \changes{v1.34}{2018/05/07}
%	{Add initialization of
%	 $\gamma_0^c\EQ\cs{pcol@columncolor@box}\cdot c$.} 
% 
%    \begin{macrocode}
  \global\@colroom\@colht \pcol@floatplacement
  \pcol@currcol\z@ \@whilenum\pcol@currcol<\pcol@ncol\do{%
    \pcol@Fb
    \@next\@currbox\@freelist{\global\setbox\@currbox\vbox{%
      \ifdim\topskip=\z@ \hrule\@height\z@\@width\z@ \fi}}\pcol@ovf
    \pcol@Fe{output@start(col)}%
    \pcol@setcurrcolnf
    \global\count\@currbox\z@
    \global\dimen\@currbox\@colroom
    \expandafter\gdef\csname pcol@shipped\number\pcol@currcol\endcsname{}%
    \pcol@ifccdefined
      {\@next\@currbox\@freelist{\global\setbox\@currbox\vbox{%
        \def\current@color{\pcol@ccuse{}}\let\aftergroup\@gobble
        \pcol@set@color}}\pcol@ovf}%
      {\def\@currbox{\voidb@x}}%
    \pcol@ccxdef{\@currbox}%
   \advance\pcol@currcol\@ne}%
%    \end{macrocode}
%
% \changes{v1.2-1}{2013/05/11}
%	{Add emptying \cs{pcol@colorstack} and the invocation of
%	 \cs{pcol@savecolorstack} at the beginnig of the first
%	 column-page to be built.}
% \changes{v1.2-2}{2013/05/11}
%	{Add insertion of \cs{footins} having footnotes to be merged if it
%	 is not void.}
% \changes{v1.22}{2013/06/30}
%	{Move emptying \cs{pcol@colorstack} to \cs{pcol@zparacol}.}
% \changes{v1.32-2}{2015/10/10}
% 	{Add \cs{pcol@Fb}/\cs{pcol@Fe} pair(s).}
% \changes{v1.35-1}{2018/12/31}
% 	{Add \cs{interlinepenalty} insertion for the first column to avoid
%	 vertical overfull due to the first item taller than the column
%	 room.} 
% 
% Finally, we let $c=\!\pcol@currcol!=0$ for the first column, and regain the
% parameters in $\cc_0$ by \!\pcol@getcurrcol!.  Then before putting
% $\cc_0(\vb^b)$ to the main vertical list by \!\unvbox! returning
% $\cc_0(\vb)$ to \!\@freelist! by \!\@cons! because it has become useless
% so far, we save the \colorctext{} just with $\Celt^c$ into $\csts$ by
% \!\pcol@savecolorstack! because $\CST$ is emptied by \!\pcol@zparacol!.
% Then we \!\insert! \!\footins! through itself to the main vertical list if
% it is not $\bot$ and thus has footnotes to be \mgfnote{}.  This
% \!\insert!ion is different from other footnote \!\insert!ion because
% \!\footins! is not \!\unvbox!ed but is put as a whole and is followed by
% \!\penalty!\!\interlinepenalty!, so that footnotes will not be broken by
% \TeX's page builder to prevent the reconnection of a broken footnote with
% innapropriate glue discarding, which we will discuss in
% \secref{sec:imp-fnote}.  We also add a penalty 10000 or
% \!\interlinepenalty! according to $\CSIndex{if@nobreak}=\true$ or not to
% allow the first column to start from a new page when its first item is
% taller than the room in the \spage{}.
% 
%    \begin{macrocode}
  \global\pcol@currcol\z@
  \pcol@getcurrcol
  \pcol@savecolorstack
  \pcol@Fb
  \@cons\@freelist\@currbox \unvbox\@currbox
  \pcol@Fe{output@start(col)}%
  \ifvoid\footins\else
    \pcol@Log\pcol@output@start{insert}\footins
    \insert\footins{\box\footins\penalty\interlinepenalty}%
  \fi
  \if@nobreak \nobreak \else \addpenalty\interlinepenalty \fi}

%    \end{macrocode}
% \end{macro}
% 
% \begin{macro}{\pcol@makenormalcol}
% \changes{v1.2-2}{2013/05/11}
%	{Replace the building operation of \cs{@outputbox} with footnotes
%	 with the invocation of \cs{pcol@combinefootins} and add the
%	 examination of \cs{ifpcol@mgfnote}.}
% \changes{v1.2-7}{2013/05/11}
%	{Turn \cs{ifpcol@lastpage} be $\string\true$ temporarily for
%	 \cs{pcol@combinefloats} to separate bottom floats in 
%	 pre-environment stuff and the multi-column stuff in
%	 \string\texttt{paracol} environment by \cs{textfloatsep}.}
% \changes{v1.3-6}{2013/09/17}
%	{Completely redesigined to use \cs{@makecol} if pre-environment
%	 stuff has bottom floats.}
% \changes{v1.32-2}{2015/10/10}
% 	{Add \cs{pcol@Fb}/\cs{pcol@Fe} pair(s).}
% 
% The macro \!\pcol@makenormalcol! is invoked solely from
% \!\pcol@output@start! to let \!\@output~box! have the \preenv{} as
% \spanning{} of the \spage.  The operations this macro performs are very
% similar to those of \!\@makecol!, which in fact is used in this macro
% itself, but has the following differences.
% 
% \begin{enumerate}\def\labelenumi{(\arabic{enumi})}
% \item
% If $\CSIndex{ifpcol@mgfnote}=\true$, we exclude footnotes in \!\footins!
% from \!\@outputbox!, because they are \mgfnote{} with \Scfnote{}s given in
% columns, by saving it into \!\@tempboxa! during the building process and
% restoring it from the box register.
% 
% \item
% If \preenv{} does not have bottom floats, we build \!\@outputbox! by
% ourselves without relying on \!\@makecol! because the skips put into the
% bottom (or near it) by the macro is harmful to making \preenv{} and
% parallel columns naturally connected.  Therfore, we move
% \!\@holdpg! to \!\@outputbox! adding \!\footins! to its tail if any by
% \!\pcol@combinefootins!, and then combine top floats if any by
% \!\pcol@combinefloats!\footnote{
% 
% Since we do not have bottom floats, the order of materials in the
% resulting \!\@outputbox! being top floats, main text and footnotes should
% be consistent with other pages with any \LaTeX{} including p\LaTeX.}.
% 
% In addition, we clear \!\@midlist! and retuns its contents to
% \!\@freelist! as \!\@makecol! does.
% 
% \item
% If \preenv{} has bottom floats, on the other hand, we use \!\@makecol! to
% build \preenv{} in \!\@outputbox! moving \!\@holdpg! into \!\box!|255|
% prior to the invocation\footnote{
% 
% Therefore the order of footnotes and bottom floats is consistent with
% other pages and columns, i.e., footnote-first in the native \LaTeX{} while
% float-first in p\LaTeX{}, for example.}.
% 
% Also before invocation in addition, we temporarily let
% $\CSIndex{ifpcol@lastpage}=\true$ to let
% $\!\@combinefloats!=\!\pcol@combinefloats!$ used in \!\@makecol!  put a
% vertical skip of \!\textfloatsep! below the bottom floats so that the
% floats are well seperated from the top of multi-column stuff in the
% \spage.  We also nullify \!\@textbottom! by making it \!\let!-equal to
% \!\relax! because it is unncessary to put an infinitely stretchable skip
% at the bottom\footnote{
% 
% Even if unharmful.},
% 
% and let $\!\vbadness!=10000$ to avoid an inevitable underfull message
% because \!\@makecol! lets \!\@outputbox! as tall as \!\textheight!.
% 
% \item
% In both cases but especially that with bottom floats, resulting
% \!\@outputbox! is decapsulated by \!\unvbox! to make its height {\em
% natural}.
% \end{enumerate}
% 
% Note that the special function for \sync{}ed \colpage{} in
% \!\pcol@combinefloats! used directly or indirectly in this macro, on the
% other hand, is not active in the invocation because \!\pcol@zparacol!
% initialized $\!\pcol@textfloatsep!=\infty$ to mean we have no \sync{}ation
% points.  Also
% note that bottom floats and non-\mgfnote{} footnotes are put in
% \!\@outputbox! and thus they will not appear at the bottom of the page
% but above the \colpage{}s in the page\footnote{
% 
% We could put them at the bottom by keeping them somewhere and insert them
% in \!\pcol@outputcolumns!, but it will cause another problem that the
% numbers of the figures and footnotes are smaller than those in \colpage{}s
% which are above them.}.
% 
%    \begin{macrocode}
\def\pcol@makenormalcol{%
  \ifpcol@mgfnote \setbox\@tempboxa\box\footins \fi
  \begingroup
  \ifx\@botlist\@empty
    \ifvoid\footins \setbox\@outputbox\box\@holdpg
    \else           \pcol@combinefootins\@holdpg\footins
    \fi
    \pcol@Fb
    \let\@elt\relax \xdef\@freelist{\@freelist\@midlist}%
    \pcol@Fe{makenormalcol}%
    \global\let\@midlist\@empty
    \pcol@combinefloats
  \else
    \pcol@lastpagetrue
    \setbox\@cclv\box\@holdpg \let\@textbottom\relax \vbadness\@M
    \@makecol
  \fi
  \global\setbox\@outputbox\vbox{\unvbox\@outputbox}%
  \endgroup
  \ifpcol@mgfnote \setbox\footins\box\@tempboxa \fi}

%    \end{macrocode}
% \end{macro}
% 
% 
% 
% \subsection{Column-Switching}
% \label{sec:imp-sout-switch}
% 
% \begin{macro}{\pcol@output@switch}
% \changes{v1.2-1}{2013/05/11}
%	{Add \cs{pcol@clearcst@unvbox} to add coloring \cs{special}s at the
%	 top and bottom of the column-page to be saved.}
% \changes{v1.3-1}{2013/09/17}
%	{Add the capture of a spanning text when it is closed.}
% \changes{v1.3-3}{2013/09/17}
%	{Add $\pi^s(p)\EQ\cs{pcol@sptextlist}$ to the argument of
%	 \cs{pcol@defcurrpage}.}
% \changes{v1.3-4}{2013/09/17}
%	{Add $\pi^m(p)\EQ\cs{pcol@mparbottom}$ to the argument of
%	 \cs{pcol@defcurrpage}.}
% \changes{v1.32-2}{2015/10/10}
% 	{Add \cs{pcol@Fb}/\cs{pcol@Fe} pair(s).}
% \changes{v1.33-2}{2016/11/19}
% 	{Let \cs{dimen@} have the height of \cs{pcol@prespan} if it is not
%	 $\bot$, or 0 if $\bot$ for the sake of clarity.}
% 
% The macro \!\pcol@output@switch! is invoked from
% \!\pcol@specialoutput! to process the special \!\output! request made in
% \!\pcol@switchcol!, \!\pcol@visitallcols! and \!\pcol@flushclear!, for a
% \cswitch{} from $c=\!\pcol@currcol!$ to $d=\!\pcol@nextcol!$ which can be
% $c$.  The macro is also invoked from \!\pcol@makeflushedpage!  to \sync{}e
% and to flush all \ccolpage{}s but staying in $c$.
% 
% First, we examine if the \cswitch{} is to close a \mctext, i.e.,
% $\CSIndex{ifpcol@sptext}={\true}$ and $c=0$, and if so we do the
% following; let $h_p$ be the height of \!\pcol@prespan! having \prespan{}
% if it is not $\bot$, or 0 if $\bot$ to mean we have had a page break in
% the \mctext; add $h_p$ to \!\@colroom! which we temporarily shrank when
% the \mctext{} starts; add an elemnent $\spt(H,h)$ to the tail of
% $\pp^s(\ptop)=\!\pcol@sptextlist!$ by \!\pcol@getcurrpage! and
% \!\pcol@defcurrpage!, where $H$ is $h_p$ plus the total height of top
% floats measured by \!\pcol@addflhd!, and $h$ is the height-plus-depth of
% \!\@holdpg! having (a part of) \mctext, represented in the form of
% integers and thus expanded with \!\number!; shift \!\@holdpg! left by
% \!\pcol@shiftspanning! if the column-0 is not the leftmost due to \cswap;
% and then put \prespan{} and (maybe shifted) \!\@holdpg! into \!\@holdpg!
% itself so as to let \!\@holdpg! have everything in the \colpage{} 0 as
% usual.
% 
% Note that it can be $\!\pcol@prespan!=\bot$ if \mctext{} had a page break
% (or multiple ones) in it as shown above.  This empty \prespan{}, however,
% does not always means that we have no top floats because the page break in
% the \mctext{} can produce a \colpage{} with top floats which are deferred
% from the previous page(s), or though unlikely the \mctext{} itself has
% float environments.  Therefore, the measurement of the total height of top
% floats are always necessary.  Also note that we perform these operations
% at the first \cswitch{} for \cscan{}ing from $c=0$ with
% $\CSIndex{ifpcol@sptext}=\true$, i.e., prior to the \sync{}ation itself
% which takes place afterword, as explained shortly.
% 
% Then regardless of the operations above, we acquire an \!\insert! from
% \!\@freelist! by \!\@next! for $\cc_c(\vb)$ to store the \ccolpage{} in
% (maybe modified) \!\@holdpg! by \!\pcol@clearcst@unvbox! to add uncoloring
% \!\special!s to rewind the \colorstack{} $\CST^c$ at the bottom and
% possibly coloring ones to establish that saved in $\csts$ at the top as
% the \colorctext{} for the \colpage{} when it has the first item.
% 
% Then if $\!\footins!\neq\bot$, we perform one of the followings.
% 
% \begin{itemize}
% \item
% If \Scfnote{} typesetting is in effect and $p=\ptop$, we save \!\footins!
% into $\pp^f(p)$ by the sequence of \!\pcol@getcurrpinfo! to get $\pp(p)$,
% \!\pcol@savefootins! to move it in $\pp^f(p)$, and \!\pcol@defcurrpage! to
% update $\pp(p)$ with $\pp^f(p)$.
% 
% \item
% If \Scfnote{} typesetting is in effect but $p<\ptop$, we simply discard
% the contents of \!\footins!  by making it $\bot$, because \!\footins!
% should have $\pp^f(p)$ which has been already fixed.
% 
% \item
% If \Mcfnote{} typesettinng is in effect, by \!\pcol@savefootins! we save
% \!\footins! into \!\pcol@currfoot!, which should be $\bot$ in other cases,
% so that it will be saved into $\cc_c(\ft)$ by \!\pcol@setcurrcol!
% afterward.
% \end{itemize}
% %
% Then if $c=0$, we invoke \!\pcol@setpageno! to reflect the jump of
% \!\c@page! made in the building process of the \colpage{} to $\pp(q)$ for
% all $q\in[p,\ptop]$.  After that, we save $c$'s \cctext{} into $\cc_c$ by
% \!\pcol@setcurrcol! and let $\cc_c(\vb^p)=p$ and
% $\cc_c(\vb^r)=\!\@colroom!$.
% 
%    \begin{macrocode}
%% Special Output Routines: Column-Switching

\def\pcol@output@switch{%
  \ifpcol@sptext\ifnum\pcol@currcol=\z@
    \ifvoid\pcol@prespan \dimen@\z@ \else \dimen@\ht\pcol@prespan \fi
    \global\advance\@colroom\dimen@
    \pcol@addflhd\@toplist\pcol@textfloatsep
    \pcol@getcurrpinfo\@tempcnta\@tempdima\@tempskipa
    \@tempdimb\ht\@holdpg \advance\@tempdimb\dp\@holdpg
    \@cons\pcol@sptextlist{{\number\dimen@}{\number\@tempdimb}}%
    \pcol@defcurrpage{\number\@tempcnta}\pcol@spanning\pcol@footins
                     {\pcol@sptextlist}{\pcol@mparbottom}%
    \pcol@shiftspanning\@holdpg
    \setbox\@holdpg\vbox{\unvbox\pcol@prespan \unvbox\@holdpg}%
  \fi\fi
  \pcol@Fb
  \@next\@currbox\@freelist{\global\setbox\@currbox\vbox{
    \pcol@clearcst@unvbox\@holdpg}}\pcol@ovf
  \pcol@Fe{output@switch}%
  \def\pcol@currfoot{\voidb@x}%
  \ifvoid\footins\else
    \ifpcol@scfnote
      \ifnum\pcol@page=\pcol@toppage
        \pcol@getcurrpinfo\@tempcnta\@tempdima\@tempskipa
        \pcol@Log\pcol@output@switch{save}\footins
        \pcol@Fb
        \pcol@savefootins\pcol@footins
        \pcol@Fe{output@switch(pagefn)}%
        \pcol@defcurrpage{\number\@tempcnta}\pcol@spanning\pcol@footins
                         {\pcol@sptextlist}{\pcol@mparbottom}%
      \else
        \pcol@Log\pcol@output@switch{discard}\footins
        \setbox\@tempboxa\box\footins
      \fi
    \else
      \pcol@Log\pcol@output@switch{save}\footins
      \pcol@Fb
      \pcol@savefootins\pcol@currfoot
      \pcol@Fe{output@switch(colfn)}%
    \fi
  \fi
  \ifnum\pcol@currcol=\z@ \pcol@setpageno \fi
  \pcol@setcurrcol
  \global\count\@currbox\pcol@page
  \global\dimen\@currbox\@colroom
%    \end{macrocode}
% 
% Next, we examine if $\CSIndex{ifpcol@sptext}=\true$ and $c=0$ again, and
% if so we {\em broadcast} \CSIndex{if@nobreak} and
% \CSIndex{if@afterindent}, or in other words $\cc_c(\sw)$, and tokens in
% $\!\everypar!=\cc_c(\ep)$, to pretend all columns follow the \mctext.
% That is, for each column $e$, we restore its \cctext{} from $\cc_e$ by
% \!\pcol@getcurrcol!, let \CSIndex{if@nobreak} and \CSIndex{if@afterindent}
% have the values for $c$ ($\in\{0,\C{-}1\}$) and $\!\everypar!=\cc_c(\ep)$,
% and then save the context to $\cc_e$ by \!\pcol@setcurrcol! so that
% $\cc_e(\sw)=\cc_c(\sw)$ and $\cc_e(\ep)=\cc_c(\ep)$.  After that, we
% \!\global!ly turn $\CSIndex{ifpcol@sptext}=\false$ to give it the default
% state.
% 
% Note that this broadcast is essential when the \mctext{} has sectioning
% commands to have consistent settings of the page break inhibition, the
% skip above the another sectioning command following them, and the
% indentation of the first paragraph, for all columns.  On the other hand,
% broadcasting of \!\everypar! is natural even when it does not have
% sectioning commands because all columns may be considered following the
% \mctext.  Also note that, as mentioned in the explanation of the first
% examination at the beginning of this macro, we perform these operations at
% the first \cswitch{} for \cscan{}ning from $c=0$ with
% $\CSIndex{ifpcol@sptext}=\true$ prior to the \sync{ation} following the
% \mctext.  This means, if $\CSIndex{if@nobreak}={\true}$,
% $\!\penalty!=10000$ is inserted at the top and bottom end of the space for
% \mctext{} in the columns such that $c\neq0$, the former by this \cscan{}
% and the latter by the \cswitch{} to $c$ made after the synchronization.
% Therefore, if our \sync{ation} mechanism and \TeX's page builder once
% agreed both end can be in a page, both end will not chosen as page
% break points\footnote{
% 
% As for $c=0$, its top end of \mctext{} is a feasible break point to make
% the penalty insertion asymmetric.  Therefore, we need to reinvestigate if
% the condition of the broadcast is really appropriate, and, if
% inappropriate, have to go back to the old implementation in which
% \cs{ifpcol@sync} is included in the condition.  Otherwise, if proved
% appropriate, we will have to consider to make the penalty insertion
% symmetric by adding \cs{nobreak} at the top of \mctext{} in $c=0$.}.
% 
% \changes{v1.0}{2011/10/10}
%	{Restrict the broadcast of \cs{if@nobreak} and \cs{everypar} only
%	 when a column-switching is accompanied with spanning text.}
% \changes{v1.2-7}{2013/05/11}
%	{Modify broadcasting of $\kappa_c(\sigma)$ so that \cs{@afterindent}
%	 is broadcasted with \cs{@nobreak}.}
% \changes{v1.3-1}{2013/09/17}
%	{Rename \cs{ifpcol@mctext} as \cs{ifpcol@sptext}.}
% \changes{v1.3-6}{2013/09/17}
%	{Modify the condition of broadcasting $\kappa_0(\sigma)$ and
%	 $\kappa_0(\varepsilon)$ accompanied with \cs{ifpcol@sptext} from
%	 \cs{ifpcol@sync} to $c\EQ0$ so that the broadcast is made in the
%	 first column-switching in column-scanning.}
% 
%    \begin{macrocode}
  \let\reserved@a\@nobreakfalse \let\reserved@b\@afterindentfalse
  \ifpcol@sptext\ifnum\pcol@currcol=\z@
    \if@nobreak \let\reserved@a\@nobreaktrue \fi
    \if@afterindent \let\reserved@b\@afterindenttrue \fi
    \@temptokena\everypar
    \pcol@currcol\z@ \@whilenum\pcol@currcol<\pcol@ncol\do{%
      \pcol@getcurrcol \reserved@a \reserved@b \everypar\@temptokena
      \pcol@setcurrcol
     \advance\pcol@currcol\@ne}%
    \global\pcol@sptextfalse
  \fi\fi
%    \end{macrocode}
% 
% \changes{v1.2-2}{2013/05/11}
%	{Add the case
%	 $\cs{ifpcol@clear}\EQ\cs{ifpcol@sync}\EQ\string\mathit{true}$ for
%	 the invocation of \cs{pcol@restartcolumn} for pre-flushing column
%	 height check.}
% \changes{v1.3-6}{2013/09/17}
%	{Modify the code structure to let
%	 $\cs{if@tempswa}\EQ\string\mathit{true}$ according to the
%	 modification of the broadcast of $\kappa_0(\sigma)$ and
%	 $\kappa_0(\varepsilon)$.}
% 
% Finally we invoke \!\pcol@sync! for the \sync{}ation if
% \CSIndex{ifpcol@sync} or \CSIndex{ifpcol@clear} is $\true$, and then
% \!\pcol@restartcolumn! to restart the \ccolpage{} $d$ if
% $\CSIndex{ifpcol@clear}={}\false$ to mean ordinary (but possibly
% \sync{}ed) \cswitch{} or $\CSIndex{ifpcol@clear}={}\true$ but
% $\CSIndex{ifpcol@sync}={}\true$ too to mean pre-flushing column height
% check, before finishing \!\output!  routine letting
% $\CSIndex{ifpcol@sync}=\false$ for next \cswitch.
% 
%    \begin{macrocode}
  \@tempswafalse \ifpcol@sync \@tempswatrue \fi \ifpcol@clear \@tempswatrue \fi
  \if@tempswa \pcol@sync \fi
  \@tempswatrue
  \ifpcol@clear \ifpcol@sync\else \@tempswafalse \fi\fi
  \if@tempswa \pcol@restartcolumn \fi
  \global\pcol@syncfalse}

%    \end{macrocode}
% \end{macro}
% 
% \begin{macro}{\pcol@shiftspanning}
% \changes{v1.3-1}{2013/09/17}
%	{Introduced to shift a spanning text to left if the column-0 is not
%	 leftmost due to column-swapping.}
% 
% The macro $\!\pcol@shiftspanning!\arg{b}$ is used in \!\pcol@makecol! and
% \!\pcol@output@switch! to let box register $b$ have itself but shifted
% left by $\WT-w_0=\!\textwidth!-\!\columnwidth!$ so that the left edge of
% its contents \mctext{} is aligned to the left edge of the leftmost column
% being different from column-0 due to \cswap, i.e., if
% $\CSIndex{ifpcol@swapcolumn}=\true$ and $\!\c@page!\bmod2=0$.  Note that
% \!\c@page! is {\em not} obtained from $\pp^p(p)$ by the invokers but have
% the value when the \!\output! request is made to let invokers work, and
% thus have the correct value even when a jump occurs prior to the request.
% 
%    \begin{macrocode}
\def\pcol@shiftspanning#1{%
  \ifpcol@swapcolumn\ifodd\c@page\else
    \setbox#1\vbox{\@tempdima\textwidth \advance\@tempdima-\columnwidth
      \moveleft\@tempdima\box#1}
  \fi\fi}

%    \end{macrocode}
% \end{macro}
% 
% \begin{macro}{\pcol@restartcolumn}
% \changes{v1.0}{2011/10/10}
%	{Add \cs{pcol@getcurrfoot} to restore parameters of $\kappa_c(\tau)$
%	 into \cs{footins}.}
% \changes{v1.32-2}{2015/10/10}
% 	{Add \cs{pcol@Fb}/\cs{pcol@Fe} pair(s).}
% 
% The macro \!\pcol@restartcolumn! is invoked from \!\pcol@output@switch! or
% \!\pcol@freshpage! to restart the \ccolpage{} $d=\!\pcol@nextcol!$ which
% becomes $c=\!\pcol@currcol!$ by the very first assignment in this macro.
% Then we restore the \cctext{} in $\cc_c$ by \!\pcol@getcurrcol! and let
% $p=\cc_c(\vb^p)$ and $\!\@colroom!=\cc_c(\vb^r)$ before returning
% $\cc_c(\vb)$ to \!\@freelist! by \!\@cons! because it has become useless
% so far.  We also restore the \pctext{} of $p$ by \!\pcol@getcurrpage!.
% 
%    \begin{macrocode}
\def\pcol@restartcolumn{%
  \global\pcol@currcol\pcol@nextcol
  \pcol@getcurrcol
  \global\pcol@page\count\@currbox
  \global\@colroom\dimen\@currbox
  \pcol@Fb
  \@cons\@freelist\@currbox
  \pcol@Fe{restartcolumn(col)}%
  \pcol@getcurrpage
%    \end{macrocode}
% 
% \changes{v1.2-2}{2013/05/11}
%	{Redesign the footnote insertion mechanism to cope with
%	 page-wise footnotes.}
% \changes{v1.2-1}{2013/05/11}
%	{Add \cs{pcol@restorecst@restart} to return main vertical list
% 	 with coloring operations.}
% \changes{v1.3-1}{2013/09/17}
%	{Rename \cs{pcol@restorecst@restart} as \cs{pcol@putbackmvl}.}
% \changes{v1.3-6}{2013/09/17}
%	{Change the code structure to move the insertion of page break
%	 penalty for ordinary column-wise footnotes from below the main
%	 text to below the footnotes.}
% \changes{v1.3-3}{2013/09/17}
%	{Add \cs{@colht} and \cs{@tempdimb} as the first and third argument
%	 of \cs{pcol@shrinkcolbyfn}.}
% 
% Then we perform footnote \!\insert!ion as follows.
% 
% \begin{enumerate}\def\labelenumi{(\arabic{enumi})}
% \item\label{item:scftop}
% If footnote typesetting is \scfnote{} and $p=\ptop$, we do the followings.
% 
% \begin{enumerate}\def\labelenumii{(\alph{enumii})}
% \item\label{item:scftop-main}
% Put the contents of $\cc_c(\vb^b)$ by \!\pcol@putbackmvl! to make
% the \colorctext{} in |.dvi| consistent with the current |.tex|'s one, and
% to save \prespan{} into \!\pcol@prespan! if we are opening a \mctext{}.
% 
% \item\label{item:scftop-pen}
% Put $\!\penalty!=10000$ by \!\nobreak! if $\CSIndex{if@nobreak}=\true$ or
% \!\interlinepenalty! by \!\addpenalty!\footnote{
% 
% As done in \!\@specialoutput! but \!\penalty!\!\interlinepenalty! should
% be sufficient.}
% 
% otherwise, as the page break penalty at the returning point.  Note that
% adding the \!\penalty! will be nullified by \TeX{} if $\cc_c(\vb^b)$ has
% nothing and thus, if the \colpage{} is still empty when we leave from it,
% its emptiness without any items is assured.  Also note that the penalty
% insertion here {\em looks} essential to keep \TeX's page builder from
% confusing with \scfnote{} footnotes which it has not seen in a
% \colpage\footnote{
% 
% At least a test with tall \scfnote{} footnotes gave us a confusing
% result.}.
% 
% \item\label{item:scftop-fnote}
% If $\pp^f(p)\neq\bot$, let \!\pcol@currfoot! and then \!\footins! have
% the footnotes in it by an \!\edef! and \!\pcol@getcurrfoot!, return the
% \!\insert! for them to \!\@freelist!, invoke \!\pcol@shrinkcolbyfn! to
% shrink \!\@colht! temporarily by their total height and to remember the
% exsistence of them with $\!\@tempdimb!=-\!\skip!\!\footins!$, and then
% \!\insert!  the footnotes so that it contributes to the building process
% of the \colpage{} to be restarted.  Otherwise, i.e. if $\pp^f(p)=\bot$,
% \!\@colht! is unchanged and $\!\@tempdimb!=0$.
% 
% \item\label{item:scftop-defer}
% Invoke \!\pcol@deferredfootins! to \!\insert! deferred footnotes in $\df$
% until their total height reaches (possibly shrunk) \!\@colht!.  This
% height capping is to keep \TeX's page builder from holding too large
% number of footnotes unprocessed causing confused ordering on presenting
% them to \!\output! routine.
% \end{enumerate}
% 
% \item
% If footnote typesetting is \scfnote{} but $p<\ptop$, we do the followings.
% 
% \begin{enumerate}\def\labelenumii{(\alph{enumii})}
% \item
% If $\pp^f(p)\neq\bot$, get it into \!\footins! as done in
% (\ref{item:scftop-fnote}) but giving \!\copy! to \!\pcol@getcurrfoot!
% because $\pp^f(p)$ has been fixed and thus will be kept until it is
% shipped out, and then \!\insert! it.
% 
% \item
% Put $\cc_c(\vb^b)$ and the penalty as done in (\ref{item:scftop-main}) and
% (\ref{item:scftop-pen}).
% \end{enumerate}
% 
% Thes order of footnotes, main vertical list and then penalty is essential
% to ensure that the \colpage{} in $p<\ptop$ has room for footnotes whose
% residence in $p$ has already been fixed.
% 
% \item
% If footnote typesetting is \mcfnote{}, we do the followings.
% 
% \begin{enumerate}\def\labelenumii{(\alph{enumii})}
% \item
% Put $\cc_c(\vb^b)$ as done in (\ref{item:scftop-main}).
% 
% \item
% If $\cc_c(ft)\neq\bot$, get it by \!\pcol@getcurrfoot! returning the
% \!\insert! to \!\@freelist!, and then \!\insert! it.
% 
% \item
% Put a penalty as done in (\ref{item:scftop-pen}).
% \end{enumerate}
% 
% The order of main vertical list, footnotes and then penalty is appropriate
% for \Mcfnote{}s because they definitely have space in the \colpage{} and
% \TeX{} will break the page below the insertion, possibly just below thanks
% to the penalty, to keep the footnotes and references to them in a page.
% \end{enumerate}
% 
% \changes{v1.32-2}{2015/10/10}
% 	{Add \cs{pcol@Fb}/\cs{pcol@Fe} pair(s).}
% 
%    \begin{macrocode}
  \ifpcol@scfnote
    \edef\pcol@currfoot{\pcol@footins}%
    \ifnum\pcol@page=\pcol@toppage
      \@tempdima\@colht \@tempdimb\z@
      \pcol@putbackmvl
      \if@nobreak \nobreak \else \addpenalty\interlinepenalty \fi
      \ifvoid\pcol@footins\else
        \pcol@Fb
        \pcol@getcurrfoot\box \@cons\@freelist\pcol@currfoot
        \pcol@Fe{restartcolumn(pagefn)}%
        \pcol@Log\pcol@restartcolumn{insert}\footins
        \pcol@shrinkcolbyfn\@colht\footins\@tempdimb
        \insert\footins{\unvbox\footins}%
      \fi
      \pcol@deferredfootins\pcol@restartcolumn
      \@colht\@tempdima
    \else
      \ifvoid\pcol@footins\else
        \pcol@getcurrfoot\copy
        \pcol@Log\pcol@restartcolumn{insdmy}\footins
        \insert\footins{\unvbox\footins}%
      \fi
      \pcol@putbackmvl
      \if@nobreak \nobreak \else \addpenalty\interlinepenalty \fi
    \fi
  \else
    \pcol@putbackmvl
    \ifvoid\pcol@currfoot\else
      \pcol@Fb
      \pcol@getcurrfoot\box \@cons\@freelist\pcol@currfoot
      \pcol@Fe{restartcolumn(colfn)}%
      \pcol@Log\pcol@restartcolumn{insert}\footins
      \insert\footins{\unvbox\footins}%
    \fi
    \if@nobreak \nobreak \else \addpenalty\interlinepenalty \fi
  \fi}

%    \end{macrocode}
% \end{macro}
% 
% \KeepSpace{1}
% \begin{macro}{\pcol@getcurrcol}
% \changes{v1.1}{2012/05/11}
% 	{Add assignment $w_c$ to \cs{columnwidth}, \cs{hsize} and
%	 \cs{linewidth}.} 
% \changes{v1.2-5}{2013/05/11}
% 	{Move assignment $w_c$ to \cs{hsize} and \cs{linewidth} to
%	 \cs{pcol@invokeoutput}.}
% \begin{macro}{\pcol@igetcurrcol}
% \begin{macro}{\pcol@iigetcurrcol}
% \changes{v1.0}{2011/10/10}
%	{Add restoration of \cs{pcol@textfloatsep}.}
% \changes{v1.3-4}{2013/09/17}
%	{Remove the argument for $\kappa_c(\mu)\EQ\cs{@mparbottom}$ because it
%	 is no longer in the column context.}
% 
% The macro \!\pcol@getcurrcol! is invoked from the following macros to
% restore the typesetting parameters of the column $c=\!\pcol@currcol!$ from
% $\cc_c$, and to let \!\columnwidth! have
% $\w_c=|\pcol@columnwidth|{\cdot}c$\footnote{
% 
% \cs{hsize} and \cs{linewidth} are let have $\w_c$ and $\w_c-\lrm$
% respectively in \cs{pcol@invokeoutput}.}.
% 
% \SpecialArrayIndex{c}{\pcol@columnwidth}
% 
% \begin{itemize}\item[]\begin{tabular}{lll}
% \!\pcol@output@start!&\!\pcol@output@switch!&\!\pcol@restartcolumn!\\
% \!\pcol@flushcolumn!&\!\pcol@measurecolumn!&\!\pcol@synccolumn!\\
% \!\pcol@makeflushedpage!&\!\pcol@imakeflushedpage!&\!\pcol@iflushfloats!\\
% \!\pcol@freshpage!&\!\pcol@output@end!
% \end{tabular}
% \end{itemize}
% 
% Since we represent $\cc_c$ as;
% 
% \begin{eqnarray*}
% &&\Arg{\cc_c(\vb)}\Arg{\cc_c(\ft)}\Arg{\cc_c(\pd)}\Arg{\cc_c(\tl)}
% \Arg{\cc_c(\ml)}\Arg{\cc_c(\bl)}\Arg{\cc_c(\dl)}\Arg{\cc_c(\tf)}|%|\\
% &&\Arg{\Arg{\cc_c(\fh)}\Arg{\cc_c(\tn)}\Arg{\cc_c(\tr)}
%        \Arg{\cc_c(\bn)}\Arg{\cc_c(\br)}\Arg{\cc_c(\cn)}\Arg{\cc_c(\sw)}
%        \Arg{\cc_c(\ep)}}
% \end{eqnarray*}
% 
% in the body of $|\pcol@col|{\cdot}c$,
% 
% \SpecialArrayIndex{c}{\pcol@col}
% 
% we restore first eight by \!\pcol@igetcurrcol! giving everything above as
% its arguments by the expansion of
% $$
% \!\csname!| pcol@col|\!\number!\!\pcol@currcol!\!\endcsname!
% $$
% and then of the resulting control sequence.  Then this macro gives its
% nineth argument to \!\pcol@iigetcurrcol! which restores the last eight.  We
% also do
% $$
% \!\global!\!\columnwidth!|\pcol@columnwidth|{\cdot}c
% $$
% 
% \SpecialArrayIndex{c}{\pcol@columnwidth}
% 
% by a pair of \!\expandafter! for the first two control sequences.
% 
% Note that the restore operations are \!\global!, except for $\cc_c(\vb)$
% and $\cc_d(\ft)$ because they are referred to only in \!\output!,
% including \CSIndex{if@nobreak} for which \!\@nobreaktrue!  and
% \!\@nobreakfalse! are defined \!\global! by \LaTeX.  Also note that
% \!\dimen!-type parameters are saved in the form of integers and thus
% restoring them needs to specify the unit |sp|.
% 
%    \begin{macrocode}
\def\pcol@getcurrcol{%
  \expandafter\expandafter\expandafter\pcol@igetcurrcol
    \csname pcol@col\number\pcol@currcol\endcsname
  \expandafter\global\expandafter\columnwidth
    \csname pcol@columnwidth\number\pcol@currcol\endcsname}
\def\pcol@igetcurrcol#1#2#3#4#5#6#7#8#9{%
  \def\@currbox{#1}\def\pcol@currfoot{#2}\global\pcol@prevdepth#3sp\relax
  \gdef\@toplist{#4}\gdef\@midlist{#5}\gdef\@botlist{#6}\gdef\@deferlist{#7}%
  \global\pcol@textfloatsep#8sp\pcol@iigetcurrcol#9}
\def\pcol@iigetcurrcol#1#2#3#4#5#6#7#8{%
  \global\@textfloatsheight#1sp\relax
  \global\@topnum#2\relax \global\@toproom#3sp\relax
  \global\@botnum#4\relax \global\@botroom#5sp\relax
  \global\@colnum#6\relax
  \global\@afterindentfalse \@nobreaktrue
  \ifcase#7
    \@nobreakfalse \or
    \global\@afterindenttrue \else
    \relax
  \fi
  \global\everypar{#8}}
%    \end{macrocode}
% \end{macro}\end{macro}\end{macro}
% 
% \begin{macro}{\pcol@getcurrfoot}
% \changes{v1.2-2}{2013/05/11}
%	{Add an argument being \cs{box} or \cs{copy}}
% 
% The macro $\!\pcol@getcurrfoot!\arg{com}$ is invoked from
% \!\pcol@startcolumn! (\!\copy!, $\pp^f(p)$), 
% \!\pcol@restartcolumn! (\!\copy!\slash\!\box!, $\pp^f(p)$\slash$\cc_c(\ft)$),
% \!\pcol@flushcolumn! (\!\box!, $\cc_c(\ft)$) and  \!\pcol@imakeflushedpage!
% (\!\box!, $\cc_c(\ft)$) to put everything in $\!\pcol@currfoot!$, having
% the second element in the parens following macro names, into \!\footins!
% using $\arg{com}$ shown as the first element in parens for the \!\box!
% component.  That is, if the source $\pp^f(p)$ or $\cc_c(\ft^b)$ is void,
% we let $\!\box!{\cdot}\!\footins!$ be so.  Otherwise, we move \!\box!,
% \!\count!, \!\dimen! and \!\skip! of the source into those of
% \!\footins!\footnote{
% 
% Moving \!\count!, \!\dimen! and \!\skip! is redundant almost always
% because it is very unlikely that these footnote parameters are modified
% dynamically.  Moreover, dynamic modification of them is hardly consistent
% with repetitive self-\!\insert!ion of \!\footins! in
% \!\pcol@restartcolumn! and \!\@reinserts! of \LaTeX.  However, we dare to
% move them in order to, for example, allow each column has its own footnote
% parameters.}.
% 
% 
%    \begin{macrocode}
\def\pcol@getcurrfoot#1{%
  \ifvoid\pcol@currfoot \global\setbox\footins\box\voidb@x
  \else
    \global\setbox\footins#1\pcol@currfoot
    \global\count\footins\count\pcol@currfoot
    \global\dimen\footins\dimen\pcol@currfoot
    \global\skip\footins\skip\pcol@currfoot
  \fi}
%    \end{macrocode}
% \end{macro}
% 
% \begin{macro}{\pcol@setcurrcol}
% \changes{v1.0}{2011/10/10}
%	{Add save of \cs{pcol@textfloatsep}.}
% \changes{v1.3-4}{2013/09/17}
%	{Remove $\kappa_c(\mu)\EQ\cs{@mparbottom}$ from the body of
%	 $\cs{pcol@col}{\cdot}c$ because it is no longer in the column
%	 context.}
% 
% \begin{macro}{\pcol@setcurrcolnf}
% The macro \!\pcol@setcurrcol! is invoked from \!\pcol@output@switch!,
% \!\pcol@measurecolumn! and \!\pcol@synccolumn! to save \cctext{} of
% $c=\!\pcol@currcol!$ in $\cc_c$.  It is also used in \!\pcol@setcurrcolnf!
% invoked from \!\pcol@output@start!, \!\pcol@flushcolumn!,
% \!\pcol@imakeflushedpage!, \!\pcol@iflushfloats! and \!\pcol@freshpage! for
% the saving when the \colpage{} is known to have no footnotes.
% 
% The macro \!\pcol@setcurrcol! at first calculates the combined code for
% \CSIndex{if@nobreak} and \CSIndex{if@afterindent}, and then saves
% parameters into $\cc_c$ by \!\xdef! to have the sequence shown in the
% description of \!\pcol@getcurrcol!.  Note that \!\dimen!-type parameters
% are saved by expansions with \!\number! and thus as decimal integers.
% 
% The macro \!\pcol@setcurrcolnf! \!\def!ines $\cc_c(\ft)=\!\pcol@currfoot!$
% as \!\voidb@x!, and then invoke \!\pcol@setcurrcol! for saving.
% 
%    \begin{macrocode}
\def\pcol@setcurrcol{{\let\@elt\relax
  \@tempcnta\if@nobreak\if@afterindent\@ne\else\tw@\fi\else\z@\fi
  \expandafter\xdef\csname pcol@col\number\pcol@currcol\endcsname{%
    {\@currbox}{\pcol@currfoot}{\number\pcol@prevdepth}%
    {\@toplist}{\@midlist}{\@botlist}{\@deferlist}{\number\pcol@textfloatsep}%
    {{\number\@textfloatsheight}%
     {\number\@topnum}{\number\@toproom}{\number\@botnum}{\number\@botroom}%
     {\number\@colnum}{\number\@tempcnta}{\the\everypar}}}}}
\def\pcol@setcurrcolnf{\def\pcol@currfoot{\voidb@x}\pcol@setcurrcol}

%    \end{macrocode}
% \end{macro}\end{macro}
% 
% \begin{macro}{\pcol@putbackmvl}
% \changes{v1.2-1}{2013/05/11}
% 	{Introduced to restart a column with coloring.}
% \changes{v1.3-1}{2013/09/17}
%	{Renamed from \cs{pcol@restorecst@restart} and the operations to
%	 save pre-spanning-text stuff is added.}
% \changes{v1.33-2}{2016/11/19}
% 	{Add {\tt\%} to the end of the line to open \cs{vbox} for
%	 \cs{pcol@prespan} to obey the coding convention.}
% \changes{v1.34}{2018/05/07}
%	{Change nullification of
%	 $\mathit{\Gamma}_s\EQ\cs{pcol@colorstack@saved}$ from \cs{gdef} to
%	 \cs{box}-assignment of $\bot$ because it is now a \cs{vbox}.}
% 
% The macro \!\pcol@putbackmvl!, solely used in
% \!\pcol@restartcolumn!, has two functions;  \colorstack{} restoration and
% \prespan{} preservation.  It examines the emptiness of the \colpage{} of the
% column $c$ to be restarted in $\cc_c(\vb)=\!\@currbox!$.  If so, the
% \colorstack{} $\CST^c$ is saved into $\csts=\!\pcol@colorstack@saved!$ by
% \!\pcol@savecolorstack! as the opening \colorctext{} of the \colpage{},
% and \!\pcol@prespan! for \prespan{} is made $\bot$.
% 
% Otherwise, $\csts$ is let $\bot$ because the opening \colorctext{} has
% already been put when we left from the \colpage{}.  Then if
% $\CSIndex{ifpcol@sptextstart}=\true$ to mean a \mctext{} is to start, we
% save $\cc_c(\vb)$ into \!\pcol@prespan! adding the colorling \!\special!s
% to restore \colorctext{} from $\CST^c$ by
% \!\pcol@restorecolorstack!\footnote{
% 
% If \cs{pcol@prespan} is connected to (the first part of) the \mctext,
% the reestablishment of the \colorstack{} here correctly places coloring
% \cs{special}s in \texttt{.dvi}.  On the other hand, if the \mctext{} is
% slown away to the next page as a whole, the reestablishment here is
% essential for the correct paring of the pushes and pops, the latter of
% which are at the bottom of the \colpage{} whose tail is \cs{pcol@prespan}.}.
% 
% We also shrink $\cc_c(\vb^r)=\!\@colroom!$ by the height of the \prespan{}
% so that \mctext{} will be captured by \!\pcol@makecol! if it is broken
% into two (or more) pages, and put a invisible \!\hrule! to the main
% vertical list letting $\!\topskip!=0$ to supress \!\topskip! insertion
% prior to the \mctext{} but instead to make the text led by
% \!\baselineskip! (or \!\lineskip!) according to the \!\prevdepth!  being
% the depth of the tallest column and the height of the first \!\hbox! in
% the \mctext.  Otherwise, i.e., if $\CSIndex{ifpcol@sptextstart}=\false$,
% we simply put back $\cc_c(\vb)$ into the main vertical list and then the
% coloring \!\special!s by \!\pcol@restorecolorstack!.
% 
% Note that $\CSIndex{ifpcol@sptextstart}$ is temporarily made $\false$ by
% this macro if the \!\pcol@output@switch! invoking \!\pcol@restartcolumn!
% did not make \sync{}ed \cswitch{}, i.e., if $\CSIndex{ifpcol@flush}=\true$
% to mean the page is flushed before the \sync{}ation, or
% $\CSIndex{ifpcol@sync}=\false$ for \cscan{}ning prior to the \sync{}ation.
% 
%    \begin{macrocode}
\def\pcol@putbackmvl{%
  \ifpcol@flush \pcol@sptextstartfalse \fi
  \ifpcol@sync\else \pcol@sptextstartfalse \fi
  \pcol@ifempty\@currbox
   {\pcol@savecolorstack
    \ifpcol@sptextstart \global\setbox\pcol@prespan\box\voidb@x \fi}%
   {\global\setbox\pcol@colorstack@saved\box\voidb@x
    \ifpcol@sptextstart
      \global\setbox\pcol@prespan\vbox{%
        \unvbox\@currbox \pcol@restorecolorstack}%
      \global\advance\@colroom-\ht\pcol@prespan
      \global\topskip\z@ \hrule\@height\z@\@width\z@
    \else
      \unvbox\@currbox \pcol@restorecolorstack
    \fi}}

%    \end{macrocode}
% \end{macro}
% 
% 
% 
% \KeepSpace{3}
% \subsection{Color Management}
% \label{sec:imp-sout-color}
% 
% \changes{v1.2-1}{2013/05/11}
%	{Add the subsection ``Color Management'' to describe newly
%	 introduced macros for coloring.}
% \changes{v1.3-6}{2013/09/17}
%	{Move commands outside \cs{output} routine to the newly introduced
%	 section ``Commands for Text Coloring'' to distinguish
%	 macros inside and outside \cs{output} routine.}
% \changes{v1.22}{2013/06/30}
% 	{\cs{pcol@output@cpush} was introduced for color stack pushing but
%	 removed in v1.34.}
% \changes{v1.34}{2018/05/07}
%	{\cs{pcol@output@cpush} is removed according to the change of text
%	 coloring from \cs{output} to \cs{insert}.}
% \changes{v1.22}{2013/06/30}
% 	{\cs{pcol@output@icpush} was introduced to implement
%	 \cs{pcol@output@cpush} but removed in v1.34.} 
% \changes{v1.34}{2018/05/07}
%	{\cs{pcol@output@icpush} is removed according to the change of text
%	 coloring from \cs{output} to \cs{insert}.}
% \changes{v1.24}{2013/07/27}
%	{\cs{pcol@output@mcpush} was introduced for coloring specified in
%	 math mode but removed in v1.34.}
% \changes{v1.34}{2018/05/07}
%	{\cs{pcol@output@mcpush} is removed according to the change of text
%	 coloring from \cs{output} to \cs{insert}.}
% \changes{v1.24}{2013/07/27}
%	{\cs{pcol@output@imcpush} was introduced for coloring specified in
%	 math mode but removed in v1.34.}
% \changes{v1.34}{2018/05/07}
%	{\cs{pcol@output@imcpush} is removed according to the change of text
%	 coloring from \cs{output} to \cs{insert}.}
% \changes{v1.22}{2013/06/30}
% 	{\cs{pcol@output@cpop} was introduced for color stack popping but
%	 removed in v1.34.}
% \changes{v1.34}{2018/05/07}
%	{\cs{pcol@output@cpop} is removed according to the change of text
%	 coloring from \cs{output} to \cs{insert}.}
% \changes{v1.2-1}{2013/05/11}
% 	{\cs{pcol@reset@color@elt} was introduced to implement
%	 \cs{pcol@reset@color@pop} but removed in v1.34.} 
% \changes{v1.22}{2013/06/30}
% 	{\cs{pcol@reset@color@elt} was moved from the position where
%	 \cs{pcol@reset@color@pop} was defined 
%	 because it became to be used only by \cs{pcol@output@cpop}, but
%	 removed in v1.34.}
% \changes{v1.34}{2018/05/07}
%	{\cs{pcol@reset@color@elt} is removed according to the change of text
%	 coloring from \cs{output} to \cs{insert}.}
% \changes{v1.24}{2013/07/27}
%	{\cs{pcol@output@mcpop} was introduced for coloring specified in
%	 math mode but removed in v1.34.} 
% \changes{v1.34}{2018/05/07}
%	{\cs{pcol@output@mcpop} is removed according to the change of text
%	 coloring from \cs{output} to \cs{insert}.}
% \changes{v1.24}{2013/07/27}
%	{\cs{pcol@output@mcpop@elt} was introduced for coloring specified in
%	 math mode but removed in v1.34.} 
% \changes{v1.34}{2018/05/07}
%	{\cs{pcol@output@mcpop@elt} is removed according to the change of text
%	 coloring from \cs{output} to \cs{insert}.}
% \changes{v1.22}{2013/06/30}
% 	{\cs{pcol@output@cset} was introduced to set $\gamma_0^c$ but
%	 removed in v1.34.}
% \changes{v1.34}{2018/05/07}
%	{\cs{pcol@output@cset} is removed according to the change of text
%	 coloring from \cs{output} to \cs{insert}.}
% \changes{v1.22}{2013/06/30}
% 	{\cs{pcol@output@icset} was introduced to implement
%	 \cs{pcol@output@cset} but removed in v1.34.} 
% \changes{v1.34}{2018/05/07}
%	{\cs{pcol@output@icset} is removed according to the change of text
%	 coloring from \cs{output} to \cs{insert}.}
% \changes{v1.22}{2013/06/30}
% 	{\cs{pcol@return@from@color} was introduced to implement
%	 \cs{pcol@output@cpush}, \cs{pcol@output@cpop} and
%	 \cs{pcol@output@cset} but removed in v1.34.} 
% \changes{v1.34}{2018/05/07}
%	{\cs{pcol@return@from@color} is removed according to the change of
%	 text coloring from \cs{output} to \cs{insert}.}
% 
% \begin{macro}{\pcol@magicpenalty}
% \begin{macro}{\pcol@ifempty}
% \changes{v1.2-1}{2013/05/11}
% 	{Introduced to examine the emptiness of the box, extracting the code
% 	 from \cs{pcol@measurecolumn} so as to be used for coloring as
%	 well.}
% 
% The macro $\!\pcol@ifempty!\arg{box}\arg{then}\arg{else}$ is used in
% \!\pcol@putbackmvl!, \!\pcol@clearcst@unvbox! and
% \!\pcol@measurecolumn! to examine if $\arg{box}$ is empty, and to perform
% $\arg{then}$ if so or $\arg{else}$ otherwise.  Since \TeX{} does not
% provide any convenient way for the examination unfortunately, we perform a
% series of tricky operations to put the followings into \!\@tempboxa!; a
% penalty of $\!\pcol@magicpenalty!=12345$ whose existence in the
% $\arg{box}$ is (almost) impossible; contents of $\arg{box}$ put by
% \!\unvcopy!; and then a \!\global! \!\def!inition of \!\@gtempa! to let it
% have the decimal representation of \!\lastpenalty!.  Since \!\lastpenalty!
% has $\arg{pen}$ if the last item is $\!\penalty!\arg{pen}$, or 0
% otherwise, $\!\@gtempa!=\!\pcol@magicpenalty!$ iff $\arg{box}$ is empty.
% 
%    \begin{macrocode}
%% Special Output Routines: Color Management

\def\pcol@magicpenalty{12345}
\def\pcol@ifempty#1#2#3{%
  \setbox\@tempboxa\vbox{\penalty\pcol@magicpenalty
    \unvcopy#1\xdef\@gtempa{\number\lastpenalty}}%
  \ifnum\@gtempa=\pcol@magicpenalty\relax \def\reserved@a{#2}%
  \else                                   \def\reserved@a{#3}%
  \fi
  \reserved@a}

%    \end{macrocode}
% \end{macro}\end{macro}
% 
% \begin{macro}{\pcol@clearcst@unvbox}
% \changes{v1.2-1}{2013/05/11}
% 	{Introduced to put coloring \cs{special}s above and below of a
%	 column-page.}
% \begin{macro}{\pcol@clearcolorstack}
% \changes{v1.2-1}{2013/05/11}
% 	{Introduced to clear color context.}
% \changes{v1.34}{2018/05/07}
%	{Completely change its definition according to the new text coloring
%	 with \cs{insert}.}
% 
% The macro \!\pcol@clearcst@unvbox!$\arg{box}$, invoked from \!\pcol@opcol!
% and \!\pcol@output@switch!, puts the following above and below
% $\arg{box}$ containing the main vertical list of a \colpage{} from which
% we are now leaving, if \!\pcol@ifempty! judges that $\arg{box}$ is not
% empty.  Above the $\arg{box}$, we put coloring \!\special!s to establish
% the \colorstack{} of |.dvi| saved in $\csts=\!\pcol@colorstack@saved!$ by
% \!\pcol@restorecst! as the opening \colorctext{} of the \colpage{}.  The
% stack $\csts$, however, can be $\bot$ if $\arg{box}$ already has the
% \!\special!s, i.e., when we visit the \colpage{} it had already had
% some items.  Below $\arg{box}$, on the other hand, we put uncoloring
% \!\special!s by \!\pcol@clearcolorstack! to rewind $\CSTraw^c$ to clear the
% \colorctext{} of the \colpage{} in |.dvi| temporarily so that afterward it
% is made consistent with that in |.tex|.
% 
% The macro \!\pcol@clearcolorstack!, solely invoked from
% \!\pcol@clearcst@unvbox! shown above\footnote{
% 
% But we have this macro to avoid the complication in \cs{def}ining
% \cs{reserved@a} and \cs{reserved@b} with an argument if we did it in the
% argument of \cs{pcol@ifempty} in \cs{pcol@clearcst@unvbox}.},
% 
% scans $\CSTraw^c=(\Celt^c,\cstraw)$ by \!\pcol@scancst! giving
% $\cstraw=\!\pcol@colorins!$ to it as its argument.  Since we gives
% $\cstraw$ to the macro, this scan includes removals of all $\celtpop_i$
% and $\mceltpop_{i,m}$, all $\celt_i$ having matching $\celtpop_i$, all
% $\mcelt_{i,m}$ having matching $\mceltpop_{i,m}$, and all elements to
% update $\Celt^c$, from $\cstraw$ to let \!\pcol@colorins! have $\cst$.
% Prior to this invocation, we \!\def!ine $\!\reserved@a!\<\celt_i\>$ and
% $\!\reserved@b!\<\Celt^c\>$ to let them have \!\reset@color! so that
% uncoloring \!\special! will be put into the main vertical list for each
% $\celt_i\in\cst$ and $\Celt^c$ before update if any, regardless of
% coloring \!\special! they have.  That is, we invoke \!\reset@color! as
% many times as the appearance of $\celt_i\in\cst$ and once if
% $\Celt^c\neq\bot$ before the invocation, ignoring the color information in
% each element and the order of elements, expecting \!\reset@color! just
% pops printer's \colorstack{} to rewind it as we intend.
% 
% Note that in some printer |.def|inition could \!\def!ine \!\reset@color! to
% let printer's text color be \!\current@color! to make the stack rewinding
% resulting in the sequence of coloring operations with \!\current@color! at
% the invocation of \!\output!.  This meaningless operations might cause a
% problem when a colored \colpage{} of $c_1$ is physically followed by another
% \colpage{} of the succeeding column $c_2$ without any coloring, because the
% \colpage{} of $c_2$ will be colored with \!\current@color! at the page break
% in $c_1$.  If this problem is serious, we could initialize $\Celt^c$ with
% \!\current@color! at \beginparacol{} for all $c$ such that $\Celtshadow^c$
% is undefined, in order to make sure that any \colpage{} has at least one
% coloring \!\special! with $\Celt^c$ at its beginning so that, for example,
% coloring operations at the tail of the \colpage{} of $c_1$ is overridden
% by that of the default color of $c_2$ placed at the head of its \colpage{}.
% 
%    \begin{macrocode}
\def\pcol@clearcst@unvbox#1{%
  \pcol@ifempty#1\relax
   {\pcol@restorecst\pcol@colorstack@saved \unvbox#1\pcol@clearcolorstack}}
\def\pcol@clearcolorstack{%
  \def\reserved@a##1{\reset@color}\def\reserved@b##1{\reset@color}%
  \pcol@scancst\pcol@colorins}

%    \end{macrocode}
% \end{macro}\end{macro}
% 
% \changes{v1.2-1}{2013/05/11}
% 	{\cs{pcol@set@color@elt} was introduced to implement color context
%	 reestablishment but removed in v1.34.} 
% \changes{v1.34}{2018/05/07}
%	{\cs{pcol@set@color@elt} is removed according to the change of text
%	 coloring from \cs{output} to \cs{insert}.}
% 
% \begin{macro}{\pcol@restorecolorstack}
% \changes{v1.2-1}{2013/05/11}
% 	{Introduced to reestablish color context.}
% \changes{v1.34}{2018/05/07}
%	{Completely change its definition according to the new text coloring
%	 with \cs{insert}.}
% \begin{macro}{\pcol@restorecst}
% \changes{v1.2-1}{2013/05/11}
% 	{Introduced to reestablish color context.}
% \changes{v1.34}{2018/05/07}
%	{Completely change its definition according to the new text coloring
%	 with \cs{insert}.}
% 
% The macro \!\pcol@restorecolorstack!, used in \!\pcol@putbackmvl!
% and \!\pcol@output@end!, makes \colorctext{} in
% |.dvi| consistent with that in |.tex| by giving $\cst=\!\pcol@colorins!$
% to \!\pcol@restorecst! to let it scan $\CST^c=(\Celt^c,\cst)$.
% The callee macro $\!\pcol@restorecst!\arg{box}$, also used in
% \!\pcol@clearcst@unvbox! with $\arg{box}=\csts=\!\pcol@colorstack@saved!$,
% invokes \!\pcol@scancst! after \!\def!ining $\!\reserved@a!\<\celt_i\>$ to
% apply \!\unvbox! to $\celt_i$ in $\cst$ or $\csts$ and
% $\!\reserved@b!\<\Celt^c\>$ to apply \!\unvcopy! to $\Celt^c$ so that
% coloring \!\special!s they have will be put into the main vertical list.
% 
%    \begin{macrocode}
\def\pcol@restorecolorstack{\pcol@restorecst\pcol@colorins}
\def\pcol@restorecst{%
  \def\reserved@a##1{\unvbox##1}\def\reserved@b##1{\unvcopy##1}%
  \pcol@scancst}

%    \end{macrocode}
% \end{macro}\end{macro}
% 
% \begin{macro}{\pcol@scancst}
% \changes{v1.34}{2018/05/07}
%	{Introduced to implement new text coloring with \cs{insert}.}
% \begin{macro}{\pcol@iscancst}
% \changes{v1.34}{2018/05/07}
%	{Introduced to implement new text coloring with \cs{insert}.}
% 
% \begin{Sloppy}{1500}
% The macro $\!\pcol@scancst!\arg{box}$ is invoked from
% \!\pcol@clearcolorstack! and \!\pcol@restorecst!.  In the former
% invocation, we have $\arg{box}=\cstraw=\!\pcol@colorins!$ to rewind
% $\CSTraw=(\Celt^c,\cstraw)$ with $\!\reserved@a!\arg{\celt_i}$ and
% $\!\reserved@b!\arg{\Celt^c}$ having \!\reset@color!.  In the latter one,
% we have
% $\arg{box}\in\{\cst{=}\!\pcol@colorins!,\,\csts{=}\!\pcol@colorstack@saved!\}$
% to reestablish $\CST^c=(\Celt^c,\cst)$ or $\csts$ with
% $\!\reserved@a!\arg{\celt_i}$ to apply \!\unvbox! to $\celt_i$ and 
% $\!\reserved@b!\arg{\Celt^c}$ to apply \!\unvcopy! to $\Celt^c$.
% Therefore, if $\arg{box}=\!\pcol@colorins!$, we at first put (un)coloring
% \!\special! for $\Celt^c$, unless it is $\bot$, to the main vertical
% list applying \!\reserved@b! to it.  This means the \!\special! for
% $\Celt^c$ is put first prior to those for elements in $\cstraw$ or $\cst$
% consistently in reestablishing but not in rewinding.  However as we
% discussed in the description of \!\pcol@clearcolorstack!, the order of
% rewinding does not affect the result for almost all printers because only
% the number of pop operations is siginficant for them\footnote{
% 
% And even for the minority because multiple updates of printer's color with
% one particular color are independent of the order of them.}.
% 
% Then if $\arg{box}\neq\bot$ we invoke \!\pcol@iscancst! to examine the
% contents of $\arg{box}$ from its bottom to top.  Prior to the invocation,
% we do the following;  let \!\@tempboxa! have an empty \!\vbox! as its initial
% value of reformed $\arg{box}$;  let \!\pcol@tempboxa! have an empty
% \!\vbox! as its initial value of the sequence of \!\special!s to be put
% into the main vertical list; let $\npop=\!\@tempcntb!=0$; let
% $M=\!\reserved@b!=()$ as its initial value of the list of identifiers of
% math-mode pops; $\CSIndex{if@tempswa}=\true$ to mean the first \!\vbox! to
% update $\Celt^c$ found in the scan (i.e., the bottommost one) is
% effective.
% \end{Sloppy}
% 
% In the macro \!\pcol@iscancst!, we repeatedly examine the last \!\vbox! in
% $\arg{box}$ taken by \!\lastbox! into $\celt=\!\pcol@tempboxb!$ until
% $\celt$ becomes $\bot$, and perform one of the following for $\celt$.
% 
% \begin{enumerate}\def\labelenumi{(\arabic{enumi})}
% \def\HT{\mathit{height}}\def\DP{\mathit{depth}}\def\WD{\mathit{width}}
% \item
% If $\HT(\celt)=0$ and $\WD(\celt)=0$ to mean $\celt=\celtpop_i$, increment
% $\npop$.
% 
% \item
% If $\HT(\celt)=0$ and $\WD(\celt)=m>0$ to mean $\celt=\mceltpop_{i,m}$, let
% $M=(m,M)$.
% 
% \item
% If $\HT(\celt)\neq0$, $\DP(\celt)=0$ and $\WD(\celt)=0$ to mean
% $\celt=\celt_i$, decrement $\npop$ if $\npop>0$, or otherwise add $\celt$
% to the head of \!\@tempboxa! and apply $\!\reserved@a!\arg{\celt}$ to add
% its result to the head of \!\pcol@tempboxa!.
% 
% \item
% If $\HT(\celt)\neq0$, $\DP(\celt)=0$ and $\WD(\celt)=m>0$ to mean
% $\celt=\mcelt_{i,m}$, do nothing if $m\in M$, or otherwise add $\celt$
% to the head of \!\@tempboxa! and apply $\!\reserved@a!\arg{\celt}$ to add
% its result to the head of \!\pcol@tempboxa!.
% 
% \item
% If $\HT(\celt)\neq0$, $\DP(\celt)\neq0$ to mean $\celt$ has a \!\special!
% with which $\Celt^c$ is updated.  If $\CSIndex{if@tempswa}=\true$ to mean
% $\celt$ is the first (bottommost) occurence, $\Celt^c$ is updated
% acquiring an \!\insert! from \!\@freelist! if it was $\bot$.  In this
% case of $\bot$, we have to put an uncoloring \!\special! by
% \!\reset@color!, because \!\pcol@scancst! did not do it but
% \!\columncolor! or \!\normalcolumncolor! pushed the corresponding color
% \!\special!.  Then we let $\CSIndex{if@tempswa}=\false$.
% 
% Otherwise, i.e., if $\CSIndex{if@tempswa}=\false$ for second or succeeding
% occurences, we do nothing because updates by them are overridden by the
% first one.
% \end{enumerate}
% 
% Note that the cases other than (3) and (4) happen only in rewinding, and
% thus in reestablishing we only have (3) and (4) with $\npop=0$ and $M=()$
% always so that every $\celt$ is kept into new $\arg{box}$ and a coloring
% \!\special! for it will be put into the main vertical list.
% 
% Then we go back to \!\pcol@scancst! to let $\arg{box}=\!\@tempboxa!$,
% meaningless in reestablishing but not harmful, and put the contents of
% \!\pcol@tempboxa! to the main vertical list.
% 
% 
%    \begin{macrocode}
\def\pcol@scancst#1{%
  \@tempcnta#1\relax
  \ifnum\@tempcnta=\pcol@colorins
    \ifvoid\pcol@ccuse{@box}\else
      \reserved@b{\pcol@ccuse{@box}}\fi
  \fi
  \ifvoid\@tempcnta\else
    \setbox\@tempboxa\vbox{}\setbox\pcol@tempboxa\vbox{}\@tempcntb\z@
    \def\reserved@b{}\let\@elt\relax \@tempswatrue \pcol@iscancst
    \global\setbox\@tempcnta\box\@tempboxa \unvbox\pcol@tempboxa
  \fi}
\def\pcol@iscancst{%
  \setbox\@tempcnta\vbox{%
    \unvbox\@tempcnta \global\setbox\pcol@tempboxb\lastbox}%
  \ifvoid\pcol@tempboxb \let\reserved@c\relax
  \else
    \let\reserved@c\pcol@iscancst
    \ifdim\ht\pcol@tempboxb=\z@
      \ifdim\wd\pcol@tempboxb=\z@ \advance\@tempcntb\@ne
      \else \edef\reserved@b{\@elt{\number\wd\pcol@tempboxb}\reserved@b}%
      \fi
    \else\ifdim\dp\pcol@tempboxb=\z@
      \ifdim\wd\pcol@tempboxb=\z@
        \ifnum\@tempcntb>\z@ \advance\@tempcntb\m@ne
        \else
          \setbox\@tempboxa\vbox{\copy\pcol@tempboxb \unvbox\@tempboxa}%
          \setbox\pcol@tempboxa\vbox{%
            \reserved@a\pcol@tempboxb \unvbox\pcol@tempboxa}%
        \fi
      \else
        \count@\wd\pcol@tempboxb \chardef\reserved@d\z@
        \def\@elt##1{\ifnum##1=\count@ \chardef\reserved@d\@ne \fi}%
        \reserved@b \let\@elt\relax
        \ifnum\reserved@d=\z@
          \setbox\@tempboxa\vbox{\copy\pcol@tempboxb \unvbox\@tempboxa}%
          \setbox\pcol@tempboxa\vbox{%
            \reserved@a\pcol@tempboxb \unvbox\pcol@tempboxa}%
        \fi
      \fi
    \else\if@tempswa
      \ifvoid\pcol@ccuse{@box}%
        \@next\@currbox\@freelist{\global\setbox\@currbox\vbox{}}\pcol@ovf
        \pcol@ccxdef{\@currbox}\reset@color
      \fi
      \global\setbox\pcol@ccuse{@box}\vbox{\unvbox\pcol@tempboxb}%
      \@tempswafalse
    \fi\fi\fi
  \fi
  \reserved@c}

%    \end{macrocode}
% \end{macro}\end{macro}
% 
% \begin{macro}{\pcol@savecolorstack}
% \changes{v1.2-1}{2013/05/11}
% 	{Introduced to save the opening color context of a column-page.}
% \changes{v1.34}{2018/05/07}
%	{Completely change its definition according to the new text coloring
%	 with \cs{insert}.}
% 
% The macro \!\pcol@savecolorstack! is used in \!\pcol@startcolumn!,
% \!\pcol@output@start! and \!\pcol@putbackmvl! to save the opening
% \colorctext{} in $\CST^c$ of a \ccolpage{} $c$ known to be or found empty
% into $\csts=\!\pcol@colorstack@saved!$.  If both of
% $\Celt^c=|\pcol@columncolor@box|\cdot c$ and $\cst=\!\pcol@colorins!$ are
% $\bot$, $\csts$ is let be $\bot$.  Otherwise, $\csts$ is let be a \!\vbox!
% having a \!\vbox! for $\Celt^c$ at the top if it is not $\bot$ and then the
% contents of $\cst$ if it is not $\bot$.
% 
% \SpecialArrayIndex{c}{\pcol@columncolor@box}
% 
%    \begin{macrocode}
\def\pcol@savecolorstack{%
  \ifvoid\pcol@colorins \@tempswafalse \else \@tempswatrue \fi
  \ifvoid\pcol@ccuse{@box}%
    \setbox\@tempboxa\box\voidb@x
  \else
    \setbox\@tempboxa\vbox{\unvcopy\pcol@ccuse{@box}}%
    \ht\@tempboxa1sp \dp\@tempboxa\z@ \wd\@tempboxa\z@
    \@tempswatrue
  \fi
  \if@tempswa
    \global\setbox\pcol@colorstack@saved\vbox{%
      \ifvoid\@tempboxa\else \box\@tempboxa \fi
      \ifvoid\pcol@colorins\else \unvcopy\pcol@colorins \fi}
  \else \global\setbox\pcol@colorstack@saved\box\voidb@x
  \fi}

%    \end{macrocode}
% \end{macro}
% 
% \changes{v1.2-1}{2013/05/11}
% 	{\cs{pcol@colorstack@full} was introduced to represent
%	 $\string\mathit{\Gamma}^c$ but removed in v1.34.} 
% \changes{v1.34}{2018/05/07}
%	{\cs{pcol@colorstack@full} is removed according to the change of text
%	 coloring from \cs{output} to \cs{insert}.}
% 
% \KeepSpace{3}
% \begin{macro}{\pcol@ccuse}
% \changes{v1.34}{2018/05/07}
%	{Introduced to implement new text coloring with \cs{insert}.}
% \begin{macro}{\pcol@ifccdefined}
% \changes{v1.34}{2018/05/07}
%	{Introduced to implement new text coloring with \cs{insert}.}
% \begin{macro}{\pcol@ccxdef}
% \changes{v1.34}{2018/05/07}
%	{Introduced to implement new text coloring with \cs{insert}.}
% 
% The macro $\!\pcol@ccuse!\arg{pfx}$ is to expand a macro
% $|\pcol@columncolor|\cdot\arg{pfx}\cdot c$ for the current column $c$.  It
% is used in \!\pcol@output@start! and \!\pcol@scancst@shadow! with
% $\arg{pfx}=\hbox{`'}$ to have $\Celtshadow^c$, and in \!\pcol@scancst!,
% \!\pcol@iscancst!, \!\pcol@savecolorstack!, \!\pcol@output@end! and
% \!\pcol@icolumncolor! with $\arg{pfx}=|@box|$ to have $\Celt^c$.
% 
% \SpecialArrayIndex{c}{\pcol@columncolor}
% \SpecialArrayIndex{c}{\pcol@columncolor@box}
% 
% The macro $\!\pcol@ifccdefined!\arg{then}\arg{else}$ is used in
% \!\pcol@output@start! and \!\pcol@scancst@shadow! to examine whether
% $\Celtshadow^c=|\pcol@columncolor|\cdot c$ is defined and to do
% $\arg{then}$ if so or $\arg{else}$ otherwise.
% 
% \SpecialArrayIndex{c}{\pcol@columncolor}
% 
% The macro $\!\pcol@ccxdef!\arg{body}$ \!\xdef!ines a macro
% $\Celt^c=|\pcol@columncolor@box|\cdot c$ as $\arg{body}$ for the current
% column $c$.  It is used in \!\pcol@output@start!, \!\pcol@iscancst! and
% \!\pcol@icolumncolor!  with $\arg{body}=\!\@currbox!$ when an \!\insert!
% for $\Celt^c$ is acquired, and in \!\pcol@output@end! with
% $\arg{body}=\!\voidb@x!$ after releasing $\Celt^c$ to \!\@freelist!.
% 
% \SpecialArrayIndex{c}{\pcol@columncolor@box}
% 
%    \begin{macrocode}
\def\pcol@ccuse#1{\@nameuse{pcol@columncolor#1\number\pcol@currcol}}
\def\pcol@ifccdefined#1#2{%
  \expandafter\ifx\csname pcol@columncolor\number\pcol@currcol\endcsname\relax
  #2\else#1\fi}
\def\pcol@ccxdef#1{%
  \expandafter\xdef
    \csname pcol@columncolor@box\number\pcol@currcol\endcsname{#1}}

%    \end{macrocode}
% \end{macro}\end{macro}\end{macro}
% 
% 
% 
% \subsection{Footnote Handling}
% \label{sec:imp-sout-scfnote}
% \changes{v1.2-2}{2013/05/11}
%	{Add the subsection ``Footnote Handling'' to describe newly
%	 introduced macros for page-wise footnotes.}
% 
% \begin{macro}{\pcol@savefootins}
% \changes{v1.2-2}{2013/05/11}
% 	{Introduced to save footnotes in multiple occasions.}
% 
% The macro $\!\pcol@savefootins!\arg{cs}$, invoked from \!\pcol@makecol!
% for \Scfnote{}s and from \!\pcol@output@switch! for both \mcfnote{} and
% \Scfnote{}s, saves \!\footins! to an \!\insert! register obtained by
% \!\@next! from \!\@freelist!, and \!\def!ines $\arg{cs}$ being
% \!\pcol@currfoot! or \!\pcol@footins! so that it has the register as its
% body and the register is then saved into $\pp^f(p)$ or $\cc_c(\ft)$.  We
% save not only \!\box! component of \!\footins! but also \!\count!,
% \!\dimen! and \!\skip!\footnote{
% 
% Knowing these three components are virtually constants.}.
% 
%    \begin{macrocode}
%% Special Output Routines: Footnote Handling

\def\pcol@savefootins#1{%
  \@next#1\@freelist{%
    \global\setbox#1\box\footins
    \global\count#1\count\footins
    \global\dimen#1\dimen\footins
    \global\skip#1\skip\footins}{\def#1{\voidb@x}\pcol@ovf}}

%    \end{macrocode}
% \end{macro}
% 
% \begin{macro}{\pcol@shrinkcolbyfn}
% \changes{v1.2-2}{2013/05/11}
% 	{Introduced to shrink \cs{@colht} temporarily by the
%	 height-plus-depth of page-wise footnotes and the natural size
% 	 of the skip above them.}
% \changes{v1.3-3}{2013/09/17}
%	{Add the first argument $\langle\string\mathit{height}\rangle$ for
%	 the user \cs{pcol@ioutputelt}.}
% \changes{v1.3-6}{2013/09/17}
%	{Add the third argument $\langle\string\mathit{skip}\rangle$ to
%	 avoid accidental destruction of \cs{@tempdimb} which was modified
%	 unconditionally.}
% 
% The macro $\!\pcol@shrinkcolbyfn!\arg{height}\arg{ins}\arg{skip}$, invoked
% from \!\pcol@makecol!, \!\pcol@ioutputelt!, \!\pcol@startcolumn!,
% \!\pcol@restartcolumn!, \!\pcol@flushcolumn! and \!\pcol@makeflushedpage!,
% shrinks $\arg{height}\in\{\!\@colht!,\!\@tempdima!\}$ temporarily so that a
% column or the set of all columns resides in a page with \Scfnote{}s in the
% \!\insert! $\arg{ins}$.  The shrinkage is calculated by adding the sum of
% the height plus depth of all footnotes, i.e., that of $\arg{ins}$, and the
% natural component of $\!\skip!\arg{ins}$, i.e., the vertical space
% inserted between the columns and the footnotes.  Note that the stretch and
% shrink components of $\!\skip!\arg{ins}$ cannot be incorporated into the
% calculation but their contribution to each \colpage{} is taken care of by
% the following macro \!\pcol@unvbox@cclv! if required.  Also note that the
% inverse of $\!\skip!\arg{ins}$ is kept in $\arg{skip}$ if it is not
% \!\relax! so that \!\pcol@deferredfootins! knows $\arg{ins}\neq\bot$ when
% this macro is invoked from \!\pcol@startcolumn! and
% \!\pcol@restartcolumn! with $\arg{skip}=\!\@tempdimb!$.
% 
%    \begin{macrocode}
\def\pcol@shrinkcolbyfn#1#2#3{%
  \ifx#3\relax\else #3-\skip#2\relax \fi
  \advance#1-\ht#2\advance#1-\dp#2\advance#1-\skip#2\relax}
%    \end{macrocode}
% \end{macro}
% 
% \begin{macro}{\pcol@unvbox@cclv}
% \changes{v1.2-2}{2013/05/11}
% 	{Introduced to add stretch/shrink components of
%	 \cs{skip}\cs{footins} at the bottom of a column-page if the page has
%	 page-wise footnotes.}
% 
% The macro $\!\pcol@unvbox@cclv!\arg{ins}$, invoked from \!\pcol@makecol! and
% \!\pcol@flushcolumn! when they work on a column-page with \Scfnote{}s,
% adds the stretch and shrink components of $\!\skip!\arg{ins}$ at the end
% of \!\box!|255|, where $\arg{ins}$ is non-void $\pp^f(p)$ having
% \Scfnote{}s.  Before the addition, the macro goes back to the baseline of
% \!\box!|255|\footnote{
% 
% The comparison of the depth of \cs{box}\texttt{255} and \cs{@maxdepth} and
% taking the latter if it is smaller is really just-in-case.}
% 
% to nullify the baseline progress mechanism so as to make it sure the exact
% amount of the vertical skip is added.  Then it adds the stretch and shrink
% by at first adding the skip itself and then the negative amount of its
% natural component.
% 
%    \begin{macrocode}
\def\pcol@unvbox@cclv#1{%
  \@tempdima\dp\@cclv \unvbox\@cclv
  \vskip \ifdim\@tempdima>\@maxdepth -\@maxdepth \else -\@tempdima \fi
  \vskip\skip#1\@tempdima\skip#1\vskip-\@tempdima}

%    \end{macrocode}
% \end{macro}
% 
% \begin{macro}{\pcol@deferredfootins}
% \changes{v1.2-2}{2013/05/11}
% 	{Introduced to insert deferred footnotes.}
% \changes{v1.3-6}{2013/09/17}
%	{Fix the bug that the height cap was underestimated by the duplicated
%	 subtraction of \cs{skip}\cs{footins} if the page has already have
%	 non-deferred footnotes.}
% 
% The macro $\!\pcol@deferredfootins!\arg{macro}$, invoked from
% \!\pcol@startcolumn! and \!\pcol@restartcolumn!, tries to \!\insert! some
% of leading deferred footnotes in $\df$ through \!\footins!\footnote{
% 
% The argument $\arg{macro}$ has the invoker itself shown in debug
% messages.}.
% 
% In order to avoid that \!\footins! has footnotes across three or more
% pages to make confusion in the order of footnotes kept inside of \TeX, we
% cap the total height of footnotes by $h=\!\@colht!$ if it has already
% shrunk by non-deferred footnotes in the page we are working on indicated
% by $\!\@tempdimb!<0$, or $h=\!\@colht!-\!\skip!\!\footins!$ if the page
% does not have non-deferred footnotes indicated by $\!\@tempdimb!=0$.  That
% is, we extract leading elements of $\df$ by \!\vsplit! until their total
% height reaches $h$ and, if some elements are obtained in \!\@tempboxa!,
% \!\insert!  them through \!\footins!.  As for the remaining elements if
% any, we add a leading \!\penalty!\!\interlinepenalty! which should have
% been removed by \!\vsplit!.
% 
% Note that we temporarily let $\!\splitmaxdepth!=\!\@maxdepth!$,
% $\!\splittopskip!=0$ and $\!\vbadness!=\infty$ at the \!\vsplit! so that
% the depth of the split first half is \!\@maxdepth! at deepest, the second
% half does not have any skip at its top, and \TeX{} will not complain of
% (almost) inevitable underfull.  Also note that the successful extraction
% of the leading elements is examined by checking \!\ht!\!\@tempboxa! and
% thus we need to \!\unvbox!  it in itself because \!\vsplit! makes the
% height $h$ regardless of its contents.
% 
%    \begin{macrocode}
\def\pcol@deferredfootins#1{%
  \ifdim\@tempdimb=\z@ \@tempdimb\@colht \advance\@tempdimb-\skip\footins
  \else \@tempdimb\@colht
  \fi
  \ifvoid\pcol@topfnotes\else \ifdim\@tempdimb>\z@
    \begingroup
      \splitmaxdepth\@maxdepth \splittopskip\z@ \vbadness\@M
      \setbox\@tempboxa\vsplit\pcol@topfnotes to\@tempdimb
      \ifvoid\pcol@topfnotes\else
        \global\setbox\pcol@topfnotes\vbox{\penalty\interlinepenalty
          \unvbox\pcol@topfnotes}%
      \fi
      \setbox\@tempboxa\vbox{\unvbox\@tempboxa}%
      \ifdim\ht\@tempboxa>\z@
        \pcol@Log#1{add}\@tempboxa
        \insert\footins{\unvbox\@tempboxa}%
      \fi
    \endgroup
  \fi\fi}

%    \end{macrocode}
% \end{macro}
% 
% \begin{macro}{\pcol@combinefootins}
% \changes{v1.2-2}{2013/05/11}
% 	{Introduced to put footnotes into pre-environment stuff.}
% \changes{v1.35-4}{2018/12/31}
% 	{Add the insertion of vertical skip \cs{belowfootnoteskip}.}
% \begin{macro}{\pcol@putfootins}
% \changes{v1.2-2}{2013/05/11}
% 	{Introduced to put page-wise footnotes to a page.}
% \changes{v1.3-2}{2013/09/17}
%	{Change users \cs{pcol@outputelt} to \cs{pcol@ioutputelt}.}
% \changes{v1.3-6}{2013/09/17}
%	{Remove \cs{pcol@output@end} from users.}
% 
% The macro $\!\pcol@combinefootins!\arg{b}\arg{f}$, invoked solely from
% \!\pcol@makenormalcol!\footnote{
% 
% And thus having the arguments $\arg{b}$ and $\arg{f}$ is unnecessary, but
% we keep this implementation to avoid unnecessary recoding from a
% development version.},
% 
% constructs the \preenv{} in \!\@outputbox!, combining the stuff in the box
% $b$ and pre-environment footnotes in the \!\insert! $f$.  It is almost
% equivalent to the \cs{else} part of \!\@makecol! in which the main
% vertical list and \Mcfnote{}s are combined.  The operation to put
% footnotes is almost equivalent to the second half of the \cs{else} part
% except that \!\insert!  is not always \!\footins! but $f$ and is done by
% \!\pcol@putfootins! in which a null vertical skip is added after $f$ but
% this skip is removed by \!\unskip! after the invocation.  The other
% difference is that the vertical space of \!\belowfootnoteskip! is added if
% it is not 0\footnote{
% 
% We avoid null space insertion to minimize the difference from older
% versions in traced output.}
% 
% to have some space between non-merged \Preenv{} footnotes and the first
% lines in a \env{paracol} environment.
% 
% The null vertical skip is put for \Scfnote{}s, for which the macro
% \!\pcol@putfootins! is invoked from \!\pcol@ioutputelt! and
% \!\pcol@makeflushedpage!.  Since we shrink the height of \colpage{}s by
% the height-plus-depth of \Scfnote{}s, the natural height of the box in
% which \colpage{}s and \Scfnote{}s are combined would be less than
% \!\textheight! due to the depth of the last footnote line if we simply
% made the footnotes the last items of the box.  Though this shortage at
% most \!\maxdepth! is expected to be covered by the stretch factor of
% \!\skip!\!\footins! without too large badness causing an underfull
% message\footnote{
% 
% With default settings of $\cs{maxdimen}=5\,|pt|$ and the stretch factor
% 4\,|pt| of \cs{skip}\cs{footins}, the badness $100\times(5/4)^3\approx195$
% is significantly less than the default $\cs{vbadness}=1000$.},
% 
% someday we could face an underfull with some unusual settings of
% \!\maxdimen!, \!\skip!\!\footins! and/or \!\vbadness!.  Therefore, we put a
% null vertical skip so that the real bottom of the footnotes, instead of
% the last baseline, is placed at the baseline of the box, to make the
% natural height of the box is \!\textheight! exactly.  Note that this
% shifting \Scfnote{}s up will not make last baselines of footnotes among
% pages unaligned, because the last line have a strut.
% 
%    \begin{macrocode}
\def\pcol@combinefootins#1#2{%
  \setbox\@outputbox\vbox{%
    \boxmaxdepth\@maxdepth
    \unvbox#1\relax
    \pcol@putfootins#2\unskip
    \ifdim\belowfootnoteskip=\z@\else \vskip\belowfootnoteskip \fi}}
\def\pcol@putfootins#1{%
  \vskip\skip#1\relax
  \color@begingroup
    \normalcolor
    \footnoterule
    \unvbox#1\relax
  \color@endgroup \vskip\z@}

%    \end{macrocode}
% \end{macro}\end{macro}
% 
% 
% 
% \KeepSpace{3}
% \subsection{Marginal Notes}
% \label{sec:imp-sout-mpar}
% \changes{v1.3-4}{2013/09/17}
%	{Add this section ``Marginal Notes''.}
% 
% \begin{macro}{\@addmarginpar}
% \changes{v1.3-4}{2013/09/17}
%	{Made \cs{let}-equal to \cs{pcol@addmarginpar} in
%	 \string\texttt{paracol} environments.}
% \begin{macro}{\pcol@addmarginpar}
% \changes{v1.3-4}{2013/09/17}
%	{Introduced to make a margin sharable by marginal notes from
%	 different columns.}
% \changes{v1.35-3}{2018/12/31}
% 	{Fix the bug referring to \cs{@marbox} inappropriately.}
% \changes{v1.35-3}{2018/12/31}
% 	{Add vertical shifting of marginal note to emulate of
%	 \cs{marginnote}.}
% \begin{macro}{\pcol@@addmarginpar}
% \changes{v1.3-4}{2013/09/17}
%	{Introduced to keep the original definition of \cs{@addmarginpar}.}
% \def\swap{\mathit{swap}}
% 
% The macro \!\pcol@addmarginpar! is our own version of \!\@addmarginpar!,
% which \!\pcol@zparacol! makes \!\let!-equal to \!\pcol@addmarginpar!
% keeping its original definition in \!\pcol@@addmarginpar!.  Therefore, in
% an \env{paracol} environment, the \!\output! request made by \LaTeX's
% \!\marginpar! in the column $c$ and page $p$ is processed by
% \!\pcol@addmarginpar! through our own \!\output! routine being
% \!\pcol@output!, \!\pcol@specialoutput! and \LaTeX's \!\@specialoutput!.
% What we do in this macro are as follows;  determine the margin which a
% marginal note goes to; temporarily modify the parameter
% $m_w=\!\marginparwidth!$ or $m_s=\!\marginparsep!$ according to the margin
% and the column; determine the position to place the marginal note
% consulting $\pp^m(p)=\!\pcol@mparbottom!$ obtained by
% \!\pcol@getcurrpage!: and update $\pp^m(p)$ according to the postion.
% 
% First, there are the following parameters to determine the margin, and thus
% the value of $\CSIndex{if@firstcolumn}$ referred to in \LaTeX's
% \!\@addmarginpar! and meaning left if $\true$ or right if $\false$.
% 
% \begin{enumerate}\def\labelenumi{(\arabic{enumi})}
% \item
% The macros \!\pcol@mpthreshold@l! and \!\pcol@mpthreshold@r! defined by
% \!\marginpar~threshold!\ give us the threshold of the column number to let
% columns less than it go to the left margin while those equal to or greater
% than it to the right, according to the \parapag{}e the column belongs to.
% Therefore, we let $\CSIndex{if@firstcolumn}=\true$ iff
% $c<\!\pcol@mpthreshold@l!\land c<\CL$ or $c<\!\pcol@mpthreshold@r!\land
% c\geq\CL$, as the fundamental setting.
% 
% \item
% If $\CSIndex{ifpcol@swapmarginpar}=\true$ because the specifier `|m|' is
% given to \!\twosided! explicitly or implicitly, the decision in (1) is
% reversed if $\page(p)\bmod2=0$, and then this decision is reversed again
% if $c\geq\CL$ and $\CSIndex{ifpcol@paired}=\false$ to mean $c$ is in a
% \npaired{} right \parapag{}e.  The second reversal is done because
% $\page(p)$ is \emph{common} to both left and right \parapag{}es and is for
% the left page in usual cases without page number jumps.  Therefore,
% $(\page(p)\bmod2)$ is for left \parapag{}es and thus for right counterparts
% the decision should be made by $((\page(p)+1)\bmod2)$ and thus the result
% should be reversed.
% 
% \item
% If \CSIndex{if@reversemargin} is made $\true$ by \LaTeX's
% \!\reversemarginpar!, the decision made by (1) and then possibly reversed
% by (2) is finally reversed.
% \end{enumerate}
% 
% Second, we calculate
% $$
% D=\!\@tempdima!=\sum_{d=\Cfrom}^{\swap(c)-1}(w_{\swap(d)}+g_{\swap'(d)})
% $$
% where $x'=swap(x)$ is given by
% $\!\pcol@swapcolumn!\<x\>\<x'\>\<\Cfrom\>\<\Cto\>$ to let
% $x'=\Cto-(x-\Cfrom)-1$ if $\CSIndex{ifpcol@swapcolumn}=\true$ and
% $\page(p)\bmod2=0$ or $x'=x$ otherwise,
% $\swap'(x)=\!\pcol@colsepid!\in\{\swap(x)-1,x\}$ according to swapped or
% not, $(\Cfrom,\Cto)\in\{(0,\CL),(\CL,\C)\}$ according to $c<\CL$ or not,
% and $\w_x$ and $\gap_x$ are the width of $x$-th column and gap given by
% $|\pcol@columnwidth|{\cdot}x$ and $|\pcol@columnsep|{\cdot}x$
% respectively.
% 
% \SpecialArrayIndex{c}{\pcol@columnwidth}
% \SpecialArrayIndex{c}{\pcol@columnsep}
% 
% That is, $D$ is the distance from the left edge of the column $c$ to that
% of leftmost column in the (paralell-)
% 
% \Index{parallel-paging}
% 
% page in which $c$ resides.  Then if $\CSIndex{if@firstcolumn}=\true$ to
% let the margial note go to the left margin, we add $D$ to
% $m_w=\!\marginparwidth!$ so that \LaTeX's \!\@addmarginpar! being
% \!\pcol@@addmarginpar! aligns the left edge of the marginal note at the
% point apart from the column's left edge by $(D+m_w)+m_s$ rather than
% $m_w+m_s$, or in other words $m_w+m_s$ apart from the left edge of the
% leftmost column.  On the other hand if $\CSIndex{if@firstcolumn}=\false$,
% we add $\WT-(D+w_c)$ where $\WT$ is \!\textwidth!, being the distance from
% the right edge of the column $c$ to that of the rightmost column, to
% $m_s=\!\marginparsep!$ so that \!\pcol@@addmarginpar!  aligns the left
% edge of the marginal note at the point apart from the column's right edge
% by $\WT-(D+w_c)+m_s$ rather than $m_s$, or in other words $m_s$ apart from
% the right edge of the rightmost column.
% 
% Third, we let \Midx{\!\pcol@marbox!} be the first element of \!\@currlist!
% obtained by \!\@xnext! for the right marginal note if
% $\CSIndex{if@firstcolumn}=\false$, or \!\@currbox! for the left marginal
% note otherwise.  Then we let $t=\!\@tempdima!$ be the distance from the
% top edge of the column to that of the marginal note, namely \!\@pageht!
% minus the height of \!\pcol@marbox! plus |\dimen|\!\@currbox! being
% downward shift amount optionally given by \!\marginnote!.  We also let
% $h=\!\@tempdimb!$ be the height-plus-depth of the box \!\pcol@marbox!
% plus \!\marginparpush!, or in other words the vertical space the marginal
% note requires.  Then we give $t$ and $h$ to \!\pcol@getmparbottom! to let
% \!\@mparbottom! have the bottom of the exisiting marginal text below which
% we put the margial text in \!\pcol@marbox!, and to let \!\pcol@mpblist!
% have the new list to be replaced with $\mpb_{\{L,R\}}^{\{l,r\}}$ in
% $\pp^m(p)$.
% 
% Forth, we update $\pp^m(p)$ with the new list in \!\pcol@mpblist! by a
% process similar to \!\pcol@setpageno! but with \!\pcol@setmpbelt! to scan
% the list of pages $\PPP$.
% 
% Fifth, we shift down \!\pcol@marbox! by |\dimen|\!\@currbox! and, if the
% shift amount is greater than the height of the box and thus the height of
% shifting result is zero, decrement \!\@mparbottom! by the amount to
% deceive \LaTeX's \!\@addmarginpar! into believing \!\@mparbottom! is above
% the real one by the amount.  In other words, by the decrement we let
% \!\@addmarginpar! see the top edge of the shifted marginal note in the
% box, rather than that of the box itself, for the placement with
% \!\@mparbottom!.
% 
% Finally, we invoke \LaTeX's original \!\@addmarginpar! being
% \!\pcol@@addmarginpar! to put the marginal note according to
% \CSIndex{if@firstcolumn}, temporarily modified \!\marginparsep! and
% \!\marginparwidth!, and \!\@mparbottom!.  Note that since
% \!\pcol@zparacol! lets $\CSIndex{if@twocolumn}\~=\true$,
% \!\pcol@@addmarginpar! determine the margin only by
% \CSIndex{if@firstcolumn}.  Also note that it can be $\!\@mparbottom!>t$
% (before decrement with the positive shift amount) to mean the marginal
% note should be shifted down from its natural position, and if so
% \!\pcol@@addmarginpar!  will give us a warning as the correct consequence
% of the placement.
% 
%    \begin{macrocode}
%% Special Output Routines: Marginal Notes

\def\pcol@addmarginpar{%
  \pcol@getcurrpage \@firstcolumntrue
  \ifnum\pcol@currcol<\pcol@ncolleft
    \let\reserved@a\z@ \let\reserved@b\pcol@ncolleft
    \ifnum\pcol@mpthreshold@l>\pcol@currcol\else \@firstcolumnfalse \fi
  \else
    \let\reserved@a\pcol@ncolleft \let\reserved@b\pcol@ncol
    \ifnum\pcol@mpthreshold@r>\pcol@currcol\else \@firstcolumnfalse \fi
  \fi
  \ifpcol@swapmarginpar
    \ifodd\c@page\else
      \if@firstcolumn \@firstcolumnfalse \else \@firstcolumntrue \fi
    \fi
    \ifpcol@paired\else \ifnum\pcol@currcol<\pcol@ncolleft\else
      \if@firstcolumn \@firstcolumnfalse \else \@firstcolumntrue \fi
    \fi\fi
  \fi
  \if@reversemargin
    \if@firstcolumn \@firstcolumnfalse \else \@firstcolumntrue \fi
  \fi
  \pcol@swapcolumn\pcol@currcol\count@\reserved@a\reserved@b
  \@tempdima\z@
  \@tempcnta\reserved@a \@whilenum\@tempcnta<\count@\do{%
    \pcol@swapcolumn\@tempcnta\@tempcntb\reserved@a\reserved@b
    \advance\@tempdima\csname pcol@columnwidth\number\@tempcntb\endcsname\relax
    \advance\@tempdima\csname pcol@columnsep\pcol@colsepid\endcsname\relax
   \advance\@tempcnta\@ne}%
  \if@firstcolumn \advance\marginparwidth\@tempdima
  \else
    \advance\marginparsep\textwidth \advance\marginparsep-\@tempdima
    \advance\marginparsep-\columnwidth
  \fi
  \expandafter\@xnext\@currlist\@@\pcol@marbox\@gtempa
  \if@firstcolumn\let\pcol@marbox\@currbox \fi
  \@tempdima\@pageht \advance\@tempdima-\ht\pcol@marbox
  \advance\@tempdima\dimen\@currbox
  \@tempdimb\ht\pcol@marbox \advance\@tempdimb\dp\pcol@marbox
  \advance\@tempdimb\marginparpush
  \pcol@getmparbottom\@tempdima\@tempdimb
  \begingroup
    \@tempcnta\pcol@page \advance\@tempcnta-\pcol@basepage
    \edef\reserved@a{\pcol@pages\pcol@currpage}%
    \global\let\pcol@pages\@empty \global\let\pcol@currpage\@empty
    \let\@elt\pcol@setmpbelt \reserved@a
  \endgroup
  \ifdim\dimen\@currbox=\z@\else
    \ifdim\dimen\@currbox>\ht\pcol@marbox
       \advance\@mparbottom-\dimen\pcol@marbox
    \fi
    \setbox\pcol@marbox\hbox{\lower\dimen\@currbox\box\pcol@marbox}%
  \fi
  \pcol@@addmarginpar}

%    \end{macrocode}
% \end{macro}\end{macro}\end{macro}
% 
% \KeepSpace{3}
% \begin{macro}{\pcol@getmparbottom}
% \changes{v1.3-4}{2013/09/17}
%	{Introduced to find the space where a marginal note is placed.}
% \begin{macro}{\pcol@getmparbottom@i}
% \changes{v1.3-4}{2013/09/17}
%	{Introduced to find the space where a marginal note is placed.}
% \begin{macro}{\pcol@getmpbelt}
% \changes{v1.3-4}{2013/09/17}
%	{Introduced to find the space where a marginal note is placed.}
% \changes{v1.33-1}{2016/11/19}
% 	{Fix the bug by which $t_k$ such that $t_k\geq t$ and $t_k-t\geq h$
%	 but $t_k-b_{k-1}<h$ is found.}
% 
% The macro $\!\pcol@getmparbottom!\<t\>\<h\>$ is solely used in
% \!\pcol@addmarginpar!\footnote{
% 
% Thus giving $t$ and $h$ as arguments is not necessary but we dare to do
% it.}
% 
% to let $B=\!\@mparbottom!$ point the bottom of the marginal note below
% which a new marginal note $m$, whose natural top is at $t$ and occupancy
% height is $h$, in the current column $c$ and the page $p$ is placed, or 0
% if the side margin has had no marginal notes yet.  It is also \!\def!ines
% $M'=\Midx{\!\pcol@mpblist!}$ having $\mpar(t',t'{+}h)$ for the new marginal
% note $m$ placed at $t'$ in it and being replaced with one of
% $\mpb_{\{L,R\}}^{\{l,r\}}$ in $\pp^m(p)$ by \!\pcol@setmpbelt!.
% 
% First we examine if $\pp^m(p)=\!\pcol@mparbottom!$ is empty and if so we
% simply let $B=0$ and $M'=(\mpar(t,t+h))$ because there are no marginal
% notes in the page $p$ at all.  Otherwise we obtain
% $M\in\Set{\mpb_X^x}{X\in\{L,R\},\;x\in\{l,r\}}$ in \!\reserved@a! according
% to the side margin to which the new marginal note $m$ goes to, i.e.,
% according to $\CSIndex{if@firstcolumn}$ and $c<\CL$ or not, by
% \!\pcol@getmparbottom@i! giving it the body of $\pp^m(p)$ by
% \!\expandafter!.  Then we apply $\!\pcol@getmpbelt!$ to each
% $\mpar(t_i,b_i)\in M$ to have
% $t_k=\min\Set{t_i}{t_i\geq{t},\;t_i-\max(t,b_{i-1})\geq{h}}$ and let
% $B=b_{k-1}$ and
% 
% \begin{eqnarray*}
% M'&=&(\mpar(t_1,b_1),\;\ldots,\;\mpar(t_{k-1},b_{k-1}),\;
%       \mpar(\max(t,B),\,\max(t,B)+h),\\
% &&\phantom{\LP}
%       \mpar(t_k,b_k),\;\ldots,\;\mpar(t_n,b_n))
% \end{eqnarray*}
% 
% where $n=\Abs{M}$, assuming $b_0=0$.  That is, we try to find the first
% avilable gap between marginal notes below $t$ to accommodate the
% marginal note $m$ of $h$ tall.  Then if such $t_k$ is not found because
% $t>t_n$ to mean $m$ appears below the last marginal note as in natural
% cases, or $t_i-\max(t,b_{i-1})<h$ for all $t_i$ s.t.\ $t_i>t$ to mean
% there are no available gaps, we let $B=t_n$ and
% $M'=(M,\mpar(\max(t,B),\,\max(t,B)+h))$ to place $m$ at the bottom.
% 
%    \begin{macrocode}
\def\pcol@getmparbottom#1#2{%
  \global\@mparbottom\z@
  \ifx\pcol@mparbottom\@empty
    \begingroup
      \@tempdimc#2\relax \advance\@tempdimc#1\relax \let\@elt\relax
      \xdef\pcol@mpblist{\@elt{\number#1}{\number\@tempdimc}}%
    \endgroup
  \else
    \expandafter\pcol@getmparbottom@i\pcol@mparbottom
    \begingroup
      \@tempdima#1\relax \@tempdimb#2\relax \@tempswafalse
      \let\@elt\pcol@getmpbelt \global\let\pcol@mpblist\@empty \reserved@a
      \if@tempswa\else
        \ifdim\@tempdima<\@mparbottom \@tempdima\@mparbottom \fi
        \advance\@tempdimb\@tempdima
        \@cons\pcol@mpblist{{\number\@tempdima}{\number\@tempdimb}}%
      \fi
    \endgroup
  \fi}
\def\pcol@getmparbottom@i#1#2#3#4{%
  \ifnum\pcol@currcol<\pcol@ncolleft
    \if@firstcolumn \def\reserved@a{#1}\else\def\reserved@a{#2}\fi
  \else
    \if@firstcolumn \def\reserved@a{#3}\else\def\reserved@a{#4}\fi
  \fi}
\def\pcol@getmpbelt#1#2{%
  \ifdim#1sp<\@tempdima
    \global\@mparbottom#2sp\relax \@cons\pcol@mpblist{{#1}{#2}}%
    \ifdim\@tempdima<#2sp\relax \@tempdima#2sp\relax \fi
  \else
    \@tempdimc\@tempdima \advance\@tempdimc\@tempdimb
    \ifdim#1sp<\@tempdimc
      \@tempdima#2sp\relax \global\@mparbottom#2sp\relax
      \@cons\pcol@mpblist{{#1}{#2}}%
    \else
      \@cons\pcol@mpblist{{\number\@tempdima}{\number\@tempdimc}\@elt{#1}{#2}}%
      \@tempswatrue
      \def\pcol@getmpbelt##1##2{\@cons\pcol@mpblist{{##1}{##2}}}
    \fi
  \fi}

%    \end{macrocode}
% \end{macro}\end{macro}\end{macro}
% 
% \begin{macro}{\pcol@setmpbelt}
% \changes{v1.3-4}{2013/09/17}
%	{Introduced to update $\pi^m(p)$.}
% \begin{macro}{\pcol@setmpbelt@i}
% \changes{v1.3-4}{2013/09/17}
%	{Introduced to update $\pi^m(p)$.}
% 
% The macro $\!\pcol@setmpbelt!
% \Arg{\pp^p(q)}\<\pp^i(q)\>\<\pp^f(q)\>\Arg{\pp^s(q)}\Arg{\pp^m(q)}$ is
% used solely in \!\pcol@addmarginpar! and is applied to $\PPP$ to update an
% element $M\in\Set{\mpb_X^x}{X\in\{L,R\},x\in{l,r}}$ in $\pp^m(p)$ with
% $M'=\!\pcol@mpblist!$ given by \!\pcol@getmparbottom!.  The structure of
% the macro is similar to \!\pcol@setpnoelt! to upadate $\pp^p(q)$ s.t.\
% $q\geq p$, but in this macro the target of the update is only $p$.  Then
% for $q=p$, we invoke \!\pcol@setmpbelt@i! giving it the body of $\pp^m(p)$
% being
% $\Arg{\mpb_L^l}\Arg{\mpb_L^r}\Arg{\mpb_R^l}\Arg{\mpb_R^r}$, or
% $\<\emptyset\>\<\emptyset\>\<\emptyset\>\<\emptyset\>$ if
% $\pp^m(p)=\emptyset$, to obtain what $\pp^m(p)$ should have in
% $\pp_{\mathit{new}}^m(p)=\!\reserved@a!$ in which $\mpb_X^x$ is replaced
% with $M'$, where $X=L$ or $X=R$ if $c<\CL$ or not respectively, and $x=l$
% or $x=r$ if $\CSIndex{if@firstcolumn}=\true$ or not respectively, and then
% update $\pp^m(p)$ by $\!\pcol@defcurrpage!
% \Arg{\pp^p(p)}\<\pp^i(p)\>\<\pp^f(p)\>\Arg{\pp^s(p)}
% \Arg{\pp_{\mathit{new}}^m(q)}$.
% 
%    \begin{macrocode}
\def\pcol@setmpbelt#1#2#3#4#5{%
  {\let\@elt\relax \xdef\pcol@pages{\pcol@pages\pcol@currpage}}%
  \ifnum\@tempcnta=\z@
    \def\reserved@a{#5}%
    \ifx\reserved@a\@empty \pcol@setmpbelt@i{}{}{}{}\else \pcol@setmpbelt@i#5\fi
    \pcol@defcurrpage{#1}{#2}{#3}{#4}{\reserved@a}%
  \else \gdef\pcol@currpage{\@elt{#1}#2#3{#4}{#5}}%
  \fi
  \advance\@tempcnta\m@ne}
\def\pcol@setmpbelt@i#1#2#3#4{%
  \ifnum\pcol@currcol<\pcol@ncolleft
    \if@firstcolumn \def\reserved@a{{\pcol@mpblist}{#2}{#3}{#4}}%
    \else           \def\reserved@a{{#1}{\pcol@mpblist}{#3}{#4}}%
    \fi
  \else
    \if@firstcolumn \def\reserved@a{{#1}{#2}{\pcol@mpblist}{#4}}%
    \else           \def\reserved@a{{#1}{#2}{#3}{\pcol@mpblist}}%
    \fi
  \fi}

%    \end{macrocode}
% \end{macro}\end{macro}
% 
% \begin{macro}{\pcol@mparbottom@zero}
% \changes{v1.3-4}{2013/09/17}
%	{Introduced to give the default of \cs{pcol@mparbottom@out}.}
% \begin{macro}{\pcol@mparbottom@out}
% \changes{v1.3-4}{2013/09/17}
%	{Introduced to keep the last elements of $\pi^m(p_t)$ at
%	 \cs{end}\string\texttt{\char`\{paracol\char`\}.}}
% 
% The macro $\Uidx\mpboutz=\!\pcol@mparbottom@zero!$ is used in
% \!\@outputpage!, \!\pcol@getmparbottom@last! and \!\pcol@output@end! to
% give the default value of $\Uidx\mpbout=\!\pcol@mparbottom@out!$ with
% $\mpar(0,0)$ for all elements $M\in\Set{\mpb_X^x}{X\in\{L,R\},x=\{l,r\}}$
% to mean a page has no marginal notes carrying over from the preceding
% \env{paracol} environments.
% 
% As for $\mpbout$, besides the top level initialization to make it
% $\mpboutz$, it is updated in \!\pcol@output@end! through macros
% \!\pcol@getmparbottom@last! and \!\pcol@bias@mpbout! to have the last
% element of each $M\in\Set{\mpb_X^x}{X\in\{L,R\},x=\{l,r\}}$ in
% $\pp^m(\ptop)$ with transformation from coordinates of columns to that of
% text area, or directly with $\mpboutz$ if the \lpage{} will not have
% \postenv.  The resulting $\mpbout$ is at first used in \!\pcol@output@end!
% itself through \!\pcol@do@mpbout! to let \!\@mparbottom! have the value
% $b$ of $\mpar(t,b)$ in $\mpb_L^l$ or $\mpb_L^r$ according to the side
% margin which marginal notes in \postenv{} goes to.
% 
% Then $\mpbout$ is passed to the next \env{paracol} environment if it
% resides in the page where the previous environment also resides, to be
% referred to by \!\pcol@output@start! which also performs
% \!\pcol@do@mpbout! and \!\pcol@bias@mpbout! to let $\pp^m(0)$ have the
% lists according to $\mpbout$ and $\!\@mparbottom!$ which can be modified
% in \postenv{} of the previous environment or in other word
% in \preenv{} of starting envionment.  By this setting of $\pp^m(0)$,
% marginal note placement in the \spage{} is aware of the marginal notes
% having been placed in previous environments and in \preenv{} and thus can
% correctly examines if a marginal note to be added in a margin collide 
% the last one in the margin.  On the other hand, if the \postenv{}
% encounters a page break before a new environment starts, our own
% \!\@outputpage! should be invoked at the page break to let
% $\mpbout=\mpboutz$ because the marginal notes in previous environments do
% not afffect those in the new environment.
% 
% Note that $\mpbout$ is also referred to and updated by
% \!\pcol@do@mpb@all@i! and \!\pcol@do@mpb@all@ii! because they are used in
% \!\pcol@bias@mpbout! and \!\pcol@getmparbottom@last! through
% \!\pcol@do@mpb@all!.
% 
%    \begin{macrocode}
\gdef\pcol@mparbottom@zero{{\@elt{0}{0}}{\@elt{0}{0}}{\@elt{0}{0}}{\@elt{0}{0}}}
\global\let\pcol@mparbottom@out\pcol@mparbottom@zero

%    \end{macrocode}
% \end{macro}\end{macro}
% 
% \KeepSpace{2}
% \begin{macro}{\pcol@do@mpbout}
% \changes{v1.3-4}{2013/09/17}
%	{Introduced to do specified operations on $\string\cal{M}$ and its
%	 element $M_L^x$ acccording to the side margin for marginal notes
%	 outside \texttt{paracol} environments.}
% \begin{macro}{\pcol@do@mpbout@i}
% \changes{v1.3-4}{2013/09/17}
%	{Introduced to do specified operations on $\string\cal{M}$ and its
%	 element $M_L^x$ acccording to the side margin for marginal notes
%	 outside \texttt{paracol} environments.}
% \begin{macro}{\pcol@do@mpbout@whole}
% \changes{v1.3-4}{2013/09/17}
%	{Introduced to do a specified operation on $\string\cal{M}$.}
% \begin{macro}{\pcol@do@mpbout@elem}
% \changes{v1.3-4}{2013/09/17}
%	{Introduced to do a specified operation on an element $M_L^x$ in
%	 $\string\cal{M}$.}
% 
% The macro \!\pcol@do@mpbout! is used in \!\pcol@output@start! and
% \!\pcol@output@end! to perform operations specified by
% \!\pcol@do@mpbout@whole!  and \!\pcol@do@mpbout@elem!.  The macro just
% invokes \!\pcol@do@mpbout@i! giving it all
% $M\in\Set{\mpb_X^x}{X\in\{L,R\},x\in\{l,r\}}$ by \!\expandafter!.
% 
% Then \!\pcol@do@mpbout@i! determines the side margin $x\in\{l,r\}$ letting
% $x=l$ iff $\CSIndex{if@mparswitch}=\true$, $\page(p)\bmod2=0$ and
% $\CSIndex{if@reversemargin}=\false$, to invoke \!\pcol@do@mpbout@whole!
% giving it all $M$ but $\mpb_L^x$ whose sole element $\mpar(t,b)$
% may be modified by $\!\pcol@do@mpbout@elem!\!\@elt!\Arg{t}\Arg{b}$.
% 
% In \!\pcol@output@start!, \!\pcol@do@mpbout@whole! is to \!\xdef!ine
% $\mpbout$ with all $M$ and \!\pcol@do@mpbout@elem! is expanded to
% $\mpar(0,B)$, where $B=\!\@mparbottom!$, regardless of $\mpb_L^x$ so that
% it is replaced with $(\mpar(0,B))$ in the modified $\mpbout$ keeping other
% elements unchanged.  In \!\pcol@output@end!, \!\pcol@do@mpbout@whole! is
% to throw all $M$ away into a \!\hbox! while \!\pcol@do@mpbout@elem! lets
% $B=b$ so that the bottom of the last marginal note in the side margin
% specified by $x$ is passed to \postenv{} through \!\@mparbottom!.
% 
%    \begin{macrocode}
\def\pcol@do@mpbout{\expandafter\pcol@do@mpbout@i\pcol@mparbottom@out}
\def\pcol@do@mpbout@i#1#2#3#4{\@tempcnta\@ne
  \if@mparswitch \ifodd\c@page\else \@tempcnta\m@ne \fi\fi
  \if@reversemargin \@tempcnta-\@tempcnta \fi
  \ifnum\@tempcnta<\z@
    \pcol@do@mpbout@whole{\pcol@do@mpbout@elem#1}{#2}{#3}{#4}%
  \else
    \pcol@do@mpbout@whole{#1}{\pcol@do@mpbout@elem#2}{#3}{#4}%
  \fi}

%    \end{macrocode}
% \end{macro}\end{macro}\end{macro}\end{macro}
% 
% \begin{macro}{\pcol@bias@mpbout}
% \changes{v1.3-4}{2013/09/17}
%	{Introduced to perform coordinate transformation of the elements in
% 	 $\string\cal{M}$.}
% \begin{macro}{\pcol@bias@mpbout@i}
% \changes{v1.3-4}{2013/09/17}
%	{Introduced to perform coordinate transformation of the elements in
% 	 $\string\cal{M}$.}
% 
% The macro $\!\pcol@bias@mpbout!\Arg{y}$ is used in \!\pcol@output@start!
% with $-y$ being the heigh-plus-depth of \preenv{}, in
% \!\pcol@output@end! with $y$ being that of \spanning{} in the \lpage, and
% in $\!\pcol@getmparbottom@last!\Arg{y}$ with its argument $y$.  The macro
% modifies $\mpar(t,b)$ in all $M\in\Set{\mpb_X^x}{X\in\{L,R\},x\in\{l,r\}}$
% of $\mpbout$ so that they have $\mpar(t{+}y,b{+}y)$ for the transformation
% from text area coordinates to columns in the first and third, while for
% the reverse transformation in the second, by invoking \!\pcol@do@mpb@all!
% giving it $\mpbout$ and letting \!\reserved@a! have
% $\!\pcol@bias@mpbout@i!\Arg{y}$.  That is,
% $\!\pcol@bias@mpbout@i!\Arg{y}\!\@elt!\Arg{t}\Arg{b}\!\@nil!$ is then
% invoked in \!\pcol@do@mpb@all@ii! with $t$ and $b$ from $\mpar(t,b)$ in
% each $M$, and then it \!\def!ines \!\reserved@b! with $\mpar(t{+}y,b{+}y)$
% so that updated $M$ has it.
% 
%    \begin{macrocode}
\def\pcol@bias@mpbout#1{\def\reserved@a{\pcol@bias@mpbout@i{#1}}%
  \pcol@do@mpb@all\pcol@mparbottom@out}
\def\pcol@bias@mpbout@i#1\@elt#2#3\@nil{%
  \dimen@#2sp\relax \advance\dimen@#1\relax
  \dimen@ii#3sp\relax \advance\dimen@ii#1\relax
  \def\reserved@b{\@elt{\number\dimen@}{\number\dimen@ii}}}

%    \end{macrocode}
% \end{macro}\end{macro}
% 
% \begin{macro}{\pcol@getmparbottom@last}
% \changes{v1.3-4}{2013/09/17}
%	{Introduced to let $\string\cal{M}$ have the occupancy information of
%	 the bottom marginal note in each margin.}
% \begingroup\let\small\footnotesize
% \begin{macro}{\pcol@getmparbottom@last@i}
% \changes{v1.3-4}{2013/09/17}
%	{Introduced to let $\string\cal{M}$ have the occupancy information of
%	 the bottom marginal note in each margin.}
% 
% The macro $\!\pcol@getmparbottom@last!\Arg{y}$ is used solely in
% \!\pcol@output@end! to let
% $\mpbout=\Arg{m_L^l}\Arg{m_L^r}\Arg{m_R^l}\Arg{m_R^r}$, where
% $m_X^x=\mpar(t_n,b_n)\in\mpb_X^x$, $n=\Abs{\mpb_X^x}$ asuming
% $\mpar(t_0,b_0)={\mpar}(y,y)$, and $y$ is the negative counterpart of the
% height-plus-depth of the \spanning{} in the \lpage.  Therefore, $\mpbout$
% is let have the occupancy information of the last marginal note if any, or
% the top edge of text area otherwise, in each margin.
% 
% The macro at first examines if $\pp^m(\ptop)=\emptyset$ and, if so, lets
% all elements in $\mpbout$ have $(\mpar(y,y))$ by letting it $\mpboutz$ and
% then adding $y$ to each $t=b=0$ by \!\pcol@bias@mpbout! giving it $y$.
% Otherwise, i.e., if $\pp^m(\ptop)\neq\emptyset$, it invokes
% \!\pcol@do@mpb@all! giving it $\pp^m(\ptop)$ and letting \!\reserved@a!
% have $\!\pcol@getmparbottom@last@i!\Arg{y}$.  That is,
% $\!\pcol@getmparbottom@last@i!\Arg{y}\~\mpar(t_1,b_1)\cdots
% \mpar(t_n,b_n)\!\@nil!$ is then invoked in \!\pcol@do@mpb@all@ii! for each
% $\mpb_X^x$ to let \!\reserved@b! have $\mpar(y,y)$ at first and then to
% let it have $\mpar(t_i,b_i)$ for all $i\in[1,n]$.  Therefore,
% \!\reserved@b! should finally have $\mpar(t_n,b_n)$ assuming $t_0=b_0=y$,
% and then becomes $m_X^x$.
% \end{macro}\endgroup
% 
%    \begin{macrocode}
\def\pcol@getmparbottom@last#1{%
  \ifx\pcol@mparbottom\@empty
    \global\let\pcol@mparbottom@out\pcol@mparbottom@zero
    \pcol@bias@mpbout{#1}%
  \else
    \def\reserved@a{\pcol@getmparbottom@last@i{#1}}%
    \pcol@do@mpb@all\pcol@mparbottom
  \fi}
\def\pcol@getmparbottom@last@i#1#2\@nil{%
  \edef\reserved@b{\@elt{\number#1}{\number#1}}%
  \def\@elt##1##2{\def\reserved@b{\@elt{##1}{##2}}}%
  #2\let\@elt\relax}

%    \end{macrocode}
% \end{macro}
% 
% \KeepSpace{1}
% \begin{macro}{\pcol@do@mpb@all}
% \changes{v1.3-4}{2013/09/17}
%	{Introduced to implement \cs{pcol@bias@mpbout} and
%	 \cs{pcol@getmparbottom@last}.}
% \begin{macro}{\pcol@do@mpb@all@i}
% \changes{v1.3-4}{2013/09/17}
%	{Introduced to implement \cs{pcol@bias@mpbout} and
%	 \cs{pcol@getmparbottom@last}.}
% \begin{macro}{\pcol@do@mpb@all@ii}
% \changes{v1.3-4}{2013/09/17}
%	{Introduced to implement \cs{pcol@bias@mpbout} and
%	 \cs{pcol@getmparbottom@last}.}
% 
% The macro $\!\pcol@do@mpb@all!\<L\>$ is used in \!\pcol@bias@mpbout! with
% $L=\mpbout$ and in \!\pcol@getmparbottom@last! with $L=\pp^m(\ptop)$, to
% process all four elements $\mpb_X^x$ in $L$ applying \!\reserved@a! to
% each of them and to let $\mpbout$ have the result through \!\reserved@b!.
% The macro simply invokes \!\pcol@do@mpb@all@i! giving it the body of $L$
% by \!\expandafter!.  Then $\!\pcol@do@mpb@all@i!
% \Arg{\mpb_L^l}\Arg{\mpb_L^r}\Arg{\mpb_R^l}\Arg{\mpb_R^r}$ initialize
% $\mpbout=\emptyset$ and then invokes \!\pcol@do@mpb@all@ii! four times
% giving it each $\mpb_X^x$.  Then
% $\!\pcol@do@mpb@all@ii!\,\mpar(t_1,b_1)\cdots\~{\mpar}(t_n,b_n)\!\@nil!$
% invokes \!\reserved@a! giving it all of $\mpar(t_i,b_1)$ at once to
% process them and to have the result in \!\reserved@b! being added to
% $\mpbout$.
% 
%    \begin{macrocode}
\def\pcol@do@mpb@all#1{\expandafter\pcol@do@mpb@all@i#1}
\def\pcol@do@mpb@all@i#1#2#3#4{\begingroup \let\@elt\relax
  \gdef\pcol@mparbottom@out{}%
  \pcol@do@mpb@all@ii#1\@nil\pcol@do@mpb@all@ii#2\@nil
  \pcol@do@mpb@all@ii#3\@nil\pcol@do@mpb@all@ii#4\@nil
  \endgroup}
\def\pcol@do@mpb@all@ii#1\@nil{%
  \reserved@a#1\@nil
  \xdef\pcol@mparbottom@out{\pcol@mparbottom@out{\reserved@b}}}

%    \end{macrocode}
% \end{macro}\end{macro}\end{macro}
% 
% 
% 
% \subsection{Synchronization}
% \label{sec:imp-sout-sync}
% 
% \begin{macro}{\pcol@sync}
% \changes{v1.0}{2011/10/10}
%	{Add measurement of $D_T$.}
% \changes{v1.2-3}{2013/05/11}
%	{Modify the action on the page overflow to return from \cs{output}
%	 without flushing so that the page is broken outside \cs{output} to
%	 place top floats above the synchronization point set in the next
%	 page.}
% \changes{v1.2-2}{2013/05/11}
%	{Revise reflecting the redesign of page context.}
% \changes{v1.2-2}{2013/05/11}
% 	{Add pre-flushing column height check taking page-wise
%	 footnotes into account.}
% \changes{v1.3-6}{2013/09/17}
%	{Add the initialization of
%	 $\cs{ifpcol@dfloats}\EQ\string\mathit{false}$ before invoking
%	 \cs{pcol@measurecolumn}.}
% 
% The macro \!\pcol@sync! is invoked solely from \!\pcol@output@switch! for
% \exsync{} in the following three cases in which
% $\CSIndex{ifpcol@sync}\lor\CSIndex{ifpcol@clear}=\true$ commonly.
% 
% \begin{itemize}
% \item
% $\CSIndex{ifpcol@sync}\land\lnot\CSIndex{ifpcol@clear}$ to mean ordinary
% \sync{}ed \cswitch.
% 
% \item
% $\CSIndex{ifpcol@sync}\land\CSIndex{ifpcol@clear}$ to mean \pfcheck.
% 
% \item
% $\lnot\CSIndex{ifpcol@sync}\land\CSIndex{ifpcol@clear}$ to mean page
% flushing.
% \end{itemize}
% 
% In any cases\footnote{
% 
% In the last case of page flushing, invoking \!\pcol@flushcolumn! is
% redundant because it is made $p=\ptop$ by \pfcheck{} always preceding the
% flushing, but the invocation is harmless.},
% 
% first we invoke \!\pcol@flushcolumn! for all $c\In0\C$ to flush the
% \ccolpage{} of $c$ into $\S_c$ if the \colpage{} is not in $\ptop$, i.e.,
% $\cc_c(\vb^p)<\ptop$ and then, if we have deferred floats, to ship out
% following \fpage{}s up to $\ptop-1$ into $\S_c$ and to place them in
% $\ptop$.  This float placement in $\ptop$ is only for top and bottom
% floats in \sync{}ed \cswitch{}, while a \fcolumn{} may be made in other
% cases.  Then we ship out all pages $p$ such that $p<\ptop$ by
% \!\pcol@outputcolumns! giving argument 1.  After that, we obtain the
% \pctext{} of $\ptop$ by \!\pcol@getcurrpinfo!.
% 
% Next, we measure the vertical sizes of the contents in the \ccolpage{} of
% $c$ which is now in $\ptop$ for all $c\In0\C$ by \!\pcol@measurecolumn! as
% follows, where $h(x)$ and $d(x)$ are the height and depth of an object $x$
% respectively.
% 
% \begin{eqnarray*}
% \sigma&=&\rlap{\!\floatsep!}\hskip8em
% \sigma_t=\cases{\cc_c(\tf)&     $\cc_c(\tf)<\infty$\cr
% 		    \!\textfloatsep!&$\cc_c(\tf)=\infty$}\qquad
% \sigma_b=\!\textfloatsep!\\
% \Uidx\fc(t)&=&\rlap{$\displaystyle(\cc_c(\tl)\neq())$}\hskip8em
%     \fc(m)=(\cc_c(\vb)\neq\!\vbox!|{}|)\\
% \fc(f)&=&\rlap{$\displaystyle(\cc_c(\ft)\neq\bot)$}\hskip8em
%     \fc(b)=\fc(b')=(\cc_c(\bl)\neq())\\
% \Uidx\Fc(X)&=&\rlap{$\displaystyle\exists x\in X:\fc(x)$}\hskip8em
%     f_b=\CSIndex{ifpcol@bfbottom}\\
% \Uidx\vc(t)&=&\!\skip!{\cdot}\cc_c(\vb)=
% 	\sum_{\Sub{\phi\in\cc_c(\tl)}}(h(\phi)+d(\phi))+
% 	(\Abs{\cc_c(\tl)}-1)\cdot\sigma+\sigma_t\\
% \vc(m)&=&h(\cc_c(\vb))+d(\cc_c(\vb))\\
% \vc(f)&=&h(\cc_c(\ft))+d(\cc_c(\ft))\\
% \vc(b)&=&\sum_{\Sub{\phi\in\cc_c(\bl)}}(h(\phi)+d(\phi))+
% 	(\Abs{\cc_c(\bl)}-1)\cdot\sigma+\sigma_b\\
% \vc(b')&=&\vc(b)+\sigma_b\\
% \pd_c&=&\cases{\cc_c(\pd)&$\fc(m)$\cr
%                \infty   &$\lnot \fc(m)$}\\
% \Uidx\size_c(x)&=&\cases{\vc(x)&$\fc(x)$\cr
%                     0     &$\lnot \fc(x)$}\\
% \Uidx\Size_c(X)&=&\cases{\displaystyle\sum_{x\in X}\size_c(x)&$\Fc(X)$\cr
%                          -\infty&                         $\lnot \Fc(X)$}\\
% \Uidx\VT&=&\!\@tempdima!=\max_{0\leq c<\C}\{\Size_c(\{t,m\})\}\\
% \Uidx\VB&=&\!\@tempdimb!=\max_{0\leq c<\C}\{\size_c(f)+\size_c(b)\}\\
% \Uidx\VP&=&\!\@pageht!=\max_{0\leq c<\C}\{\Size_c(\{t,m,f,b\})\}\\
% \Uidx\VPP&=&\!\pcol@colht!=\max_{0\leq c<\C}\{\Size_c(\{t,m,f,b'\})\}\\
% \Uidx\DT&=&\!\@tempdimc!=\min\Set{\pd_c}{\Size_c{\{t,m\}=\VT}}\\
% \Uidx\dc&=&\cases{
%   0&			$\fc(b)\land(\lnot \fc(f)\lor f_b)$\cr
%   d(\cc_c(\ft))&	$\fc(f)\land(\lnot \fc(b)\lor\lnot f_b)$\cr
%   d(\cc_c(\pd))&	$\lnot \fc(b)\land\lnot \fc(f)$}\\
% \Uidx\DP&=&\!\@pagedp!=\min\Set{\dc}{\Size_c(\{t,m,f,b'\})=\VPP}\\
% \Uidx\cmax&=&\!\@tempcntb!=\ARG\max_{0\leq c<\C}\{\Size_c(\{t,m,f,b\})\}
% \end{eqnarray*}
% 
% That is, $\VT$ is the maximum of combined vertical size (height plus
% depth) of the top floats and the main vertical list, $\VB$ is that of
% the footnotes and bottom floats, and $\VP$ is that of all items.
% $\VPP$ is similar to $\VP$ but we add \!\textfloatsep! to the size of
% bottom floats.  Note that $\VT$, $\VP$ and $\VPP$ are $-\infty$ if any
% \colpage{}s don't have corresponding items, while $\VB=0$ if so.
% Also note that $\cmax$ is the ordinal of the column whose size is $\VP$.
% 
% $\DT$ and $\DP$ are the minimum $\pd_c$ and $\dc$ respectively among those
% gives $\VT$ and $\VPP$ respectively, where $\pd_c$ is $\cc_c(\pd)$ if
% $f_m=\true$ or $\infty$ otherwise, and $\dc$ is 0 if $c$ has bottom float,
% or the depth of the last footnote if any and without any bottom float, or
% $\cc_c(\pd)$ otherwise.  The reason why $\DT$ and $\DP$ have minimums is
% that they are set into \!\prevdepth! for the items just following the
% \sync{}ation point, and thus a smaller value results in a larger interline
% skip and the special value $-1000\,|pt|$ to inhibit the skip by, e.g.,
% \!\nointerlineskip!, is given the highest priority.
% 
% Note that $\VPP$ and $\DP$ are only for the \lpage{} and thus referred to
% by \!\pcol@output@end! to close the environment, and the former is done by
% \!\pcol@makeflushedpage! if it works on the page.  The reason why we add
% \!\textfloatsep! to $\VPP$ is to make the last page well separated from
% the \postenv{} if the tallest column, taking the addition into account,
% has bottom floats.  Also note that we let
% $\CSIndex{ifpcol@dfloats}=\false$ before scanning columns with
% \!\pcol@measurecolumn! so that the switch becomes $\true$ after the scan
% iff a column has deferred floats (in the \lpage{}).
% 
%    \begin{macrocode}
%% Special Output Routines: Synchronization

\def\pcol@sync{%
  \pcol@currcol\z@ \@whilenum\pcol@currcol<\pcol@ncol\do\pcol@flushcolumn
  \pcol@outputcolumns\@ne
  \pcol@getcurrpinfo{\global\c@page}{\global\@colht}{\global\topskip}%
  \@tempdima-\maxdimen \@tempdimb-\maxdimen \pcol@colht-\maxdimen
  \@pageht-\maxdimen \@tempdimc\maxdimen \@pagedp\maxdimen \@tempcntb\z@
  \pcol@dfloatsfalse
  \pcol@currcol\z@ \@whilenum\pcol@currcol<\pcol@ncol\do\pcol@measurecolumn
%    \end{macrocode}
% 
% As described above, any items can be empty, naturally for top floats,
% footnotes and bottom floats, but also including main vertical lists if the
% \ccolpage{}s were not in $\ptop$ before the invocation of
% \!\pcol@flushcolumn!.  Moreover, all main vertical lists can be empty if
% all \lcolpage{}s just have started (by \!\newpage!, for example).  More
% weirdly, the case of all-empty main vertical lists can be accompanied with
% other non-empty items when columns have floats or footnotes which cannot
% be in $\ptop-1$ but are found their places in $\ptop$.
%
% Taking it into account that any items can be empty and the other item
% of \Scfnote{}s, we have to determine whether the following two operations
% are taken;  invocation of \!\pcol@synccolumn! for each \colpage{} to set a
% \sync{}ation point or to add an infinite stretch (and shrink) to its
% bottom;  and the examination of the height of each \colpage{}, taking the
% \sync{}ation point to be set into account, to tell the necessity of
% explicit page break with $\CSIndex{ifpcol@flush}=\true$.  For the latter
% we calculate the examination target $V=\!\@tempdimb!$ to be compared with
% $\pp^h(\ptop)$, while for the former we determine the value of a switch
% $f=\CSIndex{if@tempswa}$ so that we invoke \!\pcol@synccolumn! iff
% $f=\true$ and $V\geq0$.
% 
% For \sync{}ed \cswitch{} with $\CSIndex{ifpcol@clear}=\false$, we let
% $f=(\VT\geq0)$ to mean at least one \colpage{} has a top float or
% non-empty main vertical list, i.e., $\Fc(\{t,m\})={\true}$ for some
% $c\In0\C$.  That is if $\VT<0$, since the next items added to all
% \colpage{}s are placed at the top of the page\footnote{
% 
% In usual cases, but it can mean some of them have negative vertical
% sizes.  Even though we can detect such a very unlikely special case, it is
% very tough to define the reasonable \sync{}ation point above the top of
% $\ptop$.  Therefore, we assume the point is at the top of $\ptop$ and thus
% do nothing.},
% 
% we don't need to set \sync{}ation points in them.  As for $V$, we let
% $V=\max(\VT,0)+\max(\VB,0)+v^f$ where $v^f$ is the sum of
% height-plus-depth of $\pp^f(\ptop)$ and $\!\skip!{\cdot}\pp^f(\ptop)$ if
% $\ptop$ has \Scfnote{}s, or 0 otherwise\footnote{
% 
% In the real implementation, $V=-\infty$ if $\VT=\VB=-\infty$ and no
% \Scfnote{}s are presented, but this difference does not affect the
% decisoins because $f\land(V\geq0)=(\VT<0)\land(V\geq0)=\false$ and
% $V\leq\pp^f(\ptop)$ with either $V=0$ or $V=-\infty$.}.
% 
% Note that it can be $\VP+v^f\leq\pp^h(\ptop)<\VT+\VB+v^f=V$ to mean
% setting the \sync{}ation point at $\VT$ below $\ptop$'s top edge would
% push bottom stuff beyond its bottom edge and thus we need an explicit page
% break to place the point at the top of $\ptop+1$ (in usual cases).
% 
% For \pfcheck{} or page flushing with $\CSIndex{ifpcol@clear}=\true$ on
% the other hand, we let $f=\lnot\!\pcol@sync!$ to invoke \!\pcol@synccolumn!
% only for page flushing and thus not for \pfcheck{}.  As for $V$, we let
% $V'=\VPP$ or $V'=\VP$ according as we working on \lpage{} or not, and
% then let $V=\max(V',0)+v^f$ or $V=V'$ according as $\ptop$ has \Scfnote{}s
% or not.  That is, we have to invoke \!\pcol@synccolumn! unless $\ptop$ is
% perfectly empty.
% 
% Then if $\CSIndex{ifpcol@clear}=\false$ and
% $\max(V,\;V-D'_T+\VE)>\pp^h(\ptop)$ where $D'_T=\DT$ if $0\leq\DT<\infty$ or
% 0 otherwise, or $\CSIndex{ifpcol@clear}=\true$ and
% $V>\pp^h(\ptop)$\footnote{
% 
% The examination is redundant in page flushing with
% $\CSIndex{ifpcol@sync}=\false$ because it is assured that no overflow
% happens in any \colpage{} by \pfcheck{} and explicit page breaking, but is
% not harmful.},
% 
% we flush the page.  That is, if the condition above holds,
% we let $\CSIndex{ifpcol@flush}=\true$ and $d=\!\pcol@nextcol!={\cmax}$ to
% tell \!\pcol@switchcol! or \!\pcol@flushclear! to make an explicit page
% break in the column $\cmax$ from which we restart, and $f=\false$ to skip
% \!\pcol@synccolumn! to postpone the \exsync{}.  Note that the bias
% $\VE=\!\pcol@@ensurevspace!$ in \sync{}ed \cswitch{} is to avoid breaking a
% \colpage{} just below the \sync{}ation point due to too small space below
% the point, less than \!\baselineskip! in default but can be other
% threshold explicitly defined by \!\ensurevspace!.  That is,
% since $V-\DT+k\!\baselineskip!$ usually means the vertical position at
% which $k$-th baseline below the \sync{}ation point is placed, the flushing
% condition with $\VE=k\!\baselineskip!$ ensures that the page is flushed
% iff the space below the point cannot accommodate $k$ lines.  Also note that
% necessary flushing with $V>\pp^h(p)$ assuredly takes place even when
% $\DT$ is unusually large and/or $\VE$ is negative to make $-\DT+\VE<0$.
% 
% Finally if $V\geq0$ and $f=\true$, we invoke \!\pcol@synccolumn! for each
% column $c\In0\C$ to set a \sync{}ation point in it or to add an infinite
% stretch (and shrink) at its bottom for flushing.
% 
% \changes{v1.3-6}{2013/09/17}
%	{Modify the flushing condition of synchronized column switching from
%	 $V\GT\pi^h(p)$ to $\max(V,V-D_T+V_E)\GT\pi^h(p)$ to avoid page
%	 break just below the synchronization point as much as possible.}
% 
%    \begin{macrocode}
  \@tempswatrue \global\pcol@flushfalse
  \ifpcol@clear
    \ifpcol@lastpage \@tempdimb\pcol@colht \else \@tempdimb\@pageht \fi
    \ifpcol@sync \@tempswafalse \fi
  \else
    \ifdim\@tempdima<\z@      \@tempswafalse
    \else\ifdim\@tempdimb<\z@ \@tempdimb\@tempdima
    \else                     \advance\@tempdimb\@tempdima
    \fi\fi
  \fi
  \ifpcol@scfnote\ifvoid\pcol@footins\else
    \ifdim\@tempdimb<\z@ \@tempdimb\z@ \fi
    \advance\@tempdimb\ht\pcol@footins \advance\@tempdimb\dp\pcol@footins
    \advance\@tempdimb\skip\pcol@footins
  \fi\fi
  \dimen@\@tempdimb
  \ifpcol@clear\else \ifdim\dimen@<\z@\else
    \ifdim\@tempdimc=\maxdimen\else \ifdim\@tempdimc<\z@\else
      \advance\dimen@-\@tempdimc
    \fi\fi
    \advance\dimen@\pcol@@ensurevspace
    \ifdim\dimen@<\@tempdimb \dimen@\@tempdimb \fi
  \fi\fi
  \ifdim\dimen@>\@colht
    \global\pcol@flushtrue \@tempswafalse \pcol@nextcol\@tempcntb
  \fi
  \ifdim\@tempdimb<\z@\else \if@tempswa
    \pcol@currcol\z@ \@whilenum\pcol@currcol<\pcol@ncol\do\pcol@synccolumn
  \fi\fi}

%    \end{macrocode}
% \end{macro}
% 
% \begin{macro}{\pcol@flushcolumn}
% \changes{v1.0}{2011/10/10}
%	{Rename \cs{pcol@maxpage} as \cs{pcol@toppage}.}
% \changes{v1.0}{2011/10/10}
%	{Add \cs{vfil} at the bottom of flushed column-page.}
% \changes{v1.0}{2011/10/10}
%	{Change order of the garbage collection of \cs{pcol@currfoot} and
%	 \cs{pcol@getcurrfoot}.} 
% \changes{v1.2-2}{2013/05/11}
% 	{Revise reflecting the redesign of \cs{pcol@getcurrfoot}.}
% \changes{v1.2-2}{2013/05/11}
% 	{Add operations for page-wise footnotes.}
% \changes{v1.2-2}{2013/05/11}
% 	{Add the examination of $\kappa_c(\rho)\EQ\infty$ to cope with a
%	 rare-case interaction of pre-flushing column height check and float
%	 columns in last pages.}
% \changes{v1.2-7}{2013/05/11}
% 	{Save \cs{ifpcol@lastpage} into \cs{ifpcol@lastpagesave} and turn
% 	 \cs{ifpcol@lastpage} $\string\mathit{false}$ temporarily during
% 	 the macro works on non-top and thus non-last pages to fix the bug
%	 that \cs{@makecol} and \cs{pcol@makefcolumn} misunderstand the page
%	 they work on is last.}
% \changes{v1.2-7}{2013/05/11}
% 	{Replace \cs{@makecol} with \cs{pcol@@makecol} to cap the depth of
%	 \cs{@outputbox} by \cs{@maxdepth} even with p\string\LaTeX.}
% \changes{v1.21}{2013/06/06}
%	{Add page and column numbers to logging.}
% \changes{v1.3-6}{2013/09/17}
%	{Fix the problem that a flushed column in a non-top page causes
% 	 overfull due to its hight-plus-depth greater than $\pi^h(p)$.}
% \changes{v1.3-3}{2013/09/17}
%	{Add \cs{@colht} and \cs{relax} as the first and third argument
%	 of \cs{pcol@shrinkcolbyfn} for all of three invocations of it.}
% \changes{v1.3-6}{2013/09/17}
%	{Add \cs{@maxdepth} as the first argument of \cs{pcol@@makecol}.}
% \changes{v1.32-2}{2015/10/10}
% 	{Add \cs{pcol@Fb}/\cs{pcol@Fe} pair(s).}
% 
% The macro \!\pcol@flushcolumn! is invoked for each column $c\In0\C$ from
% \!\pcol@sync! to ship out the \ccolpage{} of $c$ into $\S_c$ if it is not
% leading one\Index{leading column-page}, i.e., $p=\cc_c(\vb^p)<\ptop$.  The
% macro also ships out \fpage{}s from $p+1$ up to $\ptop-1$ if we have
% deferred floats to fill them and, if this float flushing still leaves
% deferred floats, puts some of them to the \lcolpage{} being
% current\Index{current column-page}{} now as its top and/or bottom floats.
% 
% First we obtain the \cctext{} in $\cc_c$ by \!\pcol@getcurrcol! and
% examines if $p=\cc_c(\vb^p)<\ptop$.  If it does not hold to mean $c$ has
% \lcolpage{}, we do nothing.
% 
% Otherwise, we save \CSIndex{ifpcol@lastpage} into
% \CSIndex{ifpcol@lastpagesave} turning the former switch $\false$ because we
% are working on a non-\lcolpage{} definitely in a non-\lpage.  Then we put
% the contents of the \ccolpage{} $\cc_c(\vb^b)$ adding \!\vfil! at its tail
% into \!\box!|255| being the \TeX's standard interface to carry the main
% vertical list for \!\output!  routine.  We also move everything in
% $\cc_c(\ft)$ obtained by \!\pcol@getcurrfoot! into \!\footins! and return
% $\cc_c(\ft)$ to \!\@freelist! if $\cc_c(\ft^b)$ is not void.
% 
% Then we obtain $p$'s \pctext{} by \!\pcol@getcurrpage! and, if it has
% \Scfnote{}s in $\pp^f(p)$, we shrink \!\@colht! by the space required for
% the footnotes using \!\pcol@shrinkcolbyfn! and add the stretch\slash
% shrink compontents of $\!\skip!\cdot\pp^f(p)$ at the bottom of
% \!\box!|255| by \!\pcol@unvbox@cclv!, as we did in \!\pcol@makecol!.
% Otherwise we take a special care of the case that the height-plus-depth of
% $\cc_c(\vb^b)$ is greater than $\pp^h(p)$ due to that its height is almost
% equal to $\pi^h(p)$ and thus its depth makes the hight-plus-depth
% exceeding $\pi^h(p)$.  This excess is revealed by the \!\vfil! we just
% have added making the height-plus-depth the height of \!\box!|255|, and
% would cause overfull in \!\@makecol! and \!\pcol@@makecol! because they
% need the height, i.e., not height-plus-depth, of \!\box!|255| not
% exceeding $\pp^h(p)$.  Therefore if it happens, we we remove the \!\vfil!
% and cap the height of \!\box!|255| by \!\@maxdepth!  to pretend as if the
% box is directly passed from \TeX's page builder.
% 
% Next we examine if $\cc_c(\tr)$ was made $\infty$ by \!\pcol@makefcolumn!
% invoked from this macro itself when it processed the \colpage{} in the
% previous \pfcheck{} for environment closing, which found a page break
% should be done.  That is, $\cc_c(\tr)=\infty$ means the \ccolpage{} was
% once judged to be in the \lpage{} but \pfcheck{} forced a page break to
% make it non-last, it should have all deferred floats now listed in
% $\cc_c(\tl)$ at \Endparacol{}, but their total size is less than the
% threshold to make a usual \fcolumn{} for the \lpage.  If so, since the page
% is not last now, we put all floats in $\cc_c(\tl)$ by \!\pcol@makefcolpage!
% as the ship-out image in \!\@outputbox!, ignoring the contents added to
% \!\box!|255| in the operations above because $\cc_c(\vb^b)$ should be
% empty, and letting $\cc_c(\tr)=0$ to mean the floats have been
% processed\footnote{
% 
% $\cc_c(\tl)$ can have any values other than $\infty$ because definitely it
% will not be referred to in its inherent sense in the situation with no
% further float additions and no deferred floats.}.
% 
% Otherwise, since the \colpage{} can be put as usual, we invoke
% \!\pcol@@makecol!\footnote{
% 
% Neither \cs{pcol@makecol} because \!\box!\texttt{255} has \!\vfil! at its
% tail and the \colpage{} should be short enough, nor \cs{@makecol} because
% we need to ensure the depth of resulting \cs{@outputbox} is capped.}
% 
% giving \!\@maxdepth! to it to build the complete \colpage{} in
% \!\@outputbox! with depth capping and with the following setting\footnote{
% 
% \LaTeX's has another \!\insert! named \!\@kludgeins! for \!\enlargepage!
% but \Paracol{} does not cares about it.}.
% $$
% \begin{array}{lll}
% \!\box!|255|=\cc_c(\vb^b)&
% \!\footins!=\cc_c(\ft)&
% \!\@colht!=\pp^h(\cc_c(\vb^p))\\
% \!\@midlist!=\cc_c(\ml)&
% \!\@toplist!=\cc_c(\tl)&
% \!\@botlist!=\cc_c(\bl)
% \end{array}
% $$
% Finally, regardless of $\cc_c(\tr)=\infty$ or not, the resulting
% \!\@outputbox! becomes $\s_c(p)$ and is added to the tail of $\S_c$
% by \!\@cons!.
% 
%    \begin{macrocode}
\def\pcol@flushcolumn{%
  \pcol@getcurrcol
  \ifnum\count\@currbox<\pcol@toppage
    \ifpcol@lastpage \pcol@lastpagesavetrue \else \pcol@lastpagesavefalse \fi
    \pcol@lastpagefalse
    \pcol@page\count\@currbox
    \setbox\@cclv\vbox{\unvbox\@currbox \vfil}%
    \ifvoid\pcol@currfoot\else
      \pcol@Fb
      \@cons\@freelist\pcol@currfoot
      \pcol@Fe{flushcolumn(colfn)}%
    \fi
    \pcol@getcurrfoot\box
    \pcol@getcurrpage
    \ifvoid\pcol@footins
      \ifdim\ht\@cclv>\@colht
        \setbox\@cclv\vbox{\boxmaxdepth\@maxdepth \unvbox\@cclv \unskip}%
      \fi
    \else
      \pcol@shrinkcolbyfn\@colht\pcol@footins\relax
      \setbox\@cclv\vbox{\pcol@unvbox@cclv\pcol@footins}%
    \fi
    \pcol@Logstart{\pcol@flushcolumn(\number\c@page:\number\pcol@currcol)}
    \ifdim\@toproom=\maxdimen
      \setbox\@outputbox\pcol@makefcolpage \global\@toproom\z@
    \else
      \pcol@@makecol\@maxdepth
    \fi
    \pcol@Logend\pcol@flushcolumn
    \global\setbox\@currbox\box\@outputbox
    \expandafter\@cons\csname pcol@shipped\number\pcol@currcol\endcsname
      \@currbox
%    \end{macrocode}
% 
% Then for each $q\in[p{+}1,\ptop{-}1]$, we repeat the followings; get $q$'s
% \pctext{} by \!\pcol@getcurrpage!; shrink \!\@colht! by
% \!\pcol@shrinkcolbyfn! if $q$ has \Scfnote{}s; try to make a \fcolumn{} in
% \!\@outputbox! by \!\@makefcolumn! giving it \!\@deferlist! being
% $\cc_c(\dl)$ at initial but will be shrunk; if the \fcolumn{} is made,
% acquire an \!\insert! by \!\@next! to keep the contents of \!\@outputbox!
% and to be added to the tail of $\S_c$ by \!\@cons!.
% 
% \changes{v1.2-2}{2013/05/11}
%	{Add \cs{@colht} shrinking by page-wise footnotes.}
% \changes{v1.32-2}{2015/10/10}
% 	{Add \cs{pcol@Fb}/\cs{pcol@Fe} pair(s).}
% 
%    \begin{macrocode}
    \advance\pcol@page\@ne
    \ifx\@deferlist\@empty\else
      \@whilenum\pcol@page<\pcol@toppage\do{%
        \pcol@getcurrpage
        \ifvoid\pcol@footins\else
          \pcol@shrinkcolbyfn\@colht\pcol@footins\relax
        \fi
        \@makefcolumn\@deferlist
        \if@fcolmade
          \pcol@Fb
          \@next\@currbox\@freelist{\global\setbox\@currbox\box\@outputbox}%
            \pcol@ovf
          \pcol@Fe{flushcolumn(fcol)}%
          \expandafter\@cons
            \csname pcol@shipped\number\pcol@currcol\endcsname\@currbox
        \fi
       \advance\pcol@page\@ne}%
    \fi
%    \end{macrocode}
% 
% Next, since we reach $\ptop$ and thus restore \CSIndex{ifpcol@lastpage}
% from \CSIndex{ifpcol@lastpagesave} because the \tpage{} can be last.
% Then we acquire the \ccolpage{} of $c$ which is now in $\ptop$ and thus
% empty, by \!\@next!.  Then we let $\!\@colht!=\!\@colroom!=\pp^h(\ptop)$
% by \!\pcol@getcurrpinfo! but shrinking them by \!\pcol@shrinkcolbyfn! if
% $\ptop$ has \Scfnote{}s, and reinitialize the float placement parameters by
% \!\pcol@floatplacement!.  Then, if $\cc_c(\dl)$ still has some floats,
% we make a \fcolumn{} for some of them in the \tpage{} by
% \!\pcol@makefcolumn! if $\CSIndex{ifpcol@clear}={}\true$ meaning flushing,
% or try to move some of them to $\cc_c(\tl)=\!\@toplist!$ and/or
% $\cc_c(\bl)=\!\@botlist!$ by \!\pcol@trynextcolumn! otherwise.  Note that
% since \!\@colroom! is used in \!\pcol@makefcolumn! as a working register,
% we let $\!\@colroom!=\pp^h(\ptop)$ again after its invocation.  After that
% we save \cctext{} including those given by \!\pcol@floatplacement! and
% modified by \!\pcol@makefcolumn! or \!\pcol@trynextcolumn! into $\cc_c$ by
% \!\pcol@setcurrcolnf!  because all footnotes are shipped out, and let
% $\cc_c(\vb^p)=\ptop$.  We also let $\cc_c(\vb^r)=\!\@colroom!$ possibly
% modified by \!\pcol@trynextcolumn! but after canceling the shrinkage of
% \!\@colht! due to \Scfnote{}s, i.e.,
% $\cc_c(\vb^r)\gets\cc_c(\vb^r)+(H-H')$ where $H$ and $H'$ are \!\@colht!
% before and after the shrinkage respectively.
% 
% \changes{v1.0}{2011/10/10}
%	{Add $\cs{@colht}\EQ\pi^h(p)$.}
% \changes{v1.0}{2011/10/10}
%	{Replace \cs{pcol@trynextcolumn} with \cs{pcol@makefcolumn} for the
%	 case of $\cs{ifpcol@clear}\EQ\mathit{true}$.}
% \changes{v1.2-2}{2013/05/11}
%	{Revise reflecting the redesign of page context.}
% \changes{v1.2-2}{2013/05/11}
%	{Add \cs{@colht} shrinking by page-wise footnotes.}
% \changes{v1.2-7}{2013/05/11}
% 	{Add the restore of \cs{ifpcol@lastpage} from
%	 \cs{ifpcol@lastpagesave}.}
% \changes{v1.21}{2013/06/06}
%	{Fix the bug that $\kappa_c(\beta^p)$ is let have \cs{pcol@page},
%        which can be less than $p_t\EQ\cs{pcol@toppage}$, to cause the column
%	 $c$ is lost or moved to a wrong page.}
% \changes{v1.32-2}{2015/10/10}
% 	{Add \cs{pcol@Fb}/\cs{pcol@Fe} pair(s).}
% 
%    \begin{macrocode}
    \ifpcol@lastpagesave \pcol@lastpagetrue \fi
    \pcol@Fb
    \@next\@currbox\@freelist{\global\setbox\@currbox\vbox{}}\pcol@ovf
    \pcol@Fe{flushcolumn(col)}%
    \pcol@getcurrpinfo\@tempcnta{\global\@colht}\@tempskipa
    \@pageht\@colht
    \ifvoid\pcol@footins\else \pcol@shrinkcolbyfn\@colht\pcol@footins\relax \fi
    \global\@colroom\@colht \pcol@floatplacement
    \ifx\@deferlist\@empty\else
      \ifpcol@clear
        \pcol@makefcolumn \global\@colroom\@colht
      \else
        \pcol@trynextcolumn
    \fi\fi
    \pcol@setcurrcolnf
    \global\count\@currbox\pcol@toppage
    \advance\@pageht-\@colht \advance\@pageht\@colroom
    \global\dimen\@currbox\@pageht
  \fi %\ifnum\count\@currbox<\pcol@toppage
  \advance\pcol@currcol\@ne}

%    \end{macrocode}
% \end{macro}
% 
% \begin{macro}{\pcol@makefcolumn}
% \changes{v1.0}{2011/10/10}
%	{Introduced to take special care of the float-column in the last
%	 page. }
% \changes{v1.2-2}{2013/05/11}
%	{Encapsulate the float column in a \cs{vbox} of \cs{@colht} so that
% 	 vertical skips at the top and bottom are not lost when the column
%	 is put back to the main vertical list, and makes the assignment to
%	 \cs{@currbox} global because the box is now referred to after
%	 \cs{output} is completed.}
% \changes{v1.2-7}{2013/05/11}
% 	{Replace the sequence of operations to make a usual float column
%	 with \cs{@toplist} with the newly introduced
%	 \cs{pcol@makefcolpage}.}
% \changes{v1.33-2}{2016/11/19}
% 	{Remove a space after the \cs{vbox} to be assigned to \cs{@currbox}
%	 to obey the coding convention.}
% 
% The macro \!\pcol@makefcolumn! is invoked solely from \!\pcol@flushcolumn!
% to put deferred floats in the currently empty \colpage{} of $c$ in the
% \tpage{} $\ptop$ which is being flushed.  Since we have to take special care
% of the case of environment closing, we cannot do this operation by
% \!\@makefcolumn!, while in other cases for \!\flushpage! and \!\clearpage!
% we also have to pay a small attention.
% 
% First, we scan the copy of $\cc_c(\dl)$ applying \!\pcol@makefcolelt! to
% each element to have the floats to be put in \!\@toplist!, which is
% assuredly empty because the \ccolpage{} of $c$ has already been shipped
% out to empty it, and those still deferred in $\cc_c(\dl)$.  Prior to scan,
% we let $H_r=\pp^h(\ptop)\footnotemark+\alpha$\footnotetext{
% 
% $\pp^h(\ptop)$ referred in this macro may have been shrunk by \Scfnote{}s
% in \cs{pcol@flushcolumn}.},
% 
% as the space initially available for
% floats each of which being $\phi$ is assumed to consume
% $v(\phi)=h(\phi)+d(\phi)+\alpha$, and $H_t=\!\@colroom!=-\alpha$ as the
% initial value of the accumulated size of $v(\phi)$ for all $\phi$ to be
% put, where $\alpha=\!\floatsep!$ if $\CSIndex{ifpcol@lastpage}=\true$
% 
% \SpecialIndex{\ifpcol@lastpage}
% 
% as discussed afterward, or $\alpha=\!\@fpsep!$ otherwise.
% 
% Then if the resulting \!\@toplist! is not empty\footnote{
% 
% It can happen if the first float is larger than $\pp^h(\ptop)$.},
% 
% we examine if $\ptop$ is the \lpage{} ($|\ifpcol@|\~|lastpage|=\true$)
% 
% \SpecialIndex{\ifpcol@lastpage}
% 
% and $\cc_c(\dl)$ is empty to mean the last \colpage{} have all deferred
% floats.  This case is subtle because if we make the \colpage{} a \fcolumn{}
% it can be sparse and unnecessarily throw the \postenv{} to the next
% page.  Therefore, we intend to pack floats to the page top as top floats
% and thus have let $\alpha=\!\floatsep!$ in the building process of
% \!\@toplist! above\footnote{
% 
% We dare to do it knowing the natural component of \!\floatsep! is a little
% bit (4pt) larger than that \!\@fpsep! and the possibility of having
% fewer floats than those given by \!\@makefcolumn!.},
% 
% but it may make a too tall \colpage{} only having floats shrinking
% the \postenv{} in the page.  In addition, if other columns have
% \fpage{}s in the \lpage, packing the floats as top floats should give
% inconsistent appearance but we don't know whether the columns following
% $c$ has \fpage{}s.  Therefore, we performs this float packing if making
% the \fpage{} gives too sparse, more specifically if
% $H_t<\!\floatpagefraction!\times\pp^h(\ptop)=\!\@fpmin!$, but postpone the
% final decision until the \colpage{} is eventually shipped out by
% \!\pcol@flushcolumn! or \!\pcol@makeflushedpage!.  That is, if the all
% conditions above hold, we keep \!\@toplist! so that it is saved in
% $\cc_c(\tl)$ and let $\cc_c(\tr)=\infty$ to indicate that the column has
% pending floats.  Note that the floats are shipped out by
% \!\pcol@flushcolumn! if the \pfcheck{} currently being performed finds a 
% too tall column in $\ptop$ to force a page break making $\ptop$ non-last.
% Also note that, in any cases, letting $\cc_c(\tr)=\infty$ is safe because
% no longer we will have any float additions to the \colpage.
% 
% Otherwise, i.e., if we are working on a non-\lpage{} to be flushed or a
% \fcolumn{} is to be made for the \lpage, we put all floats in \!\@toplist!
% in a \!\vbox! of $\pp^h(\ptop)$ tall by \!\pcol@makefcolpage! and then 
% let $\cc_c(\vb^b)$ be another \!\vbox! having it.  This encapsulation of
% the \fcolumn{} is necessary because $\cc_c(\vb^b)$ can be put back to the
% main vertical list after the \pfcheck{} to remove skips above and below
% the floats, namely \!\@fptop! and \!\@fpbot!, if the contents were not
% encapsulated.
% 
%    \begin{macrocode}
\def\pcol@makefcolumn{%
  \ifpcol@lastpage \@tempdimc\floatsep \else \@tempdimc\@fpsep \fi
  \@tempdima\@colht \advance\@tempdima\@tempdimc \global\@colroom-\@tempdimc
  \begingroup
    \let\@elt\pcol@makefcolelt
    \let\reserved@b\@deferlist
    \global\let\@deferlist\@empty
    \reserved@b
  \endgroup
  \ifx\@toplist\@empty\else
    \@tempswatrue
    \ifpcol@lastpage \ifx\@deferlist\@empty \ifdim\@colroom<\@fpmin
      \@tempswafalse \global\@toproom\maxdimen
    \fi\fi\fi
    \if@tempswa \global\setbox\@currbox\vbox{\pcol@makefcolpage}\fi
  \fi}
%    \end{macrocode}
% \end{macro}
% 
% \begin{macro}{\pcol@makefcolelt}
% \changes{v1.0}{2011/10/10}
%	{Introduced to take special care of the float-column in the last
%	 page. }
% 
% The macro $\!\pcol@makefcolelt!\<\phi\>$ is invoked solely from
% \!\pcol@makefcolumn! to be applied to each float element $\phi$ in (the
% copy of) $\cc_c(\dl)$.  We examine if $v(\phi)=h(\phi)+d(\phi)+\alpha\leq
% H_r$ to mean the \fcolumn{} being built has room large enough for the float
% $\phi$.  If so, we add $\phi$ to \!\@toplist! by \!\@cons!, and let
% $H_r\gets H_r-v(\phi)$ and $H_t\gets H_t+v(\phi)$.  Otherwise, we add
% $\phi$ to $\cc_c(\dl)$ by \!\@cons! to make it deferred again, and
% let $H_r=-\infty$ so that the examinations for any succeeding elements
% fail.
% 
%    \begin{macrocode}
\def\pcol@makefcolelt#1{%
  \@tempdimb\ht#1{}\advance\@tempdimb\dp#1{}\advance\@tempdimb\@tempdimc
  \ifdim\@tempdimb>\@tempdima \@cons\@deferlist#1\relax
    \@tempdima-\maxdimen
  \else \@cons\@toplist#1\relax
    \advance\@tempdima-\@tempdimb \global\advance\@colroom\@tempdimb
  \fi}
%    \end{macrocode}
% \end{macro}
% 
% \begin{macro}{\pcol@makefcolpage}
% \changes{v1.2-7}{2013/05/11}
% 	{Introduced to implement the operations to make a float column
%	 performed in three macros.}
% \changes{v1.32-2}{2015/10/10}
% 	{Add \cs{pcol@Fb}/\cs{pcol@Fe} pair(s).}
% \changes{v1.33-2}{2016/11/19}
% 	{Add {\tt\%} to the end of the line to open \cs{vbox} to obey the
%	 coding convention.} 
% 
% The macro \!\pcol@makefcolpage! is invoked from \!\pcol@flushcolumn!,
% \!\pcol@makefcolumn! and \!\pcol@imakeflushedpage! to build a \fcolumn{}
% having floats in \!\@toplist!, which is then returned to \!\@freelist! and
% then emptied.  The floats are put in a \!\vbox! of \!\@colht! tall with
% vertical skips of \!\@fptop!, \!\@fpsep! and \!\@fpbot! above, between and
% below them respectively.  The box is then let be $\s_c(p)$ or
% $\cc_c(\vb^b)$ explictly or implicitly by the invokers, with an
% encapsulation in the case of \!\pcol@makefcolumn!.
% 
%    \begin{macrocode}
\def\pcol@makefcolpage{\vbox to\@colht{%
    \vskip\@fptop \vskip-\@fpsep
    \def\@elt##1{\vskip\@fpsep\box##1}\@toplist \vskip\@fpbot}%
  \pcol@Fb
  \xdef\@freelist{\@freelist\@toplist}\global\let\@toplist\@empty
  \pcol@Fe{makefcolpage}%
}

%    \end{macrocode}
% \end{macro}
% 
% \begin{macro}{\pcol@measurecolumn}
% \changes{v1.0}{2011/10/10}
%	{Drastically changed to measure $D_T$, to deal with empty
%	 main vertical list, and to omit \cs{topfigrule} and \cs{botfigrule}
%	 from float size measurement.} 
% 
% The macro \!\pcol@measurecolumn! is invoked for each column $c\In0\C$ from
% \!\pcol@sync! to measure the sizes of the top floats $\size_c(t)$, main
% vertical list $\size_c(m)$, footnotes $\size_c(f)$ and bottom floats
% $\size_c(b)$ in the \ccolpage{} in the page $\ptop$.  After obtaining the
% \cctext{} in $\cc_c$ by \!\pcol@getcurrcol!, we calculate
% $\size_c(t)=\!\skip!{\cdot}\cc_c(\vb)$ by \!\pcol@addflhd! giving it
% $\cc_c(\tl)$ and $\cc_c(\tf)=\!\pcol@textfloatsep!$ as its arguments also
% to have $\CSIndex{if@tempswa}={\fc(t)}$.  Note that $\cc_c(\tf)=\infty$ means
% the \colpage{} does not have any \sync{}ation points yet and thus
% \!\textfloatsep! is used in the calculation as the skip between the top
% floats and main vertical list, while the value itself possibly for a
% \mvlfloat{} discussed later is used if $\cc_c(\tf)<\infty$.  We also
% calculate $\size_c(m)=h(\cc_c(\vb))+d(\cc_c(\vb))$, and then the sum of it
% and $\size_c(t)$.
% 
%    \begin{macrocode}
\def\pcol@measurecolumn{%
  \pcol@getcurrcol
  \@tempswafalse
  \dimen@\z@ \pcol@addflhd\@toplist\pcol@textfloatsep
  \global\skip\@currbox\dimen@
  \advance\dimen@\ht\@currbox \advance\dimen@\dp\@currbox \dimen@ii\dimen@
%    \end{macrocode}
% 
% Next we examine if the main vertical list $\cc_c(\vb^b)$ is empty by
% \!\pcol@ifempty! and, if so, we let $\pd_c=\cc_c(\pd)=\infty$ and save it
% (together with others) by \!\pcol@setcurrcol! so that, if the column
% defines the $\VT$ finally by its top floats, $\DT$ is let $\infty$ and the
% fact that the column has empty list is remembered.  Otherwise, we let
% $\pd_c=\cc_c(\pd)$ and $\CSIndex{if@tempswa}=\true$ to represent
% $\Fc(\{t,m\})=f_t\lor f_m$ because $f_m=\true$.  Then we invoke
% \!\pcol@measureupdate! to let $\VT=\size_c(t)+\size_c(m)=\Size_c(\{t,m\})$
% and $\DT=\pd_c$ if $\Fc(\{t,m\})=\true$ and $\VT<\Size_c(\{t,m\})$.
% 
%    \begin{macrocode}
  \pcol@ifempty\@currbox
   {\global\pcol@prevdepth\maxdimen \pcol@setcurrcol}%
   {\@tempswatrue}%
  \pcol@measureupdate\@tempdima\dimen@ii\@tempdimc\pcol@prevdepth
%    \end{macrocode}
% 
% \changes{v1.2-2}{2013/05/11}
%	{Add calculation of $V_P$ and $c_{\string\max}$.}
% 
% Next we let $\size_c(f)=0$ if $\cc_c(\ft^b)$ is void, or otherwise let
% $\size_c(f)=h(\cc_c(\ft^b))+d(\cc_c(\ft^b))+{}\cc_c(\ft^s)$\footnote{
% 
% We ignore the height and depth of \!\footnoterule! because they are
% expected to be 0 and are so in the default setting.}
% 
% and $\CSIndex{if@tempswa}=\true$ because $\fc(f)=\true$ and thus
% $\Fc(\{t,m,f\})=\true$.  After that, we calculate $\size_c(f)+\size_c(b)$
% by \!\pcol@addflhd! giving it $\cc_c(\bl)$ and $\infty$ to mean
% \!\textfloatsep! should be used for the calculation as its argument also
% to have $\CSIndex{if@tempswa}={\Fc}(\{t,m,f,b\})$.  Then we let
% $\VB=\size_c(f)+\size_c(b)$ if $\VB<\size_c(f)+\size_c(b)$, and let
% $\VP=\size_c(t)+\size_c(m)+\size_c(f)+\size_c(b)=\Size_c(\{t,m,f,b\})$ and
% $c_{\max}=c$ if $\Fc(\{t,m,f,b\})=\true$ and $\VP<\Size_c(\{t,m,f,b\})$.
% 
%    \begin{macrocode}
  \ifvoid\pcol@currfoot \dimen@\z@
  \else
    \dimen@\ht\pcol@currfoot \advance\dimen@\dp\pcol@currfoot
    \advance\dimen@\skip\pcol@currfoot
    \@tempswatrue
  \fi
  \pcol@addflhd\@botlist\maxdimen
  \ifdim\dimen@>\@tempdimb \@tempdimb\dimen@ \fi
  \advance\dimen@\dimen@ii
  \if@tempswa \ifdim\dimen@>\@pageht
    \@pageht\dimen@ \@tempcntb\pcol@currcol
  \fi\fi
%    \end{macrocode}
% 
% Next, we let $\dc$ be the depth of the lowest non-empty items among the
% main vertical list, footnotes and bottom floats.  That is, we let
% $\dc\gets\cc_c(\pd)$ at first, and then, if
% $\CSIndex{ifpcol@bfbottom}={\true}$, override it by $\dc\gets
% d(\cc_c(\ft^b))$ if there are footnotes, and finally override it by
% $\dc\gets0$ for the bottom floats if exist adding \!\textfloatsep! to
% $\Size_c(\{t,m,f,b\})$ to have $\Size_c(\{t,m,f,b'\})$.  This overriding
% order of $d(\cc_c(\ft^b))$ and then 0 by bottom floats is reversed when
% $\CSIndex{ifpcol@bfbottom}=\false$ according to the implementation of
% \!\@makecol!.  Then, we invoke \!\pcol@measureupdate! again to let
% $\VPP=\Size_c(\{t,m,f,b'\})$ and $\DP=\dc$ if $\Fc(\{t,m,f,b\})=\true$ and
% $\VPP<\Size_c(\{t,m,f,b'\})$.  It also lets $\DP=\dc$ if
% $\Fc(\{t,m,f,b\})=\true$, $\VPP=\Size_c(\{t,m,f,b'\})$ and $\DP>\dc$.
% 
% Finally, we let $\CSIndex{ifpcol@dfloats}=\true$ if
% $\cc_c(\dl)\neq\emptyset$ to tell \!\pcol@makeflushedpage! that the
% \lpage{} must be {\em full size} and \!\pcol@output@end! to flush the
% deferred \cwise{} floats.
% 
% \changes{v1.3-6}{2013/09/17}
%	{Revise the mechanism to tell \cs{pcol@makeflushedpage} and
%	 \cs{pcol@output@end} that a column in a last page has deferred
%	 column-wise floats with newly introduced \cs{ifpcol@dfloats}.}
% 
%    \begin{macrocode}
  \dimen@ii\pcol@prevdepth
  \ifpcol@bfbottom
    \ifvoid\pcol@currfoot\else \dimen@ii\dp\pcol@currfoot \fi
    \ifx\@botlist\@empty\else \dimen@ii\z@ \advance\dimen@\textfloatsep \fi
  \else
    \ifx\@botlist\@empty\else \dimen@ii\z@ \advance\dimen@\textfloatsep \fi
    \ifvoid\pcol@currfoot\else \dimen@ii\dp\pcol@currfoot \fi
  \fi
  \pcol@measureupdate\pcol@colht\dimen@\@pagedp\dimen@ii
  \ifx\@deferlist\@empty\else \pcol@dfloatstrue \fi
  \advance\pcol@currcol\@ne}
%    \end{macrocode}
% \end{macro}
% 
% \begin{macro}{\pcol@addflhd}
% \changes{v1.0}{2011/10/10}
%	{Drastically changed to omit \cs{topfigrule} and \cs{botfigrule}
%	 from float size measurement, to take care of top float enlargement,
%	 to add the measurement of $D_T$, and to revise the definition of
%	 $D_P$.}
% 
% \begin{macro}{\pcol@hdflelt}
% \def\xl{\lambda_x}
% \changes{v1.3-3}{2013/09/17}
%	{Add a user \cs{pcol@makecol}.}
% 
% The macro $\!\pcol@addflhd!\arg{list}\arg{tfs}$ is invoked twice from
% \!\pcol@measurecolumn! for a column $c$ to measure $\size_c(x)$
% ($x\in\{t,b\}$) of top ($x=t$) or bottom ($x=b$) floats.
% The arguments and registers referred to in the macro have the folloiwngs
% according to $x=t$ or $x=b$.
% $$
% \nosv\begin{array}{l|ll}
% &		x=t&			x=b\\\hline
% \arg{list}&	\cc_c(\tl)&		\cc_c(\bl)\\
% \arg{tfs}&	\!\pcol@textfloatsep!&	\!\maxdimen!\\
% \CSIndex{if@tempswa}&
%		\false&			\Fc(\{t,m,f\})\\
% \!\dimen@!&	0&			\size(f)
% \end{array}
% $$
% The macro is also used in \!\pcol@makecol! and \!\pcol@output@switch! for
% $x=t$ but with \!\dimen@! having the height of \!\pcol@prespan! for the
% measurement of the total height of \prespan{} including top
% floats\footnote{
% 
% In these invocations, \CSIndex{if@tempswa} is meaningless and not examined
% by the invokers.}.
% 
% The macro at first examines if $\cc_c(\xl)=\arg{list}$ is empty and does
% nothing if so.  Otherwise, \CSIndex{if@tempswa} is turned $\true$ to have
% $\fc(t)=\true$ for $x=t$ or $\Fc(\{t,m,f,b\})=\true$ for $x=b$.  Then we
% scan all floats in $\arg{list}$ applying \!\pcol@hdflelt! to each float
% $\phi$ to add $h(\phi)+d(\phi)+\!\floatsep!$ to \!\dimen@!, from/to which
% we then subtract \!\floatsep! and add $\sigma_x$ because the last\slash
% first float is followed\slash preceded by the vertical skip of $\sigma_x$
% instead of \!\floatsep!, to have $\size_c(t)$ for $x=t$ or
% $\size_c(f)+\size_c(b)$ for $x=b$ being {\em returned} to
% \!\pcol@measurecolumn!.
% 
% Note that $\sigma_t$ is $\arg{tfs}=\!\pcol@textfloatsep!$ if it is less
% than $\infty$ or \!\textfloatsep! otherwise, while
% $\sigma_b=\!\textfloatsep!$ always because $\arg{tfs}=\!\maxdimen!$.  Also
% note that $\sigma_t$ can be biased by 10000\,|pt| and thus larger than
% 5000\,|pt|, if we have a \mvlfloat{} in top floats as discussed later.
% Another caution is that we ignore the contribution by \!\topfigrule! nor
% \!\botfigrule! because they should insert vertical items whose total
% height and depth are 0.
% 
%    \begin{macrocode}
\def\pcol@addflhd#1#2{%
  \ifx#1\@empty\else
    \@tempswatrue
    \let\@elt\pcol@hdflelt
    #1\advance\dimen@-\floatsep
    \ifdim#2=\maxdimen \advance\dimen@\textfloatsep
    \else
      \advance\dimen@\pcol@textfloatsep
      \ifdim\pcol@textfloatsep>5000\p@ \advance\dimen@-\@M\p@ \fi
    \fi
    \let\@elt\relax
  \fi}
\def\pcol@hdflelt#1{\advance\dimen@\ht#1\advance\dimen@\dp#1%
  \advance\dimen@\floatsep}
%    \end{macrocode}
% \end{macro}\end{macro}
% 
% \begin{macro}{\pcol@measureupdate}
% \changes{v1.0}{2011/10/10}
%	{Introduced to let $D_T$ and $D_P$ have the minimum depth of items
%	 among those which give $V_T$ and $V_P$.}
% 
% The macro $\!\pcol@measureupdate!\<V\>\<v\>\<D\>\<d\>$ is invoked
% twice in \!\pcol@measurecolumn! for $c$ to update $V\in\{\VT,\VPP\}$
% and $D=\{\DT,\DP\}$ as follows if \CSIndex{if@tempswa}, being
% $\Fc(\{t,m\})$ for $V=\VT$ or $\Fc(\{t,m,f,b\})$ for $V=\VPP$, is $\true$.
% $$
% V\gets\max(V,v)\qquad
% D\gets\cases{\min(D,d)&$V=v$\cr
%              D&        $V\neq v$}
% $$
% The arguments $v$ and $d$ have the followings according to $V$.
% $$
% \arraycolsep0pt
% \begin{array}{rll}
% V=\VT \;{:}\;&v=\Size_c(\{t,m\})&\quad d=\delta_c\\
% V=\VPP\;{:}\;&v=\Size_c(\{t,m,f,b'))&\quad d=\dc
% \end{array}
% $$
% 
%    \begin{macrocode}
\def\pcol@measureupdate#1#2#3#4{\if@tempswa
  \ifdim#1<#2\relax#1#2\relax#3#4\relax
  \else\ifdim#1=#2\ifdim#3>#4\relax#3#4\fi\fi\fi\fi}

%    \end{macrocode}
% \end{macro}
% 
% \begin{macro}{\pcol@synccolumn}
% \changes{v1.0}{2011/10/10}
%	{Drastically changed to correctly implement the top float
%	 enlargement and MVL-float.}
% \changes{v1.2-2}{2013/05/11}
%	{Change code structure removing the case for overflown synchronized
%	 pages.}
% \changes{v1.2-2}{2013/05/11}
%	{Remove \cs{penalty}\string\texttt{-10000} made unnecessary by the
%	 redesign of overflow synchronized pages.}
% \changes{v1.2-7}{2013/05/11}
%	{Add an shrink of $1/10000\,\string\texttt{fil}$ to the bottom of
%	 flushed column pages to cancel finite shrinks just below
%	 synchronization points.}
% 
% The macro \!\pcol@synccolumn! is invoked for each column $c\In0\C$ from
% \!\pcol@sync! to set a \sync{}ation point at $\VT$ from the top of the
% \ccolpage{} of $c$ if $\CSIndex{ifpcol@clear}={\false}$, or flush it
% otherwise.  After obtaining $c$'s \cctext{} $\cc_c$ by
% \!\pcol@getcurrcol!, we process one of the following three cases.
% 
% The first case is for flushing with $\CSIndex{ifpcol@clear}=\true$.  In
% this case we simply add \!\vfil! at the tail of the main vertical list in
% $\cc_c(\vb^b)$ to make the whole \colpage{} possibly with other items fit
% in a box of \!\@colht! tall and, if $\cc_c(\tf)\neq\infty$ to mean the
% column to be flushed has a \sync{}ation point, we also add an infitite
% shrink of $1/10000\,|fil|$ so as to cancel a finite shrink just below the
% point, as done in \!\pcol@makecol!\footnote{
% 
% Just in case, because it looks impossible that the natural height of the
% \colpage{} exceeds $\pp^h(\ptop)$ with \pfcheck.}.
% 
% We also let $\cc_c(\pd)=1000\,|pt|$ to mimic \TeX's mechanism of
% \!\prevdepth! with the empty main vertical list in the next
% \colpage{}\footnote{
% 
% The author is not sure if this setting is really necessary but, at least,
% it looks working well (though other setting looks all right too).}.
% 
%    \begin{macrocode}
\def\pcol@synccolumn{%
  \pcol@getcurrcol
  \ifpcol@clear
    \global\pcol@prevdepth\@m\p@
    \global\setbox\@currbox\vbox{\unvbox\@currbox
      \ifdim\pcol@textfloatsep=\maxdimen \vfil
      \else \vskip\z@\@plus1fil\@minus.0001fil
      \fi}%
%    \end{macrocode}
% 
% The second and third cases are for \sync{}ed \cswitch{}.  The second case
% is for $\DT=\infty$ to mean the \sync{}ation point is set just below the
% top floats of a column whose main vertical list is empty because it is
% definitely $\VT\geq0>-\infty$.  In this case, we should not put anything
% back to the main vertical list, because the column having defined the
% point will restart from the top of its \colpage{} with \!\topskip! and
% thus other columns should do so for the stuff following the point.
% Therefore, we put $\cc_c(\vb^b)$ as the last top float, namely
% {\em\Uidx\mvlfloat} because it is for the main vertical list, acquiring an
% \!\insert! from \!\@freelist! by \!\@next! and assigning it to
% \Midx{\!\pcol@float!} so that we pretend main vertical lists of all
% columns are empty.
% 
% The float has zero height and depth, and contains the
% followings if we have some real floats; a vertical skip of $-\!\floatsep!$
% to go back to the bottom of the last real float; \!\topfigrule! and a skip
% of \!\textfloatsep! to separate $\cc_c(\vb^b)$ from the last real float;
% and $\cc_c(\vb^b)$ followed by \!\vss! to avoid overfull and underfull.
% Otherwise, i.e., we don't have any real floats, neither of the skips nor
% \!\topfigrule! are put in the \mvlfloat{} because we let
% $\!\floatsep!=\!\textfloatsep!=0$ and $\!\topfigrule!=\!\relax!$
% temporarily in a group.  Then we set the \sync{}ation point by enlarging
% the space below the \mvlfloat{} so that the total size of all floats
% including \!\floatsep! and \!\textfloatsep!, which may be 0 as set in the
% process above, is equal to $\VT$.  This enlarging is done by letting
% $\cc_c(\tf)=\VT-(\vc(t)-\!\textfloatsep!+\!\floatsep!)$, the second term
% of which is the vertical size of the top float sequence up to the
% \mvlfloat, and by replacing \!\textfloatsep! with $\cc_c(\tf)$ in the top
% float insertion process in \!\pcol@cflt! so that the top floats including
% the \mvlfloat{} consumes $\VT$ as a whole\footnote{
% 
% This enlarging cannot be done by making the float's height
% $\VT-\vc(t)-\!\floatsep!$ (or \!\textfloatsep!) because the height can be
% negative.}.
% 
% Note that the process above involves \!\floatsep! and \!\textfloatsep!
% with some finite stretch and shrink, but these factors will not contribute
% the final result because they are canceled by \!\vss! in the \mvlfloat{}
% and by the small inifinite stretch and shrink put by \!\pcol@makecol! and
% this macro for flushing.  Also note that $\cc_c(\tf)$ is then biased by
% 10000\,|pt| so that \!\pcol@cflt!  will not put \!\topfigrule! because it
% has been already put as a part of the \mvlfloat{} or we don't have any
% real floats.  We also let $\cc_c(\pd)=1000$ to mean the \colpage{}'s main
% vertical list is empty, so as to mimic \TeX's mechanism of \!\prevdepth!
% with an empty list again.
% 
% Another attention we have to pay is that \colpage{}s with
% $\cc_c(\tf)=\infty$ does not have any \sync{}ation points, and thus
% $\cc_c(\tf)<\infty$ means a \sync{}ation has already taken place in them.
% If this $\cc_c(\tf)<\infty$ happens with $\DT=\infty$\footnote{
% 
% This can happen when a \sync{}ation with $\DT=\infty$ is immediately
% followed by another \sync{}ation or, more unlikely, by additions of
% items whose total amount is negative and then a \sync{}ation.},
% 
% we cannot update $\cc_c(\tf)$ because \!\pcol@measurecolumn! took care of
% its value on measuring $\vc(t)$.  Therefore, we do nothing if
% $\cc_c(\tf)<\infty$ but just let succeeding stuff be added to the main
% vertical list as in \cswitch{} without \sync{}ation.
% 
% \changes{v1.32-2}{2015/10/10}
% 	{Add \cs{pcol@Fb}/\cs{pcol@Fe} pair(s).}
% \changes{v1.33-2}{2016/11/19}
% 	{Add {\tt\%} to the end of the line to open \cs{vbox} for
%	 \cs{pcol@float} to obey the coding convention.} 
% 
%    \begin{macrocode}
  \else
    \@tempdimb\@tempdima
    \advance\@tempdimb-\skip\@currbox
    \ifdim\@tempdimc=\maxdimen
      \ifdim\pcol@textfloatsep=\maxdimen \begingroup
        \ifx\@toplist\@empty
          \textfloatsep\z@ \floatsep\z@ \let\topfigrule\relax
        \fi
        \pcol@Fb
        \@next\pcol@float\@freelist{\global\setbox\pcol@float\vbox to\z@{%
          \vskip-\floatsep \topfigrule \vskip\textfloatsep
          \unvbox\@currbox \vss}}\pcol@ovf
        \pcol@Fe{synccolumn(topfloat)}%
        \@cons\@toplist\pcol@float
        \advance\@tempdimb\textfloatsep \advance\@tempdimb-\floatsep
        \advance\@tempdimb\@M\p@
        \global\pcol@prevdepth\@m\p@
        \global\pcol@textfloatsep\@tempdimb
      \endgroup \fi
%    \end{macrocode}
% 
% The third and last case is for $\DT<\infty$ and thus most usual.  In this
% case, we enclose everything in $\cc_c(\vb^b)$ in a \!\vbox!  whose height
% is $h_c^v=\VT-\vc(t)$ and let $\cc_c(\vb^b)$ have it so that the item
% following the \sync{}ation point start at $\VT$.  An attention we have to
% pay is that it can be $h_c^v<\!\topskip!$ to let \TeX{} insert a vertical
% skip of the difference between them when the box is returned to the main
% vertical list pushing down the \sync{}ation point a little bit\footnote{
% 
% This can happen not very unlikely especially with $\vc(t)$ a little bit
% less than $\VT$ and $\cc_c(\vb^b)$ being empty.}.
% 
% Therefore, if so, we let $\cc_c(\vb^b)$ have the followings; a \!\vbox! of
% \!\topskip! tall having its old contents at its top above which no
% vertical skip will be inserted; a vertical skip $-\!\topskip!$ going back
% to the page top; and a vertical skip $h_c^v$ going down to the
% \sync{}ation point.
% 
% The encapsulation of the old contents $\cc_c(\vb^b)$ in the box of $h_c^v$
% tall gives us the following two features desirable for \sync{}ation.
% First, all vertical glues in the box are {\em frozen}, nullifying finite
% stretches in them because we insert an infinite stretch of
% $1/10000\,|fil|$ at the bottom of $\cc_c(\vb^b)$ to push up its old
% contents respecting other infinite stretches if any, as done by
% \!\raggedbottom!, and also nullifying finite and infinite shrinks because
% $h_c^v\geq\vc(m)$ definitely.  This freezing and nullification keeps
% \sync{}ation points already in $\cc_c(\vb^b)$ from being observed moving a
% little bit vertically.  That is, if we have a glue just below a
% \sync{}ation point and it were {\em visible} to \TeX's page builder, the
% item below the glue could move up/down when the builder found a break
% point with some shrink\slash stretch.  Though this moving up/down is
% inhibited by the small infinite stretch\slash shrink which the \colpage{}
% will at its bottom finally, it is undesirable to make \TeX{}
% misunderstanding that the glues are strethable\slash shrinkable though
% they are not in reality.
% 
% Second, since the boxes in all \colpage{} are zero deep due to the
% infinite stretch at their bottoms and these bottoms are aligned at the
% \sync{}ation point, we have a clear view of the baseline progress after
% the \sync{}ation regardless of their contents.  That is, we let
% $\cc_c(\pd)=\DT$ to {\em broadcast} $\DT$ to all columns, so that 
% the baselines of first items following the \sync{}ation
% point are aligned \!\baselineskip! below the bottom baseline of the column
% which defines $\DT$\footnote{
% 
% Since $\DT$ is given by one of the tallest columns whose depth is smallest
% among them, it is very likely that the bottom baseline of the column is
% lowest among all columns.  However, another column can have the lowest one
% when its vertical size is a little bit shorter than $\VT$ and its depth is
% small (e.g., 0).  Though of course we can define $\DT$ being $\VT$ minus
% the height of the column having the largest height to make the first
% baseline below the \sync{}ation point apart from the lowest one by
% \cs{baselineskip}, we dare to choose the definition of $\DT$ because such
% lowest baseline often means that the column have some skip at its bottom
% to give us the impressoin that the space between the baselines of the
% tallest column and its first item is a little bit too large.},
% 
% if $\DT$ plus the hight of each item is less than \!\baselineskip!.
% 
% In addition, we let $\cc_c(\tf)=\!\textfloatsep!$ to
% indicate the \colpage{} has the \sync{}ation point we just have set, if it
% was $\infty$ to mean the point is the first one.  By this setting,
% \!\pcol@makecol! and this macro itself will know that the \colpage{} needs
% to have a small infinite shrink at its bottom to cancel finite ones below
% the \sync{}ation point, while \!\pcol@cflt! acts as \LaTeX's \!\@cflt!
% because it should be $\cc_c(\tf)\leq5000\,|pt|$ to mean the \colpage{}
% does not have a \mvlfloat.
% 
% \changes{v1.33-2}{2016/11/19}
% 	{Add {\tt\%} to the end of the line to open \cs{vbox} for
%	 \cs{@currbox} and two lines for \cs{vbox}es in it to obey the
%	 coding convention.} 
% 
%    \begin{macrocode}
    \else
      \global\pcol@prevdepth\@tempdimc
      \ifdim\pcol@textfloatsep=\maxdimen
        \global\pcol@textfloatsep\textfloatsep \fi
      \global\setbox\@currbox\vbox{%
        \ifdim\@tempdimb<\topskip
          \vbox to\topskip{\unvbox\@currbox \vskip\z@\@plus.0001fil}%
          \vskip-\topskip \vskip\@tempdimb
        \else
          \vbox to\@tempdimb{\unvbox\@currbox \vskip\z@\@plus.0001fil}%
        \fi}%
    \fi
  \fi
%    \end{macrocode}
% 
% Finally, we let $\cc_c(\tn)=0$ to inhibit futher addition of top floats
% because we have fixed the space for them\footnote{
% 
% Allowing the addition is tremendously tough even when the \colpage{} has
% sufficiently large space above the \sync{}ation point.},
% 
% and save it and other \cctext{} members into $\cc_c$ by \!\pcol@setcurrcol!.
% 
%    \begin{macrocode}
  \global\@topnum\z@ \pcol@setcurrcol
  \advance\pcol@currcol\@ne}

%    \end{macrocode}
% \end{macro}
% 
% 
% 
% \subsection{Page Flushing}
% \label{sec:imp-sout-flush}
% 
% \begin{macro}{\pcol@output@flush}
% \changes{v1.0}{2011/10/10}
%	{Rename \cs{pcol@makelastpage} as \cs{pcol@makeflushedpage}.}
% \changes{v1.0}{2011/10/10}
%	{Remove unnecessary assignment of \cs{@colht}.}
% \changes{v1.2-2}{2013/05/11}
%	{Add \cs{pcol@Logstart} and \cs{pcol@Logend}.}
% \changes{v1.2-7}{2013/05/11}
%	{Add $\cs{boxmaxdepth}\EQ\cs{@maxdepth}$ for depth capping.}
% \changes{v1.2-7}{2013/05/11}
%	{Add $\cs{boxmaxdepth}\EQ\cs{@maxdepth}$ for depth capping.}
% \changes{v1.3-2}{2013/09/17}
%	{Add depth capping of \cs{pcol@rightpage}.}
% 
% The macro \!\pcol@output@flush! is invoked solely from
% \!\pcol@specialoutput! to process the \!\output! request made by
% \!\flushpage!.  We invoke \!\pcol@makeflushedpage! giving it \!\@colht! as
% the height of each \colpage{} to have the ship-out image of the
% \tpage{} including its \spanning{} and \Scfnote{}s in \!\@outputbox! whose
% height is then set to be \!\textheight!\footnote{
% 
% Just in case because the height of source \cs{@outputbox} should be exactly
% \cs{textheight} though not specified so on its construction in
% \cs{pcol@makeflushedpage}.},
% 
% ensureing that its depth is capped by $\!\boxmaxdepth!=\!\@maxdepth!$.  We
% also perform these height setting and depth capping on \!\pcol@rightpage!
% if $\CL<\C$ to mean \parapag{}ing.  Then we invoke \!\@outputpage! for
% shipping out, and then finally \!\pcol@freshpage! to have a new page to
% start new \colpage{}s in it.
% 
%    \begin{macrocode}
%% Special Output Routines: Page Flushing

\def\pcol@output@flush{%
  \pcol@makeflushedpage\@colht
  \pcol@Logstart\pcol@output@flush
  \setbox\@outputbox\vbox to\textheight{\boxmaxdepth\@maxdepth
    \unvbox\@outputbox}%
  \ifnum\pcol@ncolleft<\pcol@ncol
    \setbox\pcol@rightpage\vbox to\textheight{\boxmaxdepth\@maxdepth
      \unvbox\pcol@rightpage}%
  \fi
  \pcol@Logend\pcol@output@flush
  \@outputpage
  \pcol@freshpage}

%    \end{macrocode}
% \end{macro}
% 
% \begin{macro}{\pcol@output@clear}
% \changes{v1.0}{2011/10/10}
%	{Rename \cs{pcol@makelastpage} as \cs{pcol@makeflushedpage}.}
% \changes{v1.0}{2011/10/10}
%	{Remove unnecessary increment of \cs{pcol@page} and assignment of
%	 \cs{@colht}.}
% \changes{v1.2-2}{2013/05/11}
%	{Add \cs{pcol@Logstart} and \cs{pcol@Logend}.}
% \changes{v1.2-7}{2013/05/11}
%	{Add $\cs{boxmaxdepth}\EQ\cs{@maxdepth}$ for depth capping.}
% \changes{v1.3-2}{2013/09/17}
%	{Add depth capping of \cs{pcol@rightpage} and building an empty
%	 right parallel-page for each page-wise float page.}
% \changes{v1.3-3}{2013/09/17}
%	{Add background painting of float pages.}
% \changes{v1.32-3}{2015/10/10}
% 	{Add $\hbox{\cs{f@depth}}\EQ0$ to override
%	 $\hbox{\cs{f@depth}}\EQ\hbox{\texttt{1sp}}$ done by
%	 \cs{@dblfloatplacement}.}
% 
% The macro \!\pcol@output@clear! is invoked solely from
% \!\pcol@specialoutput! to process the \!\output! request made by
% \!\clearpage!.  The first part up to \!\@outputpage! and the last line of
% this macro are same as \!\pcol@output@flush! to flush the \tpage{} and to
% have a newpage.  In the remaining mid part, we invoke \!\pcol@flushfloats!
% to ship out all deferred \cwise{} floats in all columns if any, and
% then do it for \pwise{} floats by the following invocations enclosed in a
% group; letting $\!\pcol@rightpage!=\bot$ for ordinary paging;
% \!\@dblfloatplacement! to set up placement parameters followed by
% $\!\f@depth!=0$ to nullify the setting $\!\f@depth!=|1sp|$ possiblly done
% by it as discussed in the item-(\ref{item:ovv-float-@dblfloatplacement})
% of \secref{sec:imp-ovv-float}; \!\@makefcolumn!  with
% \!\@dbldeferlist! to have a \fpage{} in \!\@outputbox! if any; and a loop
% of \bgpaint{} of \!\@outputbox! and, if $\CL<\C$, of empty
% \!\pcol@rightpage!, and \!\@outputpage! followed by \!\@makefcolumn!
% repeated while we have a \fpage{}, i.e., $\CSIndex{if@fcolmade}=\true$.
% 
% Note that the mid part is same as that found in \!\@doclearpage! but we
% omit various adjuncts surrouding it as follows;  examination of
% \CSIndex{if@twocolumn} because we should have multiple columns;
% examination of \CSIndex{if@firstcolumn} because we have to clear the page
% immediately even when we are not in the first column;  concatenating
% \!\@dbltoplist! with \!\@dbldeferlist! and clearing it because the author
% believes \!\@dbltoplist! must be empty on the invocation of this macro;
% and letting $\!\@colht!=\!\textheight!$ because \!\pcol@flushfloats! did
% it.
% 
%    \begin{macrocode}
\def\pcol@output@clear{%
  \pcol@makeflushedpage\@colht
  \pcol@Logstart\pcol@output@clear
  \setbox\@outputbox\vbox to\textheight{\boxmaxdepth\@maxdepth
    \unvbox\@outputbox}%
  \ifnum\pcol@ncolleft<\pcol@ncol
    \setbox\pcol@rightpage\vbox to\textheight{\boxmaxdepth\@maxdepth
      \unvbox\pcol@rightpage}%
  \fi
  \pcol@Logend\pcol@output@clear
  \@outputpage
  \pcol@flushfloats
  \begingroup
    \setbox\pcol@rightpage\box\voidb@x
    \@dblfloatplacement \let\f@depth\z@
    \@makefcolumn\@dbldeferlist
    \@whilesw\if@fcolmade\fi{%
      \def\pcol@bg@floatheight{\pcol@bg@textheight}%
      \setbox\@outputbox\vbox to\textheight{%
        \pcol@bg@paintbox{Ff}\unvbox\@outputbox}%
      \ifnum\pcol@ncolleft<\pcol@ncol
        \setbox\pcol@rightpage\vbox to\textheight{\pcol@bg@paintbox{Ff}\vfil}%
      \fi
      \@outputpage
      \@makefcolumn\@dbldeferlist}%
  \endgroup
  \pcol@freshpage}

%    \end{macrocode}
% \end{macro}
% 
% \begin{macro}{\pcol@makeflushedpage}
% \changes{v1.0}{2011/10/10}
%	{Renamed from \cs{pcol@makelastpage}.}
% \changes{v1.0}{2011/10/10}
%	{Rename \cs{ifpcol@textonly} as \cs{ifpcol@nospan}.}
% \changes{v1.0}{2011/10/10}
%	{Judge the last page is empty if $V_P\EQ-\infty$ instead of $V_P<0$.}
% \changes{v1.2-2}{2013/05/11}
%	{Revise reflecting the redesign of page context.}
% \changes{v1.3-2}{2013/09/17}
%	{Completely redesigned with new macro \cs{pcol@imakeflushedpage}.}
% \changes{v1.3-3}{2013/09/17}
%	{Add background painting of page-wise floats, and a part of
%	 opreations for column-separation rule drawing and background
%	 painting of page-wise footnotes.} 
% \changes{v1.3-6}{2013/09/17}
%	{Revise the mechanism of special care about last page introducing
%	 \cs{ifpcol@dfloats}.}
% 
% The macro $\!\pcol@makeflushedpage!\arg{ht}$ is invoked from
% \!\pcol@output@flush! or \!\pcol@output@clear! with $\arg{ht}=\!\@colht!$
% and from \!\pcol@output@end! with $\arg{ht}=\!\pcol@colht!$.  At first, we
% invoke \!\pcol@output@switch! with setting $\CSIndex{ifpcol@clear}=\true$
% to flush all pages up to $\ptop-1$ and to let $\cc_c(\vb^b)$ have the
% ship-out image of the main vertical list of each \colpage{} $c$ in
% $\ptop$.  This invocatoin also lets $\!\pcol@colht!=\VPP$ so that hereafter
% we will refer $\VPP$ throuth $\arg{ht}$ if it is $\!\pcol@colht!$ for
% \lpage.  Then after obtaining $\ptop$'s \pctext{} to have
% $\page(\ptop)=\pp^p(\ptop)$, $\!\@colht!=\pp^h(\ptop)$ and
% \CSIndex{ifpcol@nospan} by \!\pcol@getcurrpinfo!, we build the ship-out
% image of $\ptop$ in \!\@outputbox!, and \!\pcol@rightpage! if
% \parapag{}ing, taking special care of the \lpage{} case as follows.
% 
% \begin{enumerate}\def\labelenumi{(\arabic{enumi})}
% \item\label{item:mfp-nonlast}
% If $\CSIndex{ifpcol@lastpage}=\false$, each of $\cc_c(\vb^b)$ has ship-out
% image even if some or all of them are empty.  It is unecessary to be aware
% of the perfectly empty case because it should mean the page $\ptop$ is
% made blank intentionally.
% 
% \item\label{item:mfp-last-dfloats}
% If $\CSIndex{ifpcol@lastpage}=\true$ but $\CSIndex{ifpcol@dfloats}=\true$
% too, the last page must have {\em full size} because we will have parallel
% columned pages having \fcolumn{}s for deferred floats.  However, if the
% page has nothing, i.e., $\pp^i(\ptop)=\pp^f(\ptop)=\bot$ and
% $\VPP=-\infty$, we must let $\!\@outputbox!=\bot$ (and
% $\!\pcol@rightpage!=\bot$ as well) to avoid an unnecessary blank page is
% shipped out.  On the other hand, if $\pp^i(\ptop)\neq\bot$ or
% $\pp^f(\ptop)\neq\bot$ while $\VPP=-\infty$, we build a full size page as
% usual but letting $\!\@textbottom!=\!\vfil!$ temporarily to avoid underfull
% in the process of building columns.  Note that if $\pp^f(\ptop)\neq\bot$,
% the \Scfnote{}s are always put into \!\@outputbox! regardless
% \CSIndex{ifpcol@mgfnote} because the \lpage{} is not combined with
% \postenv.
% 
% \item\label{item:mfp-last-empty}
% If $\CSIndex{ifpcol@lastpage}=\true$, $\CSIndex{ifpcol@dfloats}=\false$
% and $\VPP=-\infty$, we have to let $\!\@outputbox!=\bot$ unless
% $\pp^i(\ptop)\neq\bot$ or $\pp^f(\ptop)\neq\bot$ having non-\Mgfnote{}s.
% If $\pp^f(\ptop)$ has non-\Mgfnote{}s, \!\@outputbox! and
% \!\pcol@rightpage!  must have $\pp^f(\ptop)$ possibly with $\pp^i(\ptop)$
% but without any columns, and must be put into the main vertical list as
% the leading part of \postenv{} by modifying $\VPP=0$.  On the other hand
% $\pp^f(\ptop)=\bot$ or it has \Mgfnote{}s, \!\@outputbox! must have only
% $\pp^i(\ptop)$\footnote{
% 
% \!\pcol@rightpage! must have the couterpart in right \parapag{}e if the
% \spanning{} is \preenv, while it is made $\bot$ by \!\pcol@output@end! if
% the \spanning{} are \pwise{} floats.}.
% 
% Since \pwise{} floats become ordinary floats in \postenv, we cannot paint
% its \bground{} and must remove \!\dbltextfloatsep! at the bottom of
% $\pp^i(\ptop)$.
% 
% \item\label{item:mfp-last-nonempty}
% If $\CSIndex{ifpcol@lastpage}=\true$, $\CSIndex{ifpcol@dfloats}=\false$
% and $\VPP>-\infty$, \!\@outputpage! and \!\pcol@rightpage! must have short
% columns of $\VPP$ tall, together with $\pp^i(\ptop)$ as in non-\lpage{}s
% but without $\pp^f(\ptop)$ if it has \Mgfnote{}s.
% \end{enumerate}
% 
% To implement a part of special cares above, we at first let
% $\CSIndex{if@tempswa}=\true$ iff $\cs{ifpcol@}\~|lastpage|=\false$,
% 
% \CSINDEX{ifpcol@lastpage}
% 
% $\VPP>-\infty$ or $\pp^f(\ptop)\neq\bot$.
% 
%    \begin{macrocode}
\def\pcol@makeflushedpage#1{%
  \pcol@cleartrue \pcol@output@switch \pcol@clearfalse
  \pcol@getcurrpinfo{\global\c@page}{\global\@colht}\@tempskipa
  \ifpcol@lastpage \@tempswafalse \else \@tempswatrue \fi
  \ifdim\pcol@colht=-\maxdimen\else \@tempswatrue \fi
  \ifvoid\pcol@footins\else \@tempswatrue \fi
%    \end{macrocode}
% 
% Next, if $\CSIndex{ifpcol@nospan}=\true$ to mean the page $\ptop$ does not
% have \spanning{} in $\pp^i(\ptop)$, we initialize both \!\@outputbox!  and
% \!\pcol@rightpage! to be $\bot$.  Otherwise, after letting
% $\CSIndex{if@tempswa}=\true$ if $\CSIndex{ifpcol@dfloats}=\true$ to make the
% \lpage{} full size if we are working on it as discussed in\Tie
% (\ref{item:mfp-last-dfloats}), we put $\pp^i(\ptop)$ in \!\@outputbox!,
% and paint its \bground{} by \!\pcol@bg@paintbox!  \!\edef!ining the height
% paramenter \!\pcol@bg@floatheight! with $h$ being the height-plus-depth of
% $\pp^i(\ptop)$ with the following two exceptions; one is the case of
% $\CSIndex{ifpcol@firstpage}={\true}$ to mean we are in \spage{} and thus
% the \spanning{} is \preenv{} having already been painted by
% \!\pcol@output@start!; and the other is the case of
% $\CSIndex{if@tempswa}=\false$ to mean we are working on a truly \lpage{}
% being empty except for the \spanning{} itself and thus the \pwise{}
% floats become a part of deferred floats in \postenv{} as discussed in\Tie
% (\ref{item:mfp-last-empty}).  In the latter exceptional case, excluding
% the case that the \lpage{} is also the \spage\footnote{
% 
% Extremely exceptional because the closing environment does not have
% anything.},
% 
% we also remove the last skip being \!\dbltextfloatsep! so that those
% floats are naturally connected with other floats given in
% \postenv{} also as discussed in\Tie(\ref{item:mfp-last-empty}).  Then we
% {\em pack} the \!\@outputbox! in itself by \!\vbox! so that any
% stretch\slash shrink factors in it cannot affect the ship-out image
% especially when we paint its \bground{}\footnote{
% 
% Though that hardly happens.}.
% 
% Then we do the similar procedure for \!\pcol@rightpage! and make its height
% and depth equal to those of \!\@outputbox!\footnote{
% 
% If \pwise{} floats become a part of \postenv's floats, \!\pcol@rightpage!
% will be made $\bot$ by \!\pcol@output@end! afterward.}.
% 
% Finally we temporarily add $h$ to \!\topmargin! as done in
% \!\pcol@ioutputelt! so that \bgpaint{} of columns and so on with
% \bginfext{} can reach the paper top edge.
% 
% \changes{v1.3-6}{2013/09/17}
%	{Revise the condition of leaving page-wise floats as ordinary
%	 post-environment floats using \cs{if@tempswa} with
%	 \cs{ifpcol@dfloats}.}
% \changes{v1.32-2}{2015/10/10}
% 	{Add \cs{pcol@Fb}/\cs{pcol@Fe} pair(s).}
% 
%    \begin{macrocode}
  \begingroup
    \ifpcol@nospan
      \global\setbox\@outputbox\box\voidb@x
      \global\setbox\pcol@rightpage\box\voidb@x
    \else
      \ifpcol@dfloats \@tempswatrue \fi
      \let\@elt\relax
      \edef\pcol@bg@floatheight{%
        \@elt{\number\ht\pcol@spanning sp}\@elt{\number\dp\pcol@spanning sp}}%
      \def\reserved@a{%
        \ifpcol@firstpage\else \if@tempswa \pcol@bg@paintbox{Ff}\fi\fi}%
      \@tempdima\ht\pcol@spanning \advance\@tempdima\dp\pcol@spanning
      \global\setbox\@outputbox\vbox{%
        \reserved@a \unvbox\pcol@spanning
        \ifpcol@firstpage\else \if@tempswa\else \unskip \fi\fi}%
      \global\setbox\@outputbox\vbox{\box\@outputbox}%
      \pcol@Fb
      \@cons\@freelist\pcol@spanning
      \pcol@Fe{makeflushedpage(spanning)}%
      \ifnum\pcol@ncolleft<\pcol@ncol
        \global\setbox\pcol@rightpage\vbox{%
          \ifpcol@paired\else \advance\c@page\@ne \fi
          \reserved@a \unvbox\pcol@rightpage}%
        \global\ht\pcol@rightpage\ht\@outputbox
        \global\dp\pcol@rightpage\dp\@outputbox
        \global\setbox\pcol@rightpage\vbox{\box\pcol@rightpage}%
      \fi
      \advance\topmargin\@tempdima
    \fi
%    \end{macrocode}
% 
% Next, after \!\global!ly letting $\CSIndex{ifpcol@firstpage}=\false$ because
% we will ship a page which may be the \spage{} shortly, we build the
% ship-out image of columns if required fundamentally by
% $\CSIndex{if@tempswa}=\true$.  First,
% if the page $\ptop$ has \Scfnote{}s, we shrink
% $\!\@colht!={\pp^h}(\ptop)$ by \!\pcol@shrinkcolbyfn! to keep the room for
% the footnotes, to have $H=\!\@pageht!$ being the possibly shrunk
% $\pp^h(\ptop)$ for the reference
% in \!\pcol@imakeflushedpage! after the further possible modification of
% \!\@colht! we will make shortly.  Second, if
% $\CSIndex{ifpcol@lastpage}=\true$ but $\CSIndex{ifpcol@dfloats}=\true$
% too, we turn $\CSIndex{ifpcol@lastpage}=\false$ because we need a
% full-sized \lpage{}, temporarily letting $\!\@textbottom!=\!\vfil!$ if
% $\VPP=-\infty$ to avoid underfull due to perfectly empty \colpage{}s as
% discussed in\Tie(\ref{item:mfp-last-dfloats})\footnote{
% 
% Each \colpage{} $cc_c(\vb^b)$ itself exists because the empty \colpage{}
% has been visited by \cscan{} prior to \!\output! request for environment
% closing.}.
% 
% Third, if we are working on a truly \lpage{} and $\arg{ht}<H$ to mean the
% tallest column is shorter than $H$, we let $\!\@colht!=\arg{ht}$ to let
% \!\@makecol! build short \colpage{}s.  Note that it is definitely
% $\arg{ht}\leq H$ because the \pfcheck{} on the \lpage{} makes that sure.
% Fourth and finally, unless all columns in truly \lpage{} are empty as
% discussed in (\ref{item:mfp-last-empty}), we invoke
% $\!\pcol@imakeflushedpage!\<\Cfrom\>\<\Cto\>\<b\>$ once or twice, to put
% columns in right \parapag{}e to $b=\!\pcol@rightpage!$ with
% $\LBRP\Cfrom\Cto=\LBRP\CL\C$ if $\CL<\C$, and then to put left ones in
% $b=\!\@outputbox!$ with $\LBRP\Cfrom\Cto=\LBRP0\CL$ always\footnote{
% 
% The order of right to left is not essential in this macro but we follow
% the convention in \!\pcol@outputelt!.}.
% 
% \changes{v1.0}{2011/10/10}
%	{Let \cs{@colht} be $\langle\mathit{ht}\rangle$ if the former is
%	 less than the latter.}
% \changes{v1.0}{2011/10/10}
%	{Add special care of the float column in the last page.}
% \changes{v1.2-2}{2013/05/11}
%	{Add \cs{@colht} shrinking by page-wise footnotes.}
% \changes{v1.3-3}{2013/09/17}
%	{Add \cs{@colht} and \cs{relax} as the first and third argument
%	 of \cs{pcol@shrinkcolbyfn} for all of three invocations of it.}
% \changes{v1.3-6}{2013/09/17}
%	{Remove empty column scan for \cs{if@fcolmade} because it is now
%	 unnecessary thanks to \cs{ifpcol@dfloats}.}
% \changes{v1.3-6}{2013/09/17}
%	{Revise the condition of column-page building and setting of
%	 \cs{@colht}.}
% 
%    \begin{macrocode}
    \global\pcol@firstpagefalse
    \if@tempswa
      \ifvoid\pcol@footins\else
        \pcol@shrinkcolbyfn\@colht\pcol@footins\relax
      \fi
      \let\pcol@@hfil\relax \@pageht\@colht
      \ifpcol@lastpage \ifpcol@dfloats
        \ifdim\pcol@colht<\z@ \def\@textbottom{\vfil}\fi
        \pcol@lastpagefalse
      \fi\fi
      \ifpcol@lastpage \ifdim#1<\@colht \@colht#1\fi\fi
      \ifdim\@colht<\z@ \else
        \ifnum\pcol@ncolleft<\pcol@ncol
          \pcol@imakeflushedpage\pcol@ncolleft\pcol@ncol\pcol@rightpage
        \fi
        \pcol@imakeflushedpage\z@\pcol@ncolleft\@outputbox
      \fi
    \fi
%    \end{macrocode}
%
% After putting all \colpage{}s, we examine if the page $\ptop$ has
% \Scfnote{}s in $\pp^f(\ptop)$.  If so, and unless $\ptop$ is a truly
% \lpage{} and \Mgfnote{} typesetting is in effect to mean the \Scfnote{}s
% will be merged with \postenv{}, we put the footnotes in $\pp^f(\ptop)$
% below the \colpage{}s.  Prior to this however, we let
% \Midx{\!\pcol@fnheight@lpage!} have the height-plus-depth of the footnote,
% so that \!\pcol@output@end! know the size for the \bgpaint{} of the
% footnotes, which \!\pcol@imakeflushedpage!  performed for non-\lpage{}s.
% We also put an empty box of the size into \!\pcol@rightpage! by
% \!\pcol@phantom! together with the \!\skip! component of $\pp^f(\ptop)$ to
% keep the space necessary especially when $\ptop$ is the \lpage.  Then we
% put the footnotes in $\pp^f(\ptop)$ into \!\@outputpage!  by
% \!\pcol@putfootins!, reclaiming the contents of $\pp^f(\ptop)$ and letting
% $\pp^f(\ptop)=\bot$ so that \!\pcol@output@end!  will be unaware of the
% footnotes.  We also let $\VPP=\!\pcol@colht!=0$ if $\ptop$ is a truly
% \lpage{} and it had $-\infty$ to indicate that the \lpage{} is not empty
% but has footnotes as discussed in\Tie(\ref{item:mfp-last-empty}).
% 
% \changes{v1.2-2}{2013/05/11}
%	{Add incorporation of page-wise footnotes.}
% \changes{v1.3-6}{2013/09/17}
%	{page-wise footnotes for the last page followed by pages for
%	 deferred column-wise floats are now put by this macro.}
% \changes{v1.32-2}{2015/10/10}
% 	{Add \cs{pcol@Fb}/\cs{pcol@Fe} pair(s).}
% 
%    \begin{macrocode}
    \gdef\pcol@fnheight@lpage{0pt}%
    \ifvoid\pcol@footins\else
      \@tempswatrue \ifpcol@lastpage \ifpcol@mgfnote \@tempswafalse \fi\fi
      \if@tempswa
        \pcol@Log\pcol@makeflushedpage{output}\pcol@footins
        \@tempdima\ht\pcol@footins \advance\@tempdima\dp\pcol@footins
        \xdef\pcol@fnheight@lpage{\number\@tempdima sp}%
        \ifnum\pcol@ncolleft<\pcol@ncol
          \global\setbox\pcol@rightpage\vbox{\unvbox\pcol@rightpage
            \vskip\skip\pcol@footins \nointerlineskip
            \pcol@phantom\pcol@footins \vskip\z@}%
        \fi
        \global\setbox\@outputbox\vbox{%
          \unvbox\@outputbox \pcol@putfootins\pcol@footins}%
        \pcol@Fb
        \@cons\@freelist\pcol@footins \gdef\pcol@footins{\voidb@x}%
        \pcol@Fe{makeflushedpage(pagefn)}%
        \ifdim\pcol@colht=-\maxdimen \global\pcol@colht\z@ \fi
      \fi
    \fi
  \endgroup}

%    \end{macrocode}
% \end{macro}
% 
% \begin{macro}{\pcol@imakeflushedpage}
% \changes{v1.3-2}{2013/09/17}
%	{Introduced for parallel-paging.}
% \changes{v1.3-3}{2013/09/17}
%	{Implement column-separating rule and background paiting of
%	 columns, column-separating gaps, spanning texts and page-wise
%	 footnotes.}
% \changes{v1.3-4}{2013/09/17}
%	{Implement variable-width column-separating gaps.}
% 
% The macro $\!\pcol@imakeflushedpage!\<\Cfrom\>\<\Cto\>\arg{b}$ is invoked
% solely in \!\pcol@makeflushedpage! but can be twice with
% $(\Cfrom,\Cto,b)=(\CL,\C,\!\pcol@rightpage!)$ if \parapag{}ing is in effect
% and with $(\Cfrom,\Cto,b)=(0,\CL,\!\@outputbox!)$ always, to build the
% ship-out image of the right or left \parapag{}e $\ptop$ in the box $b$
% already having \spanning{} or its blank counterpart if any, respectively.
% 
% After opening the \!\vbox! of the ship-out image for $b$, at first we
% examine if $\CSIndex{ifpcol@paired}={\false}$ and $\Cfrom>0$, and if so we
% temporarily increment $\page(\ptop)$ by one so that we check its parity
% for \mirror{}ed \bgpaint{} correctly.  Then if the page $\ptop$ has
% \Scfnote{}s in $\pp^f(\ptop)$, we paint its \bground, or that of its blank
% counterpart, by \!\pcol@bg@paintbox! \!\def!ining the parameter
% \!\pcol@bg@footnoteheight!  with the height-plus-depth of $\pp^f(\ptop)$,
% as the very first element of the ship-out image as done in
% \!\pcol@ioutputelt!, unless $\ptop$ is the truely \lpage{} for which the
% \bgpaint{} is done in \!\pcol@output@end!.  Then we put \spanning{} in $b$
% itself if any.
% 
% Next, we invoke \!\pcol@buildcolseprule! for \cseprule{} drawing and
% \bgpaint{} giving it $H$ in \!\@colht! possibly shurnk from $\pp^h(\ptop)$
% by \Scfnote{}s, $\LBRP\Cfrom\Cto$ for the set of columns to be put, and
% \!\@maxdepth! for non-\lpage{}s to paint the \bground{}s of columns and
% \csepgap{}s so that those of the last segment reach the page bottom, while
% for \lpage{} we give 0 to let the bottom be the real bottom of the
% columns.  Then we put the painted \bground{}s in \!\@tempboxa!
% immediately.
% 
%    \begin{macrocode}
\def\pcol@imakeflushedpage#1#2#3{\global\setbox#3\vbox{%
  \ifpcol@paired\else\ifnum#1=\z@\else \advance\c@page\@ne \fi\fi
  \ifvoid\pcol@footins\else \ifpcol@lastpage\else
    \def\pcol@bg@footnoteheight{%
      \@elt{\ht\pcol@footins}\@elt{\dp\pcol@footins}}%
    \pcol@bg@paintbox{Nn}%
  \fi\fi
  \unvbox#3\nointerlineskip
  \ifpcol@lastpage \pcol@buildcolseprule\@colht#1#2\z@
  \else            \pcol@buildcolseprule\@colht#1#2\@maxdepth
  \fi
  \unvbox\@tempboxa
%    \end{macrocode}
% 
% Now we put columns in a \!\hbox! of $\WT=\!\textwidth!$ wide.  That is,
% for each $c$, being $c'$ or $\C-1-c'$ for the $c'$-th iteration determined
% by \!\pcol@swapcolumn! according to the effectiveness of \cswap{} and the
% parity of $\page(\ptop)$, we obtain $c$'s \cctext{} $\cc_c$ by
% \!\pcol@getcurrcol!, move $\cc_c(\vb^b)$ into \!\box!|255|, and let
% $\!\footins!=\cc_c(\ft)$ by \!\pcol@getcurrfoot! returning it to
% \!\@freelist! if $c$ has \Mcfnote{}s.
% 
% After that we examine if $\cc_c(\tr)=\infty$ to mean we are working on the
% \lpage{} and the \colpage{} is for a \fcolumn{} whose floats can be put as
% top floats, and let $\!\topfigrule!=\!\relax!$ temporarily because the
% floats are not top ones in reality, if so.  Note that the abnormal setting
% $\cc_c(\tr)=\infty$ is not recovered because it will never be referred to
% and the register \!\@toproom! it represents will be updated with correct
% value before it is referred to in \postenv.  Then we also check
% $\arg{ht}=H$ to mean the \lpage{} is full size.  If both of them hold,
% the floats in $\cc_c(\tl)$ should be (or may be) put in the \fcolumn{} as
% usual and thus we put them in \!\@outputbox! of $H$ tall by
% \!\pcol@makefcolpage!.  Otherwise we invoke \!\pcol@@makecol!\footnote{
% 
% Not \cs{pcol@makecol} because the main vertical list has \!\vfil! and, if
% it has a \sync{}ation point, a infinite shrink by \cs{pcol@synccolumn} at
% its tail already, and we should not do any special operations for
% \Scfnote{}s.  Also it is not \cs{@makecol} because we need to ensure the
% depth of resulting \cs{@outputbox} is capped.},
% 
% to have the ship-out image of the \colpage{} in \!\@outputbox!, possibily
% only for the deferred floats in $\cc_c(\tl)$ but without \!\topfigrule! in
% this case.  Note that we don't take care of the stretch\slash shrink of
% \!\skip!\!\footins!  for \Scfnote{}s because \pfcheck{} on the \colpage{}
% makes it sure that the natural height of the \colpage{} cannot be greater
% than \!\@colht!.  Also note that we give \!\@maxdepth! to
% \!\pcol@@makecol! for non-\lpage{}s for depth capping, but for the
% \lpage{} we pass 0 to the macro because $H=\!\@colht!$ should be large
% enough to accommodate everything in the column including its last box even
% if the box is unusually deep.
% 
% Then we put the \!\@outputbox! above in a \!\hbox! of \!\columnwidth! wide
% preceded by \!\pcol@@hfil! being \!\relax! for the first column, while
% it is $\!\pcol@hfil!\<c^g\>$, where $c^g=\!\pcol@colsepid!$ being $c$ or
% $c-1$ without or with \cswap{} respectively, to put a \csepgap{} possibly
% with \cseprule{} segments in \!\pcol@tempboxa! built by
% \!\pcol@buildcolseprule!.  Finally, we save \cctext{} especially those
% for float parameters into $\cc_c$ by \!\pcol@setcurrcolnf! because all
% \Mcfnote{}s have been shipped out.
% 
% \changes{v1.2-4}{2013/05/11}
%	{Add column-swapping for even pages if specified.}
% \changes{v1.2-2}{2013/05/11}
% 	{Revise reflecting the redesign of \cs{pcol@getcurrfoot}.}
% \changes{v1.2-2}{2013/05/11}
%	{Add \cs{pcol@Logstart} and \cs{pcol@Logend}.}
% \changes{v1.2-7}{2013/05/11}
% 	{Enclose the column-page building process in a group to fix the bug
%	 which lets $\cs{topfigrule}\EQ\cs{relax}$ affecting to another
%	 column.}
% \changes{v1.2-7}{2013/05/11}
% 	{Replace the sequence of operations to make a usual float column
%	 with \cs{@toplist} with the newly introduced
%	 \cs{pcol@makefcolpage}.}
% \changes{v1.2-7}{2013/05/11}
% 	{Replace \cs{@makecol} with \cs{pcol@@makecol} to cap the depth of
%	 \cs{@outputbox} by \cs{@maxdepth} even with p\string\LaTeX.}
% \changes{v1.3-6}{2013/09/17}
%	{Add \cs{@maxdepth} or 0 as the argument of \cs{pcol@@makecol} to
%	 fix the problem that the last page is too large due to
%	 \cs{@maxdepth}, by the latter.}
% \changes{v1.3-6}{2013/09/17}
%	{Remove the examination of $\kappa_c(\lambda_d)$ for
%	 \cs{if@fcolmade} because it is made unnecesary now by
%	 \cs{ifpcol@dfloats}.}
% \changes{v1.32-2}{2015/10/10}
% 	{Add \cs{pcol@Fb}/\cs{pcol@Fe} pair(s).}
% 
%    \begin{macrocode}
  \hb@xt@\textwidth{%
    \@tempcntb#1\@whilenum\@tempcntb<#2\do{%
      \pcol@swapcolumn\@tempcntb\pcol@currcol#1#2\relax
      \pcol@getcurrcol
      \setbox\@cclv\box\@currbox
      \ifvoid\pcol@currfoot\else
        \pcol@Fb
        \@cons\@freelist\pcol@currfoot
        \pcol@Fe{imakeflushedpage(colfn)}%
      \fi
      \pcol@getcurrfoot\box
      \@tempswafalse
      \begingroup
        \ifdim\@toproom=\maxdimen
          \let\topfigrule\relax \ifdim\@colht=\@pageht \@tempswatrue \fi
        \fi
        \if@tempswa
          \pcol@Logstart{\pcol@makeflushedpage(1)}%
          \setbox\@outputbox\pcol@makefcolpage
          \pcol@Logend{\pcol@makeflushedpage(1)}%
        \else
          \pcol@Logstart{\pcol@makeflushedpage(2)}%
          \ifpcol@lastpage \pcol@@makecol\z@ \else \pcol@@makecol\@maxdepth \fi
          \pcol@Logend{\pcol@makeflushedpage(2)}%
        \fi
        \pcol@@hfil \hb@xt@\columnwidth{\box\@outputbox\hss}%
      \endgroup
      \edef\pcol@@hfil{\noexpand\pcol@hfil{\pcol@colsepid}}%
      \pcol@setcurrcolnf
     \advance\@tempcntb\@ne}}}}

%    \end{macrocode}
% \end{macro}
% 
% \begin{macro}{\pcol@flushfloats}
% \changes{v1.0}{2011/10/10}
%	{Add reinitialization of \cs{@colht}.}
% \changes{v1.2-4}{2013/05/11}
%	{Add column-swapping for even pages if specified.}
% \changes{v1.3-2}{2013/09/17}
%	{Completely redesigned with new macro \cs{pcol@iflushfloats}.}
% \begin{macro}{\pcol@iflushfloats}
% \changes{v1.3-2}{2013/09/17}
%	{Introduced for parallel-paging.}
% \changes{v1.3-3}{2013/09/17}
%	{Implement column-separating rule and background paiting of
%	 columns and column-separating gaps}
% \changes{v1.3-4}{2013/09/17}
%	{Implement variable-width column-separating gaps.}
% 
% The macro \!\pcol@flushfloats! is invoked from \!\pcol@output@clear! and
% \!\pcol@output@end! to flush all deferred \cwise{} floats in each
% column if any.  After letting $\!\@colht!=\!\textheight!$ for
% \fcolumn{}s, we iterate shipping out a page having \fcolumn{}s while
% $\CSIndex{if@fcolmade}=\exists c\In0\C:(\cc_c(\dl)\neq())$.
% 
% In the loop, we initialize $\CSIndex{if@fcolmade}=\false$, and then
% invoke \!\pcol@iflushfloats! twice or once according to $\CL<\C$ or not to
% mean \parapag{}ing is in effect or not, respectively.  That is, if
% $\CL<\C$ we invoke the macro with $\LBRP\CL\C$ and \!\pcol@rightpage! for
% the right \parapag{}e, and do it with $\LBRP0\CL$ and \!\@outputbox!
% always.  Note that if $\CL=\C$, we let $\!\pcol@rightpage!=\bot$ to tell
% \!\@outputpage!, which we invoke at the end of the loop to ship out a page
% or a \parapag{}e pair, that the \parapag{}ing is not in effect.
% 
% The macro $\!\pcol@iflushfloats!\<\Cfrom\>\<\Cto\>\<b\>$ opens
% a \!\vbox! to be set into $b$.  Then if $\CSIndex{ifpcol@paired}={\false}$
% and $\Cfrom>0$ to mean we are working on a right \npaired{} \parapag{}e,
% we temporarily add \!\c@page! by one for page parity examination for
% \mirror{}ed \bgpaint.  Then, the macro \!\pcol@buildcolseprule! is invoked
% with $\!\@colht!=\!\textheight!$ and $\LBRP\Cfrom\Cto$ for \cseprule{}
% drawing in \!\pcol@tempboxa! and \bgpaint{} for columns and \csepgap{}s in
% \!\@tempboxa! put into $b$ immediately.
% 
% Then we open a \!\hbox! of \!\textwidth! wide and initialize
% $f=\CSIndex{if@tempswa}$ to be \CSIndex{if@fcolmade}.  Then for each
% $c\In\Cfrom\Cto$, being $c'$ or $\C-1-c'$ for the $c'$-th iteration
% determined by \!\pcol@swapcolumn! according to the effectiveness of
% \cswap{} and the parity of \!\c@page!, we put an inner \!\hbox! of
% $\!\columnwidth!=\w_c$ wide preceded by \!\pcol@@hfil! being \!\relax! at
% initial or $\!\pcol@hfil!\<c_g\>$ otherwise for a \csepgap{} and
% \cseprule{}, where $c_g\in\{c,c{-}1\}$ without or with \cswap{}
% respectively.  That is, at first we obtain $c$'s \cctext{} including
% $\cc_c(\dl)$ by \!\pcol@getcurrcol!  and pass $\cc_c(\dl)$ to
% \!\@makefcolumn! to produce a \fcolumn{} in \!\@outputbox!  to be put into
% the inner \!\hbox!.  Then we do $f\gets f\lor(\cc_d(\dl)\neq\emptyset)$
% with $\cc_d(\dl)$ shrunk by \!\@makefcolumn! to let $f$ have
% $\exists{}c\In0\Cto:(\cc_c(\dl)\neq\emptyset)$ at the end of the loop for
% $c$, and then save the \cctext{} into $\cc_c$ by \!\pcol@setcurrcolnf!
% because we have no footnotes in $c$.
% 
% After the end of the loop, we move $f$ to \CSIndex{if@fcolmade} for
% the termination check of the loop in \!\pcol@flushfloats!.
% 
%    \begin{macrocode}
\def\pcol@flushfloats{%
  \global\@colht\textheight
  \@whilesw\if@fcolmade\fi{%
    \global\@fcolmadefalse
    \ifnum\pcol@ncolleft<\pcol@ncol
      \pcol@iflushfloats\pcol@ncolleft\pcol@ncol\pcol@rightpage
    \else
      \setbox\pcol@rightpage\box\voidb@x
    \fi
    \pcol@iflushfloats\z@\pcol@ncolleft\@outputbox
    \@outputpage}}
\def\pcol@iflushfloats#1#2#3{\setbox#3\vbox{%
  \ifpcol@paired\else\ifnum#1=\z@\else \advance\c@page\@ne \fi\fi
  \pcol@buildcolseprule\@colht#1#2\@maxdepth \unvbox\@tempboxa
  \hb@xt@\textwidth{%
    \let\pcol@@hfil\relax
    \if@fcolmade \@tempswatrue \else \@tempswafalse \fi
    \@tempcntb#1\@whilenum\@tempcntb<#2\do{%
      \pcol@swapcolumn\@tempcntb\pcol@currcol#1#2\relax
      \pcol@getcurrcol
      \@makefcolumn\@deferlist
      \pcol@@hfil \hb@xt@\columnwidth{%
        \if@fcolmade \box\@outputbox \else \vbox to\@colht{}\fi \hss}%
      \ifx\@deferlist\@empty\else \@tempswatrue \fi
      \edef\pcol@@hfil{\noexpand\pcol@hfil{\pcol@colsepid}}%
      \pcol@setcurrcolnf
     \advance\@tempcntb\@ne}%
    \if@tempswa \global\@fcolmadetrue \else \global\@fcolmadefalse \fi}}}

%    \end{macrocode}
% \end{macro}\end{macro}
% 
% \begin{macro}{\pcol@freshpage}
% \changes{v1.0}{2011/10/10}
%	{Rename \cs{pcol@maxpage} as \cs{pcol@toppage}.}
% \changes{v1.0}{2011/10/10}
%	{Add save and restore of \cs{@currbox}.}
% \changes{v1.0}{2011/10/10}
%	{Remove unnecessary assignment of \cs{pcol@currcol}.}
% \changes{v1.2-2}{2013/05/11}
%	{Add argument 0 for the invocations of \cs{pcol@startcolumn} to
%	 inhibit inserting deferred page-wise footnotes.}
% 
% The macro \!\pcol@freshpage! is invoked from \!\pcol@output@flush! and
% \!\pcol@output@clear! to start a new page after column flushing.  At
% first, we let $p=\pbase=\ptop=0$ and $\PP=\emptyset$ because we know no
% pages are kept.  Then we invoke \!\pcol@startpage! to start a new page
% with a \!\def!inition of $\!\pcol@currpage!=|{}|$ to indicate the invoker
% is this macro (i.e., not \!\pcol@opcol!).  Then after keeping \!\@colht!
% in $h=\!\pcol@colht!$, we do the followings for each column $c\In0\C$.
% 
% First we obtain $c$'s \cctext{} in $\cc_c$ by \!\pcol@getcurrcol! but let
% $p=0$ and $\!\@colroom!=h$, which can be modified by $c'<c$, without
% referring to $\cc_c(\vb^p)$ nor $\cc_c(\vb^r)$ because they are obsolete.
% We also save \!\@currbox! to \Midx{\!\pcol@currboxsave!} because it may be
% modified by \!\pcol@opcol! if we make \fcolumn{}s afterward.  Then we invoke
% \!\pcol@getcurrpage! to have the \pctext{} of $p=0$, because it might be
% modified by a column $c'<c$ by producing \fcolumn{}s.  After that and the
% invocation of \!\pcol@floatplacement! for setting float parametners, we
% invoke \!\pcol@startcolumn! for $c$'s \colpage{} at $p=0$, and iterate
% \!\pcol@opcol! and \!\pcol@startcolumn! while a \fcolumn{} is made by the
% latter\footnote{
% 
% Each column can have deferred floats on the invocation from
% \!\pcol@output@flush!.}.
% 
% Note that we give the argument 0 to each invocation of
% \!\pcol@startcolumn! to keep it from inserting deferred \Scfnote{}s, which
% will be taken care of by \!\pcol@restartcolumn! if any.  At last in the
% loop, we restore \!\@currbox! from \!\pcol@currboxsave!, let
% $\cc_c(\vb^b)$ be an empty \!\vbox! because the main vertical list is
% empty, and save the \cctext{} into $\cc_c$ by \!\pcol@setcurrcolnf!
% because of no footnotes obviously, after saving $p$ and \!\@colroom!,
% which might be modified by the \fcolumn{} production, into $\cc_c(\vb^p)$
% and $\cc_c(\vb^r)$.
% 
% After the loop above, finally we invoke \!\pcol@restartcolumn! to return
% to the column in which \!\flushpage! or \!\clearpage! was issued.
% 
%    \begin{macrocode}
\def\pcol@freshpage{%
  \global\pcol@page\z@ \global\pcol@toppage\z@ \global\pcol@basepage\z@
  \global\let\pcol@pages\@empty \global\let\pcol@currpage\@empty
  \pcol@startpage \pcol@colht\@colht
  \pcol@currcol\z@ \@whilenum\pcol@currcol<\pcol@ncol\do{%
    \pcol@getcurrcol \pcol@page\z@ \@colroom\pcol@colht
    \let\pcol@currboxsave\@currbox
    \pcol@getcurrpage
    \pcol@floatplacement
    \pcol@startcolumn\z@
    \@whilesw\if@fcolmade\fi{\pcol@opcol \pcol@startcolumn\z@}%
    \let\@currbox\pcol@currboxsave
    \global\setbox\@currbox\vbox{}%
    \global\count\@currbox\pcol@page \global\dimen\@currbox\@colroom
    \pcol@setcurrcolnf
   \advance\pcol@currcol\@ne}%
  \pcol@restartcolumn}

%    \end{macrocode}
% \end{macro}
% 
% 
% 
% \subsection{Last Page}
% \label{sec:imp-sout-end}
% 
% \begin{macro}{\pcol@output@end}
% \changes{v1.0}{2011/10/10}
%	{Drastically changed to take special care of float columns in the
%	 last page, to deal with the empty last page with and without
%	 deferred floats, and to try to make post-environment float pages.} 
% \changes{v1.2-2}{2013/05/11}
%	{Add incorporation of page-wise footnotes in the last ordinary
%	 pages followed by float-pages.}
% \changes{v1.2-7}{2013/05/11}
%	{Add \cs{pcol@outputfalse} to solve the \cs{output} request sneaking.}
% \changes{v1.2-7}{2013/05/11}
%	{Add $\cs{boxmaxdepth}\EQ\cs{@maxdepth}$ for depth capping knowing
%	 it is redundant.}
% \changes{v1.3-2}{2013/09/17}
%	{Add parallel-paging operations.}
% \changes{v1.3-3}{2013/09/17}
%	{Add background painting of page-wise footnotes and setting
%	 of \cs{pcol@preposttop}.}
% \changes{v1.3-4}{2013/09/17}
%	{Add opeartios to pass $\string\cal{M}$ to the next
%	 \string\texttt{paracol} environment and \cs{@mparbottom} to post
%	 environment typesetting.}
% 
% The macro \!\pcol@output@end! is invoked solely from
% \!\pcol@specialoutput! to process the \!\output! request made by
% \!\endparacol!.  We invoke \!\pcol@makeflushedpage! for the \lpage{}
% production as the setting $\CSIndex{ifpcol@lastpage}=\true$ done by
% \!\endparacol! indicates, giving it $h=\!\pcol@colht!$ in which
% \!\pcol@sync! sets the height of the tallest \colpage{} to have the
% ship-out image of the \tpage{} in \!\@outputbox!.
% 
% Next, we define $\mpbout=\!\pcol@mparbottom@out!$ as follows.  First, we
% invoke \!\pcol@getmparbottom@last! giving it $y=\VPP-\!\ht!\!\@outputbox!$
% being the negative counterpart of
% the height of \spanning{} in the \lpage, to let $\mpbout$ have the occupancy
% information of the bottom marginal note in each margin if any, or
% $\mpar(y,y)$ otherwise.  Then we transform $y$-coordinates in $\mpbout$
% from those for columns to those for text area by
% $\!\pcol@bias@mpbout!\Arg{-y}$ to have the final result.  Then to pass
% \!\@mparbottom! for post-environment typesetting,
% 
% \Index{post-environment stuff}
% 
% we invoke \!\pcol@do@mpbout! \!\def!ining \!\pcol@do@mpbout@whole! to do
% nothing and \!\pcol@do@mpbout@elem! to let $\!\@mparbottom!=b$ where
% $\mpb_L^x=(\mpar(t,b))$ and $x\in\{l,r\}$ according to the margin which
% post-environment marginal notes go to.
% 
% \Index{post-environment stuff}
% 
%    \begin{macrocode}
%% Special Output Routines: Last Page

\def\pcol@output@end{%
  \pcol@Logstart\pcol@output@end
  \pcol@makeflushedpage\pcol@colht
  \@tempdima\pcol@colht \ifdim\pcol@colht<\z@ \@tempdima\z@ \fi
  \advance\@tempdima-\ht\@outputbox
  \pcol@getmparbottom@last\@tempdima
  \pcol@bias@mpbout{-\@tempdima}
  \def\pcol@do@mpbout@whole##1##2##3##4{\setbox\@tempboxa\hbox{##1##2##3##4}}%
  \def\pcol@do@mpbout@elem\@elt##1##2{\global\@mparbottom##2sp}%
  \pcol@do@mpbout
%    \end{macrocode}
% 
% Next we process one of the following cases.
% 
% The first case is for $\CSIndex{ifpcol@dfloats}=\true$ to mean the
% \lpage{} is followed by one or more pages having deferred \cwise{}
% floats and thus \!\pcol@makeflushedpage! builds the ship-out image of the
% \lpage{} in {\em full size} in \!\@outputbox! unless the page has nothing
% perfectly.  Therefore, we ship the image out by \!\@outputbox! unless it
% is $\bot$ for perfectly empty case.  Then we invoke \!\pcol@flushfloats!
% to produce and ship out \fpage{}s, lettinng $\CSIndex{if@fcolmade}={\true}$
% to tell the macro that at least we will have one \fpage{}.  Now we have
% shipped out everyting in the closing environment and thus we let
% $\CSIndex{ifpcol@output}=\false$ to tell \!\output! routine to work
% ordinarily.  Then we let $\CSIndex{if@tempswa}=f_{\it sp}=\true$ to
% remember we will start a new page and thus $\!\@pagedp!=\pd=\DP=1000$ to
% mimic \TeX's \!\prevdepth! mechanism.  Finally we let
% $\CSIndex{@mparbottom}=0$ and $\mpbout=\mpboutz$ because no marginal notes
% are carried over to post-environment typesetting.
% 
% \Index{post-environment stuff}
% 
% \changes{v1.3-6}{2013/09/17}
%	{Simplify the case with deferred floats thanks to
%	 \cs{ifpcol@dfloats} and redesign of \cs{pcol@makeflushedpage}.}
% 
%    \begin{macrocode}
  \@tempswafalse
  \ifpcol@dfloats
    \ifvoid\@outputbox\else \@outputpage \fi
    \global\@fcolmadetrue \pcol@flushfloats
    \global\pcol@outputfalse
    \@tempswatrue \@pagedp\@m\p@ \global\@mparbottom\z@
    \global\let\pcol@mparbottom@out\pcol@mparbottom@zero
%    \end{macrocode}
% 
% Before proceeding to the second and third cases, we let
% $\CSIndex{ifpcol@output}=\false$ because we have nothing to ship out.
% 
% Then the second case is for $h=-\infty$ without deferred \cwise{}
% floats to mean all columns in the \lpage{} are empty and the page does not
% have non-merged \Scfnote{}s.  In this case, we examine if
% $\Midx{\!\pcol@firstprevdepth!}=\!\relax!$ to mean we have had at least
% one new page in \env{paracol} environment, i.e., \!\pcol@startpage! have
% been invoked at least once.  If so, we let $f_{\it sp}=\true$, $\pd=1000$,
% $\!\@mparbottom!=0$ and $\mpbout=\mpboutz$ again and put nothing to the
% main vertical list so that the \postenv{} starts from the top of the page.
% However, we have to take care of the case that $f_{\it ns}=\false$ and
% thus \!\@outputbox!  has \spanning.  If so, we acquire an \!\insert! from
% \!\@freelist! by \!\@next! to let it have the \spanning, i.e., the
% contents of \!\@outputbox!\footnote{
% 
% It does not have \!\dbltextfloatsep! at its tail because the skip has been
% removed by \!\pcol@makeflushedpage!.}.
% 
% Then the \!\insert! is added to the head of \!\@dbldeferlist! with the
% float placement code 10 to force \LaTeX's float placement mechanism to put
% it to the page to be started shortly.
% 
% On the other hand, $\!\pcol@firstprevdepth!\neq\!\relax!$ means that it
% has $\!\prevdepth!=\pd'$ just before \beginparacol{} in decimal integer
% representation.  Since we have not started any pages in the environment,
% and all columns in the \lpage{} is empty, we have almost nothing in the
% environment.  Note that the environment can have \pwise{} floats but they
% have not yet put into any pages but are kept in \!\@dbldeferlist!, or
% \Mgfnote{}s but they are merged to those in \postenv.  Therefore, the
% \preenv{} and \postenv{} must be {\em connected} naturally and thus we put
% the \preenv{} kept in \!\@outputbox! to the main vertical list by
% \!\unvbox!, letting $\pd=\pd'$ and keeping $f_{\it sp}=\false$\footnote{
% 
% The author of course know this situation is very unlikely but he is
% monomaniac.}.
% 
% In this case, the setting of \!\@mparbottom! and $\mpbout$ done at the
% beginning of this macro is correct because they describe the marginal
% notes in \preenv{} including \env{paracol} enviroments preceeding it even
% if any.
% 
% \changes{v1.3-3}{2013/09/17}
%	{Add settings for background painting of post-environment stuff.}
% \changes{v1.3-4}{2013/09/17}
%	{Add letting $\cs{@mparbottom}\EQ0$ and
%	 ${\string\cal{M}}\EQ{\string\cal{M}}_0$ for the simple empty page
%	 case.}
% \changes{v1.3-6}{2013/09/17}
%	{Remove \cs{unskip} from the operation to let page-wise floats be 
%	 ordinary ones because \cs{pcol@makeflushedpage} does it.}
% \changes{v1.32-2}{2015/10/10}
% 	{Add \cs{pcol@Fb}/\cs{pcol@Fe} pair(s).}
% 
%    \begin{macrocode}
  \else
    \global\pcol@outputfalse
    \ifdim\pcol@colht=-\maxdimen
      \ifx\pcol@firstprevdepth\relax
        \@tempswatrue \@pagedp\@m\p@ \global\@mparbottom\z@
        \global\let\pcol@mparbottom@out\pcol@mparbottom@zero
        \ifpcol@nospan\else
          \pcol@Fb
          \@next\@currbox\@freelist{\global\setbox\@currbox\box\@outputbox}%
            \pcol@ovf
          \pcol@Fe{output@end(spanning)}%
          \count\@currbox10\relax
          {\let\@elt\relax \xdef\@dbldeferlist{\@elt\@currbox\@dbldeferlist}}%
          \global\setbox\pcol@rightpage\box\voidb@x
        \fi
      \else \unvbox\@outputbox \@pagedp\pcol@firstprevdepth sp\relax
      \fi
%    \end{macrocode}
% 
% The last case without deferred floats and with some non-empty columns or
% non-merged \Scfnote{}s is
% most usual.  In this case, we may simply put \!\@outputbox! letting
% $\!\topskip!=0$ because \!\topskip! has already been inserted in
% \colpage{}s or \preenv{} in the box\footnote{
% 
% If the \lpage{} has non-merged \Scfnote{}s without any other items,
% \cs{topskip} has not been inserted, but this inconsistency without
% \cs{topskip} is acceptable.}.
% 
% However before putting the box back to the main vertical list, we have to
% take care of the \bgpaint{} as follows.  First we let
% $\CSIndex{ifpcol@havelastpage}=\true$ to let \!\@outputpage! paint the
% \bground{} of the \postenv{} when the page having the \lpage{} completes.
% Second, we let \!\pcol@bg@preposttop@left! have the height-plus-depth of
% the \!\@outputbox! having the short \lpage{} because the \bground{} of
% \postenv{}, or of \preenv{} if we have another \env{paracol} environment
% in the same page, to be painted is just below the box.  We also
% \!\pcol@bg@preposttop@right! have the same value but only if $\CL<\C$,
% because otherwise we have to keep this macro unchanged so that the
% non-existent right \parapag{}e of the closing environment can be the
% \postenv{} of a preceding environment and/or the \preenv{} of a succeeding
% one with \parapag{}ing.  Note that in the aforementioned {\em fresh page}
% cases and the perfectly empty case, we may be unaware of these macros
% because it should have been made 0 by the last invocation of
% \!\@outputpage! in the fresh page case or the \preenv{} and \postenv{} are
% contiguous in the empty case.
% 
% Third and finally, we have to paint the \bground{} of non-merged
% \Scfnote{}s because the painting is left by \!\pcol@makeflushedpage! for
% this macro.  Therefore, if $\!\pcol@fnheight@lpage!>0$ to mean we have
% footnotes whose total height-plus-depth is in the macro, we paint their
% \bground{} by \!\pcol@bg@paintbox! \!\def!ining \!\pcol@bg@footnoteheight!
% with the size and temporarily re\!\def!ining \!\pcol@bg@textheight! to be
% the height-plus-depth of \!\@outputbox!  because the foontnotes are at the
% bottom of the box instead of the page.  Note that the order of painting is
% right first and then left second if we have \parapag{}es because we refer
% the height-plus-depth of \!\@outputbox! being put into the main vertical
% list making the box $\bot$.  Also note that if the right \parapag{}e is
% \npaired, we temporarily increment \!\c@page! in \!\pcol@rightpage! to let
% \!\pcol@bg@paintbox! handle \bginfext{} to side margins correctly.
% Another remark is that we don't modify $\pd=\!\@pagedp!$ and thus it keeps
% $\DP$ in this case, and $f_{sp}$ is kept $\false$.
% 
% \changes{v1.3-3}{2013/09/17}
%	{Add background painting of page-wise footnotes.}
% 
%    \begin{macrocode}
    \else
      \global\pcol@havelastpagetrue
      \@tempdima\ht\@outputbox \advance\@tempdima\dp\@outputbox
      \xdef\pcol@bg@preposttop@left{\number\@tempdima sp}%
      \ifnum\pcol@ncolleft<\pcol@ncol
        \global\let\pcol@bg@preposttop@right\pcol@bg@preposttop@left
      \fi
      \def\pcol@bg@textheight{\@elt{\ht\@outputbox}\@elt{\dp\@outputbox}}%
      \def\reserved@a{%
        \ifdim\pcol@fnheight@lpage>\z@
          \def\pcol@bg@footnoteheight{\@elt\pcol@fnheight@lpage}%
          \pcol@bg@paintbox{Nn}%
        \fi}%
      \ifnum\pcol@ncolleft<\pcol@ncol
        \global\setbox\pcol@rightpage\vbox{%
          \ifpcol@paired\else \advance\c@page\@ne \fi
          \reserved@a \unvbox\pcol@rightpage}%
      \fi
      \topskip\z@ \vbox{\reserved@a \unvbox\@outputbox}%
    \fi
  \fi
%    \end{macrocode}
% 
% Now we have put almost everything in the \lpage{} but we may still have
% \Scfnote{}s in $\pp^f(\ptop)$ to be merged with those in \postenv.
% Therefore, we \!\insert! them through \!\footins! as a part of \postenv,
% and then do that for deferred footnotes in $\df=\!\pcol@topfnotes!$
% without using \!\pcol@deferredfootins! because we don't need the height
% capping.
% 
% \changes{v1.2-2}{2013/05/11}
%	{Add insertion of page-wise footnotes to be merged.}
% \changes{v1.32-2}{2015/10/10}
% 	{Add \cs{pcol@Fb}/\cs{pcol@Fe} pair(s).}
% 
%    \begin{macrocode}
  \ifvoid\pcol@footins\else
    \pcol@Log\pcol@output@end{insert}\pcol@footins
    \pcol@Fb
    \insert\footins{\unvbox\pcol@footins}\@cons\@freelist\pcol@footins
    \pcol@Fe{output@end(pagefn)}%
  \fi
  \ifvoid\pcol@topfnotes\else \insert\footins{\unvbox\pcol@topfnotes}\fi
%    \end{macrocode}
% 
% The following operationss are for clean-up and set-up for the \postenv;
% for all c, return $\cc_c(\vb)$ obtained by \!\pcol@getcurrcol! and
% $\Celt^c\neq\bot$ letting it $\bot$ to \!\@freelist!; reestablish the
% \colorstack{} by \!\pcol@restorecolorstack! for column-0\footnote{
% 
% It can be any other column.}
% 
% so that the \colorstack{} is just $\cst$ and is rewinded at \Endparacol,
% and let $\cst=\bot$; reload $\cc_d$ for $d=\!\pcol@lastcol!$ being the
% column in which \Endparacol{} occurs to let $\!\everypar!=\cc_d(\ep)$ and
% to let \CSIndex{if@nobreak} and \CSIndex{if@afterindent} have the value
% represented by $\cc_d(\sw)$, so that the first paragraph of the \postenv{}
% is typeset following them\footnote{
% 
% For rare cases that the last item of the closing environment is a
% sectioning command, but a user has such very unusual usage.};
% 
% let $\!\pcol@prevdepth!=\pd$ so that it is set to \!\prevdepth! by
% \!\pcol@invokeoutput!; let $\!\@colht!=\!\@colroom!=\!\textheight!$ to
% mean the single-column page does not have any floats so far because those
% produced in or before the environment have already been shipped out, are
% put to the main vertical list packed in \!\@outputbox!, or are in
% \!\@dbldeferlist!.
% 
% As for deferred \pwise{} floats produced in the environment, including
% those once put in the \lpage{} but returned to the list by the operation
% described above, we move them to \!\@deferlist! because they are now
% \cwise{} floats.  Then we invoke \!\pcol@floatplacement! to
% reinitialize float placement parameters.  Finally, if $f_{\it sp}=\true$,
% we invoke \!\@startcolumn! and then repeat invocations of \!\@opcol! and
% \!\@startcolumn! while \fpage{}s are produced, after letting
% $\CSIndex{ifpcol@lastpage}=\false$ to make \!\@combinefloats! work as
% \LaTeX's original\footnote{
% 
% Letting $\!\pcol@textfloatsep!=\infty$ is done by \!\pcol@floatplacement!.}.
% 
% \changes{v1.2-1}{2013/05/11}
% 	{Add color resetting.}
% \changes{v1.31}{2013/10/10}
%	{Add \cs{pcol@getcurrcol} for the column specified by
%	 \cs{pcol@lastcol} to pass \cs{if@nobreak}, \cs{if@afterindent} and
%	 \cs{everypar} of the column to post-environment stuff.}
% \changes{v1.32-2}{2015/10/10}
% 	{Add \cs{pcol@Fb}/\cs{pcol@Fe} pair(s).}
% \changes{v1.34}{2018/05/07}
%	{Remove nullification of $\gamma_0^0$ because it is not meaningless
%	 now, add release of $\gamma_0^c\NEQ\bot$ and then nullification of
%	 it for all $c$, move color stack reestablishment down to the loop
%	 with $c$ to ensure $\gamma_0^0\EQ\bot$, and add nullification of
%	 $\mathit{\Gamma}$.}
% 
%    \begin{macrocode}
  \pcol@currcol\z@ \@whilenum\pcol@currcol<\pcol@ncol\do{%
    \pcol@Fb
    \pcol@getcurrcol \@cons\@freelist\@currbox
    \ifvoid\pcol@ccuse{@box}\else
      \@cons\@freelist{\pcol@ccuse{@box}}%
      \pcol@ccxdef{\voidb@x}%
    \fi
    \pcol@Fe{output@end(col)}%
   \advance\pcol@currcol\@ne}%
  \pcol@currcol\z@ \pcol@restorecolorstack
  \global\setbox\pcol@colorins\box\voidb@x
  \pcol@currcol\pcol@lastcol\relax \pcol@getcurrcol
  \global\pcol@prevdepth\@pagedp
  \global\@colht\textheight
  \global\@colroom\textheight
  \global\let\@deferlist\@dbldeferlist \gdef\@dbldeferlist{}%
  \pcol@floatplacement
  \pcol@lastpagefalse
  \if@tempswa
    \@startcolumn \@whilesw\if@fcolmade\fi{\@opcol\@startcolumn}%
  \fi
  \pcol@Logend\pcol@output@end
}

%    \end{macrocode}
% \end{macro}
% 
% 
% 
% \section{Starting Environment}
% \label{sec:imp-startenv}
% 
% \begin{macro}{\pcol@invokeoutput}
% \changes{v1.0}{2011/10/10}
%	{Let \cs{pcol@prevdepth} have \cs{prevdepth} directly instead of via
%	 \cs{tempdima}.}
% \changes{v1.2-2}{2013/05/11}
%	{Add logging.}
% \changes{v1.2-5}{2013/05/11}
%	{Move the setting of \cs{linewidth} and \cs{hsize} from
%	 \cs{pcol@getcurrcol} to this macro and add \cs{parshape}.}
% \changes{v1.21}{2013/06/06}
%	{Add zero-clearing of \cs{deadcycles}.}
% 
% Before giving the definition of \env{paracol} environment and commands
% used in it, we define the macro $\!\pcol@invokeoutput!\arg{pen}$ invoked
% from them to make an \!\output!-request with \!\penalty! of
% $\arg{pen}=|\pcol@op@|{\cdot}f$.  The macro's structure is similar to that
% for the request sequence in \!\end@float! as follows; insert a \!\penalty!
% of $-10004$ to save the main vertical list in \!\@holdpg!; save
% \!\prevdepth!; insert a \!\vbox! to make it sure the following \!\penalty!
% of $\arg{pen}$ is kept even when we are at the top of a page; and finally
% restore \!\prevdepth!.
% 
% A difference is that we zero-clear \!\deadcycles! because it can be reach
% $\!\maxdeadcycles!=100$ easily if a page has many \sync{}ations and many
% columns.  Another and more important difference is in the save\slash
% restore of \!\prevdepth!.  First, the value of the register is saved in
% our own \!\pcol@prevdepth! instead of \!\@tempdima! by a \!\global!
% assignment so that \!\output!-routine refers to it\footnote{
% 
% The assignment is not necessary to be done \!\global!ly but we dare to do
% it to make all assignments to \!\pcol@prevdepth! being \!\global!
% consistently.}.
% 
% Second, the value restored to \!\prevdepth! may be different from that we
% just have saved because \!\output!-routine may update \!\pcol@prevdepth!
% to have, for example, the value saved in $\cc_c(\pd)$ when we left from
% $c$ which we are now restarting.
% 
% In addition to above, after the invocatoin of \!\output!-routine, we let
% $\!\linewidth!=\w_c-\lrm$ so that the register is shrunk from $\w_c$ by the
% total width of left and right margins of the \env{list}-like environment
% surrounding \env{paracol} if $\lrm=\!\pcol@lrmargin!>0$ to mean that.
% Then if so, we set \!\parshape! to let every line of paragraphs in the
% column $c$ is indented by \!\@totalleftmargin! and has width of
% \!\linewidth!, as \LaTeX's \!\list!  does.  We also let $\!\hsize!=\w_c$
% because it should have the width of the column even in a \env{list}-like
% environment.
% 
% The macro is invoked from \!\pcol@zparacol! ($f=|start|$),
% \!\pcol@switchcol! for \!\switch~column! and \csenv{}s ($f=|switch|$),
% \!\pcol@visitallcols! for \cscan{} prior to \sync{}ed \cswitch{} and page
% flushing ($f=|switch|$), \!\pcol@flushclear! for \pfcheck{}
% ($f=|switch|$), \!\pcol@com@flushpage!  for \!\flushpage! ($f=|flush|$),
% \!\pcol@com@clearpage! for \!\clearpage!  ($f=|clear|$), and
% \!\endparacol!  ($f=|end|$).
% \SpecialIndex{\pcol@op@start}
% \SpecialIndex{\pcol@op@switch}
% \SpecialIndex{\pcol@op@flush}
% \SpecialIndex{\pcol@op@clear}
% \SpecialIndex{\pcol@op@end}
% 
%    \begin{macrocode}
%% Starting Environment

\def\pcol@invokeoutput#1{\deadcycles\z@
  \pcol@Logstart{\pcol@invokeoutput
    {#1:\the\pcol@currcol/\the\pcol@nextcol%
     \ifnum#1=-10011:\ifpcol@sync s\fi \ifpcol@clear c\fi\fi}}%
  \penalty-\@Miv \global\pcol@prevdepth\prevdepth \vbox{}%
  \penalty#1\relax \prevdepth\pcol@prevdepth
  \linewidth\columnwidth \advance\linewidth-\pcol@lrmargin
  \ifdim\pcol@lrmargin>\z@ \parshape\@ne\@totalleftmargin\linewidth \fi
  \hsize\columnwidth
  \pcol@Logend{\pcol@invokeoutput{#1}}}

%    \end{macrocode}
% \end{macro}
% 
% \KeepSpace{2}
% \begin{macro}{\paracol}
% \changes{v1.0}{2011/10/10}
%	{Change the order of operations for sake of clarity.}
% \changes{v1.0}{2011/10/10}
%	{Add the mechanism of inter-environment local counter conservation.}
% \changes{v1.0}{2011/10/10}
%	{Let $\cs{col@number}\EQ1$ instead of $C$ to keep \cs{maketitle} from
%	 producing title with \cs{twocolumn}.}
% \changes{v1.0}{2011/10/10}
%	{Add initialization of \cs{pcol@textfloatsep}, \cs{ifpcol@lastpage},
%	 \cs{pcol@firstprevdepth} and \cs{@combinefloats}.}
% \changes{v1.0}{2011/10/10}
%	{Make API commands environment-local and inhibit nesting of
%	 \texttt{paracol}.} 
% \changes{v1.3-2}{2013/09/17}
%	{Modify to add the optinal argument $C_L$ and optional
%	 `\string\texttt{*}' for parallel-paging.}
% \begin{macro}{\pcol@xparacol}
% \changes{v1.3-2}{2013/09/17}
%	{Introduced to let $C_L\EQ C$ if the optional argument $C_L$ is not
%	 given to \cs{paracol}.}
% \begin{macro}{\pcol@yparacol}
% \changes{v1.3-2}{2013/09/17}
%	{Introduced to process the optional `\string\texttt{*}' given with
%	 the optioal argument $C_L$ of \cs{paracol}.}
% \begin{macro}{\pcol@zparacol}
% \changes{v1.3-2}{2013/09/17}
%	{Introduced to add the optinal argument $C_L$ and optional
%	 `\string\texttt{*}' to \cs{paracol} for parallel-paging and to do
%	 what had done by \cs{paracol}.}
% 
% The API macro $\!\paracol!|[|\CL|][*]|\Arg{\C}$\oarg{text} is invoked by
% \beginparacol{} to start a \Midx{\env{paracol}} environment.  The macro
% sipmly examines the existence of the optional argument $\CL$, whose
% default value $\C$ is given by \!\pcol@xparacol!, to decide the
% number of columns in left \parapag{}es.  Then if the optional argument is
% given, \!\pcol@yparacol! examines the existence of `|*|' following it for
% \npaired{} \parapag{}ing to let $\CSIndex{ifpcol@paired}=\false$, while
% \!\paracol! gave the default $\true$ for \paired{} one to the switch.
% Then the all other operations to start the environment is doen by
% \!\pcol@zparacol!.
% 
%    \begin{macrocode}
\def\paracol{\global\pcol@pairedtrue \@ifnextchar[%]
  \pcol@yparacol\pcol@xparacol}
\def\pcol@xparacol#1{\pcol@zparacol[#1]{#1}}
\def\pcol@yparacol[#1]{%
  \@ifstar{\global\pcol@pairedfalse \pcol@zparacol[#1]}%
          {\pcol@zparacol[#1]}}
%    \end{macrocode}
% 
% In \!\pcol@zparacol!, after making it sure to be in vertical mode by
% \CSIndex{par}, at first we examine if we are neither in a box by
% \CSIndex{ifinner} nor with ordinary two-column typesetting by
% \CSIndex{if@twocolumn}, and complain about inappropriateness unless our
% expectation is satisfied.  Then we let $\CL$ and $\C$ have the value given
% through the corresponding arguments, unless $\CL>\C$ to let us make
% $\CL=\C$ silently.  Next we examine $\CL<\C$, and if not we let
% $\CSIndex{ifpcol@paired}=\true$ regardless the setting in
% \!\pcol@yparacol! because \npaired{} typesetting is meaningless without
% \parapag{}ing.  On the other hand, if \npaired{} typesetting is specified,
% we let $\CSIndex{ifpcol@swapcolumn}={\false}$ but {\em not} \!\global!ly
% because \cswap{} is meaningless in \npaired{} mode in the environment now
% starting.
% 
% Second, we perform the operations done by \!\item! if
% $\CSIndex{if@newlist}=\true$ and $\CSIndex{if@inlabel}\~=\false$ to mean
% the first one in a \env{list}-like environment will appear at the very
% first line of the leftmost column.  That is, we invoke \!\@nbitem! if
% $\CSIndex{if@nobreak}=\true$, or add a penalty \!\@beginparpenalty! and a
% vertical skip $\!\@topsep!-\!\parskip!-\!\itemsep!$ so that the first
% \!\item! is \!\@topsep! apart from the last line above the environment,
% and then let $\CSIndex{if@newlist}=\false$.  The reason why we do these
% operations here is that, if the \env{paracol} environment is enclosed in a
% \env{list}-like environment without anything between two \!\begin! for
% environments, we have to align all first \!\item!s in all columns.  That
% is, if we did not do that, the {\em literally first} \!\item! would do
% that resulting in the column having the \!\item! led by the vertical skip
% of \!\@topsep! while others should have ordinary inter-\!\item! skips.
% Therefore, we perform the operations on behalf of all first \!\item!s in
% all columns to have the skip of \!\@topsep! {\em above} the \env{paracol}
% environment we are opening.  Note that if \beginparacol{} immediately
% follows a \!\begin! for a \env{trivlist}-like environment,
% $\CSIndex{if@inlabel}=\true$ because the first \!\item! was given in the
% opening macro and thus the operations shown above has already been
% performed.
% 
% \changes{v1.3-6}{2013/09/17}
%	{Add operations for the vertical skips at the beginning of a
%	 \texttt{list}-like environment.}
% \changes{v1.35-6}{2018/12/31}
% 	{Add error check with \cs{ifinner} and \cs{if@twocolumn}.}
% \changes{v1.35-2}{2018/12/31}
% 	{Add $\lnot$\cs{if@inlabel} to the condition to perform operations
%	 for first \cs{item}.}
% 
%    \begin{macrocode}
\def\pcol@zparacol[#1]#2{\par
  \ifinner \@parmoderr \fi
  \if@twocolumn \PackageError{paracol}{%
    Environment paracol cannot work with ordinary two-column
    typesetting.}\@ehb\fi
  \global\pcol@ncolleft#1\relax \global\pcol@ncol#2\relax
  \ifnum\pcol@ncolleft>\pcol@ncol \global\pcol@ncolleft\pcol@ncol \fi
  \ifnum\pcol@ncolleft<\pcol@ncol\else \global\pcol@pairedtrue \fi
  \ifpcol@paired\else \pcol@swapcolumnfalse \fi
  \if@newlist \if@inlabel\else
    \if@nobreak \@nbitem
    \else
      \addpenalty\@beginparpenalty
      \addvspace\@topsep
      \addvspace{-\parskip}\addvspace{-\itemsep}%
    \fi
    \global\@newlistfalse
  \fi\fi
%    \end{macrocode}
% 
% The third operation group is to set up lists for counters as follows.
% First we obtain $\CC=\!\cl@@ckpt!$ to have all counters defined by
% \!\newcounter! and \counter{page}.  Then we scan $\CG=\!\pcol@gcounters!$
% applying \!\pcol@remctrelt! to each element to have
% $\CTL=\!\pcol@counters!$ for all \lcounter{}s by removing all $\cg\in\CG$
% from $\CC$, and to move $\clist(\cg)=|\cl@|{\cdot}\cg$ to
% $|\pcol@cl@|{\cdot}\cg$ redefining
% $|\cl@|{\cdot}\cg=\!\pcol@stepcounter!\Arg{\cg}$
% 
% \SpecialArrayIndex{\theta}{\cl@}
% 
% to clear descendant \lcounter{}s of $\cg$, i.e., those in $\clist(\cg)$.
% 
% Next we scan $\CTL$ applying \!\pcol@thectrelt! to each element $\cl\in\CTL$
% to let $|\pcol@thectr@|{\cdot}\~\cl=|\the|{\cdot}\cl$
% 
% \SpecialArrayIndex{\theta}{\pcol@thectr@}
% \SpecialArrayIndex{\theta}{\the}
% 
% so that the former has the {\em default} \lrep{} of $\cl$.  The macro
% \!\pcol@thectrelt! also lets
% $|\the|{\cdot}\cl=|\pcol@thectr@|{\cdot}\cl{\cdot}0$
% 
% \SpecialArrayIndex{\theta{\cdot}c}{\pcol@thectr@}
% \SpecialArrayIndex{\theta}{\the}
% 
% if the \lrep{} for $\cl$ and $c=0$ has been defined by
% \!\definethecounter!.  This is necessary because in the first visit to the
% leftmost column 0 we will neither invoke \!\pcol@switchcol! nor thus scan
% $\CTL$ with \!\pcol@setctrelt! which defines \lrep{}s, unless a \mctext{}
% is specified with \beginparacol.
% 
% Next we give the initial value of $\val_c(\cl)$ for each column $c$ and
% \lcounter{} $\cl\in\CTL$ by the followings enclosed in a group.  First we
% scan $\Cc_0$ applying \!\pcol@loadctrelt! to each
% $\cl\in\Cc'_0=\Set{\theta}{\<\theta,\val_0(\theta)\>\in\Cc_0}$ to have
% the value $\val_0(\cl)$ in $|\pcol@ctr@|{\cdot}\cl$
% 
% \SpecialArrayIndex{\theta}{\pcol@ctr@}
% 
% temporarily.  Next we scan $\CTL$ to pick \lcounter{}s $\theta$ such that
% $\Uidx\Val(\theta)\neq\val_0(\theta)$ where $\Val(\theta)$ is the value of
% $\theta$ outside \env{paracol} environment, or $\theta\notin\Cc'_0$, to
% let them be listed in \!\@gtempa!.  That is, we pick \lcounter{}s which
% have been updated or defined after the closing of the previous
% \env{paracol}, or everything in $\CTL$ at the first \beginparacol{} because
% $\Cc_0=\emptyset$\footnote{
% 
% Undefined in fact.}.
% 
% Finally, we invoke \!\pcol@synccounter! giving it \!\@gtempa! as its
% argument to let $\val_c(\theta)=\Val(\theta)$ for all $c$ and for each
% $\theta$ in \!\@gtempa!.  That is, when \env{paracol} environment appears
% two or more times, the value of a \lcounter{} at the beginning of the
% second or succeeding environment is kept unchanged from that at the end of
% the previous environment, unless it has been updated between them.  Note
% that these scans above are enclosed in a group in order to make
% $|\pcol@ctr@|{\cdot}\cl$
% 
% \SpecialArrayIndex{\theta}{\pcol@ctr@}
% 
% local and thus collected as a garbage at \!\endgroup!.
% 
%    \begin{macrocode}
  \global\let\pcol@counters\cl@@ckpt
  \let\@elt\pcol@remctrelt \pcol@gcounters
  \let\@elt\pcol@thectrelt \pcol@counters
  \begingroup
    \let\@elt\pcol@loadctrelt \csname pcol@counters0\endcsname
    \let\@elt\pcol@cmpctrelt \global\let\@gtempa\@empty \pcol@counters
    \pcol@synccounter\@gtempa
  \endgroup
%    \end{macrocode}
% 
% Fourth, we set up a few \LaTeX's typesetting parameters which should have
% appropriate values in the environment.  We let
% $\CSIndex{if@twocolumn}=\true$ so that \textit{float}|*| environments work
% for \pwise{} floats and LaTeX's \!\@addmarginpar! determine the margin
% for marginal notes by \CSIndex{if@firstcolumn} whose truth value is
% determined by our own \!\@addmarginpar!.  We also let $\!\col@number!=1$
% instead of $\C$ so that its (almost surely) sole user \!\maketitle! will
% not produce the title with \!\twocolumn! which cannot be in the
% environment.
% 
% Then we invoke $\!\pcol@setcolumnwidth!\<\Cfrom\>\<\Cto\>\<r\>\<s\>$ once
% or twice with
% $(\Cfrom,\Cto,r,s)=(0,\CL,\!\pcol@columnratioleft!,\!\pcol@colwidthspecleft!)$
% always for left \parapag{}e, and with
% $(\CL,\C,\!\pcol@columnratioright!,\!\pcol@colwidthspecright!)$ for right
% one if $\CL<\C$ to define column widths $\w_c=|\pcol@columnwidth|{\cdot}c$
% 
% \SpecialArrayIndex{c}{\pcol@columnwidth}
% 
% and that of \csepgap{}s $\gap_c=|\pcol@columnsep|{\cdot}c$
% 
% \SpecialArrayIndex{c}{\pcol@columnsep}
% 
% for all $c\In0\C=\LBRP0\CL\cup\LBRP\CL\C$.
% 
% Then we initialize other variables as follows;
% $\lrm=\!\pcol@lrmargin!=\!\textwidth!-\!\linewidth!$ so that, if $\lrm>0$,
% \!\linewidth! for $c$ has $\w_c-\lrm$ reflecting the paragraph shape of the
% \env{list}-like environment surrounding \env{paracol} environment;
% $\!\pcol@topskip!=\!\topskip!$ for the second and succeeding pages;
% $\!\pcol@textfloatsep!=\infty$ to mean we don't have any \sync{}ation
% points so far; $\CSIndex{ifpcol@lastpage}=\false$ because the \spage{} is
% not the last so far; and $\!\pcol@firstprevdepth!=\!\prevdepth!$ in
% decimal integer form for the extreme empty case.
% 
% We also make the macro \!\@combinefloats! \!\let!-equal to our own
% \!\pcol@combinefloats!  throughout the environment, after saving its
% original definition into \Midx{\!\pcol@@combinefloats!} for processing
% \!\output! request sneaked from outsied of environment, so that our
% customization is in effect for any invocations including those from
% \LaTeX's own macros.  Similarly, \!\@addmarginpar! is made \!\let!-equal
% to our own \!\pcol@addmarginpar!  while its orginal definition is saved
% into \!\pcol@@addmarginpar! but in this case we need the original for the
% implementaion of our own.  On the other hand, \!\end@dblfloat! is simply
% replaced with our own \!\pcol@end@dblfloat! being what \LaTeX{} had had until
% 2014 as discussed in item-(\ref{item:ovv-float-end@dblfloat}) of
% \secref{sec:imp-ovv-float}.
% 
% \changes{v1.1}{2012/05/11}
% 	{Replace the calculation of \cs{columnwidth} with the call of
%	 \cs{pcol@setcolumnwidth} and the assignment of $w_0$ to it.}
% \changes{v1.2-4}{2013/05/11}
%	{Modify the setting of \cs{if@firstcolumn} according to
%	 column-swapping.}
% \changes{v1.2-5}{2013/05/11}
%	{Remove the setting \cs{columnwidth}, \cs{hsize} and \cs{linewidth}
%	 because they are properly set in and after \cs{pcol@output@start}.}
% \changes{v1.2-5}{2013/05/11}
%	{Add the setting of \cs{pcol@lrmargin}.}
% \changes{v1.2-7}{2013/05/11}
%	{Add the saving of \cs{@combinefloats}.}
% \changes{v1.3-4}{2013/09/17}
%	{Revise the mechanism to define the width of columns and
%	 column-separating gaps, and add local overriding definition of
%	 \cs{@addmarginpar}.}
% \changes{v1.3-2}{2013/09/17}
%	{Add operations to define the width of columns and column-separating
%	 gaps in right parallel-pages.}
% \changes{v1.32-3}{2015/10/10}
% 	{Add replacement of \cs{end@dblfloat} with \cs{pcol@end@dblfloat}.}
% 
%    \begin{macrocode}
  \global\@twocolumntrue \col@number\@ne
  \pcol@setcolumnwidth\z@\pcol@ncolleft
    \pcol@columnratioleft\pcol@colwidthspecleft
  \ifnum\pcol@ncolleft<\pcol@ncol
    \pcol@setcolumnwidth\pcol@ncolleft\pcol@ncol
      \pcol@columnratioright\pcol@colwidthspecright
  \fi
  \pcol@lrmargin\textwidth \advance\pcol@lrmargin-\linewidth
  \global\pcol@topskip\topskip
  \global\pcol@textfloatsep\maxdimen
  \pcol@lastpagefalse \xdef\pcol@firstprevdepth{\number\prevdepth}%
  \let\pcol@@combinefloats\@combinefloats \let\@combinefloats\pcol@combinefloats
  \let\pcol@@addmarginpar\@addmarginpar \let\@addmarginpar\pcol@addmarginpar
  \let\end@dblfloat\pcol@end@dblfloat
%    \end{macrocode}
%
% Fifth, we save the original definition of \!\set@color! into
% \!\pcol@set@color!, and then examine if $\!\set@color!\neq\!\relax!$
% meaning some coloring package is loaded.  If any coloring packages are not
% loaded, we make macros for \bgpaint{}, namely \!\pcol@bg@paintpage!,
% \!\pcol@bg@paintcolumns! and \!\pcol@bg@paintbox!, \!\let!-equal to
% \!\relax! for the first two and to \!\@gobble! for the last so that they do
% nothing without coloring package.
% 
% If the coloring is enabled, on the other hand, we redefine \LaTeX's
% \!\set@color!  so that it works as \!\pcol@set@color@push!  with
% \colorstack.  We also prepare the text colring mechanism to let
% $\CSIndex{ifpcol@inner}=\true$ in every \!\vbox! in the \env{paracol}
% environment as follows.  First, we let $\CSIndex{ifpcol@inner}=\false$
% because we are not in any \!\vbox!es obviously.  Then, to use
% \!\everyvbox! for our own pupose, we do the followings; (1) \!\global!ly
% assign a token \Midx{\!\pcol@dummytoken!}, which should never occurs, to
% \!\pcol@everyvbox!; (2) save \!\everyvbox!  into \!\pcol@everyvbox!
% locally; (3) let \!\everyvbox! have a \!\the!-reference to
% \!\pcol@everyvbox! and |\pcol@innerture| to let
% $\CSIndex{ifpcol@inner}={\true}$; and then (4) make \!\everyvbox!
% \!\let!-equal to \!\pcol@everyvbox!.  By the last operation (4), any
% \!\everyvbox! appearing in the \env{paracol} environment is replaced with
% \!\pcol@everyvbox! to keep the real \!\everyvbox! from modified nullifying
% our own operation |\pcol@innertrue|.  On the other hand, since both
% \!\everyvbox!  and \!\pcol@everyvbox! are registers to hold tokens and
% thus any operations applicable to \!\everyvbox! are also applicable to
% \!\pcol@everyvbox!, any updates on \!\everyvbox! and explicit references
% to it are simulated by \!\pcol@everyvbox!.  Then the initial tokens given
% to \!\pcol@everyvbox!  by the saving operation (2) or tokens given inside
% the environment are correctly processed whenever a \!\vbox! is opened,
% toghether with |\pcol@innertrue| to fulfill our own purpose, because the
% real \!\everyvbox! is let have a \!\the!-reference to \!\pcol@everyvbox!
% by (3).  We also reserve the invocation of \!\pcol@restoreeveryvbox! by
% \!\aftergroup! so that the macro is invoked just after \Endparacol{} to
% examine if any \!\global! assignments to \!\everyvbox! has been made in
% the environment.  The funny \!\global!  assignment (1) is done for this
% examination so that we detect global assignments done in the environment
% having been closed because they should have changed the global value of
% \!\pcol@everyvbox! to something different from \!\pcol@dummytoken!.
% 
% Then we continue the case having some coloring package to make \bgpaint{}
% macros \!\pcol@bg@paintpage!, \!\pcol@bg@paintbox! and
% \!\pcol@bg@paintcolumns! activated by making them \!\let!-equal to thier
% |@@| counterparts namely \!\pcol@bg@@paintpage!, \!\pcol@bg@@paintbox! and
% \!\pcol@bg@@paintcolumns! which implement \bgpaint{}.
% 
% Finally we empty the shadow \colorstack{}
% $\cstshadow={}$\!\pcol@colorstack@shadow!  to give it initial value
% regardless of the availability of coloring package.
% 
% \changes{v1.2-1}{2013/05/11}
% 	{Add redefinitions of \cs{set@color} and \cs{reset@color}.}
% \changes{v1.22}{2013/06/30}
% 	{Add a trick with \cs{everyvbox} to turn on \cs{ifpcol@inner} in
%	 every \cs{vbox}.}
% \changes{v1.22}{2013/06/30}
% 	{Move initial emptying of $\string\mathit{\Gamma}$ from
%	 \cs{pcol@output@start} to \cs{pcol@zparacol}.}
% \changes{v1.22}{2013/06/30}
% 	{Add initial emptying of $\string\hat{\mathit{\Gamma}}$ and $\chi$.}
% \changes{v1.3-3}{2013/09/17}
%	{Add definition of painting macros dependent to the availability of
%	 a coloring package.}
% \changes{v1.34}{2018/05/07}
%	{Remove the initializations of \cs{pcol@colorstack} and
%	 \cs{pcol@colorstack@buf} because they no longer exist.}
% 
%    \begin{macrocode}
  \global\let\pcol@set@color\set@color
  \ifx\set@color\relax
    \let\pcol@bg@paintpage\relax \let\pcol@bg@paintbox\@gobble
    \let\pcol@bg@paintcolumns\relax
  \else
    \let\set@color\pcol@set@color@push
    \pcol@innerfalse
    \global\pcol@everyvbox{\pcol@dummytoken}%
    \pcol@everyvbox\everyvbox
    \everyvbox{\the\pcol@everyvbox \pcol@innertrue}
    \let\everyvbox\pcol@everyvbox
    \aftergroup\pcol@restoreeveryvbox
    \let\pcol@bg@paintpage\pcol@bg@@paintpage
    \let\pcol@bg@paintbox\pcol@bg@@paintbox
    \let\pcol@bg@paintcolumns\pcol@bg@@paintcolumns
  \fi
  \gdef\pcol@colorstack@shadow{}%
%    \end{macrocode}
% 
% \changes{v1.2-2}{2013/05/11}
% 	{Add initialization of \cs{pcol@footnotebase} and
%	 \cs{pcol@nfootnotes}, and redefinitions of \cs{footnote},
%	 \cs{footnotemark}, \cs{footnotetext} and \cs{@footnotetext}.}
% 
% The sixth settings are for (mainly \mcfnote) footnotes.  We initialize two
% footnote-related count registers letting $\bf=\!\pcol@footnotebase!$ have
% \!\c@footnote! and zero-clearing $\nf=\!\pcol@nfootnotes!$.  Then we
% redefine \LaTeX's API macros \!\footnote!, \!\footnotemark! and
% \!\footnotetext! to let them be our own \!\pcol@footnote!,
% \!\pcol@footnotemark! and \!\pcol@footnotetext! so that they have
% starred-versions.  The other API macro to be redefined, if \Scfnote{}
% typesetting is in effect, is \!\footnoterule!  which lets
% $\!\columnwidth!=\!\textwidth!$ before invoking its original version saved
% in \!\pcol@footnoterule! so that it acts as in single-columned typesetting
% rateher than multi-columned.  Then we redefine \LaTeX's internal macro
% \!\@footnotetext! letting it be our own \!\pcol@fntext! for encapsulating
% a footnote in a \!\vbox! and for deferred \!\insert!ion with \Scfnote{}
% typesetting.
% 
%    \begin{macrocode}
  \pcol@footnotebase\c@footnote \global\pcol@nfootnotes\z@
  \let\footnote\pcol@footnote
  \let\footnotemark\pcol@footnotemark
  \let\footnotetext\pcol@footnotetext
  \ifpcol@scfnote
    \def\footnoterule{{\columnwidth\textwidth \pcol@footnoterule}}%
  \fi
  \let\@footnotetext\pcol@fntext
%    \end{macrocode}
% 
% Seventh, we let \!\marginpar!, \!\@mn@@marginnote! and \!\@xympar! be our
% own versions \!\pcol@marginpar!, \!\pcol@marginnote! and \!\pcol@xympar!
% respectively for the emulation of \!\marginnote!, saving the original
% version of the first and third into \!\pcol@@marginpar! and
% \!\pcol@@xympar!.  Then we inactivate API macros \!\twosided! and
% \!\footnotelayout!  together with their backward-compatible macros
% \!\swapcolumninevenpages!, \!\noswapcolumnineven~pages!,
% \!\multi~columnfootnotes!, \!\singlecolumnfootnotes! and
% \!\mergedfootnotes!, so that they commonly invoke \!\pcol@ignore! because
% their inherent operations turning corresponding switches are harmful in
% \env{paracol} environment.  Note that the inactivation of \!\twosided! is
% done by redefinition of \!\pcol@twosided! because we need optional
% argument processing by \!\twosided! even when it is inactivated.
% 
% \changes{v1.2-4}{2013/05/11}
%	{Add nullificaion of \cs{[no]swapcolumninevenpages}.}
% \changes{v1.2-2}{2013/05/11}
%	{Add nullificaton of API macros of footnote typesetting definition.}
% \changes{v1.3-3}{2013/09/17}
%	{Add new API inactivation for \cs{pcol@twosided}.}
% \changes{v1.3-4}{2013/09/17}
%	{Add new API inactivation for \cs{pcol@twosided}.}
% \changes{v1.3-5}{2013/09/17}
%	{Add new API inactivation for \cs{footnotelayout}.}
% \changes{v1.35-3}{2018/12/31}
% 	{Add local modifications of \cs{marginpar}, \cs{@mn@@marginnote} and
%	 \cs{@xympar} for the emulation of \cs{marginnote}.}
% 
%    \begin{macrocode}
  \let\pcol@@marginpar\marginpar \let\marginpar\pcol@marginpar
  \let\@mn@@marginnote\pcol@marginnote
  \let\pcol@@xympar\@xympar \let\@xympar\pcol@xympar
  \def\pcol@twosided[#1]{\pcol@ignore\twosided}%
  \def\swapcolumninevenpages{\pcol@ignore\swapcolumninevenpages}%
  \def\noswapcolumninevenpages{\pcol@ignore\noswapcolumninevenpages}%
  \def\footnotelayout#1{\pcol@ignore\footnotelayout}%
  \def\multicolumnfootnotes{\pcol@ignore\multicolumnfootnotes}%
  \def\singlecolumnfootnotes{\pcol@ignore\singlecolumnfootnotes}%
  \def\mergedfootnotes{\pcol@ignore\mergedfootnotes}%
%    \end{macrocode}
% 
% Eigth, we scan the list \!\pcol@localcommands! of $\arg{com}$ being the
% name of commands, e.g., |switchcolumn|, available only in the environment
% or customized for the environment, applying \!\pcol@defcomelt! to each
% $\arg{com}$ to let $|\|\arg{com}=|\pcol@com@|{\cdot}\arg{com}$
% 
% \SpecialArrayMainIndex{\<\string\mathit{com}\>}{\pcol@com@}
% 
% the latter of which is the real implementation of the former.  Note that
% the list does not have all \elocal{} API commands but we omit
% \!\column!(|*|) for \env{column}(|*|) environments because their
% implementations \!\pcol@com@column!(|*|) are modified after the first
% invocation.  Therefore, we \!\def!ine \!\column!(|*|) to have
% \!\pcol@com@column!(|*|) in their bodies\footnote{
% 
% We can do this for other commands instead of adhearing to \cs{let} to
% eliminate the execption, but the author loves to use \cs{let} as much as
% possible.}.
% 
% We also give the first \!\def!initions of
% \!\pcol@com@column!(|*|) to let them do nothing but re\!\def!ine
% themselves by \!\pcol@defcolumn! unless
% \!\pcol@com@column*! is given an optional \mctext{} argument which is
% directly processed by \!\pcol@sptext!, if they appear as the first
% \cswitch{} command\slash environment after \beginparacol.  Then we
% re\!\def!ine \!\paracol! itself so that it will complain of illegal
% nesting by \!\PackageError!.
% 
%    \begin{macrocode}
  \let\@elt\pcol@defcomelt \pcol@localcommands
  \def\column{\pcol@com@column}%
  \@namedef{column*}{\@nameuse{pcol@com@column*}}%
  \global\let\pcol@com@column\pcol@defcolumn
  \global\@namedef{pcol@com@column*}{\pcol@defcolumn
    \@ifnextchar[%]
     \pcol@sptext\relax}%
  \def\paracol##1{\PackageError{paracol}{%
    Environment paracol cannot be nested.}\@eha}%
%    \end{macrocode}
% 
% Ninth, we let \!\output! have our output routine \!\pcol@output! as its sole
% token, and then make \!\output! request with
% $\!\penalty!=\!\pcol@op@start!$ by \!\pcol@invokeoutput! to invoke
% \!\pcol@output@start! for initialization, after letting
% $\!\@elt!=\!\relax!$ to make it sure that any lists can be manipulated
% without unexpected application of a macro to their elements.
% 
%    \begin{macrocode}
  \output{\pcol@output}%
  \let\@elt\relax
  \pcol@invokeoutput\pcol@op@start
%    \end{macrocode}
% 
% Tenth and finally, we let $\!\pcol@nextcol!=0$ in case \beginparacol{} has
% the optional argument for \mctext{}, and invoke \!\pcol@sptext! if it has.
% Otherwise, we invoke $|\pcol@colpream|{\cdot}0$ being the \colpream{} of
% the first column 0, which will be invoked by \!\pcol@switchcol! if
% \mctext{} is given.
% 
% \SpecialArrayIndex{c}{\pcol@colpream}
% 
% \changes{v1.35-5}{2018/12/31}
% 	{Add the invocation of $\cs{pcol@colpream}{\cdot}0$.}
% 
%    \begin{macrocode}
  \pcol@nextcol\z@
  \@ifnextchar[%]
    \pcol@sptext{\@nameuse{pcol@colpream0}}}
%    \end{macrocode}
% \end{macro}\end{macro}\end{macro}\end{macro}
% 
% \begin{macro}{\pcol@paracol}
% The macro \!\pcol@paracol! has the definition of \!\paracol!, which is
% redefined in the macro itself, so that the only referrer
% \!\pcol@icolumncolor! examines if it appears in \env{paracol}, i.e.,
% $\!\pcol@paracol!\neq\!\paracol!$.
% 
%    \begin{macrocode}
\let\pcol@paracol\paracol

%    \end{macrocode}
% \end{macro}
% 
% \begin{macro}{\thecolumn}
% \changes{v1.3-5}{2013/09/17}
%	{Introduced to let users know which column they are working in.}
% 
% The API macro \!\thecolumn! gives the value of \!\pcol@currcol! to users
% to let them know which column they are working in so that, for example,
% they do some column-dependent opreations.
% 
%    \begin{macrocode}
\def\thecolumn{\number\pcol@currcol}

%    \end{macrocode}
% \end{macro}
% 
% \begin{macro}{\pcol@ignore}
% \changes{v1.2-4}{2013/05/11}
%	{Introduced for nullificaion of \cs{[no]swapcolumninevenpages}.}
% \changes{v1.2-2}{2013/05/11}
%	{Introduced for nullificaton of API macros of footnote typesetting
%	 definition.}
% 
% The macro \!\pcol@ignore!$\arg{macro}$ is to complain that the
% $\arg{macro}$ being one of the followings appears in \env{paracol}
% environment.
% 
% \begin{itemize}\raggedright\item[]
% \!\twosided!,
% \!\swapcolumninevenpages!,
% \!\noswapcolumninevenpages!,
% \!\footnotelayout!,
% \!\multicolumnfootnotes!,
% \!\singlecolumnfootnotes!,
% \!\mergedfootnotes!
% \end{itemize}
% 
% That is, these macros, except for \!\twosided!, are re\!\def!ined in
% \!\pcol@zparacol! to invoke this macro with the argument identifying
% themselves, which is shown in the warning message given by
% \!\PackageWarning!.  As for \!\twosided!, the target of the
% re\!\def!inition is \!\pcol@twosided! so that its optional argument is
% captured before the complaint.
% 
%    \begin{macrocode}
\def\pcol@ignore#1{\PackageWarning{paracol}{The command \string#1 is not
  effective in paracol environment and thus ignored}}

%    \end{macrocode}
% \end{macro}
% 
% \begin{macro}{\pcol@localcommands}
% \changes{v1.0}{2011/10/10}
%	{Introduced to make API commands environment-local.}
% \changes{v1.3-5}{2013/09/17}
%	{Add \cs{@elt}\string\texttt{\char`\{cleardoublepage\char`\}} for
%	 \cs{cleardoublepage}.} 
% 
% The macro \!\pcol@localcommands! is the list of the names of the following
% {\em\Uidx\elocal} API commands (or {\em\Uidx\lcommand{}s} in short) and is
% solely referred to by \!\pcol@zparacol!.
% 
% \begin{center}\begin{tabular}{llll}
% \!\switchcolumn!&
% \!\endcolumn!(|*|)&
% \!\nthcolumn!(|*|)&\!\endnthcolumn!(|*|)\\
% \!\leftcolumn!(|*|)&\!\endleftcolumn!(|*|)&
% \!\rightcolumn!(|*|)&\!\endrightcolumn!(|*|)\\
% \!\flushpage!&
% \!\clearpage!&
% \!\cleardoublepage!\\
% \!\synccounter!&
% \!\syncallcounters!
% \end{tabular}\end{center}
% 
% Note that we omit \!\column!(|*|) from the list as discussed in the
% description of \!\pcol@zparacol!.
% 
%    \begin{macrocode}
\def\pcol@localcommands{%
  \@elt{switchcolumn}%
  \@elt{endcolumn}\@elt{endcolumn*}%
  \@elt{nthcolumn}\@elt{endnthcolumn}\@elt{nthcolumn*}\@elt{endnthcolumn*}%
  \@elt{leftcolumn}\@elt{endleftcolumn}\@elt{leftcolumn*}\@elt{endleftcolumn*}%
  \@elt{rightcolumn}\@elt{endrightcolumn}%
    \@elt{rightcolumn*}\@elt{endrightcolumn*}%
  \@elt{flushpage}\@elt{clearpage}\@elt{cleardoublepage}%
  \@elt{synccounter}\@elt{syncallcounters}}
%    \end{macrocode}
% \end{macro}
% 
% \begin{macro}{\pcol@defcomelt}
% \changes{v1.0}{2011/10/10}
%	{Introduced to make API commands environment-local.}
% 
% The macro \!\pcol@defcomelt! is invoked solely from \!\pcol@zparacol! to be
% applied to each element $\arg{com}$ in \!\pcol@localcommands!.  Two
% lengthy \!\let!s with \!\expandafter!s are for doing
% $\!\let!|\|\arg{com}|=\pcol@com@|{\cdot}\arg{com}$ to bind the \elocal{}
% API command $|\|\arg{com}$ and its implementation
% $|\pcol@com@|{\cdot}\arg{com}$.
% 
% \SpecialArrayIndex{\<\string\mathit{com}\>}{\pcol@com@}
% 
%    \begin{macrocode}
\def\pcol@defcomelt#1{%
  \expandafter\let\expandafter\reserved@a\csname pcol@com@#1\endcsname
  \expandafter\let\csname #1\endcsname\reserved@a}

%    \end{macrocode}
% \end{macro}
% 
% \begin{macro}{\@dbldeferlist}
% \changes{v1.32-3}{2015/10/10}
% 	{Add top-level definition in case that future \LaTeX{} removes it at
%	 all.}
% \begin{macro}{\pcol@end@dblfloat}
% \changes{v1.32-3}{2015/10/10}
% 	{Added to go back to old mechanism.}
% As discussed in \secref{sec:imp-ovv-float}, 2015 version of \LaTeX{}
% no longer uses \!\@dbldeferlist! but the macro itself is still kept in
% \LaTeX{}.  However it will be removed in future to make the first
% \!\@cons! with it resulting in an error.  Therefore, here we have its top
% level definition with empty duplicatedly in case of its future
% elimination.  The macro \!\end@dblfloat!, on the other hand, is replaced
% with a new definition in the new \LaTeX{} of course, and thus we define
% \!\pcol@end@dblfloat! here to keep its old definition and to replace the
% new one in \env{paracol} environment as discussed in
% item-(\ref{item:ovv-float-end@dblfloat}) of \secref{sec:imp-ovv-float}.
% 
%    \begin{macrocode}
\gdef\@dbldeferlist{}
\def\pcol@end@dblfloat{%
  \if@twocolumn
    \@endfloatbox
    \ifnum\@floatpenalty <\z@
      \@largefloatcheck
      \@cons\@dbldeferlist\@currbox
    \fi
    \ifnum \@floatpenalty =-\@Mii \@Esphack\fi
  \else
    \end@float
  \fi
}
%    \end{macrocode}
% \end{macro}\end{macro}
% 
% 
% 
% \KeepSpace{6}
% \section{Column Width Setting}
% \label{sec:imp-cwidth}
% \changes{v1.3-4}{2013/09/17}
%	{Add the secion ``Column Width Setting'' mainly to discuss the new
%	 API \cs{setcolumnwidth}.}
% 
% \begin{macro}{\columnratio}
% \changes{v1.1}{2012/05/11}
% 	{Introduced to specify column width fractions.}
% \changes{v1.3-2}{2013/09/17}
%	{Add optional second argument for fractions in right
%	 parallel-pages.} 
% \begin{macro}{\pcol@icolumnratio}
% \changes{v1.3-2}{2013/09/17}
%	{Introduced to process the optional second argument of
%	 \cs{columnratio}.}
% \begin{macro}{\pcol@columnratioleft}
% \changes{v1.1}{2012/05/11}
% 	{Introduced to keep column width fractions.}
% \changes{v1.3-2}{2013/09/17}
%	{Renamed from \cs{pcol@columnratio} to clarify it has fractions for
%	 left parallel-pages.}
% \begin{macro}{\pcol@columnratioright}
% \changes{v1.3-2}{2013/09/17}
%	{Introduced to keep column width fractions for right
%	 parallel-pages.}
% 
% The API macro $\!\columnratio!
% \Arg{r^l_0,r^l_1,\cdots,r^l_{k^l-1}}|[|r^r_0,r^r_1,\cdots,r^r_{k^r-1}|]|$
% defines the column width fraction $r^l_c$ for column $c$ in left
% \parapag{}es and optionally $r^r_c$ for those in right \parapag{}es.  This
% macro and its callee \!\pcol@icolumnratio! just \!\global!y \!\def!ine
% macros \!\pcol@columnratioleft! and \!\pcol@columnratioright!  whose
% bodies have the first and second arguments respectively, or commonly have
% the first if the second optional one is not given, so that they are given
% to \!\pcol@setcolwidth@r! as its third argument through
% \!\pcol@setcolumnwidth! invoked in \!\pcol@zparacol!.  Both of
% \!\pcol@columnratioleft! and \!\pcol@columnratioright! are initialized to
% be empty so that all columns have same width and are separated by
% \!\columnsep! as default.  Note that \!\pcol@columnratioleft! can be made
% \!\let!-equial to \!\relax! by the related API macro \!\setcolumnwidth! so
% that \!\pcol@setcolumnwidth! knows which of specifications given by two
% API macros is effective and chooses \!\pcol@setcolwidth@r! or
% \!\pcol@setcolwidth@s!.
% 
%    \begin{macrocode}
%% Column Width Setting

\def\columnratio#1{\global\let\pcol@colwidthspecleft\relax
  \gdef\pcol@columnratioleft{#1}%
  \@ifnextchar[%]
    \pcol@icolumnratio{\gdef\pcol@columnratioright{#1}}}
\def\pcol@icolumnratio[#1]{\gdef\pcol@columnratioright{#1}}
\columnratio{}\relax

%    \end{macrocode}
% \end{macro}\end{macro}\end{macro}\end{macro}
% 
% \KeepSpace{2}
% \begin{macro}{\setcolumnwidth}
% \changes{v1.3-4}{2013/09/17}
%	{Introduced to specify column widths and column-separating gaps more
%	 detailedly.}
% \begin{macro}{\pcol@isetcolumnwidth}
% \changes{v1.3-4}{2013/09/17}
%	{Introduced to process the optional second argument of
%	 \cs{setcolumnwidth}.}
% \begin{macro}{\pcol@colwidthspecleft}
% \changes{v1.3-4}{2013/09/17}
%	{Introduced to keep column width specifications for left
%	 parallel-pages.}
% \begin{macro}{\pcol@colwidthspecright}
% \changes{v1.3-4}{2013/09/17}
%	{Introduced to keep column width specifications for right
%	 parallel-pages.}
% 
% The API macro $\!\setcolumnwidth!
% \Arg{s^l_0,s^l_1,\cdots,s^l_{k^l-1}}|[|s^r_0,s^r_1,\cdots,s^r_{k^r-1}|]|$
% defines the column width specification $s^l_c$ for column $c$ in left
% \parapag{}es and optionally $s^r_c$ for those in right \parapag{}es, where
% each specification $s^x_c$ has the form of $|[|w_c|][/[|g_c|]]|$ for width
% and gap specifier $\w_c$ and $\gap_c$.  After \!\let!ting
% $\!\pcol@columnratioleft!=\!\relax!$ to disable the setting by
% \!\columnratio! and to enable that done by this macro, the macro and its
% callee \!\pcol@isetcolumnwidth! just \!\global!y \!\def!ine macros
% \!\pcol@colwidthspecleft! and \!\pcol@colwidthspecright!  whose bodies have
% the first and second arguments respectively, or commonly have the first if
% the second optional one is not given, so that they are given to
% \!\pcol@setcolwidth@s! as its fourth argument through
% \!\pcol@setcolumnwidth! invoked in \!\pcol@zparacol!.  Both of
% \!\pcol@colwidthspecleft! and \!\pcol@colwidthspecright! are initially
% undefined because the default specification is given by
% \!\columnratio!|{}|.
% 
%    \begin{macrocode}
\def\setcolumnwidth#1{\global\let\pcol@columnratioleft\relax
  \gdef\pcol@colwidthspecleft{#1}%
  \@ifnextchar[%]
    \pcol@isetcolumnwidth{\gdef\pcol@colwidthspecright{#1}}}
\def\pcol@isetcolumnwidth[#1]{\gdef\pcol@colwidthspecright{#1}}

%    \end{macrocode}
% \end{macro}\end{macro}\end{macro}\end{macro}
% 
% \begin{macro}{\pcol@setcolumnwidth}
% \changes{v1.3-4}{2013/09/17}
%	{Move original functions to \cs{pcol@setcolumnwidth@r} and redefine
%	 this macro to switch \cs{pcol@setcolumnwidth@r} and
%	 \cs{pcol@setcolumnwidth@s}.}
% 
% \begin{Sloppy}{1500}
% The macro $\!\pcol@setcolumnwidth!\<\Cfrom\>\<\Cto\>\arg{ratio}\arg{spec}$
% is invoked solely from \!\pcol@zparacol! but can be twice, with;
% $$
% (\Cfrom,\Cto,\arg{ratio},\arg{spec})=
%   (0,\CL,\!\pcol@columnratioleft!,\!\pcol@colwidthspecleft!)
% $$
% always and with;
% $$
% (\Cfrom,\Cto,\arg{ratio},\arg{spec})=
%   (\CL,\C,\!\pcol@columnratioright!,\!\pcol@colwidthspecright!)
% $$
% if $\CL<\C$ for \parapag{}ing.  The macro simply invokes
% \!\pcol@setcolwidth@s! if $\!\pcol@columnratioleft!=\!\relax!$ because
% \!\setcolumnwidth! did so, or \!\pcol@setcolwidth@r! otherwise, with all
% arguments given by \!\pcol@zparacol!.
% \end{Sloppy}
% 
%    \begin{macrocode}
\def\pcol@setcolumnwidth{%
  \ifx\pcol@columnratioleft\relax \let\reserved@a\pcol@setcolwidth@s
  \else                           \let\reserved@a\pcol@setcolwidth@r
  \fi
  \reserved@a}

%    \end{macrocode}
% \end{macro}
% 
% \begin{macro}{\pcol@setcolwidth@r}
% \changes{v1.1}{2012/05/11}
% 	{Introduced to calculate $w_c$.}
% \changes{v1.3-4}{2013/09/17}
%	{Renamed from \cs{pcol@setcolumnwidth} to make it clear what the
%	 marco works for.}
% \changes{v1.3-2}{2013/09/17}
%	{Add arguments $C^0$, $C^1$, $\langle\string\mathit{ratio}\rangle$
%	 and $\langle\string\mathit{spec}\rangle$ for columns in right
%	 parallel-pages.}
% \changes{v1.3-3}{2013/09/17}
%	{Add setting of $g_c\EQ\cs{pcol@columnsep}{\cdot}c=\cs{columnsep}.}
% 
% \begin{Hfuzz}{0.3pt}%
% The macro $\!\pcol@setcolwidth@r!\<\Cfrom\>\<\Cto\>\arg{ratio}\arg{spec}$
% is invoked solely from \!\pcol@zparacol! through \!\pcol@setcolumnwidth!
% once or twice with the arguments described in the explanation of the
% latter macro.  The macro calculates $\w_c=|\pcol@columnwidth|{\cdot}c$
% 
% \SpecialArrayIndex{c}{\pcol@columnwidth}
% 
% for all $c\In\Cfrom\Cto$, from the fractions $r_0$, $r_1$, \ldots,
% $r_{k-1}$ given through the third argument $\arg{ratio}$, which was given
% to \!\columnratio! and then kept in \!\pcol@columnratioleft! or
% \!\pcol@columnratioright!.  The macro also lets
% $\gap_c=|\pcol@columnsep|{\cdot}c=\!\columnsep!$
% 
% \SpecialArrayIndex{c}{\pcol@columnsep}
% 
% for all $c\In\Cfrom\Cto$.
% \end{Hfuzz}
% 
% First, we calculate $W=\!\textwidth!-(\Cto-\Cfrom-1)\times\!\columnsep!$
% being the sum of $\w_c$ for all $c\In\Cfrom\Cto$.  Then we let
% $\w_c=r_d{}W$ and $\gap_c=\!\columnsep!$ for all $c\In\Cfrom{k'}$ where
% $k'=\min(k,\Cto-1)$, in the \!\@for! loop scanning $r_d$ for all
% $d=c-\Cfrom\In0k$.  Finally, we let
% $\w_c=(W-\sum_{d=\Cfrom}^{k'-1}w_d)/(\Cto-\Cfrom-k')$ and
% $\gap_c=\!\columnsep!$ for all $c\In{k'}\Cto$.  Note that
% $|\pcol@columnwidth|{\cdot}c$ and $|\pcol@columnsep|{\cdot}c$ are macros
% having the integer representations of $\w_c$ and $\gap_c$ with the unit
% |sp|.
% 
%    \begin{macrocode}
\def\pcol@setcolwidth@r#1#2#3#4{%
  \@tempcntb#2\advance\@tempcntb-#1\advance\@tempcntb\m@ne
  \@tempdima-\columnsep \multiply\@tempdima\@tempcntb
  \advance\@tempdima\textwidth \@tempdimb\@tempdima
  \@tempcnta#1\relax\@tempcntb#2\advance\@tempcntb\m@ne
  \@for\reserved@a:=#3\do{%
    \ifnum\@tempcnta<\@tempcntb
      \@tempdimc\reserved@a\@tempdima
      \expandafter\xdef\csname pcol@columnwidth\number\@tempcnta\endcsname{%
        \number\@tempdimc sp}%
      \global\@namedef{pcol@columnsep\number\@tempcnta}{\columnsep}%
      \advance\@tempdimb-\@tempdimc
      \advance\@tempcnta\@ne
    \fi}%
  \@tempcntb#2\advance\@tempcntb-\@tempcnta
  \divide\@tempdimb\@tempcntb
  \@whilenum\@tempcnta<#2\do{%
    \expandafter\xdef\csname pcol@columnwidth\number\@tempcnta\endcsname{%
      \number\@tempdimb sp}%
    \global\@namedef{pcol@columnsep\number\@tempcnta}{\columnsep}%
    \advance\@tempcnta\@ne}%
}

%    \end{macrocode}
% \end{macro}
% 
% \KeepSpace{3}
% \begin{macro}{\pcol@setcolwidth@s}
% \changes{v1.3-4}{2013/09/17}
%	{Introduced to calculate $w_c$ and $g_c$.}
% \begin{macro}{\pcol@setcw@c}
% \changes{v1.3-4}{2013/09/17}
%	{Introduced to process column width components in the argument
%	 $\langle\string\mathit{spec}\rangle$ of \cs{pcol@setcolwidth@s}.}
% \begin{macro}{\pcol@setcw@s}
% \changes{v1.3-4}{2013/09/17}
%	{Introduced to process column-separating gap components in the
%	 argument $\langle\string\mathit{spec}\rangle$ of
%	 \cs{pcol@setcolwidth@s}.}
% \begin{macro}{\pcol@setcw@filunit}
% \changes{v1.3-4}{2013/09/17}
%	{Introduced to define the unit of infinite stretch factors in the
%	 argument $\langle\string\mathit{spec}\rangle$ of
%	 \cs{pcol@setcolwidth@s}.}
% 
% \begin{Hfuzz}{0.3pt}%
% The macro $\!\pcol@setcolwidth@s!\<\Cfrom\>\<\Cto\>\arg{ratio}\arg{spec}$
% is invoked solely from \!\pcol@zparacol! through \!\pcol@setcolumnwidth!
% once or twice with the arguments described in the explanation of the
% latter macro.  The macro calculates $\w_c=|\pcol@columnwidth|{\cdot}c$
% 
% \SpecialArrayIndex{c}{\pcol@columnwidth}
% 
% for all $c\In\Cfrom\Cto$ and $\gap_c=|\pcol@columnsep|{\cdot}c$
% 
% \SpecialArrayIndex{c}{\pcol@columnsep}
% 
% for all $c\In\Cfrom{\Cto{-}1}$, from the column\slash gap specifications
% $s_0$, $s_1$, \ldots, $s_{k-1}$ given through the fourth argument
% $\arg{spec}$, which was given to \!\setcolumnwidth! and then kept in
% \!\pcol@colwidthspecleft! or \!\pcol@colwidthspecright!.
% \end{Hfuzz}
% 
% Each specification $s_d$ for $\w_c$ and $\gap_c$ where $c=\Cfrom+d$ has
% the form $|[|w'_d|][/[|g'_d|]]|$ to specify the natural width $w^n_d$ and
% $g^n_d$ and inifite stretch factor $w^f_d$ and $g^f_d$ of column\slash gap
% sepcfication as follows;
% 
% \begin{eqnarray*}
% w^n_d&=&\cases{0&$w'_d=\emptyset$\cr
%                0&$w'_d=f\!\fill!$\cr
%                \mathit{natural}(w'_d)&otherwise}\qquad
% w^f_d = \cases{1&$w'_d=\emptyset$\cr
%                f&$w'_d=f\!\fill!$\cr
%                \mathit{stretch}(w'_d)&otherwise}\\[1ex]
% g^n_d&=&\cases{\!\columnsep!&$g'_d=\emptyset$\cr
%                0&$g'_d=f\!\fill!$\cr
%                \mathit{natural}(g'_d)&otherwise}\qquad
% g^f_d = \cases{0&$g'_d=\emptyset$\cr
%                f&$g'_d=f\!\fill!$\cr
%                \mathit{stretch}(g'_d)&otherwise}
% \end{eqnarray*}
% 
% where $\mathit{natural}(x)$ is the natural width of the skip $x$ and
% $\mathit{stretch}(x)$ is the infinite stretch factor of $x$.  Note that
% any finite stretch factors or any shirnk factors do not affect them, and
% infinite strech units |fil|, |fill| and |filll| are not distinguished.
% From factors above, we determine $w_c$ and $\gap_c$ as follows;
% 
% \begin{eqnarray*}
% W&=&\sum_{d=0}^{m-2}(w^n_d+g^n_d)+w^n_{m-1}\\
% F&=&\sum_{d=0}^{m-2}(w^f_d+g^f_d)+w^f_{m-1}\\
% x_c&=&\cases{(\WT/W)x^n_{c-\Cfrom}&$W\geq\WT\;\lor\;F\leq0$\cr
%              x^n_{c-\Cfrom}+(x^f_{c-\Cfrom}/F)(\WT-W)&$W<\WT\;\land\;F>0$}
%     \qquad(x\in{\w,\gap})
% \end{eqnarray*}
% where $\WT=\!\textwidth!$ and $m=\Cto-\Cfrom$.
% 
% To perform the assignments above, the macro at first invoke
% $\!\pcol@setcw@scan!\<\Cfrom\>\<\Cto\>\~\ARg{spec}$ letting
% $\!\pcol@setcw@c!=\!\pcol@setcw@s!=\!\pcol@setcw@accumwd!$ and
% \!\pcol@setcw@filunit!${}=1$\,pt to scan $s_d$ for all $d\In0{m}$ and to
% accumulate $W+g^n_{m-1}$ and $F+g^f_{m-1}$ in \!\dimen@! and \!\dimen@ii!
% and then subtract $g^n_{m-1}$ and $g^f_{m-1}$ from them to have $W$ and
% $F$.  Note that $F$ is represented by a dimension with the unit of pt by
% the definition of \!\pcol@setcw@filunit!.  Then we invoke
% \!\pcol@setcw@calcfactors! to calculate $(\WT/W)=\!\pcol@setcw@scale!$ and
% $(\WT-W)/F=\!\@tempdimb!$.  Finally we scan $s_d$ again but in this case
% we let $\!\pcol@setcw@c!=\!\pcol@setcw@set!|{width}|$,
% $\!\pcol@setcw@s!=\!\pcol@setcw@set!|{sep}|$ and
% $\!\pcol@setcw@filunit!=\!\@tempdimb!=(\WT-W)/F$ to let $w_c$ and $\gap_c$
% have the values shown above.
% 
%    \begin{macrocode}
\def\pcol@setcolwidth@s#1#2#3#4{\begingroup
  \dimen@\z@ \dimen@ii\z@ \def\pcol@setcw@filunit{\@ne\p@}%
  \let\pcol@setcw@c\pcol@setcw@accumwd \let\pcol@setcw@s\pcol@setcw@accumwd
  \pcol@setcw@scan#1#2{#4}%
  \advance\dimen@-\@tempdima \advance\dimen@ii-\@tempdimb
  \pcol@setcw@calcfactors
  \def\pcol@setcw@c{\pcol@setcw@set{width}}%
  \def\pcol@setcw@s{\pcol@setcw@set{sep}}%
  \let\pcol@setcw@filunit\dimen@ii
  \pcol@setcw@scan#1#2{#4}%
  \endgroup}
%    \end{macrocode}
% \end{macro}\end{macro}\end{macro}\end{macro}
% 
% \begin{macro}{\pcol@setcw@scan}
% \changes{v1.3-4}{2013/09/17}
%	{Introduced to scan the argument
%	 $\langle\string\mathit{spec}\rangle$ of \cs{pcol@setcolwidth@s}.}
% 
% The macro $\!\pcol@setcw@scan!\<\Cfrom\>\<\Cto\>\~\ARg{spec}$ is to scan
% first $m=\Cto-\Cfrom$ elements in $\arg{spec}=s_0,s_1,\ldots$ being the
% column\slash gap specifications given to \!\setcolumnwidth!.  At first we
% add `,' as many as $m$ to the tail of $\arg{spec}$ to make it sure the
% resulting $\arg{spec}$ has $m$ or more elements.  Then we scan all
% elements in the extended $\arg{spec}$ by a \!\@for! loop having many
% \!\expandafter! but equivalent to;
% 
% \begin{quote}
% |\@for\reserved@a:=|$s_0,s_1,\ldots$|\do{|\textit{body}|}|
% \end{quote}
% 
% In the \textit{body} above, we invoke
% $\!\pcol@setcw@getspec!\;s_i|//|\!\@nil!$ to parse $s_i$ to have $w^n_i$,
% $w^f_i$ to be processed by \!\pcol@setcw@c! and $g^n_i$ and $g^f_i$ by
% \!\pcol@setcw@s!, for all $i\In0m$.
% 
%    \begin{macrocode}
\def\pcol@setcw@scan#1#2#3{\def\reserved@a{#3}%
  \@tempcnta#1\relax \@whilenum\@tempcnta<#2\do{
    \edef\reserved@a{\reserved@a,}\advance\@tempcnta\@ne}%
  \@tempcnta#1\relax
  \expandafter\@for\expandafter\reserved@a\expandafter:\expandafter=\reserved@a
    \do{%
      \ifnum\@tempcnta<#2\relax
        \expandafter\pcol@setcw@getspec\reserved@a//\@nil
      \fi
      \advance\@tempcnta\@ne}}
%    \end{macrocode}
% \end{macro}
% 
% \KeepSpace{1}
% \begin{macro}{\pcol@setcw@getspec}
% \changes{v1.3-4}{2013/09/17}
%	{Introduced to parse an element in the argument
%	 $\langle\string\mathit{spec}\rangle$ of \cs{pcol@setcolwidth@s}.}
% \begin{macro}{\pcol@setcw@getspec@i}
% \changes{v1.3-4}{2013/09/17}
%	{Introduced to parse an element in the argument
%	 $\langle\string\mathit{spec}\rangle$ of \cs{pcol@setcolwidth@s}.}
% \begin{macro}{\pcol@setcw@fill}
% \changes{v1.3-4}{2013/09/17}
%	{Introduced to extract \cs{fill} factor from an element in the
%	 argument  $\langle\string\mathit{spec}\rangle$ of
%	 \cs{pcol@setcolwidth@s}.}
% 
% The macro
% $\!\pcol@setcw@getspec!\<w'_d\>|/|\<g'_d\>|/|\arg{garbage}\!\@nil!$ is
% used solely in \!\pcol@setcw@scan! to parse a column\slash gap
% specification $s_d=|[|w'_d|][/[|g'_d|]]|$, to extract factors $w_d^n$,
% $w_d^f$, $g_d^n$ and $g_d^f$, and to process width factors by
% \!\pcol@setcw@c! and gap factors by \!\pcol@setcw@s!.  Since the macro is
% invoked with arguments in the form of $s_d|//|\!\@nil!$, if $s_d$ has
% `|/|' in it $w'_d$ and $g'_d$ should have everything preceeding and
% following the `|/|' respectively while $\arg{garbage}$ should have the
% redundant `|/|'.   Otherwise, i.e., if $s_d$ does not have `|/|',
% $w'_d=s_d$ and $g'_d=\emptyset$ while $\arg{garbage}=\emptyset$\footnote{
% 
% If $s_d$ have two or more `|/|', everything following the second one is
% thrown away into $\arg{garbage}$ together with `|//|'.  Therefore we could
% check if $s_d$ has at most one `|/|' by examining $\arg{garbage}$ but we
% abandon it simply ingoring $\arg{garbage}$.}.
% 
% Therefore, we invoke $\!\pcol@setcw@getspec@i!\arg{default}\Arg{x'_d}$ twice
% with $(\arg{default},x'_d)=(\!\fill!,w'_d)$ for a column and
% $(\arg{default},x'_d)=(\!\columnsep!,g'_d)$ for a gap, and \!\pcol@setcw@c!
% and \!\pcol@setcw@s! after each invocation respectively.
% 
% In this macro, at first we scan all tokens in $x'_d$ by \!\@tfor! to
% remove all space tokens in it\footnote{
% 
% A proper skip specification and ``$f\!\fill!$'' is always proper without
% space tokens in them.}.
% 
% Then if $x'_d$ after the space removal has nothing, we let
% $x'_d=\arg{default}$.  Next we examine if $x'_d=f\!\fill!$ in a trickey
% way by temporarily \!\let!ting $\!\@gtempa!=\!\relax!$, defining \!\fill!
% as ``|1pt|\!\gdef!\!\@gtempa!|{}|'' and making an assignment
% ``\!\@tempskipa!  $x'_d$''.  That is, if $x'_d=f\!\fill!$, \!\@tempskipa!
% will have $f{\cdot}|1pt|$ being a proper dimension and \!\@gtempa! is made
% empty.  Otherwise, \!\@tempskipa! should have $x'_d$ which must be a
% proper skip and \!\@gtempa! remains unchanged from \!\relax!.  Therefore,
% if $\!\@gtempa!=\!\relax!$ we let $\!\@tempskipa!=x'_d$ again\footnote{
% 
% Because the first assignment is done in a group.}.
% 
% Otherwise, we invoke $\!\pcol@setcw@fill!\;x'_d$ to let
% \!\@tempskipa! have $0\,|pt|\ |plus|\ f\,|fil|$ where $f$ is replaced by 1
% if $f=\emptyset$.
% 
% Now \!\@tempskipa! has $x_d^n$ as its natural component and may have some
% infinite strectch component $x_d^f$ specified explicitly or with \!\fill!.
% Therefore, we assign \!\@tempskipa! to \!\@tempdima! so that it has
% $x_d^n$, and then, after adding 0\,|pt| |plus| 1000\,|pt| |minus|
% 1000\,|pt| to \!\@tempskipa! to make it sure it has both strech and shrink
% components\footnote{
% 
% Almost sure becaues they could be $-1000$\,pt, but we ignore the
% possibility.}
% 
% keeping infinite stretch factor if any, invoke \!\pcol@extract@fil! giving
% it \!\the!-expansion of \!\@tempskipa! as the argument to let
% $\!\@tempdimb!=x_d^f\times\!\pcol@setcw@filunit!$.
% 
%    \begin{macrocode}
\def\pcol@setcw@getspec#1/#2/#3\@nil{%
  \pcol@setcw@getspec@i\fill{#1}\pcol@setcw@c
  \pcol@setcw@getspec@i\columnsep{#2}\pcol@setcw@s}
\def\pcol@setcw@getspec@i#1#2{%
  \def\reserved@a{}%
  \@tfor\reserved@b:=#2\do{\edef\reserved@a{\reserved@a\reserved@b}}
  \ifx\reserved@a\@empty \let\reserved@a#1\fi
  \let\@gtempa\relax
  {\def\fill{1\p@\gdef\@gtempa{}}\@tempskipa\reserved@a}%
  \ifx\@gtempa\relax \@tempskipa\reserved@a\relax
  \else \expandafter\pcol@setcw@fill\reserved@a
  \fi
  \@tempdima\@tempskipa
  \advance\@tempskipa0\p@\@plus\@m\p@\@minus\@m\p@\relax
  \expandafter\pcol@extract@fil\the\@tempskipa\@nil}
\def\pcol@setcw@fill#1\fill{\def\reserved@b{#1}%
  \ifx\reserved@b\@empty \let\reserved@b\@ne \fi
  \@tempskipa0\p@\@plus\reserved@b fil\relax}

%    \end{macrocode}
% \end{macro}\end{macro}\end{macro}
% 
% \begin{macro}{\pcol@setcw@accumwd}
% \changes{v1.3-4}{2013/09/17}
%	{Introduced to accumulate natural and fill factors of the element in
%	 the argument $\langle\string\mathit{spec}\rangle$ of
%	 \cs{pcol@setcolwidth@s}.} 
% 
% The macro \!\pcol@setcw@accumwd! is made \!\let!-equal to \!\pcol@setcw@c!
% and \!\pcol@setcw@s! commonly and thus invoked from \!\pcol@setcw@getspec!
% with setting $\!\@tempdima!=x_d^n$ and $\!\@tempdimb!=x_d^f\times1\,|pt|$,
% where $x\in\{w,g\}$.  The macro simply add them to \!\dimen@! and
% \!\dimen@ii! repsectively to accumurate $x_d^n$ and $x_d^f$ in them.
% 
%    \begin{macrocode}
\def\pcol@setcw@accumwd{\advance\dimen@\@tempdima \advance\dimen@ii\@tempdimb}
%    \end{macrocode}
% \end{macro}
% 
% \begin{macro}{\pcol@setcw@set}
% \changes{v1.3-4}{2013/09/17}
%	{Introduced to define $w_c$ and $g_c$ according to the element in
%	 the argument $\langle\string\mathit{spec}\rangle$ of
%	 \cs{pcol@setcolwidth@s}.} 
% 
% The macro $\!\pcol@setcw@set!\ARg{wors}$ is made the body of
% \!\pcol@setcw@c! with $\arg{wors}=|width|$ and of
% \!\pcol@setcw@s! with $\arg{wors}=|sep|$ and thus invoked from
% \!\pcol@setcw@getspec!  with setting $\!\@tempdima!=x_d^n$ and
% $$
% (\!\pcol@setcw@scale!,\!\@tempdimb!)
% \in\{(\emptyset,\,(x_d^f/F)(\WT-W)),\;(\WT/W,\,0)\}
% $$
% Therefore, we calculate
% $x_c=\!\pcol@setcw@scale!\times\!\@tempdima!+\!\@tempdimb!$ and
% \!\xdef!ine $|\pcol@column|{\cdot}\<wors\>{\cdot}c$ to let it have the
% integer representation of $x_c$ with the unit |sp|.
% 
% \SpecialArrayIndex{c}{\pcol@columnwidth}
% \SpecialArrayIndex{c}{\pcol@columnsep}
% 
%    \begin{macrocode}
\def\pcol@setcw@set#1{%
  \@tempdima\pcol@setcw@scale\@tempdima \advance\@tempdima\@tempdimb
  \expandafter\xdef\csname pcol@column#1\number\@tempcnta\endcsname{%
    \number\@tempdima sp}}

%    \end{macrocode}
% \end{macro}
% 
% \begin{macro}{\pcol@setcw@calcfactors}
% \changes{v1.3-4}{2013/09/17}
%	{Introduced to caclulate scaling and stretch factors from
%	 the argument $\langle\string\mathit{spec}\rangle$ of
%	 \cs{pcol@setcolwidth@s}.} 
% \begin{macro}{\pcol@setcw@calcf}
% \changes{v1.3-4}{2013/09/17}
%	{Introduced to caclulate scaling or stretch factor from
%	 the argument $\langle\string\mathit{spec}\rangle$ of
%	 \cs{pcol@setcolwidth@s}.} 
% \changes{v1.31}{2013/10/10}
%	{Capitalize the first word of the error message for consistency.}
% \begin{macro}{\pcol@setcw@scale}
% \changes{v1.3-4}{2013/09/17}
%	{Introduced to have the scaling factor of $w_c$ and $g_c$.}
% 
% The macro \!\pcol@setcw@calcfactors! is used solely in
% \!\pcol@setcolwidth@s! to calculate $\phi_s=\!\pcol@setcw@scale!$ and
% $\phi_f=\!\dimen@ii!$ as follows
% $$
% (\phi_s,\phi_f)=\cases{(\WT/\W,\;0)& $W\geq\WT\;\lor\;F\leq0$\cr
%                        (1,\;(\WT-W)/F)&$W<\WT\;\land\;F>0$}
% $$
% where $W=\!\dimen@!$, $F\times1\,|pt|=\!\dimen@ii!$ and $\WT=\!\textwidth!$,
% and $\phi_s=1$ is represented by empty body of \!\pcol@setcw@scale!.
% First we deal with the special and trivial case of $W=\WT$ to let
% $\phi_s=1$ and $\phi_f=0$ so as to avoid arithmetic error in the
% calculation of $\WT/W$.  If $W\neq\WT$ on the other hand, we calculate
% $\WT/W$ by $\!\pcol@setcw@calcf!\<\WT\>\<W\>\<\phi_s\>$ to have a
% provisional result.  Then if $W<\WT$ and $F>0$, we let $\phi_s=1$ and
% invoke \!\pcol@setcw@calcf! again giving it $(\WT-W)$ and $F\times1\,|pt|$
% but throw away the result $(\WT-W)/(F\times1\,|pt|)$ because \!\@tempdimb!
% should have $(\WT-W)/F$ which is then set into $\phi_f=\!\dimen@ii!$.
% Otherwise, we keep the provisional result of $\phi_s$ as the final one and
% let $\phi_f=\!\dimen@ii!=0$.
% 
% The macro $\!\pcol@setcw@calcf!\<x\>\<y\>\<z\>$ caclulates $z\approx x/y$
% and let $\!\@tempdimb!=Z=z\times1\,|pt|$ as follows.  First we find the
% following three parameters.
% 
% \begin{eqnarray*}
% k_1&=&\min\Set{k}{k\geq0,\;x\cdot2^k\geq2^{13}\,|pt|}\\
% k_2&=&\max\Set{k}{y\bmod2^k=0}\\
% k_3&=&\min\Set{k}{k\geq0,\;\lceil y/2^{k_2+k}\rceil\leq2^{15}}
% \end{eqnarray*}
% 
% With these parameters, we calculate $z'=\lfloor(x\cdot2^{k_1})/\lceil
% y/2^{k_2+k_3}\rceil\rfloor$ to have a good approximation of
% $(x/y)\cdot2^{k}$ where $k=k_1+k_2+k_3$ without arithmetic overflow.  Then
% if $z'/2^k\geq2^{14}$ or in other words $Z$ is larger than \!\maxdimen!,
% we complain that by \!\PackageError! and, in case a user dare to
% continue the typesetting process, we let $Z=10000\,|pt|$.  Otherwise, we
% calculate $Z=(z'/2^k)\cdot2^{16}=z'\cdot2^{16-k}$ to have it in
% \!\@tempdimb! by $Z=z'\times2^{16-k}$ if $k<16$, or by $Z=z'/2^{k-16}$
% otherwise.  Finally we invoke \!\pcol@extract@pt! giving it
% \!\the!-representaion of $Z$ to have $z$.
% 
% Note that it is assured $z\leq x/y$ regardless of successfulness of the
% calculation and thus the scaling $\phi_s x_d^n$ and strething
% $x_d^n+\phi_f x_d^f$ cannot exceed their exact value to make it also sure
% that $\sum_{c=\Cfrom}^{\Cto-2}(\w_c+\gap_c)+\w_{\Cto-1}\leq\WT$ and thus
% the series of columns and \csepgap{}s should not cause overfull when a
% page is shipped out with \!\hfil! added to each \csepgap{} for underfull
% avoidance.
% 
%    \begin{macrocode}
\def\pcol@setcw@calcfactors{%
  \ifdim\dimen@=\textwidth \def\pcol@setcw@scale{}\dimen@ii\z@
  \else
    \pcol@setcw@calcf\textwidth\dimen@\pcol@setcw@scale
    \ifdim\dimen@<\textwidth \ifdim\dimen@ii>\z@
      \def\pcol@setcw@scale{}%
      \@tempdimc\textwidth \advance\@tempdimc-\dimen@
      \pcol@setcw@calcf\@tempdimc\dimen@ii\reserved@a \dimen@ii\@tempdimb
    \else \dimen@ii\z@ \fi
    \else \dimen@ii\z@ \fi
  \fi}
\def\pcol@setcw@calcf#1#2#3{%
  \@tempdimb#1\@tempdima#2\@tempcnta\z@
  \ifdim\@tempdima=\z@ \@tempdima1sp\relax\fi
  \@whiledim\@tempdimb<8192\p@\do{%
    \multiply\@tempdimb\tw@ \advance\@tempcnta\@ne}%
  \@tempdimc\@tempdima
  \@whiledim\@tempdima=\@tempdimc\do{%
    \divide\@tempdimc\tw@ \multiply\@tempdimc\tw@
    \ifdim\@tempdima=\@tempdimc
      \divide\@tempdima\tw@ \divide\@tempdimc\tw@ \advance\@tempcnta\@ne
    \fi}
  \advance\@tempdima-1sp\relax
  \@whilenum\@tempdima>32768\do{\divide\@tempdima\tw@ \advance\@tempcnta\@ne}%
  \advance\@tempdima1sp\relax
  \divide\@tempdimb\@tempdima \@tempdimc\@tempdimb \@tempcntb\@tempcnta
  \@whilenum\@tempcntb>\z@\do{\divide\@tempdimc\tw@ \advance\@tempcntb\m@ne}
  \ifnum\@tempdimc>16383\relax
    \PackageError{%
      Scaling/filling factor for column/gap width is too large.}\@eha
    \@tempdimb\@M\p@
  \else
    \@tempcntb\sixt@@n \advance\@tempcntb-\@tempcnta
    \ifnum\@tempcntb<\z@
      \@whilenum\@tempcntb<\z@\do{\divide\@tempdimb\tw@ \advance\@tempcntb\@ne}%
    \else
      \@whilenum\@tempcntb>\z@\do{%
        \multiply\@tempdimb\tw@ \advance\@tempcntb\m@ne}%
    \fi
  \fi
  \expandafter\pcol@extract@pt\the\@tempdimb#3}

%    \end{macrocode}
% \end{macro}\end{macro}\end{macro}
% 
% \KeepSpace{3}
% \begin{macro}{\pcol@defkw}
% \changes{v1.3-4}{2013/09/17}
%	{Introduced to define $\cs{pcol@kw@}{\cdot}k$ where
%	 $k\in\{\string\texttt{pt},\string\texttt{plus},\string\texttt{minus},
% 	 \string\texttt{fil}\}$.}
% \begin{macro}{\pcol@kw@pt}
% \changes{v1.3-4}{2013/09/17}
%	{Introduced to have the keyword \string\texttt{pt}.}
% \begin{macro}{\pcol@kw@plus}
% \changes{v1.3-4}{2013/09/17}
%	{Introduced to have the keyword \string\texttt{plus}.}
% \begin{macro}{\pcol@kw@minus}
% \changes{v1.3-4}{2013/09/17}
%	{Introduced to have the keyword \string\texttt{minus}.}
% \begin{macro}{\pcol@kw@fil}
% \changes{v1.3-4}{2013/09/17}
%	{Introduced to have the keyword \string\texttt{fil}.}
% 
% The macro
% $\!\pcol@defkw!|1.0|\arg{pt}\verb*| |\arg{plus}\verb*| 1.0|\arg{fil}
% \verb*| |\arg{minus}\verb*| 1.0|\arg{garbage}\!\@nil!$ is used just once at
% the top level to \!\def!ine \!\pcol@kw@pt!, \!\pcol@kw@plus!,
% \!\pcol@kw@minus! and \!\pcol@kw@fil! letting them have $\arg{pt}=|pt|$,
% $\arg{plus}=|plus|$, $\arg{minus}=|minus|$ and $\arg{fil}=|fil|$ in their
% body respectively but with $\!\catcode!=12$ (other) which is used in
% \!\the!-representation of glues.  For the definition, we invoke
% \!\pcol@defkw! giving it \!\the!-representation of \!\@tempskipa! letting
% it have $1\,|pt|\ |plus|\ 1\,|fil|\ |minus|\ 1\,|fil|$ having all keywards
% we need to have\footnote{
% 
% We can do what \!\pcol@defkw! does by temporarily giving $\!\catcode!=12$
% to the characters for the keywords of course, but this method is much
% easier.}.
% 
% The macro \!\pcol@kw@pt! is used in
% $\!\pcol@extract@fil@ii!\arg{unit}\!\@nil!$ to examine if
% $\arg{unit}=|pt|$, and in \!\pcol@def@extract@pt! to \!\def!ine
% \!\pcol@extract@pt! having |pt| in its argument specification.  The macros
% \!\pcol@kw@plus! and \!\pcol@kw@minus! are used only in
% \!\pcol@def@extract@fil!, and
% \!\pcol@kw@fil! only in \!\pcol@def@extract@fil@iii!, to \!\def!ine
% \!\pcol@extract@fil! having |plus| and |minus|, and
% \!\pcol@extract@fil@iii! having |fil|, in their argument specifications
% respectively.
% 
%    \begin{macrocode}
\@tempskipa 1\p@\@plus1fil\@minus1fil\relax
\def\pcol@defkw1.0#1 #2 1.0#3 #4 1.0#5\@nil{%
  \def\pcol@kw@pt{#1}\def\pcol@kw@plus{#2}\def\pcol@kw@fil{#3}%
  \def\pcol@kw@minus{#4}}
\expandafter\pcol@defkw\the\@tempskipa\@nil

%    \end{macrocode}
% \end{macro}\end{macro}\end{macro}\end{macro}\end{macro}
% 
% \KeepSpace{4}
% \begin{macro}{\pcol@def@extract@fil}
% \changes{v1.3-4}{2013/09/17}
%	{Introduced to define \cs{pcol@extract@fil}.}
% \begin{macro}{\pcol@extract@fil}
% \changes{v1.3-4}{2013/09/17}
%	{Introduced to extract infinite stretch factor from a skip if any.}
% \begin{macro}{\pcol@extract@fil@i}
% \changes{v1.3-4}{2013/09/17}
%	{Introduced to extract the factor of \string\texttt{fil} from the
% 	 stretch factor in a skip if any.} 
% \begin{macro}{\pcol@extract@fil@ii}
% \changes{v1.3-4}{2013/09/17}
%	{Introduced to extract the factor of \string\texttt{fil} from the
% 	 stretch factor in a skip if any.} 
% \begin{macro}{\pcol@def@extract@fil@iii}
% \changes{v1.3-4}{2013/09/17}
%	{Introduced to define \cs{pcol@extract@fil@iii}.}
% \begin{macro}{\pcol@extract@fil@iii}
% \changes{v1.3-4}{2013/09/17}
%	{Introduced to extract the factor of \string\texttt{fil} from the
% 	 stretch factor in a skip.} 
% 
% The macro
% $\!\pcol@extract@fil!\<\mathit{garbge}_1\>\verb*| plus |\<s\>\verb*| minus|
% \<\mathit{garbage}_2\>\!\@nil!$ is used solely in \!\pcol@setcw@getspec@i!
% to extract the infinite stretch factor $f$ in the stretch component $s$ of
% a column\slash gap specification $x'_d$ and to let
% $\!\@tempdimb!=f\cdot{}u$ where
% $u=\!\pcol@setcw@filunit!\in\{1\,|pt|,\;\phi_f=\!\dimen@ii!\}$ if $f$
% exist or $\!\@tempdimb!=0$ otherwise.  First of all, since the macro has
% keywords |plus| and |minus| in $\!\catcode!=12$ in its argument
% specification, we \!\def!ine it using \!\pcol@def@extract@fil!, whose body
% is equivalent to
% 
% \begin{itemize}\item[]
% \let\@sverb\latex@sverb
% \verb*|\def\pcol@extract@fil#1 plus #2 minus#3\@nil|
% |{\pcol@extract@fil@i#2\@nil}|
% \end{itemize}
% 
% just once at the top level.  Then since $s$ should have the form
% $\<n\>|.|\<m\>\arg{unit}$ where $n$ and $m$ are decimal digit sequences and
% $\arg{unit}\in\{|pt|,|fil|,|fill|,|filll|\}$, we examine if
% $\arg{unit}=|pt|$ or not by a trickey way in
% $\!\pcol@extract@fil@i!\<n\>|.|\<m{\cdot}\mathit{unit}\>\!\@nil!$.  That
% is, we do $\!\count@!\<m{\cdot}\mathit{unit}\>\!\@nil!$ with
% \!\afterassignment! to invoke $\!\pcol@extract@fil@ii!\arg{unit}\!\@nil!$
% after $m$ is assigned to \!\count@! to capture $\arg{unit}$.  Then if it
% is |pt| we let $\!\@tempdimb!=0$, or otherwise invoke
% $\!\pcol@extract@fil@iii!\<f\>|fil|\arg{garbage}\!\@nil!$ giving it $s$
% because it should have a postfix being |fil|, |fill| or |filll|, to have
% $\!\@tempdimb!=f\cdot u$ finally.  Note that since
% \!\pcol@extract@fil@iii!  also has the keyword |fil| in its argument
% specification, we \!\def!ine it using \!\pcol@def@extract@fil@iii!, whose
% body is equivalent to
% 
% \begin{quote}
% \let\@sverb\latex@sverb
% \verb*|\def\pcol@extract@fil@iii#1fil#2\@nil{%|\\
% |    \@tempdimb\pcol@setcw@filunit\relax \@tempdimb#1\@tempdimb}|
% \end{quote}
% 
% just once at the top level too.
% 
%    \begin{macrocode}
\edef\pcol@def@extract@fil{%
  \def\noexpand\pcol@extract@fil
  ##1\space\pcol@kw@plus\space##2\space\pcol@kw@minus##3\noexpand\@nil{%
    \noexpand\pcol@extract@fil@i##2\noexpand\@nil}}
\pcol@def@extract@fil
\def\pcol@extract@fil@i#1.#2\@nil{\def\reserved@a{#1.#2}%
  \afterassignment\pcol@extract@fil@ii\count@#2\@nil}
\def\pcol@extract@fil@ii#1\@nil{\def\reserved@b{#1}%
  \ifx\reserved@b\pcol@kw@pt \@tempdimb\z@
  \else \expandafter\pcol@extract@fil@iii\reserved@a\@nil
  \fi}
\edef\pcol@def@extract@fil@iii{%
  \def\noexpand\pcol@extract@fil@iii##1\pcol@kw@fil##2\noexpand\@nil{%
    \@tempdimb\noexpand\pcol@setcw@filunit\relax \@tempdimb##1\@tempdimb}}
\pcol@def@extract@fil@iii

%    \end{macrocode}
% \end{macro}\end{macro}\end{macro}\end{macro}\end{macro}\end{macro}
% 
% \begin{macro}{\pcol@def@extract@pt}
% \changes{v1.3-4}{2013/09/17}
%	{Introduced to define \cs{pcol@extract@pt}.}
% \begin{macro}{\pcol@extract@pt}
% \changes{v1.3-4}{2013/09/17}
%	{Introduced to extract the factor of \string\texttt{pt} from the
%	 \cs{the} representaion of a dimension.}
% 
% The macro $\!\pcol@extract@pt!\<f\>|pt|\<scale\>$ is solely used in
% \!\pcol@setcw@calcf! to extract $f$ from a dimension in the form of
% $f|pt|$ and to let the macro $\<scale\>$ have $f$.  Since this macro has
% the keyword |pt| in its argument specification, we \!\def!ine it using
% \!\pcol@def@extract@pt!, whose body is equivalent to
% 
% \begin{quote}
% \let\@sverb\latex@sverb
% |\def\pcol@extract@pt#1pt#2{\def#2{#1}}|
% \end{quote}
% just once at the top level again.
% 
%    \begin{macrocode}
\edef\pcol@def@extract@pt{%
  \def\noexpand\pcol@extract@pt##1\pcol@kw@pt##2{\def##2{##1}}}
\pcol@def@extract@pt

%    \end{macrocode}
% \end{macro}\end{macro}
% 
% 
% 
% \section{Counter Operations}
% \label{sec:imp-counter}
% 
% \begin{macro}{\globalcounter}
% \changes{v1.2-7}{2013/05/11}
% 	{Examine if the argument counter is already in
%	 $\string\mathit{\Theta}^g$ to avoid the duplicatoin in the list
%	 which caused a bug.}
% \changes{v1.32-1}{2015/10/10}
% 	{Modified according to the introduction of \cs{globalcounter*}.}
% \begin{macro}{\pcol@globalcounter@s}
% \changes{v1.32-1}{2015/10/10}
% 	{Added for \cs{globalcounter*}.}
% \begin{macro}{\pcol@globalcounter}
% \changes{v1.32-1}{2015/10/10}
% 	{Renamed from \cs{globalcounter} according to the introduction of
%	 \cs{globalcounter*}.}
% \begin{macro}{\pcol@gcounters}
% The API macro $\!\globalcounter!\ARg{ctr}$, implemented by
% \!\pcol@globalcounter! and also used in \!\pcol@fnlayout@p! to globalize the
% counter \counter{footnote}, defines that $\arg{ctr}$ is a \gcounter, and
% thus adds it to $\CG=\Midx{\!\pcol@gcounters!}$, which has \counter{page}
% at initial.  Note that we examines if $\arg{ctr}\in\CG$ prior to the
% addition to avoid the duplication in $\CG$.  Also note that initial
% definition of \!\pcol@gcounters! is done by \!\gdef! just for consistent
% \!\global!  assignments to it.  On the other hand \!\globalcounter!|*|,
% implemented by \!\pcol@globalcounter@s!, makes all counters kept in
% \!\cl@@ckpt! global by letting \!\pcol@gcounters! have the list.
% Switching these two functionality is done by \!\globalcounter! examining
% if it is followed by a |*| by \!\@ifstar!.
% 
%    \begin{macrocode}
%% Counter Operations

\def\globalcounter{\@ifstar\pcol@globalcounter@s\pcol@globalcounter}
\def\pcol@globalcounter@s{\global\let\pcol@gcounters\cl@@ckpt}
\def\pcol@globalcounter#1{{%
  \@tempswafalse \def\reserved@a{#1}%
  \def\@elt##1{\def\reserved@b{##1}%
    \ifx\reserved@a\reserved@b \@tempswatrue \fi}%
  \pcol@gcounters
  \if@tempswa\else \@cons\pcol@gcounters{{#1}}\fi}}
\gdef\pcol@gcounters{\@elt{page}}
%    \end{macrocode}
% \end{macro}\end{macro}\end{macro}\end{macro}
% 
% \begin{macro}{\localcounter}
% \changes{v1.2-6}{2013/05/11}
%	{Introduced to remove the argument counter from
%	 $\string\mathit{\Theta}^g$.}
% 
% The API macro \!\localcounter!\marg{ctr}, also used in
% \!\pcol@fnlayout@c! to localize the counter \counter{footnote},
% declares that $\arg{ctr}$ is a \lcounter, and thus removes it from $\CG$
% by \!\pcol@removecounter! if $\arg{ctr}\neq\counter{page}$.
% 
%    \begin{macrocode}
\def\localcounter#1{%
  \expandafter\ifx\csname c@#1\endcsname\c@page\else
    \pcol@removecounter\pcol@gcounters{#1}%
  \fi}
%    \end{macrocode}
% \end{macro}
% 
% \KeepSpace{1}
% \begin{macro}{\pcol@remctrelt}
% \changes{v1.2-6}{2013/05/11}
%	{Recode to use newly introduced \cs{pcol@removecounter}.}
% \begin{macro}{\pcol@removecounter}
% \changes{v1.2-6}{2013/05/11}
%	{Introduced for the counter removal operation in
%	 \cs{localcounter} and \cs{pcol@remctrelt}.}
% \begin{macro}{\pcol@iremctrelt}
% \changes{v1.2-6}{2013/05/11}
%	{Add the first argument $\string\mathit{\Theta}'$ as the counter
%	 list from which the second argument $\theta$ is removed.}
% 
% The macro $\!\pcol@remctrelt!\Arg\cg$ is invoked solely from
% \!\pcol@zparacol!  and is applied to each \gcounter{} $\cg\in\CG$ to
% remove it from $\CC=\!\pcol@counters!$ in which we have $\CTL$ finally.
% The macro also moves $|\cl@|{\cdot}\cg=\clist(\cg)$
% 
% \SpecialArrayIndex{\theta}{\cl@}
% 
% to $|\pcol@cl@|{\cdot}\cg$
% 
% \SpecialArrayMainIndex{\theta}{\pcol@cl@}
% 
% to keep the list of the descendant \lcounter{}s of $\cg$ in it, and then
% re\!\def!ines $|\cl@|{\cdot}\cg=\!\pcol@stepcounter!\Arg\cg$
% 
% \SpecialArrayMainIndex{\theta}{\cl@}
% 
% so that it is invoked on $\!\stepcounter!\Arg\cg$ to let $\val_c(\cl)=0$
% for all $c\In0\C$ and $\cl\in\clist(\cg)$, if $\cg\neq\counter{page}$.
% These operations are performed by a lengthy sequence with many occurrences
% of \!\expandafter!, \!\csname! and \!\endcsname! but the sequence is
% equivalent to the following.
% 
% \begin{quote}
% $\!\let!|\pcol@cl@|{\cdot}\cg|=\cl@|{\cdot}\cg$\\
% $\CSIndex{ifx}|\c@|{\cdot}\cg\!\c@page!|\else |
%     |\def\cl@|{\cdot}\cg|{|\!\pcol@stepcounter!|{|\cg|}}\fi|$
% \end{quote}
% 
% As for the removal of $\cg$ from $\CC$, we invoke
% $\!\pcol@removecounter!\<\mathit{\Theta}'\>\Arg{\theta}$ giving it
% $\mathit{\Theta}'=\CC$ and $\theta=\cg$.  This macro, also invoked from
% $\!\localcounter!\Arg\cl$ with $\mathit{\Theta}'=\CG$ and $\theta=\cl$,
% does $\mathit{\Theta}''\gets\mathit{\Theta}'$,
% $\mathit{\Theta}'=\emptyset$, and then apply
% $\!\pcol@iremctrelt!\arg{\mathit{\Theta}'}$ to each
% $\theta'\in\mathit{\Theta}''$ to let
% $\mathit{\Theta}'\gets\mathit{\Theta}'\cup\{\theta'\}$ by \!\@cons! if
% $\theta'\neq\theta$.
% 
%    \begin{macrocode}
\def\pcol@remctrelt#1{%
  \expandafter\let\expandafter\reserved@a\csname cl@#1\endcsname
  \expandafter\let\csname pcol@cl@#1\endcsname\reserved@a
  \expandafter\ifx\csname c@#1\endcsname\c@page\else
    \@namedef{cl@#1}{\pcol@stepcounter{#1}}%
  \fi
  \pcol@removecounter\pcol@counters{#1}}
\def\pcol@removecounter#1#2{%
  \def\reserved@a{#2}\let\reserved@b#1\relax \global\let#1\@empty
  {\def\@elt{\pcol@iremctrelt#1}\reserved@b}}
\def\pcol@iremctrelt#1#2{%
  \def\reserved@b{#2}%
  \ifx\reserved@a\reserved@b\else \@cons#1{{#2}}\fi}

%    \end{macrocode}
% \end{macro}\end{macro}\end{macro}
% 
% \begin{macro}{\definethecounter}
% The API macro $\!\definethecounter!\<\cl\>\<c\>\arg{rep}$ define the
% \lrep{} $\arg{rep}$ for a \lcounter{} $\cl$ in a column $c$.  It
% \!\def!ines $|\pcol@thectr@|{\cdot}\cl{\cdot}c$
% 
% \SpecialArrayMainIndex{\theta{\cdot}c}{\pcol@thectr@}
% 
% to have $\arg{rep}$ in its  body.
% 
%    \begin{macrocode}
\def\definethecounter#1#2#3{\@namedef{pcol@thectr@#1#2}{#3}}
%    \end{macrocode}
% \end{macro}
% 
% \begin{macro}{\pcol@thectrelt}
% The macro $\!\pcol@thectrelt!\<\cl\>$ is invoked solely in \!\pcol@zparacol!
% and is applied to each $\cl\in\CTL$ to define its \lrep{} of
% default and that of the leftmost column 0.  It performs a lengthy sequence
% with many occurrences of \!\expandafter!, \!\csname! and \!\endcsname! but
% the sequence is equivalent to the followings.
% 
% \begin{quote}
% $\!\let!|\pcol@thectr@|{\cdot}\cl|=\the|{\cdot}\cl$\\
% $\CSIndex{ifx}|\pcol@thectr@|{\cdot}\cl{\cdot}0\!\relax!|\else |
%     \!\let!|\the|{\cdot}\cl|=\pcol@thectr@|{\cdot}\cl{\cdot}0| \fi|$
% \end{quote}
% 
% \SpecialArrayMainIndex{\theta}{\pcol@thectr@}
% \SpecialArrayIndex{\theta{\cdot}c}{\pcol@thectr@}
% \SpecialArrayMainIndex{\theta}{\the}
% 
%    \begin{macrocode}
\def\pcol@thectrelt#1{%
  \expandafter\let\expandafter\reserved@a\csname the#1\endcsname
  \expandafter\let\csname pcol@thectr@#1\endcsname\reserved@a
  \expandafter\let\expandafter\reserved@a\csname pcol@thectr@#10\endcsname
  \ifx\reserved@a\relax\else
    \expandafter\let\csname the#1\endcsname\reserved@a
  \fi}

%    \end{macrocode}
% \end{macro}
% 
% \KeepSpace{1}
% \begin{macro}{\pcol@loadctrelt}
% \changes{v1.0}{2011/10/10}
%	{Introduced for inter-environment local counter conservation.}
% \begin{macro}{\pcol@storecounters}
% \changes{v1.0}{2011/10/10}
%	{Introduced for inter-environment local counter conservation.}
% \begin{macro}{\pcol@storectrelt}
% \changes{v1.0}{2011/10/10}
%	{Introduced for inter-environment local counter conservation.}
% 
% The macro $\!\pcol@loadctrelt!\<\cl\>\<\val_c(\cl)\>$ is invoked from
% \!\pcol@zparacol! and \!\pcol@synccounter! and is applied to each element
% $\<\cl,\val_c(\cl)\>\in\Cc_c$ for a column $c$ to define a macro
% $|\pcol@ctr@|{\cdot}\cl=v(\cl)$
% 
% \SpecialArrayMainIndex{\theta}{\pcol@ctr@}
% 
% having $\val_c(\cl)$ in its body for a temporary use.  This macro or its
% redefined version is then referred to by $\!\pcol@cmpctrelt!\<\cl\>$ or
% $\!\pcol@storectrelt!\<\cl\>$.  The latter is invoked from
% \!\pcol@storecounters! via \!\pcol@sscounters! to add $\<\cl,v(\cl)\>$ to
% \!\@gtempa! by \!\@cons! to rebuild $\CC_c$ for a column $c$ in
% \!\@gtempa!.
% 
% The macro \!\pcol@storecounters! is invoked solely from
% $\!\pcol@synccounter!\<\theta\>$ to update a \lcounter{} $\theta$ with
% $\Val(\theta)$ for \csync.  That is, \!\pcol@storecounters! is used to add
% $\<\cl,v(\cl)\>$ to \!\@gtempa! for all $\cl\in\CTL$ by \!\pcol@sscounters!
% giving it \!\pcol@storectrelt! as its argument, where $v(\cl)$ is modified
% if $\cl=\theta$ or unmodifed otherwise after it is defined by
% \!\pcol@loadctrelt!.
% 
%    \begin{macrocode}
\def\pcol@loadctrelt#1#2{\@namedef{pcol@ctr@#1}{#2}}
\def\pcol@storecounters{\pcol@sscounters\pcol@storectrelt}
\def\pcol@storectrelt#1{\@cons\@gtempa{{#1}{\@nameuse{pcol@ctr@#1}}}}
%    \end{macrocode}
% \end{macro}\end{macro}\end{macro}
% 
% \begin{macro}{\pcol@savecounters}
% \changes{v1.0}{2011/10/10}
%	{Move the body to \cs{pcol@sscounters}.}
% 
% \begin{macro}{\pcol@savectrelt}
% The macro \!\pcol@savecounters! is invoked from
% \!\pcol@com@syncallcounters!, \!\pcol@stepcounter! and \!\pcol@switchcol!
% to let $\CC_c$ for a column $c$ have the list of $\<\cl,\val_c(\cl)\>$
% where $\val_c(\cl)$ is the value of $|\c@|{\cdot}\cl$ to be saved in
% the list.
% 
% \SpecialArrayIndex{\theta}{\c@}
% 
% It does this operation invoking \!\pcol@sscounters! giving it
% \!\pcol@savectrelt! as its argument.
% 
% The macro $\!\pcol@savectrelt!\<\cl\>$ adds $\<\cl,\val_c(\cl)\>$ to
% \!\@gtempa! by \!\@cons! to rebuild $\CC_c$ for a column $c$ in
% \!\@gtempa!.
% 
%    \begin{macrocode}
\def\pcol@savecounters{\pcol@sscounters\pcol@savectrelt}
\def\pcol@savectrelt#1{\@cons\@gtempa{{#1}{\number\csname c@#1\endcsname}}}
%    \end{macrocode}
% \end{macro}\end{macro}
% 
% \begin{macro}{\pcol@sscounters}
% \changes{v1.0}{2011/10/10}
%	{Introduced to implement the common operations of
%	 \cs{pcol@storecounters} and \cs{pcol@savecounters}.}
% 
% The macro $\!\pcol@sscounters!\arg{elt}$ is invoked from
% \!\pcol@storecounters! with $\arg{elt}=\!\pcol@storectrelt!$ or
% \!\pcol@savecounters! with $\arg{elt}=\!\pcol@savectrelt!$ to build
% $\CC_c=|\pcol@counters|{\cdot}c$ for a column $c$.
% 
% \SpecialArrayMainIndex{c}{\pcol@counters}
% 
% To do that, it lets $\!\@gtempa!=()$ and then apply $\arg{elt}$ to all
% $\cl\in\CTL=\!\pcol@counters!$ to have updated $\CC_c$ in \!\@gtempa!.
% Then finally, \!\@gtempa! is moved into $\CC_c$ by \!\xdef!\footnote{
% 
% It can be done by \cs{global}\cs{let} more efficiently but it is lengthy
% due to two \cs{expandafter}.}.
% 
%    \begin{macrocode}
\def\pcol@sscounters#1{\begingroup
  \global\let\@gtempa\@empty
  \let\@elt#1\relax \pcol@counters
  \let\@elt\relax
  \expandafter\xdef\csname pcol@counters\number\pcol@currcol\endcsname{%
    \@gtempa}%
  \endgroup}

%    \end{macrocode}
% \end{macro}
% 
% \begin{macro}{\pcol@cmpctrelt}
% \changes{v1.0}{2011/10/10}
%	{Introduced for inter-environment local counter conservation.}
% 
% The macro $\!\pcol@cmpctrelt!\<\theta\>$ is invoked solely from
% \!\pcol@zparacol!  and is applied to each $\theta\in\CC$ to examine if
% $\val_0(\theta)=\Val(\theta)$ where $\val_0(\theta)$ is in
% $|\pcol@ctr@|{\cdot}\theta$.
% 
% \SpecialArrayIndex{\theta}{\pcol@ctr@}
% 
% If the examination fails including due to that $|\pcol@ctr@|{\cdot}\theta$
% 
% \SpecialArrayIndex{\theta}{\pcol@ctr@}
% 
% is undefined, we add $\theta$ to the list \!\@gtempa! by \!\@cons!.
% 
%    \begin{macrocode}
\def\pcol@cmpctrelt#1{\@tempswafalse \@tempcnta\@nameuse{c@#1}%
  \expandafter\ifx\csname pcol@ctr@#1\endcsname\relax \@tempswatrue
  \else\ifnum\@nameuse{pcol@ctr@#1}=\@tempcnta\else \@tempswatrue
  \fi\fi
  \if@tempswa \@cons\@gtempa{{#1}}\fi}

%    \end{macrocode}
% \end{macro}
% 
% \begin{macro}{\synccounter}
% \changes{v1.0}{2011/10/10}
%	{Introduced as an environment-local API command.}
% \begin{macro}{\pcol@com@synccounter}
% \changes{v1.0}{2011/10/10}
%	{Introduced for the implementation of the new API command
%	 \cs{synccounter}.} 
% 
% The macro $\!\pcol@com@synccounter!\<\theta\>$, being the implementation
% of the \elocal{} API macro \!\synccounter!, lets
% $\val_c(\theta)=\Val(\theta)$ for all $c\In0\C$.  That is, the value of
% the counter $\theta$ is {\em broadcasted} to all columns for the
% {\em\Uidx\csync} of $\theta$.  This broadcast is done by
% \!\pcol@synccounter! with an argument $\!\@elt!|{|\theta|}|$ so that it
% works only on $\theta$.
% 
%    \begin{macrocode}
\def\pcol@com@synccounter#1{\pcol@synccounter{\@elt{#1}}}
%    \end{macrocode}
% \end{macro}\end{macro}
% 
% \begin{macro}{\pcol@synccounter}
% \changes{v1.0}{2011/10/10}
%	{Introduced for \cs{synccounter} and inter-environment local counter
%	 conservation.}
% \begin{macro}{\pcol@syncctrelt}
% \changes{v1.0}{2011/10/10}
%	{Introduced for \cs{synccounter} and inter-environment local counter
%	 conservation.}
% 
% The macro $\!\pcol@synccounter!\arg{lst}$ is invoked from
% \!\pcol@zparacol! with $\arg{lst}=\!\@gtempa!$ and from
% $\!\pcol@com@synccounter!\arg{ctr}$ with $\arg{lst}=\!\@elt!\Arg{ctr}$, to
% let $\val_c(\theta)=\Val(\theta)$ for all $c\In0\C$ and all $\theta$ in
% $\arg{lst}$.  To do that, at first we move $\arg{lst}$ into \!\reserved@a!
% in order to make \!\@gtempa! free so that it can be used in
% \!\pcol@storecounters!.  Next, for each $c\In0\C$, we let
% $v(\cl)=\val_c(\cl)$ by applying \!\pcol@loadctrelt! to all
% $\<\cl,\val_c(\cl)\>\in\Cc_c$, then scan $\arg{lst}$ applying
% $\!\pcol@syncctrelt!\<\theta\>$ for each $\theta$ in $\arg{lst}$ to let
% $v(\theta)=\Val(\theta)$, and finally store all $v(\cl)$ back to $\Cc_c$
% by \!\pcol@storecounters!, where $v(\theta)$ is
% $|\pcol@ctr@|{\cdot}\theta$.
% 
% \SpecialArrayIndex{\theta}{\pcol@ctr@}
% 
%    \begin{macrocode}
\def\pcol@synccounter#1{{%
  \let\@elt\relax \edef\reserved@a{#1}%
  \pcol@currcol\z@ \@whilenum\pcol@currcol<\pcol@ncol\do{%
    \let\@elt\pcol@loadctrelt \@nameuse{pcol@counters\number\pcol@currcol}%
    \let\@elt\pcol@syncctrelt \reserved@a
    \pcol@storecounters
    \advance\pcol@currcol\@ne}}}
\def\pcol@syncctrelt#1{%
    \expandafter\edef\csname pcol@ctr@#1\endcsname{\number\@nameuse{c@#1}}}

%    \end{macrocode}
% \end{macro}\end{macro}
% 
% \begin{macro}{\syncallcounters}
% \changes{v1.0}{2011/10/10}
%	{Introduced as an environment-local API command.}
% \begin{macro}{\pcol@com@syncallcounters}
% \changes{v1.0}{2011/10/10}
%	{Introduced for the implementation of the new API command
%	 \cs{syncallcounters}.} 
% 
% The macro \!\pcol@com@syncallcounters!, being the implementation of the
% \elocal{} API macro \!\syncallcounters!, makes all \lcounter{}s in all
% columns have the value in the current column.  That is, for each
% $c\In0\C$, we invoke \!\pcol@savecounters! to let $\val_c(\cl)={\Val}(\cl)$
% for all $\cl\in\CTL$.
% 
%    \begin{macrocode}
\def\pcol@com@syncallcounters{{%
  \pcol@currcol\z@ \@whilenum\pcol@currcol<\pcol@ncol\do{%
    \pcol@savecounters \advance\pcol@currcol\@ne}}}

%    \end{macrocode}
% \end{macro}\end{macro}
% 
% \begin{macro}{\pcol@setctrelt}
% \changes{v1.0}{2011/10/10}
%	{Replace \cs{setcounter} with direct assignment with \cs{csname} and
%	 \cs{endcsname}}
% 
% The macro $\!\pcol@setctrelt!\<\cl\>\<\val_c(\cl)\>$ is solely inovked
% from \!\pcol@switchcol! to swich to a column $c$ and is applied to each
% $\<\cl,\val_c(\cl)\>\in\Cc_c$ to let $\Val(\cl)=\val_c(\cl)$ \!\global!ly.
% It also define the \lrep{} of $\cl$, being $|\the|{\cdot}\cl$,
% 
% \SpecialArrayIndex{\theta}{\the}
% 
% to be $|\pcol@thectr@|{\cdot}\cl{\cdot}c$
% 
% \SpecialArrayIndex{\theta{\cdot}c}{\pcol@thectr@}
% 
% if it is defined, or otherwise to be $|\pcol@thectr@|{\cdot}\cl$ which keeps
% the original $|\the|{\cdot}\cl$.
% 
% \SpecialArrayIndex{\theta}{\the}
% 
% This \lrep{} definition is done by a lengthy sequence with many occurrences
% of \!\expandafter!, \!\csname! and \!\endcsname!, but it is equivalent to
% the followings.
% 
% \begin{quote}
% $\CSIndex{ifx}|\pcol@thectr@|{\cdot}\cl{\cdot}c\!\relax!| |
%     \!\let!|\the|{\cdot}\cl|=\pcol@thectr@|{\cdot}\cl$\\
% $|\else |\!\let!|\the|{\cdot}\cl|=\pcol@thectr@|{\cdot}\cl{\cdot}c$\\
% |\fi|
% \end{quote}
% % 
%    \begin{macrocode}
\def\pcol@setctrelt#1#2{%
  \global\csname c@#1\endcsname#2\relax
  \expandafter\ifx\csname pcol@thectr@#1\number\pcol@currcol\endcsname\relax
    \expandafter\let\expandafter\reserved@a\csname pcol@thectr@#1\endcsname
  \else
    \expandafter\let\expandafter\reserved@a
      \csname pcol@thectr@#1\number\pcol@currcol\endcsname
  \fi
  \expandafter\let\csname the#1\endcsname\reserved@a}

%    \end{macrocode}
% \end{macro}
% 
% \KeepSpace{1}
% \begin{macro}{\pcol@stepcounter}
% \changes{v1.0}{2011/10/10}
%	{Change the order of operations.}
% \changes{v1.0}{2011/10/10}
%	{Replace \cs{csname}/\cs{endcsname} with \cs{@nameuse}.}
% \begin{macro}{\pcol@stpldelt}
% \begin{macro}{\pcol@stpclelt}
% 
% The macro $\!\pcol@stepcounter!\<\cg\>$ is invoked from
% $\!\stepcounter!\<\cg\>$ for a \gcounter{} $\cg$ because
% $|\cl@|{\cdot}\cg$
% 
% \SpecialArrayIndex{\theta}{\cl@}
% 
% is modified by \!\pcol@remctrelt! so as to invoke this macro to zero-clear
% \lcounter{}s $\theta\in\clist(\cg)$.  To do that, we do the followings in
% a group for each $c\In0\C$.  First we apply
% $\!\pcol@stpldelt!\<\cl\>\<\val_c(\cl)\>$ to each
% $\<\cl,\val_c(\cl)\>\in\Cc_c$ to let $\Val(\cl)=\val_c(\cl)$ locally.
% Then we apply $\!\pcol@stpclelt!\<\theta\>$ to each $\theta\in\clist(\cg)$
% to let $\Val(\theta)=0$.  Finally, we invoke \!\pcol@savecounters! to let
% $\val_c(\cl)=\Val(\cl)$ for all $\cl\in\CTL$ to reflect the zero-clear of
% $\theta\in\clist(\cg)$.
% 
% After the operations above, we apply $\!\@stpelt!\<\theta\>$ to each
% $\theta\in\clist(\cg)$ for \!\global! zero-clearing.
% 
%    \begin{macrocode}
\def\pcol@stepcounter#1{\begingroup
  \pcol@currcol\z@ \@whilenum\pcol@currcol<\pcol@ncol\do{%
    \let\@elt\pcol@stpldelt \@nameuse{pcol@counters\number\pcol@currcol}%
    \let\@elt\pcol@stpclelt \@nameuse{pcol@cl@#1}%
    \pcol@savecounters
   \advance\pcol@currcol\@ne}%
  \endgroup
  \let\@elt\@stpelt \@nameuse{pcol@cl@#1}}
\def\pcol@stpldelt#1#2{\csname c@#1\endcsname#2\relax}
\def\pcol@stpclelt#1{\csname c@#1\endcsname\z@}

%    \end{macrocode}
% \end{macro}\end{macro}\end{macro}
% 
% 
% 
% \section{Column-Switching Commands and Environments}
% \label{sec:imp-switch}
% 
% \begin{macro}{\pcol@par}
% \changes{v1.0}{2011/10/10}
%	{Introduced for \cs{par}-if-necessary operation.}
% 
% Before giving the definition of \cswitch{} commands and environments, we
% define a commonly used macro \!\pcol@par!, which do \CSIndex{par} if
% necessary, i.e., we are not in vertical mode.  The reason why we don't
% simply do \CSIndex{par} is that it may have some definition different from
% \!\@@par! and thus an incautious repetition of \CSIndex{par} may cause
% undesirable results.  This macro is used in \!\pcol@com@switchcolumn!,
% \!\pcol@sptext!, \!\pcol@com@endcolumn!, \!\pcol@flushclear!,
% and \!\endparacol!.
% 
%    \begin{macrocode}
%% Column-Switching Commands and Environments

\def\pcol@par{\ifvmode\else \par \fi}

%    \end{macrocode}
% \end{macro}
% 
% \KeepSpace{2}
% \begin{macro}{\switchcolumn}
% \changes{v1.0}{2011/10/10}
%	{Made \cs{let}-equal to \cs{pcol@com@switchcolumn} for localization.}
% \begin{macro}{\pcol@com@switchcolumn}
% \changes{v1.0}{2011/10/10}
%	{Introduced as the implementation of \cs{switchcolumn}.}
% \changes{v1.0}{2011/10/10}
%	{Add \cs{pcol@defcolumn} to clarify the behavior of \texttt{column}.}
% \begin{macro}{\pcol@switchcolumn}
% \changes{v1.0}{2011/10/10}
%	{Add the check of $d\geq0$.}
% \changes{v1.31}{2013/10/10}
%	{Add a space before the number of columns in the error message.}
% 
% \begin{macro}{\pcol@iswitchcolumn}
% The macro $\!\pcol@com@switchcolumn!|[|d|]|$, being the implementation of the
% \elocal{} API macro \!\switchcolumn!, switches to the column $d$ if provided
% through its optional argument, or to $d=(c+1)\bmod\C$ otherwise where $c$
% is the ordinal of the current column.  After making it sure to be in
% vertical mode by \!\pcol@par!, it invokes \!\pcol@defcolumn! to give
% \!\pcol@com@column!(|*|) their
% \!\def!initions for occurrences not as the very first \cswitch{} command
% or environment of the current \env{paracol} environment.  Then, after
% calculating $d=(c+1)\bmod\C$, this macro simply invokes
% $\!\pcol@switchcol!|[|d|]|$ with or without the calculated $d$ depending
% on the existance of the optional argument delimitor `|[|'.
% 
% The macro $\!\pcol@switchcolumn!|[|d|]|$ lets $\!\pcol@nextcol!=d$ and
% confirms $0\leq d<\C$ or abort the execution by \!\PackageError! if it
% does not hold.  Then it invokes \!\pcol@iswitchcolumn! if
% $\!\switchcolumn!|[|d|]|$ is followed by a `|*|', or \!\pcol@switchcol!
% othewise.
% 
% The macro $\!\pcol@iswitchcolumn!|[|\arg{text}|]|$ invokes
% $\!\pcol@sptext!|[|\arg{text}|]|$ if the optional argument is provided, or
% \!\pcol@switchcol! otherwise, after letting $\CSIndex{ifpcol@sync}=\true$
% for \exsync{}.
% 
%    \begin{macrocode}
\def\pcol@com@switchcolumn{\pcol@par
  \pcol@defcolumn
  \@tempcnta\pcol@currcol \advance\@tempcnta\@ne
  \ifnum\@tempcnta<\pcol@ncol\else \@tempcnta\z@ \fi
  \@ifnextchar[%]
    \pcol@switchcolumn{\pcol@switchcolumn[\@tempcnta]}}
\def\pcol@switchcolumn[#1]{%
  \pcol@nextcol#1\relax
  \@tempswafalse
  \ifnum#1<\z@ \@tempswatrue \fi
  \ifnum#1<\pcol@ncol\else \@tempswatrue \fi
  \if@tempswa
    \PackageError{paracol}{%
      Column number \number#1 must be less than \number\pcol@ncol}\@eha
    \pcol@nextcol\z@
  \fi
  \@ifstar\pcol@iswitchcolumn\pcol@switchcol}
\def\pcol@iswitchcolumn{%
  \global\pcol@synctrue
  \@ifnextchar[%]
    \pcol@sptext\pcol@switchcol}

%    \end{macrocode}
% \end{macro}\end{macro}\end{macro}\end{macro}
% 
% \begin{macro}{\pcol@sptext}
% \changes{v1.0}{2011/10/10}
%	{Made \cs{long} to allow \cs{par} in its argument.}
% \changes{v1.0}{2011/10/10}
%	{Replace \cs{par} with \cs{pcol@par}.}
% \changes{v1.0}{2011/10/10}
%	{Add $\cs{ifpcol@mctext}\EQ\mathit{true}$ for restriction of the
%	 broadcast of \cs{if@nobreak} and \cs{everypar}.}
% \changes{v1.2-4}{2013/05/11}
%	{Add \cs{pcol@swapcolumn} for column-swapping.}
% \changes{v1.2-5}{2013/05/11}
% 	{Add setting of \cs{columnwidth}, \cs{linewidth} and \cs{parshape}
%	 to have indented spanning text with surrounding
%	 \string\texttt{list}-like environments.}
% \changes{v1.3-1}{2013/09/17}
%	{Add $\cs{ifpcol@sptextstart}\EQ\string\mathit{true}$ before first
% 	 synchronized column-switching to let \cs{pcol@output@switch} save
%	 pre-spanning-text stuff, move the timing of
%	 $\cs{ifpcol@sptext}\EQ\string\mathit{true}$ from the end of
%	 spanning text to the beginning so that \cs{output} routine for a
%	 page break in the text capture the pre-break portion, and remove
%	 the invocation of \cs{pcol@swapcolumn} because spanning texts are
%	 now always put into the column-0.}
% \changes{v1.3-6}{2013/09/17}
%	{Add globalization of \cs{@svsechd} and \cs{@svsec}.}
% 
% The macro $\!\pcol@sptext!|[|\arg{text}|]|$ is invoked from
% \!\pcol@zparacol! and \!\pcol@iswitchcolumn! to put a \Uidx\mctext{}
% $\arg{text}$ given as the optional argument of the former or that of
% \!\switchcolumn!|*| and its relative envrironment openers for the latter.
% The macro has \!\long!  attribute because the \mctext{} may have
% \CSIndex{par}.  Since the text is put in the column-0 regardless of its
% physical position, we let \!\pcol@nextcol!  have 0 after saving the target
% column $d=\!\pcol@nextcol!$ in \!\@tempcnta!.  Then we switch to the
% column by \!\pcol@switchcol!, after turning $\CSIndex{ifpcol@sync}=\true$
% to set a \sync{}ation point above the text and
% $\CSIndex{ifpcol@sptextstart}=\true$ to tell \!\pcol@output@switch! to
% prepare the capture of \mctext{} saveing the \prespan{}.
% 
% Next, we let $\CSIndex{ifpcol@sptextstart}=\false$ and
% $\CSIndex{ifpcol@sptext}=\true$ to indicate the main vertical list
% contains only the \mctext{} and it is to be captured by \!\output!
% routine.  Then the $\arg{text}$ is put in a group in which we let
% $\!\columnwidth!=\!\hsize!=\!\textwidth!$ and
% $\!\linewidth!=\!\textwidth!-\lrm$ with \!\parshape! to indent lines by
% \!\@totalleftmargin! if $\lrm>0$, to let \mctext{} span across all columns
% reflecting the indentation in the \env{list}-like environments surrounding
% \env{paracol} if any.  We also let $\!\col@number!=1$ to ensure again that
% \!\maketitle!  produces a title without \!\twocolumn! if it is in the
% \mctext.
% 
% Then, after invoking \!\pcol@par! to ensure to be in vertical
% mode, we \!\global!ize \!\@svsechd! and \!\@svsec! which may be defined in
% a lower-level sectioning command such as \!\paragraph! in the \mctext{} so
% that they are properly expanded in \!\everypar! inserted at the beginning
% of the first paragraph of the column to which we switch shortly, even when
% the sectioning command is used inappropriately in the \mctext.
% We also \!\global!ize \!\everypar! by a sequence with three
% \!\expandafter! so that \!\pcol@output@switch! for the \sync{}ed
% \cswitch{} we make shortly broadcasts it to other columns.  Finally after
% closing the group, we let $\!\pcol@nextcol!=d$ and
% $\CSIndex{pcol@sync}=\true$ to set another \sync{}ation point below the
% \mctext{} and to make the captured text combined with \prespan, and then
% invoke \!\pcol@switchcol!  to switch the coloumn $d$.
% 
%    \begin{macrocode}
\long\def\pcol@sptext[#1]{%
  \@tempcnta\pcol@nextcol
  \global\pcol@synctrue \pcol@nextcol\z@
  \global\pcol@sptextstarttrue
  \pcol@switchcol
  \global\pcol@sptextstartfalse \global\pcol@sptexttrue
  \begingroup
    \columnwidth\textwidth \hsize\columnwidth
    \linewidth\columnwidth \advance\linewidth-\pcol@lrmargin
    \ifdim\pcol@lrmargin>\z@ \parshape\@ne\@totalleftmargin\linewidth \fi
    \col@number\@ne #1\pcol@par
    \global\let\@svsechd\@svsechd \global\let\@svsec\@svsec
    \expandafter\global\expandafter\everypar\expandafter{\the\everypar}%
  \endgroup
  \pcol@nextcol\@tempcnta \global\pcol@synctrue \pcol@switchcol}

%    \end{macrocode}
% \end{macro}
% 
% \begin{macro}{\pcol@switchcol}
% \changes{v1.0}{2011/10/10}
%	{Add \cs{pcol@aconly} for disabling \cs{addcontentsline}.}
% \changes{v1.2-2}{2013/05/11}
% 	{Add column-scanning and page-overflow check for synchronized
%	 column-switching.}
% \changes{v1.2-4}{2013/05/11}
%	{Add \cs{pcol@swapcolumn} for column-swapping to turn
%	 \cs{if@firstcolumn}.}
% \changes{v1.3-4}{2013/09/17}
%	{Remove settig \cs{if@firstcolumn} and the invocation of
%	 \cs{pcol@swapcolumn} for it because the position of marginal notes
%	 are now controlled by \cs{pcol@addmarginpar}.}
% \changes{v1.3-5}{2013/09/17}
%	{Add setting $V_E\EQ\cs{pcol@@ensurevspace}$ and reinitialization of
%	 \cs{pcol@ensurevspace} for avoidance of post-synchronization
%	 inconsistent page break.}
% \changes{v1.35-5}{2018/12/31}
% 	{Add the invocation of $\cs{pcol@colpream}\cdot c$.}
% 
% The macro \!\pcol@switchcol! is invoked from \!\pcol@switchcolumn!,
% \!\pcol@iswitchcolumn!, \!\pcol@sptext! and \!\endparacol! to switch the
% column $d=\!\pcol@nextcol!$.  First, we save \lcounter{}s in the current
% column $c$ into $\Cc_c$ by \!\pcol@savecounters!.
%
% Next, if $\CSIndex{ifpcol@sync}=\true$, we do the followings.  At first we
% let $\VE={}$\!\pcol@@ensure~vspace! have the natrual component of
% \!\pcol@ensurevspace! which can have a glue specified by \!\ensurevspace!,
% so that it is referred to by \!\pcol@sync! as the minimum space required
% below the \sync{}ation point we are now setting.  Second, we invoke
% \!\pcol@visitallcols! temporarily turning $\CSIndex{ifpcol@sync}=\false$
% for \cscan{}ning to visit all columns but current one to give \TeX's page
% builder the chance to break \colpage{}s in the \tpage{} with \Scfnote{}s
% which could have not been presented in the last visit of the columns.
% Third, we make an \!\output! request with $\!\penalty!=\!\pcol@op@switch!$
% to invoke \!\pcol@output@switch! by \!\pcol@invokeoutput! with
% $\CSIndex{ifpcol@sync}=\true$ to make \sync{}ed switch to the column $d$.
% This invocation may result in $\CSIndex{ifpcol@flush}=\true$ to mean the
% \tpage{} should be broken before setting the \sync{}ation point.
% Therefore if so, since \!\pcol@output@switch! switched to the tallest
% column rather than $d$, we put \!\vfil! and \!\penalty!|-10000| to force
% page break, make \cscan{} with \!\newpage! put into each column to have
% some floats in the column in the new \tpage{}, and then invoke
% \!\pcol@output@switch!  again until it returns
% $\CSIndex{ifpcol@flush}=\false$ telling us it successfully sets the
% \sync{}ation point switching to the column $d$.  Then as the last operation
% specific to \sync{}ed \cswitch{}, we invoke \!\ensurevspace! with
% \!\baselineskip! to give the default of $\VE$ for the next \sync{}ation.
% 
% Otherwise, i.e., if $\CSIndex{ifpcol@sync}=\false$, we simply make the
% \!\output! request for \!\pcol@output@switch! to switch to the column $d$.
% 
% Then we scan $\Cc_d$ applying \!\pcol@setctrelt! to each
% $\<\cl,\val_d(\cl)\>\in\Cc_d$ to let $\Val(\cl)={\val_d(\cl)}$.  We also
% scan $\T=\!\pcol@aconly!$ applying \!\pcol@aconlyelt! to each
% $\<t_c,c\>\in T$ to inhibit \!\addcontentsline! to the contents file of
% type $t_d$ as specified so by $\!\addcontentsonly!\arg{t_d}\~\arg{d}$.
% After that, we let $\!\@elt!=\!\relax!$ to make it sure that any lists can
% be manipulated without unexpected application of a macro to their
% elements.
% 
% Finally, we invoke $|\pcol@colpream|{\cdot}c$, where $c=-1$ if
% $\CSIndex{ifpcol@sptextstart}=\true$ to mean the \cswitch{} is for a
% \sptext{}, or $c=d$ otherwise.
% 
% \SpecialArrayIndex{c}{\pcol@colpream}
% 
%    \begin{macrocode}
\def\pcol@switchcol{%
  \pcol@savecounters
  \ifpcol@sync
    \@tempdima\pcol@ensurevspace\relax
    \edef\pcol@@ensurevspace{\number\@tempdima sp\relax}%
    \global\pcol@syncfalse \pcol@visitallcols\@@par \global\pcol@synctrue
    \pcol@invokeoutput\pcol@op@switch
    \@whilesw\ifpcol@flush\fi{%
      \vfil \penalty-\@M
      \global\pcol@syncfalse \pcol@visitallcols\newpage \global\pcol@synctrue
      \pcol@invokeoutput\pcol@op@switch}%
    \ensurevspace{\baselineskip}%
  \else
    \pcol@invokeoutput\pcol@op@switch
  \fi
  \let\@elt\pcol@setctrelt
  \csname pcol@counters\number\pcol@currcol\endcsname
  \let\@elt\pcol@aconlyelt \pcol@aconly \let\@elt\relax
  \@nameuse{pcol@colpream\ifpcol@sptextstart-1\else\number\pcol@currcol\fi}}

%    \end{macrocode}
% \end{macro}
% 
% \begin{macro}{\pcol@visitallcols}
% \changes{v1.2-2}{2013/05/11}
% 	{Introduced for column-scanning in synchronized column-switching and
%	 page flushing.}
% 
% The macro $\!\pcol@visitallcols!\arg{cs}$, invoked from \!\pcol@switchcol!
% and \!\pcol@flushclear!, performs \cscan{}ning putting $\arg{cs}$ into the
% visited columns.  That is, we repeat the invocation of
% \!\pcol@output@switch! to visit $d$ through \!\pcol@invokeoutput! with
% $\!\penalty!=\!\pcol@op@switch!$ for all $d\In0\C-\{c\}$.  In each visit,
% we put $\arg{cs}\in\{\!\@@par!,\!\newpage!\}$ to have a chance of or to
% force page break in each visited \colpage{}.  Finally we go back to $c$ to
% restore its \cctext{} especially when we are leaving the column 0 for
% \mctext.  That is, $\cc_c(\sw)$ and $\cc_c(\ep)$ should be presented to
% \!\pcol@output@switch! to broadcast them to other columns.
% 
%    \begin{macrocode}
\def\pcol@visitallcols#1{\begingroup
  \@tempcnta\z@ \@tempcntb\pcol@currcol
  \@whilenum\@tempcnta<\pcol@ncol\do{%
    \ifnum\@tempcnta=\@tempcntb\else
      \pcol@nextcol\@tempcnta \pcol@invokeoutput\pcol@op@switch #1%
    \fi
    \advance\@tempcnta\@ne}%
  \pcol@nextcol\@tempcntb \pcol@invokeoutput\pcol@op@switch
  \endgroup}

%    \end{macrocode}
% \end{macro}
% 
% \KeepSpace{3}
% \begin{macro}{\column}
% \changes{v1.0}{2011/10/10}
%	{Made \cs{let}-equal to \cs{pcol@com@column} for localization.}
% \begin{macro}{\column*}
% \changes{v1.0}{2011/10/10}
%	{Made \cs{let}-equal to \cs{pcol@com@column*} for localization.}
% \begin{macro}{\pcol@com@column}
% \changes{v1.0}{2011/10/10}
%	{Introduced as the implemenatation of \cs{column}.}
% \begin{macro}{\pcol@com@column*}
% \changes{v1.0}{2011/10/10}
%	{Introduced as the implemenatation of \cs{column*}.}
% \begin{macro}{\pcol@defcolumn}
% \changes{v1.0}{2011/10/10}
%	{Replace \cs{column} with \cs{pcol@com@column} and \cs{switchcolumn}
%	 with \cs{pcol@switchenv}.}
% 
% The macros \!\pcol@com@column!(|*|), the implementations of the \elocal{}
% API commands \!\column!(*), starts the environment |column(*)|.
% 
% \Midx{\EnvIndex{column}}\Midx{\EnvIndex{column*}}
% 
% Basically, the macros do \!\switch~column!\~(|*|), but if the environment
% starts just after \beginparacol{} the macro have to switch to the column
% 0.  Therefore, the definitions for this very-beginning appearance are
% given in \!\pcol@zparacol! to do (almost) nothing, and then those for
% other ones are given by \!\pcol@defcolumn! invoked in
% \!\pcol@com@switchcolumn!  to invoke
% \!\pcol@switchenv!|{column|(|*|)|}|(|*|) which then invokes
% \!\switchcolumn!.  Note that the definition of non-starred
% \!\pcol@com@column! has \!\relax!  after the invocatoin of
% \!\pcol@switchenv!  so that \!\@ifnextchar! and \!\@ifstar! to examine the
% existence of `|[|' and `|*|' definitely tells us no even if the body of the
% environment starts with a `|[|' or `|*|'.
% 
%    \begin{macrocode}
\def\pcol@defcolumn{%
  \gdef\pcol@com@column{\pcol@switchenv{column}\relax}%
  \global\@namedef{pcol@com@column*}{\pcol@switchenv{column*}*}}

%    \end{macrocode}
% \end{macro}\end{macro}\end{macro}\end{macro}\end{macro}
% 
% \KeepSpace{10}
% \begin{macro}{\nthcolumn}
% \changes{v1.0}{2011/10/10}
%	{Made \cs{let}-equal to \cs{pcol@com@nthcolumn} for localization.}
% \begin{macro}{\nthcolumn*}
% \changes{v1.0}{2011/10/10}
%	{Made \cs{let}-equal to \cs{pcol@com@nthcolumn*} for localization.}
% \begin{macro}{\pcol@com@nthcolumn}
% \changes{v1.0}{2011/10/10}
%	{Introduced as the implemenatation of \cs{nthcolumn} with the
%	 inhibition of column-switching in the environment.}
% \begin{macro}{\pcol@com@nthcolumn*}
% \changes{v1.0}{2011/10/10}
%	{Introduced as the implemenatation of \cs{nthcolumn*} with the
%	 inhibition of column-switching in the environment.}
% \begin{macro}{\leftcolumn}
% \changes{v1.0}{2011/10/10}
%	{Made \cs{let}-equal to \cs{pcol@com@leftcolumn} for localization.}
% \begin{macro}{\leftcolumn*}
% \changes{v1.0}{2011/10/10}
%	{Made \cs{let}-equal to \cs{pcol@com@leftcolumn*} for localization.}
% \begin{macro}{\pcol@com@leftcolumn}
% \changes{v1.0}{2011/10/10}
%	{Introduced as the implemenatation of \cs{leftcolumn} with the
%	 inhibition of column-switching in the environment.}
% \begin{macro}{\pcol@com@leftcolumn*}
% \changes{v1.0}{2011/10/10}
%	{Introduced as the implemenatation of \cs{leftcolumn*} with the
%	 inhibition of column-switching in the environment.}
% \begin{macro}{\rightcolumn}
% \changes{v1.0}{2011/10/10}
%	{Made \cs{let}-equal to \cs{pcol@com@rightcolumn} for localization.}
% \begin{macro}{\rightcolumn*}
% \changes{v1.0}{2011/10/10}
%	{Made \cs{let}-equal to \cs{pcol@com@rightcolumn*} for localization.}
% \begin{macro}{\pcol@com@rightcolumn}
% \changes{v1.0}{2011/10/10}
%	{Introduced as the implemenatation of \cs{rightcolumn} with the
%	 inhibition of column-switching in the environment.}
% \begin{macro}{\pcol@com@rightcolumn*}
% \changes{v1.0}{2011/10/10}
%	{Introduced as the implemenatation of \cs{rightcolumn*} with the
%	 inhibition of column-switching in the environment.}
% 
% The macros $\!\pcol@com@nthcolumn!(|*|)\arg{d}$,
% \!\pcol@com@leftcolumn!(|*|) and |\pcol@com@right|\~|column|(|*|) are the
% implementations of \elocal{} API macros \!\nthcolumn!(|*|)$\arg{d}$,
% \!\leftcolumn!(|*|) and \!\rightcolumn!(|*|) respectively.  They start
% corresponding environments |nthcolumn|(|*|), |leftcolumn|(|*|), and
% |rightcolumn|(|*|)
% 
% \Midx{\EnvIndex{nthcolumn}}\Midx{\EnvIndex{nthcolumn*}}
% \Midx{\EnvIndex{leftcolumn}}\Midx{\EnvIndex{leftcolumn*}}
% \Midx{\EnvIndex{rightcolumn}}\Midx{\EnvIndex{rightcolumn*}}
% 
% simply invoking \!\pcol@switchenv!$\Arg{env}$\~(|*|), where $\arg{env}$ is
% the name of each environment, giving it $d$, 0 and 1 repectively as its
% optional argument for the target column.
% 
%    \begin{macrocode}
\def\pcol@com@nthcolumn#1{\pcol@switchenv{nthcolumn}[#1]\relax}
\@namedef{pcol@com@nthcolumn*}#1{\pcol@switchenv{nthcolumn*}[#1]*}
\def\pcol@com@leftcolumn{\pcol@switchenv{leftcolumn}[0]\relax}
\@namedef{pcol@com@leftcolumn*}{\pcol@switchenv{leftcolumn*}[0]*}
\def\pcol@com@rightcolumn{\pcol@switchenv{rightcolumn}[1]\relax}
\@namedef{pcol@com@rightcolumn*}{\pcol@switchenv{rightcolumn*}[1]*}

%    \end{macrocode}
% \end{macro}\end{macro}\end{macro}\end{macro}
% \end{macro}\end{macro}\end{macro}\end{macro}
% \end{macro}\end{macro}\end{macro}\end{macro}
% 
% \begin{macro}{\pcol@switchenv}
% \changes{v1.0}{2011/10/10}
%	{Introduced to inhibit column-switching in the environment.}
% \changes{v1.31}{2013/10/10}
%	{Fix the misspell ``swicthing'' in the error message.}
% 
% The macro $\!\pcol@switchenv!\arg{env}$ is invoked from
% $|\pcol@com@|{\cdot}\arg{env}$ where $\arg{env}\in
% \{\env{column}\~(|*|),\env{nthcolumn}(|*|),\env{leftcolumn}(|*|),
% \env{rightcolumn}(|*|)\}$ to invoke \!\switchcolumn! with the arguments
% following $\arg{env}$ given by the invoker macros.  Before invoking
% \!\switchcolumn!, we save it in \!\reserved@a! for the invocation and
% re\!\def!ine it so that it will complain the illegal usage of \cswitch{}
% commands\slash environments in the environment $\arg{env}$ by
% \!\PackageError!.
% 
%    \begin{macrocode}
\def\pcol@switchenv#1{\let\reserved@a\switchcolumn
  \def\switchcolumn{\PackageError{paracol}{%
    Column switching commands and environments cannot be used in #1}\@eha}
  \reserved@a}

%    \end{macrocode}
% \end{macro}
% 
% \KeepSpace{14}
% \begin{macro}{\endcolumn}
% \changes{v1.0}{2011/10/10}
%	{Made \cs{let}-equal to \cs{pcol@com@endcolumn} for localization.}
% \begin{macro}{\endcolumn*}
% \changes{v1.0}{2011/10/10}
%	{Made \cs{let}-equal to \cs{pcol@com@endcolumn*} for localization.}
% \begin{macro}{\pcol@com@endcolumn}
% \changes{v1.0}{2011/10/10}
%	{Introduced as the implemenatation of \cs{endcolumn} with the
%	 globalizatoin of \cs{everypar}.}
% \begin{macro}{\pcol@com@endcolumn*}
% \changes{v1.0}{2011/10/10}
%	{Introduced as the implemenatation of \cs{endcolumn*} with the
%	 globalizatoin of \cs{everypar}.}
% \begin{macro}{\endnthcolumn}
% \changes{v1.0}{2011/10/10}
%	{Made \cs{let}-equal to \cs{pcol@com@endnthcolumn} for localization.}
% \begin{macro}{\endnthcolumn*}
% \changes{v1.0}{2011/10/10}
%	{Made \cs{let}-equal to \cs{pcol@com@endnthcolumn*} for
%	 localization.}
% \begin{macro}{\pcol@com@endnthcolumn}
% \changes{v1.0}{2011/10/10}
%	{Introduced as the implemenatation of \cs{endnthcolumn} with the
%	 globalizatoin of \cs{everypar}.}
% \begin{macro}{\pcol@com@endnthcolumn*}
% \changes{v1.0}{2011/10/10}
%	{Introduced as the implemenatation of \cs{endnthcolumn*} with the
%	 globalizatoin of \cs{everypar}.}
% \begin{macro}{\endleftcolumn}
% \changes{v1.0}{2011/10/10}
%	{Made \cs{let}-equal to \cs{pcol@com@endleftcolumn} for
%	 localization.} 
% \begin{macro}{\endleftcolumn*}
% \changes{v1.0}{2011/10/10}
%	{Made \cs{let}-equal to \cs{pcol@com@endleftcolumn*} for
%	 localization.}
% \begin{macro}{\pcol@com@endleftcolumn}
% \changes{v1.0}{2011/10/10}
%	{Introduced as the implemenatation of \cs{endleftcolumn} with the
%	 globalizatoin of \cs{everypar}.}
% \begin{macro}{\pcol@com@endleftcolumn*}
% \changes{v1.0}{2011/10/10}
%	{Introduced as the implemenatation of \cs{endleftcolumn*} with the
%	 globalizatoin of \cs{everypar}.}
% \begin{macro}{\endrightcolumn}
% \changes{v1.0}{2011/10/10}
%	{Made \cs{let}-equal to \cs{pcol@com@endrightcolumn} for
%	 localization.}
% \begin{macro}{\endrightcolumn*}
% \changes{v1.0}{2011/10/10}
%	{Made \cs{let}-equal to \cs{pcol@com@endrightcolumn*} for
%	 localization.}
% \begin{macro}{\pcol@com@endrightcolumn}
% \changes{v1.0}{2011/10/10}
%	{Introduced as the implemenatation of \cs{endrightcolumn} with the
%	 globalizatoin of \cs{everypar}.}
% \begin{macro}{\pcol@com@endrightcolumn*}
% \changes{v1.0}{2011/10/10}
%	{Introduced as the implemenatation of \cs{endrightcolumn*} with the
%	 globalizatoin of \cs{everypar}.}
% 
% The macro \!\pcol@com@endcolumn! is the implementation of the \elocal{}
% API macro \!\endcolumn! to close \env{column} environment.  The macro
% makes it sure we are in vertical mode by \!\pcol@par! and \!\global!ize
% \!\everypar! so that it is saved in $\cc_c(\ep)$ of the current column
% $c$ on the switch to another column.  The macro also gives the commmon
% definition of \!\pcol@com@endcolumn*!  for \!\endcolumn*!,
% \!\pcol@com@endnthcolumn!(|*|) for \!\endnthcolumn!(|*|),
% \!\pcol@com@endleftcolumn!(|*|) for \!\endleftcolumn!(|*|), and
% \!\pcol@com@endrightcolumn!(|*|) for \!\endrightcolumn!(|*|).
% 
%    \begin{macrocode}
\def\pcol@com@endcolumn{\pcol@par
  \expandafter\global\expandafter\everypar\expandafter{\the\everypar}}
\expandafter\let\csname pcol@com@endcolumn*\endcsname\pcol@com@endcolumn
\let\pcol@com@endnthcolumn\pcol@com@endcolumn
\expandafter\let\csname pcol@com@endnthcolumn*\endcsname\pcol@com@endcolumn
\let\pcol@com@endleftcolumn\pcol@com@endcolumn
\expandafter\let\csname pcol@com@endleftcolumn*\endcsname\pcol@com@endcolumn
\let\pcol@com@endrightcolumn\pcol@com@endcolumn
\expandafter\let\csname pcol@com@endrightcolumn*\endcsname\pcol@com@endcolumn

%    \end{macrocode}
% \end{macro}\end{macro}\end{macro}\end{macro}
% \end{macro}\end{macro}\end{macro}\end{macro}
% \end{macro}\end{macro}\end{macro}\end{macro}
% \end{macro}\end{macro}\end{macro}\end{macro}
% 
% \begin{macro}{\definecolumnpreamble}
% \changes{v1.35-5}{2018/12/31}
% 	{Introduced to define a column preamble.}
% 
% The API macro \!\definecolumnpreamble!\marg{c}\marg{pream} is to define
% the \colpream{} $\arg{pream}$ for the column $c$ or that for \sptext{}s if
% $c=-1$.  After assiging $c$ to \!\@tempcnta! to ensure $c$ is a number,
% the macro $|\pcol@colpream|{\cdot}c$ is |\def|ined to have $\arg{pream}$.
% 
% \SpecialArrayMainIndex{c}{\pcol@colpream}
% 
%    \begin{macrocode}
\def\definecolumnpreamble#1#2{\@tempcnta#1\relax
  \expandafter\gdef\csname pcol@colpream\number\@tempcnta\endcsname{#2}}

%    \end{macrocode}
% \end{macro}
% 
% 
% \begin{macro}{\ensurevspace}
% \changes{v1.3-5}{2013/09/17}
%	{Introduced to declare the minimum space $V_E$ below a
%	 synchronization point to let it stay in a page.}
% \begin{macro}{\pcol@ensurevspace}
% \changes{v1.3-5}{2013/09/17}
%	{Introduced to keep $V_E$ declared by \cs{ensurevspace}.}
% \begin{macro}{\pcol@@ensurevspace}
% \changes{v1.3-5}{2013/09/17}
%	{Introduced to pass $V_E$ declared by \cs{ensurevspace} to
%	 \cs{output} routine.}
% 
% The API macro $\!\ensurevspace!\ARg{space}$ is to declare that the
% \sync{}ation point following it must be thrown to the next page unless the
% page has the vertical $\arg{space}$ below the \sync{}ation point.  The
% macro makes a dummy assignment of $\arg{space}$ to \!\@tempdima! to ensure
% the argument is a dimension including forced one, or in other words to
% raise an error if not in this macro rather than at the time $\arg{space}$
% is {\em evaluated} in \!\pcol@switchcol!.  Then $\arg{space}$ is kept in
% \!\pcol@ensurevspace! so that $\arg{space}$ is evaluated in
% \!\pcol@switchcol! for the \sync{}ation in question to pass the value to
% \!\pcol@sync!  through the macro $\!\pcol@@ensurevspace!=\Uidx\VE$,
% especially when it has register references, for example to
% \!\baselineskip!.  To give the default of \!\pcol@ensurevspace!, we invoke
% \!\ensurevspace! at the top level with \!\baselineskip!.
% 
%    \begin{macrocode}
\def\ensurevspace#1{{\@tempdima#1\relax \gdef\pcol@ensurevspace{#1}}}
\ensurevspace{\baselineskip}

%    \end{macrocode}
% \end{macro}\end{macro}\end{macro}
% 
% 
% 
% \section{Disabling \cs{addcontentsline}}
% \label{sec:imp-aconly}
% 
% \begin{macro}{\addcontentsonly}
% \changes{v1.0}{2011/10/10}
%	{Introduced for disabling \cs{addcontentsline}.}
% \begin{macro}{\pcol@aconly}
% \changes{v1.0}{2011/10/10}
%	{Introduced for disabling \cs{addcontentsline}.}
% 
% The API macro $\!\addcontentsonly!\arg{t}\arg{c}$ makes the type $t$
% contents file written by commands appearing only in the column $c$.  The
% macro simply add the pair $\<t,c\>$ to the list $\Uidx\T=\!\pcol@aconly!$
% being empty at initial, after confirming we know the type $t$, one of
% |toc|, |lof| and |lot| so far, by the fact $|\pcol@ac@def|{\cdot}t$ is
% defined, or abort execution by \!\PackageError!.
% 
%    \begin{macrocode}
%% Disabling \addcontentsline

\def\addcontentsonly#1#2{%
  \@ifundefined{pcol@ac@def@#1}
    {\PackageError{paracol}{Unknown contents type #1}\@eha}\relax
  \@cons\pcol@aconly{{#1}{#2}}}
\gdef\pcol@aconly{}

%    \end{macrocode}
% \end{macro}\end{macro}
% 
% \begin{macro}{\pcol@aconlyelt}
% \changes{v1.0}{2011/10/10}
%	{Introduced for disabling \cs{addcontentsline}.}
% 
% The macro $\!\pcol@aconlyelt!\arg{t_d}\arg{d}$ is invoked solely in
% \!\pcol@switchcol!  for the \cswitch{} to column $c$, and is applied to
% each $\<t_d,d\>\in\T$ to enable \!\addcontentsline! for $t_d$ if $d=c$ by
% the invocation of $|\pcol@ac@def@|{\cdot}t_d$
% 
% \SpecialArrayMainIndex{t}{\pcol@ac@def@}
% 
% with an argument |enable|, or to disable if $d\neq c$ with |disable|.
% 
%    \begin{macrocode}
\def\pcol@aconlyelt#1#2{%
  \ifnum#2=\pcol@currcol \@nameuse{pcol@ac@def@#1}{enable}%
  \else \@nameuse{pcol@ac@def@#1}{disable}%
  \fi}
%    \end{macrocode}
% \end{macro}
% 
% \begin{macro}{\pcol@gobblethree}
% \changes{v1.0}{2011/10/10}
%	{Introduced for disabling \cs{addcontentsline}.}
% \begin{macro}{\pcol@addcontentsline}
% \changes{v1.0}{2011/10/10}
%	{Introduced for disabling \cs{addcontentsline}.}
% 
% \hfuzz0.39pt
% The macro $\!\pcol@gobblethree!\arg{file}\arg{sec}\arg{entry}$ is used in
% \!\pcol@ac@disable@toc! and \!\pcol@ac@caption! to make
% \!\addcontentsline! \!\let!-equal to this macro, which does nothing but
% discarding three arguments, for disabling.  The macro
% \!\pcol@addcontentsline! is the \LaTeX's original \!\addcontentsline! and
% is used in the macros mentioned above to let \!\addcontentsline! act as
% original.
% 
%    \begin{macrocode}
\def\pcol@gobblethree#1#2#3{}
\let\pcol@addcontentsline\addcontentsline

%    \end{macrocode}
% \end{macro}\end{macro}
% 
% \KeepSpace{1}
% \begin{macro}{\pcol@ac@def@toc}
% \changes{v1.0}{2011/10/10}
%	{Introduced for disabling \cs{addcontentsline}.}
% \begin{macro}{\pcol@ac@enable@toc}
% \changes{v1.0}{2011/10/10}
%	{Introduced for disabling \cs{addcontentsline}.}
% \begin{macro}{\pcol@ac@disable@toc}
% \changes{v1.0}{2011/10/10}
%	{Introduced for disabling \cs{addcontentsline}.}
% 
% The macro $\!\pcol@ac@def@toc!\arg{eord}$ is invoked solely in
% $\!\pcol@aconlyelt!|{toc}|\arg{c}$ to enable or disable
% \!\addcontentsline! according to $\arg{eord}$ by making \!\@sect!
% \!\let!-equal to \!\pcol@ac@enable@toc! which is the \LaTeX's original
% \!\@sect!, or to \!\pcol@ac@disable@toc! respectively.  The macro
% $\!\pcol@ac@disable@toc!
% \<a_1\>\<a_2\>\<a_3\>\<a_4\>\<a_5\>\<a_6\>|[|\<a_7\>|]|\<a_8\>$ at first
% disables \!\addcontentsline! by making it \!\let!-equal to
% \!\pcol@gobblethree!, then invokes the original \!\@sect! saved in
% \!\pcol@ac@enable@toc! ginving it all arguments $a_1$ to $a_8$, and
% finally enables it by making it \!\let!-equal to \!\pcol@addcontentsline!.
% Note that the argument $a_7$ is surrounded by |{| and |}| on the
% invocation of \!\@sect! to conceal `|]|' in $a_7$.
% 
%    \begin{macrocode}
\def\pcol@ac@def@toc#1{%
  \expandafter\let\expandafter\@sect\csname pcol@ac@#1@toc\endcsname}
\let\pcol@ac@enable@toc\@sect
\def\pcol@ac@disable@toc#1#2#3#4#5#6[#7]#8{%
  \let\addcontentsline\pcol@gobblethree
  \pcol@ac@enable@toc{#1}{#2}{#3}{#4}{#5}{#6}[{#7}]{#8}%
  \let\addcontentsline\pcol@addcontentsline}

%    \end{macrocode}
% \end{macro}\end{macro}\end{macro}
% 
% \KeepSpace{5}
% \begin{macro}{\pcol@ac@def@lof}
% \changes{v1.0}{2011/10/10}
%	{Introduced for disabling \cs{addcontentsline}.}
% \begin{macro}{\pcol@ac@def@lot}
% \changes{v1.0}{2011/10/10}
%	{Introduced for disabling \cs{addcontentsline}.}
% \begin{macro}{\pcol@ac@caption@enable}
% \changes{v1.0}{2011/10/10}
%	{Introduced for disabling \cs{addcontentsline}.}
% \begin{macro}{\pcol@ac@caption@disable}
% \changes{v1.0}{2011/10/10}
%	{Introduced for disabling \cs{addcontentsline}.}
% \begin{macro}{\pcol@ac@caption@def}
% \changes{v1.0}{2011/10/10}
%	{Introduced for disabling \cs{addcontentsline}.}
% \begin{macro}{\pcol@ac@caption@if@lof}
% \changes{v1.0}{2011/10/10}
%	{Introduced for disabling \cs{addcontentsline}.}
% \begin{macro}{\pcol@ac@caption@if@lot}
% \changes{v1.0}{2011/10/10}
%	{Introduced for disabling \cs{addcontentsline}.}
% 
% The macro $\!\pcol@ac@def@lof!\arg{eord}$ and
% $\!\pcol@ac@def@lot!\arg{eord}$ are invoked solely in
% $\!\pcol@aconlyelt!\<t\>\<c\>$ when $t$ is |lof| or |lot|
% respectively.  They invoke \!\pcol@ac@caption@enable!$\<t\>$ or
% $\!\pcol@ac@caption@disable!\<t\>$ according to $\arg{eord}$, and then
% these macros invoke $\!\pcol@ac@caption@def!\<s\>\<t\>$ where
% $s=\!\@tempswatrue!$ or $s=\!\@tempswafalse!$ respectievely to let
% $\!\@caption!=\!\pcol@ac@caption!$ and $|\pcol@ac@caption@if@|{\cdot}t=s$
% which are \!\let!-equal to \!\@tempswatrue! in default.  That is,
% $|\pcol@ac@catption@if@|{\cdot}t$ lets $\CSIndex{if@tempswa}={}\true$ iff
% \!\addcontentsline! for $t$ is to be enable.
% 
%    \begin{macrocode}
\def\pcol@ac@def@lof#1{\@nameuse{pcol@ac@caption@#1}{lof}}
\def\pcol@ac@def@lot#1{\@nameuse{pcol@ac@caption@#1}{lot}}
\def\pcol@ac@caption@enable{\pcol@ac@caption@def\@tempswatrue}
\def\pcol@ac@caption@disable{\pcol@ac@caption@def\@tempswafalse}
\def\pcol@ac@caption@def#1#2{\let\@caption\pcol@ac@caption
 \expandafter\let\csname pcol@ac@caption@if@#2\endcsname#1}
\let\pcol@ac@caption@if@lof\@tempswatrue
\let\pcol@ac@caption@if@lot\@tempswatrue
%    \end{macrocode}
% \end{macro}\end{macro}\end{macro}\end{macro}
% \end{macro}\end{macro}\end{macro}
% 
% \begin{macro}{\pcol@ac@caption}
% \changes{v1.0}{2011/10/10}
%	{Introduced for disabling \cs{addcontentsline}.}
% \begin{macro}{\pcol@ac@caption@latex}
% \changes{v1.0}{2011/10/10}
%	{Introduced for disabling \cs{addcontentsline}.}
% 
% The macro $\!\pcol@ac@caption!\arg{type}|[|\arg{lcap}|]|\<cap\>$ is made
% \!\let!-equal to \!\@caption! by \!\pcol@ac@caption@def! to do what
% \!\@caption! do but with enabling\slash disabling \!\addcontentsline!.  At
% first, it invokes $|\pcol@ac@cation@if@|{\cdot}t$ where
% $t=|\ext@|{\cdot}\arg{type}$
% 
% \SpecialIndex{\ext@figure}\SpecialIndex{\ext@table}
% 
% to let \CSIndex{if@tempswa} be $\true$ or $\false$ according to the
% enable\slash disable status of $t$.  Then, after letting
% $\!\addcontentsline!=\!\pcol@gobblethree!$ for disabling if $\false$, we
% invoke \!\pcol@ac@caption@latex!, being the \LaTeX's orignial \!\@caption!,
% giving all three arguments of \!\pcol@ac@caption! itself surrounding
% $\arg{lcap}$ with |{| and |}| for the concealment of `|]|'.  Finally, we
% let $\!\addcontentsline!=\!\pcol@addcontentsline!$ so that other macros
% uses it with its original definition.
% 
%    \begin{macrocode}
\long\def\pcol@ac@caption#1[#2]#3{%
  \@nameuse{pcol@ac@caption@if@\@nameuse{ext@#1}}%
  \if@tempswa\else \let\addcontentsline\pcol@gobblethree \fi
  \pcol@ac@caption@latex{#1}[{#2}]{#3}%
  \let\addcontentsline\pcol@addcontentsline}
\let\pcol@ac@caption@latex\@caption

%    \end{macrocode}
% \end{macro}\end{macro}
% 
% 
% 
% \KeepSpace{7}
% \section{Page Flushing Commands}
% \label{sec:imp-flush}
% 
% \begin{macro}{\flushpage}
% \changes{v1.0}{2011/10/10}
%	{Made \cs{let}-equal to \cs{pcol@com@flushpage} for
%	 localization.}
% \begin{macro}{\pcol@com@flushpage}
% \changes{v1.0}{2011/10/10}
%	{Introduced as the implemenatation of \cs{flushpage} with the
%	 replacement of \cs{par} with \cs{pcol@par}.}
% \begin{macro}{\clearpage}
% \changes{v1.0}{2011/10/10}
%	{Made \cs{let}-equal to \cs{pcol@com@clearpage} for
%	 localization.}
% \begin{macro}{\pcol@com@clearpage}
% \changes{v1.0}{2011/10/10}
%	{Introduced as the implemenatation of \cs{clearpage} with the
%	 replacement of \cs{par} with \cs{pcol@par}.}
% \begin{macro}{\cleardoublepage}
% \changes{v1.3-5}{2013/09/17}
%	{Added as a member of local commands and made \cs{let}-equal to
%	 \cs{pcol@com@cleardoublepage}.}
% \begin{macro}{\pcol@com@cleardoublepage}
% \changes{v1.3-5}{2013/09/17}
%	{Introduced as the implemenatation of \cs{cleardoulbepage}.}
% 
% The macros \!\pcol@com@flushpage!, \!\pcol@com@clearpage! and
% \!\pcol@com@cleardoublepage! are the implementations of \elocal{} API
% macro \!\flushpage!, \!\clearpage! and \!\cleardoublepage!  respectively.
% The first two have a common structure in which we at first invoke
% \!\pcol@flushclear! for \cscan{} and \pfcheck, and then make an \!\output!
% request by \!\pcol@invokeoutput! with \!\penalty! being \!\pcol@op@flush!
% or \!\pcol@op@clear! according to the commands.  On the other hand the
% last one simply invokes \!\pcol@com@clearpage! unconditionally, and then
% \!\pcol@com@flushpage!\footnote{
% 
% Unlike \LaTeX's \!\cleardoublepage!, it is unnecessary to put an empty
% \!\hbox! before \!\flushpage! because it is active even at the top of a
% page.}
% 
% if two-sided paging is enabled by $\CSIndex{if@twoside}=\true$, we are
% in an even-numbered page, and $\CSIndex{ifpcol@paired}=\false$ to mean we
% are not doing \npaired{} \parapag{}ing.
% 
%    \begin{macrocode}
%% Page Flushing Commands

\def\pcol@com@flushpage{\pcol@flushclear\voidb@x
  \pcol@invokeoutput\pcol@op@flush}
\def\pcol@com@clearpage{\pcol@flushclear\voidb@x
  \pcol@invokeoutput\pcol@op@clear}
\def\pcol@com@cleardoublepage{\pcol@com@clearpage
  \if@twoside \ifodd\c@page\else \ifpcol@paired\else \pcol@com@flushpage
  \fi\fi\fi}
%    \end{macrocode}
% \end{macro}\end{macro}\end{macro}\end{macro}\end{macro}\end{macro}
% 
% \begin{macro}{\pcol@flushclear}
% \changes{v1.2-2}{2013/05/11}
% 	{Introduced for column-scanning in synchronized column-switching and
%	 page flushing.}
% 
% The macro $\!\pcol@flushclear!\arg{box}$, invoked from
% \!\pcol@com@flushpage!, \!\pcol@com@clear~page! and \!\endparacol!,
% performs \cscan{} and \pfcheck{} prior to page flushing or environment
% closing.  After confirming we are in vertical mode by \!\pcol@par! and
% letting $d=\!\pcol@nextcol!$ be $c=\!\pcol@currcol!$ to stay in $c$, we
% invoke \!\pcol@visitallcols! for \cscan{} to give \TeX's page builder the
% chance to break the \tpage{} prior to flushing it.
% 
% Then we repeat \pfcheck{} invoking \!\pcol@output@switch! through
% \!\pcol@invokeoutput! with $\!\penalty!=\!\pcol@op@switch!$ and
% $\CSIndex{ifpcol@clear}=|\ifpcol@|\~|sync|=\true$
% 
% \SpecialIndex{\ifpcol@sync}
% 
% until the special \!\output! routine finishes with
% $\CSIndex{ifpcol@flush}=\false$ and $\arg{box}=\bot$, where
% $\arg{box}=\df=\!\pcol@topfnotes!$ if this macro is invoked from
% \!\endparacol! with non-merged \Scfnote{} typesetting.  In the repetition,
% we put \!\vfil! and \!\penalty!|-10000| to force page break into the
% tallest column, temporarily turning $\CSIndex{ifpcol@lastpage}=\false$
% using \!\ifpcol@lastpagesave!  because the broken page is not last one,
% each time the \pfcheck{} tells us to do it.  That is, we repeat the check
% while we have too tall columns due to \Scfnote{}s or, when closing
% \env{paracol} environment, deferred non-merged \Scfnote{}s.
% 
% Finally we let $\CSIndex{ifpcol@clear}$ have its default setting, i.e.,
% $\false$.
% 
%    \begin{macrocode}
\def\pcol@flushclear#1{\pcol@par
  \pcol@nextcol\pcol@currcol
  \pcol@visitallcols\@@par
  \pcol@cleartrue \global\pcol@synctrue
  \ifpcol@lastpage \pcol@lastpagesavetrue \else \pcol@lastpagesavefalse \fi
  \pcol@invokeoutput\pcol@op@switch \ifvoid#1\else \global\pcol@flushtrue \fi
  \@whilesw\ifpcol@flush\fi{%
    \pcol@lastpagefalse
    \vfil \penalty-\@M \pcol@cleartrue \global\pcol@synctrue
    \ifpcol@lastpagesave \pcol@lastpagetrue \fi
    \pcol@invokeoutput\pcol@op@switch
    \ifvoid#1\else \global\pcol@flushtrue \fi}%
  \pcol@clearfalse}

%    \end{macrocode}
% \end{macro}
% 
% 
% 
% 
% \KeepSpace{8}
% \section{Commands for Footnotes}
% \label{sec:imp-fnote}
% \changes{v1.2-2}{2013/05/11}
%	{Add the subsection ``Commands for Footnotes'' to describe newly
%	 introduced macros for page-wise footnotes.}
% 
% \begin{macro}{\footnotelayout}
% \changes{v1.3-5}{2013/09/17}
%	{Introduced for easier declaration of footnote layout.}
% \begin{macro}{\pcol@fnlayout@c}
% \changes{v1.3-5}{2013/09/17}
%	{Introduced for easier declaration of column-wise footnotes.}
% \begin{macro}{\pcol@fnlayout@p}
% \changes{v1.3-5}{2013/09/17}
%	{Introduced for easier declaration of page-wise footnotes.}
% \begin{macro}{\pcol@fnlayout@m}
% \changes{v1.3-5}{2013/09/17}
%	{Introduced for easier declaration of merged footnotes.}
% \begin{macro}{\multicolumnfootnotes}
% \changes{v1.2-2}{2013/05/11}
% 	{Introduced to delclare the defalut column-wise footnote
%	 typesetting explicitly.}
% \begin{macro}{\singlecolumnfootnotes}
% \changes{v1.2-2}{2013/05/11}
% 	{Introduced to delclare the page-wise but non-merged footnote
%	 typesetting.}
% \begin{macro}{\mergedfootnotes}
% \changes{v1.2-2}{2013/05/11}
% 	{Introduced to delclare the page-wise and merged footnote
%	 typesetting.}
% 
% The API macros $\!\footnotelayout!\Arg{l}$ is to determine that footnotes
% are \mcfnote{} ($l=|c|$), \scfnote{} without merging ($l=|p|$), or
% \mgfnote{} and \scfnote{} ($l=m$).  The macro examines if
% $l\in\{|c|,|p|,|m|\}$ by the existence of the corresponding macro
% $|\pcol@fnlayout@|{\cdot}l$
% 
% \SpecialArrayIndex{l}{\pcol@fnlayout@}
% 
% and invokes it, or complains if not by \!\PackageError!.
% 
% The macros \!\pcol@fnlayout@c!, \!\pcol@fnlayout@p! and
% \!\pcol@fnlayout@m! turn switches $f_s=\CSIndex{ifpcol@scfnote}$,
% $f_m=\CSIndex{ifpcol@mgfnote}$ and
% $f_a=\CSIndex{ifpcol@fncounteradjustment}$ and make the counter
% \counter{footnote} global or local as follows.
% $$
% \nosv\begin{array}{l|llll}
% l&f_s&f_m&f_a&\counter{footnote}\\\hline
% \verb|c|&\false&\false&\false&\hbox{local}\\
% \verb|p|&\true&\false&\true&\hbox{global}\\
% \verb|m|&\true&\true&\true&\hbox{global}
% \end{array}
% $$
% Note that turning \CSIndex{ifpcol@fncounteradjustment} is done by
% \!\fncounteradjustment! ($\true$) or \!\nofncounteradjustment! ($\false$).
% Also note that the setting of |\ifpcol@fncounter|\~|adjustment|
% 
% \SpecialIndex{\ifpcol@fncounteradjustment}
% 
% and the globalization\slash localization of \counter{footnote} are just to
% give defaults and thus can be overridden by API macros giving non-default
% settings.  Another remark is that backward-compatible macros
% \!\multicolumnfootnotes!, \!\singlecolumnfootnotes! and
% \!\mergedfootnotes! are \!\let!-equal to \!\pcol@fnlayout@c!,
% \!\pcol@fnlayout@p! and \!\pcol@fnlayout@m! respectively.
% 
%    \begin{macrocode}
%% Commands for Footnotes

\def\footnotelayout#1{\@ifundefined{pcol@fnlayout@#1}%
  {\PackageError{paracol}{Unknown footnote layout specifier #1}}%
  {\@nameuse{pcol@fnlayout@#1}}}
\def\pcol@fnlayout@c{\global\pcol@scfnotefalse \global\pcol@mgfnotefalse
  \localcounter{footnote}\nofncounteradjustment}
\def\pcol@fnlayout@p{\global\pcol@scfnotetrue \global\pcol@mgfnotefalse
  \globalcounter{footnote}\fncounteradjustment}
\def\pcol@fnlayout@m{\pcol@fnlayout@p\global\pcol@mgfnotetrue}

\let\multicolumnfootnotes\pcol@fnlayout@c
\let\singlecolumnfootnotes\pcol@fnlayout@p
\let\mergedfootnotes\pcol@fnlayout@m

%    \end{macrocode}
% \end{macro}\end{macro}\end{macro}\end{macro}\end{macro}\end{macro}\end{macro}
% 
% \KeepSpace{2}
% \begin{macro}{\@footnotetext}
% \begin{macro}{\pcol@fntext}
% \changes{v1.2-2}{2013/05/11}
% 	{Introduced for footnote encapsulation and deferring.}
% \begin{macro}{\pcol@fntexttop}
% \changes{v1.2-2}{2013/05/11}
% 	{Introduced for footnote encapsulation.}
% \begin{macro}{\pcol@fntextother}
% \changes{v1.2-2}{2013/05/11}
% 	{Introduced for footnote encapsulation and deferring.}
% 
% The macro \!\pcol@fntext!\marg{text} is our own version of \LaTeX's
% \!\@footnotetext! used in \!\footnote! and \!\footnotetext! to \!\insert!
% the footnote $\arg{text}$ through \!\footins!.  Since the original and our
% own are made \!\let!-equal by \!\pcol@zparacol!, our own is active
% throughout the environment.  The customization is done to examine if the
% footnote should be deferred and to encapsulate the footnote in a \!\vbox!.
% 
% The deferred footnote insertion is in effect if the footnote typesetting
% is \scfnote{} and \!\footnote! or \!\footnotetext!  appears in a page
% $p<\ptop$.  If so, we put the footnote $\arg{text}$ encapsulated in a
% \!\vbox! by \!\pcol@fntextbody! to the tail of $\df=\!\pcol@topfnotes!$
% with \!\penalty!\!\interlinepenalty! preceding it for the split in
% \!\pcol@deferredfootins!, using \!\pcol@fntextother!\marg{text} whose sole
% user is this macro.  Note that the decision of deferring is done based on
% $p=\!\pcol@page!$ which could be less than that of the page in which the
% footnoted text appears because the paragraph having the text will have a
% page break before the text.  Therefore, $p$ for the footnote can be
% $\ptop$, but this misjudment will not cause problems because the footnote
% will eventually be put in $\ptop$ through $\df$ when the page break
% occurs.
% 
% Otherwise the footnote $\arg{text}$ is processed by
% \!\pcol@fntexttop!\marg{text}, also used solely in this macro, to
% \!\insert! it through \!\footins! as usual but after the encapsulation by
% \!\pcol@fntextbody! and with \!\penalty!\!\interlinepenalty! following it
% to allow \TeX's page builder to split footnotes.
% 
% Note that \!\pcol@fntexttop! and \!\pcol@fntextother! have \!\long!
% property because $\arg{text}$ may have two or more paragraphs.
% 
%    \begin{macrocode}
\def\pcol@fntext{%
  \let\reserved@a\pcol@fntexttop
  \ifpcol@scfnote \ifnum\pcol@page<\pcol@toppage
    \let\reserved@a\pcol@fntextother
  \fi\fi
  \reserved@a}
\long\def\pcol@fntexttop#1{%
  \pcol@Logfn{\pcol@fntexttop{\@thefnmark}}%
  \insert\footins{\pcol@fntextbody{#1}\penalty\interlinepenalty}}
\long\def\pcol@fntextother#1{%
  \global\setbox\pcol@topfnotes\vbox{\unvbox\pcol@topfnotes
    \penalty\interlinepenalty\pcol@fntextbody{#1}}}
%    \end{macrocode}
% \end{macro}\end{macro}\end{macro}\end{macro}
% 
% \begin{macro}{\pcol@fntextbody}
% \changes{v1.2-2}{2013/05/11}
% 	{Introduced for footnote encapsulation and height capping.}
% 
% The macro \!\pcol@fntextbody!\marg{text}, invoked from \!\pcol@fntexttop!
% and \!\pcol@fntext~other!, encapsulates the footnote $\arg{text}$ in a
% \!\vbox! whose height is $h_{\max}=\!\textheight!-\!\skip!\~\!\footins!$ at
% tallest.  The encapsulation is to inhibit page breaks in a footnote
% because the split by the break will make some skips and other items
% eliminated causing a weird result when split portions are {\em joined}.
% The height capping is thus required to find a page in which the footnote
% resides.
% 
% The macro at first does operations done in \LaTeX's 
% \!\@footnotetext! to put $\arg{text}$ in \!\@tempboxa! but with one
% exception that $\!\hsize!=\!\textwidth!$ rather than \!\columnwidth! when
% \Scfnote{} typesetting is in effect.  Note that this part is blindly copied
% from the original though it should be meaningless to set
% \!\interlinepenalty!, \!\splittopskip!, \!\splitmaxdepth! and
% \!\floatingpenalty! because $\arg{text}$ is encapsulated.
% 
% Then the height-plus-depth of the box is compared with $h_{\max}$ and, if
% it exceeds the limit, the height of the box is set $h_{\max}$, the
% footnote is made followed by a \!\vss! to avoid overfull, and a warning
% message of too tall is put by \!\PackageWarning!.  Finally, the box is put
% into \!\footins! or $\df$ by the invoker of this macro.
% 
%    \begin{macrocode}
\long\def\pcol@fntextbody#1{\setbox\@tempboxa\vbox{%
    \reset@font\footnotesize
    \interlinepenalty\interfootnotelinepenalty
    \splittopskip\footnotesep
    \splitmaxdepth\dp\strutbox \floatingpenalty\@MM
    \hsize \ifpcol@scfnote \textwidth \else \columnwidth \fi
    \@parboxrestore
    \protected@edef\@currentlabel{\p@footnote\@thefnmark}%
    \color@begingroup
    \@makefntext{%
      \rule\z@\footnotesep\ignorespaces#1\@finalstrut\strutbox}%
    \color@endgroup}%
  \@tempdima\ht\@tempboxa \advance\@tempdima\dp\@tempboxa
  \@tempdimb\textheight \advance\@tempdimb-\skip\footins
  \ifdim\@tempdima>\@tempdimb
    \setbox\@tempboxa\vbox to\@tempdimb{\unvbox\@tempboxa\vss}%
    \PackageWarning{paracol}{Too tall footnote}%
  \fi
  \box\@tempboxa}

%    \end{macrocode}
% \end{macro}
% 
% \begin{macro}{\fncounteradjustment}
% \changes{v1.2-2}{2013/05/11}
% 	{Introduced to make \string\texttt{footnote} counter is consistent
%	 with its origin at the beginning of \string\texttt{paracol} and the
%	 number of footnotes given in the environment at its end.}
% \begin{macro}{\nofncounteradjustment}
% \changes{v1.2-2}{2013/05/11}
% 	{Introduced to disable the footnote counter adjustment.}
% 
% The API macros \!\fncounteradjustment! and \!\nofncounteradjustment! turns
% |\ifpcol@|\~|fncounteradjustment|
% 
% \SpecialIndex{\ifpcol@fncounteradjustment} 
% 
% $\true$ or $\false$, to enable or disable the \counter{footnote} counter
% adjustment letting $\!\c@footnote!=\bf+\nf$ in \Endparacol, respectively.
% After the definition we disable the adjustment to give the default setting.
% 
%    \begin{macrocode}
\def\fncounteradjustment{\global\pcol@fncounteradjustmenttrue}
\def\nofncounteradjustment{\global\pcol@fncounteradjustmentfalse}
\nofncounteradjustment

%    \end{macrocode}
% \end{macro}\end{macro}
% 
% \KeepSpace{2}
% \begin{macro}{\pcol@footnoterule}
% \changes{v1.2-2}{2013/05/11}
% 	{Introduced to keep the original definition of \cs{footnoterule}.}
% \begin{macro}{\pcol@@footnote}
% \changes{v1.2-2}{2013/05/11}
% 	{Introduced to keep the original definition of \cs{footnote}.}
% \begin{macro}{\pcol@@footnotemark}
% \changes{v1.2-2}{2013/05/11}
% 	{Introduced to keep the original definition of \cs{footnotemark}.}
% \begin{macro}{\pcol@@footnotetext}
% \changes{v1.2-2}{2013/05/11}
% 	{Introduced to keep the original definition of \cs{footnotetext}.}
% 
% The macros \!\pcol@footnoterule!, \!\pcol@@footnote!,
% \!\pcol@@footnotemark! and \!\pcol@@footnotetext! are to keep the original
% defitions of \!\footnoterule!, \!\footnote!, \!\footnotemark! and
% \!\footnotetext! in them, respectively, so that we define our own versions
% with references to the originals.
% 
%    \begin{macrocode}
\let\pcol@footnoterule\footnoterule
\let\pcol@@footnote\footnote
\let\pcol@@footnotemark\footnotemark
\let\pcol@@footnotetext\footnotetext
%    \end{macrocode}
% \end{macro}\end{macro}\end{macro}\end{macro}
% 
% \KeepSpace{4}
% \begin{macro}{\footnote}
% \begin{macro}{\pcol@footnote}
% \changes{v1.2-2}{2013/05/11}
% 	{Introduced for \cs{footnote*} and footnote counter adjustment.}
% \begin{macro}{\pcol@ifootnote}
% \changes{v1.2-2}{2013/05/11}
% 	{Introduced for \cs{footnote*} and footnote counter adjustment.}
% \begin{macro}{\footnotemark}
% \begin{macro}{\pcol@footnotemark}
% \changes{v1.2-2}{2013/05/11}
% 	{Introduced for \cs{footnotemark*} and footnote counter adjustment.}
% \begin{macro}{\pcol@ifootnotemark}
% \changes{v1.2-2}{2013/05/11}
% 	{Introduced for \cs{footnotemark*} and footnote counter adjustment.}
% 
% The macros \!\pcol@footnote! and \!\pcol@footnotemark! are the
% implementations of our own versions of \!\footnote! and \!\footnotemark!
% which are made \!\let!-equal to them by \!\pcol@zparacol!, respectively.
% The reasons why we need to have our own are two-fold; to have starred
% version of them; and to maintain $\nf=\!\pcol@nfootnotes!$ for the
% \counter{footnote} counter adjustment.
% 
% The implementations of the starred versions
% \!\footnote!|*|\oarg{num}\marg{text} and 
% \!\footnotemark!|*|\~\oarg{num} have common structure in which we invoke
% \!\pcol@adjustfnctr!$\arg{macro}$\oarg{num} if `|*|' is given, to let
% \!\c@footnote! have the number relative to $\bf=\!\pcol@footnotebase!$ or
% to itself.  Then the macros \!\pcol@ifootnote! or \!\pcol@ifootnotemark!
% are invoked from \!\pcol@adjustfnctr! or the else-part of \!\@ifstar! to
% perform the operations common to both cases with and without `|*|', i.e.,
% invoking the original version \!\pcol@@footnote! or \!\pcol@@footnotemark!
% after incrementing $\nf$.  One caution is that
% $\arg{macro}=\!\pcol@ifootnote!$ for \!\footnote!, but
% $\arg{macro}=|{|\!\pcol@ifootnotemark!\!\relax!|}|$ for \!\footnotemark!
% so that \!\@ifnextchar! in \!\pcol@@footnotemark!  invoked from
% \!\pcol@ifootnotemark! eats \!\relax! to terminate space skipping and thus
% spaces following \oarg{num} are kept.
% 
%    \begin{macrocode}
\def\pcol@footnote{\@ifstar{\pcol@adjustfnctr\pcol@ifootnote}\pcol@ifootnote}
\def\pcol@ifootnote{\global\advance\pcol@nfootnotes\@ne \pcol@@footnote}
\def\pcol@footnotemark{\@ifstar
  {\pcol@adjustfnctr{\pcol@ifootnotemark\relax}}%
  \pcol@ifootnotemark}
\def\pcol@ifootnotemark{\global\advance\pcol@nfootnotes\@ne
  \pcol@@footnotemark}
%    \end{macrocode}
% \end{macro}\end{macro}\end{macro}\end{macro}\end{macro}\end{macro}
% 
% \KeepSpace{1}
% \begin{macro}{\pcol@adjustfnctr}
% \changes{v1.2-2}{2013/05/11}
% 	{Introduced for \cs{footnote*} and \cs{footnotemark*}.}
% \begin{macro}{\pcol@iadjustfnctr}
% \changes{v1.2-2}{2013/05/11}
% 	{Introduced for \cs{footnote*} and \cs{footnotemark*}.}
% \begin{macro}{\pcol@calcfnctr}
% \changes{v1.2-2}{2013/05/11}
% 	{Introduced for \cs{footnote*}, \cs{footnotemark*} and
%	 \cs{footnotetext*}.}
% 
% The macro \!\pcol@adjustfnctr!$\arg{macro}$\oarg{num}, invoked from the
% then-part of \!\@ifstar! in \!\pcol@footnote! and \!\pcol@footnotemark!,
% calculates the number to be set into \!\c@footnote! by
% \!\pcol@calcfnctr!$\arg{num}$\!\@nil! after processing the optional
% argument $\arg{num}$ by \!\pcol@iadjustfnctr! with default `|+1|', and
% then invoke $\arg{macro}$ being \!\pcol@ifootnote! or
% \!\pcol@ifootnotemark!\!\relax!.  Since \!\pcol@calcfnctr! returns the
% number \!\c@footnote! should have and the counter is incremented by
% \!\stepcounter! in \!\pcol@@footnote! or \!\pcol@@footnotemark!, we
% decrement the counter prior to invoke $\arg{macro}$.
% 
% The macro \!\pcol@calcfnctr!$\arg{num}$\!\@nil!, also invoked from
% \!\pcol@iifootnotetext!, calculate $m$ specified by $\arg{num}$ as follows,
% where $f=\!\c@footnote!$, to return it through \!\@tempcnta!.
% $$
% m=\cases{f+k&$\arg{num}=|+|k$\cr
%           f-k&$\arg{num}=|-|k$\cr
%           \bf+k&$\arg{num}=k$}
% $$
% 
%    \begin{macrocode}
\def\pcol@adjustfnctr#1{\@ifnextchar[%]
  {\pcol@iadjustfnctr{#1}}{\pcol@iadjustfnctr{#1}[+1]}}
\def\pcol@iadjustfnctr#1[#2]{\pcol@calcfnctr#2\@nil
  \global\c@footnote\@tempcnta \global\advance\c@footnote\m@ne#1}
\def\pcol@calcfnctr#1#2\@nil{\@tempcnta\c@footnote
  \def\reserved@a{#1}\def\reserved@b{+}%
  \ifx\reserved@a\reserved@b \advance\@tempcnta#2\relax
  \else \def\reserved@b{-}%
  \ifx\reserved@a\reserved@b \advance\@tempcnta-#2\relax
  \else \@tempcnta\pcol@footnotebase \advance\@tempcnta#1#2\relax
  \fi\fi}
%    \end{macrocode}
% \end{macro}\end{macro}\end{macro}
% 
% \KeepSpace{2}
% \begin{macro}{\footnotetext}
% \begin{macro}{\pcol@footnotetext}
% \changes{v1.2-2}{2013/05/11}
% 	{Introduced for \cs{footnotetext*}.}
% \begin{macro}{\pcol@ifootnotetext}
% \changes{v1.2-2}{2013/05/11}
% 	{Introduced for \cs{footnotetext*}.}
% \begin{macro}{\pcol@iifootnotetext}
% \changes{v1.2-2}{2013/05/11}
% 	{Introduced for \cs{footnotetext*}.}
% 
% The macros \!\pcol@footnotetext! is the implementation of our own versions
% of \!\footnotetext!  which is made \!\let!-equal to it by
% \!\pcol@zparacol!.  The reasons why we need to have our own is to have the
% starred version.  That is, if `|*|' is not given, we simply invoke the
% original version \!\pcol@@footnotetext!.  Otherwise we invoke
% \!\pcol@ifootnotetext! which then examines if the optional argument
% \oarg{num} is presented.  If so, we invoke \!\pcol@iifootnotetext! in
% which \!\pcol@calcfnctr!$\arg{num}$\!\@nil! is invoked to have the value
% $m$ being the footnote ordinal with which we invoke
% \!\pcol@@footnotetext!\oarg{\mbox{$m$}} with two \!\expandafter!s to
% extract $m$ from \!\@tempcnta!.  Otherwise, i.e., \oarg{num} is not given,
% we increment \!\c@footnote! by \!\stepcounter! before invoking
% \!\pcol@@footnotetext!.
% 
%    \begin{macrocode}
\def\pcol@footnotetext{\@ifstar\pcol@ifootnotetext\pcol@@footnotetext}
\def\pcol@ifootnotetext{\@ifnextchar[%]
  \pcol@iifootnotetext{\stepcounter{footnote}\pcol@@footnotetext}}
\def\pcol@iifootnotetext[#1]{\pcol@calcfnctr#1\@nil
  \expandafter\pcol@@footnotetext\expandafter[\number\@tempcnta]}

%    \end{macrocode}
% \end{macro}\end{macro}\end{macro}\end{macro}
% 
% 
% 
% \KeepSpace{13}
% \section{Commands for Marginal Notes}
% \label{sec:imp-mpar}
% \changes{v1.35-3}{2018/12/31}
% 	{Add the section ``Commands for Marginal Notes'' to describe newly
%	 introduced macros for the emulation of \cs{marginnote}.}
% 
% \begin{macro}{\marginpar}
% \changes{v1.35-3}{2018/12/31}
% 	{Locally modified in \cs{pcol@zparacol} for the emulation of
%	 \cs{marginnote}.}
% \begin{macro}{\pcol@marginpar}
% \changes{v1.35-3}{2018/12/31}
% 	{Introduced as the in-\texttt{paracol} version of \cs{marginpar} for
%	 the emulation of \cs{marginnote}.}
% \begin{macro}{\pcol@@marginpar}
% \changes{v1.35-3}{2018/12/31}
% 	{Introduced to keep the orginal version of \cs{marginpar} for
%	 the emulation of \cs{marginnote}.}
% \begin{macro}{\@mn@@marginnote}
% \changes{v1.35-3}{2018/12/31}
% 	{Locally modified in \cs{pcol@zparacol} for the emulation of
%	 \cs{marginnote}.}
% \begin{macro}{\pcol@marginnote}
% \changes{v1.35-3}{2018/12/31}
% 	{Introduced as the in-\texttt{paracol} version of
%	 \cs{@mn@@marginnote} for the emulation of \cs{marginnote}.}
% \begin{macro}{\pcol@mn@warning}
% \changes{v1.35-3}{2018/12/31}
% 	{Introduced to put a warning message to show \cs{marginnote} is
%	 emulated.}
% \begin{macro}{\@xympar}
% \changes{v1.35-3}{2018/12/31}
% 	{Locally modified in \cs{pcol@zparacol} for the emulation of
%	 \cs{marginnote}.}
% \begin{macro}{\pcol@xympar}
% \changes{v1.35-3}{2018/12/31}
% 	{Introduced as the in-\texttt{paracol} version of
%	 \cs{@xympar} for the emulation of \cs{marginnote}.}
% \begin{macro}{\pcol@@xympar}
% \changes{v1.35-3}{2018/12/31}
% 	{Introduced to keep the orginal version of \cs{@xympar} for the
%	 emulation of \cs{marginnote}.} 
% \begin{macro}{\pcol@mparoffset}
% \changes{v1.35-3}{2018/12/31}
% 	{Introduced to have the vertical offset for the emulation of
%	 \cs{marginnote}.}
% The API macro \!\marginnote!\oarg{left}\marg{right}\oarg{voffset} given by
% the package \textsf{marginnote} is emulated using
% \!\marginpar!\oarg{left}\marg{right} and \!\pcol@addmarginpar! in
% \!\output! routine.  The basic mechanism is to pass the vertical offset
% $\arg{voffset}$ to \!\pcol@addmarginpar! through $|\dimen|\arg{b}$ where
% $b$ is the \!\insert! to carry $\arg{left}$.  The offset passing is
% implemented as follows.
% 
% \begin{itemize}
% \item
% \!\marginpar! is made \!\let!-equal to our own version \!\pcol@marginpar!
% in \!\pcol@zparacol! so that it \!\let! the macro \!\pcol@mparoffset! be
% \!\z@! and then invoke \LaTeX's origincal version kept in
% \!\pcol@@marginpar!, because the marginal note given by \!\marginpar! will
% not be shifted.
% 
% \item
% The internal macro
% \!\@mn@@marginnote!\oarg{left}\marg{right}\oarg{voffset} defined in
% \textsf{marginnote} is made \!\let!-equal to our own version
% \!\pcol@marginnote! in \!\pcol@zparacol! so that it |\def|ines
% \!\pcol@mparoffset! to have $\arg{voffset}$ and then invoke
% \!\pcol@@marginpar!\oarg{left}\~\marg{right} for the emulation.  In the
% invokation, \textsf{marginnote}'s typesetting macros \!\margin~font!,
% \!\raggedleftmarginnote! and \!\raggedrightmarginnote! are attached to
% $\arg{left}$ and $\arg{right}$.
% 
% \item
% \LaTeX's internal macro \!\@xympar! for the last operations of
% \!\marginpar! is made \!\let!-equal to our own version in
% \!\pcol@zparacol! so that it assigns the offset in \!\pcol@mparoffset! to
% |\dimen|\!\@marbox! for $\arg{left}$, if $\!\@floatpenalty!<0$ to mean
% other macros for \!\marginpar! have not detected any errors.
% \end{itemize}
% 
% In addition, we raise a warning that \!\marginnote! is emulated by
% \!\pcol@mn@warning!, which is made \!\let!-equal to \!\relax! in the caller
% \!\pcol@marginnote! after the invokation so that the warning message is
% put just once.
% 
%    \begin{macrocode}
%% Commands for Marginal Notes

\def\pcol@marginpar{\let\pcol@mparoffset\z@ \pcol@@marginpar}
\long\def\pcol@marginnote[#1]#2[#3]{\endgroup
  \pcol@mn@warning \global\let\pcol@mn@warning\relax
  \def\pcol@mparoffset{#3}%
  \pcol@@marginpar[\marginfont\raggedleftmarginnote#1]%
                  {\marginfont\raggedrightmarginnote#2}}
\def\pcol@mn@warning{%
  \PackageWarning{paracol}{\string\margninnote\space is emulated by
    \string\marginpar.}}
\def\pcol@xympar{%
  \ifnum\@floatpenalty<\z@ \global\dimen\@marbox\pcol@mparoffset\relax \fi
  \pcol@@xympar}

%    \end{macrocode}
% \end{macro}\end{macro}\end{macro}\end{macro}\end{macro}
% \end{macro}\end{macro}\end{macro}\end{macro}\end{macro}
% 
% 
% \KeepSpace{9}
% \section{Two-Sided Typesetting}
% \label{sec:imp-swap}
% \changes{v1.2-4}{2013/05/11}
%	{Add the section ``Column-Swapping'' to describe newly
%	 introduced macros for column-swapping.}
% \changes{v1.3-5}{2013/09/17}
%	{Rename the section title from ``Column-Swapping'' to ``Two-Sided
%	 Typesetting''.}
% 
% \begin{macro}{\twosided}
% \changes{v1.3-5}{2013/09/17}
%	{Introduced as an easier API for various two-sided typesetting.}
% \changes{v1.3-3}{2013/09/17}
%	{Add two-sided background painting.}
% \changes{v1.3-4}{2013/09/17}
%	{Add two-sided marginal note placement.}
% \begin{macro}{\pcol@twosided}
% \changes{v1.3-5}{2013/09/17}
%	{Introduced to implement \cs{twosided}.}
% \changes{v1.31}{2013/10/10}
%	{Fix spelling ``twosiding'' replacing it with ``two-siding'' in the
%	 error message.}
% \begin{macro}{\pcol@twosided@p}
% \changes{v1.3-5}{2013/09/17}
%	{Introduced to implement \cs{twosided} with \string\texttt{[p]}.}
% \begin{macro}{\pcol@twosided@c}
% \changes{v1.3-5}{2013/09/17}
%	{Introduced to implement \cs{twosided} with \string\texttt{[c]}.}
% \begin{macro}{\pcol@twosided@m}
% \changes{v1.3-5}{2013/09/17}
%	{Introduced to implement \cs{twosided} with \string\texttt{[m]}.}
% \changes{v1.3-4}{2013/09/17}
%	{Introduced for two-sided marginal note placement.}
% \begin{macro}{\pcol@twosided@b}
% \changes{v1.3-5}{2013/09/17}
%	{Introduced to implement \cs{twosided} with \string\texttt{[b]}.}
% \changes{v1.3-3}{2013/09/17}
%	{Introduced for two-sided background painting.}
% \begin{macro}{\swapcolumninevenpages}
% \changes{v1.2-4}{2013/05/11}
%	{Introduced to enable column-swapping.}
% \begin{macro}{\noswapcolumninevenpages}
% \changes{v1.2-4}{2013/05/11}
%	{Introduced to disable column-swapping.}
% 
% \def\inornot{\mathrel{{\in}/{\notin}}}
% 
% The API macro $\!\twosided!|[|T|]|$ where $T=t_1t_2\cdots$ is to
% enable\slash disable two-sided paging with
% $\CSIndex{if@twoside}=\true/\false$ ($|p|\inornot T$), two-sided \cswap{}
% with $\CSIndex{ifpcol@swapcolumn}\~={\true/\false}$ ($|c|\inornot T$),
% two-sided % marginal note placement with
% $\CSIndex{ifpcol@swapmarginpar}={\true/\false}$ ($|m|\inornot T$), and/or
% two-sided \bgpaint{} with $\CSIndex{ifpcol@bg@swap}={\true/\false}$
% ($|b|\inornot{}T$) individually, or to enable all of them as a whole
% when the optional argument is not given.
% 
% The macro invokes \!\pcol@twosided! with the optional argument $T$ if
% provided, or with $T=|pcmb|$ otherwise to enable all features.  Then
% \!\pcol@twosided! at first turns all the switches $\false$ an then scans
% all non-space tokens $t\in T$ invoking $|\pcol@twosided@|{\cdot}t$ if it
% is defined and thus $t\in\{|p|,|c|,|m|,|b|\}$ to turn the corresponding
% switch $\true$, or complains that the feature $t$ is unknown.
% 
% Note that backward-compatible API macros 
% \!\swapcolumninevenpages! and \!\noswapcolumn~inevenpages! are still
% availabel to turn $\CSIndex{ifpcol@swapcolumn}$ $\true$ an $\false$
% respectivey.
% 
%    \begin{macrocode}
%% Column Swapping

\def\twosided{\@ifnextchar[%]
  {\pcol@twosided}{\pcol@twosided[pcmb]}}
\def\pcol@twosided[#1]{%
  \global\@twosidefalse \global\pcol@swapcolumnfalse
  \global\pcol@swapmarginparfalse \global\pcol@bg@swapfalse
  \@tfor\reserved@a:=#1\do{%
    \@ifundefined{pcol@twosided@\reserved@a}%
      {\PackageError{paracol}{Unknown two-siding feature \reserved@a}}%
      {\@nameuse{pcol@twosided@\reserved@a}}}}
\def\pcol@twosided@p{\global\@twosidetrue}
\def\pcol@twosided@c{\global\pcol@swapcolumntrue}
\def\pcol@twosided@m{\global\pcol@swapmarginpartrue}
\def\pcol@twosided@b{\global\pcol@bg@swaptrue}

\def\swapcolumninevenpages{\global\pcol@swapcolumntrue}
\def\noswapcolumninevenpages{\global\pcol@swapcolumnfalse}

%    \end{macrocode}
% \end{macro}\end{macro}\end{macro}\end{macro}\end{macro}\end{macro}
% \end{macro}\end{macro}
% 
% \begin{macro}{\pcol@swapcolumn}
% \changes{v1.2-4}{2013/05/11}
%	{Introduced to convert column ordinal and its position.}
% \changes{v1.3-2}{2013/09/17}
%	{Add two arguments $C^0$ and $C^1$ as the third and fourth ones to
%	 modifiy the calculation of $c_2$ with them for column-swapping with
%	 parallel-paging.}
% \changes{v1.3-3}{2013/09/17}
%	{Add the assignment of \cs{pcol@colsepid} to let it have $c_2-1$ if
%	 swapped or $c_2$ otherwise.}
% \changes{v1.3-4}{2013/09/17}
%	{Add a user \cs{pcol@addmarginpar} and remove \cs{paracol},
%	 \cs{pcol@sptxt} and \cs{pcol@switchcol}.} 
% \begin{macro}{\pcol@colsepid}
% \changes{v1.3-3}{2013/09/17}
%	{Introduced to be let have $c_2-1$ if swapped or $c_2$ otherwise by
%	 \cs{pcol@swapcolumn}.}
% 
% The macro \!\pcol@swapcolumn!$\<c_1\>\<c_2\>\<\Cfrom\>\<\Cto\>$ converts
% the column ordinal $c$ or position $c'$ in the \!\count!
% register\footnote{
% 
% Or the \!\dimen! register \!\z@!.}
% 
% $c_1$ to the position or ordinal to set it in the \!\count! register
% $c_2$, for a \parapag{}e having columns $c\In\Cfrom\Cto$.  That is, we let
% $c_2=(\Cto-1)-(c_1-\Cfrom)$ if $\CSIndex{ifpcol@swapcolumn}=\true$ to mean
% the \cswap{} is in effect and \!\c@page! is even, while $c_2=c_1$
% otherwise.  We also let $c^g=\!\pcol@colsepid!=c_2-1$ if swapped, or
% $c^g=c_2$ otherwise, so that it has the ordinal of the \csepgap{}
% {\em physically} following the column $c_2$.
% 
% The macro is used in \!\pcol@ioutputelt!, \!\pcol@addmarginpar!,
% \!\pcol@imakeflushedpage! and \!\pcol@iflushfloats! with
% $(c_1,c_2)=(c',c)$, and in \!\pcol@addmarginpar! (another use) with
% $(c_1,c_2)=(c,c')$.  Note that in the uses in the macros above except for
% \!\pcol@addmarginpar!, \!\c@page!  definitely has the page number for the
% page to be shipped out.  As for \!\pcol@addmarginpar! on the other hand,
% \!\c@page! can be different from the ship-out page number to produce a
% weird result if their parities are different, due to page jump.  However
% this problem is not so severe because it just affects the position of
% marginal notes which \LaTeX{} itself may misplace.
% 
%    \begin{macrocode}
\def\pcol@swapcolumn#1#2#3#4{%
  \edef\pcol@colsepid{\number#1}%
  \ifpcol@swapcolumn
    \ifodd\c@page\relax #2#1\relax
    \else
      #2#4\relax \advance#2-#1\relax \advance#2#3\relax \advance#2-\tw@
      \edef\pcol@colsepid{\number#2}%
      \advance#2\@ne
    \fi
  \else #2#1\relax
  \fi}

%    \end{macrocode}
% \end{macro}\end{macro}
% 
% \KeepSpace{2}
% \begin{macro}{\marginparthreshold}
% \changes{v1.3-4}{2013/09/17}
%	{Introduced to spacify the smallest ordinal of columns whose marginal
%	 notes go to the right margin if not swapped.}
% \begin{macro}{\pcol@marginparthreshold}
% \changes{v1.3-4}{2013/09/17}
%	{Introduced to implement \cs{marginparthreshold}.}
% \begin{macro}{\pcol@mpthreshold@l}
% \changes{v1.3-4}{2013/09/17}
%	{Introduced to keep the value specified by \cs{marginparthreshold}
%	 for columns in left parallel-pages.}
% \begin{macro}{\pcol@mpthreshold@r}
% \changes{v1.3-4}{2013/09/17}
%	{Introduced to keep the value specified by \cs{marginparthreshold}
%	 for columns in right parallel-pages.}
% 
% The API macro $\!\marginparthreshold!\Arg{t_l}|[|t_r|]|$ determines the
% smallest ordinal $t_l$ of columns in left \parapag{}es whose marginal
% notes go to the right margin in fundamental setting of marginal note
% positioning, while the threshold in right \parapag{}es is given by $t_r$
% if provided or by $t_l$ otherwise.  That is, marginal notes given in a
% column $c$ in a page $p$ s.t.\ $c\In0\CL$ (resp.\ $\LBRP\CL\C$) go left if
% $c<t_l$ (resp.\ $c<t_r$) while they go right if $c\geq t_l$ (resp.\ $c\geq
% t_r$), providing
% $$
% (\CSIndex{ifpcol@swapmarginpar}\;\land\;\page(p)\bmod2=0)
% \not\equiv\CSIndex{if@reversemargin}=\false
% $$
% or the margins are swapped otherwise.
% 
% The macro \!\def!ines \!\pcol@mpthreshold@l! to let it have $t_l$ after a
% assigning $t_l$ to \!\@tempcnta! to ensure $t_l$ gives some number, and
% then do the same for \!\pcol@mpthreshold@r! with $t_r$ by
% \!\pcol@marginparthreshold! if $t_r$ is provided, or let the marco have
% $t_l$ otherwise.  Note that at the top level we do
% \!\marginparthreshold!|{1}| to give defaults.  Also note that
% \!\pcol@mpthreshold@l! and \!\pcol@mpthreshold@r! are referred to solely
% in \!\pcol@addmarginpar!.
% 
%    \begin{macrocode}
\def\marginparthreshold#1{\@tempcnta#1\relax
  \xdef\pcol@mpthreshold@l{\number\@tempcnta}%
  \@ifnextchar[%]
    \pcol@marginparthreshold{\xdef\pcol@mpthreshold@r{\number\@tempcnta}}}
\def\pcol@marginparthreshold[#1]{\@tempcnta#1\relax
  \xdef\pcol@mpthreshold@r{\number\@tempcnta}}
\marginparthreshold{1}

%    \end{macrocode}
% \end{macro}\end{macro}\end{macro}\end{macro}
% 
% 
% \KeepSpace{5}
% \section{Commands for Text Coloring}
% \label{sec:imp-commcolor}
% \changes{v1.3-6}{2013/09/17}
%	{Add the section ``Commands for Text Coloring'' to distinguish
%	 macros inside and outside \cs{output} routine and describe latters
%	 in this section.}
% 
% \begin{macro}{\columncolor}
% \changes{v1.2-1}{2013/05/11}
% 	{Introduced to define the default color of a column.}
% \changes{v1.22}{2013/06/30}
%	{Add the definition of \cs{pcol@colorcommand} for warning in
% 	 \cs{pcol@icolumncolor}.}
% \begin{macro}{\pcol@xcolumncolor}
% \changes{v1.2-1}{2013/05/11}
% 	{Introduced to implement \cs{columncolor}.}
% \begin{macro}{\pcol@ycolumncolor}
% \changes{v1.2-1}{2013/05/11}
% 	{Introduced to implement \cs{columncolor}.}
% \begin{macro}{\pcol@columncolor}
% \changes{v1.2-1}{2013/05/11}
% 	{Introduced to implement \cs{columncolor}.}
% 
% The API macro \!\columncolor!\oarg{mode}\marg{color}\oarg{c} defines the
% default color specified by $\arg{color}$ optionally with color
% $\arg{mode}$ of the column $c$ being the current column or that specified
% by the optional argument.  After \!\def!ining \Midx{\!\pcol@colorcommand!}
% being the \!\string! of this macro itself, and processing two optional
% arguments $\arg{mode}$ and $c$ through macros \!\pcol@xcolumncolor!,
% \!\pcol@ycolumncolor! and \!\pcol@columncolor!, the macro
% \!\pcol@icolumncolor!\marg{cmd}\oarg{c} is invoked to perform real
% operations with the coloring command
% $\arg{cmd}=\!\color!$\oarg{mode}\marg{color}.
% 
%    \begin{macrocode}
%% Commands for Text Coloring

\def\columncolor{\def\pcol@colorcommand{\string\columncolor}%
  \@ifnextchar[%]
    \pcol@xcolumncolor\pcol@ycolumncolor}
\def\pcol@xcolumncolor[#1]#2{\pcol@columncolor{\color[#1]{#2}}}
\def\pcol@ycolumncolor#1{\pcol@columncolor{\color{#1}}}
\def\pcol@columncolor#1{\@ifnextchar[%]
  {\pcol@icolumncolor{#1}}{\pcol@icolumncolor{#1}[\number\pcol@currcol]}}
%    \end{macrocode}
% \end{macro}\end{macro}\end{macro}\end{macro}
% 
% \begin{macro}{\normalcolumncolor}
% \changes{v1.2-1}{2013/05/11}
% 	{Introduced to define the default color of a column is
%	 \cs{normalcolor}.} 
% \changes{v1.22}{2013/06/30}
%	{Add the definition of \cs{pcol@colorcommand} for warning in
% 	 \cs{pcol@icolumncolor}.}
% 
% The API macro \!\normalcolumncolor!\oarg{c} defines the default color of
% the column $c$, being the current column or that specified by the optional
% argument, is \!\normalcolor!.  That is, after \!\def!ining
% \Midx{\!\pcol@colorcommand!}  being the \!\string! of this macro itself,
% this macro simply invokes \!\pcol@icolumncolor!\marg{cmd}\oarg{c} to
% perform real operations with the coloring command
% $\arg{cmd}=\!\normalcolor!$.
% 
%    \begin{macrocode}
\def\normalcolumncolor{\def\pcol@colorcommand{\string\normalcolumncolor}%
  \@ifnextchar[%]
    {\pcol@icolumncolor\normalcolor}%
    {\pcol@icolumncolor\normalcolor[\number\pcol@currcol]}}
%    \end{macrocode}
% \end{macro}
% 
% \KeepSpace{1}
% \changes{v1.22}{2013/06/30}
%	{\cs{pcol@getshadowcc} was introduced for setting $\hat\gamma_0^c$
%	 into $\gamma_0^c$ locally, but removed in v1.34.} 
% \changes{v1.34}{2018/05/07}
%	{\cs{pcol@getshadowcc} is removed according to the change of text
%	 coloring from \cs{output} to \cs{insert}.}
% \begin{macro}{\pcol@icolumncolor}
% \changes{v1.2-1}{2013/05/11}
% 	{Introduced to implement \cs{columncolor} and
%	 \cs{normalcolumncolor}.} 
% \changes{v1.22}{2013/06/30}
%	{Add warning of ineffective uses of \cs{columncolor} and
%	 \cs{normalcolumncolor}, and modify the mechanism to update
%	 $\gamma_0^c$ and to rewind\slash reestablish color stack.}
% \changes{v1.23}{2013/07/08}
%	{Add an argument of \cs{relax} to \cs{pcol@color@invokeoutput}
%	 because any insertion after \cs{vadjust} is not required or
%	 justified.}
% \changes{v1.24}{2013/07/27}
%	{Add math mode to the cases of ineffective use.}
% \changes{v1.34}{2018/05/07}
%	{Remove the invocations of \cs{pcol@iicolumncolor} for
%	 $\cs{pcol@columncolor@shadow}\cdot c$ because it no longer exists,
%	 completly change the opearations in the case the target column $c$
%	 is the current one according to the new method with \cs{insert},
%	 and add immediate setting of $\gamma_0^c$ in the case $c$ is not
%	 current.}
% \begin{macro}{\pcol@iicolumncolor}
% \changes{v1.22}{2013/06/30}
%	{Introduced for setting $\gamma_0^c$ and $\hat\gamma_0^c$, and
%	 pushing $\chi$.}
% \changes{v1.34}{2018/05/07}
%	{Remove the third argument, change the second argument from a
%	 control sequence name to the target column, add a grouping to
%	 surround the entire body of the macro, and change the body of
%	 $\gamma_0^c$ so that it only has the color information.}
% \begin{macro}{\pcol@scancst@shadow}
% \changes{v1.34}{2018/05/07}
%	{Introduced to rewind or establish the color stack $\hat\Gamma^c$.}
% 
% The macro \!\pcol@icolumncolor!\marg{cmd}\oarg{c}, invoked from
% \!\pcol@columncolor! and \!\normal~columncolor!, performs the operations to
% define the default color of the column $c$ with the coloring command
% $\arg{cmd}\in\{\!\color!\hbox{\oarg{mode}}\ARg{color},\,\!\normalcolor!\}$
% as follows.  First we examine if \!\set@color! is not \!\relax! and we are
% in non-internal vertical or non-restricted horizontal mode and, if not, we
% complain the command whose name is in \!\pcol@colorcommand! is ineffective
% by \!\PackageWarning! and do nothing.
% 
% Otherwise and if we are not in a \env{paracol} environment, i.e.,
% \!\paracol! is not \!\let!-equal to \!\pcol@paracol!, we simply invoke
% \!\pcol@iicolumncolor! to let $\Celtshadow^c=|\pcol@columncolor|\cdot c$
% have the color $\chi$ specified by $\arg{cmd}$ so that the next
% \beginparacol{} will let $\Celt^c$ have the coloring \!\special!  for
% $\chi$.  If we are in a \env{paracol} environment but in a column $c'\neq
% c$, on the other hand, we also let $\Celtshadow=\chi$ but in addition let
% $\Celt^c=|\pcol@columncolor@box|\cdot c$ have the coloring \!\special! for
% $\chi$ immediately so that it is effective in the next \cswitch{} to $c$.
% This immediate setting of $\Celt^c$ is done by invoking $\arg{cmd}$ with
% the original \!\set@color!  saved in \!\pcol@set@color! and the
% nullification of \!\aftergroup!, after acquiring an \!\insert! for it if
% necessary,
% 
% \SpecialArrayIndex{c}{\pcol@columncolor}
% \SpecialArrayIndex{c}{\pcol@columncolor@box}
% 
% Otherwise, i.e., if we are in a \env{paracol} environment and in the
% column $c$, at first we invoke \!\pcol@scancst@shadow! to rewind
% $\CSTshadow^c$ applying $\!\@elt!=\!\reset@color!$ to
% $\celtshadow_i\in\CSTshadow^c$.  Then, after letting $\Celtshadow^c=\chi$,
% we invoke \!\pcol@scancst@shadow! again to reestablish $\CSTshadow^c$ with
% the new $\Celtshadow^c$ so that $\Celtshadow^c$ is at the bottom of the
% \colorstack{} in |.tex|.  In this scan $\!\@elt!\arg{\celtshadow_i}$
% \!\def!ines \!\current@color! to let it have $\celtshadow_i$ and then
% invokes \!\pcol@set@color! to put the coloring \!\special! for
% $\celtshadow_i$ nullifying \!\aftergroup!.  Then we \!\insert! a \!\vbox!,
% whose height and depth are 1\,|pt| and width is 0, having the coloring
% \!\special! for $\chi$ so that \!\output! will let $\Celt^c$ have the
% \!\special! in a synchronous manner.  After that we put a
% $\!\penalty!=10000$ if $\CSIndex{if@nobreak}=\true$ to keep the
% \!\insert!ion from being followed by a page break.
% 
% The macro $\!\pcol@iicolumncolor!\arg{cmd}\arg{c}$ at first
% invokes $\arg{cmd}$ to let \!\current@color! have the printer-specific color
% information $\chi$ of $\arg{color}$ or what \!\normalcolor! specifies,
% temporarily letting \!\set@color! be \!\relax! to let \!\color! or
% \!\normalcolor! just do the \!\def!inition of \!\current@color! without
% putting coloring \!\special!s nor preparing \colorstack{} popping.  Then
% we \!\xdef!ine $\Celtshadow^c=|\pcol@columncolor|{\cdot}c$ to have $\chi$.
% 
% \SpecialArrayIndex{c}{\pcol@columncolor}
% 
% \begingroup\hfuzz3.1pt
% The macro \!\pcol@scancst@shadow! applies \!\@elt! to $\Celtshadow^c$ to
% put a coloring or uncoloring \!\special! for it if it is defined, and then
% do the same for all
% $\celtshadow_i\in\cstshadow=\!\pcol@colorstack@shadow!$.
% \par\endgroup
% 
%    \begin{macrocode}
\def\pcol@icolumncolor#1[#2]{%
  \@tempswafalse
  \ifpcol@inner \@tempswatrue \fi
  \ifinner      \@tempswatrue \fi
  \ifmmode      \@tempswatrue \fi
  \ifx\set@color\relax
    \PackageWarning{paracol}{\pcol@colorcommand\space is not effective
      without some coloring package}%
  \else\if@tempswa
    \PackageWarning{paracol}{\pcol@colorcommand\space is not effective
      when not in outer par mode}%
  \else
    \begingroup
    \let\@elt\relax
    \ifx\pcol@paracol\paracol
      \pcol@iicolumncolor{#1}{#2}%
    \else\ifnum#2=\pcol@currcol
      \def\@elt##1{\reset@color}\pcol@scancst@shadow
      \pcol@iicolumncolor{#1}{#2}%
      \def\@elt##1{\def\current@color{##1}\let\aftergroup\@gobble
        \pcol@set@color}%
      \pcol@scancst@shadow
      \setbox\@tempboxa\vbox{\let\set@color\pcol@set@color
        \let\aftergroup\@gobble #1}%
      \ht\@tempboxa1sp \dp\@tempboxa1sp \wd\@tempboxa\z@\relax
      \insert\pcol@colorins{\box\@tempboxa}%
      \ifvmode\if@nobreak \nobreak \fi\fi
    \else
      \pcol@iicolumncolor{#1}{#2}%
      \pcol@currcol#2\relax
      \ifvoid\pcol@ccuse{@box}%
        \@next\@currbox\@freelist{}\pcol@ovf
        \pcol@ccxdef{\@currbox}%
      \fi
      \global\setbox\pcol@ccuse{@box}\vbox{\let\set@color\pcol@set@color
        \let\aftergroup\@gobble #1}%
    \fi\fi
    \endgroup
  \fi\fi
  \ignorespaces}
\def\pcol@iicolumncolor#1#2{{\let\set@color\relax #1%
  \expandafter\xdef\csname pcol@columncolor#2\endcsname{\current@color}}}
\def\pcol@scancst@shadow{%
  \pcol@ifccdefined{\@elt{\pcol@ccuse{}}}\relax
  \pcol@colorstack@shadow}

%    \end{macrocode}
% \end{macro}\end{macro}\end{macro}
% 
% \KeepSpace{1}
% \begin{macro}{\pcol@mcpushlimit}
% \changes{v1.24}{2013/07/27}
%	{Introduced for coloring specified in math mode.}
% \changes{v1.34}{2018/05/07}
%	{Move down to place it just before the \cs{def}inition of
%	 \cs{pcol@set@color@push} being the sole referrer, and change its
%	 body from 100 to 1000.}
% \begin{macro}{\set@color}
% \begin{macro}{\pcol@set@color}
% \begin{macro}{\pcol@set@color@push}
% \changes{v1.2-1}{2013/05/11}
% 	{Introduced to work as \cs{set@color}.}
% \changes{v1.22}{2013/06/30}
%	{Modified to push color stack by \cs{output} always.}
% \changes{v1.23}{2013/07/08}
%	{Add an argument of null \cs{hskip} to \cs{pcol@color@invokeoutput}
%	 so that the first word of a colored text is hyphenated.}
% \changes{v1.24}{2013/07/27}
%	{Add the mechanism special for math mode.}
% \changes{v1.3-6}{2013/09/17}
%	{Change the second argument of \cs{pcol@color@invokeoutput} from
%	 \cs{hskip}\cs{z@} to \cs{pcol@fcwhyphenate} to make null skip
%	 insertion conditional.}
% \changes{v1.34}{2018/05/07}
%	{Completely change its definition according to the new text coloring
%	 with \cs{insert}.}
% 
% The macro \!\pcol@set@color@push! is invoked whenever \LaTeX's
% counterpart \!\set@color!  appears in a \env{paracol} environment through
% coloring commands such as \!\color!, because \!\pcol@zparacol! replaces
% \LaTeX's macro with it saving the original version in \!\pcol@set@color!,
% if the original \!\set@color! is not \!\relax! to mean some coloring
% package is in use.  This original version is used through
% \!\pcol@set@color! by \!\pcol@bg@paintregion@i!, \!\pcol@output@start! and
% \!\pcol@icolumncolor!  besides this macro \!\pcol@set@color@push!,
% while \!\output! lets $\!\set@color!=\!\pcol@set@color!$ for the references
% outside of our control.
% 
% The macro at first invokes its original version being \!\pcol@set@color!
% to put an appropriate coloring \!\special! to |.dvi| and reserve the
% invocation of \!\reset@color! by \!\aftergroup!.  Then, it performs one of
% two different operations depending on TeX's mode, i.e., math mode or not.
% If we are in math mode and not in a \!\vbox!, at first we increment
% $m=\!\pcol@mcid!$ and examine if $m>\!\pcol@mcpushlimit!=1000$, and if so
% we stop the execution with \!\PackageError!\footnote{
% 
% And let $m=1$ to allow a user to continue the execution bravely.}
% 
% in order to avoid too many macros $|\pcol@reset@color@mpop@|\cdot m$ are
% defined\footnote{
% 
% \SpecialArrayIndex{m}{\pcol@reset@color@mpop@}
% 
% We could make the number of math-mode coloring operations virtually
% unlimited by putting all digits of the decimal representation of $m$
% followed by a terminator by multiple
% \cs{aftergroup}s so that \cs{pcol@reset@color@mpop} is put by
% \cs{aftergroup} prior to them to capture them as its argument, but 
% limiting with $2^{31}-1$ is still necessary and that with 1000 is
% reasonable.}. 
% 
% Otherwise, i.e., if $m\leq\!\pcol@mcpushlimit!$, we
% reserve the invocation of the macro $|\pcol@reset@color@mpop@|{\cdot}i$
% 
% \SpecialArrayMainIndex{m}{\pcol@reset@color@mpop@}
% 
% for our own pop by \!\aftergroup! defining the macro as
% $\!\pcol@reset@color@mpop!\Arg{m}$.  If we are not in math mode, on the
% other hand, and neither in a \!\vbox!  nor in restricted horizontal mode,
% we simply reserve the invocation of the macro $\pcol@reset@color@pop$.
% 
% Then, regardless that we are in math mode or not, we push the contents of
% \!\current@color!, which \!\set@color! should refer to as the color
% information to be set, into the shadow \colorstack{}
% $\cstshadow=\!\pcol@colorstack@shadow!$ for the stack rewinding\slash
% reestablishing in \!\columncolor! and \!\normalcolumncolor!.  Since this
% push is done non-\!\global!ly with \!\edef!, we save\slash restore the
% definition of \!\@elt! to\slash from \Midx{\!\pcol@elt@save!} before\slash
% after the push, resepctively\footnote{
% 
% Just in case.}.
% 
% Then we \!\insert! a \!\vbox! through \!\pcol@colorins! for the push of
% $\celt_i$ or $\mcelt_{i,m}$ to $\cstraw$ synchronous with a page break or
% \cswitch{}.  The height of the \!\vbox! is 1\,|pt|, depth is 0 and
% width is $m$\,|sp| if we are in math mode or 0 otherwise, and its
% contents is the coloring \!\special! given by \!\pcol@set@color! so that
% the \!\special! is what the macro put at the beginning of this macro but
% without the reservation of \!\reset@color!.  After the insertion, we put
% \!\pcol@fcwhyphenate!, being \!\hskip!\!\z@! when
% \!\coloredwordhyphenated!  is effective, to split the coloring \!\special!
% from the first colored word so that the word may be hyphenated if we are
% in horizontal mode.  If we are in vertical mode, on the other hand, we do
% \!\nobreak! if $\CSIndex{if@nobreak}=\true$ to keep the \!\insert!ion from
% being followed by a page break.
% 
%    \begin{macrocode}
\def\pcol@mcpushlimit{1000}
\def\pcol@set@color@push{\pcol@set@color
  \ifmmode\else\ifinner \pcol@innertrue \fi\fi
  \ifpcol@inner\else
    \ifmmode
      \global\advance\pcol@mcid\@ne
      \ifnum \pcol@mcid>\pcol@mcpushlimit\relax
        \PackageError{paracol}{Too many coloring commands in math mode}\@ehb
        \global\pcol@mdid\@ne
      \fi
      \@tempdima\pcol@mcid sp\relax
      \expandafter\aftergroup
        \csname pcol@reset@color@mpop@\number\pcol@mcid\endcsname
      \expandafter\xdef
        \csname pcol@reset@color@mpop@\number\pcol@mcid\endcsname
          {\noexpand\pcol@reset@color@mpop{\number\pcol@mcid}}%
    \else
      \aftergroup\pcol@reset@color@pop \@tempdima\z@
    \fi
    \let\pcol@elt@save\@elt \let\@elt\relax
    \edef\pcol@colorstack@shadow{\pcol@colorstack@shadow\@elt{\current@color}}%
    \let\@elt\pcol@elt@save
    \setbox\@tempboxa\vbox{\let\aftergroup\@gobble \pcol@set@color}%
    \ht\@tempboxa1sp \dp\@tempboxa\z@ \wd\@tempboxa\@tempdima
    \insert\pcol@colorins{\box\@tempboxa}\ifhmode \pcol@fcwhyphenate \fi
    \ifvmode\if@nobreak \nobreak \fi\fi
  \fi}

%    \end{macrocode}
% \end{macro}\end{macro}\end{macro}\end{macro}
% 
% \begin{macro}{\pcol@reset@color@pop}
% \changes{v1.2-1}{2013/05/11}
% 	{Introduced to work as \cs{reset@color}.}
% \changes{v1.22}{2013/06/30}
%	{Modified to pop color stack by \cs{output} and to examine if
%	 $\cs{ifpcol@output}\EQ\mathit{true}$.}
% \changes{v1.23}{2013/07/08}
%	{Add an argument of \cs{relax} to \cs{pcol@color@invokeoutput}
%	 so that the last word of a colored text is not followed by a line
%	 break candidate.}
% \changes{v1.34}{2018/05/07}
%	{Completely change its definition according to the new text coloring
%	 with \cs{insert}.}
% \begin{macro}{\pcol@reset@color@mpop}
% \changes{v1.24}{2013/07/27}
%	{Introduced for coloring specified in math mode.}
% \changes{v1.34}{2018/05/07}
%	{Completely change its definition according to the new text coloring
%	 with \cs{insert}.}
% 
% The macro \!\pcol@reset@color@pop! and its math-mode relative
% $\!\pcol@reset@color@mpop!\~\Arg{m}$ are invoked by
% \!\aftergroup! mechanism in \!\pcol@set@color@push!, directly for the
% former and through the macro $|\pcol@reset@color@mpop@|{\cdot}m$
% 
% \SpecialArrayIndex{m}{\pcol@reset@color@mpop@}
% 
% for the latter.  They \!\insert! a \!\vbox! for $\celtpop_i$ or
% $\mceltpop_{i,m}$ to add it to $\cstraw$ synchronously with a page break
% or \cswitch{}.  Therefore, the height and depth of the \!\vbox! are 0 and
% width is 0 for $\celtpop_i$ or $m$\,|sp| for $\mceltpop_{i,m}$.  The
% contents of the \!\vbox! is an uncoloring \!\special! given by
% \!\reset@color! but this is done just for debugging to show what
% \!\pcol@colorins! has by, for example, \!\pcol@ShowBox!.  Then if we are
% in vertical mode and $\CSIndex{if@nobreak}=\true$, we do \!\nobreak! to
% keep the \!\insert!ion from being followed by a page break even in
% \!\pcol@reset@color@mpop! because its corresponding
% \!\pcol@set@color@push! may have been in a displayed math construct after
% which we are in vertical mode.
% 
% One caution is that \!\pcol@reset@color@pop! can be invoked outside the
% \env{paracol} environment in which the corresponding
% \!\pcol@set@color@push! appears.  In this case with
% $\CSIndex{ifpcol@output}=\false$, we don't need to do the pop operation
% and cannot make the \!\insert!ion for it because \!\output! is not for
% \env{paracol}.
% 
%    \begin{macrocode}
\def\pcol@reset@color@pop{%
  \ifpcol@output
    \setbox\@tempboxa\vbox{\reset@color}%
    \ht\@tempboxa\z@ \dp\@tempboxa\z@ \wd\@tempboxa\z@
    \insert\pcol@colorins{\box\@tempboxa}%
    \ifvmode\if@nobreak \nobreak \fi\fi
  \fi}
\def\pcol@reset@color@mpop#1{%
  \setbox\@tempboxa\vbox{\reset@color}%
  \ht\@tempboxa\z@ \dp\@tempboxa\z@ \wd\@tempboxa#1sp\relax
  \insert\pcol@colorins{\box\@tempboxa}%
  \ifvmode\if@nobreak \nobreak \fi\fi
}

%    \end{macrocode}
% \end{macro}\end{macro}
% 
% \changes{v1.22}{2013/06/30}
% 	{\cs{pcol@color@invokeoutput} was introduced for \cs{output}
%	 request for coloring but removed in v1.34.}
% \changes{v1.23}{2013/07/08}
%	{\cs{pcol@color@invokeoutput} was modified to add second argument
%	 $s$ to insert a null skip after \cs{vadjust} only when the macro is
%	 invoked from \cs{pcol@set@color@push} in horizontal mode.}
% \changes{v1.34}{2018/05/07}
%	{\cs{pcol@color@invokeoutput} is removed according to the change of
%	 text coloring from \cs{output} to \cs{insert}.}
% \changes{v1.22}{2013/06/30}
% 	{\cs{pcol@color@invokeoutput@v} was introduced for \cs{output}
%	 request for coloring but removed in v1.34.}
% \changes{v1.34}{2018/05/07}
%	{\cs{pcol@color@invokeoutput@v} is removed according to the change of
%	 text coloring from \cs{output} to \cs{insert}.}
% 
% \KeepSpace{1}
% \begin{macro}{\coloredwordhyphenated}
% \changes{v1.3-6}{2013/09/17}
%	{Introduced to enable null skip insertion before the first word
%	 after a coloring command not always but conditionally.}
% \begin{macro}{\nocoloredwordhyphenated}
% \changes{v1.3-6}{2013/09/17}
%	{Introduced to disable null skip insertion before the first word
%	 after a coloring command.}
% \begin{macro}{\pcol@fcwhyphenate}
% \changes{v1.3-6}{2013/09/17}
%	{Introduced to enable null skip insertion before the first word
%	 after a coloring command not always but conditionally.}
% 
% The API macro \!\coloredwordhyphenated! \!\def!ines the macro
% \!\pcol@fcwhyphenate! being \!\hskip!\!\z@! so that the null space is
% inserted after the coloring \!\special! and \!\insert! put by
% \!\pcol@set@color@push! if we are in horizontal mode, so that the word
% following them can be hyphenated.  The other API macro
% \!\nocoloredwordhyphenated! makes $\!\pcol@fcwhyphenate!=\!\relax!$ to
% inhibit the insertion.  Since the null skip is a line break candidate, the
% skip might cause an unexpected and undesirable line break.  However, this
% demerit is less important than the merit of making it possible to
% hyphenate the first word in multi-column documents with narrow lines, and
% thus we make \!\coloredwordhyphenated! effective in default while giving
% users a means to disable the insertion (occasionally) by
% \!\nocoloredwordhyphenated!.
% 
%    \begin{macrocode}
\def\coloredwordhyphenated{\def\pcol@fcwhyphenate{\hskip\z@}}
\def\nocoloredwordhyphenated{\let\pcol@fcwhyphenate\relax}
\coloredwordhyphenated

%    \end{macrocode}
% \end{macro}\end{macro}\end{macro}
% 
% 
% 
% \KeepSpace{8}
% \section{Commands for Column-Separating Rule Color and Background Painting}
% \label{sec:imp-commbg}
% \changes{v1.3-3}{2013/09/17}
%	{Add the subsection ``Commands for Column-Separating Rule Color and
%	 Background Painting'' to describe newly
%	 introduced API macros to specify colors of column-separating rules
%	 and background painting.}
% 
% \begin{macro}{\colseprulecolor}
% \changes{v1.3-3}{2013/09/17}
%	{Introduced to spcify the colors of column-separating rules.}
% \begin{macro}{\pcol@defcseprulecolor@x}
% \changes{v1.3-3}{2013/09/17}
%	{Introduced to implement \cs{colseprulecolor}.}
% \begin{macro}{\pcol@defcseprulecolor@y}
% \changes{v1.3-3}{2013/09/17}
%	{Introduced to implement \cs{colseprulecolor}.}
% \begin{macro}{\pcol@defcseprulecolor}
% \changes{v1.3-3}{2013/09/17}
%	{Introduced to implement \cs{colseprulecolor}.}
% \begin{macro}{\normalcolseprulecolor}
% \changes{v1.3-3}{2013/09/17}
%	{Introduced to spcify that color of column-separating rules is normal.}
% \begin{macro}{\pcol@defcseprulecolor@i}
% \changes{v1.3-3}{2013/09/17}
%	{Introduced to implement \cs{colseprulecolor} and
%	 \cs{normalcolseprulecolor}.}
% \begin{macro}{\pcol@colseprulecolor}
% \changes{v1.3-3}{2013/09/17}
%	{Introduced to keep the color for all column-separating rules.}
% 
% The macro \!\colseprulecolor!\oarg{mode}\marg{color}$|[|c|]|$ \!\def!ines
% \!\pcol@colseprulecolor! to have the $\arg{color}$ optionally with
% coloring $\arg{mode}$ of all \cseprule{}s if the optional argument $c$ is
% not provided, or $|\pcol@colseprulecolor|{\cdot}c$
% 
% \SpecialArrayMainIndex{c}{\pcol@colseprulecolor}
% 
% that of the rule drawn between a particular column pair $c$ and $c+1$.
% After \!\def!ining \Midx{\!\pcol@colorcommand!} to be \!\colseprulecolor!
% in case we have to give a warning, the macro invokes
% \!\pcol@defcseprulecolor@x!\oarg{mode}\marg{color} or
% \!\pcol@defcseprulecolor@y!\marg{color} according to the provision of the
% optional argument $\arg{mode}$ to invoke \!\pcol@defcseprulecolor!  with
% argument \marg{cmd}${}={}$\!\color!\oarg{mode}\marg{color} so that this
% macro invokes $\!\pcol@defcseprulecolor@i!\ARg{cmd}|[|c|]|$ where
% $c=\emptyset$ if the optional arguemnt $c$ is not provided.
% 
% The macro $\!\normalcolseprulecolor!|[|c|]|$, on the other hand, defines
% $|\pcol@colseprule|\~|color|[{\cdot}c]$
% 
% \SpecialArrayIndex{c}{\pcol@colseprulecolor}
% \SpecialIndex{\pcol@colseprulecolor}
% 
% with whatever \!\normalcolor! gives, and thus it invokes
% \!\pcol@defcseprulecolor@i! letting $\arg{cmd}=\!\normalcolor!$, after
% \!\def!ining \!\pcol@colorcommand! to be \!\normalcolsep~rule~color!.
% 
% The macro $\!\pcol@defcseprulecolor@i!\ARg{cmd}|[|c|]|$ examines if
% $\!\set@color!=\!\relax!$ to mean no coloring packages have been loaded
% and, if so, do nothing giving a warning the command in
% \!\pcol@colorcommand! is not effective.  Otherwise, we examine if
% $\arg{cmd}$ has proper arguments by invoking it but temporally nullifying
% \!\set@color! and then \!\def!ine $|\pcol@colseprulecolor|[{\cdot}c]$
% 
% \SpecialArrayIndex{c}{\pcol@colseprulecolor}
% \SpecialIndex{\pcol@colseprulecolor}
% 
% to be $\arg{cmd}$.
% 
% Note that at the top level we \!\def!ine \!\pcol@colseprulecolor! to be
% \!\normalcolor! to give the default for all \cseprule{}s.  Also note that
% macros $|\pcol@colseprulecolor|{\cdot}c$
% 
% \SpecialArrayIndex{c}{\pcol@colseprulecolor}
% 
% are referred to solely in \!\pcol@hfil!
% which also uses \!\pcol@colseprulecolor! for columns for which
% $|\pcol@colseprulecolor|{\cdot}c$
% 
% \SpecialArrayIndex{c}{\pcol@colseprulecolor}
% 
% is not defined.
% 
%    \begin{macrocode}
%% Commands for Column-Separating Rule Color and Background Painting

\def\colseprulecolor{\def\pcol@colorcommand{\string\colseprulecolor}%
  \@ifnextchar[%]
    \pcol@defcseprulecolor@x\pcol@defcseprulecolor@y}
\def\pcol@defcseprulecolor@x[#1]#2{\pcol@defcseprulecolor{\color[#1]{#2}}}
\def\pcol@defcseprulecolor@y#1{\pcol@defcseprulecolor{\color{#1}}}
\def\pcol@defcseprulecolor#1{\@ifnextchar[%]
  {\pcol@defcseprulecolor@i{#1}}{\pcol@defcseprulecolor@i{#1}[]}}
\def\normalcolseprulecolor{%
  \def\pcol@colorcommand{\string\normalcolseprulecolor}%
  \@ifnextchar[%]
    {\pcol@defcseprulecolor@i\normalcolor}%
    {\pcol@defcseprulecolor@i\normalcolor[]}}
\def\pcol@defcseprulecolor@i#1[#2]{%
  \ifx\set@color\relax
    \PackageWarning{paracol}{\pcol@colorcommand\space is not effective
      without some coloring package}%
  \else
    {\let\set@color\relax #1}%
    \global\@namedef{pcol@colseprulecolor#2}{#1}%
  \fi}
\gdef\pcol@colseprulecolor{\normalcolor}

%    \end{macrocode}
% \end{macro}\end{macro}\end{macro}\end{macro}\end{macro}\end{macro}\end{macro}
% 
% \KeepSpace{8}
% \begin{macro}{\backgroundcolor}
% \changes{v1.3-3}{2013/09/17}
%	{Introduced to define colors for background painting.}
% \begin{macro}{\nobackgroundcolor}
% \changes{v1.3-3}{2013/09/17}
%	{Introduced to undefine colors for background painting.}
% \begin{macro}{\pcol@backgroundcolor@e}
% \changes{v1.3-3}{2013/09/17}
%	{Introduced to implement \cs{backgroundcolor} and
%	 \cs{nobackgroundcolor}.}
% \begin{macro}{\pcol@backgroundcolor}
% \changes{v1.3-3}{2013/09/17}
%	{Introduced to implement \cs{backgroundcolor} and
%	 \cs{nobackgroundcolor}.}
% \changes{v1.31}{2013/10/10}
%	{Fix the mispell ``colorling'' in the error message.}
% \begin{macro}{\pcol@backgroundcolor@i}
% \changes{v1.3-3}{2013/09/17}
%	{Introduced to implement \cs{backgroundcolor} and
%	 \cs{nobackgroundcolor}.}
% \begin{macro}{\pcol@bg@region}
% \changes{v1.3-3}{2013/09/17}
%	{Introduced to implement \cs{backgroundcolor} and
%	 \cs{nobackgroundcolor}.}
% \begin{macro}{\pcol@backgroundcolor@ii}
% \changes{v1.3-3}{2013/09/17}
%	{Introduced to implement \cs{backgroundcolor} and
%	 \cs{nobackgroundcolor}.}
% \begin{macro}{\pcol@backgroundcolor@iii}
% \changes{v1.3-3}{2013/09/17}
%	{Introduced to implement \cs{backgroundcolor}.}
% \begin{macro}{\pcol@backgroundcolor@iv}
% \changes{v1.3-3}{2013/09/17}
%	{Introduced to implement \cs{backgroundcolor}.}
% \begin{macro}{\pcol@backgroundcolor@v}
% \changes{v1.3-3}{2013/09/17}
%	{Introduced to implement \cs{backgroundcolor}.}
% 
% {\nosv\gdef\|{\;|\;}}
% \def\fdef{f_{\mathit{def}}}
% 
% The macro
% \!\backgroundcolor!\marg{region}\oarg{mode}\marg{color}
% defines the $\arg{color}$ optionally with $\arg{mode}$ of the
% $\arg{region}$ whose syntax is specified as follows.
% 
% % \begin{eqnarray*}
% \arg{region}&{:}{:}{=}&\arg{regionid}\arg{extension}\\
% \arg{regionid}&{:}{:}{=}&\<a\>\|\arg{corg}|[|c|]|\\
% \arg{a}&{:}{:}{=}&\arg{corg}\||s|\||S|\||t|\||T|\||l|\||L|\||r|\||R|\|
%                               |f|\||F|\||n|\||N|\||p|\||P|\\
% \arg{corg}&{:}{:}{=}&|c|\||C|\||g|\||G|\\
% \arg{extension}&{:}{:}{=}&\emptyset\||(|x_0|,|y_0|)|\|
%                                      |(|x_0|,|y_0|)(|x_1|,|y_1|)|
% \end{eqnarray*}
% % 
% On the other hand, the counteraprt macro
% \!\nobackgroundcolor!\marg{region} undefines the color of $\arg{region}$.
% Both macros invoke \!\pcol@backgroundcolor! giving all arguments to it to
% parse the argument $\arg{region}$, after letting
% $\fdef=\CSIndex{if@tempswa}=\true$ and
% $\!\pcol@backgroundcolor@e!=\!\pcol@backgroundcolor@w!$ in
% \!\backgroundcolor!, or $\fdef=\CSIndex{if@tempswa}=\true$ and
% $\!\pcol@backgroundcolor@e!=\!\pcol@backgroundcolor@z!$ in
% \!\nobackgroundcolor!.  Note that the \!\let!-assignment to
% \!\pcol@backgroundcolor@e! is effective when we found an error in the
% parse of the argument $\arg{region}$ to throw away what remains in the
% argument unprocessed and, for \!\backgroundcolor!, the other arguments
% \oarg{mode}\marg{color}.
% 
% Then the macro \!\pcol@backgroundcolor! examines if $|\pcol@bg@@|{\cdot}a$
% 
% \SpecialArrayIndex{a}{\pcol@bg@@}
% 
% is defined, and if not, raises an error that $a$ is invalid and then, in case
% the user dare to continue the execution, invokes
% \!\pcol@backgroundcolor@e! to throw all arguments away after letting
% $a'=\!\pcol@bg@region!=|xx|$ so that the undefining
% $|\pcol@bg@color|{\cdot}a'$ does not cause any troubles.
% 
% \SpecialArrayIndex{a}{\pcol@bg@color@}
% \SpecialArrayIndex{a{\cdot}\string\texttt{@}{\cdot}c}{\pcol@bg@color@}
% 
% Otherwise, i.e., if $|\pcol@bg@@|{\cdot}a$ is defined, it 
% invokes $\!\pcol@backgroundcolor@i!|[|c|]|$ or \!\pcol@backgroundcolor@ii!
% according to the provision of the optional argument $|[|c|]|$, after
% \!\def!ining $a'=\Midx{\!\pcol@bg@region!}$ to be $a$.  Then
% \!\pcol@backgroundcolor@i! examines if $|\pcol@bg@|\~|mayhavecol@|{\cdot}a$
% 
% \SpecialIndex{\pcol@bg@mayhavecol@c}\SpecialIndex{\pcol@bg@mayhavecol@C}
% \SpecialIndex{\pcol@bg@mayhavecol@g}\SpecialIndex{\pcol@bg@mayhavecol@G}
% 
% is defined, and if not, raises an error again in a similar way, or
% otherwise invokes \!\pcol@backgroundcolor@ii! after re\!\def!ining
% $a'=a{\cdot}|@|{\cdot}c$.
% 
% Then if $\fdef=\true$, the macro \!\pcol@backgroundcolor@ii! examines if
% $\!\set@color!=\!\relax!$, and if so it complains that any coloring
% packages have not been loaded and invokes \!\pcol@backgroundcolor@w! just
% for throwing away optional arguments for \bgext{} and
% \oarg{mode}\marg{color}.  Otherwise, i.e., if $\!\set@color!\neq\!\relax!$,
% it invokes \!\pcol@backgroundcolor@iii!  after adding $a'$ to the tail of
% \!\pcol@bg@defined!.  On the other hand if $\fdef=\false$, it invokes
% \!\pcol@backgroundcolor@z! without checking the availability of coloring
% macros.
% 
% The macro \!\pcol@backgroundcolor@iii! at first invokes
% $\!\pcol@bg@defext!\Arg{d}\Arg{e}$ with $e=0$ for all
% $d\in\{|l|,|r|,|t|,|b|\}$ to have default (no) \bgext{}s.  Then if
% $|(|x_0|,|y_0|)|$ is provided, \!\pcol@bg@defext! is invoked again by
% \!\pcol@backgroundcolor@iv! for all
% $d\in\{|l|,|r|,|t|,|b|\}$ but with $e=x_0$ for
% $d\in\{|l|,|r|\}$ and with $e=y_0$ for $d\in\{|t|,|b|\}$.
% Further, if % $|(|x_1|,|y_1|)|$ is also provided, \!\pcol@bg@defext! is
% invoked once again by \!\pcol@backgroundcolor@v! for all
% $(d,e)\in\{(|r|,x_1),(|b|,y_1)\}$.  Finally \!\pcol@backgroundcolor@v!
% invokes \!\pcol@backgroundcolor@x!, which is also invoked from
% \!\pcol@backgroundcolor@iii! and \!\pcol@backgroundcolor@iv! if they find no
% (further) \bgext{}s.
% 
%    \begin{macrocode}
\def\backgroundcolor#1{\@tempswatrue
  \let\pcol@backgroundcolor@e\pcol@backgroundcolor@w
  \pcol@backgroundcolor#1\@nil}
\def\nobackgroundcolor#1{\@tempswafalse
  \let\pcol@backgroundcolor@e\pcol@backgroundcolor@z
  \pcol@backgroundcolor#1\@nil}
\def\pcol@backgroundcolor#1{%
  \@ifundefined{pcol@bg@@#1}%
    {\PackageError{paracol}%
       {Invalid background coloring region identifier #1}%
     \def\pcol@bg@region{xx}\pcol@backgroundcolor@e}%
    {\def\pcol@bg@region{#1}%
     \@ifnextchar[%]
       \pcol@backgroundcolor@i \pcol@backgroundcolor@ii}}
\def\pcol@backgroundcolor@i[#1]{%
  \@ifundefined{pcol@bg@mayhavecol@\pcol@bg@region}%
    {\PackageError{paracol}%
       {Column number \number#1 is not effective for background coloring region
        \pcol@bg@region}%
     \def\pcol@bg@region{xx}\pcol@backgroundcolor@e}%
    {\edef\pcol@bg@region{\pcol@bg@region @#1}%
     \pcol@backgroundcolor@ii}}
\def\pcol@backgroundcolor@ii{%
  \if@tempswa
    \ifx\set@color\relax
      \PackageWarning{paracol}{\string\backgroundcolor\space is not effective
        without some coloring package}%
      \let\reserved@b\pcol@backgroundcolor@w
    \else
      \let\reserved@b\pcol@backgroundcolor@iii
      \@cons\pcol@bg@defined{{\pcol@bg@region}}%
    \fi
  \else
    \let\reserved@b\pcol@backgroundcolor@z
  \fi
  \reserved@b}
\def\pcol@backgroundcolor@iii{%
  \pcol@bg@defext{l}\z@ \pcol@bg@defext{r}\z@
  \pcol@bg@defext{t}\z@ \pcol@bg@defext{b}\z@
  \@ifnextchar(%)
    \pcol@backgroundcolor@iv \pcol@backgroundcolor@x}
\def\pcol@backgroundcolor@iv(#1,#2){%
  \pcol@bg@defext{l}{#1}\pcol@bg@defext{r}{#1}%
  \pcol@bg@defext{t}{#2}\pcol@bg@defext{b}{#2}%
  \@ifnextchar(%)
    \pcol@backgroundcolor@v \pcol@backgroundcolor@x}
\def\pcol@backgroundcolor@v(#1,#2){%
  \pcol@bg@defext{r}{#1}\pcol@bg@defext{b}{#2}%
  \pcol@backgroundcolor@x}
%    \end{macrocode}
% \end{macro}\end{macro}\end{macro}\end{macro}\end{macro}
% \end{macro}\end{macro}\end{macro}\end{macro}\end{macro}
% 
% \begin{macro}{\pcol@backgroundcolor@x}
% \changes{v1.3-3}{2013/09/17}
%	{Introduced to implement \cs{backgroundcolor}.}
% \begin{macro}{\pcol@backgroundcolor@y}
% \changes{v1.3-3}{2013/09/17}
%	{Introduced to implement \cs{backgroundcolor}.}
% 
% The macro \!\pcol@backgroundcolor@x! is used in
% \!\pcol@backgroundcolor@iii!, \!\pcol@backgroundcolor@iv! and
% \!\pcol@backgroundcolor@v! to define the color for \bgpaint{} of the
% region $a'=\!\pcol@bg@region!$.  Since the macro is followed by the
% arguments \oarg{mode}\marg{color} of \!\backgroundcolor!, the macro
% invokes \!\color! to let it \!\def!ine \!\current@color! but without real
% coloring operations by letting
% \!\set@color!${}={}$\!\pcol@backgroundcolor@y!.  Therefore
% \!\pcol@backgroundcolor@y! is invoked in \!\color! and it \!\xdef!ines
% $|\pcol@bg@colpr|{\cdot}a'$ to let it have whatever \!\current@color! has.
% 
% \SpecialArrayIndex{a}{\pcol@bg@color@}
% \SpecialArrayIndex{a{\cdot}\string\texttt{@}{\cdot}c}{\pcol@bg@color@}
% 
%    \begin{macrocode}
\def\pcol@backgroundcolor@x#1\@nil{\begingroup
  \let\set@color\pcol@backgroundcolor@y \color}
\def\pcol@backgroundcolor@y{%
  \expandafter\xdef\csname pcol@bg@color@\pcol@bg@region\endcsname
   {\current@color}%
  \endgroup}
%    \end{macrocode}
% \end{macro}\end{macro}
% 
% \KeepSpace{2}
% \begin{macro}{\pcol@backgroundcolor@z}
% \changes{v1.3-3}{2013/09/17}
%	{Introduced to implement \cs{nobackgroundcolor}.}
% \begin{macro}{\pcol@backgroundcolor@w}
% \changes{v1.3-3}{2013/09/17}
%	{Introduced to implement \cs{backgroundcolor}.}
% \begin{macro}{\pcol@backgroundcolor@wi}
% \changes{v1.3-3}{2013/09/17}
%	{Introduced to implement \cs{backgroundcolor} and
%	 \cs{nobackgroundcolor}.}
% \begin{macro}{\pcol@bg@color@xx}
% \changes{v1.3-3}{2013/09/17}
%	{Introduced to implement \cs{backgroundcolor} and
%	 \cs{nobackgroundcolor}.}
% 
% The macro \!\pcol@backgroundcolor@z! is invoked from
% \!\pcol@backgroundcolor@ii! directly to work for \!\nobackgroundcolor! to
% disable the \bgpaint{} for a region $a'=\!\pcol@bg@region!$, or from
% \!\pcol@backgroundcolor! and \!\pcol@backgroundcolor@i! through
% \!\pcol@backgroundcolor@e! when they find an error in the argument
% $\arg{region}$ of \!\nobackground~color!.  Similarly the macro
% \!\pcol@backgroundcolor@w! is invoked from \!\pcol@background~color@ii!
% directly when it finds no coloring packages have not been loaded, or
% \!\pcol@backgroundcolor! and \!\pcol@backgroundcolor@i! through
% \!\pcol@backgroundcolor@e! on error too but in the argument of
% \!\backgroundcolor!.  Both macros throw away whatever remains in
% $\arg{region}$ unprocessed and then invoke \!\pcol@backgroundcolor@wi!,
% but \!\pcol@backgroundcolor@z! gives it a dummy argument pair, while
% \!\pcol@backgroundcolor@w! passes \oarg{mode}\marg{color} to it.
% 
% Then \!\pcol@backgroundcolor@wi! throw all arguments away and lets
% $|\pcol@bg@color@|{\cdot}a'\~=\!\relax!$ so that the region $a'$ is
% untouched in \bgpaint{} macros.  Note that since
% $a'=|xx|$ being an absolutely non-exsistent region when this macro is used
% for error recovery, undefining \pcol@bg@color@xx is not harmful.
% 
%    \begin{macrocode}
\def\pcol@backgroundcolor@z#1\@nil{\pcol@backgroundcolor@wi[]{}}
\def\pcol@backgroundcolor@w#1\@nil{\@ifnextchar[%]
  \pcol@backgroundcolor@wi{\pcol@backgroundcolor@wi[]}}
\def\pcol@backgroundcolor@wi[#1]#2{%
  \expandafter\global\expandafter\let
    \csname pcol@bg@color@\pcol@bg@region\endcsname\relax}

%    \end{macrocode}
% \end{macro}\end{macro}\end{macro}\end{macro}
% 
% \KeepSpace{2}
% \begin{macro}{\pcol@bg@mayhavecol@c}
% \changes{v1.3-3}{2013/09/17}
%	{Introduced to implement \cs{backgroundcolor} and
%	 \cs{nobackgroundcolor}.}
% \begin{macro}{\pcol@bg@mayhavecol@C}
% \changes{v1.3-3}{2013/09/17}
%	{Introduced to implement \cs{backgroundcolor} and
%	 \cs{nobackgroundcolor}.}
% \begin{macro}{\pcol@bg@mayhavecol@g}
% \changes{v1.3-3}{2013/09/17}
%	{Introduced to implement \cs{backgroundcolor} and
%	 \cs{nobackgroundcolor}.}
% \begin{macro}{\pcol@bg@mayhavecol@G}
% \changes{v1.3-3}{2013/09/17}
%	{Introduced to implement \cs{backgroundcolor} and
%	 \cs{nobackgroundcolor}.}
% 
% The macros $|\pcol@bg@mayhavecol@|{\cdot}a$ where $a\in\{|c|,|C|,|g|,|G|\}$
% are used in \!\pcol@background~color@i! when the region specifier $a$ in
% $\arg{region}$ argument of \!\backgroundcolor! or \!\noback~groundcolor! is
% followed by optional $|[|c|]|$, so that the invoker macro examines if $a$
% can have the optional column oridinal.
% 
% \SpecialArrayMainIndex{a}{\pcol@bg@mayhavecol@}
% 
% Therefore, the macros just need not to be \!\relax! and thus commonly have
% empty bodies.
% 
%    \begin{macrocode}
\def\pcol@bg@mayhavecol@c{}
\def\pcol@bg@mayhavecol@C{}
\def\pcol@bg@mayhavecol@g{}
\def\pcol@bg@mayhavecol@G{}

%    \end{macrocode}
% \end{macro}\end{macro}\end{macro}\end{macro}
% 
% \begin{macro}{\pcol@bg@defext}
% \changes{v1.3-3}{2013/09/17}
%	{Introduced to implement \cs{backgroundcolor}.}
% The macro $\!\pcol@bg@defext!\Arg{d}\Arg{e}$ is used by
% \!\pcol@backgroundcolor@iii!,
% \!\pcol@back~groundcolor@iv! and \!\pcol@backgroundcolor@v! to \!\def!ine 
% $|\pcol@bg@ext@|{\cdot}d{\cdot}|@|{\cdot}a'$ be $e=0$ for all
% $d\in\{|l|,|r|,|t|,|b|\}$ in the first, be $e=x_0$ for $d\in\{|l|,|r|\}$
% and $e=y_0$ for $d\in\{|t|,|b|\}$ in the second, and be $e=x_0$, $x_1$,
% $y_0$ and $y_1$ for $d$ being |l|, |r|, |t|, |b| respectively in the last.
% 
% \SpecialArrayMainIndex{d{\cdot}\string\texttt{@}{\cdot}a}{\pcol@bg@ext@}
% \SpecialArrayMainIndex
%   {d{\cdot}\string\texttt{@}{\cdot}a{\cdot}\string\texttt{@}{\cdot}c}
%   {\pcol@bg@ext@}
% 
% The macro at first lets \!\@tempdima! have $e$ to confirm $e$ is a proper
% dimension and then \!\xdef!ines
% $|\pcol@bg@ext@|{\cdot}d{\cdot}|@|{\cdot}a'$ to let it have the integer
% representaion of $e$ followed by |sp|.
% 
% \SpecialArrayMainIndex{d{\cdot}\string\texttt{@}{\cdot}a}{\pcol@bg@ext@}
% \SpecialArrayMainIndex
%   {d{\cdot}\string\texttt{@}{\cdot}a{\cdot}\string\texttt{@}{\cdot}c}
%   {\pcol@bg@ext@}
% 
% 
%    \begin{macrocode}
\def\pcol@bg@defext#1#2{%
  \@tempdima#2\relax
  \expandafter\xdef\csname pcol@bg@ext@#1@\pcol@bg@region\endcsname{%
    \number\@tempdima sp}}

%    \end{macrocode}
% \end{macro}
% 
% \begin{macro}{\resetbackgroundcolor}
% \changes{v1.3-3}{2013/09/17}
%	{Introduced to disable background-paiting for all regions.}
% \begingroup \let\SMALL\small \let\small\footnotesize
% \begin{macro}{\pcol@resetbackgroundcolor}
% \changes{v1.3-3}{2013/09/17}
%	{Introduced to implement \cs{resetbackgroundcolor}.}
% \let\small\SMALL
% \begin{macro}{\pcol@bg@defined}
% \changes{v1.3-3}{2013/09/17}
%	{Introduced to implement \cs{resetbackgroundcolor}.}
% 
% The API macro \!\resetbackgroundcolor! disables \bgpaint{} of all regions
% whose colors have been specified by \!\backgroundcolor!.  Since the region
% specifiers $a'_1$, $a'_2$, \ldots, $a'_n$ for which \bgpaint{} is specified
% are recorded in $\!\pcol@bg@defined!=
% \!\@elt!\Arg{a'_1}\!\@elt!\Arg{a'_2}\cdots\!\@elt!\Arg{a'_n}$ by
% \!\pcol@backgroundcolor!, the macro invokes \!\pcol@bg@defined!
% temporarily letting $\!\@elt!=\!\pcol@resetbackgroundcolor!$ to let
% $|\pcol@bg@|\~|color@|{\cdot}a'_i$ be \!\relax! for all $i\in[1,n]$, and
% clears \!\pcol@bg@defined!, whose initial state is also empty.
% 
% \SpecialArrayIndex{a}{\pcol@bg@color@}
% \SpecialArrayIndex{a{\cdot}\string\texttt{@}{\cdot}c}{\pcol@bg@color@}
% 
%    \begin{macrocode}
\def\resetbackgroundcolor{{%
  \let\@elt\pcol@resetbackgroundcolor \pcol@bg@defined
  \gdef\pcol@bgdefined{}}}
\def\pcol@resetbackgroundcolor#1{%
  \expandafter\global\expandafter\let\csname pcol@bg@color@#1\endcsname\relax}
\gdef\pcol@bg@defined{}

%    \end{macrocode}
% \end{macro}\end{macro}\endgroup\end{macro}
% 
% 
% 
% \section{Closing Environment}
% \label{sec:imp-end}
% 
% \begin{macro}{\endparacol}
% \changes{v1.0}{2011/10/10}
%	{Replace \cs{par} with \cs{pcol@par}.}
% \changes{v1.2-2}{2013/05/11}
%	{Add pre-flushing column height check and footnote counter
%	 adjustment.}
% \changes{v1.2-5}{2013/05/11}
%	{Remove \cs{global} assignment of \cs{hsize} and \cs{linewidth}
%	 because assigments of them in \string\texttt{paracol} are now
%	 perfectly local.}
% \changes{v1.31}{2013/10/10}
%	{Add saving $c$ into \cs{pcol@lastcol} to let \cs{pcol@output@end}
%	 know the column visited last.}
% \begin{macro}{\pcol@lastcol}
% \changes{v1.31}{2013/10/10}
%	{Introduced to keep the column visited last to pass its typesetting
%	 parameters to post-environment.}
% 
% The macro \!\endparacol! is invoked from \Endparacol{} to close
% \env{paracol} environment.  After making it sure to be in vertical mode by
% \!\pcol@par!, we switch to the column 0 by \!\pcol@switchcol! to let
% \lcounter{} have the values for the column 0 so that they are referred to
% outside the environment, after saving the current column $c$ in
% \!\pcol@lastcol! to be referred to in \!\pcol@output@end! so that
% $\cc_c(\sw)$ and $\cc_c(\ep)$ are passed to \postenv.
% 
% \begin{Sloppy}{1800}
% Then we invoke \!\pcol@flushclear! for \pfcheck{}, turning
% $\CSIndex{ifpcol@lastpage}=\true$ to tell \!\pcol@output@switch! for the
% check that it works on the \lpage, and giving $\bot$ to \!\pcol@flushclear! 
% as its argument, unless footnote typesetting is \scfnote{} but not
% \mgfnote{} for which we give $\df$ to ensure all deferred footnotes are
% put in the checking process.  Note that the argument for \Mcfnote{}
% typesetting is $\df$ but it is definitely $\bot$ in this mode.  After that
% we make an \!\output! request by \!\pcol@invokeoutput! with
% $\!\penalty!=\!\pcol@op@end!$ and still with
% $\CSIndex{ifpcol@lastpage}=\true$ to build the \lpage.
% \end{Sloppy}
% 
% Next, we let $\!\columnwidth!=\!\textwidth!$ and
% $\CSIndex{if@twocolumn}=\false$ for single-column typesetting, and also
% let $\!\topskip!=\!\pcol@topskip!$ to make it sure that the parameter has
% the value used outside in \env{paracol} environment.  Finally, if the
% \counter{footnote} counter adjustment is required by
% $\CSIndex{ifpcol@fncounteradjustment}=\true$, we let
% $\!\c@footnote!=\bf+\nf$.
% 
%    \begin{macrocode}
%% Closing Environment

\def\endparacol{\pcol@par
  \edef\pcol@lastcol{\number\pcol@currcol}%
  \pcol@nextcol\z@ \pcol@switchcol
  \pcol@lastpagetrue
  \ifpcol@mgfnote \pcol@flushclear\voidb@x
  \else \pcol@flushclear\pcol@topfnotes
  \fi
  \pcol@invokeoutput\pcol@op@end
  \global\columnwidth\textwidth
  \global\@twocolumnfalse
  \global\topskip\pcol@topskip
  \ifpcol@fncounteradjustment
    \global\c@footnote\pcol@footnotebase
    \global\advance\c@footnote\pcol@nfootnotes
  \fi}

%    \end{macrocode}
% \end{macro}\end{macro}
% 
% \begin{macro}{\pcol@restoreeveryvbox}
% \changes{v1.22}{2013/06/30}
% 	{Introduced to reflect \cs{global} updates on \cs{everyvbox} in
%	 \string\texttt{paracol} environments.}
% 
% The macro \!\pcol@restoreeveryvbox! is invoked just after \Endparacol{} by
% \!\aftergroup! mechanism activated by \!\pcol@zparacol!.  It examines if
% \!\pcol@everyvbox! has tokens different from \!\pcol@dummytoken! which
% \!\pcol@zparacol! \!\global!ly assigned to the register.  Since the dummy
% token cannot be assigned to \!\everyvbox!\footnote{
% 
% Unless a very surprising coincidence happens or a user intentionally
% violates the coherence of the implementation.},
% 
% the difference means the \!\everyvbox! has been \!\global!ly updated with
% the value that \!\pcol@everyvbox! has now.  Therefore if so, we
% globally update \!\everyvbox! with \!\pcol@everyvbox! to refrect the
% global update in the environment.
% 
%    \begin{macrocode}
\def\pcol@restoreeveryvbox{%
  \expandafter\def\expandafter\reserved@a\expandafter{\the\pcol@everyvbox}%
  \def\reserved@b{\pcol@dummytoken}%
  \ifx\reserved@a\reserved@b\else \global\everyvbox\pcol@everyvbox \fi}
%    \end{macrocode}
% \end{macro}
%\iffalse
%</paracol>
%\fi
\endinput

% 
% \IndexPrologue{\newpage\section*{Index}
% \addcontentsline{toc}{part}{\protect\numberline{}{Index}}
% 
% Underlined number refers to the page where the implementation or the
% definition of the correspoinding entry is described, while italicized
% number is for the page in which the specificatoin or usage of the entry is
% explained.
% 
% To find a control sequence, remove prefixes \cs{@}, \cs{if@}, \cs{pcol@} and 
% \cs{ifpcol@} from its name if it has one of them.}
% \Finale
% \GlossaryPrologue{\newpage\section*{Revision History}
% \addcontentsline{toc}{part}{\protect\numberline{}{Revision History}}}
% \def\EQ{=} \def\GT{>} \def\BAR{|} \def\NEQ{\neq} \def\S{\Sec}
% \PrintChanges
\endinput
