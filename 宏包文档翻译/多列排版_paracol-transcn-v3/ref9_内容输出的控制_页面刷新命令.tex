 \subsection{Control of Contents Output\hfill 内容输出的控制}
 \label{sec:ref-contents}
 
 \begin{description}
 \item[\Midx{\!\addcontentsonly!}\marg{file}\marg{col}]\mbox{}\par
 The command inhibits the output of contents information to
 $\meta{file}\in\{|toc|,|lof|,|lot|\}$ from columns other than \meta{col}.

该命令禁止除 \meta{col} 外的列将内容信息输出到 $\meta{file}\in{|toc|,|lof|,|lot|}$。
 \begin{itemize}
 \item
 For example, this manual has \!\addcontentsonly!|{toc}{0}| to
 inhibit the contents information output from \!\subsection! commands
 in the right column in Section~\ref{sec:env} and~\ref{sec:float},
 or the table should have duplicated entries of sub-sections.

 例如,本手册使用 \!\addcontentsonly!|{toc}{0}| 来阻止在第~\ref{sec:env}节和~\ref{sec:float}节的右列中,由 \!\subsection! 命令输出的目录信息,否则表格将会有子节的重复条目。

 \item
 It must be $\meta{file}\in\{|toc|,|lof|,|lot|\}$, or you will have an
 error message of illegal type of contents file.

 它必须是$\meta{file}\in{|toc|,|lof|,|lot|}$,否则将会收到一个不合法的内容文件类型的错误消息。
 \end{itemize}
 \end{description}
 

 
 \subsection{Page Flushing Commands\hfill 页面刷新命令}
 \label{sec:ref-flush}
 
 \begin{description}
 \item[\Midx{\!\flushpage!}]\mbox{}\par
 The command flushes pages up to the {\em\Uidx\tpage} in which the \lcolumn{}
 resides.  Deferred floats which can be put in the pages up to the \tpage{}
 are also flushed.
 
该命令将页面刷新到包含\lcolumn{}的{\em\Uidx\tpage}。也会刷新可以放置在\tpage{}之前的页面上的延迟浮动体。
 \item[\Midx{\!\clearpage!}]\mbox{}\par
 The command does what \!\flushpage! does and then flushes all floats still
 deferred if any.  The deferred float flushing beyond the \tpage{} takes
 place at first for \cwise{} ones creating \fcolumn{}s for them, and
 then for \pwise{} ones creating {\em\Uidx\fpage{}s} only with
 \pwise{} floats, as \LaTeX's \!\clearpage! does outside \env{paracol}
 environment.

该命令执行 \!\flushpage! 的功能,然后刷新所有延迟的浮动对象(如果有的话)。在\tpage{}之后,延迟的浮动对象刷新首先针对\cwise{}的浮动对象,为它们创建\fcolumn{},然后针对\pwise{}的浮动对象,只创建包含\pwise{}浮动对象的{\em\Uidx\fpage{}},就像在\env{paracol}环境之外使用\LaTeX 的 \!\clearpage! 命令一样。
 \item[\Midx{\!\cleardoublepage!}]\mbox{}\par
 \changes{v1.3-5}{2013/09/17}
	{Add description of \cs{cleardoublepage}.}
 The command does what \LaTeX's \!\cleardoublepage! does outside
 \env{paracol}.  That is, it does \!\clearpage! always and then leaves a
 blank page if it is even-numbered and two-sided |p|(age) feature is
 enabled by |twoside| option of \!\documentclass! or \Paracol's own
 \!\twosided! command shown in Section~\ref{sec:ref-twoside}.

该命令做的是在\env{paracol}之外与\LaTeX 的 \!\cleardoublepage! 相同的操作。也就是说,它总是执行 \!\clearpage!,然后如果该页是偶数页,并且通过 \!\documentclass! 的|twoside|选项或\Paracol 的 \!\twosided! 命令(见第~\ref{sec:ref-twoside} 节)启用了双面特性,它会留下一个空白页。
 \begin{itemize}
 \item
 This command is equivalent to \!\clearpage! in \env{paracol} environments
 for \npaired{} \parapag{}ing because \!\clearpage! flushes \emph{both}
 left and right \parapag{}es.

在对于\npaired{} \parapag{}ing的\env{paracol}环境中,该命令等效于 \!\clearpage!,因为 \!\clearpage! 会刷新\emph{左侧和右侧}的\parapag{}es。
 \end{itemize}
 \end{description}