% \section{Closing \texttt{paracol} Environment and Page Flushing}
% \label{sec:man-close}

% The final example shown here is this single-column text which the author put
% after the \env{paracol} environment above is closed.  As you are seeing, a
% \env{paracol} environment can be finished at any vertical position in a
% page and can be followed by ordinary single column texts.
%

% 这里展示的最后一个例子是在上面关闭的\env{paracol}环境之后,作者放置的这个单栏文本。正如你所见,\env{paracol}环境可以在页面的任何垂直位置结束,并且可以跟随普通的单栏文本。

% \begin{paracol}{2}
% The environment may also be restarted anywhere you like as shown here.
\switchcolumn
% 此处展示了环境可以在任何位置重新开始。
% \switchcolumn*
% The last issue is to flush a page.  The ordinary \!\newpage! command works
% as you expect.  If you say \!\newpage! in the left column in a page, the
% contents following it will appear in the left column in the next page.  Note
% that this does not affect the layout of the right column.
\switchcolumn
% 最后一个问题是如何换页。普通的 \!\newpage! 命令按照你的期望工作。如果你在页面的左栏使用 \!\newpage! 命令,在它之后的内容将出现在下一页的左栏中。请注意,这不会影响右栏的布局。
\switchcolumn*% 
% To flush all columns in a page, a command \Uidx{\!\flushpage!} is
% available.  This command in $i$-th column is almost equivalent to;
\switchcolumn
% 要在页面中刷新所有栏目,可以使用命令 \Uidx{\!\flushpage!}。这个命令在第$i$栏中几乎等同于:
% \begin{itemize}\item[]
% \!\switchcolumn!|[|$i$|]*[|\!\newpage!|]|
% \end{itemize}
\switchcolumn*
% but more robust\footnotemark\label{fn:flush}.
% The ordinary page breaking command \Uidx{\!\clearpage!} may also be used
% to flush all columns and to start a fresh page, but it has a side effect
% to put all figures and tables which are not yet output.
\switchcolumn
% 但更加健壮 \footnotemark\label{fn:flush} 。普通的换页命令 \Uidx{\!\clearpage!} 也可以用于刷新所有栏目并开始新的一页,但它会导致尚未输出的所有图表被放置在同一页中。
%\end{paracol}

% Now the author will do |\flushpage| shortly to start a real binlingual
% example from the next page, after showing another example of closing 
% \env{paracol} environments in this sentence and of restarting in the next
% one, in which {\em unbalanced column width} is demonstrated using
% \Uidx{\!\columnratio!} command shown in Section~\ref{sec:ref-colwidth}.

% 现在作者将很快使用|\flushpage|命令,在下一页开始一个真正的双语示例,此前在本句中展示了另一个关闭\env{paracol}环境的例子,并在下一句中重新开始,在其中使用了在第~\ref{sec:ref-colwidth}节中展示的 \Uidx{\!\columnratio!} 命令演示了{\em 不平衡的列宽}。

% \columnratio{0.6}
\begin{Verbatim}
\columnratio{0.6}
\end{Verbatim}

% \begin{paracol}{2}
% \begin{leftcolumn}
% O.K., we have restarted \env{paracol} environment and we will see the
% effect of \!\flushpage! now!!\footnotetext{
% 
% For example \texttt{\string\switchcolumn*} may flush a page for the
% \sync{}ation and thus \texttt{\string\newpage} may leave an empty page.}
% 

% 好的,我们已经重新开始了\env{paracol}环境,现在我们将看到 \!\flushpage! 命令的效果!!\footnotetext{例如,\texttt{\string\switchcolumn*}可能会为同步而刷新页面,因此\texttt{\string\newpage}可能会留下一个空白页。}
% \end{leftcolumn}
% \begin{rightcolumn}
%\begin{Verbatim}
%\columnratio{0.6}
%\begin{paracol}{2}
%\begin{leftcolumn}
%O.K., ...
%\end{leftcolumn}
%\end{Verbatim}
% |\begin{rightcolumn}| \textit{source}\\
% |\end{rightcolumn}|
% \flushpage
% \end{rightcolumn}
% \end{paracol}