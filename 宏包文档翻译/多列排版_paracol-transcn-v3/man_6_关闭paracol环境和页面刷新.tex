\section{Closing \texttt{paracol} Environment and Page Flushing\hfill 关闭 \texttt{paracol} 环境和页面刷新}
\label{sec:man-close}

The final example shown here is this single-column text which the author put
after the \env{paracol} environment above is closed.  As you are seeing, a
\env{paracol} environment can be finished at any vertical position in a
page and can be followed by ordinary single column texts.


这里展示的最后一个例子是在上面关闭的\env{paracol} 环境之后,作者放置的这个单栏文本。正如你所见,\env{paracol} 环境可以在页面的任何垂直位置结束,并且可以跟随普通的单栏文本。

\columnratio{0.3,0.42,0.28}
\begin{paracol}{3}
\begin{VerbatimII}
\begin{paracol}{2}
\begin{leftcolumn}
The enviro ... 
\end{leftcolumn}
\begin{rightcolumn}
source
\end{rightcolumn}
\end{paracol}
Now the aurthor will do ...
\end{VerbatimII}
%%%%%%%%%%%%%%%%%%%%%%%%%%%%%%%%%%%%%%%%%%%%%%%%%%%%%%%
\switchcolumn

\switchcolumn[1]
The environment may also be restarted anywhere you like as shown here.
\switchcolumn
此处展示了环境可以在任何位置重新开始。

\switchcolumn[1]
The last issue is to flush a page.  The ordinary \!\newpage! command works
as you expect.  If you say \!\newpage! in the left column in a page, the
contents following it will appear in the left column in the next page.  Note
that this does not affect the layout of the right column.
\switchcolumn
最后一个问题是如何换页。\!\newpage! 命令按照你的期望工作。如果你在页面的左栏使用 \!\newpage! 命令,在它之后的内容将出现在下一页的左栏中。请注意,这不会影响右栏的布局。

\switchcolumn[1]
To flush all columns in a page, a command \Uidx{\!\flushpage!} is
available.  This command in $i$-th column is almost equivalent to;
\begin{itemize}\item[]
\!\switchcolumn!|[|$i$|]*[|\!\newpage!|]|
\end{itemize}
\switchcolumn
要在页面中刷新所有栏目,可以使用命令 \Uidx{\!\flushpage!}。这个命令在第$i$栏中几乎等同于:
\begin{itemize}\item[]
\!\switchcolumn!|[|$i$|]*[|\!\newpage!|]|
\end{itemize}

\switchcolumn[1]
but more robust\footnotemark\label{fn:flush}.
The ordinary page breaking command \Uidx{\!\clearpage!} may also be used
to flush all columns and to start a fresh page, but it has a side effect
to put all figures and tables which are not yet output.
\switchcolumn
但更加健壮 \footnotemark\label{fn:flush} 。普通的换页命令 \Uidx{\!\clearpage!} 也可以用于刷新所有栏目并开始新的一页,但它会导致尚未输出的所有图表被放置在同一页中。
\end{paracol}

Now the author will do |\flushpage| shortly to start a real binlingual
example from the next page, after showing another example of closing 
\env{paracol} environments in this sentence and of restarting in the next
one, in which {\em unbalanced column width} is demonstrated using
\Uidx{\!\columnratio!} command shown in Section~\ref{sec:ref-colwidth}.

现在作者将很快使用|\flushpage|命令,在下一页开始一个真正的双语示例,此前在本句中展示了另一个关闭\env{paracol}环境的例子,并在下一句中重新开始,在其中使用了在第~\ref{sec:ref-colwidth}节中展示的 \Uidx{\!\columnratio!} 命令演示了{\em 不平衡的列宽}。

\columnratio{0.3,0.42,0.28}
\begin{paracol}{3}
\begin{VerbatimII}
\columnratio{0.6} 
\begin{paracol}{2} 
\begin{leftcolumn}
O.K., ... 
\end{leftcolumn} 
\begin{rightcolumn}
source 
\end{rightcolumn}
\end{VerbatimII}
%%%%%%%%%%%%%%%%%%%%%%%%%%%%%%%%%%%%%%%%%%%%%%%%%%%%%%%
\switchcolumn

O.K., we have restarted \env{paracol} environment and we will see the
effect of \!\flushpage! now!!\footnotetext{
For example \texttt{\string\switchcolumn*} may flush a page for the
\sync{}ation and thus \texttt{\string\newpage} may leave an empty page.}
\switchcolumn
好的,我们已经重新开始了\env{paracol}环境,现在我们将看到 \!\flushpage! 命令的效果!!\footnotetext{例如,\texttt{\string\switchcolumn*}可能会为同步而刷新页面,因此\texttt{\string\newpage}可能会留下一个空白页。}
\end{paracol}

\columnratio{0.25,0.25,0.25,0.25}
\begin{paracol}{4}
\newenvironment{Gverse}{\ensurevspace{2\baselineskip}
\begin{leftcolumn*}
\begin{myverse}}
{\end{myverse}\end{leftcolumn*}}
\newenvironment{CGverse}{%
\begin{rightcolumn}
\begin{myverse}}
{\end{myverse}\end{rightcolumn}}
\newenvironment{Everse}{%
\begin{nthcolumn}{2}
\begin{myverse}}
{\end{myverse}\end{nthcolumn}}
\newenvironment{CEverse}{%
\begin{nthcolumn}{3}
\begin{myverse}}
{\end{myverse}\end{nthcolumn}}
\makeatletter
\newenvironment{myverse}{\leftmargini0pt\partopsep0pt\verse}{\endverse}


\begin{leftcolumn*}[
\centerline{\Large An Die Freude/To Joy}\label{page:bfreude}\smallskip
\centerline{\large Friedrich Schiller 弗里德里希·席勒}\smallskip
The following is the libretto of the fourth movement of Beethoven's Ninth
Symphony, his adaptation of Schiller's ode ``An Die Freude'' (or ``To Joy'' in
English). Beethoven's additions and revisions are indicated in italics.

以下是贝多芬第九交响曲第四乐章的歌剧剧本,他改编自席勒的颂歌《致欢乐》(或英文版的《To Joy》)。贝多芬的添加和修订以斜体显示。]
\end{leftcolumn*}

\begin{Gverse}
\itshape O Freunde, nicht diese T\"one! \\
Sondern la{\ss}t uns angenehmere anstimmen und freu\-denvollere
\footnote{If I had been a good student in my German class, I could find
the German translation of the right column footnote \ref{fn:right4} is
``Dieser Teil wurde van Beethoven hinzugef\"ugt'' by myself without
the kind help from a user.}.
\end{Gverse}
\begin{CGverse}
\itshape 哦,朋友们,不要这样的音调! \\
让我们唱出更美妙、更充满欢乐的旋律%
\footnote{如果我在德语课上是个好学生的话,我就可以自己找到右栏脚注 \ref{fn:right4} 的德语翻译:“Dieser Teil wurde von Beethoven hinzugefügt”,而不需要用户的友好帮助。}。
\end{CGverse}
\begin{Everse}
\itshape Oh friends, no more of these sad tones!\\
Let us rather raise our voices together\\
In more pleasant and joyful tones
\footnote{This part was added by Beethoven.\label{fn:right4}}.
\end{Everse}
\begin{CEverse}
\itshape 哦,朋友们,不要再唱这些悲伤的音调了!\\
让我们一起高声歌唱\\
用更愉悦和欢乐的音调
\footnote{这部分是贝多芬添加的。\label{fn:right4}}。
\end{CEverse}


\end{paracol}


\end{document}
%%%%%%%%%%%%%%%%%%%%%%%%%%%%%%%%%%%%%%%%%%%%%%%%%%%%%%%
%%%%%%%%%%%%%%%%%%%%%%%%%%%%%%%%%%%%%%%%%%%%%%%%%%%%%%%


\begin{leftcolumn}

