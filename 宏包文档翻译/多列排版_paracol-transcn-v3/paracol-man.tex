%% LaTeX file 'paracol-man'.
%% Copyright (C) 2005-2018
%%   Hiroshi Nakashima <h.nakashima@DOMAIN;  DOMAIN=media.kyoto-u.ac.jp>
%%   (Kyoto University)
%% This program can be redistributed and/or modified under the terms
%% of the LaTeX Project Public License distributed from CTAN
%% archives in directory macros/latex/base/lppl.txt; either
%% version 1 of the License, or any later version.

\ProvidesFile{paracol-man.tex}
[2018/12/31 v1.35 ]
\documentclass{ltxdoc}\normalmarginpar
\usepackage[heading=true
,scheme=chinese%中文方案
,fontset=none%不使用默认的字体设置
,space=auto%自动调整中英文间距
]{ctex}
\setCJKmainfont{FangZhengShuSong-GBK-1.ttf}[Path=/Users/virhuiai/hlProjects/Latex-Typesetting-Hub/font/方正/]%设置文本的中文有衬线字体
\setCJKsansfont{FangZhengHeiTi-GBK-1.ttf}[Path=/Users/virhuiai/hlProjects/Latex-Typesetting-Hub/font/方正/]%设置文本的中文无衬线字体为
\setCJKmonofont{FangZhengFangSong-GBK-1.ttf}[Path=/Users/virhuiai/hlProjects/Latex-Typesetting-Hub/font/方正/] %设置文本的中文等宽字体 
% \setCJKfamilyfont{fontKai}{LXGWWenKai-Regular.ttf}[Path=/Users/virhuiai/hlProjects/Latex-Typesetting-Hub/font/霞鹜文楷/]
\setCJKfamilyfont{fontKai}{FangZhengKaiTi-GBK-1.ttf}[Path=/Users/virhuiai/hlProjects/Latex-Typesetting-Hub/font/方正/]
\newcommand\fontKai{\CJKfamily{fontKai}}

\usepackage{color}
\usepackage{paracol}
\usepackage{newvbtm}
\DisableCrossrefs
\PageIndex
\CodelineNumbered
\RecordChanges
\OnlyDescription
\def\ONLYDESCRIPTION{}
\textwidth210mm
\advance\textwidth-40mm \oddsidemargin20mm \advance\oddsidemargin-1in
\columnsep10mm
\marginparwidth20mm \advance\marginparwidth-\marginparsep
\marginparsep.5\marginparsep
\raggedbottom

\usepackage[a3paper, hmargin=2.5cm, vmargin=1cm, 
        includeheadfoot,landscape]{geometry}
\usepackage{parskip}
\OnlyDescription
\begin{document}
\parindent=0pt
\DocInput{paracol_1.dtx}

% \title{Package \textsf{paracol}:\\
	Yet Another Multi-Column Package to Typeset Columns in
	\textit{Parallel}}

\author{Hiroshi Nakashima\\(Kyoto University) \and 翻译:virhuiai\\(福建师范大学)}
\date{version \expandafter\@gobble\fileversion: \filedate}
\setbox0\vbox{\vskip\topskip\maketitle\vskip0pt}\unitlength\ht0\copy0

\ifx\ONLYDESCRIPTION\undefined
\tableofcontents
\newpage \ifodd\c@page\else \hbox{}\newpage \fi
\vbox to\unitlength{\unvbox0\vfil}
\part{User's Manual}\label{part:man}
\let\MidxSave\Midx \let\Midx\Uidx
\fi

\begin{abstract}
\columnratio{0.55}
\begin{paracol}{2}
\noindent
This package provides a \LaTeX{} environment named |paracol| in which you
may {\em switch} and {\em synchronize} columns by a command
|\switchcolumn| and by internal environments |column|, |nthcolumn|,
|leftcolumn| and |rightcolumn|.
\ifx\ONLYDESCRIPTION\undefined\else
See p.\Tie\pageref{page:toc} for the table of contents of this manual.
\fi
\switchcolumn 
\noindent
本宏包提供了一个名为 |paracol| 的 \LaTeX{} 环境,在其中你可以通过命令 |\switchcolumn| 和内部环境 |column|、|nthcolumn|、|leftcolumn| 和 |rightcolumn| 来{\fontKai 切换}和{\fontKai 同步}列。
\ifx\ONLYDESCRIPTION\undefined\else
请参考第 \pageref{page:toc} 页的本手册目录。
\fi
\end{paracol}
\end{abstract} 
% \tracingpages1 \tracingoutput1 \showboxdepth10000 \showboxbreadth10000
% \begingroup
% \hbadness9000 \hfuzz6pt
% \columnratio{0.3,0.42,0.28}
\begin{paracol}{3}[\section{Introduction\hfill 介绍}]
\begin{VerbatimII}
\columnratio{0.3,0.42,0.28}
\begin{paracol}{3}[\section{Introduction\hfill 介绍}]
\begin{Verbatim}
左侧源码
\end{Verbatim}
\switchcolumn
This document..
\switchcolumn
本文档..
\switchcolumn[1]
Suppose ...
\switchcolumn
假设...
\end{paracol}    
\end{VerbatimII}
%%%%%%%%%%%%%%%%%%%%%%%%%%%%%%%%%%%%%%%%%%%%%%%%%%%%%%%
\switchcolumn
This document describes the usage of yet another multi-column package named
\textsf{paracol}.  The unique feature of the package is that columns are
typeset {\em in parallel.}
\switchcolumn
本文档介绍了另一个名为 \textsf{paracol} 的多栏排版宏包的使用方法。该宏包的独特特点是可以将栏以{\em 并行}的方式排版。 

\switchcolumn[1]
Suppose you are writing a bilingual document whose left column is written in
a language, say English, and right column has the translation of the left
column in another language, e.g., Japanese.  With the \textsf{paracol}
package you may write an English part of arbitrarily length and then {\em
switch} to its Japanese counterpart to place both parts side by side.  Of
course you may return to the English writing similarly.
\switchcolumn
假设你正在撰写一份双语文档,左栏使用一种语言(如英语),右栏则是左栏的另一种语言(如日语)的翻译。使用 \textsf{paracol} 宏包,你可以先写任意长度的英文部分,然后{\em 切换}到对应的日文部分,将两部分并排放置在一起。当然,你也可以类似地返回到英文撰写。

\switchcolumn[1]
The {\em\Uidx\cswitch} is always allowed when you complete an outermost
level paragraph.  You may be unaware whether a column is broken into
multiple pages before switching because the package automatically goes
back and forward to the correct page and vertical position when you switch
the column.  Moreover, you may {\em\Uidx\sync{}e} columns so that the tops
of the first paragraphs after switching in all columns are vertically
aligned.  At a \sync{}ation point, you may give a single-column text,
for example a common section header, optionally.  You may also switch
single-column and multi-column in a page arbitrary.
\switchcolumn
在外层段落完成后,总是允许使用 {\em\Uidx\cswitch} 命令。在切换之前,你可能不知道栏是否被分成多个页面,因为当你切换栏时,宏包会自动回到正确的页面和垂直位置。此外,你可以通过 {\em\Uidx\sync{}e} 命令来使列对齐,这样在切换后,所有列中第一个段落的顶部会垂直对齐。在 \sync{}ation 点,你可以选择给出单栏文本,例如一个公共的章节标题。你还可以随意在页面上切换单栏和多栏排版。

\switchcolumn[1]*
This manual itself is an example of two-column documents typeset by
\textsf{paracol}.  Since the author is not familiar with languages other
than English and Japanese and the latter should be hardly understood by
most of readers, the right column is the translation of the left English
column into a computational language.  That is, the right column is the
\LaTeX{} source code of the left column\footnote{%
Not really but its essence
is shown.\label{fn:first}}.
\switchcolumn
该手册本身是使用 \textsf{paracol} 排版的双栏文档的一个示例。由于作者对英语和日语以外的语言不熟悉,并且后者可能很难被大多数读者理解,所以右栏是左侧英文栏的计算语言翻译。也就是说,右栏是左栏的 \LaTeX{} 源代码\footnote{%
虽然不完全准确,但其本质得以展示。\label{fn:first}virhuiai在翻译时,做成了三栏,第一栏是源代码,第二栏是英文,第三栏是中文。}。
\end{paracol}


\end{document}

\begin{paracol}{2}

\switchcolumn



\switchcolumn*
This manual itself is an example of two-column documents typeset by
\textsf{paracol}.  
\switchcolumn
本手册本身就是使用 \textsf{paracol} 宏包排版的两栏文档的一个示例。

\end{paracol}
\begin{Verbatim}
\begin{paracol}{2}[\section{Introduction}]
\hbadness5000
en.....
\switchcolumn
中文....

\switchcolumn*
en.....
\switchcolumn
中文....
\switchcolumn*[\section{Basic Usage}]....
\end{paracol}
\end{Verbatim}
% 
\section{Basic Usage\hfill 基本用法}
\columnratio{0.2,0.48}
\begin{paracol}{3}
\begin{VerbatimII}
\section{Basic Usage\hfill 基本用法}
\columnratio{0.2,0.48}
\begin{paracol}{3}
\begin{Verbatim}
左侧源码
\end{Verbatim}
\switchcolumn
Loading..
\switchcolumn
加载...
\switchcolumn[1]*
The fundamental...
\switchcolumn
并列...
\end{paracol}
\end{VerbatimII}

\switchcolumn%%%%%%%%%%%%%%%%%%%%%%%%%%%%%%
Loading the package is very simple.  What you have to do is
\!\usepackage!|{|\Uidx{\env{paracol}}|}| in the preamble.  Note that
\textsf{paracol} can be used with \LaTeXe{} and does not work with
\LaTeX{} 2.09.
\switchcolumn
加载该宏包非常简单。在导言区使用 \!\usepackage!|{|\Uidx{\env{paracol}}|}| 命令即可。请注意,\textsf{paracol} 可以与 \LaTeXe{} 一起使用,不支持 \LaTeX{} 2.09。

\switchcolumn[1]
The fundamental means of parallel-column typesetting are the environment
\env{paracol} and the command \Uidx{\!\switchcolumn!}.  The \env{paracol}
environment needs an argument to specify the number of columns.  Thus the
following is the basic construct for two-parallel-column documents.
\switchcolumn
并列栏排版的基本手段是使用 \env{paracol} 环境和命令 \Uidx{\!\switchcolumn!}。\env{paracol} 环境需要一个参数来指定栏的数量。因此,以下是两栏并列文档的基本结构。

\switchcolumn[1]
\begin{quote}
\!\begin!|{|\env{paracol}|}{2}|\\
\textit{left column text}\\
\!\switchcolumn!\\
\textit{right column text}\\
\!\switchcolumn!\\
\textit{left column text}\\
\!\switchcolumn!\\
\textit{right column text}\\
\!\switchcolumn!\\
\mbox{\hspace{4em}}$\vdots$\\
\!\end!|{|\env{paracol}|}|
\end{quote}
\switchcolumn
\begin{quote}
\!\begin!|{|\env{paracol}|}{2}|\\
\textit{左栏文本}\\
\!\switchcolumn!\\
\textit{右栏文本}\\
\!\switchcolumn!\\
\textit{左栏文本}\\
\!\switchcolumn!\\
\textit{右栏文本}\\
\!\switchcolumn!\\
\mbox{\hspace{4em}}$\vdots$\\
\!\end!|{|\env{paracol}|}|
\end{quote}

\switchcolumn[1]*
The \!\switchcolumn! command may have an optional argument to specify the
column number (zero origin) to start.  That is, \!\switchcolumn!|[0]|
means to switch to the leftmost column, |\switchcolumn[1]| is to start the
second column and so on.  Thus the |\switchcolumn| without the optional
argument may be considered as \!\switchcolumn!|[|$i+1\bmod{n}$|]| where
$i$ is the ordinal of the column you are leaving from and $n$ is the
number of columns given to \env{paracol} environment.
\switchcolumn
\!\switchcolumn! 命令可以带有可选参数来指定从第几栏(从零开始计数)开始切换。也就是说,\!\switchcolumn!|[0]| 表示切换到最左边的栏,|\switchcolumn[1]| 表示从第二栏开始,依此类推。因此,不带可选参数的 |\switchcolumn| 可以视为 \!\switchcolumn!|[|$i+1\bmod{n}$|]|,其中 $i$ 是你离开的栏的序号,$n$ 是给定给 \env{paracol} 环境的栏数。

\end{paracol} 
% \section{Column Synchronization\\栏同步}\label{sec:sync}

\columnratio{0.3,0.42,0.28}
\begin{paracol}{3}
%%%%%%%%%%%%%%%%%%%%%%%%%%%%%%%%%%%%%%%%%%%%%%%%%%%%%%%
\begin{VerbatimII}
\columnratio{0.3,0.42,0.28}
\begin{paracol}{3}
第1栏
\switchcolumn
第2栏
\switchcolumn
第3栏

\switchcolumn[0]*
同步 ...
\switchcolumn
...
\end{paracol}  
\end{VerbatimII}
%%%%%%%%%%%%%%%%%%%%%%%%%%%%%%%%%%%%%%%%%%%%%%%%%%%%%%%

\switchcolumn
The \!\switchcolumn! command may also be followed by a `|*|' to
{\em\Uidx\sync{}e} columns.  After you switch from a column to another by
\!\switchcolumn!|*| (or \!\switchcolumn!|[|$i$|]*|), all the columns are
vertically aligned at the bottom of the {\em deepest} one preceding the
command.  For example, the previous section has three \!\switchcolumn!|*|
commands at which left and right columns are vertically aligned.
\switchcolumn
\!\switchcolumn! 命令后面可以加上 `|*|',用来{\em 同步}栏。当你使用 \!\switchcolumn!|*|(或 \!\switchcolumn!|[|$i$|]*|)从一栏切换到另一栏时,所有栏都会垂直对齐在该命令之前最{\em 深}的栏的底部。例如,前一节使用了三个 \!\switchcolumn!|*| 命令,使左右两栏垂直对齐。

\switchcolumn[1]
The {\em starred} version of \!\switchcolumn! may have an optional
argument to specify a single-column {\em\Uidx\mctext} whose bottom is the
vertical alignment point of columns.  For example, \!\section!
commands in this manual are given as optional arguments
of \!\switchcolumn!|*| like;
\switchcolumn
{\em 带星号}版本的 \!\switchcolumn! 命令可以带有可选参数,用来指定一个单栏的{\em 同步文本},其底部作为栏的垂直对齐点。例如,本手册中的 \!\section! 命令作为 \!\switchcolumn!|*| 的可选参数给出,如下所示:

\switchcolumn[1]
\begin{Verbatim}
\switchcolumn*[\section{Basic Usage}]
\end{Verbatim}
\switchcolumn
\begin{Verbatim}
\switchcolumn*[\section{基础用法}]
\end{Verbatim}
\switchcolumn[1]
The \env{paracol} environment may also start with a \mctext{} by
specifying it as the optional argument of \!\begin!|{|\env{paracol}|}|.
For example, at the beginning of this document, the author put;
\switchcolumn
\env{paracol} 环境也可以以一个 \mctext{} 开始,将其指定为 \!\begin!|{|\env{paracol}|}| 的可选参数。例如,在本文档的开头,作者使用了以下代码:

\switchcolumn[1]*
\begin{Verbatim}
\begin{paracol}{2}[\section{Introduction}]
\end{Verbatim}

\switchcolumn
\begin{Verbatim}
\begin{paracol}{2}[\section{介绍}]
\end{Verbatim}

\end{paracol}    
\end{document}%%%%%%
% \section{Environments for Columns\hfill 栏环境}\label{sec:env}

\columnratio{0.3,0.42,0.28}
\begin{paracol}{3}
\begin{column}
\begin{VerbatimII}
...
\begin{column*}[\section{Environments for Columns}]
...
\end{column*}
\begin{column}
...
\end{column}
\end{VerbatimII}
%%%%%%%%%%%%%%%%%%%%%%%%%%%%%%%%%%%%%%%%%%%%%%%%%%%%%%%
\end{column}

\begin{column}
\Uidx{\Index{column-switching environment}}
\subsection{Environment \texttt{column}}
The \!\switchcolumn! is simple but you may prefer to pack the contents of a
column in an environment.  The \Uidx{\env{column}} environment is
available for this well-structuralization of \LaTeX{} sources for
parallel-columned documents. A construct;
\end{column}

\begin{column}
\subsection{\ttfamily column环境}
\!\switchcolumn! 命令很简单,但你可能更喜欢将一个栏的内容封装在一个环境中。\Uidx{\env{column}} 环境可以用于在 \LaTeX{} 文档中良好地组织并列栏的内容。以下结构:
\end{column}

\switchcolumn[1]
\begin{quote}
\!\begin!|{|\env{column}|}|\\
\textit{text for a column}\\
\!\end!|{|\env{column}|}|
\end{quote}
\noindent is (almost) equivalent to;
\begin{quote}
\!\switchcolumn!\\
\textit{text for a column}
\end{quote}
\switchcolumn
\begin{quote}
\!\begin!|{|\env{column}|}|\\
\textit{栏中文字}\\
\!\end!|{|\env{column}|}|
\end{quote}
(几乎)等同于:
\begin{quote}
\!\switchcolumn!\\
\textit{栏中文字}
\end{quote}



\end{paracol}


\end{document}%%%%%%





\begin{paracol}{2}
\switchcolumn
\switchcolumn*
The \Uidx{\env{column*}} environment is also available for the column
\sync{}ation and may have an optional argument for \mctext.
\switchcolumn
\Uidx{\env{column*}} 环境也可用于栏的同步,并且可以有一个可选参数用于 \mctext。
\end{paracol}

\begin{paracol}{2}
\begin{nthcolumn}{0}
\subsection{Environment \texttt{nthcolumn}}
The \!\switchcolumn! can start an arbitrarily specified column with the
column number given through its optional argument, but the \env{column}
environment cannot do it.  If you want to start $i$-th column, you have to
do \!\begin!|{|\Uidx{\env{nthcolumn}}|}{|$i$|}| (or
\Uidx{\env{nthcolumn*}} with an optional argument to \sync{}e).
\end{nthcolumn}

\begin{nthcolumn}{1}
\subsection{\texttt{nthcolumn}环境}
\!\switchcolumn! 可以通过可选参数指定要开始的任意列的列号,但 \env{column} 环境不能这样做。如果你想要开始第 $i$ 列,你需要使用 \!\begin!|{|\Uidx{\env{nthcolumn}}|}{|$i$|}|(或带有可选参数的 \Uidx{\env{nthcolumn*}} 来进行同步)。
\end{nthcolumn}
\end{paracol}
\begin{Verbatim}
\begin{paracol}{2}
\begin{nthcolumn*}{1}
\subsection{...}
...
\end{nthcolumn*}

\begin{nthcolumn}{0}
\subsection{...}
...
\end{nthcolumn}
\end{paracol}
\end{Verbatim}


\begin{paracol}{2}
\begin{leftcolumn*}
\subsection[Environments \texttt{leftcolumn} and \texttt{rightcolumn}]
    {Environments \texttt{leftcolumn} and\\\texttt{rightcolumn}}
The environments \Uidx{\env{leftcolumn}} and \Uidx{\env{rightcolumn}} (and
their starred versions with an optional argument) are available as more
convenient means than saying \!\begin!|{|\env{nthcolumn}|}{0}| to switch
to the left(most) column and
\!\begin!|{|\env{nthcolumn}|}{1}| to the right (but may not be rightmost)
one.

\Uidx{\EnvIndex{leftcolumn*}}\Uidx{\EnvIndex{rightcolumn*}}

\end{leftcolumn*}

\begin{rightcolumn}
\subsection{\ttfamily leftcolumn 和 rightcolumn \\环境}
环境 \Uidx{\env{leftcolumn}} 和 \Uidx{\env{rightcolumn}}(以及带有可选参数的星号版本)可作为比使用 \!\begin!|{|\env{nthcolumn}|}{0}| 切换到最左栏 和 \!\begin!|{|\env{nthcolumn}|}{1}| 切换到右栏(可能不是最右)更方便的方法。
\end{rightcolumn}

\end{paracol}
% \section{Floats, Footnotes and Counters}

\columnratio{0.3,0.42,0.28}
\begin{paracol}{3}

\begin{VerbatimII}
\switchcolumn[0]*
\begin{figure*}\nosv
\def\arraystretch{0.8}
\centerline{\begin{tabular}[b]{|c|}\hline
    \hbox to.9\textwidth{}\\
    three-column figure \#1\\
    \\\hline
    \end{tabular}}
\caption{A Three-Column Figure}
\end{figure*}

\switchcolumn
\begin{figure}[t]\nosv
\def\arraystretch{0.8}
\centerline{\begin{tabular}[b]{|c|}\hline
    \hbox to.9\columnwidth{}\\\\
    single-column figure \#1\\
    \\\\\hline
    \end{tabular}}
\caption{A Single-Column Figure}
\end{figure}

\switchcolumn
\begin{figure}[t]\nosv
\def\arraystretch{0.8}
\centerline{\begin{tabular}[b]{|c|}\hline
    \hbox to.9\columnwidth{}\\
    \ttfamily single-column figure \#2\\
    \\\hline
    \end{tabular}}
\caption{\ttfamily Another Single-Column Figure}
\end{figure}
\end{VerbatimII}
%%%%%%%%%%%%%%%%%%%%%%%%%%%%%%%%%%%%%%%%%%%%%%%%%%%%%%%
\switchcolumn
\subsection{Figures and Tables}
Double-column figures\slash tables (or those
spanned multiple columns if you have three or more) may be placed by
\env{figure*} and \env{table*} environments as usual\footnote{
See Section~\ref{sec:problem} for the appearance order issue of
double-column floats.}.

\switchcolumn
\subsection{图表}
双栏图表(如果有三栏或更多栏,则为跨多栏的图表)可以像往常一样使用 \env{figure*} 和 \env{table*} 环境来放置\footnote{请参见第~\ref{sec:problem} 节有关双栏浮动体出现顺序问题的内容。}。

\switchcolumn[1]
A single-column figure\slash table will be placed in the column in which
you put \env{figure} and \env{table}.  For example, the body of a
\env{figure} environment in a \env{leftcolumn} environment is
\emph{always} placed in a left column.  That is, even if the column of the
\emph{current} page does not have enough room to place the figure, it will
not be thrown to the right column but will be placed in the left column of
the next page\footnote{Or some farther page if \LaTeX{} cannot solve the placement problem wisely.}.
\switchcolumn
单栏图表将放置在你放置 \env{figure} 和 \env{table} 环境的栏中。例如,在 \env{leftcolumn} 环境中的 \env{figure} 环境中的内容将始终放置在左栏中。也就是说,即使当前页面的栏没有足够的空间放置图表,它也不会被放置在右栏,而是会放置在下一页的左栏\footnote{如果 \LaTeX{} 无法明智地解决放置问题,则可能放置在更远的页面上。}。

\switchcolumn[1]
\begin{table}[b]\nosv
\caption{A Single-Column Table}
\centerline{\begin{tabular}[t]{|l|c|r|}\hline
An&example&of\\\hline
single&column&table\\\hline
\end{tabular}}
\end{table}
\switchcolumn
\begin{table}[b]\nosv
\caption{\ttfamily Another Single-Column Table}
\label{tab:right}
\centerline{\ttfamily \begin{tabular}[t]{|l|r|}\hline
  Another&example\\\hline
  of&single\\\hline
  column&table\\\hline
  \end{tabular}}
\end{table}

\switchcolumn[1]
Another caution about float placement is that you have to be careful when
you try to put a top-float explicitly with |t|-option or implicitly without
placement option (i.e., |tbp| in most classes) and to \sync{}e columns.
The rule is as follows; after you \sync{}e columns in a page, the page
cannot have top-floats any more.  When you \sync{}e columns,
\textsf{paracol} fixes a virtual horizontal line in the page as the
\sync{}ation barrier.  Thus no top-floats cannot be added above the
line\footnote{Even if you have enough space above, sorry.}.
\switchcolumn
关于浮动位置的另一个警告是,当你试图使用 |t| 选项显式地放置一个顶部浮动,或者不使用放置选项隐式地放置(即,在大多数类中的 |tbp|),并且要同步列时,你必须小心。规则如下:在你在一个页面中同步列后,该页面不能再有顶部浮动。当你同步列时,\textsf{paracol} 在页面中固定一个虚拟的水平线作为同步屏障。因此,不能在该线以上添加顶部浮动\footnote{即使你在上方有足够的空间,抱歉。}。

\switchcolumn[1]
Therefore, the author put two \env{figure} environments for the figures
shown in this page into the \env{leftcolumn*} and \env{rightcolumn}
environment for the previous section.
\switchcolumn
因此,作者将在上一节的 \env{leftcolumn*} 和 \env{rightcolumn} 环境中放入本页显示的两个 \env{figure} 环境。

\switchcolumn[1]
\subsection{Footnotes and Marginal Notes}
Footnotes are also put at the bottom of the column in which \!\footnote!
commands and their references reside (like this\footnote{%
Unless you specify to make footnotes {\em page-wise} as explained in
Section \ref{sec:ref-scfnote} and \ref{sec:fnnp}.}),
\switchcolumn
\subsection{脚注和边注}
脚注也会放置在包含 \!\footnote! 命令及其引用的栏的底部(如本页所示\footnote{除非你在第 \ref{sec:ref-scfnote} 节和 \ref{sec:fnnp} 节中指定将脚注{\em 按页}处理。}),

\switchcolumn[1]
as shown in page~\pageref{fn:first} and this page.  Marginal
notes behave similarly like what you are seeing in the left margin of this
sentence\marginpar{\raggedright An example of marginal note.}
\switchcolumn
如第~\pageref{fn:first}页和本页所示。边注表现类似于你看到的这句话左 margin 中的样式\marginpar{\raggedright 一个边注示例。}

\switchcolumn[1]  
and the right marginal note in this page\footnote{%
If you have three or more columns, marginal notes of the second or
succeeding columns are placed in the right margin in default setting.  The
\textsf{paracol} package solves the placement problem of marginal notes
from two or more columns sharing a side margin by moving some of them down
if they conflict over the space with each other.}.
\switchcolumn
以及本页中的右边距注释\footnote{如果你有三列或更多列,第二列或后续列的边距注释在默认设置中放置在右边距。 \textsf{paracol}包处理来自两个或更多共享侧边距的列的边距注释的放置问题,如果它们在空间上彼此冲突,将其中一些向下移动。}。
\end{paracol}


\columnratio{0.3,0.42,0.28}
\begin{paracol}{3}

\begin{VerbatimII}
\end{VerbatimII}

\switchcolumn[1]
\subsection{Local and Global Counters}
\UsageIndex{local counter}
\UsageIndex{global counter}
You probably found that the numbering of figures and tables is \emph{global}
while that of footnotes are \emph{local}.  That is, the figure in the right
column of the previous page has number~3 following its left-column
counterpart Figure~2.  The tables in the page are also numbered as 1 and 2
crossing the column boundary.  However, the footnotes in each column have
their own numbering sequence.  Moreover, the footnote numbers in left
columns are typeset in roman font while those in right columns have italic
shapes.  Similarly, subsection numbering is local and the headings in right
columns have typewriter-face numbers.
\switchcolumn
\subsection{局部和全局计数器}
你可能发现,图表的编号是\emph{全局}的,而脚注的编号是\emph{局部}的。也就是说,上一页右栏的图表在其左栏对应的图表之后编号为3,而页面上的表格也是以1和2为编号跨越栏边界。然而,每栏中的脚注有自己的编号序列。此外,左栏中的脚注号码以罗马字体排版,而右栏中的脚注号码以斜体形式排版。类似地,小节编号是局部的,右栏标题的编号使用打字机字体。

\switchcolumn[0]*
\begin{itemize}\item[]
\Uidx{\!\globalcounter!}|{figure}|\\
\!\globalcounter!|{table}|
\end{itemize}

\switchcolumn[1]
This happens because the author declared the counters \counter{figure} and
\counter{table} are \emph{global} in the preamble of this document by
saying;
\switchcolumn
这是因为作者在文档的导言部分中声明了计数器 \counter{figure} 和 \counter{table} 是\emph{全局}的,声明首栏所示。

\switchcolumn[1]
and do nothing about \counter{footnote} and \counter{subsection} counters.
By default, all the counters except for |page| are local to columns.  The
value of a \lcounter{} of a column is saved somewhere when you leave the
column, and it is restored when you revisit the column.  The initial values
of the \lcounter{}s are the values they have at
\!\begin!|{|\env{paracol}|}|.  After you close the \env{paracol}
environment, the values of the leftmost column are used for the rest of
your document until you start new \env{paracol} environment.  On a
restart, \lcounter{}s in a column have the values they had at the last
\Endparacol, except for those which have been modified outside the
environment because the modifications are \emph{broadcasted} to
\lcounter{}s in all columns.  You will see the effect of this
inter-environment counter value conservation in the footnote numbers in
the right column in page~\pageref{fn:right3} and \pageref{fn:right4}.
\switchcolumn
但对于计数器 \counter{footnote} 和 \counter{subsection} 未进行任何处理。默认情况下,除了 |page| 计数器外,所有的计数器都是局部的。当你离开栏目时,栏目的局部计数器值会被保存住,当你再次访问该栏目时,该值会被恢复。在 \env{paracol} 环境的初始值为局部计数器的值。当你关闭 \env{paracol} 环境后,剩余部分的文档将使用最左边栏的值,直到你开始新的 \env{paracol} 环境。重新开始时,栏目中的局部计数器具有最后一个 \Endparacol 时的值,除非在环境外进行了修改,因为这些修改会被\emph{广播}到所有栏的局部计数器中。你将在第\pageref{fn:right3}页和第\pageref{fn:right4}页中看到这种跨环境计数值保存的效果,表现在右栏的脚注号码上。

\switchcolumn[1]
This broadcasting of a \lcounter{} value can be done explicitly in
\env{paracol} environments by a command $\Uidx{\!\synccounter!}\Arg{ctr}$.
This command makes $\mathit{ctr}$ in all columns have the value of that in
the column in which the command appears.  In addition, another command
\Uidx{\!\syncallcounters!} performs this broadcasting for all \lcounter{}s.
\switchcolumn
可以在\env{paracol}环境中通过命令$\Uidx{\!\synccounter!}\Arg{ctr}$来显式地进行局部计数器值的广播。所有栏中的$\mathit{ctr}$都将同步为命令调用所在栏中的值。此外,另一个命令 \Uidx{\!\syncallcounters!} 可以对所有局部计数器进行这种广播操作。

\switchcolumn[1]
If you make a counter global by the command \!\globalcounter!, the
save/restore operations are not performed to the counter and thus it is
globally incremented by \verb|\[ref]|\AB|stepcounter|
\SpecialIndex{\refstepcounter}\SpecialIndex{\stepcounter}

\switchcolumn
若用 \!\globalcounter! 将某计数器声明为全局的,则不会对其执行保存/恢复操作,它会通过 \verb|\[ref]|\AB|stepcounter| 全局递增。


\end{paracol}


\end{document}
%%%%%%%%%%%%%%%%%%%%%%%%%%%%%%%%%%%%%%%%%%%%%%%%%%%%%%%
%%%%%%%%%%%%%%%%%%%%%%%%%%%%%%%%%%%%%%%%%%%%%%%%%%%%%%%










\switchcolumn*
or commands such as \!\caption! and \!\section!.  Note that the value of a
\gcounter{} depends on the place where it is incremented (or set) in
the \emph{source code} rather than where it appears in the output.  Thus
if the author put a \env{table} environment here to increment \env{table}
counter, the right-column table at the bottom of page~\pageref{tab:right}
would be Table~3 because its \env{table} environment does not appear yet
in the source code.  Note that, however, though the counter \counter{page}
is global as expected, its numbering is consistent among all columns as
far as you refer to the value by $\!\pageref!\Arg{label}$ and/or see the
values in table of contents, etc.
\switchcolumn
或者诸如 \!\caption! 和 \!\section! 等命令。请注意,一个\gcounter{}的值取决于它在\emph{源代码}中递增(或设置)的位置,而不是它在输出中出现的位置。因此,如果作者在这里放置了一个\env{table}环境来递增\env{table}计数器,那么在第\pageref{tab:right}页底部的右栏表格将被标记为表格3,因为它的\env{table}环境在源代码中尚未出现。请注意,尽管计数器\counter{page}是全局的,但只要通过 $\!\pageref!\Arg{label}$ 引用该值,或者在目录中查看值等,其编号在所有栏目中是一致的。

\switchcolumn*
Another counter which the author made global in this document is
\counter{section}.  As explained in Section~\ref{sec:sync}, an optional
\mctext{} of \cswitch{} is considered as in the leftmost column.  Since
\!\section! commands in this document are always given in \mctext{}s, so
far, it seems unnecessary to make \counter{section} global because it is
incremented correctly in the leftmost column.  However, the stepping
\counter{section} has a side effect to reset its descendent counter
\counter{subsection} and referred to from \!\thesubsection! command.  Thus
if \counter{section} were local, the right-column subsections in
Section~\ref{sec:env} would be numbered as ``0.1'', ``0.2'' and ``0.3''
because the local value of \counter{section} would be zero.  Moreover, the
right-column subsections of this section would be ``0.4'', ``0.5'' and
``0.6'' because stepping \counter{section} local to the left column would
not reset \counter{subsection} local to the right column.
\switchcolumn
在本文档中,作者还将\counter{section}计数器声明为全局的。如第~\ref{sec:sync}节所述,\cswitch{}的可选\mctext{}被视为最左边的栏目。由于本文档中的 \!\section! 命令总是在\mctext{}中给出,因此目前似乎没有必要将\counter{section}设置为全局,因为它在最左边的栏目中递增是正确的。然而,递增\counter{section}会对其子计数器\counter{subsection}产生副作用,并且从 \!\thesubsection! 命令中引用。因此,如果\counter{section}是局部的,那么在第~\ref{sec:env}节中右栏的子章节将被编号为“0.1”、“0.2”和“0.3”,因为\counter{section}的局部值将为零。此外,本节的右栏子章节将被编号为“0.4”、“0.5”和“0.6”,因为局部递增的\counter{section}不会重置右栏局部的\counter{subsection}。


\switchcolumn*
You may give a local appearance to a counter \textit{ctr} for the $i$-th
column (zero origin) by a command;
\begin{itemize}\item[]
\Uidx{\!\definethecounter!}|{|\textit{ctr}|}{|$i$|}{|\textit{def}|}|
\end{itemize}
\switchcolumn
你可以通过命令给第$i$栏目(从零开始计数)的计数器\textit{ctr}赋予局部的外观;

\switchcolumn*
where \textit{def} is to be the body of the local definition of
|\the|\textit{ctr}.  For example, the preamble of this document has the
following to give non-default defitions to \!\thefootnote! and
\!\thesubsection! for right columns.

\begin{Verbatim}
\definethecounter{footnote}{1}{%
\textit{\arabic{footnote}}}
\definethecounter{subsection}{1}{%
\texttt{%
    \arabic{section}.\arabic{subsection}}}
\end{Verbatim}
\switchcolumn
其中\textit{def}是局部定义|\the|\textit{ctr}的内容。例如,本文档的导言部分具有以下内容,为右栏的\!\thefootnote!和\!\thesubsection!赋予非默认的定义。

\end{paracol}

 
% \section{Closing \texttt{paracol} Environment and Page Flushing\hfill 关闭 \texttt{paracol} 环境和页面刷新}
\label{sec:man-close}

The final example shown here is this single-column text which the author put
after the \env{paracol} environment above is closed.  As you are seeing, a
\env{paracol} environment can be finished at any vertical position in a
page and can be followed by ordinary single column texts.


这里展示的最后一个例子是在上面关闭的\env{paracol} 环境之后,作者放置的这个单栏文本。正如你所见,\env{paracol} 环境可以在页面的任何垂直位置结束,并且可以跟随普通的单栏文本。

\columnratio{0.3,0.42,0.28}
\begin{paracol}{3}
\begin{VerbatimII}
\begin{paracol}{2}
\begin{leftcolumn}
The enviro ... 
\end{leftcolumn}
\begin{rightcolumn}
source
\end{rightcolumn}
\end{paracol}
Now the aurthor will do ...
\end{VerbatimII}
%%%%%%%%%%%%%%%%%%%%%%%%%%%%%%%%%%%%%%%%%%%%%%%%%%%%%%%
\switchcolumn

\switchcolumn[1]
The environment may also be restarted anywhere you like as shown here.
\switchcolumn
此处展示了环境可以在任何位置重新开始。

\switchcolumn[1]
The last issue is to flush a page.  The ordinary \!\newpage! command works
as you expect.  If you say \!\newpage! in the left column in a page, the
contents following it will appear in the left column in the next page.  Note
that this does not affect the layout of the right column.
\switchcolumn
最后一个问题是如何换页。\!\newpage! 命令按照你的期望工作。如果你在页面的左栏使用 \!\newpage! 命令,在它之后的内容将出现在下一页的左栏中。请注意,这不会影响右栏的布局。

\switchcolumn[1]
To flush all columns in a page, a command \Uidx{\!\flushpage!} is
available.  This command in $i$-th column is almost equivalent to;
\begin{itemize}\item[]
\!\switchcolumn!|[|$i$|]*[|\!\newpage!|]|
\end{itemize}
\switchcolumn
要在页面中刷新所有栏目,可以使用命令 \Uidx{\!\flushpage!}。这个命令在第$i$栏中几乎等同于:
\begin{itemize}\item[]
\!\switchcolumn!|[|$i$|]*[|\!\newpage!|]|
\end{itemize}

\switchcolumn[1]
but more robust\footnotemark\label{fn:flush}.
The ordinary page breaking command \Uidx{\!\clearpage!} may also be used
to flush all columns and to start a fresh page, but it has a side effect
to put all figures and tables which are not yet output.
\switchcolumn
但更加健壮 \footnotemark\label{fn:flush} 。普通的换页命令 \Uidx{\!\clearpage!} 也可以用于刷新所有栏目并开始新的一页,但它会导致尚未输出的所有图表被放置在同一页中。
\end{paracol}

Now the author will do |\flushpage| shortly to start a real binlingual
example from the next page, after showing another example of closing 
\env{paracol} environments in this sentence and of restarting in the next
one, in which {\em unbalanced column width} is demonstrated using
\Uidx{\!\columnratio!} command shown in Section~\ref{sec:ref-colwidth}.

现在作者将很快使用|\flushpage|命令,在下一页开始一个真正的双语示例,此前在本句中展示了另一个关闭\env{paracol}环境的例子,并在下一句中重新开始,在其中使用了在第~\ref{sec:ref-colwidth}节中展示的 \Uidx{\!\columnratio!} 命令演示了{\em 不平衡的列宽}。

\columnratio{0.3,0.42,0.28}
\begin{paracol}{3}
\begin{VerbatimII}
\columnratio{0.6} 
\begin{paracol}{2} 
\begin{leftcolumn}
O.K., ... 
\end{leftcolumn} 
\begin{rightcolumn}
source 
\end{rightcolumn}
\end{VerbatimII}
%%%%%%%%%%%%%%%%%%%%%%%%%%%%%%%%%%%%%%%%%%%%%%%%%%%%%%%
\switchcolumn

O.K., we have restarted \env{paracol} environment and we will see the
effect of \!\flushpage! now!!\footnotetext{
For example \texttt{\string\switchcolumn*} may flush a page for the
\sync{}ation and thus \texttt{\string\newpage} may leave an empty page.}
\switchcolumn
好的,我们已经重新开始了\env{paracol}环境,现在我们将看到 \!\flushpage! 命令的效果!!\footnotetext{例如,\texttt{\string\switchcolumn*}可能会为同步而刷新页面,因此\texttt{\string\newpage}可能会留下一个空白页。}

\newenvironment{Gverse}{\ensurevspace{2\baselineskip}
\begin{leftcolumn*}
\begin{myverse}}
{\end{myverse}\end{leftcolumn*}}
\newenvironment{Everse}{%
\begin{rightcolumn}
\begin{myverse}}
{\end{myverse}\end{rightcolumn}}
\newenvironment{Cverse}{%
\begin{nthcolumn}{2}
\begin{myverse}}
{\end{myverse}\end{nthcolumn}}
\makeatletter
\newenvironment{myverse}{\leftmargini0pt\partopsep0pt\verse}{\endverse}


\begin{leftcolumn*}[
\centerline{\Large An Die Freude/To Joy}\label{page:bfreude}\smallskip
\centerline{\large Friedrich Schiller 弗里德里希·席勒}\smallskip
The following is the libretto of the fourth movement of Beethoven's Ninth
Symphony, his adaptation of Schiller's ode ``An Die Freude'' (or ``To Joy'' in
English). Beethoven's additions and revisions are indicated in italics.

以下是贝多芬第九交响曲第四乐章的歌剧剧本,他改编自席勒的颂歌《致欢乐》(或英文版的《To Joy》)。贝多芬的添加和修订以斜体显示。]
\end{leftcolumn*}

\begin{Gverse}
\itshape O Freunde, nicht diese T\"one! \\
Sondern la{\ss}t uns angenehmere anstimmen und freu\-denvollere
\footnote{If I had been a good student in my German class, I could find
the German translation of the right column footnote \ref{fn:right4} is
``Dieser Teil wurde van Beethoven hinzugef\"ugt'' by myself without
the kind help from a user.}.
\end{Gverse}
\begin{Everse}
\itshape Oh friends, no more of these sad tones!\\
Let us rather raise our voices together\\
In more pleasant and joyful tones
\footnote{This part was added by Beethoven.\label{fn:right4}}.
\end{Everse}
\begin{Cverse}
    a
\end{Cverse}


\end{paracol}


\end{document}
%%%%%%%%%%%%%%%%%%%%%%%%%%%%%%%%%%%%%%%%%%%%%%%%%%%%%%%
%%%%%%%%%%%%%%%%%%%%%%%%%%%%%%%%%%%%%%%%%%%%%%%%%%%%%%%


\begin{leftcolumn}


% \tracingpages0 \tracingoutput0
% \newpage
% \footnotelayout{m}
% \columnratio{}
% \section{Reference Manual\\参考手册}
% \label{sec:ref}
% \subsection{Environment \texttt{paracol}\hfill \texttt{paracol}环境}
\label{sec:ref-paracol}

\begin{description}%值得学习
\item[\ENV{paracol}{\marg{num}\oarg{text}}]\mbox{}\par
\columnratio{0.6}
\begin{paracol}{2}
The environment \env{paracol} contains \meta{body} typeset in \meta{num}
columns in parallel.  The optional \meta{text} is put spanning all columns
prior to the multi-columned \meta{body}.
\switchcolumn
环境\env{paracol}中包含以 \meta{num} 栏并列排列的 \meta{body}。可选的 \meta{text} 将跨越所有栏之前放置在多栏的 \meta{body} 之前。
\end{paracol}

\begin{itemize}
\columnratio{0.6}
\begin{paracol}{2}
\item
The environment may start from \emph{any} vertical position in a page,
i.e., not necessary at the top of a page.  The single-column
{\em\Uidx\preenv} of the {\em\Uidx\spage} in which \beginparacol{} lies
are naturally connected to the beginning part of \meta{body} in each
column, unless the page has footnotes\footnote{%
With \Mgfnote{} layout shown in Section~\ref{sec:ref-scfnote}, the
footnotes in the single-column contents are merged with those in
\env{paracol} environment and are put at the bottom of the \spage{}
together as shown in this page.}

or bottom floats.  If these kinds of bottom stuff exist, they are
put above the multi-columned \meta{body}, or the spanning \meta{text}

\UsageIndex{spanning text}

if provided, with a vertical skip of \!\textfloatsep! separating them if
bottom floats exist, or of \!\belowfootnoteskip! described in
Section~\ref{sec:ref-scfnote} if only footnotes exist.  The
\emph{deferred} floats which have not yet appeared in the starting page
and thus will appear in the next or succeeding pages are considered as
\pwise{} floats given in the environment.
\switchcolumn
\item
此环境可以从页面的\emph{任何}垂直位置开始,
即不一定在页面顶部。位于 \beginparacol{} 所在的 {\em\Uidx\spage} 中的单栏 {\em\Uidx\preenv} 自然与每个栏的\meta{body}的开头部分连接在一起,除非页面有脚注\footnote{使用在第~\ref{sec:ref-scfnote}节中展示的\Mgfnote{}布局,单栏内容中的脚注与\env{paracol}环境中的脚注合并在一起,并一起放置在\spage{}的底部,就像本页所示。}
,或底部浮动体。如果存在这些底部内容,则它们将位于多栏的\meta{body}之上,或者位于跨越的\meta{text}之上(如果提供了),并使用垂直间距 \!\textfloatsep! 将它们分隔开(如果存在底部浮动体),或者使用在第~\ref{sec:ref-scfnote}节中描述的 \!\belowfootnoteskip! (仅当存在脚注时)。尚未出现在起始页面中的\emph{延迟}浮动体将被视为在环境中给出的\pwise{}浮动体,它们将出现在下一页或后续页面中。
\switchcolumn[0]*
\item
The environment can be enclosed in a \env{list}{\em-like environment} such
as \env{enumerate}, \env{itemize} and \env{description}.  If so, \!\item!s
in each column are typeset using the parameters of the surrounding
environment such as \!\leftmargin! and \!\rightmargin!.  %For example, the
following short \env{paracol} environment is included in an \env{itemize}
for this and other \!\item!s in this page.
\switchcolumn
\item
该环境可以被封装在类似于 \env{enumerate}、\env{itemize} 和 \env{description} 的\emph{类似列表}环境中。如果这样做,每个栏中的 \!\item! 将使用周围环境的参数进行排版,如 \!\leftmargin! 和 \!\rightmargin!。%例如,以下简短的\env{paracol}环境被包含在一个\env{itemize}中,用于本页和其他 \!\item!。

\end{paracol}

You are now seeing the switching to/from multi-columned and \env{itemize}d
texts are naturally connected with the last and this single-columned
sentences.  You may feel the space between two columns above is too large
but it simply results from the large total \!\leftmargin!s of the outer
\env{description} and this \env{itemize}, which make the right column
shifted right.  A simple remedy for this large space is to make
\!\columnsep! narrower, for example 0\,pt as shown below.

您现在看到的切换到/从多栏和\env{itemize}文本与上一个和本个单栏句子自然连接在一起。您可能会觉得上面两栏之间的空间太大,但这只是由于外部\env{description}和此\env{itemize}的总 \!\leftmargin! 较大,使得右栏向右偏移。修复这个大空间的简单方法是使 \!\columnsep!变窄,例如像下面显示的0\,pt。

\begin{Verbatim}
\columnsep0pt
\end{Verbatim}

\columnsep0pt
\begin{paracol}{2}
\item
This \!\item! is wider than the last \!\item! above because
\!\columnsep! is 0\,pt.

这个 \!\item! 比上面的最后一个 \!\item! 更宽,因为 \!\columnsep! 是0\,pt。
\switchcolumn

\item
Therefore, this \!\item! is shifted left a little bit to make
inter-column spece narrower.

因此,为了使栏间距更窄,这个 \!\item! 向左移动了一点。
\end{paracol}

\columnratio{0.55}
\begin{paracol}{2}
\item
All \Uidx\lcounter{}s in all columns are initialized to have the values at
\beginparacol{} on its first occurrence.  On the second and succeeding
occurrences of \beginparacol, the \lcounter{}s in each column have the
value at the last \Endparacol, unless they are modified after the
\Endparacol.  If a counter is modified (or declared by \!\newcounter!)
after the \Endparacol, the local versions of the counter in all columns
commonly have the value at \beginparacol.
\switchcolumn
\item
所有栏中的局部计数器都被设为\beginparacol{}首次出现时的值。在\beginparacol{}的第二次及后续出现中,每个栏中的局部计数器都具有上一个\Endparacol{}处的值,除非在\Endparacol{}之后对其进行了修改。如果在\Endparacol{}之后修改了计数器(或通过 \!\newcounter! 声明了计数器),所有栏中的局部计数器都通常具有\beginparacol{}处的值。
\switchcolumn[0]*
\item
The environment may end at \emph{any} vertical position in a page, i.e.,
the {\em\Uidx\postenv} being the single-column texts and others
following \Endparacol{} in the {\em\Uidx\lpage} of the environment may not
start from the top of a page.  If any columns don't have deferred
\cwise{} floats and the most advanced {\em\Uidx\lcolumn} at
\Endparacol{} has neither of footnotes\footnote{

With \Mgfnote{} layout shown in Section~\ref{sec:ref-scfnote}, the
footnotes in the closing \env{paracol} environment are merged with those
in \postenv{} and are put at the bottom of the page{} together as shown in
this page.}

nor bottom floats, its bottom is naturally connected to the \postenv{}.
If the \lcolumn{} has these kinds of bottom stuff, they are put above the
\postenv{}, with a vertical skip of \!\textfloatsep! separating them if
bottom floats exist.  All deferred \cwise{} floats given in the
environment are flushed before the \postenv{} appears, possibly creating
{\em\Uidx\fcolumn{}s} only with floats.  On the other hand, deferred
\pwise{} floats given in the environment are considered as deferred
(single-) \cwise{} floats given just after \Endparacol.
\switchcolumn
该环境可以在页面的\emph{任何}垂直位置结束,即\emph{\Uidx\postenv}是单栏文本,而在环境的\emph{\Uidx\lpage}中的\Endparacol{}之后的其他内容可能不会从页面顶部开始。如果任何栏没有延迟的\cwise{}浮动体,并且最后一个\Endparacol{}处的\emph{\Uidx\lcolumn}既没有脚注\footnote{使用在第~\ref{sec:ref-scfnote}节中展示的\Mgfnote{}布局,\env{paracol}环境中的脚注与\postenv{}中的脚注合并在一起,并一起放置在页面底部,就像本页所示。},也没有底部浮动体,则其底部自然与\postenv{}连接在一起。如果\lcolumn{}具有这些类型的底部内容,则它们将位于\postenv{}之上,如果存在底部浮动体,则它们之间使用垂直间距 \!\textfloatsep! 分隔开。在\postenv{}出现之前,环境中给出的所有延迟\cwise{}浮动体都会被清除,可能只留下具有浮动体的{\em\Uidx\fcolumn{}s} 。另一方面,环境中给出的延迟\pwise{}浮动体被视为在\Endparacol{}之后立即给出的延迟(单个)\cwise{}浮动体。
\switchcolumn[0]*
\item
The values of all \lcounter{}s in the leftmost column are used as the
initial values of them in the \postenv.
\switchcolumn
\item
左侧栏中所有\lcounter{}的值被用作\postenv{}中对应\lcounter{}的初始值。
\switchcolumn[0]*
\item
The \env{paracol} environment cannot be nested, or you will have an error
message of illegal nesting.
\switchcolumn
\item 不能嵌套使用\env{paracol}环境,否则会出现非法嵌套的错误消息。 
\switchcolumn[0]*
\item
The commands \!\switchcolumn!, \!\synccounter!, \!\syncallcounters! and
\!\flushpage!, and environments \env{column}(|*|), \env{nthcolumn}(|*|),
\env{leftcolumn}(|*|) and \env{rightcolumn}(|*|) are {\em local} to
\env{paracol} environment and thus undefined outside the
environment\footnote{

Unless you dare to define them.}.

The command \!\clearpage! is of course usable outside and inside the
environment but its function inside is a little bit different from outside.
\switchcolumn
\item 命令 \!\switchcolumn!、\!\synccounter!、\!\syncallcounters! 和 \!\flushpage!,以及环境\env{column}(|*|)、\env{nthcolumn}(|*|)、\env{leftcolumn}(|*|)和\env{rightcolumn}(|*|)是\env{paracol}环境中的{\em 局部}命令和环境,因此在环境外部是未定义的\footnote{除非你敢于定义它们。}。

命令 \!\clearpage! 当然可以在环境内外使用,但在环境内部的功能与外部略有不同。
\end{paracol}
\end{itemize}



\item[\ENV{paracol}{\oarg{numleft}\marg{num}\oarg{text}}]\mbox{}
\Item[\ENV{paracol}{\oarg{numleft}\texttt{*}\marg{num}\oarg{text}}]
\mbox{}\par
\changes{v1.3-2}{2013/09/17}
{Add description of parallel-paging.}

If a \beginparacol{} has the optional \meta{numleft} argument to specify
the number of leading columns $n_l$ together with the total $n$ given by
\meta{num}, columns in the environment are laid out across two adjacent
pages.  In this {\em\Uidx\parapag{}e} typesetting, the first $n_l$ columns
are placed in the {\em left} page while remaining $n_r=n-n_l$ columns go to
the next {\em right} page.  The pair of left and right pages is
considered as comprising a virtual {\em\Uidx\paired} page and thus shares
a common page number, unless {\em\Uidx\npaired} typesetting is specified
by the optional `|*|' following the optional \meta{numleft} argument.  In
the \npaired{} \parapag{}ing, when the leading $n_l$ columns are put in a
page $p$, the trailing $n_r$ columns are in the page $p+1$.

如果\beginparacol{}的可选参数\meta{numleft}用于指定前导列的数量$n_l$,同时总列数由\meta{num}给出,那么环境中的列会跨两个相邻的页面进行布局。在这种{\em\Uidx\parapag{}e}排版中,前$n_l$列放置在{\em 左侧}页面,而剩下的$n_r=n-n_l$列放置在下一个{\em 右侧}页面。左侧和右侧页面的配对被认为是组成一个虚拟的{\em\Uidx\paired}页面,因此它们共享一个相同的页码,除非通过在可选的\meta{numleft}参数后面添加`|*|'来指定{\em\Uidx\npaired}排版。在\npaired{} \parapag{}ing中,当前导的$n_l$列放置在页面$p$上时,后续的$n_r$列会在页面$p+1$上。

\begin{itemize}
\item
All {\em\Uidx\pwstuff}, i.e., \Preenv{} and \postenv, \pwise{} floats,
\mctext{} and (\mgfnote{} or non-merged) \Scfnote{}s, are placed only in
left \parapag{}es leaving corresponding regions in right \parapag{}es
blank\footnote{

Someday the author could devise an advanced mechanism to exploit the space
in right \parapag{}es.}.

所有的{\em\Uidx\pwstuff},即\Preenv{}和\postenv,\pwise{}浮动体,\mctext{}和(\mgfnote{}或非合并的)\Scfnote{},只会放置在左侧\parapag{}es中,让右侧\parapag{}es中相应的区域保持空白\footnote{将来作者可能会设计一个高级机制来利用右侧\parapag{}es中的空间。}。
\item
A \npaired{} left \parapag{}e is not necessary to be even-numbered, though
the printing tradition requires so if you naturally want to have a
\parapag{}e pair in a double spread.  The page number given to the first
left \parapag{}e is simply the number of the page $p_1$ in which
\beginparacol{} reside, and that for the $k$-th left \parapag{}e is
$p_1+2(k-1)$\footnote{

Unless you make some change to \counter{page} counter.}.

Therefore, to make it sure $p_1$ is even, you might need to have an
ordinary page of blank, a title, etc., or to let \counter{page} counter have
an even number by \!\setcounter!, etc., before starting a \env{paracol}
environment.

一个没有成对出现的左页不一定是偶数页,尽管印刷传统要求如果你自然地希望在双页中有一个成对的页面。第一个左页的页码只是在\beginparacol{}所在的页$p_1$的页码,而第$k$个左页的页码是$p_1+2(k-1)$\footnote{除非你对\counter{page}计数器进行了一些更改。}。

因此,为了确保$p_1$是偶数,你可能需要在开始\env{paracol}环境之前有一个普通的空白页、一个标题等,或者通过 \!\setcounter! 等方法使\counter{page}计数器的值成为一个偶数。

\item
Section~\ref{sec:ppts} shows examples of \parapag{}ing together with
related issues on two-sided typesetting.

第~\ref{sec:ppts} 节展示了 \parapag{} 的示例,以及双面排版相关问题。
\end{itemize}
\end{description}
 
% \subsection{切换栏的命令和环境 \hfill Column-Switching Command and Environments}
\label{sec:ref-switchcolumn}

\begin{description}
\item[\Midx{\!\switchcolumn!}\oarg{col}]\mbox{}
\Item[\Midx{\!\switchcolumn!}\oarg{col}\texttt{*}\oarg{text}]\mbox{}\par
\columnratio{0.6}
\begin{paracol}{2}
The command switches columns from $i$ to $j$ where $i$ and $j$ is the
zero-origin ordinals of the columns from/to which we are leaving\slash
visiting respectively.  Without the optional \meta{col}, $j=i+1\bmod n$
where $n$ is the number of columns given to \beginparacol, while
$j=\meta{col}$ with the optional argument.  If the command (or
\oarg{col} if specified) is followed by a |*|, the \cswitch{} takes
place after \sync{}ation and, if specified, the optional spanning
\meta{text} is put.
\switchcolumn
命令从第$i$列切换到第$j$列,其中$i$和$j$是我们离开/访问的列的零起始序号。如果没有可选参数\meta{col},则$j=i+1\bmod n$,其中$n$是给定给\beginparacol{}的列数,而如果有可选参数,则$j=\meta{col}$。如果命令(或如果指定了\oarg{col})后面跟着一个|*|,则\cswitch{}将在\sync{}ation之后进行,并且如果指定了可选的跨列\meta{text},则会放置它。
\Index{spanning text}
\end{paracol}

\begin{itemize}
\columnratio{0.6}
\begin{paracol}{2}
\item
Using \!\switchcolumn! in a \env{list}-like environment \emph{included} in
a \env{paracol} environment causes an ugly result without any error\slash
warning messages.  This caution is effectual for all \csenv{}s too.
\switchcolumn\item
在\env{paracol}环境中使用 \!\switchcolumn! 命令来切换到包含在\env{list}-like环境中会导致一个不美观的结果,而且没有任何错误或警告信息。同样的注意事项也适用于所有的\csenv{}。
\switchcolumn[0]*
\item
If $\meta{col}\notin\LBRP0n$, an error is reported and, if you dare to
continue, you will switch to the leftmost column 0.
\switchcolumn\item
如果 $\meta{col}\notin\LBRP0n$,将报告错误,并且如果你敢继续,将切换到最左边的列0。    
\switchcolumn[0]*
\item
The \sync{}ation point is set just below the last line of the \lcolumn{}
in a page $p$, partly taking deferred floats into account.  That is, all
deferred floats are put in the pages up to $p-1$ and at the top of $p$ if
possible.  Then, if a non-\lcolumn{} has footnotes and/or bottom floats
and they cannot be pushed down below the \sync{}ation point, the point is
moved to the next page top\footnote{

Or below top floats deferred to the page.}.
\switchcolumn\item
\sync{}ation点设置在页$p$的\lcolumn{}的最后一行的下方,部分考虑了延迟浮动。也就是说,所有延迟浮动都放在前$p-1$页和$p$页的顶部(如果可能的话)。然后,如果非\lcolumn{}有脚注和/或底部浮动,并且它们不能被推到\sync{}ation点以下,那么点就会被移动到下一页的顶部\footnote{

或下推到页面的延迟顶部浮动下方。}。
\switchcolumn[0]*
\item
In a page having one or more \sync{}ation points, stretch and shrink
factors of all vertical spaces, such as those surrounding sectionning
commands, are ignored.  Therefore, even if you specify \!\flushbottom!,
the page is typeset as if \!\raggedbottom! were specified.
\switchcolumn\item
在一个或多个\sync{}ation点的页面中,所有垂直空间的拉伸和收缩因子都被忽略,例如围绕节标题命令的空间。因此,即使您指定了 \!\flushbottom!,页面的排版也会像指定了 \!\raggedbottom! 一样进行。
\switchcolumn[0]*
\item
After a \sync{}ation point is set, no top floats will be inserted in the page
having the point, thus they will be deferred to the next page or further one.
\switchcolumn\item
在设置了同步点之后,不会在具有该点的页面中插入顶部浮动对象,因此它们将被推迟到下一页或更远的页面。
\end{paracol}
\end{itemize}


\item[\ENV{column}{}]\mbox{}
\Item[\ENV{column*}{\oarg{text}}]\mbox{}\par
\columnratio{0.55}
\begin{paracol}{2}
The environment \env{column} contains \meta{body} for the column next to
what we are in just before \!\begin!|{|\env{column}|}|.  The starred
version \env{column*} does the same after \sync{}ation and, if specified,
the optional spanning \meta{text} is put.
\switchcolumn
环境\env{column} 包含了在 \!\begin!|{|\env{column}|}| 之前我们所在的列旁边的\meta{body}。星号版本 \env{column*} 在{\fontKai 同眇}之后执行相同的操作,并且如果指定了可选的跨列\meta{text},则会放置它。

\Index{spanning text}
\end{paracol}

\begin{itemize}
\columnratio{0.55}
\begin{paracol}{2}
\item
The environments are almost equivalent to;
\begin{quote}
|{|\!\switchcolumn!\quad\meta{body}\quad\CSIndex{par}|}|\\
|{|\!\switchcolumn!|*|\oarg{text}\quad\meta{body}\quad\CSIndex{par}|}|
\end{quote}
except for their first occurrences which don't switch to the column 1
(i.e., right column if two-columned) but stay in the leftmost column 0.
More precisely, \!\begin!|{|\env{column}(|*|)|}| does not make \cswitch{}
if it is not preceded by \!\switchcolumn! nor other \csenv{}s.
\switchcolumn\item
这些环境几乎等同于:
\begin{quote}
|{|\!\switchcolumn!\quad\meta{body}\quad\CSIndex{par}|}|\\
|{|\!\switchcolumn!|*|\oarg{text}\quad\meta{body}\quad\CSIndex{par}|}|
\end{quote}
除了第一次出现的情况外,它们不会切换到列1(即双栏时的右栏),而是保持在最左边的列0。更准确地说,如果 \!\begin!|{|\env{column}(|*|)|}| 没有在 \!\switchcolumn! 或其他 \csenv{} 之前出现,就不会进行 \cswitch{}。
\switchcolumn[0]*
\item
The \meta{body} of the environments cannot have \!\switchcolumn! nor
\csenv{}s including \env{column}(|*|) themselves, or you will have an
error message of illegal use of command\slash environment.
\switchcolumn
\item 环境的\meta{body}不能包含 \!\switchcolumn! 或包含\env{column}(|*|)本身的\csenv{},否则会出现非法使用命令\slash 环境的错误消息。
\switchcolumn[0]*
\item
Column-switching\index{column-switching} does not take place at
\!\end!|{|\env{column}(|*|)|}|.  Therefore, texts following the
environments are put in the column in which \meta{body} resides until a
\cswitch{} command\slash environment is given.
\switchcolumn\item
在 \!\end!|{|\env{column}(|*|)|}| 处不会发生列切换\index{column-switching}。因此,在环境后面的文本会放置在\meta{body}所在的列中,直到出现\cswitch{}命令\slash 环境。
\end{paracol}
\end{itemize}



\item[\ENV{nthcolumn}{\marg{col}}]\mbox{}
\Item[\ENV{nthcolumn*}{\marg{col}\oarg{text}}]\mbox{}\par
\columnratio{0.55}
\begin{paracol}{2}
The environment \env{nthcolumn} contains \meta{body} for the column
\meta{col}.  The starred version \env{nthcolumn*} does the same after
\sync{}ation and, if specified, the optional spanning \meta{text} is put.
\switchcolumn
环境\env{nthcolumn}包含了第\meta{col}列的\meta{body}。星号版本\env{nthcolumn*}在\sync{}ation之后执行相同的操作,并且如果指定了可选的跨列\meta{text},则会放置它。
\Index{spanning text}
\end{paracol}

\begin{itemize}
\columnratio{0.55}
\begin{paracol}{2}
\item
The environments are equivalent to;
\begin{quote}
|{|\!\switchcolumn!\oarg{col}\quad\meta{body}\quad\CSIndex{par}|}|\\
|{|\!\switchcolumn!\oarg{col}|*|\oarg{text}\quad
    \meta{body}\quad\CSIndex{par}|}|
\end{quote}
\switchcolumn\item 
这些环境等同于:
\begin{quote}
|{|\!\switchcolumn!\oarg{col}\quad\meta{body}\quad\CSIndex{par}|}|\\
|{|\!\switchcolumn!\oarg{col}|*|\oarg{text}\quad
    \meta{body}\quad\CSIndex{par}|}|
\end{quote}
\switchcolumn[0]*
\item
The \meta{body} of the environments cannot have \!\switchcolumn! nor
\csenv{}s including \env{nthcolumn}(|*|) themselves, or you will have an
error message of illegal use of command\slash environment.
\switchcolumn
\item
环境的 \meta{body} 不能包含 \!\switchcolumn! 或包括 \env{nthcolumn}(|*|) 在内的 \csenv{},否则会出现非法使用命令\slash 环境的错误消息。
\switchcolumn[0]*
\item
Column-switching\index{column-switching} does not take place at
\!\end!|{|\env{nthcolumn}(|*|)|}|.  Therefore, texts following the
environments are put in the column in which \meta{body} resides until a
\cswitch{} command\slash environment is given.
\switchcolumn\item
列切换\index{column-switching}不会在 \!\end!|{|\env{nthcolumn}(|*|)|}| 处发生。因此,环境后的文本会被放在\meta{body}所在的列中,直到出现\cswitch{}命令\slash 环境为止。
\end{paracol}
\end{itemize}

\KeepSpace{4}
\item[\ENV{leftcolumn}{}]\mbox{}
\Item[\ENV{leftcolumn*}{\oarg{text}}]\mbox{}
\Item[\ENV{rightcolumn}{}]\mbox{}
\Item[\ENV{rightcolumn*}{\oarg{text}}]\mbox{}\par\nobreak
\columnratio{0.55}
\begin{paracol}{2}
The environment \env{leftcolumn} contains \meta{body} for the leftmost
column 0, while \env{rightcolumn} for the column 1 being the right column
in two-column typesetting.  The starred versions \env{leftcolumn*} and
\env{rightcolumn*} do the same after \sync{}ation and, if specified, the
optional spanning \meta{text} is put.
\switchcolumn
环境\env{leftcolumn}包含了最左侧列0的\meta{body},而\env{rightcolumn} 包含了在双栏排版中作为右侧列的列1的\meta{body}。星号版本\env{leftcolumn*}和\env{rightcolumn*}在\sync{}ation之后执行相同的操作,并且如果指定了可选的跨列\meta{text},则会放置它。
\Index{spanning text}
\end{paracol}


\begin{itemize}
\columnratio{0.55}
\begin{paracol}{2}
\item
The environments \env{leftcolumn}(|*|) are equivalent to;
\begin{quote}
\Env{nthcolumn}{\Arg{\texttt{0}}}\\
\Env{nthcolumn*}{\Arg{\texttt{0}}\oarg{text}}
\end{quote}
while \env{rightcolumn}(|*|) are equivalent to;
\begin{quote}
\Env{nthcolumn}{\Arg{\texttt{1}}}\\
\Env{nthcolumn*}{\Arg{\texttt{1}}\oarg{text}}
\end{quote}
\switchcolumn
\item
环境\env{leftcolumn}(|*|)等同于:
\begin{quote}
\Env{nthcolumn}{\Arg{\texttt{0}}}\\
\Env{nthcolumn*}{\Arg{\texttt{0}}\oarg{text}}
\end{quote}
而\env{rightcolumn}(|*|)等价于:
\begin{quote}
\Env{nthcolumn}{\Arg{\texttt{1}}}\\
\Env{nthcolumn*}{\Arg{\texttt{1}}\oarg{text}}
\end{quote}
\end{paracol}
\end{itemize}


\item[\Midx{\!\thecolumn!}]\mbox{}\par
\columnratio{0.55}
\begin{paracol}{2}
The command gives you the zero-origin ordinal of the column in which this
command appears.  Therefore, the following code snip;
\switchcolumn
该命令给出了此命令出现的列的零起始序号。因此,以下代码片段:
\end{paracol}

\begin{itemize}\item[]
|\begin{paracol}{3}|\\
|Column-\thecolumn.\switchcolumn|
|Column-\thecolumn.\switchcolumn|
|Column-\thecolumn.|\\
|\end{paracol}|
\end{itemize}
\columnratio{0.55}
\begin{paracol}{2}
gives us the followings.
\switchcolumn
我们得到了以下结果。
\end{paracol}
% \par\medskip
\begin{paracol}{3}
Column-\thecolumn.\switchcolumn
Column-\thecolumn.\switchcolumn
Column-\thecolumn.
\end{paracol}

\begin{itemize}
\columnratio{0.55}
\begin{paracol}{2}
\item
The command is {\em neither} a \LaTeX's counter nor \!\count! register of
native \TeX{}, and thus the value it keeps cannot be modified.  However,
it can be used wherever an integer number is required or appropriate.
Therefore for example, \!\setcounter!|{mycounter}{|\!\thecolumn!|}| works
well to give the column ordinal to the counter |mycounter|.
\switchcolumn\item
该命令既不是 \LaTeX 的计数器,也不是原生 \TeX 的 \!\count! 寄存器,因此它所保存的值无法修改。然而,它可以在需要或适当的地方使用整数值。因此,例如,将列序数赋给计数器 |mycounter|只要:\!\setcounter!|{mycounter}{|\!\thecolumn!|}|。
\end{paracol}
\end{itemize}

\item[\Midx{\!\definecolumnpreamble!}\marg{col}\marg{pream}]\mbox{}\par
\columnratio{0.55}
\begin{paracol}{2}
The command is to define the {\Uidx\colpream} \meta{pream} for the column
\meta{col}, which is inserted at every \cswitch{} to the column.  More
specifically, the command let \!\switchcolumn! to \meta{col} act as if you
sepcify;
\begin{itemize}\item[]
\!\switchcolumn! $\arg{pream\ for\ col}$
\end{itemize}
\switchcolumn
该命令用于为列 \meta{col} 定义{\Uidx\colpream} \meta{pream},该 \meta{pream} 在每次切换到该列时插入。更具体地说,该命令使得 \!\switchcolumn! 到 \meta{col} 的行为与您指定的一样。
\begin{itemize}\item[]
\!\switchcolumn! $\arg{pream\ for\ col}$
\end{itemize}
\end{paracol}



and \csenv{}s such as \env{nthcolumn} act as if you specify;

而\env{nthcolumn}等\csenv{}则会表现得好像你指定了:

\begin{itemize}\item[]
|\begin{nthcolumn}{|\meta{col}|}| $\arg{pream\ for\ col}$

\end{itemize}

\begin{itemize}
\item
\begingroup\hfuzz1.5pt
The optional \sptext{} of \!\switchcolumn!, \csenv{}s and \beginparacol{}
is considered to be in a virtual column $-1$, and thus if you need a
\Colpream{} for \sptext{}s do \!\definecolumnpreamble!|{-1}|\marg{pream}.

\!\switchcolumn!命令、\csenv{}和\beginparacol{}的可选参数\sptext{}被视为虚拟列$-1$中的内容,因此如果你需要为\sptext{}添加\Colpream{},请使用 \!\definecolumnpreamble!|{-1}|\marg{pream}。
\par\endgroup

\item
The command may appear in a \env{paracol} environment and, if so,
\meta{pream} is effective from the succeeding \cswitch{} to \meta{col}.

该命令可以出现在 \env{paracol} 环境中,如果是这样的话,\meta{pream} 从后续的 \cswitch{} 到 \meta{col} 是有效的。
\item
The definition of \meta{pream} is made globally.

\meta{pream} 的定义是全局的。
\end{itemize}



\item[\Midx{\!\ensurevspace!}\marg{len}]\mbox{}\par
\changes{v1.3-5}{2013/09/17}
{Add description of \cs{ensurevspace}.}

The command tells the first \sync{}ing \cswitch{} command (i.e.,
\!\switchcolumn!\oarg{col}|*|) or environment (i.e., \env{column*}, etc.\@)
following this command that the page must be broken before \sync{}ation
unless the \sync{}ation point has the space of \meta{len} or more below it
in the page.  If a \sync{}ation does not have the command after the
previous \sync{}ation, it is assumed that
\!\ensurevspace!|{|\!\baselineskip!|}| is given.

该命令告诉紧随该命令之后的第一个\sync{}ing \cswitch{}命令(即 \!\switchcolumn!\oarg{col}||)或环境(即\env{column}等),除非页面中\sync{}ation点下方有\meta{len}或更多的空间,否则页面必须在\sync{}ation前被分页。如果前一个\sync{}ation之后没有该命令,则假定已给出\!\ensurevspace!|{|\!\baselineskip!|}|。

\begin{itemize}
\item
This command is to be used when a \sync{}ation point would be placed near
the bottom of a page $p$ and the space below it is not sufficient for a column
$c$ to put anything in the page, while another column $c'$ can have a few
lines in the page.  If this happens, the first line after the \sync{}ation
should start at the top of the page $p+1$ in the column $c$, while that of
$c'$ is still in the page $p$, giving you an impression that the
\sync{}ation fails to align the top of all columns below it.  The fact is,
however, the \sync{}ation point is properly established near at the bottom
of the page but the first line of $c$ needs some large space due to, for
example, the followings.

当\sync{}ation点位于页面$p$的底部附近,并且其下方的空间不足以容纳列$c$中的内容,而另一列$c'$可以在页面$p$中有几行时,应使用此命令。如果发生这种情况,则\sync{}ation后的第一行应从页面$p+1$的列$c$顶部开始,而$c'$的第一行仍在页面$p$中,给您一种印象,即\sync{}ation无法使所有列的顶部对齐。然而,事实是,\sync{}ation点确实正确地建立在页面底部附近,但由于某些原因,例如以下原因,列$c$的第一行需要一些较大的空间。

\begin{itemize}
\item
The line has unusually tall stuff including larger font letters.

该行包含异常高的内容,包括较大字号的字母。
\item
The line has a footnote reference which is hardly apart from the
footnote, and thus the line and the footnote go to the next page together.

该行有一个脚注引用,与脚注之间几乎没有间隔,因此该行和脚注一起跳转到下一页。
\item
The parameter \!\clubpenalty! is too large (e.g., 10000) to break the
first and second lines into separate pages.

参数 \!\clubpenalty! 太大(例如10000),导致第一行和第二行无法分开分页。
\item
The first line follows a vertical space.

第一行后面有一个垂直间距。
\end{itemize}

\item
This manual itself has some instances of \!\ensurevspace! command in the
page \pageref{page:bfreude} and \pageref{page:efreude} in which each German
stanza is enclosed in \env{verse} and then \env{leftcolumn*} environments
and has \!\ensurevspace!|{2|\!\baselineskip!|}| before the \!\begin!ing of
the outer \env{leftcolumn*} because the first line of the stanza is
preceded by a vertical space inserted by \!\begin!|{|\env{verse}|}|.  In
fact without \!\ensurevspace!, the first two lines of the sixth English
stanza would be in the page \pageref{page:bfreude}, while corresponding
German stanza go to the next page \pageref{page:efreude} as a whole, due
to the difference of the height of footnotes in each column, i.e., German
ones are taller than English ones to narrow the space for the German
column.

本手册本身在第 \pageref{page:bfreude} 页和第\pageref{page:efreude}页有一些 \!\ensurevspace! 命令的实例,在这些页中,每个德语诗节都被包含在\env{verse}环境和\env{leftcolumn*}环境中,并且在外部\env{leftcolumn*}的 \!\begin!之前有一个 \!\ensurevspace!|{2|\!\baselineskip!|}| ,因为诗节的第一行前面有一个由 \!\begin!|{|\env{verse}|}| 插入的垂直间距。实际上,如果没有 \!\ensurevspace!,第六首英文诗节的前两行将在第\pageref{page:bfreude}页,而相应的德文诗节将作为整体移到下一页\pageref{page:efreude},这是因为每列脚注的高度不同,即德文脚注比英文脚注更高,以缩小德文列的空间。

\item
As the author does in the ``An die Freude/To Joy'' example, it is a good
tactics to have an \!\ensurevspace! with some vertical space larger than the
default \!\baselineskip! if it is sure that a column has a feature shown
above regardless of the position of the \sync{}ation point in question,
because the point goes up or down with revisions of your document and
using an \!\ensurevspace! for a \sync{}ation far above the page bottom is
perfectly harmless.  Similarly, if you find a problem in a \sync{}ation
and add an \!\ensurevspace! to solve it, keeping the command attached is
recommended even when the \sync{}ation point moves up or down to make the
command unnecessary.

正如作者在“An die Freude/To Joy”示例中所做的那样,如果确定某一列具有上述特征,无论问题点的\sync{}ation点位置如何变化,使用比默认
\!\baselineskip! 更大的一些垂直间距的 \!\ensurevspace! 是一个好策略,因为该点随着文档的修订而上下移动,并且在页面底部上方使用
\!\ensurevspace! 是完全无害的。同样,如果在\sync{}ation中发现问题并添加了 \!\ensurevspace! 来解决问题,则建议保留该命令,即使\sync{}ation点上下移动以使命令不再需要。

\end{itemize}
\end{description}

 
%  
 \subsection{用于列和间隔宽度的命令\hfill Commands for Column and Gap Width}
 \label{sec:ref-colwidth}
 
 \begin{description}
 \item[\Midx{\!\columnratio!}\Arg{$r_0,r_1,\cdots,r_k$}
                              {|[|$r'_0,r'_1,\cdots,r'_{k'}$|]|}]\mbox{}\par
\columnratio{0.55}
\begin{paracol}{2}
The command defines the width of each column by the fraction $r_i$ to
specify the portion which $i$-th ($i=0$ for the leftmost) column
occupies.  More specifically, the width $\Midx{\w}_i$ of the $i$-th column
is defined as follows, where $W$ is \!\textwidth!, $S$ is \!\columnsep!,
and $n$ is the number of columns given to \beginparacol.
\begin{eqnarray*}
W'&=&W-(n-1)S\\
w_i&=&\cases{
r_iW'
    \vrule height1.5\ht\strutbox depth1.5\dp\strutbox width0pt&$i\leq k$\cr
\displaystyle{(1-\sum_{j=0}^k r_j)W'\over n-(k+1)}&$i>k$}
\end{eqnarray*}
\switchcolumn
该命令通过分数$r_i$来定义每列的宽度,
以指定第$i$列($i=0$表示最左边的列)所占的比例。
具体而言,第$i$列的宽度$\Midx{\w}_i$定义如下,
其中$W$是 \!\textwidth!,
$S$是 \!\columnsep!,
$n$是传递给 \beginparacol 的列数。
\begin{eqnarray*}
W'&=&W-(n-1)S\\
w_i&=&\cases{
r_iW'
\vrule height1.5\ht\strutbox depth1.5\dp\strutbox width0pt&$i\leq k$\cr
\displaystyle{(1-\sum_{j=0}^k r_j)W'\over n-(k+1)}&$i>k$}
\end{eqnarray*}
\switchcolumn[0]*
 For a \env{paracol} environment with \parapag{}ing, $n$ is replaced with
 $n_l$ for the columns in left \parapag{}es, while $n$ and $w_i$ are
 replaced with $n_r$ and $w_{n_r+i}$ for those in right \parapag{}es.
 Moreover, if the optional argument having $r'_0,r'_1,\cdots,r'_{k'}$ is
 provided, $w_{n_r+i}$ for a column in right \parapag{}es is determined
 by $r'_i$ and $k'$ instead of $r_i$ and $k$.
 \switchcolumn
 对于具有\parapag{} 分页的\env{paracol}环境,对于左侧\parapag{}的列,将$n$替换为$n_l$,而对于右侧\parapag{}的列,将$n$和$w_i$替换为$n_r$和$w_{n_r+i}$。此外,如果提供了具有$r'0,r'1,\cdots,r'{k'}$的可选参数,则右侧\parapag{}的列中的$w{n_r+i}$由$r'_i$和$k'$确定,而不是由$r_i$和$k$确定。
\end{paracol}


\begin{itemize}
\columnratio{0.55}
\begin{paracol}{2}
\item
The equations above imply that $k<n-1$, $r_i>0$ and $\sum_{j=0}^k
r_j<1$.  If $k\geq n-1$, $k$ is assumed to be $n-2$ and all $r_i$ such
that $i\geq n-1$ are ignored.  If $r_i$ or its sum does not satisfy the
conditions, you will have an ugly result with ``Overfull'' messages.
\switchcolumn\item
上述方程表明$k<n-1$,$r_i>0$且$\sum_{j=0}^k r_j<1$。如果$k\geq n-1$,则假设$k$为$n-2$,并忽略所有满足$i\geq n-1$的$r_i$。如果$r_i$或其总和不满足条件,你将得到一个带有“Overfull”消息的不美观的结果。
\switchcolumn[0]*
\item
The argument $r_0,r_1,\cdots,r_k$ can be empty to mean $k=-1$ to let all
column widths be $W'/n$ as default.
\switchcolumn\item
参数$r_0,r_1,\cdots,r_k$可以为空,表示$k=-1$,使得所有列宽默认为$W'/n$。
\switchcolumn[0]*
\item
The setting of column width by the command takes effect in the |paracol|
environments following the command\footnote{%
If the command is in a \texttt{paracol} environment, the command does not
affect the column widths of the environment but does the next ones, though
such usage is very unusual.}.
\switchcolumn\item
该命令设置的列宽度在命令后的 |paracol| 环境中生效\footnote{如果该命令在 \texttt{paracol} 环境中,该命令不会影响环境的列宽度,而是影响后续的列宽度,尽管这种用法非常不常见。}。
\switchcolumn[0]*
Therefore, though placing the command in the preamble is the most natural
way\footnote{%
Or second most to not using it at all, of course.},
\switchcolumn
因此,将该命令放在导言区是最自然的方式\footnote{%
当然,第二自然的方式是不使用它。}。
\switchcolumn[0]*
you may place this command between two |paracol| environments to change
the column layout for the second one even when they appear in a page as
shown in Section~\ref{sec:man-close}.
\switchcolumn
在两个|paracol|环境之间放置此命令,即可更改第二个环境的列布局,即使它们在页面中出现,如第~\ref{sec:man-close}节所示。
%%%%%%%%
\switchcolumn[0]*
 \item
 In the $i$-th column, \!\columnwidth! has $w_i$ and, for outermost
 paragraphs in the column, \!\hsize! has $w_i$ as well.  As for
 \!\linewidth!, it has $w_i-(\!\textwidth!-l)$ where $l$ is what
 \!\linewidth! had at \beginparacol{}, i.e., the \!\linewidth! for the
 \env{list}-like environment surrounding \env{paracol} if any, or
 \!\textwidth! otherwise.
 \switchcolumn\item
 在第$i$列中,\!\columnwidth! 的值为$w_i$,对于列中的最外层段落,\!\hsize!的值也为$w_i$。至于 \!\linewidth!,它的值为 $w_i-(\!\textwidth!-l)$,其中$l$是在\beginparacol{}中 \!\linewidth! 所具有的值,即如果有的话,是包围\env{paracol}的\env{list}-like环境的 \!\linewidth!,否则是 \!\textwidth!。
 \switchcolumn[0]*
 \item
 You can specify width of each column and that of each {\em gap} between
 two columns more detailedly by \!\setcolumnwidth! shown below.  If your
 document has both of \!\columnratio! and \!\setcolumnwidth! prior to a
 \env{paracol} environment, the command given later is effective for the
 environment.
 \switchcolumn\item
 您可以通过下面的 \!\setcolumnwidth! 更详细地指定每列的宽度和每两列之间的{\em 间隙}的宽度。如果在\env{paracol}环境之前的文档中同时存在 \!\columnratio! 和 \!\setcolumnwidth!,则后给出的命令对该环境有效。
\end{paracol}
\end{itemize}
 
 
\item[\Midx{\!\setcolumnwidth!}\Arg{$s_0,s_1,\cdots,s_k$}
                              {|[|$s'_0,s'_1,\cdots,s'_{k'}$|]|}]\mbox{}\par
\columnratio{0.55}
\begin{paracol}{2}
The command defines the width of each column and that of each {\em gap}
between two columns by the column/gap specification $s_i$ for the $i$-th
column and the gap between it and the $(i{+}1)$-th column.  More
specifically, $s_i$ has the form of $\hat w_i$ or $\hat w_i\,|/|\,\hat g_i$
where each of $\hat w_i$ and $\hat g_i$ is a proper glue including a
proper dimension, or an empty string to mean $\hat w_i=\!\fill!$ and $\hat
g_i=\!\columnsep!$, to determine the width of $i$-th column $\w_i$ and that
of $i$-th gap $\Midx{\gap}_i$ as follows, where $\mathit{nat}(x)$ is the
natural width of the glue $x$, $\mathit{str}(x)$ is the infinite stretch
factor of $x$, $W$ is \!\textwidth!, and $n$ is the number of columns
given to \beginparacol.

\begin{eqnarray*}
W'&=&\sum_{i=0}^{n-2}\big(\mathit{nat}(\hat w_i)+\mathit{nat}(\hat g_i)\big)+
\mathit{nat}(\hat w_{n-1})\\
F&=&\sum_{i=0}^{n-2}\big(\mathit{str}(\hat g_i)+\mathit{str}(\hat g_i)\big)+
\mathit{str}(\hat w_{n-1})\\
x_i&=&\cases{(W/W')\mathit{nat}(\hat x_i)&$W'\geq W\;\lor\;F\leq0$\cr
\mathit{nat}(\hat x_i)+(\mathit{str}(\hat x_i)/F)(W-W')&
$W'< W\;\land\;F>0$}
\qquad(x\in\{w,g\})
\end{eqnarray*}
\switchcolumn
该命令通过列/间隔规范$s_i$定义每个列和每个{\em 间隔}的宽度,其中$s_i$是第$i$列和它与$(i{+}1)$列之间的间隔。
具体来说,$s_i$的形式为$\hat w_i$或$\hat w_i,|/|,\hat g_i$,其中$\hat w_i$和$\hat g_i$都是包含适当尺寸的适当粘连,
或者是一个空字符串来表示 $\hat w_i=\!\fill!$ 和$\hat g_i=\!\columnsep!$,以确定第$i$列$\w_i$和第$i$个间隔$\Midx{\gap}_i$的宽度,
其中$\mathit{nat}(x)$是粘连$x$的自然宽度,$\mathit{str}(x)$是$x$的无限伸展因子,$W$是 \!\textwidth!,$n$是传递给\beginparacol 的列数。

\begin{eqnarray*}
W'&=&\sum_{i=0}^{n-2}\big(\mathit{nat}(\hat w_i)+\mathit{nat}(\hat g_i)\big)+
\mathit{nat}(\hat w_{n-1})\\
F&=&\sum_{i=0}^{n-2}\big(\mathit{str}(\hat g_i)+\mathit{str}(\hat g_i)\big)+
\mathit{str}(\hat w_{n-1})\\
x_i&=&\cases{(W/W')\mathit{nat}(\hat x_i)&$W'\geq W\;\lor\;F\leq0$\cr
\mathit{nat}(\hat x_i)+(\mathit{str}(\hat x_i)/F)(W-W')&
$W'< W\;\land\;F>0$}
\qquad(x\in\{w,g\})
\end{eqnarray*}
\end{paracol}

 That is, if the total of natural widths $W'$ is larger than \!\textwidth!
 $W$ or there are no infinite stretch factors in the specification, given
 widths are scaled down or up so that the scaled total is equal to $W$.
 Otherwise, each width with an infinite stretch factor is extended
 according to its ratio in the total stretch so that the stretched total is
 equal to $W$.

 也就是说,如果自然宽度的总和$W'$大于 \!\textwidth! $W$,或者规范中没有无限伸展因子,给定的宽度将被缩小或放大,使得缩放后的总和等于$W$。否则,每个具有无限伸展因子的宽度将根据其在总伸展中的比例进行扩展,以使伸展后的总和等于$W$。
 
 For a \env{paracol} environment with \parapag{}ing, $n$ is replaced with
 $n_l$ for the columns in left \parapag{}es, while $n$, $w_i$ and $g_i$ are
 replaced with $n_r$, $w_{n_r+i}$ and $g_{n_r+i}$ for those in right
 \parapag{}es.  Moreover, if the optional argument having
 $s'_0,s'_1,\cdots,s'_{k'}$ is provided, $w_{n_r+i}$ and $g_{n_r+i}$ for a
 column in right \parapag{}es are determined by $s'_i$ instead of $s_i$.

 对于具有\parapag{}分页的\env{paracol}环境,对于左侧\parapag{}的列,将$n$替换为$n_l$,而对于右侧\parapag{}的列,将$n$,$w_i$和$g_i$分别替换为$n_r$,$w_{n_r+i}$和$g_{n_r+i}$。此外,如果提供了具有$s'0,s'1,\cdots,s'{k'}$的可选参数,则右侧\parapag{}的列中的$w{n_r+i}$和$g_{n_r+i}$由$s'_i$确定,而不是由$s_i$确定。
 \begin{itemize}
 \item
 In \env{paracol} environments having $n$ columns, $s_i$ s.t.\ $i\geq n$
 and $\hat g_{n-1}$ are ignored.  On the other hand if $k<n-1$, it is
 assumed $s_i$ is an empty string for all $i>k$.

 在具有$n$列的\env{paracol}环境中,忽略满足$i\geq n$和$\hat g_{n-1}$的$s_i$。另一方面,如果$k<n-1$,则假设对于所有$i>k$,$s_i$都是一个空字符串。
 
 \item
 Finite stretch factors and finite or infinite shrink factors in $\hat w_i$
 and $\hat g_i$ are ignored.
 
 在$\hat w_i$和$\hat g_i$中,有限的拉伸因子和有限或无限的收缩因子被忽略。
 \item
 Unlike \TeX's genuine glue addition, all infinite unit |fil|, |fill| and
 |filll| are not distinguished in the summation for $F$.  Also unlike
 \TeX's genuine scaling of a glue primitive, 
 $f\!\fill!$ means $0\,|pt|\ |plus|\ f\,|fill|$ for convenience\footnote{
 
 In \TeX's grammar, $f\!\fill!$ means a dimension rather than a glue and is
 $0\,|pt|$ because the natural component of \!\fill! is 0.}.

 与 \TeX 的真正粘连添加不同,所有无限单位的 |fil|、|fill| 和 |filll| 在 $F$ 的求和中没有区别。另外,与 \TeX 的真正粘连原语的缩放不同, $f\!\fill!$ 表示为 $0,|pt|\ |plus|\ f,|fill|$,以方便使用\footnote{在 \TeX 的语法中,$f\!\fill!$ 表示的是一个尺寸而不是粘连,并且是 $0,|pt|$,因为 \!\fill! 的自然分量为 0。}。
 
 \item
 The division $W/W'$ and $\mathit{str}(\hat x_i)/F$ can have some
 arithmetic errors and thus the total of $w_i$ and $g_i$ may not be equal to
 $W$ exactly but can be a little bit less than $W$.  This small error is,
 however, equally distributed to $g_i$ in typesetting of a page to make the
 total width of columns and gaps is exactly $W$\footnote{
 
 If we may ignore the arithmetic error inherent in \TeX.}.

 除法$W/W'$和$\mathit{str}(\hat x_i)/F$可能存在一些算术误差,因此$w_i$和$g_i$的总和可能不完全等于$W$,而可能略小于$W$。然而,在页面排版中,这个小的误差被等分给$g_i$,以确保列和间隙的总宽度恰好为$W$\footnote{
 
 如果我们可以忽略\TeX 中固有的算术误差。}。
 
 \item
 All the specifications shown in the table below give us same results for a
 \env{paracol} environment having three columns, providing
 $\!\textwidth!=360\,|pt|$ and $\!\columnsep!=S=20\,|pt|$.

 下表中显示的所有规格都可以得到相同的结果,适用于具有三列的\env{paracol}环境,其中 $\!\textwidth!=360,|pt|$和$\!\columnsep!=S=20,|pt|$。
 
 \par\hbox to\textwidth\bgroup\hfil
 \nosv \def\|{\verb|}\small\arraycolsep0pt\def\arraystretch{1.1}
 $\begin{array}[b]{l|ccccc}
 s_0,s_1,s_2&w_0&g_0&w_1&g_1&w_2\rlap{ (in \texttt{pt})}\\\hline
 \|50pt/20pt,100pt/40pt,150pt|&50&20&100&40&150\\
 \|50pt,100pt/2\columnsep,150pt|&50&S&
                                100&2S&150\\
 \|50pt/\fill,100pt/2\fill,150pt|&50&(1/3)\cdot60&100&(2/3)\cdot60&150\\
 \|,2\fill/2\columnsep,3\fill|&\ (1/6)\cdot300&S&
                              (2/6)\cdot300&2S&
                              (3/6)\cdot300\\
 \|50pt/20,50pt plus 1fil/40pt,50pt plus 2fil |&
                              50&20&50+(1/3)\cdot150&40&
                              50+(2/3)\cdot150\\
 \|5pt/2pt,10pt/4pt,15pt|&10\cdot5&10\cdot2&10\cdot10&10\cdot4&
                         10\cdot15\\
 \|100pt/40pt,200pt/80pt,300pt|&0.5\cdot100&0.5\cdot40&
                               0.5\cdot200&0.5\cdot80&
                               0.5\cdot300
 \end{array}$\hfil\egroup
 
 \item
 If your document has both of \!\columnratio! and \!\setcolumnwidth! prior
 to a \env{paracol} environment, the command given later is effective for
 the environment.

如果在\env{paracol}环境之前的文档中同时存在 \!\columnratio! 和 \!\setcolumnwidth!,则后面给出的命令对该环境有效。
 \end{itemize}
 \end{description}
 

% 
\subsection{用于双面排版和边注的放置的命令\hfill Commands for Two-Sided Typesetting and Marginal Note Placement}
\label{sec:ref-twoside}

\begin{description}
\item[\Midx{\!\twosided!}{$|[|t_1t_2\cdots t_k|]|$}]\mbox{}\par
\columnratio{0.55}
\begin{paracol}{2}
The command enables a set of two-sided typesetting features
$\Set{t_i}{t_i\in\{|p|,|c|,|m|,|b|\},\ 1\leq i\leq k}$ explicitly by the
optional argument, or all of the following four features as a whole
without the argument, in even-numbered pages.
\switchcolumn
该命令通过可选参数显式地启用一组双面排版功能$\Set{t_i}{t_i\in{|p|,|c|,|m|,|b|},\ 1\leq i\leq k}$,或者在偶数页上作为一个整体启用以下四个功能,而无需参数。    
\end{paracol}
\begin{description}
\columnratio{0.55}
\begin{paracol}{2}
\item[|p|\rm(\textit{age})]
for ordinary two-sided paging, letting the left side margin be
\!\evensidemargin!, page headers be different from those in odd-numbered
pages with |headings| or |myheadings| page style, and \!\cleardoublepage!
leave an even-numbered page blank if it is used in an odd-numbered page.
\switchcolumn
\item[|p|\rm(\textit{age})]
对于普通的双面分页,左侧边距为 \!\evensidemargin!,页面页眉与奇数页中的 |headings| 或 |myheadings| 页面样式不同,并且 \!\cleardoublepage! 在奇数页中使用时会使偶数页保持空白。
\switchcolumn[0]*
\item[|c|\rm(\textit{olumn})]
for {\em\Uidx\cswap} to \emph{print} columns in even-numbered pages in
reverse order.  This feature is sometimes preferable in typesetting
especially with unbalanced parallel columns to make, for example, a wider
columns are always \emph{inside} while narrower ones are \emph{outside}.
\switchcolumn
\item[|c|\rm(\textit{olumn})]
对于{\em\Uidx\cswap}来在偶数页上以相反的顺序\emph{打印}列。这个功能在排版中有时是可取的,特别是在不平衡的并列列中,可以使较宽的列始终位于\emph{内部},而较窄的列位于\emph{外部}。
\switchcolumn[0]*
\item[|m|\rm(\textit{arginal text})]
to place marginal notes in the side margin opposite to that specified by
the command \!\marginparthreshold! discussed shortly.
\switchcolumn
\item[|m|\rm(\textit{arginal text})]
将边注放置在与命令 \!\marginparthreshold! 指定的相反侧边缘中(稍后会讨论)。
\switchcolumn[0]*
\item[|b|\rm(\textit{ackground painting})]
to make \bgpaint, shown in Section~\ref{sec:ref-bgpaint},
\emph{\mirror{}ed} so that, for example, a color specified for the left
margin is used to paint the right margin instead.
\switchcolumn\item[|b|\rm(\textit{ackground painting})]
为了使\bgpaint(参见第~\ref{sec:ref-bgpaint}节)是\emph{\mirror{}ed}的,例如,为左边距指定的颜色将用于绘制右边距。
\end{paracol}
\end{description}

\begin{itemize}
\columnratio{0.55}
\begin{paracol}{2}
\item
The feature |p| is also enabled by the |twoside| option of
\!\documentclass! with almost all classes including |article|, |book|,
|report|, etc.  Though it is strongly recommended to make both settings by
\!\documentclass! and this command consistent, they can be inconsistent
resulting in lack of some expected functions.  For example, enabling |p|
feature by \!\twosided! without |twoside| option in \!\documentclass!
makes the format of headers and footers in all pages same even with
\!\pagestyle!|{headings}|.
\switchcolumn\item
|p|特性也可以通过 \!\documentclass! 的 |twoside| 选项启用,几乎适用于包括 |article|、|book|、|report| 等在内的所有类。虽然强烈建议通过 \!\documentclass! 和此命令使两个设置保持一致,但它们可能不一致,导致缺少某些期望的功能。例如,通过在 \!\documentclass! 中启用 |twoside| 选项而不使用 \!\twosided!,会使所有页面上的页眉和页脚的格式相同,即使使用了 \!\pagestyle!|{headings}|。
\switchcolumn[0]*
\item
The \cswap{} enabled by the feature |c| is ineffective in \npaired{}
\parapag{}ing because it is meaningless\footnote{%
Unless somebody tells the author it is meaningful.},
and thus silently ignored.
\switchcolumn\item
在\npaired{}\parapag{}ing中,由特性 |c| 启用的\cswap{}是无效的,因为它是没有意义的\footnote{除非有人告诉作者它是有意义的。},因此会被悄悄地忽略。
\switchcolumn[0]*
\item
In ordinary single-column typesetting, marginal note swapping in
even-numbered pages is enabled by the |twoside| option, while it never takes
place in ordinary two-column typesetting.  For marginal notes given in
\env{paracol} environments, however, swapping of them in
even-numbered pages is enabled by giving the feature |m| to \!\twosided!.
\switchcolumn\item
在普通的单栏排版中,通过|twoside|选项启用了在偶数页中交换边注的功能,而在普通的双栏排版中则不会出现这种情况。然而,对于在\env{paracol} 环境中给出的边注,可以通过给予 \!\twosided! 功能特性 |m| 来在偶数页中启用它们的交换。
\switchcolumn[0]*
\item\label{page:cswap}
The command has to be outside of \env{paracol} environments to decide the
action in the environments following them.  If it appears in a
\env{paracol} environment, you will have a warning message saying it is
ignored.
\switchcolumn\item
该命令必须位于\env{paracol} 环境之外,以决定其后环境中的操作。如果它出现在\env{paracol} 环境中,您将收到一个警告消息,指示它被忽略。
\end{paracol}    


\twosided[c]\columnratio{0.55}\columnsep0pt
\begin{Verbatim}
\twosided[c]\columnratio{0.55}\columnsep0pt
\end{Verbatim}
\begin{paracol}{2}
\hfuzz2pt
\item
Here is an example of column swapping.  Since this page
\pageref{page:cswap} is odd, this wider column-0 with roman font is placed
in left side and thus inside at the begining, but now we are in an even
page in which this column is in right side.
\switchcolumn
\item
这是一个列交换的示例。由于此页\pageref{page:cswap}是奇数页,因此带有罗马字体的较宽的列-0被放置在左侧,因此在开始时位于内部,但现在我们处于一个偶数页,此列位于右侧。
\switchcolumn
\item\it
This narrower, outside and italicized column-1 is at first in right
side but the page break has changed the position to the left.

\switchcolumn\item
这个较窄、位于外侧并且斜体的列1最初在右侧,但页面断页导致其位置改变到左侧。
\switchcolumn
\item
\changes{v1.2-4}{2013/05/11}
{Add description of \cs{[no]swapcolumninevenpages}.}
\changes{v1.3-5}{2013/09/17}
{Remove description of \cs{[no]swapcolumninevenpages} but mention
    they are still available.}

In old versions of \Paracol, namely 1.2 and its minor revisions 1.2x,
\cswap{} was controlled by lengthy commmands
\Midx{\!\swapcolumninevenpages!} and \Midx{\!\noswapcolumninevenpages!}.
Though they are still available and will be so forever for backward
compatibility, it is recommended to use \!\twosided! with or without the
feature |c|.  The old versions also have a problem that \spanning{}
crossing a page boundary is placed incorrectly after the page break in it,
but this problem is solved by a fix incorporated in version 1.3.
\switchcolumn\item
在旧版本的 \Paracol 中,即1.2版本及其小的修订版本1.2x中,\cswap{}通过冗长的命令\Midx{\!\swapcolumninevenpages!}和\Midx{\!\noswapcolumninevenpages!}进行控制。尽管它们仍然可用,并且将永远用于向后兼容性,但建议使用带有或不带有特性|c|的 \!\twosided!。旧版本还存在一个问题,即跨页的\spanning{}在页面断页后放置不正确,但这个问题在1.3版本中通过修复得到解决。
\switchcolumn
\item
It must be $t_i\in\{|p|,|c|,|m|,|b|\}$, or you will have an error message
of illegal two-siding feature.
\switchcolumn\item
必须是$t_i\in{|p|,|c|,|m|,|b|}$,否则会出现非法双面特性的错误消息。
\switchcolumn
\item
Section~\ref{sec:ppts} shows examples of two-sided typesetting together
with related issues on \parapag{}ing.
\switchcolumn\item
第~\ref{sec:ppts}节展示了双面排版的示例,以及与\parapag{}分页相关的问题。
\end{paracol}
\end{itemize}


\item[\Midx{\!\marginparthreshold!}$\Arg{k}{|[|k'|]|}$]\mbox{}\par
\columnratio{0.55}
\begin{paracol}{2}
The command specifies the minimum ordinal $k$ of columns whose marginal
notes are placed in right margin.  That is, marginal notes given in a
column-$i$ go to left margin if $i<k$, while they go to right if $i\geq
k$.  The optional argument $k'$, if given, is for columns in right
\parapag{}es to decide the margin where their marginal notes are placed.
In default, $k=1$ is assumed to let marginal notes from the leftmost
column-0 go to left margin while those from other columns go to right.
\switchcolumn
该命令指定了边注放置在右边页边距中的最小列序数$k$。也就是说,在列$i$中给出的边注如果$i<k$,则放置在左边页边距中,而如果$i\geq k$,则放置在右边页边距中。如果给定可选参数$k'$,则用于决定右边\parapag{}es中的列的边注放置在哪个页边距。默认情况下,假设$k=1$,左边最左列-0的边注放置在左边页边距中,而其他列的边注放置在右边页边距中。
\end{paracol}

\begin{itemize}
\columnratio{0.55}
\begin{paracol}{2}
\item
You may specify $k=0$ to let all marginal notes go to right margin, or may
give the command a large number, say 100, to place all of them in left
margin.
\switchcolumn\item
您可以将$k$指定为0,使所有边注都放在右侧边距,或者可以给命令一个较大的数,比如100,将它们全部放在左侧边距。
\switchcolumn[0]*
\item
The setting $k=0$ or $k=100$ above makes a side margin \emph{shared} by
marginal notes from different columns, and sharing is inevitable when a
(parallel-) page has three or more columns.  When a margin is shared by
marginal notes from two or more columns, it can happen that two marginal
notes from different columns conflict over the space to be occupied by each
of them.  This conflict is solved by \Paracol{} to push down the note
given later in your source |.tex| until an available space for it is
found.  Note that the marginal note to be pushed down is determined by the
position in the source rather than that in the printed result.  Also note
that \Paracol{} exploits space between two marginal notes having been
already placed in the placement of other note coming later to place it at
the natural position if possible or to minimize the amount of pushing down
otherwise.
\switchcolumn\item
上述设置$k=0$或$k=100$使得边注从不同的列共享一个侧边距,当一个(并列)页面有三个或更多列时,共享是不可避免的。当一个侧边距被来自两个或更多列的边注共享时,可能会发生两个来自不同列的边注在它们各自要占据的空间上发生冲突的情况。这个冲突通过\Paracol{}来解决,它会将后面给出的边注推到更低的位置,直到找到一个可用的空间为止。请注意,要被推到下方的边注是由源代码中的位置决定的,而不是打印结果中的位置。同时,请注意\Paracol{}利用已经放置的两个边注之间的空间,在后面的边注放置时尽可能地在自然位置上放置,或者尽量减少推下的量。
\switchcolumn[0]*
\item
In the decision of the real margin in which a marginal note is placed,
other two factors are involved;  |m| feature of \!\twosided! command and
the parity of the page; and \LaTeX's genuine command \!\reversemarginpar!.
More specifically, after the first preliminary decision is made according
to the threshold given to \!\marginparthreshold!, we have the following
two steps to modify the decision;  if |m| feature has been specified in
\!\twosided! command and the marginal note belongs to an even-numbered
page, the decision is reversed to have the second preliminary result;  and
then if \!\reversemarginpar! has been specified, the second result is
reversed (again) to have the final result.
\switchcolumn\item
在确定边注放置的实际边距时,还涉及其他两个因素:\!\twosided! 命令的 |m| 特性和页面的奇偶性;以及\LaTeX 的原始命令 \!\reversemarginpar!。具体而言,在根据 \!\marginparthreshold! 给定的阈值做出第一次初步决策后,我们有以下两个步骤来修改决策;如果 \!\twosided! 命令中指定了 |m| 特性,并且边注属于偶数页,决策将被反转为得到第二次初步结果;然后,如果指定了 \!\reversemarginpar!,第二个结果将被(再次)反转为得到最终结果。
\switchcolumn[0]*
\item
In old versions of \Paracol, namely older than 1.3, marginal note
placement was not only uncontrollable but also gave ugly results when your
document has three or more columns because the marginal notes from a column
not being leftmost or rightmost were placed in the gap following the
column rather than a margin.  This miserable {\em gap note} placement does
not happen any more, or in other words this is no more available because
the author believes nobody loves it.
\switchcolumn\item
在旧版本的\Paracol 中(即1.3之前的版本),边注的放置不仅无法控制,而且在文档具有三列或更多列时会产生丑陋的结果,因为不在最左侧或最右侧的列的边注会放置在列后的间隙中,而不是边距中。这种痛苦的{\em 间隙边注}放置不再发生,换句话说,不再可用,因为作者认为没有人喜欢它。
\switchcolumn[0]*
\item
Section~\ref{sec:ppts} shows examples of marginal note placement together
with related issues on \parapag{}ing and two-sided typesetting.
\switchcolumn\item
第~\ref{sec:ppts}节展示了边注放置的示例,以及与\parapag{}ing和双面排版相关的问题。
\end{paracol}
\end{itemize}


\item[\Midx{\!\marginnote!}\oarg{left}\marg{right}\oarg{voffset}]\mbox{}\par
\columnratio{0.55}
\begin{paracol}{2}
You may use the package \textsf{marginnote} and its command \!\marginnote!
in \env{paracol} environments as a replacement of \!\marginpar!.  However,
the command is \emph{emulated} with \!\marginpar! and \textsf{paracol}'s
own mechanism of marginal note placement.  Therefore, some of
\textsf{marginnote}'s functionality are not effective in \env{paracol}
environment except for the following features.
\switchcolumn
您可以在 \env{paracol} 环境中使用 \textsf{marginnote} 宏包及其命令 \!\marginnote! 作为 \!\marginpar! 的替代。然而,该命令是通过 \!\marginpar! 和 \textsf{paracol} 自身的边注放置机制进行\emph{模拟}的。因此,在 \env{paracol} 环境中,除了以下功能外,一些 \textsf{marginnote} 的功能是不起作用的。
\end{paracol}

\begin{itemize}
\columnratio{0.55}
\begin{paracol}{2}
\item
Shifting up/down a marginal note by the optional \meta{voffset}.
\switchcolumn\item
通过可选参数\meta{voffset}将边注上下移动。
\switchcolumn[0]*
\item
Defining fonts (and others) for marginal notes by \!\marginfont!.
\switchcolumn\item
通过 \!\marginfont! 为边注定义字体(和其他样式)。
\switchcolumn[0]*\item
Controlling the holizontal paragraph alignment by \!\raggedleftmarginnote!
and |\raggedright|\~|marginnote|\SpecialIndex{\raggedrightmarginnote}.
\switchcolumn\item
通过 \!\raggedleftmarginnote! 和|\raggedright|~|marginnote| 控制水平段落对齐方式。
\end{paracol}
\end{itemize}

\columnratio{0.55}
\begin{paracol}{2}
Note that you will see a warning message ``|\margninnote| is emulated by
|\marginpar|'' at the first in-\env{paracol} occurrence of the command to
let you know the imperfection.
\switchcolumn
请注意,在第一次使用该命令的\env{paracol}环境中,您将看到一个警告消息“|\margninnote| is emulated by |\marginpar|”,以便让您知道这种不完美的情况。
\end{paracol}

\end{description}

 
% 
 \subsection{计数器的命令\hfill Commands for Counters}
 \label{sec:ref-counter}
 
\begin{description}
\item[\Midx{\!\globalcounter!}\marg{ctr}]\mbox{}
\Item[\Midx{\!\globalcounter!}\texttt{*}]\mbox{}\par
\columnratio{0.55}
\begin{paracol}{2}
The command \!\globalcounter!\marg{ctr} declares that the counter
\meta{ctr} is global to all columns, while \!\globalcounter!|*| does so
for all counters.  An update of a \Uidx\gcounter{} in a column is seen by
any other columns.
\switchcolumn
命令 \!\globalcounter!\marg{ctr} 声明计数器\meta{ctr}在所有列中是全局的,而 \!\globalcounter!|*| 则对所有计数器都是如此。在某列中更新全局计数器会被其他列看到。 
\end{paracol}

\begin{itemize}
\columnratio{0.55}
\begin{paracol}{2}

\item
All column-local values of a descendant \lcounter{} of a \gcounter{} are
zero-cleared when the \gcounter{} is explicitly stepped by \!\stepcounter!
or \!\refstepcounter!, or implicitly by a sectioning command and so on.
\switchcolumn
当一个全局计数器被 \!\stepcounter! 或 \!\refstepcounter! 显式步进,或者通过节标题命令等隐式步进时,其子孙局部计数器的所有列局部值都会被清零。
\switchcolumn[0]*
\item
The counter \counter{page} is always global but an explicit update of it
by e.g., \!\setcounter! in a non-leftmost column is not seen by other
columns and is canceled even for the column itself after a \cswitch{} or a
page break in the column.  Therefore, if you want to make a \emph{jump} of
\counter{page}, it must be done in the leftmost column 0.  Note that a
jump from a page $p$ to $q$ can be seen in other columns even if they have
gone beyond $p$ \emph{before} the column 0 makes the jump, as far as
\counter{page} having $q$ (or its successor) is referred to by \!\pageref!
or through \emph{contents} files such as |.toc|\footnote{
Direct reference to \counter{page} may give an inconsistent result, as you
might have in ordinary \LaTeX{} documents.}.
\switchcolumn\item
计数器\counter{page}始终是全局的,但是在非最左列中通过 \!\setcounter! 进行的显式更新在其他列中是不可见的,并且在该列进行\cswitch{}或页面断页后,甚至对于该列本身也会被取消。因此,如果要进行\emph{jump}(即跳转)\counter{page},必须在最左列0中进行。请注意,即使其他列在列0进行跳转\emph{之前}已经超过了页面$p$,只要\counter{page}具有$q$(或其后继者)的值,并且通过 \!\pageref! 或通过\emph{contents}文件(如|.toc|)进行引用,其他列仍然可以看到从页面$p$跳转到$q$。\footnote{直接引用 \counter{page} 可能会导致不一致的结果,就像在普通的 \LaTeX{}文档中可能遇到的那样。}
\switchcolumn[0]*
\item
All counters except for \counter{page} are local by default.  This feature
may cause a problem with some packages including \textsf{marginnote} and
\textsf{(auto-)pst-pdf} having their own counters which must be global.
Since it is tough to find the name of such counters from package sources,
if you have something wrong with these (or other) packages, try to put
\!\globalcounter!|*| in your preamble and use \!\localcounter! shown below
to localize specific counters which you need to be local.
\switchcolumn\item
除了\counter{page}计数器外,默认情况下所有计数器都是局部的。这一特性可能会导致一些包(包括\textsf{marginnote}和\textsf{(auto-)pst-pdf})出现问题,这些包具有必须是全局的计数器。由于很难从包的源代码中找到这些计数器的名称,如果您在使用这些(或其他)包时遇到问题,请尝试在导言区中使用 \!\globalcounter!|*| 命令,并使用下面显示的 \!\localcounter! 命令将需要局部化的特定计数器局部化。
\switchcolumn[0]*
\item
Globalizing a \meta{ctr} being already global is just ignored without any
complaints.
\switchcolumn\item
如果一个已经是全局的\meta{ctr}被再次全局化,它会被静默地忽略,而不会有任何警告。
\end{paracol}
 \end{itemize}
 
 
 
 \item[\Midx{\!\localcounter!}\marg{ctr}]\mbox{}\par
 The command declares that the counter \meta{ctr} is local for each column.

这个命令声明计数器\meta{ctr}在每个栏目中都是局部的。
 \begin{itemize}
 \item
 Though this command is intended for localizing a \meta{ctr} which is once
 globalized, localizing a local counter does not causes any error but is
 just ignored.  Localizing the permanently global \counter{page} is also
 just ignored without any complaints.

 尽管该命令旨在将一次全局化的\meta{ctr}局部化,但将局部计数器局部化不会引起任何错误,只是被忽略。将永久全局\counter{page}局部化也只是被忽略,没有任何警告。
 \end{itemize}
 
 \item[\Midx{\!\definethecounter!}\marg{ctr}\marg{col}\marg{rep}]\mbox{}\par
 The command defines |\the|\meta{ctr} being \marg{rep} for the local use in
 the column \meta{col}.  That is, |\the|\meta{ctr} in the column \meta{col}
 acts as if it is defined by
 \!\renewcommand!\Arg{\cs{the}\meta{ctr}}\Arg{\meta{rep}}.

 该命令定义 |\the|\meta{ctr} 作为在列 \meta{col} 中的局部使用,其值为 \marg{rep}。也就是说,在列 \meta{col} 中,|\the|\meta{ctr} 的行为就像是通过 \!\renewcommand!\Arg{\cs{the}\meta{ctr}}\Arg{\meta{rep}} 定义的一样。
 
 
 
 \item[\Midx{\!\synccounter!}\marg{ctr}]\mbox{}\par
 The command \emph{broadcasts} the value of the \lcounter{} \meta{ctr} in
 the column in which the command appears to the values in all other columns.
 
 该命令将出现在的列中的\lcounter{} \meta{ctr}的值向所有其他列中的值进行\emph{broadcasts}(即广播)。
 \item[\Midx{\!\syncallcounters!}]\mbox{}\par
 The command broadcasts the values of all \lcounter{}s in the column in
 which the command appears to the values in all other columns.

该命令将出现在其中的列中的所有\lcounter{}的值广播到所有其他列中的相应值。
 \end{description}
 
% 
% 
% \subsection{Page-Wise Footnotes}
% \label{sec:ref-scfnote}
% \changes{v1.2-2}{2013/05/11}
% 	{Add the sub-section ``Single-Columned Footnotes'' to describe newly
%	 introducerd commands for page-wise footnotes.}
% \changes{v1.3-5}{2013/09/17}
%	{Rename the sub-section title from ``Single-Columned Footnotes'' to
% 	``Page-Wise Footnotes'' following new naming.}
% 
% \begin{description}
% \item[\Midx{\!\footnotelayout!}\marg{layout}]\mbox{}\par
% The command specifies the \meta{layout}${}\in\{|c|,|p|,|m|\}$ of footnotes
% in \env{paracol} environments as follows.

% 该命令指定了在\env{paracol}环境中脚注的 \meta{layout}${}\in\{|c|,|p|,|m|\}$,具体如下。
% \begin{description}
% \item[|c|\rm(\textit{olumn})] makes footnotes {\em\Uidx\mcfnote} (aka
% multi-columned) being default to place footnotes in each column at the
% bottom of the column and separating them from \Preenv{} and \Postenv{}
% footnotes.
% 
使脚注{\em\Uidx\mcfnote}(也称为多列脚注)默认在每列底部放置脚注,并将其与 \Preenv{} 和 \Postenv{} 的脚注分开。
% \item[|p|\rm(\textit{age})] makes footnotes {\em\Uidx\scfnote} (aka
% single-columned) so that footnotes in all columns are gathered, typeset
% spanning all columns, and placed at the bottom of the page in which they
% appear or at the end of the \env{paracol} environment they belong to, so
% that they are separated from \Preenv{} and \Postenv{} footnotes.
% 
将脚注设置为{\em\Uidx\scfnote}(也称为单列脚注),以便将所有列中的脚注聚集在一起,跨越所有列进行排版,并放置在它们所在的页面底部,或者放置在它们所属的\env{paracol}环境的末尾,以便与\Preenv{}和\Postenv{}脚注分开。
% \item[|m|\rm(\textit{erge})] makes \Scfnote{}s {\em\Uidx\mgfnote} with
% footnotes in outside of the environment but in the same page, i.e., those
% in \Preenv{} and \postenv.

在同一页的环境外但在相同页面中,即\Preenv{}和\postenv{}中,使用\Scfnote{}和{\em\Uidx\mgfnote}创建脚注。
% \end{description}
% 
% 
% \begin{itemize}
% \item
% An example of \Mgfnote{} is found in p.~\pageref{sec:ref-paracol} while
% you will see many of them in Section~\ref{sec:fnnp}\footnote{
% 
% The left-column footnote \ref{fn:flush} in p.~\pageref{fn:flush} looks like
% a merged footnote because it is at the bottom of the page and the marked
% text is above the single-column text.  However, it is an ordinary
% \mcfnote{} one produced by a trick with \cs{footnotemark} and
% \cs{footnotetext} in different \env{paracol} environments.}.

在第~\pageref{sec:ref-paracol} 页中可以找到 \Mgfnote{} 的一个示例,而在第~\ref{sec:fnnp} 节中则会看到许多这样的示例\footnote{在第~\pageref{fn:flush} 页的左列脚注 \ref{fn:flush} 看起来像是一个合并的脚注,因为它位于页面底部,而标记的文本位于单列文本之上。然而,它是由在不同的\env{paracol}环境中使用\cs{footnotemark}和\cs{footnotetext}技巧生成的普通\mcfnote{}脚注。}。
% 
% \item
% In any layouts, a footnote cannot have page breaks in it, i.e., a footnote
% is always put in a page as a whole.  This makes it impossible to have a
% footnote taller than \!\textheight! and thus you will see a warning
% message if you give a very long footnote which will be printed intruding
% into the area for page footer (or out of the paper bound).
% 
在任何布局中,脚注不会出现分页,即脚注总是作为一个整体放在一页中。这意味着脚注的高度不可能超过 \!\textheight!,因此如果您给出一个非常长的脚注,它将打印出超出页面页脚区域(或超出纸张边界)的警告消息。
% \item
% Choosing the layout |p|age-wise or |m|erged makes \counter{footnote}
% counter global and \!\fncounteradjustment!  shown below performed inside
% \!\footnotelayout!.  Choosing |c|olumn-wise let the command do the
% operations oppositely, i.e., localizes \counter{footnote} and does
% \!\nofncounteradjustment!.  Though these settings are usually appropriate
% for each footnote layout but you can override them by explicitly using
% commands like \!\localcounter!|{footnote}|.
% 
选择布局为|p|age-wise或|m|erged会使\counter{footnote}计数器变为全局,并在 \!\footnotelayout! 内执行下面的 \!\fncounteradjustment! 操作。选择|c|olumn-wise会使命令执行相反的操作,即将\counter{footnote}局部化并执行 \!\nofncounteradjustment!。虽然这些设置通常适用于每个脚注布局,但您可以通过显式使用 \!\localcounter!|{footnote}| 等命令来覆盖它们。
% \item
% The command has to be outside of \env{paracol} environments to decide the
% action in the environments following them.  If it appears in a
% \env{paracol} environment, you will have a warning message saying it is
% ignored.
% 
% 该命令必须放在\env{paracol}环境之外,以决定其后的环境中的操作。如果它出现在\env{paracol}环境中,你将收到一个警告消息,表示该命令被忽略。
% \item
% \changes{v1.3-5}{2013/09/17}
% 	{Remove description of \cs{multicolumnfootnotes},
%	 \cs{singlecolumnfootnotes}, \cs{mergedfootnotes} but mention they
%	 are still available.}
% 
% In old versions of \Paracol, namely 1.2 and its minor revisions 1.2x,
% footnote layout was controlled by a set of lengthy commands
% \Midx{\!\multicolumnfootnotes!} for |c|, \Midx{\!\singlecolumnfootnotes!}
% for |p|, and \Midx{\!\mergedfootnotes!} for |m|.
% Though they are still available and will be so forever for backward
% compatibility, it is recommended to use \!\footnotelayout!\footnote{
% 
% Not only for type saving but also for being familiar with this command
% which could have some advanced feature, for example to put gathered
% footnotes into a specific column, someday.}.
% 
在旧版本的 \Paracol 中(即 1.2 版本及其小修订版本 1.2x),脚注布局由一组冗长的命令控制:\Midx{\!\multicolumnfootnotes!} 用于 |c|,\Midx{\!\singlecolumnfootnotes!} 用于 |p|,\Midx{\!\mergedfootnotes!} 用于 |m|。虽然它们仍然可用,并且将永远保持向后兼容,但建议使用 \!\footnotelayout!\footnote{不仅为了节省输入,还为了熟悉这个命令,它可能具有一些高级功能,例如将收集的脚注放入特定的列中。}。

% % \item
% It must be $\meta{layout}\in\{|c|,|p|,|m|\}$, or you will have an error
% message of illegal layout specifier.

必须是 $\meta{layout}\in\{|c|,|p|,|m|\}$,否则您将收到非法布局说明符的错误消息。
% \end{itemize}
% 
% 
% 
% \KeepSpace{5}
% \item[\Midx{\!\footnote!}\texttt{*}\oarg{num}\marg{text}]\mbox{}
% \Item[\Midx{\!\footnotemark!}\texttt{*}\oarg{num}]\mbox{}
% \Item[\Midx{\!\footnotetext!}\texttt{*}\oarg{num}\marg{text}]\mbox{}\par
% The starred version of \!\footnote!, \!\footnotemark! and \!\footnotetext!
% are for the adjustment of the footnote numbering, the order of footnote
% marks in main texts, and the stacking order of footnotes at page
% bottom.  Their usages with various examples are given in
% Section~\ref{sec:fnnp}.
% 
\!\footnote!、\!\footnotemark! 和 \!\footnotetext! 的星号版本用于调整脚注编号、主文本中脚注标记的顺序以及页面底部脚注的堆叠顺序。其各种示例的用法详见第~\ref{sec:fnnp}节。
% 
% 
% \KeepSpace{3}
% \item[\Midx{\!\fncounteradjustment!}]\mbox{}
% \Item[\Midx{\!\nofncounteradjustment!}]\mbox{}\par
% The maintenance of \counter{footnote} with the starred footnote commands
% such as \!\footnote!|*| shown above causes out-of-order progress of the
% counter to make it hard to have a consistent counter value at
% \Endparacol.  The command \!\fncounteradjustment! is to let \Endparacol{}
% adjust the value of the counter based on its value at
% \beginparacol{} and the number of footnote commands in the environment.
% The command \!\nofncounteradjustment! is to tell \Endparacol{} to do
% nothing as in default.

使用上面展示的带星号的脚注命令(如!\footnote!|*|)来维护\counter{footnote}会导致计数器的顺序进展混乱,使得很难在\Endparacol{}处获得一致的计数器值。命令 \!\fncounteradjustment! 用于让\Endparacol{}根据其在\beginparacol{}处的值和环境中脚注命令的数量来调整计数器的值。命令 \!\nofncounteradjustment! 用于告诉\Endparacol{}不做任何调整,这是默认情况下的行为。
% 
% \begin{itemize}
% \item
% Though \!\footnotelayout! with |p|(age-wise) or |m|(erged) argument does
% \!\fncounteradjustment! while that with |c|(olumn) does
% \!\nofncounteradjustment! inside of it, you can override these settings by
% explicitly putting a counter adjustment command after \!\footnotelayout!.
% 
尽管使用|p|(age-wise)或|m|(erged)参数的 \!\footnotelayout! 会在其中执行 \!\fncounteradjustment!,而使用|c|(olumn)的 \!\footnotelayout! 会执行 \!\nofncounteradjustment!,但您可以通过在 \!\footnotelayout! 之后显式放置计数器调整命令来覆盖这些设置。
% \item
% The effect of \!\fncounteradjustment! is shown in Section~\ref{sec:fnnp}.

\!\fncounteradjustment! 的效果在第~\ref{sec:fnnp}~节中展示。
% \end{itemize}
% 

% \item[\Midx{\!\belowfootnoteskip!}]\mbox{}\par
% \changes{v1.35-4}{2018/12/31}
% 	{Add description of \cs{belowfootnoteskip}.}
% The typesetting parameter specifies the amount of the space inserted below
% footnotes of single-column \preenv{} if it does not have bottom floats.  The
% default amount is 0\,pt, i.e., no space is added.
% 
% typesetting参数指定了在单列\preenv{}的脚注下方插入的空间量,如果它没有底部浮动对象。默认的量是0,pt,即不添加任何空间。
% \end{description}
 
% \KeepSpace{6}
\subsection{用于着色文本和列分隔线的命令\\ Commands for Coloring Texts and Column-Separating Rules}
\label{sec:ref-tcolor}

\begin{description}
\item[\Midx{\!\columncolor!}\oarg{mode}\marg{color}\oarg{col}]\mbox{}
\Item[\Midx{\!\normalcolumncolor!}\oarg{col}]\mbox{}\par
\columnratio{0.55}
\begin{paracol}{2}
The command \!\columncolor! declares that the \emph{default color} of a
column is \meta{color} or what it specifies by the combination with the
optional \meta{mode}.  The command \!\normalcolumncolor! declares the
default color is what \!\normalcolor! specifies, i.e., black usually.  The
target column of these commands is that in which the commands reside, or
\meta{col} if it specified.
\switchcolumn
命令 \!\columncolor! 声明列的\emph{默认颜色}为\meta{color},或者通过与可选的\meta{mode}组合指定的颜色。命令 \!\normalcolumncolor! 声明默认颜色为 \!\normalcolor! 指定的颜色,即通常为黑色。这些命令的目标列是包含命令的列,或者如果指定了\meta{col},则为\meta{col}。
\end{paracol}

\begin{itemize}
\columnratio{0.55}
\begin{paracol}{2}
\item
The command may be outside of \env{paracol} environment.  If so and
\meta{col} is not provided, the target column is the leftmost 0.
\switchcolumn\item
该命令可以在\env{paracol}环境之外使用。如果是这样,并且未提供\meta{col},则目标列是最左边的列0。
\switchcolumn[0]*
\item
The default color declaration is \emph{global}.  Therefore, even if the
command appears in a \env{paracol} environment (and even in some grouping
structure in it), the declaration will be kept effective after
\Endparacol{} to determine the default color of the specified column in
succeeding \env{paracol} environments.
\switchcolumn\item
默认的颜色声明是\emph{全局的}。因此,即使该命令出现在\env{paracol}环境中(甚至在其中的某个分组结构中),该声明在\Endparacol{}之后仍将保持有效,以确定后续\env{paracol}环境中指定列的默认颜色。
\switchcolumn[0]*
\item
To give a color to texts (and maybe other stuff) in a column correctly,
you need to load \textsf{color} package or its relative (e.g.,
\textsf{xcolor}) which the implementation of coloring in \textsf{paracol}
relies on.
\switchcolumn\item
要正确给列中的文本(以及其他内容)着色,您需要加载\textsf{color}包或其相关包(例如\textsf{xcolor}),因为\textsf{paracol}中的着色实现依赖于它们。
\switchcolumn[0]*
\item
Coloring with \!\color!\oarg{mode}\marg{color} and other coloring commands
in \env{paracol} environments is of course allowed.  One caution is that
the \!\color! decides the color for following texts until other
specification is given or the group surrounding the command is closed.
Therefore, \!\switchcolumn! does not affect the coloring but a color given
to the texts in a column is also applied to the texts in the column to be
switched to.  This irrelativeness of coloring and \cswitch{} is shown in
the example below.
\switchcolumn\item
当然可以在\env{paracol}环境中使用 \!\color!\oarg{mode}\marg{color} 和其他着色命令。一个注意事项是 \!\color! 决定了后续文本的颜色,直到给出其他规范或关闭命令周围的分组。因此, \!\switchcolumn! 不会影响着色,但对于给定列中的文本的颜色也会应用于要切换到的列中的文本。下面的示例展示了着色和\cswitch{}的无关性。
\end{paracol}

\twosided[]\columnratio{0.5}\columnsep0pt
\tolerance5000\hbadness5000
\begin{paracol}{2}
\columncolor{blue}
This column is colored blue because\\
本栏目被着色为蓝色,因为\\
\mbox{}\qquad \!\columncolor!|{blue}|\\
is specfied.  Here we have a \!\switchcolumn!.\\
指定了。接着有一个 \!\switchcolumn!命令。
\switchcolumn
\columncolor{red}
This column is colored red because\\
本栏目被着色为红色,因为\\
\mbox{}\qquad\!\columncolor!|{red}|\\
is specified.\\
被指定了。

\begin{color}{green}
Now the color of the right column is changed to green because\\
现在右栏的颜色被更改为绿色,因为\\
\mbox{}\qquad\!\begin!|{color}{green}|\\
is given prior to this paragraph.  Now we have another \!\switchcolumn! to
go back to the left.
\\被指定了。现在我们有另一个 \!\switchcolumn!来返回到左侧。
\switchcolumn
The color of this paragraph is green because we are still in the
environment of green coloring, which we are now closing.\par
这段文字的颜色是绿色的,因为我们仍然处于绿色着色的环境中,而现在我们正在关闭它。\par
\end{color}%

Since the coloring environment has been closed, the color of this
paragraph is the default blue.  Now we have yet another and the last
\!\switchcolumn! to the right.\\
由于着色环境已关闭,这段文字的颜色是默认的蓝色。现在我们有另一个并且是最后一个 \!\switchcolumn! 向右切换。\\\
\switchcolumn
Since this paragraph is outside of the coloring environment, its color is
the default red.
\\由于这段文字在着色环境之外,它的颜色是默认的红色。
\end{paracol}

The default coloring of columns does not affect anything outside of
\env{paracol} environment of course, and thus this sentence is not
colored\footnote{%
Or colored black as \cs{normalcolor} specifies.}.
% \switchcolumn

默认的栏目着色当然不会影响\env{paracol}环境之外的任何内容,因此这个句子没有被着色\footnote{或者按照\cs{normalcolor}的指定,着色为黑色。}。

\begin{Verbatim}
\normalcolumncolor[0]\normalcolumncolor[1]
\end{Verbatim}
\normalcolumncolor[0]\normalcolumncolor[1]
% \columnratio{0.55}
% \begin{paracol}{2}
% % The default coloring of columns does not affect anything outside of
% % \env{paracol} environment of course, and thus this sentence is not
% % colored\footnote{%
% % Or colored black as \cs{normalcolor} specifies.}.
% % \switchcolumn
% % 默认的栏目着色当然不会影响\env{paracol}环境之外的任何内容,因此这个句子没有被着色\footnote{或者按照\cs{normalcolor}的指定,着色为黑色。}。
% \end{paracol}

\end{itemize}



\KeepSpace{4}
\item[\Midx{\!\coloredwordhyphenated!}]\mbox{}
\Item[\Midx{\!\nocoloredwordhyphenated!}]\mbox{}\par
\columnratio{0.55}
\begin{paracol}{2}
The command \!\coloredwordhyphenated! allows the first word following a
coloring command such as \!\color! to be hyphenated, but at the same time
make it possible that a line is broken before the word.  The command
\!\nocoloredwordhyphenated! acts oppositely and thus line breaking before
the first word and hyphenating it are inhibited.  By default,
\!\coloredwordhyphenated! is effective.
\switchcolumn
命令 \!\coloredwordhyphenated! 允许在着色命令(如 \!\color!)后的第一个单词进行连字符划分,但同时也可能在该单词之前进行换行。命令 \!\nocoloredwordhyphenated! 则具有相反的作用,从而禁止在第一个单词之前进行换行和连字符划分。默认情况下,\!\coloredwordhyphenated! 是有效的。

\end{paracol}
\begin{itemize}
\columnratio{0.55}
\begin{paracol}{2}
\item
The implementation of \textsf{color} package and its relatives makes it
impossible that \meta{word} is hyphenated when it appears like
|{|\!\color!|{red}|\meta{word} \ldots|}| or
\!\textcolor!|{|\meta{word} \ldots|}|.  This inhibition of the hyphenation
is sometimes annoying especially when the document is multi-columned and
thus a line is narrow and a column is written in a language having long
words such as German.  Therefore in \Paracol{} package, a trick is used to
allow the \meta{word} is hyphenated.  However this trick being insertion
of a null horizontal space has a side effect that the word can have a line
break before it.  Though this line break is usually unharmful, in a
special occasion the break is undesirable and
in\textcolor{red}{appropriate} by making it possible that the
\emph{half-colored} word `inappropriate' is broken between `in' and
`appropriate' without hyphenation.  Therefore, if you find such a
inappropriate break, use \!\nocoloredwordhyphenated! as follows, for example.
\switchcolumn\item
\textsf{color}宏包及其相关命令的实现方式使得在类似于 |{|\!\color!|{red}|\meta{word} \ldots|}| 或 \!\textcolor!|{|\meta{word} \ldots|}| 的情况下,无法对 \meta{word} 进行连字符划分。这种禁止连字符划分的机制有时会令人感到不便,特别是在文档具有多列布局的情况下,当一行较窄且一列使用具有较长单词的语言(如德语)时。因此,在 \Paracol{} 宏包中,使用了一个技巧来允许对 \meta{word} 进行连字符划分。然而,这个技巧是插入一个空的水平间距,这会导致单词之前出现一个换行。虽然这种换行通常没有问题,但在特殊情况下,这种换行可能是不可取的,并且\textcolor{red}{不合适},因为它使得半着色的单词“inappropriate”在“in”和“appropriate”之间断开而没有连字符划分。因此,如果您发现这样的不合适的断行,请使用 \!\nocoloredwordhyphenated!,例如以下方式。

\end{paracol}
\begin{quote}
|{\nocoloredwordhyphenated in\textcolor{red}{appropriate}}|
\end{quote}
\end{itemize}


\KeepSpace{4}
\item[\Midx{\!\colseprulecolor!}\oarg{mode}\marg{color}\oarg{col}]\mbox{}
\Item[\Midx{\!\normalcolseprulecolor!}\oarg{col}]\mbox{}\par
\changes{v1.3-3}{2013/09/17}
{Add description of \cs{colseprulecolor} and
    \cs{normalcolseprulecolor}.}

The command \!\colseprulecolor! declares the color for
{\em\Uidx\cseprule{}s}, being the vertical rules drawn at the center of
gaps between columns, is \meta{color} or what it specifies by the
combination with the optional \meta{mode}.  The command
\!\normalcolseprulecolor! declares the color of rules is what
\!\normalcolor! specifies, i.e., black usually.  If the optional argument
\meta{col} is given, these commands specifies the color of the rule in the
gap following the column whose ordinal is \meta{col}, rather than all rules.

命令 \!\colseprulecolor! 用于声明列分隔符的颜色,列分隔符是在列之间的间隙中央绘制的垂直线条。颜色可以是特定的颜色,也可以是与可选的模式组合指定的颜色。
命令 \!\normalcolseprulecolor! 将列分隔符的颜色设置为 \normalcolor 命令指定的颜色,通常为黑色。
如果给出可选参数 \meta{col},这些命令将指定在具有序号 \meta{col} 的列之后的间隙中的分隔符的颜色,而不是所有分隔符的颜色。
\begin{itemize}
\item
The rules are drawn if \LaTeX's typesetting parameter \!\columnseprule!
for the rule width has non-zero value, e.g., 0.4\,|pt| to obey the
standard rule thickness.  The rules are \emph{not} drawn on \pwstuff{},
i.e., \Preenv{} and \postenv, \pwise{} floats or (\mgfnote{} or
non-merged) \Scfnote{}s of course but also \mctext{}s.  Therefore, if a
page has \mctext{}s, the rules are {\em broken} by them as shown in the
red rule example below.

如果\LaTeX 的排版参数 \!\columnseprule! 的规则宽度具有非零值(例如,0.4,|pt|以遵守标准规则厚度),则会绘制规则。规则不会绘制在\pwstuff{}上,即\Preenv{}和\postenv,\pwise{}浮动对象或(\mgfnote{}或非合并的)\Scfnote{}上,当然也不会绘制在\mctext{}上。因此,如果页面上有\mctext{},则它们会打破规则,如下面的红色规则示例所示。
\global\unitlength\@totalleftmargin
\end{itemize}
\end{description}

\columnseprule0.4pt\colseprulecolor{red}[1]\colseprulecolor{white}[0]
\setcolumnwidth{\unitlength/0pt}
\begin{paracol}{3}\switchcolumn\noindent 
This is a left column paragraph preceding a \mctext.  Of cource the rule
separating this and the next column starts from the top of this paragraph.

这是一个左列段落,位于 \mctext 之前。当然,分隔这个段落和下一列的规则从该段落的顶部开始。
\switchcolumn\noindent
This is a right column paragraph preceding a \mctext{} given by the
\!\switchcolumn!|*| at its end.

这是一个位于右列的段落,在其末尾由 \!\switchcolumn!|*| 给出的\mctext{}之前。
\switchcolumn[1]*[\subsubsection*{\hbox to\unitlength{}
An Example of Spanning Text Given by \cs{subsubsection}|*| Command \hfill 一个由\cs{subsubsection}|*|命令给出的跨列文本示例}]
Since we have a \mctext{} above, the red rule separating this and the next
column is broken by the text.

由于上方有一个\mctext{},分隔这一列与下一列的红色分隔线被文本打断。
\switchcolumn
It is also natural that the rule separating this and the previous column is
terminated at the end of this \env{paracol} environment.

同样自然的是,分隔这一列与前一列的分隔线在\env{paracol}环境的末尾终止。
\end{paracol}
\columnseprule0pt\columnratio{}

\begin{description}
\Item[]\mbox{}
\begin{itemize}
\Item
To give a color to rules correctly, you need to load \textsf{color}
package or its relative (e.g., \textsf{xcolor}) which the implementation
of coloring in \textsf{paracol} relies on.

为了正确给分隔符上色,您需要加载 \textsf{color} 或其相关包(例如 \textsf{xcolor}),因为 \textsf{paracol} 中的着色实现依赖于它们。

\item
Once you give a color to rules in a specific gap with the optional
\meta{col}, another \!\colseprulecolor! or \!\normalcolseprulecolor!
without \meta{col} does \emph{not} change the color of the rule in the
gap.

一旦您使用可选参数 \meta{col} 为特定间隙中的分隔符指定了颜色,再次使用 \!\colseprulecolor! 或 \!\normalcolseprulecolor!,而没有使用 \meta{col},不会改变该间隙中的分隔符的颜色。
\end{itemize}
\end{description}
 
% 
\KeepSpace{7}
\subsection{Commands for Background Painting\hfill 用于背景绘制的命令}
\label{sec:ref-bgpaint}

\begin{description}
\item[\Midx{\!\backgroundcolor!}\marg{region}\oarg{mode}\marg{color}]
    \mbox{}\par
\Item[\Midx{\!\backgroundcolor!}
    \Arg{\meta{region}$|(|x_0|,|y_0|)|$}\oarg{mode}\marg{color}]
    \mbox{}\par
\Item[\Midx{\!\backgroundcolor!}
    \Arg{\meta{region}$|(|x_0|,|y_0|)||(|x_1|,|y_1|)|$}
    \oarg{mode}\marg{color}]
    \mbox{}\par
\columnratio{0.55}
\begin{paracol}{2}
The command declares that {\em\Uidx\bgpaint} of \meta{region} is performed
with \meta{color} or what it specifies by the combination of the optional
\meta{mode}.  The \meta{region} whose \bground{} is painted is one of the
following.
\switchcolumn
该命令声明使用 \meta{color} 或其由可选 \meta{mode} 组合指定的方式来执行 \meta{region} 的{\em\Uidx\bgpaint}。被着色的\bground{}的\meta{region}是以下之一。
\end{paracol}


\begin{description}
\columnratio{0.55}
\begin{paracol}{2}
\item[|c|\rm(\textit{olumn})] for all columns, or particular one if
\meta{region} is |c|\oarg{col} to specify its ordinal \meta{col}.
\switchcolumn\item[|c|\rm(\textit{olumn})]
适用于所有列,或者如果\meta{region}为|c|\oarg{col}时,可以指定特定的列序号\meta{col}。
\switchcolumn[0]*
\item[|g|\rm(\textit{ap})] for all gaps between columns, or particular one
if \meta{region} is |g|\oarg{col} to specify the ordinal \meta{col} of the
column preceding the gap.
\switchcolumn
\item[|g|\rm(\textit{ap})]
对于所有列之间的间隙,或者特定的间隙,可以使用\meta{region}参数。如果\meta{region}是|g|\oarg{col},则可以指定前一个间隙的序号\meta{col}。
\switchcolumn[0]*
\item[|s|\rm(\textit{panning})] for \mctext{}s.
\switchcolumn\item[|s|\rm(\textit{panning})]
用于\mctext{}。
\switchcolumn[0]*
\item[|f|\rm(\textit{loat})] for \pwise{} floats.
\switchcolumn\item[|f|\rm(\textit{loat})]
用于 \pwise{} 浮动体。
\switchcolumn[0]*
\item[|n|\rm(\textit{ote})] for (\mgfnote{} or non-merged) \Scfnote{}s.
\switchcolumn\item[|n|\rm(\textit{ote})]
用于 (\mgfnote{} 或非合并的) \Scfnote{}。
\switchcolumn[0]*
\item[|p|\rm(\textit{re/post})] for \Preenv{} and \postenv.
\switchcolumn\item[|p|\rm(\textit{re/post})]
用于 \Preenv{} 和 \postenv。
\switchcolumn[0]*
\item[|t|\rm(\textit{op})] for top margin.
\switchcolumn\item[|t|\rm(\textit{op})]
用于顶部边距。
\switchcolumn[0]*
\item[|b|\rm(\textit{ottom})] for bottom margin.
\switchcolumn\item[|b|\rm(\textit{ottom})]
用于底部边距。
\switchcolumn[0]*
\item[|l|\rm(\textit{eft})] for left margin.
\switchcolumn\item[|l|\rm(\textit{eft})]
用于左边距。
\switchcolumn[0]*
\item[|r|\rm(\textit{ight})] for right margin.
\switchcolumn\item[|r|\rm(\textit{ight})]
用于右边距。
\end{paracol}

\end{description}
\columnratio{0.55}
\begin{paracol}{2}
In addition, capitals of the keys above, i.e., |C|, |G|, \ldots, |L|, are
also legitimate for {\em under painting}.  For example, you may specify to
paint the \bground{} of a region, say top margin, by two
\!\backgroundcolor! with |t| and |T| and with different color arranging the
size of the region of either |t| or |T| (or both of them) by the
\emph{\bgext} option shown below.
\switchcolumn
此外,上面的键的大写字母,即|C|、|G|、\ldots、|L|,也可以用于{\em 下层绘制}。例如,您可以通过两个不同颜色的 \!\backgroundcolor!(使用|t|和|T|)和通过\emph{\bgext}选项来调整|t|或|T|(或两者)的区域大小,来指定绘制区域(例如顶部边距)的\bground{}。
\switchcolumn[0]*
The optional $|(|x_0|,|y_0|)|$ is to enlarge the region to be painted
shifting its left-top and right-bottom corner outside by
the dimension $x_0$ horizontally and $y_0$ vertically, or to shrink it
with negative dimensions.  This {\em\Uidx\bgext} can be asymmetric giving
another optional $|(|x_1|,|y_1|)|$ so that it acts on the right-bottom
corner while let $|(|x_0|,|y_0|)|$ shift only the left-top corner.
Moreover, you may make each \bgext{} {\em infinite} by giving 10000\,|pt|
(about 3.5\,m) to $x_0$, $y_0$, $x_1$ and/or $y_1$ so that the
corresponding region edge is shifted to the paper edge.  Furthermore, this
{\em\Uidx\bginfext{}} can be terminated at the point $\alpha$ inside the
corresponding paper edge by giving $10000\,|pt|-\alpha$
($\alpha\leq1000\,|pt|$) to an extension parameter $x_0$, etc.
\switchcolumn
可选的 $|(|x_0|,|y_0|)|$ 是为了扩大要着色的区域,将其左上角和右下角分别水平和垂直地移出维度 $x_0$ 和 $y_0$,或者用负维度来缩小它。这个{\em\Uidx\bgext}可以是不对称的,可以给出另一个可选的 $|(|x_1|,|y_1|)|$,让它作用于右下角,而 $|(|x_0|,|y_0|)|$ 只移动左上角。此外,您可以通过将 $x_0$、$y_0$、$x_1$ 和/或 $y_1$ 设置为 10000,|pt|(约为 3.5,m)来使每个\bgext{}变为{\em 无限},从而将相应的区域边缘移动到纸张边缘。此外,通过将扩展参数 $x_0$ 等设置为 $10000,|pt|-\alpha$($\alpha\leq1000,|pt|$),这个{\em\Uidx\bginfext{}}可以在相应的纸张边缘内的点 $\alpha$ 处终止。
\end{paracol}


\begin{itemize}
\columnratio{0.55}
\begin{paracol}{2}
\item
A region whose color is not specified is not painted and thus left blank
(or kept as painted by \!\pagecolor! if you specify it).
\switchcolumn\item
未指定颜色的区域不会被绘制,因此保持为空白(或者如果您指定了 \!\pagecolor!,则保持为 \!\pagecolor! 绘制的颜色)。
\switchcolumn[0]*\item
Under-painting of columns and gaps by |C| and |G| is made for regions
different from those over-painting |c| and |g|.  That is, under-painting
is done ignoring all \pwstuff{} and thus the height of the regions is
always $\!\textheight!+\!\maxdepth!$.  On the other hand, over-painting is
only for chunks shrunk or separated by \pwstuff.
\switchcolumn\item
对于与覆盖|c|和|g|不同的区域,通过|C|和|G|进行的列和间隙的底层绘制是独立的。也就是说,底层绘制忽略所有\pwstuff{},因此区域的高度始终为 $\!\textheight!+\!\maxdepth!$。另一方面,覆盖绘制仅适用于通过\pwstuff{}缩小或分离的块。
\switchcolumn[0]*\item
You may exploit the following painting order, where $x_i$
is the $i$-th \mctext{} ($x\in\{|s|,|S|\}$) or $i$-th chunk followed by
the $i$-th \mctext, $m$ and $n$ is the number of \mctext{}s and columns in
a page respectively, to overlay a preceding region with a succeeding
region, if your \emph{printer} allows overlaid color painting.
\begin{eqnarray*}
|T|&\to&|B|\to|L|\to|R|
    \to|G[|0|]|\to\cdots\to|G[|n{-}1|]|\to|C[|0|]|\to\cdots\to|C[|n{-}1|]|\\
&\to&|t|\to|b|\to|l|\to|r|\to|N|\to|n|\to\{|F|,|P|\}\to\{|f|,|p|\}
\to|S|_1\to\cdots\to|S|_m\\
&\to&|g|_1|[|0|]|\to\cdots|g|_1|[|n{-}2|]|\to
    |c|_1|[|0|]|\to\cdots|c|_1|[|n{-}1|]|\to|s|_1\\
&\to&\cdots\\
&\to&|g|_m|[|0|]|\to\cdots|g|_m|[|n{-}2|]|\to
    |c|_m|[|0|]|\to\cdots|c|_m|[|n{-}1|]|\to|s|_m\\
&\to&|g|_{m+1}|[|0|]|\to\cdots|g|_{m+1}|[|n{-}2|]|\to
    |c|_{m+1}|[|0|]|\to\cdots|c|_m|[|n{-}1|]|
\end{eqnarray*}
\switchcolumn\item
您可以利用以下绘制顺序,其中$x_i$是第$i$个\mctext{}($x\in{|s|,|S|}$)或第$i$个块之后的第$i$个\mctext{},$m$和$n$分别是页面上的\mctext{}和列的数量,以将前一个区域与后一个区域叠加在一起,如果您的\emph{打印机}允许叠加颜色绘制。
\begin{eqnarray*}
|T|&\to&|B|\to|L|\to|R|
    \to|G[|0|]|\to\cdots\to|G[|n{-}1|]|\to|C[|0|]|\to\cdots\to|C[|n{-}1|]|\\
&\to&|t|\to|b|\to|l|\to|r|\to|N|\to|n|\to\{|F|,|P|\}\to\{|f|,|p|\}
\to|S|_1\to\cdots\to|S|_m\\
&\to&|g|_1|[|0|]|\to\cdots|g|_1|[|n{-}2|]|\to
    |c|_1|[|0|]|\to\cdots|c|_1|[|n{-}1|]|\to|s|_1\\
&\to&\cdots\\
&\to&|g|_m|[|0|]|\to\cdots|g|_m|[|n{-}2|]|\to
    |c|_m|[|0|]|\to\cdots|c|_m|[|n{-}1|]|\to|s|_m\\
&\to&|g|_{m+1}|[|0|]|\to\cdots|g|_{m+1}|[|n{-}2|]|\to
    |c|_{m+1}|[|0|]|\to\cdots|c|_m|[|n{-}1|]|
\end{eqnarray*}
\switchcolumn[0]*
\item
If you specify |b| feature by \!\twosided!, \bgpaint{} is
{\em\Uidx\mirror{}ed} in even-numbered pages so that |l| and |L| mean
right margin, |r| and |R| mean left margin, and asymmetric extensions are
applied to right-top and left-bottom corners.
\switchcolumn\item
如果您通过 \!\twosided! 命令指定了|b|特性,那么在偶数页上\bgpaint{}会被{\em\Uidx\mirror{}反转},这样|l|和|L|表示右边距,|r|和|R|表示左边距,并且对右上角和左下角应用非对称扩展。
\switchcolumn[0]*\item
To give a color for \bgpaint{} correctly, you need to load \textsf{color}
package or its relative (e.g., \textsf{xcolor}) which the implementation
of coloring in \textsf{paracol} relies on.
\switchcolumn\item
要正确给\bgpaint{}着色,您需要加载\textsf{color}包或其相关包(例如\textsf{xcolor}),因为\textsf{paracol}中的着色实现依赖于它们。
\switchcolumn[0]*\item
To paint margins and regions having infinite extension correctly, the
parameters \!\paperwidth! and \!\paperheight! should be set properly by,
for example, a paper selection option of \!\documentclass!.
\switchcolumn\item
为了正确绘制具有无限扩展的边距和区域,\!\paperwidth! 和 \!\paperheight! 参数应该通过 \!\documentclass! 的纸张选择选项正确设置。
\switchcolumn[0]*\item
Section~\ref{sec:bgpaint} shows examples of \bgpaint{} to give you more
intutive explanations of \!\backgroundcolor! and its region specifications.
\switchcolumn\item
第~\ref{sec:bgpaint}节展示了\bgpaint{}的示例,以便更直观地解释 \!\backgroundcolor! 及其区域规范。
\end{paracol}
\end{itemize}



\item[\Midx{\!\nobackgroundcolor!}\marg{region}]\mbox{}
\Item[\Midx{\!\resetbackgroundcolor!}]\mbox{}\par
\changes{v1.3-3}{2013/09/17}
{Add description of \cs{nobackgroundcolor} and
    \cs{resetbackgroundcolor}.}

The command \!\nobackgroundcolor! declares that the \bground{} of
\meta{region} is not painted, where \meta{region} is one of legitimate
region specifiers of \!\backgroundcolor!.  The command
\!\resetbackgroundcolor! declares no regions are painted and thus gives
you the default state.

命令 \!\nobackgroundcolor! 声明\meta{region}的\bground{}不被绘制,其中\meta{region}是 \!\backgroundcolor! 的合法区域指示符之一。命令 \!\resetbackgroundcolor! 声明没有区域被绘制,从而恢复默认状态。

\begin{itemize}
\item
If you specified the \bgpaint{} of |c|\oarg{col} or |g|\oarg{col} by
\!\backgroundcolor!, the painting is \emph{not} canceled by
\!\nobackgroundcolor! with |c| or |g| but without \oarg{col}.  Similarly,
once you made declarations of \bgpaint{} of both |c| and |c|\oarg{col}
(resp.\ |g| and |g|\oarg{col}), \!\nobackgroundcolor! with |c|\oarg{col}
(resp.\ |g|\oarg{col}) cancels the painting of |c|\oarg{col} (resp.\
|g|\oarg{col}) but the region will still be painted by the color you gave
to |c| (resp.\ |g|).

如果您通过 \!\backgroundcolor! 指定了 |c|\oarg{col} 或 |g|\oarg{col} 的 \bgpaint{},则使用不带 \oarg{col} 的 |c| 或 |g| 的 \!\nobackgroundcolor! 不会取消绘制。同样,一旦您对 |c| 和 |c|\oarg{col}(或 |g| 和 |g|\oarg{col})都进行了声明,使用 |c|\oarg{col}(或 |g|\oarg{col})的 \!\nobackgroundcolor! 将取消 |c|\oarg{col}(或 |g|\oarg{col})的绘制,但区域仍然会使用您给出的颜色进行绘制。
\end{itemize}



\item[\Midx{\!\pagerim!}]\mbox{}\par
\changes{v1.3-3}{2013/09/17}
{Add description of \cs{pagerim}.}

This is a (kind of) \emph{length command}\footnote{

In reality, it is a \cs{dimen} register rather than a \cs{skip} register.}

这是一种(某种程度上的)\emph{长度命令}\footnote{实际上,它是一个\cs{dimen}寄存器,而不是\cs{skip}寄存器。}。

to have the width of the \emph{rim} area placed at each paper edge to 
inhibit \bgpaint{} in the area.  That is, the inner edges of the area are
considered as virtual paper edges to block painting of all margins and
regions having \bginfext{} to the edges, for example in order to
avoid printing troubles caused by painting the rim area too close to the
real paper edges.  The default value of \!\pagerim! is 0 to allow paint
anywhere in a paper.

为了使每个纸张边缘的\emph{边缘}区域的宽度用于抑制该区域内的\bgpaint{}。也就是说,该区域的内部边缘被视为虚拟纸张边缘,以阻止所有具有\bginfext{}到边缘的边缘和区域的着色,例如为了避免将边缘区域着色过于靠近真实纸张边缘而造成的打印问题。\!\pagerim! 的默认值为 0,允许在纸张的任何位置进行着色。
\end{description}

%  \subsection{Control of Contents Output\hfill 内容输出的控制}
 \label{sec:ref-contents}
 
 \begin{description}
 \item[\Midx{\!\addcontentsonly!}\marg{file}\marg{col}]\mbox{}\par
 The command inhibits the output of contents information to
 $\meta{file}\in\{|toc|,|lof|,|lot|\}$ from columns other than \meta{col}.

该命令禁止除 \meta{col} 外的列将内容信息输出到 $\meta{file}\in{|toc|,|lof|,|lot|}$。
 \begin{itemize}
 \item
 For example, this manual has \!\addcontentsonly!|{toc}{0}| to
 inhibit the contents information output from \!\subsection! commands
 in the right column in Section~\ref{sec:env} and~\ref{sec:float},
 or the table should have duplicated entries of sub-sections.

 例如,本手册使用 \!\addcontentsonly!|{toc}{0}| 来阻止在第~\ref{sec:env}节和~\ref{sec:float}节的右列中,由 \!\subsection! 命令输出的目录信息,否则表格将会有子节的重复条目。

 \item
 It must be $\meta{file}\in\{|toc|,|lof|,|lot|\}$, or you will have an
 error message of illegal type of contents file.

 它必须是$\meta{file}\in{|toc|,|lof|,|lot|}$,否则将会收到一个不合法的内容文件类型的错误消息。
 \end{itemize}
 \end{description}
 

 
 \subsection{Page Flushing Commands\hfill 页面刷新命令}
 \label{sec:ref-flush}
 
 \begin{description}
 \item[\Midx{\!\flushpage!}]\mbox{}\par
 The command flushes pages up to the {\em\Uidx\tpage} in which the \lcolumn{}
 resides.  Deferred floats which can be put in the pages up to the \tpage{}
 are also flushed.
 
该命令将页面刷新到包含\lcolumn{}的{\em\Uidx\tpage}。也会刷新可以放置在\tpage{}之前的页面上的延迟浮动体。
 \item[\Midx{\!\clearpage!}]\mbox{}\par
 The command does what \!\flushpage! does and then flushes all floats still
 deferred if any.  The deferred float flushing beyond the \tpage{} takes
 place at first for \cwise{} ones creating \fcolumn{}s for them, and
 then for \pwise{} ones creating {\em\Uidx\fpage{}s} only with
 \pwise{} floats, as \LaTeX's \!\clearpage! does outside \env{paracol}
 environment.

该命令执行 \!\flushpage! 的功能,然后刷新所有延迟的浮动对象(如果有的话)。在\tpage{}之后,延迟的浮动对象刷新首先针对\cwise{}的浮动对象,为它们创建\fcolumn{},然后针对\pwise{}的浮动对象,只创建包含\pwise{}浮动对象的{\em\Uidx\fpage{}},就像在\env{paracol}环境之外使用\LaTeX 的 \!\clearpage! 命令一样。
 \item[\Midx{\!\cleardoublepage!}]\mbox{}\par
 \changes{v1.3-5}{2013/09/17}
	{Add description of \cs{cleardoublepage}.}
 The command does what \LaTeX's \!\cleardoublepage! does outside
 \env{paracol}.  That is, it does \!\clearpage! always and then leaves a
 blank page if it is even-numbered and two-sided |p|(age) feature is
 enabled by |twoside| option of \!\documentclass! or \Paracol's own
 \!\twosided! command shown in Section~\ref{sec:ref-twoside}.

该命令做的是在\env{paracol}之外与\LaTeX 的 \!\cleardoublepage! 相同的操作。也就是说,它总是执行 \!\clearpage!,然后如果该页是偶数页,并且通过 \!\documentclass! 的|twoside|选项或\Paracol 的 \!\twosided! 命令(见第~\ref{sec:ref-twoside} 节)启用了双面特性,它会留下一个空白页。
 \begin{itemize}
 \item
 This command is equivalent to \!\clearpage! in \env{paracol} environments
 for \npaired{} \parapag{}ing because \!\clearpage! flushes \emph{both}
 left and right \parapag{}es.

在对于\npaired{} \parapag{}ing的\env{paracol}环境中,该命令等效于 \!\clearpage!,因为 \!\clearpage! 会刷新\emph{左侧和右侧}的\parapag{}es。
 \end{itemize}
 \end{description}
\def\Dotquad{\leavevmode\cleaders\hbox to.44em{\hss.\hss}%
  \hskip\parindent\kern0pt}
\def\fnpar#1#2{#1 paragraph\Dotfill\\\Dotfill with a footnote#2\Dotfill
  in it.\par}
\def\Fnpar#1#2#3{#1 paragraph\Dotfill\\\Dotfill with a footnote#2\Dotfill
  in it.\\#3\par}
% \section{Numbering and Placement of Page-Wise Footnotes\hfill 页注的编号和位置}
% \label{sec:fnnp}
% \changes{v1.2-2}{2013/05/11}
% 	{Add the section ``Numbering and Placement of Single-Columned
%	 Footnotes'' to describe page-wise footnotes in detail.}
% \changes{v1.3-5}{2013/09/17}
%	{Rename the section title from ``Numbering and Placement of
%	 Single-Columned Footnotes''to ``Numbering and Placement of
% 	``Page-Wise Footnotes'' following new naming.}
% 
% Here we have a simple example of \scfnote{} but not-merged
% footnotes\footnote{
% Because of the non-merged typesetting, this footnote is put above the
% example.\par\Hrule\label{fn:preenv}}.

% 这里有一个简单的示例,展示了非合并的\scfnote{}脚注\footnote{因为不是合并的排版方式,所以这个脚注放在了示例之上。\par\Hrule\label{fn:preenv}}。

% \footnotelayout{p}
% \begin{paracol}{2}
% \fnpar{First left-column 左列第一}{\footnote{First left-column footnote. 左列第一脚注。}}
% \fnpar{Second left-column 左列第2}{\footnote{Second left-column footnote. 左列第2脚注。}}
% \switchcolumn
% \fnpar{First right-column}{\footnote{First right-column footnote.}}
% \fnpar{Second right-column}{\footnote{Second right-column footnote.
% This and all other footnotes above are \scfnote{} and, since footnote
% typesetting is non-merged, they are put above the \postenv.右列第二脚注。这个脚注和上面的所有脚注都是\scfnote{},由于脚注排版是非合并的,它们放在了\postenv{}之上。 }}
% \end{paracol}
% \Hrule
% 
% As shown above, it is easy to have a reasonable result of footnote
% numbering and placement as far as your \env{paracol} environment is
% completely included in a page and you accept the numbering in
% left-column-first manner constructing the environment as follows
% exploiting the fact \counter{footnote} is made global, where $b$ is the
% value of \counter{footnote} counter at \beginparacol, i.e., the number
% given to the footnote just preceding the environment, and thus
% $b=\ref{fn:preenv}$ in the example above.

如上所示,只要您的 \env{paracol} 环境完全包含在一页中,并且您接受按左列优先的方式编号和放置脚注,那么脚注编号和放置的结果就会比较合理。可以通过以下方式构建环境,利用 \counter{footnote} 是全局的这一事实,其中 $b$ 是在 \beginparacol 处的 \counter{footnote} 计数器的值,即在环境之前的脚注的编号,因此在上面的示例中 $b=\ref{fn:preenv}$。

% \begin{quote}\vskip-1pt
% |\begin{paracol}{2}|\\
% \textit{left-column stuff having $n$ footnotes numbered $b+1$, $b+2$,
% \ldots, $b+n$}\\
% |\switchcolumn|\\
% \textit{right-column stuff having $m$ footnotes numbered $b+n+1$, $b+n+2$,
% \ldots, $b+n+m$}\\
% |\end{paracol}|
% \end{quote}\vskip-1pt

% The real life is, however, tougher than that, because the assumptions above
% are too optimistic as described in the following subsections.

然而,现实生活比上面的假设更加艰难,因为如下小节所描述的那样,这些假设过于乐观。
% \vskip-3pt\vskip0pt 
\end{document} 
% \DocInput{paracol.dtx}




 


\subsection{Multiple \cs{switchcolumn} in a Page\hfill 页面中的多个\cs{switchcolumn}}
\label{sec:fnnp-multsc}
\columnratio{0.55}
\begin{paracol}{2}
Here we have an example with three \!\switchcolumn! commands in a page
having six footnotes.  Hereafter, footnotes are typeset with
\Uidx{\!\footnotelayout!}|{m}|\footnote{And thus this footnote is merged with those in the \env{paracol}
environment.}.
\switchcolumn
下面是一个在页面中使用了三个 \!\switchcolumn! 命令的示例,其中包含六个脚注。在此之后,使用 \Uidx{\!\footnotelayout!}|{m}|\footnote{这个脚注与\env{paracol}环境中的脚注合并了。}设置脚注样式。    
\end{paracol}

\begin{Verbatim}
\footnotelayout{m}
\end{Verbatim}

\footnotelayout{m}
\Hrule
\begin{paracol}{2}
\fnpar{First left-column 左列的第一个}{\footnote{First left-column
footnote.左列的第一个脚注。 \label{fn:2L1}}}
\Fnpar{Second left-column 左列的第二个}{\footnote{Second left-column
footnote. 左列的第二个脚注。\label{fn:2L2}}}{
It is followed by a \cs{switchcolumn}. 它后面跟着一个\cs{switchcolumn}。}
\switchcolumn
\Fnpar{First right-column 右列的第一个}{\footnote{First right-column footnote but
following the second left-column one. 右列的第一个脚注,但是在第二个左列脚注之后。\label{fn:2R1}}}{It is followed by a
\cs{switchcolumn*}. 它后面跟着一个\cs{switchcolumn*}。}
\switchcolumn*
\Fnpar{Third and synchronized left-column 左列的第三个(同步的)}{\footnote{Third left-column
footnote but following the first right-column one. 左列的第三个脚注,但是在第一个右列脚注之后。\label{fn:2L3}}}{It is
followed by a \cs{switchcolumn}. 它后面跟着一个\cs{switchcolumn}。}
\switchcolumn
\fnpar{Second and synchronized right-column 右列的第二个(同步的)}{\footnote{Second right-column
footnote but following the third left-column one. 右列的第二个脚注,但是在第三个左列脚注之后。\label{fn:2R2}}}
\fnpar{Third right-column 右列的第三个}{\footnote{Third right-column
footnote.\label{fn:2R3} 右列的第三个脚注。}}
\end{paracol}
\newpage

\columnratio{0.55}
\begin{paracol}{2}
The example in the previous page should look weird because the order of
the third footnote in the left column \ref{fn:2L3} and the first in the
right \ref{fn:2R1} are reversed in their numbers and in the stack at the
page bottom.  However, the result is \emph{natural} because they are
numbered and stacked in the order of occurrence in the source |.tex| as
always done in any documents without \textsf{paracol} and with it but
\mcfnote{} footnote typesetting.  Since the \textsf{paracol} cannot
maintain the order automatically\footnote{%
So far, because the maintenance is extremely tough.  But since it is not
impossible, some day you could have an improved version of
\textsf{paracol} with the automatic ordering.},
\switchcolumn
在上一页的示例中,看起来有些奇怪,因为左列中的第三个脚注 \ref{fn:2L3} 和右列中的第一个脚注 \ref{fn:2R1} 在它们的编号和页面底部的堆栈中的顺序上是颠倒的。然而,这个结果是“自然”的,因为它们按照在源 |.tex| 中出现的顺序进行编号和堆叠,这是任何没有使用 \textsf{paracol} 或使用了 \mcfnote{} 脚注排版的文档中都会这样做的。由于 \textsf{paracol} 无法自动维护顺序\footnote{至今为止,因为维护顺序非常困难。但是,既然不是不可能,总有一天您可能会有一个改进版的 \textsf{paracol},具有自动排序功能。},
\switchcolumn[0]*
you have to maintain it by yourself.
\switchcolumn
你需要自己维护这个问题。
\switchcolumn[0]*
The problem is partly solved by using \!\footnote! with its optional
argument \oarg{num} to number the first right-column and the third
left-column footnotes explicitly, i.e., to give
$\mathit{num}=\ref{fn:2L3}$ to the former and $\mathit{num}=\ref{fn:2R1}$
to the latter.  One caution is that you have to remember that \!\footnote!
with the optional \meta{num} does not update \counter{footnote} counter
and thus you have to do
\!\setcounter!|{footnote}{|\texttt{\ref{fn:2L3}}|}| or
\!\addtocounter!|{footnote}{2}| after the third left-column footnote.
\switchcolumn
部分解决这个问题的方法是使用带有可选参数\oarg{num}的 \!\footnote! 命令,来显式地对第一个右列和第三个左列的脚注进行编号,即给前者赋值$\mathit{num}=\ref{fn:2L3}$,给后者赋值$\mathit{num}=\ref{fn:2R1}$。需要注意的是,你必须记住,带有可选参数\meta{num}的 \!\footnote! 命令不会更新\counter{footnote}计数器,因此你需要在第三个左列脚注之后使用 \!\setcounter!|{footnote}{|\texttt{\ref{fn:2L3}}|}| 或 \!\addtocounter!|{footnote}{2}|。
\switchcolumn[0]*
This remedy, however, cannot change the stacking order of these two
footnotes of course.  Therefore, you need another trick with
\!\footnotemark! and \!\footnotetext! to stack the third left-column
footnote above the first right-column one.  More specifically, you can
solve the problem inserting
\switchcolumn
然而,这种方法当然无法改变这两个脚注的堆叠顺序。因此,您需要使用 \!\footnotemark! 和  \!\footnotetext! 来将第三个左列脚注堆叠在第一个右列脚注上面。具体来说,您可以通过插入以下内容来解决这个问题:
    
\end{paracol}
\begin{quote}
\!\footnotetext!\texttt{[\ref{fn:2R1}]}
    |{|\textit{text for the third left footnote}|}|
\end{quote}

somewhere between \!\footnote! commands for the second left-column and the
first right-column ones, e.g., at the end of the second left-column
paragraph, and attaching its mark to the appropriate word for the footnote
by \!\footnotemark!\texttt{[\ref{fn:2R1}]}, to have the following.

在第二个左列的 \!\footnote! 命令和第一个右列的 \!\footnote! 命令之间的某个位置,例如在第二个左列段落的末尾,并通过  \!\footnotemark!\texttt{[\ref{fn:2R1}]} 将其标记附加到脚注所对应的单词上,可以得到以下效果。
\Hrule
\begin{paracol}{2}
\fnpar{First left-column}{\footnote{First left-column
footnote.\label{fn:3L1}}}
\Fnpar{Second left-column}{\footnote{Second left-column
footnote.\label{fn:3L2}}}{
It is followed by \cs{footnotetext}\texttt{[\ref{fn:3L3}]}\marg{text}
and a \cs{switchcolumn}.}
\addtocounter{footnote}{1}
\footnotetext[\arabic{footnote}]{Third left-column footnote given by
\cs{footnotetext}\texttt{[\ref{fn:3L3}]}\marg{text} placed at the end of
the second left-column paragraph.\label{fn:3L3}}
\switchcolumn
\addtocounter{footnote}{1}
\Fnpar{First right-column}{\footnote[\arabic{footnote}]{First right-column
footnote whose number \ref{fn:3R1} is explicitly given by
\cs{footnote}\texttt{[\ref{fn:3R1}]}\marg{text}.\label{fn:3R1}}}{It is
followed by a \cs{switchcolumn*}.}
\addtocounter{footnote}{-1}
\switchcolumn*
Third and synchroized left-column paragraph\Dotfill\\
\Dotfill with a footnote whose mark
here\footnotemark[\arabic{footnote}]\Dotfill\\
\Dotfill is given by \!\footnotemark!\texttt{[\ref{fn:3L3}]}\Dotfill in
it.\\
It is followed by \!\addtocounter!|{footnote}{2}| and a \!\switchcolumn!.
\addtocounter{footnote}{1}
\switchcolumn
\fnpar{Second and synchronized right-column}{\footnote{Second right-column
footnote correctly following the first right-column one.\label{fn:3R2}}}
\fnpar{Third right-column}{\footnote{Third right-column
footnote.\label{fn:3R3}}}
\end{paracol}
\Hrule

Though this solution gives a good result, however, it has the following
two problems.  First, you have to explicitly specify the footnote number
through the optional arguments \oarg{num} of \!\footnote!,
\!\footnotetext! and \!\footnotemark!.  This problem is quite severe
because, for example, if you add a footnote somewhere preceding the
\env{paracol} environment in question, you have to modify all
\oarg{num} arguments of footnote-related commands in the environment.
This means that when the footnote addition is done in the first page of a
100-page document having \env{paracol} environments with explicitly numbered
footnotes in every page, you have to make the corrections for environments
in 99 pages.  The other a little bit less severe problem is that you have
to keep \counter{footnote} counter having correct value by
\!\setcounter!, \!\addtocounter! or \!\stepcounter! for footnotes following
those with explicit numbering so that their numbers are given by the
default action of \!\footnote!.


虽然这种解决方案可以得到一个很好的结果,但它存在以下两个问题。首先,您必须通过\!\footnote!、\!\footnotetext! 和 \!\footnotemark! 命令的可选参数\oarg{num}显式地指定脚注编号。这个问题非常严重,因为例如,如果您在所讨论的\env{paracol}环境之前的某个地方添加了一个脚注,您必须修改环境中所有脚注相关命令的\oarg{num}参数。这意味着当在一个具有每页都有显式编号脚注的\env{paracol}环境的100页文档的第一页中进行脚注添加时,您必须对99页中的环境进行更正。另一个稍微不那么严重的问题是,您必须通过 \!\setcounter!、\!\addtocounter! 或 \!\stepcounter! 保持\counter{footnote}计数器具有正确的值,以便对那些具有显式编号的脚注之后的脚注进行默认的编号。

To cope with these two problems, \textsf{paracol} provides you with the
\emph{starred} versions of \!\footnote! and its relatives as introduced in
Section~\ref{sec:ref-scfnote} and detailedly explained in the next
Section~\ref{sec:fnnp-starred}.

为了解决这两个问题,\textsf{paracol}为您提供了\emph{带星号}的 \!\footnote! 及其相关命令,如在第~\ref{sec:ref-scfnote} 节中介绍的,并在下一节~\ref{sec:fnnp-starred} 中详细解释。

% \subsection{Commands \cs{footnote*} and Relatives\hfill \cs{footnote*} 命令及相关命令}
% \label{sec:fnnp-starred}
% 
% \begin{description}
% \item[\Midx{\!\footnote!}\texttt{*}\oarg{|+|disp}\marg{text}]\mbox{}
% \Item[\Midx{\!\footnote!}\texttt{*}\oarg{|-|disp}\marg{text}]\mbox{}
% \Item[\Midx{\!\footnote!}\texttt{*}\oarg{disp}\marg{text}]\mbox{}\par
% The command is similar to its non-starred counterpart but the explicit
% numbering with the optional argument is done in \emph{self-relative} or
% \emph{base-displacement} style.  That is, if the optional argument has a
% leading `|+|' or `|-|',  the number given to the footnote is
% $f+\meta{disp}$ or $f-\meta{disp}$ respectively where $f$ is the value of
% \counter{footnote} counter, or in other words the number given to the last
% footnote\footnote{%
% If it is put by the ordinary \cs{footnote}.}.

% 该命令与其非星号版本类似,但是使用可选参数进行的显式编号是以\emph{自相对}或\emph{基准位移}的方式进行的。也就是说,如果可选参数以`|+|'或`|-|'开头,给予脚注的编号分别为$f+\meta{disp}$或$f-\meta{disp}$,其中$f$是\counter{footnote}计数器的值,或者换句话说,是给予最后一个脚注的编号\footnote{如果它是由普通的\cs{footnote}命令放置的。}。

% Otherwise, i.e., the optional argument is a number without |+|/|-| sign,
% the number given to the footnote is $b+\meta{disp}$ where $b$ is the base
% value of \counter{footnote} counter at \beginparacol{} for the environment
% in which the command appears, or in other words the number given to the
% last \Preenv{} footnote\footnote{
% 
% Or the last footnote in the previous \env{paracol} environment,
% etc.\label{fn:4L0}}.

否则,即可选参数是一个没有 |+|/|-| 符号的数字,则给定的脚注编号是 $b+\meta{disp}$,其中 $b$ 是 \beginparacol{} 处的 \counter{footnote} 计数器的基础值,用于包含该命令的环境,或者换句话说,给定的是最后一个 \Preenv{} 脚注\footnote{或者是前一个 \env{paracol} 环境中的最后一个脚注,等等。\label{fn:4L0}}。
% 
% In addition, unlike the non-starred version, this command updates
% \counter{footnote} counter with the number given to the footnote, i.e.,
% $f\gets f+\meta{disp}$, $f\gets f-\meta{disp}$ or $f\gets b+\meta{disp}$
% is performed, so that following \!\footnote! without explicit numbering
% option have numbers $f+1$, $f+2$ and so on with new $f$.

此外,与非星号版本不同,该命令使用给定的脚注编号更新\counter{footnote}计数器,即执行$f\gets f+\meta{disp}$、$f\gets f-\meta{disp}$或$f\gets b+\meta{disp}$,以便在没有显式编号选项的情况下,后续的 \!\footnote!命令具有编号$f+1$、$f+2$等,并更新$f$的值。
% \begin{itemize}
% \item
% If the optional argument is not provided, it is assumed that |[+1]| is
% given and thus \!\footnote!|*|\marg{text} acts as \!\footnote!\marg{text}.

如果没有提供可选参数,则假定提供了|[+1]|,因此 \!\footnote!|*|\marg{text}的作用等同于 \!\footnote!\marg{text}。
% \end{itemize}
% 
% \item[\Midx{\!\footnotemark!}\rm|*[|{[|+-|]}\meta{disp}{|]|}]\mbox{}\par
% This command is a mixture of its non-starred counterpart and
% \!\footnote!|*|.  That is the number for the footnote mark is calculated
% in the way of \!\footnote!|*| and \counter{footnote} counter is updated.
% 
这个命令是它的非星号版本和 \!\footnote!||的混合体。即脚注标记的编号是根据 \!\footnote!|*|的方式计算的,并且\counter{footnote}计数器会被更新。
% \item[\Midx{\!\footnotetext!}\rm|*[|{[|+-|]}\meta{disp}{|]|}\marg{text}]
% \mbox{}\par
% Without the optional argument |[|[|+-|]\meta{disp}|]|, this command does what
% \!\footnotetext!\marg{text} does but in addition increments
% \counter{footnote} counter before that.  With the optional argument, on
% the other hand, the number given to the footnote \meta{text} is calculated
% as done in \!\footnote!, but the \counter{footnote} counter is not
% updated.

如果没有提供可选参数 |[|[|+-|]\meta{disp}|]|,则此命令的作用与 \!\footnotetext!\marg{text} 相同,但在此之前会增加\counter{footnote}计数器的值。另一方面,如果提供了可选参数,那么给定给脚注\meta{text}的编号将按照 \!\footnote! 的方式计算,但\counter{footnote}计数器不会被更新。
% \end{description}
% 
% With these starred commands, you can produce the following using the
% base-displacement mechanism without worrying about the absolute value of
% \!\footnote! counter and its change.
% 
使用这些带星号的命令,您可以使用基础位移机制生成以下内容,而无需担心 \!\footnote! 计数器的绝对值及其变化。
% \Hrule
% \begin{paracol}{2}
% \tolerance5000\hbadness5000
% \fnpar{First left-column}{\footnote{First left-column
% footnote.\label{fn:4L1}}}
% \Fnpar{Second left-column}{\footnote{Second left-column
% footnote.\label{fn:4L2}}}{
% It is followed by \cs{footnotetext}|*[3]|\marg{text} and a
% \cs{switchcolumn}.}
% \footnotetext*[3]{Third left-column footnote given by
% \cs{footnotetext}|*[3]|\marg{text} placed at the end of
% the second left-column paragraph to have
% $\ref{fn:4L3}=\ref{fn:4L0}+3$.\label{fn:4L3}}
% \switchcolumn
% \Fnpar{First right-column}{\footnote*[4]{First right-column
% footnote whose number \ref{fn:4R1} is given by
% \cs{footnote}|*[4]|\marg{text} because
% $\ref{fn:4R1}=\ref{fn:4L0}+4$.\label{fn:4R1}}}{It is followed by a
% \cs{switchcolumn*}.}
% \switchcolumn*
% Third and synchronized left-column paragraph\Dotfill\\
% \Dotfill with a footnote whose mark
% here\footnotemark*[3]\Dotfill\\
% is given by \!\footnotemark!|*[3]| because $\ref{fn:4L3}=\ref{fn:4L0}+3$.
% It is followed by a \!\switchcolumn!.
% \switchcolumn
% \fnpar{Second and synchronized right-column}{\footnote*[5]{Second right-column
% footnote produced by \cs{footnote}|*[5]|\marg{text} because
% $\ref{fn:4R2}=\ref{fn:4L0}+5$.\label{fn:4R2}}}
% \fnpar{Third right-column}{\footnote{Third right-column
% footnote produced by \cs{footnote}\marg{text} because
% $\ref{fn:4R3}=\ref{fn:4R2}+1$.\label{fn:4R3}}}
% \end{paracol}
% \newpage
% 
% The other way to produce the same result except for the absolute footnote
% numbers is to use the self-relative mechanism and to exploit the progress
% of \counter{footnote} counter as follows.
% 
另一种产生相同结果的方法(除了绝对脚注编号)是使用自相对机制,并利用 \counter{footnote} 计数器的进展,方法如下:
% \Hrule
% \begin{paracol}{2}
% \tolerance5000\hbadness5000
% \fnpar{First left-column}{\footnote{First left-column
% footnote.\label{fn:5L1}}}
% \Fnpar{Second left-column}{\footnote{Second left-column
% footnote.\label{fn:5L2}}}{
% It is followed by \cs{footnotetext}|*|\marg{text} and a
% \cs{switchcolumn}.}
% \footnotetext*{Third left-column footnote given by
% \cs{footnotetext}|*|\marg{text} placed at the end of
% the second left-column paragraph because it follows the second footnote
% \ref{fn:5L2}.\label{fn:5L3}}
% \switchcolumn
% \Fnpar{First right-column}{\footnote{First right-column
% footnote whose number \ref{fn:5R1} is given by
% \cs{footnote}\marg{text} because
% $\ref{fn:5R1}=\ref{fn:5L3}+1$ and \cs{footnotetext*} for \ref{fn:5L3} lets
% \counter{footnote} have the value.\label{fn:5R1}}}{It is followed by a
% \cs{switchcolumn*}.}
% \switchcolumn*
% Third and synchronized left-column paragraph\Dotfill\\
% \Dotfill with a footnote whose mark
% here\footnotemark*[-1]\Dotfill\\
% is given by \!\footnotemark!|*[-1]| because $\ref{fn:5L3}=\ref{fn:5R1}-1$.
% It is followed by a \!\switchcolumn!.
% \switchcolumn
% \fnpar{Second and synchronized right-column}{\footnote*[+2]{Second
% right-column footnote produced by \cs{footnote}|*[+2]|\marg{text} because
% $\ref{fn:5R2}=\ref{fn:5L3}+2$.\label{fn:5R2}}}
% \fnpar{Third right-column}{\footnote{Third right-column
% footnote produced by \cs{footnote}\marg{text} because
% $\ref{fn:5R3}=\ref{fn:5R2}+1$.\label{fn:5R3}}}
% \end{paracol}
% \Hrule
% 
% It depends on the structure of your document which of the
% base-displacement and self-relative is better.  If your document has
% frequent switching between single- and multi-column text typesetting and
% thus the contents of a \env{paracol} environment is relatively small, the
% base-displacement is a good choice because you may concentrate on one
% base value of \counter{footnote} counter.  Otherwise, especially when your
% document consists of one single and large \env{paracol} environment, the
% base-displacement is almost equivalent to maintaining absolute values and
% thus the self-relative should be preferred.
% 

% 这取决于你的文档结构,基准位移和自相对哪个更好。如果你的文档经常在单列和多列文本排版之间切换,因此\env{paracol}环境的内容相对较小,那么基准位移是一个不错的选择,因为你可以专注于\counter{footnote}计数器的一个基准值。否则,特别是当你的文档由一个单独且较大的\env{paracol}环境组成时,基准位移几乎等同于维护绝对值,因此应该优先选择自相对方式。

% Note that if the last \!\footnote! or \!\footnotemark! in a \env{paracol}
% environment is starred, the command lets \counter{footnote} counter have
% some value smaller than that for the last stacked footnote.  For example, 
% if the second and third right-column footnotes \ref{fn:5R2} and
% \ref{fn:5R3} are omitted from the example above, the last footnote-related
% command will be \!\footnotemark!|*[-1]| which makes the counter at
% \Endparacol{} \ref{fn:5L3} rather than \ref{fn:5R1}.  You may not worry
% about this problem, however, because \Endparacol{} automatically maintains
% the counter letting it have $b+n$ where $n$ is the number of \!\footnote!
% and \!\footnotemark! in the environment, if the maintenance is ordered by
% the command \!\fncounteradjustment! which is automatically executed by
% \!\footnotelayout! with the argument |p| or |m|.
% 
请注意,如果在 \env{paracol} 环境中的最后一个 \!\footnote! 或 \!\footnotemark! 带有星号,那么该命令会使 \counter{footnote} 计数器的值小于最后一个堆叠脚注的值。例如,如果上面的示例中省略了第二个和第三个右列脚注 \ref{fn:5R2} 和 \ref{fn:5R3},那么最后一个与脚注相关的命令将是 \!\footnotemark!|*[-1]|,它使得在 \Endparacol{} 处的计数器为 \ref{fn:5L3} 而不是 \ref{fn:5R1}。然而,您可能不必担心这个问题,因为 \Endparacol{} 会自动维护计数器,使其为 $b+n$,其中 $n$ 是环境中 \!\footnote! 和  \!\footnotemark! 的数量,如果维护是由命令 \!\fncounteradjustment! 规定的,该命令会在 \!\footnotelayout! 中使用参数 |p| 或 |m| 自动执行。

% 
% \subsection{Page Break\hfill 分页}
% \label{sec:fnnp-pbreak}
% 
% When a \env{paracol} environment with footnotes lays across a page boundary,
% you could have some weird result even if the environment have just one
% \!\switchcolumn! as shown below.
% 
% 当带有脚注的\env{paracol}环境跨越页面边界时,即使该环境只有一个 \!\switchcolumn!,你可能会得到一些奇怪的结果,如下所示。
% \Hrule
% \begin{paracol}{2}
% First left-column paragraph \Dotfill\\
% \Dotfill with a footnote\footnote{First left-column
% footnote.\label{fn:6L1}}\Dotfill\\
% \Dotfill\\ \Dotfill\\ \Dotfill\\ \Dotfill\\ \Dotfill\\
% \Dotfill in it.
% \par
% Second left-column paragraph \Dotfill\\
% \Dotfill with a footnote\footnote{Second left-column
% footnote.\label{fn:6L2}}
% \Dotfill in it.
% \switchcolumn
% First right-column paragraph \Dotfill\\
% \Dotfill with a footnote\footnote{First right-column
% footnote weirdly placed here while the footnoted main text is in the
% previous page.\label{fn:6R1}}\Dotfill\\
% \Dotfill\\ \Dotfill\\ \Dotfill\\ \Dotfill\\ \Dotfill\\
% \Dotfill in it.
% \par
% Second right-column paragraph \Dotfill\\
% \Dotfill with a footnote\footnote{Second right-column
% footnote whose mark in the main text gives impression that footnote
% numbering jumps from \ref{fn:6L2} to \ref{fn:6R2}.\label{fn:6R2}}
% \Dotfill in it.
% \end{paracol}
% \Hrule
% 
% Since the part of the source |.tex| for this example above is
% fundamentally same as that in p.~\pageref{sec:fnnp} at the beginning of
% this Section~\ref{sec:fnnp}, footnotes are simply numbered in
% left-column-first manner without any tricks.  However it results in
% giving an impression that two paragraphs in each of both columns at the
% bottom of the last page have footnote marks of inconsecutive numbers
% \ref{fn:6L1} and \ref{fn:6R1} due to the second left-column paragraph and
% the footnote \ref{fn:6L2} in it.  More weirdly, the first right-column
% footnote \ref{fn:6R1} is not put in the last page where its mark is shown
% but is stacked below \ref{fn:6L2} in this page.
% 
由于上述示例的源代码部分与本节开头的第 \pageref{sec:fnnp} 页上的源代码基本相同,因此脚注的编号只是按照左列优先的方式进行,没有任何技巧。然而,由于第二个左列段落和其中的脚注 \ref{fn:6L2},在最后一页的两个列中的每个段落给人的印象是脚注标记的数字不连续,即 \ref{fn:6L1} 和 \ref{fn:6R1}。更奇怪的是,第一个右列的脚注 \ref{fn:6R1} 没有放在最后一页,而是在本页的 \ref{fn:6L2} 下方堆叠。

% The reason why this happens is that a footnote is not immediately put to
% the bottom of the page where its mark resides but to the page constructing
% at the time when the footnote is processed at the end of the paragraph in
% which the corresponding \!\footnote! (or \!\footnotetext!)
% occurs\footnote{%
% More accurately, the footnote is kept in a place in \TeX{} together with
% other preceding but still unprocessed footnotes and then \TeX{} examines
% them at the end of a paragraph in which a page break is found to decide
% whether each of them is included in the page just being completed.}.

这是因为脚注不会立即放置在其标记所在的页面底部,而是在处理包含相应的 \!\footnote!(或 \!\footnotetext!)的段落末尾时,放置在正在构建的页面上。\footnote{%
更准确地说,脚注与其他尚未处理的脚注一起保存在\TeX{}中的一个位置,然后当在一个段落末尾找到页面断页时,\TeX{}会检查这些脚注,决定是否将它们包含在刚完成的页面中。}

% Therefore, it may happen even in an ordinary single-column document or a
% \env{paracol}ed multi-column one with \Mcfnote{}s that a
% footnote is thrown to the page $p+1$ next to the page $p$ in which its
% mark is left, when the mark is placed around the bottom of the page
% $p$.

因此,即使在普通的单栏文档或使用\env{paracol}进行多栏排版的文档中,当脚注标记位于页面底部附近时,脚注可能会被放置在标记所在的页面$p$的下一页$p+1$上。

% This footnote placement mechanism becomes clearly visible in the example
% above in which the footnote \ref{fn:6R1} is processed {\em after} the
% second left-column paragraph is processed to complete the last page giving
% no chance to the footnote placed in the page\footnote{%
% 
% In fact, even \cs{footnote} for the footnote is processed after the page
% break in this case.}.

在上面的示例中,脚注\ref{fn:6R1}是在处理第二个左栏段落之后才处理的,这样才能完成最后一页,并且没有机会将脚注放置在该页上\footnote{实际上,在这种情况下,即使脚注的\cs{footnote}也是在分页后处理的。}。

% Therefore, the solution of this placement problem is to let the first
% right-column footnote processed {\em before} the page is broken by the
% progress of the left-column.  That is, in the solution shown below the
% author inserted \!\switchcolumn! after the first left-column paragraph to
% let the first right-column paragraph and its footnote are processed, and
% then did \!\switchcolumn! again after the right-column paragraph to go
% back to the left-column.

因此,解决此放置问题的方法是在左列的进展导致页面分割之前,让第一个右列的脚注被处理。也就是说,在下面的解决方案中,作者在第一个左列段落之后插入了 \!\switchcolumn!,以便处理第一个右列段落及其脚注,然后在右列段落之后再次进行 \!\switchcolumn!,回到左列。

% \Hrule
% \begin{paracol}{2}
% First left-column paragraph \Dotfill\\
% \Dotfill with a footnote\footnote{First left-column
% footnote.\label{fn:7L1}}\Dotfill\\
% \Dotfill\\ \Dotfill\\ \Dotfill\\ \Dotfill\\ \Dotfill\\ \Dotfill\\ \Dotfill\\
% \Dotfill\\ \Dotfill\\ \Dotfill\\ \Dotfill\\ \Dotfill\\ \Dotfill\\ \Dotfill\\
% \Dotfill\\
% \Dotfill in it.\\
% It is followed by a \!\switchcolumn!.
% \par\switchcolumn
% First right-column paragraph \Dotfill\\
% \Dotfill with a footnote\footnote{First right-column
% footnote which is now placed in this page where its mark \ref{fn:7R1}
% resides.\label{fn:7R1}}\Dotfill\\
% \Dotfill\\ \Dotfill\\ \Dotfill\\ \Dotfill\\ \Dotfill\\ \Dotfill\\ \Dotfill\\
% \Dotfill\\ \Dotfill\\ \Dotfill\\ \Dotfill\\ \Dotfill\\ \Dotfill\\ \Dotfill\\
% \Dotfill in it.\\
% It is followed by a \!\switchcolumn! to go back to the left column.
% \par\newpage\switchcolumn
% Second left-column paragraph \Dotfill\\
% \Dotfill with a footnote\footnote{Second left-column
% footnote whose number \ref{fn:7L2} follows the right-column footnote
% \ref{fn:7R1} in the last page.\label{fn:7L2}}
% \Dotfill in it.\\
% It is also followed by a \!\switchcolumn!.
% \switchcolumn
% Second right-column paragraph \Dotfill\\
% \Dotfill with a footnote\footnote{Second right-column
% footnote whose number \ref{fn:7R2} follows the left-column footnote
% \ref{fn:7L2}.\label{fn:7R2}}
% \Dotfill in it.
% \end{paracol}
% \Hrule
% 
% Unfortunately, this tactics does not always solve the problem.  If a
% left-column paragraph has a page break in it and a footnote before the
% break, doing \!\switchcolumn! after the paragraph is too late to let
% right-column footnotes reside in the page just having been broken, while
% inserting \!\switchcolumn! before the paragraph should cause incorrect
% stacking order.

不幸的是,这种策略并不能始终解决问题。如果左列段落中有一个分页,并且在分页之前有一个脚注,在段落之后执行 \!\switchcolumn! 命令太晚了,无法让右列的脚注位于刚分页的页面中,而在段落之前插入 \!\switchcolumn! 命令会导致错误的堆叠顺序。

% The remedy for this problem is similar to that shown in
% Section~\ref{sec:fnnp-multsc} to cope with multiple \!\switchcolumn! in a
% \env{paracol} environment.  Here it is shown a little bit more formally.
% Suppose we have a page in a \env{paracol} environment in which a page
% break occurs in $p_l$-th and $p_r$-th paragraphs in the left and right
% columns respectively.  Thus we have $p_l-1$ and $p_r-1$ completed
% paragraphs in each of both columns.  Let $n_l$ (resp.\ $n_r$) be the
% number of footnotes in the pre-break left-column (resp.\ right-column)
% paragraphs, and $m_l$ (resp.\ $m_r$) be the number of pre-break footnotes
% in the $p_l$-th (resp.\ $p_r$-th) paragraph.  Thus we have $n_l+m_l$
% (resp.\ $n_r+m_r$) footnotes in the left (resp.\ right) column of the page
% before the break.  The following construct assures that those footnotes
% are properly numbered and stacked at the bottom of the page.

解决这个问题的方法类似于第 \ref{sec:fnnp-multsc} 节所示的处理在 \env{paracol} 环境中出现多个 \!\switchcolumn! 命令的方法。这里稍微更加正式地展示一下。假设我们在 \env{paracol} 环境中有一个页面,在左列和右列中分别出现了第 $p_l$ 个和第 $p_r$ 个段落的分页。因此,在每个列中有 $p_l-1$ 和 $p_r-1$ 个已完成的段落。设 $n_l$(分别为 $n_r$)为分页前左列(分别为右列)段落中的脚注数量,$m_l$(分别为 $m_r$)为第 $p_l$(分别为第 $p_r$)个段落中分页前的脚注数量。因此,在分页前的页面左列(分别为右列)中有 $n_l+m_l$(分别为 $n_r+m_r$)个脚注。以下构造确保这些脚注在页面底部以正确的编号和堆叠方式显示。

% \begin{list}{}{\rightmargin\leftmargin \itemindent-.5\leftmargin
% \listparindent\itemindent \leftmargin1.5\leftmargin \parsep0pt}\it\item
% First to $(p_l-1)$-th paragraphs with $n_l$ footnotes in total given by
% {\rm\!\footnote!\marg{text}}.\par
% {\rm\!\footnotetext!|*{|{\it 1st footnote in $p_l$-th paragraph}|}|}\par
% \mbox{\qquad}\ldots\par
% {\rm\!\footnotetext!|*{|{\it$m_l$-th footnote in $p_l$-th paragraph}|}|}\par
% \!\switchcolumn!\par
% First to $(p_r-1)$-th paragraphs with $n_r$ footnotes in total given by
% {\rm\!\footnote!\marg{text}}.\par
% {\rm\!\footnotetext!|*{|{\it 1st footnote in $p_r$-th paragraph}|}|}\par
% \mbox{\qquad}\ldots\par
% {\rm\!\footnotetext!|*{|{\it$m_r$-th footnote in $p_r$-th paragraph}|}|}\par
% \!\switchcolumn!\par
% $p_l$-th paragraph whose first footnote mark is given by
% {\rm\!\footnotemark!|*[-|$(m_l{+}n_r{+}m_r{-1})$|]|}, while second to
% $m_l$-th ones are given by \!\footnotemark! without {\rm|*|} nor optional
% {\rm\oarg{num}}.  The first subsequent footnotes beyond the page break, if
% any, is given by {\rm\!\footnote!|*[+|$(n_r{+}m_r{+1})$|]|\marg{text}}
% while further subsequent ones are given by
% {\rm\!\footnote!\marg{text}}.\par
% \!\switchcolumn!\par
% $p_r$-th paragraph whose first footnote mark is given by
% {\rm\!\footnotemark!|*[-|$(m_r{+}k_l{-1})$|]|} where $k_l$ is the number
% of left-column footnotes beyond the break, while second to $m_r$-th ones
% are given by \!\footnotemark!.  The first subsequent footnotes beyond the
% page break, if any, is given by
% {\rm\!\footnote!|*[+|$(k_l{+1})$|]|\marg{text}}, while further subsequent
% ones are given by {\rm\!\footnote!\marg{text}}.
% \end{list}
% %
% The example shown in the next two pages is for the case of
% $p_l=p_r=n_l=n_r=m_l=m_r=k_l=2$.

下面两页中的示例是当 $p_l=p_r=n_l=n_r=m_l=m_r=k_l=2$ 时的情况。
% \newpage
% \Hrule
% \begin{paracol}{2}
% First left-column paragraph with two footnotes\break
% \mbox{}\Dotquad here\footnote{First left-column footnote given by
% \cs{footnote}\marg{text}.\label{fn;8L1}} by
% \!\footnote!\marg{text}\Dotfill\\
% \Dotquad and here\footnote{Second left-column footnote also given by
% \cs{footnote}\marg{text}.\label{fn:8L2}} also by
% \!\footnote!\marg{text}\Dotfill\\
% \Dotfill\\\Dotfill\\\Dotfill\\\Dotfill\\\Dotfill\\
% \Dotfill\\\Dotfill\\\Dotfill\\\Dotfill\\\Dotfill\\
% \Dotfill\\\Dotfill\\\Dotfill\\\Dotfill\\\Dotfill\\
% followed by a series of \!\footnotetext!|*|\marg{text} and then a
% \!\switchcolumn!.
% \footnotetext*{Third left-column footnote given by
% \cs{footnotetext*}\marg{text}.\label{fn:8L3}}
% \footnotetext*{Fourth left-column footnote given by
% \cs{footnotetext*}\marg{text}.\label{fn:8L4}}
% 
% \switchcolumn
% First right-column paragraph with two footnotes\break
% \mbox{}\Dotquad here\footnote{First right-column footnote given by
% \cs{footnote}\marg{text}.\label{fn;8R1}} by
% \!\footnote!\marg{text}\Dotfill\\
% \Dotquad and here\footnote{Second right-column footnote also given by
% \cs{footnote}\marg{text}.\label{fn:8R2}} also by
% \!\footnote!\marg{text}\Dotfill\\
% \Dotfill\\\Dotfill\\\Dotfill\\\Dotfill\\\Dotfill\\
% \Dotfill\\\Dotfill\\\Dotfill\\\Dotfill\\\Dotfill\\
% \Dotfill\\\Dotfill\\\Dotfill\\\Dotfill\\\Dotfill\\
% followed by a series of \!\footnotetext!|*|\marg{text} and then a
% \!\switchcolumn!.
% \footnotetext*{Third right-column footnote given by
% \cs{footnotetext*}\marg{text}.\label{fn:8R3}}
% \footnotetext*{Fourth right-column footnote given by
% \cs{footnotetext*}\marg{text}.\label{fn:8R4}}
% 
% \switchcolumn
% Second left-column paragraph across two pages\break
% \mbox{}\Dotquad with two pre-break footnotes\Dotfill\\
% \Dotquad here\footnotemark*[-5] by \!\footnotemark!|*[-5]|
% because $m_l+n_r+m_r-1=2+2+2-1=5$ and thus
% $\ref{fn:8L3}=\ref{fn:8R4}-5$\Dotfill\\
% \Dotquad and here\footnotemark{} by \!\footnotemark!\Dotfill\\
% \Dotfill\\\Dotfill\\\Dotfill\\\Dotfill\\\Dotfill\\
% \Dotfill\\\Dotfill\\\Dotfill\\\Dotfill\\\Dotfill\\
% \Dotfill\\\Dotfill\\\Dotfill\\\Dotfill\\\Dotfill\\
% \Dotfill\\\Dotfill\\
% \Dotquad and two post-break footnotes\Dotfill\\
% \Dotquad here\footnote*[+5]{Fifth left-column footnote given by
% \cs{footnote}|*[+5]| because $n_r+m_r+1=2+2+1=5$ and thus
% $\ref{fn:8L5}=\ref{fn:8L4}+5$.\label{fn:8L5}} by
% \!\footnote!|*[+5]|\marg{text}\Dotfill\\
% \Dotquad and here\footnote{Sixth left-column foootnote given by
% \cs{footnote}\marg{text}.\label{fn:8L6}} by
% \!\footnote!\marg{text}\Dotfill\\
% followed by a \!\switchcolumn!.
% 
% \switchcolumn
% Second right-column paragraph across two pages\break
% \mbox{}\Dotquad with two pre-break footnotes\Dotfill\\
% \Dotquad here\footnotemark*[-3] by \!\footnotemark!|*[-3]|
% because $m_r+k_l-1=2+2-1=3$ and thus
% $\ref{fn:8R3}=\ref{fn:8L6}-3$\Dotfill\\
% \Dotquad and here\footnotemark{} by \!\footnotemark!\Dotfill\\
% \Dotfill\\\Dotfill\\\Dotfill\\\Dotfill\\\Dotfill\\
% \Dotfill\\\Dotfill\\\Dotfill\\\Dotfill\\\Dotfill\\
% \Dotfill\\\Dotfill\\\Dotfill\\\Dotfill\\\Dotfill\\
% \Dotfill\\\Dotfill\\
% \Dotquad and two post-break footnotes\Dotfill\\
% \Dotquad here\footnote*[+3]{Fifth right-column footnote given by
% \cs{footnote}|*[+3]| because $k_l+1=3$ and thus
% $\ref{fn:8R5}=\ref{fn:8R4}+3$.\label{fn:8R5}} by
% \!\footnote!|*[+3]|\marg{text}\Dotfill\\
% \Dotquad and here\footnote{Sixth right-column foootnote given by
% \cs{footnote}\marg{text}.\label{fn:8R6}} by
% \!\footnote!\marg{text}\Dotfill.
% \end{paracol}
% \Hrule
% 
% Note that though the remedy works well as shown above, it is not a good
% idea to do that when you are writing draft versions of your document
% because page break points go up and down by your modifications to the
% document.  Therefore, it is recommended to put all footnotes by
% non-starred \!\footnote! until your document becomes perfect except for
% footnote numbering and placement and then to adjust them by the techique
% described in this section.

请注意,尽管上述方法可以很好地解决问题,但在撰写文档的草稿版本时,不建议这样做,因为页面分页点会根据您对文档的修改而上下移动。因此,建议您在文档除了脚注编号和位置之外的其他方面完善之前,使用非星号形式的 \!\footnote! 命令放置所有脚注,然后再使用本节描述的技巧进行调整。
 
% \section{Two-Sided Typesetting and Parallel-Paging\hfill 双面排版和并列分页}
% \label{sec:ppts}
% \changes{v1.3-2}{2013/09/17}
%	{Add the section ``Two-Sided Typesetting and Parallel-Paging''.}
% \changes{v1.3-4}{2013/09/17}
%	{Add the section ``Two-Sided Typesetting and Parallel-Paging''.}
% \changes{v1.3-5}{2013/09/17}
%	{Add the section ``Two-Sided Typesetting and Parallel-Paging''.}
% 
% This and the next section are typeset with \Uidx{\!\twosided!} enabling
% features |p|, |c| and |m| and also |b| for a part of the next section.
% The effect of |p| feature can be seen by the \oddeven{left}{right}, or in
% other word inside, margin of this \oddeven{odd}{even}-numbered page is
% narrower than that of the previous pages because the author reduced the
% effective \oddeven{left}{right} side margin being calculated from
% \oddeven{\cs{oddsidemargin}}{\cs{evesidemargin}}

% 这一节和下一节使用 \Uidx{\!\twosided!} 启用特性|p|、|c|和|m|,以及部分下一节的特性|b|进行排版。通过查看此\oddeven{奇数}{偶数}页的内边距,可以看到|p|特性的效果,即比前面的页面的内边距更窄,因为作者减小了从 \oddeven{\cs{oddsidemargin}}{\cs{evesidemargin}} 计算出的有效 \oddeven{left}{right} 边距。

% 
% \SpecialIndex{\oddsidemargin}
% \SpecialIndex{\evensidemargin}
% 
% by 75\,\%\footnote{
% This document itself does not have |twoside| option in its
% \!\documentclass! but the inconsistency between the option and
% \!\twosided! is not visible because \!\pagestyle! is |plain|.}.
% This setting makes the \oddeven{right}{left} side or outside margin of
% this page enlarged by 125\,\%, as well as the \oddeven{left}{right} side
% and outside margin of the next \oddeven{even}{odd}-numbered page specified
% by \oddeven{\cs{evensidemargin}}{\cs{oddsidemargin}}.

% 由于75,%的设置\footnote{%
% 此文档本身在其 \!\documentclass! 中没有 |twoside| 选项,但是选项与 \!\twosided! 之间的不一致之处并不可见,因为 \!\pagestyle! 是 |plain|。}。
% 这个设置使得本页的\oddeven{右}{左}边缘或外侧边缘增大了125,%,以及下一页的\oddeven{左}{右}边缘和外侧边缘,由 \oddeven{\cs{evensidemargin}}{\cs{oddsidemargin}} 指定的\oddeven{偶数}{奇数}页。

% Next, we see the effects of |c| and |m| features by the \env{paracol}
% environment below for which \Uidx{\!\columnratio!}|{0.6}| and
% \Uidx{\!\marginparthreshold!}|{0}| are declared to make the \emph{inside}
% columns (\oddeven{left}{right} ones in \oddeven{odd}{even}-numbered pages)
% are wider than the \emph{outside} ones and all marginal notes go to
% outside (\oddeven{right}{left} in \oddeven{odd}{even}-numbered pages)
% margins.

% 接下来,我们看到以下 \env{paracol} 环境通过 \Uidx{\!\columnratio!}|{0.6}| 和  \Uidx{\!\marginparthreshold!}|{0}| 来实现 |c| 和 |m| 特性的效果,使得\emph{内部}列(在\oddeven{奇数}{偶数}页中的\oddeven{左}{右}列)比\emph{外部}列更宽,并且所有的边注都放在外侧边缘(在\oddeven{奇数}{偶数}页中的\oddeven{右}{左}边缘)。

% \columnratio{0.6}\marginparthreshold{0}
% 
% \par\Hrule
% \begin{paracol}{2}
% \switchcolumn
% \footnotetext*{Since the author is temporarily disabling the warning from
% marginal note placement mechanism of \LaTeX, pushing down the second
% marginal note from column-1 is silently performed when you process this
% document.}
% \switchcolumn
% This line\Marginpar{First marginal note from column-0.} of the first
% paragraph of the inside column-0 has a marginal note.  Now the author puts
% a few dummy lines to keep a space below the marginal note.\\
% \Dotfill\\ \Dotfill\\ \Dotfill\\ \Dotfill\\ \Dotfill\\ \Dotfill\\
% \Dotfill\\ \Dotfill\\ \Dotfill\\ \Dotfill\\ \Dotfill\\ \Dotfill\par
% 
% This line\Marginpar{Second marginal note from column-0.} of the second
% paragraph of the inside column-0 also has a marginal note.  Now the author
% puts a few dummy lines again but this time to go down to the bottom of the
% page.\\
% \Dotfill\\ \Dotfill\\ \Dotfill\\ \Dotfill\\ \Dotfill\\ \Dotfill\\
% \Dotfill\\ \Dotfill\\ \Dotfill\\ \Dotfill\\ \Dotfill\\ \Dotfill\\
% \Dotfill\\ \Dotfill\par
% 
% This is the third paragraph of the inside column-0 having a page break in
% it.  Since shortly we will be in an \oddeven{even}{odd}-numbered page
% \pageref{page:ppts2} (now), this wider column\Marginpar{Third marginal
% note from column-0} is now \oddeven{right}{left} one keeping it
% inside, while the marginal note given in the first line of this page goes
% to \oddeven{left}{right} and outside.  Now we will have a \!\switchcolumn!
% below this paragraph to go to the column-1 and back to the previous page
% \pageref{sec:ppts}.\label{page:ppts2}
% \switchcolumn
% \it
% This is the first paragraph in the narrower, italicized and outside
% column-1.  In this paragraph, we shortly have a marginal note, italicized
% too, which goes to the outside margin shared by all marginal notes from
% both columns.\Marginpar{\it First marginal note from column-1.}  The
% marginal note given here is placed its natural position and its first line
% is aligned to the first line of the second sentence of this paragraph by
% exploitation of the space between two marginal notes from the column-0,
% though we already have had three notes from the column.
% 
% Now\Marginpar{\it Second marginal note from column-1.} the author puts
% another marginal note whose first line would be aligned to that of this
% paragraph, but it is pushed down below the second marginal note from the
% column-0 because two notes conflict with each other over the
% space\footnotemark*[+0].  Note that since the note from this column is given
% \emph{after} that from the column-0 was given, the conflict is solved
% pushing the note from this column down rather than that from the
% column-0.  Now the author puts a few dummy lines to go to the second last
% line of this page.\\
% \Dotfill\\ \Dotfill\\ \Dotfill\\ \Dotfill\\ \Dotfill\\
% \Dotfill\par
% 
% This is the third paragraph of the outside column-1, which becomes
% \oddeven{left}{right} shortly by the page break.\Marginpar{\it Third
% marginal note from column-1.}  The third marginal note is given in the
% first line of this page, but it is pushed down again due to the conflict
% with the note from the column-0.
% \end{paracol}
% \Hrule
% 
% Note that the position of the last marginal note in the \env{paracol}
% \Marginpar{Marginal note given after \env{paracol} environment is closed.}
% environment which we just have closed affects the marginal note placement
% in \postenv.  For example, the marginal note given in the first line of
% this paragraph is pushed down.

% 请注意,在我们刚刚关闭的\env{paracol}环境中,最后一个边注的位置会影响\postenv 中的边注位置。例如,给出在本段落第一行的边注会被推下去。

% \ifodd\value{page}
% We will see a few examples of \parapag{}ing shortly, but before that we
% will have an intentional black page to make the first page of the example
% odd-numbered to avoid you have an impression that its layout is
% incorrect\footnote{%
% At least the author himself had such impression without the blank page.}
% because if it were in an even page you would see the {\em outside\/} third
% and fourth supplementary {\em columns\/} at first.

% \ifodd\value{page}
% 不久我们将看到一些\parapag{}的例子,但在此之前,我们将有一个有意留白的页面,使示例的第一页成为奇数页,以避免给您一种布局错误的印象\footnote{至少在没有空白页面的情况下,作者本人也有这样的印象。},因为如果它在偶数页,您将首先看到第三和第四个辅助{\em 列}的{\em 外侧}。

% \newpage\vspace*{\fill}\centerline{(intentionally blanked page)}\vfill
% 
% \else
% From the next page, we will see a few examples of \parapag{}ing.

从下一页开始,我们将看到一些\parapag{}的例子。
% \fi
% 
%%%% 有些问题
% 
% \newpage
% \subsection{Example of Paired Parallel-Paging\hfill 并列分页的示例}
% \label{sec:ppts-paired}
% 
% Shortly we will start a \env{paracol} environment by \beginparacol|[2]{4}|
% having four columns but two for each of left and right \paired{}
% \parapag{}es.  Since the author declares \!\columnratio!|{0.6}[0.5]|, the
% columns in left pages are made unbalanced while those in right pages are
% balanced.


% 不久我们将通过 \beginparacol|[2]{4}|开始一个\env{paracol}环境,该环境有四列,但是每个左右\paired{}\parapag{}中有两列。由于作者声明了 \!\columnratio!|{0.6}[0.5]|,左页中的列是不平衡的,而右页中的列是平衡的。

% \columnratio{0.6}[0.5]
% \par\Hrule
% \begin{paracol}[2]{4}
% This is the first paragraph of the leftmost column-0,
% \Marginpar{Marginal note from column-0.}
% whose first line has a marginal note placed in the right margin because
% the setting of \!\marginparthreshold! being 0 is still effective and we
% are in the odd-numbered page \pageref{sec:ppts-paired}.  Now we
% have a \!\switchcolumn! to the next column-1.
% 
% \switchcolumn
% \begin{Hfuzz}{1.1pt}\it
% This is the first paragraph of the second and right column-1 in the left
% \parapag{}e.  We shortly give an italicized mar\-gin\-al note carefully, so
% that it does not conflict with the marginal note from the column-0.
% \Marginpar{\it Marginal note from column-1.}
% That is, now the author puts the note.  Now we
% have a \!\switchcolumn! to the next column-2.
% \end{Hfuzz}
% \footnotetext*{This footnote is put in the left \parapag{}e together with
% another footnote below given in the column-2 in the right \parapag{}e.
% \label{fn:ppts-paired1}}
% 
% \switchcolumn
% \begingroup\sf
% This is the first paragraph of the column-2 being the left column of the
% right \parapag{}e.  Though we are in a page different from that column-0
% and 1 reside in, this page is still numbered \pageref{sec:ppts-paired}
% because the left and right page is \paired.  Therefore, the left margin of
% this page is narrower than the right margin because the page number is
% odd.
% 
% \footnotetext*{This footnote is \emph{not} put in the right \parapag{}e
% though it is given in the column-2 in the right \parapag{}e and thus its
% reference is in the column, of course.\label{fn:ppts-paired2}}
% 
% You have to notice
% \Marginpar{\sf Marginal note from column-2.}
% the first paragraph does not start from the page top
% but above it we have some space of exactly same size as the \preenv{}
% shown in the left \parapag{}e.  Therefore, the top of the first paragraphs
% in all columns are aligned.  The marginal note given in the first line of
% this paragraph goes to the right margin of this page because of the
% \!\marginparthreshold! setting and the parity of this page.  Now we have a
% \!\switchcolumn! to the next column-3.
% \par\endgroup
% \begin{figure*}\nosv
% \def\arraystretch{0.8}
% \centerline{\begin{tabular}[b]{|c|}\hline
%     \hbox to.9\textwidth{}\\
%     \sf page-wise figure given in column-2\\
%     \\\hline
%     \end{tabular}}
% \caption{A Page-Wise Figure}
% \end{figure*}
% 
% \switchcolumn
% \begingroup\sl
% This is the first paragraph
% \Marginpar{\sl Marginal note from column-3.}
% in the last rightmost column-3 whose width is equal to that of the column-2.
% The marginal note given in the first line goes to right and does not
% conflict with that from the column-2.  We are now going back to the
% column-0 by a {\rm\!\switchcolumn!|*|} with a \mctext.
% \endgroup
% 
% \switchcolumn*[\subsection*{A Spanning Text: though this is wider than the
% page width, this text does not span the boundary between the left and
% right parallel-pages.}]
% 
% We have come back to this column-0.  The space above the \mctext{} is due
% to the \sync{}ation because two paragraphs in the column-2 are
% significantly taller in total than the paragraphs in other columns.  As
% the spanning text itself says, it cannot extend to the right \parapag{}e.
% The author puts dummy lines to go to the page bottom.\\
% \Dotfill\\ \Dotfill\\ \Dotfill\\ \Dotfill\\ \Dotfill\\ \Dotfill\\
% \Dotfill\\ \Dotfill\\ \Dotfill\\ \Dotfill\\ \Dotfill\par
% 
% Now we will have a page break shortly.  You could be surprised by seeing
% this column is not in the left \parapag{}e after the break but in the
% right one.  This is because the feature |c| is enabled to swap not only
% columns in a page but also the left and right \paired{} \parapag{}es when
% they are even-numbered.  The other feature |p| makes the left outside
% margins of this right and the previous left pages wider than the right
% inside margins.\label{page:ppts-paired2}
% 
% \switchcolumn
% \begingroup\it
% We have restarted this column-1.  This paragraph has a
% footnote\footnotemark*[-1] as shown below.\\
% \Dotfill\\ \Dotfill\\ \Dotfill\\ \Dotfill\\ \Dotfill\\ \Dotfill\\
% \Dotfill\\ \Dotfill\\ \Dotfill\\ \Dotfill\\ \Dotfill\\ \Dotfill\\
% \Dotfill\\ \Dotfill\par
% 
% After the page break below, this column also goes to the right page
% together with the column-0
% \Marginpar{\it Another marginal note from column-1.}
% and is placed outside (left) in the page, as well as the marginal note
% in this right page but in the outside margin.
% \par\endgroup
% 
% \switchcolumn
% \begingroup\sf
% We have a few other materials not shown in right \parapag{}es.  The space
% above this paragraph is for the \mctext{} placed in the left \parapag{}e.
% The \Scfnote{} given here\footnotemark{} is also not in this page but in
% the left.  Finally, the author has put a page-wise figure spanning columns
% just before \!\switchcolumn! by which we left this column, but it will be
% in the right page \pageref{page:ppts-paired2} together with column-0 and
% 1.\\
% \Dotfill\\ \Dotfill\\ \Dotfill\\ \Dotfill\\ \Dotfill\\ \Dotfill\\
% \Dotfill\\ \Dotfill\\ \Dotfill\par
% 
% Though the footnote numbered \ref{fn:ppts-paired2} goes to the left page,
% its space and that of \ref{fn:ppts-paired1} make this and the next columns
% shorter in the previous page.  Similarly, we have a space above for the
% page-wise figure shown in the right page.
% \par\endgroup
% 
% \switchcolumn
% \begingroup\sl
% As expected, this line is aligned to the first line of the paragraph in
% the column-2 as well as those in column-0 and 1.  It is also consistent
% the first lines including that of this paragraph are not indented because
% the \mctext{} is given by {\rm\!\subsection!|*|} which makes first
% paragraphs unindented.\\
% \Dotfill\\ \Dotfill\\ \Dotfill\\ \Dotfill\\ \Dotfill\\ \Dotfill\\
% \Dotfill\\ \Dotfill\\ \Dotfill\\ \Dotfill\\ \Dotfill\par
%
% After the page break we will have shortly, this column becomes the
% leftmost in the left \parapag{}e, as you are seeing now,
% \Marginpar{\sl Another marginal note from column-3.}
% but still outermost as well as the marginal note in the outside left
% margin.
% \endgroup
% \end{paracol}
% \Hrule
% 
% Now you are seeing yet another material placed only in the page in which
% the column-0 resides and thus being the right page now, i.e., this
% paragraph and the next one in the \postenv.  You might be disappointed by
% the fact the \emph{outside} pages, i.e., left in this page
% \pageref{page:ppts-paired2} and right in the previous page
% \pageref{sec:ppts-paired}, cannot have \pwstuff{} but it is what the
% author can do now for the version 1.3 and thus you have to wait some
% future versions in which the author could devise a mechanism to exploit
% the corresponding space in the pages\footnote{%
% You might complain the immaturity of \parapag{}ing and might claim that it
% should be included in \Paracol{} after the author implements the
% mechanism.  In fact the author himself is frustrated current features of
% \parapag{}ing but he dared to release the version 1.3 knowing that there
% are people who happily typeset their \parapag{}ed documents with the
% current limited features.}.

现在您正在看到的是仅放置在列-0所在页面中的另一个材料,因此现在是右侧页面,即本段和下一个段落在 \postenv 中。您可能会对这样一个事实感到失望,即\emph{外部}页面,即本页的左侧(\pageref{page:ppts-paired2}页)和前一页的右侧(\pageref{sec:ppts-paired}页),无法使用 \pwstuff{},但这是作者目前版本1.3能做的,因此您必须等待未来的版本,届时作者可能会设计一种机制来利用页面上的相应空间\footnote{%
您可能会对 \parapag{} 的不成熟感到不满,并声称作者应该在实现该机制后将其包含在 \Paracol{} 中。实际上,作者自己对当前 \parapag{} 的功能感到沮丧,但他还是敢于发布版本1.3,因为他知道有人愉快地使用当前有限的功能来排版他们的 \parapag{} 文档。}。

% In addition, you might think it is weird that the |c| feature of
% \!\twosided! swaps columns \emph{and} paired pages.  However this swapping
% is a natural consequence of the combination of \cswap{} and \paired{}
% \parapag{}ing.  Therefore, you can simply disable the |c| feature (maybe
% together with other features) to have more intuitive results.

此外,您可能会觉得奇怪的是,\!\twosided! 命令的|c|功能交换了列\emph{和}配对的页面。然而,这种交换是\cswap{}和\paired{}\parapag{}的组合的自然结果。因此,您可以简单地禁用|c|功能(可能与其他功能一起禁用),以获得更直观的结果。

% In the next Section~\ref{sec:ppts-npaired}, you will see another kind of
% \parapag{}ing namely \npaired{} one.  Before that, we need a blank page to
% let the \npaired{} \parapag{}ing start from an even-numbered page so that
% a left and right page pair comprises a double spread.  A short remark on
% the blank next page is that it does not have a right counterpart
% \parapag{}e because the page is outside \env{paracol} environments and does
% not have any portion from the environments\footnote{%
% To illustrate this fact, the author dares to put a real blank page rather
% than stepping the \counter{page} counter.}.

在接下来的第~\ref{sec:ppts-npaired}节中,你将看到另一种\parapag{}分页方式,即\npaired{}分页。在此之前,我们需要一个空白页,以便让\npaired{} \parapag{}从偶数页开始,这样左右的页面对就构成一个双页展开。关于空白的下一页的一个简短说明是,它没有右侧对应的\parapag{},因为该页位于\env{paracol}环境之外,并且不包含来自这些环境的任何部分\footnote{为了说明这个事实,作者敢于放置一个真正的空白页,而不是增加\counter{page}计数器的值。}。

% \newpage\vspace*{\fill}\centerline{(intentionally blanked page)}\vfill
% 
% \newpage
% \subsection{Example of Non-Paired Parallel-Paging\hfill 非并列分页的示例}
% \label{sec:ppts-npaired}
% 
% This and following three pages are to show an example of \npaired{}
% \parapag{}ing, in which the author keeps the setting of \!\twosided!,
% \!\columnratio! and \!\marginparthreshold! unchanged.
% The arguments of \beginparacol{} for column population are also unchanged
% to have $2+2$ configuration, but the first argument is followed by |*| for
% \npaired{} typesetting.  That is, the environment below starts by
% \beginparacol|[2]*{4}|.  The contents of the environment is also almost
% same as the previous Section~\ref{sec:ppts-paired}, while
% \Emph{bold-faced} words show the difference from the \paired{}
% typesetting.

% 这页和接下来的三页是为了展示\npaired{}\parapag{}的示例,其中作者保持了 \!\twosided!、\!\columnratio! 和 \!\marginparthreshold! 的设置不变。用于列填充的\beginparacol{}的参数也保持不变,以获得$2+2$的配置,但是第一个参数后面跟着|*|表示进行\npaired{}排版。也就是说,下面的环境通过\beginparacol|[2]{4}|开始。环境的内容与前面的第\ref{sec:ppts-paired}节几乎相同,但是\Emph{加粗的}单词显示了与\paired{}排版的区别。

% \columnratio{0.6}[0.5]
% \par\Hrule
% \begin{paracol}[2]*{4}
% This is the first paragraph of the leftmost column-0,
% \Marginpar{Marginal note from column-0.}
% whose first line has a marginal note placed in the \Emph{left} margin
% because the setting of \!\marginparthreshold! being 0 is still effective
% and we are in the \Emph{even}-numbered page
% \Emph{\pageref{sec:ppts-npaired}}.  Now we have a \!\switchcolumn! to the
% next column-1.
% 
% \switchcolumn
% \begingroup\it
% This is the first paragraph of the second and right column-1 in the left
% \parapag{}e.  We shortly give an italicized mar\-gin\-al note carefully, so
% that it does not conflict with the marginal note from the column-0.
% \Marginpar{\it Marginal note from column-1.}
% That is, now the author puts the note.  Now we
% have a \!\switchcolumn! to the next column-2.
% \par\endgroup
% \footnotetext*{This footnote is put in the left \parapag{}e together with
% another footnote below given in the column-2 in the right \parapag{}e.
% \label{fn:ppts-npaired1}}
% 
% \switchcolumn
% \begingroup\sf\label{page:ppts-npaired1r}
% This is the first paragraph of the column-2 being the left column of the
% right \parapag{}e.  \Emph{Since we are in the page next to} that column-0
% and 1 reside in, this page is numbered \Emph{\pageref{page:ppts-npaired1r}}
% because the left and right page is \Emph{\npaired}.  Therefore, the left
% margin of this page is narrower than the right margin because the page
% number is odd.
% 
% \footnotetext*{This footnote is \emph{not} put in the right \parapag{}e
% though it is given in the column-2 in the right \parapag{}e and thus its
% reference is in the column, of course.\label{fn:ppts-npaired2}}
% 
% You have to notice
% \Marginpar{\sf Marginal note from column-2.}
% the first paragraph does not start from the page top
% but above it we have some space of exactly same size as the \preenv{}
% shown in the left \parapag{}e.  Therefore, the top of the first paragraphs
% in all columns are aligned.  The marginal note given in the first line of
% this paragraph goes to the right margin of this page because of the
% \!\marginparthreshold! setting and the parity of this page.  Now we have a
% \!\switchcolumn! to the next column-3.
% \par\endgroup
% \begin{figure*}\nosv
% \def\arraystretch{0.8}
% \centerline{\begin{tabular}[b]{|c|}\hline
%     \hbox to.9\textwidth{}\\
%     \sf page-wise figure given in column-2\\
%     \\\hline
%     \end{tabular}}
% \caption{A Page-Wise Figure}
% \end{figure*}
% 
% \switchcolumn
% \begingroup\sl
% This is the first paragraph
% \Marginpar{\sl Marginal note from column-3.}
% in the last rightmost column-3 whose width is equal to that of the column-2.
% The marginal note given in the first line goes to right and does not
% conflict with that from the column-2.  We are now going back to the
% column-0 by a {\rm\!\switchcolumn!|*|} with a \mctext.
% \endgroup
% 
% \switchcolumn*[\subsection*{A Spanning Text: though this is wider than the
% page width, this text does not span the boundary between the left and
% right parallel-pages.}]
% 
% We have come back to this column-0.  The space above the \mctext{} is due
% to the \sync{}ation because two paragraphs in the column-2 are
% significantly taller in total than the paragraphs in other columns.  As
% the spanning text itself says, it cannot extend to the right \parapag{}e.
% The author puts dummy lines to go to the page bottom.\\
% \Dotfill\\ \Dotfill\\ \Dotfill\\ \Dotfill\\ \Dotfill\\ \Dotfill\\
% \Dotfill\\ \Dotfill\par
% 
% Now we will have a page break shortly.  You \Emph{will not} be surprised
% by seeing this column \Emph{is still in the left \parapag{}e after the
% break.}  This is because the feature |c| is \Emph{not effective in
% \npaired{} \parapag{}ing.}  The other feature |p| \Emph{consistently makes
% the left outside margins of this and the previous page in which this
% column resides} wider than the right inside margins.
% \label{page:ppts-npaired2}
% 
% \switchcolumn
% \begingroup\it
% We have restarted this column-1.  This paragraph has a
% footnote\footnotemark*[-1] as shown below.\\
% \Dotfill\\ \Dotfill\\ \Dotfill\\ \Dotfill\\ \Dotfill\\ \Dotfill\\
% \Dotfill\\ \Dotfill\\ \Dotfill\\ \Dotfill\\ \Dotfill\par
% 
% After the page break below, this column also \Emph{stays in the left page}
% together with the column-0
% \Marginpar{\it Another marginal note from column-1.}
% and is placed \Emph{inside (right)} in the page, as well as the marginal
% note in this \Emph{left} page \Emph{still} in the outside margin.
% \par\endgroup
% 
% \switchcolumn
% \begingroup\sf
% We have a few other materials not shown in right \parapag{}es.  The space
% above this paragraph is for the \mctext{} placed in the left \parapag{}e.
% The \Scfnote{} given here\footnotemark{} is also not in this page but in
% the left.  Finally, the author has put a page-wise figure spanning columns
% just before \!\switchcolumn! by which we left this column, but it will be
% in the \Emph{left} page \Emph{\pageref{page:ppts-npaired2}} together with
% column-0 and 1.\\
% \Dotfill\\ \Dotfill\\ \Dotfill\\ \Dotfill\\ \Dotfill\\ \Dotfill\par
% 
% Though the footnote numbered \Emph{\ref{fn:ppts-npaired2}} goes to the
% left page, its space and that of \Emph{\ref{fn:ppts-npaired1}} make this
% and the next columns shorter in the previous page.  Similarly, we have a
% space above for the page-wise figure shown in the \Emph{left} page.
% \par\endgroup
% 
% \switchcolumn
% \begingroup\sl
% As expected, this line is aligned to the first line of the paragraph in
% the column-2 as well as those in column-0 and 1.  It is also consistent
% the first lines including that of this paragraph are not indented because
% the \mctext{} is given by {\rm\!\subsection!|*|} which makes first
% paragraphs unindented.\\
% \Dotfill\\ \Dotfill\\ \Dotfill\\ \Dotfill\\ \Dotfill\\ \Dotfill\\
% \Dotfill\\ \Dotfill\par
%
% After the page break we will have shortly, this column \Emph{is kept being
% the rightmost in the right \parapag{}e}, as you are seeing now,
% \Marginpar{\sl Another marginal note from column-3.}
% \Emph{and} still outermost as well as the marginal note in the outside
% \Emph{right} margin.
% \endgroup
% \end{paracol}
% \Hrule
% 
% As the \postenv{} in Section~\ref{sec:ppts-paired} is, this paragraph
% being the \postenv{} of the \npaired{} \parapag{}es appears only in the
% \parapag{}e in which the column-0 belongs to, and thus in the left
% \parapag{}e in this case.

与第\ref{sec:ppts-paired}节中的\postenv{}一样,本段作为\npaired{}个\parapag{}的\postenv{},只出现在列-0所属的\parapag{}中,因此在这种情况下是在左侧的\parapag{}中。
 

\backgroundcolor{t}[rgb]{0.7,0,0}
\backgroundcolor{b}[rgb]{0.8,0.6,0}
\backgroundcolor{l}[rgb]{0,0,0.7}
\backgroundcolor{r}[rgb]{0,0.7,0}
\backgroundcolor{c[0]}[rgb]{1,0.8,1}
\backgroundcolor{c[1]}[rgb]{1,1,0.8}
\backgroundcolor{g}[rgb]{0.8,1,1}
\backgroundcolor{f}[rgb]{0.8,0,1}
\backgroundcolor{n}[rgb]{0.8,0.6,1}
\backgroundcolor{p}[rgb]{0.8,1,0.6}
\backgroundcolor{s}[rgb]{0.8,0.8,0.8}
\pagerim5pt

\section{Examples of Background Painting\hfill 背景绘制的示例}
\label{sec:bgpaint}
\subsection{Fundamental Painting}
\label{sec:bgpaint-fund}
\twosided[pcm]

As you undoubtedly notice, this page and a few pages following it are
colorfully painted.  For this and the next three pages, the author
declared the \bground{} color of each region as follows.

正如你无疑注意到的,本页和随后的几页都是色彩斑斓的。对于这四页,作者将每个区域的\bground{}颜色声明如下。
\begin{itemize}\item[]
|\backgroundcolor{t}[rgb]{0.7,0,0}       % dark red for top margin|\\
|\backgroundcolor{b}[rgb]{0.8,0.6,0}     % dark orange for bottom margin|\\
|\backgroundcolor{l}[rgb]{0,0,0.7}       % dark blue for left margin|\\
|\backgroundcolor{r}[rgb]{0,0.7,0}       % dark green for right margin|\\
|\backgroundcolor{c[0]}[rgb]{1,0.8,1}    % pink for colunmn-0|\\
|\backgroundcolor{c[1]}[rgb]{1,1,0.8}    % cream yellow for column-1|\\
|\backgroundcolor{g}[rgb]{0.8,1,1}       % light blue for the gap|\\
|\backgroundcolor{f}[rgb]{0.8,0,1}       % purple for page-wise floats|\\
|\backgroundcolor{n}[rgb]{0.8,0.6,1}     |
    |% light purple for page-wise footnotes|\\
|\backgroundcolor{p}[rgb]{0.8,1,0.6}     |
    |% pale green for pre/post-environment|\\
|\backgroundcolor{s}[rgb]{0.8,0.8,0.8}   % light gray for spanning texts|
\end{itemize}

\SpecialUsageIndex{\backgroundcolor}

Therefore, the \bground{} of this |p|re-environment paragraph and other
stuff above is painted by pale green.

因此,这个|p|re-environment段落以及上面的其他内容的\bground{}被涂成了浅绿色。

\Index{pre-environment stuff}

Since the author set \Uidx{\!\pagerim!} to be 5\,|pt|, you will see
unpainted strips of 5\,|pt| wide at all paper edges surrounding painted
regions.  For this and the next three pages, \Uidx{\!\twosided!}|[pcm]| is
declared to enable |p|, |c| and |m| features but to disable the |b|
feature.  Therefore, though this page \pageref{sec:bgpaint} is even and
thus the left outside margin is wider than the right inside one, the
\bground{}s of |l|(eft) and |r|(ight) margins are painted by dark blue and
dark green respectively.

由于作者将 \Uidx{\!\pagerim!} 设置为5,|pt|,因此您将在所有围绕着绘制区域的纸张边缘看到宽度为5,|pt|的未涂色条纹。在这一页和接下来的三页中,\Uidx{\!\twosided!}|[pcm]| 被声明为启用|p|、|c|和|m|功能,但禁用|b|功能。因此,尽管本页\pageref{sec:bgpaint}是偶数页,左外边缘比右内边缘宽,但左边缘和右边缘的\bground{}分别被涂成深蓝色和深绿色。

\par\bigskip

\begin{paracol}{2}
This column-0 is now right and inside because of the |c| feature of
\!\twosided! is enabled.  On the other hand, the \bground{} is this column
is painted by pink because \!\backgroundcolor! for |c[0]| specifies so.
That is, the column ordinals optionally given to |c|(olumn) (and |g|(ap))
regions are \emph{logical} ones not always corresponding to their
\emph{physical} positions in a page.

\switchcolumn
\begingroup\it
As explained in the right column-0, the \bground{} of this left and
outside column-1 is painted by cream yellow as
{\rm\!\backgroundcolor!|{c[1]}|} specifies.  Now we have a
{\rm\!\switchcolumn!|*|} with a \mctext{} to show the \bgpaint{} for
it\footnote{

Since the footnotes in this \env{paracol} environment are \scfnote{} and
\mgfnote{}, and \!\backgroundcolor!\texttt{\char`\{n\char`\}} specifies
light purple, the \bground{} of this (foot)|n|(ote) region is painted by
the color.}.

\par\endgroup
\switchcolumn*[\subsection*{The background of this |s|(panning text)
region is painted by light gray}\medskip]

\begin{figure*}\nosv
\def\arraystretch{0.8}
\centerline{\begin{tabular}[b]{|c|}\hline
    \hbox to.9\textwidth{}\\
    \texttt{f}(loat) region for this page-wise figure is painted by purple\\ 
    \\\hline
    \end{tabular}}
\caption{A Page-Wise Figure}
\end{figure*}

This paragraph is to show how the first line of a paragraph just below a
\mctext{} is placed in the painted region.
\par\vfill

\switchcolumn
\begingroup\it
See the right column for the reason why this paragraph is here.
\par\vfill

See the right column for what we are now doing.
\par\endgroup
\switchcolumn

Now we have a \!\flushpage! to see the \bgpaint{} for a material not shown
in the page, i.e., a page-wise float.
\flushpage

Since we are now in an odd-numbered page \pageref{page:bgpaint2}, this
column-0 is now a left one and is still painted by pink of course.
\par\vfill\label{page:bgpaint2}

This paragraph is to show how the last line of a page without \Scfnote{}s
is placed in the painted region.
\par\newpage

This page is to show how the page without any \pwstuff{} looks like.
\par\vfill

Shortly we will close this \env{paracol} environment in the next page.
\par\newpage

Now we are closing this \env{paracol} environment to show how its
\postenv{} is painted.

\switchcolumn
\begingroup\it
As expected, the \bground{} of this column-1 is still painted by cream
yellow.
\par\vfill

See the comment in the left column.
\par\newpage

See the right column for the reason why we have this almost blank page.
\par\vfill

See the right column for what will happen shortly.
\par\newpage

See the left column for the reason why we are now closing the environment.
\endgroup
\end{paracol}
\bigskip

The \bground{} of this paragraph in |p|(ost-environment) region is also
painted by pale green, because \postenv{} can be \preenv{} at the same
time as we see shortly.  

这个段落在|p|(ost-environment)区域的\bground{}也被涂成了淡绿色,因为正如我们很快会看到的那样,\postenv{}可以同时是\preenv{}。\par\bigskip

\begin{paracol}{2}
This short \env{paracol} environment illustrates how the \preenv{} of this
environment, or the \postenv{} of the last environment in other words, is
painted.

\switchcolumn
\begingroup\it
Therefore, the author does not have much to say in this column, except for
giving a footnote here\footnote{

Since this footnote is \mgfnote{} with that in the \postenv{}, it is
considered as a part of \postenv{} and thus painted by pale green rather
than light purple.\label{fn:bgpaint1}}.
\endgroup
\end{paracol}
\bigskip

Before moving to the next example, one caution is given for \bgpaint{} of
\Mgfnote{}s.  As the footnote \ref{fn:bgpaint1} itself says, \Mgfnote{}s
given in the \lpage{} of a \env{paracol} environment are considered as
belonging to \postenv{}.  Therefore, the footnote \ref{fn:bgpaint1} is
painted by pale green as well as another footnote given now\footnote{

Since this footnote really belongs to \postenv{}, its \bground{} is painted
by pale green naturally.}.

在进入下一个示例之前,对于\Mgfnote{}的\bgpaint{}有一个注意事项。正如脚注\ref{fn:bgpaint1}本身所说的那样,在\env{paracol}环境的\lpage{}中给出的\Mgfnote{}被认为属于\postenv{}。因此,脚注\ref{fn:bgpaint1}将被绘制成浅绿色,以及现在给出的另一个脚注\footnote{由于这个脚注确实属于\postenv{},所以它的\bground{}自然会被绘制成浅绿色。}。
\par\label{page:bgpaint4}



% 
% \newpage
% \backgroundcolor{t(0pt,0pt)(0pt,-4pt)}[rgb]{0.7,0,0}
% \backgroundcolor{b(0pt,-4pt)(0pt,0pt)}[rgb]{0.8,0.6,0}
% \backgroundcolor{l(0pt,4pt)(-4pt,4pt)}[rgb]{0,0,0.7}
% \backgroundcolor{r(-4pt,4pt)(0pt,4pt)}[rgb]{0,0.7,0}
% \backgroundcolor{c[0](4pt,4pt)}[rgb]{1,0.8,1}
% \backgroundcolor{c[1](4pt,4pt)}[rgb]{1,1,0.8}
% \backgroundcolor{g(-4pt,4pt)}[rgb]{0.8,1,1}
% \backgroundcolor{f(4pt,4pt)(4pt,-4pt)}[rgb]{0.8,0,1}
% \backgroundcolor{n(4pt,-4pt)(4pt,4pt)}[rgb]{0.8,0.6,1}
% \backgroundcolor{p(4pt,4pt)}[rgb]{0.8,1,0.6}
% \backgroundcolor{s(4pt,-4pt)}[rgb]{0.8,0.8,0.8}
% 
% \subsection{Mirrored Painting and Enlarging/Shrinking/Shifting Regions\\ 镜像绘制和放大/缩小/移动区域}
% \label{sec:bgpaint-me}
% \twosided
% 
% At a glance, this and the next three pages look painted similarly to
% previous four pages, but by a careful examination you should notice
% two important differences.  The first one is found in the colors
% of left and right margins.  As the author enabled all features of
% \Uidx{\!\twosided!} including |b| for \mirror{}ing and we are now in an
% even-numbered page \pageref{sec:bgpaint-me}, the left and outside margin
% is painted by dark green for the region |r|(ight margin), while the right
% and inside one is painted by dark blue for |l|(eft margin).

% 乍一看,这页和接下来的三页看起来与前面的四页的绘画类似,但是仔细观察你应该会注意到两个重要的区别。第一个区别在于左右边距的颜色。由于作者启用了 \Uidx{\!\twosided!} 的所有特性,包括\mirror{}ing的|b|特性,并且我们现在处于一个偶数页\pageref{sec:bgpaint-me},左边和外部边距由深绿色绘制,表示|r|(ight margin),而右边和内部边距由深蓝色绘制,表示|l|(eft margin)。
% 
% The other is that regions are enlarged, shrunk or shifted by 4\,|pt| by
% the following \!\backgroundcolor! commands with extensions.

另一个是通过以下带有扩展的 \!\backgroundcolor! 命令,通过4,|pt|来扩大、缩小或移动区域。
% \begin{itemize}\item[]
% |\backgroundcolor{t(0pt,0pt)(0pt,-4pt)}[rgb]{0.7,0,0}   |
%     |% B up|\\
% |\backgroundcolor{b(0pt,-4pt)(0pt,0pt)}[rgb]{0.8,0.6,0} |
%     |% T down|\\
% |\backgroundcolor{l(0pt,4pt)(-4pt,4pt)}[rgb]{0,0,0.7}   |
%     |% R left T/B outside|\\
% |\backgroundcolor{r(-4pt,4pt)(0pt,4pt)}[rgb]{0,0.7,0}   |
%     |% L right T/B outside|\\
% |\backgroundcolor{c[0](4pt,4pt)}[rgb]{1,0.8,1}          |
%     |% all edges outside|\\
% |\backgroundcolor{c[1](4pt,4pt)}[rgb]{1,1,0.8}          |
%     |% all edges outside|\\
% |\backgroundcolor{g(-4pt,4pt)}[rgb]{0.8,1,1}            |
%     |% L/R inside & T/B outside|\\
% |\backgroundcolor{f(4pt,4pt)(4pt,-4pt)}[rgb]{0.8,0,1}   |
%     |% L/R outside & T/B up|\\
% |\backgroundcolor{n(4pt,-4pt)(4pt,4pt)}[rgb]{0.8,0.6,1} |
%     |% L/R outside & T/B down|\\
% |\backgroundcolor{p(4pt,4pt)}[rgb]{0.8,1,0.6}           |
%     |% all edges outside|\\
% |\backgroundcolor{s(4pt,-4pt)}[rgb]{0.8,0.8,0.8}        |
%     |% L/R outside & T/B inside|
% \end{itemize}
% 
% \SpecialUsageIndex{\backgroundcolor}

% In the comments above, |L|(eft), |R|(ight), |T|(op) and |B|(ottom) mean
% edges moved by a given extension.  Therefore, for example,
% ``|L/R outside & T/B up|'' for |f|(loat) region means it is enlarged
% horizontally and shifted up vertically by the asymmetric extension
% |(4pt,4pt)(4pt,-4pt)|.  These a little bit complicated setting of
% extensions are to solve the problems in the fundamental example shown in
% previous four pages, namely too strict definition of the regions to be
% painted.  That is, both vertical edges of regions having texts, e.g.,
% |c|(olumn) regions, should look too close to the first and last letters.
% Similarly both horizontal edges of those regions seem too close especially
% when the first line is tall (e.g., the section title in
% p.\Tie\pageref{sec:bgpaint} and the page-wise figure in
% p.\Tie\pageref{page:bgpaint2}) and the last line of a column is followed by
% \mctext{} or \postenv.  Therefore, the author made fine tuning moving
% inside edges of margins outside, and so on.  We will come back this issue
% after exemplifying the effect of the tuning.

在上面的注释中,|L|(eft)、|R|(ight)、|T|(op)和|B|(ottom)表示给定扩展移动的边缘。因此,例如,对于|f|(loat)区域的``|L/R outside & T/B up|''意味着它在水平方向上扩大,在垂直方向上通过不对称扩展|(4pt,4pt)(4pt,-4pt)|向上移动。这些稍微复杂的扩展设置是为了解决前面四页中所示的基本示例中的问题,即对要绘制的区域的定义过于严格。也就是说,具有文本的区域的两个垂直边缘,例如|c|(olumn)区域,看起来离第一个和最后一个字母太近了。同样,当第一行很高时(例如,在p.\Tie\pageref{sec:bgpaint}中的节标题和p.\Tie\pageref{page:bgpaint2}中的每页图)以及一列的最后一行后面跟着\mctext{}或\postenv 时,这些区域的两个水平边缘看起来也太近了。因此,作者对外部边缘内部移动进行了微调等。在示例效果之后,我们将回到这个问题。
% \par\bigskip
% 
% \advance\skip\footins4pt\relax
% \begin{paracol}{2}
% By the tuning to enlarge this |c|(olumn) region, this paragraph has
% comfortable spaces above and below it, as well as at the both side edges.
% 
% \switchcolumn
% \begingroup\it
% This paragraph is surrounded by spaces of a small but comfortable amount as
% well.\footnote{
% 
% Shifting this (foot)|n|(ote) region down a little bit, the space below this
% footnote and above the top edge of the |b|(ottom margin) region is enlarged.}.
% 
% \par\endgroup
% \switchcolumn*[\subsection*{The background of this |s|(panning text)
% region is painted by light gray and enlarged horizontally but shrunk
% vertically}\par\medskip]
% 
% \begin{figure*}\nosv
% \def\arraystretch{0.8}
% \centerline{\begin{tabular}[b]{|c|}\hline
%     \hbox to.9\textwidth{}\\
%     shifting up this \texttt{f}(loat) region gives us a small space above
%     the top edge of the rectangle\\
%     \\\hline
%     \end{tabular}}
% \caption{A Page-Wise Figure}
% \end{figure*}
% 
% This paragraph is to show how well the first line of a paragraph just below a
% \mctext{} is separated from the boundary of two painted regions.
% \par\vfill
% 
% \switchcolumn
% \begingroup\it
% See the right column for the reason why this paragraph is here.
% \par\vfill
% 
% See the right column for what we are now doing.
% \par\endgroup
% \switchcolumn
% 
% By enlarging this |c|(olumn) region and shift the (foot)|n|(ote) region
% down, this paragraph has a comfortable amount of space below it.
% \flushpage
% 
% Similarly to other paragraphs below \pwstuff, this paragraph is well
% separated from the bottom edge of the |f|(loat) region above.
% 
% \par\vfill\label{page:bgpaint-me2}
% 
% As in the case of the line above \Scfnote{}s, the last line of this
% paragraph has a sufficient space separating it from the top edge of the
% |b|(ottom margin) region.
% \par\newpage
% 
% This page is to show how the page without any \pwstuff{} looks like.  As
% you are seeing, the space above this paragraph is sufficient and
% comfortable.
% \par\vfill
% 
% Shortly we will close this \env{paracol} environment in the next page.
% \par\newpage
% 
% Now we are closing this \env{paracol} environment to show how this
% paragraph is separated from the boundary of |c|(olumn) and
% |p|(ost-environment) regions.
% 
% \switchcolumn
% \begingroup\it
% See the comment in the left column for the intention of placing this
% paragraph here.
% \par\vfill
% 
% See the comment in the left column, too.
% \par\newpage
% 
% See the right column for the reason why we have this almost blank page.
% \par\vfill
% 
% See the right column for what will happen shortly.
% \par\newpage
% 
% See the left column for the reason why we are now closing the environment.
% \endgroup
% \end{paracol}
% \bigskip

% The \bground{} of this paragraph in |p|(ost-environment) region is
% painted by pale green as done in p.\Tie\pageref{page:bgpaint4}, but its top
% and bottom edges \emph{look} shifted down and up to give spaces below and
% above the last and first paragraphs in \env{paracol} environments,
% respectively.

这个段落在|p|(ost-environment)区域的\bground{}被涂成了淡绿色,就像在第\pageref{page:bgpaint4}页上所做的那样,但它的顶部和底部边缘\emph{看起来}向下和向上移动了,以在\env{paracol}环境的最后一个段落和第一个段落之上和之下留出空间。
% \par\bigskip
% 
% \begin{paracol}{2}
% This short \env{paracol} environment illustrates how the \preenv{} of this
% environment, or the \postenv{} of the last environment in other words, is
% painted.
% 
% \switchcolumn
% \begingroup\it
% Therefore, the author does not have much to say in this column, except for
% giving a footnote here\footnote{
% 
% As the footnote \ref{fn:bgpaint1} in p.\Tie\pageref{fn:bgpaint1}, this
% \Mgfnote{} is a part of \postenv{} and thus painted by pale green rather
% than light purple.\label{fn:bgpaint-me1}}.
% \endgroup
% \end{paracol}
% \bigskip
% 
% In the setting with \!\backgroundcolor! commands in
% p.\Tie\pageref{sec:bgpaint-me}, the author carefully moved contacting edges
% of regions.  For example, to enlarge |c|(olumn) regions, the inside edges
% of |l|(eft margin) and |r|(ight margin) regions are moved outside, and both
% vertical edges of the |g|(ap) region shifted toward its inside.  As for
% the horizontal edges, the bottom edges of |t|(op margin) and |f|(loat)
% regions are moved up, the top edges of |b|(ottom margin) and
% (foot)|n|(ote) regions are moved down, and both top and bottom edges of
% the |s|(panning text) region are shifted toward its inside.
% 
在设置中,通过在第\Tie\pageref{sec:bgpaint-me}页中使用 \!\backgroundcolor! 命令,作者仔细移动了区域的接触边缘。例如,为了扩大|c|(olumn)区域,将|l|(eft margin)和|r|(ight margin)区域的内部边缘移到外部,并将|g|(ap)区域的两个垂直边缘向内移动。至于水平边缘,将|t|(op margin)和|f|(loat)区域的底部边缘向上移动,将|b|(ottom margin)和(foot)|n|(ote)区域的顶部边缘向下移动,将|s|(panning text)区域的顶部和底部边缘都向内移动。

% These edge shifting could make a region too narrow or too much shifted
% resulting in a material in it overreaching its boundary, especially in
% vertical shifting of horizontal edges.  However we can exploit some large
% space automatically or manually inserted above and/or below the material
% to avoid overreaching.  That is the author exploited the following spaces;
% \!\headsep! below the page head (though it is empty in this document);
% \!\dbltextfloatsep! below the bottom-most page-wise float; spaces that
% \!\subsection!|*| inserts above and below it together with manually
% inserted \!\medskip! below it; \!\skip!\!\footins!\footnote{%
% This is a kind of ``length command'' maybe not widely known.}

这些边缘移动可能会使区域过窄或过多移动,导致其中的内容超出其边界,特别是在水平边缘的垂直移动中。然而,我们可以利用自动或手动插入在材料上方和/或下方的一些较大空间来避免超出。也就是说,作者利用了以下空间:页面头部下方的 \!\headsep!(尽管在本文档中为空);最底部的页面浮动下方的 \!\dbltextfloatsep!;\!\subsection!|*|插入的空间以及其上下的手动插入的 \!\medskip!;在第一个脚注上方的\!\skip!\!\footins!\footnote{这是一种可能不太常见的``长度命令''。},作者临时将其放大了4,|pt|,用于本节和下一节;以及从文本区域的底边到页码的底边的 \!\footskip!。

% above the first footnote which the author enlarged by 4\,|pt| temporarily
% for this and the next subsections; and \!\footskip! from the bottom edge
% of text area to that of the page number.

在第一个脚注之上,作者通过临时将其放大4\,|pt|,为本节和下一节预留了一些空间。此外,\!\footskip! 的高度是从文本区域的底边到页码的底边。


% Now you might notice that the explanation above does not mention the |p|
% region for \Preenv{} and \postenv.  As you should find in the settings,
% this region is enlarged horizontally \emph{and vertically} so that its top
% and bottom edges are moved up and down when the region is at the top or
% bottom of a page, as you are seeing now and find in
% p.\Tie\pageref{sec:bgpaint-me}.  However, this enlargement of course has a
% side effect that the region collides against |c|(olumn) and |g|(ap) regions
% also enlarged vertically making them overlapped.  This overlap will be
% invisible with most of \emph{printers} because, as shown in
% Section\Tie\ref{sec:ref-bgpaint}, |p| region is painted \emph{before} |c|
% and |g| regions are painted.  In addition, since relatively large spaces
% of \!\bigskip! are manually inserted before each \beginparacol{} and after
% each \Endparacol{}, texts in \Preenv{} and \postenv{} are well separated
% from region boundaries.
% 
现在您可能会注意到上面的解释没有提到\Preenv{}和\postenv{}的|p|区域。正如您在设置中找到的那样,这个区域在水平上\emph{和垂直上}被放大,所以当该区域在页面的顶部或底部时,其顶部和底部边缘会向上和向下移动,就像您现在看到的并且在第\pageref{sec:bgpaint-me}页中找到的那样。然而,这种放大当然会产生一个副作用,即该区域与垂直放大的|c|(olumn)和|g|(ap)区域发生碰撞,使它们重叠在一起。这种重叠对于大多数\emph{打印机}来说是看不见的,因为如第\ref{sec:ref-bgpaint}节所示,|p|区域是在|c|和|g|区域之前绘制的。此外,由于在每个\beginparacol{}之前和每个\Endparacol{}之后手动插入了相对较大的 \!\bigskip! 空间,\Preenv{}和\postenv{}中的文本与区域边界之间有很好的分隔。

% This overlay painting |c| and |g| over |p|, however, might produce an
% unexpected result with some printer with which, for example, two colors
% are \emph{blended} in the thin overlapped strip\footnote{%
% For example, a dvi previewer |dviout| produces such a blended result with
% the default setting of coloring.}.

然而,这种|c|和|g|覆盖|p|的叠加绘制可能会在某些打印机上产生意外的结果,例如,在细小的重叠条带中\emph{混合}了两种颜色\footnote{例如,一个dvi预览器|dviout|在默认的着色设置下会产生这样的混合结果。}。

% Unfortunately, this overlay painting is inevitable in the current version
% 1.3, but in a future version, hopefully 1.4, more sophisticated
% \emph{position-dependent} region definition, for example, to shift the top
% edge of |p| region only when the region is at the top of page, could be
% introduced.

不幸的是,在当前的1.3版本中,这种叠加绘制是不可避免的,但在将来的版本中,希望是1.4版本,可以引入更复杂的\emph{位置依赖}区域定义,例如,仅当区域位于页面顶部时才移动|p|区域的顶部边缘。
% 
% Another remark is that the \mirror{}ing specified by the |b| feature of
% \!\twosided! works not only on the colors of side margins but also on
% their asymmetric shrinkage.  That is, the asymmetric shifts of vertical
% edges of |l| and |r| regions correctly performed irrespective of their
% physical positions, i.e., even when the |l| (resp.\ |r|) region is at
% the right (resp.\ left) margin and the edge to be shift is the left
% (resp.\ right) one rather than right (resp.\ left).
% 
另一个要注意的是,\!\twosided!的|b|特性所指定的\mirror{}不仅适用于侧边栏的颜色,也适用于它们的非对称收缩。也就是说,无论左侧(resp.\ 右侧)的|l|(resp.\ |r|)区域是否位于右侧(resp.\ 左侧)边缘,以及待移动的边缘是左侧(resp.\ 右侧)边缘还是右侧(resp.\ 左侧)边缘,|l|和|r|区域的垂直边缘的非对称移动都可以正确进行。
% 
\newpage \suppressfloats
\nobackgroundcolor{t}
\nobackgroundcolor{b}
\nobackgroundcolor{l}
\nobackgroundcolor{r}
\nobackgroundcolor{g}
\backgroundcolor{c[0](4pt,4pt)(0.5\columnsep,4pt)}[rgb]{1,0.8,1}
\backgroundcolor{c[1](0.5\columnsep,4pt)(4pt,4pt)}[rgb]{1,1,0.8}
\backgroundcolor{C[0](10000pt,10000pt)(0.5\columnsep,10000pt)}[rgb]{1,0.8,1}
\backgroundcolor{C[1](0.5\columnsep,10000pt)(10000pt,10000pt)}[rgb]{1,1,0.8}

\subsection{Regions with Infinite Extensions\hfill 具有无限扩展的区域}
\label{sec:bgpaint-inf}

You are now seeing another \bgpaint{} much different from previous two
examples.  That is, after disabling painting of |t|, |b|, |l|, |r| and |g|
regions by \Uidx{\!\nobackgroundcolor!}, the author gave the followings
for painting this and the next pages.

现在你看到了另一个与前两个示例非常不同的\bgpaint{}。也就是说,在通过 \Uidx{\!\nobackgroundcolor!} 禁用|t|、|b|、|l|、|r|和|g|区域的绘制之后,作者为绘制本页和下一页给出了以下设置。

\begin{itemize}\item[]
|\backgroundcolor|
    |{c[0](4pt,4pt)(0.5\columnsep,4pt)}[rgb]{1,0.8,1}|\\
|\backgroundcolor|
    |{c[1](0.5\columnsep,4pt)(4pt,4pt)}[rgb]{1,1,0.8}|\\
|\backgroundcolor|
    |{C[0](10000pt,10000pt)(0.5\columnsep,10000pt)}[rgb]{1,0.8,1}|\\
|\backgroundcolor|
    |{C[1](0.5\columnsep,10000pt)(10000pt,10000pt)}[rgb]{1,1,0.8}|
\end{itemize}

\SpecialUsageIndex{\backgroundcolor}

The first two lines above is different from the previous declaration
because inside edges of |c[0]| and |c[1]| regions are shifted toward
outside of them and thus inside of unpainted |g| region so that the edges
are contacted.  On the other hand, the last two lines are for
\emph{under-painting} of columns and has \emph{\bginfext} to make top,
bottom and outside edges of |C| regions reaching to the corresponding
paper edges.  Since this under-painting is done with colors same as those
of over-painting of |c| regions, you will have an impression that the
paper is two-toned and \pwstuff{} are pasted on the paper\footnote{%
This footnote is given outside \env{paracol} environment but its
\bground{} is painted by light purple because it is merged with the
footnote \ref{fn:bgpaint-inf2}.\label{fn:bgpaint-inf1}}.

上面的前两行与之前的声明不同,因为|c[0]|和|c[1]|区域的内侧边缘向外移动,进入未绘制的|g|区域,使边缘相接触。另一方面,最后两行是用于对列进行\emph{下层绘制},并且具有\emph{\bginfext},使|C|区域的顶部、底部和外部边缘达到相应的纸张边缘。由于此下层绘制使用的颜色与|c|区域的上层绘制相同,所以你会有一种纸张是双色的,并且\pwstuff{}被粘贴在纸张上的印象\footnote{%
这个脚注是在\env{paracol}环境之外给出的,但是它的\bground{}被浅紫色绘制,因为它与脚注\ref{fn:bgpaint-inf2}合并了。\label{fn:bgpaint-inf1}}。
\par\bigskip

\begin{figure}\nosv
\def\arraystretch{0.8}
\centerline{\begin{tabular}[b]{|c|}\hline
    \hbox to.9\textwidth{}\\
    \parbox{.8\textwidth}{
This \texttt{f}(loat) region could be extended to both side edges
and the top edge of the paper if its extension were
\texttt{(10000pt,10000pt)(10000pt,-4pt)}.}\\
    \\\hline
    \end{tabular}}
\caption{A Page-Wise Figure \emph{Imported} from Pre-Environment}
\label{fig:bgpaint-inf}
\end{figure}

\begin{paracol}{2}
Though you cannot see, the right edge of this over-painted |c[0]| region
is shifted right by 4\,|pt| to hide the small patch at the right bottom
corner of the |p| region above by overlaying.

\switchcolumn
\begingroup\it
As explained in the right column, this {\rm|c[1]|} region also has an
invisible left edge shifted left by {\rm4\,|pt|}\footnote{

This (foot)|n|(ote) region could be extended to both side edges and the
bottom edge of the paper if its extension were
\texttt{(10000pt,-4pt)(10000pt,10000pt)}.\label{fn:bgpaint-inf2}}.
\endgroup

\switchcolumn*[\subsection*{This \texttt{s}(panning text) region could be
extended to both side edges of the paper if its extension were
\texttt{(10000pt,-4pt)}.}\par\medskip]

The author does not have much to say now for this column chunk.
\par\vfill

Still nothing to say particular to the page break we will have shortly.
\par\newpage

This paragraph is just for keeping the \env{paracol} environment alive in
this page.
\switchcolumn

\begingroup\it
Little to say as well.
\par\vfill

Nothing to say as well.
\par\newpage

This paragraph is not necessary for keeping alive the environment but is
given for consistent view.
\endgroup

\begin{figure*}\nosv
\def\arraystretch{0.8}
\centerline{\begin{tabular}[b]{|c|}\hline
    \hbox to.9\textwidth{}\\
    \parbox{.8\textwidth}{
This figure is given in the \env{paracol} environment closed in the
previous page but its background is not painted.}\\
    \\\hline
    \end{tabular}}
\caption{A Page-Wise Figure \emph{Exported} to Post-Environment}
\label{fig:bgpaint-inf2}
\end{figure*}
\end{paracol}
\bigskip

Note that overlay painting is inevitable for two-toned page painting, as
far as you want to paint \bground{} of \pwstuff.

请注意,如果您希望绘制\pwstuff{}的\bground{},那么对于双色页面绘制,覆盖绘制是不可避免的。

The last issue of \bgpaint{} is about painting materials given outside
\env{paracol}.  As you have seen, \Preenv{} and \postenv{} are painted but
it is done only when they reside in a page having a portion of a
\env{paracol} environment (maybe) of course.  Therefore, the next page is
\emph{not} painted because the page does not have any parallel-columned
stuff.  Therefore, even if you wish to paint the whole of your document
including pages without \env{paracol} stuff, you cannot do it just with
\Paracol{} package, at least so far.

\bgpaint{}的最后一个问题是关于在\env{paracol}之外给出的材料的绘制。正如您所见,\Preenv{}和\postenv{}是被绘制的,但只有当它们位于具有\env{paracol}环境(可能)的页面中时才进行绘制。因此,下一页\emph{不会}被绘制,因为该页没有任何平行列的内容。因此,即使您希望绘制整个文档,包括没有\env{paracol}内容的页面,至少目前您无法仅使用\Paracol{}宏包来实现。

On the other hand, some materials given outside \env{paracol} environments
are painted as if they are given in the environment when they are
\emph{imported} into the environment.  One category has footnotes given in
\preenv{} when \!\footnotelayout!|{m}| is specified for merging, as
exemplified by the footnote \ref{fn:bgpaint-inf1} in the previous page.
Note that such a footnote is painted by the color for |n| region rather
than |p| region even when there are no footnotes in the \env{paracol}
environment.  The other category has ordinary floats given by \env{figure}
and/or \env{table}
(i.e., neither \env{figure*} nor \env{table*}) environments outside
\env{paracol} and then \emph{deferred} to a page having (a portion of)
stuff produced by \env{paracol}.  Since such a float, e.g.,
Figure\Tie\ref{fig:bgpaint-inf} in this page, is considered as a page-wise
float given in the \env{paracol} environment in this section, its
background is painted by the color for the |f| region, rather than that
for the |p| region which would be used if the float were is placed in the
previous page.  Note that such a deferred float import could occur not
only from the page having \beginparacol{} but also from pages preceding
it.  For example, if you have three \env{figure} environments in a page
$p-1$ just preceding the page $p$ in which you start a \env{paracol}
environment, it could happen that first one is placed in $p-1$ without
painting, the second is placed in $p$ and painted by the color for |p|,
and the third is placed in $p+1$ and painted by the color for |f|.

另一方面,一些在\env{paracol}环境之外给出的材料在被\emph{导入}到环境中时会被绘制,就好像它们是在环境中给出的一样。一个类别是在\preenv{}中给出的,在指定 \!\footnotelayout!|{m}|进行合并时的脚注,例如前一页的脚注\ref{fn:bgpaint-inf1}。请注意,即使在\env{paracol}环境中没有脚注,这样的脚注也会使用|n|区域的颜色而不是|p|区域的颜色进行绘制。另一类是由\env{figure}和/或\env{table}(即既不是\env{figure*}也不是\env{table*})环境给出的普通浮动体,然后被\emph{延迟}到由\env{paracol}产生的(部分)内容的页面。因为这样的浮动体,例如本页的Figure\Tie\ref{fig:bgpaint-inf},被认为是在本节的\env{paracol}环境中给出的整页浮动体,所以它的背景会使用|f|区域的颜色进行绘制,而不是如果该浮动体放在前一页上时将使用|p|区域的颜色。请注意,这样的延迟浮动体导入不仅可能来自具有\beginparacol{}的页面,也可能来自之前的页面。例如,如果在您开始一个\env{paracol}环境的页面$p$的前一页$p-1$中有三个\env{figure}环境,可能发生以下情况:第一个放置在$p-1$中而没有绘制,第二个放置在$p$中并使用|p|的颜色进行绘制,第三个放置在$p+1$中并使用|f|的颜色进行绘制。


Finally some materials \emph{exported} from a \env{paracol} environment
are painted as if they are in \postenv.  In previous two subsections, we
saw \Mgfnote{}s (e.g., \ref{fn:bgpaint1} in p.\Tie\pageref{fn:bgpaint1}
and \ref{fn:bgpaint-me1} in p.\Tie\pageref{fn:bgpaint-me1}) are painted by
the color of |p| rather than |n|.  The other kind of exportation is of
page-wise floats given in a \env{paracol} environment but deferred to the
page next to the page having \Endparacol, or further.  For example,
Figure~\ref{fig:bgpaint-inf2} is given in the \env{paracol} environment
above in this page, but its \bground{} is not painted because the next page
in which the figure is placed does not have any parallel-columned
stuff\footnote{%
If it has, the background is painted by the color for |p|.}.

最后,一些从\env{paracol}环境中\emph{导出}的材料被绘制,就好像它们在\postenv{}中一样。在前两个小节中,我们看到\Mgfnote{}(例如,\pageref{fn:bgpaint1}页上的\ref{fn:bgpaint1}和\pageref{fn:bgpaint-me1}页上的\ref{fn:bgpaint-me1})被绘制为|p|区域的颜色,而不是|n|。另一种导出的类型是在\env{paracol}环境中给出的整页浮动体,但是延迟到\Endparacol{}所在页面的下一页或更后面的页面。例如,本页上方的\env{paracol}环境中给出了Figure~\ref{fig:bgpaint-inf2},但是它的\bground{}没有被绘制,因为放置该图的下一页没有任何平行列的内容\footnote{如果有,背景将使用|p|区域的颜色进行绘制。}。

\newpage\vspace*{\fill}
\centerline{(intentionally blanked page to show this page is \emph{not}
painted)}
\vfill
\advance\skip\footins-4pt\relax




