\columnratio{0.3,0.42,0.28}
\begin{paracol}{3}[\section{Introduction\hfill 介绍}]
\begin{VerbatimII}
\columnratio{0.3,0.42,0.28}
\begin{paracol}{3}[\section{Introduction\hfill 介绍}]
\begin{Verbatim}
左侧源码
\end{Verbatim}
\switchcolumn
This document..
\switchcolumn
本文档..
\end{paracol}    
\end{VerbatimII}
%%%%%%%%%%%%%%%%%%%%%%%%%%%%%%%%%%%%%%%%%%%%%%%%%%%%%%%
\switchcolumn
This document describes the usage of yet another multi-column package named
\textsf{paracol}.  The unique feature of the package is that columns are
typeset {\em in parallel.}
\switchcolumn
本文档介绍了另一个名为 \textsf{paracol} 的多栏排版宏包的使用方法。该宏包的独特特点是可以将栏以{\em 并行}的方式排版。 

\switchcolumn[1]
Suppose you are writing a bilingual document whose left column is written in
a language, say English, and right column has the translation of the left
column in another language, e.g., Japanese.  With the \textsf{paracol}
package you may write an English part of arbitrarily length and then {\em
switch} to its Japanese counterpart to place both parts side by side.  Of
course you may return to the English writing similarly.
\switchcolumn
假设你正在撰写一份双语文档,左栏使用一种语言(如英语),右栏则是左栏的另一种语言(如日语)的翻译。使用 \textsf{paracol} 宏包,你可以先写任意长度的英文部分,然后{\em 切换}到对应的日文部分,将两部分并排放置在一起。当然,你也可以类似地返回到英文撰写。
\end{paracol}
\end{document}

\begin{paracol}{2}

\switchcolumn


\switchcolumn*
The {\em\Uidx\cswitch} is always allowed when you complete an outermost
level paragraph.  You may be unaware whether a column is broken into
multiple pages before switching because the package automatically goes
back and forward to the correct page and vertical position when you switch
the column.  Moreover, you may {\em\Uidx\sync{}e} columns so that the tops
of the first paragraphs after switching in all columns are vertically
aligned.  At a \sync{}ation point, you may give a single-column text,
for example a common section header, optionally.  You may also switch
single-column and multi-column in a page arbitrary.
\switchcolumn
在外层段落完成后,总是允许使用 {\em\Uidx\cswitch} 命令。在切换之前,你可能不知道栏是否被分成多个页面,因为当你切换栏时,宏包会自动回到正确的页面和垂直位置。此外,你可以通过 {\em\Uidx\sync{}e} 命令来使列对齐,这样在切换后,所有列中第一个段落的顶部会垂直对齐。在 \sync{}ation 点,你可以选择给出单栏文本,例如一个公共的章节标题。你还可以随意在页面上切换单栏和多栏排版。
\switchcolumn*
This manual itself is an example of two-column documents typeset by
\textsf{paracol}.  
\switchcolumn
本手册本身就是使用 \textsf{paracol} 宏包排版的两栏文档的一个示例。

\end{paracol}
\begin{Verbatim}
\begin{paracol}{2}[\section{Introduction}]
\hbadness5000
en.....
\switchcolumn
中文....

\switchcolumn*
en.....
\switchcolumn
中文....
\switchcolumn*[\section{Basic Usage}]....
\end{paracol}
\end{Verbatim}