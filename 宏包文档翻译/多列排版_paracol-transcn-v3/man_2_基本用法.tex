
% \begin{paracol}{2}[\section{Basic Usage\hfill 基本用法}]
% Loading the package is very simple.  What you have to do is
% \!\usepackage!|{|\Uidx{\env{paracol}}|}| in the preamble.  Note that
% \textsf{paracol} can be used with \LaTeXe{} and does not work with
% \LaTeX{} 2.09.
% \switchcolumn
% 加载该宏包非常简单。在导言区使用\\ \!\usepackage!|{|\Uidx{\env{paracol}}|}| 命令即可。请注意,\textsf{paracol} 可以与 \LaTeXe{} 一起使用,不支持 \LaTeX{} 2.09。
% 
% \switchcolumn*
% The fundamental means of parallel-column typesetting are the environment
% \env{paracol} and the command \Uidx{\!\switchcolumn!}.  The \env{paracol}
% environment needs an argument to specify the number of columns.  Thus the
% following is the basic construct for two-parallel-column documents.
% \switchcolumn
% 并列栏排版的基本手段是使用 \env{paracol} 环境和命令 \Uidx{\!\switchcolumn!}。\env{paracol} 环境需要一个参数来指定栏的数量。因此,以下是两栏并列文档的基本结构。
%\switchcolumn*
% \begin{quote}
% \!\begin!|{|\env{paracol}|}{2}|\\
% \textit{left column text}\\
% \!\switchcolumn!\\
% \textit{right column text}\\
% \!\switchcolumn!\\
% \textit{left column text}\\
% \!\switchcolumn!\\
% \textit{right column text}\\
% \!\switchcolumn!\\
% \mbox{\hspace{4em}}$\vdots$\\
% \!\end!|{|\env{paracol}|}|
% \end{quote}
% \switchcolumn
% \begin{quote}
% \!\begin!|{|\env{paracol}|}{2}|\\
% \textit{左栏文本}\\
% \!\switchcolumn!\\
% \textit{右栏文本}\\
% \!\switchcolumn!\\
% \textit{左栏文本}\\
% \!\switchcolumn!\\
% \textit{右栏文本}\\
% \!\switchcolumn!\\
% \mbox{\hspace{4em}}$\vdots$\\
% \!\end!|{|\env{paracol}|}|
% \end{quote}

% \switchcolumn*
% The \!\switchcolumn! command may have an optional argument to specify the
% column number (zero origin) to start.  That is, \!\switchcolumn!|[0]|
% means to switch to the leftmost column, |\switchcolumn[1]| is to start the
% second column and so on.  Thus the |\switchcolumn| without the optional
% argument may be considered as \!\switchcolumn!|[|$i+1\bmod{n}$|]| where
% $i$ is the ordinal of the column you are leaving from and $n$ is the
% number of columns given to \env{paracol} environment.
% \switchcolumn
% \!\switchcolumn! 命令可以带有可选参数来指定从第几栏(从零开始计数)开始切换。也就是说,\!\switchcolumn!|[0]| 表示切换到最左边的栏,|\switchcolumn[1]| 表示从第二栏开始,依此类推。因此,不带可选参数的 |\switchcolumn| 可以视为 \!\switchcolumn!|[|$i+1\bmod{n}$|]|,其中 $i$ 是你离开的栏的序号,$n$ 是给定给 \env{paracol} 环境的栏数。
% \end{paracol}
