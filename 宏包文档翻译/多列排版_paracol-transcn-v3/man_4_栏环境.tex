\section{Environments for Columns\hfill 栏环境}\label{sec:env}

\columnratio{0.3,0.42,0.28}
\begin{paracol}{3}
\begin{column}
\begin{VerbatimII}
...
\begin{column*}[\section{Environments for Columns}]
...
\end{column*}
\begin{column}
...
\end{column}
\end{VerbatimII}
%%%%%%%%%%%%%%%%%%%%%%%%%%%%%%%%%%%%%%%%%%%%%%%%%%%%%%%
\end{column}

\begin{column}
\Uidx{\Index{column-switching environment}}
\subsection{Environment \texttt{column}}
The \!\switchcolumn! is simple but you may prefer to pack the contents of a
column in an environment.  The \Uidx{\env{column}} environment is
available for this well-structuralization of \LaTeX{} sources for
parallel-columned documents. A construct;
\end{column}

\begin{column}
\subsection{\ttfamily column环境}
\!\switchcolumn! 命令很简单,但你可能更喜欢将一个栏的内容封装在一个环境中。\Uidx{\env{column}} 环境可以用于在 \LaTeX{} 文档中良好地组织并列栏的内容。以下结构:
\end{column}

\switchcolumn[1]
\begin{quote}
\!\begin!|{|\env{column}|}|\\
\textit{text for a column}\\
\!\end!|{|\env{column}|}|
\end{quote}
\noindent is (almost) equivalent to;
\begin{quote}
\!\switchcolumn!\\
\textit{text for a column}
\end{quote}
\switchcolumn
\begin{quote}
\!\begin!|{|\env{column}|}|\\
\textit{栏中文字}\\
\!\end!|{|\env{column}|}|
\end{quote}
(几乎)等同于:
\begin{quote}
\!\switchcolumn!\\
{\fontKai 栏中文字}
\end{quote}

\switchcolumn[1]
The \Uidx{\env{column*}} environment is also available for the column
\sync{}ation and may have an optional argument for \mctext.
\switchcolumn
\Uidx{\env{column*}} 环境也可用于栏的同步,并且可以有一个可选参数用于跨栏文本。


\switchcolumn[0]*
\subsection*{~}
\begin{Verbatim}
\begin{nthcolumn*}{1}
\subsection{Environment \texttt{nthcolumn}}    
source
\end{nthcolumn*}
\end{Verbatim}

\begin{nthcolumn}{1}
\subsection{Environment \texttt{nthcolumn}}
The \!\switchcolumn! can start an arbitrarily specified column with the
column number given through its optional argument, but the \env{column}
environment cannot do it.  If you want to start $i$-th column, you have to
do \!\begin!|{|\Uidx{\env{nthcolumn}}|}{|$i$|}| (or
\Uidx{\env{nthcolumn*}} with an optional argument to \sync{}e).
\end{nthcolumn}

\begin{nthcolumn}{2}
\subsection{\texttt{nthcolumn}环境}
\!\switchcolumn! 可以通过可选参数指定要开始的任意列的列号,但 \env{column} 环境不能这样做。若你想要开始第 $i$ 列,你要使用 \!\begin!|{|\Uidx{\env{nthcolumn}}|}{|$i$|}|(或带有可选参数的 \Uidx{\env{nthcolumn*}} 来进行同步)。
\end{nthcolumn}

\begin{leftcolumn*}
\begin{VerbatimII}
\begin{leftcolumn*}
\begin{Verbatim}
左侧源码
\end{Verbatim}
\end{leftcolumn*}   
\begin{rightcolumn}
\subsection...
The environments...
\end{rightcolumn}  
\switchcolumn
\subsection{...
环境 ...
\end{VerbatimII}
\end{leftcolumn*}   
%%%%%%%%%%%%%%%%%% 

\begin{rightcolumn}
\subsection[Environments \texttt{leftcolumn} and \texttt{rightcolumn}]
{Environments \texttt{leftcolumn} and\\\texttt{rightcolumn}}
The environments \Uidx{\env{leftcolumn}} and \Uidx{\env{rightcolumn}} (and
their starred versions with an optional argument) are available as more
convenient means than saying \!\begin!|{|\env{nthcolumn}|}{0}| to switch
to the left(most) column and
\!\begin!|{|\env{nthcolumn}|}{1}| to the right (but may not be rightmost)
one.
\Uidx{\EnvIndex{leftcolumn*}}\Uidx{\EnvIndex{rightcolumn*}}
\end{rightcolumn}  

\switchcolumn
\subsection{\ttfamily leftcolumn 和 rightcolumn \\环境}
环境 \Uidx{\env{leftcolumn}} 和 \Uidx{\env{rightcolumn}}(以及带有可选参数的星号版本)可作为比使用 \!\begin!|{|\env{nthcolumn}|}{0}| 切换到最左栏 和 \!\begin!|{|\env{nthcolumn}|}{1}| 切换到右栏(可能不是最右)更方便的方法。

\switchcolumn[0]*
\begin{figure*}\nosv
\def\arraystretch{0.8}
\centerline{\begin{tabular}[b]{|c|}\hline
    \hbox to.9\textwidth{}\\
    three-column figure \#1\\
    \\\hline
    \end{tabular}}
\caption{A Three-Column Figure}
\end{figure*}

\switchcolumn
\begin{figure}[t]\nosv
\def\arraystretch{0.8}
\centerline{\begin{tabular}[b]{|c|}\hline
    \hbox to.9\columnwidth{}\\\\
    single-column figure \#1\\
    \\\\\hline
    \end{tabular}}
\caption{A Single-Column Figure}
\end{figure}

\switchcolumn
\begin{figure}[t]\nosv
\def\arraystretch{0.8}
\centerline{\begin{tabular}[b]{|c|}\hline
    \hbox to.9\columnwidth{}\\
    \ttfamily single-column figure \#2\\
    \\\hline
    \end{tabular}}
\caption{\ttfamily Another Single-Column Figure}
\end{figure}
\end{paracol}