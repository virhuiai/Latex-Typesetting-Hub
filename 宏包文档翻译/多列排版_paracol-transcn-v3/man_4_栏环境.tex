% \begin{paracol}{2}
%\switchcolumn
% 
% \begin{column*}[\section{Environments for Columns\hfill 栏环境}\label{sec:env}]
% \Uidx{\Index{column-switching environment}}
% \subsection{Environment \texttt{column}}
% The \!\switchcolumn! is simple but you may prefer to pack the contents of a
% column in an environment.  The \Uidx{\env{column}} environment is
% available for this well-structuralization of \LaTeX{} sources for
% parallel-columned documents. A construct;
%\end{column*}
%\begin{column}
% \subsection{\ttfamily column环境}
% \!\switchcolumn! 命令很简单,但你可能更喜欢将一个栏的内容封装在一个环境中。\Uidx{\env{column}} 环境可以用于在 \LaTeX{} 文档中良好地组织并列栏的内容。以下结构:
%\end{column}
%\end{paracol}

%\begin{paracol}{3}
% \begin{column}
% \begin{quote}
% \!\begin!|{|\env{column}|}|\\
% \textit{text for a column}\\
% \!\end!|{|\env{column}|}|
% \end{quote}
%\end{column}



%\begin{column}
%\noindent is (almost) equivalent to;\\(几乎)等同于:
%\end{column}

%\switchcolumn
% 

% \begin{quote}
% \!\switchcolumn!\\
% \textit{text for a column}
% \end{quote}

%\end{paracol}

%\begin{paracol}{2}
%\switchcolumn
%\switchcolumn*
% The \Uidx{\env{column*}} environment is also available for the column
% \sync{}ation and may have an optional argument for \mctext.
%\switchcolumn
% \Uidx{\env{column*}} 环境也可用于栏的同步,并且可以有一个可选参数用于 \mctext。
%\end{paracol}

%\begin{paracol}{2}
% \begin{nthcolumn}{0}
% \subsection{Environment \texttt{nthcolumn}}
% The \!\switchcolumn! can start an arbitrarily specified column with the
% column number given through its optional argument, but the \env{column}
% environment cannot do it.  If you want to start $i$-th column, you have to
% do \!\begin!|{|\Uidx{\env{nthcolumn}}|}{|$i$|}| (or
% \Uidx{\env{nthcolumn*}} with an optional argument to \sync{}e).
% \end{nthcolumn}

% \begin{nthcolumn}{1}
% \subsection{\texttt{nthcolumn}环境}
% \!\switchcolumn! 可以通过可选参数指定要开始的任意列的列号,但 \env{column} 环境不能这样做。如果你想要开始第 $i$ 列,你需要使用 \!\begin!|{|\Uidx{\env{nthcolumn}}|}{|$i$|}|(或带有可选参数的 \Uidx{\env{nthcolumn*}} 来进行同步)。
% \end{nthcolumn}
%\end{paracol}
%\begin{Verbatim}
%\begin{paracol}{2}
%\begin{nthcolumn*}{1}
%\subsection{...}
% ...
%\end{nthcolumn*}
% 
%\begin{nthcolumn}{0}
%\subsection{...}
%...
%\end{nthcolumn}
%\end{paracol}
%\end{Verbatim}


%\begin{paracol}{2}
% \begin{leftcolumn*}
% \subsection[Environments \texttt{leftcolumn} and \texttt{rightcolumn}]
%     {Environments \texttt{leftcolumn} and\\\texttt{rightcolumn}}
% The environments \Uidx{\env{leftcolumn}} and \Uidx{\env{rightcolumn}} (and
% their starred versions with an optional argument) are available as more
% convenient means than saying \!\begin!|{|\env{nthcolumn}|}{0}| to switch
% to the left(most) column and
% \!\begin!|{|\env{nthcolumn}|}{1}| to the right (but may not be rightmost)
% one.
% 
% \Uidx{\EnvIndex{leftcolumn*}}\Uidx{\EnvIndex{rightcolumn*}}
% 
% \end{leftcolumn*}
% 
% \begin{rightcolumn}
% \subsection{\ttfamily leftcolumn 和 rightcolumn \\环境}
% 环境 \Uidx{\env{leftcolumn}} 和 \Uidx{\env{rightcolumn}}(以及带有可选参数的星号版本)可作为比使用 \!\begin!|{|\env{nthcolumn}|}{0}| 切换到最左栏 和 \!\begin!|{|\env{nthcolumn}|}{1}| 切换到右栏(可能不是最右)更方便的方法。
% \end{rightcolumn}
% 
% \end{paracol}