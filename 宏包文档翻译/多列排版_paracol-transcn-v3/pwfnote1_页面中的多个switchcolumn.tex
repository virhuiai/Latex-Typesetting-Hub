\subsection{Multiple \cs{switchcolumn} in a Page\hfill 页面中的多个\cs{switchcolumn}}
\label{sec:fnnp-multsc}
\columnratio{0.55}
\begin{paracol}{2}
Here we have an example with three \!\switchcolumn! commands in a page
having six footnotes.  Hereafter, footnotes are typeset with
\Uidx{\!\footnotelayout!}|{m}|\footnote{And thus this footnote is merged with those in the \env{paracol}
environment.}.
\switchcolumn
下面是一个在页面中使用了三个 \!\switchcolumn! 命令的示例,其中包含六个脚注。在此之后,使用 \Uidx{\!\footnotelayout!}|{m}|\footnote{这个脚注与\env{paracol}环境中的脚注合并了。}设置脚注样式。    
\end{paracol}

\begin{Verbatim}
\footnotelayout{m}
\end{Verbatim}

\footnotelayout{m}
\Hrule
\begin{paracol}{2}
\fnpar{First left-column 左列的第一个}{\footnote{First left-column
footnote.左列的第一个脚注。 \label{fn:2L1}}}
\Fnpar{Second left-column 左列的第二个}{\footnote{Second left-column
footnote. 左列的第二个脚注。\label{fn:2L2}}}{
It is followed by a \cs{switchcolumn}. 它后面跟着一个\cs{switchcolumn}。}
\switchcolumn
\Fnpar{First right-column 右列的第一个}{\footnote{First right-column footnote but
following the second left-column one. 右列的第一个脚注,但是在第二个左列脚注之后。\label{fn:2R1}}}{It is followed by a
\cs{switchcolumn*}. 它后面跟着一个\cs{switchcolumn*}。}
\switchcolumn*
\Fnpar{Third and synchronized left-column 左列的第三个(同步的)}{\footnote{Third left-column
footnote but following the first right-column one. 左列的第三个脚注,但是在第一个右列脚注之后。\label{fn:2L3}}}{It is
followed by a \cs{switchcolumn}. 它后面跟着一个\cs{switchcolumn}。}
\switchcolumn
\fnpar{Second and synchronized right-column 右列的第二个(同步的)}{\footnote{Second right-column
footnote but following the third left-column one. 右列的第二个脚注,但是在第三个左列脚注之后。\label{fn:2R2}}}
\fnpar{Third right-column 右列的第三个}{\footnote{Third right-column
footnote.\label{fn:2R3} 右列的第三个脚注。}}
\end{paracol}
\newpage

\columnratio{0.55}
\begin{paracol}{2}
The example in the previous page should look weird because the order of
the third footnote in the left column \ref{fn:2L3} and the first in the
right \ref{fn:2R1} are reversed in their numbers and in the stack at the
page bottom.  However, the result is \emph{natural} because they are
numbered and stacked in the order of occurrence in the source |.tex| as
always done in any documents without \textsf{paracol} and with it but
\mcfnote{} footnote typesetting.  Since the \textsf{paracol} cannot
maintain the order automatically\footnote{%
So far, because the maintenance is extremely tough.  But since it is not
impossible, some day you could have an improved version of
\textsf{paracol} with the automatic ordering.},
\switchcolumn
在上一页的示例中,看起来有些奇怪,因为左列中的第三个脚注 \ref{fn:2L3} 和右列中的第一个脚注 \ref{fn:2R1} 在它们的编号和页面底部的堆栈中的顺序上是颠倒的。然而,这个结果是“自然”的,因为它们按照在源 |.tex| 中出现的顺序进行编号和堆叠,这是任何没有使用 \textsf{paracol} 或使用了 \mcfnote{} 脚注排版的文档中都会这样做的。由于 \textsf{paracol} 无法自动维护顺序\footnote{至今为止,因为维护顺序非常困难。但是,既然不是不可能,总有一天您可能会有一个改进版的 \textsf{paracol},具有自动排序功能。},
\switchcolumn[0]*
you have to maintain it by yourself.
\switchcolumn
你需要自己维护这个问题。
\switchcolumn[0]*
The problem is partly solved by using \!\footnote! with its optional
argument \oarg{num} to number the first right-column and the third
left-column footnotes explicitly, i.e., to give
$\mathit{num}=\ref{fn:2L3}$ to the former and $\mathit{num}=\ref{fn:2R1}$
to the latter.  One caution is that you have to remember that \!\footnote!
with the optional \meta{num} does not update \counter{footnote} counter
and thus you have to do
\!\setcounter!|{footnote}{|\texttt{\ref{fn:2L3}}|}| or
\!\addtocounter!|{footnote}{2}| after the third left-column footnote.
\switchcolumn
部分解决这个问题的方法是使用带有可选参数\oarg{num}的 \!\footnote! 命令,来显式地对第一个右列和第三个左列的脚注进行编号,即给前者赋值$\mathit{num}=\ref{fn:2L3}$,给后者赋值$\mathit{num}=\ref{fn:2R1}$。需要注意的是,你必须记住,带有可选参数\meta{num}的 \!\footnote! 命令不会更新\counter{footnote}计数器,因此你需要在第三个左列脚注之后使用 \!\setcounter!|{footnote}{|\texttt{\ref{fn:2L3}}|}| 或 \!\addtocounter!|{footnote}{2}|。
\switchcolumn[0]*
This remedy, however, cannot change the stacking order of these two
footnotes of course.  Therefore, you need another trick with
\!\footnotemark! and \!\footnotetext! to stack the third left-column
footnote above the first right-column one.  More specifically, you can
solve the problem inserting
\switchcolumn
然而,这种方法当然无法改变这两个脚注的堆叠顺序。因此,您需要使用 \!\footnotemark! 和  \!\footnotetext! 来将第三个左列脚注堆叠在第一个右列脚注上面。具体来说,您可以通过插入以下内容来解决这个问题:
    
\end{paracol}
\begin{quote}
\!\footnotetext!\texttt{[\ref{fn:2R1}]}
    |{|\textit{text for the third left footnote}|}|
\end{quote}

somewhere between \!\footnote! commands for the second left-column and the
first right-column ones, e.g., at the end of the second left-column
paragraph, and attaching its mark to the appropriate word for the footnote
by \!\footnotemark!\texttt{[\ref{fn:2R1}]}, to have the following.

在第二个左列的 \!\footnote! 命令和第一个右列的 \!\footnote! 命令之间的某个位置,例如在第二个左列段落的末尾,并通过  \!\footnotemark!\texttt{[\ref{fn:2R1}]} 将其标记附加到脚注所对应的单词上,可以得到以下效果。
\Hrule
\begin{paracol}{2}
\fnpar{First left-column}{\footnote{First left-column
footnote.\label{fn:3L1}}}
\Fnpar{Second left-column}{\footnote{Second left-column
footnote.\label{fn:3L2}}}{
It is followed by \cs{footnotetext}\texttt{[\ref{fn:3L3}]}\marg{text}
and a \cs{switchcolumn}.}
\addtocounter{footnote}{1}
\footnotetext[\arabic{footnote}]{Third left-column footnote given by
\cs{footnotetext}\texttt{[\ref{fn:3L3}]}\marg{text} placed at the end of
the second left-column paragraph.\label{fn:3L3}}
\switchcolumn
\addtocounter{footnote}{1}
\Fnpar{First right-column}{\footnote[\arabic{footnote}]{First right-column
footnote whose number \ref{fn:3R1} is explicitly given by
\cs{footnote}\texttt{[\ref{fn:3R1}]}\marg{text}.\label{fn:3R1}}}{It is
followed by a \cs{switchcolumn*}.}
\addtocounter{footnote}{-1}
\switchcolumn*
Third and synchroized left-column paragraph\Dotfill\\
\Dotfill with a footnote whose mark
here\footnotemark[\arabic{footnote}]\Dotfill\\
\Dotfill is given by \!\footnotemark!\texttt{[\ref{fn:3L3}]}\Dotfill in
it.\\
It is followed by \!\addtocounter!|{footnote}{2}| and a \!\switchcolumn!.
\addtocounter{footnote}{1}
\switchcolumn
\fnpar{Second and synchronized right-column}{\footnote{Second right-column
footnote correctly following the first right-column one.\label{fn:3R2}}}
\fnpar{Third right-column}{\footnote{Third right-column
footnote.\label{fn:3R3}}}
\end{paracol}
\Hrule

Though this solution gives a good result, however, it has the following
two problems.  First, you have to explicitly specify the footnote number
through the optional arguments \oarg{num} of \!\footnote!,
\!\footnotetext! and \!\footnotemark!.  This problem is quite severe
because, for example, if you add a footnote somewhere preceding the
\env{paracol} environment in question, you have to modify all
\oarg{num} arguments of footnote-related commands in the environment.
This means that when the footnote addition is done in the first page of a
100-page document having \env{paracol} environments with explicitly numbered
footnotes in every page, you have to make the corrections for environments
in 99 pages.  The other a little bit less severe problem is that you have
to keep \counter{footnote} counter having correct value by
\!\setcounter!, \!\addtocounter! or \!\stepcounter! for footnotes following
those with explicit numbering so that their numbers are given by the
default action of \!\footnote!.


虽然这种解决方案可以得到一个很好的结果,但它存在以下两个问题。首先,您必须通过\!\footnote!、\!\footnotetext! 和 \!\footnotemark! 命令的可选参数\oarg{num}显式地指定脚注编号。这个问题非常严重,因为例如,如果您在所讨论的\env{paracol}环境之前的某个地方添加了一个脚注,您必须修改环境中所有脚注相关命令的\oarg{num}参数。这意味着当在一个具有每页都有显式编号脚注的\env{paracol}环境的100页文档的第一页中进行脚注添加时,您必须对99页中的环境进行更正。另一个稍微不那么严重的问题是,您必须通过 \!\setcounter!、\!\addtocounter! 或 \!\stepcounter! 保持\counter{footnote}计数器具有正确的值,以便对那些具有显式编号的脚注之后的脚注进行默认的编号。

To cope with these two problems, \textsf{paracol} provides you with the
\emph{starred} versions of \!\footnote! and its relatives as introduced in
Section~\ref{sec:ref-scfnote} and detailedly explained in the next
Section~\ref{sec:fnnp-starred}.

为了解决这两个问题,\textsf{paracol}为您提供了\emph{带星号}的 \!\footnote! 及其相关命令,如在第~\ref{sec:ref-scfnote} 节中介绍的,并在下一节~\ref{sec:fnnp-starred} 中详细解释。
