% \section{Numbering and Placement of Page-Wise Footnotes\hfill 页注的编号和位置}
% \label{sec:fnnp}
% \changes{v1.2-2}{2013/05/11}
% 	{Add the section ``Numbering and Placement of Single-Columned
%	 Footnotes'' to describe page-wise footnotes in detail.}
% \changes{v1.3-5}{2013/09/17}
%	{Rename the section title from ``Numbering and Placement of
%	 Single-Columned Footnotes''to ``Numbering and Placement of
% 	``Page-Wise Footnotes'' following new naming.}
% 
% Here we have a simple example of \scfnote{} but not-merged
% footnotes\footnote{
% Because of the non-merged typesetting, this footnote is put above the
% example.\par\Hrule\label{fn:preenv}}.

% 这里有一个简单的示例,展示了非合并的\scfnote{}脚注\footnote{因为不是合并的排版方式,所以这个脚注放在了示例之上。\par\Hrule\label{fn:preenv}}。

% \footnotelayout{p}
% \begin{paracol}{2}
% \fnpar{First left-column 左列第一}{\footnote{First left-column footnote. 左列第一脚注。}}
% \fnpar{Second left-column 左列第2}{\footnote{Second left-column footnote. 左列第2脚注。}}
% \switchcolumn
% \fnpar{First right-column}{\footnote{First right-column footnote.}}
% \fnpar{Second right-column}{\footnote{Second right-column footnote.
% This and all other footnotes above are \scfnote{} and, since footnote
% typesetting is non-merged, they are put above the \postenv.右列第二脚注。这个脚注和上面的所有脚注都是\scfnote{},由于脚注排版是非合并的,它们放在了\postenv{}之上。 }}
% \end{paracol}
% \Hrule
% 
% As shown above, it is easy to have a reasonable result of footnote
% numbering and placement as far as your \env{paracol} environment is
% completely included in a page and you accept the numbering in
% left-column-first manner constructing the environment as follows
% exploiting the fact \counter{footnote} is made global, where $b$ is the
% value of \counter{footnote} counter at \beginparacol, i.e., the number
% given to the footnote just preceding the environment, and thus
% $b=\ref{fn:preenv}$ in the example above.

如上所示,只要您的 \env{paracol} 环境完全包含在一页中,并且您接受按左列优先的方式编号和放置脚注,那么脚注编号和放置的结果就会比较合理。可以通过以下方式构建环境,利用 \counter{footnote} 是全局的这一事实,其中 $b$ 是在 \beginparacol 处的 \counter{footnote} 计数器的值,即在环境之前的脚注的编号,因此在上面的示例中 $b=\ref{fn:preenv}$。

% \begin{quote}\vskip-1pt
% |\begin{paracol}{2}|\\
% \textit{left-column stuff having $n$ footnotes numbered $b+1$, $b+2$,
% \ldots, $b+n$}\\
% |\switchcolumn|\\
% \textit{right-column stuff having $m$ footnotes numbered $b+n+1$, $b+n+2$,
% \ldots, $b+n+m$}\\
% |\end{paracol}|
% \end{quote}\vskip-1pt

% The real life is, however, tougher than that, because the assumptions above
% are too optimistic as described in the following subsections.

然而,现实生活比上面的假设更加艰难,因为如下小节所描述的那样,这些假设过于乐观。
% \vskip-3pt\vskip0pt