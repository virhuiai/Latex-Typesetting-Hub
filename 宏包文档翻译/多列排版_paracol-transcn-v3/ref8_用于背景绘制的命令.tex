
\KeepSpace{7}
\subsection{Commands for Background Painting\hfill 用于背景绘制的命令}
\label{sec:ref-bgpaint}

\begin{description}
\item[\Midx{\!\backgroundcolor!}\marg{region}\oarg{mode}\marg{color}]
    \mbox{}\par
\Item[\Midx{\!\backgroundcolor!}
    \Arg{\meta{region}$|(|x_0|,|y_0|)|$}\oarg{mode}\marg{color}]
    \mbox{}\par
\Item[\Midx{\!\backgroundcolor!}
    \Arg{\meta{region}$|(|x_0|,|y_0|)||(|x_1|,|y_1|)|$}
    \oarg{mode}\marg{color}]
    \mbox{}\par
\columnratio{0.55}
\begin{paracol}{2}
The command declares that {\em\Uidx\bgpaint} of \meta{region} is performed
with \meta{color} or what it specifies by the combination of the optional
\meta{mode}.  The \meta{region} whose \bground{} is painted is one of the
following.
\switchcolumn
该命令声明使用 \meta{color} 或其由可选 \meta{mode} 组合指定的方式来执行 \meta{region} 的{\em\Uidx\bgpaint}。被着色的\bground{}的\meta{region}是以下之一。
\end{paracol}


\begin{description}
\columnratio{0.55}
\begin{paracol}{2}
\item[|c|\rm(\textit{olumn})] for all columns, or particular one if
\meta{region} is |c|\oarg{col} to specify its ordinal \meta{col}.
\switchcolumn\item[|c|\rm(\textit{olumn})]
适用于所有列,或者如果\meta{region}为|c|\oarg{col}时,可以指定特定的列序号\meta{col}。
\switchcolumn[0]*
\item[|g|\rm(\textit{ap})] for all gaps between columns, or particular one
if \meta{region} is |g|\oarg{col} to specify the ordinal \meta{col} of the
column preceding the gap.
\switchcolumn
\item[|g|\rm(\textit{ap})]
对于所有列之间的间隙,或者特定的间隙,可以使用\meta{region}参数。如果\meta{region}是|g|\oarg{col},则可以指定前一个间隙的序号\meta{col}。
\switchcolumn[0]*
\item[|s|\rm(\textit{panning})] for \mctext{}s.
\switchcolumn\item[|s|\rm(\textit{panning})]
用于\mctext{}。
\switchcolumn[0]*
\item[|f|\rm(\textit{loat})] for \pwise{} floats.
\switchcolumn\item[|f|\rm(\textit{loat})]
用于 \pwise{} 浮动体。
\switchcolumn[0]*
\item[|n|\rm(\textit{ote})] for (\mgfnote{} or non-merged) \Scfnote{}s.
\switchcolumn\item[|n|\rm(\textit{ote})]
用于 (\mgfnote{} 或非合并的) \Scfnote{}。
\switchcolumn[0]*
\item[|p|\rm(\textit{re/post})] for \Preenv{} and \postenv.
\switchcolumn\item[|p|\rm(\textit{re/post})]
用于 \Preenv{} 和 \postenv。
\switchcolumn[0]*
\item[|t|\rm(\textit{op})] for top margin.
\switchcolumn\item[|t|\rm(\textit{op})]
用于顶部边距。
\switchcolumn[0]*
\item[|b|\rm(\textit{ottom})] for bottom margin.
\switchcolumn\item[|b|\rm(\textit{ottom})]
用于底部边距。
\switchcolumn[0]*
\item[|l|\rm(\textit{eft})] for left margin.
\switchcolumn\item[|l|\rm(\textit{eft})]
用于左边距。
\switchcolumn[0]*
\item[|r|\rm(\textit{ight})] for right margin.
\switchcolumn\item[|r|\rm(\textit{ight})]
用于右边距。
\end{paracol}

\end{description}
\columnratio{0.55}
\begin{paracol}{2}
In addition, capitals of the keys above, i.e., |C|, |G|, \ldots, |L|, are
also legitimate for {\em under painting}.  For example, you may specify to
paint the \bground{} of a region, say top margin, by two
\!\backgroundcolor! with |t| and |T| and with different color arranging the
size of the region of either |t| or |T| (or both of them) by the
\emph{\bgext} option shown below.
\switchcolumn
此外,上面的键的大写字母,即|C|、|G|、\ldots、|L|,也可以用于{\em 下层绘制}。例如,您可以通过两个不同颜色的 \!\backgroundcolor!(使用|t|和|T|)和通过\emph{\bgext}选项来调整|t|或|T|(或两者)的区域大小,来指定绘制区域(例如顶部边距)的\bground{}。
\switchcolumn[0]*
The optional $|(|x_0|,|y_0|)|$ is to enlarge the region to be painted
shifting its left-top and right-bottom corner outside by
the dimension $x_0$ horizontally and $y_0$ vertically, or to shrink it
with negative dimensions.  This {\em\Uidx\bgext} can be asymmetric giving
another optional $|(|x_1|,|y_1|)|$ so that it acts on the right-bottom
corner while let $|(|x_0|,|y_0|)|$ shift only the left-top corner.
Moreover, you may make each \bgext{} {\em infinite} by giving 10000\,|pt|
(about 3.5\,m) to $x_0$, $y_0$, $x_1$ and/or $y_1$ so that the
corresponding region edge is shifted to the paper edge.  Furthermore, this
{\em\Uidx\bginfext{}} can be terminated at the point $\alpha$ inside the
corresponding paper edge by giving $10000\,|pt|-\alpha$
($\alpha\leq1000\,|pt|$) to an extension parameter $x_0$, etc.
\switchcolumn
可选的 $|(|x_0|,|y_0|)|$ 是为了扩大要着色的区域,将其左上角和右下角分别水平和垂直地移出维度 $x_0$ 和 $y_0$,或者用负维度来缩小它。这个{\em\Uidx\bgext}可以是不对称的,可以给出另一个可选的 $|(|x_1|,|y_1|)|$,让它作用于右下角,而 $|(|x_0|,|y_0|)|$ 只移动左上角。此外,您可以通过将 $x_0$、$y_0$、$x_1$ 和/或 $y_1$ 设置为 10000,|pt|(约为 3.5,m)来使每个\bgext{}变为{\em 无限},从而将相应的区域边缘移动到纸张边缘。此外,通过将扩展参数 $x_0$ 等设置为 $10000,|pt|-\alpha$($\alpha\leq1000,|pt|$),这个{\em\Uidx\bginfext{}}可以在相应的纸张边缘内的点 $\alpha$ 处终止。
\end{paracol}


\begin{itemize}
\columnratio{0.55}
\begin{paracol}{2}
\item
A region whose color is not specified is not painted and thus left blank
(or kept as painted by \!\pagecolor! if you specify it).
\switchcolumn\item
未指定颜色的区域不会被绘制,因此保持为空白(或者如果您指定了 \!\pagecolor!,则保持为 \!\pagecolor! 绘制的颜色)。
\switchcolumn[0]*\item
Under-painting of columns and gaps by |C| and |G| is made for regions
different from those over-painting |c| and |g|.  That is, under-painting
is done ignoring all \pwstuff{} and thus the height of the regions is
always $\!\textheight!+\!\maxdepth!$.  On the other hand, over-painting is
only for chunks shrunk or separated by \pwstuff.
\switchcolumn\item
对于与覆盖|c|和|g|不同的区域,通过|C|和|G|进行的列和间隙的底层绘制是独立的。也就是说,底层绘制忽略所有\pwstuff{},因此区域的高度始终为 $\!\textheight!+\!\maxdepth!$。另一方面,覆盖绘制仅适用于通过\pwstuff{}缩小或分离的块。
\switchcolumn[0]*\item
You may exploit the following painting order, where $x_i$
is the $i$-th \mctext{} ($x\in\{|s|,|S|\}$) or $i$-th chunk followed by
the $i$-th \mctext, $m$ and $n$ is the number of \mctext{}s and columns in
a page respectively, to overlay a preceding region with a succeeding
region, if your \emph{printer} allows overlaid color painting.
\begin{eqnarray*}
|T|&\to&|B|\to|L|\to|R|
    \to|G[|0|]|\to\cdots\to|G[|n{-}1|]|\to|C[|0|]|\to\cdots\to|C[|n{-}1|]|\\
&\to&|t|\to|b|\to|l|\to|r|\to|N|\to|n|\to\{|F|,|P|\}\to\{|f|,|p|\}
\to|S|_1\to\cdots\to|S|_m\\
&\to&|g|_1|[|0|]|\to\cdots|g|_1|[|n{-}2|]|\to
    |c|_1|[|0|]|\to\cdots|c|_1|[|n{-}1|]|\to|s|_1\\
&\to&\cdots\\
&\to&|g|_m|[|0|]|\to\cdots|g|_m|[|n{-}2|]|\to
    |c|_m|[|0|]|\to\cdots|c|_m|[|n{-}1|]|\to|s|_m\\
&\to&|g|_{m+1}|[|0|]|\to\cdots|g|_{m+1}|[|n{-}2|]|\to
    |c|_{m+1}|[|0|]|\to\cdots|c|_m|[|n{-}1|]|
\end{eqnarray*}
\switchcolumn\item
您可以利用以下绘制顺序,其中$x_i$是第$i$个\mctext{}($x\in{|s|,|S|}$)或第$i$个块之后的第$i$个\mctext{},$m$和$n$分别是页面上的\mctext{}和列的数量,以将前一个区域与后一个区域叠加在一起,如果您的\emph{打印机}允许叠加颜色绘制。
\begin{eqnarray*}
|T|&\to&|B|\to|L|\to|R|
    \to|G[|0|]|\to\cdots\to|G[|n{-}1|]|\to|C[|0|]|\to\cdots\to|C[|n{-}1|]|\\
&\to&|t|\to|b|\to|l|\to|r|\to|N|\to|n|\to\{|F|,|P|\}\to\{|f|,|p|\}
\to|S|_1\to\cdots\to|S|_m\\
&\to&|g|_1|[|0|]|\to\cdots|g|_1|[|n{-}2|]|\to
    |c|_1|[|0|]|\to\cdots|c|_1|[|n{-}1|]|\to|s|_1\\
&\to&\cdots\\
&\to&|g|_m|[|0|]|\to\cdots|g|_m|[|n{-}2|]|\to
    |c|_m|[|0|]|\to\cdots|c|_m|[|n{-}1|]|\to|s|_m\\
&\to&|g|_{m+1}|[|0|]|\to\cdots|g|_{m+1}|[|n{-}2|]|\to
    |c|_{m+1}|[|0|]|\to\cdots|c|_m|[|n{-}1|]|
\end{eqnarray*}
\switchcolumn[0]*
\item
If you specify |b| feature by \!\twosided!, \bgpaint{} is
{\em\Uidx\mirror{}ed} in even-numbered pages so that |l| and |L| mean
right margin, |r| and |R| mean left margin, and asymmetric extensions are
applied to right-top and left-bottom corners.
\switchcolumn\item
如果您通过 \!\twosided! 命令指定了|b|特性,那么在偶数页上\bgpaint{}会被{\em\Uidx\mirror{}反转},这样|l|和|L|表示右边距,|r|和|R|表示左边距,并且对右上角和左下角应用非对称扩展。
\switchcolumn[0]*\item
To give a color for \bgpaint{} correctly, you need to load \textsf{color}
package or its relative (e.g., \textsf{xcolor}) which the implementation
of coloring in \textsf{paracol} relies on.
\switchcolumn\item
要正确给\bgpaint{}着色,您需要加载\textsf{color}包或其相关包(例如\textsf{xcolor}),因为\textsf{paracol}中的着色实现依赖于它们。
\switchcolumn[0]*\item
To paint margins and regions having infinite extension correctly, the
parameters \!\paperwidth! and \!\paperheight! should be set properly by,
for example, a paper selection option of \!\documentclass!.
\switchcolumn\item
为了正确绘制具有无限扩展的边距和区域,\!\paperwidth! 和 \!\paperheight! 参数应该通过 \!\documentclass! 的纸张选择选项正确设置。
\switchcolumn[0]*\item
Section~\ref{sec:bgpaint} shows examples of \bgpaint{} to give you more
intutive explanations of \!\backgroundcolor! and its region specifications.
\switchcolumn\item
第~\ref{sec:bgpaint}节展示了\bgpaint{}的示例,以便更直观地解释 \!\backgroundcolor! 及其区域规范。
\end{paracol}
\end{itemize}



\item[\Midx{\!\nobackgroundcolor!}\marg{region}]\mbox{}
\Item[\Midx{\!\resetbackgroundcolor!}]\mbox{}\par
\changes{v1.3-3}{2013/09/17}
{Add description of \cs{nobackgroundcolor} and
    \cs{resetbackgroundcolor}.}

The command \!\nobackgroundcolor! declares that the \bground{} of
\meta{region} is not painted, where \meta{region} is one of legitimate
region specifiers of \!\backgroundcolor!.  The command
\!\resetbackgroundcolor! declares no regions are painted and thus gives
you the default state.

命令 \!\nobackgroundcolor! 声明\meta{region}的\bground{}不被绘制,其中\meta{region}是 \!\backgroundcolor! 的合法区域指示符之一。命令 \!\resetbackgroundcolor! 声明没有区域被绘制,从而恢复默认状态。

\begin{itemize}
\item
If you specified the \bgpaint{} of |c|\oarg{col} or |g|\oarg{col} by
\!\backgroundcolor!, the painting is \emph{not} canceled by
\!\nobackgroundcolor! with |c| or |g| but without \oarg{col}.  Similarly,
once you made declarations of \bgpaint{} of both |c| and |c|\oarg{col}
(resp.\ |g| and |g|\oarg{col}), \!\nobackgroundcolor! with |c|\oarg{col}
(resp.\ |g|\oarg{col}) cancels the painting of |c|\oarg{col} (resp.\
|g|\oarg{col}) but the region will still be painted by the color you gave
to |c| (resp.\ |g|).

如果您通过 \!\backgroundcolor! 指定了 |c|\oarg{col} 或 |g|\oarg{col} 的 \bgpaint{},则使用不带 \oarg{col} 的 |c| 或 |g| 的 \!\nobackgroundcolor! 不会取消绘制。同样,一旦您对 |c| 和 |c|\oarg{col}(或 |g| 和 |g|\oarg{col})都进行了声明,使用 |c|\oarg{col}(或 |g|\oarg{col})的 \!\nobackgroundcolor! 将取消 |c|\oarg{col}(或 |g|\oarg{col})的绘制,但区域仍然会使用您给出的颜色进行绘制。
\end{itemize}



\item[\Midx{\!\pagerim!}]\mbox{}\par
\changes{v1.3-3}{2013/09/17}
{Add description of \cs{pagerim}.}

This is a (kind of) \emph{length command}\footnote{

In reality, it is a \cs{dimen} register rather than a \cs{skip} register.}

这是一种(某种程度上的)\emph{长度命令}\footnote{实际上,它是一个\cs{dimen}寄存器,而不是\cs{skip}寄存器。}。

to have the width of the \emph{rim} area placed at each paper edge to 
inhibit \bgpaint{} in the area.  That is, the inner edges of the area are
considered as virtual paper edges to block painting of all margins and
regions having \bginfext{} to the edges, for example in order to
avoid printing troubles caused by painting the rim area too close to the
real paper edges.  The default value of \!\pagerim! is 0 to allow paint
anywhere in a paper.

为了使每个纸张边缘的\emph{边缘}区域的宽度用于抑制该区域内的\bgpaint{}。也就是说,该区域的内部边缘被视为虚拟纸张边缘,以阻止所有具有\bginfext{}到边缘的边缘和区域的着色,例如为了避免将边缘区域着色过于靠近真实纸张边缘而造成的打印问题。\!\pagerim! 的默认值为 0,允许在纸张的任何位置进行着色。
\end{description}
