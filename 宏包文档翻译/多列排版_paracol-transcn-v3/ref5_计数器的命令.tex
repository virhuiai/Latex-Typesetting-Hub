
 \subsection{计数器的命令\hfill Commands for Counters}
 \label{sec:ref-counter}
 
\begin{description}
\item[\Midx{\!\globalcounter!}\marg{ctr}]\mbox{}
\Item[\Midx{\!\globalcounter!}\texttt{*}]\mbox{}\par
\columnratio{0.55}
\begin{paracol}{2}
The command \!\globalcounter!\marg{ctr} declares that the counter
\meta{ctr} is global to all columns, while \!\globalcounter!|*| does so
for all counters.  An update of a \Uidx\gcounter{} in a column is seen by
any other columns.
\switchcolumn
命令 \!\globalcounter!\marg{ctr} 声明计数器\meta{ctr}在所有列中是全局的,而 \!\globalcounter!|*| 则对所有计数器都是如此。在某列中更新全局计数器会被其他列看到。 
\end{paracol}

\begin{itemize}
\columnratio{0.55}
\begin{paracol}{2}

\item
All column-local values of a descendant \lcounter{} of a \gcounter{} are
zero-cleared when the \gcounter{} is explicitly stepped by \!\stepcounter!
or \!\refstepcounter!, or implicitly by a sectioning command and so on.
\switchcolumn
当一个全局计数器被 \!\stepcounter! 或 \!\refstepcounter! 显式步进,或者通过节标题命令等隐式步进时,其子孙局部计数器的所有列局部值都会被清零。
\switchcolumn[0]*
\item
The counter \counter{page} is always global but an explicit update of it
by e.g., \!\setcounter! in a non-leftmost column is not seen by other
columns and is canceled even for the column itself after a \cswitch{} or a
page break in the column.  Therefore, if you want to make a \emph{jump} of
\counter{page}, it must be done in the leftmost column 0.  Note that a
jump from a page $p$ to $q$ can be seen in other columns even if they have
gone beyond $p$ \emph{before} the column 0 makes the jump, as far as
\counter{page} having $q$ (or its successor) is referred to by \!\pageref!
or through \emph{contents} files such as |.toc|\footnote{
Direct reference to \counter{page} may give an inconsistent result, as you
might have in ordinary \LaTeX{} documents.}.
\switchcolumn\item
计数器\counter{page}始终是全局的,但是在非最左列中通过 \!\setcounter! 进行的显式更新在其他列中是不可见的,并且在该列进行\cswitch{}或页面断页后,甚至对于该列本身也会被取消。因此,如果要进行\emph{jump}(即跳转)\counter{page},必须在最左列0中进行。请注意,即使其他列在列0进行跳转\emph{之前}已经超过了页面$p$,只要\counter{page}具有$q$(或其后继者)的值,并且通过 \!\pageref! 或通过\emph{contents}文件(如|.toc|)进行引用,其他列仍然可以看到从页面$p$跳转到$q$。\footnote{直接引用 \counter{page} 可能会导致不一致的结果,就像在普通的 \LaTeX{}文档中可能遇到的那样。}
\switchcolumn[0]*
\item
All counters except for \counter{page} are local by default.  This feature
may cause a problem with some packages including \textsf{marginnote} and
\textsf{(auto-)pst-pdf} having their own counters which must be global.
Since it is tough to find the name of such counters from package sources,
if you have something wrong with these (or other) packages, try to put
\!\globalcounter!|*| in your preamble and use \!\localcounter! shown below
to localize specific counters which you need to be local.
\switchcolumn\item
除了\counter{page}计数器外,默认情况下所有计数器都是局部的。这一特性可能会导致一些包(包括\textsf{marginnote}和\textsf{(auto-)pst-pdf})出现问题,这些包具有必须是全局的计数器。由于很难从包的源代码中找到这些计数器的名称,如果您在使用这些(或其他)包时遇到问题,请尝试在导言区中使用 \!\globalcounter!|*| 命令,并使用下面显示的 \!\localcounter! 命令将需要局部化的特定计数器局部化。
\switchcolumn[0]*
\item
Globalizing a \meta{ctr} being already global is just ignored without any
complaints.
\switchcolumn\item
如果一个已经是全局的\meta{ctr}被再次全局化,它会被静默地忽略,而不会有任何警告。
\end{paracol}
 \end{itemize}
 
 
 
 \item[\Midx{\!\localcounter!}\marg{ctr}]\mbox{}\par
 The command declares that the counter \meta{ctr} is local for each column.

这个命令声明计数器\meta{ctr}在每个栏目中都是局部的。
 \begin{itemize}
 \item
 Though this command is intended for localizing a \meta{ctr} which is once
 globalized, localizing a local counter does not causes any error but is
 just ignored.  Localizing the permanently global \counter{page} is also
 just ignored without any complaints.

 尽管该命令旨在将一次全局化的\meta{ctr}局部化,但将局部计数器局部化不会引起任何错误,只是被忽略。将永久全局\counter{page}局部化也只是被忽略,没有任何警告。
 \end{itemize}
 
 \item[\Midx{\!\definethecounter!}\marg{ctr}\marg{col}\marg{rep}]\mbox{}\par
 The command defines |\the|\meta{ctr} being \marg{rep} for the local use in
 the column \meta{col}.  That is, |\the|\meta{ctr} in the column \meta{col}
 acts as if it is defined by
 \!\renewcommand!\Arg{\cs{the}\meta{ctr}}\Arg{\meta{rep}}.

 该命令定义 |\the|\meta{ctr} 作为在列 \meta{col} 中的局部使用,其值为 \marg{rep}。也就是说,在列 \meta{col} 中,|\the|\meta{ctr} 的行为就像是通过 \!\renewcommand!\Arg{\cs{the}\meta{ctr}}\Arg{\meta{rep}} 定义的一样。
 
 
 
 \item[\Midx{\!\synccounter!}\marg{ctr}]\mbox{}\par
 The command \emph{broadcasts} the value of the \lcounter{} \meta{ctr} in
 the column in which the command appears to the values in all other columns.
 
 该命令将出现在的列中的\lcounter{} \meta{ctr}的值向所有其他列中的值进行\emph{broadcasts}(即广播)。
 \item[\Midx{\!\syncallcounters!}]\mbox{}\par
 The command broadcasts the values of all \lcounter{}s in the column in
 which the command appears to the values in all other columns.

该命令将出现在其中的列中的所有\lcounter{}的值广播到所有其他列中的相应值。
 \end{description}
 