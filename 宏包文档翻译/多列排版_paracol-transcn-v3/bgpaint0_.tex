\backgroundcolor{t}[rgb]{0.7,0,0}
\backgroundcolor{b}[rgb]{0.8,0.6,0}
\backgroundcolor{l}[rgb]{0,0,0.7}
\backgroundcolor{r}[rgb]{0,0.7,0}
\backgroundcolor{c[0]}[rgb]{1,0.8,1}
\backgroundcolor{c[1]}[rgb]{1,1,0.8}
\backgroundcolor{g}[rgb]{0.8,1,1}
\backgroundcolor{f}[rgb]{0.8,0,1}
\backgroundcolor{n}[rgb]{0.8,0.6,1}
\backgroundcolor{p}[rgb]{0.8,1,0.6}
\backgroundcolor{s}[rgb]{0.8,0.8,0.8}
\pagerim5pt

\section{Examples of Background Painting\hfill 背景绘制的示例}
\label{sec:bgpaint}
\subsection{Fundamental Painting}
\label{sec:bgpaint-fund}
\twosided[pcm]

As you undoubtedly notice, this page and a few pages following it are
colorfully painted.  For this and the next three pages, the author
declared the \bground{} color of each region as follows.

正如你无疑注意到的,本页和随后的几页都是色彩斑斓的。对于这四页,作者将每个区域的\bground{}颜色声明如下。
\begin{itemize}\item[]
|\backgroundcolor{t}[rgb]{0.7,0,0}       % dark red for top margin|\\
|\backgroundcolor{b}[rgb]{0.8,0.6,0}     % dark orange for bottom margin|\\
|\backgroundcolor{l}[rgb]{0,0,0.7}       % dark blue for left margin|\\
|\backgroundcolor{r}[rgb]{0,0.7,0}       % dark green for right margin|\\
|\backgroundcolor{c[0]}[rgb]{1,0.8,1}    % pink for colunmn-0|\\
|\backgroundcolor{c[1]}[rgb]{1,1,0.8}    % cream yellow for column-1|\\
|\backgroundcolor{g}[rgb]{0.8,1,1}       % light blue for the gap|\\
|\backgroundcolor{f}[rgb]{0.8,0,1}       % purple for page-wise floats|\\
|\backgroundcolor{n}[rgb]{0.8,0.6,1}     |
    |% light purple for page-wise footnotes|\\
|\backgroundcolor{p}[rgb]{0.8,1,0.6}     |
    |% pale green for pre/post-environment|\\
|\backgroundcolor{s}[rgb]{0.8,0.8,0.8}   % light gray for spanning texts|
\end{itemize}

\SpecialUsageIndex{\backgroundcolor}

Therefore, the \bground{} of this |p|re-environment paragraph and other
stuff above is painted by pale green.

因此,这个|p|re-environment段落以及上面的其他内容的\bground{}被涂成了浅绿色。

\Index{pre-environment stuff}

Since the author set \Uidx{\!\pagerim!} to be 5\,|pt|, you will see
unpainted strips of 5\,|pt| wide at all paper edges surrounding painted
regions.  For this and the next three pages, \Uidx{\!\twosided!}|[pcm]| is
declared to enable |p|, |c| and |m| features but to disable the |b|
feature.  Therefore, though this page \pageref{sec:bgpaint} is even and
thus the left outside margin is wider than the right inside one, the
\bground{}s of |l|(eft) and |r|(ight) margins are painted by dark blue and
dark green respectively.

由于作者将 \Uidx{\!\pagerim!} 设置为5,|pt|,因此您将在所有围绕着绘制区域的纸张边缘看到宽度为5,|pt|的未涂色条纹。在这一页和接下来的三页中,\Uidx{\!\twosided!}|[pcm]| 被声明为启用|p|、|c|和|m|功能,但禁用|b|功能。因此,尽管本页\pageref{sec:bgpaint}是偶数页,左外边缘比右内边缘宽,但左边缘和右边缘的\bground{}分别被涂成深蓝色和深绿色。

\par\bigskip

\begin{paracol}{2}
This column-0 is now right and inside because of the |c| feature of
\!\twosided! is enabled.  On the other hand, the \bground{} is this column
is painted by pink because \!\backgroundcolor! for |c[0]| specifies so.
That is, the column ordinals optionally given to |c|(olumn) (and |g|(ap))
regions are \emph{logical} ones not always corresponding to their
\emph{physical} positions in a page.

\switchcolumn
\begingroup\it
As explained in the right column-0, the \bground{} of this left and
outside column-1 is painted by cream yellow as
{\rm\!\backgroundcolor!|{c[1]}|} specifies.  Now we have a
{\rm\!\switchcolumn!|*|} with a \mctext{} to show the \bgpaint{} for
it\footnote{

Since the footnotes in this \env{paracol} environment are \scfnote{} and
\mgfnote{}, and \!\backgroundcolor!\texttt{\char`\{n\char`\}} specifies
light purple, the \bground{} of this (foot)|n|(ote) region is painted by
the color.}.

\par\endgroup
\switchcolumn*[\subsection*{The background of this |s|(panning text)
region is painted by light gray}\medskip]

\begin{figure*}\nosv
\def\arraystretch{0.8}
\centerline{\begin{tabular}[b]{|c|}\hline
    \hbox to.9\textwidth{}\\
    \texttt{f}(loat) region for this page-wise figure is painted by purple\\ 
    \\\hline
    \end{tabular}}
\caption{A Page-Wise Figure}
\end{figure*}

This paragraph is to show how the first line of a paragraph just below a
\mctext{} is placed in the painted region.
\par\vfill

\switchcolumn
\begingroup\it
See the right column for the reason why this paragraph is here.
\par\vfill

See the right column for what we are now doing.
\par\endgroup
\switchcolumn

Now we have a \!\flushpage! to see the \bgpaint{} for a material not shown
in the page, i.e., a page-wise float.
\flushpage

Since we are now in an odd-numbered page \pageref{page:bgpaint2}, this
column-0 is now a left one and is still painted by pink of course.
\par\vfill\label{page:bgpaint2}

This paragraph is to show how the last line of a page without \Scfnote{}s
is placed in the painted region.
\par\newpage

This page is to show how the page without any \pwstuff{} looks like.
\par\vfill

Shortly we will close this \env{paracol} environment in the next page.
\par\newpage

Now we are closing this \env{paracol} environment to show how its
\postenv{} is painted.

\switchcolumn
\begingroup\it
As expected, the \bground{} of this column-1 is still painted by cream
yellow.
\par\vfill

See the comment in the left column.
\par\newpage

See the right column for the reason why we have this almost blank page.
\par\vfill

See the right column for what will happen shortly.
\par\newpage

See the left column for the reason why we are now closing the environment.
\endgroup
\end{paracol}
\bigskip

The \bground{} of this paragraph in |p|(ost-environment) region is also
painted by pale green, because \postenv{} can be \preenv{} at the same
time as we see shortly.  

这个段落在|p|(ost-environment)区域的\bground{}也被涂成了淡绿色,因为正如我们很快会看到的那样,\postenv{}可以同时是\preenv{}。\par\bigskip

\begin{paracol}{2}
This short \env{paracol} environment illustrates how the \preenv{} of this
environment, or the \postenv{} of the last environment in other words, is
painted.

\switchcolumn
\begingroup\it
Therefore, the author does not have much to say in this column, except for
giving a footnote here\footnote{

Since this footnote is \mgfnote{} with that in the \postenv{}, it is
considered as a part of \postenv{} and thus painted by pale green rather
than light purple.\label{fn:bgpaint1}}.
\endgroup
\end{paracol}
\bigskip

Before moving to the next example, one caution is given for \bgpaint{} of
\Mgfnote{}s.  As the footnote \ref{fn:bgpaint1} itself says, \Mgfnote{}s
given in the \lpage{} of a \env{paracol} environment are considered as
belonging to \postenv{}.  Therefore, the footnote \ref{fn:bgpaint1} is
painted by pale green as well as another footnote given now\footnote{

Since this footnote really belongs to \postenv{}, its \bground{} is painted
by pale green naturally.}.

在进入下一个示例之前,对于\Mgfnote{}的\bgpaint{}有一个注意事项。正如脚注\ref{fn:bgpaint1}本身所说的那样,在\env{paracol}环境的\lpage{}中给出的\Mgfnote{}被认为属于\postenv{}。因此,脚注\ref{fn:bgpaint1}将被绘制成浅绿色,以及现在给出的另一个脚注\footnote{由于这个脚注确实属于\postenv{},所以它的\bground{}自然会被绘制成浅绿色。}。
\par\label{page:bgpaint4}

