\subsection{Commands \cs{footnote*} and Relatives\hfill \cs{footnote*} 命令及相关命令}
\label{sec:fnnp-starred}

\begin{description}
\item[\Midx{\!\footnote!}\texttt{*}\oarg{|+|disp}\marg{text}]\mbox{}
\Item[\Midx{\!\footnote!}\texttt{*}\oarg{|-|disp}\marg{text}]\mbox{}
\Item[\Midx{\!\footnote!}\texttt{*}\oarg{disp}\marg{text}]\mbox{}\par
\columnratio{0.55}
\begin{paracol}{2}
The command is similar to its non-starred counterpart but the explicit
numbering with the optional argument is done in \emph{self-relative} or
\emph{base-displacement} style.  That is, if the optional argument has a
leading `|+|' or `|-|',  the number given to the footnote is
$f+\meta{disp}$ or $f-\meta{disp}$ respectively where $f$ is the value of
\counter{footnote} counter, or in other words the number given to the last
footnote\footnote{%
If it is put by the ordinary \cs{footnote}.}.
\switchcolumn
该命令与其非星号版本类似,但是使用可选参数进行的显式编号是以\emph{自相对}或\emph{基准位移}的方式进行的。也就是说,如果可选参数以`|+|'或`|-|'开头,给予脚注的编号分别为$f+\meta{disp}$或$f-\meta{disp}$,其中$f$是\counter{footnote}计数器的值,或者换句话说,是给予最后一个脚注的编号\footnote{如果它是由普通的\cs{footnote}命令放置的。}。
\switchcolumn[0]*
Otherwise, i.e., the optional argument is a number without |+|/|-| sign,
the number given to the footnote is $b+\meta{disp}$ where $b$ is the base
value of \counter{footnote} counter at \beginparacol{} for the environment
in which the command appears, or in other words the number given to the
last \Preenv{} footnote\footnote{

Or the last footnote in the previous \env{paracol} environment,
etc.\label{fn:4L0}}.
\switchcolumn
否则,即可选参数是一个没有 |+|/|-| 符号的数字,则给定的脚注编号是 $b+\meta{disp}$,其中 $b$ 是 \beginparacol{} 处的 \counter{footnote} 计数器的基础值,用于包含该命令的环境,或者换句话说,给定的是最后一个 \Preenv{} 脚注\footnote{或者是前一个 \env{paracol} 环境中的最后一个脚注,等等。\label{fn:4L0}}。
\switchcolumn[0]*
In addition, unlike the non-starred version, this command updates
\counter{footnote} counter with the number given to the footnote, i.e.,
$f\gets f+\meta{disp}$, $f\gets f-\meta{disp}$ or $f\gets b+\meta{disp}$
is performed, so that following \!\footnote! without explicit numbering
option have numbers $f+1$, $f+2$ and so on with new $f$.
\switchcolumn
此外,与非星号版本不同,该命令使用给定的脚注编号更新\counter{footnote}计数器,即执行$f\gets f+\meta{disp}$、$f\gets f-\meta{disp}$或$f\gets b+\meta{disp}$,以便在没有显式编号选项的情况下,后续的 \!\footnote!命令具有编号$f+1$、$f+2$等,并更新$f$的值。
\end{paracol}
\begin{itemize}
\columnratio{0.55}
\begin{paracol}{2}
\item
If the optional argument is not provided, it is assumed that |[+1]| is
given and thus \!\footnote!|*|\marg{text} acts as \!\footnote!\marg{text}.
\switchcolumn\item
如果没有提供可选参数,则假定提供了|[+1]|,因此 \!\footnote!|*|\marg{text}的作用等同于 \!\footnote!\marg{text}。
\end{paracol}
\end{itemize}

\item[\Midx{\!\footnotemark!}\rm|*[|{[|+-|]}\meta{disp}{|]|}]\mbox{}\par
\columnratio{0.55}
\begin{paracol}{2}
This command is a mixture of its non-starred counterpart and
\!\footnote!|*|.  That is the number for the footnote mark is calculated
in the way of \!\footnote!|*| and \counter{footnote} counter is updated.
\switchcolumn
这个命令是它的非星号版本和 \!\footnote!||的混合体。即脚注标记的编号是根据 \!\footnote!|*|的方式计算的,并且\counter{footnote}计数器会被更新。   
\end{paracol}
\item[\Midx{\!\footnotetext!}\rm|*[|{[|+-|]}\meta{disp}{|]|}\marg{text}]
\mbox{}\par
\columnratio{0.55}
\begin{paracol}{2}
Without the optional argument |[|[|+-|]\meta{disp}|]|, this command does what
\!\footnotetext!\marg{text} does but in addition increments
\counter{footnote} counter before that.  With the optional argument, on
the other hand, the number given to the footnote \meta{text} is calculated
as done in \!\footnote!, but the \counter{footnote} counter is not
updated.
\switchcolumn
如果没有提供可选参数 |[|[|+-|]\meta{disp}|]|,则此命令的作用与 \!\footnotetext!\marg{text} 相同,但在此之前会增加\counter{footnote}计数器的值。另一方面,如果提供了可选参数,那么给定给脚注\meta{text}的编号将按照 \!\footnote! 的方式计算,但\counter{footnote}计数器不会被更新。    
\end{paracol}
\end{description}

With these starred commands, you can produce the following using the
base-displacement mechanism without worrying about the absolute value of
\!\footnote! counter and its change.

使用这些带星号的命令,您可以使用基础位移机制生成以下内容,而无需担心 \!\footnote! 计数器的绝对值及其变化。
\Hrule
\begin{paracol}{2}
\tolerance5000\hbadness5000
\fnpar{First left-column}{\footnote{First left-column
footnote.\label{fn:4L1}}}
\Fnpar{Second left-column}{\footnote{Second left-column
footnote.\label{fn:4L2}}}{
It is followed by \cs{footnotetext}|*[3]|\marg{text} and a
\cs{switchcolumn}.}
\footnotetext*[3]{Third left-column footnote given by
\cs{footnotetext}|*[3]|\marg{text} placed at the end of
the second left-column paragraph to have
$\ref{fn:4L3}=\ref{fn:4L0}+3$.\label{fn:4L3}}
\switchcolumn
\Fnpar{First right-column}{\footnote*[4]{First right-column
footnote whose number \ref{fn:4R1} is given by
\cs{footnote}|*[4]|\marg{text} because
$\ref{fn:4R1}=\ref{fn:4L0}+4$.\label{fn:4R1}}}{It is followed by a
\cs{switchcolumn*}.}
\switchcolumn*
Third and synchronized left-column paragraph\Dotfill\\
\Dotfill with a footnote whose mark
here\footnotemark*[3]\Dotfill\\
is given by \!\footnotemark!|*[3]| because $\ref{fn:4L3}=\ref{fn:4L0}+3$.
It is followed by a \!\switchcolumn!.
\switchcolumn
\fnpar{Second and synchronized right-column}{\footnote*[5]{Second right-column
footnote produced by \cs{footnote}|*[5]|\marg{text} because
$\ref{fn:4R2}=\ref{fn:4L0}+5$.\label{fn:4R2}}}
\fnpar{Third right-column}{\footnote{Third right-column
footnote produced by \cs{footnote}\marg{text} because
$\ref{fn:4R3}=\ref{fn:4R2}+1$.\label{fn:4R3}}}
\end{paracol}
\newpage

\columnratio{0.55}
\begin{paracol}{2}
The other way to produce the same result except for the absolute footnote
numbers is to use the self-relative mechanism and to exploit the progress
of \counter{footnote} counter as follows.
\switchcolumn
另一种产生相同结果的方法(除了绝对脚注编号)是使用自相对机制,并利用 \counter{footnote} 计数器的进展,方法如下:    
\end{paracol}
\Hrule
\begin{paracol}{2}
\tolerance5000\hbadness5000
\fnpar{First left-column}{\footnote{First left-column
footnote.\label{fn:5L1}}}
\Fnpar{Second left-column}{\footnote{Second left-column
footnote.\label{fn:5L2}}}{
It is followed by \cs{footnotetext}|*|\marg{text} and a
\cs{switchcolumn}.}
\footnotetext*{Third left-column footnote given by
\cs{footnotetext}|*|\marg{text} placed at the end of
the second left-column paragraph because it follows the second footnote
\ref{fn:5L2}.\label{fn:5L3}}
\switchcolumn
\Fnpar{First right-column}{\footnote{First right-column
footnote whose number \ref{fn:5R1} is given by
\cs{footnote}\marg{text} because
$\ref{fn:5R1}=\ref{fn:5L3}+1$ and \cs{footnotetext*} for \ref{fn:5L3} lets
\counter{footnote} have the value.\label{fn:5R1}}}{It is followed by a
\cs{switchcolumn*}.}
\switchcolumn*
Third and synchronized left-column paragraph\Dotfill\\
\Dotfill with a footnote whose mark
here\footnotemark*[-1]\Dotfill\\
is given by \!\footnotemark!|*[-1]| because $\ref{fn:5L3}=\ref{fn:5R1}-1$.
It is followed by a \!\switchcolumn!.
\switchcolumn
\fnpar{Second and synchronized right-column}{\footnote*[+2]{Second
right-column footnote produced by \cs{footnote}|*[+2]|\marg{text} because
$\ref{fn:5R2}=\ref{fn:5L3}+2$.\label{fn:5R2}}}
\fnpar{Third right-column}{\footnote{Third right-column
footnote produced by \cs{footnote}\marg{text} because
$\ref{fn:5R3}=\ref{fn:5R2}+1$.\label{fn:5R3}}}
\end{paracol}
\Hrule

\columnratio{0.55}
\begin{paracol}{2}
It depends on the structure of your document which of the
base-displacement and self-relative is better.  If your document has
frequent switching between single- and multi-column text typesetting and
thus the contents of a \env{paracol} environment is relatively small, the
base-displacement is a good choice because you may concentrate on one
base value of \counter{footnote} counter.  Otherwise, especially when your
document consists of one single and large \env{paracol} environment, the
base-displacement is almost equivalent to maintaining absolute values and
thus the self-relative should be preferred.
\switchcolumn
这取决于你的文档结构,基准位移和自相对哪个更好。如果你的文档经常在单列和多列文本排版之间切换,因此\env{paracol}环境的内容相对较小,那么基准位移是一个不错的选择,因为你可以专注于\counter{footnote}计数器的一个基准值。否则,特别是当你的文档由一个单独且较大的\env{paracol}环境组成时,基准位移几乎等同于维护绝对值,因此应该优先选择自相对方式。
\switchcolumn[0]*
Note that if the last \!\footnote! or \!\footnotemark! in a \env{paracol}
environment is starred, the command lets \counter{footnote} counter have
some value smaller than that for the last stacked footnote.  For example, 
if the second and third right-column footnotes \ref{fn:5R2} and
\ref{fn:5R3} are omitted from the example above, the last footnote-related
command will be \!\footnotemark!|*[-1]| which makes the counter at
\Endparacol{} \ref{fn:5L3} rather than \ref{fn:5R1}.  You may not worry
about this problem, however, because \Endparacol{} automatically maintains
the counter letting it have $b+n$ where $n$ is the number of \!\footnote!
and \!\footnotemark! in the environment, if the maintenance is ordered by
the command \!\fncounteradjustment! which is automatically executed by
\!\footnotelayout! with the argument |p| or |m|.
\switchcolumn
请注意,如果在 \env{paracol} 环境中的最后一个 \!\footnote! 或 \!\footnotemark! 带有星号,那么该命令会使 \counter{footnote} 计数器的值小于最后一个堆叠脚注的值。例如,如果上面的示例中省略了第二个和第三个右列脚注 \ref{fn:5R2} 和 \ref{fn:5R3},那么最后一个与脚注相关的命令将是 \!\footnotemark!|*[-1]|,它使得在 \Endparacol{} 处的计数器为 \ref{fn:5L3} 而不是 \ref{fn:5R1}。然而,您可能不必担心这个问题,因为 \Endparacol{} 会自动维护计数器,使其为 $b+n$,其中 $n$ 是环境中 \!\footnote! 和  \!\footnotemark! 的数量,如果维护是由命令 \!\fncounteradjustment! 规定的,该命令会在 \!\footnotelayout! 中使用参数 |p| 或 |m| 自动执行。    
\end{paracol}

