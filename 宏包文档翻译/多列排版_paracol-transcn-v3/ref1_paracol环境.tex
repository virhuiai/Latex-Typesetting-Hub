% \subsection{Environment \texttt{paracol}\hfill \texttt{paracol}环境}
% \label{sec:ref-paracol}
% 
% \begin{description}
% \item[\ENV{paracol}{\marg{num}\oarg{text}}]\mbox{}\par
% The environment \env{paracol} contains \meta{body} typeset in \meta{num}
% columns in parallel.  The optional \meta{text} is put spanning all columns
% prior to the multi-columned \meta{body}.

% 环境\env{paracol}中包含以 \meta{num} 栏并列排列的 \meta{body}。可选的 \meta{text} 将跨越所有栏之前放置在多栏的 \meta{body} 之前。
% 
% \begin{itemize}
% \item
% \changes{v1.2-2}{2013/05/11}
% 	{Add a footnote mentioning page-wise footnotes merged with
%	 pre-environment staff.}
% \changes{v1.35-4}{2018/12/31}
% 	{Add description of \cs{belowfootnoteskip}.}
% 
% The environment may start from \emph{any} vertical position in a page,
% i.e., not necessary at the top of a page.  The single-column
% {\em\Uidx\preenv} of the {\em\Uidx\spage} in which \beginparacol{} lies
% are naturally connected to the beginning part of \meta{body} in each
% column, unless the page has footnotes\footnote{
% 
% With \Mgfnote{} layout shown in Section~\ref{sec:ref-scfnote}, the
% footnotes in the single-column contents are merged with those in
% \env{paracol} environment and are put at the bottom of the \spage{}
% together as shown in this page.}
% 
% or bottom floats.  If these kinds of bottom stuff exist, they are
% put above the multi-columned \meta{body}, or the spanning \meta{text}
% 
% \UsageIndex{spanning text}
% 
% if provided, with a vertical skip of \!\textfloatsep! separating them if
% bottom floats exist, or of \!\belowfootnoteskip! described in
% Section~\ref{sec:ref-scfnote} if only footnotes exist.  The
% \emph{deferred} floats which have not yet appeared in the starting page
% and thus will appear in the next or succeeding pages are considered as
% \pwise{} floats given in the environment.

%此环境可以从页面的\emph{任何}垂直位置开始,
%即不一定在页面顶部。位于 \beginparacol{} 所在的 {\em\Uidx\spage} 中的单栏 {\em\Uidx\preenv} 自然与每个栏的\meta{body}的开头部分连接在一起,除非页面有脚注\footnote{使用在第~\ref{sec:ref-scfnote}节中展示的\Mgfnote{}布局,单栏内容中的脚注与\env{paracol}环境中的脚注合并在一起,并一起放置在\spage{}的底部,就像本页所示。}
,或底部浮动体。如果存在这些底部内容,则它们将位于多栏的\meta{body}之上,或者位于跨越的\meta{text}之上(如果提供了),并使用垂直间距 \!\textfloatsep! 将它们分隔开(如果存在底部浮动体),或者使用在第~\ref{sec:ref-scfnote}节中描述的 \!\belowfootnoteskip! (仅当存在脚注时)。尚未出现在起始页面中的\emph{延迟}浮动体将被视为在环境中给出的\pwise{}浮动体,它们将出现在下一页或后续页面中。
% 
% \item
% \changes{v1.2-5}{2013/05/11}
% 	{Add an item to show that \string\texttt{paracol} can be enclosed in
%	 a \string\texttt{list}-like environment.}
% 
% The environment can be enclosed in a \env{list}{\em-like environment} such
% as \env{enumerate}, \env{itemize} and \env{description}.  If so, \!\item!s
% in each column are typeset using the parameters of the surrounding
% environment such as \!\leftmargin! and \!\rightmargin!.  For example, the
% following short \env{paracol} environment is included in an \env{itemize}
% for this and other \!\item!s in this page.

% 该环境可以被封装在类似于 \env{enumerate}、\env{itemize} 和 \env{description} 的\emph{类似列表}环境中。如果这样做,每个栏中的 \!\item! 将使用周围环境的参数进行排版,如 \!\leftmargin! 和 \!\rightmargin!。例如,以下简短的\env{paracol}环境被包含在一个\env{itemize}中,用于本页和其他 \!\item!。

% 
% \begin{paracol}{2}
% \item
% This is the first \!\item! in the left column.
% 

这是左栏中的第一个 \!\item!。

% \item
% This is the second \!\item! in the left column followed by a
% \!\switchcolumn!\footnote{.
% 
% This footnote is to show the footnotes in this page are merged.}.

% 这是左栏中的第二个 \!\item!,后面跟着一个 \!\switchcolumn! \footnote{这个脚注是为了展示本页中的脚注是合并在一起的。}。
% \switchcolumn
% 
% \item
% This is the first \!\item! in the right column.

% 这是右栏中的第一个 \!\item!。
% 
% \item
% This is the second \!\item! in the right column.
%
\item
这是右栏中的第二个 \!\item!。
% \item
% This is the third and last \!\item! in the right column.

% 这是右栏中的第三个也是最后一个 \!\item!。
% \end{paracol}
% You are now seeing the switching to/from multi-columned and \env{itemize}d
% texts are naturally connected with the last and this single-columned
% sentences.  You may feel the space between two columns above is too large
% but it simply results from the large total \!\leftmargin!s of the outer
% \env{description} and this \env{itemize}, which make the right column
% shifted right.  A simple remedy for this large space is to make
% \!\columnsep! narrower, for example 0\,pt as shown below.

% 您现在看到的切换到/从多栏和\env{itemize}文本与上一个和本个单栏句子自然连接在一起。您可能会觉得上面两栏之间的空间太大,但这只是由于外部\env{description}和此\env{itemize}的总 \!\leftmargin! 较大,使得右栏向右偏移。修复这个大空间的简单方法是使 \!\columnsep!变窄,例如像下面显示的0\,pt。

\begin{Verbatim}
\columnsep0pt
\end{Verbatim}

% \columnsep0pt
% \begin{paracol}{2}
% \item
% This \!\item! is wider than the last \!\item! above because
% \!\columnsep! is 0\,pt.

% 这个 \!\item! 比上面的最后一个 \!\item! 更宽,因为 \!\columnsep! 是0\,pt。
% \switchcolumn
% 
% \item
% Therefore, this \!\item! is shifted left a little bit to make
% inter-column spece narrower.

% 因此,为了使栏间距更窄,这个 \!\item! 向左移动了一点。
% \end{paracol}
% 
% \item
% All \Uidx\lcounter{}s in all columns are initialized to have the values at
% \beginparacol{} on its first occurrence.  On the second and succeeding
% occurrences of \beginparacol, the \lcounter{}s in each column have the
% value at the last \Endparacol, unless they are modified after the
% \Endparacol.  If a counter is modified (or declared by \!\newcounter!)
% after the \Endparacol, the local versions of the counter in all columns
% commonly have the value at \beginparacol.

% 所有栏中的 \Uidx\lcounter{} 都被初始化为\beginparacol{}第一次出现时的值。在\beginparacol{}的第二次及后续出现中,每个栏中的\lcounter{}都具有上一个\Endparacol{}处的值,除非在\Endparacol{}之后对其进行了修改。如果在\Endparacol{}之后修改了计数器(或通过 \!\newcounter! 声明了计数器),所有栏中的局部计数器都通常具有\beginparacol{}处的值。
% 
% \item
% \changes{v1.2-2}{2013/05/11}
% 	{Add a footnote mentioning page-wise footnotes merged with
%	 post-environment staff.}
% 
% The environment may end at \emph{any} vertical position in a page, i.e.,
% the {\em\Uidx\postenv} being the single-column texts and others
% following \Endparacol{} in the {\em\Uidx\lpage} of the environment may not
% start from the top of a page.  If any columns don't have deferred
% \cwise{} floats and the most advanced {\em\Uidx\lcolumn} at
% \Endparacol{} has neither of footnotes\footnote{
% 
% With \Mgfnote{} layout shown in Section~\ref{sec:ref-scfnote}, the
% footnotes in the closing \env{paracol} environment are merged with those
% in \postenv{} and are put at the bottom of the page{} together as shown in
% this page.}
% 
% nor bottom floats, its bottom is naturally connected to the \postenv{}.
% If the \lcolumn{} has these kinds of bottom stuff, they are put above the
% \postenv{}, with a vertical skip of \!\textfloatsep! separating them if
% bottom floats exist.  All deferred \cwise{} floats given in the
% environment are flushed before the \postenv{} appears, possibly creating
% {\em\Uidx\fcolumn{}s} only with floats.  On the other hand, deferred
% \pwise{} floats given in the environment are considered as deferred
% (single-) \cwise{} floats given just after \Endparacol.
% 

% 该环境可以在页面的\emph{任何}垂直位置结束,即\emph{\Uidx\postenv}是单栏文本,而在环境的\emph{\Uidx\lpage}中的\Endparacol{}之后的其他内容可能不会从页面顶部开始。如果任何栏没有延迟的\cwise{}浮动体,并且最后一个\Endparacol{}处的\emph{\Uidx\lcolumn}既没有脚注\footnote{使用在第~\ref{sec:ref-scfnote}节中展示的\Mgfnote{}布局,\env{paracol}环境中的脚注与\postenv{}中的脚注合并在一起,并一起放置在页面底部,就像本页所示。},也没有底部浮动体,则其底部自然与\postenv{}连接在一起。如果\lcolumn{}具有这些类型的底部内容,则它们将位于\postenv{}之上,如果存在底部浮动体,则它们之间使用垂直间距 \!\textfloatsep! 分隔开。在\postenv{}出现之前,环境中给出的所有延迟\cwise{}浮动体都会被清除,可能只留下具有浮动体的{\em\Uidx\fcolumn{}s} 。另一方面,环境中给出的延迟\pwise{}浮动体被视为在\Endparacol{}之后立即给出的延迟(单个)\cwise{}浮动体。

% \item
% The values of all \lcounter{}s in the leftmost column are used as the
% initial values of them in the \postenv.
%
% 左侧栏中所有\lcounter{}的值被用作\postenv{}中对应\lcounter{}的初始值。
% \item
% The \env{paracol} environment cannot be nested, or you will have an error
% message of illegal nesting.

% 不能嵌套使用\env{paracol}环境,否则会出现非法嵌套的错误消息。 
% \item
% The commands \!\switchcolumn!, \!\synccounter!, \!\syncallcounters! and
% \!\flushpage!, and environments \env{column}(|*|), \env{nthcolumn}(|*|),
% \env{leftcolumn}(|*|) and \env{rightcolumn}(|*|) are {\em local} to
% \env{paracol} environment and thus undefined outside the
% environment\footnote{
% 
% Unless you dare to define them.}.
% 
% The command \!\clearpage! is of course usable outside and inside the
% environment but its function inside is a little bit different from outside.

% 命令 \!\switchcolumn!、\!\synccounter!、\!\syncallcounters! 和 \!\flushpage!,以及环境\env{column}(|*|)、\env{nthcolumn}(|*|)、\env{leftcolumn}(|*|)和\env{rightcolumn}(|*|)是\env{paracol}环境中的{\em 局部}命令和环境,因此在环境外部是未定义的\footnote{除非你敢于定义它们。}。

% 命令 \!\clearpage! 当然可以在环境内外使用,但在环境内部的功能与外部略有不同。
% \end{itemize}
% 
% 
% 
% \item[\ENV{paracol}{\oarg{numleft}\marg{num}\oarg{text}}]\mbox{}
% \Item[\ENV{paracol}{\oarg{numleft}\texttt{*}\marg{num}\oarg{text}}]
%   \mbox{}\par
% \changes{v1.3-2}{2013/09/17}
%	{Add description of parallel-paging.}
% 
% If a \beginparacol{} has the optional \meta{numleft} argument to specify
% the number of leading columns $n_l$ together with the total $n$ given by
% \meta{num}, columns in the environment are laid out across two adjacent
% pages.  In this {\em\Uidx\parapag{}e} typesetting, the first $n_l$ columns
% are placed in the {\em left} page while remaining $n_r=n-n_l$ columns go to
% the next {\em right} page.  The pair of left and right pages is
% considered as comprising a virtual {\em\Uidx\paired} page and thus shares
% a common page number, unless {\em\Uidx\npaired} typesetting is specified
% by the optional `|*|' following the optional \meta{numleft} argument.  In
% the \npaired{} \parapag{}ing, when the leading $n_l$ columns are put in a
% page $p$, the trailing $n_r$ columns are in the page $p+1$.

% 如果\beginparacol{}的可选参数\meta{numleft}用于指定前导列的数量$n_l$,同时总列数由\meta{num}给出,那么环境中的列会跨两个相邻的页面进行布局。在这种{\em\Uidx\parapag{}e}排版中,前$n_l$列放置在{\em 左侧}页面,而剩下的$n_r=n-n_l$列放置在下一个{\em 右侧}页面。左侧和右侧页面的配对被认为是组成一个虚拟的{\em\Uidx\paired}页面,因此它们共享一个相同的页码,除非通过在可选的\meta{numleft}参数后面添加`|*|'来指定{\em\Uidx\npaired}排版。在\npaired{} \parapag{}ing中,当前导的$n_l$列放置在页面$p$上时,后续的$n_r$列会在页面$p+1$上。

% \begin{itemize}
% \item
% All {\em\Uidx\pwstuff}, i.e., \Preenv{} and \postenv, \pwise{} floats,
% \mctext{} and (\mgfnote{} or non-merged) \Scfnote{}s, are placed only in
% left \parapag{}es leaving corresponding regions in right \parapag{}es
% blank\footnote{
% 
% Someday the author could devise an advanced mechanism to exploit the space
% in right \parapag{}es.}.
% 
% 所有的{\em\Uidx\pwstuff},即\Preenv{}和\postenv,\pwise{}浮动体,\mctext{}和(\mgfnote{}或非合并的)\Scfnote{},只会放置在左侧\parapag{}es中,让右侧\parapag{}es中相应的区域保持空白\footnote{将来作者可能会设计一个高级机制来利用右侧\parapag{}es中的空间。}。
% \item
% A \npaired{} left \parapag{}e is not necessary to be even-numbered, though
% the printing tradition requires so if you naturally want to have a
% \parapag{}e pair in a double spread.  The page number given to the first
% left \parapag{}e is simply the number of the page $p_1$ in which
% \beginparacol{} reside, and that for the $k$-th left \parapag{}e is
% $p_1+2(k-1)$\footnote{
% 
% Unless you make some change to \counter{page} counter.}.
% 
% Therefore, to make it sure $p_1$ is even, you might need to have an
% ordinary page of blank, a title, etc., or to let \counter{page} counter have
% an even number by \!\setcounter!, etc., before starting a \env{paracol}
% environment.
%
% 一个没有成对出现的左页不一定是偶数页,尽管印刷传统要求如果你自然地希望在双页中有一个成对的页面。第一个左页的页码只是在\beginparacol{}所在的页$p_1$的页码,而第$k$个左页的页码是$p_1+2(k-1)$\footnote{除非你对\counter{page}计数器进行了一些更改。}。

% 因此,为了确保$p_1$是偶数,你可能需要在开始\env{paracol}环境之前有一个普通的空白页、一个标题等,或者通过 \!\setcounter! 等方法使\counter{page}计数器的值成为一个偶数。

% \item
% Section~\ref{sec:ppts} shows examples of \parapag{}ing together with
% related issues on two-sided typesetting.

% 第~\ref{sec:ppts} 节展示了 \parapag{} 的示例,以及双面排版相关问题。
% \end{itemize}
% \end{description}
% 