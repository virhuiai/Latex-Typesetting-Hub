
\section{Chunk Size\\块大小}

\columnratio{0.55}
\begin{paracol}{2}
\DescribeMacro{LTchunksize}
In order to \TeX\ multi-page tables, it is necessary to break up the
table into smaller chunks, so that \TeX\ does  not have to keep
everything in memory at one time. By default \env{longtable} uses 20
rows per chunk, but this can be set by the user, with e.g.,
"\setcounter{LTchunksize}{10}".\footnote
   {You can also use the plain \TeX\ syntax
   {\ttfamily\v\\LTchunksize=10.}}
These chunks do not affect page breaking,
thus if you are using a \TeX\ with a lot of memory, you can set
"LTchunksize" to be several pages of the table. \TeX\ will run
faster with a large "LTchunksize". However, if necessary,
\env{longtable} can work with "LTchunksize" set to 1, in which case
the memory taken up is negligible.
Note that if you use the commands for setting the table head or foot
(see below), the "LTchunksize" must be at least  as large as the
number of rows in each of the head or foot sections.
\switchcolumn
\DescribeMacro{LTchunksize}
为了能够在 \TeX 中处理多页表格,需要将表格分成较小的块,这样 \TeX 不必一次性将所有内容都存入内存中。默认情况下,\env{longtable} 每块包含 20 行,但用户可以通过例如 "\setcounter{LTchunksize}{10}" 的方式设置块大小。\footnote{你也可以使用普通的 \TeX 语法 {\ttfamily\v\\LTchunksize=10.}}这些块不会影响页面分割,因此如果你使用的 \TeX 具有大量内存,可以将 "LTchunksize" 设置为表格的几个页面。使用大的 "LTchunksize",\TeX 会运行得更快。但是,必要时,\env{longtable} 可以将 "LTchunksize" 设置为1,此时所占用的内存可以忽略不计。请注意,如果你使用设置表头或表尾的命令(见下文),则 "LTchunksize" 必须至少与表头或表尾中每个部分的行数一样大。

\switchcolumn[0]*
This document specifies "\setcounter{LTchunksize}{200}".  If you look
at the previous table, after the \emph{first} run of \LaTeX\  you will
see that various parts of the table do not line up.
\LaTeX\ will also have printed a warning that the column
widths had changed. \env{longtable} writes information onto the
".aux" file, so that it can line up the different chunks.
Prior to version~4 of this package, this information was not used
unless a "\setlongtables" command was issued,  however, now the
information is always used, using a new algorithm\footnote{Due to
David Kastrup.} and so "\setlongtables" is no longer needed. It is
defined (but does nothing) for the benefit of old documents that
use it.
\switchcolumn
本文档规定了"\setcounter{LTchunksize}{200}"。如果您查看前面的表格,在第一次运行\LaTeX\ 后,您会发现表格的各个部分没有对齐。\LaTeX\ 还会打印一个警告,表示列宽已更改。\env{longtable} 将信息写入".aux"文件中,以便对不同的部分进行对齐。在此软件包的版本4之前,除非发出 "\setlongtables" 命令,否则不使用此信息,但现在始终使用新算法\footnote{由David Kastrup开发。},因此不再需要 "\setlongtables"。为了方便使用旧文档的用户,它被定义(但不起作用)。
% \switchcolumn[0]*
% \begin{table*}
% \centering
% \begin{tabular}{||l|l|l||}
% \hline\hline
% A&\env{tabular}& environment\\
% \hline
% within&a floating&\env{table}\\
% \hline\hline
% \end{tabular}
% \caption{A floating table}
% \end{table*}
% \switchcolumn
% \begin{table*}
% \centering
% \begin{tabular}{||l|l|l||}
% \hline\hline
% 一个&\env{tabular}& 环境\\
% \hline
% 在&浮动&\env{table}内\\
% \hline\hline
% \end{tabular}
% \caption{一个浮动表格}
% \end{table*}
\end{paracol}

\begin{table}
\centering
\begin{tabular}{||l|l|l||}
\hline\hline
A&\env{tabular}& environment\\
\hline
within&a floating&\env{table}\\
\hline\hline
\end{tabular}
\caption{A floating table}
\end{table}
