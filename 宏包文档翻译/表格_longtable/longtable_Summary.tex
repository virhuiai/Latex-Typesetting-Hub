
\section{Summary}

% 让表格粘附到左边距。
^^A Allow the table to stick into the left margin.
\setlength{\LTleft}{0pt plus 1fill minus 1fill}
\setlength{\LTright}{0pt}

% \begin{longtable}{@{}l@{\hspace{10pt}}p{.8\linewidth}@{}}

% \end{longtable}



\begin{longtable}{@{}l@{\hspace{10pt}}p{.8\linewidth}@{}}
\caption[A summary of \env{longtable} commands]%
        {\normalsize A summary of \env{longtable} commands}\\
\multicolumn{2}{c}{\textbf{Parameters\hfill 参数}}\\*
\hline
"\LTleft"&
    Glue to the left of the table. 表格左边的间距。          \hfill("\fill")\\
"\LTright"&
    Glue to the right of the table. 表格右边的间距。         \hfill("\fill")\\
"\LTpre"&
    Glue before the table.表格前的间距。      \hfill("\bigskipamount")\\
"\LTpost"&
    Glue after the table.表格后的间距。       \hfill("\bigskipamount")\\
"\LTcapwidth"&
    The width of a parbox containing the caption.包含标题的 parbox 的宽度。 \hfill(4in)\\
"LTchunksize"&
    The number of rows per chunk. 每个块中的行数。                \hfill(20)\\[5pt]
\multicolumn{2}{c}{\textbf{Optional
     arguments to} \ttfamily\v\\begin\v{longtable\v} \hfill \textbf{\ttfamily\v\\begin\v{longtable\v}} 的可选参数}\\*
\hline
\it none& Position as specified by "\LTleft" and "\LTright". 按照 "\LTleft" 和 "\LTright" 的设置位置。\\
"[c]"&   Centre the table.居中表格。\\
"[l]"&   Place the table flush left.将表格左对齐。\\
"[r]"&   Place the table flush right.将表格右对齐。\\[5pt]
\pagebreak[2]
\multicolumn{2}{c}{\textbf{Commands
                    to end table rows}\hfill \textbf{结束表格行的命令}}\\*
\hline
"\\"&
    Specifies the end of a row 指定行的结束。\\
"\\"\oarg{dim}& %
Ends row, then adds vertical space
    (as in the \env{tabular} environment). 结束行,然后添加垂直间距(与 \env{tabular} 环境中的行为相同)。\\
"\\*"&
    The same as "\\" but disallows a page break after the row. 与 "\\" 相同,但不允许在行后进行分页。\\
"\tabularnewline"&
    Alternative to "\\" for use in the scope of "\raggedright" and
    similar commands that redefine "\\".用于在 "\raggedright" 和类似命令的作用域内代替 "\\"。\\
"\kill"&
    Row is `killed', but is used in calculating widths.行被“杀死”,但用于计算宽度。\\
"\endhead"&
    Specifies rows to appear at the top of every page.指定在每页顶部出现的行。\\
"\endfirsthead"&
    Specifies rows to appear at the top the first page.指定在第一页顶部出现的行。\\
"\endfoot"&
    Specifies rows to appear at the bottom of every page.指定在每页底部出现的行。\\
"\endlastfoot"&
    Specifies rows to appear at the bottom of the last page.指定在最后一页底部出现的行。\\[5pt]
\multicolumn{2}{c}{\textbf{\env{longtable} caption commands}\hfill \textbf{\env{longtable} 标题命令}}\\*
\hline
"\caption"\marg{caption}&
    Caption `Table ?: \meta{caption}', and a `\meta{caption}'
    entry in the list of tables. 标题为“Table ?: \meta{caption}”,并在表格列表中添加一个 \meta{caption} 条目。\\
"\caption"\oarg{lot}\marg{caption}&
    Caption `Table ?: \meta{caption}', and a `\meta{lot}'
    entry in the list of tables.标题为“Table ?: \meta{caption}”,并在表格列表中添加一个 \meta{lot} 条目。\\
"\caption[]"\marg{caption}&
    Caption `Table ?: \meta{caption}',
    but no entry in the list of tables.标题为“Table ?: \meta{caption}”,但不在表格列表中添加条目。\\
"\caption*"\marg{caption}&
    Caption `\meta{caption}', but no entry in the list of tables.标题为 \meta{caption},但不在表格列表中添加条目。\\[5pt]
\multicolumn{2}{c}{%^^A
       \textbf{Commands available at the start of a row}\hfill \textbf{在行开始处可用的命令}}\\*
\hline
"\pagebreak"&
    Force a page break.强制分页。\\*
"\pagebreak"\oarg{val}& A `hint' between 0 and 4
 of the desirability  of a break.分页的“提示”,为 0 到 4 的值,表示分页的可取性。\\
"\nopagebreak"& Prohibit a page break.禁止分页。\\*
"\nopagebreak"\oarg{val}& A `hint' between 0 and 4 of the undesirability
   of a break.分页的“提示”,为 0 到 4 的值,表示分页的不可取性。\\
"\newpage"&
    Force a page break.强制分页。\\[5pt]
\multicolumn{2}{c}{\textbf{Footnote commands
                     available inside \env{longtable}}\hfill  \textbf{\env{longtable} 中可用的脚注命令}}\\*
\hline
"\footnote"&
    Footnotes, but may not be used in the table head \& foot.脚注,但不能在表格的头部和尾部使用。\\*
"\footnotemark"&
    Footnotemark, may be used in the table head \& foot.脚注标记,可以在表格的头部和尾部使用。\\*
"\footnotetext"&
    Footnote text, use in the table body.脚注文本,在表格的正文中使用。\\[5pt]
\multicolumn{2}{c}{\textbf{Setlongtables}}\\
\hline
"\setlongtables"&  Obsolete command. Does nothing now. 过时的命令,现在不起任何作用。
\end{longtable}

