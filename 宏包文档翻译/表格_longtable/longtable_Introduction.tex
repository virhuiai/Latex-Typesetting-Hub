\section{Introduction\hfill 介绍}

\columnratio{0.55}
\begin{paracol}{2}
The \package{longtable} package defines a new environment,
\DescribeEnv{longtable}
\env{longtable}, which has most of the features of the \env{tabular}
environment, but produces tables which may be broken by \TeX's
standard page-breaking algorithm. It also shares some features with
the \env{table} environment. In particular it uses the same counter,
\texttt{table}, and has a similar "\caption" command. Also, the
standard "\listoftables" command lists tables produced by either the
\env{table} or \env{longtable} environments.
\switchcolumn
\package{longtable} 宏包定义了一个新的环境 \env{longtable},它大多数特性与 \env{tabular} 环境相同,但是能够根据 \TeX 的标准分页算法将表格分页显示。同时,它也与 \env{table} 环境共享一些特性。特别是它使用相同的计数器 \texttt{table} 和类似的 "\caption" 命令。另外,"\listoftables" 命令能够列出由 \env{table} 或 \env{longtable} 环境生成的表格。
%%%%%%%%%%%%%%%%%%%%%%%%%%%%%%%%%%%%
\switchcolumn[0]*
The following example uses most of the features of the \env{longtable}
environment. An edited listing of the input for this example appears
in Section~\ref{listing}.
\switchcolumn 下面的示例使用了\env{longtable}环境的大多数功能。本示例的输入编辑列表在第~\ref{listing}节中。

\switchcolumn[0]*
\textbf{Note:} Various parts of the following table will
\textbf{not} line up correctly until this document has been run
through \LaTeX\ several times.  This is a characteristic feature of
this package, as described below.
\switchcolumn 注意:在此表中,各部分的对齐可能需要多次运行该文档才能正确实现。这是该软件包的特征之一,如下所述。
%%%%%%%%%%%%%%%%%%%%%%%%%%%%%%%%%%%%
\end{paracol}


% \begin{longtable}{@{*}r||p{1in}@{*}}
\begin{longtable}{@{*}p{0.45\textwidth}||p{0.45\textwidth}@{*}}
KILLED & LINE!!!! \kill
\caption
[An optional table caption (used in the list of tables)一个可选的表标题(用于表格列表)]
{A long table\label{long} 一个长表格}\\
%caption 设置表格标题,方括号中的内容为可选的表格标题,花括号中的内容为实际的表格标题。
%label命令用于设置表格的引用标签。
\hline\hline% hline 命令用于在表格中插入横线。
\multicolumn{2}{@{*}c@{*}}%
     {This part appears at the top of the table 此部分显示在表格的顶部}\\%multicolomn 命令用于合并多个列。
\textsc{First 第一列}&\textsc{Second 第二列}\\
\hline\hline
\endfirsthead%endfirsthead 命令用于指定后面的内容为第一页之后的表头部,比如标题。  endfirst head 分开
\caption[]{(continued) (续)}\\
\hline\hline
\multicolumn{2}{@{*}c@{*}}%
      {This part appears at the top of every other page 此部分显示在每页的顶部}\\
\textbf{First 第一列}&\textbf{Second 第二列}\\
\hline\hline
\endhead %endhead 命令用于指定第一页之后每页的表头。  end head
\hline
This goes at the bottom.&
这部分显示在底部。\\
\hline
\endfoot %endfoot 命令用于指定表格页脚。
\hline
These lines will appear
in place of the  usual foot
at the end of the table&
这些行将会代替通常的表格页脚出现在表格的末尾。\\\hline
\endlastfoot %endlastfoot 命令用于指定最后一页的表尾。
\env{longtable}  columns  are specified in the same way as  in the \env{tabular} environment.             &  \env{longtable}的列指定方式 与\env{tabular}环境相同。\\

"@{*}r||p{1in}@{*}"&  "@{*}r||p{1in}@{*}"\\
in this case. Each row ends with a "\\" command.&%
在这个例子中,每一行以"\\" 命令结束。\\ 

The "\\"  command  has an  optional argument, just as in  the \env{tabular} environment. & %
"\\" 命令可以带有可选的参数,就像在 \env{tabular}环境中一样。\\[10pt]

See the  effect  of "\\[10pt]"& 看看 "\\[10pt]" 的效果 \\
Lots of lines like this.& 有很多这样的行。\\
Lots of lines like this.& 有很多这样的行。\\
Lots of lines like this.& 有很多这样的行。\\
Lots of lines like this.& 有很多这样的行。\\

Also  "\hline"  may be used, as in \env{tabular}. & %
也可以使用 "\hline" ,就像在\env{tabular}中一样,\\\hline

That  was a "\hline". & 这是一个 "\hline"。\\\hline\hline
That  was "\hline\hline" .&这是 "\hline\hline"\\
\multicolumn{2}{||c||}%
{This is a \ttfamily\v\\multicolumn\v{2\v}\v{||c||\v}}\\
\multicolumn{2}{||c||}%
{这是一个 \ttfamily\v\\multicolumn\v{2\v}\v{||c||\v}}\\

If a  page break  occurs at a "\hline" then a line is drawn
at the bottom of one  page  and at the top of the next.& 
如果页面在 "\hline" 处分页,则会在一页的底部画一条线,并在下一页的顶部绘制一条线。\\\hline

The  "[t] [b] [c]"  argument of \env{tabular} can  not be used.& \env{tabular}的 "[t] [b] [c]" 参数无法使用。\\
The optional argument may be  one of "[l] [r] [c]"
to specify whether  the  table  should be  adjusted
to the  left, right or centrally.&
可选参数可以是 "[l] [r] [c]" 之一,以指定表格应该向左、向右或居中调整。\\\hline
\hline\hline
Lots of lines like this.& 有很多这样的行。\\
Lots of lines like this.& 有很多这样的行。\\
Lots of lines like this.& 有很多这样的行。\\
Lots of lines like this.& 有很多这样的行。\\
Lots of lines like this.& 有很多这样的行。\\
Lots of lines like this.& 有很多这样的行。\\
Lots of lines like this.& 有很多这样的行。\\
Lots of lines like this.& 有很多这样的行。\\
Lots of lines like this.& 有很多这样的行。\\
Lots of lines like this.& 有很多这样的行。\\
Lots of lines like this.& 有很多这样的行。\\
Lots of lines like this.& 有很多这样的行。\\
Lots of lines like this.& 有很多这样的行。\\
Lots of lines like this.& 有很多这样的行。\\
Lots of lines like this.& 有很多这样的行。\\
Lots of lines like this.& 有很多这样的行。\\
Lots of lines like this.& 有很多这样的行。\\
Lots of lines like this.& 有很多这样的行。\\
Lots of lines like this.& 有很多这样的行。\\
Lots of lines like this.& 有很多这样的行。\\
Some lines may take up a lot of space, like this: 
\raggedleft This last column is a ``p'' column so this
``row'' of the table can take up several lines. Note however that
\TeX\ will  never break a page within such a row. Page breaks only
occur between rows of the table or at "\hline" commands.&
有些行可能会占用很多空间,就像这样:
\raggedleft 这一列是一个“p”列,因此表格的这一“行”可以占用多行。但请注意,\TeX 不会在这样的行内分页。页面分页只会出现在表格的行之间或者在 "\hline" 命令处。     
\tabularnewline
    %tabularnewline 命令用于结束当前行并开始新行。
Lots of lines like this.& 有很多这样的行。\\
Lots of lines like this.& 有很多这样的行。\\
Lots of lines like this.& 有很多这样的行。\\
Lots of lines like this.& 有很多这样的行。\\
Lots of lines like this.& 有很多这样的行。\\
Lots of lines like this.& 有很多这样的行。\\
Lots of lines like this.& 有很多这样的行。\\
\hline
Lots\footnote{This is a footnote.} of lines like this.
Lots   of   lines like this\footnote{\env{longtable} takes special
precautions, so that footnotes may also be used in `p' columns.}&
\footnote{这是一个脚注。}像这样。
像这样\footnote{\env{longtable}采取了特殊的预防措施,以便在`p'列中也可以使用脚注。}\\
\hline
Lots of lines like this.& 有很多这样的行。\\
Lots of lines like this.& 有很多这样的行。\\
\end{longtable}
