\section{Captions and Headings\hfill 标题和表头}
\columnratio{0.55}
\begin{paracol}{2}
\DescribeMacro{\endhead}
At the start of the table one may specify lines which are to appear at
the top of every page (under the headline, but before the other lines
of the table).
\switchcolumn 
\DescribeMacro{\endhead}
在表格开头,可以指定出现在每一页顶部(在页眉下方,但在表格的其他行之前)的行。
\switchcolumn[0]*
The lines are entered as normal, but the last "\\" command is
replaced by a "\endhead" command.
\switchcolumn 这些行像正常的行一样输入,但是最后一个"\\"命令被"\endhead"命令替换。 

\switchcolumn[0]*
\DescribeMacro{\endfirsthead}
If the first page should have a different heading, then this should be
entered in the same way, and terminated with the "\endfirsthead"
command. The "LTchunksize" should be at least as large as the
number of rows in the heading.
\switchcolumn 
\DescribeMacro{\endfirsthead}
如果第一页需要有不同的表头,则应以相同的方式输入,并以"\endfirsthead"命令终止。 "LTchunksize"应至少与页眉中的行数一样大。

\switchcolumn[0]*
\DescribeMacro{\endfoot}
There are also "\endfoot" and "\endlastfoot"
\DescribeMacro{\endlastfoot}
commands which are used in the same way (at the \emph{start} of the
table) to specify rows (or an "\hline") to appear at the bottom of
each page. In certain situations, you may want to place lines which
logically belong in the table body at the end of the \env{firsthead},
or the beginning of the \env{lastfoot}. This helps to control which
lines appear on the first and last page of the table.\switchcolumn 

还有 "\endfoot" 和 "\endlastfoot" 命令,它们的使用方式相同(在表格的\emph{开头}),用于指定出现在每一页底部的行(或 "\hline")。在某些情况下,您可能希望将逻辑上属于表格主体的行放置在\env{firsthead}的末尾或\env{lastfoot}的开头。这有助于控制哪些行出现在表格的第一页和最后一页。

\switchcolumn[0]*
\DescribeMacro{\caption}%
The "\caption{...}" command is essentially equivalent to\switchcolumn
"\caption{...}"命令本质上等同于

\switchcolumn[0]*
"\multicolumn{n}{c}{\parbox{\LTcapwidth}{...}}"
\switchcolumn
"\multicolumn{n}{c}{\parbox{\LTcapwidth}{...}}"

\switchcolumn[0]*
where \texttt{n} is the number of columns of the table. You may set
the width of the caption with a command such as
"\setlength{\LTcapwidth}{2in}"
in the preamble of your document. The default is 4in. 
"\caption" also
writes the information to produce an entry in the list of tables. As
with the "\caption" command in the \env{figure} and \env{table}
environments, an optional argument specifies the text to appear in the
list of tables if this is different from the text to appear in the
caption. Thus the caption for table \ref{long} was specified as
{\ttfamily
 "\caption"[An optional table caption
            (used in the list of tables)]\v{A long
 table"\label{long}"\v}}.
\switchcolumn%%%%
其中\texttt{n}是表格的列数。若要设置标题的宽度,您可以在文档的导言部分使用类似
"\setlength{\LTcapwidth}{2in}"的命令。
默认值为4in。
"\caption"命令还会将信息写入表格目录中的条目。与\env{figure}和\env{table}环境中的"\caption"命令一样,可选参数用于指定在表格目录中显示的文本,如果这个文本与标题中显示的文本不同。因此,表\ref{long}的标题被指定为:{\ttfamily
"\caption"[An optional table caption
           (used in the list of tables)]\v{A long
table"\label{long}"\v}}。

\switchcolumn[0]*
You may wish the caption on later pages to be different to that on the
first page. In this case put the "\caption" command in the first
heading, and put a subsidiary caption in a "\caption[]" command in
the main heading. If the optional argument to "\caption" is empty,
no entry is made in the list of tables. Alternatively, if you do not
want the table number to be printed each time, use the "\caption*"
command.\switchcolumn 
你可能希望后面的页面的标题与第一页的标题不同。在这种情况下,将"\caption"命令放在第一个标题中,并在主标题中使用"\caption[]"命令来放置一个辅助标题。如果"\caption"的可选参数为空,则不会在表格列表中创建条目。或者,如果你不想每次打印表格编号,可以使用"\caption*"命令。

\switchcolumn[0]*
The captions are set based on the code for the \package{article}
class.
If you have redefined the standard "\@makecaption" command to produce
a different format for the captions, you may
need to make similar changes to the \package{longtable} version,
"\LT@makecaption". See the code section for more details.\switchcolumn 

标题是根据\package{article}类的代码设置的。如果您重新定义了标准的"@makecaption"命令以生成不同格式的标题,则可能需要对\package{longtable}版本"\LT@makecaption"进行类似的更改。有关更多详细信息,请参见代码部分。

\switchcolumn[0]*
A more convenient method of customising captions is given by the
\package{caption(2)} package, which provides commands for customising
captions, and arranges that the captions in standard environments, and
many environments provided by packages (including \package{longtable})
are modified in a compatible manner.\switchcolumn 

通过\package{caption(2)}包提供了更方便的自定义标题的方法,它提供了自定义标题的命令,并安排标准环境中的标题以及许多由包(包括\package{longtable})提供的环境以兼容的方式进行修改。

\switchcolumn[0]*
You may use the "\label" command so that you can cross reference
\env{longtable}s with "\ref". Note however, that the "\label" command
should not be used in a heading that may appear more than once. Place
it either in the \env{firsthead}, or in the body of the table. It
should not be the \emph{first} command in any entry.\switchcolumn 

您可以使用"\label"命令,以便您可以通过"\ref"进行跨引用\env{longtable}。但是请注意,不应在可能出现多次的标题中使用"\label"命令。将其放置在\env{firsthead}或表格主体中。它不应该是任何条目中的\emph{第一个}命令。
\end{paracol}