
\section{Adjustment\hfill 调整}
\columnratio{0.55}
\begin{paracol}{2}
The optional argument of \env{longtable} controls the
horizontal alignment of the table. The possible options are "[c]",
"[r]" and "[l]", for  centring,
right  and left adjustment, respectively.
\switchcolumn 

\env{longtable}的可选参数控制表格的水平对齐方式。可能的选项有"[c]"、"[r]"和"[l]",分别表示居中、右对齐和左对齐。
\switchcolumn[0]*
\DescribeMacro{\LTleft}
Normally centring is the default, but this document specifies
\begin{verbatim}
\setlength\LTleft\parindent
\setlength\LTright\fill
\end{verbatim}
in the preamble,
\switchcolumn 
\DescribeMacro{\LTleft}
通常,默认值是居中对齐,但是在这个文档中,设置了如下的代码:
\DescribeMacro{\LTright}
\begin{verbatim}
\setlength\LTleft\parindent
\setlength\LTright\fill
\end{verbatim}
在导言部分.
\switchcolumn[0]*
\DescribeMacro{\LTright}
 which means that the tables are set flush left, but
indented by the usual paragraph indentation. Any lengths can be
specified for these two parameters, but at least one of them should be
a rubber length so that it fills up the width of the page, unless
rubber lengths are added between the columns using the
"\extracolsep" command.
For instance
\switchcolumn 

这意味着表格被设置为左对齐,但是缩进采用通常的段落缩进。这两个参数可以指定任何长度,但至少其中一个应该是弹性长度,以便填满页面的宽度,除非在列之间使用"\extracolsep"命令添加了弹性长度。例如:

\begin{verbatim}
 \begin{tabular*}{\textwidth}{@{\extracolsep{...}}...}
\end{verbatim}

\switchcolumn[0]*
produces a full width table, to get a similar effect with
\env{longtable} specify
\begin{verbatim}
\setlength\LTleft{0pt}
\setlength\LTright{0pt}
\begin{longtable}{@{\extracolsep{...}}...}
\end{verbatim}
\switchcolumn 
生成一个全宽的表格,如果要使用\env{longtable}实现类似效果,请指定如下代码:
\begin{verbatim}
\setlength\LTleft{0pt}
\setlength\LTright{0pt}
\begin{longtable}{@{\extracolsep{...}}...}
\end{verbatim}
\end{paracol}
% @{\extracolsep{\fill}}
 