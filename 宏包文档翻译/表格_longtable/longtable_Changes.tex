
\section{Changes\hfill 变更}

\columnratio{0.55}
\begin{paracol}{2}
This section highlights the major changes since version~2. A more
detailed change log may be produced at the end of the code listing
if the "ltxdoc.cfg" file specifies
\begin{verbatim}
 \AtBeginDocument{\RecordChanges}
 \AtEndDocument{\PrintChanges}
\end{verbatim}
\switchcolumn
本节主要介绍自版本2以来的主要变更。如果"ltxdoc.cfg"文件指定了以下内容,
则在代码列表的末尾可以生成更详细的变更日志:
\begin{verbatim}
 \AtBeginDocument{\RecordChanges}
 \AtEndDocument{\PrintChanges}
\end{verbatim}

\switchcolumn[0]*
Changes made between versions 2 and 3.
\switchcolumn[0]
2和3版本之间的变更:
\end{paracol}




\begin{itemize}
\columnratio{0.55}
\begin{paracol}{2}
\switchcolumn[0]*
\item  The mechanism for adding the head and foot of the table has been
completely rewritten. With this new mechanism, \env{longtable} does
not need to issue a "\clearpage" at the start of the table, and so the
table may start half way down a page. Also the "\endlastfoot" command
which could not safely be implemented under the old scheme, has been
added.
\switchcolumn \item 完全重写了添加表格头部和尾部的机制。使用这种新机制,\env{longtable}在表格开始时不需要发出"\clearpage"命令,因此表格可以从页面的中间开始。还添加了"\endlastfoot"命令,在旧方案下无法安全实现。
\switchcolumn[0]*
\item  \env{longtable} now issues an error if started in the scope of
"\twocolumn", or the \env{multicols} environment.

\switchcolumn\item \env{longtable}现在会在"\twocolumn"或\env{multicols}环境的作用域中启动时发出错误。
\switchcolumn[0]*
\item  The separate documentation file "longtable.tex" has been
merged with the package file, "longtable.dtx" using Mittelbach's
\package{doc} package.

\switchcolumn\item 将独立的文档文件"longtable.tex"与包文件"longtable.dtx"合并,使用了Mittelbach的\package{doc}宏包。
\switchcolumn[0]*
\item 
 Support for footnotes has been added. Note however that
"\footnote" will not work in the `head' or `foot' sections of the
table. In order to put a footnote in those sections (e.g., inside a
caption), use "\footnotemark" at that point, and "\footnotetext"
anywhere in the table \emph{body} that will fall on the same page.

\switchcolumn\item 添加了对脚注的支持。但是需要注意的是,"\footnote"命令在表格的"head"或"foot"部分将无法正常工作。为了在这些部分(例如标题内部)放置脚注,请在该处使用"\footnotemark",并在表格正文中的任何位置(位于同一页上)使用"\footnotetext"。
\switchcolumn[0]*
\item 
 The treatment of "\multicolumn" has changed, making
"\kill" lines unnecessary, at the price of sometimes requiring a
third pass through \LaTeX.

\switchcolumn\item "\multicolumn"的处理方式已更改,不再需要使用"\kill"命令,但有时需要通过\LaTeX 进行三次编译。
\switchcolumn[0]*
\item 
 The "\newpage" command now works inside a \env{longtable}.
 \switchcolumn \item "\newpage"命令现在在\env{longtable}环境内也可用。
\end{paracol}
\end{itemize}

\columnratio{0.55}
\begin{paracol}{2}

\switchcolumn[0]*
Changes made between versions 3 and 4.
\switchcolumn
3和4版本之间的变更:
\end{paracol}




\begin{itemize}
    \columnratio{0.55}
\begin{paracol}{2}

\switchcolumn[0]*
\item 
 A new algorithm is used for aligning chunks. As well as the
widest width in each column, \package{longtable} remembers which
chunk produced this maximum. This allows it to check that the
maximum is still achieved in later runs. As \package{longtable} can
now deal with columns shrinking as the file is edited, the
"\setlongtables" system is no longer needed and is disabled.


\switchcolumn \item 采用了一种新的算法来对齐块。除了每一列的最宽宽度外,\package{longtable}还记住了哪个块产生了这个最大宽度。这使得它可以检查在后续运行中是否仍然达到了最大宽度。由于\package{longtable}现在可以处理文件编辑时列的收缩,因此不再需要和禁用"\setlongtables"系统。
\switchcolumn[0]*
\item 
 An extra benefit of the new algorithm's ability to deal with
`shrinking' columns is that it can give better (narrower) column
widths in the case of overlapping "\multicolumn" entries in
different chunks than the previous algorithm produced.


\switchcolumn\item 新算法的另一个好处是,在不同块中有重叠的"\multicolumn"条目的情况下,它可以给出更好(更窄)的列宽,而以前的算法则不能实现这一点。
\switchcolumn[0]*
\item 
 The `draft' multicolumn system has been removed, along with
related commands such as "\LTmulticolumn".


\switchcolumn\item 删除了"draft"多列系统,以及相关的命令,如"\LTmulticolumn"。
\switchcolumn[0]*
\item 
 The disadvantage of the new algorithm is that it can take more
passes. The theoretical maximum is approximately twice the length
of a `chain' of columns with overlapping "\multicolumn" entries,
although in practice it usually converges as fast as the old
version. (Which always converged in three passes once
"\setlongtables" was  activated.)


\switchcolumn\item 新算法的缺点是可能需要更多次的编译。理论上的最大次数约为具有重叠"\multicolumn"条目的列链的长度的两倍,但在实践中,它通常与旧版本一样快速收敛。(一旦激活"\setlongtables",旧版本总是在三次编译后收敛。)
\switchcolumn[0]*
\item 
 "\\*" and "\nopagebreak" commands may be used to control page
 breaking.
\switchcolumn\item 可以使用"\\*"和"\nopagebreak"命令来控制分页。
\end{paracol}
\end{itemize}


