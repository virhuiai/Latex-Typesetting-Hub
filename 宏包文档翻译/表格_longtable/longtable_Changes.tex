
\section{Changes\hfill 变更}

\columnratio{0.55}
\begin{paracol}{2}
This section highlights the major changes since version~2. A more
detailed change log may be produced at the end of the code listing
if the "ltxdoc.cfg" file specifies
\begin{verbatim}
 \AtBeginDocument{\RecordChanges}
 \AtEndDocument{\PrintChanges}
\end{verbatim}
\switchcolumn
本节主要介绍自版本2以来的主要变更。如果"ltxdoc.cfg"文件指定了以下内容,
则在代码列表的末尾可以生成更详细的变更日志:
\begin{verbatim}
 \AtBeginDocument{\RecordChanges}
 \AtEndDocument{\PrintChanges}
\end{verbatim}

\switchcolumn[0]*
Changes made between versions 2 and 3.
\switchcolumn[0]
2和3版本之间的变更:
\begin{itemize}

\switchcolumn
\switchcolumn[0]*
\item  The mechanism for adding the head and foot of the table has been
completely rewritten. With this new mechanism, \env{longtable} does
not need to issue a "\clearpage" at the start of the table, and so the
table may start half way down a page. Also the "\endlastfoot" command
which could not safely be implemented under the old scheme, has been
added.

\switchcolumn
\switchcolumn[0]*
\item  \env{longtable} now issues an error if started in the scope of
"\twocolumn", or the \env{multicols} environment.

\switchcolumn
\switchcolumn[0]*
\item  The separate documentation file "longtable.tex" has been
merged with the package file, "longtable.dtx" using Mittelbach's
\package{doc} package.

\switchcolumn
\switchcolumn[0]*
\item 
 Support for footnotes has been added. Note however that
"\footnote" will not work in the `head' or `foot' sections of the
table. In order to put a footnote in those sections (e.g., inside a
caption), use "\footnotemark" at that point, and "\footnotetext"
anywhere in the table \emph{body} that will fall on the same page.

\switchcolumn
\switchcolumn[0]*
\item 
 The treatment of "\multicolumn" has changed, making
"\kill" lines unnecessary, at the price of sometimes requiring a
third pass through \LaTeX.

\switchcolumn
\switchcolumn[0]*
\item 
 The "\newpage" command now works inside a \env{longtable}.
\end{itemize}

Changes made between versions 3 and 4.
\begin{itemize}

\switchcolumn
\switchcolumn[0]*
\item 
 A new algorithm is used for aligning chunks. As well as the
widest width in each column, \package{longtable} remembers which
chunk produced this maximum. This allows it to check that the
maximum is still achieved in later runs. As \package{longtable} can
now deal with columns shrinking as the file is edited, the
"\setlongtables" system is no longer needed and is disabled.


\switchcolumn
\switchcolumn[0]*
\item 
 An extra benefit of the new algorithm's ability to deal with
`shrinking' columns is that it can give better (narrower) column
widths in the case of overlapping "\multicolumn" entries in
different chunks than the previous algorithm produced.


\switchcolumn
\switchcolumn[0]*
\item 
 The `draft' multicolumn system has been removed, along with
related commands such as "\LTmulticolumn".


\switchcolumn
\switchcolumn[0]*
\item 
 The disadvantage of the new algorithm is that it can take more
passes. The theoretical maximum is approximately twice the length
of a `chain' of columns with overlapping "\multicolumn" entries,
although in practice it usually converges as fast as the old
version. (Which always converged in three passes once
"\setlongtables" was  activated.)


\switchcolumn
\switchcolumn[0]*
\item 
 "\\*" and "\nopagebreak" commands may be used to control page
 breaking.

\end{itemize}

\end{paracol}
