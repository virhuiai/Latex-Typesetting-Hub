\documentclass{ltxdoc}
\usepackage[heading=true
,scheme=chinese%中文方案
,fontset=none%不使用默认的字体设置
,space=auto%自动调整中英文间距
]{ctex}
\setCJKmainfont{FangZhengShuSong-GBK-1.ttf}[Path=/Users/virhuiai/hlProjects/Latex-Typesetting-Hub/font/方正/]%设置文本的中文有衬线字体
\setCJKsansfont{FangZhengHeiTi-GBK-1.ttf}[Path=/Users/virhuiai/hlProjects/Latex-Typesetting-Hub/font/方正/]%设置文本的中文无衬线字体为
\setCJKmonofont{FangZhengFangSong-GBK-1.ttf}[Path=/Users/virhuiai/hlProjects/Latex-Typesetting-Hub/font/方正/] %设置文本的中文等宽字体 
% \setCJKfamilyfont{fontKai}{LXGWWenKai-Regular.ttf}[Path=/Users/virhuiai/hlProjects/Latex-Typesetting-Hub/font/霞鹜文楷/]
\setCJKfamilyfont{fontKai}{FangZhengKaiTi-GBK-1.ttf}[Path=/Users/virhuiai/hlProjects/Latex-Typesetting-Hub/font/方正/]
\newcommand\fontKai{\CJKfamily{fontKai}}

\usepackage[a3paper,landscape]{geometry}
\usepackage{minted}
\usepackage{paracol}


\makeatletter
\providecommand*\input@path{}
\newcommand*\addinputpath[1]{\expandafter\def\expandafter\input@path\expandafter{\input@path#1}}
\makeatother
\addinputpath{%
{/Users/virhuiai/hlProjects/Latex-Typesetting-Hub/宏包文档翻译/表格_longtable/}%
}
\usepackage{parskip}
\parindent=0pt
\usepackage{longtable}
\begin{document}
% \globalcounter*
\globalcounter{section}
\DocInput{longtable-1.dtx}

\let\package\textsf
\let\env\textsf
\providecommand\finalclearpage{\clearpage}

\DeleteShortVerb{\|}
\MakeShortVerb{\"}


\makeatletter
\def\@oddfoot{\normalfont\rmfamily\dotfill Page \thepage\dotfill}
\def\@oddhead{\dotfill{\normalfont\ttfamily longtable.sty}\dotfill}
\def\ps@titlepage{\let\@oddhead\@empty}
\makeatother


\setlength\LTleft\parindent
\setlength\LTright\fill
\setcounter{LTchunksize}{10}

\def\v{\char`}

%  \title{The \textsf{longtable} package\thanks{This file
       has version number \fileversion, last
       revised \filedate.}}
\author{David Carlisle\thanks{%
\noindent 
The new algorithm for aligning `chunks'
of a table used in version 4 of this package was devised coded
and documented by David Kastrup.\\
\hspace*{2em}在这个包的第4个版本中,用于对表格的“块”进行对齐的新算法是由David Kastrup设计、编码和文档化的。
}\and 翻译:virhuiai}
\date{\filedate}

\MaintainedByLaTeXTeam{tools}
\maketitle

\begin{abstract}
\columnratio{0.55}
\begin{paracol}{2}
\noindent This package defines the \env{longtable} environment, a multi-page
version of \env{tabular}.
\switchcolumn
这个包定义了\env{longtable}环境,是\env{tabular}的多页版本。
\end{paracol}
\end{abstract} 
% ^^A: 这是一个特殊的字符,表示空格。在 LaTeX 中,^^A 用于在代码中插入注释,表示忽略该行后面的内容。
% \vbox to 100pt: 这是一个 LaTeX 命令,用于创建一个垂直的盒子(box),并指定其高度为100pt(点)。
% ^^A \vbox to 100pt makes the page breaks the same on the first run.
% \noindent\begin{minipage}[t][130pt]{\textwidth}
% \listoftables
% \end{minipage}
% \section{Introduction\hfill 介绍}

\columnratio{0.55}
\begin{paracol}{2}
The \package{longtable} package defines a new environment,
\DescribeEnv{longtable}
\env{longtable}, which has most of the features of the \env{tabular}
environment, but produces tables which may be broken by \TeX's
standard page-breaking algorithm. It also shares some features with
the \env{table} environment. In particular it uses the same counter,
\texttt{table}, and has a similar "\caption" command. Also, the
standard "\listoftables" command lists tables produced by either the
\env{table} or \env{longtable} environments.
\switchcolumn
\package{longtable} 宏包定义了一个新的环境 \env{longtable},它大多数特性与 \env{tabular} 环境相同,但是能够根据 \TeX 的标准分页算法将表格分页显示。同时,它也与 \env{table} 环境共享一些特性。特别是它使用相同的计数器 \texttt{table} 和类似的 "\caption" 命令。另外,"\listoftables" 命令能够列出由 \env{table} 或 \env{longtable} 环境生成的表格。
%%%%%%%%%%%%%%%%%%%%%%%%%%%%%%%%%%%%
\switchcolumn[0]*
The following example uses most of the features of the \env{longtable}
environment. An edited listing of the input for this example appears
in Section~\ref{listing}.
\switchcolumn 下面的示例使用了\env{longtable}环境的大多数功能。本示例的输入编辑列表在第~\ref{listing}节中。

\switchcolumn[0]*
\textbf{Note:} Various parts of the following table will
\textbf{not} line up correctly until this document has been run
through \LaTeX\ several times.  This is a characteristic feature of
this package, as described below.
\switchcolumn 注意:在此表中,各部分的对齐可能需要多次运行该文档才能正确实现。这是该软件包的特征之一,如下所述。
%%%%%%%%%%%%%%%%%%%%%%%%%%%%%%%%%%%%

\end{paracol}



\begin{longtable}{@{*}r||p{1in}@{*}}
KILLED & LINE!!!! \kill
\caption
[An optional table caption (used in the list of tables)]
{A long table\label{long}}\\
\hline\hline
\multicolumn{2}{@{*}c@{*}}%
     {This part appears at the top of the table}\\
\textsc{First}&\textsc{Second}\\
\hline\hline
\endfirsthead
\caption[]{(continued)}\\
\hline\hline
\multicolumn{2}{@{*}c@{*}}%
      {This part appears at the top of every other page}\\
\textbf{First}&\textbf{Second}\\
\hline\hline
\endhead
\hline
This goes at the&bottom.\\
\hline
\endfoot
\hline
These lines will&appear\\
in place of the & usual foot\\
at the end& of the table\\
\hline
\endlastfoot
\env{longtable}  columns  are specified& in the \\
same way as  in the \env{tabular}& environment.\\
"@{*}r||p{1in}@{*}"& in this case.\\
Each row ends with a& "\\" command.\\
The "\\"  command  has an& optional\\
argument, just as in& the\\
 \env{tabular}&environment.\\[10pt]
See the  effect  of "\\[10pt]"&?\\
Lots of lines& like this.\\
Lots of lines& like this.\\
Lots of lines& like this.\\
Lots of lines& like this.\\
Also  "\hline"  may be used,&  as in \env{tabular}.\\
\hline
That  was a "\hline"&.\\
\hline\hline
That  was "\hline\hline"&.\\
\multicolumn{2}{||c||}%
{This is a \ttfamily\v\\multicolumn\v{2\v}\v{||c||\v}}\\
If a  page break  occurs at a "\hline" then& a line is drawn\\
at the bottom of one  page  and at the& top of the next.\\
\hline
The  "[t] [b] [c]"  argument of \env{tabular}& can  not be used.\\
The optional argument may be  one of& "[l] [r] [c]"\\
to specify whether  the  table  should be& adjusted\\
to the  left, right& or centrally.\\
\hline\hline
Lots of lines& like this.\\
Lots of lines& like this.\\
Lots of lines& like this.\\
Lots of lines& like this.\\
Lots of lines& like this.\\
Lots of lines& like this.\\
Lots of lines& like this.\\
Lots of lines& like this.\\
Lots of lines& like this.\\
Lots of lines& like this.\\
Lots of lines& like this.\\
Lots of lines& like this.\\
Lots of lines& like this.\\
Lots of lines& like this.\\
Lots of lines& like this.\\
Lots of lines& like this.\\
Lots of lines& like this.\\
Lots of lines& like this.\\
Lots of lines& like this.\\
Lots of lines& like this.\\
Some lines may take up a lot of space, like this: &
    \raggedleft This last column is a ``p'' column so this
    ``row'' of the table can take up several lines. Note however that
    \TeX\ will  never break a page within such a row. Page breaks only
    occur between rows of the table or at "\hline" commands.
    \tabularnewline
Lots of lines& like this.\\
Lots of lines& like this.\\
Lots of lines& like this.\\
Lots of lines& like this.\\
Lots of lines& like this.\\
Lots of lines& like this.\\
Lots of lines& like this.\\
\hline
Lots\footnote{This is a footnote.} of lines& like this.\\
Lots   of   lines& like this\footnote{\env{longtable} takes special
    precautions, so that footnotes may also be used in `p' columns.}\\
\hline
Lots of lines& like this.\\
Lots of lines& like this.
\end{longtable}
 
% 
\section{Chunk Size\\块大小}

\columnratio{0.55}
\begin{paracol}{2}
\DescribeMacro{LTchunksize}
In order to \TeX\ multi-page tables, it is necessary to break up the
table into smaller chunks, so that \TeX\ does  not have to keep
everything in memory at one time. By default \env{longtable} uses 20
rows per chunk, but this can be set by the user, with e.g.,
"\setcounter{LTchunksize}{10}".\footnote
   {You can also use the plain \TeX\ syntax
   {\ttfamily\v\\LTchunksize=10.}}
These chunks do not affect page breaking,
thus if you are using a \TeX\ with a lot of memory, you can set
"LTchunksize" to be several pages of the table. \TeX\ will run
faster with a large "LTchunksize". However, if necessary,
\env{longtable} can work with "LTchunksize" set to 1, in which case
the memory taken up is negligible.
Note that if you use the commands for setting the table head or foot
(see below), the "LTchunksize" must be at least  as large as the
number of rows in each of the head or foot sections.
\switchcolumn
\DescribeMacro{LTchunksize}
为了能够在 \TeX 中处理多页表格,需要将表格分成较小的块,这样 \TeX 不必一次性将所有内容都存入内存中。默认情况下,\env{longtable} 每块包含 20 行,但用户可以通过例如 "\setcounter{LTchunksize}{10}" 的方式设置块大小。\footnote{你也可以使用普通的 \TeX 语法 {\ttfamily\v\\LTchunksize=10.}}这些块不会影响页面分割,因此如果你使用的 \TeX 具有大量内存,可以将 "LTchunksize" 设置为表格的几个页面。使用大的 "LTchunksize",\TeX 会运行得更快。但是,必要时,\env{longtable} 可以将 "LTchunksize" 设置为1,此时所占用的内存可以忽略不计。请注意,如果你使用设置表头或表尾的命令(见下文),则 "LTchunksize" 必须至少与表头或表尾中每个部分的行数一样大。

\switchcolumn[0]*
This document specifies "\setcounter{LTchunksize}{200}".  If you look
at the previous table, after the \emph{first} run of \LaTeX\  you will
see that various parts of the table do not line up.
\LaTeX\ will also have printed a warning that the column
widths had changed. \env{longtable} writes information onto the
".aux" file, so that it can line up the different chunks.
Prior to version~4 of this package, this information was not used
unless a "\setlongtables" command was issued,  however, now the
information is always used, using a new algorithm\footnote{Due to
David Kastrup.} and so "\setlongtables" is no longer needed. It is
defined (but does nothing) for the benefit of old documents that
use it.
\switchcolumn
本文档规定了"\setcounter{LTchunksize}{200}"。如果您查看前面的表格,在第一次运行\LaTeX\ 后,您会发现表格的各个部分没有对齐。\LaTeX\ 还会打印一个警告,表示列宽已更改。\env{longtable} 将信息写入".aux"文件中,以便对不同的部分进行对齐。在此软件包的版本4之前,除非发出 "\setlongtables" 命令,否则不使用此信息,但现在始终使用新算法\footnote{由David Kastrup开发。},因此不再需要 "\setlongtables"。为了方便使用旧文档的用户,它被定义(但不起作用)。
% \switchcolumn[0]*
% \begin{table*}
% \centering
% \begin{tabular}{||l|l|l||}
% \hline\hline
% A&\env{tabular}& environment\\
% \hline
% within&a floating&\env{table}\\
% \hline\hline
% \end{tabular}
% \caption{A floating table}
% \end{table*}
% \switchcolumn
% \begin{table*}
% \centering
% \begin{tabular}{||l|l|l||}
% \hline\hline
% 一个&\env{tabular}& 环境\\
% \hline
% 在&浮动&\env{table}内\\
% \hline\hline
% \end{tabular}
% \caption{一个浮动表格}
% \end{table*}
\end{paracol}

\begin{table}
\centering
\begin{tabular}{||l|l|l||}
\hline\hline
A&\env{tabular}& environment\\
\hline
within&a floating&\env{table}\\
\hline\hline
\end{tabular}
\caption{A floating table}
\end{table}
 
% \section{Captions and Headings\hfill 标题和表头}
\columnratio{0.55}
\begin{paracol}{2}
\DescribeMacro{\endhead}
At the start of the table one may specify lines which are to appear at
the top of every page (under the headline, but before the other lines
of the table).
\switchcolumn 
\DescribeMacro{\endhead}
在表格开头,可以指定出现在每一页顶部(在页眉下方,但在表格的其他行之前)的行。
\switchcolumn[0]*
The lines are entered as normal, but the last "\\" command is
replaced by a "\endhead" command.
\switchcolumn 这些行像正常的行一样输入,但是最后一个"\\"命令被"\endhead"命令替换。 

\switchcolumn[0]*
\DescribeMacro{\endfirsthead}
If the first page should have a different heading, then this should be
entered in the same way, and terminated with the "\endfirsthead"
command. The "LTchunksize" should be at least as large as the
number of rows in the heading.
\switchcolumn 
\DescribeMacro{\endfirsthead}
如果第一页需要有不同的表头,则应以相同的方式输入,并以"\endfirsthead"命令终止。 "LTchunksize"应至少与页眉中的行数一样大。

\switchcolumn[0]*
\DescribeMacro{\endfoot}
There are also "\endfoot" and "\endlastfoot"
\DescribeMacro{\endlastfoot}
commands which are used in the same way (at the \emph{start} of the
table) to specify rows (or an "\hline") to appear at the bottom of
each page. In certain situations, you may want to place lines which
logically belong in the table body at the end of the \env{firsthead},
or the beginning of the \env{lastfoot}. This helps to control which
lines appear on the first and last page of the table.\switchcolumn 

还有 "\endfoot" 和 "\endlastfoot" 命令,它们的使用方式相同(在表格的\emph{开头}),用于指定出现在每一页底部的行(或 "\hline")。在某些情况下,您可能希望将逻辑上属于表格主体的行放置在\env{firsthead}的末尾或\env{lastfoot}的开头。这有助于控制哪些行出现在表格的第一页和最后一页。

\switchcolumn[0]*
\DescribeMacro{\caption}%
The "\caption{...}" command is essentially equivalent to\switchcolumn
"\caption{...}"命令本质上等同于

\switchcolumn[0]*
"\multicolumn{n}{c}{\parbox{\LTcapwidth}{...}}"
\switchcolumn
"\multicolumn{n}{c}{\parbox{\LTcapwidth}{...}}"

\switchcolumn[0]*
where \texttt{n} is the number of columns of the table. You may set
the width of the caption with a command such as
"\setlength{\LTcapwidth}{2in}"
in the preamble of your document. The default is 4in. 
"\caption" also
writes the information to produce an entry in the list of tables. As
with the "\caption" command in the \env{figure} and \env{table}
environments, an optional argument specifies the text to appear in the
list of tables if this is different from the text to appear in the
caption. Thus the caption for table \ref{long} was specified as
{\ttfamily
 "\caption"[An optional table caption
            (used in the list of tables)]\v{A long
 table"\label{long}"\v}}.
\switchcolumn%%%%
其中\texttt{n}是表格的列数。若要设置标题的宽度,您可以在文档的导言部分使用类似
"\setlength{\LTcapwidth}{2in}"的命令。
默认值为4in。
"\caption"命令还会将信息写入表格目录中的条目。与\env{figure}和\env{table}环境中的"\caption"命令一样,可选参数用于指定在表格目录中显示的文本,如果这个文本与标题中显示的文本不同。因此,表\ref{long}的标题被指定为:{\ttfamily
"\caption"[An optional table caption
           (used in the list of tables)]\v{A long
table"\label{long}"\v}}。

\switchcolumn[0]*
You may wish the caption on later pages to be different to that on the
first page. In this case put the "\caption" command in the first
heading, and put a subsidiary caption in a "\caption[]" command in
the main heading. If the optional argument to "\caption" is empty,
no entry is made in the list of tables. Alternatively, if you do not
want the table number to be printed each time, use the "\caption*"
command.\switchcolumn 
你可能希望后面的页面的标题与第一页的标题不同。在这种情况下,将"\caption"命令放在第一个标题中,并在主标题中使用"\caption[]"命令来放置一个辅助标题。如果"\caption"的可选参数为空,则不会在表格列表中创建条目。或者,如果你不想每次打印表格编号,可以使用"\caption*"命令。

\switchcolumn[0]*
The captions are set based on the code for the \package{article}
class.
If you have redefined the standard "\@makecaption" command to produce
a different format for the captions, you may
need to make similar changes to the \package{longtable} version,
"\LT@makecaption". See the code section for more details.\switchcolumn 

标题是根据\package{article}类的代码设置的。如果您重新定义了标准的"@makecaption"命令以生成不同格式的标题,则可能需要对\package{longtable}版本"\LT@makecaption"进行类似的更改。有关更多详细信息,请参见代码部分。

\switchcolumn[0]*
A more convenient method of customising captions is given by the
\package{caption(2)} package, which provides commands for customising
captions, and arranges that the captions in standard environments, and
many environments provided by packages (including \package{longtable})
are modified in a compatible manner.\switchcolumn 

通过\package{caption(2)}包提供了更方便的自定义标题的方法,它提供了自定义标题的命令,并安排标准环境中的标题以及许多由包(包括\package{longtable})提供的环境以兼容的方式进行修改。

\switchcolumn[0]*
You may use the "\label" command so that you can cross reference
\env{longtable}s with "\ref". Note however, that the "\label" command
should not be used in a heading that may appear more than once. Place
it either in the \env{firsthead}, or in the body of the table. It
should not be the \emph{first} command in any entry.\switchcolumn 

您可以使用"\label"命令,以便您可以通过"\ref"进行跨引用\env{longtable}。但是请注意,不应在可能出现多次的标题中使用"\label"命令。将其放置在\env{firsthead}或表格主体中。它不应该是任何条目中的\emph{第一个}命令。
\end{paracol}
% 
\section{Multicolumn entries\hfill 多列条目}

\columnratio{0.55}
\begin{paracol}{2}

The "\multicolumn" command may be used in \env{longtable} in exactly
the same way as for \env{tabular}. So you may want to skip this
section, which is rather technical, however coping with "\multicolumn"
is one of the main problems for an environment such as
\env{longtable}. The main effect that a user will see is that
certain combinations of "\multicolumn" entries will result in a
document needing more runs of \LaTeX\ before the various `chunks' of
a table align.
\switchcolumn
在\env{longtable}中,"\multicolumn"命令可以与\env{tabular}完全相同的方式使用。因此,您可能希望跳过这一节,因为处理 "\multicolumn" 是 \env{longtable} 等环境的主要问题之一。用户将看到的主要效果是,某些 "\multicolumn" 条目的组合将导致文档需要多次运行\LaTeX,以使表格的各个"块"对齐。

\switchcolumn[0]*
The examples in this section are set with "LTchunksize" set to the
minimum value of one, to demonstrate the effects when "\multicolumn"
entries occur in different chunks.
\switchcolumn
本节中的示例设置"LTchunksize"为最小值1,以演示在不同块中出现"\multicolumn"条目时的效果。

\end{paracol}

\begin{table}[!htp]
\begin{center}
\LTchunksize=1
 \makeatletter

 \global\let\LT@save@row\relax
 \let\LT@warn\@gobble
 \let\LT@final@warn\relax

 \newcommand\ltexample[1]{
 \stepcounter{LT@tables}
 \expandafter\let\csname LT@\romannumeral\c@LT@tables\endcsname
                  \LT@save@row
 \addtocounter{LT@tables}{-1}
 \begin{longtable}{|c|c|c|}
 \caption{A difficult {\cs{multicolumn}} combination:
                               pass #1\label{pass#1}}\\
   \hline
   1&2&3\\
   \multicolumn{3}{|c|}{wide multicolumn spanning 1--3}\\
   \multicolumn{2}{|c|}{multicolumn 1--2}&3\\
   wide 1&2&3\\
   \hline
\end{longtable}
}

\ltexample{1}

\ltexample{2}

\ltexample{3}

\ltexample{4}

\end{center}
\end{table}

\columnratio{0.55}
\begin{paracol}{2}

Consider Table~\ref{pass1}.
In the second chunk, \env{longtable}  sees the wide
multicolumn entry.  At this point it thinks that the first two
columns are very narrow. All
the width of the multicolumn entry is assumed to be in the
third column. (This is a `feature' of \TeX's primitive "\halign"
command.) \env{longtable} then passes the information that there
is a wide third column to the later chunks, with the result that the
first pass over the table is too wide.
\switchcolumn 考虑表格~\ref{pass1}。在第二个块中,\env{longtable}看到宽的跨列条目。此时,它认为前两列非常窄。所有跨列条目的宽度都被假定在第三列中。(这是\TeX 原始的"\halign"命令的一个"特性")。然后,\env{longtable}将存在宽第三列的信息传递给后续块,导致对表格的第一次遍历过宽。

\switchcolumn[0]*
If the `saved row' from this first pass was re-inserted into the
table on the next pass, the table would line up in two passes, but
would be much two wide.
\switchcolumn 如果在下一次遍历中将这个"保存的行"重新插入到表格中,表格将在两次遍历中排列,但宽度会变得更宽。

\switchcolumn[0]*
\DescribeMacro{\kill}%
The solution to this problem used in Versions 1~and~2, was to use a
"\kill" line. If a line is "\kill"ed, by using "\kill" rather than
"\\" at the end of the line, it is used in calculating
column widths, but removed from the final table. Thus entering
"\kill"ed copies of the last two rows before the wide multicolumn
entry would  mean that "\halign" `saw' the wide entries in the first
two columns, and so would not widen the third column by so much to
make room for the multicolumn entry.
\switchcolumn
在版本1和2中用于解决这个问题的方法是使用"\kill"行。如果一行使用"\kill"而不是"\\"结束,那么它将在计算列宽时使用,但在最终表格中将被删除。因此,在宽的多列输入之前输入"\kill"的最后两行的副本意味着"\halign"看到了前两列中的宽输入,因此不会将第三列扩展得太多以为多列输入腾出空间。

\switchcolumn[0]*
In Version~3, a new solution was introduced. If the saved row in
the ".aux" file was not being used, \env{longtable} used a special
`draft' form of "\multicolumn", this modified the definition, so the
spanning entry was never considered to be wider than the columns it
spanned. So after the first pass, the ".aux" file stored the
widest normal entry for each column, no column was widened due to
"\span"ned columns. By default \env{longtable} ignored the ".aux"
file, and so each run of \LaTeX\ was considered a first pass. Once the
"\setlongtables" declaration was given, the saved row in the ".aux"
file, and the proper definition of "\multicolumn" were used. If any
"\multicolumn" entry caused one of the columns to be widened, this
information could not be passed back to earlier chunks, and so the
table would not correctly line up until the third pass. This algorithm
always converged in three passes as described above, but in examples
such as the ones in Tables \ref{pass1}--\ref{pass4}, the final
widths were not optimal as the width of column~2, which is
determined by a "\multicolumn" entry was not known when the final
width for column~3 was fixed, due to the fact that \emph{both}
"\multicolumn" commands were switched from `draft' mode to `normal'
mode at the same time.
\switchcolumn 在版本3中,引入了一种新的解决方案。如果".aux"文件中保存的行没有被使用,\env{longtable}会使用特殊的"草稿"形式的"\multicolumn",这会修改定义,使得跨越的条目永远不会被认为比其跨越的列更宽。因此,在第一遍扫描后,".aux"文件存储了每列最宽的普通条目,没有任何一列由于跨越的列而被加宽。默认情况下,\env{longtable}忽略".aux"文件,因此每次运行\LaTeX 都被视为第一遍扫描。一旦给出了"\setlongtables"声明,就会使用".aux"文件中保存的行和正确的"\multicolumn"定义。如果任何一个"\multicolumn"条目导致其中一列加宽,这个信息不能传递回早期的块,因此在第三遍扫描之前,表格不会正确地对齐。如上所述,这个算法总是在三遍扫描中收敛,但在表\ref{pass1}--\ref{pass4}中的示例中,最终宽度不是最优的,因为第2列的宽度是由"\multicolumn"条目确定的,当第3列的最终宽度被固定时,由于\emph{两个}"\multicolumn"命令同时从"草稿"模式切换到"正常"模式,第2列的宽度是未知的。

\switchcolumn[0]*
Version~4 alleviates the problem considerably.
The first pass of the table will
indeed have the third column much too wide. However, on the next pass
\env{longtable} will notice the error and reduce the column width
accordingly. If this has to propagate to chunks before the
"\multicolumn" one, an additional pass will, of course, be
needed. It is possible to construct tables where this rippling up of
the correct widths takes several passes to `converge' and produce a
table with all chunks aligned. However in
order to need many passes one needs to construct a table with
many overlapping "\multicolumn" entries, all being wider than the
natural widths of the columns they span, and all occurring in
different chunks. In the typical case the algorithm will converge
after three or four passes, and, the benefits of not needing to edit
the document before the final run to add "\setlongtables", and the
better choice of final column widths in the case of multiple
"\multicolumn" entries  will hopefully more than pay for the extra
passes that may possibly be needed.
\switchcolumn
版本4大大缓解了这个问题。表的第一遍传递确实会使第三列太宽。然而,在下一次传递中,\env{longtable}将注意到这个错误并相应地缩小列宽。如果这必须传播到"\multicolumn"之前的块中,当然需要额外的传递。可以构建表格,其中正确宽度的这种波动需要几次传递才能"收敛",并且产生所有块对齐的表格。但是,为了需要多次传递,需要构建一个表格,其中包含许多重叠的"\multicolumn"条目,所有这些条目都比它们跨越的列的自然宽度更宽,并且所有这些条目都出现在不同的块中。在典型情况下,算法将在三到四次传递后收敛,并且不需要在最终运行之前编辑文档以添加"\setlongtables"的好处,以及在多个"\multicolumn"条目的情况下更好的选择最终列宽将有望超过可能需要的额外传递的好处。

\switchcolumn[0]*
So Table~\ref{pass1} converges after 4~passes, as seen in
Table~\ref{pass4}.
\switchcolumn  因此,在第4次迭代之后,表\ref{pass1} 收敛,如表\ref{pass4} 所示。

\switchcolumn[0]*
You can still speed the convergence by introducing judicious "\kill"
lines, if you happen to have constellations like the above.
\switchcolumn 如果您恰好具有像上面那样的星座,您仍然可以通过引入明智的"\kill"行来加速收敛。

\switchcolumn[0]*
If you object even to \LaTeX-ing a file twice, you should
make the first line of
every \env{longtable} a "\kill" line that contains the widest entry
to be used in each column. All chunks will then line up on the first
pass.
\switchcolumn
如果你甚至反对将文件\LaTeX 编译两次,那么你应该在每个\env{longtable}的第一行添加一个包含每列中要使用的最宽条目的"\kill"行。所有的块都会在第一遍排列好。

\end{paracol}
% 
\section{Adjustment\hfill 调整}
\columnratio{0.55}
\begin{paracol}{2}
The optional argument of \env{longtable} controls the
horizontal alignment of the table. The possible options are "[c]",
"[r]" and "[l]", for  centring,
right  and left adjustment, respectively.
\switchcolumn 

\env{longtable}的可选参数控制表格的水平对齐方式。可能的选项有"[c]"、"[r]"和"[l]",分别表示居中、右对齐和左对齐。
\switchcolumn[0]*
\DescribeMacro{\LTleft}
Normally centring is the default, but this document specifies
\begin{verbatim}
\setlength\LTleft\parindent
\setlength\LTright\fill
\end{verbatim}
in the preamble,
\switchcolumn 
\DescribeMacro{\LTleft}
通常,默认值是居中对齐,但是在这个文档中,设置了如下的代码:
\DescribeMacro{\LTright}
\begin{verbatim}
\setlength\LTleft\parindent
\setlength\LTright\fill
\end{verbatim}
在导言部分.
\switchcolumn[0]*
\DescribeMacro{\LTright}
 which means that the tables are set flush left, but
indented by the usual paragraph indentation. Any lengths can be
specified for these two parameters, but at least one of them should be
a rubber length so that it fills up the width of the page, unless
rubber lengths are added between the columns using the
"\extracolsep" command.
For instance
\switchcolumn 

这意味着表格被设置为左对齐,但是缩进采用通常的段落缩进。这两个参数可以指定任何长度,但至少其中一个应该是弹性长度,以便填满页面的宽度,除非在列之间使用"\extracolsep"命令添加了弹性长度。例如:

\begin{verbatim}
 \begin{tabular*}{\textwidth}{@{\extracolsep{...}}...}
\end{verbatim}

\switchcolumn[0]*
produces a full width table, to get a similar effect with
\env{longtable} specify
\begin{verbatim}
\setlength\LTleft{0pt}
\setlength\LTright{0pt}
\begin{longtable}{@{\extracolsep{...}}...}
\end{verbatim}
\switchcolumn 
生成一个全宽的表格,如果要使用\env{longtable}实现类似效果,请指定如下代码:
\begin{verbatim}
\setlength\LTleft{0pt}
\setlength\LTright{0pt}
\begin{longtable}{@{\extracolsep{...}}...}
\end{verbatim}
\end{paracol}
% @{\extracolsep{\fill}}
 
% 
\section{Changes\hfill 变更}

\columnratio{0.55}
\begin{paracol}{2}
This section highlights the major changes since version~2. A more
detailed change log may be produced at the end of the code listing
if the "ltxdoc.cfg" file specifies
\begin{verbatim}
 \AtBeginDocument{\RecordChanges}
 \AtEndDocument{\PrintChanges}
\end{verbatim}
\switchcolumn
本节主要介绍自版本2以来的主要变更。如果"ltxdoc.cfg"文件指定了以下内容,
则在代码列表的末尾可以生成更详细的变更日志:
\begin{verbatim}
 \AtBeginDocument{\RecordChanges}
 \AtEndDocument{\PrintChanges}
\end{verbatim}

\switchcolumn[0]*
Changes made between versions 2 and 3.
\switchcolumn[0]
2和3版本之间的变更:
\end{paracol}




\begin{itemize}
\columnratio{0.55}
\begin{paracol}{2}
\switchcolumn[0]*
\item  The mechanism for adding the head and foot of the table has been
completely rewritten. With this new mechanism, \env{longtable} does
not need to issue a "\clearpage" at the start of the table, and so the
table may start half way down a page. Also the "\endlastfoot" command
which could not safely be implemented under the old scheme, has been
added.
\switchcolumn \item 完全重写了添加表格头部和尾部的机制。使用这种新机制,\env{longtable}在表格开始时不需要发出"\clearpage"命令,因此表格可以从页面的中间开始。还添加了"\endlastfoot"命令,在旧方案下无法安全实现。
\switchcolumn[0]*
\item  \env{longtable} now issues an error if started in the scope of
"\twocolumn", or the \env{multicols} environment.

\switchcolumn\item \env{longtable}现在会在"\twocolumn"或\env{multicols}环境的作用域中启动时发出错误。
\switchcolumn[0]*
\item  The separate documentation file "longtable.tex" has been
merged with the package file, "longtable.dtx" using Mittelbach's
\package{doc} package.

\switchcolumn\item 将独立的文档文件"longtable.tex"与包文件"longtable.dtx"合并,使用了Mittelbach的\package{doc}宏包。
\switchcolumn[0]*
\item 
 Support for footnotes has been added. Note however that
"\footnote" will not work in the `head' or `foot' sections of the
table. In order to put a footnote in those sections (e.g., inside a
caption), use "\footnotemark" at that point, and "\footnotetext"
anywhere in the table \emph{body} that will fall on the same page.

\switchcolumn\item 添加了对脚注的支持。但是需要注意的是,"\footnote"命令在表格的"head"或"foot"部分将无法正常工作。为了在这些部分(例如标题内部)放置脚注,请在该处使用"\footnotemark",并在表格正文中的任何位置(位于同一页上)使用"\footnotetext"。
\switchcolumn[0]*
\item 
 The treatment of "\multicolumn" has changed, making
"\kill" lines unnecessary, at the price of sometimes requiring a
third pass through \LaTeX.

\switchcolumn\item "\multicolumn"的处理方式已更改,不再需要使用"\kill"命令,但有时需要通过\LaTeX 进行三次编译。
\switchcolumn[0]*
\item 
 The "\newpage" command now works inside a \env{longtable}.
 \switchcolumn \item "\newpage"命令现在在\env{longtable}环境内也可用。
\end{paracol}
\end{itemize}

\columnratio{0.55}
\begin{paracol}{2}

\switchcolumn[0]*
Changes made between versions 3 and 4.
\switchcolumn
3和4版本之间的变更:
\end{paracol}




\begin{itemize}
    \columnratio{0.55}
\begin{paracol}{2}

\switchcolumn[0]*
\item 
 A new algorithm is used for aligning chunks. As well as the
widest width in each column, \package{longtable} remembers which
chunk produced this maximum. This allows it to check that the
maximum is still achieved in later runs. As \package{longtable} can
now deal with columns shrinking as the file is edited, the
"\setlongtables" system is no longer needed and is disabled.


\switchcolumn \item 采用了一种新的算法来对齐块。除了每一列的最宽宽度外,\package{longtable}还记住了哪个块产生了这个最大宽度。这使得它可以检查在后续运行中是否仍然达到了最大宽度。由于\package{longtable}现在可以处理文件编辑时列的收缩,因此不再需要和禁用"\setlongtables"系统。
\switchcolumn[0]*
\item 
 An extra benefit of the new algorithm's ability to deal with
`shrinking' columns is that it can give better (narrower) column
widths in the case of overlapping "\multicolumn" entries in
different chunks than the previous algorithm produced.


\switchcolumn\item 新算法的另一个好处是,在不同块中有重叠的"\multicolumn"条目的情况下,它可以给出更好(更窄)的列宽,而以前的算法则不能实现这一点。
\switchcolumn[0]*
\item 
 The `draft' multicolumn system has been removed, along with
related commands such as "\LTmulticolumn".


\switchcolumn\item 删除了"draft"多列系统,以及相关的命令,如"\LTmulticolumn"。
\switchcolumn[0]*
\item 
 The disadvantage of the new algorithm is that it can take more
passes. The theoretical maximum is approximately twice the length
of a `chain' of columns with overlapping "\multicolumn" entries,
although in practice it usually converges as fast as the old
version. (Which always converged in three passes once
"\setlongtables" was  activated.)


\switchcolumn\item 新算法的缺点是可能需要更多次的编译。理论上的最大次数约为具有重叠"\multicolumn"条目的列链的长度的两倍,但在实践中,它通常与旧版本一样快速收敛。(一旦激活"\setlongtables",旧版本总是在三次编译后收敛。)
\switchcolumn[0]*
\item 
 "\\*" and "\nopagebreak" commands may be used to control page
 breaking.
\switchcolumn\item 可以使用"\\*"和"\nopagebreak"命令来控制分页。
\end{paracol}
\end{itemize}




\section{Summary}

% 让表格粘附到左边距。
^^A Allow the table to stick into the left margin.
\setlength{\LTleft}{0pt plus 1fill minus 1fill}
\setlength{\LTright}{0pt}

% \begin{longtable}{@{}l@{\hspace{10pt}}p{.8\linewidth}@{}}

% \end{longtable}



\begin{longtable}{@{}l@{\hspace{10pt}}p{.8\linewidth}@{}}
\caption[A summary of \env{longtable} commands]%
        {\normalsize A summary of \env{longtable} commands}\\
\multicolumn{2}{c}{\textbf{Parameters\hfill 参数}}\\*
\hline
"\LTleft"&
    Glue to the left of the table. 表格左边的间距。          \hfill("\fill")\\
"\LTright"&
    Glue to the right of the table. 表格右边的间距。         \hfill("\fill")\\
"\LTpre"&
    Glue before the table.表格前的间距。      \hfill("\bigskipamount")\\
"\LTpost"&
    Glue after the table.表格后的间距。       \hfill("\bigskipamount")\\
"\LTcapwidth"&
    The width of a parbox containing the caption.包含标题的 parbox 的宽度。 \hfill(4in)\\
"LTchunksize"&
    The number of rows per chunk. 每个块中的行数。                \hfill(20)\\[5pt]
\multicolumn{2}{c}{\textbf{Optional
     arguments to} \ttfamily\v\\begin\v{longtable\v} \hfill \textbf{\ttfamily\v\\begin\v{longtable\v}} 的可选参数}\\*
\hline
\it none& Position as specified by "\LTleft" and "\LTright". 按照 "\LTleft" 和 "\LTright" 的设置位置。\\
"[c]"&   Centre the table.居中表格。\\
"[l]"&   Place the table flush left.将表格左对齐。\\
"[r]"&   Place the table flush right.将表格右对齐。\\[5pt]
\pagebreak[2]
\multicolumn{2}{c}{\textbf{Commands
                    to end table rows}\hfill \textbf{结束表格行的命令}}\\*
\hline
"\\"&
    Specifies the end of a row 指定行的结束。\\
"\\"\oarg{dim}& %
Ends row, then adds vertical space
    (as in the \env{tabular} environment). 结束行,然后添加垂直间距(与 \env{tabular} 环境中的行为相同)。\\
"\\*"&
    The same as "\\" but disallows a page break after the row. 与 "\\" 相同,但不允许在行后进行分页。\\
"\tabularnewline"&
    Alternative to "\\" for use in the scope of "\raggedright" and
    similar commands that redefine "\\".用于在 "\raggedright" 和类似命令的作用域内代替 "\\"。\\
"\kill"&
    Row is `killed', but is used in calculating widths.行被“杀死”,但用于计算宽度。\\
"\endhead"&
    Specifies rows to appear at the top of every page.指定在每页顶部出现的行。\\
"\endfirsthead"&
    Specifies rows to appear at the top the first page.指定在第一页顶部出现的行。\\
"\endfoot"&
    Specifies rows to appear at the bottom of every page.指定在每页底部出现的行。\\
"\endlastfoot"&
    Specifies rows to appear at the bottom of the last page.指定在最后一页底部出现的行。\\[5pt]
\multicolumn{2}{c}{\textbf{\env{longtable} caption commands}\hfill \textbf{\env{longtable} 标题命令}}\\*
\hline
"\caption"\marg{caption}&
    Caption `Table ?: \meta{caption}', and a `\meta{caption}'
    entry in the list of tables. 标题为“Table ?: \meta{caption}”,并在表格列表中添加一个 \meta{caption} 条目。\\
"\caption"\oarg{lot}\marg{caption}&
    Caption `Table ?: \meta{caption}', and a `\meta{lot}'
    entry in the list of tables.标题为“Table ?: \meta{caption}”,并在表格列表中添加一个 \meta{lot} 条目。\\
"\caption[]"\marg{caption}&
    Caption `Table ?: \meta{caption}',
    but no entry in the list of tables.标题为“Table ?: \meta{caption}”,但不在表格列表中添加条目。\\
"\caption*"\marg{caption}&
    Caption `\meta{caption}', but no entry in the list of tables.标题为 \meta{caption},但不在表格列表中添加条目。\\[5pt]
\multicolumn{2}{c}{%^^A
       \textbf{Commands available at the start of a row}\hfill \textbf{在行开始处可用的命令}}\\*
\hline
"\pagebreak"&
    Force a page break.强制分页。\\*
"\pagebreak"\oarg{val}& A `hint' between 0 and 4
 of the desirability  of a break.分页的“提示”,为 0 到 4 的值,表示分页的可取性。\\
"\nopagebreak"& Prohibit a page break.禁止分页。\\*
"\nopagebreak"\oarg{val}& A `hint' between 0 and 4 of the undesirability
   of a break.分页的“提示”,为 0 到 4 的值,表示分页的不可取性。\\
"\newpage"&
    Force a page break.强制分页。\\[5pt]
\multicolumn{2}{c}{\textbf{Footnote commands
                     available inside \env{longtable}}\hfill  \textbf{\env{longtable} 中可用的脚注命令}}\\*
\hline
"\footnote"&
    Footnotes, but may not be used in the table head \& foot.脚注,但不能在表格的头部和尾部使用。\\*
"\footnotemark"&
    Footnotemark, may be used in the table head \& foot.脚注标记,可以在表格的头部和尾部使用。\\*
"\footnotetext"&
    Footnote text, use in the table body.脚注文本,在表格的正文中使用。\\[5pt]
\multicolumn{2}{c}{\textbf{Setlongtables}}\\
\hline
"\setlongtables"&  Obsolete command. Does nothing now. 过时的命令,现在不起任何作用。
\end{longtable}


\end{document}
 
\DocInput{longtable.dtx}


\end{document}






% \input{longtable_
% .tex}

% \input{longtable_
% .tex} 


% \input{longtable_
% .tex}