\documentclass{ltxdoc}
\usepackage[heading=true
,scheme=chinese%中文方案
,fontset=none%不使用默认的字体设置
,space=auto%自动调整中英文间距
]{ctex}
\setCJKmainfont{FangZhengShuSong-GBK-1.ttf}[Path=/Users/virhuiai/hlProjects/Latex-Typesetting-Hub/font/方正/]%设置文本的中文有衬线字体
\setCJKsansfont{FangZhengHeiTi-GBK-1.ttf}[Path=/Users/virhuiai/hlProjects/Latex-Typesetting-Hub/font/方正/]%设置文本的中文无衬线字体为
\setCJKmonofont{FangZhengFangSong-GBK-1.ttf}[Path=/Users/virhuiai/hlProjects/Latex-Typesetting-Hub/font/方正/] %设置文本的中文等宽字体 
% \setCJKfamilyfont{fontKai}{LXGWWenKai-Regular.ttf}[Path=/Users/virhuiai/hlProjects/Latex-Typesetting-Hub/font/霞鹜文楷/]
\setCJKfamilyfont{fontKai}{FangZhengKaiTi-GBK-1.ttf}[Path=/Users/virhuiai/hlProjects/Latex-Typesetting-Hub/font/方正/]
\newcommand\fontKai{\CJKfamily{fontKai}}

\usepackage[a3paper,landscape]{geometry}
\usepackage{minted}
\usepackage{paracol}


\makeatletter
\providecommand*\input@path{}
\newcommand*\addinputpath[1]{\expandafter\def\expandafter\input@path\expandafter{\input@path#1}}
\makeatother
\addinputpath{%
{/Users/virhuiai/hlProjects/Latex-Typesetting-Hub/宏包文档翻译/表格_longtable/}%
}
\usepackage{parskip}
\parindent=0pt
\usepackage{longtable}
\begin{document}
% \globalcounter*
\globalcounter{section}
\DocInput{longtable-1.dtx}

\let\package\textsf
\let\env\textsf
\providecommand\finalclearpage{\clearpage}

\DeleteShortVerb{\|}
\MakeShortVerb{\"}


\makeatletter
\def\@oddfoot{\normalfont\rmfamily\dotfill Page \thepage\dotfill}
\def\@oddhead{\dotfill{\normalfont\ttfamily longtable.sty}\dotfill}
\def\ps@titlepage{\let\@oddhead\@empty}
\makeatother


\setlength\LTleft\parindent
\setlength\LTright\fill
\setcounter{LTchunksize}{10}

\def\v{\char`}

%  \title{The \textsf{longtable} package\thanks{This file
       has version number \fileversion, last
       revised \filedate.}}
\author{David Carlisle\thanks{%
\noindent 
The new algorithm for aligning `chunks'
of a table used in version 4 of this package was devised coded
and documented by David Kastrup.\\
\hspace*{2em}在这个包的第4个版本中,用于对表格的“块”进行对齐的新算法是由David Kastrup设计、编码和文档化的。
}\and 翻译:virhuiai}
\date{\filedate}

\MaintainedByLaTeXTeam{tools}
\maketitle

\begin{abstract}
\columnratio{0.55}
\begin{paracol}{2}
\noindent This package defines the \env{longtable} environment, a multi-page
version of \env{tabular}.
\switchcolumn
这个包定义了\env{longtable}环境,是\env{tabular}的多页版本。
\end{paracol}
\end{abstract} 
% ^^A: 这是一个特殊的字符,表示空格。在 LaTeX 中,^^A 用于在代码中插入注释,表示忽略该行后面的内容。
% \vbox to 100pt: 这是一个 LaTeX 命令,用于创建一个垂直的盒子(box),并指定其高度为100pt(点)。
% ^^A \vbox to 100pt makes the page breaks the same on the first run.
% \noindent\begin{minipage}[t][130pt]{\textwidth}
% \listoftables
% \end{minipage}
% \section{Introduction\hfill 介绍}

\columnratio{0.55}
\begin{paracol}{2}
The \package{longtable} package defines a new environment,
\DescribeEnv{longtable}
\env{longtable}, which has most of the features of the \env{tabular}
environment, but produces tables which may be broken by \TeX's
standard page-breaking algorithm. It also shares some features with
the \env{table} environment. In particular it uses the same counter,
\texttt{table}, and has a similar "\caption" command. Also, the
standard "\listoftables" command lists tables produced by either the
\env{table} or \env{longtable} environments.
\switchcolumn
\package{longtable} 宏包定义了一个新的环境 \env{longtable},它大多数特性与 \env{tabular} 环境相同,但是能够根据 \TeX 的标准分页算法将表格分页显示。同时,它也与 \env{table} 环境共享一些特性。特别是它使用相同的计数器 \texttt{table} 和类似的 "\caption" 命令。另外,"\listoftables" 命令能够列出由 \env{table} 或 \env{longtable} 环境生成的表格。
%%%%%%%%%%%%%%%%%%%%%%%%%%%%%%%%%%%%
\switchcolumn[0]*
The following example uses most of the features of the \env{longtable}
environment. An edited listing of the input for this example appears
in Section~\ref{listing}.
\switchcolumn 下面的示例使用了\env{longtable}环境的大多数功能。本示例的输入编辑列表在第~\ref{listing}节中。

\switchcolumn[0]*
\textbf{Note:} Various parts of the following table will
\textbf{not} line up correctly until this document has been run
through \LaTeX\ several times.  This is a characteristic feature of
this package, as described below.
\switchcolumn 注意:在此表中,各部分的对齐可能需要多次运行该文档才能正确实现。这是该软件包的特征之一,如下所述。
%%%%%%%%%%%%%%%%%%%%%%%%%%%%%%%%%%%%

\end{paracol}



\begin{longtable}{@{*}r||p{1in}@{*}}
KILLED & LINE!!!! \kill
\caption
[An optional table caption (used in the list of tables)]
{A long table\label{long}}\\
\hline\hline
\multicolumn{2}{@{*}c@{*}}%
     {This part appears at the top of the table}\\
\textsc{First}&\textsc{Second}\\
\hline\hline
\endfirsthead
\caption[]{(continued)}\\
\hline\hline
\multicolumn{2}{@{*}c@{*}}%
      {This part appears at the top of every other page}\\
\textbf{First}&\textbf{Second}\\
\hline\hline
\endhead
\hline
This goes at the&bottom.\\
\hline
\endfoot
\hline
These lines will&appear\\
in place of the & usual foot\\
at the end& of the table\\
\hline
\endlastfoot
\env{longtable}  columns  are specified& in the \\
same way as  in the \env{tabular}& environment.\\
"@{*}r||p{1in}@{*}"& in this case.\\
Each row ends with a& "\\" command.\\
The "\\"  command  has an& optional\\
argument, just as in& the\\
 \env{tabular}&environment.\\[10pt]
See the  effect  of "\\[10pt]"&?\\
Lots of lines& like this.\\
Lots of lines& like this.\\
Lots of lines& like this.\\
Lots of lines& like this.\\
Also  "\hline"  may be used,&  as in \env{tabular}.\\
\hline
That  was a "\hline"&.\\
\hline\hline
That  was "\hline\hline"&.\\
\multicolumn{2}{||c||}%
{This is a \ttfamily\v\\multicolumn\v{2\v}\v{||c||\v}}\\
If a  page break  occurs at a "\hline" then& a line is drawn\\
at the bottom of one  page  and at the& top of the next.\\
\hline
The  "[t] [b] [c]"  argument of \env{tabular}& can  not be used.\\
The optional argument may be  one of& "[l] [r] [c]"\\
to specify whether  the  table  should be& adjusted\\
to the  left, right& or centrally.\\
\hline\hline
Lots of lines& like this.\\
Lots of lines& like this.\\
Lots of lines& like this.\\
Lots of lines& like this.\\
Lots of lines& like this.\\
Lots of lines& like this.\\
Lots of lines& like this.\\
Lots of lines& like this.\\
Lots of lines& like this.\\
Lots of lines& like this.\\
Lots of lines& like this.\\
Lots of lines& like this.\\
Lots of lines& like this.\\
Lots of lines& like this.\\
Lots of lines& like this.\\
Lots of lines& like this.\\
Lots of lines& like this.\\
Lots of lines& like this.\\
Lots of lines& like this.\\
Lots of lines& like this.\\
Some lines may take up a lot of space, like this: &
    \raggedleft This last column is a ``p'' column so this
    ``row'' of the table can take up several lines. Note however that
    \TeX\ will  never break a page within such a row. Page breaks only
    occur between rows of the table or at "\hline" commands.
    \tabularnewline
Lots of lines& like this.\\
Lots of lines& like this.\\
Lots of lines& like this.\\
Lots of lines& like this.\\
Lots of lines& like this.\\
Lots of lines& like this.\\
Lots of lines& like this.\\
\hline
Lots\footnote{This is a footnote.} of lines& like this.\\
Lots   of   lines& like this\footnote{\env{longtable} takes special
    precautions, so that footnotes may also be used in `p' columns.}\\
\hline
Lots of lines& like this.\\
Lots of lines& like this.
\end{longtable}
 

\end{document}
 
\DocInput{longtable.dtx}


\end{document}