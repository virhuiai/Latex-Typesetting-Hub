\columnratio{0.55}
\begin{paracol}{2}
\begin{optionlist}
\switchcolumn[0]*%%%%%%%%%%%%
\item[breakafter (string) (\meta{none})]
Break lines after specified characters, not just at spaces, when \texttt{breaklines=true}.  Does not apply to |\mintinline|.
\switchcolumn
\item[breakafter (字符串) (\meta{none})]
在指定字符之后换行,而不仅仅是在空格处换行,当 \texttt{breaklines=true} 时。不适用于 |\mintinline|。
\switchcolumn[0]*%%%%%%%%%%%%
For example, \texttt{breakafter=-/} would allow breaks after any hyphens or slashes.  Special characters given to \texttt{breakafter} should be backslash-escaped (usually \texttt{\hashchar}, \texttt{\{}, \texttt{\}}, \texttt{\%}, \texttt{[}, \texttt{]}; the backslash \texttt{\textbackslash} may be obtained via \texttt{\textbackslash\textbackslash}).
\switchcolumn
例如,\texttt{breakafter=-/} 允许在任何连字符或斜杠之后换行。给 \texttt{breakafter} 提供的特殊字符应该进行反斜杠转义(通常是 \texttt{\hashchar}、\texttt{\{}、\texttt{\}}、\texttt{\%}、\texttt{[}、\texttt{]};反斜杠 \texttt{\textbackslash} 可以通过 \texttt{\textbackslash\textbackslash} 获得)。
\switchcolumn[0]*%%%%%%%%%%%%
For an alternative, see \texttt{breakbefore}.  When \texttt{breakbefore} and \texttt{breakafter} are used for the same character, \texttt{breakbeforegroup} and \texttt{breakaftergroup} must both have the same setting.
\switchcolumn
对于一个替代方案,请参见\texttt{breakbefore}。当\texttt{breakbefore}和\texttt{breakafter}同时用于同一个字符时,\texttt{breakbeforegroup}和\texttt{breakaftergroup}必须具有相同的设置。
\switchcolumn[0]*%%%%%%%%%%%%
\begin{longexample}
    \begin{minted}[breaklines, breakafter=d]{python}
    some_string = 'SomeTextThatGoesOnAndOnForSoLongThatItCouldNeverFitOnOneLine'
    \end{minted}
\end{longexample}
\switchcolumn
\begin{longexample}
    \begin{minted}[breaklines, breakafter=d]{python}
    some_string = 'SomeTextThatGoesOnAndOnForSoLongThatItCouldNeverFitOnOneLine'
    \end{minted}
\end{longexample}
\switchcolumn[0]*%%%%%%%%%%%%
\item[breakaftergroup] (boolean) (true)
When \texttt{breakafter} is used, group all adjacent identical characters together, and only allow a break after the last character.  When \texttt{breakbefore} and \texttt{breakafter} are used for the same character, \texttt{breakbeforegroup} and \texttt{breakaftergroup} must both have the same setting.
\switchcolumn
\item[breakaftergroup] (布尔值) (true)
当使用\texttt{breakafter}时,将所有相邻的相同字符分组在一起,并且只允许在最后一个字符之后换行。当\texttt{breakbefore}和\texttt{breakafter}同时用于同一个字符时,\texttt{breakbeforegroup}和\texttt{breakaftergroup}必须具有相同的设置。
\switchcolumn[0]*%%%%%%%%%%%%
\item[breakaftersymbolpre (string) (\string\,\string\footnotesize\string\ensuremath\{\_\string\rfloor\}, \,\footnotesize\ensuremath{_\rfloor})]
The symbol inserted pre-break for breaks inserted by \texttt{breakafter}.
\switchcolumn
\item[breakaftersymbolpre (字符串) (\string\,\string\footnotesize\string\ensuremath\{\_\string\rfloor\}, \,\footnotesize\ensuremath{_\rfloor})]
使用\texttt{breakafter}插入的换行符前的符号。
\switchcolumn[0]*%%%%%%%%%%%%
\item[breakaftersymbolpost (string) (\meta{none})]
The symbol inserted post-break for breaks inserted by \texttt{breakafter}.
\switchcolumn
\item[breakaftersymbolpost (字符串) (\meta{none})]
使用\texttt{breakafter}插入的换行符后的符号。
\end{optionlist}
\end{paracol}
