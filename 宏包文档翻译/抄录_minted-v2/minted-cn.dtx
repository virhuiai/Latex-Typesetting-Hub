例如,\texttt{breakbefore=A}允许在大写A之前断行。\texttt{breakbefore}所给出的特殊字符应该以反斜杠转义(通常为\texttt{\hashchar},\texttt{\{},\texttt{\}},\texttt{\%},\texttt{[},\texttt{]};反斜杠\texttt{\textbackslash}可以通过\texttt{\textbackslash\textbackslash}获得)。

对于另一种选择,请参见\texttt{breakafter}。当\texttt{breakbefore}和\texttt{breakafter}同时用于同一个字符时,\texttt{breakbeforegroup}和\texttt{breakaftergroup}必须具有相同的设置。

\begin{longexample}
    \begin{minted}[breaklines, breakbefore=A]{python}
    some_string = 'SomeTextThatGoesOnAndOnForSoLongThatItCouldNeverFitOnOneLine'
    \end{minted}
\end{longexample}

\begin{itemize}
  \item[breakbeforegroup](布尔值)(true)
  当使用\texttt{breakbefore}时,将所有相邻的相同字符分组在一起,并且只允许在第一个字符之前断行。当\texttt{breakbefore}和\texttt{breakafter}同时用于同一个字符时,\texttt{breakbeforegroup}和\texttt{breakaftergroup}必须具有相同的设置。

  \item[breakbeforesymbolpre](字符串)(\string\,\string\footnotesize\string\ensuremath\{\_\string\rfloor\},\,\footnotesize\ensuremath{_\rfloor})
  在\texttt{breakbefore}插入的断行符号之前插入的符号。

  \item[breakbeforesymbolpost](字符串)(\meta{none})
  在\texttt{breakbefore}插入的断行符号之后插入的符号。

  \item[breakbytoken](布尔值)(false)
  仅在不在标记内部的位置断行,防止标记被分割成多行。默认情况下,\texttt{breaklines}会在最靠近边缘的空格处断行。虽然这最小化了所需的断行数量,但如果断行发生在字符串或类似标记的中间,可能会不方便。

  此选项与\texttt{draft}模式不兼容。可以在\url{http://pygments.org/docs/tokens/}上找到完整的Pygments标记列表。如果由\texttt{breakbytoken}提供的断行发生在意外的位置,可能表示语言的Pygments词法分析器存在错误或不足。

  \item[breakbytokenanywhere](布尔值)(false)
  类似于\texttt{breakbytoken},但也允许在相邻标记之间进行断行,而不仅仅是在由空格分隔的标记之间。在使用\texttt{breakanywhere}时,使用\texttt{breakbytokenanywhere}是多余的。

  \item[breakindent](尺寸)(\meta{breakindentnchars})
  当一行被断行时,将连续行缩进这个量。当|breakautoindent|和|breakindent|一起使用时,缩进会相加。此缩进与|breaksymbolindentleft|相结合,给出总的实际左缩进。

  不适用于\texttt{\string\mintinline}。

  \item[breakindentnchars](整数)(0)
  这允许将|breakindent|指定为整数个字符而不是作为尺寸(假设使用等宽字体)。

  \item[breaklines](布尔值)(false)
  在\texttt{minted}环境和\texttt{\string\mint}命令中自动断行长行,并在\texttt{\string\mintinline}中换行例如,\texttt{breakbefore=A}允许在大写字母A之前换行。给定给\texttt{breakbefore}的特殊字符应该进行反斜杠转义(通常为\texttt{\hashchar},\texttt{\{},\texttt{\}},\texttt{\%},\texttt{[},\texttt{]};通过\texttt{\textbackslash\textbackslash}可以获得反斜杠\texttt{\textbackslash})。

另一种选择是使用\texttt{breakafter}。当\texttt{breakbefore}和\texttt{breakafter}同时用于同一个字符时,\texttt{breakbeforegroup}和\texttt{breakaftergroup}必须具有相同的设置。

\begin{longexample}
    \begin{minted}[breaklines, breakbefore=A]{python}
    some_string = 'SomeTextThatGoesOnAndOnForSoLongThatItCouldNeverFitOnOneLine'
    \end{minted}
\end{longexample}

\begin{itemize}
  \item[breakbeforegroup](布尔值)(true)
  当使用\texttt{breakbefore}时,将所有相邻的相同字符分组在一起,并且只允许在第一个字符之前换行。当\texttt{breakbefore}和\texttt{breakafter}同时用于同一个字符时,\texttt{breakbeforegroup}和\texttt{breakaftergroup}必须具有相同的设置。

  \item[breakbeforesymbolpre](字符串)(\string\,\string\footnotesize\string\ensuremath\{\_\string\rfloor\},\,\footnotesize\ensuremath{_\rfloor})
  由\texttt{breakbefore}插入的换行前插入的符号。

  \item[breakbeforesymbolpost](字符串)(\meta{none})
  由\texttt{breakbefore}插入的换行后插入的符号。

  \item[breakbytoken](布尔值)(false)
  仅在不在标记内部的位置换行,防止标记被分割。默认情况下,\texttt{breaklines}在距离边缘最近的空格处换行。尽管这样可以最小化所需的换行数量,但如果在字符串或类似标记的中间发生换行,则可能会不方便。

  这与\texttt{draft}模式不兼容。完整的Pygments标记列表可在\url{http://pygments.org/docs/tokens/}上找到。如果由\texttt{breakbytoken}提供的换行发生在意外的位置,则可能表示语言的Pygments词法分析器存在错误或不足。

  \item[breakbytokenanywhere](布尔值)(false)
  类似于\texttt{breakbytoken},但也允许在相邻标记之间换行,而不仅仅是在由空格分隔的标记之间。与\texttt{breakanywhere}一起使用\texttt{breakbytokenanywhere}是多余的。

  \item[breakindent](尺寸)(\meta{breakindentnchars})
  当一行被换行时,缩进连续行的量。当同时使用|breakautoindent|和|breakindent|时,缩进值相加。此缩进与|breaksymbolindentleft|结合,给出实际左缩进的总和。

  不适用于\texttt{\string\mintinline}。

  \item[breakindentnchars](整数)(0)
  这允许将|breakindent|指定为整数字符数而不是作为尺寸(假设使用等宽字体)。

  \item[breaklines](布尔值)(false)
  自动在\texttt{minted}环境和\texttt{\string\mint}命令中换行长行,并在\texttt{\string\mintinline}中换行。