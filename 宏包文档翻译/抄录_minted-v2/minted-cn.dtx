\section{选项}

\subsection{宏包选项}

\DescribeMacro{chapter}
在加载\pkg{minted}宏包时,可以通过传递|section|或|chapter|选项来控制\LaTeX{}如何计数|listing|浮动体。例如,以下代码将使得列表按章节计数:

\mint[frame=lines]{latex}/\usepackage[chapter]{minted}/


\DescribeMacro{cache=\meta{boolean} (默认值:true)}
\pkg{minted}通过将代码保存到临时文件中,使用\app{Pygments}对代码进行高亮,并将输出保存到另一个临时文件中,然后将输出插入到\LaTeX{}文档中。如果需要高亮显示多个代码块,这个过程可能会变得非常慢。为了避免这种情况,该宏包提供了一个|cache|选项,默认情况下为开启状态。

|cache|选项会在文档的根目录下创建一个名为|_minted-|\meta{jobname}的目录(可以使用|cachedir|选项自定义目录名)。\footnote{实际上,该目录的命名是使用了“清理”过的\meta{jobname}的副本,其中空格和星号被替换为下划线,双引号被删除。如果文件名包含空格,\texttt{\string\jobname}将包含带引号的名称,除了在旧版本的MiKTeX中,该名称将使用将空格替换为星号的名称。使用“清理”过的\meta{jobname}比适应各种转义约定更简单。} 高亮显示的代码文件将存储在该目录中,以便以后不需要再次进行高亮显示。在大多数情况下,缓存会显著加快文档的编译速度。

不再使用的缓存文件会自动删除。\footnote{这取决于主辅助文件未被删除或损坏。如果发生这种情况,您只需删除缓存目录并重新开始。}


\DescribeMacro{cachedir=\meta{directory} (默认值:\_minted-\meta{jobname})}
该选项允许指定存储缓存文件的目录。路径应该使用正斜杠,即使在Windows下也是如此。

特殊字符必须进行转义。例如,|cachedir=~/mintedcache|是无效的,因为波浪号|~|会被转换为非换行空格的\LaTeX{}命令,而不是按字面意义对待。相反,使用|\string~/mintedcache|、|\detokenize{~/mintedcache}|或类似的解决方案。

路径可以包含空格,但只有在整个\meta{directory}被放在花括号|{}|中,并且空格被引用时才可以。例如,
\begin{Verbatim}
cachedir = {\detokenize{~/"minted cache"/"with spaces"}}
\end{Verbatim}

请注意,如果指定了|outputdir|,则缓存目录是相对于|outputdir|的。


\DescribeMacro{finalizecache=\meta{boolean} (默认值:false)}
在某些情况下,可能希望在不允许|-shell-escape|的环境中使用\pkg{minted}。例如,文档可能会被提交给出版商、预印版本服务器或与不支持|-shell-escape|的在线服务一起使用。只要不需要修改\pkg{minted}内容,就可以做到这一点。

使用|finalizecache|选项编译缓存以供在不需要|-shell-escape|的环境中使用。\footnote{通常,缓存文件的命名是使用高亮设置和高亮文本的MD5哈希值。使用\texttt{finalizecache}选项,会使用\texttt{listing<number>.pygtex}方案重命名缓存文件。这样可以更容易地匹配文档内容和缓存文件,并且对于没有内置MD5功能的XeTeX引擎(在TeX Live 2016之前),这是必需的。}完成此操作后,可以将|finalizecache|选项替换为|frozencache|选项,以后就可以在不需要|-shell-escape|的情况下使用冻结(静态)缓存。

\DescribeMacro{fontencoding=\meta{encoding} (默认值:\meta{doc~encoding})}
设置用于排版代码的字体编码。

例如,|fontencoding=T1|。

\DescribeMacro{frozencache=\meta{boolean} (默认值:false)}
使用使用|finalizecache|选项创建的冻结(静态)缓存。当开启|frozencache|时,无需|-shell-escape|,也不需要Python和Pygments。此外,通过|\inputminted|访问的任何外部文件也不再需要。

\textbf{请谨慎使用此选项。在启用|frozencache|之前,文档在\pkg{minted}看来必须是最终形式,而且必须使用|finalizecache|编译文档。开启此选项后,除非直接编辑缓存文件,否则无法修改\pkg{minted}内容。不可能更改任何需要Pygments或Python的\pkg{minted}设置。如果在开启|frozencache|后错误地修改了\pkg{minted}内容,\pkg{minted}将无法检测到这些修改。}

如果使用|frozencache|,并且希望验证\pkg{minted}的设置或内容是否以无效的方式进行了修改,可以使用以下步骤测试缓存。
\begin{enumerate}
\item 获取使用|frozencache|的缓存的副本。
\item 在支持|-shell-escape|的环境中使用|finalizecache=true|和|frozencache=false|编译文档。这实际上重新生成了冻结(静态)缓存。
\item 将原始缓存与新生成的缓存进行比较。在Linux和OS X下,可以使用|diff|命令;在Windows下,可能需要使用|fc|命令。
\end{enumerate}

***文档的其余内容已被截断。***