\columnratio{0.55}
\begin{paracol}{2}
\section{Basic usage}
\switchcolumn
\section{基本使用}
\switchcolumn[0]*%%%%%%%%%%%%
\subsection{Preliminary}
\label{sec:basic:preliminary}
\switchcolumn
\subsection{准备工作}
\switchcolumn[0]*%%%%%%%%%%%%
Since \pkg{minted} makes calls to the outside world (that is, \app{Pygments}), you need to tell the
\LaTeX{} processor about this by passing it the |-shell-escape| option or it won't allow such calls.
In effect, instead of calling the processor like this:
\switchcolumn
由于\pkg{minted}会调用外部程序(即\app{Pygments}),您需要通过传递|-shell-escape|选项告诉\LaTeX{}处理器,否则它不会允许这样的调用。实际上,您需要像这样调用处理器:
\switchcolumn[0]*%%%%%%%%%%%%
\begin{Verbatim}[commandchars=\\\{\}]
\$ latex input
\end{Verbatim}
\switchcolumn
\begin{Verbatim}[commandchars=\\\{\}]
\$ latex input
\end{Verbatim}
\switchcolumn[0]*%%%%%%%%%%%%
you need to call it like this:
\switchcolumn
您需要像这样调用它:
\switchcolumn[0]*%%%%%%%%%%%%
\begin{Verbatim}[commandchars=\\\{\}]
\$ latex -shell-escape input
\end{Verbatim}
\switchcolumn
\begin{Verbatim}[commandchars=\\\{\}]
\$ latex -shell-escape input
\end{Verbatim}
\switchcolumn[0]*%%%%%%%%%%%%
The same holds for other processors, such as |pdflatex| or |xelatex|.
\switchcolumn
其他处理器,如|pdflatex|或|xelatex|,也是一样的。
\switchcolumn[0]*%%%%%%%%%%%%
You should be aware that using |-shell-escape| allows \LaTeX\ to run potentially arbitrary commands on your system.  It is probably best to use |-shell-escape| only when you need it, and to use it only with documents from trusted sources.
\switchcolumn
您应该知道,使用|-shell-escape|允许\LaTeX\ 在您的系统上运行潜在的任意命令。最好只在需要时使用|-shell-escape|,并且只与来自可信源的文档一起使用。
\switchcolumn[0]*%%%%%%%%%%%%
\subsubsection*{Working with OS X}
\switchcolumn
\subsubsection*{在OS X上使用}
\switchcolumn[0]*%%%%%%%%%%%%
If you are using \pkg{minted} with some versions/configurations of OS X, and are using caching with a large number of code blocks ($>256$), you may receive an error like
\begin{Verbatim}
OSError: [Errno 24] Too many open files:
\end{Verbatim}
\switchcolumn
如果您在某些版本/配置的OS X上使用\pkg{minted},并且使用大量代码块($>256$)进行缓存,您可能会遇到以下错误:
\begin{Verbatim}
OSError: [Errno 24] Too many open files:
\end{Verbatim}
\switchcolumn
This is due to the way files are handled by the operating system, combined with the way that caching works.  To resolve this, you may use the OS X commands |launchctl limit maxfiles| or |ulimit -n| to increase the number of files that may be used.
\switchcolumn
这是由操作系统处理文件的方式以及缓存工作方式造成的。要解决此问题,您可以使用OS X命令|launchctl limit maxfiles|或|ulimit -n|来增加可使用的文件数量。
\switchcolumn[0]*%%%%%%%%%%%%
\subsection{A minimal complete example}
\switchcolumn
\subsection{一个最小的完整示例}
\switchcolumn[0]*%%%%%%%%%%%%
The following file |minimal.tex| shows the basic usage of \pkg{minted}.
\switchcolumn
以下文件|minimal.tex|显示了\pkg{minted}的基本使用方法。
\switchcolumn[0]*%%%%%%%%%%%%
\begin{VerbatimOut}[gobble=1]{minted.doc.out}
  \documentclass{article}

  \usepackage{minted}

  \begin{document}
  \begin{minted}{c}
  int main() {
      printf("hello, world");
      return 0;
  }
  \end{minted}
  \end{document}
\end{VerbatimOut}
\inputminted[frame=lines]{latex}{minted.doc.out}
\switchcolumn
\inputminted[frame=lines]{latex}{minted.doc.out}
\switchcolumn[0]*%%%%%%%%%%%%
By compiling the source file like this:
\switchcolumn
通过这样编译源文件:
\switchcolumn[0]*%%%%%%%%%%%%
\begin{Verbatim}[commandchars=\\\{\}]
\$ pdflatex -shell-escape minimal
\end{Verbatim}
\switchcolumn
\begin{Verbatim}[commandchars=\\\{\}]
\$ pdflatex -shell-escape minimal
\end{Verbatim}
\switchcolumn[0]*%%%%%%%%%%%%
we end up with the following output in |minimal.pdf|:
\switchcolumn
我们得到了以下输出结果在|minimal.pdf|中:
\end{paracol}

\hfill
\colorbox{minted@samplebg}{\begin{minipage}{0.6\textwidth}
    \inputminted[firstline=7,lastline=10]{c}{minted.doc.out}
\end{minipage}}
\hfill\hfill



\columnratio{0.55}
\begin{paracol}{2}
\switchcolumn[0]*%%%%%%%%%%%%
\subsection{Formatting source code}
\switchcolumn
\subsection{格式化源代码}
\switchcolumn[0]*%%%%%%%%%%%%
\DescribeEnv{minted}
Using \pkg{minted} is straightforward. For example, to highlight some Python source code we might use
the following code snippet (result on the right):
\switchcolumn
\DescribeEnv{minted}
使用\pkg{minted}非常简单。例如,要高亮一些Python源代码,我们可以使用以下代码片段(右侧是结果):

\end{paracol}

\begin{example}
    \begin{minted}{python}
    def boring(args = None):
        pass
    \end{minted}
\end{example}

\columnratio{0.55}
\begin{paracol}{2}
\switchcolumn[0]*%%%%%%%%%%%%
Optionally, the environment accepts a number of options in |key=value| notation, which are described
in more detail below.
\switchcolumn
可选地,该环境接受以|key=value|符号表示的一些选项,下面将对其进行详细描述。
\switchcolumn[0]*%%%%%%%%%%%%
\DescribeMacro{\mint}
For a single line of source code, you can alternatively use a shorthand notation:
\switchcolumn
\DescribeMacro{\mint}
对于一行源代码,您也可以使用简写符号:
\end{paracol}

\begin{example}
    \mint{python}|import this|
\end{example}

\columnratio{0.55}
\begin{paracol}{2}



\switchcolumn[0]*%%%%%%%%%%%%
This typesets a single line of code using a command rather than an environment, so it saves a little typing, but its output is equivalent to that of the |minted| environment.
\switchcolumn
这使用了一个命令而不是一个环境来排版一行代码,因此它节省了一些输入,但其输出与|minted|环境的输出是等效的。
\switchcolumn[0]*%%%%%%%%%%%%
The code is delimited by a pair of identical characters, similar to how  |\verb| works.  The complete syntax is \\
\cmd\mint\oarg{options}\marg{language}\meta{delim}\meta{code}\meta{delim}\\,
where the code delimiter can be almost any punctuation character.
The \meta{code} may also be delimited with matched curly braces |{}|, so long as \meta{code} itself does not contain unmatched curly braces.
Again, this command supports a number of options described below.
\switchcolumn
代码由一对相同的字符界定,类似于|\verb|的工作方式。完整的语法是\\
\cmd\mint\oarg{options}\marg{language}\meta{代码界定符}\meta{code}\meta{代码界定符}\\,其中代码界定符可以是几乎任何标点字符。如果\meta{code}本身不包含不匹配的花括号,则\meta{code}也可以由匹配的花括号|{}|界定。同样,该命令支持一些下面描述的选\subsection{代码块选项}
\switchcolumn[0]*%%%%%%%%%%%%
Note that the |\mint| command \textbf{is not for inline use}.  Rather, it is a shortcut for |minted| when only a single line of code is present.  The |\mintinline| command is provided for inline use.
\switchcolumn
请注意,|\mint| 命令\textbf{不能用于内联使用}。相反,它是|minted|的快捷方式,仅适用于单行代码。|\mintinline|命令用于内联使用。
\switchcolumn[0]*%%%%%%%%%%%%
\DescribeMacro{\mintinline}
Code can be typeset inline:
\switchcolumn
\DescribeMacro{\mintinline}
代码可以内联排版:
\end{paracol}

\begin{example}
    X\mintinline{python}{print(x**2)}X
\end{example}

\columnratio{0.55}
\begin{paracol}{2}
\switchcolumn[0]*%%%%%%%%%%%%
The syntax is  \cmd\mintinline\oarg{options}\marg{language}\meta{delim}\meta{code}\meta{delim}.  The delimiters can be a pair of characters, as for \cmd\mint.  They can also be a matched pair of curly braces, |{}|.
\switchcolumn
该命令的语法是 \cmd\mintinline\oarg{options}\marg{language}\meta{delim}\meta{code}\meta{delim}. 分隔符可以是一对字符,就像 \cmd\mint 一样。也可以是一对匹配的花括号,|{}|。
\switchcolumn[0]*%%%%%%%%%%%%
The command has been carefully crafted so that in most cases it will function correctly when used inside other commands.\footnote{For example, \mintinline{latex}{\mintinline} works in footnotes!  The main exception is when the code contains the percent \texttt{\%} or hash \texttt{\#} characters, or unmatched curly braces.}
\switchcolumn
该命令已经被精心设计,以便在大多数情况下,当在其他命令中使用时,它可以正确地工作。\footnote{例如,\mintinline{latex}{\mintinline} 可以在脚注中工作!唯一的例外是代码中包含百分号 \texttt{\%} 或井号 \texttt{\#} 字符,或者未匹配的花括号。}
\switchcolumn[0]*%%%%%%%%%%%%
\DescribeMacro{\inputminted}
Finally, there's the |\inputminted| command to read and format whole files.
Its syntax is \cmd\inputminted\oarg{options}\marg{language}\marg{filename}.

\switchcolumn
最后,还有一个 |\inputminted| 命令用于读取和格式化整个文件。其语法是 \\\cmd\inputminted\oarg{options}\marg{language}\marg{filename}。
\switchcolumn[0]*%%%%%%%%%%%%
\subsection{Using different styles}
\switchcolumn
\subsection{使用不同的样式}
\switchcolumn[0]*%%%%%%%%%%%%
\DescribeMacro{\usemintedstyle}
Instead of using the default style you may choose another stylesheet provided by \app{Pygments}.  This may be done via the following:
\switchcolumn
\DescribeMacro{\usemintedstyle}
您可以选择使用 \app{Pygments} 提供的其他样式表,而不是使用默认样式。可以通过以下方式实现:
\switchcolumn[0]*%%%%%%%%%%%%
\mint[frame=lines]{latex}/\usemintedstyle{name}/
\switchcolumn
\mint[frame=lines]{latex}/\usemintedstyle{name}/
\switchcolumn[0]*%%%%%%%%%%%%
The full syntax is \cmd\usemintedstyle\oarg{language}\marg{style}.  The style may be set for the document as a whole (no language specified), or only for a particular language.  Note that the style may also be set via \cmd\setminted\ and via the optional argument for each command and environment.\footnote{Version 2.0 added the optional language argument and removed the restriction that the command be used in the preamble.}
\switchcolumn
完整的语法是 \cmd\usemintedstyle\oarg{language}\marg{style}。样式可以为整个文档设置(不指定语言),也可以为特定语言设置。请注意,样式也可以通过 \cmd\setminted\ 和每个命令和环境的可选参数设置。\footnote{版本2.0添加了可选的语言参数,并删除了该命令必须在导言区使用的限制。}
\switchcolumn[0]*%%%%%%%%%%%%
To get a list of all available stylesheets, see the online demo at the \href{http://pygments.org/demo/}{Pygments website} or execute the following command on the command line:
\switchcolumn
要获取所有可用样式表的列表,请参阅 \href{http://pygments.org/demo/}{Pygments 网站}上的在线演示,或在命令行上执行以下命令:
\switchcolumn[0]*%%%%%%%%%%%%
\begin{Verbatim}[commandchars=\\\{\}]
\$ pygmentize -L styles
\end{Verbatim}
\switchcolumn
\begin{Verbatim}[commandchars=\\\{\}]
\$ pygmentize -L styles
\end{Verbatim}
\switchcolumn[0]*%%%%%%%%%%%%
Creating your own styles is also easy. Just follow the instructions provided on the
\href{http://pygments.org/docs/styles/#creating-own-styles}{\pkg{Pygments} website}.
\switchcolumn
创建自己的样式也非常简单。只需按照 \href{http://pygments.org/docs/styles/#creating-own-styles}{\pkg{Pygments} 网站}上提供的说明进行操作。
\switchcolumn[0]*%%%%%%%%%%%%
\subsection{Supported languages}
\switchcolumn
\subsection{支持的语言}
\switchcolumn[0]*%%%%%%%%%%%%
\app{Pygments} supports over 300 different programming languages, template languages, and other markup languages.  To see an exhaustive list of the currently supported languages, use the command
\begin{Verbatim}[commandchars=\\\{\}]
\$ pygmentize -L lexers
\end{Verbatim}
\switchcolumn
\app{Pygments} 支持超过300种不同的编程语言、模板语言和其他标记语言。要查看当前支持的语言的详尽列表,请使用以下命令:
\begin{Verbatim}[commandchars=\\\{\}]
\$ pygmentize -L lexers
\end{Verbatim}
\end{paracol}