\begin{optionlist}
\columnratio{0.55}
\begin{paracol}{2}
\switchcolumn[0]*%%%%%%%%%%%%
\item[firstline (integer) (1)]
The first line to be shown.
All lines before that line are ignored and do not appear in the output.
\switchcolumn
\item[firstline (整数) (1)]
要显示的第一行。在此行之前的所有行将被忽略,不会出现在输出中。
\switchcolumn[0]*%%%%%%%%%%%%
\item[lastline (integer) (\meta{last line of input})]
The last line to be shown.
\switchcolumn
\item[lastline (整数) (\meta{输入的最后一行})]
要显示的最后一行。

\switchcolumn[0]*%%%%%%%%%%%%
\item[firstnumber (auto \| last \| integer) (auto = 1)]
Line number of the first line.
\switchcolumn
\item[firstnumber (auto \| last \| 整数) (auto = 1)]
第一行的行号。

\switchcolumn[0]*%%%%%%%%%%%%
\item[stepnumber (integer) (1)]
  Interval at which line numbers appear.
  \switchcolumn
  \item[stepnumber (整数) (1)]
  出现行号的间隔。

  \switchcolumn[0]*%%%%%%%%%%%%

\item[stepnumberfromfirst (boolean) (false)]
By default, when line numbering is used with |stepnumber| $\ne 1$, only line numbers that are a multiple of |stepnumber| are included.  This offsets the line numbering from the first line, so that the first line, and all lines separated from it by a multiple of |stepnumber|, are numbered.

\switchcolumn
\item[stepnumberfromfirst (布尔值) (false)]
默认情况下,当使用|stepnumber| $\ne 1$进行行编号时,只包括|stepnumber|的倍数的行号。这会使行编号从第一行偏移,以便第一行和其后以|stepnumber|的倍数分隔的所有行都被编号。
\switchcolumn[0]*%%%%%%%%%%%%
\item[stepnumberoffsetvalues (boolean) (false)]
By default, when line numbering is used with |stepnumber| $\ne 1$, only line numbers that are a multiple of |stepnumber| are included.  Using |firstnumber| to offset the numbering will change which lines are numbered and which line gets which number, but will not change which \emph{numbers} appear.  This option causes |firstnumber| to be ignored in determining which line numbers are a multiple of |stepnumber|.  |firstnumber| is still used in calculating the actual numbers that appear.  As a result, the line numbers that appear will be a multiple of |stepnumber|, plus |firstnumber| minus 1.
\switchcolumn
\item[stepnumberoffsetvalues (布尔值) (false)]
默认情况下,当使用|stepnumber| $\ne 1$进行行编号时,只包括|stepnumber|的倍数的行号。使用|firstnumber|进行偏移将改变编号的行和行获得的编号,但不会改变出现的编号。此选项导致在确定哪些行号是|stepnumber|的倍数时忽略|firstnumber|。仍然使用|firstnumber|来计算实际出现的数字。结果是,出现的行号将是|stepnumber|的倍数,加上|firstnumber|减1。
\switchcolumn[0]*%%%%%%%%%%%%

\switchcolumn[0]*%%%%%%%%%%%%
\item[numberfirstline (boolean) (false)]
Always number the first line, regardless of |stepnumber|.
\switchcolumn
\item[numberfirstline (布尔值) (false)]
 总是对第一行进行编号,而不管|stepnumber|如何。


 \switchcolumn[0]*%%%%%%%%%%%%
  \item[linenos (boolean) (false)]
    Enables line numbers.
    In order to customize the display style of line numbers, you need to redefine the |\theFancyVerbLine| macro:
    \switchcolumn
    \item[linenos (布尔值) (false)]
    启用行号。要自定义行号的显示样式,需要重定义|\theFancyVerbLine|宏:
    \switchcolumn[0]*%%%%%%%%%%%%
\begin{example}
    \renewcommand{\theFancyVerbLine}{\sffamily
        \textcolor[rgb]{0.5,0.5,1.0}{\scriptsize
        \oldstylenums{\arabic{FancyVerbLine}}}}

    \begin{minted}[linenos,
        firstnumber=11]{python}
    def all(iterable):
        for i in iterable:
            if not i: 
                return False
        return True
    \end{minted}
\end{example}
\switchcolumn[0]*%%%%%%%%%%%%
\item[numbers (left \| right \| both \| none) (none)]
Essentially the same as |linenos|, except the side on which the numbers appear may be specified.
\switchcolumn
\item[numbers (left \| right \| both \| none) (none)]
 与|linenos|基本相同,只是可以指定数字显示的位置。

\switchcolumn[0]*%%%%%%%%%%%%
  \item[numberblanklines (boolean) (true)]
    Enables or disables numbering of blank lines.
    \switchcolumn
    \item[numberblanklines (布尔值) (true)]
    启用或禁用空行的编号。
\switchcolumn[0]*%%%%%%%%%%%%
  \item[numbersep (dimension) (12pt)]
    Gap between numbers and start of line.
    \switchcolumn
    \item[numbersep (尺寸) (12pt)]
    数字和行的起始位置之间的间距。
  


\end{paracol}

\end{optionlist}
