\begin{optionlist}
\columnratio{0.55}
\begin{paracol}{2}
\switchcolumn[0]*%%%%%%%%%%%%
\item[breaksymbol (string) (breaksymbolleft)]
Alias for \texttt{breaksymbolleft}.
\switchcolumn
\item[breaksymbol (string) (breaksymbolleft)]
是\texttt{breaksymbolleft}的别名。
\switchcolumn[0]*%%%%%%%%%%%%
\item[breaksymbolleft (string) (\string\tiny\string\ensuremath\{\string\hookrightarrow\}, {\tiny\ensuremath{\hookrightarrow}})]
The symbol used at the beginning (left) of continuation lines when \texttt{breaklines=true}.  To have no symbol, simply set \texttt{breaksymbolleft} to an empty string (``\texttt{=,}'' or ``\texttt{=\{\}}'').  The symbol is wrapped within curly braces \texttt{\{\}} when used, so there is no danger of formatting commands such as \texttt{\string\tiny} ``escaping.''
\switchcolumn
\item[breaksymbolleft (string) (\string\tiny\string\ensuremath\{\string\hookrightarrow\}, {\tiny\ensuremath{\hookrightarrow}})]
在\texttt{breaklines=true}时,用于表示连续行开头的符号。要没有符号,只需将breaksymbolleft设置为空字符串(``\texttt{=,}'' 或 ``\texttt{=\{\}}'') 。当使用时,符号将被包裹在花括号\texttt{\{\}}中,因此不会出现格式命令(如 \texttt{\string\tiny})“逃逸”的危险。

\switchcolumn[0]*%%%%%%%%%%%%
The \texttt{\string\hookrightarrow} and \texttt{\string\hookleftarrow} may be further customized by the use of the \texttt{\string\rotatebox} command provided by \pkg{graphicx}.  Additional arrow-type symbols that may be useful are available in the \pkg{dingbat} (\texttt{\string\carriagereturn}) and \pkg{mnsymbol} (hook and curve arrows) packages, among others.
\switchcolumn

\texttt{\string\hookrightarrow}和\texttt{\string\hookleftarrow}可以通过使用\pkg{graphicx}提供的\texttt{\string\rotatebox}命令进行进一步自定义。其他可能有用的箭头类型符号可以在\pkg{dingbat}(\texttt{\string\carriagereturn})和\pkg{mnsymbol}(hook和curve arrows)等包中找到,还有其他一些包也提供了这样的符号。

\switchcolumn[0]*%%%%%%%%%%%%
Does not apply to \texttt{\string\mintinline}.
\switchcolumn
不适用于 \texttt{\string\mintinline}。

\switchcolumn[0]*%%%%%%%%%%%%
\item[breaksymbolright (string) (\meta{none})]
The symbol used at breaks (right) when \texttt{breaklines=true}. Does not appear at the end of the very last segment of a broken line.
\switchcolumn
\item[breaksymbolright (string) (\meta{none})]
在\texttt{breaklines=true}时,用于表示断行(右边)的符号。不会出现在最后一行的最后一个片段的末尾。    
\switchcolumn[0]*%%%%%%%%%%%%
\item[breaksymbolindent (dimension) (\meta{breaksymbolindentleftnchars})]
Alias for |breaksymbolindentleft|.
\switchcolumn
\item[breaksymbolindent (dimension) (\meta{breaksymbolindentleftnchars})]
别名为 |breaksymbolindentleft|。
\switchcolumn[0]*%%%%%%%%%%%%
\item[breaksymbolindentnchars (integer) (\meta{breaksymbolindentleftnchars})]
Alias for |breaksymbolindentleftnchars|.
\switchcolumn
\item[breaksymbolindentnchars (整数) (\meta{breaksymbolindentleftnchars})]
别名为 |breaksymbolindentleftnchars|。
\switchcolumn[0]*%%%%%%%%%%%%
\item[breaksymbolindentleft (dimension) (\meta{breaksymbolindentleftnchars})]
The extra left indentation that is provided to make room for |breaksymbolleft|.  This indentation is only applied when there is a |breaksymbolleft|.
\switchcolumn
\item[breaksymbolindentleft (尺寸) (\meta{breaksymbolindentleftnchars})]
提供额外的左缩进以为 |breaksymbolleft| 腾出空间。只有当存在 |breaksymbolleft| 时才应用该缩进。
\switchcolumn[0]*%%%%%%%%%%%%
Does not apply to \texttt{\string\mintinline}.
\switchcolumn
不适用于 \texttt{\string\mintinline}。
\switchcolumn[0]*%%%%%%%%%%%%
\item[breaksymbolindentleftnchars (integer) (4)]
This allows |breaksymbolindentleft| to be specified as an integer number of characters rather than as a dimension (assumes a fixed-width font).
\switchcolumn
\item[breaksymbolindentleftnchars (整数) (4)]
允许将 |breaksymbolindentleft| 指定为字符的整数数量,而不是作为尺寸(假设使用等宽字体)。
\switchcolumn[0]*%%%%%%%%%%%%
\item[breaksymbolindentright (dimension) (\meta{breaksymbolindentrightnchars})]
The extra right indentation that is provided to make room for |breaksymbolright|.  This indentation is only applied when there is a |breaksymbolright|.
\switchcolumn
\item[breaksymbolindentright (尺寸) (\meta{breaksymbolindentrightnchars})]
提供额外的右缩进以为 |breaksymbolright| 腾出空间。只有当存在 |breaksymbolright| 时才应用该缩进。
\switchcolumn[0]*%%%%%%%%%%%%
\item[breaksymbolindentrightnchars (integer) (4)]
This allows |breaksymbolindentright| to be specified as an integer number of characters rather than as a dimension (assumes a fixed-width font).
\switchcolumn
\item[breaksymbolindentrightnchars (整数) (4)]
允许将 |breaksymbolindentright| 指定为字符的整数数量,而不是作为尺寸(假设使用等宽字体)。
\switchcolumn[0]*%%%%%%%%%%%%
\item[breaksymbolsep (dimension) (\meta{breaksymbolsepleftnchars})]
Alias for |breaksymbolsepleft|.
\switchcolumn
\item[breaksymbolsep (尺寸) (\meta{breaksymbolsepleftnchars})]
别名为 |breaksymbolsepleft|。
\switchcolumn[0]*%%%%%%%%%%%%
\item[breaksymbolsepnchars (integer) (\meta{breaksymbolsepleftnchars})]
Alias for |breaksymbolsepleftnchars|.
\switchcolumn
\item[breaksymbolsepnchars (整数) (\meta{breaksymbolsepleftnchars})]
别名为 |breaksymbolsepleftnchars|。
\switchcolumn[0]*%%%%%%%%%%%%
\item[breaksymbolsepleft (dimension) (\meta{breaksymbolsepleftnchars})]
The separation between the |breaksymbolleft| and the adjacent text. 
\switchcolumn
\item[breaksymbolsepleft (尺寸) (\meta{breaksymbolsepleftnchars})]
|breaksymbolleft| 与相邻文本之间的间距。
\switchcolumn[0]*%%%%%%%%%%%%
\item[breaksymbolsepleftnchars (integer) (2)]
Allows |breaksymbolsepleft| to be specified as an integer number of characters rather than as a dimension (assumes a fixed-width font).
\switchcolumn
\item[breaksymbolsepleftnchars (整数) (2)]
允许将 |breaksymbolsepleft| 指定为字符的整数数量,而不是作为尺寸(假设使用等宽字体)。
\switchcolumn[0]*%%%%%%%%%%%%
\item[breaksymbolsepright (dimension) (\meta{breaksymbolseprightnchars})]
The \emph{minimum} separation between the |breaksymbolright| and the adjacent text.  This is the separation between |breaksymbolright| and the furthest extent to which adjacent text could reach.  In practice, |\linewidth| will typically not be an exact integer multiple of the character width (assuming a fixed-width font), so the actual separation between the |breaksymbolright| and adjacent text will generally be larger than |breaksymbolsepright|.  This ensures that break symbols have the same spacing from the margins on both left and right.  If the same spacing from text is desired instead, |breaksymbolsepright| may be adjusted.  (See the definition of |\FV@makeLineNumber| in \pkg{fvextra} for implementation details.)
\switchcolumn
\item[breaksymbolsepright (尺寸) (\meta{breaksymbolseprightnchars})]
|breaksymbolright| 与相邻文本之间的\emph{最小}间距。这是 |breaksymbolright| 与相邻文本可能达到的最远范围之间的间距。在实践中,|\linewidth| 通常不是字符宽度的精确整数倍(假设使用等宽字体),因此 |breaksymbolright| 与相邻文本的实际间距通常大于 |breaksymbolsepright|。这确保了断行符号与左右两侧边距的间距相同。如果希望与文本具有相同的间距,可以调整 |breaksymbolsepright|。(有关实现细节,请参见 \pkg{fvextra} 中的 |\FV@makeLineNumber| 的定义。)
\switchcolumn[0]*%%%%%%%%%%%%
\item[breaksymbolseprightnchars (integer) (2)]
Allows |breaksymbolsepright| to be specified as an integer number of characters rather than as a dimension (assumes a fixed-width font).
\switchcolumn
\item[breaksymbolseprightnchars (整数) (2)]
允许将 |breaksymbolsepright| 指定为字符的整数数量,而不是作为尺寸(假设使用等宽字体)。
\end{paracol}
\end{optionlist}
