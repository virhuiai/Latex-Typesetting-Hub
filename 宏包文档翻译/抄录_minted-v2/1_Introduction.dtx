%
^^A % \addvspace{\parskip}
^^A% \begin{parcolumns}[rulebetween=true,colwidths={1=.6\linewidth}]{2}
{\pkg{minted} is a package that allows formatting source code in \LaTeX.
% For example:}

\pkg{minted} 是一个允许在 \LaTeX 中格式化源代码的包。例如:

^^A% \end{parcolumns}
%
^^A TODO 这是在哪里生成的 minted.doc.out
% \begin{VerbatimOut}[gobble=1]{minted.doc.out}
%   \begin{minted}{<language>}
%     <code>
%   \end{minted}
% \end{VerbatimOut}
% \inputminted[gobble=2,frame=lines]{latex}{minted.doc.out}
%
^^A % \addvspace{\parskip}
^^A% \begin{parcolumns}[rulebetween=true,colwidths={1=.6\linewidth}]{2}
{will highlight a piece of code in a chosen language.
% The appearance can be customized with a number of options and color schemes.}
% \par{将会高亮显示所选语言的一段代码。外观可以使用许多选项和颜色方案进行自定义。}
^^A 将会以选定的语言高亮显示一段代码。
^^A 外观可以通过许多选项和配色方案进行定制。
^^A% \end{parcolumns}
%
%
^^A % \addvspace{\parskip}
^^A% \begin{parcolumns}[rulebetween=true,colwidths={1=.6\linewidth}]{2}
{Unlike some other packages, most notably \pkg{listings}, \pkg{minted} requires
the installation of additional software, \app{Pygments}.
This may seem like a disadvantage, but there are also significant advantages.}
% \par{与其他一些包(特别是 \pkg{listings})不同,\pkg{minted} 需要安装额外的软件 \app{Pygments}。这可能看起来像是一个劣势,但也有显著的优势。}
^^A% \end{parcolumns}
^^A 与其他一些软件包不同(最著名的是\pkg{listings}),\pkg{minted}需要额外安装Pygments。%
^^A 这似乎是一个缺点,但也有显著的优点。
^^A % todo pip3安装,在minted.sty中也有体现
^^A % \addvspace{\parskip}
^^A% \begin{parcolumns}[rulebetween=true,colwidths={1=.6\linewidth}]{2}
% \par{\app{Pygments} provides superior syntax highlighting 
compared to conventional packages.
For example, \pkg{listings} basically only highlights strings, comments and keywords.
\app{Pygments}, on the other hand, 
can be completely customized to highlight any kind of token 
the source language might support.
This might include special formatting sequences inside strings, numbers, different kinds of
identifiers and exotic constructs such as HTML tags.}
% \par{相比于传统的包,\app{Pygments} 提供了更好的语法高亮。例如,\pkg{listings} 基本上只高亮字符串、注释和关键字。另一方面,\app{Pygments} 可以完全定制以高亮源语言可能支持的任何类型的标记。这可能包括字符串内的特殊格式序列、数字、不同类型的标识符和像 HTML 标记这样的奇特构造。}
^^A 与常规的代码高亮宏包相比,\app{Pygments}提供了卓越的语法高亮功能。%
^^A 例如,\pkg{listings}基本上只突出显示字符串、注释和关键词。%
^^A 而\app{Pygments}, 则可以完全定制要突出显示的,源语言可能支持的任何类型的token。
^^A 这其中包括字符串中的特殊格式化序列、数字、不同种类的标识符和特殊的结构,如HTML标签。
^^A% \end{parcolumns}
%
^^A % \addvspace{\parskip}
^^A% \begin{parcolumns}[rulebetween=true,colwidths={1=.6\linewidth}]{2}
{Some languages make this especially desirable.
Consider the following Ruby code as an extreme, but at the same time typical, example:}
% \par{对于某些语言,这特别有用。考虑以下 Ruby 代码作为一个极端但同时也是典型的例子:}
^^A 有些语言特别需要这样做。%
^^A 考虑一下以下Ruby代码,作为一个极端、同时也是典型的的例子:
^^A% \end{parcolumns}
%
% \begin{minted}[gobble=3]{ruby}
%   class Foo
%     def init
%       pi = Math::PI
%       @var = "Pi is approx. #{pi}"
%     end
%   end
% \end{minted}
%
^^A % \addvspace{\parskip}
^^A% \begin{parcolumns}[rulebetween=true,colwidths={1=.6\linewidth}]{2}
{Here we have four different colors for identifiers (five, if you count keywords) and escapes from
inside strings, none of which pose a problem for \app{Pygments}.}
% \par{这里有四种不同的标识符颜色(如果您关键字也算上则为五种),以及来自字符串内部的转义,它们都对 \app{Pygments} 没有问题。}
^^A 在这里,我们有四种不同颜色的标识符(如果算上关键字,则是五种),以及字符串内部的转义,
^^A %完成这些需求,\app{Pygments} 都没问题。
^^A \app{Pygments} 都可以完成这些需求。
^^A% \end{parcolumns}
%
^^A % \addvspace{\parskip}
^^A% \begin{parcolumns}[rulebetween=true,colwidths={1=.6\linewidth}]{2}
{Additionally, installing \app{Pygments} is actually incredibly easy (see the next section).}
% \par{此外,安装 \app{Pygments} 实际上非常简单(见下一节)。}
^^A 此外,安装\app{Pygments}%实际上
^^A 是非常容易的(请参阅下一节)。
^^A% \end{parcolumns}
%