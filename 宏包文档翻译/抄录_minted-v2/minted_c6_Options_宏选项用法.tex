\columnratio{0.55}
\begin{paracol}{2}
\subsection{Macro option usage}
\switchcolumn
\subsection{宏选项用法}
\switchcolumn[0]*%%%%%%%%%%%%
All \pkg{minted} highlighting commands accept the same set of options.
Options are specified as a comma-separated list of |key=value| pairs.
For example, we can specify that the lines should be numbered:
\switchcolumn
所有的 \pkg{minted} 高亮命令都接受相同的选项集合。
选项以逗号分隔的 |key=value| 对的形式指定。
例如,我们可以指定行数的显示:
\end{paracol}

\begin{example}
        \begin{minted}[linenos=true]{c++}
        #include <iostream>
        int main() {
            std::cout << "Hello "
                        << "world"
                        << std::endl;
        }
        \end{minted}
\end{example}

\columnratio{0.55}
\begin{paracol}{2}
\switchcolumn[0]*%%%%%%%%%%%%
An option value of |true| may also be omitted entirely (including the ``|=|'').
To customize the display of the line numbers further, override the |\theFancyVerbLine| command.
Consult the \pkg{fancyvrb} documentation for details.
\switchcolumn
也可以完全省略选项值为 |true| 的部分(包括``|=|''符号)。
要进一步自定义行数的显示方式,请覆盖 |\theFancyVerbLine| 命令。
有关详细信息,请参阅 \pkg{fancyvrb} 文档。
\switchcolumn[0]*%%%%%%%%%%%%
|\mint| accepts the same options:
\switchcolumn
|\mint| 命令也接受相同的选项:
\end{paracol}
\begin{example}
    \mint[linenos]{perl}|$x=~/foo/|
\end{example}
\columnratio{0.55}
\begin{paracol}{2}
\switchcolumn[0]*%%%%%%%%%%%%
Here's another example: we want to use the \LaTeX{} math mode inside comments:
\switchcolumn
下面是另一个例子:我们希望在注释中使用 \LaTeX{} 的数学模式:
\end{paracol}
\begin{example}
    \begin{minted}[mathescape]{python}
    # Returns $\sum_{i=1}^{n}i$
    def sum_from_one_to(n):
        r = range(1, n + 1)
        return sum(r)
    \end{minted}
\end{example}
\columnratio{0.55}
\begin{paracol}{2}

\switchcolumn[0]*%%%%%%%%%%%%
To make your \LaTeX{} code more readable you might want to indent the code inside a |minted|
environment.
The option |gobble| removes these unnecessary whitespace characters from the output.  There is also an |autogobble| option that detects the length of this whitespace automatically.
\switchcolumn
为了使你的 \LaTeX{} 代码更具可读性,你可能希望在 |minted| 环境中对代码进行缩进。
选项 |gobble| 可以从输出中删除这些不必要的空白字符。还有一个 |autogobble| 选项,可以自动检测这些空白字符的长度。
\end{paracol}
\begin{example}
    \begin{minted}[gobble=2,showspaces]{python}
        def boring(args = None):
            pass
    \end{minted}
\end{example}

\begin{example}
    \begin{minted}[showspaces]{python}
    def boring(args = None):
        pass
    \end{minted}
\end{example}
%与

\columnratio{0.55}
\begin{paracol}{2}
\switchcolumn[0]*%%%%%%%%%%%%
\DescribeMacro{\setminted}
You may wish to set options for the document as a whole, or for an entire language.  This is possible via \cmd\setminted\oarg{language}\marg{key=value,...}.  Language-specific options override document-wide options.  Individual command and environment options override language-specific options.
\switchcolumn
\DescribeMacro{\setminted}
您可能希望为整个文档或整个语言设置选项。这可以通过\\ \cmd\setminted\oarg{language}\marg{key=value,...} 实现。语言特定的选项会覆盖文档范围的选项。单独的命令和环境选项会覆盖语言特定的选项。
\switchcolumn[0]*%%%%%%%%%%%%
\DescribeMacro{\setmintedinline}
You may wish to set separate options for \cmd\mintinline, either for the document as a whole or for a specific language.  This is possible via \cmd\setmintedinline.  The syntax is\\
\cmd\setmintedinline\oarg{language}\marg{key=value,...}.  Language-specific options override document-wide options. Individual command options override language-specific options.  All settings specified with \cmd\setmintedinline\ override those set with \cmd\setminted.  That is, inline settings always have a higher precedence than general settings.
\switchcolumn
\DescribeMacro{\setmintedinline}
您可能希望为 \cmd\mintinline 设置单独的选项,无论是为整个文档还是为特定的语言。这可以通过 \cmd\setmintedinline 来实现。其语法为\\
\cmd\setmintedinline\oarg{language}\marg{key=value,...}。语言特定的选项会覆盖文档范围的选项。单独的命令选项会覆盖语言特定的选项。所有使用 \cmd\setmintedinline 设置的选项都会覆盖使用 \cmd\setminted 设置的选项。也就是说,内联设置始所有的 \LaTeX{} 代码都会比一般设置具有更高的优先级。
\end{paracol}