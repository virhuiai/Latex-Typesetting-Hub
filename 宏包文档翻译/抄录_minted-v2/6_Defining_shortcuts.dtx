% Large documents with a lot of listings will nonetheless use the same source language and the
% same set of options for most listings.
% Always specifying all options is redundant, a lot to type and makes performing changes hard.

^^A % 有大量列表的大型文件仍然会使用相同的源语言和
^^A % 大多数列表使用相同的选项集。
^^A % 总是指定所有的选项是多余的,要打很多字,而且很难进行修改。

%对于有大量源码列表的大型文档,尽管使用相同的源语言和相同的选项集,但仍然需要为大多数源码列表指定所有选项,这是冗余的,需要输入很多内容,并且使更改变得困难。 

% One option is to use \cmd\setminted, but even then you must still specify the language each time.

%一种选项是使用\cmd\setminted ,但即使如此,您仍然必须每次指定语言。
%
% \pkg{minted} therefore defines a set of commands that lets you define shortcuts for the highlighting commands.
% Each shortcut is specific for one programming language.
%
%因此,\pkg{minted}定义了一组命令,让您为高亮命令定义快捷方式。每个快捷方式都特定于一种编程语言。

% \DescribeMacro{\newminted}
% |\newminted| defines a new alias for the |minted| environment:
%
|\newminted|定义了一个|minted|环境的新别名:

% \begin{example}
%   \newminted{cpp}{gobble=2,linenos}
%
%   \begin{cppcode}
%     template <typename T>
%     T id(T value) {
%         return value;
%     }
%   \end{cppcode}
% \end{example}
%
% If you want to provide extra options on the fly, or override existing default options, you can do that, too:
%
%如果您想要在操作过程中提供额外的选项,或覆盖默认的选项设置,您也可以这样做:

% \begin{example}
%   \newminted{cpp}{gobble=2,linenos}
%
%   \begin{cppcode*}{linenos=false,
%                    frame=single}
%     int const answer = 42;
%   \end{cppcode*}
% \end{example}
%
% Notice the star ``|*|'' behind the environment name---due to restrictions in \pkg{fancyvrb}'s handling
% of options, it is necessary to provide a \emph{separate} environment that accepts options, and the options
% are \emph{not} optional on the starred version of the environment.
%
% 请注意环境名称后面的星号“|*|”,由于\pkg{fancyvrb}处理选项的限制,需要提供一个\emph{单独的}接受选项的环境,而这些选项在星号版本的环境中\emph{不是}可选的。

% The default name of the environment is \meta{language}|code|.
% If this name clashes with another environment or if you want to choose an own name for another reason, you may
% do so by specifying it as the first argument:

%环境的默认名称为\meta{language}|code|。如果此名称与另一个环境冲突,或者出于其他原因想要选择自己的名称,可以将其指定为第一个参数:

\cmd\newminted\oarg{environment name}\marg{language}\marg{options}。

%
% Like normal \pkg{minted} environments, environments created with |\newminted| may be used within other environment definitions.  Since the \pkg{minted} environments use \pkg{fancyvrb} internally, any environment based on them must include the \pkg{fancyvrb} command |\VerbatimEnvironment|.  This allows \pkg{fancyvrb} to determine the name of the environment that is being defined, and correctly find its end.  It is best to include this command at the beginning of the definition.  For example,

与普通的\pkg{minted}环境一样,使用|\newminted|创建的环境可以在其他环境定义中使用。由于\pkg{minted}环境在内部使用\pkg{fancyvrb},因此基于它们的任何环境都必须包括\pkg{fancyvrb}命令|\VerbatimEnvironment|。这使得\pkg{fancyvrb}能够确定正在定义的环境的名称,并正确找到其结尾。最好在定义的开头包含此命令。例如,

%\begin{Verbatim}
%\newminted{cpp}{gobble=2,linenos}
%\newenvironment{env}{\VerbatimEnvironment\begin{cppcode}}{\end{cppcode}}
%\end{Verbatim}
%
% \DescribeMacro{\newmint}
% The above macro only defines shortcuts for the |minted| environment.
% The main reason is that the short command form |\mint| often needs different options---at the very least, it
% will generally not use the |gobble| option.
% A shortcut for |\mint| is defined using \cmd\newmint\oarg{macro name}\marg{language}\marg{options}.
% The arguments and usage are identical to |\newminted|.
% If no \meta{macro name} is specified, \meta{language} is used.
%
%上面的宏仅为|minted|环境定义了快捷方式。 主要原因是|mint|的短命令形式通常需要不同的选项——至少,它通常不使用|gobble|选项。 使用\cmd\newmint\oarg{宏名}\marg{语言}\marg{选项}定义了一个|\mint|的快捷方式。 参数和用法与|\newminted|相同。 如果没有指定\meta{macro name},则使用\meta{language}。

% \begin{example}
%   \newmint{perl}{showspaces}
%
%   \perl/my $foo = $bar;/
% \end{example}
%
%
% \DescribeMacro{\newmintinline}
% This creates custom versions of \cmd\mintinline.  The syntax is the same as that for \cmd\newmint:

%这将创建\cmd\mintinline的自定义版本。其语法与\cmd\newmint的语法相同:

\cmd\newmintinline\oarg{macro~name}\marg{language}\marg{options}.  

If a \meta{macro~name} is not specified, then the created macro is called |\|\meta{language}|inline|.
%
如果未指定\meta{macro~name},则创建的宏将被称为|\|\meta{language}|inline|。
 
% \begin{example}
%   \newmintinline{perl}{showspaces}
%
%   X\perlinline/my $foo = $bar;/X
% \end{example}
%
% \DescribeMacro{\newmintedfile}
% This creates custom versions of \cmd\inputminted.  The syntax is

%这将创建自定义版本的\cmd\inputminted。语法为

% \begin{center}
% \cmd\newmintedfile\oarg{macro~name}\marg{language}\marg{options}
% \end{center}
% If no \meta{macro name} is given, then the macro is called |\|\meta{language}|file|.
%
 如果没有给出\meta{macro name},则该宏被称为||\meta{language}|file|。