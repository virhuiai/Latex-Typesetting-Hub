\columnratio{0.55}
\begin{paracol}{2}
\section{Installation} 
\switchcolumn
\section{安装}
\switchcolumn[0]*%%%%%%%%%
\subsection{Prerequisites}
\switchcolumn
\subsection{先决条件}
\switchcolumn[0]*%%%%%%%%%
\app{Pygments} is written in Python, so make sure that you have Python 2.6 or later installed on your system.  This may be easily checked from the command line:

\begin{Verbatim}[gobble=3,commandchars=\\\{\}]
  \$ python --version
  Python 2.7.5
\end{Verbatim}

\switchcolumn
\switchcolumn[0]*%%%%%%%%%
If you don't have Python installed, you can download it from the \href{http://www.python.org/download/}{Python website} or
use your operating system's package manager.

\switchcolumn
\switchcolumn[0]*%%%%%%%%%
Some Python distributions include \pkg{Pygments} (see some of the options under ``Alternative Implementations'' on the Python site).  Otherwise, you will need to install \pkg{Pygments} manually. This may  be done by installing \href{http://pypi.python.org/pypi/setuptools}{\app{setuptools}}, which facilitates the distribution of Python applications.  You can then install \app{Pygments} using the following command:
\begin{Verbatim}[gobble=3,commandchars=\\\{\}]
  \$ sudo easy_install Pygments
\end{Verbatim}
Under Windows, you will not need the |sudo|, but may need to run the command prompt as administrator.  \pkg{Pygments} may also be installed with |pip|:
\begin{Verbatim}[gobble=3,commandchars=\\\{\}]
  \$ pip install Pygments
\end{Verbatim}

\switchcolumn
\switchcolumn[0]*%%%%%%%%%
If you already have \app{Pygments} installed, be aware that the latest version is recommended (at least 1.4 or later).  Some features, such as |escapeinside|, will only work with 2.0+.  \pkg{minted} may work with versions as early as 1.2, but there are no guarantees.

\switchcolumn
\switchcolumn[0]*%%%%%%%%%
\subsection{Required packages}

\switchcolumn
\switchcolumn[0]*%%%%%%%%%
\pkg{minted} requires that the following packages be available and reasonably up to date on your system.  All of these ship with recent \TeX\ distributions.

\begin{multicols}{3}
\begingroup
\setlength\parskip{0pt}
\setlength\topsep{0pt}
\begin{list}{\textrm{\labelitemi}}{\ttfamily}
  \item keyval
  \item kvoptions
  \item fancyvrb
  \item fvextra
  \item upquote
  \item float
  \item ifthen
  \item calc
  \item ifplatform
  \item pdftexcmds
  \item etoolbox
  \item xstring
  \item xcolor
  \item lineno
  \item framed
  \item shellesc (for luatex 0.87+)
  \item catchfile

  ~
\end{list}
\endgroup
\end{multicols}

\switchcolumn
\switchcolumn[0]*%%%%%%%%%
\subsection{Installing \pkg{minted}}
\label{sec:installing:installing}

\switchcolumn
\switchcolumn[0]*%%%%%%%%%
You can probably install \pkg{minted} with your \TeX\ distribution's package manager.  Otherwise, or if you want the absolute latest version, you can install it manually by following the directions below.

\switchcolumn
\switchcolumn[0]*%%%%%%%%%
You may download |minted.sty| from the
\href{https://github.com/gpoore/minted}{project's homepage}.  We have to install the file so that \TeX\ is able to find it.
In order to do that, please refer to the
\href{http://www.tex.ac.uk/cgi-bin/texfaq2html?label=inst-wlcf}{\TeX{} FAQ}.
If you just want to experiment with the latest version, you could locate your current |minted.sty| in your \TeX\ installation and replace it with the latest version.  Or you could just put the latest |minted.sty| in the same directory as the file you wish to use it with.

\end{paracol}