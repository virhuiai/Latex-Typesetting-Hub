% \columnratio{0.55}
\begin{paracol}{2}
\section{Options}
\switchcolumn
\section{选项}
\switchcolumn[0]*%%%%%%%%%%%%
\subsection{Package options}
\switchcolumn
\subsection{宏包选项}
\switchcolumn[0]*%%%%%%%%%%%%
\DescribeMacro{chapter}
To control how \LaTeX{} counts the |listing| floats, you can pass either the
|section| or |chapter| option when loading the \pkg{minted} package.
For example, the following will cause listings to be counted by chapter:
\switchcolumn
\DescribeMacro{chapter}
在加载\pkg{minted}宏包时,可以通过传递|section|或|chapter|选项来控制\LaTeX{}如何计数|listing|浮动体。例如,以下代码将使得列表按章节计数:
\switchcolumn[0]*%%%%%%%%%%%%
\mint[frame=lines]{latex}/\usepackage[chapter]{minted}/
\switchcolumn
\mint[frame=lines]{latex}/\usepackage[chapter]{minted}/
\switchcolumn[0]*%%%%%%%%%%%%
\DescribeMacro{cache=\meta{boolean} (default:~true)}
\pkg{minted} works by saving code to a temporary file, highlighting the code via \app{Pygments} and saving the output to another temporary file, and inputting the output into the \LaTeX\ document.  This process can become quite slow if there are several chunks of code to highlight.  To avoid this, the package provides a |cache| option.  This is on by default.
\switchcolumn
\DescribeMacro{cache=\meta{boolean} (默认值:true)}
\pkg{minted}通过将代码保存到临时文件中,使用\app{Pygments}对代码进行高亮,并将输出保存到另一个临时文件中,然后将输出插入到\LaTeX{}文档中。如果需要高亮显示多个代码块,这个过程可能会变得非常慢。为了避免这种情况,该宏包提供了一个|cache|选项,默认情况下为开启状态。
\switchcolumn[0]*%%%%%%%%%%%%
The |cache| option creates a directory |_minted-|\meta{jobname} in the document's root directory (this may be customized with the |cachedir| option).\footnote{The directory is actually named using a ``sanitized'' copy of \meta{jobname}, in which spaces and asterisks have been replaced by underscores, and double quotation marks have been stripped.  If the file name contains spaces, \texttt{\string\jobname} will contain a quote-wrapped name, except under older versions of MiKTeX which used the name with spaces replaced by asterisks.  Using a ``sanitized'' \meta{jobname} is simpler than accomodating the various escaping conventions.}  Files of highlighted code are stored in this directory, so that the code will not have to be highlighted again in the future.  In most cases, caching will significantly speed up document compilation.
\switchcolumn
|cache|选项会在文档的根目录下创建一个名为|_minted-|\meta{jobname}的目录(可以使用|cachedir|选项自定义目录名)。\footnote{实际上,该目录的命名是使用了“清理”过的\meta{jobname}的副本,其中空格和星号被替换为下划线,双引号被删除。如果文件名包含空格,\texttt{\string\jobname}将包含带引号的名称,除了在旧版本的MiKTeX中,该名称将使用将空格替换为星号的名称。使用“清理”过的\meta{jobname}比适应各种转义约定更简单。} 高亮显示的代码文件将存储在该目录中,以便以后不需要再次进行高亮显示。在大多数情况下,缓存会显著加快文档的编译速度。
\switchcolumn[0]*%%%%%%%%%%%%
Cached files that are no longer in use are automatically deleted.\footnote{This depends on the main auxiliary file not being deleted or becoming corrupted.  If that happens, you could simply delete the cache directory and start over.}
\switchcolumn
不再使用的缓存文件会自动删除。\footnote{这取决于主辅助文件未被删除或损坏。如果发生这种情况,您只需删除缓存目录并重新开始。}
\switchcolumn[0]*%%%%%%%%%%%%
\DescribeMacro{cachedir=\meta{directory} (def:~\_minted-\meta{jobname})}
This allows the directory in which cached files are stored to be specified.  Paths should use forward slashes, even under Windows.
\switchcolumn
\DescribeMacro{cachedir=\meta{directory} (默认值:~\_minted-\meta{jobname})}
该选项允许指定存储缓存文件的目录。路径应该使用正斜杠,即使在Windows下也是如此。
\switchcolumn[0]*%%%%%%%%%%%%
Special characters must be escaped.  For example, |cachedir=~/mintedcache| would not work because the tilde |~| would be converted into the \LaTeX\ commands for a non-breaking space, rather than being treated literally.  Instead, use |\string~/mintedcache|, |\detokenize{~/mintedcache}|, or an equivalent solution.
\switchcolumn
特殊字符必须进行转义。例如,|cachedir=~/mintedcache|是无效的,因为波浪号|~|会被转换为非换行空格%的\LaTeX{}命令
,而不是按字面意义对待。相反,使用|\string~/mintedcache|、|\detokenize{~/mintedcache}|或类似的解决方案。
\switchcolumn[0]*%%%%%%%%%%%%
Paths may contain spaces, but only if the entire \meta{directory} is wrapped in curly braces |{}|, and only if the spaces are quoted.  For example,
\begin{Verbatim}
cachedir = {\detokenize{~/"minted cache"/"with spaces"}}
\end{Verbatim}
\switchcolumn
路径可以包含空格,但只有在整个\meta{directory}被放在花括号|{}|中,并且空格被引用时才可以。例如,
\begin{Verbatim}
cachedir = {\detokenize{~/"minted cache"/"with spaces"}}
\end{Verbatim}
\switchcolumn[0]*%%%%%%%%%%%%
Note that the cache directory is relative to the |outputdir|, if an |outputdir| is specified.
\switchcolumn
请注意,如果指定了|outputdir|,则缓存目录是相对于|outputdir|的。
\switchcolumn[0]*%%%%%%%%%%%%
\DescribeMacro{finalizecache=\meta{boolean} (default:~false)}
In some cases, it may be desirable to use \pkg{minted} in an environment in which |-shell-escape| is not allowed.  A document might be submitted to a publisher or preprint server or used with an online service that does not support |-shell-escape|.  This is possible as long as \pkg{minted} content does not need to be modified.
\switchcolumn
\DescribeMacro{finalizecache=\meta{boolean} (默认值:false)}
在某些情况下,可能希望在不允许|-shell-escape|的环境中使用\pkg{minted}。例如,文档可能会被提交给出版商、预印版本服务器或与不支持|-shell-escape|的在线服务一起使用。只要不需要修改\pkg{minted}内容,就可以做到这一点。
\switchcolumn[0]*%%%%%%%%%%%%
Compiling with the |finalizecache| option prepares the cache for use in an environment without |-shell-escape|.\footnote{Ordinarily, cache files are named using an MD5 hash of highlighting settings and highlighted text.  \texttt{finalizecache} renames cache files using a \texttt{listing<number>.pygtex} scheme.  This makes it simpler to match up document content and cache files, and is also necessary for the XeTeX engine since prior to TeX Live 2016 it lacked the built-in MD5 capabilities that pdfTeX and LuaTeX have.}  Once this has been done, the |finalizecache| option may be swapped for the |frozencache| option, which will then use the frozen (static) cache in the future, without needing |-shell-escape|.
\switchcolumn
使用|finalizecache|选项编译缓存以供在不需要|-shell-escape|的环境中使用。\footnote{通常,缓存文件的命名是使用高亮设置和高亮文本的MD5哈希值。使用\texttt{finalizecache}选项,会使用\texttt{listing<number>.pygtex}方案重命名缓存文件。这样可以更容易地匹配文档内容和缓存文件,并且对于没有内置MD5功能的XeTeX引擎(在TeX Live 2016之前),这是必需的。}完成此操作后,可以将|finalizecache|选项替换为|frozencache|选项,以后就可以在不需要|-shell-escape|的情况下使用冻结(静态)缓存。
\switchcolumn[0]*%%%%%%%%%%%%
\DescribeMacro{fontencoding=\meta{encoding} (default:~\meta{doc~encoding})}
Set font encoding used for typesetting code.
\switchcolumn
\DescribeMacro{fontencoding=\meta{encoding} (默认值:\meta{doc~encoding})}
设置用于排版代码的字体编码。
\switchcolumn[0]*%%%%%%%%%%%%
For example, |fontencoding=T1|.
\switchcolumn
例如,|fontencoding=T1|。
\switchcolumn[0]*%%%%%%%%%%%%
\DescribeMacro{frozencache=\meta{boolean} (default:~false)}
Use a frozen (static) cache created with the |finalizecache| option.  When |frozencache| is on, |-shell-escape| is not needed, and Python and Pygments are not required.  In addition, any external files accessed through |\inputminted| are no longer necessary.
\switchcolumn
\DescribeMacro{frozencache=\meta{boolean} (默认值:false)}
使用使用|finalizecache|选项创建的冻结(静态)缓存。当开启|frozencache|时,无需|-shell-escape|,也不需要Python和Pygments。此外,通过|\inputminted|访问的任何外部文件也不再需要。
\switchcolumn[0]*%%%%%%%%%%%%
\textbf{This option must be used with care.  A document \emph{must} be in final form, as far as \pkg{minted} is concerned, \emph{before} \texttt{frozencache} is turned on, and the document \emph{must} have been compiled with \texttt{finalizecache}.   When this option is on, \pkg{minted} content cannot be modified, except by editing the cache files directly.  Changing any \pkg{minted} settings that require Pygments or Python is not possible.  If \pkg{minted} content is incorrectly modified after \texttt{frozencache} is turned on, \pkg{minted} \emph{cannot} detect the modification.}
\switchcolumn
\textbf{请谨慎使用此选项。在启用|frozencache|之前,文档在\pkg{minted}看来必须是最终形式,而且必须使用|finalizecache|编译文档。开启此选项后,除非直接编辑缓存文件,否则无法修改\pkg{minted}内容。不可能更改任何需要Pygments或Python的\pkg{minted}设置。如果在开启|frozencache|后错误地修改了\pkg{minted}内容,\pkg{minted}将无法检测到这些修改。}
\switchcolumn[0]*%%%%%%%%%%%%
If you are using |frozencache|, and want to verify that \pkg{minted} settings or content have not been modified in an invalid fashion, you can test the cache using the following procedure.
\switchcolumn
如果使用|frozencache|,并且希望验证\pkg{minted}的设置或内容是否以无效的方式进行了修改,可以使用以下步骤测试缓存。
\begin{enumerate}
\switchcolumn[0]*%%%%%%%%%%%%
\item Obtain a copy of the cache used with |frozencache|.
\switchcolumn
\item 获取使用|frozencache|的缓存的副本。
\switchcolumn[0]*%%%%%%%%%%%%
\item Compile the document in an environment that supports |-shell-escape|, with |finalizecache=true| and |frozencache=false|.  This essentially regenerates the frozen (static) cache.
\switchcolumn
\item 在支持|-shell-escape|的环境中使用|finalizecache=true|和|frozencache=false|编译文档。这实际上重新生成了冻结(静态)缓存。
\switchcolumn[0]*%%%%%%%%%%%%
\item Compare the original cache with the newly generated cache.  Under Linux and OS X, you could use |diff|; under Windows, you probably want |fc|.  If \pkg{minted} content and settings have not been modified in an invalid fashion, all files will be identical (assuming that compatible versions of Pygments are used for both caches).
\switchcolumn
\item 将原始缓存与新生成的缓存进行比较。在Linux和OS X下,可以使用|diff|命令;在Windows下,可能需要使用|fc|命令。
\end{enumerate}

\switchcolumn[0]*%%%%%%%%%%%%
\DescribeMacro{draft=\meta{boolean} (default:~false)}
This uses \pkg{fancyvrb} alone for all typesetting; \app{Pygments} is not used.  This trades syntax highlighting and some other \app{minted} features for faster compiling.  Performance should be essentially the same as using \pkg{fancyvrb} directly; no external temporary files are used.  Note that if you are not changing much code between compiles, the difference in performance between caching and draft mode may be minimal.  Also note that |draft| settings are typically inherited from the document class.
\switchcolumn
\DescribeMacro{draft=\meta{boolean} (默认值:~false)}
这个选项只使用\pkg{fancyvrb}进行排版,而不使用\app{Pygments}进行语法高亮和其他一些\app{minted}特性,以加快编译速度。性能应该与直接使用\pkg{fancyvrb}相同,不使用外部临时文件。请注意,如果在编译之间没有改变太多的代码,则缓存和草稿模式之间的性能差异可能很小。还要注意,草稿模式的设置通常从文档类继承而来。
\switchcolumn[0]*%%%%%%%%%%%%
Draft mode does not support |autogobble|.  Regular |gobble|, |linenos|, and most other options not related to syntax highlighting will still function in draft mode.
\switchcolumn
在草稿模式下,不支持|autogobble|选项。普通的|gobble|、|linenos|和大多数与语法高亮无关的其他选项仍然可以在草稿模式下使用。
\switchcolumn[0]*%%%%%%%%%%%%
Documents can usually be compiled without shell escape in draft mode.  The \pkg{ifplatform} package may issue a warning about limited functionality due to shell escape being disabled, but this may be ignored in almost all cases.  (Shell escape is only really required if you have an unusual system configuration such that the |\ifwindows| macro must fall back to using shell escape to determine the system.  See the \pkg{ifplatform} documentation for more details:  \url{http://www.ctan.org/pkg/ifplatform}.)
\switchcolumn
在草稿模式下,通常可以不使用shell escape编译文档。但是,\pkg{ifplatform}宏包可能会发出有关功能受限的警告,因为shell escape被禁用了,但在几乎所有情况下都可以忽略这个警告。(只有在您的系统配置非常特殊,以至于|\ifwindows|宏必须回退到使用shell escape来确定系统时,才真正需要shell escape。有关详细信息,请参阅\pkg{ifplatform}的文档:\url{http://www.ctan.org/pkg/ifplatform}。)
\switchcolumn[0]*%%%%%%%%%%%%
If the |cache| option is set, then all existing cache files will be kept while draft mode is on.  This allows caching to be used intermitently with draft mode without requiring that the cache be completely recreated each time.  Automatic cleanup of cached files will resume as soon as draft mode is turned off.  (This assumes that the auxiliary file has not been deleted in the meantime; it contains the cache history and allows automatic cleanup of unused files.)
\switchcolumn
如果设置了|cache|选项,那么在草稿模式下会保留所有现有的缓存文件。这样可以在草稿模式和缓存模式之间交替使用缓存,而无需每次都完全重建缓存。一旦关闭草稿模式,未使用的缓存文件的自动清理将重新开始。(前提是辅助文件在此期间没有被删除;它包含缓存历史记录,并允许自动清理未使用的文件。)
\switchcolumn[0]*%%%%%%%%%%%%
\DescribeMacro{final=\meta{boolean} (default:~true)}
This is the opposite of |draft|; it is equivalent to |draft=false|.  Again, note that |draft| and |final| settings are typically inherited from the document class.
\switchcolumn
\DescribeMacro{final=\meta{boolean} (默认值:~true)}
这是|draft|的相反,相当于|draft=false|。同样,注意|draft|和|final|的设置通常从文档类继承而来。
\switchcolumn[0]*%%%%%%%%%%%%
\DescribeMacro{kpsewhich=\meta{boolean} (default:~false)}
This option uses |kpsewhich| to locate files that are to be highlighted.  Some build tools such as |texi2pdf| function by modifying |TEXINPUTS|; in some cases, users may customize |TEXINPUTS| as well.  The |kpsewhich| option allows \pkg{minted} to work with such configurations.
\switchcolumn
\DescribeMacro{kpsewhich=\meta{boolean} (默认值:~false)}
此选项使用|kpsewhich|来定位需要进行语法高亮的文件。一些构建工具(如|texi2pdf|)通过修改|TEXINPUTS|来进行操作;在某些情况下,用户也可以自定义|TEXINPUTS|。|kpsewhich|选项允许\pkg{minted}与此类配置一起工作。
\switchcolumn[0]*%%%%%%%%%%%%
This option may add a noticeable amount of overhead on some systems, or with some system configurations.
\switchcolumn
此选项可能会在某些系统或某些系统配置下增加明显的开销。
\switchcolumn[0]*%%%%%%%%%%%%
This option does \emph{not} make \pkg{minted} work with the |-output-directory| and |-aux-directory| command-line options for \LaTeX.  For those, see the |outputdir| package option.
\switchcolumn
此选项\emph{不会}使\pkg{minted}与\LaTeX 的|-output-directory|和|-aux-directory|命令行选项兼容。对于这些选项,请参阅|outputdir|包选项。
\end{paracol}

%gpt
\begin{quote}
使用kpsewhich可以快速查找TeX系统中的文件,例如:

\begin{multicols}{2}
\begin{itemize}
\item
    查找一个宏包的路径:kpsewhich \textless package.sty\textgreater{}
\item
    查找一个字体文件的路径:kpsewhich \textless font.ttf\textgreater{}
\item
    查找一个文档类的路径:kpsewhich \textless class.cls\textgreater{}
\item 于编写脚本或Makefile,以自动查找和引用TeX相关文件。
\end{itemize}
\end{multicols}
\end{quote}

\columnratio{0.55}
\begin{paracol}{2}
\switchcolumn[0]*%%%%%%%%%%%%
\DescribeMacro{langlinenos=\meta{boolean} (default:~false)}
\pkg{minted} uses the \pkg{fancyvrb} package behind the scenes for the code typesetting.  \pkg{fancyvrb} provides an option |firstnumber| that allows the starting line number of an environment to be specified.  For convenience, there is an option |firstnumber=last| that allows line numbering to pick up where it left off.  The |langlinenos| option makes |firstnumber| work for each language individually with all |minted| and |\mint| usages.  For example, consider the code and output below.
\switchcolumn
\DescribeMacro{langlinenos=\meta{boolean} (默认值:~false)}
\pkg{minted}在代码排版时使用\pkg{fancyvrb}宏包。 \pkg{fancyvrb}提供了一个选项|firstnumber|,允许指定环境的起始行号。为了方便起见,还有一个选项|firstnumber=last|,允许行号从上一个环境的位置继续编号。|langlinenos|选项使得|firstnumber|在每种语言的所有|minted|和|\mint|使用中都起作用。例如,考虑以下代码和输出。
\switchcolumn[0]*%%%%%%%%%%%%
\begin{VerbatimOut}[gobble=1]{minted.doc.out}
  \begin{minted}[linenos]{python}
  def f(x):
      return x**2
  \end{minted}

  \begin{minted}[linenos]{ruby}
  def func
      puts "message"
  end
  \end{minted}

  \begin{minted}[linenos, firstnumber=last]{python}
  def g(x):
      return 2*x
  \end{minted}
\end{VerbatimOut}
\inputminted[frame=single, rulecolor=minted@linkcolor]{latex}{minted.doc.out}
\switchcolumn
\hfill
\colorbox{minted@samplebg}{\begin{minipage}{0.4\textwidth}
  \input{minted.doc.out}
\end{minipage}}
\hfill\hfill

\switchcolumn[0]*%%%%%%%%%%%%
Without the |langlinenos| option, the line numbering in the second Python environment would not pick up where the first Python environment left off.  Rather, it would pick up with the Ruby line numbering.
\switchcolumn
如果没有使用|langlinenos|选项,第二个Python环境中的行号不会从第一个Python环境的行号继续,而是从Ruby行号开始编号。
\switchcolumn[0]*%%%%%%%%%%%%
\DescribeMacro{newfloat=\meta{boolean} (default:~false)}
By default, the |listing| environment is created using the \pkg{float} package.  The |newfloat| option creates the environment using \pkg{newfloat} instead.  This provides better integration with the \pkg{caption} package.
\switchcolumn
\DescribeMacro{newfloat=\meta{boolean} (默认值:~false)}
默认情况下,使用\pkg{float}宏包创建|listing|环境。|newfloat|选项改用\pkg{newfloat}创建该环境,以更好地与\pkg{caption}宏包集成。
\switchcolumn[0]*%%%%%%%%%%%%
\DescribeMacro{outputdir=\meta{directory} (default:~\meta{none})}
The |-output-directory| and |-aux-directory| (MiKTeX) command-line options for \LaTeX\ cause problems for \pkg{minted}, because the \pkg{minted} temporary files are saved in |<outputdir>|, but \pkg{minted} still looks for them in the document root directory.  There is no way to access the value of the command-line option so that \pkg{minted} can automatically look in the right place.  But it is possible to allow the output directory to be specified manually as a package option.
\switchcolumn
\DescribeMacro{outputdir=\meta{directory} (默认值:~\meta{none})}
\LaTeX 的|-output-directory|和|-aux-directory|(MiKTeX)命令行选项会给\pkg{minted}带来问题,因为\pkg{minted}临时文件保存在|<outputdir>|中,但\pkg{minted}仍然在文档根目录中查找它们。无法访问命令行选项的值,以使\pkg{minted}能自动在正确的位置查找。但可以允许手动指定输出目录作为包选项。
\switchcolumn[0]*%%%%%%%%%%%%
The output directory should be specified using an absolute path or a path relative to the document root directory.  Paths should use forward slashes, even under Windows.  Special characters must be escaped, while spaces require quoting and need the entire \meta{directory} to be wrapped in curly braces |{}|.  See |cachedir| above for examples of escaping and quoting.
\switchcolumn
输出目录应使用绝对路径或相对于文档根目录的路径来指定。路径应该使用斜杠,即使在Windows下也是如此。特殊字符必须转义,而空格需要用引号引起来,并且需要将整个目录\meta{directory}用花括号 |{}| 括起来。参见上面的 |cachedir| 的示例来了解转义和引用的用法。
\switchcolumn[0]*%%%%%%%%%%%%
\DescribeMacro{section}
To control how \LaTeX{} counts the |listing| floats, you can pass either the
|section| or |chapter| option when loading the \pkg{minted} package.
\switchcolumn
\DescribeMacro{section}
要控制\LaTeX{}对 |listing| 浮动体的计数方式,可以在加载 \pkg{minted} 宏包时传递 |section| 或 |chapter| 选项。
\end{paracol}


\columnratio{0.55}
\begin{paracol}{2}
\subsection{Macro option usage}
\switchcolumn
\subsection{宏选项用法}
\switchcolumn[0]*%%%%%%%%%%%%
All \pkg{minted} highlighting commands accept the same set of options.
Options are specified as a comma-separated list of |key=value| pairs.
For example, we can specify that the lines should be numbered:
\switchcolumn
所有的 \pkg{minted} 高亮命令都接受相同的选项集合。
选项以逗号分隔的 |key=value| 对的形式指定。
例如,我们可以指定行数的显示:
\end{paracol}

\begin{example}
        \begin{minted}[linenos=true]{c++}
        #include <iostream>
        int main() {
            std::cout << "Hello "
                        << "world"
                        << std::endl;
        }
        \end{minted}
\end{example}

\columnratio{0.55}
\begin{paracol}{2}
\switchcolumn[0]*%%%%%%%%%%%%
An option value of |true| may also be omitted entirely (including the ``|=|'').
To customize the display of the line numbers further, override the |\theFancyVerbLine| command.
Consult the \pkg{fancyvrb} documentation for details.
\switchcolumn
也可以完全省略选项值为 |true| 的部分(包括``|=|''符号)。
要进一步自定义行数的显示方式,请覆盖 |\theFancyVerbLine| 命令。
有关详细信息,请参阅 \pkg{fancyvrb} 文档。
\switchcolumn[0]*%%%%%%%%%%%%
|\mint| accepts the same options:
\switchcolumn
|\mint| 命令也接受相同的选项:
\end{paracol}
\begin{example}
    \mint[linenos]{perl}|$x=~/foo/|
\end{example}
\columnratio{0.55}
\begin{paracol}{2}
\switchcolumn[0]*%%%%%%%%%%%%
Here's another example: we want to use the \LaTeX{} math mode inside comments:
\switchcolumn
下面是另一个例子:我们希望在注释中使用 \LaTeX{} 的数学模式:
\end{paracol}
\begin{example}
    \begin{minted}[mathescape]{python}
    # Returns $\sum_{i=1}^{n}i$
    def sum_from_one_to(n):
        r = range(1, n + 1)
        return sum(r)
    \end{minted}
\end{example}
\columnratio{0.55}
\begin{paracol}{2}

\switchcolumn[0]*%%%%%%%%%%%%
To make your \LaTeX{} code more readable you might want to indent the code inside a |minted|
environment.
The option |gobble| removes these unnecessary whitespace characters from the output.  There is also an |autogobble| option that detects the length of this whitespace automatically.
\switchcolumn
为了使你的 \LaTeX{} 代码更具可读性,你可能希望在 |minted| 环境中对代码进行缩进。
选项 |gobble| 可以从输出中删除这些不必要的空白字符。还有一个 |autogobble| 选项,可以自动检测这些空白字符的长度。
\end{paracol}
\begin{example}
    \begin{minted}[gobble=2,showspaces]{python}
        def boring(args = None):
            pass
    \end{minted}
\end{example}

\begin{example}
    \begin{minted}[showspaces]{python}
    def boring(args = None):
        pass
    \end{minted}
\end{example}
%与

\columnratio{0.55}
\begin{paracol}{2}
\switchcolumn[0]*%%%%%%%%%%%%
\DescribeMacro{\setminted}
You may wish to set options for the document as a whole, or for an entire language.  This is possible via \cmd\setminted\oarg{language}\marg{key=value,...}.  Language-specific options override document-wide options.  Individual command and environment options override language-specific options.
\switchcolumn
\DescribeMacro{\setminted}
您可能希望为整个文档或整个语言设置选项。这可以通过\\ \cmd\setminted\oarg{language}\marg{key=value,...} 实现。语言特定的选项会覆盖文档范围的选项。单独的命令和环境选项会覆盖语言特定的选项。
\switchcolumn[0]*%%%%%%%%%%%%
\DescribeMacro{\setmintedinline}
You may wish to set separate options for \cmd\mintinline, either for the document as a whole or for a specific language.  This is possible via \cmd\setmintedinline.  The syntax is\\
\cmd\setmintedinline\oarg{language}\marg{key=value,...}.  Language-specific options override document-wide options. Individual command options override language-specific options.  All settings specified with \cmd\setmintedinline\ override those set with \cmd\setminted.  That is, inline settings always have a higher precedence than general settings.
\switchcolumn
\DescribeMacro{\setmintedinline}
您可能希望为 \cmd\mintinline 设置单独的选项,无论是为整个文档还是为特定的语言。这可以通过 \cmd\setmintedinline 来实现。其语法为\\
\cmd\setmintedinline\oarg{language}\marg{key=value,...}。语言特定的选项会覆盖文档范围的选项。单独的命令选项会覆盖语言特定的选项。所有使用 \cmd\setmintedinline 设置的选项都会覆盖使用 \cmd\setminted 设置的选项。也就是说,内联设置始所有的 \LaTeX{} 代码都会比一般设置具有更高的优先级。
\end{paracol}

\end{document}
\subsection{Available options}

\newcommand\appliesto[1]{\textsf{[For #1 only]}}

Following is a full list of available options.
For more detailed option descriptions please refer to the \pkg{fancyvrb} and \app{Pygments} documentation.

\begin{optionlist}
  \item[autogobble (boolean) (false)]
    Remove (gobble) all common leading whitespace from code.  Essentially a version of |gobble| that automatically determines what should be removed.  Good for code that originally is not indented, but is manually indented after being pasted into a \LaTeX\ document.

\begin{example}
  ...text.
  \begin{minted}[autogobble]{python}
      def f(x):
          return x**2
  \end{minted}
\end{example}

  \item[baselinestretch (dimension) (\meta{document default})]
    Value to use as for baselinestretch inside the listing.


\item[beameroverlays (boolean) (false)]
Give the |<| and |>| characters their normal text meanings when used with |escapeinside| and |texcomments|, so that \pkg{beamer} overlays of the form |\only<1>{...}| will work.


  \item[breakafter (string) (\meta{none})]
    Break lines after specified characters, not just at spaces, when \texttt{breaklines=true}.  Does not apply to |\mintinline|.

For example, \texttt{breakafter=-/} would allow breaks after any hyphens or slashes.  Special characters given to \texttt{breakafter} should be backslash-escaped (usually \texttt{\hashchar}, \texttt{\{}, \texttt{\}}, \texttt{\%}, \texttt{[}, \texttt{]}; the backslash \texttt{\textbackslash} may be obtained via \texttt{\textbackslash\textbackslash}).

For an alternative, see \texttt{breakbefore}.  When \texttt{breakbefore} and \texttt{breakafter} are used for the same character, \texttt{breakbeforegroup} and \texttt{breakaftergroup} must both have the same setting.

\begin{longexample}
  \begin{minted}[breaklines, breakafter=d]{python}
  some_string = 'SomeTextThatGoesOnAndOnForSoLongThatItCouldNeverFitOnOneLine'
  \end{minted}
\end{longexample}

  \item[breakaftergroup] (boolean) (true)
    When \texttt{breakafter} is used, group all adjacent identical characters together, and only allow a break after the last character.  When \texttt{breakbefore} and \texttt{breakafter} are used for the same character, \texttt{breakbeforegroup} and \texttt{breakaftergroup} must both have the same setting.

  \item[breakaftersymbolpre (string) (\string\,\string\footnotesize\string\ensuremath\{\_\string\rfloor\}, \,\footnotesize\ensuremath{_\rfloor})]
    The symbol inserted pre-break for breaks inserted by \texttt{breakafter}.

  \item[breakaftersymbolpost (string) (\meta{none})]
    The symbol inserted post-break for breaks inserted by \texttt{breakafter}.

  \item[breakanywhere (boolean) (false)]
    Break lines anywhere, not just at spaces, when \texttt{breaklines=true}.  Does not apply to |\mintinline|.

\begingroup
\fvset{xrightmargin=1em}
\begin{longexample}
  \begin{minted}[breaklines, breakanywhere]{python}
  some_string = 'SomeTextThatGoesOnAndOnForSoLongThatItCouldNeverFitOnOneLine'
  \end{minted}
\end{longexample}
\endgroup

  \item[breakanywheresymbolpre (string) (\string\,\string\footnotesize\string\ensuremath\{\_\string\rfloor\}, \,\footnotesize\ensuremath{_\rfloor})]
    The symbol inserted pre-break for breaks inserted by \texttt{breakanywhere}.

  \item[breakanywheresymbolpost (string) (\meta{none})]
    The symbol inserted post-break for breaks inserted by \texttt{breakanywhere}.

  \item[breakautoindent (boolean) (true)]
    When a line is broken, automatically indent the continuation lines to the indentation level of the first line.  When \texttt{breakautoindent} and \texttt{breakindent} are used together, the indentations add.  This  indentation is combined with \texttt{breaksymbolindentleft} to give the total actual left indentation.  Does not apply to \texttt{\string\mintinline}.

  \item[breakbefore (string) (\meta{none})]
    Break lines before specified characters, not just at spaces, when \texttt{breaklines=true}.  Does not apply to |\mintinline|.

For example, \texttt{breakbefore=A} would allow breaks before capital A's.  Special characters given to \texttt{breakbefore} should be backslash-escaped (usually \texttt{\hashchar}, \texttt{\{}, \texttt{\}}, \texttt{\%}, \texttt{[}, \texttt{]}; the backslash \texttt{\textbackslash} may be obtained via \texttt{\textbackslash\textbackslash}).

 For an alternative, see \texttt{breakafter}.  When \texttt{breakbefore} and \texttt{breakafter} are used for the same character, \texttt{breakbeforegroup} and \texttt{breakaftergroup} must both have the same setting.

\begin{longexample}
  \begin{minted}[breaklines, breakbefore=A]{python}
  some_string = 'SomeTextThatGoesOnAndOnForSoLongThatItCouldNeverFitOnOneLine'
  \end{minted}
\end{longexample}

  \item[breakbeforegroup] (boolean) (true)
    When \texttt{breakbefore} is used, group all adjacent identical characters together, and only allow a break before the first character.  When \texttt{breakbefore} and \texttt{breakafter} are used for the same character, \texttt{breakbeforegroup} and \texttt{breakaftergroup} must both have the same setting.

  \item[breakbeforesymbolpre (string) (\string\,\string\footnotesize\string\ensuremath\{\_\string\rfloor\}, \,\footnotesize\ensuremath{_\rfloor})]
    The symbol inserted pre-break for breaks inserted by \texttt{breakbefore}.

  \item[breakbeforesymbolpost (string) (\meta{none})]
    The symbol inserted post-break for breaks inserted by \texttt{breakbefore}.

  \item[breakbytoken (boolean) (false)]
  Only break lines at locations that are not within tokens; prevent tokens from being split by line breaks.  By default, \texttt{breaklines} causes line breaking at the space nearest the margin.  While this minimizes the number of line breaks that are necessary, it can be inconvenient if a break occurs in the middle of a string or similar token.

  This is not compatible with \texttt{draft} mode.  A complete list of Pygments tokens is available at \url{http://pygments.org/docs/tokens/}.  If the breaks provided by \texttt{breakbytoken} occur in unexpected locations, it may indicate a bug or shortcoming in the Pygments lexer for the language.

  \item[breakbytokenanywhere (boolean) (false)] 
  Like \texttt{breakbytoken}, but also allows line breaks between immediately adjacent tokens, not just between tokens that are separated by spaces.  Using \texttt{breakbytokenanywhere} with \texttt{breakanywhere} is redundant.

\item[breakindent (dimension) (\meta{breakindentnchars})]
When a line is broken, indent the continuation lines by this amount.  When |breakautoindent| and |breakindent| are used together, the indentations add.  This  indentation is combined with |breaksymbolindentleft| to give the total actual left indentation.

Does not apply to \texttt{\string\mintinline}.

\item[breakindentnchars (integer) (0)]
This allows |breakindent| to be specified as an integer number of characters rather than as a dimension (assumes a fixed-width font).

  \item[breaklines (boolean) (false)] 
    Automatically break long lines in \texttt{minted} environments and \texttt{\string\mint} commands, and wrap longer lines in \texttt{\string\mintinline}.

By default, automatic breaks occur at space characters.  Use \texttt{breakanywhere} to enable breaking anywhere; use \texttt{breakbytoken}, \texttt{breakbytokenanywhere}, \texttt{breakbefore}, and \texttt{breakafter} for more fine-tuned breaking.  Currently, only \texttt{breakbytoken} and \texttt{breakbytokenanywhere} work with |\mintinline|.  Using \texttt{escapeinside} to escape to \LaTeX\ and then insert a manual break is also an option.  For example, use \texttt{escapeinside=||}, and then insert \texttt{|\textbackslash\textbackslash|} at the appropriate point.  (Note that \texttt{escapeinside} does not work within strings.)

\begin{example}
  ...text.
  \begin{minted}[breaklines]{python}
  def f(x):
      return 'Some text ' + str(x)
  \end{minted}
\end{example}

    Breaking in \texttt{minted} and \texttt{\string\mint} may be customized in several ways.  To customize the indentation of broken lines, see \texttt{breakindent} and \texttt{breakautoindent}.  To customize the line continuation symbols, use \texttt{breaksymbolleft} and \texttt{breaksymbolright}.  To customize the separation between the continuation symbols and the code, use \texttt{breaksymbolsepleft} and \texttt{breaksymbolsepright}.  To customize the extra indentation that is supplied to make room for the break symbols, use \texttt{breaksymbolindentleft} and \texttt{breaksymbolindentright}.  Since only the left-hand symbol is used by default, it may also be modified using the alias options \texttt{breaksymbol}, \texttt{breaksymbolsep}, and \texttt{breaksymbolindent}.  Note than none of these options applies to \texttt{\string\mintinline}, since they are not relevant in the inline context.

An example using these options to customize the \texttt{minted} environment is shown below.  This uses the \texttt{\string\carriagereturn} symbol from the \pkg{dingbat} package.

\begingroup
\fvset{breaklines, xleftmargin=2em, xrightmargin=2em}
\begin{longexample}
  \begin{minted}[breaklines,
                 breakautoindent=false,
                 breaksymbolleft=\raisebox{0.8ex}{
                   \small\reflectbox{\carriagereturn}},
                 breaksymbolindentleft=0pt,
                 breaksymbolsepleft=0pt,
                 breaksymbolright=\small\carriagereturn,
                 breaksymbolindentright=0pt,
                 breaksymbolsepright=0pt]{python}
  def f(x):
      return 'Some text ' + str(x) + ' some more text ' + str(x) + ' even more text that goes on for a while'
  \end{minted}
\end{longexample}
\endgroup

    Automatic line breaks are limited with \app{Pygments} styles that use a colored background behind large chunks of text.  This coloring is accomplished with \texttt{\string\colorbox}, which cannot break across lines.  It may be possible to create an alternative to \texttt{\string\colorbox} that supports line breaks, perhaps with \pkg{TikZ}, but the author is unaware of a satisfactory solution.  The only current alternative is to redefine \texttt{\string\colorbox} so that it does nothing.  For example,
\begin{Verbatim}
\AtBeginEnvironment{minted}{\renewcommand{\colorbox}[3][]{#3}}
\end{Verbatim}
    uses the \pkg{etoolbox} package to redefine \texttt{\string\colorbox} within all \texttt{minted} environments.

    Automatic line breaks will not work with \texttt{showspaces=true} unless you use \texttt{breakanywhere} or \texttt{breakafter=\string\space}.

  \item[breaksymbol (string) (breaksymbolleft)]
    Alias for \texttt{breaksymbolleft}.

  \item[breaksymbolleft (string) (\string\tiny\string\ensuremath\{\string\hookrightarrow\}, {\tiny\ensuremath{\hookrightarrow}})]
    The symbol used at the beginning (left) of continuation lines when \texttt{breaklines=true}.  To have no symbol, simply set \texttt{breaksymbolleft} to an empty string (``\texttt{=,}'' or ``\texttt{=\{\}}'').  The symbol is wrapped within curly braces \texttt{\{\}} when used, so there is no danger of formatting commands such as \texttt{\string\tiny} ``escaping.''

    The \texttt{\string\hookrightarrow} and \texttt{\string\hookleftarrow} may be further customized by the use of the \texttt{\string\rotatebox} command provided by \pkg{graphicx}.  Additional arrow-type symbols that may be useful are available in the \pkg{dingbat} (\texttt{\string\carriagereturn}) and \pkg{mnsymbol} (hook and curve arrows) packages, among others.

    Does not apply to \texttt{\string\mintinline}.

  \item[breaksymbolright (string) (\meta{none})]
    The symbol used at breaks (right) when \texttt{breaklines=true}. Does not appear at the end of the very last segment of a broken line.


\item[breaksymbolindent (dimension) (\meta{breaksymbolindentleftnchars})]
Alias for |breaksymbolindentleft|.

\item[breaksymbolindentnchars (integer) (\meta{breaksymbolindentleftnchars})]
Alias for |breaksymbolindentleftnchars|.


\item[breaksymbolindentleft (dimension) (\meta{breaksymbolindentleftnchars})]
The extra left indentation that is provided to make room for |breaksymbolleft|.  This indentation is only applied when there is a |breaksymbolleft|.

    Does not apply to \texttt{\string\mintinline}.

\item[breaksymbolindentleftnchars (integer) (4)]
This allows |breaksymbolindentleft| to be specified as an integer number of characters rather than as a dimension (assumes a fixed-width font).

\item[breaksymbolindentright (dimension) (\meta{breaksymbolindentrightnchars})]
The extra right indentation that is provided to make room for |breaksymbolright|.  This indentation is only applied when there is a |breaksymbolright|.

\item[breaksymbolindentrightnchars (integer) (4)]
This allows |breaksymbolindentright| to be specified as an integer number of characters rather than as a dimension (assumes a fixed-width font).


\item[breaksymbolsep (dimension) (\meta{breaksymbolsepleftnchars})]
Alias for |breaksymbolsepleft|.

\item[breaksymbolsepnchars (integer) (\meta{breaksymbolsepleftnchars})]
Alias for |breaksymbolsepleftnchars|.


\item[breaksymbolsepleft (dimension) (\meta{breaksymbolsepleftnchars})]
The separation between the |breaksymbolleft| and the adjacent text. 

\item[breaksymbolsepleftnchars (integer) (2)]
Allows |breaksymbolsepleft| to be specified as an integer number of characters rather than as a dimension (assumes a fixed-width font).


\item[breaksymbolsepright (dimension) (\meta{breaksymbolseprightnchars})]
The \emph{minimum} separation between the |breaksymbolright| and the adjacent text.  This is the separation between |breaksymbolright| and the furthest extent to which adjacent text could reach.  In practice, |\linewidth| will typically not be an exact integer multiple of the character width (assuming a fixed-width font), so the actual separation between the |breaksymbolright| and adjacent text will generally be larger than |breaksymbolsepright|.  This ensures that break symbols have the same spacing from the margins on both left and right.  If the same spacing from text is desired instead, |breaksymbolsepright| may be adjusted.  (See the definition of |\FV@makeLineNumber| in \pkg{fvextra} for implementation details.)

\item[breaksymbolseprightnchars (integer) (2)]
Allows |breaksymbolsepright| to be specified as an integer number of characters rather than as a dimension (assumes a fixed-width font).


  \item[bgcolor (string) (\meta{none})]
    Background color of the listing.  Be aware that this option has several limitations (described below); see ``Framing alternatives'' below for more powerful alternatives.

    The value of this option must \emph{not} be a color command. Instead, it must be a color \emph{name}, given as a string, of a previously-defined color:

\begin{example}
  \definecolor{bg}{rgb}{0.95,0.95,0.95}
  \begin{minted}[bgcolor=bg]{php}
  <?php
    echo "Hello, $x";
  ?>
  \end{minted}
\end{example}

This option puts \texttt{minted} environments and \texttt{\string\mint} commands in a \texttt{snugshade*} environment from the \pkg{framed} package, which supports breaks across pages.  (Prior to \pkg{minted} 2.2, a \texttt{minipage} was used, which prevented page breaks and gave undesirable spacing from surrounding text.)  Be aware that if \texttt{bgcolor} is used with \texttt{breaklines=true}, and a line break occurs just before a page break, then text may extend below the colored background in some instances.  It is best to use a more advanced framing package in those cases; see ``Framing alternatives'' below.

This option puts \texttt{\string\mintinline} inside a \texttt{\string\colorbox}, which \textbf{does not allow line breaks}.  If you want to use \texttt{\string\setminted} to set background colors, and only want background colors on \texttt{minted} and \texttt{\string\mint}, you may use \texttt{\string\setmintedinline\{bgcolor=\{\}\}} to turn off the coloring for inline commands.

~

\textbf{Framing alternatives}

If you want more reliable and advanced options for background colors and framing, you should consider a more advanced framing package such as \pkg{mdframed} or \pkg{tcolorbox}.  It is easy to add framing to \pkg{minted} commands and environments using the \pkg{etoolbox} package, which is automatically loaded by \pkg{minted}.  For example, using \pkg{mdframed}:
\begin{Verbatim}
\BeforeBeginEnvironment{minted}{\begin{mdframed}}
\AfterEndEnvironment{minted}{\end{mdframed}}
\end{Verbatim}
Some framing packages also provide built-in commands for such purposes.  For example, \pkg{mdframed} provides a \texttt{\string\surroundwithmdframed} command, which could be used to add a frame to all \texttt{minted} environments:
\begin{Verbatim}
\surroundwithmdframed{minted}
\end{Verbatim}
\pkg{tcolorbox} even provides a built-in framing environment with \pkg{minted} support.  Simply use \texttt{\string\tcbuselibrary\{minted\}} in the preamble, and then put code within a \texttt{tcblisting} environment:
\begin{Verbatim}
\begin{tcblisting}{<tcb options>,
                  minted language=<language>,
                  minted style=<style>,
                  minted options={<option list>} }
<code>
\end{tcblisting}
\end{Verbatim}
\pkg{tcolorbox} provides other commands and environments for fine-tuning listing appearance and for working with external code files.

  \item[codetagify (list of strings) (\textrm{highlight \texttt{XXX}, \texttt{TODO}, \texttt{BUG}, and \texttt{NOTE}})]
    Highlight special code tags in comments and docstrings.

\item[curlyquotes (boolean) (false)]
By default, the backtick \texttt{\textasciigrave} and typewriter single quotation mark \texttt{\textquotesingle} always appear literally, instead of becoming the left and right curly single quotation marks \texttt{`'}.  This option allows these characters to be replaced by the curly quotation marks when that is desirable.

  \item[encoding (string) (\meta{system-specific})]
    Sets the file encoding that \app{Pygments} expects.  See also |outencoding|.
  \item[escapeinside (string) (\meta{none})]
  Escape to \LaTeX\ between the two characters specified in \texttt{\string(string\string)}.  All code between the two characters will be interpreted as \LaTeX\ and typeset accordingly.  This allows for additional formatting.  The escape characters need not be identical.  Special \LaTeX\ characters must be escaped when they are used as the escape characters (for example, \texttt{escapeinside=\textbackslash\#\textbackslash\%}).  Requires \app{Pygments} 2.0+.

\textbf{Escaping does not work inside strings and comments (for comments, there is \texttt{texcomments}).  As of Pygments 2.0.2, this means that escaping is ``fragile'' with some lexers.}  Due to the way that Pygments implements \texttt{escapeinside}, any ``escaped'' \LaTeX\ code that resembles a string or comment for the current lexer may break \texttt{escapeinside}.  There is a \href{https://bitbucket.org/birkenfeld/pygments-main/issue/1118}{Pygments issue} for this case.  Additional details and a limited workaround for some scenarios are available on the \href{https://github.com/gpoore/minted/issues/70#issuecomment-111729930}{\pkg{minted} GitHub site}.

\begingroup  ^^A Need to prevent active "|" from causing problems
\catcode`\|=11
\begin{example}
  \begin{minted}[escapeinside=||]{py}
  def f(x):
      y = x|\colorbox{green}{**}|2
      return y
  \end{minted}
\end{example}
\endgroup

\textbf{Note that when math is used inside escapes, any active characters beyond those that are normally active in verbatim can cause problems.}  Any package that relies on special active characters in math mode (for example, \pkg{icomma}) will produce errors along the lines of \texttt{TeX capacity exceeded} and \texttt{\string\leavevmode \string\kern \string\z@}.  This may be fixed by modifying \texttt{\string\@noligs}, as described at \url{http://tex.stackexchange.com/questions/223876}.
  \item[firstline (integer) (1)]
    The first line to be shown.
    All lines before that line are ignored and do not appear in the output.
  \item[firstnumber (auto \| last \| integer) (auto = 1)]
    Line number of the first line.
  \item[fontfamily (family name) (tt)]
    The font family to use.
    |tt|, |courier| and |helvetica| are pre-defined.
  \item[fontseries (series name) (auto \textrm{-- the same as the current font})]
    The font series to use.
  \item[fontsize (font size) (auto \textrm{-- the same as the current font})]
    The size of the font to use, as a size command, e.g. |\footnotesize|.
  \item[fontshape (font shape) (auto \textrm{-- the same as the current font})]
    The font shape to use.
  \item[formatcom (command) (\meta{none})]
    A format to execute before printing verbatim text.
  \item[frame (none \| leftline \| topline \| bottomline \| lines \| single) (none)]
    The type of frame to put around the source code listing.
  \item[framerule (dimension) (0.4pt)]
    Width of the frame.
  \item[framesep (dimension) (\cmd\fboxsep)]
    Distance between frame and content.
  \item[funcnamehighlighting (boolean) (true)] \appliesto{PHP}
    If |true|, highlights built-in function names.
  \item[gobble (integer) (0)]
    Remove the first $n$ characters from each input line.

\item[highlightcolor (string) (LightCyan)]
Set the color used for |highlightlines|, using a predefined color name from \pkg{color} or \pkg{xcolor}, or a color defined via |\definecolor|.


\item[highlightlines (string) (\meta{none})]
This highlights a single line or a range of lines based on line numbers.  For example, |highlightlines={1, 3-4}|.  The line numbers refer to the line numbers that would appear if |linenos=true|, etc.  They do not refer to original or actual line numbers before adjustment by |firstnumber|.

The highlighting color can be customized with |highlightcolor|.


  \item[keywordcase] (string) (lower)
    Changes capitalization of keywords.  Takes |lower|, |upper|, or |capitalize|.
  \item[label (string) (\emph{empty})]
    Add a label to the top, the bottom or both of the frames around the code.
    See the \pkg{fancyvrb} documentation for more information and examples.
    \emph{Note:} This does \emph{not} add a |\label| to the current listing.
    To achieve that, use a floating environment (section \ref{sec:float}) instead.
  \item[labelposition (none \| topline \| bottomline \| all) (topline, all,  \textrm{or} none)]
    Position where to print the label (see above;  default: |topline| if one label is defined, |all| if two are defined, |none| else).
    See the \pkg{fancyvrb} documentation for more information.
  \item[lastline (integer) (\meta{last line of input})]
    The last line to be shown.
  \item[linenos (boolean) (false)]
    Enables line numbers.
    In order to customize the display style of line numbers, you need to redefine the |\theFancyVerbLine| macro:

\begin{example}
  \renewcommand{\theFancyVerbLine}{\sffamily
    \textcolor[rgb]{0.5,0.5,1.0}{\scriptsize
    \oldstylenums{\arabic{FancyVerbLine}}}}

  \begin{minted}[linenos,
    firstnumber=11]{python}
  def all(iterable):
      for i in iterable:
          if not i: 
              return False
      return True
  \end{minted}
\end{example}

\item[numberfirstline (boolean) (false)]
Always number the first line, regardless of |stepnumber|.


  \item[numbers (left \| right \| both \| none) (none)]
    Essentially the same as |linenos|, except the side on which the numbers appear may be specified.
  \item[mathescape (boolean) (false)]
    Enable \LaTeX{} math mode inside comments.
    Usage as in package \pkg{listings}.
    See the note under \texttt{escapeinside} regarding math and ligatures.
  \item[numberblanklines (boolean) (true)]
    Enables or disables numbering of blank lines.
  \item[numbersep (dimension) (12pt)]
    Gap between numbers and start of line.
  \item[obeytabs (boolean) (false)]
    Treat tabs as tabs instead of converting them to spaces---that is, expand them to tab stops determined by |tabsize|.  \textcolor{DarkRed}{\textbf{While this will correctly expand tabs within leading indentation, usually it will not correctly expand tabs that are preceded by anything other than spaces or other tabs.  It should be avoided in those case.}}

  \item[outencoding (string) (\meta{system-specific})]
    Sets the file encoding that \app{Pygments} uses for highlighted output.  Overrides any encoding previously set via |encoding|.
  \item[python3 (boolean) (false)] \appliesto{PythonConsoleLexer}
    Specifies whether Python 3 highlighting is applied.
  \item[resetmargins (boolean) (false)]
    Resets the left margin inside other environments.
  \item[rulecolor (color command) (black)]
    The color of the frame.
  \item[samepage (boolean) (false)]
    Forces the whole listing to appear on the same page, even if it doesn't fit.
  \item[showspaces (boolean) (false)]
    Enables visible spaces: \verb*/visible spaces/.
  \item[showtabs (boolean) (false)]
    Enables visible tabs---only works in combination with |obeytabs|.


\item[space (macro) (\string\textvisiblespace, \textvisiblespace)]
Redefine the visible space character.  Note that this is only used if |showspaces=true|.


\item[spacecolor (string) (none)]
Set the color of visible spaces.  By default (|none|), they take the color of their surroundings.


  \item[startinline (boolean) (false)] \appliesto{PHP}
    Specifies that the code starts in PHP mode, i.e., leading |<?php| is omitted.
  \item[style] (string) (\meta{default})
    Sets the stylesheet used by \app{Pygments}.
  \item[stepnumber (integer) (1)]
    Interval at which line numbers appear.


\item[stepnumberfromfirst (boolean) (false)]
By default, when line numbering is used with |stepnumber| $\ne 1$, only line numbers that are a multiple of |stepnumber| are included.  This offsets the line numbering from the first line, so that the first line, and all lines separated from it by a multiple of |stepnumber|, are numbered.


\item[stepnumberoffsetvalues (boolean) (false)]
By default, when line numbering is used with |stepnumber| $\ne 1$, only line numbers that are a multiple of |stepnumber| are included.  Using |firstnumber| to offset the numbering will change which lines are numbered and which line gets which number, but will not change which \emph{numbers} appear.  This option causes |firstnumber| to be ignored in determining which line numbers are a multiple of |stepnumber|.  |firstnumber| is still used in calculating the actual numbers that appear.  As a result, the line numbers that appear will be a multiple of |stepnumber|, plus |firstnumber| minus 1.


  \item[stripall (boolean) (false)]
    Strip all leading and trailing whitespace from the input.
  \item[stripnl (boolean) (false)]
    Strip leading and trailing newlines from the input.


\item[tab (macro) ({\rmfamily\pkg{fancyvrb}'s} \string\FancyVerbTab, \FancyVerbTab)]
Redefine the visible tab character.  Note that this is only used if |showtabs=true|.  |\rightarrowfill|, \hbox to 2em{\rightarrowfill}, may be a nice alternative.


\item[tabcolor (string) (black)]
Set the color of visible tabs.  If |tabcolor=none|, tabs take the color of their surroundings.  This is typically undesirable for tabs that indent multiline comments or strings.


  \item[tabsize (integer) (8)]
    The number of spaces a tab is equivalent to.  If |obeytabs| is \emph{not} active, tabs will be converted into this number of spaces.  If |obeytabs| is active, tab stops will be set this number of space characters apart.
  \item[texcl (boolean) (false)]
    Enables \LaTeX{} code inside comments.
    Usage as in package \pkg{listings}.  See the note under \texttt{escapeinside} regarding math and ligatures.
  \item[texcomments (boolean) (false)]
    Enables \LaTeX{} code inside comments.  The newer name for |texcl|.  See the note under |escapeinside| regarding math and ligatures.

    As of Pygments 2.0.2, \texttt{texcomments} fails with multiline C/C++ preprocessor directives, and may fail in some other circumstances.  This is because preprocessor directives are \href{http://pygments.org/docs/tokens/}{tokenized as \texttt{Comment.Preproc}}, so \texttt{texcomments} causes preprocessor directives to be treated as literal \LaTeX\ code.  \href{https://bitbucket.org/birkenfeld/pygments-main/issue/1086/wrong-processing-of-in-c-c-macros-if-is}{An issue has been opened} at the Pygments site; additional details are also available on the \href{https://github.com/gpoore/minted/issues/66}{\pkg{minted} GitHub site}.
  \item[xleftmargin (dimension) (0)]
    Indentation to add before the listing.
  \item[xrightmargin (dimension) (0)]
    Indentation to add after the listing.

\end{optionlist}



