
% \begin{optionlist}
\columnratio{0.55}
\begin{paracol}{2}
\section{Defining shortcuts}
\switchcolumn
\section{定义快捷方式}
\switchcolumn[0]*%%%%%%%%%%%%
Large documents with a lot of listings will nonetheless use the same source language and the
same set of options for most listings.
Always specifying all options is redundant, a lot to type and makes performing changes hard.
\switchcolumn
大型文档中有许多源码清单,但大多数源码清单使用相同的源语言和相同的选项。始终指定所有选项是多余的,输入的内容很多且更改困难。
\switchcolumn[0]*%%%%%%%%%%%%
One option is to use \cmd\setminted, but even then you must still specify the language each time.
\switchcolumn
一个选项是使用\cmd\setminted,但是您仍然需要每次都指定语言。
\switchcolumn[0]*%%%%%%%%%%%%
\pkg{minted} therefore defines a set of commands that lets you define shortcuts for the highlighting commands.
Each shortcut is specific for one programming language.
\switchcolumn
\pkg{minted}因此定义了一组命令,可让您为高亮命令定义快捷方式。每个快捷方式特定于一种编程语言。
\switchcolumn[0]*%%%%%%%%%%%%
\DescribeMacro{\newminted}
|\newminted| defines a new alias for the |minted| environment:
\switchcolumn
\DescribeMacro{\newminted}
|\newminted|定义了一个新的别名,用于|minted|环境:
\end{paracol}


\begin{example}
    \newminted{cpp}{gobble=2,linenos}

    \begin{cppcode}
        template <typename T>
        T id(T value) {
            return value;
        }
    \end{cppcode}
\end{example}

\columnratio{0.55}
\begin{paracol}{2}
If you want to provide extra options on the fly, or override existing default options, you can do that, too:
\switchcolumn
如果您要在使用中即时提供额外的选项,或覆盖现有的默认选项,也可以这样做:
\end{paracol}


\begin{example}
    \newminted{cpp}{gobble=2,linenos}

    \begin{cppcode*}{linenos=false,
                    frame=single}
        int const answer = 42;
    \end{cppcode*}
\end{example}

\columnratio{0.55}
\begin{paracol}{2}
Notice the star ``|*|'' behind the environment name---due to restrictions in \pkg{fancyvrb}'s handling
of options, it is necessary to provide a \emph{separate} environment that accepts options, and the options
are \emph{not} optional on the starred version of the environment.
\switchcolumn
注意环境名后的星号 ``|*|'' —— 由于 \pkg{fancyvrb} 对选项处理的限制,有必要提供一个接受选项的\emph{单独}环境,而且选项在环境的星号版本上\emph{不是}可选的。
\switchcolumn[0]*%%%%%%%%%%%%
The default name of the environment is \meta{language}|code|.
If this name clashes with another environment or if you want to choose an own name for another reason, you may
do so by specifying it as the first argument: \cmd\newminted\oarg{environment name}\marg{language}\marg{options}.
\switchcolumn
环境的默认名称是\meta{language}|code|。
如果此名称与其他环境冲突,或者出于其他原因希望选择自己的名称,可以将其作为第一个参数指定:\cmd\newminted\oarg{环境名称}\marg{语言}\marg{选项}。
\switchcolumn[0]*%%%%%%%%%%%%
Like normal \pkg{minted} environments, environments created with |\newminted| may be used within other environment definitions.  Since the \pkg{minted} environments use \pkg{fancyvrb} internally, any environment based on them must include the \pkg{fancyvrb} command |\VerbatimEnvironment|.  This allows \pkg{fancyvrb} to determine the name of the environment that is being defined, and correctly find its end.  It is best to include this command at the beginning of the definition.  For example,
\begin{Verbatim}
\newminted{cpp}{gobble=2,linenos}
\newenvironment{env}{\VerbatimEnvironment\begin{cppcode}}{\end{cppcode}}
\end{Verbatim}
\switchcolumn
与普通的\pkg{minted}环境一样,使用|\newminted|创建的环境可以在其他环境定义中使用。由于\pkg{minted}使用\pkg{fancyvrb}内部实现的环境,因此基于它们的任何环境都必须包含\pkg{fancyvrb}命令|\VerbatimEnvironment|。这允许\pkg{fancyvrb}确定正在定义的环境的名称,并正确找到其结束。最好在定义的开始处包含此命令。例如,
\begin{Verbatim}
\newminted{cpp}{gobble=2,linenos}
\newenvironment{env}%
{\VerbatimEnvironment\begin{cppcode}}{\end{cppcode}}
\end{Verbatim}
\end{paracol}

\columnratio{0.55}
\begin{paracol}{2}
\switchcolumn[0]*%%%%%%%%%%%%
\DescribeMacro{\newmint}
The above macro only defines shortcuts for the |minted| environment.
The main reason is that the short command form |\mint| often needs different options---at the very least, it
will generally not use the |gobble| option.
A shortcut for |\mint| is defined using \cmd\newmint\oarg{macro name}\marg{language}\marg{options}.
The arguments and usage are identical to |\newminted|.
If no \meta{macro name} is specified, \meta{language} is used.
\switchcolumn
\DescribeMacro{\newmint}
上述宏仅为|minted|环境定义了快捷方式。
主要原因是简短的命令形式|\mint|通常需要不同的选项-至少通常不会使用|gobble|选项。
使用\cmd\newmint\oarg{宏名称}\marg{语言}\marg{选项}定义|\mint|的快捷方式。
参数和用法与|\newminted|相同。
如果未指定\meta{宏名称},则使用\meta{语言}。
\end{paracol}



\begin{example}
    \newmint{perl}{showspaces}

    \perl/my $foo = $bar;/
\end{example}

\columnratio{0.55}
\begin{paracol}{2}
\DescribeMacro{\newmintinline}
This creates custom versions of \cmd\mintinline.  The syntax is the same as that for \cmd\newmint: \cmd\newmintinline\oarg{macro~name}\marg{language}\marg{options}.  If a \meta{macro~name} is not specified, then the created macro is called |\|\meta{language}|inline|.
\switchcolumn
\DescribeMacro{\newmintinline}
这将创建 {\cmd\mintinline} 的自定义版本。语法与\cmd\newmint 相同:\cmd\newmintinline\oarg{宏名称}\marg{语言}\marg{选项}。
如果未指定\meta{宏名称},则创建的宏将被称为|\|\meta{语言}|inline|。

\end{paracol}

\begin{example}
    \newmintinline{perl}{showspaces}

    X\perlinline/my $foo = $bar;/X
\end{example}


\columnratio{0.55}
\begin{paracol}{2}
\DescribeMacro{\newmintedfile}
This creates custom versions of \cmd\inputminted.  The syntax is
\begin{center}
\cmd\newmintedfile\oarg{macro~name}\marg{language}\marg{options}
\end{center}
If no \meta{macro name} is given, then the macro is called |\|\meta{language}|file|.
\switchcolumn
\DescribeMacro{\newmintedfile}
这将创建 \cmd\inputminted 的自定义版本。语法为
\begin{center}
\cmd\newmintedfile\oarg{宏名称}\marg{语言}\marg{选项}
\end{center}
如果未指定\meta{宏名称},则将使用|\|\meta{语言}|file|作为宏名称。




\end{paracol}
% \end{optionlist}
