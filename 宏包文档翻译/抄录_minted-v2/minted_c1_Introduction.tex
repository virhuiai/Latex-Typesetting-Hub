\columnratio{0.55}
\begin{paracol}{2}
\section{Introduction}
\switchcolumn
\section{介绍}
\switchcolumn[0]*%%%%%%%%%%%%%%
\pkg{minted} is a package that allows formatting source code in \LaTeX.
For example:
\switchcolumn
\pkg{minted} 是一个允许在 \LaTeX 中格式化源代码的包。例如:
\switchcolumn[0]*%%%%%%%%%%%%%%
%% 以下输出到文件
\begin{VerbatimOut}[gobble=1]{minted.doc.out}
  \begin{minted}{<language>}
    <code>
  \end{minted}
\end{VerbatimOut}
%% 以下输入从文件
\inputminted[gobble=2,frame=lines]{latex}{minted.doc.out}
\switchcolumn
\inputminted[gobble=2,frame=lines]{latex}{minted.doc.out}
\switchcolumn[0]*%%%%%%%%%%%%%%
will highlight a piece of code in a chosen language.
The appearance can be customized with a number of options and color schemes.
\switchcolumn
\switchcolumn[0]*%%%%%%%%%%%%%%
Unlike some other packages, most notably \pkg{listings}, \pkg{minted} requires
the installation of additional software, \app{Pygments}.
This may seem like a disadvantage, but there are also significant advantages.
\switchcolumn
\switchcolumn[0]*%%%%%%%%%%%%%%
\app{Pygments} provides superior syntax highlighting compared to conventional packages.
For example, \pkg{listings} basically only highlights strings, comments and keywords.
\app{Pygments}, on the other hand, can be completely customized to highlight any kind of token the
source language might support.
This might include special formatting sequences inside strings, numbers, different kinds of
identifiers and exotic constructs such as HTML tags.

Some languages make this especially desirable.
Consider the following Ruby code as an extreme, but at the same time typical, example:

\begin{minted}[gobble=3]{ruby}
  class Foo
    def init
      pi = Math::PI
      @var = "Pi is approx. #{pi}"
    end
  end
\end{minted}

Here we have four different colors for identifiers (five, if you count keywords) and escapes from
inside strings, none of which pose a problem for \app{Pygments}.

Additionally, installing \app{Pygments} is actually incredibly easy (see the next section).

\end{paracol}