 
\title{The \pkgname{eso-pic} package\thanks{This document
  corresponds to \pkgname{eso-pic}~\fileversion, dated \filedate.}}
\author{Rolf Niepraschk \\ \texttt{Rolf.Niepraschk@gmx.de}\and 翻译:virhuiai\\\texttt{virhuiai@qq.com}}
 
\date{}

\maketitle

\columnratio{0.55}
\begin{paracol}{2}
\section{Introduction}
\switchcolumn
\section{介绍}
\switchcolumn[0]*%%%%%%%%%%%%
 This package makes it easy to add some picture commands to every page
 at absolute positions.
\switchcolumn
该包使得在每个页面的绝对位置添加一些图片命令变得容易。
\switchcolumn[0]*%%%%%%%%%%%%
\section{Usage}
\switchcolumn
\section{使用方法}
\switchcolumn[0]*%%%%%%%%%%%%
See also the example \LaTeX\ documents (|eso-*.tex|).
\switchcolumn
另请参阅示例的 \LaTeX\ 文档(|eso-*.tex|)。%% todo /usr/local/texlive/2022/texmf-dist/doc/latex/eso-pic
\switchcolumn[0]*%%%%%%%%%%%%  
 \subsection{Basic commands for adding \LaTeX{} stuff to the page background}
 \DescribeMacro{\AddToShipoutPictureBG} All the picture commands which are
 parameters of an \cs{AddToShipoutPictureBG} command will be added to the
 internal macro \cs{ESO@HookIBG}. This macro is part of a
 zero-length \texttt{picture} environment with basepoint at the lower left
 corner of the page (default) or at the upper left corner
 (package option "texcoord"). The \texttt{picture} environment will be
 shipped out at every new page.
 \switchcolumn
 \subsection{添加 \LaTeX{} 内容到页面背景的基本命令}
 \DescribeMacro{\AddToShipoutPictureBG} 所有作为 \cs{AddToShipoutPictureBG} 命令参数的图片命令将被添加到内部宏 \cs{ESO@HookIBG} 中。该宏是一个以页面左下角(默认)或页面左上角(选项 ``texcoord'')为基点的零长度的 \texttt{picture} 环境的一部分。该 \texttt{picture} 环境将在每个新页面上进行输出。
 \switchcolumn[0]*%%%%%%%%%%%% 
 \DescribeMacro{\AddToShipoutPictureBG*} \cs{AddToShipoutPictureBG*} works like
 \cs{AddToShipoutPictureBG} but only for the current page. It used the internal
 macro \cs{ESO@HookIIBG} which contents will be removed
 automatically.
 \switchcolumn
 \DescribeMacro{\AddToShipoutPictureBG*} \cs{AddToShipoutPictureBG*} 的工作方式类似于 \cs{AddToShipoutPictureBG},但仅作用于当前页面。它使用内部宏 \cs{ESO@HookIIBG},其内容将自动被清除。
 \switchcolumn[0]*%%%%%%%%%%%% 
 For compatibility the macros \cs{AddToShipoutPicture} and
 \cs{AddToShipoutPicture*} are aliases for \cs{AddToShipoutPictureBG}
 and \cs{AddToShipoutPictureBG*}.
 \switchcolumn
 为了兼容,宏 \cs{AddToShipoutPicture} 和 \cs{AddToShipoutPicture*} 是 \cs{AddToShipoutPictureBG} 和 \cs{AddToShipoutPictureBG*} 的别名。
 \switchcolumn[0]*%%%%%%%%%%%%
 \DescribeMacro{\AddToShipoutPictureFG}
 \DescribeMacro{\AddToShipoutPictureFG*} Works like
 \cs{AddToShipoutPictureBG} but the picture commands are on the top
 oft the normal document content.
 \switchcolumn
 \DescribeMacro{\AddToShipoutPictureFG}
 \DescribeMacro{\AddToShipoutPictureFG*} 的工作方式类似于 \cs{AddToShipoutPictureBG},但图片命令位于正常文档内容的顶部。 
 \switchcolumn[0]*%%%%%%%%%%%%
 \DescribeMacro{\ClearShipoutPictureBG} A call of
 \cs{ClearShipoutPictureBG}
 clears the contents of \cs{ESO@HookIBG}.
 \switchcolumn
 \DescribeMacro{\ClearShipoutPictureBG} 调用 \cs{ClearShipoutPictureBG} 会清除 \cs{ESO@HookIBG} 的内容。
 \switchcolumn[0]*%%%%%%%%%%%%
 For compatibility the macro \cs{ClearShipoutPicture}
 is an alias for \cs{ClearShipoutPictureBG}.
 \switchcolumn
 为了兼容,宏 \cs{ClearShipoutPicture} 是 \cs{ClearShipoutPictureBG} 的别名。
 \switchcolumn[0]*%%%%%%%%%%%%
 \DescribeMacro{\ClearShipoutPictureFG} A call of
 \cs{ClearShipoutPictureFG}
 clears the contents of \cs{ESO@HookIFG}.
 \switchcolumn
 \DescribeMacro{\ClearShipoutPictureFG} 调用 \cs{ClearShipoutPictureFG} 会清除 \cs{ESO@HookIFG} 的内容。
 \switchcolumn[0]*%%%%%%%%%%%%
  \DescribeMacro{\LenToUnit} [Allows a length as parameter to a picture
   command.] Note that this macro exist only for compatibility to older 
   versions of this package. A recent \LaTeX\ version allows dimensions 
   in picture commands.
   \switchcolumn
\DescribeMacro{\LenToUnit} [允许将长度作为图片命令的参数。] 请注意,该宏仅用于与该包的旧版本兼容。较新版本的 \LaTeX\ 允许在图片命令中使用尺寸。
   \switchcolumn[0]*%%%%%%%%%%%%
\DescribeMacro{\gridSetup}
\cmd{\gridSetup}\oarg{gridunitname}\marg{gridunit}\marg{labelfactor}^^A
\marg{griddelta} \marg{gridDelta}\marg{gap}. For details see the
implementation section.

\switchcolumn
\DescribeMacro{\gridSetup}
\cmd{\gridSetup}\oarg{gridunitname}\marg{gridunit}\marg{labelfactor}^^A
\marg{griddelta} \marg{gridDelta}\marg{gap}。详细信息请参见实现部分。
\end{paracol}
\par\clearpage

\columnratio{0.55}
\begin{paracol}{2}
\switchcolumn[0]*%%%%%%%%%%%%
 \DescribeMacro{\AtPageUpperLeft}
 \DescribeMacro{\AtPageLowerLeft}
 \DescribeMacro{\AtPageCenter}
 \DescribeMacro{\AtTextUpperLeft}
 \DescribeMacro{\AtTextLowerLeft}
 \DescribeMacro{\AtTextCenter}
\DescribeMacro{\AtStockUpperLeft}
\DescribeMacro{\AtStockLowerLeft}
\DescribeMacro{\AtStockCenter}
Helper macros for easier positioning on the page.
\switchcolumn
\DescribeMacro{\AtPageUpperLeft}
\DescribeMacro{\AtPageLowerLeft}
\DescribeMacro{\AtPageCenter}
\DescribeMacro{\AtTextUpperLeft}
\DescribeMacro{\AtTextLowerLeft}
\DescribeMacro{\AtTextCenter}
\DescribeMacro{\AtStockUpperLeft}
\DescribeMacro{\AtStockLowerLeft}
\DescribeMacro{\AtStockCenter}
辅助宏,用于更容易地定位在页面上。
   \switchcolumn[0]*%%%%%%%%%%%%
 \subsection{Package options}
 \begin{center}
 \begin{tabular}{@{}>{\ttfamily}llp{.4\textwidth}p{.4\textwidth}@{}}
   \textnormal{Option} & Value & Comments \\ \hline
   pscoord & empty or \textit{true}, \textit{false} & The zero point of
     the coordinate system is the lower left corner of the page
     (the default). & 坐标系的零点是页面左下角(默认)。\\
   texcoord & empty or \textit{true}, \textit{false} & The zero point of
     the coordinate system is the upper left corner of the page. & 坐标系的零点是页面左上角。\\
   grid & empty or \textit{true}, \textit{false} & A grid is drawing;
     default: false. & 绘制网格;默认值为 false。\\
   gridBG & empty or \textit{true}, \textit{false} & Put the grid in the
     background; default: false.& 将网格放在背景中;默认值为 false。\\
   gridunit & \textit{mm}, \textit{in}, \textit{bp}, \textit{pt} & Distance
     between gridlines are multiples of \texttt{gridunit}. default: mm.& 网格线之间的距离是 \texttt{gridunit} 的倍数;默认值为 mm。\\
   gridcolor & a valid color name & Color of the main gridlines;
                                    default: black  & 主网格线的颜色;默认值为黑色。\\
   subgridcolor & a valid color name & Color of the subgridlines;
                                    default: black.  & 次网格线的颜色;默认值为黑色。\\
   subgridstyle & \textit{solid} or \textit{dotted} & `dotted' is very slow!
     default: solid. & `dotted' 速度较慢!默认值为 solid。\\
   colorgrid & empty or \textit{true}, \textit{false} & Shortcut for
         `grid=true', `gridcolor=red' and `subgridcolor=green';
         default: false.& 快捷方式,相当于 grid=true'、gridcolor=red' 和 subgridcolor=绿色';默认值为 false。\\
   dvips        & empty or \textit{true}, \textit{false} & Is also true
     if the package option \texttt{dvips} is
     used by \pkgname{color} or \pkgname{graphics}. If true package
     \pkgname{eepic} is loaded for better performance of dotted lines.& 如果 \pkgname{color} 或 \pkgname{graphics} 使用了 \texttt{dvips} 选项,则也为 true。如果为 true,则加载 \pkgname{eepic} 以提高虚线的性能。

 \end{tabular}
 \end{center}

\StopEventually{\PrintChanges\PrintIndex}
\end{paracol}