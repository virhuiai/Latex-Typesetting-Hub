%%^^A%%  fontspec-doc-opentype.tex -- part of FONTSPEC <wspr.io/fontspec>


\documentclass[a4paper]{l3doc}
\usepackage{fontspec-doc-style}
\showexamplesfalse
\begin{document}

\part{OpenType}
\label{sec:opentype-features}

\section{Introduction\\介绍}
\label{sec:opentype-features-intro}

OpenType fonts (and other `smart' font technologies such as AAT and Graphite) can change the appearance of text in many different ways.
These changes are referred to as font features.
When the user applies a feature~--- for example, small capitals~--- to a run of text, the code inside the font makes appropriate substitutions and small capitals appear in place of lowercase letters.
However, the use of such features does not affect the underlying text.
In our small caps example, the lowercase letters are still stored in the document; only the appearance has been changed by the OpenType feature.
This makes it possible to search and copy text without difficulty.
If the user selected a different font that does not support small caps, the `plain' lowercase letters would appear instead.

OpenType 字体(和其他“智能”字体技术,如 AAT 和 Graphite)可以以许多不同的方式改变文本的外观。这些更改称为字体特征。当用户将特征(例如小型大写字母)应用于一段文本时,字体内部的代码会进行适当的替换,并在小型大写字母取代小写字母。但是,使用这些特征不会影响底层文本。在我们的小型大写字母示例中,小写字母仍存储在文档中;只有通过 OpenType 特性更改了外观。这使得可以轻松搜索和复制文本。如果用户选择不支持小型大写字母的不同字体,则“普通”的小写字母将出现。


Some OpenType features are required to support particular scripts, and these features are often applied automatically.
The Indic scripts, for example, often require that characters be reshaped and reordered after they are typed by the user, in order to display them in the traditional ways that readers expect.
Other features can be applied to support a particular language.
The Junicode font for medievalists uses by default the Old English shape of the letter thorn, while in modern Icelandic thorn has a more rounded shape.
If a user tags some text as being in Icelandic, Junicode will automatically change to the Icelandic shape through an OpenType feature that localises the shapes of letters.

一些 OpenType 特性是必需的,以支持特定的脚本,并且这些特性通常会自动应用。例如,印度脚本经常要求用户在输入字符后进行重塑和重新排序,以便以读者期望的传统方式显示它们。其他特性可用于支持特定语言。中世纪学者使用的 Junicode 字体默认使用字母 thorn 的古英语形状,而现代冰岛语的 thorn 具有更圆润的形状。如果用户将一些文本标记为冰岛语,则 Junicode 将通过本地化字母形状的 OpenType 特性自动更改为冰岛语形状。

There are a large group of OpenType features, designed to support high quality typography a multitude of languages and writing scripts.
Examples of some font features have already been shown in previous sections; the complete set of OpenType font features supported by \pkg{fontspec} is described below in \ref{sec:ot-feat}.

有一个大的 OpenType 特性组,旨在支持众多语言和书写脚本的高质量排版。一些字体特性的示例已在以前的章节中显示; \pkg{fontspec} 支持的完整 OpenType 字体特性集在 \ref{sec:ot-feat} 中进行了描述。

The OpenType specification provides four-letter codes (e.g., \texttt{smcp} for small capitals) for each feature.  The four-letter codes are given below along with the \pkg{fontspec} names for various features, for the benefit of people who are already familiar with OpenType.  You can ignore the codes if they don't mean anything to you.

OpenType 规范为每个特性提供四个字母的代码(例如 \texttt{smcp} 表示小型大写字母)。这些四字代码以及各种特性的 \pkg{fontspec} 名称如下所示,以方便熟悉 OpenType 的人使用。如果这些代码对您没有意义,则可以忽略它们。

\subsection{How to select font features\\如何选择字体特性}

Font features are selected by a series of \meta{feature}=\meta{option}
selections. Features are (usually) grouped logically; for example, all
font features relating to ligatures are accessed by writing \verb|Ligatures={...}| with the appropriate argument(s), which could be \texttt{TeX}, \texttt{Rare}, etc., as shown below in \ref{sec:ot-feat-liga}.

字体特征是通过一系列\meta{feature}=\meta{option}选择来选择的。这些特征通常按逻辑分组;例如,所有与连字有关的字体特征都可以通过编写\verb|Ligatures={...}|来访问相应的参数(如\texttt{TeX}、\texttt{Rare}等),如\ref{sec:ot-feat-liga}中所示。

Multiple options may be given to
any feature that accepts non-numerical input, although doing so will
not always work. Some options will override others in generally
obvious ways; \Verb|Numbers={OldStyle,Lining}| doesn't make much
sense because the two options are mutually exclusive, and \XeTeX\
will simply use the last option that is specified (in this case
using \opt{Lining} over \opt{OldStyle}).

可以给任何接受非数值输入的特征提供多个选项,尽管这样做并不总是有效的。某些选项会以通常很明显的方式覆盖其他选项;\Verb|Numbers={OldStyle,Lining}|并不合理,因为这两个选项是互斥的,\XeTeX\ 将仅使用指定的最后一个选项(在本例中使用\opt{Lining}而非\opt{OldStyle})。

If a feature or an option is requested that the font does not have,
a warning is given in the console output. As mentioned in \vref{sec:quiet-warnings}
these warnings can be suppressed by selecting the \texttt{[quiet]} package option.

如果请求字体没有的特征或选项,则会在控制台输出中给出警告。如\vref{sec:quiet-warnings}所述,可以通过选择\texttt{[quiet]}包选项来抑制这些警告。

\subsection{How do I know what font features are supported by my fonts?\\我如何知道我的字体支持哪些字体特征?}

Although I've long desired to have a feature within \pkg{fontspec} to display the OpenType features within a font, it's never been high on my priority list.
One reason for that is the existence of the document |opentype-info.tex|, which is available on \textsc{ctan} or typing |kpsewhich opentype-info.tex| in a Terminal window.
Make a copy of this file and place it somewhere convenient.
Then open it in your regular \TeX\ editor and change the font name to the font you'd like to query; after running through plain \XeTeX, the output \textsc{pdf} will look something like this:

虽然我长期以来一直希望在\pkg{fontspec}中有一个显示字体的OpenType特征的功能,但这从未成为我的优先事项之一。其中一个原因是存在名为|opentype-info.tex|的文档,可以在\textsc{ctan}上获取或在终端窗口中键入|kpsewhich opentype-info.tex|。复制该文件并将其放在方便的位置。然后在常规的\TeX\ 编辑器中打开它,并将字体名称更改为您要查询的字体;经过普通的\XeTeX\ 运行后,输出的\textsc{pdf}将类似于以下内容:

\begin{framed}

%%% From OpenType-info.tex %%%
\DeleteShortVerb\"
\def\myfontname{[Asana-Math.otf]}
\font\testfont="\myfontname/OT" at 10pt\relax

\rightskip=0pt plus 1fil

\font\titlefont=ec-lmssbx10 at 12pt
\font\tenrm=ec-lmss10 at 9pt \tenrm
\font\tentt=ec-lmtt10 at 9pt

\def\fourcharcode#1{\begingroup
 \count0=#1\count1=\count0
 \ifnum\count0=0
  $\langle$default$\rangle$%
 \else
  \tentt
  '%
  \divide\count0 by "1000000
  \char\count0
  \multiply\count0 by "1000000
  \advance\count1 by -\count0
  \count0=\count1
  \divide\count0 by "10000
  \char\count0
  \multiply\count0 by "10000
  \advance\count1 by -\count0
  \count0=\count1
  \divide\count0 by "100
  \char\count0
  \multiply\count0 by "100
  \advance\count1 by -\count0
  \ifnum\count1=32 \ \else \char\count1 \fi
  '%
 \fi
 \endgroup
}

\newcount\scriptcount
\newcount\scriptindex
\newcount\scripttag
\newcount\langcount
\newcount\langindex
\newcount\langtag
\newcount\featurecount
\newcount\featureindex
\newcount\featuretag

\leftline{\titlefont OpenType Layout features found in `\myfontname'}
\nobreak

\scriptcount=\XeTeXOTcountscripts\testfont
\ifnum\scriptcount=0 \noindent None\par\fi

\loop
 \ifnum\scriptindex<\scriptcount
  \smallskip
  \scripttag=\XeTeXOTscripttag\testfont\scriptindex
  \noindent script = \fourcharcode{\scripttag}\endgraf\nobreak
  \langcount=\XeTeXOTcountlanguages\testfont\scripttag
  \advance\langcount by 1 % one extra to get the default language system
  {\loop
    \langtag=\XeTeXOTlanguagetag\testfont\scripttag\langindex
    \indent language = \fourcharcode{\langtag}\endgraf\nobreak
    \featurecount=\XeTeXOTcountfeatures\testfont\scripttag\langtag
    {\indent\indent \hangindent=3\parindent \hangafter=1 features = \loop
      \featuretag=\XeTeXOTfeaturetag\testfont\scripttag\langtag\featureindex
      \fourcharcode{\featuretag}
      \advance\featureindex by 1
      \ifnum\featureindex<\featurecount \repeat\endgraf}
    \advance\langindex by 1
  \ifnum\langindex<\langcount \medskip \repeat}
  \advance\scriptindex by 1
\repeat
\end{framed}

\noindent I intentionally picked a font above that by design contains few font features; `regular' text fonts such as Latin Modern Roman contain many more, and I didn't want to clutter up the document too much.
After finding the scripts, languages, and features contained within the font, you'll then need to cross-check the OpenType tags with the `logical' names used by \pkg{fontspec}.

\noindent 我故意选择了一个设计上包含很少字体特征的字体;例如,拉丁现代罗马体包含更多字体特征,而我不想使文档太混乱。
在找到脚本、语言和字体中包含的特性之后,您需要使用\pkg{fontspec}使用的“逻辑”名称交叉检查OpenType标记。

\paragraph{otfinfo}
Alternatively, and more simply, you can use the command line tool |otfinfo|, which is distributed with \TeX{}Live.
Simply type in a Terminal window, say:

另外,更简单的方法是使用命令行工具|otfinfo|,它与\TeX{}Live一起分发。只需在终端窗口中输入:

\begin{Verbatim}
  otfinfo -f `kpsewhich lmromandunh10-oblique.otf`
\end{Verbatim}
which results in:
\begin{Verbatim}[frame=single]
aalt	Access All Alternates
cpsp	Capital Spacing
dlig	Discretionary Ligatures
frac	Fractions
kern	Kerning
liga	Standard Ligatures
lnum	Lining Figures
onum	Oldstyle Figures
pnum	Proportional Figures
size	Optical Size
tnum	Tabular Figures
zero	Slashed Zero
\end{Verbatim}


\section{OpenType scripts and languages\\OpenType 脚本和语言}\label{sec:ot}

Fonts that include glyphs for various scripts and languages may contain different font features for the different character sets and languages they support, and different font features may behave differently depending on the script or language chosen.
When multilingual fonts are used, it is important to select which language
they are being used for, and more importantly what script is being used.

包含各种脚本和语言的字形的字体可能针对支持的不同字符集和语言包含不同的字体特征,并且不同的字体特征可能根据所选择的脚本或语言而表现不同。当使用多语言字体时,重要的是要选择使用它们的语言,更重要的是要选择使用哪种脚本。

The `script' refers to the alphabet in use; for example, both English
and French use the Latin script. Similarly, the Arabic script can be used
to write in both the Arabic and Persian languages.

“脚本”指使用的字母表;例如,英语和法语都使用拉丁字母表。同样,阿拉伯字母表可以用于写阿拉伯语和波斯语。

The
\feat{Script} and \feat{Language} features are used to designate this information. The possible options are
tabulated in \vref{tab:ot-scpt} and \vref{tab:ot-lang},
respectively. When a script or language is requested that is not
supported by the current font, a warning is printed in the console output.
See \vref{sec:newscriptlang} for methods to create new \feat{Script} or \feat{Language}
options if required.

\feat{Script} 和 \feat{Language} 特征用于指定此信息。可用选项列在 \vref{tab:ot-scpt} 和 \vref{tab:ot-lang} 中。当请求不受当前字体支持的脚本或语言时,会在控制台输出中打印警告。如果需要,可以参见 \vref{sec:newscriptlang} 中的方法创建新的 \feat{Script} 或 \feat{Language} 选项。

Because these font features can
change which features are able to be selected for the font, the \feat{Script} and \feat{Language}
settings are automatically selected
by \pkg{fontspec} before all others, and, if \XeTeX\ is being used, will
specifically select the \opt{OpenType}
renderer for this font, as described in \vref{sec:renderer-xetex}.

由于这些字体特征可以改变可以选择的字体特征,因此在所有其他特征之前,\pkg{fontspec} 会自动选择 \feat{Script} 和 \feat{Language} 设置,并且如果使用 \XeTeX,则会专门选择此字体的 \opt{OpenType} 渲染器,如 \vref{sec:renderer-xetex} 中所述。

OpenType fonts can make available different font features depending on the Script and
Language chosen.
In addition, these settings can also set up their own font behaviour
and glyph selection (one example is differences in style between some of the letters in the
alphabet used for Bulgarian, Serbian, and Russian).
The \pkg{fontspec} feature \feat{LocalForms} \texttt{=} \opt{Off} will disable some of
these substitutions if desired for some reason.
It is important to note that \feat{LocalForms} \texttt{=} \opt{On} is a default not of \pkg{fontspec}
but of the underlying font shaping engines in both \XeTeX\ and \LuaTeX/\pkg{otfload}.

OpenType 字体可以根据所选择的脚本和语言提供不同的字体特征。此外,这些设置还可以设置自己的字体行为和字形选择(一个示例是用于保加利亚语、塞尔维亚语和俄语的字母表中某些字母之间的样式差异)。如果需要,\pkg{fontspec} 功能 \feat{LocalForms} \texttt{=} \opt{Off} 将禁用其中一些替换。重要的是要注意,\feat{LocalForms} \texttt{=} \opt{On} 不是 \pkg{fontspec} 的默认设置,而是 \XeTeX 和 \LuaTeX/\pkg{otfload} 中的基础字体成形引擎的默认设置。


\subsection{\feat{Script} and \feat{Language} examples\\\feat{Script} 和 \feat{Language} 示例}

In the examples shown in \exref{script-lang},
the Code2000 font\note{\url{http://www.code2000.net/}}
is used to typeset various input texts with and without the OpenType Script
applied for various alphabets.
The text is only rendered correctly in the second case;
many examples of incorrect diacritic spacing as
well as a lack of contextual ligatures and rearrangement can be
seen.
Thanks to \name{Jonathan Kew}, \name{Yves Codet} and
\name{Gildas Hamel} for their contributions towards these examples.

在\exref{script-lang}所示的例子中,使用Code2000字体\note{\url{http://www.code2000.net/}}对各种字母表的输入文本进行排版,同时应用或不应用OpenType脚本。只有在第二种情况下文本才能正确呈现;可以看到许多错误的重音间距以及缺乏上下文连字和重新排列的例子。感谢Jonathan Kew,Yves Codet和Gildas Hamel对这些示例的贡献。

\begin{Xexample}[firstline=14,lastline=23]{script-lang}{An example of various Scripts and Languages.\\各种脚本和语言的例子。}
\def\testfeature#1#2{%^^A
  \fontspec{\examplefont}#2 & \fontspec[#1]{\examplefont}#2\\[1ex]}
\def \examplefont{CODE2000.TTF}
\def \arabictext{العربي}
\def \devanagaritext{हिन्दी}
\def \bengalitext{লেখ}
\def \gujaratitext{મર્યાદા-સૂચક નિવેદન}
\def \malayalamtext{നമ്മുടെ പാരബര്യ}
\def \gurmukhitext{ਆਦਿ ਸਚੁ ਜੁਗਾਦਿ ਸਚੁ}
\def \tamiltext{தமிழ் தேடி}
\def \hebrewtext{רִדְתָּֽהּ}
\def \vietnamesetext{cấp số mỗi}
\begin{tabular}{r@{\quad}l}
  \testfeature{Script=Arabic}{\arabictext}
  \testfeature{Script=Devanagari}{\devanagaritext}
  \testfeature{Script=Bengali}{\bengalitext}
  \testfeature{Script=Gujarati}{\gujaratitext}
  \testfeature{Script=Malayalam}{\malayalamtext}
  \testfeature{Script=Gurmukhi}{\gurmukhitext}
  \testfeature{Script=Tamil}{\tamiltext}
  \testfeature{Script=Hebrew}{\hebrewtext}
  \def\examplefont{DoulosSILR.ttf}
  \testfeature{Language=Vietnamese}{\vietnamesetext}
\end{tabular}
\end{Xexample}



\begin{table}[!hbp]
  \caption{Defined \opt{Script}s for OpenType fonts. Aliased names are shown in adjacent positions marked with red pilcrows ({\sffamily\textcolor{red}{\P}}).\\定义了OpenType字体的\opt{Script}。重命名的名称在相邻的位置用红色段落符号({\sffamily\textcolor{red}{\P}})标出。}
  \label{tab:ot-scpt}
\def\dup{\makebox[0pt][r]{\textcolor{red}{\P}}}
\setlength\columnseprule{0pt}
  \hrule
  \begin{multicols}{4}\setlength\parindent{0pt}
    \sffamily\scriptsize
    Adlam \par
    Ahom \par
    Anatolian Hieroglyphs \par
    Arabic \par
    Armenian \par
    Avestan \par
    Balinese \par
    Bamum \par
    Bassa Vah \par
    Batak \par
    Bengali \par
    Bhaiksuki \par
    Bopomofo \par
    Brahmi \par
    Braille \par
    Buginese \par
    Buhid \par
    Byzantine Music \par
    Canadian Syllabics \par
    Carian \par
    Caucasian Albanian \par
    Chakma \par
    Cham \par
    Cherokee \par
    \dup CJK \par
    \dup CJK Ideographic \par
    Coptic \par
    Cypriot Syllabary \par
    Cyrillic \par
    Default \par
    Deseret \par
    Devanagari \par
    Duployan \par
    Egyptian Hieroglyphs \par
    Elbasan \par
    Ethiopic \par
    Georgian \par
    Glagolitic \par
    Gothic \par
    Grantha \par
    Greek \par
    Gujarati \par
    Gurmukhi \par
    Hangul Jamo \par
    Hangul \par
    Hanunoo \par
    Hatran \par
    Hebrew \par
    \dup Hiragana and Katakana \par
    \dup Kana \par
    Imperial Aramaic \par
    Inscriptional Pahlavi \par
    Inscriptional Parthian \par
    Javanese \par
    Kaithi \par
    Kannada \par
    Kayah Li \par
    Kharosthi \par
    Khmer \par
    Khojki \par
    Khudawadi \par
    Lao \par
    Latin \par
    Lepcha \par
    Limbu \par
    Linear A \par
    Linear B \par
    Lisu \par
    Lycian \par
    Lydian \par
    Mahajani \par
    Malayalam \par
    Mandaic \par
    Manichaean \par
    Marchen \par
    \dup Math \par
    \dup Maths \par
    Meitei Mayek \par
    Mende Kikakui \par
    Meroitic Cursive \par
    Meroitic Hieroglyphs \par
    Miao \par
    Modi \par
    Mongolian \par
    Mro \par
    Multani \par
    Musical Symbols \par
    Myanmar \par
    \dup N'Ko \par
    \dup N'ko \par
    Nabataean \par
    Newa \par
    Ogham \par
    Ol Chiki \par
    Old Italic \par
    Old Hungarian \par
    Old North Arabian \par
    Old Permic \par
    Old Persian Cuneiform \par
    Old South Arabian \par
    Old Turkic \par
    \dup Oriya \par
    \dup Odia \par
    Osage \par
    Osmanya \par
    Pahawh Hmong \par
    Palmyrene \par
    Pau Cin Hau \par
    Phags-pa \par
    Phoenician \par
    Psalter Pahlavi \par
    Rejang \par
    Runic \par
    Samaritan \par
    Saurashtra \par
    Sharada \par
    Shavian \par
    Siddham \par
    Sign Writing \par
    Sinhala \par
    Sora Sompeng \par
    Sumero-Akkadian Cuneiform \par
    Sundanese \par
    Syloti Nagri \par
    Syriac \par
    Tagalog \par
    Tagbanwa \par
    Tai Le \par
    Tai Lu \par
    Tai Tham \par
    Tai Viet \par
    Takri \par
    Tamil \par
    Tangut \par
    Telugu \par
    Thaana \par
    Thai \par
    Tibetan \par
    Tifinagh \par
    Tirhuta \par
    Ugaritic Cuneiform \par
    Vai \par
    Warang Citi \par
    Yi
  \end{multicols}
  \hrule
\end{table}

\begin{table}[p]
  \vspace*{-3cm}
  \hspace{-3cm}
  \def\dup{\makebox[0pt][r]{\textcolor{red}{\P}}}
  \begin{minipage}{\linewidth+4cm}
  \caption{Defined \opt{Language}s for OpenType fonts. Aliased names are shown in adjacent positions marked with red pilcrows ({\sffamily\textcolor{red}{\P}}).\\定义了 OpenType 字体的 \opt{Language} 选项。已定义的别名显示在相邻位置,并用红色段落符号 ({\sffamily\textcolor{red}{\P}}) 标出。}
  \label{tab:ot-lang}
  \setlength\columnseprule{0pt}
  \hrule
  \begin{multicols}{6}
    \everypar{\setlength\parindent{0pt}\setlength\hangindent{2em}}
    \sffamily\footnotesize\raggedright
    Abaza \par
    Abkhazian \par
    Adyghe \par
    Afrikaans \par
    Afar \par
    Agaw \par
    Altai \par
    Amharic \par
    Arabic \par
    Aari \par
    Arakanese \par
    Assamese \par
    Athapaskan \par
    Avar \par
    Awadhi \par
    Aymara \par
    Azeri \par
    Badaga \par
    Baghelkhandi \par
    Balkar \par
    Baule \par
    Berber \par
    Bench \par
    Bible Cree \par
    Belarussian \par
    Bemba \par
    Bengali \par
    Bulgarian \par
    Bhili \par
    Bhojpuri \par
    Bikol \par
    Bilen \par
    Blackfoot \par
    Balochi \par
    Balante \par
    Balti \par
    Bambara \par
    Bamileke \par
    Breton \par
    Brahui \par
    Braj Bhasha \par
    Burmese \par
    Bashkir \par
    Beti \par
    Catalan \par
    Cebuano \par
    Chechen \par
    Chaha Gurage \par
    Chattisgarhi \par
    Chichewa \par
    Chukchi \par
    Chipewyan \par
    Cherokee \par
    Chuvash \par
    Comorian \par
    Coptic \par
    Cree \par
    Carrier \par
    Crimean Tatar \par
    Church Slavonic \par
    Czech \par
    Danish \par
    Dargwa \par
    Woods Cree \par
    German \par
    Default \par
    Dogri \par
    Divehi \par
    Djerma \par
    Dangme \par
    Dinka \par
    Dungan \par
    Dzongkha \par
    Ebira \par
    Eastern Cree \par
    Edo \par
    Efik \par
    Greek \par
    English \par
    Erzya \par
    Spanish \par
    Estonian \par
    Basque \par
    Evenki \par
    Even \par
    Ewe \par
    French Antillean \par
    \dup Farsi \par
    \dup Parsi \par
    \dup Persian \par
    Finnish \par
    Fijian \par
    Flemish \par
    Forest Nenets \par
    Fon \par
    Faroese \par
    French \par
    Frisian \par
    Friulian \par
    Futa \par
    Fulani \par
    Ga \par
    Gaelic \par
    Gagauz \par
    Galician \par
    Garshuni \par
    Garhwali \par
    Ge'ez \par
    Gilyak \par
    Gumuz \par
    Gondi \par
    Greenlandic \par
    Garo \par
    Guarani \par
    Gujarati \par
    Haitian \par
    Halam \par
    Harauti \par
    Hausa \par
    Hawaiin \par
    Hammer-Banna \par
    Hiligaynon \par
    Hindi \par
    High Mari \par
    Hindko \par
    Ho \par
    Harari \par
    Croatian \par
    Hungarian \par
    Armenian \par
    Igbo \par
    Ijo \par
    Ilokano \par
    Indonesian \par
    Ingush \par
    Inuktitut \par
    Irish \par
    Irish Traditional \par
    Icelandic \par
    Inari Sami \par
    Italian \par
    Hebrew \par
    Javanese \par
    Yiddish \par
    Japanese \par
    Judezmo \par
    Jula \par
    Kabardian \par
    Kachchi \par
    Kalenjin \par
    Kannada \par
    Karachay \par
    Georgian \par
    Kazakh \par
    Kebena \par
    Khutsuri Georgian \par
    Khakass \par
    Khanty-Kazim \par
    Khmer \par
    Khanty-Shurishkar \par
    Khanty-Vakhi \par
    Khowar \par
    Kikuyu \par
    Kirghiz \par
    Kisii \par
    Kokni \par
    Kalmyk \par
    Kamba \par
    Kumaoni \par
    Komo \par
    Komso \par
    Kanuri \par
    Kodagu \par
    Korean Old Hangul \par
    Konkani \par
    Kikongo \par
    Komi-Permyak \par
    Korean \par
    Komi-Zyrian \par
    Kpelle \par
    Krio \par
    Karakalpak \par
    Karelian \par
    Karaim \par
    Karen \par
    Koorete \par
    Kashmiri \par
    Khasi \par
    Kildin Sami \par
    Kui \par
    Kulvi \par
    Kumyk \par
    Kurdish \par
    Kurukh \par
    Kuy \par
    Koryak \par
    Ladin \par
    Lahuli \par
    Lak \par
    Lambani \par
    Lao \par
    Latin \par
    Laz \par
    L-Cree \par
    Ladakhi \par
    Lezgi \par
    Lingala \par
    Low Mari \par
    Limbu \par
    Lomwe \par
    Lower Sorbian \par
    Lule Sami \par
    Lithuanian \par
    Luba \par
    Luganda \par
    Luhya \par
    Luo \par
    Latvian \par
    Majang \par
    Makua \par
    Malayalam Traditional \par
    Mansi \par
    Marathi \par
    Marwari \par
    Mbundu \par
    Manchu \par
    Moose Cree \par
    Mende \par
    Me'en \par
    Mizo \par
    Macedonian \par
    Male \par
    Malagasy \par
    Malinke \par
    Malayalam Reformed \par
    Malay \par
    Mandinka \par
    Mongolian \par
    Manipuri \par
    Maninka \par
    Manx Gaelic \par
    Moksha \par
    Moldavian \par
    Mon \par
    Moroccan \par
    Maori \par
    Maithili \par
    Maltese \par
    Mundari \par
    Naga-Assamese \par
    Nanai \par
    Naskapi \par
    N-Cree \par
    Ndebele \par
    Ndonga \par
    Nepali \par
    Newari \par
    Nagari \par
    Norway House Cree \par
    Nisi \par
    Niuean \par
    Nkole \par
    N'ko \par
    Dutch \par
    Nogai \par
    Norwegian \par
    Northern Sami \par
    Northern Tai \par
    Esperanto \par
    Nynorsk \par
    Oji-Cree \par
    Ojibway \par
    Oriya \par
    Oromo \par
    Ossetian \par
    Palestinian Aramaic \par
    Pali \par
    Punjabi \par
    Palpa \par
    Pashto \par
    Polytonic Greek \par
    Pilipino \par
    Palaung \par
    Polish \par
    Provencal \par
    Portuguese \par
    Chin \par
    Rajasthani \par
    R-Cree \par
    Russian Buriat \par
    Riang \par
    Rhaeto-Romanic \par
    Romanian \par
    Romany \par
    Rusyn \par
    Ruanda \par
    Russian \par
    Sadri \par
    Sanskrit \par
    Santali \par
    Sayisi \par
    Sekota \par
    Selkup \par
    Sango \par
    Shan \par
    Sibe \par
    Sidamo \par
    Silte Gurage \par
    Skolt Sami \par
    Slovak \par
    Slavey \par
    Slovenian \par
    Somali \par
    Samoan \par
    Sena \par
    Sindhi \par
    Sinhalese \par
    Soninke \par
    Sodo Gurage \par
    Sotho \par
    Albanian \par
    Serbian \par
    Saraiki \par
    Serer \par
    South Slavey \par
    Southern Sami \par
    Suri \par
    Svan \par
    Swedish \par
    Swadaya Aramaic \par
    Swahili \par
    Swazi \par
    Sutu \par
    Syriac \par
    Tabasaran \par
    Tajiki \par
    Tamil \par
    Tatar \par
    TH-Cree \par
    Telugu \par
    Tongan \par
    Tigre \par
    Tigrinya \par
    Thai \par
    Tahitian \par
    Tibetan \par
    Turkmen \par
    Temne \par
    Tswana \par
    Tundra Nenets \par
    Tonga \par
    Todo \par
    Turkish \par
    Tsonga \par
    Turoyo Aramaic \par
    Tulu \par
    Tuvin \par
    Twi \par
    Udmurt \par
    Ukrainian \par
    Urdu \par
    Upper Sorbian \par
    Uyghur \par
    Uzbek \par
    Venda \par
    Vietnamese \par
    Wa \par
    Wagdi \par
    West-Cree \par
    Welsh \par
    Wolof \par
    Tai Lue \par
    Xhosa \par
    Yakut \par
    Yoruba \par
    Y-Cree \par
    Yi Classic \par
    Yi Modern \par
    Chinese Hong Kong \par
    Chinese Phonetic \par
    Chinese Simplified \par
    Chinese Traditional \par
    Zande \par
    Zulu
  \end{multicols}
  \hspace{4pt}
  \hrule
 \end{minipage}
\end{table}


\section{OpenType font features\\OpenType 字体特征}
\label{sec:ot-feat}

There are a finite set of OpenType font features, and \pkg{fontspec} provides an
interface to around half of them.
Full documentation will be presented in the following sections, including how to
enable and disable individual features, and how they interact.

有限的一组 OpenType 字体特征,\pkg{fontspec} 提供了其中约一半的接口。
下面的章节将提供完整的文档,包括如何启用和禁用单个特征以及它们之间的交互。


A brief reference is provided (\vref{tab:all-ot}) but note that this is an incomplete
listing --- only the `enable' keys are shown, and where alternative interfaces are
provided for convenience only the first is shown.
(E.g., |Numbers=OldStyle| is the same as |Numbers=Lowercase|.)

提供了简要参考信息(\vref{tab:all-ot}),但请注意这是不完整的列表——仅显示了“启用”键,并且在提供方便的替代界面的情况下,仅显示了第一个界面。
(例如,|Numbers=OldStyle| 与 |Numbers=Lowercase| 相同。)

For completeness, the complete list of OpenType features \emph{not} provided with
a \pkg{fontspec} interface is shown in \vref{tab:none-ot}.
Features omitted are partially by design and partially by oversight;
for example, the |aalt| feature is largely useless in \TeX\ since it is designed
for providing a \textsc{gui} interface for selecting `all alternates' of a glyph.
Others, such as optical bounds for example, simply haven't yet been considered
due to a lack of fonts available for testing.
Suggestions welcome for how/where to add these missing features to the package.

为了完整起见,\emph{未}提供 \pkg{fontspec} 接口的完整的 OpenType 特征列表在 \vref{tab:none-ot} 中显示。
省略的功能部分是出于设计,部分是由于疏忽;例如,|aalt| 特征在 \TeX\ 中大多无用,因为它旨在为选择字形的“所有替代”提供 \textsc{gui} 接口。
其他例如光学边界的特征之所以未考虑是由于缺乏可用于测试的字体。
欢迎提供建议,以便将这些缺失的特征添加到该软件包中的何处。
\ExplSyntaxOn
\def\allOTfeat{
  \prop_map_inline:Nn \g__fontspec_all_opentype_feature_names_prop
    { \opentypefeature{##1}{##2} }
}
\newcommand\opentypefeature[2]{
  \prop_get:NnNT \g__fontspec_OT_features_prop {#1} \tmpa
      {
        \raggedright
        \hangindent=5.2cm
        \makebox[1cm][l]{\textsc{#1}}
        \makebox[4.2cm][l]{
          \int_compare:nT { \tl_count:N \tmpa > 25 } {\ttcondensed} {\ttfamily}
          \tmpa
        }
        \textit{#2}
        \par
        \vspace{2pt}
     }
}
\ExplSyntaxOff

\begin{table}
\caption{Summary of OpenType features in \textsf{fontspec}, alphabetic by feature tag.\\按功能标签字母顺序列出的\textsf{fontspec}中OpenType功能的摘要。}
\label{tab:all-ot}
\centerline{%
\begin{minipage}{18cm}
\small
\hrule\smallskip
\begin{multicols}{2}
\parindent =0pt
\allOTfeat
\end{multicols}
\vspace*{-\smallskipamount}
\hrule
\end{minipage}}
\end{table}

\ExplSyntaxOn
\renewcommand\opentypefeature[2]{
  \prop_get:NnNF \g__fontspec_OT_features_prop {#1} \tmpa
      {
        \raggedright
        \hangindent=0.9cm
        \makebox[0.9cm][l]{\textsc{#1}}%
        \textit{#2}
        \par
     }
}
\ExplSyntaxOff

\begin{table}
\caption{List of \emph{unsupported} OpenType features.}
\label{tab:none-ot}
\bigskip
\centerline{%
\begin{minipage}{15cm}
\hrule\smallskip
\begin{multicols}{3}
\parindent =0pt
\allOTfeat
\end{multicols}
\vspace*{-\smallskipamount}
\hrule
\end{minipage}}
\end{table}

\subsection{Tag-based features\\基于标记的特征}


\subsubsection{Alternates --- \texttt{salt}\\替代形式 --- \texttt{salt}}

The \feat{Alternate} feature, alias \feat{StylisticAlternates}, is used to access alternate font glyphs when variations exist in the font, such as in \exref{salt}.
It uses a numerical selection, starting from zero, that will be different for each font.
Note that the \texttt{Style=Alternate} option is equivalent
to \texttt{Alternate=0} to access the default case.

当字体存在多种变体时(如在例 \exref{salt} 中),\feat{Alternate} 特征(别名 \feat{StylisticAlternates})用于访问替代字形。它使用数字选择,从零开始,并对每种字体都不同。请注意,选项 \texttt{Style=Alternate} 等效于 \texttt{Alternate=0},用于访问默认情况。

\begin{Xexample}[firstline=2]{salt}{The \feat{Alternate} feature.\\\feat{Alternate} 特征。}
  \huge
  \fontspec{LinLibertine_R.otf}
  \textsc{a} \& h \\
  \addfontfeature{Alternate=0}
  \textsc{a} \& h
\end{Xexample}

Note that the indexing starts from zero.
With the \LuaTeX\ engine, |Alternate=Random| selects a random alternate.

请注意,索引从零开始。在 \LuaTeX\ 引擎中,|Alternate=Random| 会选择随机替代。

See \vref{sec:newfeatures} for a way to assign names to alternates if desired.

如需按名称为替代字形分配名称,请参阅 \vref{sec:newfeatures}。

\subsubsection{Character Variants --- \texttt{cvNN}\\字符变体 --- \texttt{cvNN}}

`Character Variations' are selected
numerically to adjust the output of (usually) a single character for the
particular font. These correspond to the OpenType features |cv01| to |cv99|.

“字符变体”用数字选择来调整特定字体中(通常)单个字符的输出。这些对应于 OpenType 特征 |cv01| 到 |cv99|。

For each character that can be varied, it is possible to select among
possible options for that particular glyph.
For example, in the hypothetical example below, variants are chosen for glyphs `4' and `5',
and the trailing |:|\meta{n} corresponds to which variety to choose.

对于每个可以变化的字符,可以在该特定字形的可能选项中进行选择。例如,在以下假设的示例中,为字形“4”和“5”选择了变体,并且后面的 |:|\meta{n} 对应于要选择的种类。

\begin{Verbatim}
  \fontspec{CV Font}[CharacterVariant={4,5:2}] \& violet
\end{Verbatim}
The numbering is entirely font-specific. Glyph `5' might be the character `v', for example.
Character variants are specifically designed not to conflict with each
other, so you can enable them individually per character.
(Unlike stylistic alternates, say.)
Note that the indexing starts from zero.

编号完全取决于字体。例如,字形“5”可能是字符“v”。字符变体专门设计为彼此不冲突,因此您可以针对每个字符单独启用它们。请注意,索引从零开始。

\subsubsection{Contextuals\\上下文相关替换}
This feature refers to substitutions of glyphs that vary `contextually' by their relative position in a word or string of characters;
features such as contextual swashes are accessed via the options shown in \ref{feat:Contextuals}.

该特征指的是字形在单词或一串字符中的相对位置而变化的“上下文”替换;可以通过 \ref{feat:Contextuals} 中显示的选项访问上下文替换,如上下文花边等。

\begin{features}{Contextuals}
\otf*{Swash}{cswh}
\otf*{Alternate}{calt}
\otf*{WordInitial}{init}
\otf*{WordFinal}{fina}
\otf*{LineFinal}{falt}
\otf*{Inner}{medi}
\cmidrule{2-4}
\otf{ResetAll}{}
\end{features}

Historic forms are accessed in OpenType
fonts via the feature \feat{Style=Historic}; this is generally \emph{not}
contextual in OpenType, which is why it is not included in this feature.

历史形式可以通过 OpenType 字体中的 \feat{Style=Historic} 特征访问;这通常在 OpenType 中不是上下文相关的,因此未包括在此特征中。

\subsubsection{Diacritics\\变音符号}

Specifies how combining diacritics should be placed.
These will usually be controlled automatically
according to the Script setting.

指定如何放置组合变音符号。
通常会根据脚本设置自动控制。

\begin{features}{Diacritics}
\otf*{MarkToBase}{mark}
\otf*{MarkToMark}{mkmk}
\otf*{AboveBase}{abvm}
\otf*{BelowBase}{blwm}
\cmidrule{2-4}
\otf{ResetAll}{}
\end{features}


\subsubsection{Fractions --- \texttt{frac}\\分数 --- \texttt{frac}}

\begin{features}{Fractions}
\otf{On}{+frac}
\otf{Off}{-frac}
\otf{Reset}{}
\cmidrule{2-4}
\otf*{Alternate}{afrc}
\cmidrule{2-4}
\otf{ResetAll}{}
\end{features}

Activates the construction of `vulgar' fractions using precomposed glyphs and/or
subscript and superscript characters from within the font.
Coverage will vary by font; see \exref{ot-frac}.
Some (Asian fonts predominantly) also provide for the \opt{Alternate} option.

激活使用预先组成的字形和/或字体内的上标和下标字符构造“真分数”的功能。不同字体的覆盖范围会有所不同;请参见\exref{ot-frac}。有些字体(主要是亚洲字体)还提供了\opt{Alternate}选项。

\begin{Lexample}{ot-frac}{The \feat{Fractions} feature.}
\setsansfont{Lato}[Fractions=On]
\setmonofont{IBM Plex Mono}[Fractions=On]

\sffamily 1/2 47/11 1/1000 \par
\ttfamily 1/2 47/11
\end{Lexample}



\subsubsection{Kerning --- \texttt{kern}\\字距 --- \texttt{kern}}
\label{sec:kerning}

Specifies how inter-glyph spacing should behave.
Well-made fonts include information for how differing
amounts of space should be inserted between separate character pairs.
This kerning space is inserted automatically but in rare
circumstances you may wish to turn it off.

指定字形之间的间距应该如何表现。制作精良的字体包括不同的字形对之间应该插入多少空格的信息。这种字距空间会自动插入,但在极少数情况下,您可能希望关闭它。

\begin{features}{Kerning}
\otf{On}{+kern}
\otf{Off}{-kern}
\otf{Reset}{}
\cmidrule{2-4}
\otf*{Uppercase}{cpsp}
\cmidrule{2-4}
\otf{ResetAll}{}
\end{features}

As briefly mentioned previously at the end of \vref{sec:letters},
the \opt{Uppercase} option will add a small amount of tracking between
uppercase letters, seen in \exref{kernup}, which uses the Romande
fonts\note{\url{http://arkandis.tuxfamily.org/adffonts.html}}
(thanks to Clea F. Rees for the suggestion).
The \opt{Uppercase} option acts separately to the regular kerning
controlled by the \opt{On}/\opt{Off} options.

如\ref{sec:letters}节末尾简要提到的,\opt{Uppercase}选项会在大写字母之间添加一小段跟踪,如\exref{kernup}所示,使用了Romande字体(感谢Clea F. Rees提供建议)。\opt{Uppercase}选项与由\opt{On}/\opt{Off}选项控制的常规字距相分离。

\begin{Xexample}[firstline=2]{kernup}{Adding extra kerning for uppercase letters. (The difference is usually very small.)\\为大写字母添加额外的字距(差异通常非常小)。}
  \large
  \fontspec{RomandeADFStd-DemiBold.otf}
   UPPERCASE EXAMPLE \\
  \addfontfeature{Kerning=Uppercase}
   UPPERCASE EXAMPLE
\end{Xexample}


\subsubsection{Letters\\字母} \label{sec:letters}
The \opt{Letters} feature specifies how the letters in the current font
will look. OpenType fonts may contain the following options:
\opt{SmallCaps}, \opt{PetiteCaps},
\opt{UppercaseSmallCaps}, \opt{UppercasePetiteCaps}, and
\opt{Unicase}.

\opt{Letters} 特性规定了当前字体中字母的外观。OpenType 字体可以包含以下选项:\opt{SmallCaps}、\opt{PetiteCaps}、\opt{UppercaseSmallCaps}、\opt{UppercasePetiteCaps} 和 \opt{Unicase}。
下面的代码展示了这些选项,你可以在你的文档中使用它们:

\begin{features}{Letters}
\otf*{SmallCaps}{smcp}
\otf*{PetiteCaps}{pcap}
\otf*{UppercaseSmallCaps}{c2sc}
\otf*{UppercasePetiteCaps}{c2pc}
\otf*{Unicase}{unic}
\cmidrule{2-4}
\otf{ResetAll}{}
\end{features}

Petite caps are smaller than small caps.
\opt{SmallCaps} and \opt{PetiteCaps}
turn lowercase letters into the smaller caps letters,
whereas the \opt{Uppercase...} options turn the \emph{capital} letters into
the smaller
caps (good, \eg, for applying to already uppercase acronyms like
`NASA').
This difference is shown in \exref{caps}.
`Unicase' is a weird hybrid of upper and lower case letters.

小小型大写字母比小型大写字母更小。选项 \opt{SmallCaps} 和 \opt{PetiteCaps} 将小写字母转换为更小的大写字母,而选项 \opt{Uppercase...} 将\emph{大写}字母转换为更小的大写字母(例如,适用于已经大写的缩写词,如“NASA”)。这种差异在 \exref{caps} 中展示。
选项 \opt{Unicase} 是大写和小写字母的奇怪混合体。

\begin{Lexample}{caps}{Small caps from lowercase or uppercase letters.\\小型大写字母和小型小写字母}
  \fontspec{texgyreadventor-regular.otf}[Letters=SmallCaps]
   THIS SENTENCE no verb                \\
  \fontspec{texgyreadventor-regular.otf}[Letters=UppercaseSmallCaps]
   THIS SENTENCE no verb
\end{Lexample}

\subsubsection{Ligatures\\连字}
\label{sec:ot-feat-liga}

\feat{Ligatures} refer to the replacement of two separate characters
with a specially drawn glyph for functional or \ae sthetic reasons.
The list of options, of which multiple may be selected at one time,
is shown in \ref{feat:Ligatures}.
A demonstration with the Linux Libertine fonts\note{\url{http://www.linuxlibertine.org/}} is shown in \exref{lig}.

\feat{Ligatures}(连字)指的是为了实现某些功能或美观效果而将两个分开的字符替换为特殊绘制的字形。可以同时选择多个选项,选项列表如 \ref{feat:Ligatures} 所示。使用 Linux Libertine 字体的演示见 \exref{lig}。

Note the additional features accessed with \verb|Ligatures=TeX|. These are
not actually real OpenType features, but additions provided by \pkg{luaotfload} (i.e., \LuaTeX\ only) to emulate \TeX's behaviour for \textsc{ascii} input of curly quotes and punctuation. In \XeTeX\ this is achieved with the \feat{Mapping} feature (see \vref{sec:mapping}) but for consistency \verb|Ligatures=TeX| will perform the same function as \verb|Mapping=tex-text|.

请注意,可以通过 \verb|Ligatures=TeX| 访问其他特性。这些实际上不是真正的 OpenType 特性,而是由 \pkg{luaotfload} 提供的补充功能(仅适用于 \LuaTeX),以模拟 \textsc{ascii} 输入的花式引号和标点符号在 \TeX 中的行为。在 \XeTeX 中,可以使用 \feat{Mapping} 特性(见 \vref{sec:mapping})来实现此功能,但为了保持一致性,\verb|Ligatures=TeX| 将执行与 \verb|Mapping=tex-text| 相同的功能。

\begin{features}{Ligatures}
\otf*{Required}{rlig}
\otf*{Common}{liga}
\otf*{Contextual}{clig}
\otf*{Rare/Discretionary}{dlig}
\otf*{Historic}{hlig}
\otf*{TeX}{tlig}
\cmidrule{2-4}
\otf{ResetAll}{}
\end{features}

\begin{Lexample}[firstline=2]{lig}{An example of the \feat{Ligatures} feature.\\\feat{连字} 特性的例子。}
   \Huge\centering
   \def\test#1#2{%
     #2 $\to$ {\addfontfeature{#1} #2}\\}
   \fontspec{LinLibertine_R.otf}
   \test{Ligatures=Historic}{strict}
   \test{Ligatures=Rare}{wurtzite}
   \test{Ligatures=CommonOff}{firefly}
\end{Lexample}


\subsubsection{Localised Forms --- \texttt{locl}\\本地化表单 --- \texttt{locl}}

\begin{features}{LocalForms}
\otf{On}{+locl}
\otf{Off}{-locl}
\otf{Reset}{}
\end{features}

This feature enables and disables glyph substitutions, etc., that are specific to the
\feat{Language} selected in the font.
This feature is automatically activated by default when present,
so it should not be generally necessary to use \feat{LocalForms} \texttt{=} \opt{On}.
In certain scenarios it may be important to turn it \opt{Off} (although nothing specifically springs to mind).

此特性启用和禁用特定于字体中所选 \feat{语言} 的字形替换等。
该特性在存在时默认自动激活,
因此通常不需要使用 \feat{本地化表单} \texttt{=} \opt{打开}。
在某些情况下,关闭它可能很重要(尽管没有什么具体想法)。


\subsubsection{Numbers\\数字}

The \feat{Numbers} feature defines how numbers will look in the
selected font, accepting options shown in \ref{feat:Numbers}.

\feat{数字} 特性定义所选字体中数字的外观,接受 \ref{feat:Numbers} 中显示的选项。

\begin{features}{Numbers}
\otf*{Uppercase}{lnum}
\otf*{Lowercase}{onum}
\otf*{Lining}{lnum}
\otf*{OldStyle}{onum}
\otf*{Proportional}{pnum}
\otf*{Monospaced}{tnum}
\otf*{SlashedZero}{zero}
\otf*{Arabic}{anum}
\cmidrule{2-4}
\otf{ResetAll}{}
\end{features}

The synonyms
\opt{Uppercase} and \opt{Lowercase} are equivalent to \opt{Lining} and
\opt{OldStyle}, respectively.
The differences have been shown previously
in \vref{sec:addfontfeatures}.
The \opt{Monospaced} option is useful for tabular material when digits need
to be vertically aligned.

同义词 \opt{大写} 和 \opt{小写} 分别等效于 \opt{标准} 和 \opt{古老}。
差异在 \vref{sec:addfontfeatures} 中已经显示过了。
\opt{等宽} 选项在数字需要垂直对齐的表格材料中很有用。

The \opt{SlashedZero} option
replaces the default zero with a slashed version to prevent
confusion with an uppercase `O', shown in \exref{slashzero}.

\opt{零带斜线} 选项将默认零替换为带斜线的版本,以防与大写字母 `O' 混淆,如 \exref{slashzero} 所示。

\begin{Lexample}{slashzero}{The effect of the \opt{SlashedZero} option.\\\opt{零带斜线} 选项的效果。}
  \fontspec[Numbers=Lining]{texgyrebonum-regular.otf}
   0123456789
  \fontspec[Numbers=SlashedZero]{texgyrebonum-regular.otf}
   0123456789
\end{Lexample}

The \opt{Arabic} option (with tag \verb|anum|) maps regular numerals to their Arabic script or Persian equivalents
based on the current \opt{Language} setting (see \vref{sec:ot}).
This option is based on a \LuaTeX\ feature of the \pkg{luaotfload} package,
not an OpenType feature. (Thus, this feature is unavailable in \XeTeX.)
This feature should be considered deprecated; while there are no plans to remove it from this package,
if its support is dropped from the font loader it could disappear from \pkg{fontspec} with little notice.

\opt{Arabic}选项(标签为\verb|anum|)基于当前的\opt{Language}设置(见\vref{sec:ot}),将常规数字映射到阿拉伯文或波斯文数字的等效符号。此选项基于\pkg{luaotfload}软件包的\LuaTeX\ 功能,而不是OpenType功能。(因此,在\XeTeX 中无法使用此功能。)此功能应被视为已弃用;虽然没有计划从该软件包中删除它,但如果从字体加载程序中删除它的支持,则可能很快从\pkg{fontspec}中消失。

\subsubsection{Ornament --- \texttt{ornm}\\装饰性 --- \texttt{ornm}}

Ornaments are selected with the \feat{Ornament} feature (OpenType feature |ornm|),
selected numerically such as for the \feat{Annotation} feature.

使用\feat{Ornament}特性(OpenType 特性|ornm|)来选择装饰,可以使用数字选择,例如\feat{Annotation}特性。

\subsubsection{Style\\风格}
\label{sec:ot-feat-style}

\begin{features}{Style}
\otf*{Alternate}{salt}
\otf*{Cursive}{curs}
\otf*{Historic}{hist}
\otf*{Italic}{ital}
\otf*{Ruby}{ruby}
\otf*{Swash}{swsh}
\otf*{Titling}{titl}
\otf*{Uppercase}{case}
\otf*{HorizontalKana}{hkna}
\otf*{VerticalKana}{vkna}
\cmidrule{2-4}
\otf{ResetAll}{}
\end{features}

`Ruby' refers to a small optical size, used in
Japanese typography for annotations.
For fonts with multiple |salt| OpenType features,
use the fontspec \feat{Alternate} feature instead.

`Ruby' 指的是日本排版中用于注释的小字体。对于带有多个|salt| OpenType 特性的字体,使用 fontspec 的 \feat{Alternate} 特性代替。

\Exref{style-alt} shows an example of a font feature that involves glyph substitution
for particular letters within an alphabet.
Other options in these categories operate in similar ways, with the choice of how
particular substitutions are organised with which feature largely up to the font designer.

示例\ref{style-alt}展示了涉及字母表中特定字母的字形替换的字体特性的示例。这些类别中的其他选项以类似的方式运作,特定替换的选择大多由字体设计师决定。

 \begin{Xexample}[firstline=2]{style-alt}{Example of the \opt{Alternate} option of the \feat{Style} feature.\\\feat{Style}特性的 \opt{Alternate} 选项的示例}
  \Large
  \fontspec{Quattrocento-Regular.otf}
   M Q W                      \\
  \addfontfeature{Style=Alternate}
   M Q W
\end{Xexample}

The \opt{Uppercase} option is designed to select various
uppercase forms for glyphs such as accents and dashes, such as shown
in \exref{style-uppercase}; note the raised position of the hyphen
to better match the surrounding letters.
It will (probably) not actually map letters to uppercase.
 \note{If you want automatic uppercase letters, look to \LaTeX's
      \cmd\MakeUppercase\ command.}
This option used to be selected under the \feat{Letters} feature, but moved here
as it generally does not actually affect the letters themselves.
The \feat{Kerning} feature also contains an \opt{Uppercase} option,
which adds a small amount of spacing in between letters (see \vref{sec:kerning}).

\opt{Uppercase} 选项旨在选择大写字母的各种形式,例如用于重音符号和破折号的字形,如示例\ref{style-uppercase}所示;请注意连字符的上升位置,以更好地匹配周围的字母。它(可能)不会实际映射字母到大写。如果你想要自动大写字母,请使用 \LaTeX 的 \cmd\MakeUppercase 命令。
该选项曾在 \feat{Letters} 特性下选择,但移至此处,因为它通常实际上并不影响字母本身。 \feat{Kerning} 特性还包含一个 \opt{Uppercase} 选项,它在字母之间添加了一小段间距(参见\vref{sec:kerning})。

\begin{Lexample}{style-uppercase}{An example of the \opt{Uppercase} option of the \feat{Style} feature.\\\feat{Style}特性的 \opt{Uppercase} 选项的示例}
  \fontspec{LinLibertine_R.otf}
   UPPER-CASE example \\
  \addfontfeature{Style=Uppercase}
   UPPER-CASE example
\end{Lexample}


In other features, larger breadths of changes can be seen, covering
the style of an entire alphabet.
See \exref{style-itrub}; here, the \opt{Italic} option affects the Latin text
and the \opt{Ruby} option the Japanese.

在其他功能中,可以看到更大的变化幅度,涵盖整个字母表的风格。
参见 \exref{style-itrub};在这里,\opt{Italic} 选项影响拉丁文本,\opt{Ruby} 选项影响日语。

\begin{Xexample}[firstline=2]{style-itrub}{Example of the \opt{Italic} and \opt{Ruby} options of the \feat{Style} feature.}
  \Large \def\kana{ようこそ ワカヨタレソ}
  \fontspec{Hiragino Mincho Pro}
   Latin \kana        \\
  \addfontfeature{Style={Italic, Ruby}}
   Latin \kana
\end{Xexample}

Note the difference here between the default and the horizontal style kana
in \exref{style-hvkana}: the horizontal style is slightly wider.

请注意,在 \exref{style-hvkana} 中默认样式和水平样式假名之间的差异:水平样式略微宽。

\begin{Xexample}[firstline=2]{style-hvkana}{Example of the \opt{HorizontalKana} and \opt{VerticalKana} options of the \feat{Style} feature.\\\feat{Style} 功能的 \opt{HorizontalKana} 和 \opt{VerticalKana} 选项的示例。}
  \Large \def\kana{ようこそ ワカヨタレソ}
   \fontspec{Hiragino Mincho Pro}
    \kana   \\
  {\addfontfeature{Style=HorizontalKana}
    \kana } \\
  {\addfontfeature{Style=VerticalKana}
    \kana }
\end{Xexample}


\subsubsection{Stylistic Set variations --- \texttt{ssNN}\\风格集变体 --- \texttt{ssNN}}
\label{sec:ot-ss}

This feature selects a `Stylistic Set' variation,
which usually corresponds to an alternate glyph style for a range of
characters (usually an alphabet or subset thereof).
This feature is specified numerically. These correspond to OpenType
features |ss01|, |ss02|, etc.

该功能选择“风格集”变体,通常对应于一系列字符的备用字形样式(通常是一个字母表或其子集)。该功能以数字形式指定。这些对应于 OpenType 功能 |ss01|、|ss02| 等。

Two demonstrations from the Junicode
font\note{\url{http://junicode.sf.net}}
are shown in \exref{ss} and \exref{ss2}; thanks to Adam
Buchbinder for the suggestion.

在 Junicode 字体\note{\url{http://junicode.sf.net}} 中展示了两个示例,感谢 Adam Buchbinder 的建议,分别在 \exref{ss} 和 \exref{ss2} 中。

\begin{Lexample}{ss}{Insular letterforms, as used in medieval Northern Europe, for the Junicode font accessed with the \feat{StylisticSet} feature.\\Junicode 字体(使用 \feat{StylisticSet} 功能)的 Insular 字形,用于中世纪北欧。}
  \fontspec{Junicode}
   Insular forms. \\
  \addfontfeature{StylisticSet=2}
   Insular forms. \\
\end{Lexample}

\begin{Lexample}{ss2}{Enlarged minuscules (capital letters remain unchanged) for the Junicode font, accessed with the \feat{StylisticSet} feature.\\Junicode 字体的放大小写字母(大写字母保持不变),使用 \feat{StylisticSet} 功能访问。}
  \fontspec{Junicode}
   ENLARGED Minuscules. \\
  \addfontfeature{StylisticSet=6}
   ENLARGED Minuscules. \\
\end{Lexample}

Multiple stylistic sets may be selected simultaneously by writing, e.g.,
|StylisticSet={1,2,3}|.

多个风格集可以同时选择,例如,写成 |StylisticSet={1,2,3}|。

The |StylisticSet| feature is a synonym of the \feat{Variant} feature for \AAT\ fonts.
See \vref{sec:newfeatures} for a way to assign names to stylistic sets, which should be done on a per-font basis.

|StylisticSet| 特性是 \AAT\ 字体中 \feat{Variant} 特性的同义词。关于如何为字体样式集分配名称,请参见 \vref{sec:newfeatures},这应该在每个字体的基础上完成。

\subsubsection{Vertical Position\\垂直位置}

\begin{features}{VerticalPosition}
\otf*{Superior}{sups}
\otf*{Inferior}{subs}
\otf*{Numerator}{numr}
\otf*{Denominator}{dnom}
\otf*{ScientificInferior}{sinf}
\otf*{Ordinal}{ordn}
\cmidrule{2-4}
\otf{ResetAll}{}
\end{features}

The \feat{VerticalPosition} feature is used to access things like
subscript (\opt{Inferior}) and superscript (\opt{Superior}) numbers and
letters (and a small amount of punctuation, sometimes).
The \opt{Ordinal} option will only raise characters that are used
in some languages directly after a number.
The \opt{ScientificInferior} feature will move glyphs
further below the baseline than the \opt{Inferior} feature.
These are shown in \exref{vertpos}

\feat{VerticalPosition} 特性用于访问下标 (\opt{Inferior}) 和上标 (\opt{Superior}) 数字、字母 (以及一些标点符号),\opt{Ordinal} 选项仅会提升一些语言中直接在数字后面使用的字符。\opt{ScientificInferior} 特性会使字形相比 \opt{Inferior} 特性进一步下降基线。这些内容显示在 \exref{vertpos} 中。

\opt{Numerator} and \opt{Denominator} should only be used for creating
arbitrary fractions (see next section).

\opt{Numerator} 和 \opt{Denominator} 应仅用于创建任意分数(见下一节)。

\begin{Lexample}{vertpos}{The \feat{VerticalPosition} feature.\\\feat{VerticalPosition} 特性。}
  \fontspec{LibreCaslonText-Regular.otf}[VerticalPosition=Superior]
   Superior: 1234567890                                   \\
  \fontspec{LibreCaslonText-Regular.otf}[VerticalPosition=Numerator]
   Numerator: 12345                                       \\
  \fontspec{LibreCaslonText-Regular.otf}[VerticalPosition=Denominator]
   Denominator: 12345                                     \\
  \fontspec{LibreCaslonText-Regular.otf}[VerticalPosition=ScientificInferior]
   Scientific Inferior: 12345
\end{Lexample}

The \pkg{realscripts} package
(which is also loaded by \pkg{xltxtra} for \XeTeX)
redefines the \cmd\textsubscript\ and
\cmd\textsuperscript\ commands to use the above font features automatically,
including for use in footnote labels.
If this is the only feature of \pkg{xltxtra} you wish to use, consider
loading \pkg{realscripts} on its own instead.

\pkg{realscripts} 宏包(也被 \XeTeX 中的 \pkg{xltxtra} 加载)会自动重定义 \cmd\textsubscript\ 和 \cmd\textsuperscript\ 命令,使用以上的字体特性,包括用于脚注标签。如果这是您想要使用 \pkg{xltxtra} 的唯一功能,请考虑单独加载 \pkg{realscripts}。

\subsection{CJK features\\CJK 特性}

This section summarises the features which are largely intending for Chinese, Korean,
and Japanese typesetting.

本节总结了主要面向中文、韩文和日文排版的特性。

\subsubsection{Annotation --- \texttt{nalt} \\注释 --- \texttt{nalt}}

Some fonts are equipped with an extensive range of
numbers and numerals in different forms. These are accessed with the
\feat{Annotation} feature (OpenType feature |nalt|), selected numerically as shown in
\exref{ot-annot}. Note that the indexing starts from zero.

一些字体配备了不同形式的广泛数字和数字字符。可以使用 \feat{Annotation} 特性(OpenType 特性 |nalt|)来访问它们,如 \exref{ot-annot} 中所示进行数字选择。请注意,索引从零开始。

\begin{Xexample}{ot-annot}{Annotation forms for OpenType fonts.\\OpenType 字体的注释形式。}
  \fontspec{Hiragino Maru Gothic Pro}
   1 2 3 4 5 6 7 8 9
  \def\x#1{\\{\addfontfeature{Annotation=#1}
            1 2 3 4 5 6 7 8 9 }}
  \x0\x1\x2\x3\x4\x5\x6\x7\x7\x8\x9
\end{Xexample}


\subsubsection{Character width\\字符宽度}\label{sec:CharacterWidth}

Many Asian fonts are equipped with variously spaced characters for
shoe-horning into their generally monospaced text.
These are
accessed through the \feat{CharacterWidth} feature.

许多亚洲字体配备了各种间隔不同的字符以适应它们通常的等宽文本排版。这些特征可以通过 \feat{CharacterWidth} 功能来实现。

\begin{features}{CharacterWidth}
\otf*{Proportional}{pwid}
\otf*{Full}        {fwid}
\otf*{Half}        {hwid}
\otf*{Third}       {twid}
\otf*{Quarter}     {qwid}
\otf*{AlternateProportional}{palt}
\otf*{AlternateHalf}{halt}
\cmidrule{2-4}
\otf{ResetAll}{}
\end{features}

Japanese alphabetic glyphs (in Hiragana or Katakana) may be typeset
proportionally, to better fit horizontal measures, or monospaced, to
fit into the rigid grid imposed by ideographic typesetting. In this
latter case, there are also half-width forms for squeezing more kana
glyphs (which are less complex than the kanji they are amongst) into
a given block of space. The same features are given to roman letters
in Japanese fonts, for typesetting foreign words in the same style
as the surrounding text.

日文假名(平假名或片假名)可以按比例排版以更好地适应横向度量,也可以等宽排版以适应象形字排版所强加的刚性网格。在后一种情况下,还有半角形式,以将更多假名字母(它们比它们之间的汉字更简单)挤入给定的空间块中。日文字体中的罗马字母也具有相同的特征,可用于以与周围文本相同的风格排版外来词汇。

\begin{Xexample}[firstline=2]{charwdprop}{Proportional or fixed width forms.\\比例或固定宽度形式。}
  \def\texta{ようこそ}\def\textb{ワカヨタレソ}
  \def\test{\makebox[2cm][l]{\texta}%
            \makebox[2.5cm][l]{\textb}%
            \makebox[2.5cm][l]{abcdef}}
  \fontspec{Hiragino Mincho Pro}
  {\addfontfeature{CharacterWidth=Proportional}\test}\\
  {\addfontfeature{CharacterWidth=Full}\test}\\
  {\addfontfeature{CharacterWidth=Half}\test}
\end{Xexample}

The same situation occurs with numbers, which are provided in
increasingly illegible compressed forms seen in \exref{charwd}.

同样的情况也出现在数字中,这些数字以越来越难以辨认的压缩形式呈现,如 \exref{charwd} 所示。

\begin{Xexample}[firstline=2]{charwd}{Numbers can be compressed significantly.\\数字可以显著压缩。}
  \centering
  \fontspec[Renderer=AAT]{Hiragino Mincho Pro}
  {\addfontfeature{CharacterWidth=Full}
   ---12321---}\\
  {\addfontfeature{CharacterWidth=Half}
   ---1234554321---}\\
  {\addfontfeature{CharacterWidth=Third}
   ---123456787654321---}\\
  {\addfontfeature{CharacterWidth=Quarter}
   ---12345678900987654321---}
\end{Xexample}


\subsubsection{CJK shape\\CJK 形状}

\begin{features}{CJKShape}
\otf{Traditional}{trad}
\otf{Simplified} {smpl}
\otf{JIS1978}    {jp78}
\otf{JIS1983}    {jp83}
\otf{JIS1990}    {jp90}
\otf{Expert}     {expt}
\otf{NLC}        {nlck}
\end{features}

There have been many standards for how CJK ideographic
glyphs are `supposed' to look. Some fonts will contain many alternate
glyphs available in order to be able to display these gylphs
correctly in whichever form is appropriate. Both \AAT\ and OpenType
fonts support the following \feat{CJKShape} options:
\opt{Traditional}, \opt{Simplified}, \opt{JIS1978}, \opt{JIS1983},
\opt{JIS1990}, and \opt{Expert}. OpenType also supports the \opt{NLC} option.

有许多标准规定CJK表意字应该长成什么样子。一些字体包含了许多备用字形,以便能够以适当的形式正确显示这些字形。AAT和OpenType字体都支持以下CJKShape选项:Traditional、Simplified、JIS1978、JIS1983、JIS1990和Expert。OpenType还支持NLC选项。

\begin{Xexample}[firstline=2]{ot-cjk-shape}{Different standards for CJK ideograph presentation.\\不同的CJK表意字符表示标准。}
  \LARGE\def\text{ 唖噛躯 妍并訝}
  \fontspec{Hiragino Mincho Pro}
  {\addfontfeature{CJKShape=Traditional}
  \text }                          \\
  {\addfontfeature{CJKShape=NLC}
  \text }                          \\
  {\addfontfeature{CJKShape=Expert}
  \text }
\end{Xexample}


\subsubsection{Vertical typesetting\\竖排版}

\begin{features}{Vertical}
\otf*{RotatedGlyphs}         {vrt2}
\otf*{AlternatesForRotation} {vrtr}
\otf*{Alternates}            {vert}
\otf*{KanaAlternates}        {vkna}
\otf*{Kerning}               {vkrn}
\otf*{AlternateMetrics}      {valt}
\otf*{HalfMetrics}           {vhal}
\otf*{ProportionalMetrics}   {vpal}
\cmidrule{2-4}
\otf{ResetAll}{}
\end{features}

OpenType provides a plethora of features for accommodating the varieties of possibilities
needed for vertical typesetting (CJK and others).
No capabilities for achieving such vertical typesetting are provided by \pkg{fontspec},
however; please get in touch if there are improvements that could be made.

OpenType提供了大量功能来适应竖排版(CJK和其他语言)的各种可能性。然而,\pkg{fontspec}并没有提供实现这种竖排版的功能;如果有改进的空间,请联系我们。

\end{document}

% /©
% ------------------------------------------------
% The FONTSPEC package  <wspr.io/fontspec>
% ------------------------------------------------
% Copyright  2004-2022  Will Robertson, LPPL "maintainer"
% Copyright  2009-2015  Khaled Hosny
% Copyright  2013       Philipp Gesang
% Copyright  2013-2016  Joseph Wright
% ------------------------------------------------
% This package is free software and may be redistributed and/or modified under
% the conditions of the LaTeX Project Public License, version 1.3c or higher
% (your choice): <http://www.latex-project.org/lppl/>.
% ------------------------------------------------
% ©/
