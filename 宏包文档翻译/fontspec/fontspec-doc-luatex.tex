%%^^A%%  fontspec-doc-luatex.tex -- part of FONTSPEC <wspr.io/fontspec>

\documentclass[a4paper]{l3doc}
\usepackage{fontspec-doc-style}
\showexamplesfalse
\begin{document}

\part{\LuaTeX-only font features\\\LuaTeX-专有字体特性}
\label{sec:luatex-features}

\section{Different font technologies and shapers\\不同的字体技术和渲染器}
\label{sec:renderer-luatex}

\LuaTeX\ does not directly support any font rendering technologies out of the box, it
requires additional functionality to be added to properly support and control technologies such as OpenType.

\LuaTeX\ 并不直接支持任何字体渲染技术,需要添加额外的功能以适当支持和控制诸如 OpenType 等技术。

Using the \feat{Renderer} feature, there are a number of options that \pkg{fontspec} can pass to the engine to control which font technology is being used.
Pre-2019, there were two options provided by \pkg{luaotfload} that generally did not require user intervention.

使用 \feat{Renderer} 特性,\pkg{fontspec} 可以向引擎传递一些选项以控制正在使用的字体技术。
2019年之前,\pkg{luaotfload} 提供了两个选项,通常不需要用户干预。
\begin{itemize}
\item \feat{Renderer} = \opt{Node} : the default `mode' for typesetting OpenType fonts.
\\\feat{Renderer} = \opt{Node}:排版 OpenType 字体的默认“模式”。
\item \feat{Renderer} = \opt{Base} : a simplified mode useful only in a limited number of situations such as mathematics typesetting.
\\\feat{Renderer} = \opt{Base}:仅在少数情况下有用的简化模式,例如数学排版。
\end{itemize}


From 2019 the possibility of using the Harfbuzz text shaping engine within \LuaTeX\ has
been developed by Khaled Hosny. When running a suitable \LuaTeX\ engine with Harfbuzz support, \pkg{fontspec} provides the following options:

自2019年以来,Khaled Hosny 开发了在 \LuaTeX\ 中使用 Harfbuzz 文本整形引擎的可能性。
在运行带有 Harfbuzz 支持的适当 \LuaTeX\ 引擎时,\pkg{fontspec} 提供以下选项:
\begin{itemize}
\item \feat{Renderer} = \opt{HarfBuzz} : use the Harfbuzz engine without an explicit `shaper' (the old \opt{Harfbuzz} name is kept for compatibility).
\\ \feat{Renderer} = \opt{HarfBuzz}:使用 Harfbuzz 引擎而无需显式“整形器”(旧的 \opt{Harfbuzz} 名称仍保留以保持兼容性)。
\item \feat{Renderer} = \opt{OpenType} : use the Harfbuzz engine with the OpenType shaper.
\\ \feat{Renderer} = \opt{OpenType}:使用 OpenType 整形器的 Harfbuzz 引擎。
\item \feat{Renderer} = \opt{AAT} : use the Harfbuzz engine with the AAT shaper.
\\ \feat{Renderer} = \opt{AAT}:使用 AAT 整形器的 Harfbuzz 引擎。
\item \feat{Renderer} = \opt{Graphite} : use the Harfbuzz engine with the Graphite shaper.
\\ \feat{Renderer} = \opt{Graphite}:使用 Graphite 整形器的 Harfbuzz 引擎。
\item \feat{Renderer} = \meta{foo} : use the Harfbuzz engine with the \meta{foo} shaper.
\\ \feat{Renderer} = \meta{foo}:使用 \meta{foo} 整形器的 Harfbuzz 引擎。
\end{itemize}

Support for the Harfbuzz renderer is preliminary and may be improved over time.
Please treat the interface for Harfbuzz fonts as subject to change.

对于 Harfbuzz 渲染器的支持是初步的,可能会随着时间的推移得到改进。
请将 Harfbuzz 字体的界面视为可能会改变的内容。
\section{Custom font features\\自定义字体特性}

\LuaTeX, via the \pkg{luaotfload} package, allows the definition and re-definition of custom OpenType features for a selected font. This facility is particularly useful to implement custom substitutions or to disable unwanted but not all ligatures.

通过 \pkg{luaotfload} 包,\LuaTeX\ 允许为选定的字体定义和重新定义自定义 OpenType 特性。
此功能特别有用,可用于实现自定义替换或禁用不需要但不是所有连字符。

Figure~\ref{fig:featurefile} shows an minimal example of this type of functionality.
This example creates a new OpenType feature, \texttt{oneb}, which substitutes the glyph
when typesetting `\texttt{1}' for the named glyph \texttt{one.ss01}. The glyph names
are font specific and can be interrogated with third-party software such as \emph{FontForge}.

图~\ref{fig:featurefile}展示了此类型功能的一个最简示例。该示例创建了一个新的OpenType特性\texttt{oneb},当排版`\texttt{1}'时,将其替换为名为\texttt{one.ss01}的字形。字形名称是字体特定的,可以使用第三方软件(如\emph{FontForge})来查询。

A third-party collection of additional examples are maintained in the repository `\texttt{fonts-in-luatex}'\footnote{\url{https://github.com/mewtant/fonts-in-luatex}}.
These examples are intended to correct or adjust font features in a range of commercial fonts and provide a good introduction to some of the possibilities that \LuaTeX\ affords.

第三方的额外示例集合存储在`\texttt{fonts-in-luatex}'\footnote{\url{https://github.com/mewtant/fonts-in-luatex}}中。这些示例旨在纠正或调整商业字体的特性,并为\LuaTeX\ 提供了一些可能性的良好介绍。

Please refer to the \LuaTeX/\pkg{luaotfload} documentation for more details.

有关更多详细信息,请参阅\LuaTeX/\pkg{luaotfload}文档。
% \\图中代码示例展示了自定义字体特性的示例。
\begin{figure}
\caption{An example of custom font features.\\自定义字体特性的示例。}
\label{fig:featurefile}
\hrule
\begin{Verbatim}
\documentclass{article}
\usepackage{fontspec}
\directlua{
    fonts.handlers.otf.addfeature {
        name = "oneb",
        type = "substitution",
        data = {
                ["1"] = "one.ss01",
        }
    }
}
\setmainfont{Vollkorn-Regular.otf}[RawFeature=+oneb]
\begin{document}
1234567890
\end{document}
\end{Verbatim}
\hrule
\end{figure}

\end{document}

% /©
% ------------------------------------------------
% The FONTSPEC package  <wspr.io/fontspec>
% ------------------------------------------------
% Copyright  2004-2022  Will Robertson, LPPL "maintainer"
% Copyright  2009-2015  Khaled Hosny
% Copyright  2013       Philipp Gesang
% Copyright  2013-2016  Joseph Wright
% ------------------------------------------------
% This package is free software and may be redistributed and/or modified under
% the conditions of the LaTeX Project Public License, version 1.3c or higher
% (your choice): <http://www.latex-project.org/lppl/>.
% ------------------------------------------------
% ©/

