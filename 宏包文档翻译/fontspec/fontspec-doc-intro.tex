%%^^A%%  fontspec-doc-intro.tex -- part of FONTSPEC <wspr.io/fontspec>
% !TEX root =/Users/virhuiai/hlProjects/2023_LaTeX翻译/fontspec/fontspec-document.tex
\documentclass[a4paper]{l3doc}
\usepackage{fontspec-doc-style}
\showexamplesfalse
\begin{document}

\part{Getting started\\开始}

\section{History\\历史}

This package began life as a \LaTeX\ interface to select system-installed
\MacOSX\ fonts in \name{Jonathan Kew}'s \XeTeX, the first widely-used
Unicode extension to \TeX. Over time, \XeTeX\ was extended to support OpenType
fonts and then was ported into a cross-platform program to run also on Windows
and Linux.

这个宏包最初是作为一个 \LaTeX\ 接口,用于在 \name{Jonathan Kew} 的 \XeTeX 中选择安装在系统中的 \MacOSX\ 字体,这是第一个被广泛使用的 Unicode 扩展 \TeX。随着时间的推移,\XeTeX\ 得到了扩展,支持了 OpenType 字体,然后被移植成一个跨平台程序,可以在 Windows 和 Linux 上运行。


More recently, \LuaTeX\ is fast becoming the \TeX\ engine of the day; it
supports Unicode encodings and OpenType fonts and opens up the internals of
\TeX\ via the Lua programming language. Hans Hagen's Con\TeX t Mk.\,IV is a
re-write of his powerful typesetting system, taking full advantage of
\LuaTeX's features including font support; a kernel of his work in this area
has been extracted to be useful for other \TeX\ macro systems as well, and
this has enabled \pkg{fontspec} to be adapted for \LaTeX\ when run with the
\LuaTeX\ engine.

最近,\LuaTeX\ 快速成为当今的 \TeX\ 引擎;它支持 Unicode 编码和 OpenType 字体,并通过 Lua 编程语言打开了 \TeX\ 的内部。Hans Hagen 的 Con\TeX t Mk.,IV 是他强大的排版系统的重写,充分利用了 \LuaTeX\ 的特性,包括字体支持;他在这个领域的一些工作被提取出来,可以用于其他 \TeX\ 宏系统,并且这使得 \pkg{fontspec} 在使用 \LuaTeX\ 引擎时适用于 \LaTeX。

\section{Introduction\\介绍}

The \pkg{fontspec} package allows users of either \XeTeX\ or \LuaTeX\ to
load OpenType fonts in a \LaTeX\ document. No font installation is necessary,
and font features can be selected and used as desired throughout the document.

\pkg{fontspec}包允许\LaTeX\ 文档中使用\XeTeX\ 或\LuaTeX\ 加载OpenType字体。无需安装字体,文档中可随意选择和使用字体特性。

Without \pkg{fontspec}, it is necessary to write cumbersome font definition
files for \LaTeX, since \LaTeX's font selection scheme (known as the
`\textsc{nfss}') has a lot going on behind the scenes to allow easy
commands like \cmd\emph\ or \cmd\bfseries. With an uncountable number of
fonts now available for use, however, it becomes less desirable to have to
write these font definition (|.fd|) files for every font one wishes to use.

如果没有使用 \pkg{fontspec},则需要为 \LaTeX 写冗长的字体定义文件,因为 \LaTeX 的字体选择方案(称为“\textsc{nfss}”)在幕后进行了许多处理,以允许使用简单的命令,如 \cmd\emph 或 \cmd\bfseries。然而,由于现在有无数可用的字体,因此不希望为希望使用的每个字体编写这些字体定义(|.fd|)文件。


Because \pkg{fontspec} is designed to work in a variety of modes, this
user documentation is split into separate sections that are designed to be
relatively independent. Nonetheless, the basic functionality all behaves in
the same way, so previous users of \pkg{fontspec} under \XeTeX\ should have
little or no difficulty switching over to \LuaTeX.

由于 \pkg{fontspec} 设计为在各种模式下工作,因此本用户文档分为单独的部分,这些部分被设计为相对独立。尽管如此,所有基本功能的行为方式都相同,因此之前在 \XeTeX 下使用 \pkg{fontspec} 的用户应该很容易切换到 \LuaTeX。

This manual can get rather in-depth, as there are a lot of details
to cover. See the documents \path{fontspec-example.tex} for a complete minimal example
to get started quickly.

由于本手册包含了许多详细信息,因此可能会深入到细节。要想获得完整的最小示例,以便快速入门。可以查看文档 \path{fontspec-example.tex}。

\subsection{Acknowledgements\\致谢}

This package could not have been possible without the early and continued support
the author of \XeTeX, Jonathan Kew. When I started this package, he steered
me many times in the right direction.

感谢 \XeTeX 的作者Jonathan Kew在早期和持续支持中的作用,否则本宏包无法存在。当我开始编写此软件包时,他在许多方面给予我指导。


I've had great
feedback over the years on feature requests, documentation queries, bug reports, font suggestions, and so on from lots of people all around the world.
Many thanks to you all.

多年来,我从世界各地的许多人那里得到了很好的反馈,例如功能请求、文档查询、错误报告、字体建议等等。感谢你们所有人。

Thanks to David Perry and Markus B\"ohning for numerous documentation
improvements and David Perry again for contributing the text for one of the
sections of this manual.

感谢David Perry和Markus B\"ohning为文档做出的大量改进,David Perry再次为本手册的一部分贡献了文本。

Special thanks to Khaled Hosny, who was the driving force behind the support for \LuaLaTeX, ultimately leading to version 2.0 of the package.

特别感谢Khaled Hosny,他是支持 \LuaLaTeX 的驱动力,最终导致了该软件包的2.0版本。


\section{Package loading and options\\包加载和选项}

For basic use, no package options are required:

对于基本用法,不需要任何包选项:
\begin{Verbatim}
  \usepackage{fontspec}
\end{Verbatim}
Package options will be introduced below; some preliminary details are discussed first.

包选项将在下面介绍;首先讨论一些预备细节。

\subsection{Font encodings\\字体编码}

The (default) \texttt{tuenc} package option switches the \textsc{nfss} font encoding to \texttt{TU}.
\texttt{TU} is a new Unicode font encoding, intended for both \XeTeX\ and \LuaTeX\ engines, and automatically contains support for symbols covered by \LaTeX's traditional \texttt{T1} and \texttt{TS1} font encodings (for example, |\%|, |\textbullet|, |\"u|, and so on).
Some additional features are provided by \pkg{fontspec} to customise some encoding details; see Part~\vref{part:enc} for further details.

(默认的)\texttt{tuenc}宏包选项将\textsc{nfss}字体编码切换为\texttt{TU}。
\texttt{TU}是一种新的Unicode字体编码,旨在为\XeTeX 和\LuaTeX 引擎提供支持,并自动包含\LaTeX 传统的\texttt{T1}和\texttt{TS1}字体编码所涵盖的符号的支持(例如|%|,|\textbullet|,|"u|等)。
一些附加功能由\pkg{fontspec}提供以自定义一些编码细节;更多详细信息请参见第~\vref{part:enc} 部分。

Pre-2017 behaviour can be achieved with the \texttt{euenc} package option.
This selects the \texttt{EU1} or \texttt{EU2} encoding (\XeTeX/\LuaTeX, resp.) and loads the \pkg{xunicode} package for symbol support.
Package authors and users who have referred explicitly to the encoding names \texttt{EU1} or \texttt{EU2} should update their code or documents.
(See internal variable names described in \vref{sec:api} for how to do this properly.)

可以通过使用\texttt{euenc}宏包选项来实现2017年之前的行为。
这将选择\texttt{EU1}或\texttt{EU2}编码(分别为\XeTeX/\LuaTeX),并加载\pkg{xunicode}宏包以支持符号。
已明确引用编码名称\texttt{EU1}或\texttt{EU2}的宏包作者和用户应更新其代码或文档。(有关如何正确执行此操作的方法,请参见第\vref{sec:api}节中描述的内部变量名称。)

\subsection{Maths fonts adjustments\\数学字体调整}
By default, \pkg{fontspec} adjusts \LaTeX's default maths setup in order to maintain the correct Computer Modern symbols when the roman font changes.
However, it will attempt to avoid doing this if another maths font package is loaded (such as \pkg{mathpazo} or the \pkg{unicode-math} package).

默认情况下,\pkg{fontspec}会调整\LaTeX 的默认数学设置,以保持正确的Computer Modern符号,当罗马字体发生更改时。
但是,如果加载了另一个数学字体宏包(例如\pkg{mathpazo}或\pkg{unicode-math}宏包),则它将尝试避免这样做。

If you find that \pkg{fontspec} is incorrectly changing the maths font when it shouldn't be, apply the |no-math| package option to manually suppress its behaviour here.

如果您发现\pkg{fontspec}在不应该更改数学字体的情况下错误地更改了它,请在此处应用|no-math|宏包选项以手动抑制其行为。

\subsection{Configuration\\配置}
\label{sec:config}

If you wish to customise any part of the
\pkg{fontspec} interface, this should be done by creating your own
\texttt{fontspec.cfg} file,
which will be automatically loaded if it is found by \XeTeX\ or \LuaTeX.
A |fontspec.cfg| file is distributed with \pkg{fontspec} with a small number of defaults set up within it.

如果您希望自定义\pkg{fontspec}接口的任何部分,应创建自己的\texttt{fontspec.cfg}文件,如果\XeTeX 或\LuaTeX 找到它,它将自动加载。
\pkg{fontspec}附带了一个|fontspec.cfg|文件,其中设置了一些默认值。

To customise \pkg{fontspec} to your liking, use the standard |.cfg| file as a starting point or write your own from scratch, then either place it in the same folder as the main document for isolated cases, or in a location
that \XeTeX\ or \LuaTeX\ searches by default; \eg\ in Mac\TeX: \path{~/Library/texmf/tex/latex/}.

要将\pkg{fontspec}自定义为所需的样子,请使用标准的|cfg|文件作为起点或从头开始编写自己的文件,然后将其放在与主文档相同的文件夹中,以进行隔离的情况,或者将其放在\XeTeX 或\LuaTeX 默认搜索的位置中;例如,在Mac\TeX 中:\path{~/Library/texmf/tex/latex/}。

The package option |no-config| will suppress the loading of the |fontspec.cfg| file under all circumstances.

包选项|no-config|将在任何情况下都禁止加载|fontspec.cfg|文件。

\subsection{Warnings\\警告}
\label{sec:quiet-warnings}

This package can give some warnings that can be harmless if you know what
you're doing. Use the |quiet| package option to write these warnings to the
transcript (\texttt{.log}) file instead.

这个包可能会产生一些警告,如果你知道你在做什么,这些警告是无害的。你可以使用 |quiet| 包选项将这些警告写入转录 (\texttt{.log}) 文件中。

Use the |silent| package option to completely suppress these warnings if you
don't even want the |.log| file cluttered up.

如果你甚至不想让 |.log| 文件变得混乱,你可以使用 |silent| 包选项来完全抑制这些警告。

\section{Interaction with \LaTeXe\ and other packages\\与\LaTeXe 和其他软件包的交互}

This section documents some areas of adjustment that \pkg{fontspec} makes
to improve default behaviour with \LaTeXe\ and third-party packages.

这一节介绍了 \pkg{fontspec} 对 \LaTeXe\ 和第三方包的默认行为进行了一些调整,以改善其效果。

\subsection{Commands for old-style and lining numbers\\旧式数字和直立数字的命令}

\DescribeMacro{\oldstylenums}
\DescribeMacro{\liningnums}
\LaTeX's definition of \cs{oldstylenums} relies on strange font encodings.
We provide a \pkg{fontspec}-compatible alternative and while we're at it
also throw in the reverse option as well. Use \cs{oldstylenums}\marg{text}
to explicitly use old-style (or lowercase) numbers in \meta{text}, and
the reverse for \cs{liningnums}\marg{text}.

\LaTeX 的 \cs{oldstylenums} 定义依赖于奇怪的字体编码。 我们提供了一个与 \pkg{fontspec} 兼容的替代方案,顺便也加入了反向选项。使用 \cs{oldstylenums}\marg{text} 来显式地在 \meta{text} 中使用旧式(或小写)数字,反之亦然。

\subsection{Italic small caps\\斜体小型大写字母}

Support now provided by \LaTeXe\ in 2020.

在2020年由\LaTeXe 支持。


\subsection{Emphasis and nested emphasis\\强调和嵌套强调}

Support now provided by \LaTeXe\ in 2020.

在2020年由\LaTeXe 支持。

\subsection{Strong emphasis\\强烈强调}

\DescribeMacro{\strong}
\DescribeMacro{\strongenv}
The \cs{strong} macro is used analogously to \cs{emph} but produces variations in weight.
If you need it in environment form, use |\begin{strongenv}...\end{strongenv}|.

\cs{strong}宏类似于\cs{emph},但产生字体粗细的变化。如果您需要使用环境形式,请使用|\begin{strongenv}...\end{strongenv}|。

As with emphasis, this font-switching command is intended to move through a range
of font weights. For example, if the fonts are set up correctly it allows usage such as
|\strong{...\strong{...}}| in which each nested \cs{strong} macro increases the
weight of the font.

与强调一样,该字体切换命令旨在移动到一系列字体重量中。例如,如果正确设置了字体,则允许使用如下用法:|\strong{...\strong{...}}|,其中每个嵌套的 \cs{strong} 宏都增加了字体的重量。

\DescribeMacro{\strongfontdeclare}
Currently this feature set is somewhat experimental and there is no syntactic sugar
to easily define a range of font weights using \pkg{fontspec} commands.
Use, say, the following to define first bold and then black (|k|) font faces for \cs{strong}:

目前,该功能集有点实验性,没有使用\pkg{fontspec}命令轻松定义一系列字体权重的简洁语法。例如,使用以下内容定义 \cs{strong} 的第一个粗体,然后是黑体(|k|)字体面:

\begin{Verbatim}
  \strongfontdeclare{\bfseries,\fontseries{k}\selectfont}
\end{Verbatim}

\DescribeMacro{\strongreset}
If too many levels of \cs{strong} are reached, \cs{strongreset} is inserted.
By default this is a no-op and the font will simply remain the same.
Use \cs{renewcommand}\cs{strongreset}|{\mdseries}| to start again from the beginning if desired.

如果达到太多级别的\cs{strong},则插入\cs{strongreset}。默认情况下,这是一个无操作,并且字体将保持不变。如果需要,使用\cs{renewcommand}\cs{strongreset}|{\mdseries}|从头开始再次启动。

An example for setting up a font family for use with \cs{strong} is discussed in \vref{sec:strong-example}.

有关为使用\cs{strong}设置字体系列的示例,请参见\ref{sec:strong-example}。


\end{document}

% /©
% ------------------------------------------------
% The FONTSPEC package  <wspr.io/fontspec>
% ------------------------------------------------
% Copyright  2004-2022  Will Robertson, LPPL "maintainer"
% Copyright  2009-2015  Khaled Hosny
% Copyright  2013       Philipp Gesang
% Copyright  2013-2016  Joseph Wright
% ------------------------------------------------
% This package is free software and may be redistributed and/or modified under
% the conditions of the LaTeX Project Public License, version 1.3c or higher
% (your choice): <http://www.latex-project.org/lppl/>.
% ------------------------------------------------
% ©/
