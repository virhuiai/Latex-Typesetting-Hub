%%^^A%%  fontspec-doc-xetex.tex -- part of FONTSPEC <wspr.io/fontspec>

\documentclass[a4paper]{l3doc}
\usepackage{fontspec-doc-style}
\showexamplesfalse
\begin{document}

\part{Fonts and features with \XeTeX\\使用 \XeTeX 的字体和特性}
\label{sec:xetex-features}

\section{\XeTeX-only font features\\\XeTeX 专有的字体特性}

The features described here are available for any font
selected by \pkg{fontspec}.

此处描述的特性适用于任何由 \pkg{fontspec} 选定的字体。

\subsection{Mapping\\映射}
\label{sec:mapping}

The \feat{Mapping} feature enables a \XeTeX\ text-mapping scheme, with an example shown in \exref{mapping}.

\feat{Mapping} 特性启用了 \XeTeX 的文本映射方案,如 \exref{mapping} 中所示。


\begin{Xexample}{mapping}{\XeTeX's \feat{Mapping} feature.\\\XeTeX 的 \feat{Mapping} 特性示例。}
  \fontspec{texgyrepagella-regular.otf}[Mapping=tex-text]
  ``!`A small amount of---text!''
\end{Xexample}

Only one mapping can be active at a time and a second call to \feat{Mapping}
will override the first.
Using the |tex-text| mapping is also equivalent to writing |Ligatures=TeX|.
The use of the latter syntax is recommended for better compatibility with
\LuaTeX\ documents.

一次只能激活一个映射,第二次调用 \feat{Mapping} 将覆盖第一次调用。
使用 |tex-text| 映射等价于写 |Ligatures=TeX|。建议使用后者的语法,以更好地兼容 \LuaTeX 文档。

\subsection{Different font technologies: \AAT, OpenType, and Graphite\\不同的字体技术:\AAT、OpenType 和 Graphite}\label{sec:renderer-xetex}

\textbf{Note that from 2020 it appears that \XeTeX\ can no longer support \AAT\ fonts in \MacOSX.\\请注意,从2020年开始,在 \MacOSX 上似乎已不再支持 \AAT 字体。}

\XeTeX\ supports three rendering technologies for typesetting, selected with
the \feat{Renderer} font feature. The first, \opt{AAT}, is
that provided only by \MacOSX.
The second, \opt{OpenType}, is an open source OpenType interpreter.
It provides greater support for OpenType features, notably contextual arrangement, over \opt{AAT}.
The third is \opt{Graphite}, which is an alternative to OpenType with particular features for less-common languages and the capability for more powerful font options.
Features for \opt{OpenType} have already been discussed in \vref{sec:opentype-features};
\opt{Graphite} and \opt{AAT} features are discussed later in \vref{sec:graphite-features} and \vref{sec:aat-features}.

\XeTeX 支持三种排版渲染技术,可通过 \feat{Renderer} 字体特性选择。第一种是 \opt{AAT},仅由 \MacOSX 提供。第二种是开源的 OpenType 解释器 \opt{OpenType}。它提供了更好的 OpenType 特性支持,尤其是上下文排版功能,超过了 \opt{AAT}。第三种是 \opt{Graphite},是 OpenType 的替代品,具有特定的支持较少见语言的特性,以及更强大的字体选项功能。\opt{OpenType} 特性已在 \vref{sec:opentype-features} 中讨论;\opt{Graphite} 和 \opt{AAT} 特性将在 \vref{sec:graphite-features} 和 \vref{sec:aat-features} 中讨论。

Unless you have a particular need, the \feat{Renderer} feature is rarely explicitly required: for OpenType
fonts, the \opt{OpenType} renderer is used automatically, and for \AAT\ fonts,
\opt{AAT} is chosen by default. Some fonts, however, will contain font tables
for multiple rendering technologies, such as the Hiragino Japanese fonts
distributed with \MacOSX, and in these cases one over the other may be preferred.

除非您有特定需求,否则通常不需要显式指定 \feat{Renderer} 特性:对于 OpenType 字体,将自动使用 \opt{OpenType} 渲染器,而对于 \AAT 字体,默认选择 \opt{AAT}。然而,一些字体将包含多个渲染技术的字体表,例如随 \MacOSX 发行的 Hiragino 日文字体,在这些情况下,可能会更喜欢其中之一。

Among some other font features only available through a specific renderer,
\opt{OpenType} provides for the \feat{Script} and \feat{Language} features, which allow
different font behaviour for different alphabets and languages; see \vref{sec:ot}
for the description of these features. {\em Because these font features can
change which features are able to be selected for the font instance, they are selected
by \pkg{fontspec} before all others and will automatically and without warning
select the \opt{OpenType} renderer.}

在某些其他字体特性中,只有通过特定渲染器才能使用,\opt{OpenType} 提供了 \feat{Script} 和 \feat{Language} 特性,允许在不同的字母表和语言中使用不同的字体行为;详见\vref{sec:ot} 中对这些特性的描述。{\em 因为这些字体特性可能会改变可以选择的字体实例的特性,所以它们会在所有其他特性之前由 \pkg{fontspec} 选择,并且将自动且没有警告地选择 \opt{OpenType} 渲染器。}


\subsection{Optical font sizes\\光学字体大小} \label{sec:aat-opticalsize}

Multiple Master fonts are parameterised over
orthogonal font axes, allowing continuous selection along such
features as weight, width, and optical size.
Whereas an OpenType font will have only a few separate
optical sizes, a Multiple Master font's optical size can be
specified over a continuous range. Unfortunately, this flexibility makes
it harder to create an automatic interface through \LaTeX, and the
optical size for a Multiple Master font must always be specified
explicitly.

多主字体是在正交字体轴上参数化的,允许沿着这些特性,例如重量、宽度和光学大小进行连续选择。而 OpenType 字体只有几个独立的光学大小,多主字体的光学大小可以在连续的范围内指定。不幸的是,这种灵活性使得通过 \LaTeX 自动创建接口变得更加困难,并且必须始终明确指定多主字体的光学大小。
\begin{Verbatim}
  \fontspec{Minion MM Roman}[OpticalSize=11]
   MM optical size test                    \\
  \fontspec{Minion MM Roman}[OpticalSize=47]
   MM optical size test                    \\
  \fontspec{Minion MM Roman}[OpticalSize=71]
   MM optical size test                    \\
\end{Verbatim}


\subsection{Vertical typesetting\\纵向排版}

\XeTeX\ provides for vertical typesetting simply with the ability to rotate
the individual glyphs as a font is used for typesetting, as shown in
\exref{vert}.

\XeTeX\ 提供了纵向排版的简单方法,只需在用于排版的字体中旋转单个字形即可,如 \exref{vert} 所示。

\begin{Xexample}[firstline=2]{vert}{Vertical typesetting.\\纵向排版。}
  \def\verttext{共産主義者は}
  \fontspec{Hiragino Mincho Pro}
  \verttext

  \fontspec{Hiragino Mincho Pro}[Renderer=AAT,Vertical=RotatedGlyphs]
  \rotatebox{-90}{\verttext}% requires the graphicx package
\end{Xexample}

No actual provision is made for typesetting top-to-bottom
languages; for an example of how to do this, see the vertical Chinese
example provided in the \XeTeX\ documentation.

实际上没有提供排版自上而下的语言的功能;有关如何实现这一点的示例,请参见 \XeTeX\ 文档中提供的纵向中文示例。
\section{The Graphite renderer\\Graphite 渲染器}
\label{sec:graphite-features}

Since the Graphite renderer is designed for less common scripts and languages, usually with
specific or unique requirements, Graphite features are not standard across fonts.

由于 Graphite 渲染器设计用于不太常见的书写和语言,通常具有特定或独特的要求,因此字体的 Graphite 特性在各种字体之间是不标准的。

Currently \pkg{fontspec} does not support a convenient interface to select Graphite font
features and all selection must be done via `raw' font feature selection.

目前,\pkg{fontspec} 不支持方便的界面选择 Graphite 字体特性,所有选择都必须通过“原始”字体特性选择完成。

Here's an example:

以下是一个例子
\begin{Verbatim}
  \fontspec{Charis SIL}[
    Renderer=Graphite,
    RawFeature={Uppercase Eng alternates=Large eng on baseline}]
  Ŋ
\end{Verbatim}

Here's another:

以下是另一个例子:

\begin{Verbatim}
\fontspec{AwamiNastaliq-Regular.ttf}[Renderer=Graphite] ^^^^06b5
\addfontfeature{RawFeature={Lam with V=V over bowl}}    ^^^^06b5
\end{Verbatim}

\section{\MacOSX's \AAT\ fonts\\\MacOSX 的 \AAT 字体}
\label{sec:aat-features}

\begin{quote}\itshape
\textbf{Warning!\\警告!}
\XeTeX's implementation on \MacOSX\ is currently in a state of flux and the information contained below may well be wrong from 2013 onwards.
There is a good chance that the features described in this section will not be available any more as \XeTeX's completes its transition to a cross-platform--only application.
All examples in this section have now been removed.

\XeTeX 在 \MacOSX 上的实现目前处于变化状态,下面的信息从2013年开始可能已经不正确。
随着 \XeTeX 完成其过渡到跨平台应用程序,这一节描述的功能很可能不再可用。
本节中的所有示例现已被删除。
\end{quote}

\MacOSX's font technology began life before the ubiquitous-OpenType era
and revolved around the Apple-invented `\AAT' font format. This format
had some advantages (and other disadvantages) but it never became widely
popular in the font world.

\MacOSX 的字体技术在普及的 OpenType 时代之前诞生,并且围绕着由苹果发明的“\AAT”字体格式。这种格式有一些优点(以及其他缺点),但在字体世界中并没有广泛流行。

Nonetheless, this is the font format that was first supported by \XeTeX\
(due to its pedigree on \MacOSX\ in the first place) and was the first
font format supported by \pkg{fontspec}. A number of fonts distributed with
\MacOSX\ are still in the \AAT\ format, such as `Skia'.

尽管如此,这是 \XeTeX 最早支持的字体格式(由于它在 \MacOSX 上的渊源),也是 \pkg{fontspec} 支持的第一个字体格式。一些随 \MacOSX 分发的字体仍然是 \AAT 格式,例如“Skia”。
\subsection{Ligatures\\连字}

\feat{Ligatures} refer to the replacement of two separate characters
with a specially drawn glyph for functional or \ae sthetic reasons.
For \AAT\ fonts, you may choose from any combination of \opt{Required},
\opt{Common}, \opt{Rare} (or \opt{Discretionary}), \opt{Logos}, \opt{Rebus},
\opt{Diphthong}, \opt{Squared}, \opt{AbbrevSquared}, and \opt{Icelandic}.

\feat{连字}是指出于功能或美观原因替换两个独立字符为特殊绘制的字形。对于 \AAT 字体,您可以选择任何组合的 \opt{Required}、\opt{Common}、\opt{Rare}(或 \opt{Discretionary})、\opt{Logos}、\opt{Rebus}、\opt{Diphthong}、\opt{Squared}、\opt{AbbrevSquared} 和 \opt{Icelandic}。

Some other Apple \AAT\ fonts have those `Rare' ligatures contained in
the \opt{Icelandic} feature. Notice also that the old \TeX\ trick of
splitting up a ligature with an empty brace pair does not work in
\XeTeX; you must use a 0\,pt kern or \cs{hbox} (\eg, \cs{null}) to
split the characters up if you do not want a ligature to be performed (the usual examples for when this might be desired are words like `shelf\null full').

一些其他的 Apple \AAT 字体包含在 \opt{Icelandic} 功能中的这些“Rare”连字。请注意,旧版的 \TeX\ 技巧使用空的花括号对来分解连字在 \XeTeX 中不起作用;如果您不希望执行连字,则必须使用 0,pt kern 或 \cs{hbox}(例如 \cs{null})来分解字符(通常需要执行这种操作的例子是像“shelf\null full”这样的单词)。

\subsection{Letters\\字母} \label{sec:aat-letters}
The \opt{Letters} feature specifies how the letters in the current font
will look. For \AAT\ fonts, you may choose from \opt{Normal},
\opt{Uppercase}, \opt{Lowercase}, \opt{SmallCaps}, and
\opt{InitialCaps}.

\opt{Letters} 功能指定当前字体中字母的外观。对于 \AAT 字体,您可以选择 \opt{Normal}、\opt{Uppercase}、\opt{Lowercase}、\opt{SmallCaps} 和 \opt{InitialCaps}。

\subsection{Numbers\\数字}
The \feat{Numbers} feature defines how numbers will look in the
selected font. For \AAT\ fonts, they may be a
combination of \opt{Lining} or \opt{OldStyle} and \opt{Proportional} or
\opt{Monospaced} (the latter is good for tabular material). The synonyms
\opt{Uppercase} and \opt{Lowercase} are equivalent to \opt{Lining} and
\opt{OldStyle}, respectively. The differences have been shown previously
in \vref{sec:addfontfeatures}.

\feat{数字}特性定义了在所选字体中数字的外观。对于 \AAT\ 字体,它们可以是\opt{Lining}或\opt{OldStyle}以及\opt{Proportional}或\opt{Monospaced}的组合(后者适用于表格材料)。同义词\opt{Uppercase}和\opt{Lowercase}分别等同于\opt{Lining}和\opt{OldStyle}。这些差异在\vref{sec:addfontfeatures}中已经展示过了。

\subsection{Contextuals\\上下文相关} \label{sec:contextuals}
This feature refers to glyph substitution that vary by their position;
things like contextual swashes are implemented here.
The options for \AAT\ fonts are
\opt{WordInitial}, \opt{WordFinal} (\exref{wordcx}), \opt{LineInitial},
\opt{LineFinal}, and \opt{Inner} (\exref{longsaat}, also called `non-final' sometimes). As
non-exclusive selectors, like the ligatures, you can turn them off
by prefixing their name with \opt{No}.

该特性涉及按位置变化的字形替换;例如实现上下文花押的就在这里。\AAT\ 字体的选项包括\opt{WordInitial}、\opt{WordFinal}(\exref{wordcx})、\opt{LineInitial}、\opt{LineFinal}和\opt{Inner}(\exref{longsaat},有时也称为“非终止”)。像连字一样,它们也可以通过在它们的名称前缀中添加\opt{No}来关闭。

\subsection{Vertical position\\垂直位置}
The \feat{VerticalPosition} feature is used to access things like
subscript (\opt{Inferior}) and superscript (\opt{Superior}) numbers and
letters (and a small amount of punctuation, sometimes).
The \opt{Ordinal} option is (supposed to be)
contextually sensitive to only raise characters that appear directly
after a number.

\feat{VerticalPosition}特性用于访问上下标(\opt{Inferior}和\opt{Superior})数字、字母(有时也包括少量标点符号)。\opt{Ordinal}选项是(应该是)仅在数字之后直接出现的字符上下文敏感的。

The \pkg{realscripts} package redefines the \cmd\textsubscript\ and
\cmd\textsuperscript\ commands to use the above font features,
including for use in footnote labels.

\pkg{realscripts}宏包重新定义了\cmd\textsubscript\ 和 \cmd\textsuperscript\ 命令,使用上述字体特性,包括用于脚注标签。

\subsection{Fractions\\分数}
Many fonts come with the capability to typeset various forms of
fractional material. This is accessed in \pkg{fontspec} with the
\feat{Fractions} feature, which may be turned \opt{On} or \opt{Off}
in both \AAT\ and OpenType fonts.

许多字体都具有排版各种形式的分数材料的能力。这可以通过在\pkg{fontspec}中使用\feat{Fractions}特性来访问,可以在\AAT\ 和 OpenType 字体中将其打开(\opt{On})或关闭(\opt{Off})。


In \AAT\ fonts, the `fraction slash' or solidus character, is
to be used to create fractions. When \feat{Fractions} are turned
\opt{On}, then only pre-drawn fractions will be used.

在 \AAT\ 字体中,将使用“分数斜杠”或实线符号来创建分数。当打开 \feat{Fractions}时,只使用预绘制的分数。

Using the \opt{Diagonal} option (\AAT\ only), the font will attempt
to create the fraction from superscript and subscript
characters.

使用\opt{Diagonal}选项(仅限 \AAT ),字体将尝试从上下标字符中创建分数。

Some (Asian fonts predominantly) also provide for the
\opt{Alternate} feature.

一些(主要是亚洲字体)还提供了\opt{Alternate}特性。

\subsection{Variants\\变体}
The \feat{Variant} feature takes a single numerical input for
choosing different alphabetic shapes.
See \vref{sec:newfeatures} for a way to assign names to variants,
which should be done on a per-font basis.

\feat{Variant}特性采用单个数字输入来选择不同的字母形状。有关将变体命名的方法,请参见\vref{sec:newfeatures},这应该根据字体而定。

\subsection{Alternates\\替代}
Selection of \feat{Alternate}s again must be done numerically.
See \vref{sec:newfeatures} for a way to assign names to alternates,
which should be done on a per-font basis.

选择\feat{Alternate}必须再次通过数字进行。有关将备用项命名的方法,请参见\vref{sec:newfeatures},这应该根据字体而定。


\subsection{Style\\样式}
The options of the \feat{Style} feature
are defined in \AAT\ as one of the following: \opt{Display},
\opt{Engraved}, \opt{IlluminatedCaps}, \opt{Italic},
\opt{Ruby},\footnotemark\ \opt{TallCaps}, or \opt{Titling}.
\footnotetext{`Ruby' refers to a small optical size, used in
Japanese typography for annotations.}

\feat{Style} 功能的选项在 \AAT\ 中被定义为以下之一:\opt{Display}、\opt{Engraved}、\opt{IlluminatedCaps}、\opt{Italic}、\opt{Ruby}、\opt{TallCaps} 或 \opt{Titling}。

Typical examples for these features are shown in \ref{sec:ot-feat-style}.

这些功能的典型示例显示在 \ref{sec:ot-feat-style} 中。

\subsection{CJK shape\\CJK 形状}
There have been many standards for how CJK ideographic
glyphs are `supposed' to look. Some fonts will contain many alternate
glyphs in order to be able to display these gylphs
correctly in whichever form is appropriate. Both \AAT\ and OpenType
fonts support the following \feat{CJKShape} options:
\opt{Traditional}, \opt{Simplified}, \opt{JIS1978}, \opt{JIS1983},
\opt{JIS1990}, and \opt{Expert}. OpenType also supports the \opt{NLC} option.

有许多标准规定了 CJK 表意文字的“应该”样式。一些字体包含许多替代字形,以便能够以适当的形式正确显示这些字形。\AAT\ 和 OpenType 字体都支持以下 \feat{CJKShape} 选项:\opt{Traditional}、\opt{Simplified}、\opt{JIS1978}、\opt{JIS1983}、\opt{JIS1990} 和 \opt{Expert}。OpenType 还支持 \opt{NLC} 选项。
\subsection{Character width\\字符宽度}
See \vref{sec:CharacterWidth} for relevant examples; the features are
the same between OpenType and \AAT\ fonts.
\AAT\ also allows \feat{CharacterWidth}|=|\opt{Default} to return to
the original font settings.

相关示例请参见 \vref{sec:CharacterWidth};OpenType 和 \AAT\ 字体之间的特性是相同的。\AAT\ 还允许 \feat{CharacterWidth}|=|\opt{Default} 返回到原始字体设置。

\subsection{Diacritics\\变音符号}
Diacritics are marks, such as the acute accent or the tilde, applied to letters; they usually indicate a change in pronunciation.
In Arabic scripts, diacritics are used to indicate vowels.
You may either choose
to \opt{Show}, \opt{Hide} or \opt{Decompose} them in \AAT\ fonts.
The \opt{Hide} option is for scripts such as Arabic which may be
displayed either with or without vowel markings. E.g.,
\verb|\fontspec[Diacritics=Hide]{...}|

变音符号是应用于字母的标记,例如锐音符号或波浪符号,它们通常表示发音的变化。在阿拉伯语书写中,变音符号用于表示元音。在 \AAT\ 字体中,您可以选择 \opt{Show}、\opt{Hide} 或 \opt{Decompose}。 \opt{Hide} 选项适用于像阿拉伯语这样可以显示带或不带元音标记的书写系统。例如,\verb|\fontspec[Diacritics=Hide]{...}|。


Some older fonts distributed with \MacOSX\ included `|O/|' \etc\ as shorthand for writing `\O' under the label of the \feat{Diacritics} feature. If you come across such fonts, you'll
want to turn this feature off (imagine typing |hello/goodbye| and
getting `hell\o goodbye' instead!) by decomposing the two characters
in the diacritic into the ones you actually
want. I recommend using
the proper \LaTeX\ input conventions for obtaining such characters
instead.

某些旧版的配有 \MacOSX\ 的字体使用 |O/|' 等作为书写 \feat{Diacritics} 功能标签下 \O' 的速记符号。如果您遇到这样的字体,您需要将此功能关闭(想象一下输入 |hello/goodbye| 却得到 `hell\o goodbye' 的结果!),通过将变音符号中的两个字符分解成您实际需要的字符。我建议使用适当的 \LaTeX\ 输入约定来获取这样的字符。


\subsection{Annotation\\注释}
Various Asian fonts are equipped with a more extensive range of
numbers and numerals in different forms. These are accessed through
the \feat{Annotation} feature with the following
options: \opt{Off},
\opt{Box}, \opt{RoundedBox}, \opt{Circle}, \opt{BlackCircle},
\opt{Parenthesis}, \opt{Period}, \opt{RomanNumerals}, \opt{Diamond},
\opt{BlackSquare}, \opt{BlackRoundSquare}, and \opt{DoubleCircle}.

各种亚洲字体配备了更广泛的不同形式的数字和数字符号。这些是通过 \feat{Annotation} 功能访问的,具有以下选项:\opt{Off}、\opt{Box}、\opt{RoundedBox}、\opt{Circle}、\opt{BlackCircle}、\opt{Parenthesis}、\opt{Period}、\opt{RomanNumerals}、\opt{Diamond}、\opt{BlackSquare}、\opt{BlackRoundSquare} 和 \opt{DoubleCircle}。

\end{document}

% /©
% ------------------------------------------------
% The FONTSPEC package  <wspr.io/fontspec>
% ------------------------------------------------
% Copyright  2004-2022  Will Robertson, LPPL "maintainer"
% Copyright  2009-2015  Khaled Hosny
% Copyright  2013       Philipp Gesang
% Copyright  2013-2016  Joseph Wright
% ------------------------------------------------
% This package is free software and may be redistributed and/or modified under
% the conditions of the LaTeX Project Public License, version 1.3c or higher
% (your choice): <http://www.latex-project.org/lppl/>.
% ------------------------------------------------
% ©/
