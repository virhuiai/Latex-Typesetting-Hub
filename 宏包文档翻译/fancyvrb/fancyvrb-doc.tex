\documentclass[twoside]{article}
\usepackage[heading=true]{ctex}
\usepackage{cprotect}
\usepackage{libertinus-otf}
\setmonofont[Scale=MatchLowercase,FakeStretch=0.9]{AnonymousPro-Regular.ttf}
\newif\ifChangeBar \ChangeBarfalse
\usepackage{fancyvrb}
\usepackage{fancyvrb-ex}
\usepackage{fancyhdr}
\pagestyle{fancy}
\fancyhead[RE,LO]{}
\fancyhead[RO]{\leftmark}

\usepackage{xurl}

\usepackage[colorlinks,linktocpage]{hyperref}
\parskip=3pt plus 1pt minus 1pt
%

\newcommand\FBoxPackage{`\textsf{fancybox}'}
\newcommand\FVrbPackage{`\textsf{fancyvrb}'}
\newcommand\cs[1]{\texttt{\textbackslash#1}}
%
%  From ltugboat.cls
%
%  Typeset the name of an environment
\providecommand\env[1]{\textsf{#1}}
\providecommand\clsname[1]{\textsf{#1}}
\providecommand\pkgname[1]{\textsf{#1}}
\providecommand\optname[1]{\textsf{#1}}
\providecommand\progname[1]{\textsf{#1}}
%
%  A list of options for a package/class
\newenvironment{optlist}{\begin{description}%
  \renewcommand\makelabel[1]{%
    \descriptionlabel{\mdseries\optname{##1}}}%
  \itemsep0.25\itemsep}%
 {\end{description}}
%
%  Utility macros
%
%  Special dashes

\makeatletter
\def\thinskip{\hskip 0.16667em\relax}
\def\endash{--}
\def\emdash{\endash-}
\def\d@sh#1#2{\unskip#1\thinskip#2\thinskip\ignorespaces}
\def\dash{\d@sh\nobreak\endash}
\def\Dash{\d@sh\nobreak\emdash}
\makeatother
\newif\ifPostScriptFonts \PostScriptFontstrue
%
\setlength{\columnseprule}{0.6pt}

\newcommand\verbatimTxt{抄录}
%
\newif\ifmulticols
%\IfFileExists{multicol.sty}{\multicolstrue}{}
\newif\ifpstricks
\pstrickstrue
%
\title{The `\textsf{fancyvrb}' package\\Fancy Verbatims in \LaTeX}
\author{Timothy Van Zandt\\Princeton University\\Princeton -- USA\\[5mm]
%        {\footnotesize \url{tvz@Princeton.EDU}}\\[5mm]
        {Packaging and documentation by}\\
        {\footnotesize Denis Girou (CNRS/IDRIS -- France),}\\[-2mm]
        {\footnotesize Sebastian Rahtz\dagger\ (Elsevier -- GB)}\\[-2mm]
        {\footnotesize and}\\[-2mm]
        {\footnotesize Herbert Voß\ (FU Berlin -- DE)}\\[-2mm]
        {\footnotesize 翻译:virhuiai@qq.com}
        }
\date{\today}

\fvset{frame=single,xrightmargin=0cm,numbers=left,numbersep=3pt,commentchar=Z}

\begin{document}
\maketitle
%
\begin{abstract}
\textsf{fancyvrb}宏包提供了高度可配置的抄录环境。用户可以对抄录环境做的有:修改
字体、修改字号、为抄录内容加上行号、使用带边框的代码示例环境、高亮抄录的文本、 以及有条件的处理抄录文本等等。
\end{abstract}

\begin{center}
Report bugs to \url{hvoss@tug.org}
\end{center}

\vfill
Thanks to 
Claudio Beccari,
Jean-François Burnol, 
Philippe Esperet, %\url{esperet@marie.polytechnique.fr}, 
Michael Friendly, %\url{friendly@hotspur.psych.yorku.ca}, 
Mario Hassler, %\url{HASSLER@ippnv2.ipp.kfa-juelich.de}, 
Ohad Kammar, % \url{ohad.kammar@ed.ac.uk},
Mikhail Kolodin, %\url{myke@morrigan.spb.su}, 
Andreas Matthias,
Frank Mittelbach,
Rolf Niepraschk, %\url{Rolf.Niepraschk@gmx.de}, 
Will Robertson,
Ulrich M. Schwarz, 
Thomas Siegel, % \url{siegel@aix520.informatik.uni-leipzig.de}, 
Clemens Steinke,
and 
Vladimir Volovich. %\url{vvv@vvv.vsu.ru}.
\clearpage
\tableofcontents
\clearpage
%\part{\texttt{fancyvrb} package by Timothy Van Zandt}


\part{Package \texttt{fancyvrb}}


\section{介绍}
%
  \FVrbPackage{} is the development of the \emph{verbatim} macros of the
\FBoxPackage{} package,  Section 11 of \cite{FancyBox}. 同标准的 \LaTeX{}
\textsf{verbatim} 环境相比, \FVrbPackage{} 提供了6种拓展功能,:

\begin{enumerate}
  \item {\em\verbatimTxt}命令可以在脚注中使用,
  \item 一些{\verbatimTxt}命令得到了加强,
  \item 宏包自带了一些{\verbatimTxt}环境, 配有许多参数来控制内容的输出格式;它也让你能够更方便的定义新的符合自身需求的{\verbatimTxt}环境,
  \item 提供了保存和取用{\verbatimTxt}文本和环境方法,
  \item 在{\verbatimTxt}模式提供了命令以便读写文件, 这是很有用的。
  \item 你可以 build \emph{example} 环境 (同时展示出排版的结果和所使用的排版的代码), 
  with the same versatility as normal verbatim.%TODO
\end{enumerate}

% 该软件包通过从环境或文件中一次扫描一行来工作。
该宏包对抄录环境或从文件中读取文本是,每次只扫描一行。 %
这样的方式允许它 to pre-process each line, rejecting it, 移除空格, numbering it, etc, 
before going on to execute the body of the line with the appropriate catcodes set.

\section{脚注中的Verbatim material}

  调用 \cs{VerbatimFootnotes} 宏声明之后 (在the preamble之后使用),
就可以将 verbatim 命令和环境
(the \LaTeX{} or \FVrbPackage{} ones) 放到脚注中了, 而标准的
\LaTeX 中是不允许这样做的:

\begin{Example}
  \VerbatimFootnotes
  我们可以将抄录\footnote{\verb+_Yes!_+} 文本放到脚注中了.
\end{Example}

\section{改进的{\verbatimTxt}命令}
%
\cs{DefineShortVerb}命令让我们可以指定一个符号作为抄录命令的简化方式,这个符号包围住要{\verbatimTxt}的文本。命令 \cs{UndefineShortVerb} 
用来取消这个效果 (%
在 \LaTeX{} 的 `\textsf{shortvrb}' 宏包中也提供了这两个命令):

\fvset{xrightmargin=4.8cm}

\begin{SideBySideExample}
  \DefineShortVerb{\|}
  We can simply write \Verb+_verbatim_+
  material using a single |_delimiter_|
  \UndefineShortVerb{\|}
  \DefineShortVerb{\+}
  And we can +_change_+ the character.
\end{SideBySideExample}

更强大的(使排版效果多多样化)是, 我们可以指定在抄录文本中的 \emph{escape} (转义)字符%
 (using the \cs{Verb} macro or with a `shortverb' character
defined), 以执行类似格式化这样内容, 请使用在\ref{sec:commandchars} 中介绍的,抄录环境的\texttt{commandchars} 参数 .

\section{Verbatim 环境}

提供了不少的 {\verbatimTxt} 环境, 都有一堆的参数可以用来配置。
下面的例子我使用了 \texttt{Verbatim} 环境, 它相当于标准 \LaTeX{} 的 \texttt{verbatim} 环境。
配置参数可以使用 \cs{fvset} 命令全局设置,
也可以作为可选参数在环境的开始处指定\footnote{For clarification in this paper, note that we
generally indent each verbatim line with two
spaces.}$^,$\footnote{This mechanism uses the
`\textbf{keyval}' package from the standard \LaTeX{} graphics
distribution, written by David \textsc{Carlisle}.}.

\begin{SideBySideExample}
  \begin{Verbatim}
    First verbatim line.
    Second verbatim line.
  \end{Verbatim}
\end{SideBySideExample}

\subsection{{\verbatimTxt}环境的设置}

{\verbatimTxt}环境的输出效果,有不少是可以定制的; 下面列出我们可以设置的参数。

\subsubsection{注释}

\begin{optlist}
  \item[commentchar (字符)]: 定义{\verbatimTxt}环境中的注释字符,
若某行以此字符开头,在输出中你整行都不会看到 (\emph{默认: empty})。
\end{optlist}

\begin{SideBySideExample}
  \begin{Verbatim}[commentchar=!]
    A comment
    Verbatim line.!一
    ! 二、这行注释在输出中你是看不到的
  \end{Verbatim}
\end{SideBySideExample}

  要注意一种特殊的情况,当注释符不是第一个非空白字符时: 这是因为我们定义的这个注释符是被等同于 \TeX{} 的注释符了, 这意味着换行符也会被忽略. 因此,这种情况下, 当前行和下一行会合为一行了, 另外, 如果最后一行包含注释符,最后一行的内容也不会输出, 因为 \FVrbPackage{} 宏包的每一行的输出是在遇到结束符才执行的,而最后一行包含注释符这个情况下永远不会遇到结束符\ldots

\begin{SideBySideExample}
  \begin{Verbatim}[commentchar=\%]
    第一行. % First line
    第二行.
    第三行. % Third line lost...
  \end{Verbatim}
\end{SideBySideExample}

\begin{SideBySideExample}
  \begin{Verbatim}[commentchar=\%]
    第一行. 
    第二行. 
    最后一行包含注释符. % 不输出
  \end{Verbatim}
\end{SideBySideExample}

\subsubsection{忽略起始的字符数量}

\begin{optlist}
  \item[gobble (integer)]: 每行开始跳过的字符数 (最大值为9), 
这在包含有空格缩进的内容时很有用%
(\emph{默认: empty} \Dash 没有字符被忽略)。
\end{optlist}

\begin{SideBySideExample}
  \begin{Verbatim}
    Verbatim line.
  \end{Verbatim}

  \begin{Verbatim}[gobble=2]
    Verbatim line.
  \end{Verbatim}

  \begin{Verbatim}[gobble=8]
    Verbatim line.
  \end{Verbatim}
\end{SideBySideExample}

\subsubsection{指定修饰格式}

\begin{optlist}
  \item[formatcom (命令)]: 指定输出 {\verbatimTxt} 文本之前运行的命令。
  (\emph{默认: empty}).
\end{optlist}

\begin{SideBySideExample}
  \begin{Verbatim}[formatcom=\color{red}]
    First verbatim line.
    Second verbatim line.
  \end{Verbatim}
\end{SideBySideExample}

\subsubsection{修改每行的格式化方式}

\cs{FancyVerbFormatLine}命令定义了每行的格式化方式。
默认值是\marginpar{\small 译注:原样输出}\\ \verb+\def\FancyVerbFormatLine#1{#1}+, 我们也可以
按我们的需要,重新定义它({\verbatimTxt}环境中,行号的计数器名是\texttt{FancyVerbLine}):

{\fvset{fontsize=\small}
\begin{SideBySideExample}
  Z\fvset{fontsize=\normalsize}
  \renewcommand{\FancyVerbFormatLine}[1]{%
    \makebox[0cm][l]{$\Rightarrow$}#1}
  \begin{Verbatim}
    First verbatim line.
    Second verbatim line.
    Third verbatim line.
  \end{Verbatim}
\end{SideBySideExample}

\begin{SideBySideExample}
  \renewcommand{\FancyVerbFormatLine}[1]{%
    \ifodd\value{FancyVerbLine}%
      \MakeUppercase{#1}\else#1\fi}
  \begin{Verbatim}
    First verbatim line.
    Second verbatim line.
    Third verbatim line.
  \end{Verbatim}
\end{SideBySideExample}
}

\subsubsection{字体}

\begin{optlist}
  \item[fontfamily (family name)]: 使用的字体。 预定义的有:
  \texttt{tt}、 \texttt{courier} 和 \texttt{helvetica} 
  (\emph{默认:~tt}).
\end{optlist}

\ifPostScriptFonts  我们可以猜测,PostScript字体是可用的\marginpar{\small\tt{\textbackslash}ifPostScriptFonts}

\begin{SideBySideExample}
  \begin{Verbatim}[fontfamily=helvetica]
    Verbatim line.
  \end{Verbatim}
\end{SideBySideExample}

\else
\begin{quote}
    \textbf{\large Warning!} PostScript fonts seems not available on your
  platform (we are looking for the file \texttt{times.sty}, so we will
  not show the examplaes using such PostScript fonts.
\end{quote}
\fi

\begin{optlist}
  \item[fontsize (font size)]: 要使用的字体的大小
  (\emph{默认: auto} \Dash 同当前字号一致). 如果你也使用了 `\textsf{relsize}' 宏包,
  您可以将之改成与当前字体成比例的大小
  (例如:
  \verb+fontsize=\relsize{-2}+).
\end{optlist}

\ifPostScriptFonts  我们可以猜测,PostScript字体是可用的

\begin{SideBySideExample}
  \begin{Verbatim}[fontsize=\small]
    Verbatim line.
  \end{Verbatim}

  \begin{Verbatim}[fontfamily=courier,
                   fontsize=\large]
    Verbatim line.
  \end{Verbatim}
\end{SideBySideExample}

\else
\begin{SideBySideExample}
  \begin{Verbatim}[fontsize=\small]
    Verbatim line.
  \end{Verbatim}
\end{SideBySideExample}
\fi

\begin{optlist}
  \item[fontshape (字形)]: 使用的字形
  (\emph{默认: auto} \Dash 同当前字形一致).
\end{optlist}

\ifPostScriptFonts  我们可以猜测,PostScript字体是可用的

\begin{SideBySideExample}
  \begin{Verbatim}[fontfamily=courier,
                   fontshape=it]
    Verbatim line.
  \end{Verbatim}
\end{SideBySideExample}

\else

\begin{SideBySideExample}
  \begin{Verbatim}[fontshape=it]
    Verbatim line.
  \end{Verbatim}
\end{SideBySideExample}
\fi

\begin{optlist}
  \item[fontseries (series name)]: 使用的 \LaTeX{} 字体 `series'
  (\emph{默认: auto} \Dash 同当前字体一致).
\end{optlist}

\ifPostScriptFonts  我们可以猜测,PostScript字体是可用的

\begin{SideBySideExample}
  \begin{Verbatim}[fontfamily=courier,
                   fontseries=b]
    Verbatim line.
  \end{Verbatim}
\end{SideBySideExample}

\else
\begin{SideBySideExample}
  \begin{Verbatim}[fontseries=b]
    Verbatim line.
  \end{Verbatim}
\end{SideBySideExample}
\fi

\subsubsection{边框的类型和characteristics}

\begin{optlist}
  \item[frame 
  (none$\mid$leftline$\mid$topline$\mid$bottomline$\mid$lines$\mid$single)]:
  抄录文本的四周的边框类型
  (\emph{默认: none} \Dash 无边框). 使用 \textsf{leftline} 和
  \textsf{single} 模式时, \LaTeX{} 的
  \cs{fboxsep} 命令指定的宽度的空白会被添加到左边的垂直线和文本之间。
\end{optlist}

%  Problem at the top of a page...
\begin{SideBySideExample}
  \begin{Verbatim}[frame=leftline]
    抄录文本,leftline
  \end{Verbatim}
\end{SideBySideExample}

\begin{SideBySideExample}
  \begin{Verbatim}[frame=topline]
    抄录文本,topline
  \end{Verbatim}
\end{SideBySideExample}

\begin{SideBySideExample}
  \begin{Verbatim}[frame=bottomline]
    抄录文本,bottomline
  \end{Verbatim}
\end{SideBySideExample}

\begin{SideBySideExample}
  \begin{Verbatim}[frame=lines]
    抄录文本,lines
  \end{Verbatim}
\end{SideBySideExample}

\begin{SideBySideExample}
  \begin{Verbatim}[frame=single]
    抄录文本,single
  \end{Verbatim}
\end{SideBySideExample}

\begin{optlist}
  \item[framerule (dimension)]: 设置边框线的宽度
  (\emph{默认: 0.4pt ,如果有指定边框}).
\end{optlist}

\begin{SideBySideExample}
  \begin{Verbatim}[frame=single,
                   framerule=1mm]
    Verbatim line.
  \end{Verbatim}
\end{SideBySideExample}

\begin{optlist}
  \item[framesep (dimension)]:设置抄录文本与边框的距离 (\emph{默认: \cs{fboxsep}}).
\end{optlist}

\begin{SideBySideExample}
  \begin{Verbatim}[frame=single,
                   framesep=5mm]
    Verbatim line.
  \end{Verbatim}
\end{SideBySideExample}

\begin{optlist}
  \item[rulecolor (color command)]: 设置边框的颜色, 标准的 \LaTeX{} 描述方式
  (\emph{默认: black})。
\end{optlist}

\begin{SideBySideExample}
  \begin{Verbatim}[frame=single,
                   rulecolor=\color{red}]
    Verbatim line.
  \end{Verbatim}
\end{SideBySideExample}

\begin{optlist}
  \item[fillcolor (color command)]: 设置抄录文本与边线之间空白的颜色 (宽度为前面介绍的
  \texttt{framesep}的值)
  (\emph{默认: none} \Dash no color).
\end{optlist}

\begin{SideBySideExample}
  \begin{Verbatim}[frame=single,
         framerule=1mm,framesep=3mm,
         rulecolor=\color{red},
         fillcolor=\color{yellow}]
    Verbatim line.
  \end{Verbatim}
\end{SideBySideExample}

\subsubsection{Label for the环境}

\begingroup
\newcommand\Vitem{\SaveVerb[aftersave={\item[\UseVerb{Vitem}]}]{Vitem}}
\DefineShortVerb{\|}

\begin{optlist}
  \Vitem|label ({[string]string})|: 指定顶部的标签、底部的标签,或同时指定顶部和底部的标签,用于描述抄录的内容
  (\emph{默认: empty} \Dash no label)。
  如果标签文字包含特殊字符, 比如逗号、等号,
  那就必须放到组里%(\verb|label={含特殊字符的标签}|)
  。
  如果只指定了一个标签字符串, 各种情况下使用的标签内容是一致的。
  但如果可选的标签也给出了(2个标签字符串), 可选的标签文字会用到顶部,第二个用到底部。 
  要注意的是,如果frame参数使用了topline、bottomline、lines或single以外的其他值,则不会打印标签。
%  Note also that, if another value
%  than \textsf{topline}, \textsf{bottomline}, \textsf{lines} or
%  \textsf{single} is used for the \textsf{frame} parameter, the label(s)
%  will not be printed.
\end{optlist}
\endgroup

%  Problem at the top of a page...

\begin{SideBySideExample}
  \fvset{gobble=2}
  \begin{Verbatim}[frame=single,
                   label=My text]
    First verbatim line.
    Second verbatim line.
  \end{Verbatim}

  \begin{Verbatim}[frame=topline,
      framesep=4mm,
      label=\fbox{\Large\emph{The code}}]
    First verbatim line.
    Second verbatim line.
  \end{Verbatim}
\end{SideBySideExample}

\begin{optlist}
  \item[labelposition (none$\mid$topline$\mid$bottomline$\mid$all)]:
  设置标签的位置,如果设置了标签文字的话,which must be
  coherent with the kind of frame chosen
  (\emph{默认: none if the label is empty, topline if one label is
  defined and all if two are defined}).
  当然, 如果配置成冲突的了 (比如
  \textsf{frame=topline,labelposition=bottomline}) 将不显示标签.
\end{optlist}

%  Problem at the top of a page...

\begin{SideBySideExample}
  \fvset{gobble=2}
  \begin{Verbatim}[frame=single,
         framesep=2mm,
         label=Text,labelposition=all]
    frame=single,
    label=Text,
    labelposition=all
  \end{Verbatim}
\end{SideBySideExample}

\begin{SideBySideExample}
  \fvset{gobble=2}
  \begin{Verbatim}[frame=lines,
         label=Text,labelposition=topline]
    First verbatim line.
    Second verbatim line.
  \end{Verbatim}
\end{SideBySideExample}

\begin{SideBySideExample}
  \begin{Verbatim}[frame=bottomline,
         framesep=3mm,
         label=\textit{Code included},
         labelposition=bottomline]
    First verbatim line.
    Second verbatim line.
  \end{Verbatim}
\end{SideBySideExample}

\begin{SideBySideExample}
  \begin{Verbatim}[frame=lines,
                   framesep=3mm,
   label={[Beginning of code]End of code}]
    First verbatim line.
    Second verbatim line.
  \end{Verbatim}
\end{SideBySideExample}

\subsubsection{行号}

\begin{optlist}
  \item[numbers (none$\mid$left$\mid$right)]: 抄录的文本行的行号
  (\emph{默认: none} \Dash no numbering). 如果启用, 行号放在抄录环境的边框之\emph{外}。
\end{optlist}

\begin{SideBySideExample}
  \begin{Verbatim}[gobble=2,numbers=left]
    左:numbers=left
  \end{Verbatim}
\end{SideBySideExample}

\begin{SideBySideExample}
  \begin{Verbatim}[gobble=2,
         numbers=right,numbersep=0pt]
    右:numbers=right
  \end{Verbatim}
\end{SideBySideExample}

\begin{optlist}
  \item[numbersep (dimension)]: 行号和{\verbatimTxt}文本行间的距离
  (\emph{默认: 12pt}).
\end{optlist}

\begin{SideBySideExample}
  \begin{Verbatim}[gobble=2,
         numbers=left,numbersep=2pt]
    First verbatim line.
    Second verbatim line.
  \end{Verbatim}
\end{SideBySideExample}

\begin{optlist}
  \item[firstnumber (auto$\mid$last$\mid$数字)]: 指定行号的起始数字
  (\emph{默认: auto} \Dash 从1开始). \textsf{last} 指定行号续着上一个
  {\verbatimTxt}环境的行号。 如果指定的是一个数字, 行号就从这个数字开始增加。
\end{optlist}

\begin{SideBySideExample}
  \fvset{gobble=2,
         numbers=left,numbersep=3pt}
  \begin{Verbatim}
    Verbatim line.
  \end{Verbatim}

  \begin{Verbatim}[firstnumber=last]
    Verbatim line.
  \end{Verbatim}
\end{SideBySideExample}

\begin{SideBySideExample}
  \fvset{gobble=2,
         numbers=left,numbersep=3pt}
  \begin{Verbatim}[firstnumber=100]
    Verbatim line.
  \end{Verbatim}
\end{SideBySideExample}

\begin{optlist}
  \item[stepnumber (数字)]: 每多少行显示行号
  (\emph{默认:~1} \Dash 在设置了行号的情况下,所有行的行号都会显示出来)。
\end{optlist}

\begin{SideBySideExample}
  \begin{Verbatim}[gobble=2,numbers=left,
         numbersep=3pt,stepnumber=2]
    First verbatim line.
    Second verbatim line.
    Third verbatim line.
  \end{Verbatim}
\end{SideBySideExample}

  \cs{theFancyVerbLine}命令定义了行号的排版格式,
  行号使用的计数器的名称是 \texttt{FancyVerbLine}:

\begin{SideBySideExample}
  \renewcommand{\theFancyVerbLine}{%
    \textcolor{red}{\small
      8.\alph{FancyVerbLine}}}
  \begin{Verbatim}[gobble=2,
         numbers=left,numbersep=2pt]
    First verbatim line.
    Second verbatim line.
    Third verbatim line.
  \end{Verbatim}
\end{SideBySideExample}

\ifChangeBar\begin{changebar}\fi
\begin{optlist}
  \item[numberblanklines (boolean)]: 是否对空行编号
  (全空的行,或只包含空白字符的行)
  (\emph{默认: true} \Dash 所有行都会被编号).
\end{optlist}

\begin{SideBySideExample}
  \begin{Verbatim}[gobble=2,numbers=left,
         numbersep=3pt,
         numberblanklines=false]
    First verbatim line.


    Second verbatim line.
  \end{Verbatim}
\end{SideBySideExample}
\ifChangeBar\end{changebar}\fi
%
\subsubsection{选择要排版的起始行和终止的行}

\begin{optlist}
  \item[firstline (integer)]: 要排版的起始行号
  (\emph{默认: empty} \Dash 从第一行开始).
  \item[lastline (integer)]: 要排版的终止行号
  (\emph{默认: empty} \Dash 一直到最后一行).
\end{optlist}


\begin{SideBySideExample}
  \begin{Verbatim}[gobble=2,
         firstline,lastline,
         numbers=left,numbersep=2pt]
    First verbatim line.
    Second verbatim line.
    Third verbatim line.
  \end{Verbatim}
\end{SideBySideExample}



\begin{SideBySideExample}
  \begin{Verbatim}[gobble=2,firstline=2,
         numbers=left,numbersep=2pt]
    First verbatim line.
    Second verbatim line.
    Third verbatim line.
  \end{Verbatim}
\end{SideBySideExample}


\begin{SideBySideExample}
  \begin{Verbatim}[gobble=2,lastline=1,
         numbers=left,numbersep=2pt]
    First verbatim line.
    Second verbatim line.
    Third verbatim line.
  \end{Verbatim}
\end{SideBySideExample}

除了指定 firstline 参数来控制开始的行号范围, 你也可以指定一个{\em 开始字符串}; 开始的第一行就是第一个完全匹配这个字符串的那一行.  (当指定了忽略的起始字符数时,是会有区别的) {\em 结束字符串}也是类似的. 
你可以混合使用指定行号和指定字符串这两种方式, e.g.\ 用 firstline 指定开始的行号, 用 {\em 结束字符串} 来指定结束的位置。%
指定字符串有点笨拙%  Specifying the strings is a bit klunky
。首先你需要用以下的 \cs{newcommand*} 命令定义相应的字符串:

\begin{SideBySideExample}[gobble=2]
  \newcommand*\FancyVerbStartString{FROM}
  \newcommand*\FancyVerbStopString{TO}
  \begin{Verbatim}
    First verbatim line.
  FROM
    Second verbatim line.
  TO
    Third verbatim line.
  \end{Verbatim}
\end{SideBySideExample}

\noindent 在重新定义{\em 开始字符串}、{\em 结束字符串}时,你要使用 \cs{renewcommand*} 命令重新定义。

若设置了 \texttt{lastline=} 或 \texttt{lastline=0} 所有行都不会输出。

\begin{SideBySideExample}[gobble=2]
  foo
  \begin{Verbatim}[frame=none,lastline=]
    First verbatim line.
    Second verbatim line.
    Third verbatim line.
  \end{Verbatim}
  bar
\end{SideBySideExample}

\subsubsection{空格和 tab characters}

\begin{optlist}
  \item[showspaces (boolean)]: 是否将每个空格用可视的字符展示出来。
  (\emph{默认: false}).
\end{optlist}

\begin{SideBySideExample}[gobble=0]
\begin{Verbatim}[showspaces=true]
  Verbatim line.
\end{Verbatim}
\end{SideBySideExample}

  In practice, 所有的 verbatim 环境有相应的带~\texttt{*} 变体, 设置了 \texttt{showspaces=true}:

\begin{SideBySideExample}[gobble=0]
  \begin{Verbatim*}
    Verbatim line.
  \end{Verbatim*}
\end{SideBySideExample}

还有一些参数用来配置制表符的输出方式 (使用{\em 制表符}实际上是相当过时的编程风格了):

\begin{optlist}
  \item[showtabs (boolean)]: 显式显示制表符
  (\emph{默认: false} \Dash 不显示制表符).
\end{optlist}

\begin{optlist}
  \item[obeytabs (boolean)]: %是否将tabs仅作为tab而不转换为空格。
根据制表符定位字符
  (\emph{默认: false} \Dash tab characters are added to the current position).
\end{optlist}

\begin{optlist}
  \item[tabsize (integer)]: 一个制表符对应的空格数
  (\emph{默认:~8}).
\end{optlist}

\subsubsection{行间距}

\begin{optlist}
  \item[baselinestretch (auto$\mid$dimension)]: 设置抄录环境中,\LaTeX{}的  `baselinestretch'  参数的值。
  (\emph{默认: auto} \Dash 它在抄录命令之前的当前值).
\end{optlist}

\begin{SideBySideExample}
  \begin{Verbatim}[baselinestretch=2]
    First verbatim line.
    Second verbatim line.
  \end{Verbatim}
\end{SideBySideExample}

\subsubsection{用于插入命令的转义字符}

\begin{optlist}
  \item[commandchars (指定三个字符)]: 指定开始命令的字符、命令参数的组的开始和结束符号\cprotect\marginpar{\small 译注:通常三个符号为\verb|\\\{\}|};
  我们介绍下在抄录代码中使用 \emph{转义} 序列 . 
  当然,最好是选择抄录文本中未出现的字符来使用! (\emph{默认: empty}).
  \label{sec:commandchars}
\end{optlist}

\begin{SideBySideExample}
  \begin{Verbatim}[commandchars=\\\{\}]
    \textit{This is a comment}
    First verbatim line.
    \fbox{Second} verbatim line.
    \textcolor{red}{Third} verbatim line.
  \end{Verbatim}

  \begin{Verbatim}[commandchars=+\[\]]
    +textit[\textbf{Verbatim} line].
  \end{Verbatim}
\end{SideBySideExample}

\ifChangeBar\begin{changebar}\fi
  这样,指定了commandchars后,就可以放置{\em 标签}了, \verb|\ref|{\em  引用}的时候,显示的是在{\verbatimTxt}环境中的{\em 行号}:

\begin{SideBySideExample}
  \begin{Verbatim}[commandchars=\\\{\},
         numbers=left,numbersep=2pt]
    First verbatim line.
    Second line.\label{vrb:Important}
    Third verbatim line.
  \end{Verbatim}

    As I previously shown
  line~\ref{vrb:Important}, it is...
\end{SideBySideExample}
\ifChangeBar\end{changebar}\fi

\subsubsection{边距}

\begin{optlist}
  \item[xleftmargin (长度)]: 每行的开头添加的缩进
  (\emph{默认:~0pt}).
\end{optlist}

\begin{SideBySideExample}
  \begin{Verbatim}[frame=single,
                   xleftmargin=5mm]
    Verbatim line.
  \end{Verbatim}
\end{SideBySideExample}

\begin{optlist}
  \item[xrightmargin (dimension)]: 在每行之后添加的右边距
  (\emph{默认:~0pt} \Dash).
\end{optlist}

\begin{SideBySideExample}
  \begin{Verbatim}[frame=single,
                   xrightmargin=1cm]
    Verbatim line.
  \end{Verbatim}
\end{SideBySideExample}

\begin{optlist}
  \item[resetmargins (boolean)]: 重置 left margin,   
  当在其他的带有缩进的环境中时,这很有用(\emph{默认: false} \Dash no
  reset of the margin).
\end{optlist}

{%
\DefineVerbatimEnvironment{Verbatim}{Verbatim}{gobble=0,numbers=none}
\begin{SideBySideExample}
  \begin{itemize}
    \item First item
    \begin{Verbatim}[frame=single]
  Verbatim line.
    \end{Verbatim}
    \item Second item
    \begin{Verbatim}[frame=single,
                     resetmargins=true]
  Verbatim line.
    \end{Verbatim}
  \end{itemize}
\end{SideBySideExample}
}

\subsubsection{溢出盒子信息}

\begin{optlist}
  \item[hfuzz (dimension)]: value to give to the \TeX{} \cs{hfuzz} dimension
  for text to format. 这可以避免显示一些不重要的
  \emph{Overfull box} 信息 (\emph{默认:~2pt}).
\end{optlist}

\subsubsection{跨页}

\begin{optlist}
  \item[samepage (boolean)]: 在非常特殊的情况下, 
  我们希望,{\verbatimTxt}环境中的内容不要分到不同的页面, even if it does not fit on the current page. 要避免跨页, 可以将
  \texttt{samepage} 设置为 \emph{true} (\emph{默认: false}).
\end{optlist}

\subsubsection{Catcode characters}

\begin{optlist}
  \item[codes (macro)]: 指定 \emph{catcode} 变更 (\emph{默认:
  empty}).
\end{optlist}

例如,下面这样设置后,我们就可以在抄录文本中包含数学公式了:

{\fvset{fontsize=\small}
\begin{SideBySideExample}
  Z\fvset{fontsize=\normalsize}
  \begin{Verbatim}[commandchars=\\\{\},
         codes={\catcode`$=3\catcode`^=7}]
    x=1/sqrt(z**2) ! $\frac{1}{\sqrt{z^2}}$
  \end{Verbatim}
\end{SideBySideExample}
}  %$


\subsubsection{Active characters}

\begin{optlist}
  \item[defineactive (macro)]: to define the effect of \emph{active}
  characters
  (\emph{默认: empty}).
\end{optlist}

  This allows us to do some devious tricks: see the example in Section
\ref{sec:VerbatimInclude} on page~\pageref{sec:VerbatimInclude}.


\subsubsection{引用标签}

\begin{optlist}
  \item[reflabel (<标签>)]:设置标签,以便 \verb|\pageref| 引用命令显示页数。
\end{optlist}

\begin{SideBySideExample}
  \begin{Verbatim}[reflabel=verb0]
    First verbatim line.
    Second verbatim line.
  \end{Verbatim}
  See the verbatim on 
  page~\pageref{verb0}.
\end{SideBySideExample}




\subsection{不同类型的{\verbatimTxt}环境}

\subsubsection{Verbatim 环境}

  这是我们上面已经用过的`normal' {\verbatimTxt}环境.

\subsubsection{BVerbatim 环境}

  这个环境将{\verbatimTxt}内容放到一个 \TeX{} 盒子中。 有些前面介绍过的参数在这个环境是不存在的 (尤其是和边框相关的参数),
不过,也有两个新的参数可用于配置:

\begin{optlist}
  \item[boxwidth (auto$\mid$dimension)]: 使用的盒子的宽度
  (\emph{默认: auto} \Dash the width of the longest line is used).
\end{optlist}

\begin{optlist}
  \item[baseline (b$\mid$c$\mid$t)]: 对齐用的基线位置
   (on the
  \texttt{baseline}, the \texttt{center} or the \texttt{top} of the box)
  (\emph{默认: b}).
\end{optlist}

\begin{SideBySideExample}
  \fvset{gobble=2}
  \begin{BVerbatim}
    First
    Second
  \end{BVerbatim}
  \begin{BVerbatim}[baseline=c]
    First
    Second
  \end{BVerbatim}
\end{SideBySideExample}

\begin{SideBySideExample}
  \begin{BVerbatim}[boxwidth=2cm]
    First
    Second
  \end{BVerbatim}
  \begin{BVerbatim}[boxwidth=2cm,
                    baseline=t]
    First
    Second
  \end{BVerbatim}
\end{SideBySideExample}

\subsubsection{LVerbatim 环境}

  这个环境将抄录内容放到\LaTeX{} 左右模式 (实际上这在
\TeX{} 里也叫 \emph{受限水平模式}).

\subsubsection{自定义{\verbatimTxt}环境}

\ifChangeBar\begin{changebar}\fi
\VerbatimFootnotes
要定义自己的{\verbatimTxt}环境是很容易的。 
  
你可以用 \cs{RecustomVerbatimEnvironment} 命令重新定义已经存在的环境,
或使用 \cs{DefineVerbatimEnvironment} 命令\footnote{%
  \ifChangeBar\begin{changebar}\fi
    也存在{\verbatimTxt}命令 \cs{CustomVerbatimCommand} 和
  \cs{RecustomVerbatimCommand} 命令; 例如:
  
  \noindent
  \verb+\RecustomVerbatimCommand{\VerbatimInput}{VerbatimInput}{frame=lines}+
  \ifChangeBar\end{changebar}\fi} 创建新的环境。这两种方式
, 你需要指定新环境的名称、它所基于的环境的{\em 名字},  
以及一组选项的初始值。 
这些选项的值可以在使用时传入的可选参数覆盖:



\begin{SideBySideExample}
  \RecustomVerbatimEnvironment
    {Verbatim}{Verbatim}
    {gobble=2,frame=single}
  \begin{Verbatim}
    First verbatim line.
    Second verbatim line.
  \end{Verbatim}
\end{SideBySideExample}
\ifChangeBar\end{changebar}\fi

\begin{SideBySideExample}
  \DefineVerbatimEnvironment%
    {MyVerbatim}{Verbatim}
    {gobble=2,numbers=left,numbersep=2mm,
     frame=lines,framerule=0.8mm}
  \begin{MyVerbatim}
    First verbatim line.
    Second verbatim line.
  \end{MyVerbatim}

  \begin{MyVerbatim}[numbers=none,
                     framerule=1pt]
    First verbatim line.
    Second verbatim line.
  \end{MyVerbatim}
\end{SideBySideExample}

\section{保存和取用{\verbatimTxt}文本和环境}

\cs{SaveVerb} 和 \cs{UseVerb} 命令让我们可以保存和取用
{\verbatimTxt}内容。 \cs{UseVerb}是robust的:

%  \DefineShortVerb{\|}
%  \SaveVerb{Verb}|_verbatim_|
\begin{SideBySideExample}
  \DefineShortVerb{\|}
  \SaveVerb{Verb}|_verbatim_|
  I have saved \UseVerb{Verb} and reuse 
  it later as many times as I want
  \subsection*{Using \UseVerb{Verb}}
  \UseVerb{Verb}.
\end{SideBySideExample}

  这也为,向那些通常不允许放置抄录命令的\LaTeX{}命令中放置抄录文本提供了一种方案: % (used package \texttt{marginnote}):

{\fvset{frame=single,xrightmargin=0cm}
\begin{Example}
  \DefineShortVerb{\|}\SaveVerb{Verb}|_OK^| \marginpar{\UseVerb{Verb}}
\end{Example}
}

  有个很有用的情况是:在描述列表中,使用抄录文本作为条目文本
 (\LaTeX 里,这通常是不允许放置抄录命令的), 使用
\texttt{aftersave}参数:

\begin{optlist}
  \item[aftersave (macro)]: 用来动态保存一些抄录文本的命令
  (\emph{默认: empty}).
\end{optlist}

\begin{SideBySideExample}
  \newcommand{\Vitem}{%
    \SaveVerb[aftersave={%
      \item[\UseVerb{Vitem}]}]{Vitem}}
  \DefineShortVerb{\|}
  \begin{description}
    \Vitem|\MyCommand|: my command
  \end{description}
\end{SideBySideExample}

\begin{quotation}
译者注:
\begin{description}
  \item[1] 保存 \verb|\MyCommand| 到 Vitem
  \item[2] 继续执行aftersave指定的 \verb|\item[\UseVerb{Vitem}]|, 以及后续的 \verb|: my command|
\end{description}
\end{quotation}

  同样的, 我们可以保存和取用(盒模式使用\cs{UseVerbatim}, \cs{BUseVerbatim},\cs{LUseVerbatim}使用左右模式)
整个的抄录环境:

\begin{SideBySideExample}
  \fvset{gobble=0,numbers=none}
  \begin{SaveVerbatim}{抄录保存名}
    Verbatim line.
  \end{SaveVerbatim}
  \UseVerbatim{抄录保存名}
  and \UseVerbatim{抄录保存名}
\end{SideBySideExample}

\begin{SideBySideExample}
  \fvset{gobble=0,numbers=none}
  \begin{SaveVerbatim}[gobble=5]%
    {VerbEnv}
1234First
    Second
  \end{SaveVerbatim}

  \fbox{\BUseVerbatim{VerbEnv}}
  and \BUseVerbatim{VerbEnv}.

  \LUseVerbatim{VerbEnv} and
  \LUseVerbatim{VerbEnv}
\end{SideBySideExample}


\begin{SideBySideExample}
\begin{SaveVerbatim}{FOO}
然而,如果我使用
SaveVerbatim
定义了一个可复用的
Verbatim,
开启了行号。
\end{SaveVerbatim}

\begin{Verbatim}[numbers=left]
This works.
我直接使用
Verbatim环境.
\end{Verbatim}

\UseVerbatim[numbers=left]{FOO}
\end{SideBySideExample}


\clearpage

\section{从文件读写抄录}
\label{sec:VerbatimInclude}

  \cs{VerbatimInput}命令 (变种 \cs{BVerbatimInput}、
\cs{LVerbatimInput} 也) 让我们可以从文件中读取内容并以{\verbatimTxt}处理输出。当然, 我们前面介绍过的那些配置参数也还可以使用:

%The file we will use for \VerbatimInput

\typeout{*************************************}
\typeout{* Created files: hello.f90, file.txt}
\typeout{* See fancyhdr.dvi for an explanation}
\typeout{*************************************}

\begin{VerbatimOut}{hello.f90}
  ! A "hello" program

  program hello
    print *,"Hello world"
  end program hello
\end{VerbatimOut}

\begin{SideBySideExample}
  \RecustomVerbatimCommand{\VerbatimInput}
  {VerbatimInput}{gobble=4}
  \fvset{fontsize=\small}
  \VerbatimInput{hello.f90}

  \fvset{frame=single,numbers=left,
         numbersep=3pt}
  \VerbatimInput{hello.f90}

  \VerbatimInput[firstline=3,
      rulecolor=\color{green}]
    {hello.f90}

  \VerbatimInput[frame=lines,
      fontshape=sl,fontsize=\footnotesize]
    {hello.f90}
\end{SideBySideExample}

我们可以用 `defineactive' 参数将程序文本中的注释行设置为不同的样式:

\begin{SideBySideExample}
  \RecustomVerbatimCommand{\VerbatimInput}
  {VerbatimInput}{gobble=4}
  \def\ExclamationPoint{\char33}
  \catcode`!=\active
  \VerbatimInput%
    [defineactive=%
      \def!{\color{cyan}\itshape
        \ExclamationPoint}]
    {hello.f90}
\end{SideBySideExample}

  要注意的是,如果文件内容在当前面放不下了, 它将根据需要,自动分解跨页 (除非 \texttt{samepage} 参数被设置为 \texttt{true} 了).

  类似的,\verb+VerbatimOut+ 环境则用用来将{\verbatimTxt}内容写入到外部的文件中, in the same way:

\begin{SideBySideExample}
  \begin{VerbatimOut}{file.txt}
    I write that.
    And that too.
  \end{VerbatimOut}

  \VerbatimInput[frame=single,
    numbers=left,numbersep=6pt]{file.txt}
\end{SideBySideExample}

\ifChangeBar\begin{changebar}\fi
\section{自动高亮排版}

显然, 自动 \emph{高亮排版}不在此包的范围内。
尽管如此, this is specially interesting for verbatim
inclusion of programming 代码文件或片段. 在\LaTeX{}世界
(not speaking of the \emph{literate programming} way), there are software for some special languages, as the `\textsf{C++2LaTeX}' package from Norbert
\textsc{Kiesel}, but mainly two generic ones, which use completely different 
modes (an external preprocessor written in C and a \TeX{} based solution):
the `\textsf{LGrind}'~\cite{LGrind} system, currently maintened by Michael
\textsc{Piefel}, and the `\textsf{listings}'~\cite{Listings} package from
Carsten \textsc{Heinz}.

   \FVrbPackage{} 和 `\textsf{listings}' 包未来版本有计划进行合作, 
   which will offer great advantages to both users of the
two actual packages, and will 允许 \FVrbPackage{} 的用户拥有自动漂亮排版编程语言代码。

\section{已知的问题}
  
\begin{itemize}
  \item Vladimir \textsc{Volovich} \verb+<vvv@vvv.vsu.ru>+ 有上报提到 the
  special character \verb+\th+, 在 T1 编码是可用的, 但在 \FVrbPackage 宏包的作为抄录内容时却不可用。It can be true for other special
  characters too.
\end{itemize}

\iffalse

\section{Thanks}

  For interesting comments and suggestions, we would like to thank specially
(alphabetic order): Philippe \textsc{Esperet}
\url{esperet@marie.polytechnique.fr}, Michael \textsc{Friendly}
\url{friendly@hotspur.psych.yorku.ca}, Rolf \textsc{Niepraschk}
\url{niepraschk@gmx.de} and for bug reports Mario \textsc{Hassler}
\relax\unskip\break
\url{HASSLER@ippnv2.ipp.kfa-juelich.de}, Mikhail \textsc{Kolodin}
\relax\unskip\break
\url{myke@morrigan.spb.su}, Andreas Matthias, Ulrich M. Schwarz, and Vladimir \textsc{Volovich}
\verb+<vvv@vvv.vsu.ru>+.


\fi
\section{小结}

  There are a few other possibilities that 我们在这没介绍到.
  特别注意,我们可以定义一些自定义的内容到
(\texttt{fancyvrb.cfg})文件,宏包会在最后时加载这个文件(如果有), 
保存你自定义的命令和环境以及重定义一些已有的属性。


\clearpage


\part{\texttt{fancyvrb-ex}包}
这个宏包定义了一些 example 环境,可以将源码和排版效果并排或上下输出。它们在这个\texttt{fancyvrb}文档中也被运用到。

\section{\texttt{Example}环境}

\begin{Verbatim}[gobble=2]
  \begin{Example}
    First verbatim line.
    Second verbatim line.
    Third verbatim line.
  \end{Example}
\end{Verbatim}

\begin{Example}
  First verbatim line.
  Second verbatim line.
  Third verbatim line.
\end{Example}

\begin{Verbatim}[gobble=2]
  \begin{Example}[frame=lines,framerule=1mm,
          numbers=left]
    First verbatim line.
    Second verbatim line.
    Third verbatim line.
  \end{Example}
\end{Verbatim}

\begin{Example}[frame=lines,framerule=1mm,numbers=left]
  First verbatim line.
  Second verbatim line.
  Third verbatim line.
\end{Example}

\newpage

\section{\texttt{CenterExample}环境}

\begin{Verbatim}[gobble=2]
  \begin{CenterExample}[frame=single,
       numbers=right]
    First verbatim line.
    Second verbatim line.
    居中示例
  \end{CenterExample}
\end{Verbatim}

\begin{CenterExample}[frame=single,numbers=right]
  First verbatim line.
  Second verbatim line.
  居中示例
\end{CenterExample}


\section{\texttt{SideBySideExample}环境}

\begin{Verbatim}[gobble=2]
  \begin{SideBySideExample}[
      xrightmargin=5cm,
      frame=lines, numbers=left]
    First verbatim line.
    Second verbatim line.
    Third verbatim line.
    并排示例
  \end{SideBySideExample}
\end{Verbatim}

\begin{SideBySideExample}[xrightmargin=5cm,frame=single,numbers=left]
  First verbatim line.
  Second verbatim line.
  并排示例
\end{SideBySideExample}



\iffalse



\section{\texttt{PCenterExample}环境}

\begin{Verbatim}
  \fvset{frame=lines,framerule=0.5mm,numbers=left}

  \begin{PCenterExample}(-0.5,-0.5)(0.5,0.5)
    \setlength{\unitlength}{1cm}
    \put(0,0){\circle{1}}
  \end{PCenterExample}

  \showgrid
  \begin{PCenterExample}(-1,-1)(1,1)
    \setlength{\unitlength}{1cm}
    \put(0,0){\circle{1}}
  \end{PCenterExample}
\end{Verbatim}

{\fvset{frame=lines,framerule=0.5mm,numbers=left}
\begin{PCenterExample}(-0.5,-0.5)(0.5,0.5)
  \setlength{\unitlength}{1cm}
  \put(0,0){\circle{1}}
\end{PCenterExample}
\showgrid
\begin{PCenterExample}(-1,-1)(1,1)
   \setlength{\unitlength}{1cm}
   \put(0,0){\circle{1}}
\end{PCenterExample}
}


\section{\texttt{PSideBySideExample}环境}

\begin{Verbatim}[gobble=2]
  \fvset{frame=single,xrightmargin=5cm}
  \begin{PSideBySideExample}(-2,-1)(2,1)
    \psellipse*[linecolor=yellow](2,1)
  \end{PSideBySideExample}
  \showgrid
  \begin{PSideBySideExample}(-2,-1)(2,1)
    \psellipse[linestyle=dashed](2,1)
  \end{PSideBySideExample}
\end{Verbatim}

{\fvset{frame=single,xrightmargin=5cm}
\begin{PSideBySideExample}(-2,-1)(2,1)
  \psellipse*[linecolor=yellow](2,1)
\end{PSideBySideExample}

\showgrid
\begin{PSideBySideExample}(-2,-1)(2,1)
  \psellipse[linestyle=dashed](2,1)
\end{PSideBySideExample}
}

\begin{Verbatim}[gobble=2]
  \fvset{frame=single,xrightmargin=5cm}
  \begin{PSideBySideExample}(-2,-1)(2,1)
    \psellipse[linestyle=dashed](2,1)
  \end{PSideBySideExample}
  \begin{PSideBySideExample}[numbers=right](-2,-1)(2,1)
    \psellipse[linestyle=dotted](2,1)
  \end{PSideBySideExample}
\end{Verbatim}

{\fvset{frame=single,xrightmargin=5cm}
\begin{PSideBySideExample}(-2,-1)(2,1)
  \psellipse[linestyle=dashed](2,1)
\end{PSideBySideExample}

\begin{PSideBySideExample}[numbers=right](-2,-1)(2,1)
  \psellipse[linestyle=dotted](2,1)
\end{PSideBySideExample}
}


\fi




\begin{thebibliography}{1}
  \bibitem{FancyBox} Timothy \textsc{van Zandt},
  \textit{Documentation for `fancybox': Box tips and tricks for \LaTeX}.
  Available from \url{CTAN: macros/latex/contrib/supported/fancybox}, 1993.

  \bibitem{FancyVrb} Timothy \textsc{van Zandt},
  \textit{`fancyvrb': Fancy Verbatims in \LaTeX}.
  Available from \url{CTAN: macros/latex/contrib/supported/fancyvrb}, 1998.

  \bibitem{LGrind} Various authors (current maintainer: Michael
  \textsc{Piefel}),
  \textit{The `LGrind' package}.
  Available from \url{CTAN: support/lgrind}, 1998.

  \bibitem{Listings} Carsten \textsc{Heinz},
  \textit{The `Listings' package}.
  Available from \url{CTAN: macros/latex/contrib/supported/listings}, 1996-1997.
\end{thebibliography}








\end{document}
