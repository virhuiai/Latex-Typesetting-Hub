\documentclass{ltxdoc}
\usepackage[%
,scheme=chinese%中文方案
,fontset=none%不使用默认的字体设置
,space=auto%自动调整中英文间距
]{ctex}
\setCJKmainfont{方正书宋_GBK}%方正书宋_GBK.TTF  设置文本的中文有衬线字体为“方正书宋_GBK”
\setCJKsansfont{方正黑体简体}%方正黑体_GBK.TTF  设置文本的中文无衬线字体为“方正黑体简体”
\setCJKmonofont{方正书宋简体}%方正仿宋_GBK.TTF  设置文本的中文等宽字体为“方正书宋简体”

% \usepackage[all]{tcolorbox}
%\tcbuselibrary{documentation,minted}
% \tcbset{listing engine=minted}

\usepackage{parcolumns}
\EnableCrossrefs
\CodelineIndex
\RecordChanges
% Modification of verbatim for tabs in listings
\makeatletter
{\catcode`\ =\active%
\catcode`\^^I=\active%
\gdef\@vobeyspaces{%
\catcode`\ \active\let \@xobeysp%
\catcode`\^^I\active\def^^I{~~}%
}}%
\makeatother

\usepackage{parskip}
\begin{document}
\parindent=0pt
	\DocInput{\jobname.dtx}
\end{document}
% xelatex -file-line-error -synctex=1  -shell-escape -output-directory=/Volumes/RamDisk/latex_output


%\vspace*{-2\parskip}
% \begin{parcolumns}[rulebetween=false,colwidths={1=.6\linewidth}]{2}
% \colchunk{}
% \colchunk{}
% \end{parcolumns}

parcolumns