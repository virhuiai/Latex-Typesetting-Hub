\PassOptionsToPackage{no-math}{fontspec}%禁用了使用fontspec宏包中的数学字体功能。
\PassOptionsToPackage{AutoFakeBold=true,AutoFakeSlant=true}{xeCJK}%让xeCJK宏包自动产生伪粗体和伪斜体效果。

\documentclass{book}
% \usepackage[paperwidth=9.0cm,paperheight=11.7cm,%
% margin=0in,left=0.1cm,right=0.1cm,top=0.1cm,bottom=0.2cm
% ]{geometry}

\usepackage[heading=true
,scheme=chinese%中文方案
,fontset=none%不使用默认的字体设置
,space=auto%自动调整中英文间距
]{ctex}
\CTEXsetup[name={第~,~篇}]{chapter}
% \CTEXsetup[name={第~,~篇\newline\small}]{chapter}
\CTEXsetup[name={第,节},number={\chinese{section}}]{section}
\CTEXsetup[name={,、},number={\chinese{subsection}}]{subsection}

\setCJKmainfont[Path=/Users/virhuiai/hlProjects/Latex-Typesetting-Hub/font/方正/]{FangZhengShuSong-GBK-1.ttf}%设置文本的中文有衬线字体
\setCJKsansfont[Path=/Users/virhuiai/hlProjects/Latex-Typesetting-Hub/font/方正/]{FangZhengHeiTi-GBK-1.ttf}%设置文本的中文无衬线字体为
\setCJKmonofont[Path=/Users/virhuiai/hlProjects/Latex-Typesetting-Hub/font/方正/]{FangZhengFangSong-GBK-1.ttf} %设置文本的中文等宽字体 
% 我们首先设置了三种主要的字体:正文 (mainfont)、无衬线字体 (sansfont) 和等宽字体 (monofont)。

% \usepackage{newclude}%后面 \include* 引用就不会自动分页了
%\includeonly{1}%加上这句,就只有1的两个引入有效果了

% \makeatletter
% \providecommand*\input@path{}
% \newcommand*\addinputpath[1]{\expandafter\def\expandafter\input@path\expandafter{\input@path#1}}
% \makeatother

% \addinputpath{%
% {/Users/virhuiai/hlProjects/Latex-Typesetting-Hub/练习书本排版/iText_in_Action_Second_Edition}%
% }

\usepackage[all]{tcolorbox}
% % \tcbuselibrary{documentation}

\begin{document}

%第一篇
\chapter{婴幼儿的发育特点}

%第一节
\section{婴幼儿年龄分期}

婴幼儿时期是体格发育和神经智能发育最旺盛和最迅速的时期。随着身体各个器官不断发育长大,生理功能逐步成熟,身心几乎每月甚至每天都在不断发生变化,各个时期的解剖、生理和病理特点又都是极不相同的。为了便于了解和区分不同年龄婴幼儿生长发育的特点和规律,我们将婴幼儿分为新生儿期、婴儿期和幼儿期三个时期,现简单介绍如下。


% 一、
\subsection[新生儿期]{新生儿期-\small 从出生到满28天}

从出生到满28天,这个时期的宝宝被称为新生儿。新生儿从母体温暖安静的小环境中突然进入到寒冷嘈杂的大环境中,是十月怀胎后的第一场人生考验,对于宝宝来说需要全身心地投入和机体的动员以适应外部世界。由于内外环境的突然巨变,小宝宝机体内尚未建立完整和健全的调节能力和适应能力,特别容易发生心、脑、肺及全身性疾病而导致死亡。在许多发达国家都建立了围生期保健网和新生儿监护中心,大大降低了新生儿的死亡率。在我国由于地区不同,其医疗卫生条件、经济状况及地理环境等的不同而差异非常大,如北京、上海等大城市婴儿的死亡率就明显低于农村。

经专家研究发现,新生儿不仅具有视听能力,还有惊人的模仿和记忆能力,早期开发新生儿大脑功能已经得到越来越多的专家和学者注意,并得到了广泛的社会关注。在新生儿医学方面,已经逐步由单一的体格发育、营养发育的研究转为智力潜能开发的研究。


%第二节
\section{新生儿的生长发育和智力评估}
%第三节
\section{婴儿期各阶段的生长发育特点}
%第四节
\section{幼儿期的生长发育特点}

%第二篇
\chapter{婴幼儿的喂养指导}
%第三篇
\chapter{幼儿的营养方案}
%第四篇
\chapter{婴幼儿的日常护理}
%第五篇
\chapter{婴幼儿的早期数育}
%第六篇
\chapter{婴幼儿的疾病防治}


% %%%sumBgColor%%%%%%%%%%%%%%%%%%%%%%%%%%%%%%%%%%%%%%%%%%%%%%%%%%%%%%%%%%%%%%%%%
% %abstractlist
% \definecolor{sumBgColor}{RGB}{236, 236, 211}
% \definecolor{sumTxtColor}{RGB}{167, 33, 22}
% \newenvironment{itemizeSum}{%
% \begin{itemize}}{\end{itemize}}
% \tcolorboxenvironment{itemizeSum}{
% %before skip和after skip选项分别表示在环境前后添加6pt的垂直空白。
% % before skip=6pt,after skip=6pt,
% breakable,%
% opacityframe=0.0,colback=sumBgColor,colupper=sumTxtColor,grow to left by=2em,before upper={\textbf{This chapter covers\\本章涵盖了}}
% }
% %%%%%%%%%%%%%%%%%%%%%%%%%%%%%%%%%%%%%%%%%%%%%%%%%%%%%%%%%%%%%%%%%%%%

% \newtcolorbox[blend into=figures]{myfigure}[2][]{float=htb,capture=hbox,
% title={#2},every float=\centering,#1}

在
% 
\end{document}
%cd /Volumes/RamDisk/ &&  xelatex --output-directory=/Volumes/RamDisk/ -synctex=1 -shell-escape  /Users/virhuiai/hlProjects/Latex-Typesetting-Hub/练习书本排版/iText_in_Action_Second_Edition/iText_in_Action_Second_Edition.tex