\PassOptionsToPackage{no-math}{fontspec}%禁用了使用fontspec宏包中的数学字体功能。
\PassOptionsToPackage{AutoFakeBold=true,AutoFakeSlant=true}{xeCJK}%让xeCJK宏包自动产生伪粗体和伪斜体效果。

\documentclass{book}
\usepackage[heading=true
,scheme=chinese%中文方案
,fontset=none%不使用默认的字体设置
,space=auto%自动调整中英文间距
]{ctex}
\CTEXsetup[name={第~,~篇}]{chapter}

\setCJKmainfont[Path=/Users/virhuiai/hlProjects/Latex-Typesetting-Hub/font/方正/]{FangZhengShuSong-GBK-1.ttf}%设置文本的中文有衬线字体
\setCJKsansfont[Path=/Users/virhuiai/hlProjects/Latex-Typesetting-Hub/font/方正/]{FangZhengHeiTi-GBK-1.ttf}%设置文本的中文无衬线字体为
\setCJKmonofont[Path=/Users/virhuiai/hlProjects/Latex-Typesetting-Hub/font/方正/]{FangZhengFangSong-GBK-1.ttf} %设置文本的中文等宽字体 
% 我们首先设置了三种主要的字体:正文 (mainfont)、无衬线字体 (sansfont) 和等宽字体 (monofont)。

% \usepackage{newclude}%后面 \include* 引用就不会自动分页了
%\includeonly{1}%加上这句,就只有1的两个引入有效果了

% \makeatletter
% \providecommand*\input@path{}
% \newcommand*\addinputpath[1]{\expandafter\def\expandafter\input@path\expandafter{\input@path#1}}
% \makeatother

% \addinputpath{%
% {/Users/virhuiai/hlProjects/Latex-Typesetting-Hub/练习书本排版/iText_in_Action_Second_Edition}%
% }

\usepackage[all]{tcolorbox}
% % \tcbuselibrary{documentation}

\begin{document}

%第一篇
\chapter{婴幼儿的发育特点}
%第二篇
\chapter{婴幼儿的喂养指导}
%第三篇
\chapter{幼儿的营养方案}
%第四篇
\chapter{婴幼儿的日常护理}
%第五篇
\chapter{婴幼儿的早期数育}
%第六篇
\chapter{婴幼儿的疾病防治}


% %%%sumBgColor%%%%%%%%%%%%%%%%%%%%%%%%%%%%%%%%%%%%%%%%%%%%%%%%%%%%%%%%%%%%%%%%%
% %abstractlist
% \definecolor{sumBgColor}{RGB}{236, 236, 211}
% \definecolor{sumTxtColor}{RGB}{167, 33, 22}
% \newenvironment{itemizeSum}{%
% \begin{itemize}}{\end{itemize}}
% \tcolorboxenvironment{itemizeSum}{
% %before skip和after skip选项分别表示在环境前后添加6pt的垂直空白。
% % before skip=6pt,after skip=6pt,
% breakable,%
% opacityframe=0.0,colback=sumBgColor,colupper=sumTxtColor,grow to left by=2em,before upper={\textbf{This chapter covers\\本章涵盖了}}
% }
% %%%%%%%%%%%%%%%%%%%%%%%%%%%%%%%%%%%%%%%%%%%%%%%%%%%%%%%%%%%%%%%%%%%%

% \newtcolorbox[blend into=figures]{myfigure}[2][]{float=htb,capture=hbox,
% title={#2},every float=\centering,#1}

在
% 
\end{document}
%cd /Volumes/RamDisk/ &&  xelatex --output-directory=/Volumes/RamDisk/ -synctex=1 -shell-escape  /Users/virhuiai/hlProjects/Latex-Typesetting-Hub/练习书本排版/iText_in_Action_Second_Edition/iText_in_Action_Second_Edition.tex