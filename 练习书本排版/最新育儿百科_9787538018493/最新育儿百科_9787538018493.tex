\PassOptionsToPackage{no-math}{fontspec}%禁用了使用fontspec宏包中的数学字体功能。
\PassOptionsToPackage{AutoFakeBold=true,AutoFakeSlant=true}{xeCJK}%让xeCJK宏包自动产生伪粗体和伪斜体效果。

\documentclass{book}%The current text width is \the\textwidth.%345.0pt.
% \setlength{\textwidth}{40em}%400pt
% \setlength{\textwidth}{410pt}
\usepackage[textwidth=410pt]{geometry}

% \usepackage[paperwidth=9.0cm,paperheight=11.7cm,%
% margin=0in,left=0.1cm,right=0.1cm,top=0.1cm,bottom=0.2cm
% ]{geometry}

\usepackage[heading=true
,scheme=chinese%中文方案
,fontset=none%不使用默认的字体设置
,space=auto%自动调整中英文间距
]{ctex}
\CTEXsetup[name={第~,~篇}]{chapter}
% \CTEXsetup[name={第~,~篇\newline\small}]{chapter}
\CTEXsetup[name={第,节},number={\chinese{section}}]{section}
\CTEXsetup[name={,、},number={\chinese{subsection}}]{subsection}
\setcounter{secnumdepth}{3}
% \renewcommand{\thesubsubsection}{\arabic{subsubsection}}
\CTEXsetup[name={,.},number={\arabic{subsubsection}}]{subsubsection}

\setCJKmainfont[Path=/Users/virhuiai/hlProjects/Latex-Typesetting-Hub/font/方正/]{FangZhengShuSong-GBK-1.ttf}%设置文本的中文有衬线字体
\setCJKsansfont[Path=/Users/virhuiai/hlProjects/Latex-Typesetting-Hub/font/方正/]{FangZhengHeiTi-GBK-1.ttf}%设置文本的中文无衬线字体为
\setCJKmonofont[Path=/Users/virhuiai/hlProjects/Latex-Typesetting-Hub/font/方正/]{FangZhengFangSong-GBK-1.ttf} %设置文本的中文等宽字体 
% 我们首先设置了三种主要的字体:正文 (mainfont)、无衬线字体 (sansfont) 和等宽字体 (monofont)。
\newCJKfontfamily\PinyinTi{ToneOZ-Pinyin-WenKai}[
Path=/Users/virhuiai/hlProjects/Latex-Typesetting-Hub/font/澳声通拼音文楷/,
Extension      = .ttf,
UprightFont    = *-Regular,
BoldFont       = *-Bold,
% ItalicFont
% BoldItalicFont
% BoldFont = ⟨font name⟩ 
% ItalicFont = ⟨font name⟩ 
% BoldItalicFont = ⟨font name⟩ 
% SlantedFont = ⟨font name⟩ 
% BoldSlantedFont = ⟨font name⟩ 
% SwashFont = ⟨font name⟩ 
% BoldSwashFont = ⟨font name⟩ 
% SmallCapsFont = ⟨font name⟩ 
% UprightFont = ⟨font name⟩
]


% \usepackage{newclude}%后面 \include* 引用就不会自动分页了
%\includeonly{1}%加上这句,就只有1的两个引入有效果了

% \makeatletter
% \providecommand*\input@path{}
% \newcommand*\addinputpath[1]{\expandafter\def\expandafter\input@path\expandafter{\input@path#1}}
% \makeatother

% \addinputpath{%
% {/Users/virhuiai/hlProjects/Latex-Typesetting-Hub/练习书本排版/iText_in_Action_Second_Edition}%
% }

\usepackage[all]{tcolorbox}
% % \tcbuselibrary{documentation}

\begin{document}

\newtcolorbox{mybox}[2][]{colback=red!5!white, colframe=red!75!black,fonttitle=\bfseries, colbacktitle=red!85!black,enhanced,
attach boxed title to top center={yshift=-2mm},
before upper={\parindent2em}, % 设置段落缩进
title={#2},#1}

% \begin{mybox}[colback=yellow]{}
% \subsubsection*{}
% \parindent=2em
% \end{mybox}


% % 第一篇婴幼儿的发育特点.tex

%第一篇婴幼儿的发育特点
\chapter{婴幼儿的发育特点}

%第一节
\section{婴幼儿年龄分期}

婴幼儿时期是体格发育和神经智能发育最旺盛和最迅速的时期。随着身体各个器官不断发育长大,生理功能逐步成熟,身心几乎每月甚至每天都在不断发生变化,各个时期的解剖、生理和病理特点又都是极不相同的。为了便于了解和区分不同年龄婴幼儿生长发育的特点和规律,我们将婴幼儿分为新生儿期、婴儿期和幼儿期三个时期,现简单介绍如下。

% 一、
\subsection[新生儿期]{新生儿期-\small 从出生到满28天}

从出生到满28天,这个时期的宝宝被称为新生儿。新生儿从母体温暖安静的小环境中突然进入到寒冷嘈杂的大环境中,是十月怀胎后的第一场人生考验,对于宝宝来说需要全身心地投入和机体的动员以适应外部世界。由于内外环境的突然巨变,小宝宝机体内尚未建立完整和健全的调节能力和适应能力,特别容易发生心、脑、肺及全身性疾病而导致死亡。在许多发达国家都建立了围生期保健网和新生儿监护中心,大大降低了新生儿的死亡率。在我国由于地区不同,其医疗卫生条件、经济状况及地理环境等的不同而差异非常大,如北京、上海等大城市婴儿的死亡率就明显低于农村。

经专家研究发现,新生儿不仅具有视听能力,还有惊人的模仿和记忆能力,早期开发新生儿大脑功能已经得到越来越多的专家和学者注意,并得到了广泛的社会关注。在新生儿医学方面,已经逐步由单一的体格发育、营养发育的研究转为智力潜能开发的研究。


% 二、
\subsection[婴儿期]{婴儿期-\small 28天到1岁}

婴儿期是指从生后28天到1岁的时期,这个时期的宝宝称为婴儿。因为这个时期的宝宝主要以\textbf{乳制品}为主要食物,也有人称为乳儿期。这个时期宝宝体格发育和智能发育的速度是最快的,几乎每天都有新的变化令父母惊奇和高兴。宝宝的体重可增加2\textasciitilde4倍,身长增加大约是出生时的50\%,脑重量的发育从400克达到1000克左右,各个器官已经具有初步的调节功能和适应能力。宝宝从每天单一枯燥的仰卧位姿势发展到能够随意爬行,独自站立和迈步等运动形式。已经学会用手势表达自己的愿望,并发出单音节音表示喜怒哀乐。此阶段是大脑早期开发的最佳时期,家长必须重视。

婴儿期最好采用\textbf{母乳}喂养,母乳中的大量\textbf{抗体}和从\textbf{母体输送给宝宝的抗体}足够宝宝维持出生后\textbf{半年}的抗病能力,而半年后则需要依靠宝宝自身的抵抗力和预防接种来抵御各种疾病。婴儿期的合理喂养和疾病预防是家长至关重要的理解项目。

% 三、
\subsection[幼儿期]{幼儿期-\small 1\textasciitilde3岁}

幼儿期是指1\textasciitilde3岁的时期,这个时期内的宝宝被称为幼儿。在这期间宝宝的体格发育速度明显减慢,但脑神经的发育仍在快速发展,到2岁时宝宝的脑细胞数量已经增殖完毕。通过与外界的接触和交流,宝宝的认知能力,语言和思维能力,分析和判断能力明显增强。因此,幼儿期增加与外界的交流,接触大量的与人类有关的事物是开发宝宝智力的关键。

幼儿期宝宝的饮食从单纯母乳喂养逐渐过渡到与成人相似的普通膳食。\textbf{合理}的食品营养搭配,防止营养摄入不平衡而出现营养物质的缺乏和过剩是本阶段的重点。正确引导宝宝\textbf{建立良好的饮食习惯},才能为今后体格发育和智能发育打下良好基础。

\subsection{婴幼儿生长发育特点}

大自然中任何事物的发生和发展都有其一定的规律,人类的生长发育也同样遵循自然发展的规律。

%1
\subsubsection{器官发育顺序和速度}

人类自怀胎开始首先发育的是大脑和神经系统,其次是胸腹内脏器官和肢体骨骼系统。在胎儿形成过程中,胎儿大脑发育最早,神经管在怀孕3个月发育成型,出生时大脑有$350\sim{}400$克,$6$个月时增加到$600$克左右,而到$1$岁时已经增加至$1000$克左右(成人约为$1400$克),约为成人脑重量的$70\%$,而1岁宝宝的体重仅为成人的$15\%\sim{}20\%$。这足以说明人类发育初期大脑的发育速度是非常惊人的,这也是婴幼儿早期智力开发的重要依据之一。

肢体骨骼发育的年龄越小,发育的速度越快,以体重为例,出生第一年增长最快,为出生时的$2\sim{}4$倍,$2$岁以后增长速度明显减慢,每年增加2千克左右。在出生后的$2\sim{}3$年时间里,婴儿大脑发育速度明显快于体格发育,到3岁以后大脑细胞已经增殖完毕,宝宝已经具备了基本的运动、语言和思维能力,而体格发育并未达到相适应的程度。因此,有一些父母会有这样的感觉,``这么小的孩子怎么会说出这么成熟的话''。其实,这正是人类发育中的正常现象,千万不要误认为自己的宝宝比别人的宝宝聪明而放松教育。【\textbf{不要迷之自信},是正常】


%2
\subsubsection{运动和智能发育特点}%\label{ux8fd0ux52a8ux548cux667aux80fdux53d1ux80b2ux7279ux70b9}}

\begin{description}
\item[由上到下]
婴儿从头到脚的顺序开始生长发育。比如,3个月时会趴着抬起头,4个月时抬起头的同时还能带动胸部抬高,5个月时可以用胳臂支撑翻过身来,6个月会坐,10个月能站立,1岁会主动用脚迈步等。一系列运动发育就是\textbf{由颈部逐渐发展到上肢},又进一步发展到下肢的延伸过程。

\item[由近到远]
从身体的中心部位开始发展到四肢的过程。例如,新生儿时听到声音或朝有光亮的方向转头,这是颈背部肌肉在运动的结果,$1\sim{}2$个月看人或物时会扭动肩部和腰部来试图与人交流,$3\sim{}4$个月用手臂够喜欢的物品,$5\sim{}6$个月时会主动用手掌抓握东西,$7\sim{}8$个月时用大拇指和其他手指抓住小东西等。这些都是从身体中心的大肌肉逐渐发展延伸到四肢末端小肌肉的过程表现。

\item[由粗到细]
最典型的例子就是由粗大的动作发展到精细的动作过程。宝宝开始只能用手大把抓东西,不知不觉中就能拇指分开捡起床面或桌面上散落的小颗粒物品。年轻的父母看到宝宝这些以前尚未出现的动作时会情不自禁地说``我们的宝宝又进步了''。宝宝眼睛的分辨过程也是这样,刚出生的宝宝只能看到人面部的轮廓,到2个月时已经能清楚辨认人的面部,3个月时又进一步能区分母亲与其他亲人的面孔。宝宝由粗糙的大体视野发展到细致的局部视野过程其实就是一个由粗到细的观察过程。【渐渐迭代,先框架,后细节\ldots】

\item[从简单到复杂]
在宝宝的运动、语言发育过程和认识事物过程中处处体现了这种发展规律。比如从牙牙学语到会背诵诗歌,用笔乱画到有意识地画出直线和圆圈,再发展到能画出漂亮的图形等。

\item[从低级到高级]
这是人类与动物的最大区别所在。动物能听到自然界的声音,但是它们不会运用声音创造出表达完整意思的语言。而人类则不同,从仅能听到声音发展到会模仿声音,并创造出最优美最动听的能够表达任何意愿的语言。从仅仅会使用简单的工具到会制造复杂的工具,直到现在人类发明的电子信息技术,都是一部人类由低级到高级的发展史。
\end{description}

%3
\subsubsection{体格和智能发育的个体差异}%\label{ux4f53ux683cux548cux667aux80fdux53d1ux80b2ux7684ux4e2aux4f53ux5deeux5f02}}

世间每一件事物都是不尽相同的,万物生长各有其不同的特点。虽然都是人类,但由于基因的不同,人与人之间有着千差万别,这就是个体差异。个体差异受到遗传基因、居住环境、营养状况及教育程度等诸多方面因素的影响。例如,身高方面,北方人普遍比南方人偏高,这与遗传和居住环境有很大的关系。再比如,具有某些``天赋''的人,像音乐神童莫扎特这样的音乐天才毕竟只是凤毛麟角,这些人的智商和一般人有区别,不能盲目效仿。有些父母望子成龙,从宝宝很小就开始拼命加码,希望孩子能够成为莫扎特式的人物。然而,结果往往不仅没有达到预期的目的,反而给宝宝幼小的心灵造成了很大的伤害,甚至影响了孩子的终生。因此,对于自己的宝宝的体格发育和智能发育,一定要结合孩子的兴趣、家庭和生活居住环境等多种因素来培养和教育,使宝宝在德智体三方面得到均衡发展。


% \begin{quote}
% gpt:

% 个体差异主要表现在以下几个方面:

% \begin{itemize}
% \item
% 发育速度:即便是同一种群体,每个个体的发育速度也会有差异。有的孩子可能会早早学会走路,而有的孩子可能会稍微晚一些。这并不意味着发展晚的孩子就比发展早的孩子差,只是每个人的发展速度不同而已。
% \item
% 特长和优点:每个人都有自己的特长和优点,这是由于遗传和环境因素共同作用的结果。例如,有的孩子可能有很好的语言能力,而有的孩子可能在数学或者绘画上有独到之处。
% \item
% 性格特质:每个人的性格都是独一无二的,这也是个体差异的一部分。有的孩子可能生性活泼开朗,善于交朋友;而有的孩子可能内向腼腆,喜欢独处。
% \item
% 学习方式:人们的学习方式也各不相同。有的人通过观察和实践来学习,而有的人则通过阅读和理论学习。理解和接受新知识的方式也会因人而异。
% \end{itemize}

% 这些个体差异是非常正常的,我们应该尊重并欣赏这些差异,因为正是这些差异使我们每个人都是独一无二的。在教育和培养孩子的过程中,我们应该注意到并尊重他们的个体差异,鼓励他们充分发挥自己的特长和优点,同时帮助他们改进弱点和不足。
% \end{quote}



%五、
\subsection{影响婴幼儿生长和发育的因素}

%1
\subsubsection{遗传因素}%

婴幼儿生长发育的特征,如皮肤和头发的颜色、身高、体重、相貌、性格等均受到父母双方遗传因素的影响。遗传性疾病,如染色体异常(先天愚型)等和代谢性疾病(苯丙酮尿症等)对孩子的生长发育均有显著的影响。


\subsubsection{性别因素}%2 

男孩和女孩在婴幼儿时期的生长发育特点是有很大区别的。男孩比较调皮,不守规矩,注意力不容易集中,因而男孩的模仿能力发育一般较女孩的发育慢一些,如语言发育,背诵诗歌,学唱儿歌等。男孩比较好动,肌肉骨骼发育相对旺盛,在运动发育方面,一部分男孩可能要比女孩快一些。从体格发育情况来看,刚刚出生的男宝宝平均体重会比女宝宝重一些,身长也略微长一些,以后整个婴幼儿期都是这样一种发育规律。因此,在评价婴幼儿的体格和智能发育水平时,一定要参照男孩和女孩的不同标准进行判断。


\subsubsection{营养因素}%3. 

营养因素对于婴幼儿来讲非常重要。营养是人类生命物质的基础,是婴幼儿身体健康必不可少的关键条件。营养充足可以使机体发育达到最佳状态,尤其对大脑的发育更是如此。从怀孕期到生后两周岁的这段时间里,供给足够的营养物质会促进脑细胞的生长发育。怀孕期营养不良的胎儿出生时不仅身材矮小、体重轻,而且还会影响脑神经的正常发育,严重者可导致脑神经损害后遗症。生后第1\textasciitilde2年如果出现营养不良,就会影响婴幼儿脑细胞的进一步增殖和增大,引起脑细胞数量的匮乏和细胞体积的缩小,使婴幼儿生长发育受到严重影响。


\subsubsection{家庭环境和社会环境}%4.

在一个家庭中,父母亲的教育程度,兄弟姐妹之间的亲情,家庭的氛围对宝宝的生长发育和心理发育是非常重要的。良好的生活环境和充满爱心的正确引导会使宝宝的体格和智力潜能得到最佳的发展。周围环境和宝宝所处的时代也很重要。


\subsubsection{怀孕时机与孕期情况}%5. 

受孕时机最好选择在父母双方生理状态最好和心理情况最稳定的时候。受孕前后,如果过度饮酒或抽烟都会影响宝宝的顺利出生。怀孕期间,孕妇的家庭环境、周围生活环境、营养状况、心理情绪及有无疾病等都会对即将出生的宝宝有影响。怀孕早期如果叶酸缺乏可导致胎儿神经管畸形;病毒感染可引起先天性心脏病、脑发育畸形和早产;服用某些有毒药物、有害化学物质侵袭及放射线辐射等都可以影响胎儿或出生后的宝宝生长发育。精神状态对宝宝的影响也是很大的,抑郁的母亲可能使自己的宝宝性格发生异常;精神有障碍的母亲的宝宝往往胆小懦弱,缺乏自信,同时容易出现精神异常等。


\subsubsection{疾病}%6. 

许多先天性疾病都会影响宝宝的智能发展。先天性心脏病不仅影响宝宝的身体发育,长期缺氧还会影响宝宝的大脑发育,使宝宝反应迟钝,接受事物能力差;某些染色体异常、畸形或先天性代谢性疾病可导致宝宝大脑发育迟缓,智力低下,终身不能独立生活。后天的营养也是至关重要的,缺钙导致佝倭病、营养不良性贫血、微量元素的缺乏等,都会给宝宝的体格和智力发育带来不利的影响。

‍
% % 四、
  


% \setcounter{chapter}{1}
% \setcounter{section}{2}
% \setcounter{subsection}{4}
% \setcounter{subsection}{4}
% \setcounter{page}{8}
\setcounter{chapter}{1}
\setcounter{section}{2}
\setcounter{subsection}{11}
\setcounter{subsubsection}{3}
\setcounter{page}{10}



% 4.
\subsubsection{头部肿物}

头部肿物有两种:①产瘤。是胎儿在分娩过程中由于产道的挤压导致头部皮下的淤血。从表面上来看,单侧或者双侧头部肿胀,没有边界感。一般\texttt{1\textasciitilde{}2}\hspace{0pt}天就可以吸收,无需治疗。②头颅血肿。胎儿头部在经过产道挤压时,引起头皮下血管破裂,导致颅骨和骨膜下出血。出生后即可触及,\texttt{1\textasciitilde{}2}\hspace{0pt}个月自行消失。有时合并严重黄疸。一般小血肿可以自行吸收,过大的血肿需要到医院处理,以防止严重的黄疸造成脑后遗症。

% 5.
\subsubsection{新生儿生理黄疸}%ux65b0ux751fux513fux751fux7406ux9ec4ux75b8}}

生理性黄疸是新生儿普遍出现的一种生理现象,一般出现在生后\texttt{2\textasciitilde{}3}\hspace{0pt}天,最迟可到第5天出现,皮肤黄染从脸部开始,逐渐波及胸、腹及四肢上端,呈浅黄色,皮肤红润,有光泽,一般情况在出生后\texttt{7\textasciitilde{}14}\hspace{0pt}日自然消失。早产儿的生理性黄疸更加深一些,消退时间亦可能更长,需要\texttt{7\textasciitilde{}21}\hspace{0pt}日。

生理性黄疸特点是皮肤黄染比较浅淡,通常只出现在脸、躯干及四肢上端,\textbf{不累及巩膜}。如果皮肤黄染在24小时之内出现,或皮肤颜色比较深,或呈黄绿色,四肢下端及手、足心皮肤均有黄染,而且巩膜也有黄染,说明黄疸已经超过了生理性黄疸的范围,应及时抱到医院诊治。

生理性黄疸不需治疗,提早开奶及频繁喂奶可使黄疸程度减轻及消退加快。

有一些合并有疾病或血型不合的新生儿发生黄疸时需要严密观察,如出现窒息、缺氧、严重感染,ABO型或Rh血型不合者,应及时到医院就医,以免延误治疗。

生理性黄疸无须治疗即可自愈。但当皮肤的黄疸已经超过了生理性黄疸的范围,就变成了病理性黄疸。医院测定血清胆红素高于205.2毫摩尔/升就属于病理性黄疸。一些危重病儿的血清胆红素可能高于342毫摩尔/升。

血液中的胆红素非常容易与富含脑磷脂的脑组织相结合,干扰神经细胞的正常代谢,对脑细胞产生不可逆的毒性作用。轻者可能出现轻微脑功能失调,体格发育迟缓,牙齿发育不良。重者可能出现严重的神经系统症状,如脑性瘫痪,智力障碍,以及语言、行为、识别能力紊乱及学习障碍等。

% 6. 
\subsubsection{母乳性黄疸}%ux6bcdux4e73ux6027ux9ec4ux75b8}}

母乳性黄疸是由于母乳喂养而导致的新生儿皮肤出现黄染。母乳性黄疸儿有以下几个特点:

\begin{enumerate}
% \def\labelenumi{\arabic{enumi}.}
\item
  皮肤黄染颜色呈淡黄色,皮肤色泽好,面色红润。
\item
  出生3\textasciitilde7日,皮肤黄染不减轻。
\item
  出生3\textasciitilde7日黄疸减退后又加重。
\item
  大约在生后2周时皮肤黄染开始减轻,但黄疸减退缓慢,有的婴儿可持续30\textasciitilde90日。
\item
  如果停喂母乳改换人工喂养\texttt{2\textasciitilde{}3}\hspace{0pt}日,皮肤黄染明显减轻,如果再继续喂母乳,
  皮肤黄染又会略有加重。
\item
  新生儿除黄疸之外一般情况好,无贫血及溶血表现。无感染等病史。
\end{enumerate}

\begin{mybox}[colback=yellow]{育儿小百科}
\subsubsection*{正确对待母乳黄疸}
许多家长一听说自己的孩子患母乳性黄疸,就会担心这会对孩子有什么影响,其实这些担心都是多余的。目前,国内外的诸多临床观察中尚未见到一例因母乳性黄疸而导致死亡或遗留有后遗症,有专家建议将母乳性黄疸不归入疾病类之中,而当作一种特殊的生理状态来对待。因此,如果怀疑为母乳性黄疸,皮肤黄染明显时可暂停母乳喂养2
\textasciitilde{}
3日,待黄疸减轻一些再恢复母乳喂养。如果皮肤黄染不明显可继续母乳喂养观察至2\textasciitilde3个月,如果3个月之后黄疸仍不消失,则应到医院做相应检查。
\end{mybox}




% %ux8fc7ux5206ux54edux95f9ux4e0eux8fc7ux5206ux5b89ux9759}{%
% \subsubsection{7.
% 过分哭闹与过分安静}%ux8fc7ux5206ux54edux95f9ux4e0eux8fc7ux5206ux5b89ux9759}}

% \hspace{0pt}\includegraphics[width=3.32867in,height=2.93706in]{media/rId154.png}\hspace{0pt}

% 婴儿出生后最重要的表达方式就是啼哭。啼哭对婴儿是有好处的,首先\textbf{啼哭}对孩子来说是一种\textbf{全身的运动},啼哭时孩子往往要四肢舞动,躯干晃动,同时还能增加肺活量,加强呼吸功能。一些家长生怕宝宝大声哭闹,只要一哭就赶快将孩子抱起,其实没有必要,孩子啼哭是向父母亲表达意愿的一种方式。孩子感到饥饿、口渴、困倦、排尿、排便、寒冷、燥热、寂寞时候都会用哭声表达,这些都是属于生理性表现,只要护理得当,婴儿大多会很快安静下来。

% 如果孩子有病,如发热、腹泻、胆红素脑病等身体感到不舒服的时候,往往哭闹很难阻止,只有当疾病痊愈以后,孩子才会恢复正常。

% 这里需要提到的一点是这种过分长时间哭闹与小婴儿定时定点哭闹是不同的。经常碰到家长问这样的问题,``我的孩子很奇怪,每天总是定时定点哭闹,怎么哄也不行,哭闹一段时间后,不知怎的自己也就好了''。有的家长因宝宝哭闹不止急急忙忙抱孩子上医院,可孩子一坐上车就安静了,到医院时已经睡得很香了,叫都叫不醒。其实,\textbf{孩子这种定时定点哭闹,完全可以看作是一种正常的生理现象。}表现有下述几种:

% \begin{enumerate}
% \def\labelenumi{\arabic{enumi}.}
% \item
%   傍晚哭闹:傍晚哭闹的孩子多数\textbf{白天睡眠不好},到傍晚时非常\textbf{疲乏},要求安静睡眠,而这时往往又是父亲或其他家人下班归来的时候,因而孩子很厌倦,这样的孩子往往在傍晚5
%   \textasciitilde{} 7点哭闹一阵。
% \item
%   夜哭:有些宝宝白天很好,可夜里的某一时刻哭得很厉害,令父母非常不安,担心是不是孩子得了什么病,如佝偻病。有些专家认为,这可能是由于宝宝\textbf{白天生活不规律或者母亲离开宝宝时间过长},导致孩子惊慌或焦躁不安所致。还有人认为,可能是\textbf{一种姿势躺得时间过长},压迫后背肌肉引起不适所致。这时家长不必惊慌,如果孩子脸色正常,\textbf{抱起来哄片刻就可以安静入睡}。不断喂奶,不断爱抚,反而加重宝宝的厌烦情绪,使宝宝的夜哭时间更长。
% \item
%   哭闹有规律:一些孩子哭闹非常有规律,每天在固定的某一时刻连续哭闹不止。啼哭时,孩子绷直双腿,小胳膊也在奋力挥舞,使出最大的气力啼哭,给奶不吃,给水不喝,经过一段时间后哭声逐渐变小,最后疲乏地进入睡眠状态。医学上认为,这种情况可能是因为喂奶时\textbf{吞气过多},肠蠕动过快,致使肠胀气,或便秘所致。这种孩子往往在不哭的时候表现很好,生长发育正常,3个月以后哭闹就逐渐减少了。
% \end{enumerate}

% \begin{quote}
% 你知道吗?

% \textbf{婴儿的哭闹与性格有关。}

% 婴儿的哭闹和安静程度当然也和婴儿的性格有关,有些婴儿\textbf{性格比较急},哭闹起来自然也比较厉害,安抚起来相对困难一些。一些婴儿性格比较平和,哭闹起来声音可能比较温和,安抚起来也比较容易。但是,如果婴儿一天中除了吃喝拉撒睡,总是默默无声,很少哭闹,过分安静也应该引起注意,如先天性耳聋、甲状腺功能低下等疾病早期表现就是过分安静。
% \end{quote}

% %ux524dux56dfux8fc7ux5c0fux6216ux524dux56dfux8fc7ux5927}{%
% \subsubsection{8.前囟过小或前囟过大}%ux524dux56dfux8fc7ux5c0fux6216ux524dux56dfux8fc7ux5927}}

% 我们都知道,人的头颅骨不是一块完整的骨头,而是由几块骨头拼装而成的。前额部的骨头称为额骨,刚出生时在双眉上方一分为二,两块骨头中间的缝隙叫矢状缝。头顶部的骨头称为顶骨,也是左右两块,顶骨与额骨之间的骨缝(与脸平行方向)叫冠状缝。这四块骨头相连接的中心部位就称之为前囟,出生时为1.5厘米x
% 2厘米。自然顺产的孩子经产道挤压,可以使这几块骨头之间的缝隙缩小,甚至可能出现重叠,因此经产道分娩的宝宝刚出生时前囟可以偏小,但随着脑组织和颅骨的发育又可能增大一些,而剖宫产的孩子由于没有经过产道挤压,所以出生时骨头之间的缝隙较宽,这时的前囟往往偏大,甚至可达到3厘米x
% 3厘米以上。骨缝间隙也可达到0.5厘米以上。

% 所以,刚出生时的前囟过大或过小并不能表明有什么异常,可以观察半年左右。前囟闭合时间非常重要,婴幼儿通常在1-1.5岁闭合。如果闭合过早(特别是小于6个月),将使脑组织扩充变小,影响大脑的整体发育,一些孩子形成``小头畸形'',造成智力障碍。超过1岁半仍未闭合为闭合延迟,多因佝偻病或先天性甲状腺功能低下所致。

% %ux65b0ux751fux513fux5148ux5929ux6027ux75beux75c5ux7684ux7b5bux67e5}{%
% \subsubsection{9.
% 新生儿先天性疾病的筛查}%ux65b0ux751fux513fux5148ux5929ux6027ux75beux75c5ux7684ux7b5bux67e5}}

% 在医院出生的新生儿都要做足跟血检查。很多家长不明白这是为什么?其实足跟血的检查主要是为了筛查能够引起小儿智力障碍的某些先天性疾病。从20世纪80年代起,国际上就能同时筛查出多种疾病。目前,我国主要筛查两种疾病,即苯丙酮尿症和先天性甲状腺功能减低症(俗称``呆小症'')。

% \begin{enumerate}
% \def\labelenumi{\arabic{enumi}.}
% \item
%   苯丙酮尿症:苯丙酮尿症是一种先天性氨基酸代谢异常性疾病。刚出生时外貌正常,到3〜6月时渐渐出现智力发育不全的表现。因症状不典型,家长往往到1岁以后才发现自己的孩子比别的孩子发育差,而这时孩子早已过了治疗的最佳时期,再经努力也无济于事,孩子越大智力障碍越明显,存活的儿童基本没有生活自理能力。导致大脑损害的主要原因是体内的苯丙氨酸浓度过高,进入到脑组织,引起神经系统损害。苯丙氨酸在牛奶及肉类、鸡蛋类食品中浓度很高,如果出生后尽早做筛查,早期诊断本病就可以严格限制苯丙氨酸食品的摄入,如哺喂低苯丙氨酸奶粉,添加辅食时以淀粉类、蔬菜和水果为主,就可以减少体内苯丙氨酸含量。治疗及时,方法得当,孩子完全可以发育成为正常人。
% \item
%   甲状腺功能减低症(呆小症):这是由于甲状腺激素缺乏引起的疾病。1岁以内症状不典型,往往到行走时才发现智力落后于正常婴幼儿。随年龄增长逐渐出现智力低下、学习成绩差和身材矮小。如果出生后能够早期发现,早期治疗,小儿发育完全可以达到正常水平。
% \end{enumerate}

% \begin{quote}
% gpt:

% 因此,新生儿的足跟血检查是非常重要的,可以早期发现并预防一些可能导致智力障碍的疾病。
% \end{quote}

% 1

% %ux7b2cux4e09ux8282-ux5a74ux513fux671fux5404ux9636ux6bb5ux7684ux751fux957fux53d1ux80b2ux7279ux70b9}{%
% \subsection{3第三节
% 婴儿期各阶段的生长发育特点}%ux7b2cux4e09ux8282-ux5a74ux513fux671fux5404ux9636ux6bb5ux7684ux751fux957fux53d1ux80b2ux7279ux70b9}}

% 婴儿期是指宝宝来到世界的第一年。在这一年中宝宝经历了大量的事情和收集到大量信息,可以说是人生中发育最迅速的阶段。在不断地与成人的交往中,大脑和身体逐渐发育,同时各个器官的功能也不断完善和成熟。在这一年里,宝宝学会了俯卧抬头、抬肩和抬胸,学会了翻身和爬行,独自坐位玩耍,站立和迈步等粗大动作。从用手掌大把抓到拇指与食指分开对捏拿一些很小的东西,如花生、纽扣等精细动作。语言发育方面能够听懂大人说话,并用手势或表情表达自己的意志和愿望,开始学用语言和周围人进行交往。一年中孩子天天都在进步,时时都在变化,本文将婴儿一年中的发育过程按月进行归纳和总结,便于家长了解和掌握自己宝宝的生长发育情况,更好地教育和培养他们,使每一位宝宝都能茁壮成长。

% %ux4e001ux4e2aux6708ux5a74ux513fux751fux957fux53d1ux80b2ux72b6ux51b5}{%
% \subsection{一、1个月婴儿生长发育状况}%ux4e001ux4e2aux6708ux5a74ux513fux751fux957fux53d1ux80b2ux72b6ux51b5}}

% 刚刚度过新生儿期,宝宝又有了很多的进步。头部在仰卧位时能够自由转动,头部可以自行竖立2
% \textasciitilde{} 3秒钟,俯卧位时宝宝可以抬起头部1\textasciitilde{}
% 2秒钟。宝宝可以抓住大人的手指头,能够分辨人脸的轮廓,注视红颜色的物体,能够辨别甜和酸的味道,能发出细小的喉音,听到父母的声音会露出微笑。

% %ux4f53ux683cux53d1ux80b2}{%
% \subsubsection{1. 体格发育}%ux4f53ux683cux53d1ux80b2}}

% 1个月的宝宝体重可以增长\texttt{700\textasciitilde{}1000}\hspace{0pt}克,母乳充足的宝宝体重能增长1000克以上,身高可以增长\texttt{2\textasciitilde{}5}\hspace{0pt}厘米,头围可以增长\texttt{2\textasciitilde{}4}\hspace{0pt}厘米,部分宝宝后囟闭合。

% %ux8fd0ux52a8ux53d1ux80b2}{%
% \subsubsection{2.运动发育}%ux8fd0ux52a8ux53d1ux80b2}}

% \textbf{头颈部转动}:宝宝有了一定的运动能力,主要表现在头颈部。仰卧位时能非常自如地转动头颈部,当听到声音时会很快转动脖子向声音方向望去。大人轻轻握住宝宝的两只手腕,可以很轻松地将宝宝从仰卧位拉到坐位。

% \textbf{头竖立}:宝宝的头部可以自行竖立2 \textasciitilde{}
% 3秒钟。俯卧位时,宝宝可以抬起头部1\textasciitilde2秒钟,但不能离开床面。竖着抱起时,宝宝可以将头竖立。

% \textbf{手握持}:仰卧位时,大人将食指放入宝宝手掌中,宝宝能用小拳头紧紧握住食指不放。

% %ux89c6ux89c9ux5473ux89c9ux53d1ux80b2}{%
% \subsubsection{3.
% 视觉味觉发育}%ux89c6ux89c9ux5473ux89c9ux53d1ux80b2}}

% \textbf{分辨人脸}:刚刚出生的宝宝就能够分辨人脸的轮廓,能够清楚地分辨大人是否戴眼镜或戴口罩。有人观察发现,当宝宝所熟悉的人突然戴上眼镜或者口罩时,宝宝就会露出非常惊讶的表情。

% \textbf{注视红色}:婴儿首先认识的是红颜色,之后才会逐渐认识黄、蓝、绿色等,因此对以红颜色为主的图画及玩具会注视很久。当妈妈身穿红色衣服进入宝宝视线时,宝宝会用眼睛跟着妈妈走动的方向来回转动。

% \textbf{辨别甜酸味}:宝宝能够辨别甜和酸的味道,如果开始就给宝宝饮用酸甜的果汁,再给宝宝喝一般的白开水就会遭到拒绝,以哭闹来表示自己的意愿。

% %ux8bedux8a00ux53d1ux80b2}{%
% \subsubsection{4. 语言发育}%ux8bedux8a00ux53d1ux80b2}}

% 能发出细小的喉音。有的妈妈说:``我一说话宝宝就`欧、欧'个不停,好像他明白我的话。''这种理解也许有道理。但是,目前认为这种喉音是无意识的,与大人说话无关。有人认为,这是婴儿对声响的条件反射,这个观点还有待于进一步的研究。

% %ux8ba4ux77e5ux4e0eux751fux6d3bux4ea4ux5f80ux80fdux529b}{%
% \subsubsection{5.
% 认知与生活交往能力}%ux8ba4ux77e5ux4e0eux751fux6d3bux4ea4ux5f80ux80fdux529b}}

% 宝宝喜欢听到父母或亲人的声音。一旦听到亲人的声音,宝宝就会露出微笑。当宝宝清醒时,如果妈妈把脸贴近宝宝,并与宝宝微笑或说话,宝宝也会回应微笑,或者露出高兴的表情。

% %ux4e8c2ux4e2aux6708ux5a74ux513fux751fux957fux53d1ux80b2ux72b6ux51b5}{%
% \subsection{二、2个月婴儿生长发育状况}%ux4e8c2ux4e2aux6708ux5a74ux513fux751fux957fux53d1ux80b2ux72b6ux51b5}}

% 宝宝能够俯卧抬头,头部很容易竖直片刻。宝宝可以用手握住玩具棒,发出``啊、欧、哦''等声音;可以看清楚稍微远一点的地方,用眼睛跟随着大人的身影转动。对声音有反应,头部的转动能超过180度;注意力会十分集中注视鲜艳的物体。喜欢有大人陪伴。

% %ux4f53ux683cux53d1ux80b2-1}{%
% \subsubsection{1. 体格发育}%ux4f53ux683cux53d1ux80b2-1}}

% 从第2个月至第6个月阶段,宝宝的体重增长速度会较第1个月减慢,平均每月增长\texttt{600\textasciitilde{}800}\hspace{0pt}克,身高平均每月增长\texttt{2\textasciitilde{}3}\hspace{0pt}厘米,头围平均每月增长\texttt{2\textasciitilde{}2.5}\hspace{0pt}厘米。到6个月时,头围可以达到\texttt{42\textasciitilde{}44}\hspace{0pt}厘米。后囟已经闭合。

% %ux8fd0ux52a8ux53d1ux80b2-1}{%
% \subsubsection{2. 运动发育}%ux8fd0ux52a8ux53d1ux80b2-1}}

% \begin{itemize}
% \item
%   \textbf{俯卧抬头}:颈部运动变得更加灵活。当宝宝俯卧时,头部可以离开床面2\texttt{\textasciitilde{}}\hspace{0pt}3秒,有的宝宝甚至还能转动一下头部。从本月开始可以练习俯卧抬头,选择宝宝清醒时、喂奶1\texttt{\textasciitilde{}}\hspace{0pt}2小时后的时间,将宝宝俯卧平放到床上,双手放到头部的两侧。父母站到婴儿头部的一侧,与宝宝说话或者用颜色鲜艳的玩具逗引宝宝,鼓励宝宝抬头。有些2个月的婴儿,由于家长未注意训练宝宝的抬头动作,到门诊时检查发现宝宝俯卧抬头动作非常不好,但经过上述方法的训练,2\texttt{\textasciitilde{}}\hspace{0pt}3天之后宝宝就会抬头了。
% \item
%   \textbf{头能竖直片刻}:将宝宝从仰卧位拉到坐位时,宝宝的头部可以自行竖立5秒钟以上。抱起宝宝头部很容易竖直片刻,但要注意抱起宝宝时一定要保护好宝宝的身体,可以将宝宝的前胸或后背紧紧靠住妈妈的身体,用一只手托住宝宝的臀部,然后试着竖起宝宝的头部。如果宝宝头颈部不稳,千万不要强行竖立,以免发生颈部和脊柱损伤。有一部分婴儿可能不会做上述的动作,但没有关系,只要加强训练,大部分的宝宝都能学会。
% \item
%   \textbf{晃动玩具棒}:宝宝手抓握的能力也有所进步,如果将玩具棒塞到宝宝手中,宝宝能够晃动几下。
% \end{itemize}

% %ux8bedux8a00ux53d1ux80b2-1}{%
% \subsubsection{3. 语言发育}%ux8bedux8a00ux53d1ux80b2-1}}

% 在宝宝清醒的时候,边笑边逗引宝宝,大部分的宝宝都会发出声音来,如``啊、欧、哦''等。

% %ux8ba4ux77e5ux548cux751fux6d3bux4ea4ux5f80ux80fdux529b}{%
% \subsubsection{4.
% 认知和生活交往能力}%ux8ba4ux77e5ux548cux751fux6d3bux4ea4ux5f80ux80fdux529b}}

% \begin{itemize}
% \item
%   \textbf{眼睛跟随人影转动}:宝宝这时的视听能力已经有了很大的进步。视力不仅仅局限在
%   20厘米的范围,已经可以看清楚很远的地方,当大人在宝宝房间里来回走动,宝宝就会用眼睛跟随着大人的身影转动,听到屋里有声音就会转过头来寻找声音发出的地方,头部的转动能超过180度。
% \item
%   \textbf{注视物体片刻}:在宝宝头部上方放置气球等鲜艳玩具时,宝宝立刻就会注意力十分集中地观看片刻。
% \item
%   \textbf{喜欢大人陪伴}:喜欢大人在身边,看见妈妈等亲人时,会露出微笑。
% \end{itemize}

% %ux4e093ux4e2aux6708ux5a74ux513fux751fux957fux53d1ux80b2ux72b6ux51b5}{%
% \subsection{三、3个月婴儿生长发育状况}%ux4e093ux4e2aux6708ux5a74ux513fux751fux957fux53d1ux80b2ux72b6ux51b5}}

% 俯卧时,头胸部可以同时抬起,头、肩、胸部可以转动;部分婴儿能翻身。双手握在一起,能注视数秒钟;部分婴儿能笑出声音,发出音节。他们能够很清楚地看见身边的人和物体,眼神灵活,表情丰富。看见亲人会表现出很高兴的样子,能认识妈妈,且有主动与人交往的意愿。

% %ux8fd0ux52a8ux53d1ux80b2-2}{%
% \subsubsection{1. 运动发育}%ux8fd0ux52a8ux53d1ux80b2-2}}

% \begin{itemize}
% \item
%   \textbf{头胸部抬起}:运动能力从颈部发展到胸部。例如,在俯卧抬头时,头部抬起的同时可以带动胸部抬起45度\texttt{\textasciitilde{}}\hspace{0pt}90度。
% \item
%   \textbf{翻身动作}:仰卧位时不仅头部可以转动自如,肩部和胸部也可以转动。因此,这个时期的婴儿可以从仰卧位变为侧卧位。也就是说,如果婴儿发育得好,这个月是可以会翻身的。
% \item
%   \textbf{头竖立稳当}:抱起婴儿时能将头部稳稳地竖直10秒以上。
% \item
%   \textbf{双手握在一起}:双手从前两个月的握拳状态开始能够张开,并用手摸东西。如果给宝宝玩具棒,宝宝可以握住玩具棒达半分钟以上。宝宝能够将双手握在一起,注视并玩弄自己的手,持续3\texttt{\textasciitilde{}}\hspace{0pt}4秒钟。
% \end{itemize}

% %ux8bedux8a00ux53d1ux80b2-2}{%
% \subsubsection{2. 语言发育}%ux8bedux8a00ux53d1ux80b2-2}}

% 当大人逗引时,这个月的婴儿会随着大人的声音发出``欧、欧''的声音,有的婴儿还能笑出声音来。

% %ux8ba4ux77e5ux4e0eux751fux6d3bux4ea4ux5f80ux80fdux529b-1}{%
% \subsubsection{3.
% 认知与生活交往能力}%ux8ba4ux77e5ux4e0eux751fux6d3bux4ea4ux5f80ux80fdux529b-1}}

% \textbf{眼睛分辨物体清楚}:这个月宝宝的眼睛分辨能力有了很大的进步,能够很清楚看见身边的人和物体,而且还能很灵敏地跟随人或物体。当大人将一个玩具拿开,去拿另一个玩具时,宝宝很快就会将视线转移到另一个玩具上,这叫``视线转移''。最近新的研究表明,3个月龄的婴儿知道物体离开视线以后依然存在,同时还确信这个物体必然要在时空中继续存在下去。因此,当大人把宝宝喜欢的玩具拿开之后,他会表示出不满意,也许还会通过大哭来表达,当大人把玩具给他之后,宝宝立即就会停止哭闹。

% \textbf{认识亲人}:眼神灵活,能够东张西望。表情也逐渐丰富起来。见到亲人非常高兴,尤其见到妈妈时常常会笑出声音来。当看见亲人熟悉的面孔时会手舞足蹈地表示高兴。

% \textbf{有与人交往的意愿}:看见有人靠近身边,宝宝会表现出主动与人交往的意愿。

% %ux56db-4ux4e2aux6708ux5a74ux513fux751fux53d1ux80b2ux72b6ux51b5}{%
% \subsection{四、
% 4个月婴儿生发育状况}%ux56db-4ux4e2aux6708ux5a74ux513fux751fux53d1ux80b2ux72b6ux51b5}}

% 大部分的宝宝都会翻身,头和胸部抬高90度,用双手拿玩具玩耍,会自己晃动玩具棒。他们会发出七到八种不同的声音,可以分辨远近不同的物体,能将看见的物体和听到的声音联系在一起。大部分的婴儿能认识母亲。

% %ux8fd0ux52a8ux53d1ux80b2-3}{%
% \subsubsection{1. 运动发育}%ux8fd0ux52a8ux53d1ux80b2-3}}

% \begin{itemize}
% \item
%   \textbf{翻身运动}:4个月的婴儿会翻身了,这是婴儿时期运动发育的一大进步。婴儿在仰卧玩耍时,逐渐挪动着身体,眼睛不断东张西望。一旦发现身旁有自己喜欢的鲜艳玩具时,势必要想方设法拿到它。婴儿身体不断变换着姿势,不知不觉中就将身体趴到了床上。有时候婴儿在翻身时会把一只手臂压在身体下面,这时家长一定要帮助婴儿拿出手臂或者帮助婴儿选择最好和最舒服的姿势翻身,可以\textbf{练习左右翻身},使宝宝从翻身运动中体会乐趣。
% \item
%   \textbf{头胸部抬高90度}:宝宝从仰卧位翻到俯卧位后,可以很自如地将头和胸部抬高90度,用手掌支撑床面。有时婴儿还可以用手臂支撑床面,用双手拿玩具玩耍。
% \item
%   \textbf{坐位练习}:这个月龄的宝宝运动逐渐发展到胸和腰部。家长可以将宝宝从仰卧位拉到坐位。抱起宝宝时可以将宝宝的背部靠在母亲的前胸,用另一只手护住婴儿的腰部,让宝宝的臀部稳稳的坐在大人的双腿之上,婴儿的脸朝向前方。这种坐位可以使婴儿的视野从母亲的怀里转向周围,即扩大了婴儿视野,又增加了宝宝对周围物品的乐趣。将玩具棒塞入婴儿手中时,婴儿能够拿到眼前注视几秒,并努力摇晃。如果听到摇晃后发出的声音会更加起劲地晃动玩具棒,有的婴儿还会将玩具棒塞进嘴里。
% \end{itemize}

% %ux8bedux8a00ux53d1ux80b2-3}{%
% \subsubsection{2. 语言发育}%ux8bedux8a00ux53d1ux80b2-3}}

% 婴儿会做出咿呀的声音,能够发出七到八种不同的声音,虽然这些声音并无实际意义。

% %ux8ba4ux77e5ux548cux751fux6d3bux4ea4ux5f80ux80fdux529b-1}{%
% \subsubsection{3.
% 认知和生活交往能力}%ux8ba4ux77e5ux548cux751fux6d3bux4ea4ux5f80ux80fdux529b-1}}

% \begin{itemize}
% \item
%   \textbf{分辨远近不同的物体:}婴儿的视力非常清晰,能够调节视焦距,分辨远近不同的物体。他们能看到小纽扣、小糖块等。如果大人用手拿着鲜艳的小纽扣,婴儿会注视片刻再离开视线。
% \item
%   \textbf{视听觉之间的联系能力:}婴儿能将视觉与听觉很好的结合起来。例如,他们会发现母亲发出的声音是和她运动的嘴唇有关系的,这样婴儿就将视听觉联系到了一起。在宝宝的大脑中,整个世界就像一个由视觉、听觉和触觉组成的万花筒,这些事物之间有着千丝万缕的联系。宝宝接触到的东西越多,大脑中的世界就会越神奇,好奇感就会越强。因此,从这个月开始就应该采取各种方式尽可能地让宝宝看到或者听到各种丰富多彩的人间事物,如经常改变房间里的布局,床头放置一些彩画,经常不断变换。给宝宝看动画片,听广播,听音乐。加强户外活动,让宝宝看到外面的精彩世界。
% \item
%   \textbf{认识亲人:}从这个月起,宝宝开始认识自己的母亲等最亲近的人的脸。当母亲走到宝宝身旁时,宝宝会表现出非常兴奋的样子来迎接妈妈,当妈妈离开时,宝宝会用眼睛一直望着妈妈远去的身影。
% \end{itemize}

% %ux4e945ux4e2aux6708ux5a74ux513fux751fux957fux53d1ux80b2ux72b6ux51b5}{%
% \subsection{五、5个月婴儿生长发育状况}%ux4e945ux4e2aux6708ux5a74ux513fux751fux957fux53d1ux80b2ux72b6ux51b5}}

% 宝宝的运动发育扩展到身体的各个部位。他们能自如地翻身,扶着宝宝可以稍坐片刻,会进行双腿支撑跳跃的动作。宝宝会主动用手摸东西,能抓到面前的物体,并送到嘴中。他们会发出多种音节的声音,大部分的婴儿能够认识母亲。宝宝会有寻找眼前消失的物体的意愿。

% %ux8fd0ux52a8ux53d1ux80b2-4}{%
% \subsubsection{1. 运动发育}%ux8fd0ux52a8ux53d1ux80b2-4}}

% \begin{itemize}
% \item
%   \textbf{翻身自如}:运动发育已经由头部发展到整个身体的各个部位。能够非常容易地翻身,从仰卧位翻到俯卧位,再从俯卧位翻回仰卧位,或者翻到侧卧位等等。宝宝翻过身趴着的时候,能够很容易地抬起胸来,头部可以自由转动,但是手还是不能动。
% \item
%   \textbf{扶着独坐}:扶着宝宝可以稍坐片刻,如果让宝宝靠着沙发靠垫等可以坐得时间更加长一些。有一些婴儿能够独坐一会儿,但是身体会向前倾斜。
% \item
%   \textbf{双腿支撑跳跃}:将这个月的婴儿抱起来立在大人腿上,婴儿的双腿会主动出现支撑动作,这时大人用双手扶在婴儿的腋下,婴儿会出现跳跃动作。很多细心的家长会发现,婴儿在跳跃的运动中显得非常高兴和愉快。
% \item
%   \textbf{主动抓握}:宝宝的四肢已经更加灵活和有力量。能够用手摸一摸他们感觉新鲜和有趣的东西,并且可以用手准确抓握。一些婴儿甚至能将递到面前的饼干抓到自己的手里,并送到嘴里。
% \end{itemize}

% %ux8bedux8a00ux53d1ux80b2-4}{%
% \subsubsection{2. 语言发育}%ux8bedux8a00ux53d1ux80b2-4}}

% 婴儿的语言发育仍然以咿呀学语为主,能无意识地发出十几种不同的声音。

% %ux8ba4ux77e5ux548cux751fux6d3bux4ea4ux5f80ux80fdux529b-2}{%
% \subsubsection{3.
% 认知和生活交往能力}%ux8ba4ux77e5ux548cux751fux6d3bux4ea4ux5f80ux80fdux529b-2}}

% \begin{itemize}
% \item
%   \textbf{认识母亲:}这个月的宝宝已经\textbf{完全}能够\textbf{认识自己的母亲}了。宝宝看见人脸时可以发出微笑,但是如果别人抱他,他会不高兴,甚至还会哭出声来。然而,如果妈妈向他拍手,伸出双手做出要拥抱宝宝的样子时,宝宝也会不由自主地伸出双手迎接妈妈。每当这时,年轻的妈妈都会高兴半天,认为宝宝有了进步。
% \item
%   \textbf{对物体的认识:}物体在宝宝的脑海中已经形成了印象,所以当物体突然消失时,宝宝会到处寻找。一旦重新发现,宝宝会高兴得手舞足蹈。这时适当给予鼓励,会激发宝宝的好奇心和兴趣感。
% \end{itemize}

% %ux516d6ux4e2aux76eeux5a74ux513fux751fux957fux53d1ux80b2ux72b6ux51b5}{%
% \subsection{六、6个目婴儿生长发育状况}%ux516d6ux4e2aux76eeux5a74ux513fux751fux957fux53d1ux80b2ux72b6ux51b5}}

% 宝宝能够独坐玩耍,会连续翻身,会用手够取玩具。他们开始有拇指和其他指分开对捏的动作,会用双手积木传手。宝宝能听懂大人的简单语言,发出元音和辅音,能够很清楚地辨认父母、亲人和陌生人。他们会表达高兴、满意或者愤怒等情感意愿;会用双手撕纸,会和大人玩``藏猫猫''游戏。

% \begin{quote}
% v:元音和辅音?这是翻译过来的吧?
% \end{quote}

% %ux8fd0ux52a8ux53d1ux80b2-5}{%
% \subsubsection{1.运动发育}%ux8fd0ux52a8ux53d1ux80b2-5}}

% \begin{itemize}
% \item
%   \textbf{独坐玩耍}:6个月的婴儿已经学会了自己坐在床上,大人稍微用手拉住婴儿的双手,婴儿就会顺着大人拉行的方向很轻松地坐起来,而且还能玩一会儿摆放在面前的玩具。
% \item
%   \textbf{连续翻身}:这个月的婴儿不仅能够自由翻身,还可以在床上玩翻身打滚游戏。如果大人拿着婴儿喜爱的玩具逗引时,宝宝会移动上身和下肢想去够取玩具,身体会不由自主地移动或者转圈。
% \item
%   \textbf{拇指和其他指分开动作}:手的动作又有了新的进步。从全手大把抓逐渐进步到拇指和其他指对捏捏取。因此,可以给婴儿拿一些小方块积木等东西让婴儿练习捏取动作。
% \item
%   \textbf{双手传递积木}:双手可以配合完成积木传手的动作。这是一项非常重要的精细动作。让婴儿坐在床上,母亲递给孩子一块小积木,待宝宝拿稳之后,再递给他另一块积木让他用同一只手拿,聪明的宝宝就会将手中原有的积木传递到另一只手中,再来拿另一块积木。
% \end{itemize}

% %ux8bedux8a00ux53d1ux80b2-5}{%
% \subsubsection{2.语言发育}%ux8bedux8a00ux53d1ux80b2-5}}

% \begin{itemize}
% \item
%   \textbf{理解简单名词:}在这个年龄阶段,宝宝已经能够听懂许多话了。经常叫宝宝的名字时,他就会意识到这是在叫自己,会迅速地把头转向声音的来源,并用眼睛注视着叫他的人。对于日常生活中的一些用品,尤其是宝宝经常使用的东西,他会逐渐听懂这些物品的名称。比如,经常告诉他,房顶上发出光亮的物体是``灯'',经过多次重复,宝宝就会明白头顶上发出光亮的东西叫``灯'',并记住``灯''的特征。因此,当大人说出``灯''这个词时,宝宝会抬头向上看,表示他明白了``灯''的含义。
% \item
%   \textbf{发出元音和辅音:}在这个阶段,婴儿的发音不仅包括元音,还增加了许多辅音,例如``b''、``m''、p、d、k等。将这些音组合起来,宝宝就会发出``ba''、``ma''、da、ka等音节。当宝宝发出``ba''和``ma''音节时,父母往往会惊喜万分,以为孩子已经会说``爸爸''和``妈妈''了。即使家长明白孩子只是无意识地发音,他们也应该表现出高兴的情绪,因为这对宝宝来说是一种鼓励,\textbf{家长的这些愉快的情绪会促进宝宝的语言和智力发育}。
% \end{itemize}

% %ux8ba4ux77e5ux548cux751fux6d3bux4ea4ux5f80ux80fdux529b-3}{%
% \subsubsection{3.认知和生活交往能力}%ux8ba4ux77e5ux548cux751fux6d3bux4ea4ux5f80ux80fdux529b-3}}

% \textbf{辨认父母、亲人和陌生人}:从这个月起,孩子能够很清楚地辨认父母、亲人和陌生人。看见父母或熟悉的人时,宝宝会非常高兴,而看到陌生人时可能会产生些许的怯生感。当父母离开时,宝宝会用哭声表示不高兴。当父母用笑声引逗宝宝时,宝宝会发出爽朗的笑声来回应父母;当父母表情不高兴时,宝宝也会表情严肃,停止微笑等。因此,从这个月开始,父母在宝宝面前应该多以快乐为主,让宝宝多多享受愉快的气氛,并且尽可能让宝宝与更多不认识的陌生人交往,以练习宝宝的适应能力。

% \textbf{表达愿望}:6个月的宝宝在遇到自己不能做某件事情时,会用哭声来表示需要他人的帮助。例如,想要够取非常喜欢的玩具而没有能力够到的时候,看到别人吃东西,自己也想吃而吃不到的时候;对父母的离开表示不满的时候等等。孩子有时表现得比较烦躁或者大声哭闹,这时一定要采取耐心、温柔和循循善诱的方式,帮助宝宝渡过难关。切记不要大包大揽或者采取冷漠无情、视而不见的态度,前者会使宝宝形成一种依赖性,日后养成懒惰的习惯,形成遇事不能自立和缺乏信心的性格,而后者则又会使宝宝感到孤独和害怕,以后可能会形成冷漠和孤僻的性格。总之,父母的一言一行会影响宝宝的一生。

% \textbf{撕纸}:这个月的宝宝如果手里有一张纸,他会将纸撕成两半或者撕成几块,这个游戏实际上是训练宝宝双手协调的能力。

% \textbf{``藏猫猫''游戏}:``藏猫猫''是这个月宝宝应该会的游戏。具体过程是这样的:让母亲怀抱婴儿,父亲站在母亲的身后,将自己的脸藏在母亲的背后。父亲先呼叫婴儿的名字,引起婴儿的注意后,从一侧露出面孔,反复发出``喵、喵''的声音,逗引小儿笑出声音来。这时父亲又将脸藏到母亲背后,继续呼叫宝宝的名字。孩子会把头转向刚才父亲露出脸的方向进行寻找,而这次父亲则从另一侧露出脸来,令婴儿感到很新鲜和有趣。``藏猫猫''游戏可以练习宝宝的反应速度,增加愉快感,促进感情的交流。

% %ux4e037ux4e2aux6708ux5a74ux513fux751fux957fux53d1ux80b2ux72b6ux51b5}{%
% \subsection{七、7个月婴儿生长发育状况}%ux4e037ux4e2aux6708ux5a74ux513fux751fux957fux53d1ux80b2ux72b6ux51b5}}

% 在7\texttt{\textasciitilde{}}\hspace{0pt}12个月期间,婴儿身高、体重和头围的增长速度明显减慢,大部分孩子开始长出牙齿。他们能独坐玩耍很长时间,开始出现爬行动作,轮流用双手抓握玩具,并会用拇指和其他手指对捏在一起。

% 这个阶段的婴儿能够完全听懂大人所指的某些东西和表达的意愿。他们的发音数量明显增多,还可以用手势来与大人进行交流和沟通。他们能认识身边的物品和事物,明白数字的概念。此外,他们能够自己吃饼干,自己抱着奶瓶喝奶。

% %ux4f53ux683cux53d1ux80b2-2}{%
% \subsubsection{1.体格发育}%ux4f53ux683cux53d1ux80b2-2}}

% 在7\texttt{\textasciitilde{}}\hspace{0pt}12个月这个阶段,身高每月平均增长1.0-1.5厘米,体重每月可增长250\texttt{\textasciitilde{}}\hspace{0pt}400克。头围在7\texttt{\textasciitilde{}}\hspace{0pt}12个月期间可增长2\texttt{\textasciitilde{}}\hspace{0pt}4厘米,1岁时头围可以达到46厘米。从这个月开始,大部分的婴儿开始长出牙齿。

% %ux8fd0ux52a8ux53d1ux80b2-6}{%
% \subsubsection{2.运动发育}%ux8fd0ux52a8ux53d1ux80b2-6}}

% \begin{itemize}
% \item
%   \textbf{坐立稳定:}这个月的婴儿已经能稳稳地坐在床上,并且能独自玩耍很长时间。上身基本可以坐成直位,已经没有上个月的前倾位,双腿可以自由摆放得很平稳。头部可以随意地转动到各个位置,眼睛可以看到四周的任何一个角落。因此,孩子从卧位发展到坐位是一个很大的进步,不仅可以直接观察世界,同时也可以与人进行面对面的交流,真正体验到世界的丰富多彩。
% \item
%   \textbf{爬行动作:}爬行动作是从这个月开始出现的。爬行是人类运动发展中的重要步骤和过程。爬行时需要调动全身各个部位的骨骼和肌肉参与运动,需要肢体运动和相互协调。爬行时必须要保持平衡,才能稳定地向前挪动,因此反复练习爬行可以促进小脑平衡功能的发育,也可以让婴儿的大脑接受更多的运动信息,促进大脑的发育。在这个月的初期,婴儿学习爬行时,只能用上肢和腹部慢慢挪动着向前爬行,这种被医学称为``腹爬''。这种爬行很费力,婴儿需要使出全身力气来完成爬行动作。当宝宝的上下肢不能很好地协调时,家长可以用双手顶住婴儿的双下肢,帮助宝宝努力向前进。\\
%   \hspace{0pt}\includegraphics[width=2.93706in,height=2.62937in]{media/rId224.png}\hspace{0pt}
% \item
%   "轮流用双手抓握玩具和倒手:这是本月婴儿应该会的项目。当大人递给宝宝一个方积木时,宝宝可以用一只手拿起来。当大人再给他另一个方积木时,起初宝宝只能扔下手中的方积木去取另一块积木。但经过反复练习后,宝宝在不知不觉中渐渐能够用双手同时拿两块积木了。有的宝宝这个月还能将一块积木从一只手倒到另一只手中。这些都是婴儿手部精细动作发展的特征。
% \end{itemize}

% \begin{quote}
% 专家讲堂

% \textbf{宝宝进入``语音理解阶段''}

% 从这个月开始,婴儿进入了``语音理解阶段''。宝宝的发音数量明显增多,同时宝宝还可以用手势来与大人进行交流和沟通。例如,当爸爸要上班,出门前说"再见'',同时摇动手臂多次,宝宝明白了摇动手臂就是"再见''的意思。当下次爸爸出门说``再见"时,宝宝立刻就会摇动自己的手臂和爸爸"再见''。宝宝会用"哼哼呀呀''等大人听不明白的语言表达意愿,这些语言属于说话前的萌芽语言。

% \hspace{0pt}\includegraphics[width=2.96503in,height=2.7972in]{media/rId229.png}\hspace{0pt}

% 然而,有些家长对婴儿这个阶段的"哼哼呀呀''表现得很不耐烦,认为这是在打扰大人的正常休息,把孩子扔在一边不予理睬是很错误的。有的家长认为这个阶段宝宝不会说话,因而不主动与宝宝交谈,这可能会造成宝宝的语言发育障碍,影响宝宝今后的语言发育。在这个时期,家长应该\textbf{不厌其烦}地反复和宝宝唠叨,不管\textbf{拿到什么东西都要说出它的名字},\textbf{做动作时也要说出其相关的动作词汇},使宝宝总能\textbf{不断}地接受到语言的\textbf{刺激},这会对宝宝的语言发育有很大的促进作用。
% \end{quote}

% 捏取动作与上个月差不多,宝宝会用拇指和其他手指对捏在一起,捏取比较小的物品,如小方块积木、小塑料玩具等。

% %ux8bedux8a00ux53d1ux80b2-6}{%
% \subsubsection{3.语言发育}%ux8bedux8a00ux53d1ux80b2-6}}

% 尽管这个月的婴儿还不会说话,但他们已经能够完全听懂大人所指的某些东西和表达的意愿。例如,当你告诉她``这是娃娃'',几次之后宝宝就记住了娃娃的模样和发音。当下一次你再问她:``娃娃在哪里''时,宝宝就会用眼睛看着娃娃,表示告诉你这是娃娃。有的宝宝还能用手指向娃娃的方向,同时还不断发出某些音节来,好像在说``这是娃娃''。

% %ux8ba4ux77e5ux548cux751fux6d3bux4ea4ux5f80ux80fdux529b-4}{%
% \subsubsection{4.认知和生活交往能力}%ux8ba4ux77e5ux548cux751fux6d3bux4ea4ux5f80ux80fdux529b-4}}

% \begin{itemize}
% \item
%   \textbf{对语言和事物的认识和理解能力}:宝宝半岁以后开始具有对人类语言和世间事物的认识和理解能力。首先表现在认识自己身边的非常熟悉的日常生活用品,如奶瓶、毛巾、衣服、小被子、彩色玩具等。当感到饥饿时,看到奶瓶会表现得很高兴,说明宝宝知道奶瓶是用来喝奶的工具,看到鲜艳的玩具会非常高兴和兴奋,宝宝知道这种漂亮的玩具会发出令人愉快和好听的声音,摇晃起来也会非常有趣。这些都说明宝宝对自己身边的物体已经具有一定的认识和理解能力。
% \end{itemize}

% \begin{itemize}
% \item
%   宝宝认识和熟悉父母及周围经常在一起的护理人员,宝宝明白父母无论白天或者晚间都会陪伴在自己的身边;当自己饥饿或口渴时,总是父母第一个来到身边;当自己感到不舒服或遇到解决不了的困难时,总是父母第一个帮助自己渡过难关。因此,当这些人在宝宝身边或和宝宝说话及玩耍的时候,宝宝通常表现得很高兴、很随和,在他非常幼小的心中一定会觉得非常踏实和安全;而与头一次接触的生人则表现出陌生和胆怯感,因为宝宝觉得没有安全感,宝宝不知道这些人是干什么的,是否能够领走妈妈和爸爸,一旦妈妈和爸爸都走了,这些不认识的人会不会保护自己,一想到这里,宝宝就会产生恐惧感,因而就会大声哭泣起来。如果年轻的父母经历到此种情景时,一定会觉得宝宝长大了,同时也会感到欣慰,因为宝宝终于知道妈妈在宝宝的心目中占有多么重要的地位。
% \end{itemize}

% \begin{itemize}
% \item
%   \textbf{对数字的认识能力}:最近,国外专门研究婴幼儿心理学的实验室对7个月大的婴儿进行了有关数字方面的测试。结果发现7个月的婴儿大脑具有认识和理解数字的能力。研究指出,7个月的婴儿可以从照顾他们的亲人人数的多少来开始接触到数念。这个月的宝宝知道他们有一个妈妈和一个爸爸,当双亲其中一个离开时,宝宝知道双亲当中有一个人不见了,当剩下的人再走开时,他会意识到身边已经没有人了。宝宝还知道比一个多的数学概念。给7个月的婴儿做``加减法试验''非常有趣。试验过程是这样的:

%   \begin{enumerate}
%   \def\labelenumi{\alph{enumi}.}
%   \item
%     \textbf{加法试验}。
%   \end{enumerate}

%   \begin{itemize}
%   \item
%     我们进行一个加法试验。在挡板后面先后放上两个玩具娃娃,让宝宝仔细看清楚。拿开挡板时,婴儿看到两个玩具娃娃,表情很一般,认为这是理所应当的事情,因为他已经看见先后放了两个玩具娃娃
%     ------ ``1+1=2''。

%     第二次试验,研究人员先放进去一个娃娃,又放进去一个娃娃,宝宝大脑中已经形成了两个的概念。可是当拉开挡板时,婴儿只看到一个玩具娃娃,孩子感到非常惊讶,简直难以置信。他睁着大大的眼睛盯着这一个玩具娃娃,似乎知道这个答案是不正确的。
%   \end{itemize}

%   \begin{enumerate}
%   \def\labelenumi{\alph{enumi}.}
%   \setcounter{enumi}{1}
%   \item
%     \textbf{减法试验}。
%   \end{enumerate}

%   \begin{itemize}
%   \item
%     减法试验的过程是这样的。首先,让婴儿清楚地看到挡板后面放置了两个玩具娃娃。然后,拿走了一个玩具娃娃。此时,婴儿计算出了他自己的答案``2-1
%     =
%     1''。但当挡板拿开时,他看见仍然有两个玩具娃娃。婴儿这时感到很困惑,使劲把眼睛睁得大大的,似乎是在向大人提示这绝对不是他们所预料的情况。

%     在第二次试验中,将两个玩具娃娃放好后,拉上挡板并从挡板后面拿走一个。这时婴儿认为只会留下一个。结果拉开挡板一看,果然只剩下一个。婴儿觉得这种结果是正常的,他觉得很无聊,打起了``哈欠'',似乎在说这游戏太不吸引人了。
%   \end{itemize}
% \end{itemize}

% 婴儿的注意力只有在加减法不正确的时候,才会被吸引住。这表明即使婴儿只有7个月大,他们就可能具有初步的数学判断能力,这是多么令人惊奇的能力。从这项小小的``加减法试验''中,我们就可以看出婴儿大脑具有多么大的潜力。

% 在生活能力方面,这个月的婴儿能够自己吃饼干,自己抱着奶瓶喝奶等。

% %ux516b8ux4e2aux6708ux5a74ux513fux751fux957fux53d1ux80b2ux72b6ux51b5}{%
% \subsection{八、8个月婴儿生长发育状况}%ux516b8ux4e2aux6708ux5a74ux513fux751fux957fux53d1ux80b2ux72b6ux51b5}}

% 8个月龄的婴儿坐和爬行非常自如,完全能够自由支配躯干和肢体的运动,有站立的意识。他们可以捏起花生、米粒、小药丸、曲别针等比较小的物品,并可以用手拨弄键盘,还可以扒拉大算盘珠子。

% 此外,他们会发出各种语调的声音,并且有想与大人说话的意识。他们能够通过手势很好地与大人进行沟通,会用手势表达``欢迎''、``再见''、``谢谢''等。此外,他们还能记住五官的名称。

% %ux8fd0ux52a8ux53d1ux80b2-7}{%
% \subsubsection{1.运动发育}%ux8fd0ux52a8ux53d1ux80b2-7}}

% 婴儿进入生长发育的第8个月时,身体和肢体的活动更加灵活,坐位和爬行动作基本达到自如的程度。

% \textbf{爬行运动}:8个月的婴儿爬行完全能够自由支配躯干和肢体运动的程度。可以做前后、左右爬行运动。7个月时主要以匍匐爬行为主,肚子贴在床上向前挪动,最初孩子只能通过手和上臂前端或者胳臂的反复挪动慢慢向前爬行。到了
% 8个月时,四肢的运动
% 越来越协调,可以用手和腿交替向前挪动了。婴儿逐渐感觉到仅仅用手和腿交替挪动已
% 经不能满足自己的某种愿望时,就会努力用双手和双侧膝盖来支撑身体,使身体离开床面或者地面,这样可以加快爬行速度。上述这种爬行方式是大部分婴儿的爬行发育过程,学会这种爬行会使宝宝进一步加快运动发育。

% \begin{quote}
% 育儿小百科

% \textbf{家长不要限制婴儿的自由}

% 许多孩子学会爬行之后,父母最担心的就是害怕孩子从床上摔到地上,或者家中的物品会被损坏。而孩子最喜欢的事情就是爬到床边,将拿到的玩具和其他物品扔到地上。其实,大部分孩子扔东西纯粹就是为了听到不同的声音而已。父母对孩子的这种行为非常不满,同时又害怕孩子摔到地上,家长经常限制孩子的爬行,令孩子感到很不自由,这样不利于婴儿的运动发育。我们应该给他创造条件,不要限制婴儿的自由,鼓励婴儿爬行,充分发挥孩子的运动才能。
% \end{quote}

% 还有一种爬行方式是``倒爬''方式。婴儿在学会用四肢支撑身体开始爬行时,先要向后倒爬几步之后,再向前爬行。许多家长为自己孩子出现的这种怪现象而感到很奇怪,反复多次询问保健医生,担心这种爬行会有碍于孩子的运动发育。其实,孩子的这种发育是完全正常的。根据作者的经验,在门诊咨询这类爬行的孩子以女孩为多见。

% 最少见的一种爬行是``桥式爬行''。这通常发生在孩子的头部上方有某件他特别喜欢的东西时。孩子呈仰卧位,首先抬起臀部,连带着腰也抬了起来,这样腰部和臀部就出现了悬空状态。在腰部和臀部悬空的同时,孩子会用一只腿弯曲着蹬着床面,而另一条腿伸直。也有一部分婴儿会用两条腿同时弯曲向上挪动。虽然爬行非常辛苦,但当孩子爬到自己想要去的地方时,或者经过努力终于拿到自己想要的玩具时,他们是多么高兴。他们不会感到辛苦和疲劳,而是会非常投入地玩耍起来。

% \textbf{扶站运动}:这个月的婴儿刚刚有站立的意识。如果大人用双手扶在婴儿的腋下,婴儿会非常自然地伸直双腿站立起来。一般情况下,当大人拉住婴儿的双手时,婴儿会自动顺势坐起来,大人继续用力抬起孩子的双手,孩子就会抬起臀部站立起来。在这项运动中,有些孩子干脆没有坐的过渡,直接就站立起来。经常听到一些父母这样说:``我的孩子根本就不想坐,只想站着。''其实这只是婴儿学会站立前的一种欲望的表现。孩子急于站起来是因为他们觉得站立起来看得更高、更远、更多和更丰富。家长不要惧怕孩子站立过早。早期站立的练习可以这样做:让婴儿双手扶住栏杆呈站立姿势,以胸部不贴栏杆为最佳状态,至少保持5秒钟以上,如果婴儿能够达到上述要求,则说明孩子已经做好了站立的准备,开始向独站的运动发展。

% \textbf{手指精细动作}:婴儿在这个月仍然以拇指和食指捏起为主要抓握动作。可以捏起比较小的物品,如花生、米粒、小药丸、曲别针等,可以用手拨弄键盘,使其发出声音,还可以扒拉大算盘的珠子。有的婴儿对一些小洞和小孔更感兴趣,看见电脑桌上的小洞就一定要用手指抠一抠。婴儿玩方积木时,可以同时用双手各拿1块方积木,但还是有想拿第三块积木的欲望,虽然不一定能拿到,但可以出现用手要继续够的动作。

% %ux8bedux8a00ux53d1ux80b2-7}{%
% \subsubsection{2.语言发育}%ux8bedux8a00ux53d1ux80b2-7}}

% \hspace{0pt}\includegraphics[width=2.48951in,height=2.16783in]{media/rId241.png}\hspace{0pt}

% 明显看出宝宝有想与大人聊聊的欲望,在大人说话时,宝宝会不时地发出各种语调的声音,同时还会用手使劲地敲打桌面或者床面,以引起他人的注意。更有意思的是,当听到大人咳嗽时,宝宝也会跟着发出``咳咳''的声音。这种现象说明宝宝已经具备了非常完整的模仿和发声的能力,要经常与宝宝对话,如在给宝宝穿衣时,发出``衣服,穿衣''等声音,让宝宝知道这些动作是穿衣服的意思。大人在室内活动时,要多对宝宝慢慢地反复说一些名词和动词,发音准确和清晰。只要抓紧时机反复练习,宝宝在不久就会发出准确的音节了。

% %ux8ba4ux77e5ux548cux751fux6d3bux4ea4ux5f80ux80fdux529b-5}{%
% \subsubsection{3.认知和生活交往能力}%ux8ba4ux77e5ux548cux751fux6d3bux4ea4ux5f80ux80fdux529b-5}}

% \hspace{0pt}\includegraphics[width=3.1049in,height=2.30769in]{media/rId245.png}\hspace{0pt}

% 宝宝这个月已经能够通过手势很好地与大人进行沟通了。反复训练之后,宝宝会用手势很好地表达``欢迎''、``再见''、``谢谢''等词汇。经常给宝宝指大人的眼睛、鼻子、耳朵,边指边说出五官的名称,不知不觉中宝宝就记住了五官的名称,当大人再次说出眼睛、鼻子和耳朵时,宝宝就会准确地指出各自的位置。

% %ux4e5d9ux4e2aux6708ux5a74ux513fux751fux957fux53d1ux80b2ux72b6ux51b5}{%
% \subsection{九、9个月婴儿生长发育状况}%ux4e5d9ux4e2aux6708ux5a74ux513fux751fux957fux53d1ux80b2ux72b6ux51b5}}

% 四肢和躯干更加灵活,能够变换各种体位。拇指和食指可以分开抓握。对语言有所了解,对一些身边比较熟悉的东西,基本都能有所认识。记忆时间延长,会用手握住勺子,将盛放在勺子里的饭菜放入口中。并且,他们会表达``要''与``不要''的意愿。

% %ux8fd0ux52a8ux53d1ux80b2-8}{%
% \subsubsection{1.运动发育}%ux8fd0ux52a8ux53d1ux80b2-8}}

% 这个月龄的婴儿的大运动形式不断转换,更加灵巧;精细动作发展到拇指和食指分开抓握。婴儿从大运动发展到精细动作,运动功能出现了飞跃发展。

% \textbf{体位的转换}:孩子可以到处爬行,同时可以从爬位转到坐位,再从坐位变到爬行位,可以很自如地取到在自己身边前后左右的玩具等东西。扶着宝宝可以从坐位拉到站位,从站位扶至蹲位、坐位。

% \textbf{整个过程可以这样练习}:用双手扶着宝宝,他会站立几秒钟,但是这种站立还相当不稳;站立一会儿后可以扶着宝宝慢慢蹲下,然后再坐下,待坐稳之后让宝宝玩耍一会儿。也可以在宝宝的前后左右不远处放置一些玩具,让宝宝从坐位变到爬行位来够取玩具;等宝宝回到坐位之后,再扶着宝宝,双手拉着宝宝站起来。上述动作可以每天反复练习数次,练习时一定要耐心,动作要柔和,如果动作过快,孩子会感到接受不了,下一次练习时宝宝就会反感而不配合,或者大声哭闹以表示拒绝,这些都会影响大人和孩子的情绪,妨碍正常的练习。

% \textbf{拇食指抓握}:婴儿从出生起就有抓握的意识,但这种抓握是被动的,只有将东西塞到手中时婴儿才能握紧。这种被动抓握一般要持续3\texttt{〜}\hspace{0pt}4个月才消失。

% 4 \texttt{\textasciitilde{}}\hspace{0pt}
% 5个月时婴儿开始有目的抓握东西了,如吃奶时旁边的大人可以帮助孩子挑选他最喜爱的玩具,调整孩子的情绪,或者将玩具塞到孩子的手中,让孩子知道手中有这么多的玩具该有多快活,增强孩子够取更多玩具的欲望和信心,促进孩子的手运动发展。

% 7 \texttt{\textasciitilde{}}\hspace{0pt}
% 8个月的婴儿的拇指和其他手指能够分开取物了,并且能够抓到体积比较小的东西,如花生米、玉米粒、小药丸等,有时候婴儿还能抓起饭桌上残留的大米粒、面包渣等,并很快放入嘴中。这时候婴儿最多的动作就是抓取小物品放入嘴中。

% 9\textasciitilde10个月的孩子可以用拇指和食指分开抓取细小的东西了。拇、食指抓握是人类手部精细运动的重要发展,拇、食指分开意味着婴儿开始会使用工具,而且这种工具并不同于一般的简单工具,而是比较复杂的精细工具。许多患有脑神经系统疾病和脑瘫的病儿均不会拇、食指抓握动作,直到五六岁时还是用手掌大把抓握东西。因此,本月的重点就是练习用拇、食指抓握东西,一定要不厌其烦地反复练习,练习次数越多,孩子的手指就会越灵活,将来说不定还能成为一个小雕塑家呢。

% 综上所述,我们可以看出孩子的抓握过程实际上就是人类运动从粗大笨拙到精细灵巧的过程,这种精细灵巧的动作正是人类与动物的区别所在,是人类不断发展,不断进化的演变过程。

% %ux8bedux8a00ux53d1ux80b2-8}{%
% \subsubsection{2.语言发育}%ux8bedux8a00ux53d1ux80b2-8}}

% 9个月婴儿进入语言萌芽阶段。这个时期孩子对语言基本有了一些了解。知道大人发出的音节代表一定的意思,对一些名词类词汇好像更加敏感一些,尤其对身边一些比较熟悉的东西,基本都能有所明白,但对动词和形容词还不清楚,对于连贯的语言还不明白。因此,大人在说话时要\textbf{以简单的名词发音为主},语调平稳,吐字清晰,让孩子清清楚楚、明明白白地听到准确的发音。这时候孩子也在努力地倾听大人的发音,同时嘴里也在嘟囔着什么,实际上他这是在练习发音。孩子说话的同时也在倾听自己的声音,也许在什么时候孩子就会突然说出他想要说出的语言。每当这时孩子都会欣喜无比,手舞足蹈。如果大人\textbf{也跟着欢呼雀跃},孩子会受到鼓舞,一定会更加努力地练习发音。

% %ux8ba4ux77e5ux548cux751fux6d3bux4ea4ux5f80ux80fdux529b-6}{%
% \subsubsection{3.认知和生活交往能力}%ux8ba4ux77e5ux548cux751fux6d3bux4ea4ux5f80ux80fdux529b-6}}

% \textbf{记忆能力}:婴儿自出生起就具有记忆能力。刚出生的新生儿就会对妈妈的脸部轮廓有记忆。当妈妈戴上口罩或眼镜时,婴儿就会觉得与平时记忆中的妈妈好像不一样,露出惊讶的眼神,反复注视妈妈面孔许久才离开。

% 1个月的婴儿能记住每天看到的奶瓶,每当妈妈举起奶瓶时,婴儿都会露出非常高兴的神态。

% 3个月的婴儿会记住自己的母亲,每当母亲来到宝宝身边时,宝宝都会非常兴奋和高兴,有时还会手舞足蹈地迎接妈妈。

% 6个月的婴儿会记住带有光亮、色彩鲜艳的东西,如房顶上悬挂的彩灯,墙壁上贴的大头娃娃像等,有些聪明的宝宝还会记住``灯''的发音,当大人发出``灯''的音节时,婴儿就会把眼睛转向有光的灯的方向。

% 8个月的婴儿已经能够记住两个以上音节的词汇,如``谢谢、再见、欢迎''等,而且还能记住用什么手势来模仿这些词汇的意思。这说明从这个月龄起,婴儿已经能将记忆和模仿结合在一起,与大人进行交流。但是,这个阶段婴儿的记忆非常短暂,也许现在记住的内容一会儿就会忘掉,今天看见的事物明天就会全部忘光。有时家长会发现孩子上午还能表达的手势下午就忘记了,这时家长千万不要着急,本阶段的婴儿就是这样,别看孩子暂时忘记了,但是再教起来会很容易就记住,反复训练多次,随着宝宝年龄不断增大,他的记忆也会越来越强,记住的内容也会越来越多。

% 进入9个月的婴儿的记忆有了明显的长进,主要是记忆时间延长了。例如,告诉宝宝眼睛的发音和位置,宝宝会很长时间不忘记,甚至可以记忆1\textasciitilde2个月以上。对于一些细微的手势动作,宝宝也会记忆许久。

% \textbf{思维判断能力}:对于6\textasciitilde8个月的婴儿来说,在他面前把玩具藏起来,他是不会找到的。在婴儿面前将玩具猫藏在毛毯下面,然后让婴儿自己寻找刚才看见的玩具猫,这时婴儿只会看着周围的东西,而不知应该到哪去寻找。实际上寻找东西的过程是一个非常复杂的过程,需要很多的大脑思考过程和手眼协调过程。首先婴儿要对消失后的玩具形象完整地保留在脑海中,还要记住刚才大人做过的动作,记住玩具藏匿的地方,当大人让他找些这件玩具时,婴儿必须要通过大脑来制订计划,如何用自己的手拿开毛毯等,这需要大量的思考和思维活动。有人给8个月婴儿做了这样一项试验,在婴儿面前把一个玩具藏在一块屏风后面,然后让这名8个月的婴儿去寻找,玩具近在咫尺,但婴儿却不知道寻找,只是瞪着大眼睛看着大人。

% 进入9个月,婴儿的意识和思维就有了突飞猛进的发展。9个月的婴儿不仅能够在同样的游戏中找到玩具,而且还能准确地拿回来。

% 通过大脑的思维判断,本月龄的婴儿开始具有解决问题的能力了。有人做了这样的试验,让妈妈抱着婴儿,把一只小熊放到婴儿面前的小桌子上,让他玩耍一会儿,在桌子上面铺一块桌布,把小熊放到离婴儿稍远的桌子上,距离要以婴儿够不到为准,观察婴儿如何够取这只小熊。可见到婴儿在试着够取小熊,没有够到,婴儿很着急,用手去拉桌布,无意中将桌布拉向了自己的一方,同时挪动了小熊,缩短了小熊与婴儿之间的距离,婴儿很容易就拿到了小熊,婴儿很高兴。当大人再重复这项游戏时,婴儿就会很快地拉动桌布,拿到玩具。通过这个事例我们可以看出,婴儿实际是在运用大脑思维来判断,寻找解决问题的方法。因此有专家建议,给9
% \textasciitilde{}
% 10个月月龄的婴儿\textbf{设置一些比较容易的障碍型游戏},如在婴儿面前把玩具放在衣服下面、被子下面,或者大人将拿东西的手藏在背后,让孩子自己寻找,婴儿通过类似的游戏可以增强思维活动,培养判断事物和解决问题的能力。

% \textbf{使用工具}:婴儿最初接触到的工具是吃饭用的勺子。9个月的婴儿可以用手握住勺子,有时候也会将盛放在勺子里的饭菜放入口中。但孩子拿勺子时方向不分,左右不分,往往是胡乱拿,结果往往是大部分的食物撒满了桌面,沾满了嘴唇和面部,仅有一小部分食物塞进口中。尽管是这样,婴儿还是乐此不疲,只要将勺子放在他面前,他都会用手抓起勺子,放入口中。孩子通过这项活动体会到使用勺子就会吃到自己想吃的食物。

% 实际上,用勺子把食物放到嘴里是一个相当复杂的过程,需要大脑的思维判断,手眼运动的协调和足够的耐心,要真正学会还需要很长一段时间。一些性急的婴儿感觉用勺子吃饭比较费力,干脆放弃勺子,直接用手来抓着吃。喜欢喊叫的婴儿则使用大嗓门叫唤以寻求帮助。其实,让孩子练习用勺子吃饭是训练婴儿早期使用工具的最好方法。首先家长\textbf{必须有足够的耐心,不怕辛苦,不怕麻烦},一些爱清洁的父母看到婴儿用小勺吃得满身和满嘴都是食物,感觉很脏,收拾起来也很费力,干脆不让孩子用勺子,直接由大人来喂给婴儿,久而久之孩子养成了饭来张口的习惯。

% \textbf{正确的方法可以采用如下方式}:开始由大人将勺子中盛上食物,摆放在婴儿的面前,让婴儿自己拿起勺子吃,也可以手把手教孩子用小勺取一些饭菜放入口中,只要\textbf{耐心和有规律的练习},孩子到1岁以后很快就会自己用勺子吃饭了,不要小看用勺子吃饭的动作,这是培养孩子自己动手能力的最好方法。

% \textbf{``要''与``不要''的意愿表达}:9个月的孩子已经能够很清楚地表达``要''与``不要''的意愿了。当陌生人走到面前要抱起他(她)时,婴儿会马上把脸转向妈妈,表示不要别人抱,到医院看见穿白大衣的医生将听诊器放在自己的小胸脯上,马上就会摇头或者用手推开;看见大人把一片药放到小勺上,准备喂入口中时,婴儿就会大哭表示不要或者抗议大人的举动,而当妈妈把一件孩子从未见过的新鲜玩具给他时,婴儿马上就会破涕为笑,用双手接过玩具玩耍起来。

% %ux534110ux4e2aux6708ux5a74ux513fux751fux957fux53d1ux80b2ux72b6ux51b5}{%
% \subsection{十、10个月婴儿生长发育状况}%ux534110ux4e2aux6708ux5a74ux513fux751fux957fux53d1ux80b2ux72b6ux51b5}}

% %ux8fd0ux52a8ux53d1ux80b2-9}{%
% \subsubsection{1.运动发育}%ux8fd0ux52a8ux53d1ux80b2-9}}

% 在9个月婴儿发育中可以明显看出婴儿运动形式在不断发生变换,通过仰卧、俯卧、爬行、坐位、扶站等各种运动形式,感受运动的乐趣。进入10个月时,婴儿运动又有了新的进步,学会了扶栏站立,并出现了迈步的意识。精神运动发育方面也趋于更加熟练和成熟。

% \textbf{扶栏站立}:9个月时婴儿需要在大人帮助下站立数秒或片刻,进入10个月之后,如果把婴儿放到有栏杆的床旁边,婴儿可以从坐位开始用双手扶住栏杆下部,以此为支撑点试着站起来。刚开始婴儿感到很费力,经常是双手扶栏杆,屁股脱离床面后失去平衡,结果又一屁股坐回到床上,有的婴儿可能反复出现数次,家长看着很心疼。其实,这些动作对于婴儿来说可能反而会觉得很有趣,为了能够扶栏杆站立起来,孩子会使尽全身力气反复去做上述动作。当有一天他突然扶着栏杆站立起来之后,婴儿会觉得扶着栏杆站起来并不是很难的事情,而且,自己扶栏杆站起来是多么愉快的事情。

% \hspace{0pt}\includegraphics[width=3.09091in,height=2.36364in]{media/rId254.png}\hspace{0pt}

% 家长看见婴儿试图要扶栏杆站起来时,开始可以适当地帮一把,以后逐渐减少帮助,尽量让孩子自己站立起来。这个月的婴儿大多是这样运动的,从床的一边爬到另一边,坐下,再扶着栏杆下部,支撑身体蹲起来,再用一只手扶住栏杆上部,用另一只手支撑床面或下部将身体直立起来。还有一种方式就是直接由爬行位转到扶栏杆站立,用一只手扶住栏杆,另一只手支撑床面,撅起屁股,挪动下肢,逐渐站立起来。婴儿刚刚站立起来时双手会紧紧地抓住栏杆,如果大人用手试图扒开婴儿的手,婴儿会感到非常恐惧,说明孩子知道他是借助于栏杆才站立起来的,没有栏杆他就会摔倒,这也是婴儿学会使用工具的典型事例之一。

% \textbf{扶栏迈步}:婴儿扶栏杆站立起来之后,首先他会感到面前的视野又开阔了许多,能看见的东西更多、更远、更清楚,与大人交流起来更加方便,这使孩子更增强了要站立起来的愿望和信念。当他看到比较远的地方有更加可爱的玩具时,就会试图迈步,走向要去的地方,这就是迈步的开始。家长可以站在距离婴儿扶栏杆稍远的地方,手拿一些婴儿喜欢的玩具,鼓励孩子向前迈步拿取。

% 孩子通常会抬起一条腿,试着去迈步,家长可以协助孩子,用手抓住孩子双手或者躯干,让孩子感到很安全,他会很放心的抬起一条腿,做出迈步的姿势。迈步成功后,家长应该立即拍手叫好鼓励继续向前走。

% 有一部分孩子10个月时刚刚会扶栏,但还不能迈步,这根本没有什么关系,本月龄的孩子有相当一部分还不能达到扶栏迈步的程度,但家长一定要有训练意识,从坐位到扶栏站立,再开始迈步等一系列动作,需要一步一步,一个动作一个动作反复练习。一段时间练习一个动作,待动作熟练后再练习下一个动作,经过反复练习孩子就会逐渐掌握要领,尝到甜头,达到训练的目标了。

% 实际上,从坐位到扶栏站立及迈步过程是一个身体多个部位,多种动作的综合运动,需要大脑、小脑的共同协调才能完成,如果大脑和小脑功能有障碍,孩子扶栏就会找不到平衡,当然也就不会顺利站立和迈步。因此,这项训练也是培养孩子大脑和小脑功能的重要训练方式,而这个月正是抓紧训练的最佳时机。

% \textbf{拇、食指抓握训练}:9个月以后的婴儿开始会用拇、食指对捏,但还不够熟练。进入10个月之后,大部分的婴儿都会很熟练、很迅速地用拇、食指捏起小东西如花生粒等,一部分婴儿抓起之后总是愿意将物品放入嘴里,因此在训练捏取食物的过程中,一定要有专门看护人员,防止东西吃到嘴中,误入咽部,后果将不堪设想。【尽量用安全的,如可化食物?】

% %ux8bedux8a00ux53d1ux80b2-9}{%
% \subsubsection{2.语言发育}%ux8bedux8a00ux53d1ux80b2-9}}

% 婴儿基本能明白大部分名词性语言的意思,知道自己最亲近的人是爸爸、妈妈,一部分婴儿能够准确发出``爸爸、妈妈''的音节,但大部分的婴儿仍然以手势为主。当大人发出``马、熊猫、大象''的声音时,婴儿就会指认相应的图像;把各种水果摆在婴儿的面前,说出``苹果、香蕉''等词汇时,婴儿就会用手拿起苹果和香蕉等水果来,示意这是大人说的水果。本月语言发育与9个月一样属于说话萌芽阶段,大量练习视觉和听觉,造就丰富的语言环境是非常重要的。要注意以下几点。

% \textbf{声音与眼神结合}:说话或发出声音时要用眼睛看着孩子的眼睛,感觉到孩子在看、在听时,努力张大口型,慢慢发出声音,让孩子看得明白和清楚,让他初步理解这种口型下的发音是什么意思。

% \textbf{说话与动作结合}:说话时可以结合家中的实物,让婴儿用手一边触摸,一边告诉他这是什么,让这些名词牢牢记在孩子的脑海中。吐字要清楚,速度要缓慢。

% \textbf{创造好的语言环境}:努力创造丰富的语言环境,如经常播放一些好听的儿歌、儿童故事等,吃饭、穿衣、洗澡及玩耍时,嘴里要反复唠叨各种词汇。

% \textbf{拓宽视野}:经常带孩子到户外接触大自然与其他人或事物,开阔孩子视野,丰富感官印象。保持快乐心情。

% %ux8ba4ux77e5ux548cux751fux6d3bux4ea4ux5f80ux80fdux529b-7}{%
% \subsubsection{3.认知和生活交往能力}%ux8ba4ux77e5ux548cux751fux6d3bux4ea4ux5f80ux80fdux529b-7}}

% 本月的婴儿开始产生好奇心,对许多事物开始细心观察。应该开始有步骤地训练孩子的生活\textbf{自理能力},如坐盆大小便,配合大人穿衣裤和小鞋等,会与亲人交流,表达喜怒哀乐的情感。下面分别加以叙述。

% \textbf{好奇心与新鲜感}:10个月的婴儿内心世界已经很丰富,求知欲望越来越强烈。由于肢体活动项目越来越多,范围越来越广泛,可以熟练地完成坐、爬、扶栏站立等动作,一部分婴儿还能扶栏迈步。在婴儿眼前的世界越来越精彩,他们想了解一切,在他们看来,周围有看不完的五彩缤纷、色彩斑斓的新鲜东西,它们是多么奇妙和不可思议,每当孩子拿起一件他认为是很有意思的东西时,都会拿起来看一看,然后放入嘴中尝一尝。当他看见大人把某件玩具拆开之后能安装上时,孩子就会让大人反复安装和拆卸。当孩子听见玩具摔到地上的声音很好听时,就会反复将玩具扔到地上,聆听摔到地上的声音。只要他认为是新鲜、有趣、奇妙和没有见过的东西,哪怕是一支笔,一根绳,一只牙膏他都会独自玩耍许久,直到厌烦为止。有的家长每天看护婴儿感觉很疲劳,看到孩子的这些``无聊''动作和``无理''要求不予理睬,有的干脆将玩具拿开,或者大声训斥孩子,这些都是不可取的行为,这些行为会大大减低和遏制孩子的好奇心与新鲜感,使孩子对周围事物渐渐失去兴趣。孩子正是通过这些拆拆卸卸,捡起和扔掉东西来体会周围事物,感知人类活动。那么家长应该怎么做才能满足孩子的好奇心呢?可以给孩子准备一个没有盖子的大塑料箱,里边放一些日常生活中常用的东西,而且这些东西最好是没有危险性,触摸起来比较圆滑,感觉很舒适的小东西,如形态各异的小积木、小吊环、头绳、彩色卡片、各式小花布、小装饰品、钥匙链,以及不怕摔、不怕打的塑料或木制工艺品等均可,让孩子随意扔取,充分满足孩子的好奇心和求知欲望。

% \hspace{0pt}\includegraphics[width=2.04196in,height=2.6014in]{media/rId259.png}\hspace{0pt}

% \textbf{大小便训练}:当婴儿在坐位时能够独立自如拿取东西,上身随意摆动不摔倒的情况下,就可以开始训练坐盆大小便了。刚开始坐盆时,孩子有时候会摔倒,需要大人扶着,慢慢等孩子习惯坐便盆之后,就可以不用扶了。选择便盆时要挑选颜色鲜艳,有一些可爱造型的婴儿用便具。现在市场上卖一些带有小动物形象或者卡通形象的便具很受孩子喜爱。便盆一般都放置在屋中比较显眼的位置,当婴儿想要大小便时能看到它。反复告诉他这个活泼可爱的小动物便盆就是用来``尿尿''和``拉臭臭''的。经过反复多次的练习,待孩子明白坐盆的过程和坐盆的意义之后,就可以定时让孩子坐盆。坐盆时间不宜过长,因为坐盆时间过长可以引起脱肛和痔疮。如果坐盆一段时间还没有解出大便,可以等一段时间再重新坐盆。这样反复训练多次后,孩子就会形成一种条件反射,只要有尿意或者便意,就会主动要求坐盆。坐盆时不要逗引孩子或和孩子玩耍分散孩子精力。应该让孩子集中精力完成解大小便过程。孩子在吃饭中或者睡觉中尽量避免坐盆,以免影响孩子食欲和睡眠。

% \textbf{坐盆洗澡}:洗澡是对婴儿非常有益的一种水中运动。胎儿在母亲子宫内的羊水中生活了整整10个月,充分享受了水中生活的乐趣。刚出生不久的新生儿对洗澡比较恐惧,并不是因为水的缘故,而是由于大人在给婴儿洗澡时往往\textbf{动作过重},或者在洗头时将\textbf{水溅到婴儿脸上},引起婴儿\textbf{恐惧}所造成的。当婴儿稍微大一些,特别是当婴儿能坐位玩耍时,就特别愿意坐到水中嬉戏玩耍。纯净透明不断变化的水,让婴儿感受到无穷的乐趣,温暖适度的水与皮肤接触,让婴儿产生愉快感,水的不断流动和变化让婴儿产生好奇和新鲜。有时候婴儿用小手拍打着水,当水花跳出水面,落到脸上、身上,或者旁边大人身上时,婴儿会非常高兴。不要以为这是孩子的恶作剧,孩子正是通过这种动作和玩耍来了解水的特性,了解大自然的规律。但是一定要记住千万不要玩耍时间过长,洗澡完毕之后要马上用干软毛巾擦干身上的水,并尽快穿好衣服,穿衣服时不要过分嬉笑逗乐,影响穿衣速度。以免因为洗澡时间过长,或者洗澡后不愿意穿衣而感冒。

% \begin{quote}
% gpt

% 注意,洗澡的水温不宜过高,一般在\textbf{37-38摄氏度}为适宜。水的深度以能够覆盖婴儿的肚脐为准。洗澡的时间以10-15分钟为宜,不要过长,避免婴儿受凉。

% 洗澡结束后,要迅速把婴儿从水中抱出来,用干净的毛巾把婴儿的身体擦干,尤其是脚趾间的水分要擦干净,防止引起脚趾间皮肤湿疹。最后,给婴儿穿上舒适干净的衣服,让他取得良好的休息。
% \end{quote}

% \hspace{0pt}\includegraphics[width=2.6014in,height=2.41958in]{media/rId262.png}\hspace{0pt}

% \textbf{穿衣练习}:这个月的婴儿已经懂得温暖和寒冷的意思,明白穿衣的用途。因此,早上当妈妈和爸爸掀开被窝时,婴儿都会感觉到寒冷,要求穿衣服。大人可以利用这个时候训练孩子配合大人穿衣服。有些家长将小衣服袖口卷成小圆圈,让婴儿伸出小手钻进去,逗引婴儿发笑,使他感到很有趣,下次婴儿一见到这样的小圆圈就会把手伸出来,配合大人穿衣服。把两只裤腿挽成短裤形状,让孩子很轻松地将腿从裤腿中钻出来,几次以后婴儿就会知道穿衣服和裤子并不是一件很费力的事情,与爸爸妈妈共同穿衣裤是一件很愉快的事情。切记不要给孩子穿过紧的内衣内裤,尤其是松紧带一定要松一些。

% 10个月的婴儿不宜穿连体衣裤,这样的衣裤妨碍婴儿肢体的伸展和运动,穿脱起来也比较费力,还影响皮肤和周身血液循环,夏天妨碍出汗,冬天感到燥热,易引起婴儿情绪不安。宽松的衣服不仅可以使婴儿能够充分舒展肢体和利于运动,而且还使皮肤经受锻炼,不易感冒,穿脱起来比较容易。

% \hspace{0pt}\includegraphics[width=2.43357in,height=2.96503in]{media/rId265.png}\hspace{0pt}

% %ux5341ux4e0011ux4e2aux6708ux5a74ux513fux751fux957fux53d1ux80b2ux72b6ux51b5}{%
% \subsection{十一、11个月婴儿生长发育状况}%ux5341ux4e0011ux4e2aux6708ux5a74ux513fux751fux957fux53d1ux80b2ux72b6ux51b5}}

% 11个月的宝宝能够扶栏杆站得很稳,还能扶着栏杆迈出几步,扶栏杆蹲下和站起,双手能抓起玩具并摆弄许久。能听懂许多词汇的意思,用手势表达出身边的许多物品,准确指认眼睛、鼻子、嘴和耳朵等面部器官,发出的音节更多更丰富了,但不能准确表达。能使用勺子准确将饭菜放入口中,有时还会用杯子喝水,会打开小盒的盖子或者盖上。会观察身边的各种人或事物,开始有与同龄儿玩耍的愿望。

% %ux8fd0ux52a8ux53d1ux80b2-10}{%
% \subsubsection{1.运动发育}%ux8fd0ux52a8ux53d1ux80b2-10}}

% \textbf{独站练习}:进入11个月月龄的婴儿基本能够很自如地扶栏杆站立起来。如果在上个月加强扶栏杆训练,那么本月就可以脱离栏杆,独自站立片刻了。婴儿学会独站是有一个过程的,在会站之前,婴儿必须要很灵活的从卧位到坐位,从坐起到扶栏杆,甚至不扶栏杆也可以站立数秒钟。婴儿必须能从站立位很顺畅地蹲下,然后坐下,再到爬行,卧位等。这一系列动作是婴儿肢体在大脑支配下连贯活动的整个运动过程。如果有某一环节不熟练,动作不连贯,或者四肢不协调、不灵活,站立过程中婴儿就容易跌倒。绝大部分的婴儿对反复跌倒并不在乎,爬起来继续玩耍。有些比较娇气的婴儿跌倒后往往会大哭,遇到这种情景时可以轻轻抱起,稍加安抚即可,不必过分在意。在家中,可以将孩子放置到有栏杆的婴儿床中,让孩子练习坐下、站起来等动作。当看到孩子能够双手脱离栏杆独自站立起来时,父母或者身边的大人一定要拍手向宝宝表示祝贺,让孩子明白独自站立是令父母感到高兴的事情,激发孩子向上进取的愿望。

% \begin{quote}
% 育儿小百科

% \textbf{谨慎对待学步车}

% 近些年来出现了多种样式、美观大方,深受婴儿和家长喜爱的学步车。家长让孩子使用学步车的目的,首先是想让自己的宝宝早点学会走路。其次就是想避免婴儿在学习走路的过程中突然出现跌倒或摔跤的事故。但是,学步车对婴儿学习走路究竟有无好处,目前国内外专家还存在很大的争议。国外有一些专家认为,婴儿脊柱的发育还未完全成熟,背部肌肉的发育尚不完善,加之婴儿骨骼还在继续增长,一旦遇到不良刺激,就会出现骨骼发育异常。因此,过早使用学步车,婴儿下肢不能完全支撑躯干及上肢,身体的重量都集中在车内的坐垫上,久而久之就会引起脊柱和下肢骨骼的弯曲畸形,如``0"型腿或``X''型腿。另外,由于站立和行走均依靠学步车的支撑,反而影响了全身骨骼,特别是下肢骨骼的生长发育和进一步锻炼,导致独自站立和行走的时间反而推迟,因此学习站立和行走阶段,还是应该细心看护为好,因为学步车在婴儿移动的过程中,由于孩子不会掌握重心,导致学步车重心不稳,也会引起翻车等事故而伤及婴儿。
% \end{quote}

% \textbf{蹲下取物}:孩子在站立时,如果发现身边有更好的玩具,就会不知不觉慢慢弯曲双腿蹲下来,够取他们所喜欢的东西。有的父母看到婴儿会蹲下,通常会感到很惊讶,惊叹孩子又有进步了。孩子在蹲下的过程中,有时会摇摇晃晃,身体向前或者向后倾斜,这时候一定要注意保护好婴儿,以免头部受伤。

% \textbf{行走前的准备}:从婴儿扶栏杆迈步到独自站立是很短暂的过程,下一阶段就是行走的开始。一些家长担心孩子走路不稳或者摔倒,想用学步车代替婴儿的早期行走。

% %ux8bedux8a00ux53d1ux80b2-10}{%
% \subsubsection{2.语言发育}%ux8bedux8a00ux53d1ux80b2-10}}

% 有一小部分的婴儿会面向亲人发出``妈妈''和``爸爸''的声音,这类语言发育比较早的婴儿能够明白这是在叫喊自己的亲人。一部分婴儿还不知道这其中的意思,只是朦胧觉得发出这种声音之后,身边的亲人都会喜笑颜开,非常高兴。有一些婴儿会发出一连串令人听不懂的语言,其实这是在表达自己的意思。婴儿听到大人说话\textbf{很想与之交流},看见身边发生的事情很想发表自己的意见,但苦于不能表达,只好胡乱发出声音,引起大人的注意,这就是现在某些人所说的``\textbf{儿语}''。1岁左右孩子的``儿语''比较简单,一般为2-3个音节。但细心的家长会发现每当孩子发出``儿语''时,他都会很专注地注视某个人或者某件东西,也许还会露出笑容等,这正是婴儿独特的表达方式,每当大人看到这种情景时,应该\textbf{及时与宝宝对答},至于对答的内容是什么无关紧要,因为宝宝并不完全明白大人在说什么,可是宝宝知道大人回应了他所说出的``儿语'',他会更加起劲地唠叨``儿语'',从中体会与外界交流的感受。

% 这个月语言发育还有更加重要的一点,婴儿能够明白和理解一些简单的要求性语言,如``把娃娃给妈妈'',如果孩子愿意就会把娃娃递过来。能够理解``不''的意思,告诉宝宝``不要把手指放到嘴里'',听话的宝宝就会把放到嘴里的手指拿出来。当宝宝想拿一些不应该拿的东西时,妈妈如果说:``宝宝不动。''宝宝就会放下手中的东西。有一些宝宝还能明白一些带有询问性的语言,如妈妈问宝宝:``宝宝的衣服到哪里去了?''宝宝就会四处寻找,当发现衣服时就会用手指向其衣服的方向,或者直接指自己身上穿的衣服。以上情景说明宝宝已经对语言有了更加深刻的理解。

% \hspace{0pt}\includegraphics[width=2.90909in,height=2.6014in]{media/rId271.png}\hspace{0pt}

% %ux8ba4ux77e5ux548cux751fux6d3bux4ea4ux5f80ux80fdux529b-8}{%
% \subsubsection{3.认知和生活交往能力}%ux8ba4ux77e5ux548cux751fux6d3bux4ea4ux5f80ux80fdux529b-8}}

% \textbf{寻找东西的能力}:11个月的婴儿已经具有找寻看不见但记忆中已经有保留的东西的能力。这是因为在婴儿的脑海中,身边的物品已经留有很深刻的印象,而且这种记忆通常会持续一段时间。例如,妈妈或爸爸离开几天甚至十几天之后突然回来,大部分的婴儿都能一眼认出妈妈或是爸爸,表现出极大的欣喜和高兴。在婴儿面前用小花布包住一块颜色鲜艳孩子经常拿着玩的积木,记忆能力好的婴儿就会知道这块积木藏在小花布里,就会主动地用手打开小花布,拿出其中的积木。把小圆珠放进不透明的塑料盒子里,盖上盖子,对婴儿说:``刚才的小圆珠哪里去了?''大多数的婴儿听到这种询问时,会向周围看一看,然后注视塑料盒,他这是在回忆刚才大人的动作,当注视小塑料盒一会儿,他会试着去打开这个塑料盒的盖子,当发现小圆珠在塑料盒中时,婴儿会倒出盒子中的小圆珠。有一些记忆力比较好的婴儿甚至能够找到不在眼前放置的非常熟悉的玩具,如妈妈和婴儿对坐在小床上,在婴儿能看见的视野之外把玩具熊藏到被子底下,询问婴儿:``刚才玩的玩具熊哪里去了?''那么婴儿就会判断妈妈并没有离开这里,玩具熊一定就在床上,因此婴儿就会掀开枕头或者被子寻找,一旦发现玩具熊他会很高兴,并主动将玩具熊递给妈妈。

% \textbf{从容器内拿出物品或放进物品}:这个月的婴儿已经明白了小东西可以放入大容器中。实际上这就是大和小的概念,婴儿知道大东西绝对不会放到比它小的容器之内,只有小东西才会放到大容器中。在婴儿面前放置一个大杯子和几块积木,大人先将一块积木放到杯子中,婴儿会目不转睛地看着,鼓励婴儿模仿大人拿积木放到杯子里。婴儿这种理解能力非常快,几次玩耍下来,婴儿就会很迅速很熟练地将多块积木放入杯子中或从杯子中拿出来。这种能力的培养非常重要,它反映婴儿对物体大小和形状的理解,同时还能培养判断能力和逻辑思维能力。

% \textbf{对复杂动作的模仿能力}:婴儿从出生的第一天就开始具有模仿能力。\textbf{模仿能力是一个循序渐进的从简单到复杂的过程},这个过程需要婴儿不断接触外界事物,在不断与人的交流中学习,模仿发音、模仿动作、模仿使用工具等,这些都是人类高级活动的体现。11个月的孩子已经能模仿许多大人的动作,现举例说明。

% \begin{enumerate}
% \def\labelenumi{\arabic{enumi}.}
% \item
%   \textbf{模仿推玩具小车}:大人将玩具小车放置在小桌上,用手推着向前进或转弯,婴儿看过几次之后就会自己用手推着小车前进或转弯。大人将玩具小车放到地上,婴儿会在大人的协助之下,蹲下来继续用手推着前进。
% \item
%   模仿搭积木:本月龄的孩子会模仿大人将积木一块一块叠搭起来。刚开始仅仅会叠搭12块积木,慢慢就会叠搭34块,或者5\textasciitilde6块积木而不倒。\\
%   \hspace{0pt}\includegraphics[width=2.53147in,height=2.62937in]{media/rId276.png}\hspace{0pt}
% \item
%   \textbf{模仿拍娃娃}:大人抱着娃娃,用手拍着娃娃的胸部说:``小宝乖乖,赶快睡觉。''婴儿看到大人这个动作一般都要注视一会儿,然后自己也会抱起娃娃,用手拍打,一些婴儿嘴里还不停地唠叨着大人听不懂的``儿语''。
% \item
%   \textbf{套筒游戏}:现在市场上出售非常漂亮的套筒玩具,套筒大小规格各异,一般按大小尺寸逐一套在带有底座的小柱子上,形成一个多层的套筒。这种游戏可以帮助孩子识别哪个是大的,哪个是小的,并根据大小排列顺序,培养孩子逻辑思维能力。\\
%   \hspace{0pt}\includegraphics[width=2.61538in,height=2.72727in]{media/rId281.png}\hspace{0pt}
% \item
%   \textbf{模仿画画}:这个月龄的婴儿可以模仿大人画出竖道或者圆圈。大人先在一张白纸上用彩笔画出一条竖道或者圆圈,然后将彩笔递给孩子,鼓励孩子画出竖道或者圆圈,有一些婴儿胡乱画一些横道或竖道,这没有关系,只要孩子能握住笔画出道道就基本可以了。\\
%   \hspace{0pt}\includegraphics[width=2.7972in,height=2.72727in]{media/rId285.png}\hspace{0pt}
% \item
%   \textbf{学习与同龄儿交往的能力}:快1岁的孩子开始出现想与同龄儿进行交往的愿望。最初孩子只是用眼睛注视对方,当看见对方穿着五颜六色的服装时,会表现出极大的兴趣,如微笑,用手摸对方的衣服,或用手拉对方的手,或把自己手中的玩具递给对方。如果将两个婴儿同时放在床上,他们会爬到一起互相拍打对方、拉扯对方。这时大人可以在旁边帮助两个婴儿共同玩耍,让他们碰碰头,摸摸手,或者交换玩具。不要小看这些非常普通的动作,对于孩子今后的社会交往,人际关系具有很重要的影响。父母应经常带孩子到户外运动,多接触同龄的孩子,尽力创造与人交往的环境和条件,以利于孩子的智能发育。\\
%   \hspace{0pt}\includegraphics[width=2.74126in,height=2.76923in]{media/rId289.png}\hspace{0pt}
% \end{enumerate}

% %ux5341ux4e8c12ux4e2aux6708ux5a74ux513fux751fux957fux53d1ux80b2ux72b6ux51b5}{%
% \subsection{十二、12个月婴儿生长发育状况}%ux5341ux4e8c12ux4e2aux6708ux5a74ux513fux751fux957fux53d1ux80b2ux72b6ux51b5}}

% 时间过得飞快,孩子已经来到人世间整整一年了。孩子在体格和智能发育方面有了飞速发展,如从站立到开始独立行走,能说出一个到几个词,开始具有逻辑思维能力,萌发要参与社会交往,与人交往的意愿,这些都标志着孩子开始进入了身心发展的一个重要时期。

% %ux4f53ux683cux53d1ux80b2-3}{%
% \subsubsection{1.体格发育}%ux4f53ux683cux53d1ux80b2-3}}

% 婴儿体重可以达到9\texttt{\textasciitilde{}}\hspace{0pt}10千克,身长可以达到75\texttt{\textasciitilde{}}\hspace{0pt}80厘米,体重或身长超过或低于平均10\%都属于正常范围。头围到满周岁时可以长到44\textasciitilde46厘米,男婴略大于女婴。一般认为,周岁以内的婴儿头围比胸围大属于正常,如果婴儿发育比较好,到1岁时胸围能赶上头围,为46厘米左右。

% \textbf{囟门}:1岁时的前囟门未闭合是完全正常的,前囟门闭合通常在12\texttt{\textasciitilde{}}\hspace{0pt}18个月之间。牙齿发育良好的婴儿已经长出6\texttt{\textasciitilde{}}\hspace{0pt}8颗牙,按照正常顺序应该从正中切齿(上下各2颗)开始萌出,然后左右切齿各1颗(上下各2颗)共8颗,个别婴儿还可能长出乳磨齿。

% \textbf{脊柱的发育}:刚出生的新生儿脊柱几乎是直位的,到2\texttt{\textasciitilde{}}\hspace{0pt}3个月时婴儿能抬头,脊柱出现了第一个生理性弯曲,即颈椎向前凸起。颈椎骨的弯曲有助于婴儿头部的前后左右活动,比如抬头或转头;6\texttt{\textasciitilde{}}\hspace{0pt}7个月时婴儿开始会坐,形成了脊柱第二个生理性弯曲,即胸椎向后凹起,有助于婴儿坐位时的活动;11\texttt{\textasciitilde{}}\hspace{0pt}12个月时,婴儿身体能够直立或者行走,形成脊柱第三个生理性弯曲,即腰椎向前凸起。脊柱的三个生理性弯曲是婴儿从卧位向坐、站立、行走发展过程中,随着脊柱的增长形成的,有利于身体的平衡。一直到6\texttt{\textasciitilde{}}\hspace{0pt}7岁韧带发育后,这些弯曲才固定下来。坐、站、行走姿势不正确,或者骨骼有病变,如佝偻病等都可以引起脊柱发育异常或造成畸形,家长应该按照婴儿的发育和发展顺序进行运动训练,保证婴儿脊柱的正常发育。

% %ux8fd0ux52a8ux53d1ux80b2-11}{%
% \subsubsection{2.运动发育}%ux8fd0ux52a8ux53d1ux80b2-11}}

% \textbf{学会站立}:一部分11个月婴儿会自己站立数秒,到1岁时大部分婴儿均已经能够独自站立数秒以上。家长先扶着婴儿站好,然后轻轻松开双手,站立稳的婴儿通常可以自己站立数秒钟以上。但有的婴儿站立不稳,甚至不能离开大人的支撑,这也是正常的,只是还需要加强训练。

% 学会站立是学习走路的基础。在站的基础上学会直立行走,曾经是从类人猿到人的转变中重要的一步。它不仅对于婴儿的生存具有重要意义,而且对于婴儿的心理发育有重大影响。站立使婴儿的视觉器官处在最高位,婴儿的视野更加开阔,进入脑海中的事物更加丰富;站立使婴儿与其他人的交流更加方便和自如;站立彻底解放了婴儿的双手,双手可以更加自由地够取东西,摆弄东西,为人类认识问题和解决问题提供了先决条件。站立还为语言的发展创造了最好的发声条件。

% \textbf{开始学走}:这个月龄的婴儿可以扶着栏杆行走,也可以在大人的牵扶下行走几步。开始牵扶婴儿行走时稍稍用力支撑即可,当婴儿站立很稳想要迈步时,大人只需牵住婴儿的手就可以了。这个时期婴儿已经可以自己移动双腿迈步,只是平衡功能比较差,需要借成人的一只手来保持身体的平衡,防止跌倒。

% 开始学走也是人生道路上的重要一步。它有助于婴儿身体运动功能的发育和发展,使身体各部位运动更加灵活和自如,它可以扩大婴儿对自然事物的认识范围,使婴儿不但能主动地接触事物,还可以从不同角度去观察和认识它们。行走为空间知觉、初步的思维活动形成准备了一个重要条件,因为空间知觉主要是动觉和视觉的联系,行走有助于这种联系的形成。另外,在多方面接触事物的过程中,行走可以丰富感官的认识,为初步思维活动的产生创造条件;行走还可以为有目的的活动,如各种游戏等准备条件。

% \textbf{视觉功能与双手的协调功能}:从新生儿期开始,婴儿就开始具有视觉、听觉和肢体活动能力。但在早期视听和运动功能是分开的,到2\texttt{\textasciitilde{}}\hspace{0pt}3个月时,视听能力能够结合起来,如婴儿听到一种声音时,会把眼睛转向声音的方向,也就是说通过声音的方向寻找发出声音的东西。将视、听觉和肢体运动结合起来的发育过程需要很长一段时间的形成过程。8\texttt{\textasciitilde{}}\hspace{0pt}9个月时婴儿把看起来很喜欢的玩具摔到地板上,聆听发出的声音,从中体会快乐动作,这些就是一种视听和肢体运动协调的过程。11\textasciitilde12个月时婴儿又有了进步,视听功能可以与双手的精细动作组合一起参与人类活动,如按琴键,婴儿看到大人用手指按压琴键能够发出悦耳的声音,他会观察一会儿,然后用手去按一按试试,当听到发出同样的声音时他会很高兴,反复不断地按压琴键,从中体会各种不同的声音;敲小鼓,大人先做示范给婴儿看,然后给婴儿一手拿小鼓,一手拿鼓槌,让他自己去敲小鼓,当击鼓位置正确时,才能发出与大人敲鼓同样的声音,如果鼓槌撞击到鼓面以外的位置,将会发出不同的声音。因此,婴儿必须手眼要协调,眼睛看着,拿着鼓槌的手必须对准小鼓才行。婴儿为了能够敲出与大人同样的声音,会不厌其烦地反复敲鼓,当他听到敲出了与大人刚才同样的声音,他明白敲击这个位置是对的,于是下一次再敲鼓时他会直接拿鼓槌敲击鼓面。上述这种练习对婴儿至关重要,它对婴儿将来学习使用工具都有很大的好处,所谓``心灵手巧''就是这样练习出来的。

% \textbf{学会使用工具}:经过一年的观察和学习,婴儿手的精细动作已经很灵活,如自如抓握玩具,会自己使用奶瓶喝奶,会用小勺吃饭,会将小东西放到大箱子中并盖上盖子(整理能力),能搭几块积木,会使用彩笔画道道等。虽然这些都是极为简单的工具,但这是使用复杂工具的基础,是以后适应成人生活,进行创造性活动的基础。

% %ux8bedux8a00ux53d1ux80b2-11}{%
% \subsubsection{3.语言发育}%ux8bedux8a00ux53d1ux80b2-11}}

% \textbf{能够听懂许多词汇}:满1岁的婴儿可以听懂一批词汇,周围接触到的东西名称,如食品类名称``香蕉、苹果、饼干'',服装类名称``帽子、衣服、裤子'',以及和某种动作联系到一起的动作性词汇,如``再见、谢谢、摇头、吃饭、尿尿''等。

% \textbf{能够说出少量词汇}:接近1岁的婴儿能够说出数量很少的词汇,一般认为不超过10个,最先说出的词汇是``妈妈''或``爸爸''一类的词,因为这些词发音容易,代表的事物又和婴儿最亲近。另外,在自然状态下,父母总希望婴儿尽快会喊``妈妈''或``爸爸'',因此首先会反复教婴儿说这类词汇。

% \textbf{完成了学习语言的准备阶段}:婴儿通过一年的语言、听力和发音练习,大脑中已经对周围大人所说的语言有了深刻的印象。尽管婴儿能说的词汇很少,但只要发出声音来就意味着语言的开始。

% %ux8ba4ux77e5ux548cux751fux6d3bux4ea4ux5f80ux80fdux529b-9}{%
% \subsubsection{4.认知和生活交往能力}%ux8ba4ux77e5ux548cux751fux6d3bux4ea4ux5f80ux80fdux529b-9}}

% \textbf{思维发育水平}:思维作为一种经过大脑活动的、概括的、间接的反应形式,不同于直接的反应形式------感觉。感觉是与生俱来的感知。一个新生儿只要有健全的大脑和感觉器官,就可以直接对外部世界作出反应。思维则不同,它不是生来就有的,而是在婴儿来到世上的第一年,在逐步复杂的感知觉活动的基础上产生的。

% \textbf{自我意识的发育}:自我意识即对自己的理解和认识,婴儿出生后有一个逐步认识自己的过程。

% 婴儿开始时并不了解自己,不知道自己的独立存在,分不清自己和自己周围其他客观事物的界限。例如,一个9个月以前的婴儿,吸吮自己的手指就像吸吮其他玩具一样,并不理解手指是自己身体的一部分;如果你将一个玩具紧紧贴在他的肚皮上,他也不知道伸手拿开接触到他身体上的东西。当婴儿11\textasciitilde12个月时,开始意识到自己的独立存在,这是从理解自己的动作和力量开始的。例如,婴儿用手扔一个皮球,皮球滚出很远,妈妈给拾了回来,他又扔一次,皮球又滚走了,如此反复数次,婴儿逐渐理解了自己的动作和动作带来的力量,同时这种动作和妈妈的动作是不一样的。如果妈妈拾球次数多了,会对婴儿说``不要扔球了'',婴儿会意识到妈妈不喜欢自己这种动作,从中更加明确这种动作是自己独立的动作。通过类似这样大量的事物,婴儿逐渐把自己从客观物体中分离出来。1岁的婴儿会拒绝任何外来的物体接触到自己的身体,如到医院看医生,当医生把听诊器放到婴儿小胸脯上的时候,他会伸手拿开听诊器以表示拒绝。

% \textbf{交往活动}:12个月的婴儿交往意识更加浓厚,不仅同父母或者家庭中其他成员交往,而且追求与同龄小伙伴交往的愿望日益强烈。

% 1

% %ux7b2cux56dbux8282-ux5e7cux513fux671fux7684ux751fux957fux53d1ux80b2ux7279ux70b9}{%
% \subsection{4第四节
% 幼儿期的生长发育特点}%ux7b2cux56dbux8282-ux5e7cux513fux671fux7684ux751fux957fux53d1ux80b2ux7279ux70b9}}

% 幼儿期是指宝宝1\textasciitilde3周岁之间的年龄阶段。一个人在一生的发育当中,幼儿期占有非常重要的地位。这是因为在这个阶段,孩子的智能发育和变化尤为突出,家庭和周围环境及教育将会影响孩子终身。所谓``3岁看到老''、``3岁所学,终身受用''等多年来在民间广为流传的谚语就充分说明了这一点。因此,有人也称这一时期为获得人类经验和学习的最重要的关键时期。

% 下面我们将幼儿期分为几个发育阶段,分别叙述各阶段的生长发育特点。

% %ux4e0013-15ux4e2aux6708ux5e7cux513fux751fux957fux53d1ux80b2ux72b6ux51b5}{%
% \subsection{一、13-15个月幼儿生长发育状况}%ux4e0013-15ux4e2aux6708ux5e7cux513fux751fux957fux53d1ux80b2ux72b6ux51b5}}

% %ux524dux56df}{%
% \subsubsection{1.前囟}%ux524dux56df}}

% 大部分的宝宝前囟已经关闭。如果到15个月还没有关闭,应该及时到医院接受检查,大多数的宝宝可能因缺乏维生素D所致,还有一部分是因为脑积水所致。

% %ux8fd0ux52a8ux53d1ux80b2-12}{%
% \subsubsection{2. 运动发育}%ux8fd0ux52a8ux53d1ux80b2-12}}

% 这个年龄阶段的宝宝大部分自己能够站得很稳,并能独立行走。两手自如拿起喜爱的玩具进行组合,如叠搭积木,对小孔形态的洞洞比较感兴趣,经常将手指伸进去,并反复注视,大把握笔,在白纸上乱涂,乱画。

% (1)\textbf{独立行走}:每个孩子学会独立行走的发育速度是不相同的,但一旦能够稳稳站住,孩子就会松一口气,明白自己能够站立了。下一步就是迈步的问题了。刚开始迈步行走时,孩子往往要借助一些外来的支撑,如床栏杆、大人的手、手推车、拖车等,反复行走多次后,孩子胆子大了起来,终于有一天不用大人牵着,或者不依靠外力就能够独立行走了。孩子会感到无比的高兴,他觉得世界在他面前突然开阔了许多。他可以自己围着小桌转上几圈,可以自己走出房间去寻找妈妈或爸爸,也可以去拿想要的玩具。

% \hspace{0pt}\includegraphics[width=2.58741in,height=2.25175in]{media/rId302.png}\hspace{0pt}

% 有一些孩子从学会走路开始,就不停地行走。经常听到家长这样抱怨:``天天看孩子这样走路,真是头都大了。''\,``天天走这么多路,难道他不知道累吗?''这些家长想对了,刚刚学会走路的孩子与大人刚刚学会骑自行车或刚刚学会开车一样,他们就是喜欢走路,\textbf{走路对于他们来说就是一种快乐},一种自豪。这一阶段,孩子可能会对其他方面暂时失去兴趣,而是专心致志地练习走路。

% 孩子在刚开始练习走路时,往往走得东倒西歪的,令人感觉要摔倒一样,这主要是孩子还不能掌握好身体重心的缘故。有一些孩子为了保持重心,会把双腿叉开行走,如果同时再摇晃上身,简直就像走``鸭步''。一些孩子刚开始只会用足尖走路,取不到自己想得到的玩具,反复几次之后,孩子终于可以蹲下了,取到了他喜欢的玩具。

% 从蹲位到站起来的过程就容易多了,但是在这种复位的运动中,如果孩子身体重心把握不好,反而容易一屁股坐到地上,胆子比较小的孩子可能会哭,胆子大的孩子根本是无所谓的,他会用双手支撑着再度站起来,或者直接再度蹲下继续玩耍。家长应该在旁边鼓励孩子,为孩子能够再度爬起来而拍手叫好。对于胆小的孩子可以帮助安抚一下,目的是为了减轻孩子的压力,使他能够再度爬起重新蹲下,继续他喜欢的游戏。

% \textbf{(2)搭积木}:搭积木是练习双手精细动作与手眼协调的最好游戏,一般要先从小方积木的叠搭开始,可以选择1\textasciitilde2厘米大小的积木,放到孩子的面前,先做一些示范动作让孩子看,然后孩子把一块积木放在另一块积木的上面,当两块积木叠搭稳,再示意孩子继续把积木放在两块积木的上方,鼓励孩子尽量多叠搭积木。这个月龄的孩子一般都能搭上两块积木,但孩子搭不上积木的,不要着急,反复练习,孩子总会找到窍门的。

% \textbf{(3)练习把小丸投入小瓶子中并拿出}:让孩子坐好,拿出一个较小的玻璃瓶子,同时在瓶子的旁边摆放几粒小糖球,告诉把小糖球放到瓶子中,孩子很可能费了很大的力气才放进去一个,或者根本放不进去,没有关系,可以反复做示范动作。这个练习实际上也是培养孩子的耐力和毅力,有些孩子开始玩时小手老是对不准瓶口,结果糖球总是掉在瓶子的旁边,孩子这时会很不耐烦,也许会扔掉糖球干脆不玩了。每当这时,家长可以随着孩子的意愿玩一些孩子感兴趣的玩具和游戏,下一次当孩子高兴或情绪好的时候,再与他玩这种游戏,当孩子把小糖球放入小瓶子中之后,可以示意孩子将瓶子翻过来,把糖球倒出来。倒出这种动作看起来容易,但要知道倒出的动作意识是一个大脑综合判断的复杂过程,孩子必须明白瓶子一定要翻过来东西才能倒出来这个道理。因此,1岁的婴儿可以知道往小瓶子里放东西,但是知道从小瓶子中倒出东西还需要有一段时间的大脑思维发育过程,大人可以反复做示范动作,让孩子明白瓶子在什么位置小球才能倒出来。一旦孩子有一天终于完成了投进和倒出的一系列动作,就会引发他喜欢上这种游戏。

% \hspace{0pt}\includegraphics[width=2.54545in,height=2.67133in]{media/rId305.png}\hspace{0pt}

% 总之,在与孩子的共同玩耍中,一定要注意练习双手的动作,这对于孩子今后双手的灵活性和准确性是非常重要的。

% %ux8bedux8a00ux53d1ux80b2-12}{%
% \subsubsection{3.语言发育}%ux8bedux8a00ux53d1ux80b2-12}}

% 这个年龄阶段的孩子能够有意识地说简单的单词性话语,如``爸爸、妈妈、姨、奶、姥''等,这些既简单又有代表性的语言是这个年龄阶段孩子表达的主要方式。通常表达意思的同时,大多要加上各种手势,目的是为了让别人明白自己要表达的意思。比如说``妈妈'',可能是想让妈妈过来干点什么,如果同时张开双臂则表示想让妈妈抱,如果同时用手指来指着奶瓶,说明孩子这时想喝奶。又比如,当孩子对某些食物比较喜欢吃的时候,大人无意中说出``好吃''的单词,孩子印象非常深刻,一下就把这个单词记住了,以后每当他想要吃某种食品时,都会说``好吃'',并用手指着他喜欢的食物,反复数次之后,大人也就明白了孩子说``好吃''的意思就是代表要吃某种他喜欢吃的食品。

% 孩子在这个时期说出的语言的准确性是很不可靠的,家长应把孩子说话时附带的手势、表情、体态等许多情景性表现作为参考的因素,来揣摩孩子所要表达的意思。这个时期孩子因为感到说话很费力,因此不太愿意过多说话,而是尽量用手势表达。但是,当大人说出完整的语言,孩子基本都能理解,而且还能按照去做。

% 研究结果表明,一个1岁零1个月的孩子就可以听懂成人的如下问话:``要吃奶吗?要吃就点头。''孩子点头,``和爸爸睡吗?''孩子摇头;``和姐姐睡好吗?''孩子点头。当孩子要把花生往嘴里塞,大人说:``不要吃!''孩子就会停止往嘴里塞花生的动作。孩子正要伸手打人时,大人说:``不好。''孩子就会把手放下来。这些现象充分说明孩子完全能够理解大人的语言,并能够在大人的语言支配下进行各项活动。

% 孩子已经会有意识地叫出自己周围的亲人,如``妈妈、爸爸、奶奶、爷爷''等。每当孩子奶声奶气地叫喊父母或爷爷、奶奶时,大人们一定要高高兴兴大声答应着,让孩子感受到喊叫大人时他们是多么的愉快,促进孩子与亲人的交流。

% \hspace{0pt}\includegraphics[width=3.76224in,height=2.47552in]{media/rId309.png}\hspace{0pt}

% %ux8ba4ux77e5ux548cux751fux6d3bux4ea4ux5f80ux80fdux529b-10}{%
% \subsubsection{4.认知和生活交往能力}%ux8ba4ux77e5ux548cux751fux6d3bux4ea4ux5f80ux80fdux529b-10}}

% 1岁以后的孩子非常渴望与大人或同龄儿交流,其中更愿意与同龄儿玩耍。但是有一部分孩子,尤其是女孩开始知道害羞或认生感很强,这时需要大人的\textbf{协调和帮助},要\textbf{经常}带孩子到户外活动,加强与他人的交往,培养孩子与人交往的能力。

% \textbf{(1)握笔乱画}:看到孩子拿不好笔时,应当想办法先让孩子正确握笔,告诉他细头的部分是笔尖,应该是冲下方的。大人可以先画几道,让孩子懂得通过笔尖与纸的接触可以画出不同颜色的道道来。待孩子学会拿笔之后,示意他用笔尖去接触纸面,让孩子用力画出道道。反复多次的练习,孩子就会对握笔画道发生兴趣,到了某一天,孩子就会主动拿起笔并画出各种各样的横道或竖道。孩子画出的道道并没有什么意义,大人也没有必要要求他们画出什么,\textbf{主要是训练孩子握笔的能力},也许孩子从随便乱涂的横道竖道中想象出点什么,这对孩子将来的学习和创作是非常有好处的。

% \textbf{(2)对于欲望的表达}:孩子已经可以明确表达愿意要、愿意做,或者不愿意要、不愿意做的欲望,如给孩子喜欢的玩具时,他会抓在手里不放,而对于孩子不喜欢的玩具他会看也不看就马上扔掉,对于孩子不想吃的东西,他是坚决不吃的。

% \hspace{0pt}\includegraphics[width=2.86713in,height=2.53147in]{media/rId313.png}\hspace{0pt}

% \textbf{(3)与大人共同玩球}:当孩子能够自己行走坐卧时,大人就可以与孩子玩球了。不要买太大的球,只要孩子能握住就可以了。两人对坐在床上,大人将球抛起,让孩子看到小球在大人手中一起一落多么有意思,引起孩子的兴趣之后,将球扔给孩子,让他自己玩球。刚开始孩子只会用手握住球,或者让球从手中滑落,而不会扔球的动作,渐渐地出现把球扔到地上的动作,终于有一天孩子把球朝上扔了出去,一旦发现孩子出现这种动作,家长一定要及时把球扔回给孩子,反复练习数次之后,孩子便会运用自如,他觉得这个游戏很有意思,由此也会引发孩子对球类游戏的爱好。这个游戏是训练孩子手眼协调,以及四肢的灵活程度和跑跳能力。

% \hspace{0pt}\includegraphics[width=2.96503in,height=2.22378in]{media/rId316.png}\hspace{0pt}

% %ux4e8c1618ux4e2aux6708ux5e7cux513fux751fux957fux53d1ux80b2ux72b6ux51b5}{%
% \subsection{二、16\textasciitilde18个月幼儿生长发育状况}%ux4e8c1618ux4e2aux6708ux5e7cux513fux751fux957fux53d1ux80b2ux72b6ux51b5}}

% 接近1岁6个月的孩子已经能走得很稳了。看见地上有玩具时,可以弯腰蹲下玩耍一会儿,然后站起来继续行走。扶着栏杆可以爬几步楼梯,开始有跑步意识。

% 模仿大人翻书或看书,认识图画中的一些动物、食物、常见日用品等,模仿大人画出横道或竖道;模仿大人撕纸;会灵巧地将小物品放入小瓶子中,或从瓶子中倒出来,会搭4\textasciitilde5块积木。能说出最常见的物品名称,开始会使用动词,如``吃、喝、抱''等,能听懂大人说出的完整句子。开始有意识使用工具,如用杯子喝水,用小勺吃饭等。

% %ux8fd0ux52a8ux53d1ux80b2-13}{%
% \subsubsection{1.运动发育}%ux8fd0ux52a8ux53d1ux80b2-13}}

% \textbf{(1)行走稳当}:经过几个月的练习,孩子已经能够自己行走得很好了,一部分孩子行走过程中,如果看到地上有他喜欢的玩具,他会停下脚步,慢慢弯下腰,或慢慢蹲下来玩一会儿。有时候孩子感觉到有些累,就会一屁股坐下来接着玩耍一会儿。当孩子想到其他地方去的时候,他就会爬起来和站起来,继续向前走。这种走走停停、弯腰坐下、爬起站立的动作每天不知道要重复多少次。

% \textbf{(2)上下楼梯}:上下楼梯实际上是运动能力和深部感觉能力的结合过程。首先,孩子要明白楼梯的位置是高于或低于现在孩子本身所在的位置,而且这种高度是可以触及的位置。当大人第一次牵着孩子的小手爬上楼梯时,孩子知道必须要抬起一条腿,才能够着高出的台阶,但是他不会两腿交替使用,他必须要把两只脚同时迈到同一位置之后才能再上一层楼梯。因此,家长开始带孩子上楼梯时不必着急,让孩子一步一步登上楼梯,孩子会觉得登上一层就会高一点儿,从中体会登高的喜悦和快乐。上楼梯比下楼梯要容易一些,好多胆小的孩子往往学会了上楼梯,但很长时间内还不敢下楼梯。有的孩子上楼梯不使用旁边的栏杆,而喜欢爬着上去,这样的孩子一般说来属于运动发育相对慢一些的类型,或者说是胆子比较小的孩子。家长不必介意,可以让孩子随意,只要孩子喜欢,任何一种运动方式都是对孩子有利的。在这里需要提醒家长注意的是,在孩子下楼梯时一定要保护好孩子,以免孩子摔下去。

% \textbf{(3)搭4〜5块积木}:本月龄应该继续训练孩子搭积木能力。接近1岁半的孩子已经不能满足搭2块积木了,他们逐渐能搭上3块、4块、5块积木而不倒。

% %ux8bedux8a00ux53d1ux80b2-13}{%
% \subsubsection{2. 语言发育}%ux8bedux8a00ux53d1ux80b2-13}}

% \textbf{(1)使用动词}:孩子有一天突然向妈妈伸出手说出了``抱''字,某一天在饭桌上突然对妈妈说出了``吃''字,这些动词的说出正是本月龄的特点。本月龄的孩子能说出的动词是很有限的,只会说日常生活中常见的基本词汇,如``抱、走、吃、喝''等,大约有十几个单词。家长可以反复在孩子面前重复这类单词,增加孩子动词类的语言词汇量。

% \textbf{(2)儿歌训练}:为了进一步促进孩子的语言能力发育,从这个月开始应该经常给孩子念一些儿歌,儿歌中有许多孩子熟悉的名词和动词,有一些句子合辙押韵听起来很好听。如果再配一些图画,更会引起孩子兴趣。例如,``小白兔,白又白,两只耳朵竖起来,爱吃萝卜爱吃菜\ldots\ldots''。虽然孩子并不能说出儿歌中完整的句子,但是其中重要的发音会给孩子留下很深刻的印象,一旦在其他语言中涉及儿歌中的词汇时,孩子就会不由自主地发出这个音节。反复朗诵儿歌多遍之后,孩子也会随着大人的语调发出声音,听到孩子对某个词汇有印象时,可以放慢这个词的速度,有意识地让孩子跟着发出声音,逐渐孩子就会说出更多的词汇。教儿歌是培养语言能力和记忆能力的最好方法。

% %ux8ba4ux77e5ux548cux751fux6d3bux4ea4ux5f80ux80fdux529b-11}{%
% \subsubsection{3.认知和生活交往能力}%ux8ba4ux77e5ux548cux751fux6d3bux4ea4ux5f80ux80fdux529b-11}}

% \textbf{(1)模仿大人撕纸}:6个月以后的婴儿开始出现撕纸动作,婴儿拿起一张纸往往要端详半天,然后开始用手大把去抓,在抓纸的过程中,无意中撕开了纸,婴儿对撕纸时发出的声响感到非常快乐和惊奇。同时婴儿还会发现,通过自己发出动作,将面前的纸改变了形状,变得与先前不一样了,这是多么不可思议的事情啊。1岁以后孩子撕纸有了进步,他会用双手同时拿纸来撕,有时还会将书上的纸直接撕下来,撕下来后他会拿在手里注视一会儿,然后扔掉继续撕,如果家长不理会,可能他会把一本书都撕开,撕得满床满地都是。1岁半的孩子撕纸有了进步,他会把大张的纸撕成小纸块,再撕成小纸屑。

% \hspace{0pt}\includegraphics[width=2.85315in,height=2.43357in]{media/rId323.png}\hspace{0pt}

% 孩子的这种撕纸行为,实际是一种手部精细动作和灵巧性的训练,孩子通过撕纸可以初步感受到,自己的动作能够改变外界事物,从中得到乐趣,同时还可以训练手部的精细动作和手眼协调能力,刺激大脑的发育和成熟。因此,孩子撕纸是对大脑发育非常有利的玩耍游戏,家长不仅不要阻止孩子撕纸,相反还要鼓励孩子多撕纸。

% 1岁半的孩子手的精细动作更加灵巧和精确了,而且这个月龄的孩子大多明白了各种各样的形状。因此,家长可有意识地教孩子撕一些简单的物体形状,如方形、圆形、三角形等,或用彩笔画出太阳、月亮、五角星等形状,让孩子按照形状一点儿一点儿撕开。

% \textbf{(2)模仿大人翻书或看书}:家长给孩子一本书,让他看书中的图画,并对他讲一些与图画中有关系的内容,也许刚开始孩子并不太喜欢听,这是因为他听不太懂,大人说的话他不十分明白。但是反复让孩子看同一个画面,同时反复说出同样的语言,渐渐孩子就明白了这幅图画中的意思。下一次一旦把这本书放到孩子面前,他就会不由自主地拿起书学着大人的样子翻起书来,看到熟悉的画面,他会欣喜万分,嘴里还不停地唠叨着什么,看到自己不太熟悉或根本没有见过的画面,他会多注视一会儿,努力辨认图中的画面,一旦感到没有意思,孩子就会把书撇到一边,玩起其他游戏。孩子看书的时间非常短,有的仅能坚持数秒,最长时间也就几分钟而已,因此看见孩子翻书或者看书时尽量要多陪着孩子,给他讲解书中的内容,如认识图画中的一些动物、食物、常见日用品,引起孩子的兴趣,延长看书时间。

% \hspace{0pt}\includegraphics[width=2.81119in,height=2.29371in]{media/rId326.png}\hspace{0pt}

% \textbf{(3)模仿大人画出横道或竖道}:1岁的孩子会握笔乱画,画出道道根本看不出是一条直线,歪歪扭扭,断断续续,这是孩子握笔方式不对或因胳膊力量不足而导致握笔不稳的表现。接近1岁的孩子已经能够很稳当地握住笔了,开始时大人可以把着孩子手,教他画出直线,如横道或竖道。手把手教的目的主要有两点,一是教会孩子握笔方式,二是让孩子知道如何用力才能画出道。经过耐心细致的训练,孩子逐渐学会了如何握笔和如何用力,如果让孩子自己去画,他就会自然而然地画出横道和竖道了。早期培养孩子握笔画道的能力可以引发孩子对绘画的兴趣,培养孩子集中注意力能力。

% \textbf{(4)准确使用杯子和小勺喝水}:在婴儿时期,婴儿就会用小勺吃饭,但那只是使用小勺的初级阶段,仅仅能将小勺拿起,把盛放在勺子里的饭菜放入口中。婴儿拿勺子时方向不分,左右不分,胡乱拿起就往嘴里塞,结果大部分的食物都撒满了桌面,沾满了嘴唇和面部,仅有一小部分食物放入了口中。而对于1岁多的孩子来说,可以根据勺子的不同方向用手准确地将勺子中的饭菜放入口中。有的孩子还会学着大人的样子拿着小杯喝水。

% \textbf{(5)使用工具能力的进步}:孩子大脑已经开始具有初步认识和思考能力。他们用自己对周围工具的认识进行思考,试图解决他们自己认为能力达不到的事情,如看见桌子上有一件可爱的玩具,由于个子矮小够不着,孩子就会拿一个板凳,登着板凳上去够东西。

% \textbf{(6)初步辨认物体颜色、形状和特征}:1岁半的孩子能够从图中或实物中辨认物体的形状、颜色和用途。例如,买一些画有各种水果的图画书,反复告诉他长长的香蕉是黄颜色的、圆圆的苹果是红颜色的、大小不等的葡萄有紫色和绿色的,待大人拿出真的水果时,孩子就会知道长长的黄颜色的水果是香蕉,圆圆的红颜色水果是苹果等,这些都是可以吃的东西。

% 经常给孩子看一些各种动物的图画书,告诉他长耳朵、红眼睛、浑身白毛的动物是小白兔;全身黑、眼圈白、胖乎乎的是大熊猫;高个子、长脖子、身穿花衣的是长颈鹿等。狗见人``汪汪''叫,猫见人``咪咪''跑,逐渐孩子就明白了这些动物的特征,一旦看见这些图画或见到这些动物时,他就会一眼认出来,甚至还发出``汪汪''的叫声等。

% \hspace{0pt}\includegraphics[width=2.74126in,height=2.27972in]{media/rId329.png}\hspace{0pt}

% 知道物体的形状也是这个月龄孩子应该具有的能力,如给孩子买一些能够安装三角形、长方形和圆形的积木,让孩子将三角形积木放到三角形木框里,长方形积木放到长方形木框里等,这种游戏能使孩子了解到物体的固有形状,在头脑中形成立体概念。

% %ux4e091921ux4e2aux6708ux5e7cux513fux751fux957fux53d1ux80b2ux72b6ux51b5}{%
% \subsection{三、19\textasciitilde21个月幼儿生长发育状况}%ux4e091921ux4e2aux6708ux5e7cux513fux751fux957fux53d1ux80b2ux72b6ux51b5}}

% 走路很稳当,开始会跑,并能控制跑和走,会用脚尖走几步,一部分孩子会倒退走几步;能扶墙上楼;会把5\textasciitilde8块的积木搭成小塔。

% 知道并说出身体的2\textsubscript{3个部位,如``肚肚、屁股''等,会用``我'',知道``我''和其他人的区别,能够说出2}3个连贯的词或短句子,如``妈妈抱、奶奶走、没有''等,能回答最简单的问题。

% %ux8fd0ux52a8ux53d1ux80b2-14}{%
% \subsubsection{1.运动发育}%ux8fd0ux52a8ux53d1ux80b2-14}}

% \begin{enumerate}
% \def\labelenumi{\arabic{enumi}.}
% \item
%   \textbf{学会跑}:1岁半以后的孩子平衡能力开始逐渐成熟,运动能力也越来越完善,孩子不仅仅满足于迈步和走路,他们需要以更快的速度前进,奔跑对于刚刚会跑的孩子来说是一种身心的良性刺激。在跑步的过程中,周围环境的摇动和变换,令他们感到心情愉悦,如果在游戏中奔跑,还可以使孩子产生竞争意识,刺激脑神经系统发育。孩子在刚刚学会走路时由于掌握不好身体的重心,迈步时双脚不能稳稳落地,因此行走起来步伐显得比走路要快一些,家长往往会误解为孩子是跑步,当他能比较好的控制住自己的身体,掌握好身体重心,能平稳走路后即开始学跑。刚开始跑步时,孩子动作比较僵硬,速度也比较慢,待孩子经过不断地练习,逐渐跑得稳当起来,上下肢协调,速度也逐渐加快。大人带孩子跑的时候,可以有意识地学习转弯,或者围绕障碍物跑,不仅可以让孩子练习大脑和四肢的协调能力,同时也可以增加跑步的乐趣。
% \item
%   \textbf{脚尖行走和倒退行走}:大人可以先进行示范动作,脚跟不着地,只用脚尖行走,或倒退着走几步。大部分孩子看几遍之后就能够学会,但刚开始走的时候多半走不稳,还会经常拌倒,这正说明孩子的平衡功能尚不健全,应该反复练习。经过反复练习后,孩子适应了这种运动形式,掌握了平衡功能,就会自如地用脚尖走路,并可以倒着走。
% \item
%   \textbf{利用扶手和扶墙上楼}:孩子从这个月起有了登高的愿望,看见楼梯就想模仿大人爬上去,可是孩子并不会单腿抬起来,必须要借助于扶手或楼梯旁边的墙壁才能登上楼梯,这也是孩子使用工具的一种能力。孩子最初爬楼梯的时候不会想到去扶墙,只会上下肢着地式的爬行,一个台阶一个台阶爬上楼梯,家长可以在旁边鼓励孩子往上爬,同时提示他用旁边的扶手或墙壁,让孩子感受到用扶手或扶墙登高更容易。这样,孩子就会主动借助于扶手或墙壁爬上楼梯。
% \item
%   用塑料丝穿扣子眼:本月龄的孩子已经能够很好地使用双手协调,做一些日常生活中非常精细的动作了。当孩子用小手捏起细细的塑料丝时,大人应该配合孩子拿起带有小孔或小洞洞的东西,如小纽扣,小塑料管等类似的东西,先给孩子做一下示范动作,让孩子看明白塑料丝是如何穿过小孔的。首先,孩子必须明白塑料丝一定要对准扣眼,而且这个小孔一定要比塑丝绳要大一些才能穿过。\\
%   用塑料丝穿扣眼这项游戏,可以练习孩子的手眼协调功能,手的精细动作能力,以及了解东西的粗细概念等。培养孩子这种能力是需要耐心的,有一些孩子需要反复练习,多次示范才能成功。一旦孩子克服了困难,终于穿进去,他该是多么高兴呀,也许在孩子的内心世界中会充满了喜悦和成功感,而这些成功的喜悦,会增强孩子今后的工作和生活自信心,希望家长帮助孩子多多练习。
% \end{enumerate}

% %ux8bedux8a00ux53d1ux80b2-14}{%
% \subsubsection{2.语言发育}%ux8bedux8a00ux53d1ux80b2-14}}

% \begin{enumerate}
% \def\labelenumi{\arabic{enumi}.}
% \item
%   \textbf{能够说出2\textasciitilde3个连贯的词或短句子}:孩子从这个月开始已经不仅仅使用一个字来表达意思了。他们的大脑更加灵活,语言更加丰富,表达形式也是多种多样。最重要的是孩子可以将名词和动词连接起来,形成一些简单的句子表达意思,如``妈妈抱抱、爸爸上班、奶奶走''等。同时还可以使用一些复合句子,如``宝宝要吃饭、阿姨抱抱我、妈妈不上班''等语句。孩子这个时期主要以简单句为主,复合句的比例占得很小,大部分的词都在5个字以内。孩子表达的语句越完整,大人对此做出的反应就越迅速,使孩子感受到用语言表达给他带来的快乐。因此,在这个时期,孩子说话的积极性大大提高,各种词汇和语言不断从孩子的口中说出,给父母及亲人带来无穷无尽的快乐。
% \item
%   \textbf{会回答最简单的问题}:孩子能够运用语言来表达他们的要求和愿望,对父母提出的问题能够表达愿意或不愿意。例如,妈妈问:``宝宝想睡觉吗?''孩子可以回答
%   ``想''或``不想''。妈妈又问:``宝宝的小车在哪里?''宝宝会指着地上或玩具盒子说:``在那儿。''上午孩子吃完饭以后,爸爸问宝宝:``我们一起出去玩好吗?''宝宝会很高兴地答应,有的宝宝还会把自己的衣服拿来,递给爸爸,示意让爸爸帮忙穿上。这个年龄段的孩子开始出现自我意识,因此无论做什么事情最好要用商量的口吻,征求孩子的意见,千万不要强行命令孩子去做他不喜欢的事情,对孩子的心理造成不良影响。
% \end{enumerate}

% %ux8ba4ux77e5ux548cux751fux6d3bux4ea4ux5f80ux80fdux529b-12}{%
% \subsubsection{3.认知和生活交往能力}%ux8ba4ux77e5ux548cux751fux6d3bux4ea4ux5f80ux80fdux529b-12}}

% \begin{enumerate}
% \def\labelenumi{\arabic{enumi}.}
% \item
%   \textbf{主动模仿成人做事情}:孩子的模仿动作已经从被动转为主动,当孩子与大人一起进餐时,他会看着大人如何用勺子盛饭和菜,他也用同样的动作来盛饭和菜,慢慢地只要孩子一上桌子就会主动拿起饭勺自己吃饭。家长可以有意识地多让孩子自己拿勺吃饭,不要怕他把饭菜撒在地上或只吃进去一半,多练几次孩子就会越来越熟练地使用勺子吃饭了。
% \item
%   \textbf{记忆能力增强}:1岁半以后的孩子记忆能力有了非常明显的进步,孩子能够记住大人做过的动作,能够记住自己的东西放在哪里,如自己的衣服、被褥,自己的喝水杯子等。一些发育比较早的孩子还能够记住奶奶、爷爷、小姨、大姑等亲人的称呼,问他:``大姑在哪?''孩子会迅速地用手指向大姑,而问他``那小姨呢?''孩子立即把手指向小姨。\\
%   记忆能力的增强还表现在语言方面,本月龄的孩子能够记住30个以上的词汇,而且还能说出3
%   \textasciitilde{}
%   5个单词组成的句子。一些接受早期教育的婴幼儿可能还会记住更多的句子,甚至能够辨认一些常见汉字。
% \item
%   \textbf{自我意识}:婴儿期孩子是没有自我意识概念的。1岁以后,随着孩子的活动量增加,他们开始认识到自己使用的东西与父母是不一样的,自己的身体与大人的身体也有区别。当孩子具有一定的语言表达能力和对事物理解能力时,他们就意识到自己与其他人是不一样的。父母应该利用一切机会尽可能教育孩子,让他们建立起最初级的自我意识。
% \end{enumerate}

% %ux5bf9ux5916ux754cux4e8bux7269ux7684ux597dux5947ux5fc3ux548cux5174ux8da3}{%
% \subsubsection{4.
% 对外界事物的好奇心和兴趣}%ux5bf9ux5916ux754cux4e8bux7269ux7684ux597dux5947ux5fc3ux548cux5174ux8da3}}

% 5\textasciitilde2岁之间,孩子的思维有了突飞猛进的发展,家长一定要利用各种各样的机会对孩子讲解周围的事物。上街看见公交车、小汽车、自行车时,要告诉他这些车的区别;到超市看见家中没有的物品时,要告诉孩子这种东西的用途,让他知道除了家中的东西以外,还有许多他不知道的东西。有时候还要给孩子做示范,做动作,让孩子了解东西的用途。【疑问,区间可能是2-5岁】

% 这个年龄段的孩子由于对东西的性能不能掌握和理解,因此经常会摔坏东西。同时孩子不知道危险,很容易发生意外。因此,家长一定要将孩子放在安全的地方,危险物品要放到孩子够不到的地方,防止孩子发生意外。

% %ux56db2224ux4e2aux6708ux5e7cux513fux751fux957fux53d1ux80b2ux72b6ux51b5}{%
% \subsection{四、22\textasciitilde24个月幼儿生长发育状况}%ux56db2224ux4e2aux6708ux5e7cux513fux751fux957fux53d1ux80b2ux72b6ux51b5}}

% 孩子2岁了,从一个一无所知的婴儿,成长为一个活蹦乱跳的,会说会唱的活泼可爱的小大人。孩子的跑跳能力又有了很大的进步,可以连续跑出3〜4米的距离;一些孩子会用双腿蹦(双脚同时离开地面)。上下楼梯时可以用一只手扶着。

% 能够说出一些比较连贯的词语,如念成语,会念几首儿歌等,并能够回答一些生活上比较简单的问题。

% 能数出简单的数,能够按照顺序一页一页地翻书,能在大人提示下做某件事情,知道生活中常见东西的用途,听大人讲故事后能够理解大概意思,并能讲出人物和发生的事情。

% %ux8fd0ux52a8ux53d1ux80b2-15}{%
% \subsubsection{1.运动发育}%ux8fd0ux52a8ux53d1ux80b2-15}}

% \begin{enumerate}
% \def\labelenumi{\arabic{enumi}.}
% \item
%   \textbf{跑步}:孩子2岁时已经能够跑出很长一段距离,这时孩子已经不满足于在屋子里跑,他非常愿意到外面去跑,愿意围着障碍物跑,愿意和小朋友互相追逐着奔跑。其实跑跳是孩子的天性,尤其是一些喜欢运动的孩子更喜欢跑跑跳跳,他觉得这是玩耍,是娱乐。家长应该尽可能不要限制孩子的跑跳自由,可以让孩子在小花园中奔跑蹦跳,自由玩耍,充分发挥孩子的运动才能,培养他们各种各样的运动能力。
% \item
%   双腿蹦:某一天家长突然发现自己的宝宝能够用双腿同时离开地面蹦起来。这种动作是孩子经过很多天的行走和跑跳,双下肢反复练习逐渐发展的成效,如果出现了这种动作,说明孩子已经掌握了身体重心,具备很好的平衡功能。双足抬起的能力对于每一个孩子来讲,发展速度是不一样的。有的孩子2岁就可以完成双腿蹦的动作,而有的孩子到2岁半还不能完成双足抬起。这只是运动发育的差异,千万不要着急。有一些孩子胆子比较小,害怕摔倒,因而不敢同时抬起双脚,可以鼓励孩子多做跑跳运动,帮助孩子有目的的练习,如学习小兔蹦跳等游戏。双腿蹦一定要以离开地面为准。\\
%   \hspace{0pt}\includegraphics[width=2.20979in,height=2.83916in]{media/rId349.png}\hspace{0pt}
% \item
%   \textbf{一手扶楼梯扶手自己上下楼梯}:如果家中有楼梯,孩子非常热衷于上下楼梯玩耍。这是因为孩子可以用一只手扶着楼梯扶手自己上下楼梯。上下楼梯不仅使孩子的高度感觉发生变化,还会使孩子的视觉产生变化。让孩子爬楼梯其实也是一种运动游戏,它可以训练孩子的胆量,培养他们的毅力和信心,当孩子站在楼梯的最高点时,他的心情是多么愉快,说不定孩子还会有一种成就感。爬楼梯对于刚刚满2岁的孩子来说还需要有一个训练的过程,开始时可以先少爬几层,待孩子熟练了自己爬楼梯之后再让他多爬几层,逐渐爬完整层楼梯。下楼梯时一定要扶好孩子以免摔倒,造成身体伤害。\\
%   \hspace{0pt}\includegraphics[width=2.39161in,height=2.55944in]{media/rId353.png}\hspace{0pt}
% \end{enumerate}

% %ux8bedux8a00ux53d1ux80b2-15}{%
% \subsubsection{2.语言发育}%ux8bedux8a00ux53d1ux80b2-15}}

% 2岁孩子在语言方面有了很大的进步。有的孩子甚至能说出几句成语,令大人惊讶万分。这个年龄的孩子能够说出很完整的简单句式。当看见妈妈上班,孩子会说``妈妈上班'';想要喝奶时,孩子会说``我要喝奶'';想要玩球时,孩子会说:``我要玩球''等。但是有一些孩子的发音不是很清楚,要随时注意纠正孩子的发音,尽量让孩子吐字清晰。如何培养孩子的语言能力尤为重要。可以采取以下几方面措施来加强对孩子的语言训练。

% \begin{enumerate}
% \def\labelenumi{\arabic{enumi}.}
% \item
%   \textbf{创造良好的语言环境}:2岁孩子生活环境主要是在家中和出去与外人的玩耍。不管任何时候,父母和孩子的亲人应该多注意与孩子说话,尽量创造出良好的语言环境,让孩子有机会听到各种各样的语言和说话内容。
% \item
%   \textbf{丰富孩子的生活内容}:有人说现在的孩子越来越聪明,这说明人类的生长发育是和社会进步同时发展的。孩子从电视、收音机里可以听到各种各样丰富的语言,获取大量的信息,这对孩子是至关重要的语言训练手段,从小反复听到过的语言,长大以后会很容易记住。家长不要只是忙于自己的工作,应该多利用业余时间经常带孩子出去转一转,多看、多听、多认识,新奇的事物一定会激发孩子要表达的愿望,这样可以促使他们多多讲话。通过讲述这些激动人心的经历,还能使孩子体会到谈话的乐趣。父母和家庭中其他的成员,或者是老师,都应该成为孩子讲话的听众,耐心细致地听孩子讲述。千万不要流露出不耐烦的情绪,这样会影响孩子的兴致,成人的良好态度是鼓励孩子讲述和表达的重要因素。\\
%   \hspace{0pt}\includegraphics[width=2.81119in,height=2.54545in]{media/rId359.png}\hspace{0pt}
% \item
%   \textbf{选择合适的口语训练方式}:训练孩子的口语方式有很多,如看图说话、讲故事、说儿歌等。\\
%   ①
%   看图说话。看图的内容不要过多,最好以单篇幅为好。孩子的观察和理解能力比较差,因此应讲一些比较简单的内容,如``这是什么'',``小熊在跳舞'',``小朋友在歌唱''等。当孩子已经能够理解上述内容之后,还可以进一步讲解稍微多的内容,如``小熊在和谁跳舞,旁边有什么人在伴奏'',``小朋友唱的是什么歌,复述歌中的内容''等。\\
%   ②
%   说儿歌。2岁孩子说的儿歌应该是非常短小、顺口,内容活泼有趣,而且一定要形象和个性鲜明。比如,``小白兔白又白,两只耳朵竖起来\ldots\ldots''等。\\
%   ③
%   讲故事。经常给孩子讲一些他比较熟悉的人物和动物的故事,可以以小熊、小猫、小白兔等动物为例,如果同时还有一些鲜艳的图画和丰富的表情、手势等就更好了。
% \item
%   \textbf{唱歌}:2岁时发出的语音语调还不是很清楚,尚不能识别各种音调,因此孩子唱起歌来往往会跑调,唱出的歌词也基本听不懂,但这没有关系,只要孩子想唱就让他唱,千万不要笑话孩子,扼杀孩子学习音乐的兴趣,要经常和他一起唱一些比较简单的儿歌,训练孩子的语音语调。\\
%   \hspace{0pt}\includegraphics[width=2.72727in,height=2.68531in]{media/rId364.png}\hspace{0pt}
% \item
%   \textbf{叙述和表达}:从这个年龄段开始应该注意培养孩子的语言表达和叙述能力。可以买一些简单的图画书,结合书中的图画给孩子讲述一些情节比较简单的小故事,一方面让孩子理解图画的意思,另一方面有意识地让孩子复述故事的人名、事件和结果等。这种能力的培养对孩子以后语言表达极为重要,家长一定要耐心,不厌其烦地反复教孩子说和表达,经过多次的练习,孩子就能够用很正规的简单句子来表达故事中的情节和人物了。
% \end{enumerate}

% %ux8ba4ux77e5ux548cux751fux6d3bux4ea4ux5f80ux80fdux529b-13}{%
% \subsubsection{3.认知和生活交往能力}%ux8ba4ux77e5ux548cux751fux6d3bux4ea4ux5f80ux80fdux529b-13}}

% \begin{enumerate}
% \def\labelenumi{\arabic{enumi}.}
% \item
%   \textbf{能数出简单的数}:2岁左右的孩子开始能够知道``1''个苹果、``1''个足球,同时他们还能听懂和运用数词``1''和``2、3''。可以从自己脸上找出``1''个鼻子、``1''个嘴巴、``2''只耳朵、``2''只眼睛等,能数出1\textasciitilde5个数字。
% \item
%   \textbf{和大人一起做事}:从这个年龄段开始应该有意识地培养孩子与大人共同做事的能力。要千方百计地鼓励孩子参与家中的各种活动,让孩子享受到家人共同劳动的愉快,这也是培养孩子积极参与社会交往,独立做事,努力为他人服务的意识。
% \item
%   \textbf{与同龄儿一起玩耍}:家长要经常带孩子出去玩耍,还应该让孩子去参加一些托幼机构办的小小班,尽量提供孩子与同龄儿童玩耍的机会,可以让孩子带一些他认为是比较好玩的玩具,鼓励孩子给别的小朋友玩,或者与别人一起玩,学会与他人相处,与人友好合作的愿望和意念。
% \end{enumerate}

% %ux4e942527ux4e2aux6708ux5e7cux513fux751fux957fux53d1ux80b2ux72b6ux51b5}{%
% \subsection{五、25\textasciitilde27个月幼儿生长发育状况}%ux4e942527ux4e2aux6708ux5e7cux513fux751fux957fux53d1ux80b2ux72b6ux51b5}}

% 自己能够不扶楼梯扶手上楼或者下楼,双腿蹦跳稳当,能够单腿站立数秒钟,能够自己迈过障碍物,能够控制活动方向,会玩不同形状的积木,摆出各种形状。

% 会念几首儿歌,会说一些比较简短的句子,会用几个形容词,会问``这是什么'',掌握常用词汇200
% \textasciitilde{} 300个。

% 知道``大''和``小''的概念,知道物体的形状和颜色,有简单的是非观念,能分辨``好''与``不好''的事情,会自己吃饭,会自己脱外衣和鞋袜,会自己解大小便,愿意与小朋友玩游戏。

% %ux8fd0ux52a8ux53d1ux80b2-16}{%
% \subsubsection{1.运动发育}%ux8fd0ux52a8ux53d1ux80b2-16}}

% \begin{enumerate}
% \def\labelenumi{\arabic{enumi}.}
% \item
%   \textbf{独自上下楼}:孩子由于胆怯往往总是用一只手先去扶墙,然后再上楼梯。先鼓励孩子不扶墙上楼,也可以稍加帮助,让孩子尽量不扶着墙上楼,一旦独自上楼成功,孩子胆子就大了,下一次孩子就敢自己上楼了。当孩子会自己上楼之后,开始帮助孩子自己下楼,开始大人先拉着孩子从楼梯上下来,到最下面1阶时,松开手,让孩子自己下,然后改为让孩子自己下最下面的2阶、3阶,直至4阶等,可以训练孩子两足交替使用下楼梯。上下楼梯是锻炼孩子下肢肌肉和增强平衡能力的好办法,有条件时可以反复训练。
% \item
%   \textbf{单腿站立}:这项运动也是训练孩子平衡功能的方法。大人与孩子面对面站着,大人先抬起一只脚,让孩子明白即使用一只脚也能站得稳,然后鼓励孩子抬起一条腿,一只脚离开地面就可以了,单腿站立时孩子上身可能会摇晃,大人可以扶住孩子的身体,然后轻轻放开,让孩子自己独自单腿站立数秒钟,然后告诉孩子轻轻放下脚,反复做上述动作,如果孩子身体重心掌握得比较好,就会稳稳地单腿站立,如果身体重心掌握不好或者平衡功能差,那么孩子往往不会用单腿站。因此,反复训练此项运动对孩子的平衡功能非常有好处。\\
%   \hspace{0pt}\includegraphics[width=2.37762in,height=2.64336in]{media/rId376.png}\hspace{0pt}
% \item
%   自己迈过障碍物:这是培养孩子大脑判断能力和四肢协调能力的非常好的一项运动。可以在家中练习,也可以在户外练习。开始设置的障碍物高度要低一些,在地上画一条线,放置一根绳,或者一根棍等,让孩子迈过去,然后逐渐增加高度,可以放一只小板凳,或者两只板凳中间放一块板,让孩子迈过去,比较高的障碍物可以用纸盒来代替,如果孩子迈不过去,也不至于有危险。练习时可以先协助孩子,先拉孩子的一只手帮助孩子迈过障碍物,然后再让孩子自己迈。一般这个月龄的孩子都能迈过去,但也有一部分孩子迈不过去,这并不能说孩子运动能力差,大多数的原因是因为孩子胆小,个别是因为孩子的平衡功能不完善,或者缺乏锻炼所致,只要多锻炼、多做,孩子最终都会按要求做到。\\
%   \hspace{0pt}\includegraphics[width=2.23776in,height=2.74126in]{media/rId380.png}\hspace{0pt}
% \item
%   会搭各种形状的积木:会搭各种形状的积木:这个月龄的孩子不仅仅满足于只把积木叠搭起来,有的孩子能够凭借对多种物体的感观认识,搭出各式各样的东西,如小房子,还可以搭出一列小火车等。刚开始搭积木时需要大人协助,首先和孩子商量说:``我们搭积木好吗?''孩子如果高兴或感兴趣是最好的时机,可给孩子提个建议:``我们搭一列火车可以吗?''一边说一边让孩子搭,尽量让孩子自己搭建出一种他认为是火车样的东西,如果搭得不像,也不要打断他,让孩子自己搭完。当孩子停下来看大人的时候,大人可以问他:``你搭得是火车吗?''如果搭得很像,他自己就会说:``这是火车。''如果搭得不像,孩子会犹豫不决,或者干脆推倒重来,这时大人一定要鼓励孩子重新树立信心,让孩子按照自己的想象力搭出来,搭完之后,要用赞美的语言鼓励孩子。有时候孩子搭出的东西大人看不出来,这时可以问孩子:``你搭的是什么呀?''孩子可能一时说不上来,帮助他想一个名称,让孩子凭借大脑的记忆和联想对号。这种游戏不仅训练孩子的动手能力,同时也培养了孩子的想象力。
% \end{enumerate}

% %ux8bedux8a00ux53d1ux80b2-16}{%
% \subsubsection{2.语言发育}%ux8bedux8a00ux53d1ux80b2-16}}

% \begin{enumerate}
% \def\labelenumi{\arabic{enumi}.}
% \item
%   会背诵儿歌:前面已经反复多次提到关于儿歌的项目,实际上大部分的孩子真正能够很完整地背诵儿歌,基本都是在2-3岁的这个年龄段中。因此,家长可以有意识地教给孩子背诵一些好听、易懂、易学的儿歌,可以挑选一些健康、活泼、形象的儿歌让孩子反复背诵。有一些家长喜欢让孩子背诵唐诗,结果背了半天,孩子根本不知道是什么意思,而且2\textasciitilde3岁的孩子长期记忆能力差,好不容易记住的几首唐诗过一段时间再一问,大多数都忘记了。所以说这个年龄背一些比较形象的顺口的儿歌更好。现举例如下:``大公鸡,真美丽,红红的鸡冠,花花衣,清早起,喔喔啼,告诉宝宝早早起。''
%   ``我的小手帕,真呀真美丽,天天带着它,擦嘴擦鼻涕。''\\
%   \hspace{0pt}\includegraphics[width=2.18182in,height=2.26573in]{media/rId386.png}\hspace{0pt}
% \item
%   会用形容词:孩子的语言发育非常快,不仅能够正确使用名词和动词,而且开始使用形容词来表达自己对某件事情的看法,如看见妈妈换了一件新衣服,他会说:``妈妈好漂亮。''到马路上看见小汽车,孩子会说:``小车跑得真快。''看见红苹果时,他会说:``我要吃大红苹果。''大人在平时说话时,应该多注意加用一些形容词类的语言,让孩子慢慢体会其中的意思,用过几次,孩子就会使用这些词汇了。
% \end{enumerate}

% %ux8ba4ux77e5ux548cux751fux6d3bux4ea4ux5f80ux80fdux529b-14}{%
% \subsubsection{3.认知和生活交往能力}%ux8ba4ux77e5ux548cux751fux6d3bux4ea4ux5f80ux80fdux529b-14}}

% \begin{enumerate}
% \def\labelenumi{\arabic{enumi}.}
% \item
%   知道``大''和``小''的概念:孩子能够分辨``大''和``小'',说明已经有了对比和比较的思维。孩子在日常生活中经常会碰到大小不同的食物、用品及玩具等,比如对孩子说,``爸爸用大碗,宝宝用小碗'',``妈妈用大杯喝水,宝宝用小杯喝水''。坐车时告诉宝宝,``大卡车大,小汽车小''等,使宝宝对大和小有一个初步的印象。
% \item
%   准确辨认形状和颜色:孩子到了这个年龄一般可以辨认不同形状的物体了,如方形、圆形等。家长可以制作一些颜色鲜艳的圆形纸板和方形纸板,告诉孩子这是圆形的,或这是方形的,等孩子能够辨认圆形和方形之后,可以教孩子将两个方形纸板摞在一起,圆形纸板摞在一起,这样反复训练,直到孩子能够自己将同样形状的纸板摞在一起。孩子能说出两种不同形状的名称,并能正确匹配两种不同形状的纸板。
% \item
%   有简单的是非观念:孩子已经对日常生活中一些人和事物有了一个初步的了解,知道``好''与``不好''的事情,这些是非观念对孩子今后的人生成长非常重要,一定要让孩子懂得正确和错误的事情。父母教育孩子做事情时,一定要用自己的行动去说明自己的态度,鼓励孩子去做正确的事情,制止孩子去做错误的事情,如果孩子不明白时,应该耐心说服,千万不要训斥或者打骂孩子,引起孩子的反感,产生逆反心理,影响孩子将来的发展。
% \item
%   自己做事情:一般这个年龄段的孩子大多会自己做很多事情了,如自己吃饭,自己穿外衣和鞋袜,自己解大小便等,但是孩子的动作往往很慢,这时性急的家长就会拿起饭碗喂给孩子吃;孩子穿衣很慢,家长怕孩子感冒,于是代替孩子穿衣裤和袜子,这些都影响了孩子自己动手的能力。孩子这些生活自理能力的培养是一个循序渐进的过程,需要一个磨练的过程,应该尽量让孩子自己做自己的事情,从小培养孩子独立生活能力,这对于孩子将来的学习和工作是很有帮助的。
% \end{enumerate}

% %ux516d2830ux4e2aux6708ux5e7cux513fux751fux957fux53d1ux80b2ux72b6ux51b5}{%
% \subsection{六、28\textasciitilde30个月幼儿生长发育状况}%ux516d2830ux4e2aux6708ux5e7cux513fux751fux957fux53d1ux80b2ux72b6ux51b5}}

% 2岁6个月的孩子跑跳自如,还能参加许多与跑跳有关的活动,如踢球、跳远等运动,会举起手臂做投掷动作,从楼梯底层跳下,可以模仿大人做``跳舞''动作,会用塑料丝穿扣眼。会画横线和竖线。

% 会用人称代词``你、你们、他、他们''等,会用连接词``和、跟''等,能说出比较复杂的语句。

% 知道``少和多'',``长和短'',``上和下''等抽象概念:会脱衣服,解衣扣,知道数字``1\textasciitilde10'',``1''和``许多''的意思,自己会洗手,会折纸。

% %ux8fd0ux52a8ux53d1ux80b2-17}{%
% \subsubsection{1. 运动发育}%ux8fd0ux52a8ux53d1ux80b2-17}}

% \begin{enumerate}
% \def\labelenumi{\arabic{enumi}.}
% \item
%   踢球和投球:大部分的孩子从婴儿期就开始对球感兴趣,因为球是可以滚动的,看着满地滚的球,孩子会觉得新鲜和不可思议,看着大人拿球抛来抛去,或者球在脚下滚来滚去的样子,孩子的脑海中充满了刺激和乐趣。利用孩子的好奇心和新鲜感,通过玩球促进孩子的运动发育是再好不过的活动了。踢球是一项孩子比较易于接受的玩耍形式。刚开始孩子踢球时并不要求必须有方向,只要能将球踢出去就行,大人与孩子面对面站着,先示范踢球动作,然后让孩子伸出脚,慢慢做踢球动作,当孩子能踢出去以后,再告诉他把球踢到某一个方向。训练投球动作也可以这样,与孩子面对面站着,把球扔给孩子,再让孩子把球扔回来,来回多次,孩子就会有目的、有方向地投球了。\\
%   \hspace{0pt}\includegraphics[width=2.1958in,height=2.58741in]{media/rId398.png}\hspace{0pt}
% \item
%   \textbf{立定跳远}:这项运动是孩子必须会双腿离地蹦起来之后才能学会的运动项目。其实蹦出多远并没有多大意义,目的主要是训练孩子双腿协调的能力。
% \item
%   \textbf{熟练地用塑料丝穿扣眼}:孩子到1岁6个月以后就应该训练用塑料丝穿扣眼,但是真正穿得很熟练、很迅速的年龄一般都在2岁6个月左右。可以与孩子这样做游戏,找两根颜色鲜艳的塑料丝和十几个小扣子,对孩子说:``我们比赛吧,看谁穿得多。''大人有意识地稍微慢一些,使大人穿进的数目与孩子穿进的数目差不多,然后说:``宝宝穿进的扣子真多,都快赶上妈妈了,再穿一次好吗?''当看到孩子穿过去的数目多于妈妈时,妈妈要及时鼓励孩子:``宝宝真棒,都超过妈妈了。''这项游戏对训练孩子十指的灵活性是有帮助的。
% \item
%   \textbf{模仿和自己画横道、竖道}:画横道和竖道是训练孩子精细动作的有效方法。孩子2岁以后能很好地拿笔了,但是尚不能准确画出横平竖直的道道来,需要反复训练才能学会此项较为复杂的精细动作。
% \end{enumerate}

% %ux8bedux8a00ux53d1ux80b2-17}{%
% \subsubsection{2.语言发育}%ux8bedux8a00ux53d1ux80b2-17}}

% \begin{enumerate}
% \def\labelenumi{\arabic{enumi}.}
% \item
%   学用代名词:孩子已经能够区别自己和其他人的关系,作为这个年龄段的孩子来讲,准确理解``我、你、他'',``我们、你们、他们''这些概念是非常重要的。2岁多的孩子常常把这些代名词弄混,日常生活中家长应该多注意训练。
% \item
%   会问``这是什么'':孩子的好奇心很强,想象力也极为丰富,遇到不明白的问题总想问一问,因此教会孩子提问题是帮助孩子满足好奇心的首要方法。
% \end{enumerate}

% %ux8ba4ux77e5ux548cux751fux6d3bux4ea4ux5f80ux80fdux529b-15}{%
% \subsubsection{3.认知和生活交往能力}%ux8ba4ux77e5ux548cux751fux6d3bux4ea4ux5f80ux80fdux529b-15}}

% \begin{enumerate}
% \def\labelenumi{\arabic{enumi}.}
% \item
%   \textbf{知道``多、少'',``长、短'',``上、下''等概念}:在日常生活中可以设置一些有对应关系的事物或环境,让孩子理解和接受这些概念。
% \item
%   \textbf{懂得``1''和``许多''}。
% \item
%   \textbf{能数1\textasciitilde10}:利用生活中的一切机会告诉孩子这些数字代表的意思,还可以用一些表示数字的图片等加深孩子对数字的印象。
% \item
%   \textbf{懂得表扬与批评}:孩子到了这个年龄段已经能够清楚地表达自己的意愿了,做完事情之后,如果得到大人的表扬,他会表现得兴高采烈,如果事情没有做好,这时如果大人不高兴批评了几句,孩子顿时情绪低落,甚至哭闹不安。
% \item
%   与小伙伴争夺玩具:当孩子自我意识开始萌芽时,他首先表现出玩具不愿意给别人,和别人一块玩,也不愿意让别人拿着,只能自己拿着。其实这并不是一件坏事,这说明孩子已经知道自己和别人是有区别的,自己的东西应该自己拿着。当发生这种情况时,家长不要着急,可以用一些孩子能接受的动作和语言来告诉孩子说:``宝宝的玩具可以借给XX吗?他是妈妈的好朋友,也是你的好朋友嘛。''
%   ``把积木借给XX玩,好吗?他会帮你搭一列长长的火车,你们一块玩''等等,一旦孩子与别人玩了几次感到很愉快,下一次他就会主动找别的小朋友去玩耍了。但有一些孩子当他的个人欲望开始发展时,他会去抢别的小朋友的玩具,甚至打人,这些都是值得注意的问题,要及时教育孩子应该与他人友好相处,可以把自己的玩具拿来与他人交换,或者把自己的玩具混到他人的玩具中间一起玩耍,目的就是要培养孩子与他人友好合作的精神,避免产生以自我为中心及自私自利的倾向性。\\
%   \hspace{0pt}\includegraphics[width=2.78322in,height=2.44755in]{media/rId413.png}\hspace{0pt}
% \end{enumerate}

% %ux4e033136ux4e2aux6708ux5e7cux513fux751fux957fux53d1ux80b2ux72b6ux51b5}{%
% \subsection{七、31\textasciitilde36个月幼儿生长发育状况}%ux4e033136ux4e2aux6708ux5e7cux513fux751fux957fux53d1ux80b2ux72b6ux51b5}}

% 孩子快3岁了,无论从动作方面还是语言方面都感觉像个``小大人''了。这时孩子已经能够行走、跑跳自如,还能做一些比较复杂的动作。

% 能够说出复杂句,会背诵十首以上的儿歌,掌握词汇1000个左右,主要以口语为主;能够说出大部分图画中的东西和用品名称。

% 明白性别,懂得``里、外'',可以准确地用勺子吃饭不撒,喝水不洒,饭前自己洗手并擦干,自己擦鼻涕,会穿衣解扣,会折纸,会玩``过家家''游戏等。对外界环境有表示,如有冷、热、渴、饿、累、烦等感觉。

% %ux8fd0ux52a8ux53d1ux80b2-18}{%
% \subsubsection{1.运动发育}%ux8fd0ux52a8ux53d1ux80b2-18}}

% \begin{enumerate}
% \def\labelenumi{\arabic{enumi}.}
% \item
%   \textbf{双脚交替上下楼}:接近3岁的孩子双下肢的协调能力还不是特别完善,因此应该注意锻炼孩子的双脚交替动作。可以先拉着孩子的一只手,让孩子轮流抬起左脚或者右脚,然后让孩子先用一只脚登上第一层阶梯,拉着孩子的手再让孩子用另一只脚登上第二层阶梯,轮流反复多次,让孩子体会到用双脚交替上楼既快又轻松;下楼时也用同样的方法训练孩子。当孩子可以很灵活地用双脚交替上下楼时,说明孩子四肢已经具有很好的协调能力了。
% \item
%   \textbf{按口令做操}:主要让孩子模仿大人的动作,如两胳臂向上伸展,向下抬起,下肢踢腿、伸展,全身转动及蹦跳运动等都是非常好的身体活动项目,让孩子跟着模仿,不仅培养孩子迅速反应能力,同时还能培养孩子的身体协调功能。
% \item
%   \textbf{画圆圈}:画圆圈的难度比画横道和竖道要大得多,孩子学画圆圈是要有一个过程的。首先要让孩子学会模仿画圆,家长先画一圆形,让孩子照着圆圈描画一遍,然后再让孩子自己画。刚开始画圆时往往画不成形,画出个多角形,或者不知道如何收拢口。这时可以协助孩子把画出的多角形改为圆形,或者帮助封上口,给孩子做出示范。当孩子画圆时,随时要提醒孩子注意画出圆弧线等,也许孩子一笔画不完,要耐心等他分几笔画完。等孩子画完之后,拿一个事先裁剪好的圆纸片让孩子比一比,用手在圆纸片周围画几圈,体会圆的感觉。这时孩子就会感觉出他画的圆圈是否是圆的,下一次再画圆的时候,他一定会注意画线的方向,尽量画出圆圈来。\\
%   \hspace{0pt}\includegraphics[width=2.75524in,height=2.92308in]{media/rId421.png}\hspace{0pt}
% \item
%   \textbf{自己插(搭)积木}:到了3岁的年龄,几乎都能自己玩一些能插和能搭出形状的积木了,如市场上卖的各种塑料插积木,能够插出许多各类小房子、各种小汽车的形状。一定想方设法让孩子插出一个成形的东西,使孩子有一种成就感,这样会大大激发孩子的想象力和创造力,同时还能锻炼孩子的毅力,使孩子树立自信心。
% \item
%   学习折纸:折纸游戏主要是训练孩子手指的灵活度,同时还可以激发孩子的想象力和创造力。先训练孩子折长方形或四方形,大人与孩子各拿一张正方形或长方形的彩纸,大人可以先将纸对折一次,让孩子跟着模仿,如果孩子一次能模仿成功,就可以让孩子自己继续练习折纸,如果孩子不会,大人要手把手教孩子对折一次或几次,直到孩子自己能够对折为止。开始对折时不要求折得很准确,只要把纸折一下就可以了,下一步练习将纸压平,以后慢慢提高要求,达到能够将纸对齐、折好、压平为准,以后再教孩子折两折、三折等。孩子会折方形后,开始训练孩子折三角形及其他形状等。
% \item
%   骑小三轮车:骑车是一项练习全身协调功能的运动游戏,这里包括手和眼的协调,上下肢与身体的平衡等多种能力。孩子刚开始骑车的时候,可能不会控制身体的平衡,需要家长的帮助。可以帮孩子扶着小三轮车,让孩子学会双足蹬踏动作,然后告诉孩子双手把好前扶手,眼睛看着前方,大人边推边鼓励孩子向前骑。有一些孩子胆子小,不敢用力蹬,大人可以先推他走一圈,一边推一边让孩子用双足做蹬踏动作,让孩子体会骑车并不是一件很害怕的事情,反而会带来很多乐趣。以后逐渐训练孩子骑车拐弯、躲过障碍物等随机应变能力。\\
%   \hspace{0pt}\includegraphics[width=2.39161in,height=2.88112in]{media/rId427.png}\hspace{0pt}
% \end{enumerate}

% %ux8bedux8a00ux53d1ux80b2-18}{%
% \subsubsection{2.语言发育}%ux8bedux8a00ux53d1ux80b2-18}}

% \begin{enumerate}
% \def\labelenumi{\arabic{enumi}.}
% \item
%   \textbf{会说复杂语句}:孩子接近3岁时词汇量大增,已经初步掌握了日常生活中的口头用语,词汇近1000个左右。除了常用的名词、动词以外,还有副词、形容词、数词、代名词、连词等各类词汇。孩子说的句子一般比较短,以5\textasciitilde10个词为多,如``妈妈带我上公园''、``我要玩积木''、``妈妈为什么还不回来''等语句,口语化比较强,可以同成人进行最基本的语言交流,一般以对话语言为主。
% \item
%   \textbf{会背多首儿歌}:这个年龄段的儿歌一般多以四句为多,每句的字数通常在4\textasciitilde8个,孩子背诵的能力比前几个月有明显的进步,经常练习背诵的孩子甚至能背诵十几首唐诗。在这里要提醒家长的是孩子背诵多少儿歌是不重要的,重要的是孩子背诵时一定要发音清楚,有一些孩子语言能力相对差一些,发音比较含糊,节奏感也不强,勉强让孩子背下来的儿歌很快就会忘掉。如果遇到这种情况,家长可以找一些短小简单的儿歌让孩子学习跟读,重点要纠正发音,同时还要告诉孩子这些句子的意思,让孩子既学习了发音,又学习了知识,一举两得,背诵一首儿歌就应该让孩子有所收获。这样做比盲目追求孩子背诵儿歌数量所取得的效果要大得多。有时候在与孩子做游戏时,边玩边教孩子说儿歌也常常会有较好的效果。
% \item
%   \textbf{教孩子提问题}:开始提问题是孩子智力进步的又一表现,这说明孩子已经能够用大脑思考问题。家长可利用日常生活中的一些习惯提问题,如:``为什么早上起来要刷牙?''
%   ``为什么饭前便后要洗手?''利用与孩子到户外玩耍的机会给孩子提问题,如:``为什么有的动物会飞,有的动物不会飞?''\,``为什么春天小树会发芽?''给孩子讲解图画中的内容启发孩子提问题,如讲解老狼和小兔的故事时,启发孩子思考:``为什么小兔妈妈不让小兔给老狼开门?''
%   ``夜里兔妈妈不在时小兔们应该怎么办?''等等。告诉孩子如果有不明白的问题时可以提出``为什么''或者``怎么办'',逐渐培养孩子爱提问题的习惯,这对孩子今后学习是非常有帮助的。
% \end{enumerate}

% %ux8ba4ux77e5ux548cux751fux6d3bux4ea4ux5f80ux80fdux529b-16}{%
% \subsubsection{3.认知和生活交往能力}%ux8ba4ux77e5ux548cux751fux6d3bux4ea4ux5f80ux80fdux529b-16}}

% \begin{enumerate}
% \def\labelenumi{\arabic{enumi}.}
% \item
%   \textbf{辨认性别}:孩子对性别的认识是朦胧的,家长可以通过妈妈、爸爸及奶奶、爷爷等不同的性别特征来告诉孩子他是男孩还是女孩,也可以与邻居家的小孩作对比,从外部的特征来了解男孩和女孩的不同。3岁左右的孩子大多能了解性别的不同,这对于孩子今后的身心发育是非常重要的。
% \end{enumerate}

% \begin{itemize}
% \item
%   \hspace{0pt}\includegraphics[width=2.81119in,height=2.78322in]{media/rId436.png}\hspace{0pt}
% \end{itemize}

% \begin{enumerate}
% \def\labelenumi{\arabic{enumi}.}
% \setcounter{enumi}{1}
% \item
%   \textbf{语言交往意识}:孩子已经会说很多话了,这时要注意和其他人的交往,遇见和妈妈爸爸说话的大人时一定要告诉孩子喊``阿姨、叔叔、奶奶、爷爷''等,打招呼时要说``你好'',离开时要说``再见''。养成习惯以后孩子遇到认识的大人就会主动打招呼,和小朋友见面时也要互相称呼问好等,使孩子从小形成讲文明和礼貌待人的良好习惯。
% \item
%   \textbf{小群体交往活动}:1\textasciitilde2岁的孩子基本上是以个体活动为主,虽然他们有想与其他小伙伴交往的意识,但因为语言和肢体活动的限制,大多是自己玩为主。当接近3岁时,孩子语言开始丰富,活动范围也日趋广泛,这时孩子往往不满足于自己玩耍了,需要和其他人共同进行活动,丰富玩耍内容。小群体活动通常以两三个人在一起玩耍为主。这种交往活动在游戏活动中尤为明显,如当医生给病人看病,当司机开车拉乘客等,均为一对一的交往活动,或者是两三个人的交往。尽管人数不多,但却是群体活动的萌芽状态,它是孩子以后参加多群体活动的基础。因此,家长要尽量创造一些让孩子与其他小朋友交往玩游戏的机会,培养孩子的群体适应性,为今后孩子的社会交往打下良好的基础。\\
%   \hspace{0pt}\includegraphics[width=3.52448in,height=2.33566in]{media/rId441.png}\hspace{0pt}
% \item
%   自我意识的发展:接近3岁的孩子不仅知道有一个``自我''(即自己身体独立存在),而且开始明白``自我''在社会群体中的存在,认识到自己是存在于社会中的一个人,``我''和别人是不同的。随着年龄的增长,孩子的动作和活动能力,语言和社会交往能力增强,孩子逐步意识到自己的能力和价值,并产生一种强烈的要求独立行动的愿望,通过语言或行动常常表示出``我要、我想\ldots\ldots''等,他要自己吃饭,自己穿衣,自己拿东西,自己出去等。但是,由于能力和经验有限,时常事与愿违,做不好自己想要做的事情。即使这样,家长也应看到,这是一种积极向上的要强行为,是一种独立性、主动性的表现,应该给予鼓励。
% \item
%   具备生活自理能力:2\textasciitilde3岁是孩子独立性开始形成和发展的时期。如果孩子发育正常,3岁的孩子应该能自己用勺子吃饭不撒,用杯子喝水不洒到外面,在大人的帮助下自己穿衣服和鞋袜,有一些孩子还可以解开扣子,或者系按扣;自己会在流水的龙头下用肥皂洗手,并用毛巾擦干,用水洗脸;能自己上厕所;能自己盖被子睡觉,夜里基本不遗尿。这些生活技能正是为他们上幼儿园,适应集体生活创造条件。
% \item
%   对数字的理解:3岁左右的孩子可以从1数到10,但不一定理解这些数字的概念。他们还不能手口一致地点数物品,往嘴里数的数和手的动作不一致,有时往往会重数或漏数。
% \item
%   会玩``过家家''游戏:这是3岁孩子最喜欢玩的游戏形式。这些孩子能模仿一些家务劳动的动作,如抱娃娃,给娃娃看病等。
% \end{enumerate}

% 1

% %ux7b2cux4e8cux7bc7-ux5a74ux5e7cux513fux7684ux5582ux517bux6307ux5bfc}{%
% \section{2第二篇
% 婴幼儿的喂养指导}%ux7b2cux4e8cux7bc7-ux5a74ux5e7cux513fux7684ux5582ux517bux6307ux5bfc}}

% 第一节 新生儿的喂养指导

% 第二节 1〜2个月婴儿的喂养指导

% 第三节 2〜3个月婴儿的喂养指导

% 第四节 3〜4个月婴儿的喂养指导

% 第五节 4〜5个月婴儿的喂养指导

% 第六节 5 \textasciitilde{} 6个月婴儿的喂养指导

% 第七节 6〜7个月婴儿的喂养指导

% 第八节7 \textasciitilde{} 8个月婴儿的喂养指导

% 第九节 8-9个月婴儿的喂养指导

% 第十节 9-10个月婴儿的喂养指导

% 第十一节10〜11个月婴儿的喂养指导

% 第十二节11〜12个月婴儿的喂养指导

% %ux7b2cux4e00ux8282-ux65b0ux751fux513fux7684ux5582ux517bux6307ux5bfc}{%
% \subsection{01第一节
% 新生儿的喂养指导}%ux7b2cux4e00ux8282-ux65b0ux751fux513fux7684ux5582ux517bux6307ux5bfc}}

% %ux4e00ux6bcdux4e73ux5582ux517bux7684ux8d8bux52bf}{%
% \subsection{一、母乳喂养的趋势}%ux4e00ux6bcdux4e73ux5582ux517bux7684ux8d8bux52bf}}

% 近来,国际上已将保护、促进和支持母乳喂养作为妇幼卫生工作的一个重要内容。在1990年的世界儿童问题首脑会议上,提高\textbf{四个月的纯母乳喂养率}被列为全球奋斗目标。

% 母乳喂养的情况是多种多样的。1989年联合国儿童基金会主办的母乳喂养定义会上,将母乳喂养分为以下几类:

% %ux5168ux90e8ux6bcdux4e73ux5582ux517b}{%
% \subsubsection{1.
% 全部母乳喂养}%ux5168ux90e8ux6bcdux4e73ux5582ux517b}}

% 全部母乳喂养又分为两种:

% \begin{enumerate}
% \def\labelenumi{\arabic{enumi}.}
% \item
%   纯母乳喂养,指除母乳外,不给婴儿吃其他任何液体或固体食物。
% \item
%   几乎纯母乳喂养,指除母乳外,还给婴儿吃维生素、水、果汁,但每天不超过2次,每次不超过1\textasciitilde2口。
% \end{enumerate}

% %ux90e8ux5206ux6bcdux4e73ux5582ux517b}{%
% \subsubsection{2.
% 部分母乳喂养}%ux90e8ux5206ux6bcdux4e73ux5582ux517b}}

% 部分母乳喂养分为3种:

% \begin{enumerate}
% \def\labelenumi{\arabic{enumi}.}
% \item
%   高比例母乳喂养,指母乳占婴儿全部食物的80\%以上,
% \item
%   中等比例母乳喂养,指母乳占婴儿全部食物的20\% \textasciitilde{} 80\%,
% \item
%   低比例母乳喂养,指母乳占婴儿全部食物的20\%以下。
% \end{enumerate}

% %ux8c61ux5f81ux6027ux6bcdux4e73ux5582ux517b}{%
% \subsubsection{3.象征性母乳喂养}%ux8c61ux5f81ux6027ux6bcdux4e73ux5582ux517b}}

% 这种母乳喂养只给婴儿提供小部分需要。能使产后子宫早日恢复,从而减少产后并发症。

% %ux4e8cux7eafux6bcdux4e73ux54faux517bux80fdux6ee1ux8db3ux5a74ux513fux751fux957fux53d1ux80b2ux7684ux9700ux8981}{%
% \subsection{二、纯母乳哺养能满足婴儿生长发育的需要}%ux4e8cux7eafux6bcdux4e73ux54faux517bux80fdux6ee1ux8db3ux5a74ux513fux751fux957fux53d1ux80b2ux7684ux9700ux8981}}

% 据研究表明,大多数6个月以内的纯母乳喂养婴儿生长适宜。母乳是婴儿\textbf{必需}的和\textbf{理想}的食品,其所含的各种营养物质最适合婴儿消化吸收,而且具有最高的生物利用率。母乳的质与量随着婴儿的生长和需要呈现相应的改变。孩子越吸得勤,乳汁便分泌得越多。一般公认婴儿6周时乳母每日分泌700毫升乳汁,到3个月时可增加到800毫升。据报道,纯母乳喂养时,7个月的婴儿每日可从母亲乳房吮吸到1500毫升乳汁。

% 母亲的乳汁含有丰富的营养成分,如脂肪、乳糖、矿物质、微量元素等。母亲一时营养供给不足,不会影响乳汁成分。但是如母亲长期营养摄入不足,可影响到乳汁营养素的含量,尤其是维生素\(B_{6}\)、维生素\(B_{12}\)和维生素A和维生素D,出现婴儿营养不良现象。

% %ux4e09ux6bcdux4e73ux5582ux517bux7684ux597dux5904}{%
% \subsection{三、母乳喂养的好处}%ux4e09ux6bcdux4e73ux5582ux517bux7684ux597dux5904}}

% 母乳营养丰富,所含各种营养物质最适合宝宝消化吸收,具有最高生物利用率,是宝宝最理想的天然食品。

% 母乳含有丰富而独特的营养元素及活性物质,其复杂而合理的养分搭配完全适合宝宝需求。

% %ux6bcdux4e73ux53efux4ee5ux6839ux636eux5b9dux5b9dux9700ux8981ux4e0dux65adux8c03ux6574}{%
% \subsubsection{1.母乳可以根据宝宝需要不断调整}%ux6bcdux4e73ux53efux4ee5ux6839ux636eux5b9dux5b9dux9700ux8981ux4e0dux65adux8c03ux6574}}

% 每一位母亲的乳汁,都是为她自己宝宝量身定制的,以满足宝宝成长需求。没有两位母亲的乳汁成分是一模一样的。每一位母亲的乳汁都会根据自己宝宝成长情况,每天有所调整,甚至一天之内随时调整。比如,早产儿母亲分泌的乳汁比足月儿母亲的乳汁含有更多免疫球蛋白。

% %ux6bcdux4e73ux53efux4ee5ux4fc3ux8fdbux5927ux8111ux53d1ux80b2}{%
% \subsubsection{2.
% 母乳可以促进大脑发育}%ux6bcdux4e73ux53efux4ee5ux4fc3ux8fdbux5927ux8111ux53d1ux80b2}}

% 所有哺乳动物都分泌乳汁。所有乳汁的基本成分都是水、蛋白质、乳糖、维生素、矿物质,还有消化酶以及荷尔蒙等。然而,不同物种根据自己成长速度、行为和需求,所分泌的乳汁成分也各不相同。比如,乳糖促进大脑发育。智商越高的哺乳动物,其乳汁中乳糖含量越高,母乳则含有最高的乳糖。

% %ux6bcdux4e73ux4e2dux542bux6709ux5929ux7136ux7684ux80c6ux56faux9187}{%
% \subsubsection{3.母乳中含有天然的胆固醇}%ux6bcdux4e73ux4e2dux542bux6709ux5929ux7136ux7684ux80c6ux56faux9187}}

% 虽然成年人比较``惧怕''胆固醇,其实它对于宝宝头两年的成长发育,尤其是大脑和神经系统发育以及维生素D的生成,是必不可少的。缺乏胆固醇和维生素D会导致成年人心脏和中枢神经系统疾病。多年的调查研究表明,母乳喂养宝宝的平均智商高于人工喂养宝宝,而且接受母乳喂养时间越长,相对智力优势也越高。

% %ux6bcdux4e73ux4e2dux6709ux4e30ux5bccux6d3bux6027ux514dux75abux56e0ux5b50ux4e3aux5b9dux5b9dux63d0ux4f9bux6297ux4f53}{%
% \subsubsection{4.
% 母乳中有丰富活性免疫因子,为宝宝提供``抗体''}%ux6bcdux4e73ux4e2dux6709ux4e30ux5bccux6d3bux6027ux514dux75abux56e0ux5b50ux4e3aux5b9dux5b9dux63d0ux4f9bux6297ux4f53}}

% 母乳中的抗体在宝宝体内设立一道天然屏障,保护宝宝不受疾病侵扰,尤其是呼吸道、肠道、耳道感染等宝宝常见病。多年大量调查表明,人工喂养宝宝患以上常见病的患病率和死亡率相比母乳喂养宝宝高十几倍。一项在芝加哥进行的调查结果表明,人工喂养宝宝死于呼吸道感染的几率比母乳喂养宝宝高120倍。另外一项在美国进行的调查表明,母乳喂养宝宝在一岁之内患病率仅为25\%,而人工喂养宝宝患病率则高达94\%。因此医生们将富含免疫球蛋白、白细胞和各种抗发炎因子的母乳称为``白色血液''。

% %ux6bcdux4e73ux5582ux517bux7684ux5b9dux5b9dux6781ux5c11ux53d1ux751fux8d2bux8840}{%
% \subsubsection{\texorpdfstring{\textbf{5、母乳喂养的宝宝极少发生贫血}}{5、母乳喂养的宝宝极少发生贫血}}%ux6bcdux4e73ux5582ux517bux7684ux5b9dux5b9dux6781ux5c11ux53d1ux751fux8d2bux8840}}

% 虽然母乳中铁含量比较少,但它是活性铁,吸收率极高,可达75\%。母乳中含有更多乳糖和维生素C,有助于铁的吸收。根据美国儿科专家的研究和调查,纯母乳喂养至7个月的宝宝群体中没有发现一例贫血,纯母乳喂养至7个月的宝宝在1岁时,比那些较早添加了含强化铁奶粉或米粉的宝宝,血红蛋白高出许多。为避免缺铁性贫血、过敏症、营养不良等后果,国际母乳协会、美国儿科学会推荐母亲们坚持纯母乳喂养宝宝至6个月以上,不建议6个月以前添加辅食。

% \begin{quote}
% \textbf{你知道吗?}

% 母乳中免疫因子会根据宝宝的身体状况进行调整。当宝宝身体受到新的病菌或病毒侵袭时,会通过吸吮乳汁传送到妈妈身体里。妈妈身体会立刻根据``敌情''制造免疫白细胞和球蛋白,再通过乳汁传送给宝宝,在宝宝体内建立屏障,保护宝宝不受感染。

% 母乳会自动浓缩抗体,在宝宝身体内存活七八年之久。母乳喂养至一岁以后,宝宝辅食吃得多,母乳摄入量相对减少,母乳会自动浓缩抗体,将这种重要养分完全输入宝宝体内。这些活性细胞甚至会在宝宝身体内存活长达七八年,有效降低宝宝在整个童年期患病几率。母乳喂养时间较长的女婴,成年后罹患乳腺癌几率也明显减少。
% \end{quote}

% %ux5b9dux5b9dux6c38ux8fdcux4e0dux4f1aux5bf9ux6bcdux4e73ux8fc7ux654f}{%
% \subsubsection{6.宝宝永远不会对母乳过敏}%ux5b9dux5b9dux6c38ux8fdcux4e0dux4f1aux5bf9ux6bcdux4e73ux8fc7ux654f}}

% 母乳中的蛋白质完全适合宝宝。牛奶中的蛋白质却往往给宝宝带来麻烦,如腹痛、呕吐、腹泻、鼻塞、咳嗽、烦躁不安、无法入睡等。母乳保护宝宝不对其他食物过敏,减少罹患其他过敏引起的疾病(如哮喘、湿疹、花粉过敏、慢性肠炎等)的几率。母乳喂养可以长期预防湿疹、食物过敏以及呼吸道过敏症。

% %ux6bcdux4e73ux7ed9ux5b9dux5b9dux6f02ux4eaeux7684ux7259ux9f7f}{%
% \subsubsection{7.
% 母乳给宝宝漂亮的牙齿}%ux6bcdux4e73ux7ed9ux5b9dux5b9dux6f02ux4eaeux7684ux7259ux9f7f}}

% 牙医们鼓励长期母乳喂养,因为吸吮动作有助于宝宝面部肌肉和牙床的发育,有助于宝宝牙齿健康和脸蛋儿漂亮。而奶粉喂养的宝宝则容易发生龋齿和口腔变形。

% %ux6bcdux4e73ux5582ux517bux7701ux65f6ux7701ux529bux7701ux94b1}{%
% \subsubsection{8.母乳喂养省时、省力、省钱}%ux6bcdux4e73ux5582ux517bux7701ux65f6ux7701ux529bux7701ux94b1}}

% 没有什么代乳品比母乳更方便。宝宝一哭,妈妈可以马上用母乳满足他。\textbf{宝宝的心理特点之一是一旦有需求,就必须马上满足}。他们未成熟的身体既不懂得也不适合等待。肚子饿时马上吃到香甜的母乳,会让宝宝更好地建立起对人生的信任感。无论白天黑夜,母乳随时随地生产,存储在乳房里,永远新鲜,也不会枯竭。喂母乳既能省去准备奶粉的麻烦,也能够免除宝宝哭声引起母亲内疚与焦虑。特别是在夜间,喂母乳能够让全家人都睡得更安稳。尤其方便的是在旅行途中,不必担心开水供应、奶瓶消毒、喂奶用具的清洁等问题。

% %ux6bcdux4e73ux5582ux517bux80fdux591fux4fc3ux8fdbux5b9dux5b9dux65e9ux671fux667aux529bux5f00ux53d1}{%
% \subsubsection{9.
% 母乳喂养能够促进宝宝早期智力开发}%ux6bcdux4e73ux5582ux517bux80fdux591fux4fc3ux8fdbux5b9dux5b9dux65e9ux671fux667aux529bux5f00ux53d1}}

% 哺乳过程中,妈妈的声音、心音、气味和肌肤接触能刺激宝宝大脑,促进宝宝早期智力开发。此外,通过抚摸、拥抱、对视等,能使宝宝获得满足感和安全感,能够增加母子间的感情,这也是母乳喂养的最大优点。

% \textbf{母乳主要营养成分}

% \begin{itemize}
% \item
%   蛋白质:大部分是易于消化的乳清蛋白,且含有代谢过程所需的酶以及抵抗感染的免疫球蛋白和溶菌素。
% \item
%   脂肪:含有大量不饱和脂肪酸,并且脂肪球较小,易于吸收。
% \item
%   糖:主要是乳糖,在宝宝消化道内变成乳酸,可以促进消化,有利于钙、铁、锌等的吸收,也能促进肠道内乳酸杆菌的大量繁殖,增强消化道抗感染能力。
% \item
%   矿物质:虽然含量不高,但比例恰当,易于吸收。
% \end{itemize}

% %ux56dbux73cdux8d35ux7684ux521dux4e73}{%
% \subsection{四、珍贵的初乳}%ux56dbux73cdux8d35ux7684ux521dux4e73}}

% 产妇最初分泌的乳汁叫初乳,似黄油。与成熟乳比较,初乳中含有丰富的蛋白质、脂溶性维生素、钠和锌。还含有人体所需要的各种酶类、抗氧化剂等。相对而言含乳糖、脂肪、水溶性维生素较少。初乳中的抗体可以覆盖在婴儿未成熟的肠道表面,阻止细菌、病毒的附着。初乳虽然不多但浓度很高,有促进排泄作用,减少黄疸的发生。所以初乳被人们称为第一次免疫。妈妈一定要抓住给孩子初乳喂养的机会。此外,早产乳也具有最适合喂养早产儿的特点。早产乳乳糖较少,蛋白质、矿物质、乳铁蛋白较多,最适合早产儿生长发育的需要,请不要忽视这点。

% 健康母亲的哺乳量:

% \begin{longtable}[]{@{}
%   >{\raggedright\arraybackslash}p{(\columnwidth - 4\tabcolsep) * \real{0.3333}}
%   >{\raggedright\arraybackslash}p{(\columnwidth - 4\tabcolsep) * \real{0.3333}}
%   >{\raggedright\arraybackslash}p{(\columnwidth - 4\tabcolsep) * \real{0.3333}}@{}}
% \toprule()
% \begin{minipage}[b]{\linewidth}\raggedright
% 产后时间
% \end{minipage} & \begin{minipage}[b]{\linewidth}\raggedright
% 每次哺乳量(毫升)
% \end{minipage} & \begin{minipage}[b]{\linewidth}\raggedright
% 每日平均哺乳量(毫升)
% \end{minipage} \\
% \midrule()
% \endhead
% 第1周 & 8\textasciitilde45 & 250 \\
% 第2周 & 30\textasciitilde90 & 400 \\
% 第4周 & 45\textasciitilde140 & 550 \\
% 第6周 & 60\textasciitilde150 & 700 \\
% 第3月 & 75\textasciitilde160 & 750 \\
% 第4月 & 90\textasciitilde180 & 800 \\
% 第6月 & 120\textasciitilde220 & 1000 \\
% \bottomrule()
% \end{longtable}

% \textbf{人乳、初乳及牛乳成分比较}

% \begin{longtable}[]{@{}
%   >{\raggedright\arraybackslash}p{(\columnwidth - 6\tabcolsep) * \real{0.2500}}
%   >{\raggedright\arraybackslash}p{(\columnwidth - 6\tabcolsep) * \real{0.2500}}
%   >{\raggedright\arraybackslash}p{(\columnwidth - 6\tabcolsep) * \real{0.2500}}
%   >{\raggedright\arraybackslash}p{(\columnwidth - 6\tabcolsep) * \real{0.2500}}@{}}
% \toprule()
% \begin{minipage}[b]{\linewidth}\raggedright
% 成分
% \end{minipage} & \begin{minipage}[b]{\linewidth}\raggedright
% 人乳(每100克)
% \end{minipage} & \begin{minipage}[b]{\linewidth}\raggedright
% 人初乳(每100克)
% \end{minipage} & \begin{minipage}[b]{\linewidth}\raggedright
% 牛乳(每100克)
% \end{minipage} \\
% \midrule()
% \endhead
% 水 & 88 & 87 & 88 \\
% 蛋白质 & 0.9 & 2.3 & 3.3 \\
% 酪蛋白 & 0.4 & 1.2 & 2.7 \\
% 乳白蛋白 & 0.2 & 1.5 & 0.2 \\
% 乳球蛋白 & & & \\
% 脂肪 & 3.8 & 2.9 & 3.8 \\
% 不饱和脂肪酸 & 8.0 & 7.0 & 2.0 \\
% 乳糖 & 7.0 & 5.3 & 4.8 \\
% \bottomrule()
% \end{longtable}

% \begin{longtable}[]{@{}
%   >{\raggedright\arraybackslash}p{(\columnwidth - 6\tabcolsep) * \real{0.2500}}
%   >{\raggedright\arraybackslash}p{(\columnwidth - 6\tabcolsep) * \real{0.2500}}
%   >{\raggedright\arraybackslash}p{(\columnwidth - 6\tabcolsep) * \real{0.2500}}
%   >{\raggedright\arraybackslash}p{(\columnwidth - 6\tabcolsep) * \real{0.2500}}@{}}
% \toprule()
% \begin{minipage}[b]{\linewidth}\raggedright
% 矿物质
% \end{minipage} & \begin{minipage}[b]{\linewidth}\raggedright
% 人乳 (毫克/100克)
% \end{minipage} & \begin{minipage}[b]{\linewidth}\raggedright
% 初乳 (毫克/100克)
% \end{minipage} & \begin{minipage}[b]{\linewidth}\raggedright
% 牛乳 (毫克/100克)
% \end{minipage} \\
% \midrule()
% \endhead
% 钙 & 34 & 30 & 117 \\
% 磷 & 15 & 15 & 32 \\
% 钠 & 15 & 135 & 58 \\
% 钾 & 55 & 275 & 138 \\
% 镁 & 4 & 4 & 12 \\
% 铜 & 0.04 & 0.06 & 0.03 \\
% 铁 & 0.21 & 0.01 & 0.21 \\
% 锌 & 0.4 & 0.6 & 0.4 \\
% 碘 & 0.003 & 0.012 & 0.05 \\
% \bottomrule()
% \end{longtable}

% \begin{longtable}[]{@{}
%   >{\raggedright\arraybackslash}p{(\columnwidth - 6\tabcolsep) * \real{0.2500}}
%   >{\raggedright\arraybackslash}p{(\columnwidth - 6\tabcolsep) * \real{0.2500}}
%   >{\raggedright\arraybackslash}p{(\columnwidth - 6\tabcolsep) * \real{0.2500}}
%   >{\raggedright\arraybackslash}p{(\columnwidth - 6\tabcolsep) * \real{0.2500}}@{}}
% \toprule()
% \begin{minipage}[b]{\linewidth}\raggedright
% 维生素(毫克/100毫升)
% \end{minipage} & \begin{minipage}[b]{\linewidth}\raggedright
% 人乳
% \end{minipage} & \begin{minipage}[b]{\linewidth}\raggedright
% 初乳
% \end{minipage} & \begin{minipage}[b]{\linewidth}\raggedright
% 牛乳
% \end{minipage} \\
% \midrule()
% \endhead
% A & 190国际单位 & 100国际单位 & - \\
% B1 & 0.016 & 无 & 0.044 \\
% B2 & 0.036 & 无 & 0.175 \\
% 烟酸 & 0.147 & 无 & 0.094 \\
% B2 & 0.01 & 无 & 0.064 \\
% 叶酸 & 0.052 & 无 & 0.055 \\
% B12 & 0.00003 & 无 & 0.0004 \\
% C & 4.3 & 无 & 1.5 \\
% D & 4\textasciitilde11国际单位 & 无 & 0.3-4.0国际单位 \\
% E & 0.2 & 无 & 0.04 \\
% K & 0.0015 & 无 & 0.006 \\
% \bottomrule()
% \end{longtable}

% \begin{quote}
% gpt:

% "国际单位"(International Units,
% IU)是一种测量生物活性或效力的单位,常用于药物、疫苗、维生素、激素等物质的测量。国际单位的定义和换算因物质的性质和效力而异,对于每种物质都有一个国际协议定义了其国际单位的计算方法。例如,维生素A、D、E的国际单位就是根据这些维生素的生物活性或效力来定义的。

% 需要注意的是,国际单位并不是一个通用的、可以互换的单位,比如一个国际单位的维生素A并不等同于一个国际单位的维生素D。换句话说,国际单位的定义是依赖于它所测量的特定生物物质的。
% \end{quote}

% %ux4e94ux6bcdux4e73ux5582ux517bux5173ux952eux8bcd}{%
% \subsection{五、母乳喂养关键词}%ux4e94ux6bcdux4e73ux5582ux517bux5173ux952eux8bcd}}

% %ux5c3dux65e9ux5f00ux5976}{%
% \subsubsection{1.尽早开奶}%ux5c3dux65e9ux5f00ux5976}}

% 宝宝吮吸妈妈乳头是乳汁产生的关键。一般来说,只要母子都没有异常,宝宝出生后半小时就可以开始吸吮妈妈乳汁了。

% \hspace{0pt}\includegraphics[width=2.85315in,height=2.99301in]{media/rId482.png}\hspace{0pt}

% %ux6309ux9700ux5582ux5976}{%
% \subsubsection{2.按需喂奶}%ux6309ux9700ux5582ux5976}}

% 每当宝宝饿了、渴了或妈妈感到乳房胀时,就应喂奶。宝宝出生后\(2 - 7\)天,每\(1 - 3\)\hspace{0pt}小时喂1次,也可更多些,间隔不要超过3小时。妈妈下奶后,通常每24小时喂8\textasciitilde12次,\textbf{夜间不应停止}哺乳。

% %ux54faux4e73ux65f6ux95f4}{%
% \subsubsection{3.哺乳时间}%ux54faux4e73ux65f6ux95f4}}

% 一次的授乳时间,开始时有的只用1-2分钟,以后逐渐延长,最终的标准是15-20分钟。如果超过30分钟,宝宝和妈妈都会疲劳。

% %ux54faux4e73ux65b9ux6cd5}{%
% \subsubsection{4.哺乳方法}%ux54faux4e73ux65b9ux6cd5}}

% 给宝宝喂奶时,不能只让其含住乳头,而应将\textbf{乳头及大部分乳晕}一起塞进宝宝口中,因为乳晕下面的乳窦是储存乳汁的重要部位。为了保持充沛乳汁,妈妈应该坚持两侧乳房轮流喂奶,并且应在每次充分\textbf{哺乳后挤净乳房内余奶},促进乳汁分泌。

% %ux8fdbux98dfux5145ux8db3ux5976ux91cf}{%
% \subsubsection{5.进食充足奶量}%ux8fdbux98dfux5145ux8db3ux5976ux91cf}}

% 一般来说,在不添加任何辅助食品(包括水、饮料、汤)的情况下,宝宝每日有6次或者6次以上小便,这表明宝宝已经进食足够奶量。

% %ux9632ux6b62ux4e73ux5934ux9519ux89c9}{%
% \subsubsection{6.防止乳头错觉}%ux9632ux6b62ux4e73ux5934ux9519ux89c9}}

% 使用胶皮奶嘴,宝宝吮吸时不需费力,若再改为吮吸母乳,宝宝有可能只吸乳头而不含乳晕,吮吸起来比较费力,也就不愿再吸了。因此,如果确实需要喂牛奶开水等,要用小勺或者注射器等,以防宝宝乳头错觉。

% \begin{quote}
% gpt

% "乳头错觉"(Nipple
% confusion)是一个描述新生儿或婴儿在母乳喂养和瓶喂之间可能出现混淆的术语。这种情况通常在婴儿同时使用母乳和瓶喂(包括奶嘴或吸管)进行喂养时发生。吮吸乳头和吮吸奶瓶的机制是不同的,婴儿可能会难以在两者之间进行切换。

% 婴儿在吮吸乳头时需要用到不同的口腔和舌部肌肉,而吮吸奶瓶则相对容易,因为奶水可以直接流入婴儿的口中。如果婴儿过早或过频繁地使用奶瓶,他们可能会适应奶瓶的喂奶方式,从而导致他们在后续的母乳喂养中出现困难。

% 乳头错觉并非所有婴儿都会经历,但是对于一些试图维持母乳喂养的母亲来说,这可能是一个挑战。为了避免乳头错觉,一些专家建议在婴儿完全适应母乳喂养后(通常需要3-4周的时间),再尝试引入奶瓶。
% \end{quote}

% %ux9632ux6b62ux6ea2ux5976}{%
% \subsubsection{7. 防止溢奶}%ux9632ux6b62ux6ea2ux5976}}

% 哺乳结束时,应抱起宝宝轻拍背部,帮助他把吸吮时吞咽的空气从口中排除。然后让宝宝侧身躺一会儿,以防止溢出的奶被宝宝误吸,引起其他疾病。

% 新生宝宝怎么吃

% \begin{longtable}[]{@{}
%   >{\raggedright\arraybackslash}p{(\columnwidth - 6\tabcolsep) * \real{0.2500}}
%   >{\raggedright\arraybackslash}p{(\columnwidth - 6\tabcolsep) * \real{0.2500}}
%   >{\raggedright\arraybackslash}p{(\columnwidth - 6\tabcolsep) * \real{0.2500}}
%   >{\raggedright\arraybackslash}p{(\columnwidth - 6\tabcolsep) * \real{0.2500}}@{}}
% \toprule()
% \begin{minipage}[b]{\linewidth}\raggedright
% \end{minipage} & \begin{minipage}[b]{\linewidth}\raggedright
% 第1周
% \end{minipage} & \begin{minipage}[b]{\linewidth}\raggedright
% 第2周
% \end{minipage} & \begin{minipage}[b]{\linewidth}\raggedright
% 第3、4周
% \end{minipage} \\
% \midrule()
% \endhead
% 喂哺次数 & 10-12次 & 8-10次 & 7-8次 \\
% 主要食物 & 母乳、配方奶粉 & 母乳、配方奶粉 & 母乳、配方奶粉 \\
% \bottomrule()
% \end{longtable}

% \begin{quote}
% 育儿小百科

% \textbf{哪些母亲不宜哺乳}

% 母乳喂养固然有许多优点,但还是有少数母亲因健康原因不宜哺乳。例如,母亲生产时失血过多或患有败血症;患有结核病、肝炎等传染病;患严重心脏病、肾脏疾病、糖尿病、癌症或身体极度虚弱者。患急性传染病、乳头皲裂或乳腺脓肿者,可暂时停止哺乳。在暂停哺乳期间,要将乳汁用吸奶器吸出来。这有两个好处,一方面可以消除肿胀;另一方面可以使病愈后哺乳时,仍有足量的乳汁。在暂停哺乳期间,可以用牛奶代替母乳。
% \end{quote}

% %ux516dux5f71ux54cdux6bcdux4e73ux5206ux6cccux7684ux56e0ux7d20}{%
% \subsection{六、影响母乳分泌的因素}%ux516dux5f71ux54cdux6bcdux4e73ux5206ux6cccux7684ux56e0ux7d20}}

% 母乳分泌量的多少受许多因素影响,主要有:

% \begin{enumerate}
% \def\labelenumi{\arabic{enumi}.}
% \item
%   母亲的营养状况:母亲营养良好,热量充足,各种营养素充足,其乳汁的分泌质量就高且数量也多;反之,则质劣量少。
% \item
%   乳母的精神情绪:精神情绪因素起一定作用,如焦虑、悲伤、紧张、不安都可使乳汁突然减少。因此,乳母应该有一个宁静、愉快的生活环境。
% \item
%   乳母的休息时间:乳母要有充分的休息时间,休息时间不足,可使乳汁分泌减少。
% \item
%   乳母的健康状况:乳母生病也会使乳汁减少。每次乳汁不能完全排空或每日的哺乳次数过少,使乳房内乳汁积郁,会抑制乳汁分泌。
% \end{enumerate}

% \begin{quote}
% 育儿小百科

% \textbf{哪些药物影响哺乳}

% 以下药物影响泌乳:
% \end{quote}

% \begin{enumerate}
% \def\labelenumi{\arabic{enumi}.}
% \item
%   \begin{quote}
%   生物碱代谢药可影响泌乳素的产生,从而抑制泌乳。
%   \end{quote}
% \item
%   \begin{quote}
%   止痛药。止痛药,如可待因、安乃近应避免使用。因为这些药可通过乳汁分泌出来。可选择扑热息痛。如母亲用了安定、巴比妥酸盐等药后可影响泌乳。
%   \end{quote}
% \end{enumerate}

% \begin{quote}
% 以下药物在哺乳期最好不用,如必须用时,就要考虑停止哺乳:

% 金刚烷胺、抗癌药物、溴化物、放射性同位素等。
% \end{quote}

% %ux4e03ux6bcdux4e73ux4e0dux8db3ux7684ux8868ux73b0}{%
% \subsection{七、母乳不足的表现}%ux4e03ux6bcdux4e73ux4e0dux8db3ux7684ux8868ux73b0}}

% \begin{enumerate}
% \def\labelenumi{\arabic{enumi}.}
% \item
%   母亲感觉乳房空。
% \item
%   宝宝吃奶时间长,用力吸吮却听不到连续的吞咽声,有时突然放开奶头啼哭不止。
% \item
%   宝宝睡不香甜,出现吃完奶不久就哭闹,来回转头寻找奶头的现象。
% \item
%   宝宝大小便次数少,量也少。
% \item
%   体重不增加或增加缓慢。
% \end{enumerate}

% 大多数自认为``没有奶''的母亲并非真正母乳不足,应及时查明原因,排除障碍,并采取积极的催奶办法,千万不要轻易放弃母乳喂养。

% %ux516bux6bcdux4e73ux5582ux517bux5e94ux6ce8ux610fux7684ux95eeux9898}{%
% \subsection{八、母乳喂养应注意的问题}%ux516bux6bcdux4e73ux5582ux517bux5e94ux6ce8ux610fux7684ux95eeux9898}}

% 孩子出生后1\textasciitilde2小时内,妈妈就要做好抱婴准备。

% 掌握正确的哺乳姿势。让孩子把乳头及部分乳晕含接在口中,孩子吃起来很香甜。孩子吃奶姿势正确,也可防止乳头皲裂和不适当的供乳。

% 纯母乳喂养的孩子,除母乳外不添加任何食品,包括不用喂水,孩子什么时候饿了什么时候吃。纯母乳喂哺最好坚持6个月左右。

% 孩子出生后头几个小时和头几天要多吸母乳,以达到促进乳汁分泌的目的。孩子饥饿时或母亲感到乳房充满时,可随时喂哺,哺乳间隔是由宝宝和母亲感觉决定的,这也叫按需哺乳。

% 孩子出生后2-7天内,喂奶次数频繁,以后通常每日喂8-12次,当婴儿睡眠时间较长\textbf{或母亲感到乳胀时},可叫醒宝宝随时喂哺。

% 哺乳前母亲应先做好准备,将手洗干净,用温开水清洗乳头。哺乳时母亲最好坐在椅子上,将孩子抱在怀中,如孩子的头依偎于母亲左侧手臂,则先喂左侧乳房,吸空后换另一侧。这样可使两侧乳房都有排空的机会。哺乳完毕后,以软布擦洗乳头,并盖于其上。再将孩子抱直,头靠肩,用手轻拍孩子背部,使孩子打几个嗝,胃内空气排出,以防溢奶,然后将婴儿放在床上,向右倾卧位,头略垫高。

% %ux4e5dux6bcdux4e73ux5582ux517bux7684ux6b63ux786eux59ffux52bf}{%
% \subsection{九、母乳喂养的正确姿势}%ux4e5dux6bcdux4e73ux5582ux517bux7684ux6b63ux786eux59ffux52bf}}

% %ux5582ux54faux4e2dux6bcdux4eb2ux7684ux6b63ux786eux59ffux52bf}{%
% \subsubsection{1.喂哺中母亲的正确姿势}%ux5582ux54faux4e2dux6bcdux4eb2ux7684ux6b63ux786eux59ffux52bf}}

% \textbf{体位舒适}。喂哺可采取不同姿势,重要的是母亲应了解,心情愉快、体位舒适和全身肌肉松弛有益于乳汁排出。

% \hspace{0pt}\includegraphics[width=2.61538in,height=2.93706in]{media/rId508.png}\hspace{0pt}

% \textbf{母婴必须紧密相贴}。无论怎样抱婴儿,喂哺时婴儿的身体与母亲的身体相贴。婴儿的头与双肩朝向乳房,嘴处于乳头相同水平的位置。

% \textbf{防止婴儿鼻部受压}。喂哺全过程保持婴儿头和颈略微伸张,以免鼻部压到乳房而影响呼吸,但也要防止婴儿头部与颈部过度伸展造成吞咽困难。

% \textbf{母亲手的正确姿势}:应将\textbf{拇指和四指分别放在乳房上、下方},托起整个乳房喂哺。除非在奶流过急,婴儿有呛奶时,避免剪刀式夹托乳房。这种手势会反向推乳腺组织,阻碍婴儿将大部分乳晕含入口内,不利于充分挤压乳窦内的乳汁。

% \textbf{母亲喂哺常取姿势}:不要躺着喂奶,母亲喂哺的常取姿势一般应为以下两种之一。

% \begin{itemize}
% \item
%   \textbf{第一种是卧位哺乳,即侧卧或仰卧位。}\\
%   \hspace{0pt}\includegraphics[width=3.25874in,height=2.64336in]{media/rId511.png}\hspace{0pt}
% \item
%   \textbf{第二种是坐位喂乳}。要求椅子高度合适,且没有把手用于支托婴儿,椅子不宜太软。椅背不宜\uline{后倾},否则使婴儿含吮\uline{不易定位}。喂哺时母亲应紧\textbf{靠椅背}促使背部和双肩处于放松姿势。用枕支托婴儿,还可在母亲足下添加脚凳以帮助身体舒适、松弛,有益于排乳反射不被抑制。
% \end{itemize}

% 采用坐位``环抱式''喂哺,尤其适用于剖腹产及双胎婴儿,因此式可避免伤口受压疼痛,也可使双胎婴儿同时授乳。

% %ux5582ux54faux4e2dux5a74ux513fux7684ux6b63ux786eux59ffux52bf}{%
% \subsubsection{2.喂哺中婴儿的正确姿势}%ux5582ux54faux4e2dux5a74ux513fux7684ux6b63ux786eux59ffux52bf}}

% \textbf{正确的含接姿势}:每次喂哺先将乳头触及婴儿口唇,\textbf{诱发觅食反射},当婴儿口张大、舌向下的一\textbf{瞬间},即将婴儿靠向母亲,使其能大口地\textbf{把乳晕也吸入口内}。这样婴儿在吸吮时能\textbf{充分挤压乳晕下的乳窦},使乳汁排出,又能有效地\textbf{刺激乳头上的感觉神经末梢},促进泌乳和排乳反射。

% \textbf{紧密相贴}:婴儿的嘴及下颌部紧贴乳房,身体紧靠母亲。

% \textbf{出现典型的颌部动作}:颌部肌肉作出缓慢而有力,并伴有节律地向后作伸展运动,直至耳部。如出现两面颊向内缩的动作,说明婴儿含接姿势不正确。

% \textbf{特别提醒}:初乳是母亲产后最初几日产生的乳汁,含有丰富的抗体,对多种细菌、病毒具有抵抗作用,应当及时让小宝宝吃上母亲的初乳。

% %ux5341ux600eux6837ux77e5ux9053ux5b69ux5b50ux662fux5426ux5403ux9971}{%
% \subsection{十、怎样知道孩子是否吃饱}%ux5341ux600eux6837ux77e5ux9053ux5b69ux5b50ux662fux5426ux5403ux9971}}

% \begin{enumerate}
% \def\labelenumi{\arabic{enumi}.}
% \item
%   喂奶前乳房丰满,喂奶后乳房较柔软。
% \item
%   喂奶时可听见吞咽声。
% \item
%   母亲有下乳的感觉。
% \item
%   尿布24小时湿6次及6次以上。
% \item
%   孩子大便软,呈金黄色、糊状。每天2\textasciitilde4次。
% \item
%   在两次喂奶之间,婴儿很满足、安静。
% \item
%   孩子体重平均每天增长18-30克或每周增加125-210克。
% \end{enumerate}

% %ux5341ux4e00ux6bcdux4e73ux4e0dux8db3ux65f6ux7684ux5582ux517bux65b9ux6cd5}{%
% \subsubsection{十一、母乳不足时的喂养方法}%ux5341ux4e00ux6bcdux4e73ux4e0dux8db3ux65f6ux7684ux5582ux517bux65b9ux6cd5}}

% 母乳不足时,需加牛奶或其他乳制品进行混合喂养。混合喂养虽不如母乳喂养效果好,但要比完全人工喂养好得多。混合喂养时,每次应先哺母乳,将乳房吸空后,再给孩子补充其他乳品,当然最好的代乳品是婴儿配方奶粉。这种喂养方法叫做补授法。这样每次按时哺乳吸空,有利于刺激乳汁的再分泌,否则会使母乳量逐渐减少。补授的乳汁量要按孩子食欲情况与母乳分泌量多少而定,原则是孩子吃饱为宜。补授开始需观察儿天,以便掌握每次补授的奶量及孩子有无消化异常现象。以无腹泻、吐奶等现象为宜。

% %ux5341ux4e8cux5256ux5babux4ea7ux5988ux5988ux600eux6837ux8fdbux884cux6bcdux4e73ux5582ux517b}{%
% \subsection{十二、剖宫产妈妈怎样进行母乳喂养}%ux5341ux4e8cux5256ux5babux4ea7ux5988ux5988ux600eux6837ux8fdbux884cux6bcdux4e73ux5582ux517b}}

% %ux9996ux5148ux5256ux5babux4ea7ux540eux80fdux4e0dux80fdux7acbux5373ux5582ux5976}{%
% \subsubsection{1.
% 首先,剖宫产后能不能立即喂奶?}%ux9996ux5148ux5256ux5babux4ea7ux540eux80fdux4e0dux80fdux7acbux5373ux5582ux5976}}

% 麻药会不会进入到奶水里?目前,剖宫产产妇通常使用的是硬膜外麻醉,也就是腹腔麻醉,麻醉药剂的剂量不会对奶水造成影响。等到产妇清醒和肢体能够活动时,麻药也已经代谢得差不多了(对于那些不能马上下奶的产妇更不存在这个问题)。

% %ux4ea7ux540eux8f93ux6db2ux6d88ux708eux662fux5426ux4f1aux5ef6ux8fdfux5f00ux5976}{%
% \subsubsection{2.产后输液消炎是否会延迟开奶?}%ux4ea7ux540eux8f93ux6db2ux6d88ux708eux662fux5426ux4f1aux5ef6ux8fdfux5f00ux5976}}

% 产后输液通常是消炎或预防感染,使用的是静脉注射抗生素,一般不会影响乳汁分泌和成分。目前妇产医院在割宫产产后输液的药品上也会尽量选用对乳汁没有影响的药品。

% %ux5b9dux5b9dux7684ux5438ux542eux4f1aux4fc3ux8fdbux5b50ux5babux6536ux7f29ux90a3ux4e48ux4f1aux4e0dux4f1aux5f71ux54cdux4f24ux53e3ux6108ux5408ux5462}{%
% \subsubsection{3.宝宝的吸吮会促进子宫收缩,那么会不会影响伤口愈合呢?}%ux5b9dux5b9dux7684ux5438ux542eux4f1aux4fc3ux8fdbux5b50ux5babux6536ux7f29ux90a3ux4e48ux4f1aux4e0dux4f1aux5f71ux54cdux4f24ux53e3ux6108ux5408ux5462}}

% 宝宝的吸吮确实是可以促进子宫收缩,与妈妈的担心恰恰相反,这种收缩其实会减少子宫出血。子宫收缩得越快,复原得也越快。因此,医生都会鼓励新妈妈们让宝宝多多吸吮。

% %ux672fux540e24ux5c0fux65f6ux4e0dux80fdux5403ux4e1cux897fux4f1aux4e0dux4f1aux59a8ux788dux4e73ux6c41ux5206ux6ccc}{%
% \subsubsection{4.术后24小时不能吃东西会不会妨碍乳汁分泌?}%ux672fux540e24ux5c0fux65f6ux4e0dux80fdux5403ux4e1cux897fux4f1aux4e0dux4f1aux59a8ux788dux4e73ux6c41ux5206ux6ccc}}

% 剖宫产产妇身体受损和体内泌乳素的迟至,都会为最初的哺乳造成影响。很多人对剖宫产都有一些基本的了解,就是剖宫产后乳汁分泌不如自然分娩快。的确,那是因为母体没有经历自然分娩的过程,体内的泌乳素一时达不到迅速催乳的程度。与产后是否及时喝汤、喝水没有关系。况且24小时后就可以喝一些利于排气的汤。要想加快乳汁的产出,宝宝的吸吮其实才是最有效的``武器''。

% %ux54faux4e73ux7684ux59ffux52bfux600eux6837}{%
% \subsubsection{5.哺乳的姿势怎样}%ux54faux4e73ux7684ux59ffux52bfux600eux6837}}

% 剖宫产产妇由于伤口原因,起初很难采取一般产妇的哺乳姿势,即横抱式。同时也很难采取标准的侧卧位,而使宝宝含乳姿势不标准,容易造成乳头疼痛或乳头皲裂。因此,对于剖宫产产妇,学会正确哺乳姿势,对宝宝和妈妈都很重要。下面是两种剖宫产产妇常用的比较有效的体位。

% \textbf{床上坐位哺乳}:妈妈背靠床头坐或半坐卧,将背后垫靠舒服。把枕头或棉被叠放在身体一侧,其高度约在乳房下边缘(产妇根据个人情况自行调节)。将宝宝臀部放在高的枕头或棉被上,腿朝向妈妈身后。妈妈用胳膊抱住宝宝,使他的胸部紧贴妈妈胸部。妈妈用另一只手以"C"字型托住乳房,让宝宝含住乳头和大部分乳晕。

% \textbf{床下坐位哺乳}:妈妈坐在床边椅子上,尽量坐得舒服,身体靠近床沿,并与床沿成一夹角。把宝宝放在床上,用枕头或棉被把宝宝垫到适当高度,使他的嘴能刚好含住乳头。妈妈就可以抱住宝宝,用另一只手呈C字型托住乳房给宝宝哺乳。

% 需要明确的是,最初正确的哺乳姿势,更重要意义在于让宝宝对乳头进行有效吸吮,以促进哺乳反射和泌乳素分泌,让宝宝适应和习惯妈妈的乳头。正确舒适的体位和宝宝衔乳姿势的正确,还能够增强妈妈哺乳信心,从而达到良性循环,使得乳汁更加充沛。

% %ux5f00ux59cbux4e73ux6c41ux4e0dux8db3ux600eux4e48ux529e}{%
% \subsubsection{6.
% ---开始乳汁不足怎么办}%ux5f00ux59cbux4e73ux6c41ux4e0dux8db3ux600eux4e48ux529e}}

% 这是妈妈们最着急的事。剖宫产妈妈比自然分娩妈妈下奶晚,使得她们会生出许多担心:宝宝吃到奶了吗?会不会吃不饱,饿坏了吗?我会不会就是没有奶啊?

% 由于伤口的疼痛、哺乳体位的困难,乃至乳头皲裂的遭遇,也使得一些剖宫产妈妈很快就产生了畏惧心理,从而丧失或放弃了母乳喂养的信心。这些不论是担心着急,还是主动放弃,恰恰是造成乳汁更加不足的罪魁祸首,这些妈妈的母乳喂养梦想更加艰难甚至最终破灭。

% 在宝宝\textbf{初生两三天里,他们其实不会太饿},在这几天里他们正忙着排出胎便和羊水。妈妈们需要做的就是每天保证能让宝宝在24小时之内吸吮乳头至少8\textasciitilde12次。充分吸吮既能让宝宝吃到富含抗体的初乳,也能刺激妈妈更快下奶。

% 如果实在没有奶水,宝宝哭闹不止,可以给他冲调一些很淡的奶粉或葡萄糖水,用\textbf{针管}喂给他(切忌用奶瓶,以免造成宝宝乳头错觉)。一两天、两三天后,剖宫产妈妈就会下奶了。妈妈们对于母乳喂养总还是有些紧张,尽管它是那么自然而然的一件事,尤其是剖宫产的新妈妈。而母乳喂养最大的天敌,不是年龄,不是机体差异,更不可能是外界阻力,而是来自妈妈内心的沮丧、娇气或脆弱。你的信心、你的无私、你的坚韧,就是乳汁源源不断的根源。只要你想喂,你就得做好吃苦受累的准备。

% %ux5341ux4e09ux4ebaux5de5ux5582ux517b}{%
% \subsection{十三、人工喂养}%ux5341ux4e09ux4ebaux5de5ux5582ux517b}}

% 人工喂养是指由于各种原因造成的主观上不愿进行母乳喂养,或者客观上限制了母乳喂养,而只好采用其他乳品和代乳品进行喂哺婴儿的一种方法。人工喂养相对前两种喂养方法复杂一些,但只要细心,同样会收到较满意的喂养效果。

% %ux5341ux56dbux4ebaux5de5ux5582ux517bux8981ux6ce8ux610fux7684ux95eeux9898}{%
% \subsection{十四、人工喂养要注意的问题}%ux5341ux56dbux4ebaux5de5ux5582ux517bux8981ux6ce8ux610fux7684ux95eeux9898}}

% 最好为孩子选购直式奶瓶,便于洗刷,奶头软硬应适宜,乳孔大小可根据孩子吸吮能力情况而定,一般在乳头上扎两个孔,最好扎在\textbf{侧面不易呛奶}。奶头孔扎好后,将奶瓶盛水倒置,以连续滴出为宜。

% 奶瓶、奶头、杯子、碗、匙等食具,每次用后要清洗,并消毒。市场有专门的奶瓶消毒锅,也可以给孩子准备一个干净的锅专门消毒用,把奶瓶放入锅中加水在火上煮沸20分钟即可。

% 每次喂哺前要看乳汁的温度,过热、过凉对婴儿都不利。可将奶滴于腕、手背部,以不烫为宜。

% 喂奶时将奶瓶倾斜45度,使乳头中充满乳汁,以避免冲力太大或空气吸入。

% %ux5341ux4e94ux4ebaux5de5ux5582ux517bux53caux5176ux5e94ux5582ux591aux5c11ux6c34}{%
% \subsection{十五、人工喂养及其应喂多少水}%ux5341ux4e94ux4ebaux5de5ux5582ux517bux53caux5176ux5e94ux5582ux591aux5c11ux6c34}}

% 1个月以内的婴儿在两次喂奶中间喂1次水或米汤,每日每千克体重喂40\textasciitilde50毫升水。但孩子到底需要多少水,与天气、活动量、身体情况有关。

% %ux6df7ux5408ux5582ux517bux7684ux65b9ux6cd5}{%
% \subsubsection{混合喂养的方法}%ux6df7ux5408ux5582ux517bux7684ux65b9ux6cd5}}

% 需要混合喂养时,可以采取补授法或代授法。

% \textbf{补授法}:每天喂哺母乳次数照常。每次喂完母乳后,补喂配方奶。

% \textbf{代授法}:以配方奶代替1次或几次母乳喂哺,总次数以不超过每天哺乳次数的一半为宜。

% 以上两种方法,妈妈可以根据自己的实际情况选择,以补授法比较好。

% 混合喂养的规律及喂奶次数,与母乳喂养相同。

% 最重要的是,妈妈在乳汁少时不要轻易放弃母乳喂养。有的妈妈一看到乳汁少,马上就急着用配方奶来代替,生怕饿着宝宝。其实,随着宝宝吸吮乳房次数增多,乳汁还会逐渐增多。妈妈应该这样想:``这只不过是我临时的解决方法,我的奶水会多起来的,很快就不用配方奶粉代替了。''有了这种想法,妈妈就会树立信心,千方百计地想办法增加乳汁。否则,乳汁真的很快就完全没有了。

% \textbf{新生儿期母子每日生活时间表}

% \begin{longtable}[]{@{}
%   >{\raggedright\arraybackslash}p{(\columnwidth - 2\tabcolsep) * \real{0.5000}}
%   >{\raggedright\arraybackslash}p{(\columnwidth - 2\tabcolsep) * \real{0.5000}}@{}}
% \toprule()
% \begin{minipage}[b]{\linewidth}\raggedright
% 时间
% \end{minipage} & \begin{minipage}[b]{\linewidth}\raggedright
% 内容
% \end{minipage} \\
% \midrule()
% \endhead
% 05:00 & 给婴儿换尿布,喂奶然后放回床上 \\
% 07:00 & 母亲吃早餐 \\
% 08:00 & 喂奶,换尿布 \\
% 09:00 & 给婴儿洗澡并喂水1次,放回床上(洗澡也可放在下午5:00) \\
% 11:00 &
% 喂奶,然后放回床上,婴儿出生4周后,如天气适宜,可放入小车到室外1小时 \\
% 12:00 & 母亲进午餐 \\
% 12:30 & 母亲午休 \\
% 14:00 & 喂奶,放回床上,4周后,如天气适宜,可放入小车到室外1小时 \\
% 16:00 & 喂水,生后第2周可喂果汁水,并预备好婴儿睡觉 \\
% 17:00 & 喂奶,把婴儿放回床上 \\
% 18:30 & 母亲晚餐 \\
% 20:00 & 喂奶,把婴儿放回床上 \\
% 23:00 &
% 喂奶,然后让婴儿入睡到第2天5:00,如果夜间醒了,换尿布后还啼哭,在没有什么不舒服的情况下,可喂一点水。 \\
% \bottomrule()
% \end{longtable}

% 牛奶喂哺参考表

% \begin{longtable}[]{@{}
%   >{\raggedright\arraybackslash}p{(\columnwidth - 8\tabcolsep) * \real{0.2000}}
%   >{\raggedright\arraybackslash}p{(\columnwidth - 8\tabcolsep) * \real{0.2000}}
%   >{\raggedright\arraybackslash}p{(\columnwidth - 8\tabcolsep) * \real{0.2000}}
%   >{\raggedright\arraybackslash}p{(\columnwidth - 8\tabcolsep) * \real{0.2000}}
%   >{\raggedright\arraybackslash}p{(\columnwidth - 8\tabcolsep) * \real{0.2000}}@{}}
% \toprule()
% \begin{minipage}[b]{\linewidth}\raggedright
% 年龄
% \end{minipage} & \begin{minipage}[b]{\linewidth}\raggedright
% 一日奶量(毫升)
% \end{minipage} & \begin{minipage}[b]{\linewidth}\raggedright
% 加水量(毫升)
% \end{minipage} & \begin{minipage}[b]{\linewidth}\raggedright
% 喂次数
% \end{minipage} & \begin{minipage}[b]{\linewidth}\raggedright
% 一次奶量(毫升)
% \end{minipage} \\
% \midrule()
% \endhead
% 第1周 & 140 & 280 & 7 & 60 \\
% 第2周 & 280 & 280 & 7 & 80 \\
% 第3周 & 400 & 200 & 6\textasciitilde7 & 85-100 \\
% 第4周 & 500 & 150 & 6\textasciitilde7 & 90-110 \\
% \bottomrule()
% \end{longtable}

% \begin{quote}
% 重点提示

% \textbf{如何补充钙剂和维生素D}

% 婴幼儿时期是人体生长发育最迅速时期,尤其是骨骼增长很快。补充钙剂和维生素D预防佝倭病发生就显得尤为重要。那么如何补充钙片和鱼肝油滴剂呢?

% 根据世界卫生组织规定,纯母乳喂养的宝宝在4个月前是不需添加任何营养素的(包括钙和维生素D),因为母乳中所含营养成分完全可以满足4个月内宝宝需要。

% 由于我国饮食结构不同于西方国家,许多孕妇及乳母自身就缺钙,所以提倡女性在孕期和哺乳期就应注意钙的补充,多吃些含钙多的食物,如海带、虾皮、豆制品、芝麻酱等。牛奶中钙含量也是很高的,可以坚持每日喝500克牛奶,也可以补充钙片,多晒太阳以利于钙的吸收。如果乳母不缺钙,母乳喂养婴儿在3个月内可以不吃钙片,只需要从出生后3周开始补充鱼肝油,尤其是寒冷季节出生的宝宝。

% 如果是人工喂养的宝宝应在出生后两周就开始补充鱼肝油和钙剂。鱼肝油中含有丰富维生素A和维生素D,我们通常使用的是浓鱼肝油,开始时可每日1次,每次2滴。根据宝宝消化状况,如果食欲、大小便等无异常改变,可逐渐增至每日2次,每次2-3滴,平均每日5-6滴,维生素D的补充每日不能超过800国际单位,长期过量补充会发生中毒反应。如果是早产儿更应及时、足量补充。补充鱼肝油滴剂时,可以用滴管直接滴入宝宝口中。
% \end{quote}

% %ux7b2cux4e8cux8282-12ux4e2aux6708ux5a74ux513fux7684ux5582ux517bux6307ux5bfc}{%
% \subsection{02第二节
% 1〜2个月婴儿的喂养指导}%ux7b2cux4e8cux8282-12ux4e2aux6708ux5a74ux513fux7684ux5582ux517bux6307ux5bfc}}

% 满月起,宝宝进入一个快速生长时期,对各种营养需求也迅速增加。生长发育所需热能占总热量的25\%
% \textasciitilde{} 30\%,每天热量供给约需397.7千焦/千克体重。

% %ux4e00ux6839ux636eux5b9dux5b9dux9700ux6c42ux8fdbux884cux5582ux54fa}{%
% \subsection{一、根据宝宝需求进行喂哺}%ux4e00ux6839ux636eux5b9dux5b9dux9700ux6c42ux8fdbux884cux5582ux54fa}}

% 宝宝出生的头几个星期里,母婴之间要建立起恰如其分的喂养方式,宝宝要以频繁的吸吮来刺激母亲乳汁分泌。宝宝吃得越频繁,乳汁分泌量越旺盛。在大约三个星期和六个星期,宝宝会经历``\textbf{猛长期}'',需要的营养比平常多,也会通过频繁吸吮来提高母乳分泌量。这是大自然安排好的供需关系,因此母亲要在宝宝需要时及时喂奶。

% \hspace{0pt}\includegraphics[width=2.76923in,height=2.6014in]{media/rId542.png}\hspace{0pt}

% 每一个宝宝都有其天生的\textbf{独特性格},首先就表现在吃奶方式上。有的宝宝吃得\textbf{又快又猛},吃饱后可以等待两三个小时再吃下一顿。有的宝宝却吃得\textbf{温柔缓慢},一顿奶要吃上半个小时甚至一个小时,有时还会吃着吃着就睡着了,过一会儿醒来又要吃。而且大多数宝宝在吃饱肚子后还会在奶头上耽搁一阵,获得\textbf{娱乐性吸吮}。新生儿的胃口也随时有所变化,有时吃得多,有时需要的少。每一位妈妈的乳汁都是为自己宝宝的独特性而设计的,根据宝宝不同的需要情况,每一次喂奶时,乳汁的分泌量、浓度和成分都有所调节。因此哺乳妈妈要按照自己宝宝的需要来喂奶。

% 母乳较奶粉更容易消化和吸收。母乳喂养的宝宝需要喂食的次数也自然多一些。

% 作为新手妈妈,最好做好思想准备。你的宝宝在最初几个星期里,基本上要平均每两个小时吃1次奶,或者是在24小时内吃10\textasciitilde12次奶。一般3个月以后,乳汁分泌量达到宝宝要求。在按需喂养基础上,母婴之间建立起令双方满意的喂养关系,宝宝也会接受频繁吃奶之外的安抚方式。至于多长时间喂1次、每次喂多久这样的疑问就不会再出现了。

% 就像一位母亲所说的那样:``成功地分泌乳汁是每一位女人女性气质的自然表现,她不需要计算给宝宝喂奶的次数,\textbf{就像她不需要计算亲吻宝宝的次数一样}。''

% %ux4e8cux591cux95f4ux54faux4e73}{%
% \subsection{二、夜间哺乳}%ux4e8cux591cux95f4ux54faux4e73}}

% \hspace{0pt}\includegraphics[width=2.71329in,height=2.44755in]{media/rId546.png}\hspace{0pt}

% 经常听到一些妈妈抱怨,宝宝夜里总要醒来吃几次奶,弄得自己疲惫不堪。有些妈妈担心这样下去,会影响宝宝睡眠。也有些妈妈十分厌烦夜间被宝宝吵醒,干脆发出疑问``怎样才能改掉这个坏毛病?''在这些妈妈心目中,让宝宝睡一整夜觉不醒来,是最理想的。

% 有关夜间哺乳,可以分几个阶段来看。

% %ux4e2aux6708ux751fux957fux53d1ux80b2ux8fc5ux731b}{%
% \subsubsection{1-5个月,生长发育迅猛}%ux4e2aux6708ux751fux957fux53d1ux80b2ux8fc5ux731b}}

% 宝宝出生后头几个月里,由于生长迅猛,夜里起码要起来吃1次奶。很早的一项研究表明,即使10个月大的宝宝,对于母乳的汲取也有25\%是在夜间。另外,如果夜里间隔5-6个小时不喂奶,妈妈乳房会因胀奶而肿胀,到了早晨再喂时,宝宝会叼不住奶头。这样也会引起乳汁分泌量下降。

% 应对宝宝夜间吃奶的一个妙方就是学会\textbf{躺着喂奶}。有些妈妈在一开始可能不大习惯,需要尝试各种不同的卧姿,背后多垫几个枕头。一旦找到适合自己的姿势,做母亲的辛劳就被大大缓解了。

% %ux4e2aux6708ux540eux5988ux5988ux4e0aux73edux4e86}{%
% \subsubsection{6个月后妈妈上班了}%ux4e2aux6708ux540eux5988ux5988ux4e0aux73edux4e86}}

% 一般在宝宝6-10个月时,妈妈的产假就会结束,需要重返职场。这个时期的宝宝,已经结束纯母乳喂养阶段,在\textbf{6个月左右开始添加辅食}。但是母乳仍然是宝宝最主要的营养来源。有能力和条件的母亲,应该将母乳挤出来,交给白天的看护人喂给宝宝。这个时期,也最容易出现宝宝在夜间频繁醒来吃奶的现象。

% 可能不少妈妈都遇见过这样的情况:下班回家后,看护人很得意地向你汇报:``这宝宝可乖啦,上午一觉睡三个小时,下午一觉睡四个小时,一点儿麻烦都没有。''但是入夜之后,麻烦来了,宝宝一夜醒七八回要吃奶。本来上班就很辛苦,夜里也不得安宁。这是为什么呢?

% 原来,宝宝是在故意将白天黑夜睡颠倒了,把他珍贵的清醒时间留到夜里跟妈妈亲近的时候。你可能觉得这很烦,但是要记住,宝宝所做的一切都有原因,宝宝不是成心跟你过不去,而是在告诉你:``妈妈,我要更多跟你在一起的时间!''

% 解决这个问题的最佳方法就是和宝宝同床而眠,享受和宝宝亲密无间的夜晚。你会发现,宝宝醒来的次数很快就减少了,而且即使醒来,你也不用费什么劲就能满足他的需求。夜间哺乳对于保持乳汁分泌量来说十分重要。因为你白天一整天都不在家,喂不了宝宝,乳汁产量会因此下降。

% 千万不要因为怕夜里喂奶而采取断奶这种办法,因为你白天已经没有和宝宝亲密接触了,再不喂奶,就会失去建立牢固亲情关系的大好机会。不用担心夜间哺乳会影响你的睡眠,其实恰恰相反,\textbf{母乳中含有天然催眠成分},有利于宝宝睡眠。夜间哺乳能提高妈妈体\textbf{内有镇静作用的荷尔蒙}的增加,你也会更加放松。

% %ux4e09ux600eux6837ux4e3aux5b9dux5b9dux9009ux62e9ux5976ux7c89}{%
% \subsection{三、怎样为宝宝选择奶粉}%ux4e09ux600eux6837ux4e3aux5b9dux5b9dux9009ux62e9ux5976ux7c89}}

% 有的妈妈生下宝宝后,由于各种原因造成奶水缺乏,不能用母乳喂哺自己的宝宝,心里感到十分内疚,担心宝宝的健康会受到影响,因此就想不惜代价给宝宝选择最好的奶粉。是不是最贵的奶粉就一定是最好的奶粉呢?不一定。选择奶粉时要考虑奶粉的质量、宝宝的适应程度及家庭经济\textbf{承受能力}。

% \hspace{0pt}\includegraphics[width=2.64336in,height=2.44755in]{media/rId552.png}\hspace{0pt}

% 现在市场上宝宝奶粉在价格上有高、中、低三个档次,最贵的是进口奶粉,每450克在350元左右。中等奶粉多为合资产品,每450克在100元左右。国产名牌奶粉价格仅十几元。如果按一个宝宝平均每周吃完一大罐奶粉(450克)计算,选择不同奶粉对家庭来说经济负担就会相差很多。选择进口奶粉一个月需1000多元,选合资奶粉一个月500多元,选国产奶粉一个月则只需不到百元。

% 这些奶粉在营养配比上差距不大,国内一些知名品牌的婴幼儿奶粉各项指标均达到标准。在这些婴幼儿配方奶粉中,不但有适合不同年龄段婴幼儿食用的采用不同配方生产的产品,而且还有高乳糖配方和低乳糖配方之分。高乳糖配方奶粉乳糖含量达50\%,接近母乳,适合大部分婴幼儿食用。低乳糖奶粉乳糖含量20\%左右,适合那些喝牛奶会出现腹胀、腹泻等不适的宝宝食用。以上奶粉可以根据家庭经济状况不同进行选择。

% 此外,不论何种奶粉,只要宝宝吃了能够健康生长发育,不会引起过敏等不良反应,就是适合宝宝的奶粉。应注意,奶粉\textbf{不宜经常更换,否则可能引起宝宝消化系统紊乱}。

% \begin{quote}
% gpt:

% 婴儿的消化系统是非常敏感的,它们需要时间去适应新的食物。如果经常更换奶粉,婴儿的肠道可能会因为不断适应新的食物成分而出现问题。

% 不同品牌和类型的奶粉,其成分和配方可能会有所不同。例如,某些奶粉可能会含有更多的蛋白质,某些可能含有特定的预生物或益生菌,而另一些可能会添加特定的脂肪酸。这些差异可能会影响宝宝的消化过程,如果经常更换,可能会导致宝宝的肠道需要不断调整以处理不同的奶粉,这可能导致腹泻、便秘、胀气或其他消化问题。

% 此外,如果宝宝对某个奶粉的成分过敏,而这个成分在另一个奶粉中并不存在,频繁更换可能会增加宝宝暴露于过敏原的机会,从而引发过敏反应。

% 因此,除非医生建议或宝宝对当前的奶粉有不良反应,否则不建议经常更换奶粉。如果需要更换,建议逐渐切换,让宝宝的消化系统有足够的时间适应新的奶粉。
% \end{quote}

% 在选购婴幼儿配方奶粉时,既不要对国内知名品牌奶粉质量不信任,不惜花高价购买进口奶粉,也不要过分考虑价格因素,忽视质量因素,选购价格便宜、质量没有保证的奶粉。总之,质量高、价格合理又适合宝宝年龄、身体状况的奶粉才是最佳选择。

% %ux56dbux7ed9ux5a74ux513fux8c03ux5236ux5976ux7c89}{%
% \subsection{四、给婴儿调制奶粉}%ux56dbux7ed9ux5a74ux513fux8c03ux5236ux5976ux7c89}}

% %ux5f00ux542fux5a74ux513fux914dux65b9ux5976ux7c89}{%
% \subsubsection{1.
% 开启婴儿配方奶粉}%ux5f00ux542fux5a74ux513fux914dux65b9ux5976ux7c89}}

% \textbf{罐装}:罐装铝箔的封口设计有易拉型和非易拉型两种。使用非易拉型铝箔罐装婴幼儿配方奶粉时要注意:

% 不要用剪刀在铝箔上来回划动,这样容易使铝箔屑掉入奶粉中,开罐前应检查开启工具是否完好无损,是否洁净;开启后,应该将所有的铝箔撕干净,不要残留在罐上,尽量用力均匀防止铝箔破损脱落铝箔碎屑。

% \textbf{袋装}:用剪刀沿封口不重复剪开,以免将包装屑掉入奶粉中。

% %ux600eux6837ux51b2ux8c03ux5976ux7c89}{%
% \subsubsection{2.
% 怎样冲调奶粉}%ux600eux6837ux51b2ux8c03ux5976ux7c89}}

% 先洗净双手,事先准备好冲配奶粉的各种用具,取出消毒过的奶嘴、奶瓶与奶粉和冲泡奶粉所需的水;根据要泡的奶量,取准备好的50\textasciitilde60度的热水\textbf{2/3}量倒入奶瓶中;用配方奶粉附带的小匙,按照说明加入适量奶粉;晃动奶瓶,让配方奶粉充分溶化,不要结块;剩余\textbf{1/3}热水加入奶瓶中,把奶瓶放平,通过刻度察看是否够量;盖上奶瓶盖后再轻轻晃动一次,直至配方奶粉彻底溶化。

% \begin{quote}
% \textbf{小贴士}

% 晃奶瓶时不要太用力,以免起泡沫,使奶液溢出瓶外。冲完奶粉后,将小匙消毒并放置在干净的容器里,不要直接放入奶粉罐。因为手的细菌可能粘在小匙上,如果用后仍放在奶粉罐里,容易污染奶粉。

% 专家提醒

% \textbf{奶粉浓度要适宜}

% 在冲调奶粉时要避免两个误区,一是冲成的牛奶太浓,这样会造成婴儿消化不良,且过多的蛋白质在体内会产生大量的尿酸和尿素,加重了肾脏的负担。在夏天,过浓的牛奶还可能导致脱水、发热,称为蛋白热。另一个是配制的牛奶太稀,长期食用后会导致婴儿营养不良。因此,奶粉一定要冲调得浓淡适宜。
% \end{quote}

% %ux6c34ux7684ux8981ux6c42}{%
% \subsubsection{3. 水的要求}%ux6c34ux7684ux8981ux6c42}}

% 冲调奶粉的水必须清洁,了解当地饮用水是否含任何可能损害婴儿健康的物质,如铅、硝酸盐、细菌、杀虫剂或其他化学物质。

% 水要经过消毒:把冷水煮沸1\textasciitilde2分钟,然后凉至适当的温度。将水滴至你的\textbf{手腕内侧},感觉与体温差不多即可。【内侧是不是比外侧更灵敏】

% 煮沸时间不宜超过5分钟,因为它可能使水中的铅和硝酸盐浓缩。

% 使用自来水最好让它流出2分钟后接水,会使水中的铅含量减少。

% 冲奶粉的水一定不要用开水,水温过高会使奶粉中的乳清蛋白产生凝块,影响消化吸收。某些对热不稳定的维生素将被破坏,免疫活性物质会被全部破坏。应正确冲调,避免营养物质损失。

% %ux5582ux54fa}{%
% \subsubsection{4 喂哺}%ux5582ux54fa}}

% 一定按照说明书的浓度来冲调,过浓或过淡都会影响宝宝的健康;不要让宝宝含奶瓶睡觉,奶中糖类会腐蚀宝宝的牙齿,易造成蛀牙;养成每餐固定间隔及固定时间的良好喂食习惯;注意奶嘴洞口大小及宝宝吸吮的用力程度,以免吸入过多空气;随着奶逐渐减少,要注意增加奶瓶倾斜度,以免吸入太多空气,喂奶时奶嘴常出现扁缩,阻塞出孔,影响奶汁流出。此时把奶瓶盖松开一点,让空气进入瓶内,即可解决奶嘴扁缩的现象。

% \begin{quote}
% 小贴士

% 宝宝喝完奶后,应竖着抱起并让小脑袋搭在肩上,轻轻拍拍小后背,好把吃奶时吞进胃里的空气排出去。

% 每次吃剩下的奶一定要倒掉,不能留到下一餐再吃。因为,牛奶是很好的细菌培养基,可能导致宝宝腹泻、食物中毒。
% \end{quote}

% %ux6e05ux6d17ux5976ux5177}{%
% \subsubsection{5. 清洗奶具}%ux6e05ux6d17ux5976ux5177}}

% 把冲泡奶粉及喂奶时所用的用具都浸泡在热水中,水一定要覆盖住所有用具。

% 轻轻旋转奶瓶,用奶瓶专用刷彻底清洗每一个部分,玻璃奶瓶宜用尼龙奶刷,奶瓶要用海绵奶刷,最后用清水冲净奶瓶。

% 清洗奶嘴时先把奶嘴反面翻开,并用奶嘴刷清洗;然后刷洗奶嘴外面,注意奶嘴孔是否通畅,刷洗完后用清水冲干净。

% \textbf{注意:靠近奶嘴处的地方较薄,清洗时要小心裂开。}

% %ux6d88ux6bd2ux5976ux5177}{%
% \subsubsection{6. 消毒奶具}%ux6d88ux6bd2ux5976ux5177}}

% \textbf{煮沸消毒}:洗净的奶瓶放入加水的消毒锅里,水煮沸后再放入奶瓶盖、奶嘴煮15\textasciitilde20分钟。文火煮,以免水煮干,使塑胶融化。

% \textbf{蒸汽消毒}:可使用普通蒸汽锅或使用电气式蒸汽消毒器。使用前者消毒需要加热10分钟,使用后者消毒需要加热5分钟。

% \textbf{注意:用消毒镊子把消毒过的喂奶用具一个个取出,放在干净地方风干。然后放进一个清洁有盖容器中存放,准备下次再用。}

% \begin{quote}
% 你知道吗?

% \textbf{沖好的奶吃不完时怎么办?}

% 父母常遇到的另一个问题是,由于婴儿每次吃的奶量并不完全一致,所以经常会发生冲好的奶吃不完的情况,这时该怎么办呢?

% 一般地说,如果这瓶奶已冲好但婴儿由于种种原因没有吃,那么可以将之密封,暂时贮存在冰箱的冷藏室中,但贮存的时间不得超过24小时。如果这瓶奶已给婴儿吃过,那么应记录残余奶量后,将多余的奶倒掉,因为婴儿的口腔中有细菌,而牛奶是细菌的良好培养基,细菌会不断地繁殖。记录残余奶量是为了知道婴儿到底吃了多少,方便下次冲调牛奶时掌握好奶量。
% \end{quote}

% %ux4e94ux5976ux74f6ux4e0eux5976ux5634ux7684ux9009ux62e9}{%
% \subsection{五、奶瓶与奶嘴的选择}%ux4e94ux5976ux74f6ux4e0eux5976ux5634ux7684ux9009ux62e9}}

% %ux5976ux74f6ux7684ux9009ux62e9}{%
% \subsubsection{1. 奶瓶的选择}%ux5976ux74f6ux7684ux9009ux62e9}}

% 目前市场上销售的奶瓶制作材料可分为两种:合成树脂和玻璃。

% 合成树脂制的奶瓶轻,不易碎,适合外出及较大宝宝自己拿着用,但不耐磨、不耐洗。玻璃奶瓶则正好相反,更适合妈妈拿着喂宝宝。

% 奶瓶依容量分为大、中、小三号,母乳喂养的宝宝喝水时最好用小号,储存母乳可用大号的。用其他方式喂养的宝宝则应用大号的,让宝宝1次吃饱。

% %ux5976ux5634ux7684ux9009ux62e9}{%
% \subsubsection{2. 奶嘴的选择}%ux5976ux5634ux7684ux9009ux62e9}}

% 奶嘴有橡胶、乳胶和硅胶三种材料制成的,最常见的材料是乳胶和硅胶。乳胶奶嘴富有弹性,质感近似妈妈的乳头。硅胶奶嘴没有乳胶的异味,容易被宝宝接纳,而且不易老化,抗热、抗腐蚀性较好。宝宝\textbf{吸奶时间应为20\textasciitilde30分钟},时间太长或太短都不利于宝宝口腔的发育,因此选择合适的奶嘴型号非常重要。

% 常见的奶嘴型号如下:

% \begin{itemize}
% \item
%   圆孔小号:适合于尚不能控制奶量的新生儿。
% \item
%   圆孔中号:适合于2\textasciitilde3个月,用小号奶嘴费时太长的宝宝。用此奶嘴吸奶与吸吮妈妈乳房时所吸出的奶量所做的吸吮运动次数非常接近。
% \item
%   圆孔大号:适合于用以上两种奶嘴喂奶时间太长,吸奶量不足而致体重过轻的宝宝。
% \item
%   ``Y''字形孔:适合于可以自我控制吸奶量,边喝边玩的宝宝使用。
% \item
%   ``十''字形孔:适合于吸果汁、米糊或其他粗颗粒饮品时,也可以用来吃奶。
% \end{itemize}

% 在选购奶嘴时,应特别注意奶嘴的说明书,看看奶嘴里的亚硝胺和双酚A的含量。因为前者是致癌物质,后者可导致性早熟。

% %ux516dux6d4bux8bd5ux725bux5976ux7684ux6e29ux5ea6}{%
% \subsection{六、测试牛奶的温度}%ux516dux6d4bux8bd5ux725bux5976ux7684ux6e29ux5ea6}}

% 牛奶温度过高会烫伤婴儿,过低会刺激胃肠道蠕动,造成腹泻,影响营养素的吸收。在实际生活中,可以采用下列简便方法测试牛奶的温度。

% \begin{enumerate}
% \def\labelenumi{\arabic{enumi}.}
% \item
%   用手腕感觉:\textbf{手腕的温度感觉比手背灵敏得多},所以可以将牛奶先滴几滴在手腕上试试,如果手腕部皮肤感到奶滴不冷不热或略微偏热,说明牛奶温度与体温相近,奶温是合适的。
% \item
%   用面颊感觉:把盛有牛奶的奶瓶摇匀,片刻后贴在面颊上,如果不感到烫或冷,说明与体温相近,可以用来哺喂。以上方法十分简便,父母们不妨一试。需要注意的是,成人不要用嘴去尝牛奶,这样会将口腔中的细菌带到奶嘴上。
% \end{enumerate}

% \begin{quote}
% 育儿百科

% \textbf{孩子为什么经常吐奶}

% 有的孩子生后就有吐奶的毛病,到第2个月还是经常吐奶,有的吃完一会儿就吐,有的吃完奶20分钟左右吐,这是为什么呢?原来,人的胃有两个口,上口叫贲门,下口叫幽门。贲门和食管相连接,幽门和十二指肠相连接。孩子在生长中,贲门肌发育较松弛,而幽门肌容易痉挛。孩子吐出的奶呈\textbf{豆腐脑状},这是奶蛋白在胃酸作用下形成乳块的结果。

% 对常常吐奶的孩子要少喂一些,喂奶以后要多抱一会儿,抱的姿势是使孩子上半身直立,趴在大人肩上,然后用手轻轻拍打孩子背部,直到孩子打嗝将胃内所含的空气排出为止。这时轻轻把孩子放在床上,枕部高一些,向右侧卧,这样可以减少吐奶。吐奶是生理现象,随着年龄的增长,身体不断发育会自行缓解。如果吐奶频繁且呈喷射状,吐出的除了乳块还伴有黄绿色液体及其他东西,一定不要忽视,要及时到医院检查。

% \hspace{0pt}\includegraphics[width=2.04196in,height=2.68531in]{media/rId573.png}\hspace{0pt}
% \end{quote}

% %ux4e03ux5b9dux5b9dux7684ux6bcdux4e73ux66ffux4ee3ux54c1ux6709ux54eaux4e9b}{%
% \subsection{七、宝宝的母乳替代品有哪些}%ux4e03ux5b9dux5b9dux7684ux6bcdux4e73ux66ffux4ee3ux54c1ux6709ux54eaux4e9b}}

% 对于宝宝来说,没有一种食品比得上母乳更优越。如果母亲缺乏乳汁或者由于某种原因不能给宝宝按时哺乳,就只能用母乳替代品了。

% %ux5a74ux513fux914dux65b9ux5976ux7c89}{%
% \subsubsection{1.婴儿配方奶粉}%ux5a74ux513fux914dux65b9ux5976ux7c89}}

% 婴儿配方奶粉是专为婴儿生产的替代母乳的奶粉。婴儿奶粉有两个配方,都是以牛奶为主,在牛奶的基础上进行了改良,使之更适宜宝宝生长发育需要。其配方一号叫婴儿奶粉,配方二号叫母乳化奶粉。

% \textbf{婴儿奶粉(配方一号):}在牛奶基础上添加了大豆和饴糖,并强化了铁、维生素等,经过喷雾干燥法制成干粉便于保存和携带,也易于消化。

% \textbf{母乳化奶粉(配方二号)}:是在牛奶基础上加入适量脱盐乳清粉、矿物质(铁、锌等)和维生素(A、D、C、E等),使之成分更接近于母乳。

% 配方奶粉比牛奶的蛋白质、脂肪、碳水化合物的比例更为合理,更适合宝宝的需要。其中还添加了牛奶中缺乏的铁、锌及维生素A、维生素D等,有利于宝宝的生长发育,是缺乏母乳者的良好替代品。

% 目前市售的婴儿奶粉品种很多,其中有许多为进口奶粉,与国产奶粉相比,进口奶粉价格要高出许多,但并不是每一种进口奶粉的质量都与价钱成正比的。有些奶粉的口感很差,有些还可能存在质量问题。所以,在选购时,要看清有无经过我国进出口食品检疫,有无出厂日期、成分表,不要盲目购买。当然,不少进口奶粉的工艺较好,溶解快,易于调配。

% %ux5a74ux513fux7c73ux7c89}{%
% \subsubsection{3.婴儿米粉}%ux5a74ux513fux7c73ux7c89}}

% 婴儿米粉是根据婴儿生长发育需要而研制的以谷类(大米、面粉)为主的宝宝食品,常见的有糕干粉、婴宝、赐你力、赐你爱、亨氏米粉等。其主要成分均是谷类,其中加有黄豆粉、蛋黄粉、蔗糖、植物油、维生素、矿物质(食盐、钙、磷、铁等),有一些米粉中加入乳类。

% 在选用米粉时要仔细阅读其成分表,看清都强化了什么,强化量是多少。如铁,按国家标准强化的,每100克中含铁为6-10毫克,包括婴宝和赐你爱、蛋黄宝宝米粉等。也有特殊强化的,例如血宝和亨氏米粉,其中血宝为每100克含铁20毫克,亨氏米粉中每100克含铁40毫克等。如果宝宝本身就喝着强化铁的奶粉,就不必再吃特殊强化铁的米粉。如果是吃母乳或喝牛奶,可在满4个月后选用特殊强化铁的米粉,但\textbf{不宜长期食用,以防铁过量。}在宝宝4-12个月期间,父母可以将标准强化米粉和特殊强化米粉搭配起来给宝宝吃,既避免了宝宝食欲下降和铁过量,又可防止缺铁。

% 宝宝米粉较之奶粉便宜,但对缺乏母乳的小宝宝而言,宝宝配方奶粉更适宜充当``主食'',而宝宝米粉更适宜作为添加辅食。

% \begin{quote}
% \textbf{专家提示}\\
% 孩子一般从生后第15天开始服用鱼肝油和钙片,浓鱼肝油滴剂每次1-2滴,每天3次。钙片1-2片,每天3次。
% \end{quote}

% %ux516bux5a74ux513fux4e00ux5929ux6240ux9700ux7684ux5976ux91cf}{%
% \subsection{八、婴儿一天所需的奶量}%ux516bux5a74ux513fux4e00ux5929ux6240ux9700ux7684ux5976ux91cf}}

% 一般地说,婴儿全天所需的奶量大约是体重的1/10。人工喂养的宝宝根据孩子的食欲情况而定奶量。一般全天的奶量在500
% \textasciitilde{} 750毫升,按每天喂6次计算,每次喂75 \textasciitilde{}
% 125毫升,孩子的活动量不同,每个孩子的食量也不同,这要根据每个孩子的具体情况确定,不能强求一致。母亲平时要注意孩子的大便情况,体重增长情况,孩子的精神状态等等。人工喂养的孩子与母乳喂养的孩子不同,不要孩子一哭就以为是饿了,马上喂奶,要养成按时喂养的好习惯。

% 每日喂奶的时间可以安排在早晨5:00,上午9:00,中午13:00,下午17:00,晚上21:00,夜间1:00。白天在两次喂奶中间,应加喂蔬菜水、鲜果汁水,每次25\textasciitilde50毫升。

% 当然,每个婴儿的胃口不一样,在这个计算方法的基础上适当增减是正常的。检查奶量是否合适的最好方法是每月称重一次,如果体重能按正常速度增加,则说明喂养得当。

% 2个月宝宝1日营养计划

% \begin{longtable}[]{@{}
%   >{\raggedright\arraybackslash}p{(\columnwidth - 2\tabcolsep) * \real{0.5000}}
%   >{\raggedright\arraybackslash}p{(\columnwidth - 2\tabcolsep) * \real{0.5000}}@{}}
% \toprule()
% \begin{minipage}[b]{\linewidth}\raggedright
% 时间
% \end{minipage} & \begin{minipage}[b]{\linewidth}\raggedright
% 食物类型
% \end{minipage} \\
% \midrule()
% \endhead
% 早上6:00 & 母乳或配方奶80-100毫升 \\
% 上午9:00 & 母乳或配方奶80-100毫升 \\
% 中午12:00 & 母乳或配方奶80-100毫升 \\
% 下午3:00 & 母乳或配方奶80-100毫升 \\
% 下午6:00 & 母乳或配方奶80-100毫升 \\
% 晚上9:00 & 母乳或配方奶80-100毫升 \\
% 夜间12:00 & 母乳或配方奶80-100毫升 \\
% 鱼肝油 & 每日1次,参照说明或遵医嘱 \\
% 温开水 & 人工喂养的宝宝在白天两次喂奶中间适量添加 \\
% \bottomrule()
% \end{longtable}

% %ux4e5dux7ed9ux5b9dux5b9dux559dux852cux83dcux6c34ux548cux679cux6c41}{%
% \subsection{九、给宝宝喝蔬菜水和果汁}%ux4e5dux7ed9ux5b9dux5b9dux559dux852cux83dcux6c34ux548cux679cux6c41}}

% 蔬菜水和果汁是给孩子增加维生素C的主要食品,由于各种乳品中维生素C的含量都不多,即便鲜牛奶,在煮沸过程中,所含的维生素损耗也很大,所剩无几,奶粉类食品更不必说了。所以蔬菜水与鲜果汁就成了给婴儿补充体内所需的维生素的最好食品。

% %ux5341ux9c9cux679cux6c41ux5236ux4f5cux65b9ux6cd5}{%
% \subsection{十、鲜果汁制作方法}%ux5341ux9c9cux679cux6c41ux5236ux4f5cux65b9ux6cd5}}

% 如果家里有榨果汁机的话,可将橘子、广柑等水果\textbf{洗净去皮},加工后\textbf{去渣},加水和少量白糖,放入奶瓶中喂孩子。

% \textbf{番茄汁制作法}:将熟透的番茄洗净,放在开水中\textbf{烫}一下,\textbf{去皮}切碎,用干净的\textbf{纱布}包好,用力挤\textbf{压},使鲜汁流出,加少许白糖和水,用小匙喂孩子喝。

% \textbf{蔬菜水制作方法}:取少许新鲜蔬菜,如菠菜、油菜、胡萝卜、白菜等,洗净切碎,放入小锅中,放少量水煮沸,再煮3\textasciitilde5分钟,菠菜可少煮一会儿,胡萝卜可多煮一会儿,放置到不烫手时,将汁倒出,加少量白糖,放入奶瓶给孩子食用。给孩子饮用果汁、菜汁时,开始时量要小。加水要多,孩子适应之后,逐渐增加浓度。

% %ux5341ux4e00ux54faux4e73ux5988ux5988ux7684ux996eux98df}{%
% \subsection{十一、哺乳妈妈的饮食}%ux5341ux4e00ux54faux4e73ux5988ux5988ux7684ux996eux98df}}

% 随着哺乳妈妈的热能及营养素的需要量增加,加之想让乳汁分泌旺盛、乳汁营养成分良好,妈妈需要每天多吃几餐,以4\textasciitilde5餐为宜。特别需要注意的是,要注意多喝一些\textbf{催乳}汤类,如排骨汤、鸡汤、猪蹄汤、豆腐汤、青菜汤等。如果出现乳汁分泌很少,千万不要灰心,不要轻易放弃母乳喂养,可以采用一些催乳餐来促进乳汁分泌。

% %ux9ca4ux9c7cux50acux4e73ux7ca5}{%
% \subsubsection{鲤鱼催乳粥}%ux9ca4ux9c7cux50acux4e73ux7ca5}}

% 材料:活鲤鱼1条,粳米、小米各20克。

% 做法:将鲤鱼去除内脏后切成小块,与粳米或小米一起煮粥,但粥里不要放盐,\textbf{淡食}催乳效果较佳。

% 功效:鲤鱼富含蛋白质,具有开胃健脾、消除寒气、催生乳汁之功效。因此,既能补养身体又能催乳。如果用鲤鱼煮汤,不放盐,吃肉喝汤,催乳的效果也不错。

% %ux9cabux9c7cux50acux4e73ux7ca5}{%
% \subsubsection{\hspace{0pt}鲫鱼催乳粥}%ux9cabux9c7cux50acux4e73ux7ca5}}

% 材料:活鲫鱼500克,通草5克。

% 做法:将鲫鱼去除内脏后,加通草煮成汤。吃鱼喝汤,每天2次,连喝3\textasciitilde5天。有很好的催乳效果。

% 功效:鲫鱼能补虚、利水、顺气,具有通乳功效;通草可以通气下乳。鲫鱼与通草搭配煮汤,不仅催乳效果更佳,而且有利于妈妈身体复原。注意汤要清淡。

% %ux732aux8e44ux50acux4e73ux7ca5}{%
% \subsubsection{猪蹄催乳粥}%ux732aux8e44ux50acux4e73ux7ca5}}

% 材料:猪蹄1只,通草3克。

% 做法:将猪蹄剁成小块,和通草一同放入砂锅里,加清水煮汤。先用急火,水开后改成文火,煮1-2个小时。每天用1只猪蹄煮汤,每天喝2次,连续喝3-5天。

% 功效:猪蹄含有丰富的蛋白质、脂肪,具有较强补血、活血作用。通草可利水、通乳汁。这两种食品搭配在一起食用,既可促进妈妈身体尽快恢复,通乳效果也非常好。

% %ux732aux9aa8ux50acux4e73ux6c64}{%
% \subsubsection{猪骨催乳汤}%ux732aux9aa8ux50acux4e73ux6c64}}

% 材料:新鲜猪骨(腔骨、排骨、腿骨均可)500克,通草6克。

% 做法:将猪骨剁成小块,和通草一起放入砂锅中加2000克水煮1-2小时,煮至猪骨汤约1小碗,加少许酱油调味。每天1次喝完,连续喝3-5天。

% 功效:猪骨具有补气、补血、生乳作用,有助于妈妈身体康复,加上通草后催乳效果更佳。

% %ux9c9cux7f8eux84b8ux86cb}{%
% \subsubsection{\texorpdfstring{\textbf{鲜美蒸蛋}}{鲜美蒸蛋}}%ux9c9cux7f8eux84b8ux86cb}}

% \textbf{材料}:鸡蛋6个,鲜姜50克,红糖、醋各适量。

% \textbf{做法}:鲜姜洗净,用刀拍松,切块。锅中水烧开,加入红糖、醋、姜块,煮5分钟,倒出,拣出姜块,凉凉姜糖水备用。鸡蛋打入大碗中搅散,倒入凉凉的姜糖水,搅匀,再倒入小碗中,入笼蒸10分钟即可。趁热分顿服食。

% \textbf{功效}:鸡蛋滋阴润燥,养血熄风,含有优质蛋白质及较多的钙、磷、铁、维生素A、维生素D等营养物质;红糖、生姜除了提供糖分外,还兼有活血祛痰,温中散寒的作用。此道菜能滋阴养血,祛痰散寒,可预防产后发生的风寒、淤血。

% 一日食谱推荐

% \begin{longtable}[]{@{}
%   >{\raggedright\arraybackslash}p{(\columnwidth - 2\tabcolsep) * \real{0.5000}}
%   >{\raggedright\arraybackslash}p{(\columnwidth - 2\tabcolsep) * \real{0.5000}}@{}}
% \toprule()
% \begin{minipage}[b]{\linewidth}\raggedright
% 时间
% \end{minipage} & \begin{minipage}[b]{\linewidth}\raggedright
% 推荐食物
% \end{minipage} \\
% \midrule()
% \endhead
% 8:00-9:00 & 面包、牛奶、鸡蛋、小米粥 \\
% 11:00-12:00 & 米饭、鸡汤、猪肝炒青菜、豆制品 \\
% 14:00-15:00 & 蛋糕、牛奶、水果 \\
% 18:00-19:00 & 馒头、猪蹄汤、炒青菜 \\
% 22:00-23:00 & 粥、青菜、鸡蛋 \\
% \bottomrule()
% \end{longtable}

% 每餐之间,可以适量吃一些\textbf{应季}水果。

% %ux5341ux4e8cux54faux4e73ux6bcdux4eb2ux8981ux591aux52a0ux4f11ux606f}{%
% \subsection{十二、哺乳母亲要多加休息}%ux5341ux4e8cux54faux4e73ux6bcdux4eb2ux8981ux591aux52a0ux4f11ux606f}}

% 怀孕和分娩,使孕妇全身发生了很大的变化。分娩后,机体逐渐恢复正常。这段恢复期需要6\textasciitilde8周,也就是俗称的``坐月子''。这时孕妇生殖系统的自然防御功能发生了变化,子宫内有未愈合的创面,产妇又因分娩过程中体力消耗而抵抗力降低。因此,要重视孕妇产后的护理,预防各种疾病,加速恢复产妇的身体健康。在产后恢复期要注意:

% \begin{enumerate}
% \def\labelenumi{\arabic{enumi}.}
% \item
%   室内安静清洁,保持空气新鲜,使产妇得到充分的休息,同时要避免受凉。
% \item
%   产后第2天即可开始做抬腿运动和仰卧起坐运动,并逐渐增加运动幅度,锻炼腹肌,加速恢复。
% \item
%   产后最初几天,以\textbf{易消化、高营养}、\textbf{不油腻}的食物为好,以后改为普通饮食。
% \item
%   产后3\textasciitilde4天可出现腹痛,为子宫复旧收缩引起,无需处理,可自行好转。如果疼痛较重,可按摩腹部或用止痛药。
% \item
%   保持外阴清洁,防止感染。
% \item
%   产后经阴道排出的血性液体,称为恶露,正常恶露有血腥味,但不臭。一般血性恶露3\textasciitilde7天,以后颜色渐淡,3周左右停止。如恶露有臭味,或血性恶露持续2周以上,应到医院就诊治疗。
% \item
%   产后8小时可开始哺乳,首次哺乳时间为5分钟,以后逐渐延长,每次15分钟左右,间隔4小时左右。
% \item
%   产后42天应到医院进行产后检查,了解全身恢复情况。
% \item
%   在``坐月子''期间,应避免性生活,因为产妇的子宫尚未完全恢复,阴道也可能因生产而出现裂伤,容易造成细菌感染。
% \item
%   注意避孕。特别是在哺乳期,千万不要认为产妇没有月经来潮就不会怀孕。产后最好选择避孕套、避孕药、药膜等方式避孕,因为口服避孕药会影响产妇泌乳,并影响母乳质量。
% \end{enumerate}

% %ux7b2cux4e09ux8282-23ux4e2aux6708ux5a74ux513fux7684ux5582ux517bux6307ux5bfc}{%
% \subsection{03第三节
% 2〜3个月婴儿的喂养指导}%ux7b2cux4e09ux8282-23ux4e2aux6708ux5a74ux513fux7684ux5582ux517bux6307ux5bfc}}

% 这个月,宝宝依然是快速生长时期,对各种营养需求也迅速增加。此阶段继续提倡母乳喂养。如果母乳量足,完全可以不添加其他配方奶。如果母乳不足或者由于妈妈体力不支,不能完全母乳喂养时,首先应当选择混合喂养,采取补授法。当补授法也不能坚持时,再采用代授法。

% 对2个月的宝宝仍应继续坚持母乳喂养。无条件哺乳的,仍应每隔4小时喂奶1次,每天喂6次,用配方奶粉喂养的宝宝奶量每次100毫升左右,即使吃得多的宝宝,全天总奶量也不能超过1000毫升。

% 浓鱼肝油仍每天3次,每次2滴,钙片每天2\textasciitilde3次,每次2片。

% 对混合喂养和人工喂养的宝宝,应酌量添加蔬菜和新鲜的果汁,用以补充牛奶在加工过程中损失的维生素C。一般每日2次,每次1\textasciitilde2匙,在喂奶间隙喂入。

% %ux4e00ux600eux6837ux4fddux8bc1ux6bcdux4e73ux5582ux517bux7684ux6210ux529f}{%
% \subsection{一、怎样保证母乳喂养的成功}%ux4e00ux600eux6837ux4fddux8bc1ux6bcdux4e73ux5582ux517bux7684ux6210ux529f}}

% %ux54faux4e73ux671fux95f4ux8425ux517bux8981ux8ddfux4e0a}{%
% \subsubsection{1.哺乳期间营养要跟上}%ux54faux4e73ux671fux95f4ux8425ux517bux8981ux8ddfux4e0a}}

% 在怀孕期间,妈妈的体内会额外增加2 \textasciitilde{}
% 3千克的脂肪来储存能量,为哺乳做好准备,其他维生素和微量元素也会有相应的储备。但对于漫长的哺乳过程来说,这些显然是不够的。及时地补充丰富的营养和能量是迫在眉睫的事情。

% \hspace{0pt}\includegraphics[width=2.65734in,height=2.36364in]{media/rId603.png}\hspace{0pt}

% 很多妈妈在分娩后的一个月内体重都会有较大的下降,所以需要吃热量丰富的饮食以维持体重。现在是崇尚美丽的时代,但妈妈们要注意,如果你想减轻体重,减轻的幅度应该限制在每周0.5千克以内,确保饮食中含有6276千焦以上的热量。每周体重下降超过0.5千克以上及严重限制热量的摄入将导致营养不良,从而损害到母亲的健康并会导致产奶不足。

% 喂奶期间的饮食需要含有更多的蛋白质、钙、维生素和更多的水分。最容易缺乏的营养成分包括维生素B6、维生素B1、叶酸、钙、锌和镁。乳汁中维生素B6、维生素B1和叶酸的水平与母亲的饮食或补充摄入直接相关。这些营养成分的摄入不足可减少母亲体内的储存,并且对母亲和宝宝的健康都将造成危害。另外要多饮水保持体内水分的充足。如果你观察到自己的尿液呈淡黄色,说明水分摄入是比较充足的,如果呈现深黄的颜色或尿量很少,就应该喝更多的水。

% %ux4e0dux8981ux7528ux5976ux74f6ux5582ux517bux4e0dux7528ux5b89ux6170ux5976ux5634}{%
% \subsubsection{2.
% 不要用奶瓶喂养,不用安慰奶嘴}%ux4e0dux8981ux7528ux5976ux74f6ux5582ux517bux4e0dux7528ux5b89ux6170ux5976ux5634}}

% 有些妈妈认为用奶瓶给宝宝喂奶,奶嘴开大一点,这样宝宝吃起来既不费力又节省时间,比吃妈妈奶头好多了。但是你知道吗,这样做可能会造成很严重的后果,甚至使母乳喂养失败。

% 奶瓶橡皮奶头的感觉和真正吸吮乳头的感觉是完全不一样的,在最初的几周内宝宝可能在一定程度上难以适应不同的吸吮方式和流量的差别。他可能发现奶瓶更容易,因为他不必怎么花力气就可以吃到乳汁,而相比起来,吸吮妈妈的乳头简直就是``费力不讨好'',并开始拒绝母乳喂养,这种情况也被称为``乳头错觉''。而让一个已经习惯了橡皮奶头的婴儿再改为吸吮乳头会很困难,这对于希望母乳喂养的母亲可是一个不小的打击,或者就是因为短短几天的偷懒,进行母乳喂养的伟大计划将付之东流。

% \begin{quote}
% v:由俭入奢易,由奢入俭难
% \end{quote}

% 因此,在母乳喂养开始时要避免使用奶瓶,不然会妨碍哺乳的实施,还会推迟妈妈的乳汁旺盛分泌。妈妈摄入的营养如果不能转化成乳汁分泌出去,还会导致产后塑身效果不佳。

% %ux4e0dux6dfbux52a0ux6bcdux4e73ux4ee5ux5916ux7684ux996eux54c1}{%
% \subsubsection{3.
% 不添加母乳以外的饮品}%ux4e0dux6dfbux52a0ux6bcdux4e73ux4ee5ux5916ux7684ux996eux54c1}}

% 补充瓶装配方奶、糖水和水不仅是不必要的,还具有一些不利的因素。配方奶和糖水可减弱宝宝的饥饿感,并\textbf{干扰}他吃奶的愿望,宝宝的饮食\textbf{习惯会被打乱}。而不能按时将乳房吸空,乳汁淤滞,会使乳腺分泌减少,甚至引起乳腺炎的发生。

% \begin{quote}
% 育儿小百科

% \textbf{不要滥服鱼肝油}

% 有的妈妈觉得鱼肝油是维生素A,多吃几滴只有好处没有坏处。殊不知维生素A或维生素D过量会造成中毒。孩子维生素A、维生素D急性中毒,可引起颅内压增高、头痛、恶心、呕吐、烦躁、精神不振、前囟隆起,常被误认为是患了脑膜炎。慢性中毒表现为食欲不好、发烧、腹泻、口角糜烂、头发脱落、皮肤瘙痒、贫血、多尿等。如发现以上症状,要停服鱼肝油,少晒太阳,立即到医院进行检查。
% \end{quote}

% %ux4e8cux8981ux7ed9ux5b69ux5b50ux591aux5582ux6c34}{%
% \subsection{二、要给孩子多喂水}%ux4e8cux8981ux7ed9ux5b69ux5b50ux591aux5582ux6c34}}

% 水是人体中不可缺少的重要部分,也是组成细胞的重要成分。人体中的新陈代谢,如营养物质的输送、废物的排泄、体温的调节、呼吸等都离不开水。水被摄入人体后,有1\%\textasciitilde2\%存在体内供组织生长的需要,其余经过肾脏、皮肤、呼吸、肠道等器官排出体外。水的需要量与人体的代谢和饮食成分相关,孩子的新陈代谢比成人旺盛,需水量也就相对要多。

% 三个月以内的婴儿肾脏浓缩尿的能力差,如摄入食盐过多时,就会随尿排出,因此需水量就要增多。母乳中含盐量较低,但牛奶中含蛋白质和盐较多,故用\textbf{牛乳喂养的孩子需要多喂一些水},来补充代谢的需要。

% 总之,孩子年龄越小,水的需要量就相对要多。一般婴幼儿每日每千克体重需要100\textasciitilde150毫升水,如5千克重的孩子,每日需水量是500
% \textasciitilde{} 750毫升,这里包括喂奶量在内。

% %ux4e09ux5a74ux513fux62d2ux54faux7684ux539fux56e0}{%
% \subsection{三、婴儿拒哺的原因}%ux4e09ux5a74ux513fux62d2ux54faux7684ux539fux56e0}}

% \begin{enumerate}
% \def\labelenumi{\arabic{enumi}.}
% \item
%   \textbf{奶头不适}:如人工喂奶的奶瓶奶头太硬,或上面的吸孔太小,吮乳费力,从而使婴儿厌吮。
% \item
%   \textbf{疾病}:婴儿患一些疾病,如消化道疾病,面颊硬肿时,均有不同程度地出现厌吮。
% \item
%   \textbf{鼻塞}:因为婴儿鼻塞后,就得用嘴呼吸,如果吮乳,必然妨碍呼吸,往往乍吮又止。【木木出现过】
% \item
%   \textbf{生理缺陷}:如唇、腭裂等生理缺陷,其吸吮困难,也会出现拒吮现象。
% \item
%   \textbf{口腔感染}:此因疼痛而害怕吮乳,原因是婴儿口腔黏膜柔嫩,分泌液少,口腔比较干燥,再加上不适当地擦拭口腔或饮料过热,常使婴儿的口腔发生感染。口腔感染后,吮奶时即可产生疼痛,从而出现拒吮。
% \item
%   \textbf{早产儿}:原因是其身体尚未发育完善,吸吮机能低下,故常表现出口含奶头不吮或稍吮即止现象。
% \end{enumerate}

% %ux56dbux7ed9ux5b69ux5b50ux8865ux5145ux7ef4ux751fux7d20}{%
% \subsection{四、给孩子补充维生素}%ux56dbux7ed9ux5b69ux5b50ux8865ux5145ux7ef4ux751fux7d20}}

% 维生素C主要来源于新鲜蔬菜和水果,因婴儿不能食入蔬菜,所以容易造成维生素C的缺乏。一般每100毫升母乳含维生素C
% 2\textasciitilde6毫克。但牛奶中维生素C含量较少,经过加热煮沸,又被破坏了一部分,就所剩无几了。【不能煮沸】

% 所以要注意给孩子增加一些绿叶菜汁、番茄汁、橘子汁和鲜水果泥等,这些食品中均含有较丰富的维生素C。维生素C在接触氧、高温、碱或铜器时,容易被破坏,因而给孩子制作这些食品要用新鲜水果蔬菜,\textbf{现做现吃},既要注意卫生,又要避免过多地破坏维生素C。

% %ux4e94ux852cux83dcux6c34ux548cux9c9cux679cux6c41ux7684ux6dfbux52a0ux65b9ux6cd5}{%
% \subsection{五、蔬菜水和鲜果汁的添加方法}%ux4e94ux852cux83dcux6c34ux548cux9c9cux679cux6c41ux7684ux6dfbux52a0ux65b9ux6cd5}}

% 对混合喂养和人工喂养的孩子,应适量添加蔬菜水和新鲜的果汁,用以补充牛奶加工过程中损失的维生素C。

% 喂果汁和菜汤的时间,可以选在\textbf{两次喂奶之间}、散步、洗澡后或口渴时,并且使用小勺。不仅能让宝宝品尝到母乳和配方奶以外的多种味道,也可以让宝宝\textbf{练习用小勺喝汤},这样以后吃辅食就容易多了。

% 原则上,饮用的果汁应该先用温开水稀释。因为直接喝浓缩的果汁会增加宝宝的身体负担,而且喝果汁过多,还会影响吃奶的量。所以,每日果汁饮用量不应超过30毫升(2大勺)。另外,喂奶时间还应该坚持3\textasciitilde4小时一次。

% %ux516dux5b66ux505aux51e0ux79cdux679cux83dcux6c41}{%
% \subsection{六、学做几种果菜汁}%ux516dux5b66ux505aux51e0ux79cdux679cux83dcux6c41}}

% \textbf{番茄汁}:将成熟的新鲜番茄洗净,用开水\textbf{烫}软后去皮切碎,放入果汁机中打成汁,过滤后将白糖放入汁中,再用适量温开水冲调后即可饮用。

% \textbf{黄瓜汁}:将黄瓜洗净去皮,切小块,放入果汁机中打成汁,过滤后盛入杯中即可。

% \textbf{橘子汁}:将橘子去核,放入果汁机中打成汁,过滤后盛入杯中加入适量水调匀即可。

% \textbf{苹果汁}:选用熟透的苹果洗净,去皮、核,切小块,放入果汁机中打成汁,过滤后盛入杯中即可。

% \textbf{甜瓜汁}:将甜瓜去皮和瓤后切成小块,放入果汁机中打成汁,过滤后盛入杯中即可。

% \textbf{橙汁}:取橙子1个,将外皮洗净,切成两半。分别置于挤汁器盘上旋转几次,果汁即可流入槽内,过滤后即成。每个橙子约得果汁40毫升。饮用时可加1倍水和少量糖。

% \textbf{西瓜汁}:将西瓜瓤100克放入碗内,用匙捣烂,再用纱布过滤即成。

% \textbf{胡萝卜水}:胡萝卜50克,清水50克。将胡萝卜洗净切碎,放入不锈钢锅(不要用铁、铝制品)内,加入水,上火煮沸2\textasciitilde3分钟,用纱布过滤去渣即可。

% \textbf{3个月宝宝1日营养计划}

% \begin{longtable}[]{@{}
%   >{\raggedright\arraybackslash}p{(\columnwidth - 2\tabcolsep) * \real{0.5000}}
%   >{\raggedright\arraybackslash}p{(\columnwidth - 2\tabcolsep) * \real{0.5000}}@{}}
% \toprule()
% \begin{minipage}[b]{\linewidth}\raggedright
% 时间
% \end{minipage} & \begin{minipage}[b]{\linewidth}\raggedright
% 食物类型
% \end{minipage} \\
% \midrule()
% \endhead
% 早上6:00 & 母乳或配方奶120\textasciitilde150毫升 \\
% 上午9:00 & 母乳或配方奶120\textasciitilde150毫升 \\
% 上午11:00 & 蔬菜汁或水果汁50毫升 \\
% 中午12:00 & 母乳或配方奶120\textasciitilde150毫升 \\
% 下午3:00 & 母乳或配方奶120\textasciitilde150毫升 \\
% 下午5:00 & 水果汁50毫升 \\
% 晚上6:00 & 母乳或配方奶120\textasciitilde150毫升 \\
% 晚上9:00 & 母乳或配方奶120\textasciitilde150毫升 \\
% 夜间12:00 & 母乳或配方奶120\textasciitilde150毫升 \\
% 鱼肝油 & 每日1次,参照说明或遵医嘱 \\
% \bottomrule()
% \end{longtable}

% %ux7b2cux56dbux8282-34ux4e2aux6708ux5a74ux513fux7684ux5582ux517bux6307ux5bfc}{%
% \subsection{04第四节
% 3〜4个月婴儿的喂养指导}%ux7b2cux56dbux8282-34ux4e2aux6708ux5a74ux513fux7684ux5582ux517bux6307ux5bfc}}

% 这一时期仍提倡纯母乳喂养。婴儿在这一时期里生长发育是很迅速的,食量增加。当然每个孩子因胃口、体重等差异,食入量也有很大差别。做父母的,不但要注意到奶量多少,而且还要注意奶的质量高低。母乳喂养要注意提高奶的质量,有的母亲只注意在月子中吃得好,忽略哺乳期的饮食或因减肥而节食,这是错误的。孩子要吃妈妈的奶,妈妈就必须保证营养的摄入量,否则,奶中营养不丰富,直接影响到婴儿的生长发育。\textbf{3个月是孩子脑细胞发育的第二个高峰期}(第一个高峰期在胎儿期第二十周),也是身体各个方面发育生长的高峰,营养的好坏关系到今后的智力和身体发育,因此一定要提高母乳的质量。

% %ux4e00ux5982ux4f55ux6dfbux52a0ux86cbux9ec4}{%
% \subsection{一、如何添加蛋黄}%ux4e00ux5982ux4f55ux6dfbux52a0ux86cbux9ec4}}

% 对3-4个月的孩子应该添加\textbf{含铁较丰富}、又能被婴儿消化吸收的食品,鸡蛋黄是最适合的食品之一。开始时将鸡蛋煮熟,取1/4蛋黄用开水或米汤调糊状,用小匙喂,以锻炼婴儿用匙进食的能力。婴儿食后无腹泻等不适后,再逐渐增加蛋黄的量,半岁后便可食用整个蛋黄了。人工喂养的婴儿,最好在第2个月开始加蛋黄,可将1/8个蛋黄加少许牛奶调为糊状,然后将一天的奶量倒入调好的糊中,搅拌均匀。煮沸后,再用文火煮5-10分钟,分次给孩子食用。如婴儿无不良反应,可逐渐增加一些蛋黄的量,直至加到1个蛋黄为止。应当注意的是,奶煮熟后放凉,要存入冰箱中,每次食用时都要煮开,以免孩子食入变质的牛奶引起不良的后果,另外不要随意增加蛋黄的食用量。【煮开这个...】

% %ux4e8cux4e73ux6bcdux5e94ux591aux5403ux5065ux8111ux98dfux54c1}{%
% \subsection{二、乳母应多吃健脑食品}%ux4e8cux4e73ux6bcdux5e94ux591aux5403ux5065ux8111ux98dfux54c1}}

% 孩子从出生到1周岁,宝宝的脑发育是很快的,几乎每月平均增长约1克。在头6个月内,平均每分钟增加约20万个脑细胞。也就是出生后第3个月是脑细胞生长的第二个高峰。所以为了宝宝的聪明程度,每个哺乳的妈妈一是要注意营养,以提高自己母乳的质量。

% 下面介绍几种供母亲食用的,有利于促进孩子健脑益智的食品:

% 动物脑、肝、鱼肉、鸡蛋、牛奶、大豆及豆制品、苹果、橘子、香蕉、核桃、芝麻、花生、榛子、各种瓜子、胡萝卜、黄花菜、菠菜、小米、玉米等。

% %ux4e09ux54faux4e73ux5988ux5988ux7684ux98dfux8c31}{%
% \subsection{三、哺乳妈妈的食谱}%ux4e09ux54faux4e73ux5988ux5988ux7684ux98dfux8c31}}

% \textbf{乌鸡白风汤}

% 材料:乌骨鸡1只,白凤尾菇50克

% 调味料:黄酒10克,葱姜各5克,味精、盐各适量。

% 做法:乌骨鸡去血、去毛、去内脏,洗净,白凤尾菇漂洗后撕成条。锅中清水加姜片煮沸,放入乌骨鸡,加入黄酒、葱节,用小火焖煮至酥,放入白凤尾菇,加味精、盐调味后,沸煮3分钟起锅,趁热饮汤食肉、菜。

% 功效:乌骨鸡有较强的滋补肝肾的功效。白凤尾菇能益气、补脾胃,舒筋活络,可增强机体免疫力。这道菜有利于养精益髓,下乳增奶,对产后补益、增乳效果明显。

% \textbf{乌鸡白风汤}

% 材料:乌骨鸡1只,白凤尾菇50克

% 调味料:黄酒10克,葱姜各5克,味精、盐各适量。

% 做法:乌骨鸡去血、去毛、去内脏,洗净,白凤尾菇漂洗后撕成条。锅中清水加姜片煮沸,放入乌骨鸡,加入黄酒、葱节,用小火焖煮至酥,放入白凤尾菇,加味精、盐调味后,沸煮3分钟起锅,趁热饮汤食肉、菜。

% 功效:乌骨鸡有较强的滋补肝肾的功效。白凤尾菇能益气、补脾胃,舒筋活络,可增强机体免疫力。这道菜有利于养精益髓,下乳增奶,对产后补益、增乳效果明显。

% %ux56dbux6dfbux52a0ux8f85ux98dfux7684ux539fux5219}{%
% \subsection{四、添加辅食的原则}%ux56dbux6dfbux52a0ux8f85ux98dfux7684ux539fux5219}}

% \begin{enumerate}
% \def\labelenumi{\arabic{enumi}.}
% \item
%   \textbf{由一种到多种的原则}。开始时不要几种食物一起加,应先试加一种,让宝宝从口感到胃肠道功能都逐渐适应后再加第二种。如宝宝拒绝食入就不要勉强,可过一天再试,三五次后婴儿一般就接受了。
% \item
%   \textbf{由少到多的原则}。添加辅食应从少量开始,待婴儿愿意接受,大便也正常后,才可再增加量。如果婴儿出现大便异常,应暂停辅食,待大便正常后,再以原量或小量开始试喂。
% \item
%   \textbf{由稀到稠的原则}。食品应从汁到泥,由果蔬类到肉类。如从果蔬汁到果蔬泥再到碎菜碎果,由米汤到稀粥再到稠粥。
% \item
%   \textbf{应使用小匙食用},而不要放在奶瓶中吸吮。这样也为孩子断奶以后的进食打下良好的基础。
% \item
%   孩子患病时,应暂缓添加,以免加重其胃肠道的负担。
% \item
%   最好给孩子添加专门为其制作的食品,即不要只是简单地把大人的饭做得软烂一些给宝宝食用。因为孩子的肝脏肾脏还很娇嫩,功能还没有发育完善,咀嚼吞咽功能还不够强。他们的食物应以\textbf{尽量少加盐,甚至不加盐为原则},以免增加孩子肝、肾的负担。颗粒尽量小,以免噎住卡住喉咙。
% \end{enumerate}

% %ux4e94ux8f85ux98dfux7684ux5236ux4f5c}{%
% \subsection{五、辅食的制作}%ux4e94ux8f85ux98dfux7684ux5236ux4f5c}}

% \textbf{蛋黄泥:}
% 取鸡蛋放入冷水中微火煮沸,剥去壳,取出蛋黄,加开水少许用汤匙捣烂调成糊状即可。把蛋黄泥混入牛奶、米汤、茶水中调和喂吃。

% \textbf{猪肝泥:}
% 将生猪肝去筋切成碎末,加少许酱油泡一会。在锅中放少量水煮开,将肝末放入煮5分钟即可。还可用油炒熟混入牛奶、茶水、米汤内调和喂吃。

% \textbf{菜泥:}
% 蔬菜种类很多,可交替给孩子食用。如胡萝卜、土豆、白薯等,可将它们洗净后,用锅蒸熟或用水煮软,碾成细泥状喂婴儿。菜类可选用白菜心、油菜、菠菜等。把菜洗净后,切成细末,再用少许植物油炒熟即可食用。应该注意的是,\textbf{菠菜}中含草酸较多,草酸容易与钙质结合形成\textbf{草酸钙},不能被人体所吸收。所以在制作菠菜时,要先将洗净的菠菜用水\textbf{烫一下},再放入冷水中浸泡15分钟,切成细末,放在炉火上继续煮2
% \textasciitilde{}
% 3分钟才可食用。这样便可去掉菠菜中大部分草酸,减少草酸与人体中钙的结合。

% 不管给孩子食用何种蔬菜,都要注意既要新鲜,又要\textbf{多样}。初始时要少量,从一小匙开始,逐渐增多,同时注意观察孩子身体是否适应,如出现呕吐和腹泻的情况,要立即停止食用,找出原因。在各种蔬菜中,\textbf{胡萝卜是孩子最理想的食物},胡萝卜营养丰富,是合成人体内维生素A的主要来源。要知道,人体如缺了维生素A,眼睛发育会出现障碍,易患夜盲症并伴有皮肤粗糙等病变。

% 除了上述食品外,给孩子一些蜂蜜是很必要的,尤其是便秘的孩子。由于不能吃泻药,给孩子食用适量的蜂蜜可以起到促消化、润肠、通便的作用。蜂蜜中含有许多人体所需的矿物质如钾、锌、钙、铁、铜、磷等,并含有各种维生素。这些营养素可以强健孩子的身体,促进脑细胞的发育,还能促进孩子牙齿与骨骼的发育生长,提高机体的抗病能力。但一定要选择新鲜卫生的蜂蜜喂孩子,千万不能给孩子食用污染变质的蜂蜜,因为在这些蜂蜜中含有肉毒杆菌,食用后对孩子的身体会产生很大的危害。

% 4个月宝宝1日营养计划

% \begin{longtable}[]{@{}
%   >{\raggedright\arraybackslash}p{(\columnwidth - 2\tabcolsep) * \real{0.5000}}
%   >{\raggedright\arraybackslash}p{(\columnwidth - 2\tabcolsep) * \real{0.5000}}@{}}
% \toprule()
% \begin{minipage}[b]{\linewidth}\raggedright
% 时间
% \end{minipage} & \begin{minipage}[b]{\linewidth}\raggedright
% 营养计划
% \end{minipage} \\
% \midrule()
% \endhead
% 早上6:00 & 母乳或配方奶120-150毫升 \\
% 上午8:00 & 鲜橙汁或番茄汁80毫升 \\
% 上午10:00 & 米粉 \\
% 中午12:00 & 新鲜水果汁或蔬菜汁80毫升 \\
% 下午3:00 & 母乳或配方奶120-150毫升 \\
% 下午6:30 & 水果泥20克 \\
% 晚上9:00 & 母乳或配方奶120-150毫升 \\
% 夜间12:00 & 母乳或配方奶120-150毫升 \\
% 鱼肝油 & 每日1次,参照说明或遵医嘱 \\
% \bottomrule()
% \end{longtable}

% \begin{quote}
% 育儿小百科

% \textbf{孩子发烧不爱吃奶怎么办}

% 人体发热可引起胃肠功能紊乱,交感神经活动增强,消化酶的分泌减少。虽然食入量很少,但食物在胃肠内停留的时间很长。所以,孩子在发烧时食欲减退,有时还肚子胀。

% 怎么办呢?可以让孩子每次食入\textbf{量少}一点,\textbf{多}吃几餐。而且要吃一些稀释且清淡的有助于消化吸收的食品,在牛奶中加一些米汤或水,并注意给孩子多喂水,保证足够的液体供给。发烧时体内水分消耗较多,如不注意给孩子喂水,一方面发烧不容易退,另一方面,容易引起代谢紊乱。在补充水时,特别要注意补充些鲜果汁水或菜水等。
% \end{quote}

% %ux7b2cux4e94ux8282-45ux4e2aux6708ux5a74ux513fux7684ux5582ux517bux6307ux5bfc}{%
% \subsection{05第五节
% 4〜5个月婴儿的喂养指导}%ux7b2cux4e94ux8282-45ux4e2aux6708ux5a74ux513fux7684ux5582ux517bux6307ux5bfc}}

% 4个月的婴儿在行为上和生理上会发出准备学习新进食技巧的讯号。在这个阶段可添加固体食物,这标志着宝宝的成长迈上一个新台阶。接触新的口感和味道时,刺激宝宝学习在嘴里移动食物。另外在这一年龄时期添加食物的另一重要因素是,宝宝从母体内带来的铁含量已开始逐渐减少,需要从饮食中得到补充。单纯母乳喂养已经不能满足孩子生长的需要了,如果您发觉宝宝体重不再增加,\textbf{吃完奶后还意犹未尽,这可能就是该添加固体食物的时候了}。不过您最好请医生指导一下。4个多月的孩子食入量差别较大,4个多月的孩子除了吃奶以外,要逐渐增加半流质的食物,为以后吃固体食物作准备。婴儿随年龄增长,胃里分泌的消化酶类增多,可以食用一些淀粉类半流质食物,先从1\textasciitilde2匙开始,以后逐渐增加,孩子不爱吃就不要喂,千万不能勉强。加大米粥等食物的那一餐,可以停喂一次婴儿米粉。

% %ux4e00ux907fux514dux7f3aux94c1ux6027ux8d2bux8840}{%
% \subsection{一、避免缺铁性贫血}%ux4e00ux907fux514dux7f3aux94c1ux6027ux8d2bux8840}}

% 4个多月的孩子容易出现贫血,这是因为从母体带来的微量元素铁,已经消耗掉。如果日常食物比较单一,便跟不上身体生长的需要。因此要在辅食中注意增补含铁量高的食物,例如蛋黄中铁的含量就较高,可以在牛奶中加上蛋黄搅拌均匀,煮沸以后食用。贫血较重的孩子,可由医生指导,口服宝宝补血蜜果,千万不要自己乱给孩子服用铁剂药物,以免产生不良反应。

% 为补充体内维生素C的需要,除了继续给孩子吃水果汁和新鲜蔬菜水以外,还可以做一些菜泥和水果泥喂孩子。在添加辅食的过程中,要注意孩子的大便是否正常以及有没有不适应的情况,每次添加的量不宜过多,使孩子的消化系统逐渐适应。

% 喂养时间可在上午6:00、10:00,下午14:00、18:00,晚上22:00,夜间可以不喂,在两次喂食之间加喂一次鲜水果汁、水等。钙片一天可喂3次,每次2片。鱼肝油一天喂2次,每次2\textasciitilde3滴。

% %ux4e8cux6dfbux52a0ux8f85ux98dfux5e94ux6ce8ux610fux7684ux95eeux9898}{%
% \subsection{二、添加辅食应注意的问题}%ux4e8cux6dfbux52a0ux8f85ux98dfux5e94ux6ce8ux610fux7684ux95eeux9898}}

% \begin{enumerate}
% \def\labelenumi{\arabic{enumi}.}
% \item
%   孩子吃惯了奶,对一种新的食物不接受。妈妈费劲做的辅食,他不张嘴吃。遇到这种情况不要勉强,不吃就下回换一种食物再喂。
% \item
%   对于孩子的饮食\textbf{不要死按书本,不要太教条}。孩子跟孩子不同,吃多吃少,吃哪种食物还要根据孩子的食欲和爱好灵活掌握。
% \item
%   给孩子添加食物一定要讲究卫生,原料要新鲜,现做现吃,吃剩的不要再吃。
% \item
%   不要把大人的剩饭菜煮烂给孩子当辅食。
% \item
%   孩子吃某种食物腹泻,可停止添加。
% \item
%   孩子吃西红柿、西瓜、胡萝卜后大便可能会有红色,或吃青菜有绿色,这是正常的。再做辅食时可做得再细些。
% \item
%   孩子如果出现湿疹,可能是对某种蛋白质过敏。
% \item
%   孩子的主食仍然是乳。
% \end{enumerate}

% %ux4e09ux5a74ux513fux98dfux54c1ux7684ux5236ux4f5c}{%
% \subsection{三、婴儿食品的制作}%ux4e09ux5a74ux513fux98dfux54c1ux7684ux5236ux4f5c}}

% %ux725bux5976ux9999ux8549ux7cca}{%
% \subsubsection{\texorpdfstring{\textbf{牛奶香蕉糊}}{牛奶香蕉糊}}%ux725bux5976ux9999ux8549ux7cca}}

% 材料:香蕉1/2根,玉米面5克(要选择研磨得极细的玉米面)、白糖4克、牛奶30毫升

% 做法:将玉米面、白糖、牛奶放入小锅内搅匀。锅置火上,加热煮沸后,改为文火并不断搅拌,以防糊锅底和外溢。待玉米糊煮熟后放入捣碎的香蕉调匀即成。

% 妈妈喂养经:香蕉中含有丰富的钾和镁,其他维生素和糖分、蛋白质、矿物质的含量也很高,此粥不仅是很好的强身健脑食品,更是\textbf{便秘}宝宝的最佳食物。

% %ux7c73ux7c89}{%
% \subsubsection{\texorpdfstring{\textbf{米粉}}{米粉}}%ux7c73ux7c89}}

% 材料:婴儿营养米粉,温水适量

% 做法:按照产品说明书,取适量米粉放入小碗中,加入适量温水,用筷子按照顺时针方向调成糊状即可。

% 妈妈喂养经:市售的米粉种类繁多,营养非常丰富,没有时间自己做的时候,使用起来非常方便,而且口感润滑,宝宝也很喜欢吃。但是冲调时一定要按照产品说明,根据宝宝体重来确定米粉的量,切不可冲调过稠或过稀。

% %ux9c7cux6ce5ux80e1ux841dux535cux6ce5ux7c73ux7c89}{%
% \subsubsection{\texorpdfstring{\textbf{鱼泥胡萝卜泥米粉}}{鱼泥胡萝卜泥米粉}}%ux9c7cux6ce5ux80e1ux841dux535cux6ce5ux7c73ux7c89}}

% 材料:河鱼或是海鱼,胡萝卜1/2根

% 做法:选择河鱼或是海鱼,蒸熟,取出肉,并小心将鱼刺全部除去,压成泥即可。将做好的少量鱼泥,连同胡萝卜泥一起放在米粉里,加适量温水调匀即可。

% 妈妈喂养经:给宝宝选择鱼类食物时,一定要选择刺少的鱼类,以免伤害到宝宝。另外,市场上也有\textbf{成品鱼泥}、肝泥,可以挑选几种搭配胡萝卜、土豆一起喂食。

% %ux725bux5976ux86cbux9ec4ux7c73ux6c64ux7ca5}{%
% \subsubsection{\texorpdfstring{\textbf{牛奶蛋黄米汤粥}}{牛奶蛋黄米汤粥}}%ux725bux5976ux86cbux9ec4ux7c73ux6c64ux7ca5}}

% 原料:米汤半小碗,奶粉2勺,鸡蛋黄1/3个

% 做法:在烧大米粥时,将上面的米汤盛出半碗;鸡蛋煮熟,取蛋黄1/3个研碎。将奶粉冲调好,放入蛋黄、米汤,调匀即可。

% 妈妈喂养经:此粥富含蛋白质和钙质,蛋黄中还含有丰富的卵磷脂,对宝宝生长和大脑发育有好处。

% %ux86cbux9ec4ux7ca5}{%
% \subsubsection{\texorpdfstring{\textbf{蛋黄粥}}{蛋黄粥}}%ux86cbux9ec4ux7ca5}}

% 材料:软饭1小碗,肉汤(或鱼汤、菜汤),鸡蛋1个

% 做法:

% \begin{enumerate}
% \def\labelenumi{\arabic{enumi}.}
% \item
%   当煮成人饭时,把米和适量水放入煲内,中间放入干净的小碗,里面放入米和多量的水,煮成饭后,中心的米便成软饭。
% \item
%   把适量的软饭研磨成糊状。把肉汤或鱼汤、菜汤滤去渣,如鱼汤要特别小心以防有细小的刺,除去汤面上的油。
% \item
%   把汤和饭糊放入小煲内煲滚,用慢火煲成稀糊状,然后放入1/4熟蛋黄(先搓成蓉)搅匀煮滚,盛入碗中。待温度合适时,便可喂宝宝。
% \end{enumerate}

% 妈妈喂养经:4个月的宝宝,只适宜学习吞咽半流质的食物,让他知道奶以外的``味道'',训练他接受食物的习惯。大人的汤不适宜宝宝食用的话,可以用配方奶或水煮成稀糊,一定要注意是极烂的稀糊,否则,宝宝难以消化。试验食物热度的方法:妈妈可以将少许粥滴在手背上,感觉到有一点点温度就可以喂宝宝了。

% %ux6c34ux871cux6843ux6c41}{%
% \subsubsection{水蜜桃汁}%ux6c34ux871cux6843ux6c41}}

% \textbf{材料}:水蜜桃50克

% \textbf{做法}:水蜜桃用清水洗净,去皮,把果肉切成小块,用榨汁机榨汁即可。

% \textbf{妈妈喂养经}:4个月大的宝宝,每次喂食1\textasciitilde2小匙。水蜜桃的维生素C含量很高,如果要给宝宝补充维生素C高的水果,最好是选择\textbf{本地水果},而不要盲目购买进口水果。因为水果从采摘下来后维生素就开始流失,在经过\textbf{长途}运输的过程中维生素C的含量会更少。

% %ux82f9ux679cux6c41}{%
% \subsubsection{苹果汁}%ux82f9ux679cux6c41}}

% \textbf{材料}:苹果50克

% \textbf{做法}:选用熟透的苹果,洗净切成两半。将苹果皮和核去掉,切成小块,将小块苹果放入榨汁机中榨汁。喂食时加入适量温水稀释。

% \textbf{妈妈喂养经}:适合4个月左右的宝宝,每次喂食1\textasciitilde2勺。如果在苹果汁中添加胡萝卜后,果汁特别容易变质,不宜久放。苹果含有丰富的矿物质和多种维生素,宝宝常吃苹果,可以预防佝偻病。

% %ux767dux841dux535cux751fux68a8ux6c41}{%
% \subsubsection{白萝卜生梨汁}%ux767dux841dux535cux751fux68a8ux6c41}}

% \textbf{材料}:小白萝卜一个,梨半个

% \textbf{做法}:将白萝卜切成细丝,梨切成薄片。将白萝卜倒入锅内加清水烧开,用微火炖10分钟后,加入梨片再煮5分钟取汁即可食用。

% \textbf{妈妈喂养经}:白萝卜富含维生素C、蛋白质、铁等素等营养成分,具有止咳润肺,帮助消化等保健作用。

% %ux9c9cux7389ux7c73ux7cca}{%
% \subsubsection{鲜玉米糊}%ux9c9cux7389ux7c73ux7cca}}

% \textbf{材料}:新鲜玉米半个

% \textbf{做法}:用刀将玉米粒削下来,放入果汁机中搅拌成浆。用纱布将玉米汁过滤出来,煮成黏稠状即可。

% \textbf{妈妈喂养经}:玉米富含钙、镁、硒、维生素B2、维生素B6、卵磷脂和18种氨基酸等30多种营养活性物质,能提高人体免疫力,增强脑细胞的活动,健康益智。

% %ux51b0ux7cd6ux96eaux68a8ux7cca}{%
% \subsubsection{冰糖雪梨糊}%ux51b0ux7cd6ux96eaux68a8ux7cca}}

% \textbf{材料}:梨1个,冰糖适量

% \textbf{做法}:将梨去皮去核,切碎,与冰糖一起入锅煮,待梨酥烂以后,一边煮一边用勺子碾压,成糊状即可。

% \textbf{妈妈喂养经}:不仅补充维生素和矿物质,而且对咳嗽的宝宝有辅助治疗的作用。

% %ux56dbux4ebaux5de5ux5582ux517bux7684ux95eeux9898}{%
% \subsection{四、人工喂养的问题}%ux56dbux4ebaux5de5ux5582ux517bux7684ux95eeux9898}}

% 母乳喂养要按需哺乳,孩子什么时候要吃便吃,吃饱就可以了。许多人工喂养的孩子吃奶,妈妈都有一个参照量,如果孩子达不到量,便不断将奶嘴往孩子嘴里塞,强逼他吃。这种做法不会给孩子带来好处,孩子往往因母亲的强迫,对吃奶产生厌烦情绪,食欲减退,消化能力也减弱,反而使孩子摄取的营养满足不了需要而影响发育。孩子在最早时知道自己需要多少食物,饿的时候他会哭,要求妈妈来喂,吃饱了就对乳头和奶瓶不感兴趣了。孩子厌食,多数是父母喂养不当造成的。【\textbf{恩恩是不是有类似?}】

% 小婴儿有时吃着吃着奶就睡着了,睡一会儿醒了又吃。出现这种情况可能是喂奶时孩子吸进了空气,\textbf{空气到胃里使孩子感到饱了}。也可能是孩子食欲缺乏。如果孩子剩下的奶不多,就让他睡,下次多准备些奶就行了。如果剩的多,可\textbf{揪揪}耳朵把他叫醒,让他接着吃。不要给孩子养成一瓶奶分两次吃的习惯。妈妈也\textbf{不要一边喂奶一边与别人聊天或看电视},喂奶时要关注自己的宝宝,对他说说话。【\textbf{要不要看手机用耳机}?】

% 妈妈不要指望孩子把奶瓶中的奶喝光,只要孩子心满意足就行了。剩下点奶不要紧,不要勉强他都吃下去。吃得太多孩子会发生肥胖。孩子吃完了也许还叼着奶嘴玩,妈妈可轻轻地将小指滑入孩子嘴角,即可中断孩子吸奶的动作,将奶瓶拿走。

% %ux4e94ux5a74ux513fux5582ux517bux8befux533a}{%
% \subsection{五、婴儿喂养误区}%ux4e94ux5a74ux513fux5582ux517bux8befux533a}}

% \begin{enumerate}
% \def\labelenumi{\arabic{enumi}.}
% \item
%   小粒食品\\
%   孩子咀嚼能力差,舌的运动也不协调。小粒食品极易误吸进气管,造成危险。如花生米、玉米花、黄豆、榛子仁等都不宜给孩子吃。
% \item
%   带骨的肉食\\
%   不要给孩子吃排骨,排骨的骨渣易刺伤口腔黏膜,或卡在喉部。吃鱼最好吃\textbf{海鱼},家长把刺挑净,压成鱼泥给孩子吃,虾要把皮剥净。【以后污染了,这。。。】
% \item
%   少吃不易消化吸收的食物\\
%   如竹笋、炒黄豆、生萝卜、白薯等。
% \item
%   不吃太咸的食物\\
%   如咸菜、咸蛋等,孩子的饭菜宜清淡。
% \item
%   不吃太过油腻的食物\\
%   如肥肉、油炸食品等。
% \item
%   不吃不卫生的食品\\
%   特别是街头小摊的食物。
% \item
%   不吃辛辣刺激性食品\\
%   如酒精饮料、咖啡、可乐、浓茶及各种饮料,还有辣椒、大蒜等。
% \end{enumerate}

% 5个月宝宝1日营养计划

% \begin{longtable}[]{@{}
%   >{\raggedright\arraybackslash}p{(\columnwidth - 2\tabcolsep) * \real{0.5000}}
%   >{\raggedright\arraybackslash}p{(\columnwidth - 2\tabcolsep) * \real{0.5000}}@{}}
% \toprule()
% \begin{minipage}[b]{\linewidth}\raggedright
% 时间
% \end{minipage} & \begin{minipage}[b]{\linewidth}\raggedright
% 食物类型
% \end{minipage} \\
% \midrule()
% \endhead
% 早上6:00 & 母乳或配方奶150\textasciitilde200毫升 \\
% ±午9:00 & 米糊或婴儿营养米粉20克 \\
% 上午11:00 & 新鲜蔬菜泥20克 \\
% 中午12:30 & 母乳或配方奶150\textasciitilde200毫升 \\
% 下午3:00 & 鸡蛋羹10克 \\
% 下午5:30 & 菜水60毫升 \\
% 晚上8:00 & 母乳或配方奶150\textasciitilde200毫升 \\
% 夜间12:00 & 母乳或配方奶150\textasciitilde200毫升 \\
% 鱼肝油 & 每日1次,参照说明或遵医嘱 \\
% \bottomrule()
% \end{longtable}

% %ux7b2cux516dux8282-5-6ux4e2aux6708ux5a74ux513fux7684ux5582ux517bux6307ux5bfc}{%
% \subsection{06第六节 5 \textasciitilde{}
% 6个月婴儿的喂养指导}%ux7b2cux516dux8282-5-6ux4e2aux6708ux5a74ux513fux7684ux5582ux517bux6307ux5bfc}}

% 5个多月的孩子,由于活动量增加,热量的要求也随之增加,以前认为只吃母乳或牛奶远不能满足孩子生长发育的需要,\textbf{现认为}纯母乳喂养可以满足孩子生长发育的需要。

% 如果必须人工喂养,5个月的孩子主食喂养仍以乳类为主,牛奶每次可吃到200毫升,除了加些米粉,健儿粉类外,还可将蛋黄加到1个,在大便正常的情况下,粥和菜泥都可以增加一点,可以用水果泥来代替果汁。已经长牙的婴儿,可以试吃一点饼干,锻炼阻嚼能力,促进牙齿和颌骨的发育。

% 本月在辅食上还可以增加一些鱼类,如平鱼、黄鱼、巴鱼等,此类鱼肉多、刺少,便于加工成肉糜。鱼肉含磷脂、蛋白质很高,并且细嫩易消化,适合婴儿发育的营养需要。但一定要选购新鲜的鱼。

% 在喂养时间上,仍可按上月的安排进行。只是在辅食添加种类与量上略多一些。鱼肝油每次仍吃2滴,每天3次,钙片每次2片,每天2
% \textasciitilde{} 3次。

% %ux4e00ux7ed9ux5b69ux5b50ux6dfbux52a0ux8f85ux98dfux5e94ux6ce8ux610fux7684ux95eeux9898}{%
% \subsection{一、给孩子添加辅食应注意的问题}%ux4e00ux7ed9ux5b69ux5b50ux6dfbux52a0ux8f85ux98dfux5e94ux6ce8ux610fux7684ux95eeux9898}}

% \begin{enumerate}
% \def\labelenumi{\arabic{enumi}.}
% \item
%   由少量开始,逐渐增多。当孩子愿意吃并能正常消化时,再逐渐增多。如孩子不肯吃,就不要勉强地喂,可以过2\textasciitilde3天再喂。
% \item
%   辅食要由稀到干,由细到粗,由软到硬,由淡到浓,\textbf{循序渐进}逐步增加,要使孩子有一段逐渐适应的过程。
% \item
%   要根据季节和孩子身体状态来添加辅食,并要一样一样地增加,逐渐增加到多种。如孩子大便变稀不正常,要暂停增加,等恢复正常后再增加。另外,在炎热的夏季和身体不好的情况下,不要添加辅食,以免孩子产生不适。
% \item
%   辅食宜在孩子吃奶前饥饿时添加,这样孩子更容易接受。随着辅食的逐渐增加,可由每天代替半顿奶逐步过渡到代替1顿奶。
% \item
%   要注意卫生,婴儿餐具要固定专用,除注意认真洗刷外,还要每日消毒。喂饭时,家长不要用嘴边吹边喂,更不要咀嚼后再喂给宝宝。这种做法极不卫生,很容易把疾病传染给孩子。【可以买专用小风扇,如我公司那个】
% \item
%   喂辅食时,要锻炼孩子逐步适应使用餐具,为以后独立用餐具做好准备。
% \item
%   家长在喂婴儿辅食时,要有耐心,还要想办法让孩子对食物产生兴趣。
% \end{enumerate}

% %ux4e8cux5b9dux5b9dux6bcfux65e5ux9700ux8981ux591aux5c11ux70edux91cf}{%
% \subsection{二、宝宝每日需要多少热量}%ux4e8cux5b9dux5b9dux6bcfux65e5ux9700ux8981ux591aux5c11ux70edux91cf}}

% 小儿生长的特点是年龄越小,生长发育越快。因此,这期间所需要的营养物质和水分相对成人也较高。

% 婴幼儿虽然需要的热量高,但他们的消化功能并不好,体内分泌的消化酶也不足,容易出现消化不良情况。如腹泻、呕吐以致造成脱水和酸中毒等。

% %ux4e09ux5b9dux5b9dux8f85ux98dfux7684ux5582ux517bux65b9ux6cd5}{%
% \subsection{三、宝宝辅食的喂养方法}%ux4e09ux5b9dux5b9dux8f85ux98dfux7684ux5582ux517bux65b9ux6cd5}}

% 进入5个月,要教会宝宝不让进入口腔的食物再从嘴边流出来。当食物从嘴边流出时,宝宝就会有意识地减慢进食速度。正确的方法是将小勺轻轻放在宝宝的下唇,宝宝就会自动张开嘴巴,可以帮助宝宝闭上嘴唇,使之将食物顺利咽下,之后便可逐渐减少食物中的水分,进入喂泥状辅食阶段。

% %ux56dbux5b9dux5b9dux8f85ux98dfux7684ux7c97ux7ec6ux7a0bux5ea6}{%
% \subsection{四、宝宝辅食的粗细程度}%ux56dbux5b9dux5b9dux8f85ux98dfux7684ux7c97ux7ec6ux7a0bux5ea6}}

% 以南瓜粥为例:将南瓜去皮放入水中煮软,研碎,然后用热水或汤汁将其制成光滑糊状。刚开始时应做得稍微稀一些,以便宝宝吞咽。\textbf{浓稀程度应以盛在盘子里用勺子划线,线痕会立即消失为宜。}

% %ux4e94ux6bcfux6b21ux8f85ux98dfux5582ux591aux5c11}{%
% \subsection{五、每次辅食喂多少}%ux4e94ux6bcfux6b21ux8f85ux98dfux5582ux591aux5c11}}

% 从1勺喂起,宝宝需要的话可以喂三四勺。刚开始时,可以视宝宝的情况进行。大便有些变稀,次数有所增加,这些都不要紧,只要宝宝情绪良好,食欲旺盛就没有问题。

% 如果宝宝不喜欢吃的话,妈妈应该看看食物是否光滑,宝宝情绪如何,改天再试一试。

% 开始后,可以每隔一两天增加1勺的量。过五六天后,宝宝已经习惯吃粥并能吃到三四勺时,可以给宝宝喂些胡萝卜、南瓜之类的蔬菜。这些蔬菜也要从1勺喂起,每隔一两天增加1勺,到五六天完全适应后,可以再增加些豆腐等蛋白质食品。增加新食物时要慎重小心,应一点点地加量。

% 开始吃断奶食物2周后,如果宝宝可以食用谷类、蔬菜和蛋白质类食品这三大营养源的食品,就没有问题,妈妈应该进一步逐渐增加食品种类,以保持营养均衡。

% 当宝宝完全能够适应后,可以每天增加一次进食,让宝宝每天吃2次断奶食物。

% %ux516dux98dfux8c31ux8bbeux8ba1ux8981ux70b9}{%
% \subsection{六、食谱设计要点}%ux516dux98dfux8c31ux8bbeux8ba1ux8981ux70b9}}

% 很多妈妈在宝宝刚进入断奶餐阶段,往往不知道该喂些什么食物,其实只要选择宝宝容易习惯的食物就可以,如过滤的果汁、蔬菜汤或米粥等,也可以选择一些配方辅食。

% 宝宝长到5个月时,可以在宝宝情绪好的日子里开始给宝宝喂断奶食物。刚开始时,建议给宝宝喂些谷类,特别是喂粥为宜,因为这样不会对宝宝肠胃造成负担,而且也是婴儿比较容易习惯的味道。把粥煮软后研碎,用小勺喂给宝宝。不过,不管怎么煮,米粥都不会特别光滑,对于喉咙特别敏感的宝宝来说,应该将粥过滤一下,让粥变得光滑易咽。也可以选择购买婴儿营养米粉,加入热水后即可冲调成光滑的米粥。

% 这个阶段除了继续观察皮肤和粪便情况之外,也要慎重选择水果,不要给宝宝口感太强烈的品种,如:榴莲、荔枝、芒果等。给予果汁的时间最好是在白天,以利于宝宝出现过敏反应或不适症状时,父母能有足够的时间寻求医生帮助。另外,这个时期还要注意宝宝的营养均衡,五谷杂粮、蔬菜类和水果类、蛋白质食物要逐渐增加。食量增加的同时,宝宝的大便形状也会改变,有时会出现便秘和腹泻,只要宝宝情绪和食欲好,就不必担心。

% \textbf{6个月宝宝1日营养计划}

% \begin{longtable}[]{@{}
%   >{\raggedright\arraybackslash}p{(\columnwidth - 2\tabcolsep) * \real{0.5000}}
%   >{\raggedright\arraybackslash}p{(\columnwidth - 2\tabcolsep) * \real{0.5000}}@{}}
% \toprule()
% \begin{minipage}[b]{\linewidth}\raggedright
% 时间
% \end{minipage} & \begin{minipage}[b]{\linewidth}\raggedright
% 食物类型
% \end{minipage} \\
% \midrule()
% \endhead
% 早上6:00 & 母乳或配方奶150-200毫升 \\
% 上午8:00 & 果汁:鲜橙汁或番茄汁80毫升 \\
% 上午10:00 & 营养米粉:鸡蛋籴粉20克、蛋黄10克 \\
% 中午12:30 & 母乳或配方奶150-200毫升 \\
% 下午2:00 & 水果泥 \\
% 下午4:00 & 蔬菜汁80毫升 \\
% 晚上6:00 & 母乳或配方奶150-200毫升 \\
% 晚上10:00 & 母乳或配方奶150-200毫升 \\
% 凌晨2:00 & 母乳或配方奶150-200毫升 \\
% 每天 & 鱼肝油1次,参照说明或遵医嘱 \\
% \bottomrule()
% \end{longtable}

% %ux7b2cux4e03ux8282-67ux4e2aux6708ux5a74ux513fux7684ux5582ux517bux6307ux5bfc}{%
% \subsection{07第七节
% 6〜7个月婴儿的喂养指导}%ux7b2cux4e03ux8282-67ux4e2aux6708ux5a74ux513fux7684ux5582ux517bux6307ux5bfc}}

% 为了孩子的健康成长,妈妈应该坚持母乳喂养满6个月。

% 如果条件不允许,可人工喂养,奶量不再增加。每天喂养3-4次。每次喂150-200毫升。可以在早上6:00、中午11:00、下午17:00、晚上22:00各喂1次奶。上午9:00-10:00及下午15:00-16:00添加两次辅食。

% 6个月的孩子每天可吃两次粥,每次1/2-1小碗,可以吃少量烂面片,鸡蛋黄应保证每天1个,每日要喂些菜泥、鱼泥、肝泥等,但要从少到多,逐渐增加辅食。

% 6个多月孩子正是出牙的时候,所以,应该给孩子一些固体食物如烤馒头片、面包干、饼干等练习咀嚼,磨磨牙床,以促进牙齿生长。

% \textbf{蛋黄粥}:大米2小匙洗净加水约120毫升泡1-2小时,然后用微火煮40-50分钟,再把蛋黄1/4个研碎后加入粥锅内,再煮10分钟左右即可。

% \textbf{水果麦片粥}:把麦片3大匙放入锅内,加入牛奶1大匙后用微火煮2\textasciitilde3分钟,煮至黏稠状,停火后加切碎的水果1大匙(可用切碎的香蕉加蜂蜜,也可以用水果罐头做)。

% \textbf{面包粥}:把1/3个面包切成均匀的小碎块,和肉汤2大匙一起放入锅内煮,面包变软后即停火。

% \textbf{牛奶藕粉}:藕粉或淀粉1/2大匙,水1/2杯,牛奶1大匙一起放入锅内,均匀混合后用微火熬,边熬边搅拌,直到透明糊状为止。

% \textbf{奶蛋粥}:蛋黄1/2个,淀粉1/2大匙加水放入锅内均匀混合后上火熬,边熬边搅拌,熬至黏稠状时加入牛奶3匙,停火后放凉时再加蜂蜜少许。

% %ux4e00ux7ed9ux5a74ux513fux505aux98dfux7269ux5e94ux6ce8ux610fux4ec0ux4e48}{%
% \subsection{一、给婴儿做食物应注意什么}%ux4e00ux7ed9ux5a74ux513fux505aux98dfux7269ux5e94ux6ce8ux610fux4ec0ux4e48}}

% \begin{enumerate}
% \def\labelenumi{\arabic{enumi}.}
% \item
%   所需原料互相搭配以便营养成分互补。
% \item
%   所需蔬菜水果要新鲜、干净,并要煮3 \textasciitilde{} 5分钟。
% \item
%   为便于婴儿吞咽食物,可做得稀一些。
% \item
%   彻底清洁厨房和做食物的用具,以免污染婴儿食品。
% \item
%   不要让婴儿吃上顿剩下的食物。
% \end{enumerate}

% %ux4e8cux7f3aux94c1ux6027ux8d2bux8840ux5b69ux5b50ux7684ux5582ux517b}{%
% \subsection{二、缺铁性贫血孩子的喂养}%ux4e8cux7f3aux94c1ux6027ux8d2bux8840ux5b69ux5b50ux7684ux5582ux517b}}

% 孩子贫血大多数是由缺铁而引起的缺铁性贫血。铁是造血的主要原料之一,人体内的铁主要来源于食物,缺铁性贫血是营养性贫血。轻度贫血无异常表现,常在血红蛋白很低时才发现孩子患了贫血。贫血的孩子面色苍白、唇及眼睑色淡,抵抗力低下,生长发育缓慢。

% 对缺铁性贫血主要在于预防,在孩子喂养上应注意以下方面:

% \textbf{(1)人工喂养的小婴儿及时加辅食}。因奶中含铁量低,远不能满足孩子生长的需要,而从母体得到的铁,至3\textasciitilde4个月时,都已用尽,因此必须及时补充。

% (2)选择含铁丰富,铁吸收率高的食物。

% 部分食品含铁量(单位:毫克/100克食物)

% \begin{longtable}[]{@{}
%   >{\raggedright\arraybackslash}p{(\columnwidth - 6\tabcolsep) * \real{0.2500}}
%   >{\raggedright\arraybackslash}p{(\columnwidth - 6\tabcolsep) * \real{0.2500}}
%   >{\raggedright\arraybackslash}p{(\columnwidth - 6\tabcolsep) * \real{0.2500}}
%   >{\raggedright\arraybackslash}p{(\columnwidth - 6\tabcolsep) * \real{0.2500}}@{}}
% \toprule()
% \begin{minipage}[b]{\linewidth}\raggedright
% 食品
% \end{minipage} & \begin{minipage}[b]{\linewidth}\raggedright
% 铁
% \end{minipage} & \begin{minipage}[b]{\linewidth}\raggedright
% 食品
% \end{minipage} & \begin{minipage}[b]{\linewidth}\raggedright
% 铁
% \end{minipage} \\
% \midrule()
% \endhead
% 小米 & 4.7 & 油菜 & 3.4 \\
% 大米 & 0.7-0.8 & 菠菜 & 2.5 \\
% 芹菜 & 8.5 & 黄豆 & 11.0 \\
% 菜花 & 1.8 & 蚕豆 & 7.0 \\
% 白萝卜 & 1.9 & 瘦猪肉 & 2.4 \\
% 胡萝卜 & 1.9 & 牛肉 & 3.2 \\
% 海带 & 158.0 & 羊肉 & 3.0 \\
% 紫菜 & 32.0 & 猪肝 & 25.0 \\
% 黑木耳 & 185.0 & 鸡肉 & 1.5 \\
% 香菇 & 23.0 & 牛乳 & 0.1 \\
% 蛋黄 & 7.0 & 人乳 & 0.1 \\
% \bottomrule()
% \end{longtable}

% 一般来说,\textbf{动物性食品的铁吸收率较高},为20\%左右,而植物性食物的吸收率较低,约在10\%以下。虽然\textbf{鸡蛋}中的铁吸收率较低,但不能仅依赖吃鸡蛋来获取铁质。\textbf{大豆}中的铁吸收率较高,因此可\textbf{适量}食用。

% (3)积极治疗胃肠疾病,对慢性腹泻要彻底治愈,以免铁吸收不良。在发生缺铁性贫血的情况下,一方面应注意以上几点,调理好孩子的饮食;另一方面,在医生的指导下,服用铁剂,多吃含铁量高的食物,防止感染其他疾病。一般来说,经过治疗后,血红蛋白可以恢复到正常水平。

% \textbf{孩子血红蛋白正常值}

% \begin{longtable}[]{@{}
%   >{\raggedright\arraybackslash}p{(\columnwidth - 2\tabcolsep) * \real{0.5000}}
%   >{\raggedright\arraybackslash}p{(\columnwidth - 2\tabcolsep) * \real{0.5000}}@{}}
% \toprule()
% \begin{minipage}[b]{\linewidth}\raggedright
% 年龄阶段
% \end{minipage} & \begin{minipage}[b]{\linewidth}\raggedright
% 血红蛋白正常值
% \end{minipage} \\
% \midrule()
% \endhead
% 新生儿 & 180-190g/L \\
% 婴儿 & 110-120g/L \\
% 儿童 & 120-140g/L \\
% \bottomrule()
% \end{longtable}

% \textbf{7个月宝宝1日营养计划}

% \begin{longtable}[]{@{}
%   >{\raggedright\arraybackslash}p{(\columnwidth - 2\tabcolsep) * \real{0.5000}}
%   >{\raggedright\arraybackslash}p{(\columnwidth - 2\tabcolsep) * \real{0.5000}}@{}}
% \toprule()
% \begin{minipage}[b]{\linewidth}\raggedright
% 时间
% \end{minipage} & \begin{minipage}[b]{\linewidth}\raggedright
% 主食及用量
% \end{minipage} \\
% \midrule()
% \endhead
% 早上6:00 & 配方奶200\textasciitilde220毫升 \\
% 上午9:30 & 饼干20克,母乳或配方奶120毫升 \\
% 上午10:00 & 果泥50克,白开水100毫升 \\
% 中午12:00 & 肝泥粥60克,蔬菜1\textasciitilde2匙 \\
% 下午2:00 & 母乳或配方奶120毫升,蛋糕20克,水果泥20克 \\
% 下午4:00 & 白开水 \\
% 晚上6:00 & 番茄鸡蛋面60\textasciitilde80克 \\
% 晚上9:00 & 母乳或配方奶200\textasciitilde220毫升 \\
% 鱼肝油 & 每日1-2滴 \\
% \bottomrule()
% \end{longtable}

% %ux7b2cux516bux82827-8ux4e2aux6708ux5a74ux513fux7684ux5582ux517bux6307ux5bfc}{%
% \subsection{08第八节7 \textasciitilde{}
% 8个月婴儿的喂养指导}%ux7b2cux516bux82827-8ux4e2aux6708ux5a74ux513fux7684ux5582ux517bux6307ux5bfc}}

% 不管是母乳喂养还是人工喂养的孩子,在7个多月时每天的奶量仍不变,分3\textasciitilde4次喂进。辅食除每天给孩子两顿粥或煮烂的面条之外,还可添加一些豆制品,仍要吃菜泥、鱼泥、肝泥等。鸡蛋可以蒸或煮,仍然只吃蛋黄。在孩子出牙期间,还要继续给他吃小饼干、烤馒头片等,让他练习咀嚼。

% \begin{quote}
% gpt:
% \end{quote}

% \begin{longtable}[]{@{}
%   >{\raggedright\arraybackslash}p{(\columnwidth - 2\tabcolsep) * \real{0.5000}}
%   >{\raggedright\arraybackslash}p{(\columnwidth - 2\tabcolsep) * \real{0.5000}}@{}}
% \toprule()
% \begin{minipage}[b]{\linewidth}\raggedright
% 食物类别
% \end{minipage} & \begin{minipage}[b]{\linewidth}\raggedright
% 每日摄入量
% \end{minipage} \\
% \midrule()
% \endhead
% 奶类 & 保持不变,分3\textasciitilde4次喂进 \\
% 粥或煮烂的面条 & 每天两顿 \\
% 豆制品 & 适量 \\
% 菜泥、鱼泥、肝泥 & 适量 \\
% 蒸或煮的鸡蛋 & 只吃蛋黄 \\
% 小饼干、烤馒头片等 & 适量,用于练习咀嚼 \\
% \bottomrule()
% \end{longtable}

% 7个多月的幼儿婴儿每天应进食多少各种食物才能满足他的生长发育需要呢?具体的营养量请参照下表。

% \begin{longtable}[]{@{}
%   >{\raggedright\arraybackslash}p{(\columnwidth - 6\tabcolsep) * \real{0.2500}}
%   >{\raggedright\arraybackslash}p{(\columnwidth - 6\tabcolsep) * \real{0.2500}}
%   >{\raggedright\arraybackslash}p{(\columnwidth - 6\tabcolsep) * \real{0.2500}}
%   >{\raggedright\arraybackslash}p{(\columnwidth - 6\tabcolsep) * \real{0.2500}}@{}}
% \toprule()
% \begin{minipage}[b]{\linewidth}\raggedright
% 食品
% \end{minipage} & \begin{minipage}[b]{\linewidth}\raggedright
% 全日总量
% \end{minipage} & \begin{minipage}[b]{\linewidth}\raggedright
% 次数
% \end{minipage} & \begin{minipage}[b]{\linewidth}\raggedright
% 可替代的食品
% \end{minipage} \\
% \midrule()
% \endhead
% 母奶或牛奶 & 750毫升 & 3-4次 & \\
% 粥 & 1碗 & 分2次 &
% 研碎的面包片1片,烂面条1碗,麦片4大匙:土豆半个:白薯1/3个(煮软研碎) \\
% 蛋黄 & 1个 & 分2次 & 鸡胸肉2小块(研碎),肉末2小匙:豆腐1/5块(研碎) \\
% 鱼 & 20克 & 1次 & 鱼肉泥2大匙 \\
% 水果 & 50克 & 分2次 & 苹果1/4个;桃1/3个; 香蕉1/2个;橘子1/3个 \\
% 蔬菜 & 30克 & 分2次 & 可选胡萝卜、圆白菜、黄瓜、白菜、番茄、茄子等 \\
% \bottomrule()
% \end{longtable}

% %ux4e00ux5582ux8f85ux98dfux7684ux65b9ux6cd5}{%
% \subsection{一、喂辅食的方法}%ux4e00ux5582ux8f85ux98dfux7684ux65b9ux6cd5}}

% 在喂宝宝辅食时,妈妈要注意方法,避免出现喂食不当的情况。

% \begin{enumerate}
% \def\labelenumi{\arabic{enumi}.}
% \item
%   先将宝宝抱在妈妈膝上,围上围嘴。
% \item
%   用柔软的湿巾擦宝宝的嘴和手。
% \item
%   将小勺移到宝宝嘴巴正前方,当宝宝嘴张开时,趁此机会将小匙放在宝宝舌上,再把食物粘在上腭上。
% \item
%   问问宝宝``好吃吗?''观察宝宝的反应,看到宝宝咽下去了,再继续喂第二口。
% \item
%   抱起宝宝轻轻拍背,让宝宝打嗝。
% \item
%   吃完后,给宝宝擦擦嘴,再喂母乳或配方奶。
% \end{enumerate}

% \begin{quote}
% 119(128)
% \end{quote}

% %ux4e8cux8f85ux52a9ux98dfux54c1ux7684ux5236ux4f5c}{%
% \subsection{二、辅助食品的制作}%ux4e8cux8f85ux52a9ux98dfux54c1ux7684ux5236ux4f5c}}

% 制作辅助食品除了要讲究烹调方法,使食物的色香味俱全外,最需要注意的是保证安全卫生,防止病从口入。为此,在为宝宝准备辅助食品时,要做到以下几点。

% \begin{enumerate}
% \def\labelenumi{\arabic{enumi}.}
% \item
%   清洁\\
%   准备辅助食品所用的案板、锅、铲、碗、勺等用具应当用清洁剂洗净,充分漂洗,用沸水或消毒柜消毒后再用。最好能为宝宝单独准备一套烹饪用具,避免交叉感染。
% \item
%   选择优质的原料\\
%   制作辅助食品的原料最好是没有化学物质污染的绿色食品,尽可能新鲜,并精心选择和清洗。
% \item
%   单独制作\\
%   宝宝的辅助食品一般都要求\textbf{细烂}、清淡,所以不要将宝宝辅助食品与成人食品混在一起制作。
% \item
%   用合适的烹饪方法\\
%   制作宝宝辅助食品时,应避免长时间烧煮、油炸、烧烤,以减少营养素的流失。应根据宝宝咀嚼和吞咽能力及时调整食物的质地。食物的调味也要根据宝宝需要来调整,不能以成人的喜好来决定。
% \item
%   现做现吃\\
%   隔顿食物味道和营养都大打折扣,还容易被细菌污染,因此不要让宝宝吃上顿吃剩下的食物。为了方便,在准备生的原料(如肉末、碎菜等)时,可以一次多准备些,然后根据宝宝每次的食量,用保鲜膜分开包装后放入冰箱保存。但是,这样保存食品的时间也不应超过两个星期。
% \end{enumerate}

% %ux4e09ux5a74ux513fux5c0fux98dfux54c1ux5236ux4f5c}{%
% \subsection{三、婴儿小食品制作}%ux4e09ux5a74ux513fux5c0fux98dfux54c1ux5236ux4f5c}}

% %ux852cux83dcux732aux809dux6ce5}{%
% \subsubsection{蔬菜猪肝泥}%ux852cux83dcux732aux809dux6ce5}}

% 胡萝卜煮软切碎1小匙,菠菜叶1/2匙加少量盐煮后切碎,和切碎的猪肝2小匙一起放入锅内,加酱油1小匙用微火煮,停火前加牛奶1大匙。

% %ux9999ux8549ux7ca5}{%
% \subsubsection{香蕉粥}%ux9999ux8549ux7ca5}}

% 1/6根香蕉去皮后,用勺子背把香蕉研成糊状,放入锅内加牛奶1大匙混合后上火煮,边煮边搅拌均匀,停火后加入少许蜂蜜。

% %ux756aux8304ux732aux809d}{%
% \subsubsection{番茄猪肝}%ux756aux8304ux732aux809d}}

% 切碎的猪肝2小匙,切碎的葱头1小匙同时放入锅内,加水或肉汤煮,然后加洗净剥皮切碎的番茄2小匙,盐少许。

% %ux56dbux5b66ux505aux51e0ux6837ux83dcux6ce5}{%
% \subsection{四、学做几样菜泥}%ux56dbux5b66ux505aux51e0ux6837ux83dcux6ce5}}

% %ux83dcux6ce5}{%
% \subsubsection{菜泥}%ux83dcux6ce5}}

% 油菜(白菜、菜花、菠菜等)300克。将鲜菜洗净切碎,加少量水煮烂,捞出,放箩上,用匙碾碎,筛落者为菜泥。

% %ux5357ux74dcux6ce5}{%
% \subsubsection{南瓜泥}%ux5357ux74dcux6ce5}}

% 南瓜(红薯、胡萝卜、土豆等)300克。将原料去皮洗净切小块,放水煮烂,捣成泥。

% %ux86cbux9ec4ux6ce5}{%
% \subsubsection{蛋黄泥}%ux86cbux9ec4ux6ce5}}

% 鸡蛋1个。将鸡蛋洗净煮熟,去壳取蛋黄,用匙搅烂加少许开水。可将蛋黄泥调入牛奶、米汤、菜水中食用。

% %ux9c7cux6ce5}{%
% \subsubsection{鱼泥}%ux9c7cux6ce5}}

% 净鱼肉100克。将鱼洗净放开水煮,取出剥皮,去骨刺,把肉研碎。再加水将鱼肉煮烂。

% \begin{quote}
% \textbf{育儿小百科}

% 儿童营养学家向家长推荐的``十佳儿童食品'':
% \end{quote}

% \begin{enumerate}
% \def\labelenumi{\arabic{enumi}.}
% \item
%   \begin{quote}
%   新鲜水果
%   \end{quote}
% \item
%   \begin{quote}
%   绿色蔬菜
%   \end{quote}
% \item
%   \begin{quote}
%   脱脂奶
%   \end{quote}
% \item
%   \begin{quote}
%   去皮鸡肉
%   \end{quote}
% \item
%   \begin{quote}
%   鱼肉
%   \end{quote}
% \item
%   \begin{quote}
%   谷类
%   \end{quote}
% \item
%   \begin{quote}
%   瘦牛肉
%   \end{quote}
% \item
%   \begin{quote}
%   全麦饼干
%   \end{quote}
% \item
%   \begin{quote}
%   玉米片、土豆片
%   \end{quote}
% \item
%   \begin{quote}
%   爆玉米花
%   \end{quote}
% \end{enumerate}

% %ux4e94ux8981ux9632ux6b62ux5b69ux5b50ux5403ux76d0ux8fc7ux591a}{%
% \subsection{五、要防止孩子吃盐过多}%ux4e94ux8981ux9632ux6b62ux5b69ux5b50ux5403ux76d0ux8fc7ux591a}}

% 一些家长在给儿童调剂食物时,常以大人的口味来调剂孩子的日常饮食,让孩子长期处于被动高盐之中,这对儿童健康极为不利。

% 对一些学龄儿童进行调查发现,吃含盐量过高食物的儿童有11\%\textasciitilde13\%患了高血压。此外,食入盐分太多,还会导致体内的钾从尿中丧失,而钾对人体活动时肌肉的收缩、放松是必需的,钾丧失过多,能引起心脏衰弱而死亡。

% 当然,适量的食盐对维护人体健康起着重要的生理作用,这不仅因为食盐是人们生活中不可缺少的调味品,又能为人体提供重要的营养元素钠和氯,且能维护人体的酸碱平衡及渗透压平衡,是合成胃酸的重要物质,可促进胃液、唾液的分泌,增强唾液中淀粉酶的活性,增进食欲,因此,孩子不可缺食盐。但孩子机体功能尚未健全,肾脏功能发育不够完善,没有能力充分排出血液中过多的钠,而过多的钠能潴留体内水液,促使血量增加,血管呈高压状态,于是发生血压升高,心脏负担加重。

% 家长们一定要注意在给儿童做食物时,应稍微淡点,千万不要以自己的味觉为准。

% %ux516dux5b9dux5b9dux708aux6c34ux8981ux79d1ux5b66}{%
% \subsection{六、宝宝炊水要科学}%ux516dux5b9dux5b9dux708aux6c34ux8981ux79d1ux5b66}}

% \begin{enumerate}
% \def\labelenumi{\arabic{enumi}.}
% \item
%   不喝冰水
% \end{enumerate}

% 孩子活动量大,浑身是汗,十分口渴,总喜欢喝一杯冰汽水,尽管当时喝着舒服,但喝冰水易引起胃黏膜血管收缩,影响消化,还能刺激胃肠蠕动加快,出现肠痉挛,引起腹痛。

% \begin{enumerate}
% \def\labelenumi{\arabic{enumi}.}
% \setcounter{enumi}{1}
% \item
%   睡前不喝水
% \end{enumerate}

% 不少孩子没有养成晚上自己控制排尿的习惯,在大量喝水后,很易遗尿,若是因尿憋醒,会影响睡眠质量。

% \begin{enumerate}
% \def\labelenumi{\arabic{enumi}.}
% \setcounter{enumi}{2}
% \item
%   可适当喝些饮料
% \end{enumerate}

% 最好是喝点果汁,如橘子汁、橙汁、西瓜汁、番茄汁等,这些饮料热量低,营养素多;此外,牛奶也可多喝,因为其营养价值高。

% \begin{enumerate}
% \def\labelenumi{\arabic{enumi}.}
% \setcounter{enumi}{3}
% \item
%   不要喝生水
% \end{enumerate}

% 孩子性急,当口渴难忍而又没有开水、凉开水时,有的孩子就要喝生水,尤其是农村的孩子,这样易发生胃肠道疾病。

% \begin{enumerate}
% \def\labelenumi{\arabic{enumi}.}
% \setcounter{enumi}{4}
% \item
%   喝水不要过快
% \end{enumerate}

% 小孩子不喝水则已,一喝水常一口气喝上一大碗,极易造成急性胃扩张,也不利于水的吸收。要给孩子讲清不宜喝水快的道理,养成慢慢喝,一口口喝的习惯。

% \textbf{8个月宝宝1日营养计划}

% \begin{longtable}[]{@{}
%   >{\raggedright\arraybackslash}p{(\columnwidth - 2\tabcolsep) * \real{0.5000}}
%   >{\raggedright\arraybackslash}p{(\columnwidth - 2\tabcolsep) * \real{0.5000}}@{}}
% \toprule()
% \begin{minipage}[b]{\linewidth}\raggedright
% 时间
% \end{minipage} & \begin{minipage}[b]{\linewidth}\raggedright
% 主食及用量
% \end{minipage} \\
% \midrule()
% \endhead
% 早上6:00 & 配方奶200\textasciitilde220毫升 \\
% 上午9:30 & 馒头20克,鸡蛋羹20克,母乳或配方奶120毫升 \\
% 上午10:00 & 果泥50克,白开水100毫升 \\
% 中午12:00 & 菜肉馄饨或鲜虾水饺50克,饺子汤50毫升 \\
% 下午3:00 &
% 母乳或配方奶120毫升,蛋糕20克,水果适量20\textasciitilde30克 \\
% 晚上6:00 & 肉末胡萝卜豆腐羹60克 \\
% 晚上9:00 & 配方奶100毫升 \\
% 鱼肝油 & 每日1-2滴 \\
% 备注 &
% 只要宝宝需要可多给宝宝喝果汁。喂配方奶应逐渐增加,直至第10个月完全代替母乳。辅食品种应搭配食用。 \\
% \bottomrule()
% \end{longtable}

% %ux7b2cux4e5dux8282-8-9ux4e2aux6708ux5a74ux513fux7684ux5582ux517bux6307ux5bfc}{%
% \subsection{09第九节
% 8-9个月婴儿的喂养指导}%ux7b2cux4e5dux8282-8-9ux4e2aux6708ux5a74ux513fux7684ux5582ux517bux6307ux5bfc}}

% 一般来说,8\textasciitilde12个月是宝宝断奶的\textbf{最佳时期}。宝宝断奶并不是不吃奶,而是在饮食上以饭菜为主,以奶制品为辅。8个月的宝宝一天可以添加三次辅食。宝宝每天的辅食应包括蛋、豆、鱼、肉、五谷、蔬菜及水果等,以达到营养平衡的目的。尽量使宝宝从一日三餐的辅食中摄取2/3的营养,其余1/3从奶中补充。辅食应以软嫩、半流质食物为主,口味要清淡。

% 这一时期的宝宝每天可以只吃两次母乳,时间可安排在早晨6点起床后和晚上9点睡觉前。母乳充足的妈妈可以喂三次,但必须保证让宝宝从辅食中获取至少2/3的营养,以后逐渐减少母乳量,让宝宝\textbf{循序}进入正式的断奶期。

% 这一时期的宝宝还应保证一定量的牛奶,每次吃完辅食后,最好给宝宝喝100\textasciitilde150毫升的奶,而且全天总奶量(包括母乳)不得少于600毫升。

% 人工喂养的孩子每天需750毫升牛奶,分3次喂,上、下午各喂1顿辅食。

% %ux4e00ux98dfux8c31ux8bbeux8ba1ux8981ux70b9}{%
% \subsection{一、食谱设计要点}%ux4e00ux98dfux8c31ux8bbeux8ba1ux8981ux70b9}}

% 孩子8个月时,消化蛋白质的胃液已经充分发挥作用了,所以可多吃一些蛋白质食物,如豆腐、奶制品、鱼、瘦肉末等。孩子吃的肉末,必须是新鲜的瘦肉,可剁碎后加作料蒸烂吃。应该注意,增加辅食时每次只增加一种,当孩子已经适应了,并且没有什么不良反应时,再增加另外一种。此外,只有当孩子处于饥饿状态时,才更容易接受新食物。所以,新增加的辅食应该在奶前吃,喂完辅食之后再喂奶。

% %ux4e8cux65adux5976ux7684ux51c6ux5907}{%
% \subsection{二、断奶的准备}%ux4e8cux65adux5976ux7684ux51c6ux5907}}

% 第8个月时,妈妈乳汁的质和量都已经开始下降,难以完全满足宝宝生长发育的需要。所以添加辅食显得更为重要。从这个阶段起,可以让宝宝尝尝配方奶的味道,为断掉母乳后添加乳类食品做好准备。辅食方面,可以让宝宝尝试更多种类的食品。由于此阶段大多数宝宝都在学习爬行,体力消耗也较多,所以应该供给更多的碳水化合物、脂肪和蛋白质类食品。

% 每餐都要保持营养均衡。与上个月相比,这个时期的辅食中增加了可吃的食物材料,并可以吃肉了。从脂肪较少的鸡胸脯肉开始,还可以吃鲑鱼或金枪鱼。从这个月的后半期开始可以吃蛋白,可以给宝宝吃整个鸡蛋了。每天的食谱中,应包括谷类、蛋白质及适当蔬菜和水果。

% %ux4e09ux5b9dux5b9dux7684ux98dfux7269ux5f62ux6001}{%
% \subsection{三、宝宝的食物形态}%ux4e09ux5b9dux5b9dux7684ux98dfux7269ux5f62ux6001}}

% 这个阶段,应为宝宝添加柔嫩、半固体的食物,如碎菜、鸡蛋、粥、面条、鱼、肉末等。在为宝宝制作米粥时,应以米加7倍的水熬制而成。有的宝宝在此时并不喜欢吃粥,而对成人吃的米饭感兴趣,这时,妈妈也可以给宝宝喂一些米饭,但是米饭一定要做得稍微软烂一些。

% %ux56dbux8f85ux98dfux7684ux786cux5ea6ux6807ux51c6}{%
% \subsection{四、辅食的硬度标准}%ux56dbux8f85ux98dfux7684ux786cux5ea6ux6807ux51c6}}

% 虽然有的宝宝已经出牙,但也不能马上给宝宝吃硬的食物。对于难以吃下硬食物的宝宝,可以在以前黏糊状食物中加入一种粒状的食物,让宝宝\textbf{逐渐}习惯起来。在此时期,比起黏糊食物,有许多宝宝更喜欢有形状的容易弄碎的食物。可以在煮熟的南瓜、滑溜的土豆泥中加入碎蔬菜或剔下的鱼肉等给宝宝吃。

% 硬度可以以豆腐为标准。大人用手指弄碎来试,使其达到能轻易碎开的程度。这是一个宝宝用舌头弄碎粒状或有形的食物,同时有意识地去咬的时期。如果总给宝宝吃糊糊状的食物,就不能锻炼宝宝咬的能力。

% %ux4e94ux786cux5ea6ux7684ux5927ux5c0fux548cux6807ux51c6}{%
% \subsection{五、硬度的大小和标准}%ux4e94ux786cux5ea6ux7684ux5927ux5c0fux548cux6807ux51c6}}

% 以南瓜为例:把南瓜煮软。开始的标准是呈黏糊状,小心地弄碎后,残有少量固体颗粒。后半期的标准是用刀叉等轻轻地弄碎,全部是固态颗粒状。

% %ux516dux5a74ux513fux5c0fux98dfux54c1ux5236ux4f5c}{%
% \subsection{六、婴儿小食品制作}%ux516dux5a74ux513fux5c0fux98dfux54c1ux5236ux4f5c}}

% \textbf{香蕉玉米面糊}:把玉米面2大匙和半杯牛奶一起放入锅内,上火煮至玉米面熟了为止,再加剥皮后的香蕉1根切成薄片和少许蜂蜜煮片刻。

% \textbf{肉面条}:把面条放入热水中煮后切成小段,和2小匙猪肉末一起放入锅内,加海味汤后用微火煮,再加适量酱油,把淀粉用水调匀后倒人锅内搅拌均匀后停火。

% \textbf{虾糊}:把虾剥去外壳,洗干净后用开水煮片刻,然后研碎,再放入锅内加肉汤煮,煮熟后加入用水调匀淀粉和少量盐,使其呈糊状后停火。

% \textbf{奶油鱼}:把收拾干净的鱼放热水中煮过后研碎,把酱油倒入锅内加少量肉汤,再加切碎的鱼肉上火煮,边煮边搅拌,煮好后放入少许奶油和切碎的芹菜即可。

% %ux4e03ux5b66ux505aux51e0ux79cdux7ca5}{%
% \subsection{七、学做几种粥}%ux4e03ux5b66ux505aux51e0ux79cdux7ca5}}

% %ux732aux809dux7ca5}{%
% \subsubsection{猪肝粥}%ux732aux809dux7ca5}}

% 原料:猪肝泥25克,豆腐25克,胡萝卜泥20克,软饭或烂粥一小碗。\\
% 制法:将豆腐压碎,与猪肝泥、胡萝卜泥放入软饭中煮沸,煮软,即可。

% %ux7ea2ux67a3ux5c71ux836fux7ca5}{%
% \subsubsection{红枣山药粥}%ux7ea2ux67a3ux5c71ux836fux7ca5}}

% 原料:红枣15个,山药250克,大米500克,白糖适量。

% 制法:

% 红枣泡发,去核切丁,山药去皮切丁,

% 将大米熬粥,后将红枣、山药放入,熟烂即可。

% %ux841dux535cux7cb3ux7c73ux7ca5}{%
% \subsubsection{萝卜粳米粥}%ux841dux535cux7cb3ux7c73ux7ca5}}

% 原料:萝卜500克,粳米30克。\\
% 制法:

% 将萝卜煮熟,取汁,

% 煮粥,加入萝卜汁同煮。

% %ux6842ux82b1ux751cux85d5ux7ca5}{%
% \subsubsection{桂花甜藕粥}%ux6842ux82b1ux751cux85d5ux7ca5}}

% 原料:糯米250克,嫩藕1节,桂花酱少许。\\
% 制法:

% 将藕刨成浆,去渣留汁;

% 藕汁和糯米加水煮,煮熟加入桂花酱。

% %ux867eux84c9ux7ca5}{%
% \subsubsection{虾蓉粥}%ux867eux84c9ux7ca5}}

% 原料:鲜虾100克,白米50克,香菜末、葱花、盐适量。

% 制法:

% 将米洗净加水煮;

% 虾去壳、去筋,用生粉拌匀;

% 粥快热时放入虾再煮,离火时放入香菜末和葱花。

% %ux7f8aux809dux80e1ux841dux535cux7ca5}{%
% \subsubsection{羊肝胡萝卜粥}%ux7f8aux809dux80e1ux841dux535cux7ca5}}

% 原料:羊肝50克,胡萝卜30克,大米30克,蒜、葱、姜、盐适量。\\
% 制法:

% 羊肝去筋洗净刮成蓉;

% 胡萝卜蒸熟去皮碾成泥,

% 热油爆蒜蓉后,倒入肝泥略炒;

% 将大米熬成粥,加入胡萝卜泥,焖15\textasciitilde20分钟,再加肝泥、盐、葱花即可。

% %ux809dux9ec4ux7ca5}{%
% \subsubsection{肝黄粥}%ux809dux9ec4ux7ca5}}

% 原料:猪肝50克,鸡蛋1个,大米粥50克。

% 制法:

% 猪肝去筋洗净刮成蓉;

% 鸡蛋煮熟取黄碾成泥,

% 将肝蓉和蛋黄泥加入大米粥,小火煮10\textasciitilde15分钟。

% %ux732aux8840ux7ca5}{%
% \subsubsection{猪血粥}%ux732aux8840ux7ca5}}

% 原料:猪血200克,大米50克,葱、胡椒粉适量。

% 制法:

% 先将米煮粥,

% 将猪血切块,放清水中浸泡;

% 粥快熟时加入猪血,煮开,

% 撒上葱花及胡椒粉。

% %ux7effux8c46ux7ca5}{%
% \subsubsection{绿豆粥}%ux7effux8c46ux7ca5}}

% 原料:绿豆50克,粳米100克,白糖50克。

% 制法:

% \begin{enumerate}
% \def\labelenumi{\arabic{enumi}.}
% \item
%   将绿豆洗净,温水浸涨,与米同放锅内,加清水煮;
% \item
%   食用时加白糖调匀。
% \end{enumerate}

% \textbf{9个月宝宝1日营养计划}

% \begin{longtable}[]{@{}
%   >{\raggedright\arraybackslash}p{(\columnwidth - 2\tabcolsep) * \real{0.5000}}
%   >{\raggedright\arraybackslash}p{(\columnwidth - 2\tabcolsep) * \real{0.5000}}@{}}
% \toprule()
% \begin{minipage}[b]{\linewidth}\raggedright
% 时间
% \end{minipage} & \begin{minipage}[b]{\linewidth}\raggedright
% 主食及用置
% \end{minipage} \\
% \midrule()
% \endhead
% 早上6:00 & 母乳或配方奶200毫升 \\
% 上午8:00 & 米粥,鸡蛋羹1/2-1碗;面包 \\
% 上午10:00 & 白开水100毫升,饼干2块 \\
% 中午12:00 & 软饭或稠粥1/2-1碗,鸡蛋1个,蔬菜2-3大匙 \\
% 下午3:00 & 配方奶150-200毫升,小点心1块,水果50-80克 \\
% 晚上6:00 & 排骨汤面,鱼肉,蔬菜1小碗 \\
% 晚上9:00 & 配方奶100毫升 \\
% 鱼肝油 & 每日1-2滴 \\
% 备注 &
% 只要宝宝需要可多给宝宝喝白开水,吃水果。中午吃的蔬菜可选菠菜,大白菜,胡萝卜等 \\
% \bottomrule()
% \end{longtable}

% %ux7b2cux5341ux8282-9-10ux4e2aux6708ux5a74ux513fux7684ux5582ux517bux6307ux5bfc}{%
% \subsection{10第十节
% 9-10个月婴儿的喂养指导}%ux7b2cux5341ux8282-9-10ux4e2aux6708ux5a74ux513fux7684ux5582ux517bux6307ux5bfc}}

% 此阶段喂哺原则与第8个月大致相同,喂奶次数应逐渐从3次减到2次。每天哺乳600\textasciitilde800毫升就足够了。而辅食要逐渐增加,\textbf{为断奶做好准备}。从现在起可以增加一些粗纤维食物,如茎秆类蔬菜,但要把粗、老的部分去掉。9个月的宝宝已经长牙,有咀嚼能力了,可以让其啃食硬一点的东西,这样有利于乳牙的萌出。

% 饮食中应注意添加面粉类的食物,其中的碳水化合物可为宝宝提供每天活动和生长的热量,其中含有的蛋白质可促进宝宝身体组织的生长发育。

% 宝宝在饥饿状态下更易接受新食物,因此,添加新的辅食一定要在喂奶之前喂食,两餐的辅食内容最好不一样,以便让宝宝逐渐适应各种不同的味道。

% 9个多月的婴儿应增加一些土豆、白薯等含糖较多的根茎类食物,增加一些粗纤维的食物如蔬菜,但要把粗的老的部分去掉。9个多月的孩子已经长牙,有咀嚼能力了,可以让他啃硬一点的东西。

% %ux4e00ux98dfux8c31ux8bbeux8ba1ux8981ux70b9-1}{%
% \subsection{一、食谱设计要点}%ux4e00ux98dfux8c31ux8bbeux8ba1ux8981ux70b9-1}}

% 从9个月起,宝宝可以接受的食物明显增多,应试着逐渐增加宝宝的饭量,使宝宝主要营养的摄取由以奶为主转为以辅食为主。由于宝宝的食物构成逐渐发生变化,选择食物要得当,烹调食物要尽量做到色、香、味俱全,以适应宝宝的消化能力,并引起宝宝的食欲。

% 进入第9个月,就不要再给宝宝食用流食了,应制作一些有利于训练咀嚼的辅食,可以逐渐增加食物的硬度和大小。千万不要因为宝宝吐出来或囫囵吞下就放弃,否则,就不能达到训练咀嚼的目的了。

% %ux4e8cux5b9dux5b9dux8f85ux98dfux81eaux5df1ux5236ux4f5cux597dux8fd8ux662fux8d2dux4e70ux6210ux54c1ux597d}{%
% \subsection{二、宝宝辅食自己制作好还是购买成品好}%ux4e8cux5b9dux5b9dux8f85ux98dfux81eaux5df1ux5236ux4f5cux597dux8fd8ux662fux8d2dux4e70ux6210ux54c1ux597d}}

% 因为很忙,很多妈妈不能亲自给宝宝制作辅食,又会担心市场上的产品出现问题,心里会觉得内疚。其实,市场上销售的婴儿食品品种丰富,搭配的种类也很多,可以充分地利用。比如有五谷强化营养糊(大米、江米、小米、薏米、黑米和小麦粉)、五豆强化营养糊(黄豆、绿豆、红小豆、豌豆、黑豆、小麦粉),五仁强化营养糊(核桃仁、花生仁、杏仁、莲子、黑芝麻),就是选用多种五谷杂粮进行科学搭配后精制而成,另外还有以优质大米、全脂奶粉和大豆分离蛋白为主要原料,以果糖低聚糖(双歧因子)、苹果粉、猕猴桃粉、胡萝卜粉、菠菜粉、番茄粉、豌豆粉为辅料制造而成的营养米粉,在产品质量有所保证的前提下,利用这些辅食喂养宝宝非常方便而且营养丰富。妈妈可以在忙的时候喂购买的婴儿食品,有空的时候亲手给宝宝制作一些``家常饭'',比如煮面条、面片、菜粥、肉粥等,把多样的食物组合起来给宝宝吃,既增加辅食的口味又能保证营养均衡,宝宝也喜欢吃。

% %ux4e09ux968fux65f6ux8c03ux6574ux98dfux7269ux611fux89c9}{%
% \subsection{三、随时调整食物感觉}%ux4e09ux968fux65f6ux8c03ux6574ux98dfux7269ux611fux89c9}}

% 观察宝宝的大便。如果出现腹泻,可能是宝宝对添加的新食物或食物的硬度不能接受,以致消化不良,需要停止添加这种食物。如果大便中带有未消化的食物,需要降低食物的摄入量或将食物做得更细小些。

% \begin{quote}
% 你知道吗?

% \textbf{婴儿辅食少盐、不放糖}

% 宝宝的饭菜尽量不要放糖,糖在肠道中发酵,会使宝宝肠胀气、腹泻,大便呈绿色,同时会使宝宝产生饱腹感,不爱吃饭,情绪烦躁。8个月后,宝宝的食物中可少量放盐,程度以成人觉得没有什么咸味为宜。
% \end{quote}

% %ux56dbux5a74ux513fux5c0fux98dfux54c1ux5236ux4f5c}{%
% \subsection{四、婴儿小食品制作}%ux56dbux5a74ux513fux5c0fux98dfux54c1ux5236ux4f5c}}

% %ux716eux767dux85af}{%
% \subsubsection{煮白薯}%ux716eux767dux85af}}

% 把白薯洗干净去皮后切成4个\textbf{薄片},把苹果洗净去皮除核后也切成薄片,然后把白薯和苹果的薄片先后放入锅内,加入少许水后用微火煮,煮好后放入蜂蜜。

% %ux829dux9ebbux8c46ux8150}{%
% \subsubsection{芝麻豆腐}%ux829dux9ebbux8c46ux8150}}

% 豆腐1/6块用开水紧后控去水分,然后研碎再加入炒熟的芝麻、豆酱、淀粉各一匙混合均匀后做成饼状,再放入容器中用锅蒸15分钟即可。此食品的特点是非常松软,
% 易消化。

% %ux4e94ux5206ux6e05ux662fux7eafux9178ux5976ux8fd8ux662fux4e73ux9178ux996eux6599}{%
% \subsection{五、分清是纯酸奶还是乳酸饮料}%ux4e94ux5206ux6e05ux662fux7eafux9178ux5976ux8fd8ux662fux4e73ux9178ux996eux6599}}

% 有的孩子消化功能低下,可以\textbf{自制酸奶}给孩子吃,即在100毫升煮沸消毒过的冷牛奶中,加乳酸0.5\textasciitilde0.8毫升,或6毫升橘子汁,用滴管将酸奶慢慢加入,一边滴一边搅。这种酸奶较牛奶容易吸收。市售的酸奶是用乳酸菌加入鲜奶,使乳糖转化成乳酸,因此它的营养成分与牛奶不同,孩子不能用这种酸奶完全代替牛奶。另外还有许多乳制品饮料,不是全由牛奶制成,其中含有一定比例的水,营养素只相当于牛奶的1/3,这类饮品是饮料,虽然味道好,孩子喜欢喝,但不能用它当作奶喂孩子。

% \begin{quote}
% 育儿小百科

% \textbf{宝宝断奶时的注意事项}

% 9个月的宝宝可以断奶了,但在断奶时,家长要注意以下一些事项:
% \end{quote}

% \begin{enumerate}
% \def\labelenumi{\arabic{enumi}.}
% \item
%   \begin{quote}
%   断奶时间不宜选在夏季,夏天气候炎热,宝宝胃肠道功能减弱,断奶后改吃奶品以外的食物容易引起消化不良,而且夏天细菌繁殖快,食物容易腐败,稍有不慎,就可能引起胃肠道疾病。
%   \end{quote}
% \item
%   \begin{quote}
%   宝宝生病期间不要断奶,因为生病时往往食欲减退,消化功能降低,这时完全断奶改用其他饮食,会使宝宝难以适应,不利于宝宝康复。
%   \end{quote}
% \item
%   \begin{quote}
%   切忌强行断奶。有的家庭为了尽快给宝宝断奶,采取在乳头上抹辣椒、黄连,甚至强迫母子分离一段时间,这样只会使宝宝产生恐惧,影响其身心健康,是不可取的。
%   \end{quote}
% \end{enumerate}

% %ux4e2aux6708ux5b9dux5b9d1ux65e5ux8425ux517bux8ba1ux5212}{%
% \subsubsection{10个月宝宝1日营养计划}%ux4e2aux6708ux5b9dux5b9d1ux65e5ux8425ux517bux8ba1ux5212}}

% \begin{longtable}[]{@{}
%   >{\raggedright\arraybackslash}p{(\columnwidth - 2\tabcolsep) * \real{0.5000}}
%   >{\raggedright\arraybackslash}p{(\columnwidth - 2\tabcolsep) * \real{0.5000}}@{}}
% \toprule()
% \begin{minipage}[b]{\linewidth}\raggedright
% 时间
% \end{minipage} & \begin{minipage}[b]{\linewidth}\raggedright
% 主食及用量
% \end{minipage} \\
% \midrule()
% \endhead
% 早上6:00 & 母乳或配方奶250毫升 \\
% 上午8:00 & 米粥或鸡蛋羹1/2---1碗:面包两片 \\
% 上午10:00 & 白开水100毫升,饼干2块 \\
% 中午12:00 & 软饭或稠粥1/2-1碗,鸡蛋1个,蔬菜2\textasciitilde3大匙 \\
% 下午3:00 & 配方奶150-200毫升,小点心1块,水果50-80克 \\
% 晚上6:00 & 排骨汤面,鱼肉,蔬菜1小碗 \\
% 晚上9:00 & 配方奶100毫升 \\
% 每日 & 鱼肝油1\textasciitilde2滴 \\
% \bottomrule()
% \end{longtable}

% 备注:只要宝宝需要可多给宝宝喝白开水,吃水果,中午吃的蔬菜可选菠菜、大白菜、胡萝卜等。

% %ux7b2cux5341ux4e00ux82821011ux4e2aux6708ux5a74ux513fux7684ux5582ux517bux6307ux5bfc}{%
% \subsection{11第十一节10〜11个月婴儿的喂养指导}%ux7b2cux5341ux4e00ux82821011ux4e2aux6708ux5a74ux513fux7684ux5582ux517bux6307ux5bfc}}

% 这个月可以给宝宝断奶了,但要用自然断奶法,即通过逐步增加辅食的次数和数量,慢慢减少喂哺母乳的次数,在1\textasciitilde2个月的时间内使宝宝断奶。在断奶的过程中,应让宝宝有一个适应的过程。开始时每天先少喂一次母乳,再代之其他的食物,在之后的几周内慢慢减少喂奶次数,并相应增加辅食,逐渐将辅食变成主食,直至最后断掉母乳。刚开始断奶时,宝宝可能会不习惯,若无特殊情况,一定要耐心加喂辅食,坚持按期断奶。

% 10个月的宝宝一般已长出了4\textasciitilde6颗牙齿,有的宝宝出牙较晚,此时才刚刚长出第一颗牙齿。虽然牙齿还很少,但已经会用牙床咀嚼食物,这个动作也能更好地促进宝宝牙齿的发育,这在前几个月的准备下,宝宝进入断奶后期,此时辅食的添加次数也要相应增加。可以把哺乳次数进一步降低为每天不多于2次,让宝宝进食更丰富的食品,以利于各种营养元素的摄入。可以让宝宝尝试全蛋、软饭和各种绿叶菜。既增加营养又锻炼咀嚼能力,同时仍要注意微量元素的添加。

% %ux4e00ux5a74ux513fux5c0fux98dfux54c1ux5236ux4f5c}{%
% \subsection{一、婴儿小食品制作}%ux4e00ux5a74ux513fux5c0fux98dfux54c1ux5236ux4f5c}}

% %ux7599ux7629ux6c64}{%
% \subsubsection{疙瘩汤}%ux7599ux7629ux6c64}}

% 把1/4个鸡蛋和少量水放入一大匙面粉之中,用筷子搅拌成小疙瘩。把切碎的葱头、胡萝卜、圆白菜各2小匙放入肉汤内煮软后,再把面疙瘩一点一点放入肉汤中煮,煮熟之后放少许酱油。

% %ux84b8ux9c7cux997c}{%
% \subsubsection{蒸鱼饼}%ux84b8ux9c7cux997c}}

% 把1/2条鱼去皮和骨、刺后,研碎,与豆腐泥混合均匀做成小饼,放蒸锅内蒸。把鱼汤煮开后加少许作料,最后把蒸过的鱼饼放入鱼汤内煮熟。蒸鱼饼的特点是能够保持鱼肉中的营养成分不被破坏。

% %ux867eux8c46ux8150}{%
% \subsubsection{虾豆腐}%ux867eux8c46ux8150}}

% 小虾2条,豆腐1/10块,嫩豌豆苗2\textasciitilde3根煮后切碎,放入锅内,加切碎的生香菇1/4,加海味汤煮,加白糖和酱油各1小匙,熟时薄薄的勾一点芡。

% \begin{quote}
% 你知道吗?

% \textbf{婴儿忌食什么}
% \end{quote}

% \begin{enumerate}
% \def\labelenumi{\arabic{enumi}.}
% \item
%   \begin{quote}
%   柿子:难消化。
%   \end{quote}
% \item
%   \begin{quote}
%   栗子:服后腹胀,不宜多吃。
%   \end{quote}
% \item
%   \begin{quote}
%   枣:不可生食,可煮粥。
%   \end{quote}
% \item
%   \begin{quote}
%   李子:多食伤脾胃。
%   \end{quote}
% \item
%   \begin{quote}
%   杏:性温,不宜食。
%   \end{quote}
% \item
%   \begin{quote}
%   韭菜:辛辣温热,不易消化。
%   \end{quote}
% \item
%   \begin{quote}
%   黏食:不易消化。
%   \end{quote}
% \item
%   \begin{quote}
%   虾蟹:发物,过敏孩子不宜吃。
%   \end{quote}
% \end{enumerate}

% %ux4e8cux5b66ux505aux51e0ux79cdux679cux6c41}{%
% \subsection{二、学做几种果汁}%ux4e8cux5b66ux505aux51e0ux79cdux679cux6c41}}

% %ux80e1ux841dux535cux82f9ux679cux6c41}{%
% \subsubsection{胡萝卜苹果汁}%ux80e1ux841dux535cux82f9ux679cux6c41}}

% 原料:胡萝卜50克,苹果1个,柠檬2片,蜂蜜2匙,凉开水100克。

% 制法:

% \begin{enumerate}
% \def\labelenumi{\arabic{enumi}.}
% \item
%   原料洗净切碎,榨汁,挤入柠檬汁,搅拌均匀;
% \item
%   将果汁加入蜂蜜,凉开水调匀。
% \end{enumerate}

% %ux8349ux8393ux679cux83dcux6c41}{%
% \subsubsection{草莓果菜汁}%ux8349ux8393ux679cux83dcux6c41}}

% 原料:草莓10个,卷心菜1/6个,胡萝卜1/3个,苹果1/2个,白糖50克,凉开水100克。

% 制法:

% \begin{enumerate}
% \def\labelenumi{\arabic{enumi}.}
% \item
%   将胡萝卜洗净、切碎,榨汁;
% \item
%   将草莓、卷心菜、苹果洗净,切碎,榨汁;
% \item
%   将果汁混合,加入糖、凉开水。
% \end{enumerate}

% %ux9ec4ux74dcux6c41}{%
% \subsubsection{黄瓜汁}%ux9ec4ux74dcux6c41}}

% 原料:黄瓜3条,白糖适量。

% 制法:

% \begin{enumerate}
% \def\labelenumi{\arabic{enumi}.}
% \item
%   将黄瓜洗净切碎,榨汁,加白糖。
% \end{enumerate}

% %ux83e0ux841dux6c41}{%
% \subsubsection{菠萝汁}%ux83e0ux841dux6c41}}

% 原料:菠萝200克,白糖50克,凉开水250克。

% 制法:菠萝去皮,榨汁,加白糖,凉开水调匀。

% \begin{quote}
% v:担心,还小,先不要吧
% \end{quote}

% %ux8461ux8404ux6c41}{%
% \subsubsection{葡萄汁}%ux8461ux8404ux6c41}}

% 原料:鲜葡萄1000克,白糖100克,凉开水500克。

% 制法:

% \begin{enumerate}
% \def\labelenumi{\arabic{enumi}.}
% \item
%   葡萄洗净,去皮,榨汁;
% \item
%   葡萄汁中加水,白糖混匀。
% \end{enumerate}

% %ux8378ux8360ux6c41}{%
% \subsubsection{荸荠汁}%ux8378ux8360ux6c41}}

% 原料:鲜荸荠500克,冰糖250克,水1000克。

% 制法:

% \begin{enumerate}
% \def\labelenumi{\arabic{enumi}.}
% \item
%   荸荠洗净去皮,切碎,榨汁;
% \item
%   将冰糖溶化,加入荸荠汁,再放凉开水中调匀。
% \end{enumerate}

% \begin{quote}
% v:担心寄生虫,大一些再吧
% \end{quote}

% %ux4e09ux9c9cux6c41}{%
% \subsubsection{三鲜汁}%ux4e09ux9c9cux6c41}}

% 原料:鸭梨250克,荸荠250克,鲜藕250克。

% 制法:

% \begin{enumerate}
% \def\labelenumi{\arabic{enumi}.}
% \item
%   原料洗净收拾好;
% \item
%   分别切碎用榨汁机榨汁;
% \item
%   将白糖加入果汁。
% \end{enumerate}

% %ux4e09ux4e0dux8981ux7a81ux7136ux65adux5976}{%
% \subsection{三、不要突然断奶}%ux4e09ux4e0dux8981ux7a81ux7136ux65adux5976}}

% 一般所说的断奶是指断母乳。在半岁以内,母乳是宝宝最好的天然食品。如果1岁以后仍不断奶,母乳就难以满足宝宝生长发育的营养需要了。另外,继续哺乳对宝宝的心理发育也可能产生不良影响。因此,需要适时断奶。

% 需要强调的是,断奶应该是一个长期的、\textbf{逐渐适应}的过程,而不应靠突然让宝宝和母亲分开几天来掐断。一般情况下,断奶的整个过程应该持续半年左右。从宝宝\textbf{4\textasciitilde6}个月起,就应开始为断奶做准备。及时地按一定的时间、顺序为宝宝添加辅食,培养宝宝用勺、碗、杯子吃饭、喝水,同时逐渐减少母乳的喂哺量和喂哺次数,用配方奶替代母乳。

% 一般到10\textasciitilde12个月时,母乳喂养的宝宝应该完全断母乳,也就是断奶。此时的宝宝应该形成1日3次正餐、加餐、加点心的饮食格局,能吃稠粥、烂饭、面条、碎菜、肉末等食物,每天喂奶的次数减到3次左右,用配方奶来替代母乳。

% %ux56dbux8badux7ec3ux5b9dux5b9dux95edux53e3ux5480ux56bc}{%
% \subsection{四、训练宝宝闭口咀嚼}%ux56dbux8badux7ec3ux5b9dux5b9dux95edux53e3ux5480ux56bc}}

% 闭口咀嚼期是宝宝将食物放在舌上,在上腭处将食物碾碎后吞下的时期。这时,宝宝的好奇心会很旺盛,对汤匙、餐具及食物等都很感兴趣,什么都用手抓过来就往嘴里塞。此时期应训练宝宝完成完全闭口上下运动的动作。

% 进入用牙龈咀嚼期时,宝宝应该很熟悉吃饭的动作了,可以让宝宝试着拿吃饭用的汤匙。

% 这个时期宝宝必须学会利用舌头将食物推到左或右边牙龈处,再利用上下牙龈碾碎食物的动作。这对宝宝来说是件挺困难的事,通常要达到成人般熟练地咬食,大概要到2岁左右。

% 由于宝宝已能拿着东西吃了,一定要把水果洗净削皮,并把宝宝的手洗干净再把水果给宝宝。

% %ux4e94ux8f85ux98dfux7684ux91cf}{%
% \subsection{五、辅食的量}%ux4e94ux8f85ux98dfux7684ux91cf}}

% 一般而言,辅食的分量是以宝宝的食量为依据的。只要宝宝有食欲,多喂一些也无妨。但必须\textbf{留意蛋白质的摄取量},如果摄取过多的蛋白质,将会对宝宝尚未发育完全的\textbf{肾脏}造成负担。

% 米饭类的主食和蔬菜就没有摄取上的顾虑了。有些妈妈会担心宝宝会不会因此过胖,其实大可放心,辅食阶段就算把宝宝养得胖胖的,并不代表长大后宝宝就一定会发胖,两者间并无直接关联。如果真的担心宝宝吃得过多,可将辅食稍微调理硬一点,或者将食物的体积切大一些,减少稀饭的水量。由于要花费力气咀嚼,宝宝自然就不会吃太多了。【吃得累吧】

% %ux516dux98dfux8c31ux8bbeux8ba1ux8981ux70b9-1}{%
% \subsection{六、食谱设计要点}%ux516dux98dfux8c31ux8bbeux8ba1ux8981ux70b9-1}}

% 这个阶段最好训练宝宝练习将食物咬成一口口适合自己吞咽的大小。食谱中最好能加上一些可以让宝宝自己拿着吃的食物,例如:香蕉、煮软的胡萝卜等,可以切成细长的条状喂食,食物的硬度要达到宝宝能用牙床压碎的程度。

% 随着每日三餐辅食量的增加,餐后的母乳或配方奶量会逐渐减少。一般来讲,每日可给2次母乳或配方奶,不喜欢喝配方奶的,可以换其他的乳类食品。这个阶段的宝宝每天摄取配方奶的量以500
% \textasciitilde{} 800毫升为宜。

% 这个时期,要注意不要给宝宝吃鳗鱼等\textbf{脂肪}较多的鱼,或如咸鲑鱼等\textbf{盐}分高的食物。另外,当宝宝越来越喜欢活动,就连吃饭也不安分时,不要边追边喂食物。\textbf{当他开始玩时,就要停止喂食},只要宝宝饿了,自然会吃的,不要养成边玩边喂的习惯。

% %ux4e03ux4e0aux684cux5403ux996d}{%
% \subsection{七、上桌吃饭}%ux4e03ux4e0aux684cux5403ux996d}}

% 许多孩子到这个月龄,爱吃饭不爱吃奶,对上桌与父母同吃有极大的兴趣。妈妈可以将孩子抱上桌,在他面前也放一份饭菜,他的饭菜要\textbf{单做},比大人的要\textbf{软些、烂些}。让孩子自己吃,能用勺更好,不能就用手抓东西。尽管吃一点,撒了多半,对孩子的训练是十分重要的。实际上妈妈喂孩子吃比叫他自己吃简单得多,但是虽然带来很多麻烦,妈妈\textbf{还是要给孩子训练的机会}。

% 孩子上桌吃饭,除了吃自己的,他还要爸爸妈妈的菜。可以给他一点尝尝,告诉他什么是酸、甜、苦、辣。但不可以大家你一口我一口地无节制的喂他吃,一是不卫生,二是孩子还没有这份消化能力。

% %ux516bux65adux5976ux665aux597dux4e0dux597d}{%
% \subsection{八、断奶晚好不好}%ux516bux65adux5976ux665aux597dux4e0dux597d}}

% 不好。宝宝断奶的时间最好是在出生后的第8-12个月。过早断奶,就必须添加过多的辅食,此时宝宝的消化功能尚未发育完全,容易引起消化不良、腹泻,影响健康。过晚断奶,因母乳的量及所含营养物质都逐渐减少,不能满足生长发育的需要,常会导致宝宝发生各种营养缺乏症。因此,断奶不宜太晚。应在宝宝4-6个月的时候开始添加辅食,使宝宝养成习惯吃母乳(或配方奶)以外的食物,经过一段适应过程,逐步用辅食代替母乳,需要半年左右的时间,宝宝由吃母乳(或配方奶)转成吃饭,逐渐完成断奶。假如不做好断奶的准备,认定断奶时间到了,就突然不给宝宝吃奶,这种做法会使宝宝感到不适应,影响宝宝情绪,也容易引起疾病。

% \textbf{11个月宝宝1日营养计划}

% \begin{longtable}[]{@{}
%   >{\raggedright\arraybackslash}p{(\columnwidth - 2\tabcolsep) * \real{0.5000}}
%   >{\raggedright\arraybackslash}p{(\columnwidth - 2\tabcolsep) * \real{0.5000}}@{}}
% \toprule()
% \begin{minipage}[b]{\linewidth}\raggedright
% 时间
% \end{minipage} & \begin{minipage}[b]{\linewidth}\raggedright
% 主食及用量
% \end{minipage} \\
% \midrule()
% \endhead
% 早上6:00 & 配方奶250毫升 \\
% 上午8:00 & 馒头片20克,番茄虾仁60克,紫菜汤80毫升 \\
% 上午10:00 & 白开水150毫升,饼干2块 \\
% 中午12:00 & 油菜香菇面1碗 \\
% 下午3:00 & 配方奶150-200毫升,小点心1块,水果50-80克 \\
% 晚上6:00 & 软米饭30克,红烧平鱼50克,蔬菜1小碗 \\
% 晚上9:00 & 配方奶100毫升 \\
% 鱼肝油 & 每日1-2滴 \\
% \bottomrule()
% \end{longtable}

% 备注:只要宝宝需要可多给宝宝喝白开水,吃水果。蔬菜可选南瓜、芹菜,油菜等

% %ux7b2cux5341ux4e8cux82821112ux4e2aux6708ux5a74ux513fux7684ux5582ux517bux6307ux5bfc}{%
% \subsection{12第十二节11〜12个月婴儿的喂养指导}%ux7b2cux5341ux4e8cux82821112ux4e2aux6708ux5a74ux513fux7684ux5582ux517bux6307ux5bfc}}

% 11个多月的孩子仍应每天早晚喂奶,3餐喂饭。

% 孩子出生之后是以乳类为主食,经过1年的时间要逐渐过渡到以谷类为主食。快1岁的孩子可以吃软饭、面条、小包子、小饺子了。每天\textbf{3餐应变换花样},使孩子有食欲。

% \begin{quote}
% 可以写个程序

% 1.可以随机

% 2.带一些统计,出现的少的,权重大一些

% 3.要灵活,可以有默认的,也可以添加,删除

% 甚至,以这个书为理论基础,出一个app...
% \end{quote}

% %ux4e00ux589eux52a0ux8425ux517bux4fc3ux8fdbux751fux957f}{%
% \subsection{一、增加营养促进生长}%ux4e00ux589eux52a0ux8425ux517bux4fc3ux8fdbux751fux957f}}

% 11个月的宝宝普遍已长出了上下切牙,能咬下较硬的食物。这个阶段的喂哺也要逐步向幼儿期过渡,餐数适当减少,每餐量相应增加。婴儿期最后两个月是宝宝身体生长较迅速的时期,需要更多的碳水化合物、脂肪和蛋白质。

% 这个月的宝宝断奶已接近完成期,母乳喂哺应尽量减少,或者干脆断掉。这个月如果不及时给宝宝断奶,可能会影响宝宝的食欲。断奶后,可以让宝宝和大人一样在早、午、晚按时进食,并让宝宝养成在固定时间进食牛奶、饼干、水果等食物的习惯,可以在宝宝吃完辅食之后喂牛奶,一次喂100-200毫升,每天的总奶量应控制在500-600毫升。

% 为了避免造成宝宝偏食,应尽可能让宝宝品尝到各种食物。另外,还要注意宝宝的营养均衡,应尽量使宝宝的饮食含有优质蛋白质、足够的钙和维生素等营养成分。

% %ux4e8cux8ba9ux5b9dux5b9dux548cux5927ux4ebaux4e00ux8d77ux7528ux9910}{%
% \subsection{二、让宝宝和大人一起用餐}%ux4e8cux8ba9ux5b9dux5b9dux548cux5927ux4ebaux4e00ux8d77ux7528ux9910}}

% \hspace{0pt}\includegraphics[width=2.96503in,height=2.67133in]{media/rId815.png}\hspace{0pt}

% 这个月里,宝宝的进餐已经接近规律,每日三餐可以和大人的进餐时间安排在一起。当宝宝看到大人吃饭的样子时,宝宝的嚼食动作也会有所进步。

% 当宝宝和大人一同进餐时,吃饭时间要以宝宝为准,如果有家庭成员回家较晚也不要再等了,应按原定时间吃饭,以便养成宝宝规律进餐的习惯。在吃饭时,妈妈要先喂宝宝,然后自己再吃。

% 有时宝宝会想吃大人的食物,但是不要给他,因为大人的食物对宝宝来说又硬又咸,不适合宝宝吃。

% \begin{quote}
% 你知道吗

% \textbf{宝宝腹泻不宜过于限制饮食}

% 当宝宝腹泻时,为了配合治疗而适量限制宝宝的饭量,可使消化道充分休息,减少腹泻次数。但长时间控制饮食,会使宝宝\textbf{食量变得过小,胃肠功能减弱},如再增加乳量,会引起腹泻,这种腹泻称为饥饿性腹泻。此时患儿大便多呈黏液便,不成形。虽然次数多,但每次量少,化验无异常,大便培养呈阴性。这说明这种腹泻是非感染性的,不需要用药,只要逐渐加强营养,改善喂养方法,增加辅食即可好转,绝不能滥用抗生素或反复限制饮食。
% \end{quote}

% %ux4e09ux571fux8c46ux7684ux5904ux7406ux65b9ux6cd5}{%
% \subsection{三、土豆的处理方法}%ux4e09ux571fux8c46ux7684ux5904ux7406ux65b9ux6cd5}}

% 土豆质地细软,是此阶段宝宝不可或缺的辅食。土豆是断奶期宝宝的优良食品。土豆素有``植物之王''的美誉,被称做``第二面包''、``十全十美的上等食品''。土豆热量很低,碳水化合物大部分是优质淀粉,很容易消化,不伤肠胃,而且土豆所含的氨基酸也易被吸收。土豆的营养成分非常丰富,土豆蛋白质含量高而且质量好,接近动物性蛋白。它含有特殊的黏蛋白,不但有润肠作用,还有脂类代谢作用,能帮助胆固醇代谢。

% (一)\textbf{土豆的营养集中在皮的附近,所以土豆不宜削皮后食用}。最好在加热后,再将皮剥掉,保留有效成分。

% (二)烹调时,可以将土豆洗净,切成条状,放入水中浸泡5分钟,取出后放入碗中上蒸熟至\textbf{牙签可轻易穿插}为宜。

% \begin{quote}
% gpt,另:

% 在加热过程中,土豆皮可以起到保护内部营养成分不受热量损害的作用。待土豆熟透后,再将皮剥掉。这样可以最大程度地保留土豆的营养价值。然而,需要注意的是,土豆皮对某些人来说可能难以消化,因此,需要根据个人的消化情况来决定是否食用土豆皮。
% \end{quote}

% %ux56dbux5a74ux513fux98dfux54c1ux5236ux4f5c}{%
% \subsection{四、婴儿食品制作}%ux56dbux5a74ux513fux98dfux54c1ux5236ux4f5c}}

% %ux756aux8304ux996dux5377}{%
% \subsubsection{番茄饭卷}%ux756aux8304ux996dux5377}}

% 将1/2个鸡蛋调匀后放平锅内摊成薄片,将切碎的胡萝卜和葱头各1/2小匙用油炒软,再加入软米饭1小碗,和番茄2小匙拌匀,将混合后的米饭平摊在蛋皮上,然后卷成卷儿,切成小卷子状食用。

% %ux8089ux4e38ux7ca5}{%
% \subsubsection{肉丸粥}%ux8089ux4e38ux7ca5}}

% 鸡肉末1大匙,将1大匙葱头放油在锅内炒过,再与鸡肉末一起混合做成鸡肉丸子,把鸡汤倒入锅内加鸡肉丸子煮,开锅后再将米饭放入一起煮,煮熟时加少许盐。

% %ux8089ux677eux996d}{%
% \subsubsection{肉松饭}%ux8089ux677eux996d}}

% 鸡肉末1大匙放入锅内,加少许白糖、酱油、料酒,边煮边搅拌使之均匀混合,煮好后放米饭1碗焖熟,熟后切一片花型胡萝卜在上面做装饰。

% %ux8c46ux8150ux996d}{%
% \subsubsection{豆腐饭}%ux8c46ux8150ux996d}}

% 把半块豆腐放在开水中煮一下,切成小方块,将1碗米饭放入锅内加海味汤一起煮,煮软后入豆腐和少许酱油,最后撒少许青菜末,再稍煮片刻即可。

% %ux4e94ux53d8ux6362ux82b1ux6837ux7ed9ux5b69ux5b50ux591aux5403ux86cb}{%
% \subsection{五、变换花样给孩子多吃蛋}%ux4e94ux53d8ux6362ux82b1ux6837ux7ed9ux5b69ux5b50ux591aux5403ux86cb}}

% 完整的记忆是事物在中枢神经系统留下的痕迹,记忆力的强弱与\textbf{乙酰胆碱}有关。乙酰胆碱对大脑有兴奋作用,使大脑维持觉醒状态并具有一定的反应性,也可促使条件反射巩固,从而改善人们的记忆力。蛋黄含有卵磷脂和甘油三酯,卵磷脂在肠内被消化液中的酶消化后,释放出胆碱,胆碱直接进入脑部后与醋酸结合生成有助于改善记忆的乙酰胆碱。这里主张儿童要多吃蛋,是为了使他们的智力发展得更快更好。孩子总吃煮鸡蛋,就厌烦了,可以变花样做了给他们吃。

% %ux51e4ux51f0ux86cb}{%
% \subsubsection{凤凰蛋}%ux51e4ux51f0ux86cb}}

% 原料:鸡蛋50克,瘦肉50克。盐、酱油、味精、淀粉、胡椒粉、葱末、白糖、马蹄。

% 制法:先将鸡蛋煮熟去皮,用盐水腌备用。瘦肉剁成泥,加酱油、味精、盐、白糖、淀粉、水打起劲。马蹄切成小丁。葱切成小丁,放入馅中搅拌。取出鸡蛋控净水,用肉馅裹上,滚上淀粉。锅放宽油上旺火,将鸡蛋炸呈金黄色捞出,每个鸡蛋切成4块装盘。

% %ux51e4ux773cux9e4cux9e51ux86cb}{%
% \subsubsection{凤眼鹌鹑蛋}%ux51e4ux773cux9e4cux9e51ux86cb}}

% 原料:鹌鹑蛋5个,虾饺50克,面包50克,生抽、花生油适量。

% 制法:鹌鹑蛋煮熟去壳,每个切成两半,面包切成片。将虾饺放在面包上,鹌鹑蛋镶嵌在虾饺中间,蛋黄向上,如凤眼状。下入油锅中炸至金黄色。

% %ux693fux82bdux70d8ux86cb}{%
% \subsubsection{椿芽烘蛋}%ux693fux82bdux70d8ux86cb}}

% 原料:鸡蛋1个,嫩香椿10克,精盐、湿淀粉各适量。

% 制法:将鸡蛋磕入碗内搅匀加湿淀粉,再放入精盐,香椿(切碎)继续搅匀。炒锅置中火上,上油烧热,移至小火上,舀起油待用,倒入蛋液,并将舀起的油淋人蛋液中间,回盖,烘约6分钟,翻面,再烘3分钟,滗去余油,盛入盘内即成。

% %ux9ec4ux57d4ux7092ux86cb}{%
% \subsubsection{黄埔炒蛋}%ux9ec4ux57d4ux7092ux86cb}}

% 原料:去壳鸡蛋50克,精盐、味精各适量。

% 制法:鸡蛋液加入味精,精盐,搅成蛋浆。炒锅洗净放在中火上,下油搪锅后倒回油盆,再下油倒入蛋浆,边倒边铲动,边下油,炒至刚熟装盘。

% %ux73caux745aux9526ux7ee3ux86cb}{%
% \subsubsection{珊瑚锦绣蛋}%ux73caux745aux9526ux7ee3ux86cb}}

% 原料:鹌鹑蛋4个,鲜菇50克,菜心15\textasciitilde20克,蟹肉20克,蟹黄20克,味精、酱油、清汤、花生油、香油、胡椒粉、生抽、老抽、淀粉、蚝油适量。

% 制法:先将鲜菇煨好,菜心炒好。鹌鹑蛋放水中煮熟去壳,上点生油,锅放宽油烧热,将蛋炸金黄色捞出,而后煨制入味。鲜菇也要煨制入味。装盘时鲜菇在中,菜心围在边,鹌鹑蛋排在鲜菇中间,砌成圆形。锅放清汤、盐、胡椒粉、生抽、蚝油,加蟹肉、蟹黄、淀粉勾薄芡,浇在菇面上,以露出蛋、菜心为好。

% %ux8d5bux8783ux87f9}{%
% \subsubsection{赛螃蟹}%ux8d5bux8783ux87f9}}

% 原料:黄花鱼肉100克,鸭蛋黄1个,葱末、姜末、姜汁、湿淀粉、料酒、味精、花生油、精盐、清汤适量。

% 制法:将黄花鱼去皮,上笼蒸熟,用手撕碎。鸭蛋黄倒入碗内用筷子搅匀。另一碗内放清汤、精盐、湿淀粉、料酒、葱末、姜汁,兑成``爆汁''。炒锅内放油,旺火烧热,倒入鸭蛋黄搅炒,再放上鱼肉炒,把已兑好的``爆汁''倒入炒锅内,颠翻一下盛入盘中撒上姜末。

% \textbf{12个月宝宝1日营养计划}

% %ux7b2cux4e09ux7bc7-ux5e7cux513fux7684ux8425ux517bux65b9ux6848}{%
% \section{3第三篇
% 幼儿的营养方案}%ux7b2cux4e09ux7bc7-ux5e7cux513fux7684ux8425ux517bux65b9ux6848}}

% 第一节 1岁1〜3个月宝宝的营养方案

% 第二节1岁4〜6个月宝宝的营养方案

% 第三节1岁7〜9个月宝宝的营养方案

% 第四节1岁10〜12个月宝宝的营养方案

% 第五节2岁1\textasciitilde3个月宝宝的营养方案

% 第六节2岁4〜6个月宝宝的营养方案

% 第七节2岁7〜9个月宝宝的营养方案

% 第八节2岁10〜12个月宝宝的营养方案

% %ux7b2cux4e00ux8282-1ux5c8113ux4e2aux6708ux5b9dux5b9dux7684ux8425ux517bux65b9ux6848}{%
% \subsection{01第一节
% 1岁1〜3个月宝宝的营养方案}%ux7b2cux4e00ux8282-1ux5c8113ux4e2aux6708ux5b9dux5b9dux7684ux8425ux517bux65b9ux6848}}

% 宝宝处于以乳类为主食向普通食物转化的时期,这个阶段的喂哺原则是营养要全面,以保证身体生长需要。三餐热量要根据幼儿活动的规律合理分配。食物品种要多样化,一周内的食谱尽量不重复,以保证宝宝良好的食欲。

% %ux4e00ux98dfux8c31ux8bbeux8ba1ux8981ux70b9-2}{%
% \subsection{一、食谱设计要点}%ux4e00ux98dfux8c31ux8bbeux8ba1ux8981ux70b9-2}}

% 1岁左右的孩子,逐渐变为以一日三餐为主,早、晚牛奶为辅,慢慢过渡到完全断奶。如果正好在夏天,为了不影响孩子的食欲,可以略向后推迟12个月再断奶,最晚\textbf{不要超过15月龄}。

% 以三餐为主之后,家长一定要注意保证孩子辅食的质量。如肉泥、蛋黄、肝泥、豆腐等含有丰富的蛋白质,是孩子身体发育必需的食品,而米粥、面条等主食是孩子补充热量的来源,蔬菜可以补充维生素、矿物质和纤维素,促进新陈代谢,促进消化。孩子的主食主要有米粥、软饭、面片、龙须面、馄饨、豆包、小饺子、馒头、面包、糖三角等。周岁孩子每日的膳食量大致可以这样供给:粮食100克左右,牛奶500毫升加糖25克(分早晚两次喝),瘦肉类30克,猪肝泥20克,鸡蛋1个,植物油5克,蔬菜150-200克,水果150克。

% 1岁的孩子,鱼肝油要加3滴、每日2次,钙片每次1克,每日2次。

% %ux4e8cux5e7cux513fux81b3ux98dfux5236ux6cd5}{%
% \subsection{二、幼儿膳食制法}%ux4e8cux5e7cux513fux81b3ux98dfux5236ux6cd5}}

% %ux4e09ux9c9cux86cbux7fb9}{%
% \subsubsection{三鲜蛋羹}%ux4e09ux9c9cux86cbux7fb9}}

% 把1-2个鸡蛋打入碗中,加少许食盐和凉开水打匀,放入锅中蒸熟,然后再切几个新鲜虾仁与炒好的肉菜末放进碗中搅匀,再继续蒸5-8分钟,停火后即可食用。

% %ux6df7ux5408ux83dcux7ccaux9762}{%
% \subsubsection{混合菜糊面}%ux6df7ux5408ux83dcux7ccaux9762}}

% 将土豆、胡萝卜洗净,上锅蒸熟去皮压烂成泥。番茄用开水烫去皮,切成碎块,放入锅中煸炒,再加上少许食盐与土豆胡萝卜泥、肝泥和熟肉末一起炒熟后食用。

% %ux679cux7fb9}{%
% \subsubsection{果羹}%ux679cux7fb9}}

% 将苹果、百合、山药、梨、莲子洗净去皮去核切成小片加上琼脂一同放在火上加水煮热。离火加白糖凉后食用。没有琼脂可用藕粉代替。

% %ux4e09ux5b9dux5b9dux65adux5976ux540eux7684ux98dfux8c31ux7279ux70b9}{%
% \subsection{三、宝宝断奶后的食谱特点}%ux4e09ux5b9dux5b9dux65adux5976ux540eux7684ux98dfux8c31ux7279ux70b9}}

% 从每天保证600毫升奶,逐渐过渡到以粮食、奶、蔬菜、鱼、肉、蛋为主的混合食品,这些食品是满足孩子生长发育必不可少的。适当喂养面条、米粥、馒头、小饼干等,以提高热量。经常给宝宝吃各种蔬菜、水果、海产品,提供足够的维生素和无机盐,以供代谢的需要,达到营养平衡的目的。常食用些动物血、肝类,以保证铁的供应。烹制方法多样化,注意色、香、味、形,且要细、软、碎。不宜煎、炒、爆,以利消化。

% %ux56dbux57f9ux517bux5b9dux5b9dux826fux597dux7684ux996eux98dfux4e60ux60ef}{%
% \subsection{四、培养宝宝良好的饮食习惯}%ux56dbux57f9ux517bux5b9dux5b9dux826fux597dux7684ux996eux98dfux4e60ux60ef}}

% \hspace{0pt}\includegraphics[width=2.65734in,height=2.64336in]{media/rId840.png}\hspace{0pt}

% 一般情况下,12\textasciitilde18个月的宝宝已能独立进餐,只要训练得当,宝宝可以吃得很好。反之,如果父母总是担心宝宝吃不好,吃不饱,一直喂宝宝吃饭,那就会使宝宝产生依赖性。等到3岁后再改变吃饭习惯就比较困难了。因此,父母应及早培养宝宝的饮食习惯。

% 当宝宝满1周岁时,就可以让他试着拿小勺吃饭了,虽然刚开始宝宝只能吃几口,主要还需要大人喂,但只要坚持下去,宝宝到1岁半左右就能独立吃饭了。

% 循循善诱,鼓励宝宝自己吃饭。当宝宝已能独立吃几口时,大人应该及时鼓励,提高宝宝自己吃饭的兴趣和自信心。

% 如果宝宝的依赖性很强,可采取这样的方法:连续几天给宝宝做他最喜欢吃的饭菜,把饭菜盛好放在宝宝面前,成人暂时离开几分钟,然后回到宝宝身边。如果宝宝能吃上几口,则给予表扬,鼓励他继续吃完。如果宝宝仍不愿意自己吃,大人也不要对宝宝发火,要帮助他把饭吃完。几天之内多次重复这种方法后,宝宝饿了自然会自己拿起餐具吃饭。

% 在训练过程中要注意,父母千万不能一看宝宝不愿意自己吃,担心宝宝饿着,饭菜凉了,就赶快妥协,主动去喂宝宝吃饭,这样只能助长宝宝的惰性。

% \begin{quote}
% 你知道吗?

% 不要给宝宝吃以下有损宝宝大脑发育的食物:

% ①过咸的食物。②含味精多的食物。③含过氧化脂质的食物,如腊肉、熏鱼等。④含铅的食物,如爆米花、松花蛋等。⑤含铝的食物,如油条、油饼等。
% \end{quote}

% \textbf{一日食谱}

% \begin{longtable}[]{@{}
%   >{\raggedright\arraybackslash}p{(\columnwidth - 2\tabcolsep) * \real{0.5000}}
%   >{\raggedright\arraybackslash}p{(\columnwidth - 2\tabcolsep) * \real{0.5000}}@{}}
% \toprule()
% \begin{minipage}[b]{\linewidth}\raggedright
% 时间
% \end{minipage} & \begin{minipage}[b]{\linewidth}\raggedright
% 食物
% \end{minipage} \\
% \midrule()
% \endhead
% 上午 8:00 & 母乳或配方奶250毫升,肉松粥1小碗,鸡蛋1个 \\
% 上午 10:00 & 3\textasciitilde4片饼干,50毫升酸奶 \\
% 中午 12:00 & 软饭1小碗,碎肝炒青椒1小盘,菜叶汤半碗 \\
% 下午 15:00 & 香蕉1根,蛋糕1块 \\
% 晚上 18:00 & 肉末胡萝卜饺子1小盘 \\
% 晚上 21:00 & 母乳或配方奶250毫升 \\
% \bottomrule()
% \end{longtable}

% %ux7b2cux4e8cux82821ux5c8146ux4e2aux6708ux5b9dux5b9dux7684ux8425ux517bux65b9ux6848}{%
% \subsection{02第二节1岁4〜6个月宝宝的营养方案}%ux7b2cux4e8cux82821ux5c8146ux4e2aux6708ux5b9dux5b9dux7684ux8425ux517bux65b9ux6848}}

% 此阶段幼儿的消化器官尚在完善中,虽然已经在吃普通食物,但不能与成人饮食相同,应强调碎、软、新鲜,忌食煎炸、过甜、过咸、过酸和刺激性食品。主食以谷类为主,要勤换花样。保证肉、蛋、奶各类蛋白质的供应,以满足这个时期身体发育的需要。

% 随着孩子乳牙的陆续萌出,咀嚼消化的功能较前成熟,在喂养上与前2个月相比略有变化,每日进餐次数为5次,3餐中间上下午各加一次点心。有条件的还可以继续每日加一个鸡蛋和250克牛奶。

% %ux4e00ux98dfux8c31ux8bbeux8ba1ux8981ux70b9-3}{%
% \subsection{一、食谱设计要点}%ux4e00ux98dfux8c31ux8bbeux8ba1ux8981ux70b9-3}}

% 孩子的膳食安排尽量做到花色品种多样化,荤素搭配,粗细粮交替,保证每日能食入足量的蛋白质、脂肪、糖类以及维生素、矿物质等。培养孩子良好的饮食习惯能使孩子保持较好的食欲,避免孩子挑食、偏食和吃过多的零食。为了保证维生素A、胡萝卜素、钙、铁等营养素的摄入,孩子应多食用黄、绿色新鲜蔬菜,如油菜、小菠菜、胡萝卜、番茄、甜柿椒、红心白薯。萝卜、白菜、芥菜头等蔬菜所含维生素、矿物质虽较黄、绿色蔬菜低,但也具有不可缺少的营养价值。每日还要吃一些水果。含维生素A较多的水果有柑橘类、枣、山楂、猕猴桃等。除此之外,每日吃鱼肝油2次,每次仍为3滴,钙片每日2次,每次1克。

% %ux4e8cux54eaux4e9bux98dfux54c1ux542bux9499ux591a}{%
% \subsection{二、哪些食品含钙多}%ux4e8cux54eaux4e9bux98dfux54c1ux542bux9499ux591a}}

% 对于小孩来说,奶类是其补充钙的最好来源,母乳中500毫升奶含钙170毫克,牛奶和羊奶含钙都很高,奶中的钙容易被消化吸收。蔬菜中含钙质高的是绿叶菜。如大家熟悉的油菜、雪里蕻、空心菜、太古菜等,食后吸收也比较好。给孩子食用绿叶菜,最好洗净后用开水烫一下,这样可以去掉大部分的草酸,有利于钙的吸收。豆类含钙也比较丰富,每100克黄豆中含360毫克的钙质,每100克豆皮中含钙284毫克。含钙特别高的食品还有海带、虾片、紫菜、麻酱、骨髓酱等。

% %ux4e09ux8425ux517bux4e0eux667aux529bux53d1ux80b2ux7684ux5173ux7cfb}{%
% \subsection{三、营养与智力发育的关系}%ux4e09ux8425ux517bux4e0eux667aux529bux53d1ux80b2ux7684ux5173ux7cfb}}

% 许多科学家对不同国家的儿童的智力发育与营养关系进行了研究,发现营养不足的孩子的反应性、想象力、智力都不如营养良好的儿童。日本的科学家曾对6对1\textasciitilde3岁的双胞胎进行对比研究,给每对中的一个改善蛋白质的质与量(补充了几种必需的氨基酸),经过两三年后,补充氨基酸的与没补充的相比其智力要高出10倍以上。营养差的孩子认识事物反应性慢,思维的能力偏低,记忆力和语言表达能力均差,自然影响孩子的学习成绩。如果发现孩子营养不良,家长应及时采取措施,在医生的指导下给孩子制订合理的食谱,改善营养状况。

% %ux56dbux5b69ux5b50ux4e0dux7231ux559dux725bux5976ux600eux4e48ux529e}{%
% \subsection{四、孩子不爱喝牛奶怎么办}%ux56dbux5b69ux5b50ux4e0dux7231ux559dux725bux5976ux600eux4e48ux529e}}

% 一般来说,1岁半的孩子每天还应喝250毫升牛奶,因为牛奶是比较好的营养品,既易消化又含有多种营养素,是婴幼儿生长发育不可缺少的食物。但是有的孩子到了1岁多,尝到五谷香,便不爱喝牛奶了。对不爱喝牛奶的孩子不要勉强,可用蛋羹、豆浆、豆乳等与奶交替喂孩子。

% %ux4e94ux6bcfux5929ux591aux5c11ux86cbux767dux8d28ux5bf9ux5b69ux5b50ux5408ux9002}{%
% \subsection{五、每天多少蛋白质对孩子合适}%ux4e94ux6bcfux5929ux591aux5c11ux86cbux767dux8d28ux5bf9ux5b69ux5b50ux5408ux9002}}

% 蛋白质是生命的物质基础。人体的每一个部位、组织、细胞都含蛋白质。如果缺乏蛋白质,人体就会代谢紊乱,发生贫血、浮肿,易患各种疾病,孩子则生长发育迟缓。

% 1岁半的孩子每天大约需要多少蛋白质呢?
% 一般在40克左右,其中至少应有一半是动物蛋白。

% 具体地说,1岁半的孩子每天最好吃250毫升牛奶,1\textasciitilde2个鸡蛋,30克瘦肉,一些豆制品,有条件再吃一些肝、排骨或鱼,这样就能够基本满足孩子生长发育的需要了。

% %ux516dux600eux6837ux6444ux5165ux86cbux767dux8d28}{%
% \subsection{六、怎样摄入蛋白质}%ux516dux600eux6837ux6444ux5165ux86cbux767dux8d28}}

% 蛋白质一般来自植物和动物类食品。植物类食品主要是豆类。动物类食品多指鱼、肉、蛋、奶等。在植物类食品中,含有优质蛋白的有大豆、芝麻、葵花子等,其他如米、面虽含有蛋白质,但量不高。在肉类食品中,虽然含蛋白质较丰富,但不可以食用过量,过量的食用会给机体代谢带来负担,产生过量的代谢产物,对身体不利。

% %ux4e03ux8102ux80aaux5bf9ux4ebaux4f53ux7684ux4f5cux7528}{%
% \subsection{七、脂肪对人体的作用}%ux4e03ux8102ux80aaux5bf9ux4ebaux4f53ux7684ux4f5cux7528}}

% 胖人不怕冷。这是因为我们能用手捏起的皮下脂肪具有保温作用。从生理角度讲,脂肪还具有以下作用:

% \begin{enumerate}
% \def\labelenumi{\arabic{enumi}.}
% \item
%   为机体产热、释放能量。脂肪是营养素主要来源,产热量要比蛋白质和糖都高,约达1.52倍。一般我们食用的脂肪,一部分被消耗利用,一部分用于体内贮存起来。当人体饥饿时脂肪就会释放能量,为身体产生热量。
% \item
%   是构成人体细胞的成分(主要是磷脂和胆固醇等),也是构成脑和神经组织的主要成分。组织细胞的各种膜叫细胞膜,都离不开脂类与蛋白质成分。
% \item
%   促进脂溶性维生素的吸收。维生素可分为脂溶性与水溶性两种,脂溶性维生素包括A、D、E、K等。此类维生素吸收条件是溶于脂肪中才能被吸收利用。
% \end{enumerate}

% %ux516bux78b3ux6c34ux5316ux5408ux7269ux5bf9ux4ebaux4f53ux7684ux4f5cux7528}{%
% \subsection{八、碳水化合物对人体的作用}%ux516bux78b3ux6c34ux5316ux5408ux7269ux5bf9ux4ebaux4f53ux7684ux4f5cux7528}}

% 碳水化合物最基本的作用就是供给机体能量。按照中国人的饮食习惯,以谷类为主,占每天膳食中所供给身体热量的60\%\textasciitilde70\%。因此,吃好每顿饭是孩子身体所需热量的重要保证。

% 另外,\textbf{患有肝炎的人常多吃些糖},因为糖在肝内可以协助肝解毒,提高抵抗细菌的能力。人体组织细胞的代谢合成都离不开糖的协助,脂肪转为能量也需要糖起作用,如果没有糖的参与,脂肪利用不完全,就可能产生一种不利于人体的酸性物质,蓄积过多就可能引起酸中毒,后患无穷。

% \textbf{一日食谱}

% \begin{longtable}[]{@{}
%   >{\raggedright\arraybackslash}p{(\columnwidth - 2\tabcolsep) * \real{0.5000}}
%   >{\raggedright\arraybackslash}p{(\columnwidth - 2\tabcolsep) * \real{0.5000}}@{}}
% \toprule()
% \begin{minipage}[b]{\linewidth}\raggedright
% 时间
% \end{minipage} & \begin{minipage}[b]{\linewidth}\raggedright
% 食物
% \end{minipage} \\
% \midrule()
% \endhead
% 上午 8:00 & 母乳或配方奶150毫升,面包25克,荷包鸡蛋1个 \\
% 上午 10:00 & 饼干少许,酸奶50\textasciitilde100毫升 \\
% 中午 12:00 & 稠米粥1碗,豆腐60克 \\
% 下午 15:00 & 香蕉或苹果100克,小点心1块 \\
% 晚上 18:00 & 软饭1碗,珍珠汤1碗 \\
% 晚上 21:00 & 母乳或配方奶250毫升 \\
% \bottomrule()
% \end{longtable}

% %ux7b2cux4e09ux82821ux5c8179ux4e2aux6708ux5b9dux5b9dux7684ux8425ux517bux65b9ux6848}{%
% \subsection{03第三节1岁7〜9个月宝宝的营养方案}%ux7b2cux4e09ux82821ux5c8179ux4e2aux6708ux5b9dux5b9dux7684ux8425ux517bux65b9ux6848}}

% 这个阶段宝宝的乳牙已经大部分出齐,消化能力进一步提高。在膳食安排上可以比照成人的饮食内容。此后,乳品不再是宝宝的主食,但尽量保证每天饮用牛奶,以获取更佳的蛋白质。宝宝的食品应当尽量细、软、烂,以利于营养成分的吸收。

% %ux4e00ux8425ux517bux4e0dux826fux4f1aux5f71ux54cdux5b9dux5b9dux89c6ux529b}{%
% \subsection{一、营养不良会影响宝宝视力}%ux4e00ux8425ux517bux4e0dux826fux4f1aux5f71ux54cdux5b9dux5b9dux89c6ux529b}}

% \hspace{0pt}\includegraphics[width=2.58741in,height=2.46154in]{media/rId859.png}\hspace{0pt}

% 眼睛是人体的重要器官,如果不注意用眼卫生或用眼过度,比如看书、看电视、看电脑时间太长,光线太暗,坐姿不正确,都可以引起眼睛的疲劳造成视力减退。尤其是宝宝年龄小,更不宜让眼睛过度疲劳,不能让宝宝长时间或近距离地观看电视。

% 此外,专家发现偏食对视力发育有非常明显的影响。由于偏食导致营养不均衡,所以影响眼球的发育。无论是蛋白质或是维生素缺乏,都可造成近视或近视的进一步发展。肉、蛋制品中含有大量的蛋白,但维生素较少,而在肝脏、蔬菜、水果中含有丰富的维生素,但蛋白质较少。因此为了眼球的正常发育,预防近视的发生,减缓近视的发展,应养成合理的饮食习惯,多吃新鲜蔬菜、水果、粗粮、豆制品、海带等,少吃糖果等,切不可偏食。

% %ux4e8cux5bf9ux773cux775bux6709ux76caux7684ux98dfux7269}{%
% \subsection{二、对眼睛有益的食物}%ux4e8cux5bf9ux773cux775bux6709ux76caux7684ux98dfux7269}}

% 含有维生素A及胡萝卜素的食物。当宝宝饮食长期缺乏维生素A或胡萝卜素的时候,眼睛对黑暗环境的适应能力会减退,严重的还容易患夜盲症、干眼病。维生素A的最好来源是各种动物的肝脏(民间常以羊肝为明目佳品)、鱼肝油、奶类和蛋类及富含胡萝卜素的植物性的食物,比如胡萝卜、苋菜、菠菜、韭菜、青椒、红薯,以及橘子、杏、柿子等胡萝卜素含量较高的水果。

% 含有维生素C的食物。维生素C是组成眼球水晶体的成分之一,如果缺乏维生素C就容易患水晶体浑浊的白内障。含维生素C丰富的食物有:各种新鲜蔬菜和水果,其中尤其以青椒、黄瓜、菜花、小白菜、鲜枣、生梨、橘子等维生素C含量为高。

% 丰富的钙质对眼睛也是有好处的。钙具有消除眼部肌肉紧张的作用。豆类、绿叶蔬菜、虾皮含钙量都比较丰富。烧排骨汤、酥鱼、糖醋排骨等烹调方法可以增加钙的吸收量。

% 蛋白质是组成细胞的主要成分,组织的修补更新需要不断地补充蛋白质,多补充蛋白质有助于眼睛消除疲劳。如瘦肉、禽肉、动物的内脏、鱼虾、奶类、蛋类、豆类等,里面都含有丰富的蛋白质。

% %ux4e09ux9884ux9632ux5b9dux5b9dux8d2bux8840}{%
% \subsection{三、预防宝宝贫血}%ux4e09ux9884ux9632ux5b9dux5b9dux8d2bux8840}}

% 据调查,我国儿童缺铁性贫血的发生率较高,0-6岁的宝宝中30\%-40\%有不同程度的贫血,其中以7个月到2岁的宝宝贫血发生率为最高。人工喂养的宝宝比母乳喂养的宝宝贫血率几乎高出一倍。

% 缺铁性贫血会影响宝宝的行为和智力发育,出现视觉和听觉发育水平和学习能力下降。有的宝宝出现异食癖,喜欢吃土块、粉笔等异物。儿童期宝宝血液中血清铁的含量也与智商成正比。因此,在为2岁宝宝安排饮食时,每周至少要给宝宝吃1-2次猪肝、猪血及动物内脏类食物,提供给宝宝容易消化吸收的铁质。

% 食物中铁的吸收率十分重要。母乳中铁含量虽少,但易于吸收,所以母乳喂养的宝宝患贫血的较少。动物食品中的血红蛋白易于吸收,所以动物血和内脏对防治宝宝贫血效果较好。

% 植物中的铁质有些不易被人体吸收,如菠菜虽含铁丰富,但菠菜中的草酸易与铁结合,使宝宝难以吸收利用。植物食品中黄豆、黑豆、黑芝麻、红果、红枣、黑木耳、深色蔬菜以及水果中都含有丰富的铁质,可以在日常饮食中多给宝宝食用。除非宝宝贫血特别严重时才需要在医生指导下用药物补铁。

% %ux56dbux8425ux517bux5931ux8861ux5bfcux81f4ux60c5ux7eeaux5f02ux5e38}{%
% \subsection{四、营养失衡导致情绪异常}%ux56dbux8425ux517bux5931ux8861ux5bfcux81f4ux60c5ux7eeaux5f02ux5e38}}

% \hspace{0pt}\includegraphics[width=2.39161in,height=2.75524in]{media/rId865.png}\hspace{0pt}

% 宝宝长期情绪多变,爱激动、喜吵闹或情绪暴躁等,应考虑是否其甜食吃得过多,诱发了这些情绪。另外,儿童肥胖症、近视、多动症、低智力、龋齿等疾病,也与甜食摄入过多有关。家长应限制宝宝食糖的摄入量,平衡宝宝的饮食。

% 如果宝宝性格郁郁寡欢、反应迟钝、表情麻木等,应考虑其体内缺乏蛋白质、多种维生素等营养素,其结果会导致机体免疫力下降、贫血、智力低下等。家长应多给宝宝补充水产品、肉类、奶制品等高蛋白食物,并给宝宝吃些含维生素丰富的番茄、橘子、苹果、青菜等蔬菜水果。

% 如果宝宝经常忧心忡忡、惊恐不安或健忘,应考虑其是否缺乏B族维生素。缺乏B族维生素会导致食欲不振、脚气病,影响生长发育,影响脑神经的反应能力,抑制思维能力等。家长可适当在饮食中补充些粗粮(如谷类)、蛋黄、猪肝、核桃仁、奶制品、土豆等含维生素B丰富的食品。

% 如果宝宝夜间常常手脚抽筋、磨牙,多为缺钙的表现。如果宝宝常感头晕目眩或气虚,可能是缺铁所致。如果宝宝有异食癖,则为缺锌、锰等微量元素所致。宝宝缺乏常量元素和微量元素,会造成发育不良及多种疾病。家长应让宝宝多食用含钙量丰富的奶制品、鱼、虾皮等。多食用含铁量丰富的海带、木耳、蘑菇等。多食用含锌、锰量丰富的禽类及牡蛎等海产品。

% \textbf{一日食谱}

% \begin{longtable}[]{@{}
%   >{\raggedright\arraybackslash}p{(\columnwidth - 2\tabcolsep) * \real{0.5000}}
%   >{\raggedright\arraybackslash}p{(\columnwidth - 2\tabcolsep) * \real{0.5000}}@{}}
% \toprule()
% \begin{minipage}[b]{\linewidth}\raggedright
% 08:00
% \end{minipage} & \begin{minipage}[b]{\linewidth}\raggedright
% 母乳或配方奶150毫升,鸡肝面条1碗
% \end{minipage} \\
% \midrule()
% \endhead
% 10:00 & 酸奶50毫升,小点心1块 \\
% 12:00 & 软饭1碗,蒸肉豆腐70克,虾皮紫菜汤1小碗 \\
% 15:00 & 香蕉或苹果100克,饼干2块,母乳或配方奶150毫升 \\
% 18:00 & 二米粥(小米、大米)1碗,黄瓜沙拉50克 \\
% 21:00 & 母乳或配方奶200毫升 \\
% \bottomrule()
% \end{longtable}

% %ux7b2cux56dbux82821ux5c811012ux4e2aux6708ux5b9dux5b9dux7684ux8425ux517bux65b9ux6848}{%
% \subsection{04第四节1岁10〜12个月宝宝的营养方案}%ux7b2cux56dbux82821ux5c811012ux4e2aux6708ux5b9dux5b9dux7684ux8425ux517bux65b9ux6848}}

% 这个阶段宝宝的主食以米、面、杂粮等谷类为主,是热能的主要来源。蛋白质主要来自肉、蛋、乳类、鱼类等食物。钙、铁和其他矿物质主要来自蔬菜,部分来自动物性食品。维生素主要来自水果、蔬菜。膳食要做到碎、软、烂,鱼肉要去骨去刺,花生、核桃要制成酱,不要吃刺激性食品。

% %ux4e00ux4e0dux8981ux7ed9ux5b9dux5b9dux5403ux8fc7ux591aux5473ux7cbe}{%
% \subsection{一、不要给宝宝吃过多味精}%ux4e00ux4e0dux8981ux7ed9ux5b9dux5b9dux5403ux8fc7ux591aux5473ux7cbe}}

% 一些年轻的父母见宝宝厌食或胃口不好而不愿吃饭时,往往会在菜肴中多加些味精,以使饭菜味道鲜美来刺激宝宝的食欲,或者让宝宝1次进食大量美味的鸡鸭鱼肉而不加控制,这种做法是不可取的。

% 味精的主要成分的是谷氨酸钠。医学专家研究发现,味精(即谷氨酸钠)进入人体后,在肝脏中被谷氨酸丙酮酸转移酶转化,生成谷氨酸后再被人体吸收。近年,德国科学家通过研究证实,过量的谷氨酸能把婴幼儿血液中的锌逐渐带走,导致机体缺锌。大量食入谷氨酸钠,能使血液里的锌转变为谷氨酸锌,从尿中过多的排泄体外。

% 锌是人体所需的重要微量元素,具有维持人体正常生长发育的作用,对于婴幼儿来说更是不可缺少的。而婴幼儿一旦缺锌,便可出现味觉迟钝,甚至厌食,日久造成智力减退、生长迟缓、性晚熟等不良后果。

% 婴儿期很多宝宝以母乳喂养为主,如果乳母食用过量味精,大量的谷氨酸可通过乳汁进入宝宝体内,同样可导致宝宝缺锌的后果。为此分娩3个月以内的用母乳喂养宝宝的妈妈也应少食味精。

% 一味追求鲜美还会使宝宝产生美味综合征。美味综合征的病因是鸡、鸭、鱼、肉等美味食品中含有较多的谷氨酸钠,它是味精的主要成分,食入过多会使新陈代谢出现异常,导致疾病的发生。美味综合征的表现一般是在进食后半小时发病,出现头昏脑涨、眩晕无力、心慌、气喘等症状。有些宝宝会表现为上肢麻木,下肢颤抖,个别的则表现为恶心及上腹部不适。

% 因此,对于正在生长发育的宝宝,美味佳肴不可1次吃得过多。宝宝的自控力差,容易偏食,父母在饮食上一定要把好关,有粗有细,荤素搭配,不可暴饮暴食,引起美味综合征。

% \begin{quote}
% 你知道吗

% \textbf{为何孩子容易缺锌?}

% 人体是一个复杂的机器,许多微量元素发挥着各自的作用。锌元素就是其中之一。锌参与人体的基础代谢,主宰味觉,同时可调节防病免疫功能。

% 锌在动物类食品中含量较高,在植物类食品中含量较低。如长期摄入植物类食品会影响对锌的吸收。这是因为植物中的粗纤维、植酸、草酸等物质会干扰人体对锌的吸收与利用。如果过多食用精制食品,会造成锌的摄入量不足。还有人在生病吃药时也会造成锌的丢失。

% 要想不缺锌,除了不偏食外,还可以多吃些干果、硬壳食物,这些食物含锌量都较高。
% \end{quote}

% %ux4e8cux4ebaux4f53ux6bcfux65e5ux9700ux8981ux591aux5c11ux6c34}{%
% \subsection{二、人体每日需要多少水?}%ux4e8cux4ebaux4f53ux6bcfux65e5ux9700ux8981ux591aux5c11ux6c34}}

% 人体每日需要饮多少水才能维持生命的活动呢?这要因年龄而异。成人每天需水量约为2,500毫升,孩子是按千克体重来计算的,一般每日每千克体重需要水60-100毫升。这里包括食物中的水分。例如,10千克重的孩子,在不吃不喝的情况下,静脉补充水分为600-1,000毫升。在发烧或剧烈活动、天气酷热的情况下,还需根据身体情况增加水分。正常情况下,水的食入量和排出量是相对平衡的。

% 大家可以注意一下:在腹泻时,尿量就少,这说明从大便排出的水分较多,尿就少了。这也可说明体内存在脱水情况,指示我们要给体内适量补充水分和无机盐。

% %ux4e09ux5404ux79cdux7ef4ux751fux7d20ux7684ux4f5cux7528}{%
% \subsection{三、各种维生素的作用}%ux4e09ux5404ux79cdux7ef4ux751fux7d20ux7684ux4f5cux7528}}

% \begin{itemize}
% \item
%   维生素A的作用可以使人体皮肤光滑健康,保护人体眼睛,并抵抗疾病的侵袭。
% \item
%   维生素D可以促进钙质的吸收和利用,预防小儿佝偻病。
% \item
%   维生素E可以保护心脏和骨骼肌肉的健康,延缓人类衰老过程,并有抵抗空气污染物对人体的影响的作用。
% \item
%   维生素B1有稳定人的情绪,增强记忆力和活力的功能。
% \item
%   维生素B2可预防口腔黏膜溃疡,是蛋白质、脂肪和糖代谢过程中需要的酶类,也是促进儿童生长必需的物质。
% \item
%   尼克酸能促进碳水化合物、脂肪和氨基酸的代谢作用,并可降低人体血液中超量的胆固醇,防止血管硬化,使人保持旺盛的精力。
% \item
%   维生素B6可调节人体中枢神经系统活动,稳定情绪,能协助产生抗体,促进皮肤健康。
% \item
%   维生素B12可协助人体神经系统工作,并维持正常红细胞的生成过程。
% \item
%   叶酸能协助形成红细胞。
% \item
%   维生素C的作用是可以促进牙齿和骨骼的生长,促进骨折和外伤的愈合过程,并能抵抗传染病和其他疾病。
% \end{itemize}

% %ux56dbux4ebaux4f53ux65e0ux673aux76d0ux542bux91cfux662fux591aux5c11}{%
% \subsection{四、人体无机盐含量是多少}%ux56dbux4ebaux4f53ux65e0ux673aux76d0ux542bux91cfux662fux591aux5c11}}

% 无机盐是多种微量元素的总称,一般是通过日常各种食物摄入体内,以满足人体的需要。

% \begin{itemize}
% \item
%   钙的需要量在800\textasciitilde1,500毫克。最佳食物来源有杏仁、甜菜、啤酒、酵母、甘薯、奶制品、鱼、豆腐等。
% \item
%   磷的需要量约800毫克。最佳食物来源是肉、坚果、鱼卵、谷粒等。
% \item
%   钾的需要量为800\textasciitilde1,300毫克。最佳食物来源是橘子、苹果、杏等食品。
% \item
%   镁需要量为300毫克。最佳食物来源为糙米、绿叶菜、蜜糖、坚果、豌豆、大豆等。
% \item
%   铁需要量在15\textasciitilde18毫克。鳄梨、牛肉、花生、牛油、番茄、土豆、葡萄干等含铁较多。铁的最佳来源是绿叶菜、瘦肉、猪肝类、杏、小麦胚等。
% \item
%   碘一日可需150微克,来源主要是海带和其他海味。
% \item
%   锌一日可摄入10\textasciitilde20毫克即可维持机体代谢平衡,主要有坚果、干豆、牡蛎、肝、肉等。
% \end{itemize}

% %ux4e94ux7f3aux9499ux7684ux56e0ux7d20ux6709ux54eaux4e9b}{%
% \subsection{五、缺钙的因素有哪些}%ux4e94ux7f3aux9499ux7684ux56e0ux7d20ux6709ux54eaux4e9b}}

% 除了孩子摄入钙不足引起的缺钙外,主要还有一些妨碍钙吸收的因素。

% \begin{enumerate}
% \def\labelenumi{\arabic{enumi}.}
% \item
%   首先,维生素摄入不足可能引起钙吸收转入障碍。
% \item
%   其次,脂肪食物摄入过多会形成钙皂不易溶解,导致随大便排出。
% \item
%   此外,摄入的钙和磷的比例不当也会影响钙的吸收。钙和磷的最佳比例应为1.3:1.5。如果含磷过高,会导致磷酸盐排出体外。此外,在碱性环境中,钙不容易溶解。
% \end{enumerate}

% %ux516dux4ebaux4f53ux9700ux8981ux54eaux4e9bux7ef4ux751fux7d20}{%
% \subsection{六、人体需要哪些维生素}%ux516dux4ebaux4f53ux9700ux8981ux54eaux4e9bux7ef4ux751fux7d20}}

% 人体需要的维生素有十几种,根据其特点归为两大类。

% 一类是水溶性维生素,它们可以在水中溶解。包括维生素B族和维生素C(也叫抗坏血酸),维生素B族的成员有维生素B1(硫胺素)、维生素B2(核黄素)、维生素B6(吡多醇)、维生素B12(氰钴胺)、尼克酸、泛酸、叶酸和生物素。

% 另一类是脂溶性维生素,它们在油脂中容易溶解。包括维生素A(视黄醇)、维生素D(生育酚)、维生素E(胆骨化醇)和维生素K。这些维生素可以在体内贮存,但不能过量,否则会引起中毒。现在科学的进展认为维生素不仅是维持正常机体的需要,还具有防病和抗病的作用。

% \begin{quote}
% gpt:
% \end{quote}

% \begin{longtable}[]{@{}
%   >{\raggedright\arraybackslash}p{(\columnwidth - 4\tabcolsep) * \real{0.3333}}
%   >{\raggedright\arraybackslash}p{(\columnwidth - 4\tabcolsep) * \real{0.3333}}
%   >{\raggedright\arraybackslash}p{(\columnwidth - 4\tabcolsep) * \real{0.3333}}@{}}
% \toprule()
% \begin{minipage}[b]{\linewidth}\raggedright
% 维生素类别
% \end{minipage} & \begin{minipage}[b]{\linewidth}\raggedright
% 特点
% \end{minipage} & \begin{minipage}[b]{\linewidth}\raggedright
% 维生素成员
% \end{minipage} \\
% \midrule()
% \endhead
% 水溶性维生素 & 可以在水中溶解 &
% 维生素B1(硫胺素)、维生素B2(核黄素)、维生素B6(吡多醇)、维生素B12(氰钴胺)、尼克酸、泛酸、叶酸和生物素 \\
% 脂溶性维生素 & 在油脂中容易溶解 &
% 维生素A(视黄醇)、维生素D(生育酚)、维生素E(胆骨化醇)和维生素K \\
% \bottomrule()
% \end{longtable}

% %ux4e03ux4fc3ux8fdbux5927ux8111ux53d1ux80b2ux7684ux8425ux517bux5143ux7d20ux4ee5ux53caux6765ux6e90}{%
% \subsection{七、促进大脑发育的营养元素以及来源}%ux4e03ux4fc3ux8fdbux5927ux8111ux53d1ux80b2ux7684ux8425ux517bux5143ux7d20ux4ee5ux53caux6765ux6e90}}

% \hspace{0pt}\includegraphics[width=2.85315in,height=2.64336in]{media/rId898.png}\hspace{0pt}

% (1)
% 不饱和脂肪酸使脑功能健全。在大脑发育快速时期,脂肪和类脂质(胆固醇、脑磷脂、卵磷脂)是神经细胞发育和增殖的基本物质。如卵磷脂在体内经过代谢后,会释放一种叫乙酰胆碱的物质,它是脑神经细胞之间传递信息的物质,少了它,大脑传递信息会受到障碍。主要来源:芝麻、核桃仁,自然状态下饲养的动物肉以及坚果类食物等。

% (2)
% 充足的维生素C使脑功能敏锐。维生素C是大脑营养物质分解酶的辅助物质,具很强的抗氧化作用,起到保护神经原的作用。主要来源:红枣、柚子、草莓、西瓜、鲜果类、黄绿色蔬菜等都富含维生素C。如果饮用果汁,一定要选择100\%的纯果汁(加半量的水稀释),不要饮用含糖的果汁饮料,而且尽量避免防腐剂、色素等。

% (3)充足的钙质能使大脑持续工作。钙能使智力发达,保证大脑顽强地工作。主要来源:牛奶、海带、小鱼类、蛋、虾皮、芝麻、香菇等食物。

% (4)糖质是脑活动的能源。大脑的主要燃料来自葡萄糖。人们餐桌上的米饭,面条中的糖在人体内能迅速转化成葡萄糖,并顺利透过脑屏障进入脑组织被脑细胞利用。由于大脑对糖有偏食的习惯,因此千万别忽视了这类日常生活中最普通的食物。主要来源:米饭、面条等各种谷物类、淀粉类食物。

% (5)蛋白质是大脑从事复杂智力活动的基本物质。蛋白质是构成大脑神经细胞的重要成分,优质蛋白质将促进细胞生长发育。在组成蛋白质的氨基酸中,色氨酸对人脑的思维活动有重要帮助。谷氨酸能解除氨对脑的毒害,对保护脑组织起到很大作用。一些氨基酸(如亮氨酸)的缺乏还可导致大脑发育不全。主要来源:瘦肉、鸡蛋、豆制品、鱼贝类食物。

% 维生素B、维生素E、维生素A同样是大脑营养物质分解酶的辅助物质,有利于大脑对糖类的利用。大脑要正常工作,少不了这些维生素的帮助。维生素B族物质能预防精神障碍,主要来源有香菇、野菜、黄绿色蔬菜、坚果类等。维生素A能促进大脑发育,主要来源有牛乳、奶粉、黄油、胡萝卜、韭菜等。维生素E能保持脑的活力,主要来源有甘薯、莴苣、肝、黄油等。

% \textbf{一日食谱}

% \begin{longtable}[]{@{}
%   >{\raggedright\arraybackslash}p{(\columnwidth - 4\tabcolsep) * \real{0.3333}}
%   >{\raggedright\arraybackslash}p{(\columnwidth - 4\tabcolsep) * \real{0.3333}}
%   >{\raggedright\arraybackslash}p{(\columnwidth - 4\tabcolsep) * \real{0.3333}}@{}}
% \toprule()
% \begin{minipage}[b]{\linewidth}\raggedright
% 时间
% \end{minipage} & \begin{minipage}[b]{\linewidth}\raggedright
% 食物
% \end{minipage} & \begin{minipage}[b]{\linewidth}\raggedright
% 份量
% \end{minipage} \\
% \midrule()
% \endhead
% 上午 8:00 & 大米豆粥 & 70克 \\
% & 小花卷 & 1个(约25克) \\
% & 花生酱 & 少许 \\
% & 母乳或配方奶 & 100\textasciitilde150毫升 \\
% 上午 10:00 & 酸奶 & 150毫升 \\
% & 小点心 & 1块 \\
% 上午 12:00 & 软饭 & 1碗 \\
% & 肉末胡萝卜、黄瓜丁 & 50克 \\
% & 菠菜汤 & 1小碗 \\
% 下午 15:00 & 香蕉或苹果 & 100克 \\
% & 煮鸡蛋 & 1个 \\
% & 配方奶 & 120毫升 \\
% 下午 18:00 & 高汤水饺 & 100\textasciitilde120克 \\
% & 橘子 & 1个 \\
% 晚上 21:00 & 母乳或配方奶 & 200毫升 \\
% \bottomrule()
% \end{longtable}

% %ux7b2cux4e94ux82822ux5c8113ux4e2aux6708ux5b9dux5b9dux7684ux8425ux517bux65b9ux6848}{%
% \subsection{05第五节2岁1\textasciitilde3个月宝宝的营养方案}%ux7b2cux4e94ux82822ux5c8113ux4e2aux6708ux5b9dux5b9dux7684ux8425ux517bux65b9ux6848}}

% 2岁以后,宝宝的营养需求比以前有了较大提高。由于胃容量的增加和消化能力的完善,从现在起每天的餐点仍为5次,每次的量适当增多。同时,父母要有意识地让宝宝接触粗纤维食品。

% %ux4e00ux8425ux517bux5145ux8db3ux80fdux63d0ux9ad8ux5b9dux5b9dux8bedux8a00ux80fdux529b}{%
% \subsection{一、营养充足能提高宝宝语言能力}%ux4e00ux8425ux517bux5145ux8db3ux80fdux63d0ux9ad8ux5b9dux5b9dux8bedux8a00ux80fdux529b}}

% \hspace{0pt}\includegraphics[width=2.39161in,height=2.57343in]{media/rId904.png}\hspace{0pt}

% 专家认为,语言能力较其他方面更能反映宝宝的智力水平,在提高宝宝语言能力的众多方法中,最重要的一条是保证宝宝获得足够的滋养大脑神经的物质,以促进语言中枢的正常发育。

% 蛋白质是脑细胞的主要成分之一,占脑干重的30\%\textasciitilde35\%,在促进语言中枢发育方面起着极其重要的作用。如果孕妇蛋白质摄入不足,不仅使胎儿脑发育发生重大障碍,还会影响到乳汁蛋白质含量及氨基酸组成,导致乳汁减少。宝宝蛋白质摄入不足,更会直接影响到脑神经细胞发育。因此,孕妇及宝宝要摄食足够的优质蛋白质食物。

% 大豆富含优质蛋白质,是植物中唯一类似于动物蛋白的完全蛋白质。大豆蛋白\textbf{不含胆固醇},并可降低人体血清中的胆固醇,这一点又优于动物蛋白。大豆蛋白中人体必需的八种氨基酸配比均衡,非常适合人体需要。人体对大豆蛋白的吸收多少跟食用方式有关,其中,干炒大豆的蛋白消化率不超过50\%,煮大豆也仅为65\%,而制成\textbf{豆浆蛋白消化率则高达95\%左右}。因此,每天喝一杯豆浆不失为摄取优质蛋白的一个有效途径。【来,九阳豆浆机,哈哈】

% %ux5b69ux5b50ux8981ux591aux5403ux6c34ux679c}{%
% \subsubsection{孩子要多吃水果}%ux5b69ux5b50ux8981ux591aux5403ux6c34ux679c}}

% 水果与蔬菜都为孩子成长提供必备的营养成分,但水果可以生吃,营养素免受加工烹调的破坏。水果中的有机酸可以帮助消化,促进其他营养成分的吸收。桃、杏等水果含有较多的铁,山楂、鲜枣含大量的维生素C。食用水果前应很好地清洗。洒过农药粉的水果,除彻底清洗外,最好削去外皮后再食用。

% %ux4e8cux4e0dux9002ux4e8eux5a74ux5e7cux513fux98dfux7528ux7684ux98dfux7269}{%
% \subsection{二、不适于婴幼儿食用的食物}%ux4e8cux4e0dux9002ux4e8eux5a74ux5e7cux513fux98dfux7528ux7684ux98dfux7269}}

% 一般生硬、带壳、粗糙、过于油腻及带刺激性的食物对幼儿都不相宜。有的食物需要加工后才能给孩子食用。

% %ux4e09ux591aux5403ux51b7ux996eux4e0dux597d}{%
% \subsection{三、多吃冷饮不好}%ux4e09ux591aux5403ux51b7ux996eux4e0dux597d}}

% 很多孩子在夏天吃冷饮没个够,冰棍、汽水、冰激凌\ldots\ldots 这样好不好呢?孩子在天气非常热的时候可以吃些冷饮,以防中暑,但是不能没有限度,因为大量的冷食进入胃内,会使胃壁的小血管收缩,血流减少,温度降低,抑制消化酶的活力,抑制胃酸分泌,造成孩子食欲下降。另外,冷饮一般含糖量比较高,甜食吃多了也会影响孩子食欲。

% %ux56dbux6c64ux6ce1ux996dux4e0dux5229ux4e8eux6d88ux5316}{%
% \subsection{四、汤泡饭不利于消化}%ux56dbux6c64ux6ce1ux996dux4e0dux5229ux4e8eux6d88ux5316}}

% 有的孩子不爱吃菜,却喜欢用汤或水泡饭吃。这样,很多饭粒还没有嚼烂就咽下去了,自然加重了胃的负担。而水又冲淡了胃液影响胃的消化功能,因此经常吃汤泡饭容易得胃病。【东东算这个不?】

% 孩子活动量大,消耗的水分多,往往因贪玩顾不上喝水,吃饭时感到干渴。家长应在饭前0.5\textasciitilde1小时让孩子喝些水,吃饭的时候不要让他用汤或水泡饭。

% %ux4e94ux5e7cux513fux98dfux54c1ux7684ux5236ux4f5c}{%
% \subsection{五、幼儿食品的制作}%ux4e94ux5e7cux513fux98dfux54c1ux7684ux5236ux4f5c}}

% %ux714eux9762ux5305}{%
% \subsubsection{煎面包}%ux714eux9762ux5305}}

% 把面包切成2厘米厚的片,每片中间再切条缝(不切断),在缝内抹一层果酱。然后将1个鸡蛋打入250毫升牛奶中搅匀,加少量糖,把夹果酱的面包放入牛奶中泡。平底锅放适量的油烧热,放入面包片煎成两面金黄色即可食用。

% %ux84b8ux86cbux7cd5}{%
% \subsubsection{蒸蛋糕}%ux84b8ux86cbux7cd5}}

% 取大油50克,温化,白糖250克,放入盆中搅拌,边搅边打入5个鸡蛋。搅成白色稠糊状,再加入300克面粉搅拌成面糊。可加入适量切碎的果料。然后将面糊倒入模具中蒸熟。

% %ux70e9ux9e21ux86cbux997a}{%
% \subsubsection{烩鸡蛋饺}%ux70e9ux9e21ux86cbux997a}}

% 原料:鸡蛋1个,肥瘦肉馅适量;葱、姜末,油菜心,植物油、酱油、精盐、料酒、味精、团粉汤。

% 制法:肉馅用葱姜末、酱油、料酒及味精拌匀。鸡蛋打散加盐调匀。金属勺内放少许油,置温火上烧热,用小匙盛蛋放入,趁蛋液未全干时,将蛋液摊成圆形,取肉饼放上,用筷子将蛋皮折合成饺子。锅内放汤加肉,放入油菜心稍煮,再下盐、味精,把蛋饺放入,温火烧2分钟,用团粉勾芡即成。

% 特点:鲜嫩可口,营养丰富。

% %ux9e33ux9e2fux86cb}{%
% \subsubsection{鸳鸯蛋}%ux9e33ux9e2fux86cb}}

% 原料:瘦肉馅30克,鸡蛋1个;葱、姜末、精盐、酱油、料酒、白糖、团粉汤、植物油。

% 制法:鸡蛋煮熟剥壳,纵切两半;瘦肉馅放入料酒;团粉、盐、姜末和水拌匀,夹抹在蛋内,合成整个蛋形(如欲紧合可在蛋与肉之间洒少许干面粉),放入热油锅过油炸透。然后,加酱油、料酒、盐和汤蒸30分钟,起笼沥出原汤放入锅中,加入团粉勾芡,撒上葱末,浇在蛋上即可。

% 特点:颜色美观,鲜嫩可口。

% %ux73cdux73e0ux8089ux5706}{%
% \subsubsection{珍珠肉圆}%ux73cdux73e0ux8089ux5706}}

% 原料:猪肉100克(瘦七肥三),糯米50克,荸荠25克;料酒、精盐、味精、姜末、水淀粉。

% 制法:将糯米淘洗干净后,放入开水锅内烫一下,倒入铜筛内控水。猪肉洗净后,切碎,再斩成肉蓉。荸荠洗干净,剥去皮,切成细末和肉蓉一起放入碗内。加入料酒、精盐、姜末、水淀粉,向同一方向搅拌均匀上劲。

% 将拌好的肉蓉,用左手挤成直径约2厘米的肉圆,放入烫好的糯米内,滚满糯米,放入盘内。这样,边挤肉圆,边滚糯米,挤完,滚好。上笼屉蒸约15分钟即熟,取出换盘,码放整齐即可。

% 特点:此菜质地松软鲜香,形似珍珠。

% %ux7cd6ux918bux9ec4ux9c7c}{%
% \subsubsection{糖醋黄鱼}%ux7cd6ux918bux9ec4ux9c7c}}

% 原料:黄鱼1条;青椒、土豆少许;油、精盐、糖、醋、料酒、姜、葱、淀粉。

% 制法:将黄鱼刮鳞、去鳃、去内脏,洗净沥干;在两面脊肉上每隔约3厘米斜切一刀(便于入味和炸透),用少许精盐、料酒略拌,挂上一层薄糊。葱姜切末,青椒、土豆切成小丁。

% 将黄鱼放入约七成热的油锅内炸至呈黄色、起软壳,再升高油温,重新油炸至熟透,装入盘内。炒锅放油少许,油热,放入青椒、土豆略煸炒,再放入葱、姜和适量汤、糖、醋,用湿淀粉勾成汁浇盘内鱼上。

% %ux5c0fux9e21ux7096ux8304ux5b50}{%
% \subsubsection{小鸡炖茄子}%ux5c0fux9e21ux7096ux8304ux5b50}}

% 原料:小鸡(带骨)50克,茄子1个,熟豆油、酱油、精盐、绍酒、肉汤、葱末、姜末。

% 制法:将鸡洗净,剁成3.5厘米见方的块,茄子洗净,去皮,切成滚刀块。勺置火上烧热,添入熟豆油,热时,用葱末、姜末炝锅,再下入鸡块煸炒透,添入酱油、绍酒,再炒片刻;然后添入肉汤,沸后,放入茄子,并用精盐、味精调好口味,移小火炖至鸡块、茄子酥烂入味时即成。

% 特点:鸡烂茄嫩,汤汁味浓,醇香诱口。

% %ux7d20ux4ec0ux9526}{%
% \subsubsection{素什锦}%ux7d20ux4ec0ux9526}}

% 原料:香菇、发菜、扁尖笋、口蘑、豆腐、面筋、豆腐衣、青菜心、栗子,白糖、精盐、酱油、料酒、花生油。

% 制法:

% \begin{enumerate}
% \def\labelenumi{\arabic{enumi}.}
% \item
%   把香菇、口蘑、扁尖笋、发菜分别用开水泡1个小时,取出洗净,去掉香菇和口蘑的沙子。泡香菇和口蘑的水留用。
% \item
%   把豆腐放在冷水锅内,盖严,用大火煮至起空时取出,切成方块。
% \item
%   再起热锅,放入花生油,下香菇、口蘑、笋、发菜、白果(或花生米)、栗子、豆腐块、豆腐衣、面筋等,加盐、酱油、糖、料酒炒一炒,取出放入砂锅内。
% \item
%   把泡香菇和口蘑的水倒入,用温火炖30分钟至1小时,起锅前放上青菜心,翻炒一下即可。
% \end{enumerate}

% 特点:色味俱佳,爽口。

% \textbf{一日食谱}

% \begin{longtable}[]{@{}
%   >{\raggedright\arraybackslash}p{(\columnwidth - 4\tabcolsep) * \real{0.3333}}
%   >{\raggedright\arraybackslash}p{(\columnwidth - 4\tabcolsep) * \real{0.3333}}
%   >{\raggedright\arraybackslash}p{(\columnwidth - 4\tabcolsep) * \real{0.3333}}@{}}
% \toprule()
% \begin{minipage}[b]{\linewidth}\raggedright
% 时间
% \end{minipage} & \begin{minipage}[b]{\linewidth}\raggedright
% 餐点
% \end{minipage} & \begin{minipage}[b]{\linewidth}\raggedright
% 食物
% \end{minipage} \\
% \midrule()
% \endhead
% 上午 & 8:00 & 二米粥100克,煮鸡蛋1个,肉松10克,拌黄瓜丁1小盘 \\
% 上午 & 10:00 & 牛奶或酸奶100毫升,饼干3块 \\
% 上午 & 12:00 & 馒头80克,白菜肉卷100克,海带汤1小碗 \\
% 下午 & 15:00 & 蛋糕1块,水果50克 \\
% 下午 & 18:00 & 米饭60克,红烧牛肉炖土豆100-120克 \\
% 晚上 & 21:00 & 牛奶250毫升,饼干2块 \\
% \bottomrule()
% \end{longtable}

% %ux7b2cux516dux82822ux5c8146ux4e2aux6708ux5b9dux5b9dux7684ux8425ux517bux65b9ux6848}{%
% \subsection{06第六节2岁4〜6个月宝宝的营养方案}%ux7b2cux516dux82822ux5c8146ux4e2aux6708ux5b9dux5b9dux7684ux8425ux517bux65b9ux6848}}

% 这个时期的喂哺原则与前阶段近似,每天所需的总热量达到5023.2
% \textasciitilde{}
% 5441.8千焦,其中蛋白质、脂肪和糖类的比例约为1:0.8:4.5。有些宝宝已经完成了每天餐点由5次向4次的转变。养成独立进食的习惯可以使宝宝专心吃好每一餐,这是保证营养充分摄入的需要。

% %ux4e00ux5e7cux513fux98dfux8c31ux7684ux5236ux4f5c}{%
% \subsection{一、幼儿食谱的制作}%ux4e00ux5e7cux513fux98dfux8c31ux7684ux5236ux4f5c}}

% %ux7092ux9762ux6761}{%
% \subsubsection{\texorpdfstring{\textbf{炒面条}}{炒面条}}%ux7092ux9762ux6761}}

% 将胡萝卜、扁豆、葱头、火腿切碎,放油锅内炒,待菜炒软后再放入煮过的细面条50克一块炒,最后加番茄酱调味。

% %ux83dcux5377ux86cb}{%
% \subsubsection{\texorpdfstring{\textbf{菜卷蛋}}{菜卷蛋}}%ux83dcux5377ux86cb}}

% 把适量圆白菜叶放在开水中煮一下,把1个鸡蛋煮熟后剥皮,外面裹上面粉,再用圆白菜叶包好放入肉汤中,加切碎的番茄2大匙及番茄酱少许煮,煮好后放入盘内切两半。

% %ux571fux8c46ux86cbux997c}{%
% \subsubsection{\texorpdfstring{\textbf{土豆蛋饼}}{土豆蛋饼}}%ux571fux8c46ux86cbux997c}}

% 土豆洗净煮熟,剥皮捣碎成泥状。面粉100克,鸡蛋2个加入土豆泥,适量糖、盐搅匀,上笼蒸或在平底锅上抹油之后烤热。

% %ux62ccux8304ux5b50}{%
% \subsubsection{\texorpdfstring{\textbf{拌茄子}}{拌茄子}}%ux62ccux8304ux5b50}}

% 原料:茄子1个

% 制法:麻酱1汤匙,精盐半汤匙,蒜1瓣,味精适量。将茄子洗净,对剖开,放蒸锅蒸熟放凉。用凉开水将麻酱调稀,加入精盐、味精备用。把蒜剁成蒜泥,与麻酱一起倒入茄子中拌匀即可。

% 特点:制作简单,软嫩清淡。

% %ux4e8cux6311ux98dfux504fux98dfux662fux574fux4e60ux60ef}{%
% \subsection{二、挑食、偏食是坏习惯}%ux4e8cux6311ux98dfux504fux98dfux662fux574fux4e60ux60ef}}

% 孩子偏食、挑食原因很多。比如,有的\textbf{家长本身就挑食,吃饭时又不注意,边吃边评论这个不好吃,那个味道差,无形之中就会影响孩子}。另外,饭菜\textbf{太单调,总不变换花样},也容易使孩子厌食、偏食。

% 当孩子挑食时,家长不要训斥孩子,不要强迫他吃,这样做会使他产生厌恶感。家长在做菜时要注意翻新花样,同样的菜这样做孩子不愿吃,换一个方法做他就爱吃,要使孩子感到新鲜,增进食欲。比如,孩子不吃煮鸡蛋可以炒,可以蒸,可以炖,还可以做成荷包蛋、摊成鸡蛋饼或做成鸡蛋糕等。不能一下子就下结论说孩子不爱吃鸡蛋。

% 另外,还可以用讲故事的方法,提高孩子的兴趣。比如孩子不爱吃萝卜,可以给他讲讲拔萝卜的故事,玩拔萝卜的游戏,让他进入角色,使他对平时根本不屑一顾的萝卜产生兴趣,慢慢就会喜欢吃了。

% 总之,纠正孩子偏食,家长既要有耐心,又要讲究方式方法,才能取得好的效果。

% %ux4e09ux5b9dux5b9dux4e3aux4ec0ux4e48ux6210ux4e86ux80d6ux513f}{%
% \subsection{三、宝宝为什么成了胖儿}%ux4e09ux5b9dux5b9dux4e3aux4ec0ux4e48ux6210ux4e86ux80d6ux513f}}

% \hspace{0pt}\includegraphics[width=2.92308in,height=2.95105in]{media/rId933.png}\hspace{0pt}

% 过胖和过瘦的宝宝

% 宝宝摄入的总热量明显超过消耗的总热量,剩余的热能就会转化为脂肪聚于体内。如体重超标20\%以上,可诊断为肥胖症。

% 能引起肥胖症的原因很多,如甲状腺、肾上腺等内分泌系统疾病,神经系统和代谢功能异常等疾病,但绝大多数肥胖患儿是没有疾病影响的单纯性肥胖。单纯性肥胖儿的食欲好,食量大,特别是爱吃肥肉和甜食。体内脂肪主要聚积于乳部、腹部、髋部、肩部。

% 轻度肥胖对宝宝的健康无明显影响。较严重肥胖儿的活动受限制,与周围事物的接触相对少,大脑通过感觉器官接受信息及相应的分析综合能力活动也减少,这对宝宝的智力发育是不利的。而且儿童时期的肥胖,可成为成年后发生高血脂、高血压、冠心病等的重要因素。

% 治疗宝宝肥胖,一是调节饮食,二是增加运动,又以限制进食为首。调节饮食的目的,是使宝宝摄入的总热量低于消耗的总热量。严重肥胖患儿,应比正常标准体重所需总热量减少30\%。

% \textbf{限制进食应注意以下几点}:

% \begin{enumerate}
% \def\labelenumi{\arabic{enumi}.}
% \item
%   宝宝处于生长发育阶段,蛋白质的供给不能过少,仍应常给宝宝吃瘦肉、鱼肉、鸡蛋、豆制品等食物。
% \item
%   另外,基本满足宝宝食欲,不使宝宝饥饿,故应多吃热量少但体积大的食物,如芹菜、萝卜等蔬菜。米饭、面食仍为主食,每天吃适量水果。
% \item
%   严格限制脂肪及甜食。当患儿体重下降到只超过正常标准体重10\%左右时,就不必继续严格限制饮食。
% \end{enumerate}

% %ux56dbux5b9dux5b9dux4e3aux4ec0ux4e48ux8fc7ux4e8eux7626ux5c0f}{%
% \subsection{四、宝宝为什么过于瘦小}%ux56dbux5b9dux5b9dux4e3aux4ec0ux4e48ux8fc7ux4e8eux7626ux5c0f}}

% 宝宝过于瘦小的主要原因有以下几点:

% \begin{enumerate}
% \def\labelenumi{\arabic{enumi}.}
% \item
%   \textbf{胎生期生长缓慢。}出生时的身高体重与其发育至儿童末期时的身高体重成正比,因此要重视孕期营养与保健,保障胎儿正常发育。
% \item
%   \textbf{新生儿期喂养不足。}动物实验证实,生命早期(生后一周最具决定性)生长如受到阻碍,以后就是给予充足的各种营养素,其生长仍继续落后。新生儿期供给的总热量不足,如开奶过迟,牛奶里配水比例过大等,是宝宝生长迟缓的重要原因,因此新生儿喂养的关键是必须让宝宝吃饱。
% \item
%   \textbf{婴幼儿期喂养不当。}一些家长不及时合理地给宝宝增加必需的辅食,致使宝宝摄入的总热量不足,不能满足其迅速生长发育的需要。有些婴幼儿挑食、偏食,家长给宝宝吃零食弥补,特别是吃糖果、糕点过多,产生虚假饱腹感,严重影响宝宝吃正餐的食欲。
% \item
%   \textbf{微量元素缺乏。}较常见的有铁、铜、锌等缺乏。最能影响宝宝生长的是锌元素。缺锌可影响宝宝味蕾细胞的代谢,使\textbf{味觉迟钝,降低食欲}。因此,身长、体重过低的宝宝,应检测体内的含锌值。
% \item
%   \textbf{疾病因素。}有肥大性幽门狭窄等先天性疾病,有缺铁性贫血、结核病、反复呼吸道感染、肾小球肾炎、风湿热等慢性疾病等,都会使宝宝成为``瘦小型''。家长应根据以上原因对症调理。
% \end{enumerate}

% 一日食谱

% \begin{longtable}[]{@{}
%   >{\raggedright\arraybackslash}p{(\columnwidth - 2\tabcolsep) * \real{0.5000}}
%   >{\raggedright\arraybackslash}p{(\columnwidth - 2\tabcolsep) * \real{0.5000}}@{}}
% \toprule()
% \begin{minipage}[b]{\linewidth}\raggedright
% 时间
% \end{minipage} & \begin{minipage}[b]{\linewidth}\raggedright
% 食物
% \end{minipage} \\
% \midrule()
% \endhead
% 上午 8:00 & 肉丁馒头50克,二米粥100克,牛奶150毫升,炒菜1小碟 \\
% 上午12:00 & 软米饭100克,白菜肉卷100克,海带汤1小碗 \\
% 下午 15:00 & 全麦面包片1\textasciitilde2片,酸奶100毫升,水果50克 \\
% 下午18:00 & 米饭60克,肉丝豆腐干蒜苗100\textasciitilde120克 \\
% 晚上 21:00 & 牛奶或配方奶250毫升(牛奶要加适量白糖),饼干2块 \\
% \bottomrule()
% \end{longtable}

% %ux7b2cux4e03ux82822ux5c8179ux4e2aux6708ux5b9dux5b9dux7684ux8425ux517bux65b9ux6848}{%
% \subsection{07第七节2岁7〜9个月宝宝的营养方案}%ux7b2cux4e03ux82822ux5c8179ux4e2aux6708ux5b9dux5b9dux7684ux8425ux517bux65b9ux6848}}

% 2岁半以后,宝宝每天所需的蛋白质、脂肪和糖类的比例约为1:0.8:4.5,总热量达到5441.8千焦。每天应当进食主餐3次,点心1次,坚持喝奶,同时适量吃些应季水果。继续强化独立进餐的良好习惯。

% %ux4e00ux5e7cux513fux98dfux8c31ux7684ux5236ux4f5c-1}{%
% \subsection{一、幼儿食谱的制作}%ux4e00ux5e7cux513fux98dfux8c31ux7684ux5236ux4f5c-1}}

% %ux7effux8c46ux82bdux7092ux97edux83dc}{%
% \subsubsection{\texorpdfstring{\textbf{绿豆芽炒韭菜}}{绿豆芽炒韭菜}}%ux7effux8c46ux82bdux7092ux97edux83dc}}

% 原料:绿豆芽50克,韭菜15克;精盐半汤匙,酱油少许,食油。

% 制法:将韭菜切成寸段。将油倒入炒锅中,烧至冒烟。将洗净的豆芽及韭菜同时放入锅中,炒几下,加入调料,大火炒2\textasciitilde3分钟即熟。

% 特点:鲜、脆。

% %ux756aux8304ux867eux4ec1}{%
% \subsubsection{\texorpdfstring{\textbf{番茄虾仁}}{番茄虾仁}}%ux756aux8304ux867eux4ec1}}

% 原料:山药,味精、团粉、面粉、番茄酱、白糖、香油少许,食盐适量,食油。

% 制法:山药洗净蒸烂后凉凉,去皮捣碎成泥,然后加入面粉、淀粉、食盐、味精、葱姜末,拌匀搓成条做成虾仁状,滚粘于淀粉备用。锅内多放油加热,倒入调制好的虾仁,快炸一下捞出。锅里留少许油,放入番茄酱,炒出香味时,加入食盐、味精、白糖及少量水,用旺火烧开,见番茄汁出现均匀小泡时,倒入炸好的虾仁,快炒几下,滴入香油即成。

% 特点:山药酥嫩,酸甜适口。

% %ux4e8cux98dfux7269ux8fc7ux654fux7684ux8868ux73b0}{%
% \subsection{二、食物过敏的表现}%ux4e8cux98dfux7269ux8fc7ux654fux7684ux8868ux73b0}}

% 食物过敏是令父母感到棘手的问题之一。有些时候,宝宝吃过某些东西后,会有不适的表现,但是父母并未意识到是因为过敏所致。有的时候,父母虽然感觉宝宝过敏是由于吃了某些食物,但是验证起来却很困难。所以,父母常被这个问题困扰。能给过敏的宝宝吃什么?给宝宝吃,怕过敏;不给宝宝吃,怕宝宝营养摄入不足。父母处于矛盾之中。

% \textbf{一般情况下,食物过敏常表现在下面3个方面}:

% \begin{enumerate}
% \def\labelenumi{\arabic{enumi}.}
% \item
%   \textbf{呼吸道表现}。流鼻涕、打喷嚏、气喘、鼻塞、流泪、久咳不愈、充血。
% \item
%   \textbf{皮肤表现}。脸部红、出疹子、荨麻疹、皮肤干燥,呈鱗状,发痒。眼皮肿胀,嘴唇红肿等。
% \item
%   \textbf{肠道表现}。黏液性腹泻,便秘、腹胀、胀气,呕吐,肠道出血,腹部不适等。
% \end{enumerate}

% 由以上不适带来的相关神经系统症状,如烦躁、夜间睡眠不踏实,醒闹,易怒,啼哭,焦虑等。

% 当宝宝有过敏表现的时候,有时很容易查找到过敏源,例如:宝宝吃完虾不久,身上起皮疹。虾就是过敏源。有时很难觉察过敏源是什么。遇到这类情况时,父母最好将宝宝吃过的每一样东西记录下来,包括正餐的食物。连续记录3
% \textasciitilde{} 4天,然后从记录单中选出最可疑的食物。

% \begin{quote}
% 育儿小百科

% \textbf{常见的过敏食物有哪些}

% 常见的过敏食品种类是乳类、小麦、蛋白、玉米、柑橘类、大豆或食物添加物。乳制品中的蛋白质是最常见的食物过敏源。食入大米基本上没有过敏的现象发生。因此,\textbf{最早给宝宝添加辅食种类之一应是米粉,而不是面粉}。择出可疑过敏物后,停止吃该种食物,观察过敏征兆与症状的变化并记录。如果怀疑得到证实,以后再给宝宝吃这类食物时便需要小心了。做这项工作的时候,\textbf{需要排除空气过敏源的存在},如花粉季节、居室装修、空气异味等。
% \end{quote}

% %ux4e09ux600eux6837ux907fux514dux8fc7ux654fux60c5ux51b5ux7684ux53d1ux751fux5c24ux5176ux662fux5316ux654fux4f53ux8d28ux7684ux5b9dux5b9d}{%
% \subsection{三、怎样避免过敏情况的发生(尤其是化敏体质的宝宝)}%ux4e09ux600eux6837ux907fux514dux8fc7ux654fux60c5ux51b5ux7684ux53d1ux751fux5c24ux5176ux662fux5316ux654fux4f53ux8d28ux7684ux5b9dux5b9d}}

% 给宝宝添加辅食时,一定要遵循``从少量开始,从单一食品开始''的原则,选择添加辅食的种类要适当,按添加顺序进行。

% 发现宝宝对某种食物过敏后,可以停一段时间再吃该食物。需从少量喂起并观察宝宝反应,直到宝宝适应此食物后,再给予正常量。

% 如果宝宝过敏轻微,没有受到太大的影响,父母可以不必太在意。但如果宝宝深受食物过敏所致的痛苦,父母需要带宝宝到医院做专项检查、治疗,日常起居也需要格外小心。

% 大多数食物过敏情况,像许多婴儿期的小毛病一样,会随着宝宝的成长,免疫系统的增强而减轻或消失。

% %ux56dbux7ed9ux5b9dux5b9dux8865ux5145ux7ef4ux751fux7d20ux7684ux539fux5219}{%
% \subsection{四、给宝宝补充维生素的原则}%ux56dbux7ed9ux5b9dux5b9dux8865ux5145ux7ef4ux751fux7d20ux7684ux539fux5219}}

% \textbf{缺什么,补什么},缺几种补几种,\textbf{不缺不补},这是补充维生素的第一条原则。补充维生素时妈妈们首先想到各类复合维生素口服片。的确,复合维生素是很诱人,小小一粒药丸将维生素家族的大部分成员一网打尽。要补的和不要补的都在那一片药丸里。但是这样补没有明确目的。全面补充,很容易产生维生素之间的失衡,该补的量不足,不需补的量偏多。

% 补充要有间断性。也就是说,如果要补充维生素,也应间断地补充,不能一直服用。长期补充复合维生素片或各类维生素容易使人体产生依赖性,尤其是水溶性维生素。长期补充不仅会降低机体对食物中维生素的吸收率,而且一旦停服,就可能出现维生素缺乏的症状。专家建议,无论补充哪类维生素,最好补一段时间,停一段时间,以保证服用的安全,例\textbf{如隔天服用1次,或服5天停2天}。

% %ux4e94ux4e3aux5b9dux5b9dux9009ux62e9ux98dfux7269ux7684ux539fux5219}{%
% \subsection{五、为宝宝选择食物的原则}%ux4e94ux4e3aux5b9dux5b9dux9009ux62e9ux98dfux7269ux7684ux539fux5219}}

% 婴幼儿时期,宝宝的生长发育较快,营养素消耗量较大,需供应足够的营养,才能满足宝宝生长发育的需要。为宝宝配制膳食,最基本的要求是注意食品安全性与食品卫生,要求各种食品新鲜、无毒、无害、无污染。

% \hspace{0pt}\includegraphics[width=2.32168in,height=2.40559in]{media/rId957.png}\hspace{0pt}

% \begin{enumerate}
% \def\labelenumi{\arabic{enumi}.}
% \item
%   去正规的大型超市或市场购买有检验、检疫合格标志的肉类。
% \item
%   选用新鲜的蛋类。将蛋打开,放入碗中,蛋清呈透明、无色无味,蛋黄完整,无异物的蛋才是新鲜的。如果蛋清呈黄绿色,即使没有异味,也可能是不新鲜的蛋,不宜食用。宝宝不宜吃腌制的各种蛋类,更不宜吃松花蛋。腌制的蛋类会刺激宝宝的胃黏膜。
% \item
%   选择鱼类食物时,要选用刺少、柔嫩、易去除腥味的新鲜鱼,如三文鱼、黄花鱼、带鱼、青鱼等。制成鱼丸、鱼泥较好。不要用油炸、油煎等烹调方法。防止鱼骨、鱼刺卡在宝宝咽部。
% \item
%   不要给宝宝鱼干、烤鱼片等小食品,这些食品渗透压高,会刺激胃黏膜,还含有亚硝胺类,影响宝宝的生长发育。
% \item
%   选用牛奶时,要注意牛奶的形状。正常的牛奶为均匀的乳白色纯净混合体,无凝块和沉淀,微甜,具有特有的奶香味。选择袋装牛奶时,要注意牛奶的生产日期。
% \item
%   选用蔬菜时,要选择嫩菜心及农药污染少的蔬菜,如萝卜、土豆等地下生长的蔬菜要比地上生长的叶菜污染少。叶菜类蔬菜洗净后用清水浸泡1小时,有助于去掉菜叶上的农药。
% \item
%   给宝宝购买包装食品时,一定要阅读产品说明,注意生产日期、保质期、生产厂家及产品所含成分,避免给予含过多食品添加剂及高脂、高糖、高钠的食品。
% \end{enumerate}

% \textbf{一日食谱}

% \begin{longtable}[]{@{}
%   >{\raggedright\arraybackslash}p{(\columnwidth - 2\tabcolsep) * \real{0.5000}}
%   >{\raggedright\arraybackslash}p{(\columnwidth - 2\tabcolsep) * \real{0.5000}}@{}}
% \toprule()
% \begin{minipage}[b]{\linewidth}\raggedright
% 时间
% \end{minipage} & \begin{minipage}[b]{\linewidth}\raggedright
% 食物
% \end{minipage} \\
% \midrule()
% \endhead
% 上午 8:00 & 五仁包子80克, 麦片粥50克, 牛奶150毫升 \\
% 中午 12:00 & 鱼肉饺子100克, 番茄鸡蛋汤1小碗 \\
% 下午 15:00 & 豆奶200毫升(加适量白糖), 饼干2块, 水果50-100克 \\
% 傍晚 18:30 & 肉末南瓜100克, 小米粥1小碗 \\
% 晚上 21:00 & 牛奶或配方奶250毫升(牛奶要加适量白糖) \\
% \bottomrule()
% \end{longtable}

% %ux7b2cux516bux82822ux5c811012ux4e2aux6708ux5b9dux5b9dux7684ux8425ux517bux65b9ux6848}{%
% \subsection{08第八节2岁10〜12个月宝宝的营养方案}%ux7b2cux516bux82822ux5c811012ux4e2aux6708ux5b9dux5b9dux7684ux8425ux517bux65b9ux6848}}

% 在这个阶段,宝宝每天所需的营养比以前略有增加,总热量可以达到5651.1千焦左右。宝宝普遍已经能够独立进餐,但会有边吃边玩的现象。父母要有耐心,让宝宝慢慢用餐,以保证宝宝真正吃饱,避免出现进食不当导致营养不良。

% %ux4e00ux7ed9ux5b9dux5b9dux5403ux5f53ux5b63ux6c34ux679c}{%
% \subsection{一、给宝宝吃当季水果}%ux4e00ux7ed9ux5b9dux5b9dux5403ux5f53ux5b63ux6c34ux679c}}

% \hspace{0pt}\includegraphics[width=3.59441in,height=1.53846in]{media/rId970.png}\hspace{0pt}

% 挑选的水果品种应选择当季的新鲜水果。现在水果保存方法越来越先进,我们经常能吃到一些反季节水果,冬天吃到夏天的西瓜已经不是什么稀罕事。有些水果,例如苹果和梨,营养虽然丰富,可如果储存时间过长,营养成分也会丢失得非常厉害。

% 购买水果时应首选当季水果,每次买的数量也不要太多,\textbf{随吃随买},防止水果霉烂或储存时间过长,降低水果的营养成分。挑选时也要选择那些新鲜、表面有光泽,没有霉点的水果。

% %ux4e8cux5403ux6c34ux679cux7684ux6700ux4f73ux65f6ux95f4}{%
% \subsection{二、吃水果的最佳时间}%ux4e8cux5403ux6c34ux679cux7684ux6700ux4f73ux65f6ux95f4}}

% 一些妈妈认为饭后吃水果可以促进食物消化,这种想法对于成人来说没错,可对于正在生长发育中的宝宝却并不适宜。这是因为,一些水果中有不少单糖物质,虽然说它们极易被小肠吸收,但若是堵在胃中,就很容易形成胃\textbf{胀气},还可能引起便秘。所以在饱餐之后不要马上给宝宝吃水果。另外,餐前也不是吃水果的最佳时间。宝宝的胃容量还比较小,如果在餐前食用,就会占据胃的空间,影响正餐的摄入。把吃水果的时间安排在\textbf{两餐之间}比较好,比如午睡醒来之后,吃一个苹果或者橘子。

% %ux4e09ux9009ux62e9ux6c34ux679cux8981ux4e0eux5b9dux5b9dux4f53ux8d28ux76f8ux9002ux5b9c}{%
% \subsection{三、选择水果要与宝宝体质相适宜}%ux4e09ux9009ux62e9ux6c34ux679cux8981ux4e0eux5b9dux5b9dux4f53ux8d28ux76f8ux9002ux5b9c}}

% 不是所有的水果宝宝都能吃。妈妈要注意挑选与宝宝的体质、身体状况相适宜的水果。比如,体质偏热容易便秘的宝宝,最好吃寒凉性水果,如梨、西瓜、香蕉、猕猴桃等,它们可以败火。

% 如果宝宝体内缺乏维生素A、维生素C,那么就多吃杏、甜瓜及柑橘,这样能给身体补充大量的维生素A和维生素C。

% 宝宝患感冒、咳嗽时,可以用梨加冰糖炖水喝,因为梨性寒,生津润肺,可以清肺热。如果宝宝腹泻就不宜吃梨。对于一些体重超标的宝宝,妈妈要注意控制水果的摄入量,可挑选一些低糖、低热量的水果【gpt:如柚子、蓝莓、草莓等。】

% %ux56dbux6c34ux679cux4e0dux80fdux4ee3ux66ffux852cux83dc}{%
% \subsection{四、水果不能代替蔬菜}%ux56dbux6c34ux679cux4e0dux80fdux4ee3ux66ffux852cux83dc}}

% 有些妈妈认为水果营养优于蔬菜,加之水果口感好,宝宝更乐于接受。因此,对一些不爱吃蔬菜的宝宝,妈妈常以水果代替蔬菜,认为这样可以弥补不吃蔬菜而对身体造成的损害。

% 其实,用水果代替蔬菜的做法并不科学。水果与蔬菜营养差异很大,与蔬菜相比,水果中的无机盐和粗纤维含量较少,不能给肠肌提供足够的``动力''。不吃蔬菜的宝宝经常会有饱腹感,食欲下降,营养摄入不足势必影响身体发育。

% \begin{quote}
% \hspace{0pt}你知道吗?

% \textbf{孩子不宜吃的食物:}
% \end{quote}

% \begin{enumerate}
% \def\labelenumi{\arabic{enumi}.}
% \item
%   \begin{quote}
%   汽水。
%   \end{quote}
% \item
%   \begin{quote}
%   汉堡包。
%   \end{quote}
% \item
%   \begin{quote}
%   热狗。
%   \end{quote}
% \item
%   \begin{quote}
%   全脂牛奶。
%   \end{quote}
% \item
%   \begin{quote}
%   黄油。
%   \end{quote}
% \item
%   \begin{quote}
%   肥肉。
%   \end{quote}
% \item
%   \begin{quote}
%   红肠。
%   \end{quote}
% \item
%   \begin{quote}
%   比萨饼。
%   \end{quote}
% \item
%   \begin{quote}
%   巧克力。
%   \end{quote}
% \item
%   \begin{quote}
%   冰淇淋。
%   \end{quote}
% \end{enumerate}

% %ux4e94ux6c34ux679cux5e76ux975eux591aux591aux76caux5584}{%
% \subsection{五、水果并非多多益善!}%ux4e94ux6c34ux679cux5e76ux975eux591aux591aux76caux5584}}

% 水果并不是吃得越多就越好,每天水果的品种不要太杂,每次吃水果的量也要有节制。一些水果中含糖量很高,吃多了不仅会造成宝宝食欲不振,还会影响宝宝的消化功能,影响其他必需营养素的摄取。

% 另外,一些水果不能与其他食物一起食用。比如柿子与红薯、螃蟹一同吃,便会在胃内形成不能溶解的硬块儿。轻者造成宝宝便秘,严重的话这些硬块不能从体内排出,便会停留在胃里,致使宝宝胃部胀痛、呕吐及消化不良。

% 熟悉各种水果的特性,每天吃的水果不超过3种,控制宝宝的水果摄入量。

% %ux516dux6c34ux679cux7684ux6e05ux6d17ux65b9ux6cd5}{%
% \subsection{六、水果的清洗方法}%ux516dux6c34ux679cux7684ux6e05ux6d17ux65b9ux6cd5}}

% 吃水果前应将水果清洗干净,并在清水中浸泡30分钟或用\textbf{淡盐水浸泡20分钟},再用流动的水冲净后食用。水果能削皮的尽量削去皮,有些水果在食用前要用毛刷\textbf{刷干净},而不能因为图方便在水龙头下冲冲了事。

% %ux4e03-ux5b9dux5b9dux5403ux96f6ux98dfux7684ux539fux5219}{%
% \subsection{七、
% 宝宝吃零食的原则}%ux4e03-ux5b9dux5b9dux5403ux96f6ux98dfux7684ux539fux5219}}

% 大部分家长认为,吃零食会影响正餐,提供的营养也不全面,因此告诫宝宝不要吃零食。研究证实,零食并非一无是处。在儿童的饮食结构中,零食扮演着不可替代的角色。它能够补充一些身体必需的营养素,尤其是矿物质、微量元素和多种维生素。同时,还能从零食中获得全天所需能量的20\%左右。这样吃零食,反而有利于膳食平衡。

% \begin{enumerate}
% \def\labelenumi{\arabic{enumi}.}
% \item
%   不要喧宾夺主
% \end{enumerate}

% 许多儿童零食不离口,走路时吃、做作业时吃、看电视时吃、聊天时还吃。这样吃零食不仅影响了正餐,甚至还以零食代替了正餐。

% \begin{enumerate}
% \def\labelenumi{\arabic{enumi}.}
% \setcounter{enumi}{1}
% \item
%   合理安排吃零食的时间
% \end{enumerate}

% 可在两餐之间,上午九十点钟和下午三四点钟,离正餐时间已有两个多小时。由于儿童代谢较成人快,此时,他们可能会有轻微的饥饿感。如果能够让他们适量地吃些零食,就会起到防止饥饿和增加营养的作用,也不会出现影响正餐进食的情况。

% \begin{enumerate}
% \def\labelenumi{\arabic{enumi}.}
% \setcounter{enumi}{2}
% \item
%   选择有营养的零食
% \end{enumerate}

% 要选择富有营养的食品作为零食,如牛奶、酸奶、水果、蛋糕、肉松、牛肉干等。各种薯片、话梅干、果冻等食品营养价值比较低,不宜长期作为儿童的零食。

% %ux516bux5065ux8111ux98dfux54c1}{%
% \subsection{八、健脑食品}%ux516bux5065ux8111ux98dfux54c1}}

% \begin{enumerate}
% \def\labelenumi{\arabic{enumi}.}
% \item
%   \textbf{豆类}:对于大脑发育来说,豆类是不可缺少的植物蛋白质,黄豆、花生米、豌豆等都有很高的营养价值。
% \item
%   \textbf{糙米杂粮}:糙米的营养成分比精白米多,黑面粉比白面粉的营养价值高,这是因为在细加工的过程中,很大一部分营养成分损失掉了。要给孩子多吃杂粮,包括糯米、玉米、小米、红小豆、绿豆等,这些杂粮的营养成分适合身体发育的需要,搭配食用能使孩子得到全面的营养,有利于大脑的发育。
% \item
%   \textbf{动物内脏}:动物肝、肾、脑、肚等,补血又健脑,是孩子很好的营养品。
% \item
%   \textbf{鱼虾类及其他}:鱼虾蛋黄等食品中含有一种胆碱物质,这种物质进入人体后,能被大脑从血液中直接吸收,在脑中转化成乙酰胆碱,可提高脑细胞的功能。尤其是蛋黄,含卵磷脂较多,被分解后能放出较多的胆碱,所以孩子最好每日吃点蛋黄和鱼肉等食品。
% \end{enumerate}

% %ux4e5dux5982ux4f55ux89c2ux5bdfux5b9dux5b9dux7684ux8425ux517bux72b6ux51b5}{%
% \subsection{九、如何观察宝宝的营养状况}%ux4e5dux5982ux4f55ux89c2ux5bdfux5b9dux5b9dux7684ux8425ux517bux72b6ux51b5}}

% 宝宝的营养状况如何,首先要看宝宝体格发育情况。此外还可通过以下4项标准来观察:

% \begin{enumerate}
% \def\labelenumi{\arabic{enumi}.}
% \item
%   \textbf{宝宝生活得愉快}:包括吃得香,睡得深沉,醒后精神好,活泼,不磨人等。
% \item
%   \textbf{宝宝的外貌}:面色红润,头发黑密而有光泽,皮肤细腻,口唇与眼皮的内面和指甲是淡红色的,这些即属正常。
% \item
%   \textbf{皮下脂肪厚度}:用拇指和食指将宝宝腹部的皮肤捏成一个皱褶,量量皮下脂肪厚度,一般应小于1厘米。皮下脂肪的厚度是表示宝宝营养状况好坏的一个重要标志。当宝宝营养不良时,皮下脂肪消减的顺序首先是腹部,其次是躯干、四肢,最后是面颊部。所以检查皮下脂肪的厚度能早期发现宝宝的异常情况。家长不要等宝宝的小脸已经消瘦再去请大夫检查。
% \item
%   \textbf{肌肉坚实}:宝宝身上的肌肉坚实。如能达到上述要求,就属于发育良好。
% \end{enumerate}

% 一日饮食

% \begin{longtable}[]{@{}
%   >{\raggedright\arraybackslash}p{(\columnwidth - 2\tabcolsep) * \real{0.5000}}
%   >{\raggedright\arraybackslash}p{(\columnwidth - 2\tabcolsep) * \real{0.5000}}@{}}
% \toprule()
% \begin{minipage}[b]{\linewidth}\raggedright
% 时间
% \end{minipage} & \begin{minipage}[b]{\linewidth}\raggedright
% 食物
% \end{minipage} \\
% \midrule()
% \endhead
% 上午 8:00 & 葡萄干蛋糕80克,牛奶150毫升,果酱10克,小盘新鲜蔬菜 \\
% 上午12:00 & 馒头60克,炖排骨100克(肉,汤各半) \\
% 下午 15:00 & 豆奶250毫升,加适量白糖,面包片2片,水果100克 \\
% 下午 18:30 & 白菜肉馅饼100克,荷叶粥1碗 \\
% 晚上 21:00 & 牛奶或配方奶250毫升(牛奶要加适量白糖) \\
% \bottomrule()
% \end{longtable}

% %ux7b2cux56dbux7bc7-ux5a74ux5e7cux513fux7684ux65e5ux5e38ux62a4ux7406}{%
% \section{4第四篇
% 婴幼儿的日常护理}%ux7b2cux56dbux7bc7-ux5a74ux5e7cux513fux7684ux65e5ux5e38ux62a4ux7406}}

% 第一节 0〜1个月婴儿的日常护理

% 第二节1〜2个月婴儿的日常护理

% 第三节2 \textasciitilde{} 3个月婴儿的日常护理

% 第四节3 \textasciitilde{} 4个月婴儿的日常护理

% 第五节4〜5个月婴儿的日常护理

% 第六节5〜6个月婴儿的日常护理

% 第七节6〜7个月婴儿的日常护理

% 第八节7\textasciitilde8个月婴儿的日常护理

% 第九节8〜9个月婴儿的日常护理

% 第十节9〜10个月婴儿的日常护理

% 第十一节10 \textasciitilde{} 11个月婴儿的日常护理

% 第十二节11〜12个月宝宝的日常护理

% 第十三节幼儿的日常生活护理

% %ux7b2cux4e00ux8282-01ux4e2aux6708ux5a74ux513fux7684ux65e5ux5e38ux62a4ux7406}{%
% \subsection{01第一节
% 0〜1个月婴儿的日常护理}%ux7b2cux4e00ux8282-01ux4e2aux6708ux5a74ux513fux7684ux65e5ux5e38ux62a4ux7406}}

% %ux4e00ux65b0ux751fux513fux8110ux5e26ux7684ux62a4ux7406}{%
% \section{一、新生儿脐带的护理}%ux4e00ux65b0ux751fux513fux8110ux5e26ux7684ux62a4ux7406}}

% 脐带是胎儿与母亲胎盘相连接的一条纽带,胎儿由此摄取营养与排出废物。胎儿出生后,脐带被结扎、切断,留下呈蓝白色的残端。几个小时后,残端就变成棕白色。以后逐渐干枯、变细,并且成为黑色。一般出生后3\textasciitilde7天内脐残端脱落。脐带初掉时创面发红、稍湿润,几天后就完全愈合了。

% 由于身体内部脐血管的收缩,皮肤牵拉、凹陷而成脐窝,也就是俗称的肚脐眼。处理脐带时,洗手后以左手捏起脐带,轻轻提起,右手用消毒酒精棉棍,围绕脐带的根部进行消毒,将分泌物及血迹全部擦掉,每日1\textasciitilde2次,以保持脐根部清洁。

% 同时,还必须勤换尿布,以免尿便污染脐部。如果发现脐根部有脓性分泌物,而且脐局部发红,说明有脐炎发生,应该请医生治疗。

% %ux4e8cux53e3ux8154ux62a4ux7406}{%
% \subsection{二、口腔护理}%ux4e8cux53e3ux8154ux62a4ux7406}}

% 新生儿口腔内有少量的唾液,能起到清洁口腔的作用。新生儿的口腔黏膜柔嫩,如果用纱布蘸白开水擦拭,会擦伤黏膜,可能会造成霉菌性口腔炎。

% 霉菌性口腔炎在新生儿中颇为多见。可以用1\%的甲紫(龙胆紫)溶液涂抹患处,每天一两次。也可用制霉菌素片(10万单位/片)溶于10毫升冷开水中,用该溶液涂抹患处,每天两三次。

% %ux4e09ux81c0ux90e8ux62a4ux7406}{%
% \subsection{三、臀部护理}%ux4e09ux81c0ux90e8ux62a4ux7406}}

% 每次大便后要清洁臀部,尤其是女婴,因为尿道口比较短,尿道、阴道与肛门比较近,大便中的细菌容易进入尿道而引起上行性的尿路感染。清洁臀部时,用清洁的毛巾从尿道口向肛门的方向擦洗。擦干后再涂护肤油或1\%的鞣酸软膏。

% %ux56dbux600eux6837ux7ed9ux65b0ux751fux513fux6d17ux6fa1}{%
% \subsection{四、怎样给新生儿洗澡}%ux56dbux600eux6837ux7ed9ux65b0ux751fux513fux6d17ux6fa1}}

% 对于新上任的父母来说,给新生儿洗澡确实是一件比较令人头疼的事情。婴儿全身软软的,滑滑的,每次洗澡就像打仗一样,弄得父母满头大汗。确实,给新生儿洗澡要有技巧。

% %ux6d17ux6fa1ux524dux7684ux51c6ux5907}{%
% \subsubsection{1.
% 洗澡前的准备}%ux6d17ux6fa1ux524dux7684ux51c6ux5907}}

% \begin{itemize}
% \item
%   \textbf{洗澡时间}:一般地说,洗澡可在喂奶前30分钟进行。不要在刚刚吃完奶时洗,这样容易导致呕吐。
% \item
%   \textbf{温度}:洗澡时的室温需维持在26°C左右,水温维持在38°C
%   \textasciitilde{} 40°C。
% \item
%   \textbf{用品}:小浴盆1个,温度表1只,干浴巾及毛巾各1条,婴儿浴皂1块,爽身粉1盒,替换的清洁衣服各1套,尿布1块。
% \item
%   \textbf{检查}:洗澡之前应检查有无排便,如果有,必须清理好后再洗。
% \end{itemize}

% %ux6d17ux6fa1ux987aux5e8f}{%
% \subsubsection{\texorpdfstring{\textbf{2.
% 洗澡顺序}}{2. 洗澡顺序}}%ux6d17ux6fa1ux987aux5e8f}}

% \begin{itemize}
% \item
%   浴盆中放半盆温水,用水温表量一下水温,将水温调节在适当的范围,或用\textbf{手肘}感觉水温,手肘感觉不凉也不热即可。
% \item
%   脱去婴儿的衣服,\textbf{腹部}用浴巾遮住,用左手固定婴儿头部,右手放在臀部,将婴儿抱稳。\\
%   \hspace{0pt}\includegraphics[width=2.08392in,height=1.8042in]{media/rId1015.png}\hspace{0pt}
% \item
%   左手仍固定头部并略抬高,用左右拇指和中指向前压住婴儿的耳屏,将耳孔盖住,避免水进入耳朵。移出右手,给婴儿头部抹些婴儿香皂洗头,清洗后,然后擦干头发。按同法清洗颈部。\\
%   \hspace{0pt}\includegraphics[width=2.11189in,height=1.87413in]{media/rId1019.png}\hspace{0pt}\\
%   \hspace{0pt}\includegraphics[width=1.97203in,height=2.11189in]{media/rId1022.png}\hspace{0pt}
% \item
%   拿掉浴巾托住婴儿的头背部和臀部,将婴儿轻轻放入浴盆中。移动左手,使婴儿头部枕在左前臂上,用右手清洗腋下。\\
%   \hspace{0pt}\includegraphics[width=2.06993in,height=1.91608in]{media/rId1026.png}\hspace{0pt}
% \item
%   左手恢复托头姿势,右手洗腹部及腹股沟处、腿部及脚部。
% \item
%   轻轻将婴儿翻转,左手托住婴儿前胸,使婴儿侧卧,头部仍略抬高,右手自上而下洗净背部、臀缝。\\
%   \hspace{0pt}\includegraphics[width=2.08392in,height=1.88811in]{media/rId1031.png}\hspace{0pt}
% \item
%   洗澡完毕,左手托住孩子的头颈部,右手抓住双足踩部,离盆,用浴巾包好,擦干,迅速穿上衣服,注意保暖。\\
%   \hspace{0pt}\includegraphics[width=2in,height=1.94406in]{media/rId1035.png}\hspace{0pt}
% \end{itemize}

% 婴儿的洗澡过程最好在10分钟内完成,否则孩子会因体力消耗而感到疲倦。另外,应避免肥皂和水入眼、入耳。

% \begin{quote}
% \textbf{如何去掉孩子的鼻屎?}

% 婴幼儿鼻腔内常有鼻屎,尤其在感冒之后或患过敏性鼻炎时。鼻屎阻塞鼻腔,影响孩子的呼吸和进食。因为孩子鼻腔狭小,成人的手指伸不进鼻腔,而用镊子夹取又怕伤到鼻黏膜。

% 去掉鼻屎的方法是\textbf{用热毛巾敷鼻部},使热蒸气进入鼻腔,鼻屎变软、松动后排出。也可以用棉签蘸些\textbf{生理盐水}或冷开水,使鼻屎变软,然后用细棉签将鼻屎卷出,或用棉签刺激鼻黏膜使孩子\textbf{打喷嚏}而将鼻屎喷出。如果有\textbf{吸鼻器},可以利用负压将鼻部分泌物吸出。【刺激和吸鼻器感觉太危险了】
% \end{quote}

% %ux6d74ux540eux62a4ux7406}{%
% \subsubsection{浴后护理}%ux6d74ux540eux62a4ux7406}}

% 给婴儿穿好上衣后,先清洁脐孔,然后扑爽身粉。2周内的新生儿洗澡时,洗澡水不要浸湿脐部,浴后可以用75\%的酒精棉签清洁脐孔,预防脐部感染。注意,粉不要扑得太多,以防止结成硬块,引起皮肤损伤。可以\textbf{将粉撒在成人的手心中,然后涂到孩子的身上}。

% %ux4e94ux65b0ux751fux513fux7761ux89c9ux4e0dux5b9cux6346ux7ed1}{%
% \subsection{五、新生儿睡觉不宜捆绑}%ux4e94ux65b0ux751fux513fux7761ux89c9ux4e0dux5b9cux6346ux7ed1}}

% 在我国民间,存在一个传统习惯,即在孩子睡觉时,将孩子的腿拉直并用布带捆绑,人们认为只有这样才不会长成罗圈腿。同时,也会将两臂贴在身体两侧固定,认为这样孩子才能睡得香甜,不会受到惊吓,于是用带子将孩子上下捆紧。然而,这种做法实际上限制了孩子在睡觉时的自如动作,固定的姿势使肌肉处于紧张状态。实际上,\textbf{罗圈腿是佝偻病的症状},非捆绑所能预防的。因此,孩子在睡觉时,四肢不应被裹紧。

% %ux516dux65b0ux751fux513fux63a5ux79cdux5361ux4ecbux82d7}{%
% \subsection{六、新生儿接种卡介苗}%ux516dux65b0ux751fux513fux63a5ux79cdux5361ux4ecbux82d7}}

% 孩子在出生后第2天即可接种卡介苗,接种后,可以获得对结核菌的一定免疫能力。卡介苗接种一般在左上臂三角肌处皮内注射,也有在皮肤上进行划痕接种,做``仆''或``井''字形,长1厘米。虽然划痕接种法方便,但因接种量不准,有效免疫力不如皮内注射法,因此目前一般不采用划痕法。

% 新生儿接种卡介苗后,无特殊情况一般不会引起发热等全身性反应。在接种后2\textasciitilde8周,局部出现红肿硬结,逐渐形成小脓疮,以后自行消退。有的脓疮穿破,形成浅表溃疡,直径不超过0.5厘米,然后结痂,痂皮脱落后,局部可留下永久性瘢痕,俗称卡疮。

% 为了判断卡介苗接种是否成功,一般在接种后8-14周,应到所属区结核病防治所再做结核菌素``OT''试验,局部出现红肿0.5-1.0厘米为正常,如果超过1.5厘米,需排除结核菌自然感染。一般新生儿接种卡介苗后,2\textasciitilde3个月就可以产生有效免疫力,约5年后,在小学一年级时,再进行``OT''试验,如呈阴性,可再次接种卡介苗。

% 值得注意的是,早产儿、难产儿以及有明显先天畸形、皮肤病等的孩子,是禁忌接种卡介苗的。

% %ux4e03ux65b0ux751fux513fux8981ux6ce8ux5c04ux4e59ux578bux809dux708eux75abux82d7}{%
% \subsection{七、新生儿要注射乙型肝炎疫苗}%ux4e03ux65b0ux751fux513fux8981ux6ce8ux5c04ux4e59ux578bux809dux708eux75abux82d7}}

% 目前在世界各国,乙型肝炎的患病率均高得令人吃惊。为此,我国有关部门研究出乙型肝炎疫苗,这种疫苗没有传染性,对乙肝病毒具有很好的免疫性能,已在新生儿中广泛应用。

% 整个免疫注射要打3针,第1针(一般由产科婴儿室医务人员注射)于孩子出生后24小时内在上臂三角肌处注射,剂量为5微克。第2针在出生后1个月注射,剂量为5微克。第3针在出生后6个月注射,剂量为5微克。全部免疫疗程结束后,有效率可达90\%-95\%。婴幼儿接种疫苗后,可获得免疫力达3-5年之久。

% 免疫疫苗接种过程简单,一般没有什么反应,个别孩子可能出现低热,有的在接种部分出现小的红晕和硬结,一般不用护理,1\textasciitilde2天可自行消失。

% %ux516bux65b0ux751fux513fux623fux95f4ux7684ux73afux5883ux536bux751f}{%
% \subsection{八、新生儿房间的环境卫生}%ux516bux65b0ux751fux513fux623fux95f4ux7684ux73afux5883ux536bux751f}}

% 我国有个传统习惯,就是把产妇和孩子严严地捂在房间里。这样做,实际上给产妇和婴儿造成了一个昏暗和污浊的环境。尤其是在夏季,室内更加闷热,很容易引起孩子发热,起脓包疹,长痱子,以及患呼吸道疾病,产妇也容易中暑。

% 科学的方法是要保持产妇和新生儿室内空气的清新。在温暖的季节,每天都要通风换气。当然,开窗之前,要给产妇和婴儿适当的遮盖,\textbf{不要让风直吹在他们身上},要避免产生对流风。在夏季要使室内温度保持在30℃以下,可在地面洒些水,既可降温,又可使室内空气保持一定湿度。冬季室温最好保持在20\textasciitilde22℃,也可以使用空气加湿器来湿化空气,防止呼吸道疾病的发生。通风要谨慎,应避免穿堂风,且不可时间过长。生火炉的家庭,一定要注意烟筒通畅,不要将没有烟筒的火炉子搬进室内,以防止发生煤气中毒。

% \hspace{0pt}\includegraphics[width=2.6014in,height=2.58741in]{media/rId1045.png}\hspace{0pt}

% %ux4e5dux65b0ux751fux513fux9700ux8981ux4fddux6696}{%
% \subsection{九、新生儿需要保暖}%ux4e5dux65b0ux751fux513fux9700ux8981ux4fddux6696}}

% 新生儿的体温调节机制还不健全,因而给孩子保暖十分重要。如何观察孩子是冷还是热呢?一般可以摸孩子露着的部位,如面额、手等,\textbf{以不凉无汗为合适}。若孩子四肢发凉,说明温度不够,要想办法加热。例如使用热水袋保暖,热水袋的温度应在\textbf{50℃左右}。要将热水袋放在孩子棉被下,\textbf{不要直接接触皮肤},以免烫伤。【还有低温烫伤的说法】

% %ux5341ux5a74ux513fux7761ux7720ux59ffux52bfux4f18ux52a3}{%
% \subsection{十、婴儿睡眠姿势优劣}%ux5341ux5a74ux513fux7761ux7720ux59ffux52bfux4f18ux52a3}}

% 据说国外的婴儿经常趴着睡,这样遵循了一种``自然规律'',即人生来是喜欢趴着睡的;有人说婴幼儿侧着睡最适合;有人说仰着睡会睡成个``扁脑袋''\ldots\ldots 究竟哪种睡姿对婴儿最有利,目前并没有科学的定论,多主张仰睡。作为父母应该了解一下三种睡姿各自的优缺点,再根据自家宝宝的情况灵活掌握。

% %ux4fefux5367}{%
% \subsubsection{1. 俯卧}%ux4fefux5367}}

% 优点:

% \begin{itemize}
% \item
%   婴儿有安全感。胎儿在妈妈子宫里就是腹面部朝内、\textbf{背部朝外的蜷曲姿势}。人体的腹面部相对于背部来说,缺少骨骼的保护,容易受到外界伤害,且比较敏感。这种姿势是最自然的自我保护姿势。\\
%   婴儿好像天生就爱趴着睡,也是基于上述原理,把身体最脆弱的部分保护起来,睡觉时更有安全感,容易睡得熟,从而减少哭闹,有利于神经系统的发育。
% \item
%   有益于婴儿胃的蠕动及消化。趴着睡时,胃溶物不易流到食道及口中,而引起呕吐;反而蠕动到小肠中,有利于消化吸收。
% \item
%   俯卧可使婴儿受抬头挺胸的带动,从而锻炼了颈部、胸部、背部及四肢等大肌肉群,有利于翻身和爬行训练。
% \end{itemize}

% 缺点:

% \begin{itemize}
% \item
%   容易引起窒息。婴儿的头较重,而颈部力量不足,在不会自如地转头或翻身时,口鼻易被枕头、毛巾等堵住,造成窒息,甚至危及生命。
% \item
%   不利于散热。婴儿胸腹部紧贴床铺,不易散热,容易引起体温升高,或者由于汗液积于胸腹而产生湿疹。
% \item
%   四肢手脚不易活动。
% \end{itemize}

% %ux4fa7ux5367}{%
% \subsubsection{2. 侧卧}%ux4fa7ux5367}}

% 优点:

% \begin{itemize}
% \item
%   \textbf{右侧卧可减少呕吐或溢奶}。婴儿胃的出口与十二指肠均在腹部右侧,右侧卧可使胃容物更易流到小肠;若万一发生呕吐,侧卧可使口腔内的呕吐物由低的嘴角流出,而不至于流到咽喉。
% \item
%   帮助肺部痰的引流。若一侧肺部发生炎症,可采取另一侧卧,使发炎部位的分泌物(痰)易于流出气管。
% \item
%   减少打鼾。婴儿睡觉打鼾,多由咽喉部分泌物及软组织相互振动而产生的。侧卧能够改变咽喉软组织的位置,减少分泌物的滞留,使婴儿的呼吸更顺畅,也就不会打鼾了。
% \end{itemize}

% \textbf{缺点}

% \begin{enumerate}
% \def\labelenumi{\arabic{enumi}.}
% \item
%   \textbf{左侧卧易引起呕吐或溢奶}。婴儿胃与食道的交界偏左侧,左侧卧时胃容物易回流到食道中。
% \item
%   维持姿态比较累。婴儿的身体是滚圆的,四肢又比较短,维持侧卧姿势并不容易,需要用枕头在前胸及后背支撑。
% \end{enumerate}

% %ux4ef0ux5367}{%
% \subsubsection{3. 仰卧}%ux4ef0ux5367}}

% 优点

% \begin{enumerate}
% \def\labelenumi{\arabic{enumi}.}
% \item
%   不必担心呼吸。婴儿口鼻直接向上接触空气,通常不会有外物遮堵而影响呼吸。
% \item
%   可直接观察婴儿睡况。口鼻是否有过多分泌物;有没有呕吐;脸上是否有怪异表情或脸色不正常等均可立即发现,并采取措施。
% \item
%   四肢活动灵活。四肢不受局限,使婴儿睡眠比较放松、自在。
% \end{enumerate}

% 缺点

% \begin{enumerate}
% \def\labelenumi{\arabic{enumi}.}
% \item
%   易发生呕吐。婴儿胃的生理结构使仰卧时胃容物易回流食道造成呕吐;同时吐出物不易流出口外,会聚集在咽喉处,容易呛入气管及肺发生危险。
% \item
%   心理上安全感较小。因为把人体较脆弱的部位都暴露在外,心理上缺乏安全感,不易熟睡。
% \item
%   容易着凉。胸腹部皮肤较薄易散热,若是没有采取适宜的保暖措施,容易引起着凉。
% \end{enumerate}

% %ux5341ux4e00ux9009ux62e9ux7761ux59ffux9700ux8981ux6ce8ux610fux7684ux4e8bux9879}{%
% \subsection{十一、选择睡姿需要注意的事项}%ux5341ux4e00ux9009ux62e9ux7761ux59ffux9700ux8981ux6ce8ux610fux7684ux4e8bux9879}}

% \begin{enumerate}
% \def\labelenumi{\arabic{enumi}.}
% \item
%   不能让婴儿总固定一种睡姿,应该三种睡姿交替进行。
% \item
%   3个月内的婴儿肌肉力量不足,应采用仰卧。
% \item
%   会自己翻身的婴儿,在身边有人观察其睡眠的情况下,可以俯卧。
% \item
%   婴儿仰卧时应该不用枕头,以免因颈部抬高,而使颈部及咽喉处弯曲,不但影响以后的体态,还可能造成呼吸困难,如用枕头,也应垫在婴儿的颈肩部。
% \item
%   稍大一些的婴幼儿可以侧卧,但身前身后用枕头或侧卧枕夹靠住。
% \item
%   极易呕吐、呼吸时咽喉有杂音的,或较神经质的婴儿可以采取俯卧。
% \item
%   \textbf{俯卧}有助于婴儿的脸型变得\textbf{较圆且偏长},\textbf{但需要经常左右更换}来达到目的。
% \item
%   \textbf{仰卧}有助于婴儿的脸型变成传统上中国人比较喜好的\textbf{方头大脸}。
% \item
%   \textbf{侧卧}可使婴儿的头型\textbf{较狭长}。
% \end{enumerate}

% %ux5341ux4e8cux65b0ux751fux5b9dux5b9dux7684ux5185ux8863}{%
% \subsection{十二、新生宝宝的内衣}%ux5341ux4e8cux65b0ux751fux5b9dux5b9dux7684ux5185ux8863}}

% %ux5e94ux4eceux4f55ux65f6ux5f00ux59cbux7a7fux5185ux8863}{%
% \subsubsection{1.
% 应从何时开始穿内衣}%ux5e94ux4eceux4f55ux65f6ux5f00ux59cbux7a7fux5185ux8863}}

% 经专家认定,宝宝应该从一出生即开始穿内衣,特别是在天气寒冷时出生的宝宝。这样做有以下几点益处:

% \textbf{能促进新生宝宝大脑和运动机能发育}

% 内衣具有保暖作用,妈妈无须担心宝宝冷而再采用包布包裹宝宝,使宝宝的胳膊和腿被捆得不便动弹,从而影响四肢的活动。若给婴儿按节令变化穿上适宜的内衣,婴儿会自由地躺在床上,小手小脚任意乱踢乱蹬,这些\textbf{主动自由的伸展动作},不但能增强肌肉和骨骼的发育,还可因此加深呼吸、促进血液循环及新陈代谢,并由于神经肌肉反射的活动而促进大脑运动机能发育。

% \textbf{保暖御寒预防宝宝受凉感冒}

% 医生经常发现,反复因受凉感冒的婴儿,大多未穿贴身内衣,身体摸上去凉冰冰,尤其是下半身。没有给婴儿穿内衣的妈妈以为,天气凉时多给孩子穿厚实的外衣即可御寒保暖。其实,厚实的外衣只有挡风效果,不能像棉内衣那样,特别是寒冷时节的双层针织棉内衣,能够吸收保留空气,使空气围护在皮肤四周而达到保温,既舒适又暖和,婴儿也不容易着凉而患感冒。

% \textbf{培养宝宝从小养成现代文明的生活习惯}

% 给新生宝宝穿内衣,使他们从一出生便开始感受舒适的感觉,这对今后形成有品质的健康生活习性大有裨益。

% %ux4ec0ux4e48ux6837ux7684ux5185ux8863ux9002ux5408ux5b9dux5b9d}{%
% \subsubsection{2.
% 什么样的内衣适合宝宝}%ux4ec0ux4e48ux6837ux7684ux5185ux8863ux9002ux5408ux5b9dux5b9d}}

% 专家表示这个问题应该从婴儿以下几方面的生理特点来考虑:

% \textbf{做工}

% 婴儿的皮肤最外层的耐磨性角质层非常薄,即使是不大的刺激也容易使皮肤变红,甚至受损。因此,内衣的质地是否柔软应该是妈妈首要考虑的因素。妈妈在选择时,可以先用手背抚摩一下是否柔软,然后翻看里边的缝边是否因制作粗糙而发硬,特别要注意腋下和领口处;若能给新生的小宝宝买到缝边朝外的内衣则对他们更为合适。【现在很多是这样子了】

% \textbf{质地}

% 婴儿处在生长发育旺盛时期,新陈代谢快,活动量也很大,非常容易出汗。如果选择具有吸汗和排汗功能的棉织品,会减少汗液对婴儿皮肤的刺激,从而避免发生各种皮肤问题。在选择时,妈妈必须注意质地标签上是否标有``100\%
% cotton(棉)''的标志。

% \textbf{保暖}

% 由于婴儿体表面积相对较大,皮肤表层易于散热的血管多,而防止体内热量散发的脂肪层却较薄,因此婴儿容易体温低下。妈妈绝对不能忽视内衣的保暖性,特别是在气温降低时出生的婴儿,需要穿上双层有伸缩性的全棉织品,外面再套上薄绒衣等。

% \textbf{方便}

% 婴儿头较大而\textbf{脖颈却较短},新生的宝宝更是脖子软绵绵,这使得衣服的穿脱非常不方便。婴儿内衣款式不仅应该简洁,而且要穿脱方便。传统式前开襟、无领、腰边系带子的和尚宝宝服非常适宜婴儿,而且它还能在胸腹处重叠,这种设计会避免婴儿腹部受凉而引起腹泻。

% \textbf{花色}

% 婴儿内衣的色泽应该是浅淡的,无花案或仅有稀疏小花案,这样便于观察尿色、便色及血迹,从而及早发现异常情况,也可以避免有色染料对婴儿皮肤的刺激。但是,如果内衣的颜色非常白,可能含有荧光增白剂等化学物质,最好不要选购。100\%纯棉内衣的颜色\textbf{白中透着淡黄},色泽看起来自然而柔和。为了确保安全和卫生,内衣在临穿前,都应先清洗一遍。

% \begin{quote}
% 温馨提示

% 给婴儿买内衣时,不能像买外衣那样总是选择大一尺码的,应该选择适合婴儿月龄或身长大小的型号,让婴儿穿得舒适。
% \end{quote}

% %ux5341ux4e09ux5b9dux5b9dux7684ux5185ux8863ux4e70ux56deux6765ux5c31ux8981ux6d17}{%
% \subsection{十三、宝宝的内衣买回来就要洗}%ux5341ux4e09ux5b9dux5b9dux7684ux5185ux8863ux4e70ux56deux6765ux5c31ux8981ux6d17}}

% 无论婴儿内衣是否有甲醛等化学物质存留,也要先下水洗涤后,再给婴儿穿。经过洗涤后,一些化学物质的残留量会有所减少;而且,也可将棉絮、细小纤维及内衣在制作、搬运、出售等过程中因经过许多人的手而带来的部分细菌和脏污除去,更能保证卫生,保护婴儿的皮肤健康。

% %ux4f7fux7528ux4e13ux7528ux6d17ux8863ux6db2}{%
% \subsubsection{1.
% 使用专用洗衣液}%ux4f7fux7528ux4e13ux7528ux6d17ux8863ux6db2}}

% 内衣直接接触婴儿娇嫩的皮肤,洗衣粉、肥皂等碱性都比较大,不适于用来洗涤婴儿的内衣,应该选用专为婴儿设计的洗衣液来清洗。这些洗衣液对婴儿身上经常会出现的奶渍、汗渍、果汁渍有特效,去污力强,易漂洗,对皮肤无刺激、无副作用;通常还是无磷、无铝、无碱、不含荧光剂的环保产品。

% 在没有专用洗衣液的时候,也必须选用纯中性且不含荧光剂的洗衣粉(液),同时注意将成人与婴儿的衣服分开洗。

% %ux79d1ux5b66ux4fddux517b}{%
% \subsubsection{2. 科学保养}%ux79d1ux5b66ux4fddux517b}}

% 婴儿虽然长得很快,衣服刚买不久就穿不上了,但妈妈也应该注意保养好婴儿内衣。先不去管以后是否还有用,至少现在能让婴儿穿得更舒服些。保养的前提是洗涤方式要正确,应该按照产品标识上的洗烫方法处理,这才是妈妈对宝宝最贴心的呵护。

% \begin{quote}
% \textbf{胎记不是病}

% 在孩子出生时以及之后的一段时间里,可能会在身上发现有青色的斑块,这就是俗称的"胎记"。胎记多出现在孩子的背部、骶骨部、臀部,少见于四肢,偶尔会出现在头部、面部,形态大小不等,颜色深浅各有差异。这种青色斑是胎儿时期色素细胞堆积的结果,对身体没有什么影响。随着年龄的增长,到儿童时期,它们会逐渐消退,不需要治疗。
% \end{quote}

% %ux5341ux56dbux89c2ux5bdfux5b69ux5b50ux56dfux95e8}{%
% \subsection{十四、观察孩子囟门}%ux5341ux56dbux89c2ux5bdfux5b69ux5b50ux56dfux95e8}}

% 孩子在1岁半之内,头盖骨还没有发育完全,头部各块颅骨之间留有缝隙。位于头顶部中间靠前一点的地方,有一块菱形间隙,一般斜径有2.5厘米左右,医学名叫前囟。用手摸上去有跳动感觉,这是头皮下的血管中血液在流动,不是病态。

% 孩子在患上某些疾病时,囟门会发生变化。例如,严重吐泻、脱水的孩子可能出现囟门凹陷的现象,脑膜炎时,由于脑压增高,囟门可以凸起。一般来说,囟门在1岁半左右闭合,如果囟门过早闭合,可能是脑发育不良或者头型畸形。若囟门过晚闭合,可能是佝倭病或者甲状腺功能低下(呆小病)。

% \begin{quote}
% 你知道吗?

% 怎样早期发现克汀病呢?母亲应注意,在新生儿期,如果孩子黄疸持续不退,吃奶不好,反应迟钝,爱睡觉,很少哭闹,经常便秘,哭声与正常孩子不一样,声音嘶哑,就应该请医生检查。如果延误诊断,到2\textasciitilde3个月时会发现更多的症状,例如舌大且常伸出口外,鼻梁塌平,脖子短,头发干而黄且稀疏,皮肤干燥粗糙,肚子相对较大。这时就不应该再拖延,一定要尽早请医生诊治。克汀病的治疗必须争分夺秒,早一天给孩子使用甲状腺素治疗,孩子的智力发育就会好一些。
% \end{quote}

% %ux5341ux4e94ux9884ux9632ux514bux6c40ux75c5}{%
% \subsection{十五、预防克汀病}%ux5341ux4e94ux9884ux9632ux514bux6c40ux75c5}}

% 克汀病是由于孩子体内缺少甲状腺素而引起的一种病。甲状腺素是人体生长发育中必不可少的内分泌激素。孩子如果缺乏这种激素,就会影响其脑细胞和骨骼的发育。如果在出生后到1岁以内不能及早发现和治疗,可能会造成孩子终身智能低下和矮小。

% 克汀病主要的病因有两种,一是某些地区缺乏微量元素碘,缺碘的妇女在怀孕后,供给胎儿的碘就不足,导致胎儿期甲状腺素的缺乏。二是孩子先天甲状腺功能发育不良。

% %ux5341ux516dux5a74ux513fux9700ux8981ux5408ux9002ux7684ux6795ux5934}{%
% \subsection{十六、婴儿需要合适的枕头}%ux5341ux516dux5a74ux513fux9700ux8981ux5408ux9002ux7684ux6795ux5934}}

% 婴幼儿的枕头长度应与其肩宽相等或稍宽些,宽度略比头长一点,高度约5厘米。枕套最好用棉布制作,以保证柔软、透气。

% 枕芯应有一定的软度,可以选择荞麦皮或蒲绒的;塑料泡沫枕芯透气性差,最好不用。质地太硬的枕头可能会使孩子的颅骨变形,不利于头颅的发育;弹性太大的枕头也不好,孩子枕时,头的重量下压,半边头皮紧贴枕头,可能会使血流不畅。木棉枕、泡沫枕通风散热性能差,不适合夏天使用。父母在为孩子选择枕头时,要从高度、硬度、通风散热排汗、不变形等各方面综合考虑。

% %ux5341ux4e03ux4f60ux4f1aux62b1ux5b9dux5b9dux5417}{%
% \subsection{十七、你会抱宝宝吗}%ux5341ux4e03ux4f60ux4f1aux62b1ux5b9dux5b9dux5417}}

% \begin{quote}
% 温馨提示:在宝宝8周以前,因为不能控制自己的头部和肌肉,所以在移动宝宝时,一定\textbf{要扶住他的身体},使他的头不致向后仰,四肢向下垂。
% \end{quote}

% 抱宝宝的姿势应随宝宝月龄的增加而变换。1\textasciitilde2个月的宝宝只能横着抱,可先将一只手放到宝宝的颈下,托起宝宝的头颈部,另一只手伸到宝宝的腰臀部,然后慢慢直起身子,用双手将宝宝轻轻地抱至胸前,横抱于怀中。整个过程中要特别当心宝宝的头部,应该用手托住或用前臂支撑住宝宝的头。

% 小月龄的宝宝就像易碎物品,必须``小心轻放''。在抱起和放下宝宝时,一定要动作轻柔。可用一手托住宝宝的头和颈部,另一手抱住腰臀部,\textbf{确保每一个过程都平稳、安全}。

% \begin{enumerate}
% \def\labelenumi{\arabic{enumi}.}
% \item
%   抱起仰卧的宝宝:把一只手轻轻地放在宝宝的头下方,另一只手从对侧轻轻地放在宝宝的下背部和臀部下方。轻轻地、慢慢地将宝宝抱起来,这样宝宝的身体才有依靠,头才不会往后仰。将宝宝的头小心地转到妈妈的肘弯或肩膀上,让宝宝的头有依靠。
% \item
%   抱起俯卧的宝宝:先将一只手轻轻地放在宝宝的胸部下方,用前臂支住他的下巴,再把另一只手放在他的臀下。慢慢地抬高宝宝,让宝宝的脸转向妈妈,靠近妈妈的身体,一只支撑宝宝头部的手向前滑动,直到宝宝的头舒适地躺在妈妈的肘弯。另一只手则放在宝宝的臀下和腿部。
% \item
%   抱起侧卧的宝宝:将一只手轻轻放在宝宝的头颈下方,另一只手放在臀下。将宝宝抱进妈妈的手中,慢慢地、轻轻地抬高宝宝。将宝宝靠着妈妈的身体一侧抱住,然后将妈妈的前臂轻轻地滑向宝宝的头下方,让宝宝的头安全地靠在妈妈的肘部。
% \item
%   放下宝宝的方法:将一只手放在宝宝的头颈下方,然后用另一只手抱住宝宝的臀部,慢慢地放下宝宝。手一直扶住他的身体,直到把宝宝完全放到床上为止。从宝宝的臀部轻轻地抽出妈妈的手,用这只手稍稍抬高宝宝的头部,然后轻轻抽出妈妈的另一只手,再慢慢地将宝宝的头放在床上。
% \item
%   侧着放下宝宝:让宝宝躺在妈妈手臂中,宝宝的头靠在妈妈的肘部。将宝宝放在床上后,轻轻地抽出妈妈置于宝宝臂下的那只手。抬高宝宝的头,抽出妈妈放在宝宝头下的那只手,然后轻轻放下宝宝的头。
% \end{enumerate}

% %ux5341ux516bux7537ux5b9dux5b9dux751fux6b96ux5668ux7578ux5f62ux65e9ux53d1ux73b0}{%
% \subsection{十八、男宝宝生殖器畸形早发现}%ux5341ux516bux7537ux5b9dux5b9dux751fux6b96ux5668ux7578ux5f62ux65e9ux53d1ux73b0}}

% 儿童的生殖器畸形有许多种,男孩常见的有包茎、阴茎包皮过长、隐睾、睾丸鞘膜积液、尿道上裂或下裂等。

% 生殖器官的异常通常可以通过外观及比较看出来,有些家长在宝宝出生后不久就会看出异常,而有的家长却在宝宝长到几岁了才有所察觉,导致宝宝因家长的疏忽而延误了治疗,不仅影响其儿童时期的正常发育,还可影响成年后的性功能和生育能力,有的还会发展成恶性疾病而危及生命。

% %ux5305ux76aeux8fc7ux957fux4e0eux5305ux830e}{%
% \subsubsection{1
% 包皮过长与包茎}%ux5305ux76aeux8fc7ux957fux4e0eux5305ux830e}}

% 包皮完全包裹龟头,不能上翻显露龟头的为包茎,能上翻显露龟头,则为包皮过长。

% 治疗:并不是所有的包茎都需要手术,特别是3岁以内的宝宝一般不主张手术。因为这一年龄段的宝宝多为生理性包茎,常可自行缓解。但若发生反复感染、尿流细、局部红肿伴尿频等症,则应该去医院看医生,医生根据具体病情进行不同的治疗。对无包皮口狭窄患者可进行包皮粘连分离,此种情况可避免手术。如果包皮口已经形成瘢痕环,应尽早选择包皮环切术。学龄期及以后的宝宝多为病理性包茎,包皮环切的指征是明确的。对包皮过长者也需手术。

% %ux9690ux533fux9634ux830e}{%
% \subsubsection{2 隐匿阴茎}%ux9690ux533fux9634ux830e}}

% \textbf{排查诊断:}正常的阴茎皮肤应该呈套袖状紧紧包裹在阴茎体上,这是阴茎的正常形态。而隐匿阴茎的基本病变则是阴茎体与阴茎皮肤相分离。尤其是在阴茎根部皮肤分离,使阴茎皮肤呈锥状,并把阴茎体隐藏于皮肤下。尽管包皮腔空虚,触不到阴茎体,但是后推阴茎根部皮肤大多可显露出阴茎体。

% \textbf{治疗:}由于此病大多数为肥胖儿,所以有人提出减轻体重有助于改善症状。隐匿阴茎一般手术年龄在学龄期或稍晚些为宜,不必在很小就行手术治疗。但是值得注意的是,隐匿阴茎千万不要误认为包茎而行单纯的包皮环切术。过多的包皮被切掉,会使病变加重,不仅会使阴茎体显露异常加重,还会使日后进行阴茎体固定成型术时,因阴茎皮肤缺乏而加大手术难度。

% %ux9690ux777e}{%
% \subsubsection{3 隐睾}%ux9690ux777e}}

% \textbf{排查诊断:}隐睾的诊断并不难,即用手在阴囊内摸不到睾丸,但要注意与气候寒冷时的睾丸回缩相区别。

% \textbf{治疗:}1岁以内的宝宝通过一些药物的应用如绒毛膜促性腺激素,有可能使睾丸降入阴囊。如果到了2岁仍然不能下降入阴囊,则要考虑手术治疗。资料显示,在2岁以前手术对睾丸的生精功能无太大影响,超过4岁则会明显影响,超过8岁则会严重影响。如果超过12岁即使做了手术,睾丸的生精功能也无法恢复。因此,隐睾下降固定术应在2岁以前进行。

% %ux5c3fux9053ux4e0bux88c2}{%
% \subsubsection{4. 尿道下裂}%ux5c3fux9053ux4e0bux88c2}}

% \textbf{排查诊断}:尿道下裂,即尿道开口异常,是宝宝常见的先天性阴茎发育畸形,在男性新生儿中的发生率约为8\%。尿道开口靠前(即阴茎头型)的宝宝虽不妨碍站立排尿,但阴茎外形多有异常,尿线向下斜。尿道开口在阴茎干、阴囊或会阴者,不仅需要下蹲排尿,还可能因伴发阴茎下曲畸形,影响阴茎正常发育。至青春期阴茎勃起时弯曲更为严重,且可能伴有疼痛及性生活困难。

% \textbf{治疗}:尿道下裂必须通过手术进行矫正。手术应在学龄前完成,或者更早一些以减少给宝宝带来的心理创伤。一般来说,3岁左右的宝宝就能达到完成手术修复的基本条件。对于短段型尿道下裂及无阴茎下曲畸形者,也可以提前进行手术。

% 对于伴有阴茎发育不良的患者,因为可能伴有内分泌缺陷,应先进行相应检查并给予激素治疗。待阴茎增大后再进行手术。

% %ux5341ux4e5dux5973ux5b9dux5b9dux751fux6b96ux5668ux72b6ux6001}{%
% \subsection{十九、女宝宝生殖器状态}%ux5341ux4e5dux5973ux5b9dux5b9dux751fux6b96ux5668ux72b6ux6001}}

% 有些女孩的阴道上皮和子宫内膜由于受出生前母体雌激素的影响而增生,然而出生后母体雌激素的影响突然中断,于是增生的阴道上皮和子宫内膜会脱落,流出的通常呈白色黏液,这就是白带。少数女孩还可能从阴道里流出血性分泌物,呈红色,这被称为假月经。这种情况父母不用过于担忧,这是正常现象,不需要任何处理就会自行消失。当然,遇到这种情况,家长需要用清洁的水和毛巾给宝宝轻轻擦洗阴部,以防产生异味。

% %ux4e8cux5341ux5973ux5b9dux5b9dux751fux6b96ux5668ux7684ux65e9ux671fux5f02ux5e38ux53d1ux73b0}{%
% \subsection{二十、女宝宝生殖器的早期异常发现}%ux4e8cux5341ux5973ux5b9dux5b9dux751fux6b96ux5668ux7684ux65e9ux671fux5f02ux5e38ux53d1ux73b0}}

% \begin{enumerate}
% \def\labelenumi{\arabic{enumi}.}
% \item
%   \textbf{处女膜闭锁(先天发育异常)}:正常的处女膜中央有一小孔,称为处女膜孔。处女膜闭锁者就是没有此孔,在阴道口处可见有一暗蓝(淡蓝)色圆形肿物,有的家长误以为是肿瘤。\\
%   治疗:该病越早治疗越好,否则会使病情复杂化,甚至导致终生不孕。
% \item
%   \textbf{先天性无阴道或阴道阙如}:这是一种先天性的发育异常。患者在膀胱、尿道与直肠间无正常的阴道,两者之间仅有一些疏松的组织,有些患者在相当于阴道外口处可见一浅窝。青春期后由于经血潴留,可能出现周期性腹痛,无月经或直至婚后才被发现。\\
%   治疗:先天性无阴道女孩在月经初潮前应进行阴道扩张术或阴道再造术。
% \item
%   \textbf{小阴唇粘连}:一般由外阴发炎所致,表现为一层灰色略透明的薄膜将小阴唇连在一起,外观上看不到阴道,只能看见有一小口排尿。主要表现是排尿费力,尿不成线。严重的可以产生排尿困难。到了青春发育期以后,还可能出现经血排出不畅,导致痛经。\\
%   治疗:确诊后施行分离手术即可,治疗并不复杂。
% \item
%   \textbf{直肠阴道瘘}:这是一种先天性肛门直肠畸形。这种情况常被忽略,往往在发现尿液内有粪便或尿液混浊时才引起注意。这些病变很容易引起膀胱炎或肾盂肾炎。新生儿出生后一天仍无胎粪排出,喂奶时发现腹胀、呕吐、哭闹不安等,应引起注意,观察有无肛门直肠畸形的可能。\\
%   治疗:由于直肠阴道瘘的成因复杂、种类繁杂、手术后感染、复发率高,术式的选择极其重要。手术的目的是恢复有正常控制能力的排便功能。
% \end{enumerate}

% %ux7b2cux4e8cux828212ux4e2aux6708ux5a74ux513fux7684ux65e5ux5e38ux62a4ux7406}{%
% \subsection{02第二节1〜2个月婴儿的日常护理}%ux7b2cux4e8cux828212ux4e2aux6708ux5a74ux513fux7684ux65e5ux5e38ux62a4ux7406}}

% %ux4e00-ux7761ux7720}{%
% \subsection{一、 睡眠}%ux4e00-ux7761ux7720}}

% 两个月的宝宝每天睡16\textasciitilde18小时,有的宝宝出现日夜颠倒的现象,即白天睡、晚上醒,生物钟与父母不同步。日夜颠倒现象会持续一段时间。一直要到第20周时,宝宝的生物钟才基本上与妈妈相似,即日醒夜睡。

% 如果日夜颠倒比较严重,则可通过与宝宝对话,做按摩或被动操等使宝宝白天少睡。到万不得已时,可在医生指导下采取药物处理。

% %ux4e8cux5a74ux513fux8981ux591aux6652ux592aux9633}{%
% \subsection{二、婴儿要多晒太阳}%ux4e8cux5a74ux513fux8981ux591aux6652ux592aux9633}}

% 不论春夏秋冬,家长每天要抱孩子晒太阳,因为在人体皮肤中含有一种维生素D源,这种物质经日光中紫外线的照射后,才能转变为维生素D,这是人体维生素D的主要来源。维生素D的作用在于促使身体吸收钙,预防佝偻病。

% 晒太阳时,要尽量暴露孩子的皮肤,才能多接受紫外线。不在室内晒太阳,因为玻璃挡住了大部分紫外线,隔着玻璃晒太阳,起不到应有的作用。在炎热的夏季,不要让孩子接受日光的直射,强烈的日光照射皮肤对人体是有害的,可以选择上午9:00-10:00和下午4:00-5:00,\textbf{避开阳光最强烈的时刻}。

% 在寒冷的\textbf{冬季,要选择天气较好的中午}抱孩子晒一晒太阳,但一定要注意保暖。

% %ux4e09ux4e0dux8981ux5243ux80ceux53d1}{%
% \subsection{三、不要剃胎发}%ux4e09ux4e0dux8981ux5243ux80ceux53d1}}

% 民间习惯给快满月的宝宝剃满月头。\textbf{剃胎发对新生儿并无好处},相反,可能使宝宝的头皮上肉眼看不到的毛孔受到损伤。如果剃刀不干净或头部不清洁,细菌很容易经过肉眼看不见的创伤进入体内,引起皮肤炎症,甚至患败血症。【看过理tu了的】

% 如果想要部分胎毛留作纪念,妈妈可以用剪刀剪些长的胎发,而\textbf{不必用剃刀剃}。如果胎发蓬乱,可以用梳子梳理一下。

% %ux56dbux526aux6307ux7532}{%
% \subsection{四、剪指甲}%ux56dbux526aux6307ux7532}}

% 刚出生的宝宝指甲长得非常快,同时两只小手还不停地动,到处乱抓,很容易把自己的小脸皮儿抓破。这对于没有经验的新妈妈来说确实是一个难题。给宝宝剪指甲时,宝宝会很不配合,使妈妈无从下手,不是将宝宝的指甲剪得太深,就是伤到手指皮肤。要想顺利进行,妈妈应该掌握一些小窍门:

% \begin{enumerate}
% \def\labelenumi{\arabic{enumi}.}
% \item
%   宝宝躺卧床上,妈妈跪坐在宝宝一旁,再将胳膊支撑在大腿上,这样可以让手部动作稳固。
% \item
%   握住宝宝的小手,将宝宝的手指尽量分开,用宝宝专用指甲刀靠着指甲剪。
% \item
%   要把指甲剪成圆弧状,不要尖。剪完后,妈妈用自己的拇指肚,摸一摸有没有不光滑的部分。
% \item
%   不要给宝宝剪得太深,以免引起疼痛。
% \item
%   不爱剪指甲的宝宝可以在他熟睡时剪。
% \item
%   最好一周剪\textbf{2\textasciitilde3次}指(趾)甲。
% \item
%   不要在宝宝玩得高兴的时候剪指甲,以免剪伤手指皮肤。
% \end{enumerate}

% %ux4e94ux8badux7ec3ux4fefux5367ux62acux5934ux52a8ux4f5c}{%
% \subsection{五、训练俯卧抬头动作}%ux4e94ux8badux7ec3ux4fefux5367ux62acux5934ux52a8ux4f5c}}

% 每天可让宝宝将两手放在胸前俯卧一会儿。这样训练一段时间后,宝宝会出现片刻的抬头动作。逐渐地,他可以将头抬到与床面呈30度的位置,时间也从几秒钟延长到1分钟左右。训练俯卧抬头是训练爬的最初准备动作。

% %ux516dux9884ux9632ux63a5ux79cd}{%
% \subsection{六、预防接种}%ux516dux9884ux9632ux63a5ux79cd}}

% 孩子满两个月的时候,应该服用第1丸小儿麻痹糖丸。这种糖丸是用来预防孩子麻痹疾病的,若不服用这种糖丸,孩子患孩子麻痹病的危险很大。

% 孩子麻痹这种病,在医学上称为``脊髓灰质炎'',是脊髓灰质炎病毒引起的。这种病毒经口进入胃肠,可侵犯脊髓前角,引起肢体瘫痪,导致终生残疾。髓灰质炎疫苗即孩子麻痹糖丸,是由\textbf{减毒}的髓灰质炎病毒制成的。孩子口服糖丸后,身体内就会形成抵抗髓灰质炎病毒的抗体,而免于此病的发生。因此每个孩子都应在规定的时间内按时服用。

% 根据免疫预防接种程序,满2个月的婴儿开始第一次服用脊髓灰质炎三价混合疫苗,满三四个月时分别服第2次和第3次,4岁时再服一次。这样就可以获得较强的抵抗脊髓灰质炎病毒的免疫力,不患小儿麻痹症了。

% \begin{quote}
% 糖丸要立即给孩子服用,不要放置,以免失效。服用的方法是:将糖丸研碎,用凉水溶化,千万不要用热水溶,以免把糖丸病毒烫死而失去免疫作用,然后用小勺给孩子喂下。服药后1小时内,不能喂热开水。

% 什么情况下不能服用糖丸呢?近期发烧、腹泻或有先天免疫缺陷及其他严重疾病的婴儿均不能服用,以免引起不良反应或加重病情。
% \end{quote}

% %ux4e03ux53eaux5403ux9499ux7247ux5e76ux4e0dux80fdux9884ux9632ux4f5dux507bux75c5}{%
% \subsection{七、只吃钙片并不能预防佝偻病}%ux4e03ux53eaux5403ux9499ux7247ux5e76ux4e0dux80fdux9884ux9632ux4f5dux507bux75c5}}

% 单纯地给孩子吃很多钙片并不能预防佝偻病,必须在适量的维生素D的促进下,才能使身体吸收的钙达到抗佝偻病的效果。

% 人服用维生素D后,经过肝脏、肾脏的代谢,转变为有活性的维生素D,才能使肠道吸收钙、磷进入血液,维持血液中钙的正常浓度,并能将钙、磷输送到达骨骼。所以说,只吃钙片不吃维生素D达不到预防佝偻病的目的。

% %ux516bux4e0dux8981ux7ed9ux5b69ux5b50ux6234ux624bux5957}{%
% \subsection{八、不要给孩子戴手套}%ux516bux4e0dux8981ux7ed9ux5b69ux5b50ux6234ux624bux5957}}

% 2个月的孩子比较活泼爱动了,但还不能控制手脚定向运动,有时把自己的小脸蛋抓破。有的家长为了防止出现这种情况,就给孩子戴上小\textbf{手套}。其实这种作法是比较危险的,这是因为手套毛边的棉线,很容易绕到嫩小的手指上。手指越动,线勒得越紧,很快婴儿的手指会因为血液循环受阻,缺血而\textbf{坏死}。轻者可引起指端脱落致残,重者可引起骨髓炎、败血症等。

% %ux4e5dux9884ux9632ux5b69ux5b50ux7761ux504fux4e86ux5934}{%
% \subsection{九、预防孩子睡偏了头}%ux4e5dux9884ux9632ux5b69ux5b50ux7761ux504fux4e86ux5934}}

% 孩子出生后,头颅都是正常对称的,但由于婴幼儿时期骨质密度低,骨骼发育又快,所以在发育过程中极易受外界条件的影响。如果总把孩子的头侧向一边,受压一侧的枕骨就变得扁平,出现头颅不对称的现象。

% 1岁之内的婴儿,睡眠时间约占每天的一半,甚至2/3的时间,因此,要防孩子睡偏了头,首先是注意孩子睡眠时的头部位置,保持枕部两侧受力均匀。另外,孩子睡觉时容易习惯于面向母亲,在喂奶时也把头转向母亲一侧。为不影响孩子颅骨发育,母亲应该经常和孩子调换位置,这样,孩子就不会总是把头转向固定的一侧。

% 如果孩子已经睡偏了头,家长应用上述方法进行纠正。若孩子超过一岁半,骨骼发育的自我调整便很困难,偏头不易纠正,影响孩子的外观美。

% %ux5341ux8b66ux60d5ux5a74ux513fux7a92ux606f}{%
% \subsection{十、警惕婴儿窒息}%ux5341ux8b66ux60d5ux5a74ux513fux7a92ux606f}}

% 婴儿自己不能照顾自己,因而家长要特别注意婴儿是否呼吸通畅,防止窒息的发生。

% 不要给婴儿玩羽绒等软枕或软靠垫。

% 婴儿不会翻身时,不要俯卧睡眠。

% 婴儿枕不要太软,以防陷进去妨碍呼吸。

% 不要把硬币、豆类、小糖粒、纽扣等给小婴儿玩,以防误入呼吸道。

% 不要让婴儿玩塑料袋类,以防套在头上,遮住口鼻造成窒息。

% 不要让孩子含着糖块,以防误入呼吸道。

% %ux5341ux4e00ux5c3fux5e03ux7684ux4f7fux7528ux548cux66f4ux6362}{%
% \subsection{十一、尿布的使用和更换}%ux5341ux4e00ux5c3fux5e03ux7684ux4f7fux7528ux548cux66f4ux6362}}

% %ux5c3fux5e03ux7684ux51c6ux5907}{%
% \subsubsection{尿布的准备}%ux5c3fux5e03ux7684ux51c6ux5907}}

% \textbf{尿布的选择}

% 尿布是婴儿的必备用品,一般可分为纸尿裤和布尿布两种。这两种尿布各有特点,父母们可以根据自身的经济条件和使用习惯选择。

% \textbf{纸尿裤}:一次性纸尿裤的优点是不用洗涤、穿脱方便、大小便不易弄脏衣物、使用时间较长(3\textasciitilde4小时),它得到很多年轻父母的青睐。不过相对布尿布来说,纸尿裤价格比较贵。

% 但是,一次性纸尿裤若使用不当,会诱发一些疾病。

% \textbf{布尿布}:布尿布的优点是\textbf{透气性好},不易引起过敏,而且价格便宜,经济实惠。布尿布的缺点是准备和洗涤都比较麻烦。新生儿和婴儿一天排尿\textbf{10余次},排便也有好几次,而使用布尿布时,每次排尿后都要及时更换,并且清洗干净,因此给忙碌的母亲增加了许多工作量。

% \begin{longtable}[]{@{}
%   >{\raggedright\arraybackslash}p{(\columnwidth - 2\tabcolsep) * \real{0.5000}}
%   >{\raggedright\arraybackslash}p{(\columnwidth - 2\tabcolsep) * \real{0.5000}}@{}}
% \toprule()
% \begin{minipage}[b]{\linewidth}\raggedright
% 疾病名称
% \end{minipage} & \begin{minipage}[b]{\linewidth}\raggedright
% 引发原因及症状
% \end{minipage} \\
% \midrule()
% \endhead
% 尿布皮炎 &
% 如果婴儿解大便后或者多次排尿后未及时更换纸尿裤,粪便与尿液中的盐会刺激皮肤,引起尿布皮炎,也就是通常所说的红臀,甚至会发生皮肤溃疡。 \\
% 泌尿系统感染 &
% 女婴尿道较短,若排便后未及时更换纸尿裤并清洗阴部,很容易发生上行性泌尿系统感染。 \\
% 生殖能力受损 &
% 国外有个别研究认为,男婴长期使用纸尿裤会使阴嚢局部温度偏高,影响睾丸发育,导致成年后精子数量和质量受损。 \\
% 皮肤过敏 &
% 各种纸尿裤的制作材料因品牌不同而各有差异。除了极少数使用纯天然的棉质材料(价格极贵),大部分用的是非天然材质,较容易引起过敏。 \\
% \bottomrule()
% \end{longtable}

% \textbf{尿布的购买与制作}

% \textbf{纸尿裤}:在购买纸尿裤时,先查看生产日期。过期产品或出厂时间太长的产品容易被霉菌或细菌污染。然后,确定所购纸尿裤的规格,过大或过小均会增加尿液渗漏的机会。父母可根据婴儿的月龄或体重选择小、中、大、特大号的纸尿裤,有些品牌的纸尿裤还分为男孩用和女孩用两种,一般在外包装袋上均有详细的说明。有些纸尿裤还有\textbf{尿湿显示的功能},这样更利于父母及时了解婴儿的排便情况。购买纸尿裤时,最重要的一点是要选择吸水率高、透气性好、不会导致过敏的。

% \textbf{布尿布}:在新生儿出生前,父母应准备25\textasciitilde30条布尿布,便于换洗。市场有专门的布尿布出售,但比较少见,价格也比较贵,父母可以自己动手制作。选择柔软、吸水、耐洗的棉织品面料,以浅色的为宜(以便能及时发现大小便的异常)。将面料裁剪成长方形或三角形即可,根据婴儿的月龄决定布尿布的大小,也制成大、中、小3种规格。

% \textbf{尿布的洗涤}

% 纸尿裤用完即扔,不用洗涤,而布尿布需要及时洗涤。

% 洗涤布尿布时,如果布尿布上沾有粪便,要先用毛刷把粪便刷掉,然后用肥皂搓洗、漂清,\textbf{再用开水烫(最好能煮沸10分钟)},最后拧干后在阳光下晒干。

% 不要用洗衣粉洗尿布,因为婴儿皮肤非常娇嫩,如果尿布上残留了未洗净的洗衣粉,就会刺激婴儿皮肤,引起皮炎。

% 雨季或冬季尿布不易干时,可用取暖器慢慢烘干,或者用电熨斗熨干。但是,刚烘烤干的尿布不能马上用,要等凉透了再用,否则也容易发生尿布皮炎。

% 洗净晾干后的尿布应整齐地摆放好,不能随意乱扔,这样既方便取用,又可以防止布尿布被污染。

% \textbf{尿布的穿戴}

% \hspace{0pt}\includegraphics[width=5.83333in,height=4.43051in]{media/rId1126.png}\hspace{0pt}

% \begin{quote}
% ①撕开胶带。男婴解开的尿片这头,有时会有尿,要将弄脏了的尿片盖好。

% ②一只手抓住婴儿的两条腿抬起屁股,擦净赃污。男婴阴囊的皱褶之间,阴茎的内侧也要用水擦洗干净。女婴的阴部皱褶也要仔细用水擦洗,一定要从前往后擦拭。

% ③擦洗干净后,将脏了的尿片扯掉,换上新的尿片,看着胶带的刻度,左右对称地固定,要注意不要碰到肚脐,将尿片包好。

% ④检查一下松紧,如果能插进两根手指即为合适。整理好尿片不要让皱褶硌着婴儿的腿。把大腿部的尿片折起拉到外面,以免漏尿。

% ⑤将脏尿片包成团,用胶带固定后扔掉,如有大便要先冲洗掉再丢弃。

% 布尿片:

% ①将尿片放在尿片垫上。男婴因阴茎在前面,前端要叠两层;女婴拉尿因都积在后面,后面应叠两层。

% ②不要使尿片遮住肚脐,放在肚子上。如果能插进妈妈两根手指,就将尿片垫的带子固定,用手整理一下,不让背部和大腿边的尿片漏出尿片垫。

% 布尿片的折叠方法:先纵向折叠,再在一半处横向折叠,洗好后就应折叠成这样(见右图1)。
% \end{quote}

% \begin{enumerate}
% \def\labelenumi{\arabic{enumi}.}
% \item
%   布尿布:使用布尿布时,要注意尿布不能太大、太厚,否则会影响婴儿腿部的运动。将布尿布叠好,垫在大腿根,前端不要超过肚脐,不要将整个腰部都裹起来。男孩的尿布前面要厚点,而女孩的尿布则后面厚一点。可以在腰部用橡皮筋或\textbf{安全别针}将尿布固定,但橡皮筋不能过紧,否则会损伤婴儿的皮肤。布尿布的外面不要用塑料布包裹,否则婴儿的小屁股会因不透气而发红、糜烂,引起尿布皮炎。
% \item
%   纸尿裤:纸尿裤的穿法很简单,一般纸尿裤的包装上都有说明。将纸尿裤穿上后,粘好腰部的粘胶带就可以了。当然,别忘了将纸尿裤的边缘拉平整,这样既使孩子感到舒服,也可避免``泄露''事件。\\
%   用纸尿裤时,父母应该密切注意尿布的干湿情况,并及时更换。另外,夏季气温高的时候,不要一直让婴儿裹着纸尿裤,尤其是男婴,最好能\textbf{经常解开}尿裤让皮肤透透气,让生殖器降降温。
% \end{enumerate}

% \begin{quote}
% 专家指导

% \textbf{如何预防尿布皮炎?}

% 尿布皮炎大多数是臀部护理不当导致的。以下是防止尿布皮炎的要点。\\
% (1)
% 清洗尿布时一定要把肥皂或\textbf{洗涤剂冲洗干净},以防残留物刺激婴儿的皮肤。\\
% (2) 保持臀部干燥,清洗臀部后应涂上糠酸软膏或鱼肝油。\\
% (3) 掉在地上的尿布不能捡起来再用,因为尿布上面沾染了细菌。\\
% (4) 对于不用尿布的较大婴儿,一定要勤换内衣裤。内衣裤最好是全棉制品。
% \end{quote}

% \textbf{臀部的清洁}

% 婴儿的小屁股需要悉心的呵护,正确的臀部护理是防止尿布皮炎和泌尿道疾病的关键。

% 对于出生后3个月以内的婴儿,最好\textbf{每次换尿布时都用温水洗一洗}小屁股,或者用婴儿专用的湿纸巾将小屁股擦拭干净。记住,每次换上新尿布前,都要把小屁股\textbf{擦干},还可以涂些护臀\textbf{膏}。婴儿应该使用新的、柔软的毛巾,毛巾要独用,以免感染各种疾病。在擦拭肛门和外阴时,动作一定要轻柔,若用力过大很容易擦破。应注意的是,擦拭女婴的肛门和外阴时要从前往后擦,以免将肛门周围的细菌带到尿道口,引起上行性泌尿系统感染。

% \begin{quote}
% 育儿小百科

% \textbf{婴儿泌尿道感染与臀部护理方法有关吗?}

% 女婴由于尿道短而宽,且尿道口与肛门、阴道距离近,所以肛门部位的细菌容易侵入泌尿道和阴道。因此,女婴的外阴护理要格外注意。当女婴大便后擦拭肛门时,如果从后面肛门处往前面阴道口擦,就会使大便中的细菌沾在外阴部,从而使细菌由尿道进入膀胱。因此,清洁女婴外阴部和肛门部时,应由前向后擦拭。除了大小便后要及时清洁小屁股外,每天睡前也要清洗外阴及肛门口。
% \end{quote}

% %ux5341ux4e8cux57f9ux517bux5a74ux513fux826fux597dux7684ux7761ux7720ux4e60ux60ef}{%
% \subsection{十二、培养婴儿良好的睡眠习惯}%ux5341ux4e8cux57f9ux517bux5a74ux513fux826fux597dux7684ux7761ux7720ux4e60ux60ef}}

% 从新生儿期就要注意培养孩子良好的睡眠习惯。良好的睡眠习惯首先是\textbf{按时睡觉,自然入睡}。

% 有的妈妈对孩子``爱不释手'',孩子吃饱后还要把他抱在怀里,摇晃着、拍着,或是让孩子叼着乳头、空奶嘴,这都不是好习惯。妈妈一定注意在孩子睡前不哄、不拍、不抱、不摇,更不要吃东西、叼奶头。\textbf{到该睡的时候},把孩子放到床上让他自己睡。小婴儿还没有养成\textbf{按时}睡的习惯,可给他放些轻柔的催眠曲,使孩子建立起睡眠的条件反射。等到孩子养成按时入睡的习惯,就不必放音乐了。

% %ux5341ux4e09ux57f9ux517bux826fux597dux7684ux6392ux4fbfux4e60ux60ef}{%
% \subsection{十三、培养良好的排便习惯}%ux5341ux4e09ux57f9ux517bux826fux597dux7684ux6392ux4fbfux4e60ux60ef}}

% 从孩子两个月起就应该训练良好的排便习惯,使他按时排便,排便最好在清晨晚上临睡前,早晨排便最好,睡前大便可使孩子夜里睡得踏实。饭前大便可使孩子吃得好,但不要饭后大便。妈妈先观察孩子排便的情况,然后根据孩子的情况,有意识定时排便。【\textbf{这怎么训练...}】

% %ux5341ux56dbux907fux514dux611fux67d3}{%
% \subsection{十四、避免感染}%ux5341ux56dbux907fux514dux611fux67d3}}

% 避免与外人的``亲密接触''。家里添了宝宝,亲朋好友都会来探望。他们会用抚摩、亲嘴等方式表示喜爱之情,这便增加了新生儿的感染机会。因此,父母应委婉地拒绝这些``亲热''行为,尤其是患病亲友的``亲热''行为。最好不办满月酒。

% %ux5341ux4e94ux9632ux6b62ux610fux5916ux4e8bux6545}{%
% \subsection{十五、防止意外事故}%ux5341ux4e94ux9632ux6b62ux610fux5916ux4e8bux6545}}

% \begin{enumerate}
% \def\labelenumi{\arabic{enumi}.}
% \item
%   窒息。宝宝不要睡太软的床,不要用大而软的枕头。最好不要与妈妈同床同被睡眠,以防堵住口鼻。宝宝的小床上也不要堆放衣物、玩具,挂玩具的绳索和窗帘绳也不能靠近小床,以免套住宝宝的颈部。【蚊帐外...】
% \item
%   烫伤。喂牛奶时要先将冲调好的牛奶滴于妈妈\textbf{手腕内侧}试温度。用热水袋保暖时,水温宜在50度左右。要拧紧塞子并用毛巾包好放在垫被下面,距宝宝皮肤10厘米左右。
% \item
%   丝线缠绕指(趾)端。每天都要\textbf{检查宝宝的手指、脚趾}是否被袜子、手套或被子上的丝线缠绕,以免血流不通、组织坏死。
% \item
%   动物咬伤。养猫、狗等小动物的家庭,应将小动物移到别处。平时要关紧门窗,以防小动物钻进室内伤害宝宝。
% \item
%   溺水。给宝宝洗澡时,不能暂时丢下宝宝去接电话、开门等。如果必须去,一定要把宝宝用浴巾包好抱在怀里,以防意外。【不离】
% \item
%   煤气中毒。冬天室内生火炉一定要安装通气管道。
% \end{enumerate}

% %ux5341ux516dux7406ux89e3ux5a74ux513fux54edux7684ux539fux56e0}{%
% \subsection{十六、理解婴儿哭的原因}%ux5341ux516dux7406ux89e3ux5a74ux513fux54edux7684ux539fux56e0}}

% 婴儿通过安静的行为表示爱,通过哭来表示害怕、不适、痛苦。

% \begin{enumerate}
% \def\labelenumi{\arabic{enumi}.}
% \item
%   孤独。大人离开,剩他一人时会哭。
% \item
%   饥饿。饿了的婴儿会不停地哭,喂奶能使他安静下来。
% \item
%   烦躁。孩子在烦躁时遇到别人引逗会不耐烦地哭。
% \item
%   疼痛。疼痛及肠胃不适会引起孩子啼哭。
% \item
%   温度。室温过低婴儿会哭,并且不能睡眠。
% \item
%   裸体。婴儿不喜欢脱光衣服,穿上衣服会停止啼哭。
% \item
%   尿布湿。\textbf{尿布湿了}不舒服他会哭。
% \item
%   睡眠被打扰。孩子被惊醒会啼哭。
% \item
%   中断喂奶。孩子没吃饱会大哭。
% \item
%   加辅食。孩子第一次加某种辅食拒吃时会哭。
% \end{enumerate}

% %ux5341ux4e03ux5e38ux89c1ux7761ux7720ux95eeux9898}{%
% \subsection{十七、常见睡眠问题}%ux5341ux4e03ux5e38ux89c1ux7761ux7720ux95eeux9898}}

% %ux65e5ux591cux98a0ux5012}{%
% \subsubsection{1.日夜颠倒}%ux65e5ux591cux98a0ux5012}}

% 这种情况多发生在出生后6个月内的宝宝。由于大脑皮质功能发育不完善,正常的生活规律尚未建立,宝宝对黑夜和白天没有时间概念,所以白天大部分时间在睡眠,而晚上清醒的时间较多,甚至在夜间啼哭不止。这种现象可能持续到8\textasciitilde9个月。

% 如果宝宝出现日夜颠倒的现象,父母可以采取以下措施:

% ① 在临睡前换上干爽的尿布,让宝宝吃饱后入睡。

% ② \textbf{晚上应避免逗引宝宝},不要让他过度兴奋。

% ③
% 宝宝半夜醒来时,不要马上把他抱起来哄,这样会彻底弄醒宝宝,而应\textbf{轻轻拍拍他,让他迷迷糊糊地继续睡觉}。【木妈妈做到了】

% ④
% 减少白天的睡眠时间。宝宝白天睡得多,夜里就会精神十足。因此,\textbf{白天应多逗宝宝},减少宝宝的睡眠时间。

% %ux542bux7740ux5976ux5634ux5165ux7761}{%
% \subsubsection{2.
% 含着奶嘴入睡}%ux542bux7740ux5976ux5634ux5165ux7761}}

% \hspace{0pt}\includegraphics[width=2.54545in,height=2.43357in]{media/rId1155.png}\hspace{0pt}

% \textbf{含着奶嘴入睡}看似没有什么特别的不良后果,如果宝宝长牙了而这坏习惯还没有改掉的话,就会导致口腔疾病。

% 含着奶嘴入睡的坏习惯常常是父母``培养''出来的。当宝宝吃着吃着就睡着了时,父母往往不忍心因拔出乳头而惊醒他,久而久之,宝宝就养成了这个习惯,往往含着乳头似睡非睡、似醒非醒地吃几口。因此,当宝宝入睡后,就应\textbf{及时将乳头拔出}。

% %ux8fb9ux62cdux8fb9ux7761ux8fb9ux6447ux8fb9ux7761}{%
% \subsubsection{边拍边睡、边摇边睡}%ux8fb9ux62cdux8fb9ux7761ux8fb9ux6447ux8fb9ux7761}}

% 边拍边睡、边摇边睡,这个习惯是父母\textbf{惯}出来的。家里添了小宝贝,父母往往爱不释手,家人,尤其是祖父母喜欢抱着宝宝、拍着宝宝或摇着宝宝睡觉。当宝宝习惯了边拍边睡或边摇边睡后,如果夜间醒来(这是正常睡眠中常见的现象)时没人拍或摇,他就会经常在夜间啼哭。因此,如果没有足够的耐心这样拍或摇宝宝两三年,并且能坚持在夜间也起来四五次,最好还是不要去``培养''这个习惯。

% %ux5341ux516bux5b9dux5b9dux4f55ux65f6ux7528ux6795ux5934}{%
% \subsection{十八、宝宝何时用枕头}%ux5341ux516bux5b9dux5b9dux4f55ux65f6ux7528ux6795ux5934}}

% 3个月内的宝宝可以不用枕头。3个月后宝宝开始会抬头,趴着时能用双手支撑起上半身,脊柱颈段出现向前的生理弯曲,此时就需要\textbf{用枕头来维持生理弯曲},保持体位舒适。

% 宝宝枕头的高度以\textbf{3〜4厘米为宜},并要根据发育状况调整枕头的高度。如果枕头太高,时间一久就会造成颈椎后凸畸形。枕头的长度应与宝宝的\textbf{肩部同宽},或略长于宝宝的肩宽。\textbf{枕心应柔软、轻便、透气、吸湿性好},可用柔软的木棉、蒲绒等作为填充材料,也可用晒干后的菊花充填。有的父母认为宝宝睡硬一些的枕头可以使头骨长得更结实,脑袋的外形长得更好看,其实这是不对的。质地过硬的枕头易使颈部肌肉疲劳,颈部软组织受损。另外,宝宝的颅骨较软,囟门和颅骨缝还未完全闭合,长期使用硬枕头易造成头颅变形,甚至两侧脸部不对称而影响美观。

% 宝宝新陈代谢旺盛,头部出汗较多,睡觉时容易浸湿枕头,汗液和头皮屑混合容易使致病的微生物黏附在枕面上。因此,宝宝的枕心要\textbf{经常在阳光下暴晒},枕套应选择\textbf{半新的棉制品},且要\textbf{常洗常换},保持清洁。

% \begin{quote}
% 育儿小百科\\
% \textbf{如何应对宝宝踢被子}

% 大多数宝宝睡觉时都会踢被子。宝宝睡眠时踢被子可能是白天玩得太累了,或临睡前玩了剧烈的游戏,入睡后大脑皮层的\textbf{兴奋}状态还未消失造成的。更主要的原因是宝宝睡觉时盖得过暖、裹得太紧,因\textbf{过热}而踢被子。睡着后皮肤表面的毛孔是张开着的,而宝宝的免疫能力又比成人差,因此,踢被子容易导致受凉。

% 因此,要想宝宝睡觉时不踢被子,睡觉时就不能给他盖得过厚、过重,也不能穿得过多。以下是应对宝宝踢被子的方法:
% \end{quote}

% \begin{enumerate}
% \def\labelenumi{\arabic{enumi}.}
% \item
%   \begin{quote}
%   选择轻软而保暖性好的棉质床垫,根据季节变换被子的厚薄。
%   \end{quote}
% \item
%   \begin{quote}
%   冬天时,临睡前可用热水袋把被窝焐热,然后拿开,这样可以使宝宝感到暖和又不过分热。
%   \end{quote}
% \item
%   \begin{quote}
%   睡觉时给宝宝穿宽大柔软的单衣裤。
%   \end{quote}
% \item
%   \begin{quote}
%   让宝宝一个人盖一条被子,盖被子时注意盖好两肩和两脚就可以了。天气不是十分寒冷时,可以将手脚露在被子外。如果裹得太紧,宝宝动弹不得,反而容易踢被子。冬天可让宝宝睡睡袋,既方便,又不会着凉。
%   \end{quote}
% \item
%   \begin{quote}
%   睡觉前应使宝宝安静下来,不要玩得太累。
%   \end{quote}
% \item
%   \begin{quote}
%   室内空气要流通,最好能开窗通风。
%   \end{quote}
% \end{enumerate}

% \begin{quote}
% 当宝宝睡得舒适时,他就不会踢被子了。
% \end{quote}

% %ux5341ux4e5dux65b0ux751fux5b9dux5b9dux7684ux4fddux6696ux8981ux6c42}{%
% \subsection{十九、新生宝宝的保暖要求}%ux5341ux4e5dux65b0ux751fux5b9dux5b9dux7684ux4fddux6696ux8981ux6c42}}

% 人类是一种恒温动物,人体的温度只能在很小的范围内波动。对于新生宝宝来说,体温调控特别重要,人体内的酶只有在很小的温度范围内才起作用,而这些酶又决定了细胞和器官的功能。若婴幼儿的体温低于适宜温度太多,一些肌体的功能就会下降到危险水平;若体温过高(发烧),就会影响体内酶活性,呼吸及新陈代谢因此加快,血液内的物质平衡会因而遭到破坏。所以说,宝宝必须要维持相对恒定的体温。

% %ux5b9dux5b9dux662fux600eux6837ux7ef4ux6301ux4f53ux6e29ux7684}{%
% \subsubsection{1.
% 宝宝是怎样维持体温的}%ux5b9dux5b9dux662fux600eux6837ux7ef4ux6301ux4f53ux6e29ux7684}}

% 先来了解成人是怎样应对温度变化的。当成人处于较热的环境中时,代谢下降,血管扩张,大量的血液流经体表,热量散发到空气中;通过出汗,蒸发散热;通过呼吸释放热量。当处于低温环境中时,体表的血管关闭以减少热量的散失,并可通过运动或寒战使肌体产生热量,还可以增加衣服达到保温的作用。

% 与成人相比,刚刚出生的婴儿在维持体温方面存在很大的问题。相对于体重而言,新生儿的体表面积很大,若直接暴露于冷空气中,损失的热量很大。而且,新生儿比成人防止热量散失的脂肪层要少得多,在寒冷环境中处境极为不利。这些因素加起来导致新生儿丢失热量的速度是成人的\textbf{4倍}。所以,新生宝宝会在出生后迅速开发出应对低温的机制,即在出生后15分钟,就能够收缩体表血管并增加热量的产生来应对低温。3小时后,就婴儿的体重而言,其对抗寒冷的代谢水平已经接近成人,但相对于其体表面积,情况就不那么乐观了。可见问题不是新生宝宝缺乏体温调节的功能,而是这项调节任务对于宝宝来说过于艰巨了。

% 新生宝宝的体温调控能力在其降生于医院的环境后,将迅速经受考验。子宫内的温度通常恒定在36.5℃左右,而迎接新生儿的充满空气的环境却要比子宫内的温度低得多,很少高于27℃,有时室温只有15℃。新生宝宝生下来身体是湿湿的,有时还要先洗个澡,通过蒸发和皮肤暴露于空气中,相当多的热量又会因此而损失掉。由于温度的陡降,宝宝产生的热量要是成人的两倍才能够应对这种情况。若是成人从温暖的浴室走到更衣室,身上挂着水珠,会感觉到特别的冷,这是因为蒸发散失的热量特别的多,即使把身子擦干,也会有发抖的倾向,穿上一两件衣服后就不那么冷了,因为衣服阻止了身体向空气中大量散热。因而不难理解新生儿面临的冲击。许多医院试图将这种冲击减少到最小,但子宫中的环境仍旧比产房要温暖得多,新生宝宝必须能够调节自己的体温以应对这种挑战。像一些早产儿,他们太小了,无法应付外界的环境,只能先呆在温箱中,等待成熟一些再出来。

% 宝宝除了能够动用生理自动调节方式外,还有两种主要的方式可以用来调节体温。一种方式是婴儿\textbf{紧贴}在看护者身上,这样成人的热量会传递给婴儿,这就是婴儿所眷恋的温暖的怀抱,与此同时,安静地躺着,可以减少热量的散失。另一种方式是当温度下降时,婴儿会\textbf{蜷起}身子,这样能减少体表面积,降低热量的散失。

% %ux600eux6837ux4e3aux65b0ux751fux7684ux5c0fux5b9dux5b9dux4fddux6696}{%
% \subsubsection{2.怎样为新生的小宝宝保暖}%ux600eux6837ux4e3aux65b0ux751fux7684ux5c0fux5b9dux5b9dux4fddux6696}}

% 对于婴儿的照顾以及需求的及时满足,不仅可以帮助宝宝抵御生理上的寒冷,也可以帮助宝宝抵御心理上的寒冷。宝宝知道只要他需要帮助,就会有人保护他,使他感到安全。在这个过程中,宝宝可以细细品味来自成人的爱和关注,会感觉到这个世界并不那么寒冷。因此,照料者应该做到:

% \begin{enumerate}
% \def\labelenumi{\arabic{enumi}.}
% \item
%   室温要保持适度,通常在22\textasciitilde26℃为宜。如果冬天无法维持这样的温度,要用其他方式为宝宝保暖。
% \item
%   给宝宝穿适当的衣物,以宝宝感觉舒适为好。衣服应该以棉质为主,这样的衣服吸汗又不刺激皮肤。给宝宝穿衣时注意松紧适度,衣袖不要过长,让宝宝\textbf{手露在外面},手是宝宝重要的感知外界的器官之一,就如同眼睛、耳朵和鼻子一样。
% \item
%   关于宝宝盖被子的问题,老一辈的人习惯是把宝宝裹在小被里,而不是盖上被子,婴儿的手脚都被绑在小被里不能移动,这样做的好处是宝宝不会将被子踢掉,不会因此着凉,父母也不必花更多的心思关注宝宝这方面的需求。但缺点是宝宝被限制了活动的空间和活动的自由,婴儿用手和脚探索外部世界的权利和机会都被剥夺了。因此,父母应该\textbf{尽量给宝宝盖被子而不是裹住宝宝},这样虽然会累一点,但对宝宝多投入些心力,会有利于宝宝的成长。
% \item
%   给宝宝更换尿布时要\textbf{及时迅速}。宝宝尿后,臀部沾满尿液,更换尿布时,臀部暴露在外面,尿液的蒸发会带走大量的热,且皮肤也会迅速向空气中散热。所以必须迅速给宝宝擦干,更换好干爽的尿布,可避免宝宝受更多寒冷环境的刺激。
% \item
%   给婴儿洗澡时要注意水温。成人的皮肤特别是手的温度的适应范围很大,成人不觉得很冷或很热的水,对宝宝来说可能已经无法适应了,因为宝宝的皮肤是非常娇嫩和敏感的。因此,试水温时用\textbf{腕部}等部位而不是用手试为好,不然,冷热的过度刺激会使宝宝惧怕洗澡。当然,最好还是准备一个测水温的\textbf{温度计}。宝宝洗完澡后,要迅速把宝宝擦干包好,宝宝皮肤的散热量过大,宝宝会感觉特别冷。
% \end{enumerate}

% 保温也要注意不能做得过分,若婴儿被捂得太热,会大量流汗,水分丧失过快,也会生病。若夏天太热,要注意保持宝宝干爽,常洗澡,以免因高温和汗水刺激起痱子或中暑。

% %ux4e8cux5341ux8bfbux61c2ux5a74ux513fux7684ux8868ux60c5ux548cux52a8ux4f5c}{%
% \subsection{二十、读懂婴儿的表情和动作}%ux4e8cux5341ux8bfbux61c2ux5a74ux513fux7684ux8868ux60c5ux548cux52a8ux4f5c}}

% 婴儿会通过自己特有的表情、情绪及动作向妈妈发出信息,只要妈妈悉心照料和观察,就会读懂婴儿的心声。

% \textbf{太困了}

% 新生宝宝双眼紧闭,呼吸平稳,规则,身体没有任何动作;或者双眼紧闭,但呼吸较快并有变化,偶尔微睁一下眼,或者脸上出现表情如笑、皱眉,有时手和脚活动一下,可实际上却睡得挺沉。这时不要叫醒宝宝吃奶、换尿布,不然宝莹会烦躁。

% \textbf{已经睡得差不多了}

% 外表看宝宝还在睡觉,眼睛似睁非睁但目光不灵活,此时叫醒宝宝他不会大闹,因此可以给宝宝喂水、喂奶或换尿布。

% \textbf{状态棒极了}

% 宝宝目光灵活,而且十分有神,对于外界的反应很专注。这时是给宝宝哺喂,对他讲话及做其他交流的最佳时机。

% \hspace{0pt}\includegraphics[width=1.94406in,height=2.36364in]{media/rId1176.png}\hspace{0pt}

% \textbf{快来哄哄我}

% 宝宝大声哭叫,情绪很不安,大概是对周围的刺激产生否定情绪,这时需要安抚他,即将宝宝换一舒适体位,一只手放在肚子上,另一只手轻\textbf{轻拍}打他,并且对他轻\textbf{柔}地讲话,这时动作要\textbf{慢且有节律},不要总变化,这样就会使宝宝心理获得安定和满足感。

% \textbf{快给我吃奶,我饿了}

% 宝宝的小脸转向妈妈,嘴里做着吸吮动作,小手抓住不放,用手指一碰他的嘴角便做出急忙地寻找食物的样子。

% \textbf{别叫我,我还不太饿}

% 若到了喂奶时间但宝宝还在睡觉,这表示他并不饿,不用去叫醒他。

% \textbf{已经吃饱了}

% 若宝宝把奶头或奶瓶推开,或者给他喂奶时头转开,并且一副四肢松弛的样子,表明宝宝已经吃饱了,不要再多给他吃东西了。

% \textbf{来和我玩玩吧}

% 宝宝面带微笑看着你,脸部表情和肢体动作活跃,适宜做视觉及听觉刺激。

% \textbf{我想休息了}

% 宝宝眼光不那么机灵了,打哈欠,头转到一边不太理睬妈妈,这时不要进行任何活动了,给宝宝换好尿布,放在舒适安静的地方,让宝宝好好睡觉,这样对宝宝的生长很有利。

% %ux4e8cux5341ux4e00ux54c4ux5b9dux5b9dux7761ux89c9ux7684ux9519ux8befux505aux6cd5}{%
% \subsection{二十一、哄宝宝睡觉的错误做法}%ux4e8cux5341ux4e00ux54c4ux5b9dux5b9dux7761ux89c9ux7684ux9519ux8befux505aux6cd5}}

% %ux6447ux7761}{%
% \subsubsection{1. 摇睡}%ux6447ux7761}}

% 婴幼儿哭闹或睡眠不安时,许多妈妈便将宝宝抱在怀中或放入摇篮里摇晃,宝宝越哭越凶,妈妈摇晃得也就越猛烈,直到宝宝入睡为止。但这种做法对宝宝十分有害,因为摇晃动作使婴儿的大脑在颅骨腔内不断晃荡,\textbf{未发育成熟的大脑}会与较硬的颅骨相撞,造成脑小血管破裂,引起``脑轻微震伤综合征'',发生脑震荡,颅内出血。轻者发生癫痫病,智力低下,肢体瘫痪,严重者出现脑水肿、脑疝而死亡。若眼睛里的视网膜受到影响,还可导致弱视或失明,由此铸成大错,特别是对于10个月内的小宝宝更危险。

% %ux966aux7761}{%
% \subsubsection{2. 陪睡}%ux966aux7761}}

% 从宝宝一出生,妈妈就应积极鼓励宝宝\textbf{独自入睡},并养成习惯,即使是新生儿,也不应与妈妈同睡一个被窝。妈妈熟睡后稍不注意就可能压在小宝宝身上,造成窒息死亡,美国一项调查资料证实了这一点,即让婴儿单独睡觉可降低60\%的突然死亡率,如果妈妈长期陪睡,宝宝会出现一种``恋母''心理,到了幼儿园甚至上小学的年龄,与妈妈分离还会很困难,宝宝日后容易患学校恐惧症、考试紧张症,对宝宝的身心发展不利。

% 培养宝宝独睡习惯通常从\textbf{1岁开始},此年龄段的宝宝入睡较快,并已有了一定程度的自主意识。利用这些特点鼓励宝宝独睡,宝宝较易接受。

% %ux4fefux7761}{%
% \subsubsection{3.俯睡}%ux4fefux7761}}

% 国外专家研究发现婴儿猝死综合征与睡眠姿势有关,尤其是颜面朝下的俯睡最具危险性。小婴儿通常不会自己翻身,并且不能主动避开口鼻前的障碍物,因而呼吸道在受阻时,只能吸收到很少的空气而缺氧,加上消化器官发育不完善,当胃蠕动、胃内压增高时,食物就会反流,阻塞本已十分狭窄的呼吸道,造成婴儿猝死。

% 宝宝最安全的睡姿是仰睡,此种睡姿可使其呼吸道畅通无阻,一定程度上避免了婴儿猝死。据统计,在美国自从推广了仰睡法后,婴儿猝死综合征的发生率随之大幅度下降,从每年约死亡5
% 000人下降到不足3 000人,值得妈妈们借鉴。

% %ux6402ux7761}{%
% \subsubsection{4.搂睡}%ux6402ux7761}}

% 许多妈妈担心宝宝在睡眠中发生意外,常常搂着睡觉,这样做恰恰增加了发生意外的机会:

% \begin{enumerate}
% \def\labelenumi{\arabic{enumi}.}
% \item
%   搂睡使宝宝难以吸收新鲜空气,吸入的多是被子里的污秽空气,容易生病;
% \item
%   容易使宝宝养成醒来就吃奶的坏习惯,不易形成定时喂养,因此妨碍宝宝的食欲与消化功能;
% \item
%   限制了宝宝睡眠时的自由活动,难以舒展身体,影响正常的血液循环,若妈妈睡得过熟,不小心奶头堵塞了宝宝的鼻孔,还可能造成窒息等严重后果。
% \end{enumerate}

% %ux8499ux7761}{%
% \subsubsection{5.蒙睡}%ux8499ux7761}}

% 通常在冬春气温较低的季节,妈妈为让宝宝暖和,常将宝宝头部蒙在棉被下,这样做有两大危害:

% \begin{enumerate}
% \def\labelenumi{\arabic{enumi}.}
% \item
%   被窝温度较高,加上宝宝代谢旺盛,容易诱发``闷热综合征'',可致宝宝大汗淋漓,甚至发生虚脱。
% \item
%   可能引起呼吸困难,或者窒息。宝宝睡觉时都应将头面部露在被子外面,以防发生不测。
% \end{enumerate}

% %ux70edux7761}{%
% \subsubsection{6.热睡}%ux70edux7761}}

% 为给宝宝保暖,很多家庭购买了电热毯。然而,电热毯加热速度较快,温度也较高,会增加宝宝不显性失水量,引起轻度脱水而影响健康。

% \textbf{宝宝不宜使用电热毯}。如果要用须正确掌握方法,即睡前通电预热,待宝宝上床后及时切断电源,切忌通宵不断电。使用过程中,若宝宝出现了哭声嘶哑、烦躁不安等表现,说明身体可能脱水,马上给宝宝多喝些白开水,一般就会平静下来,很快恢复正常。【大人都会脱水】

% %ux4eaeux7761}{%
% \subsubsection{7. 亮睡}%ux4eaeux7761}}

% 许多妈妈为了方便夜间喂奶、换尿布,常将卧室里的灯通宵开着,这对宝宝有不利影响。

% 医学研究表明,宝宝在通宵开灯的环境中睡眠,可导致睡眠不良,睡眠时间缩短,进而减慢发育速度。婴儿的神经系统尚处于发育阶段,适应环境变化的调节机能差,卧室内通夜亮着灯,势必改变了人体适应的昼明夜暗的自然规律,从而影响宝宝正常的新陈代谢,危害生长发育。以视力发育为例,据英国学者报告:睡觉时居室内开着小灯的宝宝有30\%成了近视眼,而面对灯火通明的宝宝近视眼的发生率则高达55\%。【木妈妈这个改的好】

% %ux88f8ux7761}{%
% \subsubsection{8. 裸睡}%ux88f8ux7761}}

% 夏天气温高,许多妈妈便将宝宝衣裤脱光,让宝宝光着小身子躺在床上,以求凉爽。但小宝宝体温调节功能差,容易使身体受凉,尤其是腹部一旦受凉,可使肠蠕动增强,导致腹泻发生。即使在炎热的夏天也不可以裸睡,胸腹部最好盖一层薄薄的衣被或带上小肚兜。

% \hspace{0pt}\includegraphics[width=2.85315in,height=3.03497in]{media/rId1192.png}\hspace{0pt}

% %ux7b2cux4e09ux82822-3ux4e2aux6708ux5a74ux513fux7684ux65e5ux5e38ux62a4ux7406}{%
% \subsection{03第三节2 \textasciitilde{}
% 3个月婴儿的日常护理}%ux7b2cux4e09ux82822-3ux4e2aux6708ux5a74ux513fux7684ux65e5ux5e38ux62a4ux7406}}

% %ux4e00ux5c45ux5ba4ux7684ux9009ux62e9}{%
% \subsection{一、居室的选择}%ux4e00ux5c45ux5ba4ux7684ux9009ux62e9}}

% \hspace{0pt}\includegraphics[width=3.42657in,height=1.98601in]{media/rId1198.png}\hspace{0pt}

% 经过漫长的40周子宫内生活,孩子来到了这个陌生的世界上,周围的一切都需要重新适应。孩子对卧室的环境要求是非常高的,以下条件一样也少不了。

% \begin{enumerate}
% \def\labelenumi{\arabic{enumi}.}
% \item
%   新鲜的空气:孩子的卧室最好选在绿化环境较好、远离马路和工厂的地方。不良空气中的粉尘、灰烬不仅会削弱孩子呼吸道的抵抗力,而且还会影响他们的生长发育。
% \item
%   适宜的温度与湿度:足月孩子居室的室温应为22\textasciitilde24℃,相对湿度为60\%-65\%,早产儿居室的室温为24-27℃,相对湿度为65\%以上。无论是足月儿还是早产儿,室内的温度和湿度都要保持相对的恒定,忽冷忽热、忽干忽湿的空气往往会导致疾病的发生。
% \item
%   充足的阳光:最好选择\textbf{朝南或阳光充足}的房间作婴儿的卧室。在无风的时候打开窗户,让温暖的阳光直射进来。阳光中的紫外线不仅有消毒作用,还可以促进孩子体内维生素D的合成,能预防维生素D缺乏性佝倭病。【有时木木喜欢在二姐家客厅是不是这个原因?】
% \item
%   通风良好:无论冬夏,每天都应该开几次窗户,以保持卧室内空气清新。
% \item
%   环境安静:孩子每天需要的睡眠时间很长,良好的睡眠有利于他们健康地成长。而安静的环境是良好睡眠的保障,如果周围声音过于嘈杂,必定会影响孩子的休息,孩子会因睡眠不佳而哭闹不止。
% \item
%   适当的光线:卧室的光线不宜过强,强烈的光线会对孩子的眼睛造成强烈刺激,从而影响其视觉发育。如日间阳光直接照射在小床上时,可拉上窗帘;同样,电灯也不宜装在小床上方。但室内光线也不能一直太暗,若整天拉上厚厚的窗帘将不利于孩子观察周围的事物。
% \end{enumerate}

% %ux4e8cux5c45ux5ba4ux7684ux5e03ux7f6e}{%
% \subsection{二、居室的布置}%ux4e8cux5c45ux5ba4ux7684ux5e03ux7f6e}}

% 孩子居室的布置可按父母的条件和喜好而定,但有几项原则是必须要注意的,即\textbf{安全、舒适、有利健康}。

% %ux5c0fux5e8a}{%
% \subsubsection{1. 小床}%ux5c0fux5e8a}}

% 床是孩子的小天地,也是他们的安全地带。在买婴儿床时最好选择\textbf{木}制的、有栏\textbf{杆}的床,当孩子以后会翻身、爬行站立时,栏杆可以起到很好的防护作用。栏杆的高度以\textbf{超出床面50厘米}为宜,栏杆过低容易使孩子站立时翻下床,过高则会让母亲抱起或放下孩子时感到吃力。栏杆之间的距离不宜过大或过小,一般为11厘米左右,过大容易夹住孩子的头部,过小容易夹住孩子的手脚。小床的高度可适当\textbf{低}一些,床底下还可以\textbf{铺}上一块毯子或垫子,这样万一孩子跌落下来时可起到缓冲作用。小床的位置\textbf{不要紧贴着门窗,因为那里温差较大,噪声干扰也较多}。小床应紧贴着墙,或索性离墙远点,否则万一孩子从床上跌下后会被夹在床和墙壁之间而造成更大伤害。床褥不能过于松软,否则会使孩子脊柱旁的韧带和关节负担过重,从而引起脊柱后突或侧突畸形。有些小床的设计比较合理,即床的高度、长度、栏杆可随孩子的年龄变化自动\textbf{调节},这样不仅安全,而且也经济实惠,比较受父母欢迎。

% 孩子的居室可挂些色彩艳丽的画片、玩具,以刺激其视觉的发育,但不要挂得离眼睛太近,玩具的位置也要经常\textbf{变换},否则孩子眼睛老盯住一个地方,容易发生斜视。

% %ux5e03ux7f6e}{%
% \subsubsection{2. 布置}%ux5e03ux7f6e}}

% 有些父母喜欢把墙壁或小床涂成各种颜色,殊不知很多色彩艳丽的涂料中含有大量\textbf{铅},如果孩子经常用嘴啃咬床栏杆或用手抚摸墙壁,则会不知不觉地把铅吃到肚子里,从而造成铅中毒。

% \hspace{0pt}\includegraphics[width=2.54545in,height=2.04196in]{media/rId1209.png}\hspace{0pt}

% 孩子的房间里尽量\textbf{少放食物},因为食物会招引老鼠、蟑螂等进入房间,会给孩子的健康带来危害。另外,猫、狗等宠物也最好不要进入孩子的房间,因为它们身上常携带有寄生虫及其他病原体。

% 孩子对生活环境的要求很高,室温过高或过低都容易导致孩子生病。为了使孩子能生活在四季如春的房间里,父母们会动用空调、取暖器、电风扇等电器设备。不过,这些电器一定要正确使用,否则它们将成为各种疾病的诱因。

% %ux7a7aux8c03}{%
% \subsubsection{3. 空调}%ux7a7aux8c03}}

% 长期生活在空调房间中,很容易得``空调综合征''。这是因为许多空调器里吹出来的风是房间中原来的空气,由于空调系统中各部位对负离子有吸收作用,因此换气次数越多,空气中\textbf{负离子}越少,达到一定程度后就会使人感到胸闷、心跳加快、头晕、头痛、乏力、食欲不振。另外,随着室内二氧化碳的积聚和病菌繁殖,空气污染会进一步加重。

% 因此,使用空调期间,要定期清洗空调器的过滤膜,可在室内放置一个\textbf{人工负离子发生器},以维持室内正负离子的平衡。空调的送风口不要直接对着孩子,每天要有一定的时间关闭空调,开窗通气。冬季开暖气时可在房间里放一盆水,适当提高空气湿度,以防止干燥的空气刺激呼吸道黏膜。【木妈有常通风,赞】

% %ux7535ux98ceux6247}{%
% \subsubsection{4. 电风扇}%ux7535ux98ceux6247}}

% 使用电风扇时,风扇可放在离孩子头部2米左右的地方,使用时间\textbf{不宜过长},风速\textbf{不宜过快},最好使电风扇摇头\textbf{旋转}。不能让孩子赤身躺着吹风扇,因为这会使体表汗液排泄失调。由于婴幼儿的自主神经调节功能尚未发育成熟,若汗液排泄不均衡,就很容易生病,轻者可出现鼻塞、流涕等上呼吸道感染症状,重者则会发生关节炎。

% %ux53d6ux6696ux5668}{%
% \subsubsection{5. 取暖器}%ux53d6ux6696ux5668}}

% 有些家庭会使用红外线取暖器或暖风取暖器,如果孩子较长时间近距离使用取暖器,很容易发生脱水或皮肤灼伤。因此,取暖器不能直接照着孩子的皮肤,一旦发现孩子皮肤发红或异常哭闹,应该立即将取暖器移远。使用取暖器时,房间里可放盆水,并经常给孩子喂些水或果汁,以补充体内水分。与使用空调时相同,使用取暖器时,每天应通风几次,以保证室内空气新鲜。

% %ux55d3ux58f0ux4e0eux566aux5149}{%
% \subsubsection{6. 嗓声与噪光}%ux55d3ux58f0ux4e0eux566aux5149}}

% 噪声和噪光对人体的健康有危害,对尚在生长发育中的孩子,其危害更大。

% 家庭中常见的噪声有电视机声、录音机声等,噪光有闪动的霓虹灯、电视广告中五颜六色、变化迅速的色彩和画面。长时间的噪声和噪光对孩子听力和视力有严重的影响,可以造成听力及视力的下降。研究表明,噪声是造成婴儿聋哑的原因之一。噪光可能会使孩子视觉的调节和眼的运动速度减慢,对光亮的适应能力降低,视觉和视野异常,出现眼痛、眼花等症状。\textbf{强烈的噪声和噪光还会使血压升高、心跳加快、胃酸分泌减少、胃肠道功能紊乱,出现呕吐、恶心、食欲下降等症状,甚至刺激脑细胞,破坏大脑皮质的兴奋和抑制的平衡,出现头痛、头晕、失眠、多梦、乏力和记忆力减退等表现。}

% 目前,国家标准规定住宅区的噪声白天不能超过50贝,夜间应低于40贝。但是,据测定,许多电视机、录音机所发出的声音已达60\textasciitilde80贝,高声说话也可达到60贝。这些噪声可能对孩子产生危害,不能不引起父母的重视。

% %ux4e09ux5b9dux5b9dux65e5ux5e38ux7528ux54c1ux5fc5ux5907ux6307ux6570}{%
% \subsection{\texorpdfstring{三、\textbf{宝宝日常用品必备指数}}{三、宝宝日常用品必备指数}}%ux4e09ux5b9dux5b9dux65e5ux5e38ux7528ux54c1ux5fc5ux5907ux6307ux6570}}

% \begin{longtable}[]{@{}
%   >{\raggedright\arraybackslash}p{(\columnwidth - 6\tabcolsep) * \real{0.2500}}
%   >{\raggedright\arraybackslash}p{(\columnwidth - 6\tabcolsep) * \real{0.2500}}
%   >{\raggedright\arraybackslash}p{(\columnwidth - 6\tabcolsep) * \real{0.2500}}
%   >{\raggedright\arraybackslash}p{(\columnwidth - 6\tabcolsep) * \real{0.2500}}@{}}
% \toprule()
% \begin{minipage}[b]{\linewidth}\raggedright
% 物品
% \end{minipage} & \begin{minipage}[b]{\linewidth}\raggedright
% 数量
% \end{minipage} & \begin{minipage}[b]{\linewidth}\raggedright
% 必备指数
% \end{minipage} & \begin{minipage}[b]{\linewidth}\raggedright
% 备注
% \end{minipage} \\
% \midrule()
% \endhead
% 纱布内衣 & 5\textasciitilde6件 & ★★★ & 吸汗、透气 \\
% 连身内衣 & 5\textasciitilde6件 & ★★★ & 吸汗、透气 \\
% 包巾 & 2\textasciitilde3件 & ★★★ & 厚薄依季节选择更换 \\
% 肚衣 & 2\textasciitilde4件 & ★★ & 保暖 \\
% 长袍 & 3\textasciitilde4件 & ★★★ & 棉质、耐穿 \\
% 两用服装 & 3\textasciitilde4件 & ★★★ & 长、短袖 \\
% 围兜 & 2\textasciitilde3件 & ★★★ & 吸水、易清洗 \\
% \textbf{外出服} & 3\textasciitilde4件 & ★★★ & 长袍式,两件式套装 \\
% 帽子 & 1\textasciitilde2顶 & ★★ & 透气、保暖 \\
% 袜子 & 2\textasciitilde4双 & ★★★ & 保暖、易清洗 \\
% 手套 & 3\textasciitilde4双 & ★★ & 保暖、宽松舒适 \\
% 纸尿裤 & 3打 & ★★★ & 合身、透气 \\
% 鞋子 & 2\textasciitilde3双 & ★ & 合脚、舒适 \\
% 小衣架 & 6\textasciitilde12个 & ★★★ & 适合宝宝小衣服 \\
% 儿童衣橱 & 1个 & ★★ & 与大人衣服分开放置 \\
% \bottomrule()
% \end{longtable}

% \begin{quote}
% gpt:

% 注意:必备指数中,``★''的数量代表必需程度,三个``★''表示非常必需,两个``★''表示比较必需,一个``★''表示一般必需。
% \end{quote}

% %ux56dbux5a74ux513fux7684ux7a7fux8863ux7a0bux5e8f}{%
% \subsection{四、婴儿的穿衣程序}%ux56dbux5a74ux513fux7684ux7a7fux8863ux7a0bux5e8f}}

% 给宝宝穿衣服可不是件容易的事情,特别是新生儿。新生儿身体柔嫩,脖子软软的,四肢又呈强硬的蜷曲状,更不会配合父母穿衣。给宝宝穿衣的大致方法如下:

% \begin{enumerate}
% \def\labelenumi{\arabic{enumi}.}
% \item
%   将胸前开口的衣服打开,平放在床上。
% \item
%   让宝宝平躺在衣服上,妈妈一只手将宝宝的手送入衣袖,另一只手从袖口伸进衣袖,慢慢将宝宝的手拉出衣袖。同时,妈妈的另一只手将衣袖向上拉。之后,用同样的方法穿对侧衣袖。
% \item
%   把穿上的衣服拉平,系上系带或扣上纽扣。用同样方法穿外衣。
% \item
%   穿裤子比较容易,妈妈的手从裤管中伸入,拉住宝宝的小脚,将裤子向上提,即可将裤子穿上。气温不是很低时,可不穿裤子,直接穿上尿裤。
% \end{enumerate}

% 穿宽松式连身衣时,先将连身衣纽扣解开,平放在床上,让宝宝躺在上面。\textbf{先穿裤腿},再用穿上衣的方法将手穿入袖子中,然后扣上所有的纽扣即可。连身衣穿脱方便,穿着也舒服,保暖性能也很好。

% \begin{quote}
% 育儿小百科

% \textbf{不要给宝宝穿得太多}

% 俗话说:"欲要宝宝安,常带三分饥和寒。''宝宝的汗腺分泌十分旺盛,而宝宝又喜欢活动,穿着过多时很容易出汗,而汗湿的衣服如果没有及时更换,则会引起感冒。另外,长期穿着过多还会降低宝宝的耐寒能力。夏天捂着宝宝尤其有害,宝宝极易中暑和腹泻。

% 怎样才能把握宝宝的穿衣件数呢?1岁以内的宝宝由于运动量少,穿衣的件数可参考父母,和妈妈差不多即可。不要参考祖父母,因为老年人代谢率低、产热少,穿衣往往较多。1岁以上的宝宝大多好动,容易出汗,因此穿衣应比妈妈少为宜。

% 判断宝宝所穿的衣服是否适当,要看宝宝的反应。宝宝感到冷时会显得不活泼,甚至把身子缩成一团,手足发冷。此时,应加衣。宝宝感到热时会表现得烦躁不安,爱发脾气,身上有汗,这时就应减衣或少盖一些。另外,要根据当天天气情况增减衣物。
% \end{quote}

% %ux4e94ux5982ux4f55ux5e94ux5bf9ux5b9dux5b9dux54edux95f9}{%
% \subsection{五、如何应对宝宝哭闹}%ux4e94ux5982ux4f55ux5e94ux5bf9ux5b9dux5b9dux54edux95f9}}

% \hspace{0pt}\includegraphics[width=2.39161in,height=2.32168in]{media/rId1225.png}\hspace{0pt}

% %ux54edux95f9ux7684ux539fux56e0}{%
% \subsubsection{1. 哭闹的原因}%ux54edux95f9ux7684ux539fux56e0}}

% 哭是宝宝的本能之一,也是他表达情感和需求的最重要的一种方式。在宝宝出生后的第一年,当他还未学会用语言或肢体动作来表达他的情绪或需要时,只能用哭泣来表示。宝宝哭泣所代表的信息是多层面的,大约可分为生理需求、心理反应、病理状况3种。表达这3种状况或需求的哭法是不同的,应该注意区分。

% %ux751fux7406ux9700ux6c42}{%
% \subsubsection{2. 生理需求}%ux751fux7406ux9700ux6c42}}

% 对于尿布脏了或湿了,宝宝饿了、渴了、痒了,太热或太冷、太吵,光线太亮或太暗等不适,宝宝都用哭声来表达。与宝宝朝夕相处的妈妈只要仔细倾听,就\textbf{能分辨出不同}。满足宝宝的要求后,宝宝就会停止哭闹。因此,这类哭泣是比较好解决的。

% %ux5fc3ux7406ux9700ux6c42}{%
% \subsubsection{3. 心理需求}%ux5fc3ux7406ux9700ux6c42}}

% 每个宝宝的气质都不同,有的宝宝动不动就大哭大闹,有的宝宝却是常常笑容满面\ldots\ldots 父母应多观察宝宝的行为表现,了解他先天的气质。

% 爱哭泣的宝宝比较黏人,易受惊吓,是性格上比较敏感或坚持度高、适应性差的类型。表达心理需求的哭声比较轻、低。宝宝甚至会盯着妈妈或伸出双手表示他想要抱,想要有人陪他玩。对于这种哭,父母只要抱抱他、逗弄逗弄他,就万事大吉了。【木木哭之前也有段时间是如此】

% 3\textasciitilde6个月时,宝宝开始熟悉亲近的人,高兴就笑,不高兴就哭。6个月以后,宝宝对四肢的控制更成熟,表情也更丰富,许多生理需求不必借哭来表示了。因此表达情绪的哭泣比例增加,不满、失望、害怕、生气、挫折感等都是哭泣的原因。当被妈妈拥抱时,宝宝能感到满足与愉悦。所以父母应该\textbf{抱抱}宝宝,让宝宝感受到关爱,这对他日后的情绪发展有良好的启发作用。

% 另外,还有一种哭闹是由父母\textbf{管教不当}导致的。宝宝在与父母的互动中,会以一定的方式探测父母的反应,如果这种方式很``管用'',那么宝宝就会采用这种方式达到自己的目的。例如,如果宝宝一哭闹父母就给他玩具、糖果,那么日后宝宝就会以哭闹的方式来``谋取''玩具、糖果,达到自己的目的。这种哭闹主要见于稍大的宝宝。

% %ux75c5ux75db}{%
% \subsubsection{4.病痛}%ux75c5ux75db}}

% 假如宝宝哭声比平常尖锐凄厉,或哭时握拳、蹬腿、烦躁不安,不论如何抱也无法让他安静下来,那么宝宝可能是生病了。当身体不适引起疼痛时,不会说话的宝宝就用肢体语言和哭声来表达。各种系统的疾病,只要能引起不适,宝宝都会用哭声来表达,且这种哭声往往不同于一般的哭声。常见的导致宝宝哭闹的疾病如下:

% (1) 口腔溃疡或咽部溃疡。哭闹发生在喂奶或进食时。

% (2)
% 腹痛。肠套叠、急性阑尾炎、嵌顿性腹股沟病等都可以引起哭闹,但各有特点。肠套叠时的哭闹是阵发的,宝宝面色苍白,不哭闹时两下肢蜷曲,靠近腹部,同时有呕吐,排果酱样的大便。急性阑尾炎时的哭闹是持续性的,宝宝不让别人摸腹部。如果阑尾穿孔,哭闹更加剧烈。嵌顿病是指肠子嵌住了,腹痛剧烈,伴有呕吐,宝宝常常用双手抚着嵌顿的腹股沟处,可在患处摸到嵌顿的肠段。

% (3)
% 鼻塞。鼻塞的宝宝吃奶时不能呼吸,故常吃几口奶后哭几声,哭几声后又吃几口奶,不吃奶时则不哭闹。

% (4)
% 头痛。各种原因引起的头痛都会使宝宝哭闹,脑膜炎或者新生儿颅内出血时,宝宝的哭声短促而尖利。

% (5)
% 中耳炎。患中耳炎时,宝宝耳朵贴近妈妈身体时就哭,牵拉其耳屏时哭闹更厉害。

% (6) 其他。如尿布皮炎、皮肤溃破、虫咬等。

% %ux6b63ux786eux7684ux5b89ux629aux65b9ux5f0f}{%
% \subsubsection{5.正确的安抚方式}%ux6b63ux786eux7684ux5b89ux629aux65b9ux5f0f}}

% 对初为父母的新爸妈来说,宝宝的哭闹常常令他们感到手足无措。不过,与宝宝打交道时间长了,父母们就能充分了解宝宝哭声传达的信息。

% 当宝宝哭时一定要弄清宝宝的哭因。如果属于生理方面的或心理方面的,是正常的,要用关爱的态度去安抚和满足他们。如果是疾病引起的哭闹,则必须请医师诊治。

% \textbf{(1)
% 稳定情绪、坦然面对}。父母与宝宝的情绪可互相感染,宝宝哭泣时,父母不要因此惊慌失措或发脾气。如果父母表现出不耐烦、烦躁或紧张,宝宝可能会哭得更厉害。因此,父母一定要先克制自己的情绪,让哭闹中的宝宝感觉到父母的冷静。

% \textbf{(2)
% 抱在怀中,安定情绪}。当宝宝情绪失控时,可将宝宝抱在怀中轻拍安抚。有的宝宝哭闹可能仅仅是因为\textbf{皮肤``饥饿''},需要父母抱一抱,从父母的搂抱中获得安全感,消除不安,得到安慰,父母不要置之不理。抱一会儿后,将宝宝妥善安置在小床上、小推车里,让他继续独自玩。

% \textbf{(3)
% 用温柔的语气对宝宝说话。}温柔地对宝宝说,``怎么了,为什么这么生气,还哭得这么伤心'',``你很难过,哭一下没什么关系,告诉妈妈为什么难过,让妈妈帮你''等。让宝宝感觉到妈妈的关心,烦躁情绪就会缓解。【木妈妈就是这样哇!】

% \textbf{(4)
% 利用辅助工具安抚宝宝。}宝宝闹情绪时,可以利用一些宝宝平常喜爱的玩具、布娃娃,让他抱在怀中,暂时安抚宝宝失控的情绪,然后再进行安抚、询问。

% (5)
% 一杯水、一条毛巾,让宝宝感觉到被关心。拿水给宝宝喝,拿毛巾给宝宝擦拭泪水,可以让宝宝得到安慰。

% (6)
% 站在宝宝的立场去思考问题。当宝宝闹情绪时,应了解宝宝发脾气、哭闹的真正原因,站在宝宝的立场去思考问题,即所谓移情。当宝宝对父母诉说他的感觉和想法时,除用心倾听外,还要表示同情,重复宝宝所讲的话会让宝宝感觉到父母是了解他的。

% \textbf{(7)
% 适当的管教。}如果发现宝宝是在以哭闹为手段来达到满足自己要求的目的,比如要买新玩具、要吃糖果等,父母就必须适度地管教宝宝,让宝宝知道这样的行为、举动是达不到目的的。

% %ux9519ux8befux7684ux5b89ux629aux65b9ux5f0f}{%
% \subsubsection{6.
% 错误的安抚方式}%ux9519ux8befux7684ux5b89ux629aux65b9ux5f0f}}

% \hspace{0pt}\includegraphics[width=2.40559in,height=2.68531in]{media/rId1233.png}\hspace{0pt}

% 父母在安抚哭闹的宝宝时,\textbf{表达关爱}是很重要的。一个关怀的眼神,一句温柔的话语,都可有效缓解宝宝的情绪。如果用错了安抚方法,则会让这场``战争''愈演愈烈。

% 以下是一些不恰当的安抚方法:

% \textbf{过激处理}。冲动地责怪宝宝,嫌宝宝``真是讨厌'',甚至生气地破口大骂,叫宝宝立刻停止哭闹,以威吓的方式强迫宝宝不哭。

% \textbf{过于冷漠}。对宝宝哭闹完全视若无睹,继续做自己的家事,任由宝宝在一旁哭闹。

% \textbf{``哄骗''}。拿一样吃的、玩的东西给哭闹中的宝宝,希望他能就此停止哭闹。

% \textbf{反应过度}。过于宠爱宝宝,宝宝一哭,就急忙将他抱住安慰。这种方式的后果是,宝宝日后经常以哭闹的方式来达到自己的目的。

% 要从容应对宝宝的哭闹需要一段时间,当父母越来越了解自己的宝宝时,就会掌握一套有效的应对方法。

% \begin{quote}
% 育儿小百科

% \textbf{如果宝宝一哭就抱,会不会宠坏他}

% 有的父母担心宝宝一哭就抱会宠坏宝宝。其实,对小宝宝来说,父母不用担心这个问题,他还不懂得以哭为"武器"。相反,及时给哭泣的宝宝以拥抱,对其发展很有好处。研究调查显示,父母在宝宝哭泣时立刻给予回应,会让宝宝有被重视的感觉,有助于其日后自信心与自尊心的建立,这样与宝宝沟通的技巧也比较好。而哭泣时被父母冷漠或忽视的宝宝会比一般的宝宝更爱哭,哭泣时间更长。宝宝日后的人际关系发展也受到负面影响。
% \end{quote}

% %ux516dux5982ux4f55ux5bf9ux5f85ux7231ux54edux7684ux5b9dux5b9d}{%
% \subsection{六、如何对待爱哭的宝宝}%ux516dux5982ux4f55ux5bf9ux5f85ux7231ux54edux7684ux5b9dux5b9d}}

% 有的宝宝特别爱哭,一遇到不愉快的事情就会``哇哇''大哭,哄也哄不住,常常弄得父母心烦意乱。

% 如果采取前面的安抚措施还不能让他停止哭泣时,那么就采取不予理睬的``冷处理''方法,让他哭一会儿,发泄一下。待他停止哭泣,再跟他讲道理,进行正面教育。

% 强行让宝宝不哭或一味迁就他,对他的心理发育都是不利的,前者使其心理受压抑,后者使其学会将哭当作法宝来要挟父母。

% %ux4e03ux9884ux9632ux63a5ux79cd}{%
% \subsection{七、预防接种}%ux4e03ux9884ux9632ux63a5ux79cd}}

% 孩子在2个月时已经服用了孩子麻痹糖丸(脊髓灰质炎三价混合疫苗)的第一丸,3个月时要继续服用第二丸。

% 从本月开始,要给孩子注射三联针(百日咳菌苗、白喉类毒素、破伤风类毒素混合剂)。这种预防针要打3次,每次间隔30天,在3个月内连续注射完毕,才能达到预防的目的。为什么这种针要连续打3次呢?因为注射1次产生的抗体维持的时间较短,必须连续注射,才能使机体产生一定的抗体,达到足够的抗病能力。注射三联针后,大部分孩子都有轻度的发热反应,这没关系。如果体温超过38.5℃,可服一次退热药,1\textasciitilde2日体温可恢复正常。什么情况不能注射三联针呢?孩子发热不适的情况下暂时不能注射,待病愈后再注射。另外,有过敏体质的孩子、脑神经系统发育不正常的孩子、脑炎后遗症或癫痫的孩子均不能接种,以免发生抽风等意外情况。

% %ux7b2cux56dbux82823-4ux4e2aux6708ux5a74ux513fux7684ux65e5ux5e38ux62a4ux7406}{%
% \subsection{04第四节3 \textasciitilde{}
% 4个月婴儿的日常护理}%ux7b2cux56dbux82823-4ux4e2aux6708ux5a74ux513fux7684ux65e5ux5e38ux62a4ux7406}}

% %ux4e00ux80ccux888bux7684ux4f7fux7528}{%
% \subsection{一、背袋的使用}%ux4e00ux80ccux888bux7684ux4f7fux7528}}

% 婴儿4个月以后,脖子可以挺立了,头也能竖直了,这时可以使用背袋。在背袋中,婴儿一般是竖直位,因此视野比横躺位时要开阔得多,可以``放眼世界''。对父母来说,背袋解放了双手,省事多了。

% %ux80ccux888bux7684ux4f7fux7528ux65b9ux6cd5}{%
% \subsubsection{1.
% 背袋的使用方法}%ux80ccux888bux7684ux4f7fux7528ux65b9ux6cd5}}

% 大多数背袋都可采用\textbf{前抱}式和\textbf{后背}式两种使用方法。一般地说,\textbf{小月龄的婴儿比较适合前抱式},即用背袋将婴儿放在母亲胸前。这种方法可以让母亲随时观察婴儿的情况,能随时与婴儿进行情感交流。较大月龄的婴儿则可以采取后背式,这样母亲在做家务时比较省力,但情感交流不够。另外,采取后背式方法背婴儿时,母亲要注意婴儿的安全,不能因为忙于家务而碰伤了背上的婴儿。对于小婴儿来说,由于其颈部肌肉尚未发育完全,故可采取横抱的姿势,即先将婴儿固定在背袋中,然后将双肩带合在一起斜挎起来,使婴儿斜躺在母亲胸前,母亲只需用一只手轻轻托住婴儿的头部即可。

% %ux80ccux888bux7684ux4f7fux7528ux8981ux70b9}{%
% \subsubsection{2.
% 背袋的使用要点}%ux80ccux888bux7684ux4f7fux7528ux8981ux70b9}}

% 无论采取哪种姿势背婴儿,母亲都要注意以下几点:

% \begin{enumerate}
% \def\labelenumi{\arabic{enumi}.}
% \item
%   婴儿不能连续在背袋中待两小时以上。如果\textbf{长时间}处于一个姿势会导致婴儿背部肌肉过分疲劳,从而影响他的生长发育。
% \item
%   \textbf{喂奶后30分钟}以内不能使用背袋,因为这段时间内婴儿很容易发生吐奶,而背袋可能压迫婴儿的胃部,增加了吐奶的机会。
% \item
%   背着婴儿时不能做剧烈运动或大幅度的动作,如跑、跳、弯腰等。
% \item
%   夏天外出时,母亲更要注意不能让婴儿在背袋中待得过久,因为母子之间身体接触时间过长,会使婴儿体温上升,引起中暑。
% \end{enumerate}

% %ux4e8cux65e9ux671fux9884ux9632ux80a5ux80d6ux75c7}{%
% \subsection{二、早期预防肥胖症}%ux4e8cux65e9ux671fux9884ux9632ux80a5ux80d6ux75c7}}

% 在人的生活水平不断提高和各种营养品不断产生的今天,一些家长往往不注意孩子营养的均衡,因而造成``小胖子''也越来越多。这些肥胖的孩子到成年后各种疾病的隐患也就越多。

% 什么样的程度才叫肥胖呢?医学上通常把超过同龄同身高正常体重20\%的儿童称为肥胖儿。过多的脂肪不仅对机体是一个沉重负担,对心理也会造成一定程度的损害。

% 肥胖的孩子不爱户外活动,在孩子群体中易成为同伴们取笑的对象。随着年龄的增长,容易在心理上产生压力,出现自卑感,形成孤僻的不良性格特征。到成年后还会给生理健康带来许多隐患,如高血压、糖尿病、动脉粥样硬化、冠心病、肝胆疾病及其他一系列与之密切相关的疾患。

% 肥胖孩子由于脂肪组织过多,皮肤皱褶加深,若护理不当容易因局部潮湿引起皮肤糜烂或产生疖肿。

% 此外,孩子肥胖与遗传有关。父母中1人肥胖,孩子出现肥胖率约为40\%。若父母双方均肥胖者,孩子肥胖可达70\%。预防肥胖对有肥胖家族史的孩子尤其重要。

% 预防方法主要有:

% (一) 坚持母乳喂养至少4个月。

% (二) 最好6个月前不喂固体食物。

% (三)
% 合理喂养。营养品种多样化,均衡热量摄入应按照月龄需要喂养,保证正常长发育为好。

% (四)
% 1〜3岁期间饮食需要有规律,\textbf{不要用哺喂的方法制止非饥饿性的哭闹。}

% 孩子生长发育阶段需要大量蛋白质供应,对于肥胖孩子要减少其动物性脂肪和糖类的摄入,注意及早锻炼身体,多活动。

% %ux4e09ux9884ux9632ux63a5ux79cd}{%
% \subsection{三、预防接种}%ux4e09ux9884ux9632ux63a5ux79cd}}

% 4个月的孩子应该第3次服用骨髓灰质三价混合疫苗(孩子麻痹糖丸),应按时带孩子到所属防疫部门服用。

% 4个月的孩子该注射三联针的第二针了,三联针是用来预防百日咳、白喉、破伤风等疾病的。百日咳是由百日咳杆菌引起的一种急性呼吸道传染病。咳嗽时表现为一阵阵痉挛性剧咳,使孩子非常痛苦。患上百日咳2\textasciitilde3个月才能治愈,有的可继发肺炎。白喉是白喉杆菌引起的烈性传染病。患病后婴儿咽喉部可见白色假膜,假膜沿呼吸道蔓延,病情发展快且严重,有的很快出现呼吸困难窒息死亡,后果不堪设想。破伤风是由于破伤风杆菌引起的急性传染病。孩子皮肤嫩,容易碰伤,伤口易受破伤风杆菌污染,破伤风杆菌可产生毒素,伤害人体神经系统,造成抽搐、牙关紧闭,甚至窒息死亡。这三种传染病,严重地威胁着孩子的健康成长,自从广泛进行了``白、百、破''预防针的注射后,这三种传染病的发病率明显降低。所以,一定要按时给孩子进行预防接种,以防患于未然。

% 三联针第二针的注射时间应与第一针相隔30天以上,如果此时正巧生病,可推迟几天再去接种,但最多不要超过60天。

% %ux56dbux5a74ux513fux4e3aux4ec0ux4e48ux6d41ux53e3ux6c34}{%
% \subsection{四、婴儿为什么流口水}%ux56dbux5a74ux513fux4e3aux4ec0ux4e48ux6d41ux53e3ux6c34}}

% 口水是人体口腔内唾液腺分泌的一种液体,含有丰富的酶类,是促进食物消化吸收的一种重要物质。那么为什么很少见新生儿流口水,大人也不流,只有此时的婴儿才流呢?这与孩子此阶段发育特点有关。

% 3个月以下的孩子,中枢神经系统和唾液腺发育未成熟,唾液分泌量很少。而成人呢,口腔唾液分泌与吞咽功能相协调,多余的口水在不知不觉中就咽下去了。

% 孩子到3\textasciitilde4个月的时候,中枢神经系统与唾液腺均趋向于成熟,唾液分泌逐渐增多,再加上孩子到三四个月有的已长出了牙,对口腔神经产生刺激,使唾液分泌增加了。婴儿的口腔较浅,吞咽功能又差,不能将分泌的口水吞咽下去或存在口腔中,口水就不断地顺嘴流出来。这是一种生理现象,不是病态。

% 一般到2\textasciitilde3岁流口水的现象会自然消失。但有的孩子有口腔溃疡等疾患时,也可引起流口水,常伴有不吃奶、哭闹等,这时就要请医生给孩子看病了。

% %ux4e94ux5a74ux513fux4fbfux79d8ux7684ux62a4ux7406ux65b9ux6cd5}{%
% \subsection{五、婴儿便秘的护理方法}%ux4e94ux5a74ux513fux4fbfux79d8ux7684ux62a4ux7406ux65b9ux6cd5}}

% 吃牛奶的孩子常常便秘,每次排便很痛苦,有的甚至把肛门撑破。孩子因此而哭闹,不愿大便,使家长心急如焚。怎样避免这种情况呢?可以采用下述办法试一试:

% \begin{enumerate}
% \def\labelenumi{\arabic{enumi}.}
% \item
%   在奶中适当增加糖分,100毫升牛奶中加10克白糖。
% \item
%   给孩子吃些蜂蜜水。
% \item
%   注意给孩子吃新鲜果汁水、蔬菜水和苹果泥等维生素含量高的辅食。
% \end{enumerate}

% 如果孩子大便十分费力,难以排出,用甘油栓塞入肛门,若仍便不出,可用孩子开塞露或者到医院请医生处理,千万不要随便给孩子服用泻药。

% 经常便秘的孩子,除了在饮食上调剂外,还应坚持做体操,以增加腹肌的力量,有利于排便。

% \begin{quote}
% 育儿小百科

% \textbf{孩子不能常吃小中药}

% 有许多家长常在孩子看完病后,还要求大夫加开一点小中药,如至宝锭、妙灵丹等,理由是怕孩子生病,常给孩子吃点小中药预防着。这种做法既不妥当也不科学。

% 这是因为人食入的任何\textbf{药物都要在肝脏解毒},由肾脏排泄。孩子的身体处在成长发育过程,许多脏器功能尚未成熟,肝脏解毒功能差,肾脏排泄的功能不完全,应尽量少用药,更不要随便经常滥用药。许多孩子中药制剂中,都含有朱砂,中药用来镇惊,但朱砂是炼汞的原料,长期服用,可蓄积中毒,影响孩子的生长发育。
% \end{quote}

% %ux516dux5b69ux5b50ux5435ux591cux600eux4e48ux529e}{%
% \subsection{六、孩子吵夜怎么办}%ux516dux5b69ux5b50ux5435ux591cux600eux4e48ux529e}}

% 许多孩子晚上不睡觉,但也没有其他不适的症状,这是怎么回事,怎么办呢?实际上,人类昼出夜寝的习惯是在长期的生活中形成的,是一种普遍的生活习惯。如果你有意识地培养自己白天睡觉的习惯,那么到了晚上就不会发困。孩子也不例外,如果睡够了,不管在什么时候醒来,都显得很精神。当然,如果在夜间醒来,就会扰得大人不得安宁。

% 睡眠既然是个生活习惯,就可以调节,这需要母亲有意识地训练自己的孩子,养成良好的睡眠习惯。白天让孩子尽量少睡,在夜间除了喂奶、换1-2次尿布以外,不要打扰孩子。在后半夜,如果孩子睡得很香也不哭闹,可以不喂奶。随着孩子的月龄增长,逐渐\textbf{过渡到夜间不换尿布、不喂奶}。如果妈妈总是不分昼夜地护理孩子,那么孩子也就会养成不分昼夜的生活习惯。如果以上办法都不起作用的话,可以在医生的指导下,吃点镇静药。剂量适当地吃2-3天的镇静药不会影响孩子的大脑发育,也不会引起其他不良后果。

% %ux4e03ux5b9dux5b9dux57284ux4e2aux6708ux65f6ux5c31ux5df2ux751fux5ac9ux5992ux5fc3}{%
% \subsection{七、宝宝在4个月时就已生嫉妒心}%ux4e03ux5b9dux5b9dux57284ux4e2aux6708ux65f6ux5c31ux5df2ux751fux5ac9ux5992ux5fc3}}

% 心理学家们认为,宝宝在2岁以后才会有嫉妒这种复杂的感情,但英国的心理学家最新研究却发现,4个月大的宝宝已经懂得``吃醋''。

% 在一次偶然的机会中,心理学家发现一个只有4个月的宝宝流露出嫉妒之情,于是决定对24名母亲和她们的宝宝(4\textasciitilde6个月大)进行研究。

% 心理学家先让这些母亲们互相交流,然后让她们去抱别人的宝宝抚摸、玩耍。当母亲们相互交流时只有3个宝宝大哭起来;而当自己的母亲抱起别的宝宝爱抚时,却有13个宝宝哭闹不止。【武艺...】

% \begin{quote}
% 育儿小百科

% 不愉快的家庭生活可导致宝宝偏头痛

% 芬兰的一项医学研究表明,发作性头痛家族史是儿童易患偏头痛的危险因素,而家庭生活不愉快则会增加这种危险。

% 研究发现,患偏头痛的儿童大多家庭中有问题,如环境差、父母不和,离异及父母对孩子态度不恰当,而这些家庭问题在无偏头痛的儿童中很很少见。研究者认为,遗传和环境对儿童偏头痛的发病率有影响。
% \end{quote}

% %ux516bux9002ux540846ux4e2aux6708ux5b9dux5b9dux7684ux73a9}{%
% \subsection{八、适合4\textasciitilde6个月宝宝的玩}%ux516bux9002ux540846ux4e2aux6708ux5b9dux5b9dux7684ux73a9}}

% 这时的宝宝能注视\textbf{较远距离}的物体,会通过视、听、嘴咬、吮吸、手的触摸来接受外界刺激。刚开始时,宝宝尝试着挥动手臂朝物体摇动,试图用手去接触物体,但由于动作的不协调而达不到。这时的宝宝在抓东西时,主要不是用手指的动作,而是把整只手弯起来,好像一个大钩子那样。到了6个月,他们的手眼协调已发展起来了,大拇指的动作与其他四指在逐渐分开,渐渐地把大拇指放在物体的一边,其他的四指放在另一边,这就形成所谓``五指分工''。这时,宝宝已从仰卧学会侧卧,到翻身和独自坐一会儿。这些动作能力的获得,一方面扩大了宝宝接触和探索环境的范围,同时也使双手解放出来,这对双手的协调动作和手指精细动作的发展是极为有利的。

% 这一时期宝宝开始能区别一些物体和现象。6个月时,见到熟人,就能表示高兴,并开始认生。为了进一步发展宝宝的视觉能力,父母不应该满足于婴儿躺在床上不哭不闹,这样会失去视觉能力发展的机会。

% 手的动作出现是这一时期的主要特征。家长的主要任务是在促进宝宝视、听运动协调的同时,给宝宝提供可触摸的玩具,让宝宝去抓、捏、握玩具,练习手的动作和翻身、爬行的动作。

% \textbf{玩具举例}

% %ux8badux7ec3ux624bux7684ux52a8ux4f5cux7684ux73a9ux5177}{%
% \subsubsection{训练手的动作的玩具}%ux8badux7ec3ux624bux7684ux52a8ux4f5cux7684ux73a9ux5177}}

% \hspace{0pt}\includegraphics[width=2.37762in,height=1.93007in]{media/rId1257.png}\hspace{0pt}

% \textbf{主要的玩具有}:摇棒、哗铃棒、拨浪鼓、各种环状玩具、拉串、软硬塑料和橡胶玩具。

% \textbf{玩法及成人指导}:

% (一)成人开始用玩具碰触宝宝的手,帮宝宝拿住,之后便提供材料,让宝宝自己去抓握。所放玩具大约在距离宝宝的脸部25厘米处。

% (二)可用松紧带把玩具悬挂在宝宝手能够摸到的地方,以便使宝宝能自己玩弄。

% (三)可选用一些用手捏住可发声的橡胶玩具或较轻的小型玩具。

% (四)玩具应\textbf{便于宝宝抓握},不要太大或太小,或太滑。

% (五)宝宝抓握到玩具,有时会放到嘴里吸吮,因此玩具必须坚固,经常保持干净,清洗消毒。【不易伤到人】

% (六)当宝宝想要触摸玩具失败时,不要迫不及待地把玩具塞给宝宝,而应该\textbf{给宝宝提供尝试和练习的机会},鼓励宝宝自己去触摸。

% (七)玩具可以配合躲猫猫等游戏来玩,以增加游戏的乐趣。

% %ux7ec3ux4e60ux7ffbux8eabux722cux884cux52a8ux4f5cux7684ux73a9ux5177}{%
% \subsubsection{练习翻身、爬行动作的玩具}%ux7ec3ux4e60ux7ffbux8eabux722cux884cux52a8ux4f5cux7684ux73a9ux5177}}

% (一)将玩具放在宝宝必须经过翻身才能取到的地方,锻炼宝宝向左向右翻身拿到玩具。

% (二)将玩具放在儿童俯卧的前方,逗引宝宝向前爬行,伸手抓取玩具,并逐渐加长些距离。【试过】

% (三)在学翻身时,成人也可用手轻推宝宝的背部,帮助宝宝翻过来。

% (四)成人可把手\textbf{抵在婴儿的脚后},在他们爬行时做推动来帮助。但是不可在宝宝没有做出爬行动作时去推动。【试过】

% %ux4e5dux89c2ux5bdfux5b9dux5b9dux7684ux542cux529bux662fux5426ux6b63ux5e38}{%
% \subsection{九、观察宝宝的听力是否正常}%ux4e5dux89c2ux5bdfux5b9dux5b9dux7684ux542cux529bux662fux5426ux6b63ux5e38}}

% 妈妈应仔细观察,发现宝宝在不同年龄对``声音''的反应,从而判断出宝宝的听力是否正常。

% \textbf{刚出生}:对大的声音会做出惊跳反射。站在宝宝的身后,用小铃铛、小哨子或者双手相拍,观察宝宝的反应。若宝宝眨眼睛、转身或者有惊跳的反应,说明有听力。

% \textbf{3个月后}:能转过头寻找声音的方向。

% \textbf{6\textasciitilde12个月}:能熟练判断声音的方向。

% \textbf{8\textasciitilde12个月}:对自己的名字或其他感兴趣的名词会做出反应。

% \textbf{9个月以后}:开始呼呼呀呀地练习发音,并能听懂简单的话,说出单音字。

% 如果经过反复细心的观察,仍得不到上述反应的肯定结论,就应该到医院进行进一步的检查和确诊。

% 对于儿童的听力检测有主观和客观两种方法。客观听力检查就是不需要儿童的配合,医生即可。目前大部分医院仍以客观听力测查为主要检测手段。这种检查方法对于幼儿来说,能反映出的实际听力情况有限,因为幼儿由于年龄的特点,在检查时不能像成年人一样主动配合,从而不能反映幼儿的全部听力。主观行为听力测试是根据儿童各年龄段身体及行为的发育特点,制定出的具体评估听力状况的方法。这一评估方法是在幼儿的自然状态或者是在和幼儿的游戏中进行。

% 通过给声后观察儿童的反应,这种观察既可以得到比客观测听更全面的听力情况,而且能够尽早地发现幼小儿童的听力问题,同时还可评估听力有问题儿童的助听器选配效果,为患儿的治疗和康复提供了有效的手段。

% %ux7b2cux4e94ux828245ux4e2aux6708ux5a74ux513fux7684ux65e5ux5e38ux62a4ux7406}{%
% \subsection{05第五节4〜5个月婴儿的日常护理}%ux7b2cux4e94ux828245ux4e2aux6708ux5a74ux513fux7684ux65e5ux5e38ux62a4ux7406}}

% %ux4e00ux7ed9ux5a74ux513fux6309ux6469}{%
% \subsection{一、给婴儿按摩}%ux4e00ux7ed9ux5a74ux513fux6309ux6469}}

% 孩子一生中所接受的最强有力的按摩莫过于出生时接受的``按摩''了。当母亲分娩时,子宫长时间的收缩推动胎儿通过产道,这个过程刺激了孩子的末梢神经系统和主要的器官,为他降临到外面的世界做好了准备。出生后经常给孩子类似的皮肤刺激,对孩子的生长发育有很大的帮助。

% \hspace{0pt}\includegraphics[width=2.72727in,height=2.04196in]{media/rId1265.png}\hspace{0pt}

% %ux6309ux6469ux7684ux76eeux7684}{%
% \subsubsection{1. 按摩的目的}%ux6309ux6469ux7684ux76eeux7684}}

% (一)了解孩子,学会照料孩子。

% (二)孩子安静、少哭闹,改善睡眠,促进生长发育。

% (三)及时发现孩子的肌肉和关节是否存在隐藏的紧张和僵硬。

% (四)增强孩子肌肉的力量和关节的灵活性。

% (五)帮助消化。

% (六)增强孩子的免疫力。

% (七)帮助孩子建立安全感、信任感。

% (八)加强成人和孩子之间的互动关系,增加相互的感情依恋。

% %ux6309ux6469ux65f6ux95f4ux7684ux9009ux62e9}{%
% \subsubsection{2.
% 按摩时间的选择}%ux6309ux6469ux65f6ux95f4ux7684ux9009ux62e9}}

% \textbf{两次进食之间}是进行按摩的最佳时间。如果孩子太饱,按摩的过程会令他不舒服,尤其在按摩腹部的时候或俯卧的时候,如果孩子饥饿,那么无论时间长短,他都不愿意接受按摩。还可选择晚上孩子\textbf{洗完澡后}或白天他\textbf{最放松、反应最灵敏的时候}。每天按摩的时间最好\textbf{固定}下来,这样有助于良好生活习惯的培养。【洗澡,睡觉,各种固定...】

% %ux6309ux6469ux7684ux7981ux5fcc}{%
% \subsubsection{3. 按摩的禁忌}%ux6309ux6469ux7684ux7981ux5fcc}}

% 当孩子患病或身体不适时不要进行按摩。孩子拒绝按摩时,不要强行给他按摩,亦不要为了按摩而把孩子从睡眠中弄醒。按摩时最好避开孩子接种疫苗的部位及外伤、红肿、敏感的部位。

% %ux6309ux6469ux524dux7684ux51c6ux5907}{%
% \subsubsection{4.
% 按摩前的准备}%ux6309ux6469ux524dux7684ux51c6ux5907}}

% (一)保证按摩的环境安静、避风而温暖,室温最好在26℃左右。

% (二) 成人应取下珠宝首饰,将指甲修剪至合适的长度。

% (三)
% 准备一条毛毯、一块大毛巾、一个靠枕及按摩油(最好先进行``\textbf{皮试}'',以防过敏),如葡萄籽油、甜杏仁油、橄榄油、葵花子油等。

% (四) 一块尿布和一条干净的毛巾。

% %ux6309ux6469ux6280ux5de7}{%
% \subsubsection{5.按摩技巧}%ux6309ux6469ux6280ux5de7}}

% \begin{enumerate}
% \def\labelenumi{\arabic{enumi}.}
% \item
%   按摩时应保持双手放松,手指和手掌接触孩子的皮肤。如果成人的双手僵硬,那么紧张感就会被传递给孩子。
% \item
%   当孩子逐渐适应并喜欢按摩时,成人可渐渐增加按摩的压力。
% \item
%   在按摩时最好经常亲吻和搂抱孩子,和他说话、唱歌,增强成人和孩子的交流。
% \end{enumerate}

% %ux6309ux6469ux7684ux57faux672cux52a8ux4f5c}{%
% \subsubsection{6.
% 按摩的基本动作}%ux6309ux6469ux7684ux57faux672cux52a8ux4f5c}}

% \begin{enumerate}
% \def\labelenumi{\arabic{enumi}.}
% \item
%   擦:双手放松,把手的重量轻轻压在孩子的身体或肢体上,前后来回滑动。
% \item
%   揉:在孩子身体的柔软部位,如腹部,用整个手按压,然后放开。
% \item
%   扣:双手放松呈空心掌,有节奏地用手拍。
% \end{enumerate}

% %ux6309ux6469ux7684ux59ffux52bf}{%
% \subsubsection{7. 按摩的姿势}%ux6309ux6469ux7684ux59ffux52bf}}

% 坐式:成人坐在靠垫上,双脚伸直并分开,孩子脸对着成人,仰卧在毯子上。

% 跪式:为了更舒适些,成人可以双膝跪在毛巾上,臀部和小腿间垫个软垫,向后跪坐在软垫上。

% 双腿交叉式:双腿交叉坐在软垫上,将孩子放在成人的正前方。俯身向前,先试着按摩几下,看这种姿势是否舒适。

% %ux6309ux6469ux7684ux8fc7ux7a0b}{%
% \subsubsection{8.按摩的过程}%ux6309ux6469ux7684ux8fc7ux7a0b}}

% 一般按\textbf{从脚到头}的顺序进行按摩。大多数孩子喜欢从脚和腿开始按摩,然后再过渡到身体。

% \begin{enumerate}
% \def\labelenumi{\arabic{enumi}.}
% \item
%   脚部:在成人的手上涂上按摩油,揉搓孩子的脚背及脚掌两侧,再用拇指和食指牵拉每个脚趾并轻轻分开它们,使它们呈扇形。然后,屈伸孩子的脚踝,用一只手把孩子的脚向外侧转,使脚跟伸展,另一只手按摩其小腿。左右交替,整个过程约3分钟。
% \item
%   腿部:成人的左手放在孩子右腹股沟处,右手抓住孩子右侧踝关节,左手手掌从上向下按摩孩子的右腿,直至右脚部。重复五六次后,以同样的方法按摩左腿。
% \item
%   腹部:如果孩子不适应腹部的按摩,可先抚摩腹部,挠痒痒,使腹部放松,然后把手放在孩子的腹部进行按摩。用抹过按摩油且放松的手,从孩子左下腹部以顺时针方向做圆周按摩(此为食物通过消化道的方向)。整个过程约2分钟。
% \item
%   胸部:让孩子仰卧,将双手手掌放在孩子前胸,向下、向外按摩,反复五六次。然后,把两手放在孩子胸廓上部,向上、向外按摩孩子的双肩,反复五六次。整个过程约2分钟。
% \item
%   肩和臂:孩子俯卧,成人将双手掌平展地放在孩子胸部上方及前肩部,手掌反复地在孩子的肩部上方及外侧按摩(五六次)。然后,用双手抓住孩子的双臂,由上而下反复按摩孩子的手臂(五六次)。之后,将孩子的两臂向外平展,两手手掌自孩子肩部向手指方向按摩。
% \item
%   手:把孩子的手放在成人的手掌里揉搓,然后用成人的拇指和食指按摩孩子的手掌和手背,并轻轻地揉搓孩子的手腕和手指。接下来,用成人的拇指和食指轻轻拉扯孩子的每一个手指。最后,用成人的手掌揉搓孩子的整只手,包括手掌和腕。整个过程约2分钟。
% \item
%   背部和脊椎:孩子俯卧,成人一手按住孩子的臀部,一手从孩子的肩部沿脊椎向下按摩背部,双手轮换,动作有力,不时地给孩子挠痒痒,逗他高兴。接着,将手做空心拳状,轻轻地上下扣孩子的整个背、肩和脊椎。整个过程约2分钟。
% \item
%   头和颈:头和颈的按摩有着非常显著的安慰作用,而且按摩也不受时间、地点的限制,可以随时随地见缝插针地进行。按摩时,可先用成人的指尖打圆圈环绕孩子的前额、头顶、后脑勺做按摩,接着用指腹轻揉面、颈部和两肩。
% \end{enumerate}

% %ux4e8cux5916ux51fa}{%
% \subsection{二、外出}%ux4e8cux5916ux51fa}}

% 孩子一天天地长大,他需要到室外去看一看,兜一兜,接触新鲜空气和阳光,看看外面的人和事物。

% \hspace{0pt}\includegraphics[width=3.04895in,height=2.82517in]{media/rId1291.png}\hspace{0pt}

% 孩子出行,必要的装备少不了。如果是近距离的外出,父母可以根据季节变化给孩子做好御寒或防暑准备,如带上帽子、手套或遮阳伞。两个月以内的孩子可以由母亲抱着出门,或躺在卧式的婴儿推车中,较大的孩子可以坐在母亲的背袋里,或乘坐式的婴儿车出门。外出时,母亲的包里可放一瓶奶或水、尿布、餐巾纸、湿纸巾及垃圾袋等。不要带孩子去电影院、拥挤的百货商店和超市,因为那里\textbf{空气}不流通,容易传染疾病。如果是在炎热的夏季,最好\textbf{不要抱着孩子走得太远},因为母亲\textbf{抱着孩子时,孩子身体散热不好},会导致体温过高。

% %ux4e09ux65c5ux884c}{%
% \subsection{三、旅行}%ux4e09ux65c5ux884c}}

% 一般来说,5\textasciitilde6个月的孩子已经能够坐车、乘船或坐飞机了,可以到较远的地方去旅行了。由于路上时间较长,所以父母除了做好以上的外出准备外,还要进行更周密的安排。如果是人工喂养的孩子,最好准\textbf{备好三四个消过毒并且晾干了的奶瓶}。可以在奶瓶中装入奶粉,盖上盖子,这样到时候只要用开水冲开就可以喂孩子了。\textbf{最好不要在旅途中装奶粉,因为这样容易造成奶粉污染。}在旅途中转站时,可将喝完奶的奶瓶洗干净,有条件的还可以消毒晾干以备用。需要注意的是,不要在途中随便给孩子买食物,因为这些场所的食物质量很难有保证。旅途中要经常给孩子喂些水,因为孩子比成人更容易脱水。

% 另外,不要忘记给孩子\textbf{换尿布},一时的疏忽很容易使孩子发生尿布皮\textbf{炎}。

% 旅途中天气的变化快,父母可以为孩子带些毛毯、衣服及雨具。还要准备\textbf{一些药物},如退热药、常用抗生素、包扎伤口的消毒纱布及胶布等,以备不时之需。另外,冬春季节是传染病的易发时期,所以出门前不要忘了带孩子及时完成各项预防接种。

% %ux56dbux53efux80fdux53d7ux5230ux7684ux4f24ux5bb3}{%
% \subsection{四、可能受到的伤害}%ux56dbux53efux80fdux53d7ux5230ux7684ux4f24ux5bb3}}

% %ux65e5ux5149ux6027ux76aeux708e}{%
% \subsubsection{1. 日光性皮炎}%ux65e5ux5149ux6027ux76aeux708e}}

% 夏季,躺在沙滩上晒日光浴是难得的享受。不过,长时间的直射阳光会灼伤皮肤,引起日光性皮炎,而孩子的皮肤娇嫩,更容易被晒伤。皮肤被灼伤时,先出现红斑,有的甚至起大疮、很痛。皮炎好转后会脱屑,以后留下色素沉着。强烈的紫外线对视网膜也有很大的损害。

% 因此,夏季婴幼儿在日光下照射的时间每次不应超过20分钟。在晒太阳之前,应擦防晒霜,带好太阳帽(防止日射病)及太阳眼镜(防止视网膜损伤)。如果孩子玩兴很足,可以直晒10分钟后到阴凉处歇一会儿然后再晒。

% 如果已经出现日光性皮炎,那么不要立即用热水洗澡,而要用冷水湿敷皮炎处,并请医生治疗。

% %ux610fux5916}{%
% \subsubsection{2. 意外}%ux610fux5916}}

% 旅游过程中常发生的意外包括外伤(如跌伤、骨折、关节扭伤)、溺水及孩子的走失。当孩子走失之后,由于没有父母的照顾,更容易发生外伤及溺水。因此,父母要看管好孩子。记住,不要放开孩子的手,尤其在人多的地方。

% %ux4e94ux4e0dux8981ux5267ux70c8ux6447ux6643ux5b9dux5b9d}{%
% \subsection{五、不要剧烈摇晃宝宝}%ux4e94ux4e0dux8981ux5267ux70c8ux6447ux6643ux5b9dux5b9d}}

% 这个阶段的宝宝已经知道如何和父母嬉戏。年轻的爸爸高兴时会把宝宝抛向空中,再用双手接住,这种惊险的体验让宝宝和父母都很高兴。不过,这种游戏隐藏着严重的隐患,剧烈的摇晃、震荡会引起脑组织的损伤,甚至导致脑出血、视网膜剥离等,也就是宝宝震荡综合征。另外,万一父母失手,后果更不堪设想。

% 由于摇晃而导致婴儿精神不振,表情淡漠,眼神呆滞,食欲不振等症状,称为``婴儿摇晃综合征''。

% \textbf{案例一}:4个月女婴,哭闹不止,不肯入睡。小保姆就用力上下左右摇晃,从而导致颅内出血,经抢救宝宝是活了下来,但却留下了严重的后遗症―癫痫、智力低下。

% \textbf{案例二}:8个月男婴,叔叔逗他玩耍,把他在空中抛来抛去,致使硬膜下血肿而死亡。

% 国外也常有这类事情的报道。如加拿大渥太华儿童医院对49例因摇晃而导致入院的婴儿进行了调查研究,结果显示其中10\%的婴儿因过重的摇动而死亡,其余有的表现为婴幼儿的日常护理发育迟缓,有的竟出现``脑瘫''。

% \textbf{科学研究认为}:婴儿的头部相对较大,约占身体的1/5,而支撑头部的颈部肌肉力量却较弱,加之颅内小血管比较脆弱,在用力摇晃时,未发育完善的脑组织与较硬的颅骨相碰撞,极易出现小血管破裂而引起脑震荡、颅内出血,轻者发生癫痫、智力低下、肢体瘫痪,严重者出现脑水肿、脑疝而死亡。10个月以内的孩子出现颅内出血的几率最高,后果的严重程度与摇晃的强度和时间成正比。

% 很多人并不知道用力摇晃婴儿会出现如此危险的后果,希望作为家长的注意,千万不要用力摇晃和震荡小宝宝。

% %ux516dux9884ux9632ux63a5ux79cd-1}{%
% \subsection{六、预防接种}%ux516dux9884ux9632ux63a5ux79cd-1}}

% 上个月给婴儿注射了三联针(百、白、破混合制剂)的第二针,这个月应注射第三针,这样就完成了百日咳、白喉、破伤风三种疾病的第一次免疫注射过程。以后到1岁半、6\textasciitilde7岁时,还要加强两次免疫。此后,孩子便获得了对这三种病的持续而良好的免疫力。

% 大部分孩子在注射三联针后都有些不适,常表现为发热,吃奶不好,在打针的局部稍稍有些红肿,孩子比以前爱哭闹了。这些现象家长必然很担心,其实这是正常的反应,一般2\textasciitilde3天就会好的。如果体温超过了38.5℃,可以给孩子吃点退烧药。如果发热持续3日不退,并出现皮疹、咳嗽等症状,要请医生给孩子看看有无其他不适。

% %ux4e03ux6ce8ux610fux4fddux62a4ux597dux5b69ux5b50ux7684ux4e73ux7259}{%
% \subsection{七、注意保护好孩子的乳牙}%ux4e03ux6ce8ux610fux4fddux62a4ux597dux5b69ux5b50ux7684ux4e73ux7259}}

% 人的牙齿分为乳牙与恒牙,乳牙从第4\textasciitilde6个月开始萌发,到2岁半左右出齐。从6岁开始,乳牙逐渐脱落,恒牙开始陆续萌出。

% 对5个月的孩子来说,首先,要从小培养孩子的好习惯,在睡眠前不要吃带糖分的食物,因为糖类食物在口腔细菌作用下,\textbf{发酵}产生\textbf{酸性物质},这种物质腐蚀乳牙,使其\textbf{脱钙形成龋齿}。其次,要从小培养孩子的正确睡眠姿势,有的孩子睡眠喜欢偏向一侧,这样会使正常颌骨发育受到影响,会形成一侧大一侧小,影响牙齿的发育。此外,孩子叼奶头睡、吮手指等坏习惯,都会引起牙齿排列不齐,从而影响牙齿的正常发育。

% %ux516bux4e0dux8981ux5f3aux884cux5236ux6b62ux5b69ux5b50ux54edux95f9}{%
% \subsection{八、不要强行制止孩子哭闹}%ux516bux4e0dux8981ux5f3aux884cux5236ux6b62ux5b69ux5b50ux54edux95f9}}

% 婴幼儿大脑发育不够完善,当受到惊吓、委屈或不满足时,就会哭。哭可以使孩子内心的不良情绪发泄出去,通过哭能调和人体七情,所以哭是有益于健康的。

% 有的家长在孩子哭时强行制止或进行恐吓,叫孩子把哭\textbf{憋}回去。这样做使孩子的精神受到压抑,心胸憋闷,长期下去会精神不振,影响健康。当孩子哭时家长要顺其自然。孩子哭后就能情绪稳定,嬉笑如常了。

% %ux7b2cux516dux828256ux4e2aux6708ux5a74ux513fux7684ux65e5ux5e38ux62a4ux7406}{%
% \subsection{06第六节5〜6个月婴儿的日常护理}%ux7b2cux516dux828256ux4e2aux6708ux5a74ux513fux7684ux65e5ux5e38ux62a4ux7406}}

% %ux4e00ux4e0dux5fc5ux62c5ux5fc3ux5a74ux513fux7684ux53e3ux6c34ux591a}{%
% \subsection{一、不必担心婴儿的口水多}%ux4e00ux4e0dux5fc5ux62c5ux5fc3ux5a74ux513fux7684ux53e3ux6c34ux591a}}

% 孩子6个月左右,由于出牙的刺激,唾液分泌增多。而孩子又不能及时咽下,就会出现流口水的现象,这是一种正常现象。这时要注意给围嘴,并经常洗换,保持干燥。不要用硬毛巾给孩子擦嘴、擦脸,而要用柔软干净的小毛巾或餐巾纸来擦。

% \hspace{0pt}\includegraphics[width=2.90909in,height=2.72727in]{media/rId1304.png}\hspace{0pt}

% 孩子在出牙时,除流涎外,还会出现咬奶头现象,个别孩子还会出现低烧,这都是正常现象,家长不必担心。

% %ux4e8cux6ce8ux610fux9884ux9632ux8d2bux8840}{%
% \subsection{二、注意预防贫血}%ux4e8cux6ce8ux610fux9884ux9632ux8d2bux8840}}

% 婴儿出生6个月之后,从母体得来的造血物质基本用完,若补充不及时,就易发生贫血。6个月孩子最常见的是缺铁性贫血和营养性大细胞性贫血。

% 缺铁性贫血是由于体内贮存的铁缺乏,使血红蛋白合成减少。常见的缺铁原因有以下几种:

% \begin{enumerate}
% \def\labelenumi{\arabic{enumi}.}
% \item
%   \textbf{先天性储铁不足}:由于胎儿储铁以产前3个月最多,所以早产、双胎、母亲贫血严重时,都会使新生儿储铁减少。
% \item
%   \textbf{生长发育过快}:生长发育越快,体重增加越多,身体缺血量也就越多,对造血原料铁的需要也就增多。因而生长发育过快的孩子容易发生缺铁。
% \item
%   \textbf{饮食中铁缺乏}:因为母乳、牛奶含铁很少,不够孩子生长发育的需要,所以,单纯喂奶而不增加辅食的孩子特别容易发生缺铁性贫血。
% \item
%   \textbf{疾病的影响}:孩子患某些疾病可造成身体缺铁。如孩子消化道畸形、长期腹泻,铁便不能很好地被吸收,易发生贫血。再如肠息肉、美克尔憩室、钩虫病等由于肠道经常少量失血,也会引起贫血。当孩子精神不好,食欲差,经常疲乏无力时,应观察孩子面色、口唇、牙床、皮肤黏膜是否苍白,若是,应想到孩子贫血,及时到医院检查。一经证实,就要坚持耐心按医嘱服药。家长了解了贫血的原因,就会认识合理喂养的重要性,及时给孩子添加辅食,多吃动物肝、瘦肉、鸡蛋、绿色菜等防治贫血。营养性大细胞性贫血是由于缺乏维生素B12和叶酸这些造血物质而引起的疾病。
% \end{enumerate}

% 维生素B12在动物瘦肉、肝、肾中含量较多,在奶类、蛋类中含量较少。叶酸在新鲜绿叶菜、酵母、肝、肾中含量较多。人工喂养、单纯母乳喂养不添加辅食或孩子饮食单调,缺乏肉类和各种蔬菜,容易发生营养性大细胞性贫血。不管是什么性质的贫血,都会引起孩子肝、脾、淋巴结肿大,心脏扩大,重者还会发生心功能不全。贫血还严重影响孩子的生长发育,所以,必须认真防治孩子贫血。

% %ux4e09ux6ce8ux610fux9884ux9632ux4f5dux507bux75c5}{%
% \subsection{三、注意预防佝偻病}%ux4e09ux6ce8ux610fux9884ux9632ux4f5dux507bux75c5}}

% \hspace{0pt}\includegraphics[width=2.7972in,height=2.75524in]{media/rId1313.png}\hspace{0pt}

% 当孩子体内缺乏维生素D时,会产生钙、磷代谢失常、骨样组织钙化障碍,引起一系列症状。患佝偻病的孩子夜间睡眠不稳,容易惊醒,并且多汗。由于酸性汗液刺激皮肤,造成孩子头部来回摆动摩擦枕部,使头后形成一圈脱发,医学上叫枕秃。较严重的佝偻病,颅骨出现软化,用手按上去,似乒乓球一样;逐渐出现方颅,胸廓下部肋骨呈现外翻。当孩子学走路时,由于骨骼软而吃力,致使腿部弯曲,形成O型或``X''型腿。有的还可出现脊柱弯曲等症状。患有佝偻病的孩子,走路、说话、长牙齿都比正常孩子要晚。

% 预防佝偻病的方法,首先是多给孩子晒太阳,6个月以后的孩子每天在户外活动的时间应越来越长,即使在冬天也要注意户外锻炼,让孩子接触阳光,同时还应坚持继续服用钙片和鱼肝油。已经患有佝偻病的孩子应根据医嘱使用维生素D制剂。

% %ux56dbux5b9dux5b9dux7684ux5934ux53d1ux7a00ux5c11ux600eux4e48ux529e}{%
% \subsection{四、宝宝的头发稀少怎么办}%ux56dbux5b9dux5b9dux7684ux5934ux53d1ux7a00ux5c11ux600eux4e48ux529e}}

% 刚出生的宝宝头发稀只是一个暂时的生理现象。

% 头发的多少是有个体差异的,有的新生儿头发稀疏,但到了1岁左右头发就会逐渐长出,2岁时头发就已长得相当多了,5\textasciitilde6岁时头发就会和其他宝宝一样浓密而乌黑。

% 由于宝宝头发稀疏,许多父母就不敢给宝宝洗头发,担心会使原本就稀少的头发脱落,造成头发更少。实际上,在洗发过程中脱落的头发本身就是衰老而自动脱掉的,倘若长期不洗头,就会使油脂、汗液等分泌物以及污染物刺激头皮,引起头皮发痒、起疱,甚至继发感染,这样倒使头发脱掉;还有的妈妈以为将宝宝的头发剃光,便可加速头发生长,这种方法也不可取。个别妈妈盲目地在宝宝头皮上涂擦``生发精''、``生发灵''之类的药物,希望宝宝能长出浓密的头发,须知这类药物不适用于婴幼儿稚嫩的头皮,甚至还会带来不良后果。

% %ux53d1ux73b0ux5b9dux5b9dux5934ux53d1ux7a00ux758fux65f6ux5e94ux5f53ux600eux4e48ux529e}{%
% \subsubsection{发现宝宝头发稀疏时应当怎么办}%ux53d1ux73b0ux5b9dux5b9dux5934ux53d1ux7a00ux758fux65f6ux5e94ux5f53ux600eux4e48ux529e}}

% \textbf{勤洗头}\\
% 经常为宝宝洗头,保持头发清洁卫生,使头皮得到刺激,才能促进头发生长。洗头时,必须选用婴儿专用洗发液,洗时轻轻\textbf{按摩头发},\textbf{不要揉搓头发},以防止头发纠结在一起,然后用清洁的温水冲洗干净。

% \textbf{勤梳发}\\
% 为宝宝梳理头发时,选择使用\textbf{橡胶梳子},这种梳子有弹性,很柔软,不会损伤宝宝的头皮。梳理时要按宝宝头发自然生长的方向梳理,不可强梳到一个方向。

% \textbf{营养充足}\\
% 充足而全面的营养,对宝宝的头发发育非常重要,及时按月龄让宝宝多摄入蛋白质、维生素A、维生素B、维生素C及富含矿物质的食物,这样可以通过血液循环供给毛根,使头发长得更结实,更秀丽。

% \textbf{多晒太阳}\\
% 适当的阳光照射和新鲜空气,对宝宝头发的生长大有裨益。紫外线的照射既有利于杀菌,又可以促进头皮的发育和头发的生长。

% \begin{quote}
% 温馨提示

% 以上措施,对头发浓密的宝宝同样重要,有些婴儿出生后头发很密,但几个月之后枕部头发逐渐磨掉、脱落,出现``枕秃'',同时还伴有多汗、烦躁、哭闹等症状,则可能是身体内缺钙和缺维生素D所引起的佝偻病的表现。出现这种情况,应及时到医院了解原因,对症治疗。
% \end{quote}

% %ux4e94ux8ba4ux8bc6ux5b9dux5b9dux7684ux4f53ux6001ux8bedux8a00}{%
% \subsection{五、认识宝宝的体态语言}%ux4e94ux8ba4ux8bc6ux5b9dux5b9dux7684ux4f53ux6001ux8bedux8a00}}

% %ux4e2aux6708ux4ee5ux524dux5b9dux5b9dux901aux5e38ux8868ux73b0ux7684ux4f53ux6001ux8bed}{%
% \subsubsection{1.
% 6个月以前宝宝通常表现的体态语}%ux4e2aux6708ux4ee5ux524dux5b9dux5b9dux901aux5e38ux8868ux73b0ux7684ux4f53ux6001ux8bed}}

% \begin{enumerate}
% \def\labelenumi{\arabic{enumi}.}
% \item
%   \textbf{张嘴而笑,表示兴奋愉快}:宝宝笑的形态是突然发出的,短暂而快速,口角牵动、笑容骤现,并伴随着满目发光、两手晃动,接着笑容立即停止,\textbf{等候亲脸鼓励}。这时,父母应笑脸相迎,用手轻轻抚摸宝宝的面颊,或在其面、额部亲吻一下,以示鼓励。此时此刻,宝宝会再以微笑来对父母的行动表示满意。\\
%   \hspace{0pt}\includegraphics[width=2.51748in,height=2.32168in]{media/rId1319.png}\hspace{0pt}
% \item
%   \textbf{撇嘴,表示提出要求}:宝宝撇起小嘴,好像受到委屈,也是啼哭的先兆,而实际上是对成人有所要求,比如肚子饿了要吃奶,寂寞了要人逗乐,厌烦了要大人抱起来换个环境或改变一种姿势。父母必须细心观察宝宝的要求,适时地满足宝宝的需要。
% \item
%   \textbf{撅嘴、咧嘴,表示小便的信号}:据研究,一般男婴以撅嘴来表示小便,女婴多以咧嘴或上唇紧含下唇来表示小便。父母如果能及时观察到宝宝的嘴形变化,了解要小便时的表情,就能摸清宝宝小便的规律,从而加以引导,有利于逐步培养孩子的自控能力和良好的习惯。
% \item
%   \textbf{红脸横眉,表示大便的信号}:宝宝先是眉筋突暴,然后脸部发红,而且目光发呆,有明显的``内急''反应,这是大便的信号。这时父母应立即让宝宝坐便盆,以解决``便急''之需。\\
%   \hspace{0pt}\includegraphics[width=2.41958in,height=2.32168in]{media/rId1325.png}\hspace{0pt}
% \item
%   \textbf{眼神无光,提醒父母要警惕}:健康宝宝的眼睛总是明亮有神,转动自如。如果发现宝宝眼神黯然无光,呆滞少神,很可能是宝宝身体不适,有疾病的先兆。这时,父母要特别细心地注意宝宝的身体情况,发现疑问及时去医院检查,及早采取保健医疗措施。
% \item
%   \textbf{玩弄舌头、嘴唇吐气泡,表示自己会玩}:大多数宝宝在吃饱、换了干净尿布,而且还没有睡意时,自得其乐地玩弄自己的嘴唇、舌头、吐气泡、吮手指等。这时,宝宝喜欢独自长时间地玩,成人不要去干扰他。
% \end{enumerate}

% 以上是6个月以前宝宝通常表现的体态语。6个月以后的宝宝,由于感知能力和动作能力的发展与增强,除了用面部表情代替语言来表示自己的意愿之外,还伴有各种动作的体态语来表达自己的思想感情,随着月龄的增长而有不同的表现。

% %ux4e2aux6708ux4ee5ux540eux5b9dux5b9dux901aux5e38ux8868ux73b0ux7684ux4f53ux6001ux8bed}{%
% \subsubsection{2.
% 6个月以后宝宝通常表现的体态语}%ux4e2aux6708ux4ee5ux540eux5b9dux5b9dux901aux5e38ux8868ux73b0ux7684ux4f53ux6001ux8bed}}

% \begin{enumerate}
% \def\labelenumi{\arabic{enumi}.}
% \item
%   6个月时,宝宝会张开双臂,身体扑向亲人,要求搂抱、亲热,如果陌生人想要抱,则转头将脸避开,表示不愿与陌生人交往。\\
%   \hspace{0pt}\includegraphics[width=2.65734in,height=2.43357in]{media/rId1332.png}\hspace{0pt}
% \item
%   7-8个月时,宝宝会以``拍手''和笑脸表示高兴,在父母教导下会以``点头''表示谢谢,对不爱吃的食物避开,并以``摇头''表示拒绝。
% \item
%   9-10个月时,宝宝会用小手指向去那里,或用小手拍拍头,表示要戴帽子带他出去。
% \item
%   11-12个月时,宝宝除了以面部表情和动作来表示体态语言外,还会伴有各种声音,比如哪嚇声(表示汽车),嘎嘎声(表示小鸭),以及用简单的单词音来表示自己的意愿。
% \end{enumerate}

% %ux516dux78e8ux7259ux5e8a}{%
% \subsection{六、磨牙床}%ux516dux78e8ux7259ux5e8a}}

% 6个月的孩子抓到物品后,喜欢放在嘴里啃,这为他日后自己进食打下基础。妈妈要鼓励他,不要见他往嘴里放以为不卫生就呵斥他,而是积极为他创造条件:经常给他把手洗干净,给他一些饼干、水果片、馒头干,这些食物可以帮他摩擦牙床。

% 孩子有了这种爱好以后,妈妈要检查一下他的用品和玩具:

% \begin{enumerate}
% \def\labelenumi{\arabic{enumi}.}
% \item
%   婴儿玩具要经常刷洗,保持卫生。
% \item
%   不让孩子玩涂漆的、有锐边的铁玩具,如小铲、小汽车等。
% \item
%   给婴儿买软、硬不同的,不同质地的玩具。
% \item
%   不要让他拿到直径2厘米以下的小物品,以免他将小物品塞入口中。
% \end{enumerate}

% %ux4e03ux957fux7259ux540eux7684ux62a4ux7406}{%
% \subsection{七、长牙后的护理}%ux4e03ux957fux7259ux540eux7684ux62a4ux7406}}

% 6个月的孩子已长牙了,先长下面的门牙,再长上面的门牙。有的孩子长得晚一点,并不能说明身体有什么病。刚长出的牙还不能吃饭用,因此不能给孩子硬食,但咬起母亲的乳头来还是很厉害,妈妈不要让孩子含着乳头睡觉。

% 小儿在6个月左右开始长牙。乳牙萌出前几天孩子可能会有一些异常的表现,如哭闹、口涎增多、喜欢咬手指和硬的东西、睡眠不好、食欲减退等,有的还有低热、轻度腹泻、局部牙龈充血或肿大。一般来说,以上现象持续3-4天,乳牙就穿破牙龈萌出了。

% 这个时期的口腔保健主要由母亲来完成。在喂奶以后和晚上睡觉前,母亲用纱布蘸温水轻轻地擦洗孩子的口腔黏膜、牙龈和舌面,除去附着在这些部位的乳凝块,达到清洁口腔的目的。妈妈在做这种口腔擦洗前应认真地洗手,把指甲剪断,擦洗的时候动作要轻柔,不要损伤孩子的口腔黏膜。

% 这种哺乳外的口腔刺激,可以使母亲对孩子口腔内乳牙萌出的情况有及时的了解,对小儿的牙龈形态有所认识,同时也可以增强小儿大脑的感受性。

% %ux516bux5750ux5a74ux513fux8f66}{%
% \subsection{八、坐婴儿车}%ux516bux5750ux5a74ux513fux8f66}}

% 6个月的孩子会坐了,可以经常坐在婴儿车里出去玩。带孩子出去散步,妈妈要注意尽量走平坦的路,\textbf{不要太过颠簸}。在购车时,要买车轮大些,座位高些的车,有的车座位很低,\textbf{孩子离地面太近,很不卫生}。【低了的缺点】

% %ux4e5dux77e5ux9053ux8ba4ux751f}{%
% \subsection{九、知道认生}%ux4e5dux77e5ux9053ux8ba4ux751f}}

% \hspace{0pt}\includegraphics[width=2.23776in,height=2.53147in]{media/rId1348.png}\hspace{0pt}

% 6个月以前的孩子不认生,这是因为婴儿还不能分辨人,只要是在他身边的人,对他友好的人,谁抱他都可以。过了6个月孩子就认人了,他对母亲更加依恋,不喜欢陌生人抱他,也不喜欢陌生的环境,他知道怕人,见生人对他有威胁,他会哭。如果带孩子出去,在晚上他可能不睡觉,他感到紧张,即使是到很亲近的人家也是如此。

% %ux5341ux7406ux89e3ux5b9dux5b9dux7684ux4f9dux604bux884cux4e3a}{%
% \subsection{十:理解宝宝的依恋行为}%ux5341ux7406ux89e3ux5b9dux5b9dux7684ux4f9dux604bux884cux4e3a}}

% %ux4f9dux604bux53d1ux5c55ux7684ux56dbux4e2aux9636ux6bb5}{%
% \subsubsection{1.
% 依恋发展的四个阶段}%ux4f9dux604bux53d1ux5c55ux7684ux56dbux4e2aux9636ux6bb5}}

% 第一阶段:从宝宝出生到一两个月,宝宝会对任何人做出反应。

% 第二阶段:从宝宝2个月到7个月,宝宝更偏爱熟悉的人,当父母离开时,宝宝通常还能从其他较熟的人那里获得安慰。

% 第三阶段:从7个月开始,持续到2岁或2岁半,这期间依恋是最强的,父母的离开会给宝宝造成很大的烦恼。

% 第四阶段:宝宝开始理解父母的情感和动机,他们之间建立起一种伙伴关系,尽管这个时期依恋关系仍然很强,但宝宝对父母离开不会再害怕了。

% %ux4f9dux604bux5728ux5b9dux5b9dux6210ux957fux4e2dux7684ux91cdux8981ux4f5cux7528}{%
% \subsubsection{2.依恋在宝宝成长中的重要作用}%ux4f9dux604bux5728ux5b9dux5b9dux6210ux957fux4e2dux7684ux91cdux8981ux4f5cux7528}}

% 依恋具有生物学作用,在人类进化过程中,宝宝与父母的依恋关系确保了婴儿的生存。某些特定的刺激(人的面孔、人的声音、陌生的物体)可以引发婴儿的特定行为(微笑、警惕地环视四周、哭泣)。宝宝的行为进而会使成人产生互补的行为,如宝宝的微笑可以让成人回馈以微笑,深深地吸引了成人的注意力。

% 依恋以其特有的方式帮助宝宝拥有了基本的认知方面的技巧,使宝宝行为与环境协调运作,并通过四种行为系统来完成:

% \begin{enumerate}
% \def\labelenumi{\arabic{enumi}.}
% \item
%   依恋行为系统:促使依恋关系的形成。
% \item
%   恐惧警惕系统:帮助宝宝回避那些会危及生命的人、物或环境。这个系统也就是我们熟悉的宝宝对陌生人的警惕。
% \item
%   参与行为系统:一旦对陌生人的恐惧被克服后,宝宝就会有勇气与家庭以外的人进行交流和接触。这个系统促使宝宝在社会领域的拓展,对于宝宝进入人类这个社会性群体是极为必要的。
% \item
%   探索行为系统:可以使宝宝去探索周围环境。若成长中的宝宝想获得生存竞争能力,探索环境是必要的步骤。
% \end{enumerate}

% 对于宝宝来说,存在着一个值得信任并且可靠的依恋对象可以给宝宝提供情感方面的安全,若依恋对象失去了或不可靠,宝宝就会启动恐惧警惕系统。通过探索,宝宝不仅可以从外界获得新的知识和能力,探索的意义还在于宝宝反过来还会把来自外界的收获运用于与自己熟悉的慈爱的人的交往上。

% %ux4ebaux683cux7684ux5f62ux6210}{%
% \subsubsection{3.人格的形成}%ux4ebaux683cux7684ux5f62ux6210}}

% 建立了安全依恋关系的宝宝面对问题时,不仅有热情,也有毅力。当宝宝遇到无法解决的难题时,很少乱发脾气,愿从父母那里获得倾心的帮助。这样的宝宝长大后,则更可能成为领导者,善于主持游戏活动,成为他人愿意结交的伙伴,对其他人的痛苦有同情心。这样的宝宝有更强的自我决定能力,对新事物表现出更大的好奇心,更喜欢学习新技能,追求目标时更有魄力,不会轻言放弃。这并不是说早期没有建立良好的依恋关系就会对儿童的发展起到不可逆转的决定作用,或许有些可以在成长过程中,能够通过其他方式修复当初由于缺乏安全的依恋关系所造成的不良影响。但可以肯定的是为此所付出的代价一定不菲。

% %ux7b2cux4e03ux828267ux4e2aux6708ux5a74ux513fux7684ux65e5ux5e38ux62a4ux7406}{%
% \subsection{07第七节6〜7个月婴儿的日常护理}%ux7b2cux4e03ux828267ux4e2aux6708ux5a74ux513fux7684ux65e5ux5e38ux62a4ux7406}}

% %ux4e00ux4e0dux8981ux7ed9ux5b69ux5b50ux76d6ux539aux88ab}{%
% \subsection{一、不要给孩子盖厚被}%ux4e00ux4e0dux8981ux7ed9ux5b69ux5b50ux76d6ux539aux88ab}}

% 如果孩子在夜间睡着了之后总是踢被,家长应该注意不要给孩子盖得太多、太厚。特别是在孩子刚入睡时,更要少盖一点,等到夜里冷了再盖。稍微盖薄一些,孩子不会冻坏,盖得太厚,孩子感觉燥热,踢掉了被子,反而容易着凉感冒。

% %ux4e8cux4e0dux8981ux7ed9ux5b69ux5b50ux7a7fux5f97ux592aux6696}{%
% \subsection{二、不要给孩子穿得太暖}%ux4e8cux4e0dux8981ux7ed9ux5b69ux5b50ux7a7fux5f97ux592aux6696}}

% 孩子穿的衣服薄厚也应适宜。穿得太少,孩子的手、脚都发凉,容易生病;穿得太多,活动起来不方便,一动就会出汗,出汗之后,再一受风更容易着凉。俗话说,``\textbf{要想孩子安,三分饥和寒}''。也就是说,要想让孩子平安不生病,只需要吃七分饱,穿七分暖就行了,若吃得过饱,穿得过暖,反而容易生病。

% %ux4e09ux5b69ux5b50ux7231ux51faux6c57ux7684ux539fux56e0}{%
% \subsection{三、孩子爱出汗的原因}%ux4e09ux5b69ux5b50ux7231ux51faux6c57ux7684ux539fux56e0}}

% 孩子比较爱出汗,这是因为孩子体内的新陈代谢旺盛,产热多。出汗是体内散热的主要方式,再加上孩子神经系统发育不完善,调节功能差,因此爱出汗。

% 如果孩子只是出汗多,但精神、面色、食欲均很好,吃、喝、玩、睡都正常,一般就不是有病。但患有活动性佝倭病、结核病和其他神经血管疾病以及慢性消耗性疾病的孩子汗多,特别是夜间入睡后出汗多,同时伴有其他症状,如低烧、食欲缺乏、睡眠不稳、易惊等,应该去医院检查,找出病因,及时治疗。

% %ux56dbux7528ux6d17ux53d1ux6db2ux7c7bux7269ux54c1ux65f6ux5e94ux6ce8ux610f}{%
% \subsection{四、用洗发液类物品时应注意}%ux56dbux7528ux6d17ux53d1ux6db2ux7c7bux7269ux54c1ux65f6ux5e94ux6ce8ux610f}}

% 随着科学的发展,各式各样的洗发用品琳琅满目、层出不穷,用起来香味飘溢。但是,任何洗发用品都有含碱性的化学物质。有的人对某种化学物质过敏,当使用这种洗发液时就会出现痒感;有的人洗发时不小心,把洗发水弄到眼睛里,结果出现眼睛疼痛、流泪、怕光、不敢睁眼等症状,检查眼睛时可发现角膜被损伤,进一步发展将影响角膜的透明度,出现混浊,影响视力。所以在使用这些物品时,千万不要弄到孩子眼内,如不小心进了眼内,要立即用清水冲洗干净,以免化学品长时间刺激眼组织,引起眼损伤,如遇到此情况时,应上医院治疗。

% %ux4e94ux5f53ux5b9dux5b9dux53d1ux751fux610fux5916ux65f6}{%
% \subsection{五、当宝宝发生意外时}%ux4e94ux5f53ux5b9dux5b9dux53d1ux751fux610fux5916ux65f6}}

% \begin{enumerate}
% \def\labelenumi{\arabic{enumi}.}
% \item
%   流血
% \end{enumerate}

% 可以用清洁的棉花或纱布直接压在伤口上,或用手指紧压最靠近伤口的大动脉。如果一直不停地流血,甚至用布带紧压伤口还是止不住,就要立刻把宝宝送进医院或诊所了。

% \begin{enumerate}
% \def\labelenumi{\arabic{enumi}.}
% \setcounter{enumi}{1}
% \item
%   流鼻血
% \end{enumerate}

% 可以用凉毛巾敷在宝宝的\textbf{额}头上,然后用手轻压鼻梁,让宝宝的头颈向下而不是后仰,这样才能将流出的血吐出,不至咽下。

% \begin{enumerate}
% \def\labelenumi{\arabic{enumi}.}
% \setcounter{enumi}{2}
% \item
%   轻微跌伤
% \end{enumerate}

% 首先应该检查伤口是否出血,情况如何,倘若有轻微的擦伤,洗干净后可用创可贴进行简单包扎。

% \begin{enumerate}
% \def\labelenumi{\arabic{enumi}.}
% \setcounter{enumi}{3}
% \item
%   严重跌伤
% \end{enumerate}

% 让宝宝平躺着,检查其四肢有没有屈曲、变形和疼痛的反应。如果怀疑骨折,就不能随便移动宝宝跌伤的肢体,而应尽量固定好,立刻去医院。倘若宝宝头部受了伤,应进行24小时观察,留意以下征兆:

% \begin{enumerate}
% \def\labelenumi{\arabic{enumi}.}
% \item
%   耳鼻溢液流出;
% \item
%   恶心、呕吐;
% \item
%   剧烈头痛;
% \item
%   视力模糊;
% \item
%   四肢平衡失调;
% \item
%   昏迷、丧失知觉。
% \end{enumerate}

% 若有以上症状出现,则可能是脑部受伤,需立刻送医院治疗。

% %ux516dux4e86ux89e3ux5b9dux5b9dux7684ux4e0dux540cux4e2aux6027}{%
% \subsection{六、了解宝宝的不同个性}%ux516dux4e86ux89e3ux5b9dux5b9dux7684ux4e0dux540cux4e2aux6027}}

% 宝宝从诞生时起,就具有鲜明的、互不相同的个性。身体发育自不必说,精神和气质也各不相同。吃奶的样子、哭泣的样子、闹人的方式,一个宝宝一个样儿。

% 有的宝宝很难从沉睡中被唤醒,有的则直接从睡梦中惊醒,号啕大哭。有些宝宝很容易镇定,大人抚慰时不需要很多精力;有些则很不容易镇定,在他哭闹时,仅用一种方法如轻摇、拥抱或喂奶是无法使其镇定的。有些宝宝非常安静,很少哭闹;而有的宝宝从生下第一天开始就是个``哭虫''。\textbf{非常好动}的宝宝在吮奶时啧啧有声,``贪婪地''大口大口地吞咽,然后再吐出一些来,而安静的宝宝吮奶时很平静,吃完后又立即悄然入睡\ldots\ldots{}

% 体现出每个宝宝不同特点的个性,形成了宝宝各自生活的格调,而这种格调又决定着身心发展的情况。之所以强调婴儿的个性,是因为担心父母如果不了解这种个性,从零岁开始的教育就可能以失败告终。

% 所有的母亲都必须特别注意宝宝在身体和精神上的满足感。充足而愉快地哺乳、安静地睡眠、顺利地排便、舒适地皮肤接触、适当地刺激等,能使宝宝身心安定,从而保持健康的状态。如果身心健康,宝宝的日常生活就会有规律,相应的,母亲的生活节奏也会和谐起来。

% 零岁是宝宝感觉机能和运动机能形成基础,思想意识出现萌芽的时间。如果父母采取那种给予宝宝身心以满足感的接触方式,宝宝便会迅速而健康地成长起来。

% 做父母的要了解宝宝的个性,要把自己的孩子与别人的孩子加以比较,掌握同龄儿童之间有什么不同。所有的育儿书籍\textbf{仅能}提供有关婴幼儿发育的一些基本常识,那些叙述仅是一种普遍性,而不是对每个宝宝而言的,不要期望其中的种种现象都与自己的孩子相似。教育自己的宝宝的最好方法是\textbf{研究其本身},掌握自己孩子的生理、心理特点,把握孩子的发展趋势,并采用适合宝宝个性的科学喂养方法精心培育。

% %ux7b2cux516bux828278ux4e2aux6708ux5a74ux513fux7684ux65e5ux5e38ux62a4ux7406}{%
% \subsection{08第八节7\textasciitilde8个月婴儿的日常护理}%ux7b2cux516bux828278ux4e2aux6708ux5a74ux513fux7684ux65e5ux5e38ux62a4ux7406}}

% %ux4e00ux600eux6837ux5bf9ux5f85ux79bbux4e0dux5f00ux5988ux5988ux7684ux5b9dux5b9d}{%
% \subsection{一、怎样对待离不开妈妈的宝宝}%ux4e00ux600eux6837ux5bf9ux5f85ux79bbux4e0dux5f00ux5988ux5988ux7684ux5b9dux5b9d}}

% 一般来说,宝宝出生1周以后就能盯住人的眼睛看了,6周以后就会对人微笑了。在四五个月时,已经能够认识父母,但是,宝宝并不在意父母的离开。其实,这时感到舍不得离开的是父母而不是婴儿。到了8个月时,宝宝虽然会对最常出现在身边的人表示喜爱,可无论那个人是否在身边,宝宝都不会感到不安。对宝宝来说看不见的东西就不存在。

% 从这时开始宝宝就能够意识到身边的人走开了,以及走开的是谁,能够辨认熟人的面孔并开始对陌生人产生戒心,开始不信任生人并且渴望从父母那里寻求保护,特别是当宝宝看见陌生人或到了一个陌生地方的时候。

% %ux542fux8499ux5b9dux5b9dux72ecux7acbux610fux8bc6}{%
% \subsubsection{1.启蒙宝宝独立意识}%ux542fux8499ux5b9dux5b9dux72ecux7acbux610fux8bc6}}

% \textbf{让宝宝有属于自己的地方}

% 婴儿的小床是最早专门属于宝宝自己的地方,这里对宝宝有特殊的意义,是启蒙独立意识的起点。

% \textbf{做游戏}

% 对婴儿来说,不在眼前的东西就是``不存在''。可以和宝宝做一个游戏,用手蒙住脸,使宝宝看不见妈妈,然后再把手移开。妈妈再露面时宝宝会惊喜的。5个月左右的宝宝很喜欢这个游戏。这个游戏能够使宝宝感到愉快并能训练宝宝对外部事物的信心,同时也能使宝宝形成\textbf{妈妈会回来}的意识。

% \textbf{藏猫猫}:宝宝长大一些时,妈妈和宝宝玩``藏猫猫''的游戏,并逐渐延长躲起来的时间,甚至可以走得远一点。这种游戏会通过玩来帮助宝宝适应与妈妈分开。但应注意躲起来的时间不要太长,也不要走得太远,千万不能使宝宝感到害怕,以免给宝宝心灵留下阴影。

% \textbf{为宝宝布置玩耍的环境}

% 当宝宝学会爬或走以后,宝宝能爬向妈妈或走到妈妈身边,这是很有益的活动。因此必须把房间\textbf{安排得很安全},这样当妈妈没有目不转睛地盯着宝宝时,也可以自己活动一下,已经能让不满1岁的孩子自己跑开多达一两分钟,当宝宝又跑到妈妈跟前时,妈妈要很高兴地欢迎宝宝并和宝宝玩一会儿。

% \textbf{出生后即开发社交能力}

% 妈妈和宝宝不可能总呆在一起,应让宝宝学会并且知道,尽管妈妈出门离开了,但妈妈会回来的。从小就让宝宝和其他成年人及小朋友多交往,以便发展宝宝的社交能力。可以让宝宝在襁褓中就经常接触生面孔,这样的宝宝自主能力较早出现,能力也较强,父母离开时,很少感到害怕。

% %ux7269ux8272ux5408ux9002ux7684ux4fddux59c6}{%
% \subsubsection{2.
% 物色合适的保姆}%ux7269ux8272ux5408ux9002ux7684ux4fddux59c6}}

% 当妈妈离开时一定要把宝宝留给一个会照料孩子并且爱孩子的人。在宝宝看到妈妈要离开而大哭时,妈妈\textbf{不要表现过分不安},因为妈妈的\textbf{神情会影响到宝宝},不要感到对不起孩子。但是,妈妈要记住教宝宝做到离开他时,宝宝能高高兴兴地同妈妈再见,让妈妈放心地离开他。

% \textbf{让宝宝和保姆熟悉}

% 如果是一位新保姆,可以请她在妈妈需要离开宝宝的早些时候来,这时妈妈也在场,指导她们彼此适应,当妈妈感到放心时,在保姆和宝宝玩的时候,可以试着离开一会儿。

% \textbf{保持宝宝的习惯}

% 告诉保姆宝宝平时的起居、玩耍规律以及宝宝的爱好,做到妈妈不在家,宝宝的生活规律也不会改变。

% \textbf{趁宝宝玩时离开}

% 趁宝宝不注意时离开,这样做比较好。比如宝宝正忙着做游戏或吃饭时,注意力不在妈妈身上,妈妈可以在这时离开。宝宝正在缠着保姆和他做游戏也是妈妈离开的好机会。

% \textbf{不要不辞而别}

% 外出时悄悄地溜走对宝宝是不公平的,这会使宝宝非常不安。有些宝宝喜欢两次告别,他们对妈妈说再见后,还要站在门口向妈妈挥手告别。另外一些宝宝发明出自己的告别程序,例如,每次和爸爸妈妈分手时都要郑重其事地拥抱和亲吻他们。

% %ux4e3aux5b9dux5b9dux53bbux65b0ux5730ux65b9ux505aux597dux51c6ux5907}{%
% \subsubsection{\texorpdfstring{3.
% \textbf{为宝宝去新地方做好准备}}{3. 为宝宝去新地方做好准备}}%ux4e3aux5b9dux5b9dux53bbux65b0ux5730ux65b9ux505aux597dux51c6ux5907}}

% 倘若要把宝宝托给别人照料,事先要让宝宝先熟悉一下环境。如果我们知道要去哪儿,以及去那里后的情况,我们就会感到心里有数,宝宝也是如此。

% \textbf{事先访问}

% 在把宝宝送走之前,先带宝宝去做一次适应性访问,让宝宝熟悉一下环境,再让宝宝同将要照看他的人玩一玩。

% \textbf{向将要照看宝宝的人介绍情况}

% 必须向准备看管宝宝的人详细介绍宝宝的所有情况,如宝宝的喜好、个性、习惯等。

% \textbf{给宝宝找个伙伴}

% 可能的话,帮助宝宝结识一个将要和他上同一个托儿所或幼儿园的小伙伴。

% 同宝宝谈谈将要发生的事情:告诉宝宝在新住处会碰到哪些情况,将在那里看见什么和做些什么事情。

% \textbf{使事情显得不同寻常}

% 把宝宝去托儿所或去幼儿园当作一件了不起的美事,夸奖宝宝穿戴得多么漂亮,给宝宝一些新的用品,使宝宝感到去这个新地方很有趣。

% %ux7b80ux77edux7684ux544aux522b}{%
% \subsubsection{4. 简短的告别}%ux7b80ux77edux7684ux544aux522b}}

% 当离开宝宝时不要犹豫或感到担心和内疚,对宝宝说过再见后就迅速离开,\textbf{不要让宝宝感到离开是一件可以商量的事情},或者觉得只要哭闹得足够厉害,妈妈就会改变主意,留下来陪他。

% \textbf{内心明白}

% 如果宝宝因为正玩得高兴而对妈妈的离开表现得无所谓,妈妈不要觉得受了伤害,妈妈应该很自信,知道宝宝是爱妈妈和信任妈妈的。

% \textbf{安慰但不迁就宝宝}

% 如果宝宝一想到妈妈要离开,就时而表现出焦虑不安,则应该和宝宝好好谈一谈,告诉他你很理解他,但妈妈必须要走。可以非常平静地对宝宝说:``我知道你现在很不安,但你一会儿就会好的,你会和别人玩得很开心。我爱你,我会来看你的。''说完后立即离开,例如,开车送宝宝去托儿所或幼儿园,宝宝不肯下车,妈妈提出两种方案供他选择,妈妈对他说:``要么我送你进去,要么你和其他孩子一起进去,你自己决定吧。''给宝宝5秒钟的考虑时间,然后按说过的话去做。

% \textbf{对付"第二天的忧郁"}

% 有时候,妈妈第一天离开宝宝时,宝宝并不在意,但是,第二天就不愿意让妈妈离开了。这并不是因为他离开你后过得不愉快,而是因为意识到要和妈妈分开,只要妈妈每次都很自信地同他告别,宝宝的感觉也就好了。

% %ux57f9ux517bux5b69ux5b50ux7684ux72ecux7acbux6027}{%
% \subsubsection{5.
% 培养孩子的独立性}%ux57f9ux517bux5b69ux5b50ux7684ux72ecux7acbux6027}}

% \textbf{离开宝宝时给宝宝多多的爱}

% 如果宝宝抱住妈妈不放,不要断定宝宝已经被``惯坏了'',宝宝这样做仅仅是想得到更多的安全感,只要宝宝的独立性在逐渐发展,这一切都是正常的。平时要更多地同宝宝在一起,如果妈妈离开家时他没有哭闹,妈妈要好好地亲吻和拥抱宝宝。

% \textbf{鼓励宝宝的独立精神}

% 不要事事都替宝宝操心,而是让宝宝尽量自己照顾自己,当宝宝确实做不好时才帮助一下,要尽可能鼓励宝宝自己完成,并表扬他的努力和成功。但是,如果宝宝未能完成一件事情,妈妈绝不要表现出失望或生气。

% \textbf{奖励}

% 告诉宝宝,如果能够乖乖地向妈妈告别,妈妈要给他奖励,如他可以得到一份小礼物,或者应允做什么事,宝宝果真这样做了一定要及时给予兑现,\textbf{不要向宝宝许诺妈妈做不到的事情或推迟时间去做},这样会影响宝宝的独立性的形成。

% %ux4e8cux9884ux9632ux6ce8ux5c04}{%
% \subsection{二、预防注射}%ux4e8cux9884ux9632ux6ce8ux5c04}}

% 孩子8个月时应到所属地段医院保健科、街道保健站、农村卫生院注射麻疹预防针。麻疹是一种病毒引起的急性传染病,发病时可有高烧、眼结膜充血、流泪、流鼻涕、打喷嚏等症状,持续3\textasciitilde5天,全身出现皮疹。出麻疹的孩子全身抵抗力降低,这时若护理不好,或环境卫生不良,很容易发生合并症。最常见的是麻疹合并肺炎、喉炎、脑炎或心肌损害,严重者可以死亡。得过麻疹的人可以终身免疫。

% 注射麻疹预防针的目的是提高孩子血中抗麻疹病毒的抗体水平,使之对麻疹产生免疫力,避免发病。个别情况即使发病也很轻微,不至于危及生命。

% %ux4e09ux5b69ux5b50ux53d1ux70e7ux4e0eux4f53ux6e29ux7684ux6d4bux91cf}{%
% \subsection{三、孩子发烧与体温的测量}%ux4e09ux5b69ux5b50ux53d1ux70e7ux4e0eux4f53ux6e29ux7684ux6d4bux91cf}}

% 当家长感到孩子不活泼、不爱玩或吃饭不香时,别忘了给他测测体温,看他是否发烧了。

% 有的家长只用手摸摸孩子的前额,这是很不准确的。有时候孩子体温正常,摸着他的头也许感觉热。有时孩子低烧,摸着感觉是正常的。还有的时候是家长的手太凉或太热,所以不能正确估计出孩子是否发烧。最准确的方法是测量体温。

% 给孩子测量体温不能放在口里,因为他也许会把体温计弄破,割破口、舌或咽下水银,这是很危险的。

% 给婴儿测体温只能从腋下或肛门测量。在量体温之前,先将体温计中的水银柱甩到35℃以下,然后把体温计夹在孩子腋下,体温表要紧贴孩子皮肤,不要隔着衣服。由家长扶着孩子的手臂3-5分钟,取出观察体温表上的度数。孩子的正常体温是36-37℃(腋下)。

% 如果孩子发烧,应让他卧床休息,多喝开水,体温太高可以物理降温,如酒精擦浴、冷毛巾湿敷、头枕冷水袋等,也可服退烧药片。

% 家长还要观察一下孩子其他的症状,如是否呕吐、腹泻、咳嗽、气喘等,以便带他去医院看病时给医生详细地介绍,协助医生作出正确的诊断。看病之后,就要按医嘱吃药,只要没有出现特殊情况,就不要接连不断地再去医院。

% %ux56dbux6b63ux786eux7167ux6599ux5b9dux5b9dux7684ux65b9ux6cd5}{%
% \subsection{四、正确照料宝宝的方法}%ux56dbux6b63ux786eux7167ux6599ux5b9dux5b9dux7684ux65b9ux6cd5}}

% \begin{enumerate}
% \def\labelenumi{\arabic{enumi}.}
% \item
%   看护宝宝的保姆在雇佣之前必须做健康检查。\\
%   ①
%   化验血做澳抗试验,澳抗阳性者不能看护宝宝;如是乙肝病毒携带者,会通过与宝宝的密切的生活接触传染给宝宝。\\
%   ②
%   胸部透视或验痰,做结核菌培养试验。倘若是阳性,要排除有结核病的可能后才能请她作看护。\\
%   ③
%   验便。做便细菌培养试验,查清是否患有伤寒病或者是伤寒病的健康带菌者。阳性者不宜看护宝宝,因为她们体内隐藏的伤寒病菌会污染食物,传染给宝宝。\\
%   ④
%   即使健康检查合格了,但当保姆患上细菌性痢疾时,必须马上停止看护宝宝的工作,及时治疗,待病愈连续做3次便培养皆为阴性后,才可重新看护宝宝。
% \item
%   当宝宝到了7个月表现出要爬的愿望时,妈妈可让宝宝趴在床上,将腹部托起,再把宝宝的左右腿交替向腹下推入拉出,每天定时反复练习6\textasciitilde8次;当宝宝已具备一定的双腿交替前爬能力时,可在宝宝面前放一个吸引他的玩具,以此逗引宝宝主动向前爬。在练习过程中,妈妈要\textbf{经常给予赞许的眼光和鼓励的语言,积极调动宝宝学习爬行的热情,}让孩子尽早学会爬行。
% \item
%   给宝宝穿衣服、领着上下楼梯、散步或从床上和椅子上往起拉宝宝时,特别是在宝宝蹒跚学步经常摔倒的情况下,拉拽动作一定要轻柔,即使宝宝不听话也要有耐心,千万不能将宝宝猛然拉起。一旦不慎发生脱位,马上送往骨科医生处进行诊治,待医生给宝宝复位后,为了防止再次桡骨头脱出,可让宝宝坚持2\textasciitilde3天戴颈腕吊带,尽量不要再牵拉宝宝的胳膊,如果宝宝摔倒,应抱着宝宝的腰部扶起,以防范再次桡关节脱位。
% \item
%   对于喉片,在咽喉炎、扁桃体炎、鹅口疮及口臭等具有良好的作用。宝宝只有在这些部位有炎症时才可以吃喉片。因为喉片同样是药而不是糖果,哪怕里面含的是中药也应像西药一样慎重使用,随便给宝宝服用一样会带来副作用。
% \item
%   宝宝做事时喜欢用哪只手都可以,\textbf{不要强行矫正},以免使生理功能发生不协调。
% \item
%   宝宝一旦突然扭伤,若关节活动受到限制,应马上去医院就诊,拍X光片以确定有无骨折,切记在24小时之内不要用手按摩,应将扭伤部位固定不动,可用冷毛巾\textbf{湿敷}扭伤的部位,以使毛细血管收缩,减少组织出血,过了24小时后如无骨折,可进行轻轻按摩,并用热毛巾热敷,这样能够促进受伤部位的血液循环,有利于尽快康复。
% \item
%   如果妈妈在分娩前有霉菌性阴道炎,必须积极治疗,以防宝宝出生经过产道受感染;宝宝出生后照料时,要严格进行奶具消毒及口腔护理,妈妈每次喂奶前切记洗净双手,清洁乳头;一旦患上了鹅口疮,先用淡盐水清洗患处,然后涂抹0.5\%龙胆紫药水,或用制霉菌素10万单位加少许甘油涂抹,每天2-3次,也可在医生的指导下口服制霉菌素,一般5-7天便可痊愈。\textbf{千万不能使用抗生素},特别是广谱抗生素。
% \item
%   判定宝宝病情是否痊愈,必须要医生来做,绝对不可自行判断或听别人随便说。并且,\textbf{必须把什么情况下可以停用药,什么情况下必须坚持用药,以及药的名称、成分及服用方法都应向医生问清楚。}
% \item
%   绿灯举措:外出有风或天气寒冷时,应给宝宝戴上一个清洁卫生的棉纱布小口罩,不但可挡风保暖,还可预防呼吸道传染病。但每次用后,必须及时清洗干净,并在阳光下晒干,不可不清洗而又经常重复使用。
% \item
%   妈妈为了让宝宝的小脚丫凉快,可以选择虽不露跟但有镂空网眼的透气鞋,同时别忘了给宝宝穿一双薄薄的小棉袜,若光着脚穿鞋,可使鞋质的有害物质直接与宝宝幼嫩的皮肤接触,容易使宝宝的脚部皮肤变得干燥粗糙,也失去了一个保护层。
% \end{enumerate}

% %ux4e94ux6ce8ux610fux5b69ux5b50ux7684ux773cux775b}{%
% \subsection{五、注意孩子的眼睛}%ux4e94ux6ce8ux610fux5b69ux5b50ux7684ux773cux775b}}

% 婴儿出生后的头两个月里出现斜视是常见的,此时孩子的视力很低,目光不会追随事物移动,吃饱了常处于睡眠状态,往往不被人注意。

% 到3个月时,孩子就会用眼睛追随人和移动的物体了,会用目光盯着喜欢的颜色,这时就容易发现有无斜视的症状,如发现存在斜视,就应迅速到医院请眼科大夫检查一下。7-8个月的孩子,如果在爬动和玩玩具的时候,与同龄的孩子相比表现得笨手笨脚,动作迟缓,这时就要注意孩子的视力是否有问题,并请医生帮助诊断一下。

% %ux516dux5b69ux5b50ux5927ux4fbfux5e72ux71e5ux65f6ux7684ux62a4ux7406}{%
% \subsection{六、孩子大便干燥时的护理}%ux516dux5b69ux5b50ux5927ux4fbfux5e72ux71e5ux65f6ux7684ux62a4ux7406}}

% 大便干燥的孩子平时多饮温开水,多吃蔬菜和水果。另外,要训练孩子养成定时排便的习惯。

% %ux7b2cux4e5dux828289ux4e2aux6708ux5a74ux513fux7684ux65e5ux5e38ux62a4ux7406}{%
% \subsection{09第九节8〜9个月婴儿的日常护理}%ux7b2cux4e5dux828289ux4e2aux6708ux5a74ux513fux7684ux65e5ux5e38ux62a4ux7406}}

% %ux4e00ux5904ux7406ux5b9dux5b9dux7684ux6015ux751fux73b0ux8c61}{%
% \subsection{一、处理宝宝的怕生现象}%ux4e00ux5904ux7406ux5b9dux5b9dux7684ux6015ux751fux73b0ux8c61}}

% 一般来说,4个月左右的宝宝不怕生,对谁都会笑。5\textasciitilde7个月的宝宝开始怕生,到9个月时怕生现象明显。宝宝是否怕生与以下因素有关:

% \begin{enumerate}
% \def\labelenumi{\arabic{enumi}.}
% \item
%   \textbf{父母是否在场}。父母在场时怕生现象轻一些。
% \item
%   \textbf{陌生人的特点}。如果陌生人一副``凶相'',宝宝就容易怕生。如果陌生人面目和善,宝宝怕生的程度就会轻一些。另外,\textbf{宝宝不害怕陌生的宝宝。}
% \item
%   \textbf{环境。}在熟悉的环境中见到陌生人的害怕程度比较轻。
% \item
%   \textbf{平时接触人的机会}。与妈妈越亲密,对陌生人越容易害怕。平时与陌生人接触较多,怕生的情况就轻一些。
% \item
%   \textbf{平时抚养人的多少}。平时抚养宝宝的人越多,宝宝怕生的程度就越轻。
% \end{enumerate}

% 如果宝宝怕生,可以采取以下措施:

% \begin{enumerate}
% \def\labelenumi{\arabic{enumi}.}
% \item
%   \textbf{扩大宝宝的接触面}。父母可在休息日里带宝宝到亲戚、朋友家里去玩,也可在公园内与年龄相仿的小朋友一起玩,接触人多了怕生的情况会逐渐减轻。
% \end{enumerate}

% \begin{itemize}
% \item
%   有些情况下必须请陌生人照顾怕生的宝宝,此时可借用心理治疗的原理,使宝宝消除对陌生人的害怕情绪。具体方法是:当宝宝玩得高兴的时候,让陌生人在宝宝的视野内露一下面,让宝宝看到,但不与宝宝接触。此时由于宝宝正高兴,他可能不表现出怕生。这样重复几次,逐渐延长陌生人露面的时间,缩短与宝宝的距离直至宝宝不再怕生。
% \end{itemize}

% %ux4e8cux4fddux62a4ux7259ux9f7f}{%
% \subsection{二、保护牙齿}%ux4e8cux4fddux62a4ux7259ux9f7f}}

% 孩子出牙之后要注意保护牙齿,睡觉前不要吃食物,平时少吃甜食,每天\textbf{早晚让孩子喝白开水以清洁口腔}。

% 另外,不要让孩子养成吮手指、吃假奶头等坏习惯,这些都影响牙齿的生长。

% %ux4e09ux6253ux9488ux662fux5426ux6bd4ux5403ux836fux597d}{%
% \subsection{三、打针是否比吃药好}%ux4e09ux6253ux9488ux662fux5426ux6bd4ux5403ux836fux597d}}

% 孩子生病了,家长很着急,很多家长要求医生给孩子打针,以便使孩子好得快些。其实,吃药还是打针应根据病情及药物的性质、作用来决定。有些病口服用药效果好,如肠炎、痢疾等消化道疾病,药物通过口服进入胃肠道,保持有效浓度,能收到很好效果。还有一些药只能口服,不能注射,如咳嗽糖浆等,所以家长不能只迷信打针。药物被口服之后,大部分都能够被身体所吸收,经过血液循环运送到全身而发挥作用。通过打针注射给药,药物吸收快而规则,所以有些病是打针效果好。但是打针痛苦大,还有可能局部感染或损伤神经(虽然几率很小),反复打针,局部会有硬结,肌肉收缩能力减弱,少数发生臀大肌挛缩症,还得要进行手术治疗。所以,孩子有病,能口服药的应尽量口服为好。

% %ux56dbux5b9dux5b9dux51acux65e5ux8d77ux5c45ux8981ux9886}{%
% \subsection{四、宝宝冬日起居要领}%ux56dbux5b9dux5b9dux51acux65e5ux8d77ux5c45ux8981ux9886}}

% (1)在寒冷的冬夜起来给宝宝喂奶是件苦差事,但事先准备妥当,就可以减轻喂奶的辛苦。临睡前先把奶瓶、奶粉准备好,再备上可以随时使用的热水,喂奶的程序是不是减少了呢?

% 在没有暖气或暖风的家里,半夜喂奶很容易遇到的事情就是奶瓶里的奶越来越凉。因此,要用毛巾包住奶瓶喂奶,或是喂到一半再用温开水温热。

% 许多妈妈总担心宝宝吃了凉东西会弄坏肠胃,在冬天里喜欢给宝宝吃相对太热的食物,其实没这个必要,相反还有可能烫伤宝宝的口腔及食道。

% (2)宝宝睡觉时有被子盖着,不必再穿很多衣服,穿一件连体睡衣即可。即使室温很低,也能充分保持体温。

% 宝宝睡觉好动,妈妈担心宝宝肩膀露在外面受寒而感冒。可以把1岁内的宝宝``装''进睡袋里,大一点的宝宝如果实在太不老实,不妨在容易受凉的部位加件棉背心,总要伸出被窝的小手也可以戴上小手套来保暖。

% 冬天很冷,妈妈可以适当地给宝宝铺上比较厚的褥子。但不可过于松软,\textbf{过软的床会对宝宝脊柱的发育产生不良影响}。还有床单也要铺平、掖好,以免堵住宝宝的口鼻,造成意外。

% (3)只要家里有条件,无论大人孩子应该天天洗澡,寒冷的冬天也不例外。相对于成人,宝宝汗量大,新陈代谢快,除非是生病,不舒服,要坚持洗澡。

% 冬天宝宝洗澡的水温可以比38℃略高些,同时浴室也要有适宜的温度及通风。给小宝宝洗澡要准备好可以随时添加的热水;可以直接淋浴的宝宝只要调整好水温即可。

% 洗澡之前,应将宝宝需要的衣服、浴巾、浴袍、婴儿油、尿布等事先准备好,放在固定的地方。洗完了立即用浴巾包住宝宝身体,擦干后再穿上浴袍。北方冬天太干燥,可以适当地给宝宝抹一些儿童专用乳液、婴儿油等,但不要急着给宝宝穿衣或直接进被窝,这样容易出汗,应该等到身体不再发热时才行。

% 洗澡之后要让宝宝喝点儿水,补充流失的水分。

% (4)很多妈妈都知道,感冒的罪魁祸首并非是寒冷,而是病毒。病毒最喜欢的环境是干燥。空气越干燥,感冒病毒的活动性越强,然后附着在人体干燥的鼻及口腔黏膜上,引发感冒。因而,\textbf{冬季预防感冒最好的办法就是保持良好的室温和湿度}。

% 室温应该控制在20\textasciitilde25℃,湿度能接近60\%是最合适不过的。北方都有暖气,会加速空气干燥性,因此要保持通风,必要时使用加湿器。电暖气等不可距离宝宝太近,那样容易使宝宝身体受热不均匀,也容易上火。南方没有暖气,冬季室温较低,妈妈也会给宝宝穿得暖暖和和的,但宝宝其实没有妈妈想象的那么怕冷,而且妈妈还应该有意识训练宝宝的体温调节能力,穿太多太厚的衣服就不合适了,也妨碍宝宝的活动,不如想办法把室温提高。

% %ux7b2cux5341ux8282910ux4e2aux6708ux5a74ux513fux7684ux65e5ux5e38ux62a4ux7406}{%
% \subsection{10第十节9〜10个月婴儿的日常护理}%ux7b2cux5341ux8282910ux4e2aux6708ux5a74ux513fux7684ux65e5ux5e38ux62a4ux7406}}

% %ux4e00ux8ba9ux5b9dux5b9dux5728ux5b89ux5168ux7684ux5730ux65b9ux5c3dux60c5ux73a9ux800d}{%
% \subsection{一、让宝宝在安全的地方尽情玩耍}%ux4e00ux8ba9ux5b9dux5b9dux5728ux5b89ux5168ux7684ux5730ux65b9ux5c3dux60c5ux73a9ux800d}}

% 这一时期的宝宝急切地想自己做完一件事,比如,想自己用勺子吃饭等。宝宝这时正处于\textbf{自信心发育的阶段}。如果不加考虑地一概否定,就会扼杀宝宝的探索欲。妈妈应当收起可能对宝宝有危险的物品,并防止宝宝去厨房等危险场所。在安全的范围内,让宝宝尽情地玩耍。

% %ux4e8cux7ed9ux5b9dux5b9dux7a7fux65b9ux4fbfux6d3bux52a8ux7684ux8863ux670d}{%
% \subsection{二、给宝宝穿方便活动的衣服}%ux4e8cux7ed9ux5b9dux5b9dux7a7fux65b9ux4fbfux6d3bux52a8ux7684ux8863ux670d}}

% 这一时期,宝宝四处爬行,运动量大,因此流汗较多,衣服脏得快。如果衣服被汗水湿透,不仅容易患感冒,还容易引发皮炎,所以要经\textbf{常给宝宝换衣服}。

% 给宝宝挑选衣服时,应选择\textbf{吸汗性好的棉布衣服}。应便于穿着,不束缚宝宝的行动。太紧或太松的衣服也不便于活动。此外,还要养成\textbf{在室内给宝宝穿薄衣服}的习惯。

% %ux4e09ux7ed9ux5b9dux5b9dux7406ux53d1}{%
% \subsection{三、给宝宝理发}%ux4e09ux7ed9ux5b9dux5b9dux7406ux53d1}}

% 理发时最好由爸爸抱着宝宝,让宝宝坐在爸爸的膝上,由妈妈来理。散落的头发会让宝宝感到非常难受,因此在理发前父母应该先\textbf{用毛巾将宝宝的脖子仔细地围好},并把宝宝的\textbf{头发弄湿}。

% 理发时,妈妈应该注意用另一只手将要理的头发\textbf{轻轻地抓起,以免剪刀伤到宝宝}。在宝宝\textbf{洗澡时},迅速地给宝宝理发也是一种好办法。

% %ux56dbux6237ux5916ux6d3bux52a8}{%
% \subsection{四、户外活动}%ux56dbux6237ux5916ux6d3bux52a8}}

% 这一时期宝宝非常好动,因此父母可以偶尔带宝宝外出,到有大块草坪的地方让宝宝尽情地玩耍。特别是在宝宝对语言理解逐步加深的这一时期,户外活动可以让宝宝通过亲眼所见记住``树、狗''等词汇。看大孩子玩耍,宝宝的社会适应性也会增加。

% \hspace{0pt}\includegraphics[width=2.74126in,height=2.20979in]{media/rId1414.png}\hspace{0pt}

% %ux4e94ux5982ux4f55ux7ed9ux5b69ux5b50ux5582ux836f}{%
% \subsection{五、如何给孩子喂药}%ux4e94ux5982ux4f55ux7ed9ux5b69ux5b50ux5582ux836f}}

% 分析孩子不愿吃药的原因,主要是怕苦,其次是家长经常给予孩子吃药、打针是一种惩罚的暗示。因此,家长\textbf{不要用打针、吃药吓唬孩子},当孩子顺利地吃下药时,就要\textbf{鼓励和称赞}。另外,要尽量想方设法减少药的苦味,以便使孩子能够接受。比如,把药研成粉和白糖拌在一起,或用两层果酱夹一层药粉放在勺子里一下喂进。汤药要煎得浓浓的,分几次喂进。鱼肝油类药物可滴在饼干上给孩子吃。给孩子喂药时,可让孩子坐在怀里,将头抬起,一手轻捏下巴,一手用勺将药喂进。喂药后立即喂糖水,使孩子口中的余药全部吞下去。\textbf{只要孩子咽下之后,就要表扬,不要再去批评喂药之前的一段表现}。若孩子吃药后哭闹时间过长,引起呕吐,还应重新喂药。给孩子喂药必须注意剂量准确,一定要看清楚说明。有的家长把1/3片药错看成3片,这时就容易造成危险。给孩子喂药还要按时,每天2次的药在早晚服,每天3次的药在早、中、晚服,每天4次的药在早、中、晚和睡前服。喂药之后一定要多喂水,使药充分溶解,易于吸收。

% %ux516dux4e0dux8981ux6ee5ux7528ux6297ux751fux7d20}{%
% \subsection{六、不要滥用抗生素}%ux516dux4e0dux8981ux6ee5ux7528ux6297ux751fux7d20}}

% 当孩子生病时,很多家长迷信抗生素,坚持要给孩子吃``消炎药'',或要求注射抗生素。抗生素能够杀灭或抑制危害人体的病菌,使很多疾病得到有效的治疗,但是不能包治百病。比如,绝大多数孩子感冒发烧,都是病毒感染引起的,抗生素对病毒性疾病没有疗效。反之,常用抗生素,还会使细菌产生抗药性,给治疗疾病带来困难。滥用抗生素还增加了发生过敏和毒性反应的机会,有的孩子就因为感冒发烧注射庆大霉素,结果造成耳聋。滥用抗生素,还可使在原有疾病的基础上产生新的疾病,也就是说,大量的抗生素抑制了敏感的细菌,却使耐药的细菌乘机大量繁殖,造成机体菌群失调,发生二重感染。所以家长要切记,抗生素只能在医生的指导下使用。

% %ux4e03ux5982ux4f55ux9884ux9632ux5b69ux5b50ux7f3aux950c}{%
% \subsection{七、如何预防孩子缺锌}%ux4e03ux5982ux4f55ux9884ux9632ux5b69ux5b50ux7f3aux950c}}

% ``锌''是一种人体内必不可少的微量元素。如果锌缺乏,就会发生一些疾病或引起孩子生长障碍。缺锌的孩子一般都\textbf{食欲不好},又矮又瘦,免疫力低下,很爱生病。特别容易患消化道或呼吸道感染、口腔溃疡等。如果孩子诊断为锌缺乏症,可以服用硫酸锌治疗。缺锌的孩子平时应注重食物的合理搭配,动物食品要占一定比例。同时要养成孩子良好的饮食习惯,不偏食,不挑食。

% %ux54eaux4e9bux5b9dux5b9dux6613ux7f3aux950c}{%
% \subsubsection{1.
% 哪些宝宝易缺锌}%ux54eaux4e9bux5b9dux5b9dux6613ux7f3aux950c}}

% (1)
% 母亲在怀孕期间摄入锌不足的孩子:孕妇的血锌水平维持在89.4毫克/分升左右。如果孕妇的一日三餐中缺乏含锌的食品,势必会影响胎儿对锌的利用,使体内贮备的锌过早被应用,这样的孩子出生后就容易出现缺锌症状。

% (2)
% \textbf{早产儿}:如果孩子不能在母体内孕育足够的时间而提前出生,就容易失去在母体内贮备锌元素的黄金时间(一般是在孕末期的最后1个月),造成先天不足。

% (3)
% 非母乳喂养的孩子:母乳中含锌量大大超过普通牛奶,更重要的是其吸收率高达42\%,这是任何非母乳食品都不能比的。

% (4)
% 过分偏食的孩子:有些``素食者'',从小拒绝吃任何肉类、蛋类、奶类及其制品,这样非常容易缺锌,因此,应从小就培养良好的饮食习惯,不偏食,不挑食。

% (5)
% 过分好动的孩子:不少孩子尤其是男孩子,过分好动,经常出汗甚至大汗淋漓,而汗水也是人体排锌的渠道之一。据测定,一天中大汗淋漓可丢失锌1.3毫克。

% (6)
% 罹患佝偻病的孩子:这些孩子因治疗疾病需要而服用钙制剂,而体内钙水平升高后就会抑制肠道对锌的吸收。同时,因为这样的患儿食欲也相对较差,食物中的锌摄入减少,很容易发生缺锌。

% (7)
% 一些特殊情况下的孩子:土壤含锌过低,使当地农产品缺锌;孩子的消化吸收功能不良,一些疾病、药物如四环素等与锌结成难溶的复合物妨碍吸收。

% %ux5b9dux5b9dux7f3aux950cux7684ux8868ux73b0}{%
% \subsubsection{2
% 宝宝缺锌的表现}%ux5b9dux5b9dux7f3aux950cux7684ux8868ux73b0}}

% \begin{enumerate}
% \def\labelenumi{\arabic{enumi}.}
% \item
%   喜欢吃泥土、煤渣、纸张、墙皮、鸡蛋皮等。也就是说,喜欢进食非食物性异物,医学上称为异食癖。
% \item
%   长期食欲不振、厌食、拒食。
% \item
%   生长发育速度缓慢,但需排除遗传性疾病及某些药物中毒等。
% \item
%   复发性口腔溃疡也是缺锌的症状之一。
% \item
%   经常发生呼吸道及消化道感染,每年不少于10\textasciitilde12次。
% \item
%   有肢端皮炎、慢性湿疹、痤疮,外伤后创伤面不易愈合。
% \end{enumerate}

% 这些症状是常见的临床表现,必要时可去医院测定身体中锌的含量,有助于疾病的诊断。

% 母乳喂养是预防小儿锌营养缺乏最简单的办法。因为人的初乳中含有大量的锌,牛奶含锌虽然不亚于人乳,但吸收比较困难,这也就更显示人乳喂养的好处。

% 但是随着月份的增长,终乳的锌含量也逐渐减少。婴儿不断地生长发育,就需要丰富小儿食品的种类,单一的食品往往会造成锌的缺乏。一般动物性食物除脂肪和蛋白外,锌的含量均比植物性食物高。特别是海生贝壳类,含量更丰富。在植物性食品中,海带和紫菜含量最多,豆类和花生米等坚果也有一定的含量,水果含量较低。植物性食物中锌的吸收率为10\%-20\%或更低,而动物性食物中锌的吸收率可达35\%-40\%。所以如果能够多吃全谷类、坚果、肉类和海味,就能获得足够的锌。

% %ux7b2cux5341ux4e00ux828210-11ux4e2aux6708ux5a74ux513fux7684ux65e5ux5e38ux62a4ux7406}{%
% \subsection{11第十一节10 \textasciitilde{}
% 11个月婴儿的日常护理}%ux7b2cux5341ux4e00ux828210-11ux4e2aux6708ux5a74ux513fux7684ux65e5ux5e38ux62a4ux7406}}

% %ux4e00ux5b69ux5b50ux809dux504fux5927ux662fux4e0dux662fux75c5}{%
% \subsection{一、孩子肝偏大是不是病}%ux4e00ux5b69ux5b50ux809dux504fux5927ux662fux4e0dux662fux75c5}}

% 孩子摸上去肝偏大,这一般是正常生理现象。这是因为孩子腹肌松软,腹壁薄,容易在右肋下摸到肝脏。3岁以内孩子,肝不超过肋下2厘米,质软,边缘清楚,均属正常。孩子生长发育迅速,代谢旺盛,血容量相对比成人要高,而肝脏是人体具有加工、合成、分解、代谢功能的重要器官,所以孩子肝脏的体积相对地比成人大些。

% 但是,当孩子患营养不良、佝偻病、贫血等疾病时,也会引起肝脏偏大。

% %ux4e8cux9884ux9632ux610fux5916ux4e8bux6545ux7684ux53d1ux751f}{%
% \subsection{二、预防意外事故的发生}%ux4e8cux9884ux9632ux610fux5916ux4e8bux6545ux7684ux53d1ux751f}}

% 孩子会走以后,眼界大开,对于一切事物都感到新鲜、好奇,他们对什么都感兴趣,都想试探一下。因此,家长必须随时注意他们,防止意外事故发生。

% 1岁左右的孩子有个特点:不论见了什么,都爱放进嘴里,所以像珠子、扣子、别针、小钉子这类东西,家长要收好,不要给小孩玩,以免他们咽进肚里或塞进鼻孔、耳朵里。家里的汽油、煤油、碘酒、洗涤液等东西和大人吃的药,都要放在孩子拿不着、够不到的安全地方,以免被孩子误服后发生危险。【买个可以上锁的大箱子】

% 如果孩子从高处摔下来,要观察他的神志,若出现呕吐或昏迷等情况,应想到可能是头部受伤,要立即送医院治疗。

% %ux4e09ux5b69ux5b50ux8bf4ux8bddux5927ux820cux5934ux600eux4e48ux529e}{%
% \subsection{三、孩子说话大舌头怎么办}%ux4e09ux5b69ux5b50ux8bf4ux8bddux5927ux820cux5934ux600eux4e48ux529e}}

% 有些孩子说话发音时,一些字、音咬不清,这在1岁以内的小孩是常见的。也有一些孩子是由于舌系带过短而造成发音不清,这需要检查治疗。

% 舌系带是舌尖下方一条纵行的薄薄的黏膜,如果舌系带过短,舌头伸展受到限制,发音吐字就会受到影响。

% 检查舌系带是否过短,方法很简单,让孩子学伸舌的动作,当舌尖是尖形或圆形时,就是正常的,若是形成倒角形的舌尖,中间有一条明显的凹陷,就是舌系带过短。

% 舌系带过短大多是先天性的,也有一些是由于后天创伤引起的。若确诊为舌系带过短,可进行手术矫治。

% %ux56dbux6b63ux786eux4f7fux7528ux5b66ux6b65ux8f66}{%
% \subsection{四、正确使用学步车}%ux56dbux6b63ux786eux4f7fux7528ux5b66ux6b65ux8f66}}

% 学步车有许多优点,可以省下抱宝宝的时间。宝宝在车里不会摔倒,还可以扩大宝宝的活动范围。

% 不过,学步车也有它的弊病。首先,如果\textbf{过早}让宝宝站在学步车内,会使\textbf{两下肢过早负重},造成两下肢畸形。所以,如果要使用学步车,至少\textbf{要等到宝宝已能独站得比较稳时}。其次,在学步车内,宝宝不需要自己控制身体的平衡,也不需要掌握正确的迈步姿势,\textbf{而}这两点正是学会独立走路的关键,所以别指望学步车帮助宝宝学步。最后,学步车扩大了宝宝的活动范围,也带来了一些危险。因此,在使用学步车时,要注意安全问题。

% 使用学步车的安全要点:

% \begin{itemize}
% \item
%   在宽敞的客厅或卧室使用比较合适,厨房间、阳台等狭窄的地方不宜使用。
% \item
%   最好每天检查一下螺丝是否松动,车轮滑动是否自如,防止意外。
% \item
%   \textbf{学步车的高度以齐宝宝乳线水平为宜},太低易翻车,宝宝也容易爬出。
% \item
%   放学步车的房间应有门槛挡住,或关上门,以免宝宝``驾车''滑到其他场所而发生意外。
% \item
%   放学步车的\textbf{室内不能放热水瓶、玻璃器皿,桌角、椅角要用海绵包住}。
% \end{itemize}

% %ux4e94ux751fux6d3bux81eaux7406ux80fdux529bux8badux7ec3}{%
% \subsection{五、生活自理能力训练}%ux4e94ux751fux6d3bux81eaux7406ux80fdux529bux8badux7ec3}}

% 大小便坐盆:训练宝宝养成大小便坐盆的习惯。此时宝宝尚不能完全主动表示,可在宝宝有便意时定地点、定时协助宝宝坐盆。

% 配合穿衣:给宝宝穿衣服时要告诉他``伸手、举手、抬腿''等,让宝宝用动作配合穿衣、穿裤。如果宝宝还未听懂就用手去示范协助。经常表扬宝宝的合作行为,以后宝宝就会主动伸臂入袖、伸腿穿裤。

% %ux516dux600eux6837ux62a4ux7406ux8f93ux6db2ux7684ux5b9dux5b9d}{%
% \subsection{六、怎样护理输液的宝宝}%ux516dux600eux6837ux62a4ux7406ux8f93ux6db2ux7684ux5b9dux5b9d}}

% %ux9700ux8981ux6ce8ux610fux7684ux4e8bux9879}{%
% \subsubsection{1.需要注意的事项}%ux9700ux8981ux6ce8ux610fux7684ux4e8bux9879}}

% 一般说来,输液需要4\textasciitilde6小时,护理宝宝应该注意以下几点:

% \begin{enumerate}
% \def\labelenumi{\arabic{enumi}.}
% \item
%   始终和宝宝在一起:输液时,家长要和宝宝在一起,便于喂奶或喂食。给宝宝进食时,须抬高上半身,有呕吐的可以少量、多次地喂哺,防止食物吸入气管,导致窒息。
% \item
%   降低体温:如果宝宝高烧,可以减少衣被,多喂凉开水,来降低宝宝体温。
% \item
%   及时更换衣裤:输液过程中,宝宝的尿量明显增多,经常注意观察宝宝,及时换尿布,如果尿湿了衣裤,要及时更换。
% \item
%   控制输液速度:要根据不同病情控制输液速度,可先向医护人员请教,同时要注意观察宝宝的反应。
% \item
%   输液结束时:输入液体还剩10\textasciitilde15毫升时,要及时通知护士换药瓶或拔针,以防液体滴完后有空气进入血管内。
% \end{enumerate}

% %ux9632ux6b62ux53d1ux751fux8f93ux6db2ux53cdux5e94}{%
% \subsubsection{2.
% 防止发生输液反应}%ux9632ux6b62ux53d1ux751fux8f93ux6db2ux53cdux5e94}}

% 输液如同用药一样,人体有时会发生不良反应,即输液反应。最常见的输液反应是发热反应,约占80\%以上。一般输液反应有以下几种情况:

% \begin{enumerate}
% \def\labelenumi{\arabic{enumi}.}
% \item
%   输液过程中,宝宝突然畏寒或出现寒战,数分钟后发热,体温迅速上升至39\textasciitilde41℃或更高。
% \item
%   输液反应多发生在输液开始后30-60分钟内,滴入液体100-250毫升时。
% \item
%   通常在停止输液后数小时或使用镇静、脱敏药后,宝宝的畏寒或寒战现象突然消失,体温能较快地恢复正常。
% \end{enumerate}

% 一旦发现宝宝出现输液反应,应当立即找值班护士做紧急处理,减慢输液速度或由医生决定停止输液。如果宝宝的病情仍需要输液,要及时更换输液管、药物和液体。

% %ux968fux65f6ux76d1ux62a4ux8f93ux6db2ux72b6ux6001}{%
% \subsubsection{3.
% 随时监护输液状态}%ux968fux65f6ux76d1ux62a4ux8f93ux6db2ux72b6ux6001}}

% (1)
% 宝宝如果因发烧在输液过程中睡着,陪护家长千万不能睡。要随时监护注射部位有无药液外漏、肿胀或包扎过紧的状况,还要注意保持注射部位干燥清洁,以免污染,并随时观察宝宝有无药物和输液反应。

% \begin{quote}
% 温馨提示

% 医院普遍装有空调,夏秋季输液带上\textbf{小毯子}方便披盖;冬春季穿上便于穿脱的\textbf{棉衣},穿太多会影响发烧宝宝的散热。

% 周岁内的宝宝要带上尿布或纸尿裤,数量可视病情而定,腹泻的宝宝要多备一些。没断奶的宝宝要带上奶瓶,带上适量奶粉和少量零食,预防宝宝饥饿引起哭闹。

% 准备宝宝常玩的玩具两件,童话书若干本,以免输液时间过长宝宝无聊。

% 陪同家长要耐心对待宝宝的哭闹,配合护士做好护理。
% \end{quote}

% (2)遵照医嘱最要紧:输液效果显著,但一定要坚持医生给的用量。宝宝有可能第\textbf{一次输液后立刻变得很精神,家长往往会觉得宝宝好了,次日不再去医院输液,这样往往耽误治疗,反而会延长病愈时间。}

% (3)输液并非万能:输液效果虽好,但千万不要什么病都用输液来解决。输液毕竟有创伤,药品用量要比口服药大好几倍,副反应较多。因此,是否给宝宝输液治疗应当完全听取医生的建议。

% \begin{quote}
% 你知道吗?

% 大医院并不一定好:如果宝宝得的是常见病,无非退烧消炎,这些药在地段或小区医院也照样有,加上病人少,自然服务也好,比大医院耐心得多,而且还能避免众多生病宝宝聚集在输液室,空气差,容易交叉感染,还可以节省开支。
% \end{quote}

% %ux4e03ux5982ux4f55ux62a4ux7406ux751fux75c5ux7684ux5b9dux5b9d}{%
% \subsection{七、如何护理生病的宝宝}%ux4e03ux5982ux4f55ux62a4ux7406ux751fux75c5ux7684ux5b9dux5b9d}}

% %ux4f53ux6e29ux6d4bux8bd5ux6280ux5de7}{%
% \subsubsection{1.体温测试技巧}%ux4f53ux6e29ux6d4bux8bd5ux6280ux5de7}}

% 儿童的正常体温在36\textasciitilde37℃之间,体温低于35℃称为低体温;腋表体温高于37.5℃,就是发烧。

% 哺乳期的婴儿发烧时爱啼哭,吃奶时,婴儿母亲的奶头会有灼热的感觉。家长还可以用自己的\textbf{嘴唇}吻一下宝宝的额头,这样便可以初步判断宝宝是否有发烧。如果怀疑宝宝发烧,可以使用体温计为宝宝测量体温。

% 常用的体温计是水银式体温计,它是在玻璃管内放置适量水银制成的。在测量体温时,水银因人体的体温而受热膨胀,使水银柱上升到一定的高度。我们可以直接在体温计的刻度上读到被测者的体温。水银式体温计又分肛表、腋表和口表三种。为了安全起见,防止发生意外,\textbf{对于1岁半以下的儿童最好使用肛表}。在宝宝1岁半以后,能够与大人配合了,可以将水银式体温计放在宝宝的腋窝下测量体温。六七岁以上的儿童,你相信他不会将体温计咬碎在嘴里了,也可以将体温计放在宝宝的舌头下面,教宝宝用舌尖抵住下排的牙齿,固定口表,然后闭上嘴来测量体温(而不是用牙齿来咬住口表)。不过要注意,将体温计放在肛门、嘴里或腋下所测得的体温是不一样的。在腋下所测得的温度会比宝宝的实际体温低0.6℃。【肛表目前不适合木木】

% \hspace{0pt}\includegraphics[width=2.20979in,height=2.6014in]{media/rId1452.png}\hspace{0pt}

% 目前还有数字式体温计和液晶式体温计。无论对于多大年龄的儿童,使用这两种体温计都很安全方便,比用水银式体温计测体温要容易得多。但是,数字式体温计较贵一些;\textbf{液晶式体温计的测值则不如水银式体温计准确}。

% %ux5728ux7ed9ux5b9dux5b9dux6d4bux91cfux4f53ux6e29ux65f6ux8981ux6ce8ux610fux4ee5ux4e0bux51e0ux70b9}{%
% \subsubsection{2.在给宝宝测量体温时要注意以下几点}%ux5728ux7ed9ux5b9dux5b9dux6d4bux91cfux4f53ux6e29ux65f6ux8981ux6ce8ux610fux4ee5ux4e0bux51e0ux70b9}}

% \begin{enumerate}
% \def\labelenumi{\arabic{enumi}.}
% \item
%   宝宝刚刚剧烈运动后,不要测体温。因为儿童大脑的体温调节中枢未完全发育成熟,所以当你第一次测量体温发现宝宝发烧时,就应当在\textbf{20分钟以后再对宝宝测一次体温},以便及时发现宝宝的体温是否在骤然上升。
% \item
%   在测体温之前,先要检查体温计是否完好无损,并手持体温计没有水银的一端,用力将它往下甩,要将体温计的水银柱的位置甩到35℃以下刻度处。
% \item
%   水银式体温计使用后,要用凉水及肥皂清洗擦干,放在固定的地方保存,以便下次需要时能马上拿来使用。千万不能用热水清洗体温计,以免体温计的玻璃管爆裂。用口表测体温时,家长不要离开宝宝。万一体温计被宝宝咬碎在嘴里,务必要叮嘱宝宝将玻璃碎片及水银全部吐出来,并用干毛巾或餐巾纸擦拭宝宝的口腔。如果宝宝吞入了碎片或水银,哪怕只是一点点,都必须马上到医院去处理。
% \item
%   对于好动的宝宝,如果将体温表放在腋下测体温,家长最好还是将宝宝抱在怀里,并帮助宝宝将体温表夹紧,以免体温计滑落而测量不准或落地搞碎。
% \end{enumerate}

% \begin{quote}
% 你知道吗?
% \end{quote}

% \begin{enumerate}
% \def\labelenumi{\arabic{enumi}.}
% \item
%   \begin{quote}
%   用水银式体温计测体温一般需要5\textasciitilde10分钟,也可参照各型体温计的使用说明来使用。
%   \end{quote}
% \item
%   \begin{quote}
%   发热的分类,按腋温的高低可分为:低热37.3-38℃;中等度发热38.1-39℃;高热39.1\textasciitilde41℃;超高热41℃以上。
%   \end{quote}
% \end{enumerate}

% %ux7b2cux5341ux4e8cux82821112ux4e2aux6708ux5b9dux5b9dux7684ux65e5ux5e38ux62a4ux7406}{%
% \subsection{12第十二节11〜12个月宝宝的日常护理}%ux7b2cux5341ux4e8cux82821112ux4e2aux6708ux5b9dux5b9dux7684ux65e5ux5e38ux62a4ux7406}}

% %ux4e00ux8badux7ec312ux4e2aux6708ux5b9dux5b9dux6f31ux53e3}{%
% \subsection{一、训练12个月宝宝漱口}%ux4e00ux8badux7ec312ux4e2aux6708ux5b9dux5b9dux6f31ux53e3}}

% 漱口是利用水的冲击力和激荡水流来清除口腔内的食物残渣和口腔细菌。科学证明,这种方法可以洗去口腔内50\%的细菌。勤漱口是一种良好的卫生习惯,对保护牙齿和口腔卫生很有好处。

% 先为宝宝准备好水杯,并预备好漱口所用的温白开水(夏天可以用凉白开水)。记住不要用自来水。因为宝宝在开始时不可能马上学会漱口的动作,往往漱不好就会把水咽下去,所以刚开始最好用白开水。初学时,家长为宝宝做示范,把一口水含在嘴里做漱口动作,然后吐出,反复几次,宝宝很快就会学会。不要让宝宝仰着头漱口,这样很容易造成呛咳,甚至发生意外。

% %ux4e8cux8ba9ux5b9dux5b9dux5f00ux59cbux5237ux7259}{%
% \subsection{二、让宝宝开始刷牙}%ux4e8cux8ba9ux5b9dux5b9dux5f00ux59cbux5237ux7259}}

% \hspace{0pt}\includegraphics[width=2.46154in,height=2.62937in]{media/rId1466.png}\hspace{0pt}

% 一旦宝宝学会刷牙往外吐水,就可以给他使用牙膏了。在这之前,只用清水刷牙就行。

% \begin{enumerate}
% \def\labelenumi{\arabic{enumi}.}
% \item
%   每次买小包装的牙膏,让宝宝常换常新。
% \item
%   牙膏只用豌豆大小就足以,不必挤满宝宝的牙刷。
% \item
%   不用时,把牙膏放到宝宝够不着的地方。
% \item
%   挑选质量有保证的,并且是专为儿童设计的牙膏。
% \item
%   为宝宝买一个软毛牙刷。因为硬毛牙刷会伤害宝宝的牙龈。
% \item
%   给宝宝选择一个刷柄容易抓握的儿童牙刷,并且刷头的弧线和尺寸适合宝宝的使用。
% \item
%   正确刷牙:顺牙缝由上而下、由下而上地竖刷。上下、内外都是顺着牙根向牙尖刷,咬合面可以横刷。每次刷牙至少需要3分钟,每个面要刷15至20次,才能达到清洁牙齿的目的。
% \end{enumerate}

% \begin{quote}
% 育儿小百科

% \textbf{睡前吃东西易患``虫牙}

% 所谓虫牙,就是细菌利用碳水化合物黏附在牙齿表面,形成一层细菌膜,来侵蚀牙体。

% 睡前吃了东西,很快就要睡觉,人睡着的时候,身体内的各个器官大部分处于休息状态,口腔内口水的分泌量明显减少,就无法冲淡牙齿上的食物残渣。睡着后,牙齿不再嚼东西,也就没有什么粗糙的食物可以把牙面上粘聚的细菌膜擦掉。所以,口腔内的细菌就会密集地粘在牙面上形成菌膜。菌膜一旦形成,就像滚雪球一样,可吸附更多的细菌。细菌利用碳水化合物不断地新陈代谢,繁殖后代。这个过程会产生大量的酸,这些酸会使牙齿脱钙、弱化、分解,形成牙洞。牙洞更是堆积食物的好场所,细菌再次大量繁殖,这种恶性循环,会使牙洞越来越大。

% 因此,睡前最好不要让宝宝吃东西,如果吃了东西就一定要漱口、刷牙。
% \end{quote}

% %ux4e09ux907fux514dux6559ux5b9dux5b9dux8bf4ux8bddux7684ux8befux533a}{%
% \subsection{三、避免教宝宝说话的误区}%ux4e09ux907fux514dux6559ux5b9dux5b9dux8bf4ux8bddux7684ux8befux533a}}

% %ux8ba4ux4e3aux5b9dux5b9dux542cux4e0dux61c2}{%
% \subsubsection{1.认为宝宝听不懂}%ux8ba4ux4e3aux5b9dux5b9dux542cux4e0dux61c2}}

% 刚出生的小宝宝,对成人的话虽然听不懂,但宝宝的学习能力很强,当妈妈总是冲宝宝微笑,对宝宝说:``宝宝,我是妈妈。宝宝,这是奶,你饿了吧!''时间一长,这种语言信息就储存在了宝宝的脑子里。随着宝宝的智力发育,再经过几十次的语言重复,孩子就明白,原来总抱着我的人就是妈妈。到了1岁的时候,宝宝可能会叫``爸爸、妈妈''了,当成人对他说:``宝宝,你的球呢?''孩子会转身去找,说明孩子已经明白了话的意思。

% %ux8fc7ux5206ux6ee1ux8db3ux5b9dux5b9dux7684ux8981ux6c42}{%
% \subsubsection{2
% 过分满足宝宝的要求}%ux8fc7ux5206ux6ee1ux8db3ux5b9dux5b9dux7684ux8981ux6c42}}

% 当宝宝已经明白成人的话以后,而宝宝还不会从口中说出,如果说宝宝指着水瓶,成人马上明白,这是宝宝想喝水了,于是把水瓶递给宝宝。这种满足宝宝要求的方法使宝宝的语言发展缓慢,因为不用说话,成人就能明白自己的意图,自己的要求就已经达到了,因此宝宝失去了说话的机会。当宝宝想喝水时,可以给宝宝一个空水瓶,宝宝拿着空水瓶,想要得到水时,会努力去说``水''。仅仅说一个字,就应该鼓励宝宝,这是不小的进步,因为宝宝懂得用语言表达自己的要求了。

% %ux7528ux513fux8bedux548cux5b9dux5b9dux8bf4ux8bdd}{%
% \subsubsection{3.
% 用儿语和宝宝说话}%ux7528ux513fux8bedux548cux5b9dux5b9dux8bf4ux8bdd}}

% 儿童语言发展有其自身的阶段性,都是经历单词句(用一个词表达多种意思)、多词句(用两个以上词表达意思)、说出完整句子这几个阶段,父母对宝宝进行教育时,应了解这一规律,但又不能迁就宝宝,而应通过正确的教育引导宝宝的语言向更尚阶段发展。

% 1岁左右的宝宝,语言处于单词句阶段,宝宝经常发出一些重叠的音,如``抱抱''、``饭饭''、``打打'',结合身体动作表情来表达自己的愿望,如说抱抱时,就张开双臂面向妈妈,表示要妈妈抱。

% 到了1岁6个月左右,宝宝能用二三个词组合在一起表达意思,这就进入了多词句时期。开始时能把两个词重叠在一起,如``吃饭饭''、``妈妈抱'';快到了2岁时,出现简单句,能准确地表达自己的意思,如说出``妈妈抱宝宝''、``宝宝吃饭饭''等。在这些发展阶段中,宝宝用小儿语,是因为其语言发展限制了准确表达自己的意思

% %ux91cdux590dux5b9dux5b9dux7684ux9519ux8befux8bedux97f3}{%
% \subsubsection{4.
% 重复宝宝的错误语音}%ux91cdux590dux5b9dux5b9dux7684ux9519ux8befux8bedux97f3}}

% 刚学会说话的宝宝基本上能用语言表达自己的愿望和要求,但是有很多宝宝还存在着发音不准的现象。如把``吃''说成``七'',把``狮子''说成``希几'',``苹果''说成``苹朵''等等,这是因为宝宝发音器官发育不够完善,听觉的分辨能力和发音器官的调节能力都比较弱,还不能正确掌握某些音的发音方法,不会运用发音器官的某些部位。如在发``吃''、``狮''的音时,舌向上卷,呈勺状,有种悬空感,而小宝宝不会做这种动作,把舌放平了,错音就出来了。对于这种情况,\textbf{父母不要学宝宝的错误发音,而应当用正确的语言来和宝宝说话,时间一长,在正确语音的指导下,宝宝发音就逐渐正确了。}

% \textbf{宝宝语言发育进度表}

% \begin{longtable}[]{@{}
%   >{\raggedright\arraybackslash}p{(\columnwidth - 2\tabcolsep) * \real{0.5000}}
%   >{\raggedright\arraybackslash}p{(\columnwidth - 2\tabcolsep) * \real{0.5000}}@{}}
% \toprule()
% \begin{minipage}[b]{\linewidth}\raggedright
% 月龄
% \end{minipage} & \begin{minipage}[b]{\linewidth}\raggedright
% 语言发育水平
% \end{minipage} \\
% \midrule()
% \endhead
% 1个月 & 能发出细小的喉音 \\
% 2个月 & 能发出咕咕声,发单元音如a,o,e,交流发音 \\
% 3个月 & 能发出两个音节的音 \\
% 4个月 & 兴奋起来能呼吸加深,屏气,大声笑 \\
% 5个月 & 发咕噜声,主动和玩具"说话";头转向声音处,可发出尖叫声 \\
% 6个月 & 叫名字会转过头来,会叫喊 \\
% 7个月 & 发``ao''、``hao''、``h''的音,对熟悉的人发音 \\
% 8个月 & 发``da,da'',或相当于它的音。对``不,不''的音调或声音有反应 \\
% 9个月 & 试模仿听到的音,发"妈" "妈"的音,但无所指 \\
% 10个月 &
% 问他"妈妈在哪?爸爸在哪?"会转头找;会有意识地叫爸爸,说再见时挥手 \\
% 11个月 & 开始出现难懂的话,有意识地叫"妈妈" \\
% 12个月 & 能说4个字,能找到成人所说的东西 \\
% 13个月 & 指出一个身体部位,如"嘴",问他会指出来 \\
% 14个月 & 指出两个身体部位,问他会指出来 \\
% 15个月 & 尿布湿时会示意,对自己所要的物品会指出或发音 \\
% 18个月 & 会要吃、喝的东西,用语言表达自己的需要 \\
% 21个月 & 会说"不要",能说出两件实物的名称,指出身体7个部位 \\
% 24个月 & 能连续说出3个字,能用``你'',``我'',``他''等词 \\
% 30个月 & 能说出自己的名字,知道大和小 \\
% 36个月 & 能说出自己的性别,识别两种颜色 \\
% \bottomrule()
% \end{longtable}

% %ux8bedux8a00ux73afux5883ux590dux6742}{%
% \subsubsection{5
% 语言环境复杂}%ux8bedux8a00ux73afux5883ux590dux6742}}

% 有些特殊家庭父母、爷爷奶奶、保姆各有各的方言,语言环境复杂,多种方言并存,这会使正处于模仿成人学习语言的小宝宝产生困惑,导致说话晚。因此,在0\textasciitilde2岁这个学习语言的关键期,家人\textbf{应着重教宝宝一两种正确的语言}。

% %ux56dbux7ed9ux5b9dux5b9dux7528ux836fux768410ux4e2aux9519ux8bef}{%
% \subsection{四、给宝宝用药的10个错误}%ux56dbux7ed9ux5b9dux5b9dux7528ux836fux768410ux4e2aux9519ux8bef}}

% %ux968fux610fux670dux7528ux6b62ux75dbux836f}{%
% \subsubsection{1.
% 随意服用止痛药}%ux968fux610fux670dux7528ux6b62ux75dbux836f}}

% 宝宝时常会肚子痛,有些妈妈一着急就给服用止痛药。虽然这样可以让宝宝的肚子痛减轻了,但却有可能带来严重后果。因为,如果宝宝是因急性阑尾炎、肠套叠等引起的肚子疼痛,服用止痛药会由于疼痛减轻而掩盖真正的病情,实际上病情仍在发展,由此延误诊治时机,甚至危及生命。

% \textbf{纠正错误法}:腹痛可由多种病引起,宝宝腹痛时,在病因未明之前妈妈不可随意给宝宝乱用止痛药,以免掩盖病情,造成病情发展。

% %ux81eaux4f5cux4e3bux5f20ux52a0ux5927ux836fux91cf}{%
% \subsubsection{2.
% 自作主张加大药量}%ux81eaux4f5cux4e3bux5f20ux52a0ux5927ux836fux91cf}}

% 宝宝生病了,妈妈都会着急,恨不得一下子好起来。但疾病往往有一个发展的过程,不会很快就见好。于是,有些妈妈便自作主张地给宝宝加大药量,或再给宝宝服用另一种作用相同的药。比如,看到宝宝体温不退,便把退热药加大或再加上点别的退热药。这样虽然可能把宝宝的体温降下来了,但宝宝却由于出汗太多而发生虚脱,使身体``雪上加霜''。不仅不能使病情快些好转,病情加重,还容易导致不良结果。

% \textbf{纠正错误法}:妈妈一定要严格遵从医嘱服药,病情不见好转随时请教医生,切不可自作主张随意加大药量或乱加药。

% %ux670dux7528ux6210ux4ebaux836f}{%
% \subsubsection{3. 服用成人药}%ux670dux7528ux6210ux4ebaux836f}}

% 宝宝感冒了,家里如果暂时没有治疗儿童的感冒药,一些妈妈就会给宝宝服用大人的,觉得宝宝虽小,但其实还不是大人的缩影,用大人的药没问题,只要减点量就行了。

% 尽管某些成人药物可以给儿童服用,但这仅限于一般药物,并不是所有的成人药只要减少剂量都能给宝宝服用。因为,宝宝正常生长发育中,肝肾器官还没有完全发育成熟,排毒解毒作用较弱,易引发不良后果。

% \textbf{纠正错误法}:宝宝身体不舒服,最好在医生指导下服用儿童药物。即使服用可以使用的成人药,在服用之前务必要先请教一下医生。

% %ux628aux836fux653eux5728ux5976ux91ccux5582}{%
% \subsubsection{4.
% 把药放在奶里喂}%ux628aux836fux653eux5728ux5976ux91ccux5582}}

% 有些宝宝就是喂不进去药,一些妈妈就把药弄碎了,放在宝宝所喜欢的奶里,结果宝宝不知不觉地就把药吃进去了。但这种方法是不科学的,因为\textbf{奶类是一种中和剂},与某些药物放在一起,容易与药物结合形成另一种物质,从而降低药效,使药物不能充分发挥作用,影响治疗效果。【妈咪爱可以不?】

% \textbf{纠正错误法}:由于有些药物与奶类一起服用会影响药效,因此给宝宝用药前一定要向医生询问清楚,适宜的做法是采用单独的喂药方法。

% %ux670dux7528ux4e00ux6837ux7684ux611fux5192ux836f}{%
% \subsubsection{5.
% 服用一样的感冒药}%ux670dux7528ux4e00ux6837ux7684ux611fux5192ux836f}}

% 宝宝的免疫功能尚未完善,免不了经常被不同类型的病毒所侵扰,引起感冒。有些妈妈一看宝宝感冒了,就拿出感冒药给宝宝服用,而这种做法是不恰当的。因为感冒分风寒感冒、风热感冒等几种类型。如果不辨证用药,则不利于宝宝的病情好转,对宝宝的健康产生不利影响。

% \textbf{纠正错误法}:遇到这种情况应该先辨明病情,看宝宝是风寒感冒还是风热感冒,然后再选方用药,有的放矢地选择药物才能收到满意的疗效。

% %ux611fux5192ux670dux7528ux6297ux751fux7d20}{%
% \subsubsection{6.感冒服用抗生素}%ux611fux5192ux670dux7528ux6297ux751fux7d20}}

% 宝宝感冒发烧了,妈妈赶快给服用药物,除了抗感冒药之外,同时还给服用一些抗生素,认为这样宝宝的感冒才能好。

% 其实,\textbf{大部分感冒都是病毒感染}所致,而抗生素对病毒没有杀灭作用。所以,宝宝感冒时如果没有并发细菌感染,不需要服用抗生素,尤其是广谱抗生素。服用抗生素不仅对病毒没有什么作用,滥用还会导致体内的细菌产生抗药性。待宝宝一旦真被细菌感染了,可能细菌已经出生了一定的耐药性,从而使药效减弱,治疗时不得不加大药物剂量或联合用药,还可能会使药物副作用增大,引起药物疹、白血球减少、肝肾损坏等不良反应。

% \textbf{纠正错误法}:宝宝感冒发烧时,有条件最好做一下血液化验,看看是否是细菌感冒所致。如果只是病毒感染,没有必要给宝宝服用抗生素,可在医生指导下服用一些对病毒有抑制作用的宝宝抗感冒中成药。一旦宝宝合并了细菌感染,立即去看医生,采用足量有效的抗生素进行治疗。

% %ux957fux671fux6ee5ux7528ux6297ux751fux7d20}{%
% \subsubsection{7.长期滥用抗生素}%ux957fux671fux6ee5ux7528ux6297ux751fux7d20}}

% 宝宝的消化功能尚未发育完善,特别是添加辅食,稍有不合适就容易发烧腹泻。妈妈通常会给宝宝服用抗生素。有时宝宝的腹泻已经见好了,可生怕治疗不彻底,还一直给宝宝服用抗生素。

% 这种做法是错误的。长期不恰当地使用抗生素,会使肠道正常的寄生菌被抑制甚至被杀死,引起肠道内菌群失调,导致致病菌乘虚而入,在病症肠道内大量繁殖,加重腹泻或导致腹泻久治不愈,甚至引起全身细菌性感染。

% \textbf{纠正错误法}:宝宝腹泻时,妈妈不一定非给服用抗生素,特别是长期滥用。因为,腹泻并不一定都是细菌感染引起的,最好先到医院做一下便常规检查。如果是病毒或霉菌感染,特别是发生在\textbf{秋季轮状病毒引起的腹泻,不可使用抗生素}。即便是细菌感染引起,也应该先到医院做一下便细菌培养,确定病菌的种类,以选用有效抗生素。但给宝宝的用药时间一定要恰当,要在医生的指导下用药。

% %ux628aux5065ux813eux6d88ux98dfux7247ux5f53ux4f5cux5c0fux96f6ux98dfux7ed9ux5b9dux5b9dux5403}{%
% \subsubsection{8.把健脾消食片当作小零食给宝宝吃}%ux628aux5065ux813eux6d88ux98dfux7247ux5f53ux4f5cux5c0fux96f6ux98dfux7ed9ux5b9dux5b9dux5403}}

% 宝宝的消化功能还未发育成熟,容易产生消化不良。于是,妈妈就把一些能够促进消化的健脾消食片(一般这类药片都口味酸甜)当作小零食经常给宝宝吃。认为多吃点没关系,可以帮助宝宝消化吸收食物。专家指出,健脾消食类药物,不适宜经常当作小零食给宝宝吃。因为,无论是什么药或多或少都有\textbf{副作用},经常服用不利于宝宝的健康。

% \textbf{纠正错误法}:不要把健脾消食片当作小零食常给宝宝吃。如果宝宝食欲不佳,在服用健脾消食药的同时,还应注意培养宝宝养成定时定量饮食的规律,少食生冷、甘甜、油腻等食物,不让宝宝由着性子随便吃零食。

% %ux5b9dux5b9dux8179ux6cfbux65f6ux5988ux5988ux7ed9ux670dux7528ux5421ux54ccux9178}{%
% \subsubsection{9.
% 宝宝腹泻时,妈妈给服用吡哌酸}%ux5b9dux5b9dux8179ux6cfbux65f6ux5988ux5988ux7ed9ux670dux7528ux5421ux54ccux9178}}

% 一些妈妈在宝宝发生腹泻时,会给宝宝服用治疗腹泻的吡哌酸。吡哌酸是一种喹诺酮类抗生素,这类抗生素通常以``沙星''冠名,如环丙沙星、氧氟沙星等。大量临床资料表明,\textbf{吡哌酸会抑制软骨发育},从而影响宝宝的骨骼生长,导致身高增长受限。

% \textbf{纠正错误法}:宝宝发生腹泻时,即使确定是细菌感染也不可自行给宝宝服用吡哌酸,其中包括哺乳妈妈,一定都要慎用。

% %ux603bux7ed9ux5b9dux5b9dux670dux7528ux677fux84ddux6839ux51b2ux5242}{%
% \subsubsection{10.
% 总给宝宝服用板蓝根冲剂}%ux603bux7ed9ux5b9dux5b9dux670dux7528ux677fux84ddux6839ux51b2ux5242}}

% 板蓝根具有清热解毒、凉血消肿的作用,有些人把它称为中药中的``抗生素'',但却认为它不同于西药中的抗生素,它没有什么副作用。虽然板蓝根冲剂对于病毒性感冒、水痘疾病等有一定的防治作用,但它毕竟是药物,经常服用并不妥当。俗话说,``是药三分毒'',经常服用板蓝根会给宝宝的健康带来一些不必要的影响。

% \textbf{纠正错误法}:如果宝宝与感冒患者密切接触了,连服几天板蓝根有助于抵抗感冒病毒。但\textbf{预防感冒的最好方法是合理喂养、多带宝宝去户外活动及晒太阳}。而且,要\textbf{按时接受预防注射},以提高宝宝自身的抗病能力。

% %ux4e94ux5b66ux6e38ux6cf3ux7684ux597dux5904}{%
% \subsection{五、学游泳的好处}%ux4e94ux5b66ux6e38ux6cf3ux7684ux597dux5904}}

% 游泳是一项很好的锻炼项目,它综合了水、空气、日光的作用,通过游泳可以增强孩子的心脏收缩功能,增强孩子的肺活量,促进全身肌肉的发育,有利于增强身体抗病的能力,有利于体型美的发展,促进孩子智力发育,使孩子吃得饱、睡得香、少生病。孩子游泳的场所水质应清洁无污染,气温不要低于28℃,水温应不低于26℃,开始在水里的时间不超过2-5分钟,出水后应赶快用毛巾保暖,以后逐渐延长到每次下水10-15分钟。带孩子游泳应注意:下水前要活动一下四肢,并用水浸湿胸部和头部,然后再入水,下水后发现孩子有寒冷感觉时,应赶快出水,用毛巾擦干身上的水并保暖;不要让孩子饥饿的时候或饭后立即去游泳;出汗时不能立即下水。

% %ux516dux5982ux4f55ux7ed9ux5b69ux5b50ux9009ux62e9ux8863ux7269ux978bux5e3d}{%
% \subsection{六、如何给孩子选择衣物鞋帽}%ux516dux5982ux4f55ux7ed9ux5b69ux5b50ux9009ux62e9ux8863ux7269ux978bux5e3d}}

% \hspace{0pt}\includegraphics[width=2.46154in,height=2.41958in]{media/rId1496.png}\hspace{0pt}

% 给孩子选购衣物,首先要注意穿着舒服,厚薄合适。

% 由于孩子皮肤娇嫩,出汗多,所以给孩子穿\textbf{棉布衣服}最好。棉布衣服具有柔软、吸汗、透气性好、保暖性强、好洗等优点。孩子的内衣要穿纯棉衣裤,轻柔暖和,洗换也方便。孩子的\textbf{毛衣不要高领的},否则会刺激孩子的皮肤。冬季,孩子一般都要穿棉衣棉裤,棉花要松软,不要做得太厚,有碍孩子玩耍活动。夏季,应给孩子用浅色的小薄棉布做汗衫、短裤、背心,孩子穿着舒服、吸汗,也容易散热。不宜穿涤纶料的衣服,因为化纤制品不吸汗,有时还会产生静电刺激孩子皮肤。在款式上,要选择简单、宽松、便于脱穿和便于生活的式样,要考虑到孩子生长发育的特点。由于小孩的关节和骨骼正处在生长发育阶段,如给孩子选择类似牛仔裤、紧身衣式的服装,会影响血液循环,不利于孩子生长和发育。在选择帽子、大衣、披风时,可以选择\textbf{美观大方新颖别致}的款式,同时也要注意\textbf{脱穿方便},这样既可以体现孩子朝气蓬勃、天天向上的精神面貌,又照顾了孩子的生理特点。孩子的鞋要注意合适,跟脚、柔软、轻便,鞋面透气性要好,鞋底不宜太厚,也不宜太软。随着孩子的生长发育,\textbf{一般3个月需要换一号鞋子}。

% %ux4e03ux9884ux9632ux5b69ux5b50ux53e3ux5403}{%
% \subsection{七、预防孩子口吃}%ux4e03ux9884ux9632ux5b69ux5b50ux53e3ux5403}}

% 在孩子刚开始学说话的时候,家长就要注意预防孩子发生口吃。口吃俗称结巴,是一种常见的语言障碍。口吃男孩多于女孩,男孩约占4\%,女孩约占2\%。发生的年龄常在2-5岁,这个阶段是语言发育最为迅速的时期。口吃有没有遗传性呢?回答是没有的,不过有资料报道,双亲都有口吃的,其子女约有67\%的也患口吃,双亲中1人患口吃,其子女患口吃的约占40\%。所以父母患口吃的,孩子发生率要高些。口吃的发生与环境变化有关,像发生火灾、地震、失去亲人等引起孩子精神紧张的因素都可引起口吃。但\textbf{最常见的原因,是在孩子开始学说话时,家长操之过急,做过多的矫正或恐吓,强迫孩子说话,使孩子心理上感到压抑,说话时心急发慌,而发生口吃}。还有的是由于孩子模仿性较强,模仿他人口吃而自己也形成了口吃的习惯。总之,孩子在学说话时,家长应注意正确指导孩子说话,即使有的孩子说话慢一点也没关系,因为孩子在语言发展时期可存在4-6个月的正常差异。

% %ux516bux6ce8ux610fux996eux98dfux536bux751f}{%
% \subsection{八、注意饮食卫生}%ux516bux6ce8ux610fux996eux98dfux536bux751f}}

% \begin{enumerate}
% \def\labelenumi{\arabic{enumi}.}
% \item
%   宝宝的餐具和奶瓶、小勺、小碗、饮水杯等要独自使用,清洗时要用清水冲洗,使用前最好用开水烫一下。要定期煮沸消毒,放在清洁卫生的地方。
% \item
%   给孩子喂食时,先把手洗净。喂食时,千万\textbf{不能把小勺放在自己口中试温度后喂孩子},这样容易将大人的病菌带给孩子而致病。
% \item
%   \textbf{大人感冒时要戴上口罩照顾孩子},以免通过空气传染给孩子。家长如患有肠道疾病,一定要用肥皂洗干净手再接触孩子,最好暂时不要动孩子入口的东西,以免不慎传染给孩子。
% \end{enumerate}

% %ux4e5dux4e3aux4f55ux5468ux5c81ux8fd8ux4e0dux5f00ux53e3ux8bb2ux8bdd}{%
% \subsection{九、为何周岁还不开口讲话}%ux4e5dux4e3aux4f55ux5468ux5c81ux8fd8ux4e0dux5f00ux53e3ux8bb2ux8bdd}}

% \textbf{宝宝说出第一个词的时间差异是比较大的}。早的从9个月就能有意识地叫人,一般的宝宝到1岁可以发出简单的音,如会叫``爸爸、妈妈''等,但也有的孩子甚至到2岁仍很少开口说话,不久却突然讲话了,一下子会说好多的话,这些都属于正常。

% \textbf{宝宝语言的发展首先是听懂成人的语言,然后才自己开口说话。}如果1岁左右的宝宝对成人所说的一些词语能做出相应的反应,如问宝宝:``妈妈呢?''他会转过头看或用手去指,并且经常地咿呀学语,那么父母可尽管放心,宝宝一定会学会说话的,只是时间早晚的问题。

% \textbf{外部环境也是影响宝宝语言发展的因素之一。}成人是否积极地为宝宝创造听和说的语言环境,在照看宝宝时是沉默寡言还是经常和宝宝说话,都会影响婴儿对语言的理解及开口说话。家长应积极创造听说条件,促使宝宝语言发展。

% 还有的宝宝营养不良,发育迟缓,甚至患有慢性疾病,也会影响与成人语言交流的积极性,使语言发展落后。

% %ux5341ux8ba9ux5b9dux5b9dux53d1ux6cc4}{%
% \subsection{十、让宝宝发泄}%ux5341ux8ba9ux5b9dux5b9dux53d1ux6cc4}}

% 1岁左右的宝宝有时会向父母提出一些不合理,或父母力所不能及的要求。如,要玩剪刀,要进厨房,或非要吃刚熬出锅的粥等。当这些要求不能得到满足时,宝宝就会大哭大闹。这时,父母首先要耐心劝解,告诉他这样做危险,不能这样做。如果孩子不听,父母应想办法分散他的注意力,如用一件宝宝平日十分喜欢的物品逗引他,或带他去看画册\ldots\ldots 如果宝宝仍不肯罢休,可以采取暂不理睬的方法,让他自己去哭一阵,待发泄完毕后,再和他讲清道理。

% 宝宝在1岁左右,为发泄心中的不满,以大哭大闹的形式要挟父母,逼迫父母屈服。成人如果总是迁就他,只要一哭,就无条件地满足他的任何要求,就会使宝宝认为只要自己一发脾气,一切都会如愿以偿,以后遇到类似情况,便会变本加厉,愈闹愈凶,从而养成任性、不讲理的毛病,而坏习惯一经养成,再去纠正就不容易了。对宝宝的一切无理要求,父母必须用严肃的表情和严厉的语言加以制止,以表示父母是不赞成他这样做的。宝宝因无理要求被拒绝而哭闹发泄几次,对他的健康并不会有多大影响,父母不必为此担心。应让宝宝从小懂得,每个人都要约束自己的行为和情绪,这对培养宝宝的自制力和良好的行为习惯是十分重要的。

% %ux5341ux4e00ux5468ux5c81ux5b9dux5b9dux7684ux4f5cux606fux65f6ux95f4ux5b89ux6392}{%
% \subsection{十一、周岁宝宝的作息时间安排}%ux5341ux4e00ux5468ux5c81ux5b9dux5b9dux7684ux4f5cux606fux65f6ux95f4ux5b89ux6392}}

% 1周岁的宝宝应当建立起一种比较规律化的生活制度,这对于宝宝的健康成长是十分有益的。周岁的作息制度可参照下表安排:

% \begin{longtable}[]{@{}
%   >{\raggedright\arraybackslash}p{(\columnwidth - 2\tabcolsep) * \real{0.5000}}
%   >{\raggedright\arraybackslash}p{(\columnwidth - 2\tabcolsep) * \real{0.5000}}@{}}
% \toprule()
% \begin{minipage}[b]{\linewidth}\raggedright
% 时间
% \end{minipage} & \begin{minipage}[b]{\linewidth}\raggedright
% 活动
% \end{minipage} \\
% \midrule()
% \endhead
% 上午7:00-7:30 & 起床,清洗,排便 \\
% 上午7:30-8:00 & 早饭 \\
% 上午8:00-11:00 & 室内、外活动及玩耍 \\
% 上午11:00-11:30 & 饭前清洗及准备 \\
% 上午11:30-12:00 & 午饭 \\
% 下午12:00-3:00 & 睡眠 \\
% 下午3:00-4:00 & 室内、外活动及玩耍 \\
% 下午4:00-4:30 & 吃点心 \\
% 下午4:30-5:30 & 室内、外活动及玩耍 \\
% 下午5:30-6:00 & 饭前清洗及准备 \\
% 下午6:00-6:30 & 吃晚饭 \\
% 下午6:30-8:00 & 室内、外活动及玩耍 \\
% 晚上8:00-8:30 & 睡前清洗 \\
% 晚上8:30 & 睡眠 \\
% \bottomrule()
% \end{longtable}

% %ux5341ux4e8cux5468ux5c81ux5b9dux5b9dux5907ux5fd8ux5f55}{%
% \subsection{十二、周岁宝宝备忘录}%ux5341ux4e8cux5468ux5c81ux5b9dux5b9dux5907ux5fd8ux5f55}}

% 终于1岁了,宝宝可以蹒跚学步,同时也意味着父母更加辛苦了。

% \begin{enumerate}
% \def\labelenumi{\arabic{enumi}.}
% \item
%   如果宝宝睡有木栏的小床,需要降低床垫,以使围栏显得更高。宝宝站立在小床上,围栏的高度应在宝宝的肩膀或\textbf{下巴}处。
% \item
%   宝宝喜欢开门和关门,要小心宝宝的手指。
% \item
%   宝宝会爬楼梯了,所以注意关好房门。
% \item
%   使用不易打碎的塑料餐具。
% \item
%   远离洗衣机、吸尘器等家用电器。
% \item
%   时刻注意关好门窗。
% \item
%   家中如果种植了花草,最好问清楚是否有毒,以免宝宝误食。【只种菜吧】
% \item
%   如果陌生的动物靠近宝宝,也许会造成宝宝的惊恐,因此,家长必须在宝宝身边。
% \item
%   只有安静坐下来,才能给宝宝吃东西,\textbf{避免让孩子边吃边玩}。
% \item
%   外出吃饭时,不要让宝宝在餐馆里自己转悠,以免引起烫伤或磕碰。
% \item
%   家中有客人来访,将客人的提包等物品放到宝宝碰不到的地方。
% \end{enumerate}

% %ux7b2cux5341ux4e09ux8282ux5e7cux513fux7684ux65e5ux5e38ux751fux6d3bux62a4ux7406}{%
% \subsection{13第十三节幼儿的日常生活护理}%ux7b2cux5341ux4e09ux8282ux5e7cux513fux7684ux65e5ux5e38ux751fux6d3bux62a4ux7406}}

% %ux4e001018ux4e2aux6708ux513fux7ae5ux7684ux751fux6d3bux5b89ux6392}{%
% \subsection{一、10\textasciitilde18个月儿童的生活安排}%ux4e001018ux4e2aux6708ux513fux7ae5ux7684ux751fux6d3bux5b89ux6392}}

% 在安排1\textasciitilde3岁儿童的生活上,吃和睡是中心环节。除此之外,再配以其他活动,形成一定的制度。

% 10至18个月的儿童,每天要进餐5次,两餐间隔4小时左右。白天睡两次,每次两小时左右,晚上睡10个小时,一昼夜总计14个小时左右。

% 一日安排举例:

% \begin{longtable}[]{@{}
%   >{\raggedright\arraybackslash}p{(\columnwidth - 2\tabcolsep) * \real{0.5000}}
%   >{\raggedright\arraybackslash}p{(\columnwidth - 2\tabcolsep) * \real{0.5000}}@{}}
% \toprule()
% \begin{minipage}[b]{\linewidth}\raggedright
% 时间
% \end{minipage} & \begin{minipage}[b]{\linewidth}\raggedright
% 活动内容
% \end{minipage} \\
% \midrule()
% \endhead
% 6:00 - 7:00 & 起床、大小便、洗手脸、早饭 \\
% 7:00 - 9:00 & 游戏 \\
% 9:00 - 11:00 & 喝水、第一次睡眠 \\
% 11:00 - 11:30 & 起床、小便、洗手、午饭 \\
% 11:30 - 13:30 & 游戏、喝水 \\
% 13:30 - 15:30 & 第二次睡眠 \\
% 15:30 - 16:00 & 起床、小便、午点 \\
% 16:00 - 18:30 & 游戏、喝水 \\
% 18:30 - 19:00 & 洗手、晚饭 \\
% 19:00 - 20:00 & 盥洗、小便、准备睡眠 \\
% 20:00 - 6:00 & 夜间睡眠 \\
% \bottomrule()
% \end{longtable}

% (22:30吃奶)

% %ux4e8c1836ux4e2aux6708ux513fux7ae5ux7684ux751fux6d3bux5b89ux6392}{%
% \subsection{二、18\textasciitilde36个月儿童的生活安排}%ux4e8c1836ux4e2aux6708ux513fux7ae5ux7684ux751fux6d3bux5b89ux6392}}

% 18\textasciitilde36个月的儿童每天进餐四次,两餐间隔约4个小时,白天睡一次,大约两个半小时,晚上睡十个半小时,一昼夜总计约13个小时。

% 以下是一天的生活安排举例:

% \begin{longtable}[]{@{}
%   >{\raggedright\arraybackslash}p{(\columnwidth - 2\tabcolsep) * \real{0.5000}}
%   >{\raggedright\arraybackslash}p{(\columnwidth - 2\tabcolsep) * \real{0.5000}}@{}}
% \toprule()
% \begin{minipage}[b]{\linewidth}\raggedright
% 时间
% \end{minipage} & \begin{minipage}[b]{\linewidth}\raggedright
% 活动内容
% \end{minipage} \\
% \midrule()
% \endhead
% 6:30 - 7:30 & 起床、大小便、洗手脸 \\
% 7:30 - 8:00 & 早饭 \\
% 8:00 - 9:00 & 游戏 \\
% 9:00 - 11:00 & 喝水、小便、游戏 \\
% 11:00 - 11:30 & 洗手、午饭 \\
% 11:30 - 12:00 & 小便、准备睡眠 \\
% 12:00 - 14:30 & 午睡 \\
% 14:30 - 15:00 & 起床、洗手、午点 \\
% 15:00 - 18:00 & 游戏(16:30喝水) \\
% 18:00 - 18:30 & 洗手、晚饭 \\
% 18:30 - 19:00 & 游戏 \\
% 19:00 - 20:00 & 盥洗、小便、准备睡眠 \\
% 20:00 - 6:30 & 夜间睡眠 \\
% \bottomrule()
% \end{longtable}

% %ux4e09ux6ce8ux91cdux4e30ux5bcc13ux5c81ux513fux7ae5ux7684ux751fux6d3bux5185ux5bb9}{%
% \subsection{三、注重丰富1\textasciitilde3岁儿童的生活内容}%ux4e09ux6ce8ux91cdux4e30ux5bcc13ux5c81ux513fux7ae5ux7684ux751fux6d3bux5185ux5bb9}}

% 1\textasciitilde3岁的儿童随着年龄的增长,睡眠时间逐渐减少。怎样使儿童在清醒时保持活泼愉快的心情,既不产生厌倦情绪,也不过于疲劳呢?

% 我们知道,儿童神经系统的特点是兴奋容易扩散,而不易集中,不能持久地保持注意力。所以在孩子玩的时候,活动方式要尽量\textbf{多种多样}。例如,除了玩各种玩具以外,还可以让孩子参与一些力所能及的家务活动,即便做不好,甚至给大人添了麻烦,但会使孩子觉得新奇有趣,有助于孩子保持积极愉快的情绪。千万不要过分限制孩子的活动范围,这也不准摸,那也不准动,孩子多问几个为什么就厌烦。有的家长还认为少说、少动,成天处于消极、被动状态的孩子才是乖孩子,听话的好孩子。这是由于不了解儿童心理特点而产生的一种误解,对儿童身心健康的成长是非常不利的。

% %ux56dbux57f9ux517bux513fux7ae5ux7231ux6e05ux6d01ux8bb2ux536bux751fux7684ux4e60ux60ef}{%
% \subsection{四、培养儿童爱清洁、讲卫生的习惯}%ux56dbux57f9ux517bux513fux7ae5ux7231ux6e05ux6d01ux8bb2ux536bux751fux7684ux4e60ux60ef}}

% 爱清洁,讲卫生的习惯应从小培养。

% \hspace{0pt}\includegraphics[width=2.41958in,height=2.53147in]{media/rId1527.png}\hspace{0pt}

% 1\textasciitilde3岁的幼儿每天要按时洗脸,饭前便后洗手,晚上洗脚,女孩子要洗屁股,冬季要每周洗澡,洗头一次,夏季每天一次。儿童一般都喜欢玩水和洗澡,但也要注意避免造成个别儿童对洗澡有反感。水温一般在37°C左右,成年人手试着冷热合适即可。洗头时不要让水流进儿童的耳朵和眼睛里。对不爱洗澡的儿童,可在澡盆中放些能浮在水面的玩具。理发最好由家长来做,因为大多数儿童都不喜欢理发,去理发馆,人地两生,很容易引起大哭大闹。

% 培养儿童经常保持手脸干净,像嘴边的饭粒、菜汤、鼻涕都要随时擦净。要培养儿童有鼻涕就感到不舒服,并能主动让人擦的习惯。到1岁半就可经常提醒他用手帕自己去擦,擦不干净时,成人再帮助。手脏了随时给洗,逐渐养成自动要求洗手的习惯。人的双手接触外界机会最多,防护又最差,最容易受污染,特别是东摸西摸到处玩耍的儿童。有人观察,经常寄居在手和指甲缝里的病菌有30多种。

% 洗手最好用温水,并一定要用香皂。用香皂比不用香皂的杀菌效果要大7\textasciitilde8倍。用香皂,手上的细菌能除去90\%左右。如果从1岁开始就培养孩子早晚和饭前便后洗手的习惯,到3岁时,一般就都能学会自己洗手了。

% %ux4e94ux57f9ux517bux513fux7ae5ux7684ux72ecux7acbux751fux6d3bux80fdux529b}{%
% \subsection{五、培养儿童的独立生活能力}%ux4e94ux57f9ux517bux513fux7ae5ux7684ux72ecux7acbux751fux6d3bux80fdux529b}}

% 家长要注意从小让孩子做一些力所能及的事,逐渐培养孩子的独立生活能力。例如让1岁左右的儿童练习用碗喝水,继而再用碗喝奶,学会不呛。1岁半的儿童,就可以让他练习自己用勺吃饭。开始时,儿童虽然用不好,可是又抓着勺不放,成年人可另拿一只勺喂他,这样既可让孩子练习自己吃饭,又可保证孩子吃饱。穿脱衣服时,开始让他配合着伸胳膊伸腿,逐渐培养孩子自己穿脱简单衣物。除了让孩子逐渐学会处理自己的事以外,还要培养他帮助家里做事,如取送一些简单物品,帮助开门、关门等。在培养儿童的独立生活能力时,成年人要随时给予必要的帮助,鼓励孩子克服困难。有些父母往往嫌麻烦,认为还不如自己干省事,这样会养成孩子的依赖性。

% %ux516d3ux5c81ux513fux7ae5ux7684ux953bux70bcux65b9ux6cd5}{%
% \subsection{六、3岁儿童的锻炼方法}%ux516d3ux5c81ux513fux7ae5ux7684ux953bux70bcux65b9ux6cd5}}

% 冷水洗脸、洗手、洗脚:儿童身体的局部受寒冷刺激,会反射性地引起全身一系列复杂反应,从而增加儿童的耐寒能力。这与用温水、香皂洗手、洗脸的意义不同,前者主要是为了锻炼,后者是为了清洁卫生。晚上盥洗时,还是用温水好,避免由于冷水的作用,引起儿童神经兴奋,影响睡眠。

% \textbf{冷水擦身}:这是冷水锻炼中比较缓和的方法,先把擦洗用的毛巾,在冷水中浸透,稍稍拧干,从擦儿童的四肢开始,顺序擦颈部、胸部、腹部、背部。擦过的和尚未擦的部分都要用干的大毛巾盖好。用湿毛巾擦完后,再用干毛巾擦,开始擦时的水温最好与体温相等,每隔两三天降低1°C,冬季一般降至22°C,擦时的室温以16\textasciitilde18°C为宜,如果因故间断,重新开始时,应按间断前最后一次的水温,或稍高一些。夏季随自然温度用冷水擦身。有人主张在水里加一些盐或酒精,比例是75毫升水,加15克\textbf{盐}或15毫升酒精,盐或酒精可作用于神经末梢,能加强神经系统的紧张度,这对增进儿童的健康是有好处的。

% %ux4e03ux4e3aux5b69ux5b50ux521bux9020ux826fux597dux7684ux5c45ux4f4fux73afux5883}{%
% \subsection{七、为孩子创造良好的居住环境}%ux4e03ux4e3aux5b69ux5b50ux521bux9020ux826fux597dux7684ux5c45ux4f4fux73afux5883}}

% (1)
% 房屋朝向:如果条件许可,儿童的居室最好\textbf{门窗向南},保证有充足的阳光,室内的阳光仍有杀灭细菌或降低其毒力的能力,对预防传染性疾病有一定的作用,尤其是在冬季,会让人感到温暖、舒适。如果室内日照不足,居室经常阴暗潮湿,对儿童健康是不利的。在炎热的夏季,特别是在南方,为了避免室内过热,还要注意挂上窗帘遮阳,或采取其他降温措施。

% (2)
% 室温:冬季适宜室温在16-18°C。根据儿童的适应情况,增高或降低23°C也可以,因为医学实验证明,在这个温度范围,儿童皮肤血管舒缩状况稳定,体内各项生理机能良好,儿童的感觉也最为舒适。同时,还要注意白天与黑夜的温度差,最好不超过2\textasciitilde3°C。取暖的炉子不要太靠近儿童。

% (3)
% 湿度:为了保持室内空气有一定的湿度,冬春季节在地面上可以经常洒些清水,或用湿\textbf{拖布拖地}。

% (4)
% 通风换气:日常生活中,我们经常发现在人数较多、通风不够良好的居室内,常会出现一种难闻的气味,尤其是刚从室外突然走进屋子的时候,最易嗅到,长期在这种环境中生活,人体的正常生理机能会受到影响,身体抵抗力下降,特别是儿童更容易感染呼吸道疾病。所以,开窗通风换气,供给居室大量的新鲜空气,是保护儿童健康的重要措施。有资料证明,当室外温度在8-10°C时,打开窗户,通风30分钟,可使室内空气中的细菌污染率降低40\%,而外界温度在零下39°C时,可降低65\%。

% 有儿童的居室,一般在夏季和春秋季的大部分时间,要经常开着窗户,冬季和秋末春初可利用通风小窗或风斗式小窗换气。通风小窗是设在窗户上1/3的地方,能够单独打开的小窗户。风斗式小窗的形式最好,其构造特点是下面为轴,向内开放,回转角度为30度左右,窗框左右两侧有铁制或木制夹板,室外的冷气流经风斗窗流向天花板,呈弧形下降,不直接向儿童头部吹风,对下层气温的影响也不大。在炎热的夏季,既要特别注意室内的空气流通,又要避免穿堂风的冷气流直接吹向孩子。

% (5)
% 打扫卫生:每天清扫居室时,注意不要让尘土飞扬。因为尘土上附有很多微生物,儿童吸进去容易生病。所以扫地前要洒些水,或将扫帚冲湿再扫,擦桌椅要用湿抹布,不要用掸子,因为用掸子等于让尘土搬家,起不到彻底清洁的目的。床单和衣物等也要拿到室外去拍打和扫刷。

% (6)
% 室内的布置和布局:屋内的陈设除了必需的家具以外,如果能摆几盆生机勃勃的花草,配上里面游着金鱼的玻璃缸,墙上再悬挂几张色彩鲜艳、适合儿童年龄特点的画片,这对儿童产生良好的情绪当然很有帮助。

% 对于可以下地行走的小儿,可在居室中划分出一个安全的角落,这个角落最好距离火炉、开水壶、暖瓶等物品远一点,也不要靠门窗,并且四周的物品最好没有锐角,以免小儿不小心或摔倒后碰伤。在这个角落里,可放上小桌椅,桌上摆好孩子喜欢的玩具、画片、小图书等,作为小儿每天活动的场地。小儿在这里会玩得很愉快,并且认为这里就是自己的小天地。小儿玩完玩具或看完画片以后,家长要注意教育孩子把物品\textbf{收拾整齐},以帮助小儿建立良好的生活习惯。

% %ux516bux9009ux5236ux5408ux9002ux7684ux8863ux7740ux548cux88abux8925}{%
% \subsection{八、选制合适的衣着和被褥}%ux516bux9009ux5236ux5408ux9002ux7684ux8863ux7740ux548cux88abux8925}}

% %ux6750ux6599ux9009ux62e9}{%
% \subsubsection{1 材料选择}%ux6750ux6599ux9009ux62e9}}

% 衣着和被褥是小儿生活中接触最多和最密切的物品,小儿的皮肤特别细嫩,选择衣服被褥的材料应以柔软、不掉色、耐洗的棉布为好,并随着季节的不同,要求保持温暖或凉爽。

% %ux88abux8925ux548cux6795ux5934}{%
% \subsubsection{2 被褥和枕头}%ux88abux8925ux548cux6795ux5934}}

% 要给小儿用单独的被褥,不要与母亲同睡一个被窝,这样便于养成按时喂奶和小儿独立生活的习惯。为了防止小儿踢开被子着凉,可将被子的两角(接近头部的那一头)钉上两根布带,拴在床栏上。【可以有】

% 褥子上面的塑料布或者橡胶尿布兜等,都不要直接接触小儿皮肤,中间应有垫子或尿布隔开。这样,一方面可以避免小儿皮肤过敏,也能避免因塑料布、橡胶等不吸水,尿液长时间浸泡小儿皮肤。做尿垫时,可在里面套进一块塑料布,铺时将有塑料布的一面朝下,这样既能保持褥子的清洁卫生,又能防止尿垫在塑料布上滑脱。

% %ux8863ux670dux5f0fux6837}{%
% \subsubsection{3 衣服式样}%ux8863ux670dux5f0fux6837}}

% \hspace{0pt}\includegraphics[width=2.72727in,height=2.25175in]{media/rId1536.png}\hspace{0pt}

% 小儿的衣服应宽大、简单、舒适,不限制活动。夏季最好多裸露身体,但要随着气候的变化注意给小儿增减衣服。冬季小儿可以穿一件厚度合适的棉衣,棉衣比毛衣松软而且保暖性强。

% 小儿的上衣可用和尚服式,不钉纽扣,而用布带松松系上。罩衣用背后开口式。棉背心和连脚裤要选用适合小儿的服装式样,这样既保暖又方便。相反的,如背心裤,在北方多见,是背心与裤子在腰处缝在一起的服装,不适合小儿穿,因为这个时期的小儿生长迅速,衣服来不及修改,穿着就短小了,躯体没有缓冲的余地,容易影响孩子的生长发育。

% 另外,小儿要有衬衣、衬裤和罩衣、罩裤,以便于经常洗换,睡觉时,要脱掉罩衣裤,防止将被褥弄脏。在北方,穿开裆裤的小儿,在寒冷季节,尤其是到室外,为了防止冷风直接吹入臀部,或孩子采取坐位时屁股太凉,可以做屁股帘,帘处做成布的或棉的,上面用布带系在腰间。

% %ux5c0fux513fux7a7fux591aux5c11ux8863ux670dux5408ux9002}{%
% \subsubsection{4
% 小儿穿多少衣服合适}%ux5c0fux513fux7a7fux591aux5c11ux8863ux670dux5408ux9002}}

% 在不同季节里,小儿只要多于一般成年人的衣物就行,成年人要经常摸摸孩子的\textbf{小手和小脚是否暖和}。另外,不要让小儿随大人怕冷或怕热的习惯而给孩子穿得过多或过少。

% %ux978bux5b50}{%
% \subsubsection{5 鞋子}%ux978bux5b50}}

% 鞋子的质地应该较硬,鞋帮稍高,鞋底宽大,能略带足弓和分出左右更好。不要穿鞋过早,一是意义不大,二是麻烦,等到9\textasciitilde10个月孩子学会了站立,慢慢学步时再穿也不晚。

% 幼儿的鞋子最好是松紧口,或是一侧钉上松紧带,另一侧钉扣子,成为扣袢式样,这样便于孩子自己穿脱。

% %ux4e5dux6c99ux53d1ux5bf9ux5c0fux513fux4e0dux5b9c}{%
% \subsection{九、沙发对小儿不宜}%ux4e5dux6c99ux53d1ux5bf9ux5c0fux513fux4e0dux5b9c}}

% \hspace{0pt}\includegraphics[width=2.61538in,height=2.18182in]{media/rId1543.png}\hspace{0pt}

% \textbf{沙发}是一种软体坐卧类家具,它具有美观、舒适等优点,然而,\textbf{小孩不宜坐}。因为小孩的身体正处在生长发育阶段,关节软骨较成人为厚,关节囊较薄,关节周围的韧带薄而松弛,骨骼有机质含量较高,骨骼富有弹性,可塑性很大,因此,这个阶段是决定体形的关键时期,无论是坐、站、走都应当严格要求。一定要做到坐有坐相,站有站相。【不买】

% 由于小孩腿短,坐在沙发上往往双脚不着地。身体靠着沙发背,在重力的作用下,身体呈S形,脊柱弯曲呈弧形,这种姿势对小孩的生长发育是非常不利的,又因臀部下陷,脊柱两侧肌肉、韧带受力不均,还容易引起腰部肌肉慢性劳损。

% 所以,父母不要让孩子长时间蜷卧在沙发里,否则会影响孩子的生长发育。

% %ux5341ux6ce8ux610fux5b69ux5b50ux98dfux7269ux65b9ux9762ux7684ux5371ux9669}{%
% \subsection{十、注意孩子食物方面的危险}%ux5341ux6ce8ux610fux5b69ux5b50ux98dfux7269ux65b9ux9762ux7684ux5371ux9669}}

% %ux9910ux5177}{%
% \subsubsection{1 餐具}%ux9910ux5177}}

% 有些父母喜欢为宝宝选择色彩鲜艳、图案漂亮的餐具,却不知颜料中含有\textbf{铅}。宝宝经常使用这种餐具,摄入过量的铅,会引起铅中毒,影响健康。父母选择餐具要注意其原料无毒、无害,符合国家规定的卫生标准。适宜选用\textbf{无色或浅色}的餐具,图案、花纹在碗的外层,或盘子的边沿,比较安全。

% 宝宝不宜使用玻璃或瓷器类易碎的餐具,以免失手打\textbf{破},造成伤害。

% 不锈钢和铝制成的餐具传热较快,宝宝易被\textbf{烫}着手、嘴。

% 若想让2岁的宝宝学习用筷子,父母必须密切注意,以防不慎扎伤,筷子要短,头不可尖。

% %ux8fdbux98df}{%
% \subsubsection{2 进食}%ux8fdbux98df}}

% 宝宝吃饭慢,喜欢含饭,父母不要催促,如发现宝宝恶心或有梗塞现象,应马上让宝宝把嘴里的食物吐出来。

% 睡觉前要让宝宝把嘴里的食物吐净,睡眠时神经反射活动减弱,咽喉肌肉松弛,食物容易滑入气管,是非常危险的。

% 宝宝的食物,温度要合适。父母食用的热汤、热菜要放在宝宝够不着的地方,父母端开水、热汤进屋要提醒宝宝避让。

% %ux98dfux7269ux4e2dux6bd2}{%
% \subsubsection{3 食物中毒}%ux98dfux7269ux4e2dux6bd2}}

% 宝宝应避免吃凉拌菜,店里买来的熟食也要蒸煮后再吃,以免引起肠道感染和食物中毒。在家庭自制沙拉前,刀具、器皿必须要消毒;豆浆须煮沸后才能给宝宝喝。

% 从冰箱里拿出的食物要烧开煮透才能食用。隔夜食物、吃剩的牛奶,应避免再给孩子吃。

% 家禽的内脏和鱼、虾、蛋、奶都是容易变质的食物,必须特别注意新鲜。

% 开水反复烧开,会产生亚硝酸盐,蒸锅水、热水瓶里的隔夜水再次烧滚后也不宜再给宝宝饮用。

% %ux5341ux4e00ux5bb6ux5c45ux5b89ux5168ux7684ux6ce8ux610fux4e8bux9879}{%
% \subsection{十一、家居安全的注意事项}%ux5341ux4e00ux5bb6ux5c45ux5b89ux5168ux7684ux6ce8ux610fux4e8bux9879}}

% 日常生活中,父母脑海里有没有防患意识?有没有忽略了防患方法?事实上经常有因父母没有采取必要防患措施,而发生使宝宝受到严重伤害的事故。

% %ux5bb6ux5c45ux63d0ux793a}{%
% \subsubsection{1 家居提示}%ux5bb6ux5c45ux63d0ux793a}}

% \begin{enumerate}
% \def\labelenumi{\arabic{enumi}.}
% \item
%   把所有的药物放在上锁的药箱或抽屉里,使宝宝无法拿到。因为,有颜色的药片,最容易引诱宝宝去品尝。
% \item
%   药物和所有化学药品应清楚地写上标签,避免把这种药品放在别的瓶子里,以免错拿错用。
% \item
%   所有的电线应保持良好状态(不要散开或损坏),电线要短,不要让宝宝拉到。若有额外加长的电线,应绑至需要长度。
% \item
%   把所有电插座装上\textbf{安全盖},防止宝宝把手指或其他尖锐物体塞入插座中而导致触电。
% \item
%   家庭电器用完后记得切断电源。
% \item
%   所有楼上的窗户、阳台最好装置铁栏杆,避免把有些东西靠窗口放(如床、椅子等),防止宝宝爬上去,发生意外。
% \item
%   把尖利用具如剪刀、刀、针收好。
% \item
%   不要把喷雾式杀虫剂、喷雾发胶到处放,以免宝宝乱按而喷伤眼睛。其他如火柴和打火机,也应收好,避免宝宝拿到。
% \item
%   房门钥匙应多备一套,以防父母稍离片刻(如下楼倒垃圾时),宝宝把门锁上后又不会打开。
% \item
%   熨烫衣服时必须特别留意,别让宝宝靠近,熨衣板大多都很轻,孩子很容易弄翻后烫伤。
% \item
%   家具应\textbf{重些},以免宝宝拖、拉倒后砸伤。
% \item
%   不要把热水壶、热茶壶放在低矮的地方。
% \item
%   必须确保家中栽种的植物无毒。【只种菜好了】
% \end{enumerate}

% %ux536bux751fux95f4}{%
% \subsubsection{2 卫生间}%ux536bux751fux95f4}}

% \begin{enumerate}
% \def\labelenumi{\arabic{enumi}.}
% \item
%   剃须刀、电吹风、香水、化妆品不应随手乱放,应放在较高的宝宝拿不到的地方。
% \item
%   把所有消毒剂、漂白剂、清洁剂收好。
% \item
%   把马桶盖好。
% \item
%   如果准备好水给宝宝洗澡时,要\textbf{先倒入冷水},以预防意外烫伤,先试好水温,再把宝宝放入浴盆里。【有用!】
% \item
%   浴室不要铺太滑的地砖、板革等,以防宝宝意外跌伤。
% \end{enumerate}

% %ux53a8ux623fux63d0ux793a}{%
% \subsubsection{3 厨房提示}%ux53a8ux623fux63d0ux793a}}

% \begin{enumerate}
% \def\labelenumi{\arabic{enumi}.}
% \item
%   厨房的地面不能铺滑溜的地砖、板革等。
% \item
%   厨房地面和切菜板应保持干净,尽量减少水滴,注意菜刀不要随处放。
% \item
%   及时清理油污,以免滑溜。
% \item
%   用长柄锅煮东西时要注意看管,注意把\textbf{长柄转向炉里边},以免宝宝经过时,推倒锅而烫伤。
% \item
%   不要把东西放在桌布上,特别是水、热油、热菜,防止宝宝拉到桌布后,把这些烫东西拉翻,倒在身上。【要不不要桌布,或者钉牢】
% \item
%   用油炒、炸食物时,不要让宝宝站在旁边,以免热油溅烫。
% \item
%   把塑胶袋收好,不要让宝宝拿到,塑胶袋会意外地套住宝宝的头,造成窒息。
% \item
%   宝宝的餐具,应用不易碎的材料,如硬塑、搪瓷等,以防打碎后割伤。
% \end{enumerate}

% %ux5367ux5ba4ux63d0ux793a}{%
% \subsubsection{4 卧室提示}%ux5367ux5ba4ux63d0ux793a}}

% \begin{enumerate}
% \def\labelenumi{\arabic{enumi}.}
% \item
%   如果家具角太尖,应用布包缠。
% \item
%   把玩具放在宝宝容易拿到的高度,使宝宝不必爬上去,以免跌伤。
% \item
%   不要把玩具随地乱放,以防踏到而摔伤。
% \item
%   定期检查玩具有没有损坏或零件松脱,以防宝宝把零件放入嘴里而造成意外。
% \end{enumerate}

% %ux5341ux4e8cux6ce8ux610fux5b9dux5b9dux8863ux7269ux65b9ux9762ux7684ux5371ux9669}{%
% \subsection{十二、注意宝宝衣物方面的危险}%ux5341ux4e8cux6ce8ux610fux5b9dux5b9dux8863ux7269ux65b9ux9762ux7684ux5371ux9669}}

% (1)
% 颈部衣饰:颈部衣领应柔软、合适。领子又高又硬,宝宝转头时压迫了颈部大血管,造成脑细胞暂时缺血缺氧,硬领紧领还会擦伤宝宝颈部的皮肤。

% 颈部是需要细心保护的部位,冬天要注意透气,适宜穿无领衫。领口不要有带子,以免发生绕颈的危险。宝宝戴领巾、围巾不可过紧,也不可过松,过松容易被各种把手、攀登架、滑梯上凸出部位钩住,产生窒息的危险。

% (2)
% 衣服上小饰品:宝宝喜欢摆弄小玩意儿。衣服上的扣子、别针、小饰品一旦松动、脱落,很容易被宝宝取下含在嘴里,或被戳伤。扣子不要有尖角,不能太小,拉链的拉环要足够大。新做的衣裤上身前要检查一下,看是否留有缝衣针和大头针。照料幼小宝宝的母亲,不宜戴长耳环、项链和戒指。

% (3)
% 裤子:宝宝的裤子不可太紧,裤裆不能太短。太紧的牛仔裤、紧身裤不仅穿着不舒服,还会影响宝宝的发育。裤裆要够长,否则会勒疼勒伤宝宝的外阴部。

% (4)
% 鞋子:幼小的宝宝最好不穿皮鞋,特别不穿硬面底的窄头皮鞋,宝宝穿着不舒服,活动不便,而且容易打滑跌跤。

% 宝宝还不会系鞋带,不要穿有带子的鞋,以免松散的鞋带引起跌倒。

% %ux5341ux4e09ux5b9dux5b9dux4e03ux5927ux751fux7406ux7279ux5f81ux548cux5bf9ux6d17ux62a4ux7528ux54c1ux7684ux8981ux6c42}{%
% \subsection{十三、宝宝七大生理特征和对洗护用品的要求}%ux5341ux4e09ux5b9dux5b9dux4e03ux5927ux751fux7406ux7279ux5f81ux548cux5bf9ux6d17ux62a4ux7528ux54c1ux7684ux8981ux6c42}}

% 0\textasciitilde3岁婴幼儿的皮肤和相关的生殖器官与成人完全不同,需要的洗护用品种类及其功能都与成人产品完全不同。

% \begin{enumerate}
% \def\labelenumi{\arabic{enumi}.}
% \item
%   皮肤面积与体重比例大,易吸收洗护品。据专家统计,婴儿皮肤平均表面积为2,500平方厘米,平均体重为5千克,其比值为500,而成人皮肤平均表面积为18,000平方厘米,平均体重为65千克,其比值为270。同时,婴幼儿皮肤脂肪和脂质较多,对洗护品吸收增加,由此也可导致对过敏性物质吸收增强,所以婴幼儿洗护品必须保证比成人用品有更尚的安全性。
% \item
%   皮肤薄,易受损。婴幼儿皮肤仅有成人1/10厚,表皮层是单层细胞,而成人是多层细胞。真皮也较成人薄,弹性纤维结构稀少,缺乏弹性,更易渗透和摩擦损伤,因此适宜使用爽身粉减少摩擦。
% \item
%   皮肤表面有一层天然酸性保护膜,不可使用碱性洗护品破坏。比如,含皂质、酒精和刺激性成分的产品不能给宝宝使用。
% \item
%   色素层较薄,易被灼伤。应在必要时给宝宝使用低刺激、高品质的防晒用品。
% \item
%   体温调节能力差,易生痱子。宝宝汗腺和血液循环系统还没有发育完善,难以有效地控制体温,易产生热疹。应该注意调节室温和衣服厚薄,选用止痒防痱产品。
% \item
%   免疫系统功能低下,易出现皮肤过敏,如红斑、湿疹、脱皮等。应在医生指导下给宝宝使用安全沐浴露及药物。
% \item
%   泪腺发育不成熟。宝宝不能分泌足够泪水保护眼睛,眨眼较少,所以,必须避免洗发液对宝宝眼睛的伤害。
% \end{enumerate}

% \begin{quote}
% gpt:
% \end{quote}

% \begin{longtable}[]{@{}
%   >{\raggedright\arraybackslash}p{(\columnwidth - 2\tabcolsep) * \real{0.5000}}
%   >{\raggedright\arraybackslash}p{(\columnwidth - 2\tabcolsep) * \real{0.5000}}@{}}
% \toprule()
% \begin{minipage}[b]{\linewidth}\raggedright
% 特征
% \end{minipage} & \begin{minipage}[b]{\linewidth}\raggedright
% 要求
% \end{minipage} \\
% \midrule()
% \endhead
% 1 & 皮肤面积与体重比例大,易吸收洗护品。 \\
% 2 & 皮肤薄,易受损。 \\
% 3 & 皮肤表面有一层天然酸性保护膜,不可使用碱性洗护品破坏。 \\
% 4 & 色素层较薄,易被灼伤。 \\
% 5 & 体温调节能力差,易生痱子。 \\
% 6 & 免疫系统功能低下,易出现皮肤过敏。 \\
% 7 & 泪腺发育不成熟,避免洗发液对宝宝眼睛的伤害。 \\
% \bottomrule()
% \end{longtable}

% %ux5341ux56dbux5982ux4f55ux7ed9ux5b9dux5b9dux9009ux62e9ux6d17ux62a4ux7528ux54c1}{%
% \subsection{十四、如何给宝宝选择洗护用品}%ux5341ux56dbux5982ux4f55ux7ed9ux5b9dux5b9dux9009ux62e9ux6d17ux62a4ux7528ux54c1}}

% 目前,市场上有两大类宝宝洗护产品:清洁与护肤,即洗浴类与护肤类。

% 宝宝清洁产品主要是洗浴品,它们是:

% \begin{enumerate}
% \def\labelenumi{\arabic{enumi}.}
% \item
%   婴儿香皂:成分温和,碱性及刺激性低于成人香皂。主要用于宝宝手脸较脏时,需过水。
% \item
%   婴儿沐浴露:比婴儿香皂多增加了滋润成分,清洁护肤一次完成,用完无需过水。市场上又有婴儿滋润沐浴露,主要针对皮肤过于干燥的婴儿,如秋冬季或每天沐浴的婴儿,适宜选择无化学成分配方,必须在湿毛巾或打湿的皮肤上用。
% \end{enumerate}

% %ux5341ux4e94ux590fux5b63ux5988ux5988ux5e94ux6ce8ux610fux7684ux4e24ux5927ux8befux533a}{%
% \subsection{十五、夏季妈妈应注意的两大误区}%ux5341ux4e94ux590fux5b63ux5988ux5988ux5e94ux6ce8ux610fux7684ux4e24ux5927ux8befux533a}}

% %ux8ba9ux7535ux98ceux6247ux548cux51b7ux6c14ux5bf9ux7740ux5b9dux5b9dux5439}{%
% \subsubsection{1
% 让电风扇和冷气对着宝宝吹}%ux8ba9ux7535ux98ceux6247ux548cux51b7ux6c14ux5bf9ux7740ux5b9dux5b9dux5439}}

% 在闷热的夏季,若是室内温度太高,宝宝的体温就很容易随着气温升高而升高,表现为烦躁不安,哭闹不已,并且尿量减少。许多妈妈生怕宝宝因此发烧,就把电风扇对着宝宝的身体吹风,希望让宝宝凉快一些,殊不知结果并不如意。

% 宝宝被风吹到的体表部位,汗液会蒸发得很快,而风吹不到的部位,汗液却蒸发得很慢,因而使身体汗液排泄\textbf{失衡}。因为宝宝的体温调节系统还未发育成熟,处于一种不稳定的状态,若再加上汗液排泄不均衡而使身体的各个系统和部位发生不协调,就容易引起疾病,轻者流涕、鼻塞,重者则会导致支气管炎,乃至肺炎。

% 明智的做法是:

% \begin{enumerate}
% \def\labelenumi{\arabic{enumi}.}
% \item
%   在房间内选择出一个最佳放置电风扇或冷气的角度,既可以营造出类似自然风的环境,又不会直吹,这样便可降温。【似乎对外面吹更有效】
% \item
%   风量不要开得过大,电风扇以摇头旋转风为宜,使用时间不宜太长。
% \item
%   开电风扇或冷气时,应该给宝宝穿上薄厚适中的衣服,最好在胸前和肚子上戴个小肚兜。
% \item
%   门窗不应紧闭,经常打开通风透气。
% \end{enumerate}

% %ux7528ux516dux795eux4e38ux6765ux9884ux9632ux5b9dux5b9dux751fux70edux75f1ux548cux70edux7596}{%
% \subsubsection{2
% 用六神丸来预防宝宝生热痱和热疖}%ux7528ux516dux795eux4e38ux6765ux9884ux9632ux5b9dux5b9dux751fux70edux75f1ux548cux70edux7596}}

% 夏天高温天气,许多妈妈担心宝宝生热痱或热疖,经常给宝宝服用六神丸,认为它能清热败火,是预防宝宝生热痱或热疖的特效药。事实上,六神丸是一种用于治疗咽喉肿痛、扁桃体炎的中成药,其中重要成分为蟾蜍,具有一定的\textbf{毒}性作用,应用不当易致心律失常,配药中的雄黄含有硫化砷成分,应用过多会损伤肝肾等生命器官,不能起到预防作用。宝宝处于发育阶段,心、肝、肾功能尚未发育完全,如果长期大剂量服用六神丸,很容易造成这些器官的功能损害。

% 明智的做法是:

% \begin{enumerate}
% \def\labelenumi{\arabic{enumi}.}
% \item
%   保持居室通风、凉爽,让室内温度维持在25\textasciitilde28°C之间。
% \item
%   天气热的时候避免宝宝在阳光下玩耍,到户外嬉戏应该在阴凉地方玩。
% \item
%   每天给宝宝洗1\textasciitilde2次澡,采用温水并用刺激性小的皮肤清洗剂,使皮肤得到彻底清洁,洗后擦干水,轻轻扑上一层薄薄的婴儿爽身粉。【有些担心粉,吸入呼吸道呀】
% \item
%   给宝宝勤换衣裳,衣服应该质薄、柔软、吸汗、宽松。
% \item
%   经常查看宝宝的衣服、被单及枕巾是否已潮湿。
% \item
%   勤给经常躺着的小宝宝翻身,及时更换枕巾,头发尽量剪得薄而短些。
% \item
%   经常给宝宝喝自制的清凉解暑饮料,如绿豆汤、西瓜汁、菊花茶等。
% \end{enumerate}

% %ux5341ux516dux600eux6837ux6559ux5b69ux5b50ux4e0aux5395ux6240}{%
% \subsection{十六、怎样教孩子上厕所}%ux5341ux516dux600eux6837ux6559ux5b69ux5b50ux4e0aux5395ux6240}}

% \hspace{0pt}\includegraphics[width=2.51748in,height=2.67133in]{media/rId1613.png}\hspace{0pt}

% 在小孩具有以下条件时就可以训练孩子上厕所了:能够自由地在房间走动,有能力轻松地在马桶上坐上和下来;能够很容易地自己穿上和脱下内裤,知道许多诸如``干''、``湿''、``尿''、``屎''等词的意思,能够在不被责骂训斥的情况下按照父母的简单指示办事,能够每天排几次小便,而不是点点滴滴到处小便。

% 平时穿衣服、脱衣服,尤其是提裤子、脱裤子时,应尽量鼓励小孩自己动手,当你上街买东西时,可以带小孩去公共厕所,目的是让孩子知道使用公共厕所是很方便的。父母应该改掉一定回到家里才上厕所的习惯,因为小孩子没有那么久的控制能力。

% 在正式开始训练前,应做一些准备工作。父母要确保有一个使用方便的便盆,因为孩子在大小便时不易保持平衡。应为孩子准备很多内裤,大约10条,这样就可以应付许多意外弄脏内裤的情况。同时,洗这么多内裤比一次只洗两三条内裤经济得多。孩子的内裤不应太大或太小,使孩子能很容易地将内裤脱过臀部及大腿根部但又不至于掉下来。

% \textbf{一开始}应该教男孩子\textbf{坐下来小便}而不应站着撒尿。否则,一听到响声或受到惊吓,他们常常会把浴室弄得到处都是小便。

% 在如厕训练结束后6个月左右的时间里,孩子偶有一次反复,这是预料中的事,与如厕训练的方法无关。许多孩子在6或7岁之前仍有许多麻烦事发生。3岁以下的儿童夜尿多或尿床(遗尿),应该是意料中的事,也与如厕训练无关。孩子并非故意要弄湿他们的床,因此不应该因为尿床而责骂他们。

% 大多数孩子在大便后不能擦干净他们的肛门,尤其是在大便稀软的情况下,对很多孩子来说,这种情况要到4岁才会得到改善。如果你的孩子排便困难或便秘,甚至便后喊大腿酸痛,应该考虑每天饮食中给予更多的植物纤维或粗粮。在如厕训练中,没有比小孩子便秘引起的麻烦更糟的了。如果孩子经常便秘或排便困难,应该在进行如厕训练前带小孩子找医生看一看。

% %ux5341ux4e03ux5b9dux5b9dux51e0ux5c81ux4e0aux5e7cux513fux56ed}{%
% \subsection{十七、宝宝几岁上幼儿园}%ux5341ux4e03ux5b9dux5b9dux51e0ux5c81ux4e0aux5e7cux513fux56ed}}

% %ux5b9dux5b9dux6b63ux5e38ux4e0aux5e7cux513fux56edux7684ux5e74ux9f84ux5e94ux8be5ux5728ux4e24ux5c81ux534a}{%
% \subsubsection{1
% 宝宝正常上幼儿园的年龄应该在两岁半。}%ux5b9dux5b9dux6b63ux5e38ux4e0aux5e7cux513fux56edux7684ux5e74ux9f84ux5e94ux8be5ux5728ux4e24ux5c81ux534a}}

% 宝宝由于长期在家,环境比较封闭,平时和小朋友的接触比较少,可能有的宝宝到了幼儿园不适应那里的环境一到那里就哭闹,这时家长暂时不要强制让宝宝去幼儿园。爸爸、妈妈需要做的就是每夫安排宝宝和周边的小朋友一起玩,教会宝宝怎样和别人相处,让他和小朋友之间建立友谊,喜欢群体生活的氛围。作为家长不要太着急,要有一个过程。如果方便,要以旁观者的身份,带宝宝去幼儿园,让他逐渐对幼儿园感兴趣。

% \hspace{0pt}\includegraphics[width=4.0979in,height=3.9021in]{media/rId1617.png}\hspace{0pt}

% 宝宝该不该上幼儿园,是有学问的,还应考虑到子女心智的发展。

% 年龄并不是决定儿童入学的唯一条件。在你准备送宝宝上幼儿园之前,不妨先给他做一个小测验:

% \begin{enumerate}
% \def\labelenumi{\arabic{enumi}.}
% \item
%   会不会自己扣衣服扣子?
% \item
%   懂得如伯伯、叔叔等称谓吗?
% \item
%   能不能分辨左右?
% \item
%   可以不用练习而念出四个连续数字?
% \item
%   能画一个正方形?
% \item
%   能说出毛衣、鞋子与帽子相同的地方在哪里?
% \end{enumerate}

% 如果宝宝能够轻而易举地说出正确答案,那么便有资格快快乐乐地上学了。

% %ux57f9ux517bux5b9dux5b9dux613fux610fux4e0aux5e7cux513fux56ed}{%
% \subsubsection{2.
% 培养宝宝愿意上幼儿园}%ux57f9ux517bux5b9dux5b9dux613fux610fux4e0aux5e7cux513fux56ed}}

% 对宝宝而言,幼儿园是一个充满陌生的环境,心理上的杀戮战场。因为宝宝一生中最大的``分离焦虑''是在幼儿园产生的。

% 谈到上幼儿园,每个宝宝都余悸犹存,只因为那是一个完全陌生、看不到亲人的地方。在妈妈离去时,疯狂地喊叫哭泣,无助地趴在地上,此种恐怖的记忆对宝宝而言,有种被撕裂与割离的情愫。所以如何使宝宝能开开心心上幼儿园,对爸爸、妈妈来讲这是一道很重要而且必须要克服的难关,此时期的亲子教育对宝宝的心灵和心理发展有很大的影响,代表人生另一种转变与蜕变。

% %ux5efaux7acbux826fux597dux7684ux4ebaux9645ux5173ux7cfbux4e0eux793eux4f1aux5173ux7cfb}{%
% \subsubsection{3.
% 建立良好的人际关系与社会关系}%ux5efaux7acbux826fux597dux7684ux4ebaux9645ux5173ux7cfbux4e0eux793eux4f1aux5173ux7cfb}}

% 其实,协助宝宝愿意上幼儿园并非难事,只要稍微动一下脑筋,便可应付自如。

% 以下的策略与方法可提供爸爸、妈妈参考:

% \begin{enumerate}
% \def\labelenumi{\arabic{enumi}.}
% \item
%   为宝宝找友伴,同龄的或大一、两岁的宝宝。每天抽空让宝宝和左邻右舍年龄相近的宝宝多相处,以建立良好的人际关系与社会关系,将来上幼儿园时,可以和自己的同伴一起上学,如此才不会感到孤独与无伴。
% \item
%   养成宝宝的正常作息习惯。早睡早起,二餐定时定量,中午要睡午觉。
% \item
%   常带宝宝到户外运动、游戏,培养良好的生活习惯。
% \item
%   完成宝宝大小便的训练。让宝宝学会自己照顾自己,增加其自我自信心;
% \item
%   培养宝宝看书的兴趣。提供幼儿一些简单的图卡、画片以培养其阅读的习惯。在未上幼儿园时,先养成其规律生活的习惯与自动自发的生活,将来才能适应团体生活,不至于因为生活的改变而导致适应不良。
% \end{enumerate}

% %ux6263ux51cfux5c11ux632bux6298ux589eux5f3aux81eaux4fe1ux5fc3}{%
% \subsubsection{4
% 扣减少挫折、增强自信心}%ux6263ux51cfux5c11ux632bux6298ux589eux5f3aux81eaux4fe1ux5fc3}}

% 良好习惯的养成,必须是点点滴滴累积而成。所以爸爸、妈妈必须要花时间和心血,培养宝宝良好的生活习惯,鼓励其多与其他小孩子认识,建立友伴与同侪关系。养成宝宝正常作息和训练宝宝大小便,最终目的在于使宝宝适应团体生活,如果宝宝没有养成这些习惯,很容易在团体生活中受到挫折,降低其自尊心;其次还会增加幼儿园老师的负担,容易遭到同学们的讪笑,对宝宝幼小的心灵会造成很大的伤害,失去自信心而退缩畏怯,不敢轻易尝试与冒险,缺乏进取的精神。

% 所以在宝宝上幼儿园之前,一定要让宝宝学会自己上厕所。

% %ux7ed9ux7236ux6bcdux7684ux5efaux8bae}{%
% \subsubsection{5.
% 给父母的建议}%ux7ed9ux7236ux6bcdux7684ux5efaux8bae}}

% 爸爸、妈妈可以从以下几个方面帮助幼儿尽快地适应幼儿园的新生活。

% \begin{enumerate}
% \def\labelenumi{\arabic{enumi}.}
% \item
%   创造条件帮助幼儿尽快适应新环境。宝宝入园前,爸爸、妈妈可以带宝宝去幼儿园玩,使他逐渐熟悉这个陌生的环境。爸爸、妈妈也可以有意识地在家中和宝宝玩``幼儿园''游戏。如果爸爸、妈妈本人不熟悉幼儿园的生活,可以请邻居中已上幼儿园的小朋友来家里玩,使宝宝在游戏中熟悉幼儿园的生活,了解幼儿园的一些常规。另外,爸爸、妈妈事先可以去幼儿园了解幼儿园的作息制度来安排宝宝的生活,使宝宝的生物钟能和幼儿园的作息时间吻合,以便宝宝更容易地适应幼儿园的生活。
% \item
%   利用故事和儿歌,使宝宝向往幼儿园的生活。爸爸、妈妈可以有意识地给宝宝讲讲有关幼儿园的故事,让宝宝知道幼儿园是孩子们的乐园,是他们学习本领的地方,使他们对幼儿园的生活充满美好的憧憬。
% \item
%   鼓励宝宝出去交往。爸爸、妈妈可以采用放出去、请进来的方法,鼓励宝宝与陌生人交往,尤其是和同龄的小伙伴玩,以培养他们的交往能力和乐群性。
% \item
%   培养宝宝基本的生活自理能力。如自己吃饭、自己上厕所等,以便更好地适应集体生活。
% \item
%   主动和老师接触,使宝宝在入园前能认识老师,并帮助宝宝对老师产生好感和信任感。
% \end{enumerate}

% 进入幼儿园是宝宝生活的一个重大转折,只要引导有方,相信宝宝是会很快适应幼儿园的新生活的。

% %ux7b2cux4e94ux7bc7-ux5a74ux5e7cux513fux7684ux65e9ux671fux6570ux80b2}{%
% \section{5第五篇
% 婴幼儿的早期数育}%ux7b2cux4e94ux7bc7-ux5a74ux5e7cux513fux7684ux65e9ux671fux6570ux80b2}}

% 第一节新生儿的早期培育

% 第二节1〜2个月婴儿的早期培育

% 第三节2〜3个月婴儿的早期培育

% 第四节3〜4个月婴儿的早期培育

% 第五节4〜5个月婴儿的早期培育

% 第六节5〜6个月婴儿的早期培育

% 第七节6〜7个月婴儿的早期培育

% 第八节7〜8个月婴儿的早期培育

% 第九节8〜9个月婴儿的早期培育

% 第十节9 \textasciitilde{} 10个月婴儿的早期培育

% 第十一节10〜11个月婴儿的早期培育

% 第十二节11 \textasciitilde{} 12个月婴儿的早期培育

% %ux7b2cux4e00ux8282ux65b0ux751fux513fux7684ux65e9ux671fux57f9ux80b2}{%
% \subsection{01第一节新生儿的早期培育}%ux7b2cux4e00ux8282ux65b0ux751fux513fux7684ux65e9ux671fux57f9ux80b2}}

% %ux4e00ux65b0ux751fux513fux5148ux5929ux6027ux53cdux5c04ux79cdux7c7bux4e0eux8badux7ec3}{%
% \subsection{一、新生儿先天性反射种类与训练}%ux4e00ux65b0ux751fux513fux5148ux5929ux6027ux53cdux5c04ux79cdux7c7bux4e0eux8badux7ec3}}

% %ux53cdux5c04ux79cdux7c7b}{%
% \subsubsection{1. 反射种类}%ux53cdux5c04ux79cdux7c7b}}

% 新生儿有一些生下来就会做的动作,称为无条件反射。从这些反射可以检査出新生儿的神经系统发育是否正常。这里介绍几种家长容易检査的反射:

% \begin{enumerate}
% \def\labelenumi{\arabic{enumi}.}
% \item
%   用手指轻碰新生儿的面颊或嘴唇周围,看看他的嘴是否朝手指的方向寻找,这种反射称为``\textbf{觅食反射}''。在喂奶前做这个试验较好。
% \item
%   用手指轻轻划新生儿脚底外侧部位,新生儿的5个脚趾会分开,大拇指向上跷起。
% \item
%   将棍棒放在新生儿手心,看他是否抓得很紧,有时甚至能将新生儿提起。
% \item
%   扶住新生儿腋下,让其直立,轻轻用手按他一只脚的脚背,他就会成功地先后抬起左右脚,像走路的样子。【试试呀】
% \item
%   将刺激较重的某种气味放在鼻子附近,新生儿会将头扭向另一边,表现他的躲避能力。
% \item
%   俯卧时他会尽力把头抬一下,然后把脸扭向一边。
% \item
%   遇见强光会眨眼睛,以此来保护眼睛少受刺激。
% \item
%   扶新生儿坐直时,他的脖子摇晃不停,但不会受伤。一旦坐稳了,他的大脑袋竖直时,若头向前倒,他会纠正变成向后倒。这种倒来倒去,是一种平衡反应。
% \item
%   轻拍他一条腿,腿就会缩回去,如果挣不脱,另一只脚会来帮忙。【试试这个帮忙,哈】
% \item
%   新生儿会伸懒腰,并出现双脚踩自行车样的运动。
% \item
%   对突然刺激引起的一种惊跳反射,如大声、强光、疼痛、突然叩击枕头或床铺、或轻叩腹部、突然抬高下肢等,均可引起上、下肢伸直外展,躯干及手指伸直,拇指及食指末节屈曲,然后上肢屈曲、内收到胸前呈拥抱状,并在两臂松弛时发出哭声。该反射如在新生儿时引不出,提示有异常。
% \item
%   让新生儿仰卧,将头转向一侧,则同侧的上、下肢伸直,而对侧上下肢屈曲,若此情况持续存在,或过早消失,提示可能有脑瘫。
% \item
%   在膝关节处按新生儿一侧下肢,使其伸直,这时若刺激该侧足底,则对侧下肢出现屈曲,然后伸直和内收,如这种反射未引出,可能有神经系统损伤。
% \end{enumerate}

% 新生儿的这些反射,并不是一种无目的的运动,孩子将来的能力就是从这些反射的经验储存中发展起来的。如果做了几种试验,发现新生儿反应不灵敏,要注意观察,及时向儿保医师或神经科医师咨询,以便早期发现神经系统功能是否有障碍。

% %ux6761ux4ef6ux53cdux5c04ux7684ux8badux7ec3}{%
% \subsubsection{2.条件反射的训练}%ux6761ux4ef6ux53cdux5c04ux7684ux8badux7ec3}}

% 新生儿的条件反射功能有主动、被动之分,主动的条件反射是通过耳、眼、鼻、口及皮肤等器官感觉而形成的。被动的生理条件反射功能是一种纯本能。例如,当您用手指去触碰孩子的口角、面颊时,他就会认为有吃的东西,会顺着被触摸的方向张开小嘴,做吸吮的动作。这是一种本能,寻找食物,用以维持生命。又如,孩子具有抓握反射功能。用一个孩子能握住的玩具去触及孩子的小手时,他就会把手握得更紧。如果他拿住了这个玩具,就会牢牢地抓住,当您用力拉玩具时,会连同孩子的身体一起拉动。这两种条件反射随着神经系统的正常发育,到了3个月大的时候将会消失。

% \begin{quote}
% 育儿小百科

% \textbf{新生儿个性的发展}

% 小儿出生后,父母马上就会发现,他们在个性上存在着差异。有的新生儿非常老实,非常安静,比较好带养。他们睡眠时间长,肚子不十分饿就不会醒,当肚子饿了就咕噜咕噜地吃奶,也不怎么哭。若是吃母乳,就会把两侧奶全部吃空,若是吃牛奶也能轻松地吃掉100多毫升。小儿吃完奶就要小便,给他换尿布时显得很高兴,然后又不知不觉地睡着了。在夜里一般再醒1-2次,每次换完尿布吃完奶又马上睡着了。这样的小儿每天大便一般1-2次。称为易抚养型。

% 但是,有的新生儿就不那么老实,带养起来比较费劲儿。他们对外界刺激很敏感,有一点儿声响马上会醒,醒来后如果尿布湿了就哭,表现出不高兴,即使换了尿布,如果肚子饿了仍然哭个不停。这种孩子如果是吃母奶,吃了6-7分钟\textbf{饥饿感一消失就不再吃了,}此时小孩肚子并未吃饱。如果再硬塞奶给他吃,他就会把吃进去的奶全部吐出来,待过10来分钟他又因饿而啼哭,再吃5-6分钟才能睡去。如果是喂牛奶的,奶嘴稍有不通畅就哭,甚至把奶嘴吐出来不吃奶了。有时很庆幸把奶喂完了,刚过20来分钟他又把奶给全吐出来,这种情况多见于男孩子。由于每次吃奶量和吐奶量均不同,饥饿的时间也就不同,所以喂奶时间上也就没有规律了。称为困难型。【偶尔小困难型吧?】

% 每个小孩的个性受遗传因素的影响,但也与母亲怀孕期间的环境和生活方式有关,如母亲怀孕时行动活泼与否、说话声音的大小、母亲身体状况、生活的外部环境等。
% \end{quote}

% %ux4e8cux65b0ux751fux513fux542cux89c9ux80fdux529bux8badux7ec3}{%
% \subsection{二、新生儿听觉能力训练}%ux4e8cux65b0ux751fux513fux542cux89c9ux80fdux529bux8badux7ec3}}

% %ux542cux97f3ux4e50}{%
% \subsubsection{1. 听音乐}%ux542cux97f3ux4e50}}

% \hspace{0pt}\includegraphics[width=2.8951in,height=2.15385in]{media/rId1660.png}\hspace{0pt}

% 方法:妈妈在给宝宝喂奶时,将录音机或音响的音量\textbf{调小},播放一段旋律优美、舒缓的乐曲。此活动在宝宝出生几天后即可进行。

% \textbf{目的}:音乐可以训练听觉、乐感和注意力,陶冶孩子的性情。

% \textbf{注意}:不要给婴儿听很多不同的曲子,\textbf{一段乐曲一天中可反复播放几次,每次十几分钟,过几周后再换另一段曲子}。

% %ux5bf9ux5b9dux5b9dux8bf4ux8bdd}{%
% \subsubsection{2. 对宝宝说话}%ux5bf9ux5b9dux5b9dux8bf4ux8bdd}}

% \textbf{方法}:宝宝清醒时,妈妈可以用缓慢、柔和的语调对他说话,比如``宝贝,我是妈妈,妈妈喜欢你''等等。也可以给宝宝朗读简短的儿歌,哼唱旋律优美的歌曲。此活动可在出生20天后开始进行。

% \textbf{目的}:给婴儿听觉刺激,有助于宝宝早日开口学话,并促进母子之间的情感交流。

% \textbf{注意}:对宝宝说话时要尽量使用普通话。

% %ux4e09ux65b0ux751fux513fux89c6ux89c9ux80fdux529bux8badux7ec3}{%
% \subsection{三、新生儿视觉能力训练}%ux4e09ux65b0ux751fux513fux89c6ux89c9ux80fdux529bux8badux7ec3}}

% %ux770bux5149ux4eae}{%
% \subsubsection{1. 看光亮}%ux770bux5149ux4eae}}

% 方法:用一块红布蒙住手电筒的上端,开亮手电。将手电置于距婴儿双眼约30厘米远的地方,沿水平和前后方向慢慢移动几次。此训练可在婴儿出生半个月后开始进行。

% 目的:吸引婴儿注视灯光,进行视觉刺激。

% 注意:最好隔天进行1次,每次1-2分钟,不能不蒙红布用手电筒直照婴儿眼睛。

% %ux770bux660eux6697}{%
% \subsubsection{2. 看明暗}%ux770bux660eux6697}}

% 方法:将一张与书大小差不多的白纸对折,将一边涂黑,另一边空白。婴儿清醒时,将这张纸举在他眼前30厘米处。可在出生后半个月开始进行。

% 目的:观察婴儿的眼球是否在黑白两个画面上溜来溜去。

% 注意:若婴儿的眼球没有反应,应去医院检查,以便及早发现病情,及早治疗。

% %ux7ad6ux62b1}{%
% \subsubsection{3.竖抱}%ux7ad6ux62b1}}

% \hspace{0pt}\includegraphics[width=2.02797in,height=2.92308in]{media/rId1668.png}\hspace{0pt}

% 年龄:新生儿。

% 方法:从出生起,妈妈每次抱起宝宝时都用竖抱法。用左手扶托宝宝的臀部,让宝宝的身体直立,用右手扶托宝宝的头部,以免头向后仰,使宝宝可以看见墙上的各种装饰物、挂图和玩具。

% 目的:让宝宝看到仰卧时看不到的东西,使宝宝得到丰富的视觉刺激,同时锻炼宝宝颈部肌肉,使颈肌能托起头部的重量。

% %ux56dbux65b0ux751fux513fux89e6ux89c9ux80fdux529bux8badux7ec3}{%
% \subsection{四、新生儿触觉能力训练}%ux56dbux65b0ux751fux513fux89e6ux89c9ux80fdux529bux8badux7ec3}}

% %ux6293ux624bux6307}{%
% \subsubsection{1. 抓手指}%ux6293ux624bux6307}}

% 方法:妈妈伸出大拇指或食指,放在宝宝的手心里,让宝宝抓握。等宝宝会抓以后,再把手指从小儿的手心移到掌的边缘,看小儿是否也能去抓。

% 目的:通过训练使婴儿从最初无意识抓握到有意识地抓握。

% 注意:妈妈的指甲应该剪掉,以免刮伤婴儿。

% %ux89e6ux8138ux988a}{%
% \subsubsection{2.触脸颊}%ux89e6ux8138ux988a}}

% 方法:妈妈给孩子喂奶时,用手指轻轻触宝宝的脸颊,当触宝宝的右脸颊时,宝宝的头会往右侧转过来,当触宝宝的左脸颊时,宝宝的头会往左侧转过来。

% 目的:训练触觉能力。

% 注意:妈妈触宝宝脸颊时不要将手触到宝宝的眼、鼻处。

% %ux4e94ux65b0ux751fux513fux52a8ux4f5cux80fdux529bux8badux7ec3}{%
% \subsection{五、新生儿动作能力训练}%ux4e94ux65b0ux751fux513fux52a8ux4f5cux80fdux529bux8badux7ec3}}

% %ux770bux73a9ux5177}{%
% \subsubsection{1.看玩具}%ux770bux73a9ux5177}}

% 方法:在婴儿床上安装婴儿床挂玩具,上面有很多色彩鲜艳的小玩偶和小球,妈妈可以轻轻晃动,让宝宝看。同时还有各种不同的音乐铃声,可以放给宝宝听。

% 目的:训练宝宝追随小范围内的物体。

% 注意:妈妈晃动玩具发出音乐铃声\textbf{时间不宜长},以免宝宝疲倦、烦躁。

% %ux8f6cux5934}{%
% \subsubsection{2.转头}%ux8f6cux5934}}

% 方法:妈妈手持色彩鲜艳的玩具,最好是可摇响的,在距离孩子眼睛30厘米远的地方,慢慢地移到左边,再慢慢地移到右边。让小儿的头随着玩具做180度的转动。

% 目的:集动作训练、视觉训练和听觉训练于一体的综合训练。

% 注意:妈妈在移动玩具时,应将玩具摇响。孩子的头能朝左朝右各转动90度游戏即可停止。

% %ux4fefux5367ux62acux5934}{%
% \subsection{3、俯卧抬头}%ux4fefux5367ux62acux5934}}

% \hspace{0pt}\includegraphics[width=2.8951in,height=2.15385in]{media/rId1679.png}\hspace{0pt}

% 方法:出生7-10天的宝宝能自己向左右转头,妈妈可将宝宝俯卧在床上,用右手扶起宝宝的额部,左手摇响铃铛,宝宝会抬起眼睛去看。每天做这种练习2-3次,宝宝的头渐渐能抬得高些。到满月时,宝宝的下巴可以抬起离床3\textasciitilde4厘米。

% 目的:使宝宝颈部肌肉得到锻炼和发育,渐渐能支持头部的重量。

% %ux516dux60c5ux5546ux57f9ux517bux7ec3ux4e60}{%
% \subsection{六、情商培养练习}%ux516dux60c5ux5546ux57f9ux517bux7ec3ux4e60}}

% %ux9017ux7b11}{%
% \subsubsection{1. 逗笑}%ux9017ux7b11}}

% \begin{quote}
% 专家指导

% 宝宝越早出现逗笑就越聪明。因为大脑能早日形成条件反射,并能随着进入的信息建立更多的条件反射。以42天为临界,如果宝宝到56天还不会逗笑,就有可能是先天愚型。
% \end{quote}

% 方法:从出生第一天,爸爸妈妈就开始同宝宝逗笑,用手摸摸宝宝的小脸、胸脯、腋窝和脚心,一边逗弄一边微笑,同他说话。让宝宝感受家里的快乐气氛。

% 在2\textasciitilde3周,在妈妈逗弄时,宝宝的嘴角往上,面部松弛,出现了笑容。宝宝经常笑会使面部的表情肌肉发达而丰满,更加可爱。妈妈会发现宝宝会对某一个玩具发笑,会对爸爸做鬼脸发笑,每一种原因都能引起笑的条件反射。

% 目的:建立第一种条件反射,表示宝宝有了学习能力。\textbf{经常笑的宝宝较少啼哭,妈妈好带}。让宝宝经常笑,逐渐成为习惯,宝宝就会成为爱笑的快乐宝宝,容易与人交往。

% %ux53e3ux5507ux6e38ux620f}{%
% \subsubsection{2. 口唇游戏}%ux53e3ux5507ux6e38ux620f}}

% 方法:刚出生的宝宝就会同妈妈做口唇游戏,如果妈妈把口张大,宝宝也会张口;妈妈伸舌,宝宝也会伸出舌头来,有时妈妈咋舌,宝宝也会模仿。

% 出生后的宝宝以嘴巴为最能干的部位,因为宝宝要靠吸吮才能长大,口唇的模仿能力是最快获得的能力。

% 专家指导:口腔的各种活动是将来发展语言所必需的。妈妈要多给宝宝鼓励,教他多种玩法,促进口腔和舌头的活动。

% 除了模仿妈妈外,宝宝还会用口来自娱。比如宝宝会吐出唾液在口唇边来回玩耍等。

% 目的:学会用口腔来模仿,为以后模仿学习尤其是语言学习打基础。

% %ux8e22ux6253ux54cdux7403}{%
% \subsubsection{3. 踢打响球}%ux8e22ux6253ux54cdux7403}}

% \hspace{0pt}\includegraphics[width=2.85315in,height=2.55944in]{media/rId1685.png}\hspace{0pt}

% 方法:宝宝出生7\textasciitilde10天,用一个滚动作响的大球放在宝宝的床尾。妈妈拿着宝宝的小脚,帮助他踢打响球,宝宝听到踢打发出的声音,就会自己踢打响球,感受快乐。

% 目的:促使宝宝活动下肢。由于动作使玩具发出声音能使宝宝感到快乐,引起重复踢打动作。【原来玩具车,玩的时间似乎不要太长,根据本书前面提到的】

% %ux81eaux5a31ux4e0eux56deux5e94}{%
% \subsubsection{4.自娱与回应}%ux81eaux5a31ux4e0eux56deux5e94}}

% 方法:满月前后,宝宝觉醒的时间增加,他会自己躺着发音自娱,经常``啊姑,啊姑''地喊叫着玩,有时还会出现笑容。如果外面有猫狗叫、汽车的声音等,他会听一会儿,然后再叫。此时最好有妈妈的声音,如果妈妈能给以回应,宝宝会再次发声,脸上出现快乐的表情,夹着笑声等待着妈妈出现。

% 目的:妈妈经常回应宝宝的自娱,促进宝宝发声,出现迎接妈妈的快乐表情,这有利于宝宝与人进一步交往。

% \begin{quote}
% 专家指导

% 鼓励宝宝尽早用声音和表情迎接妈妈,不用啼哭来要求妈妈照料,有利于宝宝快乐情绪的发展。
% \end{quote}

% %ux4e03ux65b0ux751fux513fux7684ux6309ux6469ux4f53ux64cd}{%
% \subsection{七、新生儿的按摩体操}%ux4e03ux65b0ux751fux513fux7684ux6309ux6469ux4f53ux64cd}}

% %ux7b2cux4e00ux5957}{%
% \subsubsection{第一套}%ux7b2cux4e00ux5957}}

% 孩子仰卧,双臂放于体侧。妈妈用双手指面从肩到手按摩孩子胳膊4\textasciitilde6次。

% %ux7b2cux4e8cux5957}{%
% \subsubsection{第二套}%ux7b2cux4e8cux5957}}

% 孩子、妈妈姿势同上。妈妈用双手掌面按顺时针方向按摩孩子腹部6-8次,然后再用双手掌面从孩子腹部中心向两肋腰间方向抚摸6-8次。

% %ux516bux5988ux5988ux6559ux513fux6b4c}{%
% \subsection{八、妈妈教儿歌}%ux516bux5988ux5988ux6559ux513fux6b4c}}

% %ux5c0fux5a03ux5a03}{%
% \subsubsection{小娃娃}%ux5c0fux5a03ux5a03}}

% 小娃娃,小娃娃,嘴巴甜,喊爸爸,喊妈妈,喊得奶奶笑掉牙。

% %ux70b9ux70b9ux7a9dux7a9d}{%
% \subsubsection{点点窝窝}%ux70b9ux70b9ux7a9dux7a9d}}

% 点点窝窝,宝宝笑一笑,两个小酒窝。

% %ux559dux725bux5976}{%
% \subsubsection{喝牛奶}%ux559dux725bux5976}}

% 小宝宝,喝牛奶,喝了又唱还不饱,抱着奶瓶添添添。

% %ux7b2cux4e8cux828212ux4e2aux6708ux5a74ux513fux7684ux65e9ux671fux57f9ux80b2}{%
% \subsection{02第二节1〜2个月婴儿的早期培育}%ux7b2cux4e8cux828212ux4e2aux6708ux5a74ux513fux7684ux65e9ux671fux57f9ux80b2}}

% %ux4e001ux4e2aux6708ux5a74ux513fux7684ux667aux80fdux53d1ux80b2ux6c34ux5e73}{%
% \subsection{一、1个月婴儿的智能发育水平}%ux4e001ux4e2aux6708ux5a74ux513fux7684ux667aux80fdux53d1ux80b2ux6c34ux5e73}}

% 大运动:拉腕坐起头部能竖直片刻。

% 精细动作:触碰手掌会紧握住。

% 适应能力:眼跟踪红球过中线,听声音有反应。

% 语言:发出细小喉音。

% 社交行为:眼跟踪走动的人。

% 培养要点:根据新生儿心理发育的特点,应当着手进行``抬头''、``眼跟踪红绒球''、``学听声音''和``握持反射''等方面能力的培养,开始锻炼其行为能力。

% %ux4e8cux8bedux8a00ux80fdux529bux8badux7ec3}{%
% \subsection{二、语言能力训练}%ux4e8cux8bedux8a00ux80fdux529bux8badux7ec3}}

% %ux65e0ux58f0ux7684ux8bedux8a00}{%
% \subsubsection{1. 无声的语言}%ux65e0ux58f0ux7684ux8bedux8a00}}

% 方法:在宝宝情绪好时,母子面对面,相距约20厘米,孩子会紧盯着你的脸和眼睛,当你们的目光碰在一起时,和孩子对视并进行无声的语言交流,即\textbf{做出多种面部表情},如张嘴、伸舌、龇牙、鼓腮、微笑等。

% 目的:逗宝宝发笑,培养无声语言能力。

% 注意:当宝宝第一次出现逗笑时,切记记录下日期,作为宝宝心理发育的重要资料。宝宝在快乐的情绪中,各感官(眼、耳、口、鼻、舌、身等)最灵敏,接受能力也最好、最灵敏。过7\textasciitilde10天宝宝会笑出声音,这也是应该记录的日期。

% %ux56deux58f0ux5f15ux5bfcux53d1ux97f3}{%
% \subsubsection{2.
% 回声引导发音}%ux56deux58f0ux5f15ux5bfcux53d1ux97f3}}

% 方法:在宝宝啼哭之后,父母发出与宝宝哭声相同的声音。这时宝宝会试着再发声,几次回声对答,宝宝喜欢上这种游戏的叫声,渐渐地宝宝学会了叫而不是哭。这时父母可以把口张大一点,用``啊''来代替哭声诱导宝宝对答,渐渐地宝宝发出第一个元音。

% 目的:利用父母的回声引导婴儿发音,训练婴儿的语言能力。

% 注意:如果宝宝无意中出现另一个元音,无论是``啊''或``噢''都应以肯定、赞扬的语气用回声给以巩固强化,并且应当记录。

% %ux4e09ux793eux4ea4ux80fdux529bux8badux7ec3}{%
% \subsection{三、社交能力训练}%ux4e09ux793eux4ea4ux80fdux529bux8badux7ec3}}

% %ux968fux58f0ux821eux52a8}{%
% \subsubsection{1. 随声舞动}%ux968fux58f0ux821eux52a8}}

% 方法:在床前悬挂色彩鲜艳或能发声的玩具,使它们在婴儿视线内摇晃,让婴儿注视,并随着玩具发声,手足舞动。

% 目的:训练婴儿的社交能力。

% 注意:玩具不要只挂在一侧,应挂在床的四周,以免影响婴儿视力。

% %ux719fux6089ux73afux5883}{%
% \subsubsection{2. 熟悉环境}%ux719fux6089ux73afux5883}}

% \hspace{0pt}\includegraphics[width=3.09091in,height=3.18881in]{media/rId1704.png}\hspace{0pt}

% 方法:每天可将孩子竖抱片刻,使孩子能看到房间内各种形态的物品,并向孩子介绍周围景物。

% 目的:让孩子熟悉周围环境,增强社交能力。

% 注意:室内的光线不能太强,并对孩子说话和微笑。

% %ux56dbux60c5ux611fux57f9ux517bux8badux7ec3}{%
% \subsection{四、情感培养训练}%ux56dbux60c5ux611fux57f9ux517bux8badux7ec3}}

% %ux6bcfux5929ux629aux6478ux5b9dux5b9d}{%
% \subsubsection{1.
% 每天抚摸宝宝}%ux6bcfux5929ux629aux6478ux5b9dux5b9d}}

% 方法:在婴儿觉醒或和婴儿说悄悄话时,配以轻轻的皮肤抚摸。抚摸部位可以是头发、四肢、腿、腹部、背部、足背、手背、手指等。每天至少5-6次,每次3-5分钟,即每天15分钟以上。

% 目的:发育触觉,促进生长,增进父母与婴儿的感情。

% 注意:每天洗澡后一定要抚摩。抚摩婴儿前要洗手、剪好指甲、摘下手表等金属物;可隔一层衣服或柔软的毛巾轻轻抚摩,以防损伤皮肤。

% %ux8ddfux5b9dux5b9dux8bf4ux6084ux6084ux8bdd}{%
% \subsubsection{2.
% 跟宝宝说悄悄话}%ux8ddfux5b9dux5b9dux8bf4ux6084ux6084ux8bdd}}

% 方法:婴儿睡醒时,用缓慢、柔和的语调对婴儿讲些``悄悄话'',如:``噢,宝贝醒了,睡觉梦见妈妈了吗?''
% ``宝贝,我是妈妈,妈妈好爱你''等等。每天2-3次,每次2-3分钟。

% 目的:给婴儿听觉刺激,有助于婴儿早日开口说话,并促进母子之间的情感交流。

% 注意:对婴儿说话,最好用普通话反复和婴儿说,有条件的,\textbf{可以同时用外语说同样的话}。这样可以让婴儿贮存标准、丰富的语音信息,可促进语言能力的发展。

% %ux4e94ux751fux6d3bux81eaux7406ux80fdux529bux8badux7ec3-1}{%
% \subsection{五、生活自理能力训练}%ux4e94ux751fux6d3bux81eaux7406ux80fdux529bux8badux7ec3-1}}

% %ux628aux5927ux5c0fux4fbf}{%
% \subsubsection{把大小便}%ux628aux5927ux5c0fux4fbf}}

% 方法:出生半个月起,开始定时定点培养宝宝大小便的习惯。在便盆上方用``呜''声表示大便或用``嘘''声表示小便。通过视便盆,听声音,加上姿势形成排泄的条件反射,在满月前后婴儿就懂得把大小便了。

% 目的:婴儿把便既培养了与大人的合作,又能训练膀胱容量扩大,锻炼膀胱括约肌的功能,还密切了母婴关系,是一种良好习惯和能力的训练。

% 注意:大人挺胸坐正,不可压迫婴儿胸背而妨碍呼吸,当婴儿打挺表示不愿意把便时,应马上放下,停止训练,以免使婴儿疲劳。不过,只要你有耐心,孩子很快会建立起条件反射的,而且很快就不尿床了。

% %ux516dux89c6ux89c9ux80fdux529bux8badux7ec3}{%
% \subsection{六、视觉能力训练}%ux516dux89c6ux89c9ux80fdux529bux8badux7ec3}}

% %ux770bux5f69ux8272ux56fe}{%
% \subsubsection{1. 看彩色图}%ux770bux5f69ux8272ux56fe}}

% 方法:把彩图挂在四周墙壁上,每天竖抱宝宝观看,一边看,一边说图中的物名,用\textbf{愉快的口吻}引起宝宝的注意。在50-60天时,宝宝会对某一幅图做出欢快的反应,每当看到这一幅图时,不但会笑出声音,而且会手舞足蹈地表示高兴,说明宝宝对图画有了选择性。

% 目的:宝宝对某一幅图产生快乐的记忆,形成了条件反射,每次看到它时都会出现同样的反应。这种反应很专一,只对一幅图发生兴趣,证明宝宝的视觉分辨力良好,视觉记忆力清楚,能与快乐反应发生条件联系。

% 注意:让宝宝看某一幅图时,引起快乐反应,是与视觉、声音、大人当时的表情等做出的联合反应。如果大人第一次介绍这幅图时没有快乐和特殊的声音和表情,就难以让宝宝出现这种联合反应。

% %ux770bux5ba4ux5185ux5916ux7684ux7269ux4f53}{%
% \subsubsection{2.
% 看室内外的物体}%ux770bux5ba4ux5185ux5916ux7684ux7269ux4f53}}

% 方法:婴儿室内应有色彩鲜明的气球、飘带、大小玩具和挂图。在宝宝觉醒时,妈妈竖着抱起宝宝四处观看,既可以练习宝宝颈部肌肉,也可以使宝宝得到色彩和形状的视觉刺激。妈妈抱着宝宝边看边说,拿起小手去摸,使宝宝得到多种感官刺激。室外气温20°C以上时,可以带宝宝到户外观看,户外的景物经常变化,而且有动感。马路上有奔驰的汽车和走路的行人,有风吹落叶,也有鸟儿飞过、猫狗在跑,动的东西对宝宝很有吸引力,能引起宝宝的追视和兴趣。

% 目的:给宝宝多种感官的刺激。宝宝看到多种色彩和形状,听到妈妈温柔的解说声,用手摸到不同的物体引起不同的感觉,又可同时使颈部肌肉得到锻炼。

% 注意:在美丽的色彩、温和的语调和温暖的怀抱中,给宝宝以美好的印象,促进宝宝右脑的开发。

% %ux4e03ux542cux89c9ux80fdux529bux8badux7ec3}{%
% \subsection{七、听觉能力训练}%ux4e03ux542cux89c9ux80fdux529bux8badux7ec3}}

% %ux4e01ux96f6ux96f6}{%
% \subsubsection{1. 丁零零}%ux4e01ux96f6ux96f6}}

% 方法:在婴儿头部两侧摇铃,节奏时快时慢,音量时大时小。边摇边说:``铃!铃!铃!铃儿响丁当!''先不要让婴儿看到摇铃,而要观察其对铃声有无反应(如听到铃声停止哭闹或动作减少等),再训练婴儿根据铃声用眼睛寻找声源,每天2-3次。

% 目的:检验听力,提高视觉能力。

% 注意:铃声不可过响,否则影响婴儿听力。

% %ux516bux611fux89c9ux80fdux529bux8badux7ec3}{%
% \subsection{八、感觉能力训练}%ux516bux611fux89c9ux80fdux529bux8badux7ec3}}

% %ux6478ux4e00ux6478---ux89e6ux89c9ux8bd5ux9a8c}{%
% \subsubsection{1. 摸一摸 -
% 触觉试验}%ux6478ux4e00ux6478---ux89e6ux89c9ux8bd5ux9a8c}}

% 方法:轻触婴儿手心或眼睑,观察婴儿的反应。

% 目的:刺激婴儿的触觉发展。

% 注意:让婴儿触及的物品不可太冷、太热、以免伤到婴儿。

% %ux751cux7684ux8fd8ux662fux9178ux7684---ux5473ux89c9ux4e0eux55c5ux89c9ux8bd5ux9a8c}{%
% \subsubsection{2. 甜的还是酸的 -
% 味觉与嗅觉试验}%ux751cux7684ux8fd8ux662fux9178ux7684---ux5473ux89c9ux4e0eux55c5ux89c9ux8bd5ux9a8c}}

% 方法:将甜、酸、苦等各种味道的水放入婴儿口中,观察婴儿做出的不同反应。让婴儿闻香、臭及刺鼻的味道,观察其反应。

% 目的:刺激婴儿的味觉和嗅觉功能。

% 注意:不要将有强刺激性的味道放入婴儿口中试验,也不要让婴儿闻有强烈刺激性气味的物体。

% %ux7b2cux4e09ux828223ux4e2aux6708ux5a74ux513fux7684ux65e9ux671fux57f9ux80b2}{%
% \subsection{03第三节2〜3个月婴儿的早期培育}%ux7b2cux4e09ux828223ux4e2aux6708ux5a74ux513fux7684ux65e9ux671fux57f9ux80b2}}

% %ux4e002ux4e2aux6708ux5a74ux513fux7684ux667aux80fdux53d1ux80b2ux6c34ux5e73}{%
% \subsection{一、2个月婴儿的智能发育水平}%ux4e002ux4e2aux6708ux5a74ux513fux7684ux667aux80fdux53d1ux80b2ux6c34ux5e73}}

% 大运动:拉腕坐起时头可竖直片刻,俯卧时头开始可以抬离床面。

% 精细动作:拨浪鼓能在手中留握片刻。

% 适应能力:能立刻注意大玩具。

% 语言:能发a、o、e等韵母音。【可以啊】

% 社交行为:逗引孩子时有反应。

% %ux4e8cux5a74ux513fux6027ux522bux7684ux5deeux5f02ux6027}{%
% \subsection{二、婴儿性别的差异性}%ux4e8cux5a74ux513fux6027ux522bux7684ux5deeux5f02ux6027}}

% 绝大多数儿童在婴儿期就明显地显示出男女之间的个性差异。如,一般男孩子好动、胆大、说话晚些,而女孩子则较文静、胆小、说话早些。造成这些差异的原因究竟是什么呢?

% 从生物学角度看,男女之间有一些先天的有助于决定个性的差异(如男孩有睾丸、女孩有卵巢等),由此决定一个人的生理性别是``男''还是``女'',但这仅仅是为儿童提供了可能发展的方向。而实际上男女之间在对食物、庇护、爱抚等方面都有同样的基本要求,他们在智力上、体格上、言行举止、情绪上等都可能达到相同的发展水平。在现实生活中常常会见到有的男孩其言行举止等多个方面很像女孩,而有的女孩的言行举止等却很像男孩,这种差异属于心理性别上的差异。之所以会出现男女性别差异,关键还在于儿童出生后所受到的养育方式。

% 婴儿呱呱坠地,父母的第一个问题就是``孩子正常吗?''同时他们又会迫不及待地问``是男孩还是女孩?''
% 一旦他们知道了孩子的性别,他们就知道了婴儿身份的基本情况,这对孩子的发展有重要影响。

% 女孩子出生后,往往让她穿上红衣服,玩的玩具很可能是洋娃娃之类的;而男孩子出生后往往穿蓝色的衣服,玩具往往是汽车、枪之类。母亲对女儿比对儿子讲话的机会要多些、声音也更温柔些。父母对自己认为是合乎性别的行为,如女孩子玩娃娃、顺从、喜与人交往,男孩子的不愿受约束、要求自主等,可能报之以微笑、赞许和鼓励,而对他们认为不合乎性别的行为则要处罚、阻拦或限制。这些都是造成男女性别差异的重要因素。

% %ux4e09ux8bedux8a00ux80fdux529bux8badux7ec3}{%
% \subsection{三、语言能力训练}%ux4e09ux8bedux8a00ux80fdux529bux8badux7ec3}}

% %ux6a21ux4effux9762ux90e8ux52a8ux4f5c}{%
% \subsubsection{1、模仿面部动作}%ux6a21ux4effux9762ux90e8ux52a8ux4f5c}}

% \hspace{0pt}\includegraphics[width=3.21678in,height=3.04895in]{media/rId1725.png}\hspace{0pt}

% 方法:在宝宝情绪很好、很稳定的时候搂抱他,并在他面前经常张口、吐舌或做多种表情,使宝宝逐渐会模仿面部动作或微笑。

% 目的:培养语言能力。

% 注意:不要做恐怖的表情,那样不利于婴儿的心理发育。

% %ux5f15ux9017ux53d1ux97f3ux53d1ux7b11}{%
% \subsubsection{2、引逗发音发笑}%ux5f15ux9017ux53d1ux97f3ux53d1ux7b11}}

% 方法:用亲切温柔的声音,面对着宝宝,使他能看得见口型,试着对他发单个韵母a(啊)、o(喔)、u(呜)、e(鹅)的音,逗着孩子笑一笑,玩一会儿,以刺激他发出声音。快乐情绪是发音的动力。

% 目的:培养语言能力。

% 注意:\textbf{口型一定要做对,以免误导婴儿}。

% %ux56dbux793eux4ea4ux80fdux529bux8badux7ec3}{%
% \subsection{四、社交能力训练}%ux56dbux793eux4ea4ux80fdux529bux8badux7ec3}}

% %ux7b11ux4e0eux6761ux4ef6ux53cdux5c04}{%
% \subsubsection{笑与条件反射}%ux7b11ux4e0eux6761ux4ef6ux53cdux5c04}}

% 方法:在宝宝面前走过时,要轻轻抚摩或亲吻孩子的鼻子或脸蛋,并笑着对他说``宝宝笑一个'',也可用语言或带响的玩具引逗孩子,或轻轻挠他的肚皮,引起他挥手蹬脚,甚至渐渐``呀呀''发声,或发出``咯咯''笑声。

% 目的:快乐的孩子招人爱,也能合群,是具有良好性格的开端。

% 注意:注意观察哪一种动作最易引起宝宝大笑,经常有意重复这种动作,使宝宝高兴而大声地笑。

% %ux4e94ux60c5ux611fux57f9ux517bux8badux7ec3}{%
% \subsection{五、情感培养训练}%ux4e94ux60c5ux611fux57f9ux517bux8badux7ec3}}

% %ux547cux5524ux5a74ux513f}{%
% \subsubsection{1. 呼唤婴儿}%ux547cux5524ux5a74ux513f}}

% \hspace{0pt}\includegraphics[width=3.06294in,height=2.6993in]{media/rId1733.png}\hspace{0pt}

% 方法:妈妈(或爸爸)经常俯身面对婴儿微笑,让其注视自己的脸。然后,妈妈将脸移向一侧,轻声呼唤婴儿的名字,训练婴儿的视线随妈妈的脸移动。

% 目的:增强母子(父子)间的情感联络。

% 注意:妈妈(或爸爸)呼唤婴儿的声音应柔美,切勿尖高。

% %ux770bux8138ux8c31}{%
% \subsubsection{2. 看脸谱}%ux770bux8138ux8c31}}

% 方法:给婴儿看各种脸谱,如猫、狗、兔、猴及人物脸谱。

% 婴儿喜欢看脸,看到这么多形形色色、色彩鲜艳的脸谱,婴儿会高兴得``咿咿呀呀''直叫,并伸出小手去摸。

% 目的:通过看脸谱培养婴儿丰富感情。

% 注意:不要给婴儿看鬼神之类不健康的脸谱。

% %ux516dux542cux89c9ux80fdux529bux8badux7ec3}{%
% \subsection{六、听觉能力训练}%ux516dux542cux89c9ux80fdux529bux8badux7ec3}}

% %ux542cux58f0ux97f3}{%
% \subsubsection{1. 听声音}%ux542cux58f0ux97f3}}

% 方法:将各种发声体如哗铃棒、八音盒、钟表、橡皮捏响玩具等,在婴儿视线内让婴儿听,并告诉他名称。待其注意后,再慢慢移开,让婴儿追声寻源,当婴儿辨出声源后,再变换不同方向。

% 目的:用多种发声体训练听觉辨别力和方位听觉。

% 注意:选择音高、响度均不同的发声体。

% %ux8f6cux5934-1}{%
% \subsubsection{2. 转头}%ux8f6cux5934-1}}

% 方法:妈妈手持色彩鲜艳的玩具(最好是可摇响的),在离孩子眼睛30厘米远的地方,慢慢地移到左边,再慢慢地移到右边。让婴儿的头随着玩具做180度的转动。【app,选中后,摇动可以播放同一个乐器的声音】

% 目的:集动作训练、视觉训练和听觉训练于一体的综合训练。

% 注意:妈妈在移动玩具时,将玩具发出声响。婴儿的头能转90度,游戏即可停止。

% %ux4e03ux89c6ux89c9ux80fdux529bux8badux7ec3}{%
% \subsection{七、视觉能力训练}%ux4e03ux89c6ux89c9ux80fdux529bux8badux7ec3}}

% %ux770bux6c14ux7403}{%
% \subsubsection{1. 看气球}%ux770bux6c14ux7403}}

% \hspace{0pt}\includegraphics[width=2.95105in,height=2.96503in]{media/rId1742.png}\hspace{0pt}

% 方法:在婴儿睡床上方约75厘米处悬挂一个体积较大、色彩鲜艳的玩具,如彩色气球。妈妈一边用手轻轻触动气球,一边缓慢而清晰地说,``宝宝看,大气球''或``气球在哪儿啊?''

% 目的:引导婴儿用眼睛去看悬挂的玩具,训练婴儿逐渐学会用眼睛追随着视力范围内移动的物体。

% 注意:悬挂的玩具\textbf{不要长时间固定}在一个地方,以免婴儿的眼睛发生对视或斜视。悬挂的物品也不要过重或有尖锐的边角,以防不慎坠落时伤着婴儿。悬挂的玩具或物品还应\textbf{定期更换}。【七个彩色...根据前面说的从红色开始】

% %ux770bux4e16ux754c}{%
% \subsubsection{2. 看世界}%ux770bux4e16ux754c}}

% 方法:挑选一个好天气,把婴儿抱到室外,让他观察眼前出现的人和物,如大树、汽车等等,并缓慢清晰地反复说给他听。这时的婴儿会手舞足蹈地东看西看,非常开心。

% 目的:发展视觉开阔眼界,对开发婴儿智力大有好处。

% 注意:外出时间可由3-5分钟逐渐延长到15-20分钟。

% %ux516bux611fux89c9ux80fdux529bux8badux7ec3-1}{%
% \subsection{八、感觉能力训练}%ux516bux611fux89c9ux80fdux529bux8badux7ec3-1}}

% %ux6293ux73a9ux5177}{%
% \subsubsection{1. 抓玩具}%ux6293ux73a9ux5177}}

% 方法:分别把不同质地的玩具放在婴儿的手中保留一会儿。如果婴儿还不会抓握,可轻轻地从\textbf{指根到指尖抚摸他的手背},这时他的握持反射就会中断,紧握的小手就会\textbf{自然张开}。此时可把玩具塞到他的两只手里,并握住婴儿抓握玩具的手,帮助他抓握。

% 目的:提高触觉能力,训练手的技能。

% 注意:给婴儿抓握的玩具\textbf{不可尖锐}或有突出的棱角,以免误伤婴儿。

% \hspace{0pt}\includegraphics[width=3.51049in,height=3.13287in]{media/rId1748.png}\hspace{0pt}

% %ux63e1ux624bux6307}{%
% \subsubsection{2. 握手指}%ux63e1ux624bux6307}}

% 方法:把食指放在婴儿的手心让他抓握,并轻轻触动他的手向他``问好'',引起他的兴趣。待婴儿会抓后,父母再把手指从宝宝的手心移到手掌边缘,看他能否抓握。反复动作,直到熟练。

% 目的:提高触觉和手的技能。

% 注意:反复动作,直到婴儿的熟练。

% %ux4e5dux52a8ux4f5cux80fdux529bux8badux7ec3}{%
% \subsection{九、动作能力训练}%ux4e5dux52a8ux4f5cux80fdux529bux8badux7ec3}}

% %ux62acux5934}{%
% \subsubsection{1 .抬头}%ux62acux5934}}

% 方法:竖抱抬头、俯腹抬头和俯卧抬头。经过训练,婴儿不但抬起脸部观看前面响着的哗铃棒,而且下巴也能短时离床,双肩也抬起来。

% 目的:开阔了视野,丰富了视觉信息,增强了颈部张力。

% 注意:时间不宜过长,婴儿会疲惫。

% %ux5b66ux4e60ux722c}{%
% \subsubsection{2.学习``爬''}%ux5b66ux4e60ux722c}}

% 方法:在俯卧练习抬头的同时,可用手抵住宝宝的足底,虽然此时他的头和四肢尚不能离开床面,但婴儿会用全身力量向前方窜行,这种类似爬行的动作是与生俱来的本能,与8个月时爬行不同。

% 目的:不是让宝宝马上会爬,而是通过练习,促进小儿大脑感觉统合的健康发展,同时,也是开发智力潜能、激发快乐情绪的重要方法。

% 注意:如果没有这种训练,有些婴儿到11\textasciitilde12个月时才能爬,或者就根本不会爬,就直立行走,这将会导致大脑综合失调症。

% %ux7b2cux56dbux828234ux4e2aux6708ux5a74ux513fux7684ux65e9ux671fux57f9ux80b2}{%
% \subsection{04第四节3〜4个月婴儿的早期培育}%ux7b2cux56dbux828234ux4e2aux6708ux5a74ux513fux7684ux65e9ux671fux57f9ux80b2}}

% %ux4e003ux4e2aux6708ux5a74ux513fux7684ux667aux80fdux53d1ux80b2ux6c34ux5e73}{%
% \subsection{一、3个月婴儿的智能发育水平}%ux4e003ux4e2aux6708ux5a74ux513fux7684ux667aux80fdux53d1ux80b2ux6c34ux5e73}}

% %ux5927ux8fd0ux52a8}{%
% \subsubsection{大运动:}%ux5927ux8fd0ux52a8}}

% 俯卧抬头45度,抱直头稳。

% %ux7cbeux7ec6ux52a8ux4f5c}{%
% \subsubsection{精细动作:}%ux7cbeux7ec6ux52a8ux4f5c}}

% 两手握一起,拨浪鼓留握30秒。【会了】

% %ux9002ux5e94ux80fdux529b}{%
% \subsubsection{适应能力:}%ux9002ux5e94ux80fdux529b}}

% 目光跟随红球转头180度。

% %ux8bedux8a00}{%
% \subsubsection{语言:}%ux8bedux8a00}}

% 笑出声。【会了】

% %ux793eux4ea4ux884cux4e3a}{%
% \subsubsection{社交行为:}%ux793eux4ea4ux884cux4e3a}}

% 灵敏模样,见人会笑。

% %ux4e8cux638cux63e1ux5a74ux5e7cux513fux60c5ux5546ux7279ux70b9}{%
% \subsection{二、掌握婴幼儿情商特点}%ux4e8cux638cux63e1ux5a74ux5e7cux513fux60c5ux5546ux7279ux70b9}}

% \hspace{0pt}\includegraphics[width=2.41958in,height=3.48252in]{media/rId1764.png}\hspace{0pt}

% 较高水平的情商,有助于孩子创造力的发挥,它是所有学习行为的根本。一项研究表明,要预测孩子在幼儿园、在学校表现的标准,不是看小孩子积累了多少知识,而是看其情感与社会性的发展。一般来讲,高情商的孩子有以下特点:

% \begin{enumerate}
% \def\labelenumi{\arabic{enumi}.}
% \item
%   自信心强。自信心是任何成功的必要条件,是情商的重要内容。\textbf{自信是不论什么时候有何目标,都相信通过自己的努力,有能力和决心去达到。}
% \item
%   好奇心强。对许多事物都感兴趣,想弄个明白。
% \item
%   自制力强。即善于控制和支配自己行动的能力,有时是善于迫使自己去完成应当完成的任务,有时是善于抑制自己不当行为的发生。
% \item
%   人际关系良好。指能与别人友好相处,在与其他孩子相处时积极的态度和体验(如关心、喜悦、爱护等)占主导地位,而消极的态度和体验(如厌恶、破坏等)少一些。
% \item
%   具有良好的情绪。情商高的孩子活泼开朗,对人热情、诚恳,经常保持愉快。许多研究与事实也表明,良好的情绪是影响人生成就的一大原因。
% \item
%   同情心强。指能与别人在情感上发生共鸣,这是培养爱人爱物的基础。
% \end{enumerate}

% %ux4e09ux5988ux5988ux6559ux6b4c}{%
% \subsection{三、妈妈教歌}%ux4e09ux5988ux5988ux6559ux6b4c}}

% \textbf{小嘴巴}

% 小嘴巴,小嘴巴,用处大,吃饭唱歌全靠他。

% \textbf{小耳朵与小鼻子}

% 小耳朵,小耳朵,灵灵灵,样样声音听得清。

% 小鼻子,小鼻子,用处大,闻气味全靠他。

% \textbf{小眼睛}

% 小眼睛,小眼睛,亮晶晶,样样东西看得清。

% %ux56dbux8bedux8a00ux80fdux529bux8badux7ec3}{%
% \subsection{四、语言能力训练}%ux56dbux8bedux8a00ux80fdux529bux8badux7ec3}}

% %ux627eux58f0ux6e90}{%
% \subsubsection{1. 找声源}%ux627eux58f0ux6e90}}

% \hspace{0pt}\includegraphics[width=3.13287in,height=2.7972in]{media/rId1775.png}\hspace{0pt}

% 方法:拿一个拨浪鼓,在孩子前方30厘米处摇动,当孩子注意到鼓响时,对孩子说:``宝宝,看拨浪鼓在这儿。''让宝宝的眼睛盯着鼓,张开手想抓鼓。休息一会儿,在宝宝的后方,让他看不到你的脸,拿这个拨浪鼓摇动,稍停一会儿再问:``拨浪鼓在哪里呢?''再分别将拨浪鼓慢慢移到孩子能看到的左、右方摇动。

% 目的:训练婴儿听声音辨别方向,从而培养语言能力。

% 注意:注意观察宝宝的眼、耳和手的动作,看宝宝对声源方向的反应。

% %ux548cux5a74ux513fux5bf9ux8bdd}{%
% \subsubsection{2. 和婴儿``对话''}%ux548cux5a74ux513fux5bf9ux8bdd}}

% 方法:3个月的婴儿会咯咯地发笑,高兴的时候还会自发地``咿呀''、``啊呀''地讲话,这时妈妈同样``咿呀''、``啊呀''地去应答他,和他``对话'',可使其情绪得以充分地激发。

% 目的:不仅是对婴儿最初的发音训练,而且也是母子情感交流的好方式。

% 注意:在婴儿以后的成长过程中,父母应一直坚持与婴儿``对话'',积极应答他发出的各种声音。有条件的话,父母可以用普通话和外语交替着与婴儿说话。

% %ux4e94ux793eux4ea4ux80fdux529bux8badux7ec3}{%
% \subsection{五、社交能力训练}%ux4e94ux793eux4ea4ux80fdux529bux8badux7ec3}}

% %ux4f1aux51faux58f0ux642dux8bdd}{%
% \subsubsection{1. 会出声搭话}%ux4f1aux51faux58f0ux642dux8bdd}}

% 方法:在宝宝情绪愉快时,父母可用愉快的口气和表情,或用玩具,让他发出``呢、啊''声,或``咯、咯''的笑声,一旦逗引宝宝主动发声,你就要富有感情地称赞他,亲热地抚摩他,以示鼓励,并与他你一言、我一语地``对话'',诱导孩子出声搭话。

% 目的:训练婴儿的社交能力。

% 注意:逗引婴儿发笑的时间不宜太长,婴儿会累。

% %ux548cux5c0fux670bux53cbux4e00ux8d77ux73a9}{%
% \subsubsection{2.
% 和小朋友一起玩}%ux548cux5c0fux670bux53cbux4e00ux8d77ux73a9}}

% 方法:请一些2\textasciitilde5岁的小朋友来家里玩。小朋友们看见小婴儿,会觉得非常惊奇和喜爱;婴儿看见这么多喜欢他的小哥哥、小姐姐,也会高兴得手舞足蹈,忙不迭地和他们``谈话''。

% 目的:经常请小朋友来和婴儿一起玩,可以培养婴儿合群、开朗活泼的好性格。

% 注意:时刻观察小朋友们和婴儿的行为,以免互相误伤。

% %ux516dux611fux77e5ux80fdux529bux8badux7ec3}{%
% \subsection{六、感知能力训练}%ux516dux611fux77e5ux80fdux529bux8badux7ec3}}

% %ux5206ux8fa8ux5f62ux72b6}{%
% \subsubsection{1. 分辨形状}%ux5206ux8fa8ux5f62ux72b6}}

% 方法:用不同颜色的电线(红、黄、蓝、绿)弯几个直径为20厘米大小的正方形、长方形和三角形,当孩子``哼、哈''讲话时大人举起来让他看清后说:``这是正方形,这是长方形,这是三角形。''还可让小手拿一拿、攥一攥,多次反复刺激,直至长大一些会说会认了再增加新内容。

% 目的:现代研究表明,婴儿在3个月龄时已有分辨形状的能力,为此,早日开掘强化这方面智能,逐渐通过可见形象物,熟悉抽象的数学概念,初步感知基本图形概念。

% 注意:接头处用胶布缠好,勿伤着孩子皮肤。在孩子拿时,由于小手还不那么听使唤,注意别让角尖扎碰脸部。

% %ux611fux77e5}{%
% \subsubsection{2. 感知}%ux611fux77e5}}

% 方法:继续让小儿多看、多听、多摸、多嗅、多尝。如玩具物品应当轻软、有声、有色,让他能摸的都摸一摸,能摇动的都摇一摇,能发声的都听一听,如钟表声、动物叫声、风声、流水声等,结合生活起居自然地让他听音乐,让他闻闻醋,尝尝酸。

% 目的:锻炼他完整的感知事物的能力。

% 注意:婴儿所看、摸、嗅、尝的东西要有利于婴儿健康成长。

% %ux4e03ux52a8ux4f5cux80fdux529bux8badux7ec3}{%
% \subsection{七、动作能力训练}%ux4e03ux52a8ux4f5cux80fdux529bux8badux7ec3}}

% %ux6293ux548cux8e6c}{%
% \subsubsection{1. 抓和蹬}%ux6293ux548cux8e6c}}

% 方法:在婴儿床的上方,悬挂一个彩色玩具,如哗铃棒、塑料小动物等,距离以婴儿伸出手可以触到为宜。妈妈轻轻晃动悬挂的玩具,逗引孩子伸出手去抓,手抓的动作熟练以后,可以试着把玩具移到孩子脚部,让他用脚蹬一蹬。

% 目的:训练婴儿的手眼协调能力,同时发展孩子的触觉,并锻炼身体。

% 注意:开始时,妈妈应给孩子一些帮助和引导,如抬起孩子的小手去拿玩具,或有意地把玩具塞到孩子手里,引起他抓、拿玩具的兴趣。经过一段时间后,孩子自己就能挥舞着小手去抓玩具了,这时妈妈应及时表扬他:``好极了,抓着了。''激励孩子抓玩具的积极性。

% %ux5a74ux5e7cux513fux5065ux8eabux64cd}{%
% \subsubsection{2.
% 婴幼儿健身操}%ux5a74ux5e7cux513fux5065ux8eabux64cd}}

% 3个月的孩子要保证有室外活动的时间,在天气好的情况下,可以在室外活动10\textasciitilde30分钟,家长要坚持给孩子做操,除了前面学过的动作外,可以增加两个动作。

% 伸展运动:婴儿仰卧,母亲双手握住婴儿手腕,把婴儿两臂放在体侧。拉婴儿两臂在胸前呈前平举,掌心相对;然后使婴儿两臂向两侧斜上举;再拉婴儿两臂在胸前呈平举,掌心相对;最后还原。以上动作重复2遍。

% 两腿上举运动:婴儿仰卧,母亲拇指在下,其他4指在上,握住婴儿小腿,使两腿伸直。把婴儿两腿上举,与腹部成直角,然后还原,连续做2遍。

% %ux7b2cux4e94ux828245ux4e2aux6708ux5a74ux513fux7684ux65e9ux671fux57f9ux80b2}{%
% \subsection{05第五节4〜5个月婴儿的早期培育}%ux7b2cux4e94ux828245ux4e2aux6708ux5a74ux513fux7684ux65e9ux671fux57f9ux80b2}}

% %ux4e004ux4e2aux6708ux5a74ux513fux7684ux667aux80fdux53d1ux80b2ux6c34ux5e73}{%
% \subsection{一、4个月婴儿的智能发育水平}%ux4e004ux4e2aux6708ux5a74ux513fux7684ux667aux80fdux53d1ux80b2ux6c34ux5e73}}

% 4个月是智能发育的关键年龄,行为模式出现了质的变化。

% 大运动:仰卧抬头90度,扶腋可站立片刻。

% 精细动作:摇动并注视拨浪鼓。

% 适应能力:偶然注意小丸,找到声源。

% 语言:咿呀学语,高声叫。

% 社交行为:认识亲人。

% %ux4e8cux8bedux8a00ux80fdux529bux8badux7ec3-1}{%
% \subsection{二、语言能力训练}%ux4e8cux8bedux8a00ux80fdux529bux8badux7ec3-1}}

% %ux6293ux63e1ux73a9ux5177}{%
% \subsubsection{1. 抓握玩具}%ux6293ux63e1ux73a9ux5177}}

% 方法:把婴儿抱在桌前,桌面上放几种不同玩法的玩具,每次放一种,让婴儿练习抓握玩具,并教他玩法。如婴儿抓住摇铃后,你就告诉他名称``摇铃'',再抓住婴儿的手把铃摇响,边摇边说``摇摇铃,摇摇铃''。慢慢让他学着自己玩。学会后,再教另一种玩具的玩法。

% 目的:发展触觉,训练语言理解能力,训练手的抓握能力和手眼协调能力。

% 注意:准备各种质地、色彩,便于抓握的玩具,如摇铃、乒乓球、核桃、金属小圆盒、不倒翁、小方块积木、小勺、吹塑或橡皮动物、绒环或线球等。

% %ux770bux753bux7247}{%
% \subsubsection{2. 看画片}%ux770bux753bux7247}}

% 方法:给婴儿看一些色彩鲜艳的卡通画片,边看边给婴儿介绍。

% 例如:妈妈抽出一张画有一朵红花的画片,然后握住婴儿的小手指,指点着画片模仿婴儿的语气一问一答:``这是什么呀?''
% ``红花。''\,``红花下面是什么?'' ``绿叶。'' ``红花有几个花瓣呀?''
% 握住婴儿的手指一瓣一瓣地点(一瓣,两瓣,三瓣。)知道啦,红花有三个小花瓣。''这时,小婴儿会高兴地咯咯笑起来,自己用小手指在画片上点来点去,嘴里咿呀呀的,模仿刚才妈妈教的动作。

% 目的:训练语言能力,同时增强婴儿对语言的理解能力。

% 注意:当婴儿模仿妈妈的动作指点画片时,妈妈一定要对婴儿的``说话''作出反应,表扬他,称赞他,和他一起说。

% %ux4e09ux5988ux5988ux6559ux513fux6b4c}{%
% \subsection{三、妈妈教儿歌}%ux4e09ux5988ux5988ux6559ux513fux6b4c}}

% 摇篮曲

% 宝宝疲倦了,我把摇篮摇。宝宝整天玩,累得要睡觉。

% 小汽车

% 小汽车,笛笛笛,长着一对大眼睛。开过来,开过去,来来去去跑得急。

% %ux56dbux89c6ux89c9ux80fdux529bux8badux7ec3}{%
% \subsection{四、视觉能力训练}%ux56dbux89c6ux89c9ux80fdux529bux8badux7ec3}}

% %ux9017ux9017ux98de}{%
% \subsubsection{1. 逗逗飞}%ux9017ux9017ux98de}}

% 方法:让宝宝背靠在妈妈怀里,妈妈双手分别抓住孩子的两只小手,教他把两个食指尖对拢又水平分开,嘴里一边说``逗逗一飞'',如此反复数次。还可以分别将其余四指对拢又分开玩此游戏。

% 目的:锻炼宝宝的小肌肉,同时可以训练孩子的手眼协调能力。

% 注意:动作要轻缓。

% %ux597dux9ad8ux597dux9ad8}{%
% \subsubsection{2. 好高好高}%ux597dux9ad8ux597dux9ad8}}

% \hspace{0pt}\includegraphics[width=2.81119in,height=3.34266in]{media/rId1797.png}\hspace{0pt}

% 方法:用你的双手托起宝宝。将他轻轻地举上举下,转圈圈,让他从这些新的角度来观察周围的世界。当你把宝宝举向空中的时候,可以唱下面这首儿歌:

% 我是一只小小鸟,我是一只小小鸟,飞得高高地,快乐地叫,自在又逍遥。从没有忧愁,从没有烦恼,有了妈妈的陪伴,世界多美好。

% 目的:为了不让孩子老是处于他的低视野,有时候不妨把宝宝的视线提高,让他换个不同的角度来看这个世界。换个高度、观点来看平常熟悉的环境,可以提高孩子的好奇心,促进心智的成长,同时还能让宝宝学会视觉搜寻。

% 注意:动作要轻柔,同时抱紧宝宝,让他有安全感,手一定要抓牢他。

% %ux4e94ux542cux89c9ux80fdux529bux8badux7ec3}{%
% \subsection{五、听觉能力训练}%ux4e94ux542cux89c9ux80fdux529bux8badux7ec3}}

% %ux94c3ux94dbux58f0ux4eceux4f55ux5904ux6765}{%
% \subsubsection{1.
% 铃铛声从何处来}%ux94c3ux94dbux58f0ux4eceux4f55ux5904ux6765}}

% 方法:把铃铛等能发出声音的玩具或物品缝到五彩绳或橡皮套上,让宝宝仰躺在铺着柔软毯子的婴儿床上,把缝好玩具的五彩绳或橡皮套套在宝宝的手腕及脚踝上,当宝宝``手舞足蹈''的时候,你们就来一起听音乐,把这些发出声响的东西缝到宝宝的小袜子或小衣服袖子上。

% 目的:培养宝宝的语言及倾听能力。

% 注意:要仔细地把小玩具\textbf{缝牢},不致被宝宝咬下误吞。注意玩具不要过于坚硬或有尖锐的边角。

% %ux542cux97f3ux627eux7269}{%
% \subsubsection{2. 听音找物}%ux542cux97f3ux627eux7269}}

% 方法:家长敲响玩具(铃、鼓),小儿注意倾听,然后走到房子的一角敲,跟小儿说,``这是什么声音?''\,``听听声音,在哪里?''这时注意小儿的视线,是否朝着有声音的地方注视,若未注视,重复敲,直到他注视为止。

% 目的:听音找物或找人,通过发展视听提高适应能力。

% 注意:在此基础上,多给孩子倾听周围的声音,如给孩子听能发出悦耳声音的玩具(如小铃铛、八音盒、带音响的玩具等),甚至听昆虫和鸟类的啼鸣声,各种交通工具的声音等,当周围发出音响时,观察孩子的反应。

% %ux516dux52a8ux4f5cux80fdux529bux8badux7ec3}{%
% \subsection{六、动作能力训练}%ux516dux52a8ux4f5cux80fdux529bux8badux7ec3}}

% %ux7528ux624bux6491ux8d77}{%
% \subsubsection{1. 用手撑起}%ux7528ux624bux6491ux8d77}}

% 方法:让宝宝趴在床上或铺有草席或地毯的地上,在宝宝头侧用不倒翁或有声音的玩具逗引。宝宝先用肘撑起,大人把玩具从地上拿起来,逗引宝宝抬起上身。宝宝会把胳臂伸直,胸脯完全离开床铺,上身与床铺成90度角。有时宝宝的一个胳臂用手撑,另一个胳臂用肘撑,身体不平衡歪向肘撑的一侧,从肘撑的一侧翻滚成仰卧。此时并不是有意地做180度翻身,是无意的因重心不稳而偶然翻过去的,这种过大的翻动如同跌倒一样会使宝宝感到不安。所以如果宝宝只用一只手去支撑身体时,大人可以帮助他将另一只手也撑起来,使身体重心平衡,才能巩固俯卧双手支撑的练习,使宝宝感到安稳和愉快。

% 目的:宝宝俯卧用手撑起时,头可以看得更高更远,使宝宝的视觉开阔。这种姿势不但可以练习颈肌,还可以练习上肢和腰背的肌群使之强健,为以后匍行和爬行做好准备。

% 注意:要让宝宝有安全感。

% %ux62c9ux5750}{%
% \subsubsection{2. 拉坐}%ux62c9ux5750}}

% 方法:小儿在仰卧位时,家长握住小儿的手,将其拉坐起来,注意让小儿自己用力,家长仅用很小的力,以后逐渐减力,或仅握住家长的手指拉坐起来,宝宝的头能伸直,不向前倾。每日训练数次。

% 目的:训练运动能力。

% 注意:拉起时动作不要太快以免拉伤婴儿腱带。

% \begin{quote}
% 育儿小百科

% \textbf{婴幼儿健美健身方法}

% \hspace{0pt}\includegraphics[width=2.11189in,height=2.29371in]{media/rId1806.png}\hspace{0pt}

% 抱、逗、按、捏是婴儿健身简便易行的有效方法,对婴儿身心健康有着良好的作用。
% \end{quote}

% \begin{enumerate}
% \def\labelenumi{\arabic{enumi}.}
% \item
%   \begin{quote}
%   抱:抱是母子感情信息的传递,是对婴儿最轻微得体的活动。当婴儿在哭闹不止的时候,恰恰是最需要抱,从而得到精神安慰的的时候。有的父母怕惯坏了孩子而不愿意抱,这对婴儿的身心健康和生长发育是非常不利的。为了培养婴儿的感情思维,特别是在哭闹的特殊语言的要求下,不要挫伤幼儿心灵,要多抱一抱婴儿。
%   \end{quote}
% \item
%   \begin{quote}
%   逗:逗可以活跃气氛,丰富感情,是婴儿一种最好的娱乐形式。逗可以使小儿高兴得手舞足蹈,使全身的活动量进一步加强。实验证明,\textbf{常被逗戏}的婴儿不仅比长期躺在床上很少有人过问的婴儿表现得活泼可爱,而且,对周围事物的反应也显得更加灵活敏锐,不难想像,这对婴儿今后的智商有着直接的影响。
%   \end{quote}
% \item
%   \begin{quote}
%   按:按是家长用手掌对婴儿做轻微按摩。先取俯卧位,从背至臀部下肢。再取仰卧位从胸至腹部下每行10-20次。按不仅能增加胸背腹肌的锻炼,减少脂肪细胞的沉积,促进全身血液循环,还可以增强心肺活动量和肠胃的消化功能。
%   \end{quote}
% \item
%   \begin{quote}
%   捏:捏是家长用手指对婴儿捏揉,捏较按稍加用力,可以使全身和四肢肌肉更加紧实。一般先从上肢至两下肢,再从两肩至胸腹,每行10-20次。在捏揉过程中,小儿胃泌素的分泌和小肠的吸收功能均有改变,特别是对脾胃虚弱,消化功能不良的婴儿效果更加显著。
%   \end{quote}
% \end{enumerate}

% \begin{quote}
% 除了抱以外,逗、按、捏均不宜在进食当中或食后不久进行,以免食物呛入气管,时间一般应选择进食后2小时进行。操作手法要轻柔,不要过度用力,以让婴儿感到舒适为宜,并且注意不要让婴儿受凉,以防感冒。在逗婴儿时,笑态表情自然大方,不要做过多的挤眉、斜眼、歪嘴等怪诞不堪的动作,以避免婴儿模仿形成不良的病态习惯,令将来不好纠正。
% \end{quote}

% %ux7b2cux516dux828256ux4e2aux6708ux5a74ux513fux7684ux65e9ux671fux57f9ux80b2}{%
% \subsection{06第六节5〜6个月婴儿的早期培育}%ux7b2cux516dux828256ux4e2aux6708ux5a74ux513fux7684ux65e9ux671fux57f9ux80b2}}

% %ux4e005ux4e2aux6708ux5a74ux513fux7684ux667aux80fdux53d1ux80b2ux6c34ux5e73}{%
% \subsection{一、5个月婴儿的智能发育水平}%ux4e005ux4e2aux6708ux5a74ux513fux7684ux667aux80fdux53d1ux80b2ux6c34ux5e73}}

% 大运动:轻拉腕部即坐起,独坐头身前倾。

% 精细动作:抓住近处玩具。

% 适应能力:拿住一块方木,注视另一块方木。

% 语言:对人及物发声。

% 社交行为:见食物兴奋。

% %ux4e8cux6709ux8ba1ux5212ux5730ux5ff5ux513fux6b4c}{%
% \subsection{二、有计划地念儿歌}%ux4e8cux6709ux8ba1ux5212ux5730ux5ff5ux513fux6b4c}}

% 5个月的婴儿特别喜欢节奏明快的儿歌,虽然他还不懂儿歌或诗词的意思,但他喜欢儿歌有韵律的声音和欢快的节奏,更喜欢你给他念儿歌时亲切而又丰富的表情、口形和动作。

% 适合这个月龄念的儿歌应短小、朗朗上口,并做一种固定的动作,如《甜嘴巴》:(用手指婴儿的小嘴巴)``小娃娃,甜嘴巴,喊完妈妈,喊爸爸,喊得奶奶笑掉牙。''

% 每天至少要给婴儿念1-2首儿歌,每首儿歌至少要念3-4次。应当根据婴儿的日常活动并配以固定的丰富的表情和动作,使婴儿做到耳、目、手、足、脑并用,更有效地学习和记忆。

% %ux4e09ux6709ux8ba1ux5212ux5730ux8ba9ux513fux542cux97f3ux4e50}{%
% \subsection{三、有计划地让儿听音乐}%ux4e09ux6709ux8ba1ux5212ux5730ux8ba9ux513fux542cux97f3ux4e50}}

% \hspace{0pt}\includegraphics[width=2.43357in,height=2.95105in]{media/rId1818.png}\hspace{0pt}

% 5个月的婴儿对音乐能表现出特殊的爱好并能配合音乐节奏摆动四肢,也就是说,他已具有初步的音乐记忆力,并对音乐有了初步的感受能力。所以,从这个月开始,你就要有目的、有步骤地让婴儿欣赏音乐。

% \begin{enumerate}
% \def\labelenumi{\arabic{enumi}.}
% \item
%   让婴儿反复听某一乐曲,增强婴儿的音乐记忆力。
% \item
%   给婴儿听模仿动物的叫声和大自然中某些声响的音乐。你可用画有单个物体的彩色图片或实物配合,引起他的兴趣和舒畅的情绪,做到声一物一情融为一体。
% \end{enumerate}

% %ux56dbux751fux6d3bux81eaux7406ux80fdux529bux7684ux8badux7ec3}{%
% \subsection{四、生活自理能力的训练}%ux56dbux751fux6d3bux81eaux7406ux80fdux529bux7684ux8badux7ec3}}

% 给小孩一块软的能攥住的饼干,笑着对他说:``宝宝吃饼干啦。''并帮他把饼干移到嘴边放入口中,让小孩将饼干阻嚼后咽下。

% %ux4e94ux5988ux5988ux6559ux513fux6b4c}{%
% \subsection{五、妈妈教儿歌}%ux4e94ux5988ux5988ux6559ux513fux6b4c}}

% 小青蛙

% 小青眼,阔嘴巴,大眼睛,穿花褂。蹦蹦跳,呱呱呱,吃害虫,护庄稼。

% 大苹果

% 大苹果,圆又圆,尝一尝,甜又甜。

% 黄香蕉,黄香蕉,甜又甜,闻一闻,香又香。

% 橘子橘子红又红,真像一个小灯笼。像灯笼,圆又圆,它的味道酸又甜。

% 小西瓜,小西瓜,圆溜溜,红瓤黑籽在里头,水又多,味又甜,夏天吃它凉哇哇。

% 葡萄藤,爬得高,爬到架上吹泡泡,吹了一串又一串,串串都是甜葡萄。

% %ux516dux8bedux8a00ux80fdux529bux8badux7ec3}{%
% \subsection{六、语言能力训练}%ux516dux8bedux8a00ux80fdux529bux8badux7ec3}}

% %ux6a21ux4effux53d1ux97f3}{%
% \subsubsection{1、模仿发音}%ux6a21ux4effux53d1ux97f3}}

% 方法:与宝宝面对面,用愉快的口气与表情发出``wu-wu''、``ma-ma''、``ba-ba''等重复音节,逗引宝宝注视你的口形,每发一个重复音节应停顿一下给孩子模仿的机会。接着你手拿个球,问他``球在哪儿''时,把球递到孩子手里,让他亲自摸一摸、玩一玩,告诉他:``这是球。''边说,边触摸、注视、指认,每日数次。

% 目的:训练语言能力。

% 注意:发音与口形要准确。

% %ux53ebux540dux56deux5934}{%
% \subsubsection{2、叫名回头}%ux53ebux540dux56deux5934}}

% \hspace{0pt}\includegraphics[width=2.8951in,height=2.6014in]{media/rId1827.png}\hspace{0pt}

% 方法:宝宝早就能听到声音回头去看,但是能否理解自己的名字,此时可以进一步观察。带宝宝去街心、公园或有其他孩子的地方,父母可先说其他小朋友的名字,看看宝宝有无反应,然后再说宝宝的名字,看他是否回头。当孩子听名回头向你笑笑时,要将他抱起来亲吻,并说``你真棒''\,``真聪明'',以示表扬。

% 目的:让婴儿知道自己的名字,增强语言能力。

% 注意:父母应在胎教时,即在妊娠第6个月时就为宝宝取名,每次呼唤都用同一个名字。经过孕期一个月呼名训练的婴儿会在出生3个月时知道自己的名字而回头。未经训练的婴儿可在5\textasciitilde7个月时知道自己的名字。切记\textbf{要用固定的名字称呼宝宝},如果大人一会儿说``宝宝'',一会儿说``文文'',一会儿又说``闹闹'',经常更改名字,会使孩子无所适从,就会延迟叫名回头的时间。

% %ux4e03ux793eux4ea4ux80fdux529bux8badux7ec3}{%
% \subsection{七、社交能力训练}%ux4e03ux793eux4ea4ux80fdux529bux8badux7ec3}}

% %ux8868ux60c5ux53cdux5e94}{%
% \subsubsection{1、表情反应}%ux8868ux60c5ux53cdux5e94}}

% 方法:玩照镜子的游戏,和妈妈同时照镜子,看镜子里母子的五官和表情逗引宝宝发出笑声,并让宝宝和你一起做惊讶、害怕、生气和高兴等游戏。

% 目的:训练婴儿分辨面部表情。

% 注意:时间不宜长,不宜让婴儿过于兴奋。

% \hspace{0pt}\includegraphics[width=2.57343in,height=2.85315in]{media/rId1832.png}\hspace{0pt}

% %ux4e3eux9ad8}{%
% \subsubsection{2、举高}%ux4e3eux9ad8}}

% 方法:孩子最喜欢让爸爸``举高'',然后再``放低''。家长一面举一面说,以后每当大人说``举高''时,宝宝会将身体向上做相应的准备。

% 目的:提高婴儿语言与动作协调能力。

% 注意:在举起和放下动作时,要将孩子扶稳,千万不要做抛起和接住的动作,以免失手让孩子受惊或受伤。

% %ux516bux89c6ux89c9ux80fdux529bux8badux7ec3}{%
% \subsection{八、视觉能力训练}%ux516bux89c6ux89c9ux80fdux529bux8badux7ec3}}

% %ux770bux8fdcux5904ux7684ux7269ux4f53}{%
% \subsubsection{1.
% 看远处的物体}%ux770bux8fdcux5904ux7684ux7269ux4f53}}

% 方法:母亲要更多地指着周围环境中的各种物品介绍给婴儿听,不管婴儿是否能听懂,都要多次重复,让婴儿反复感知。这时,除了指认室内的家具、玩具、食物、日用品和室外的花草树木、交通工具、建筑物等以外,可以开始指认远处的行人、车辆、天上的白云、风筝、初升的月亮和落日等等。

% 目的:此时婴儿的视力已明显增强,可以看到远处的物体,及时训练可扩大认识事物的范围。

% 注意:不可带婴儿去太吵闹的地方或在恶劣的天气情况下外出。

% %ux81eaux5df1ux73a9}{%
% \subsubsection{2. 自己玩}%ux81eaux5df1ux73a9}}

% \hspace{0pt}\includegraphics[width=2.1958in,height=3.1049in]{media/rId1839.png}\hspace{0pt}

% 方法:用被子把婴儿``围''起来,或者把婴儿放在带围栏的小床上。在婴儿面前放上会发声的橡皮玩具、可以抱的布娃娃或其他小动物玩具,让婴儿自己玩玩具。或母亲走过去,帮他把玩具弄出声响来,再把玩具放到不同的地方,逗引婴儿变换体位,抓握玩具。

% 目的:提高婴儿认识物体和寻找物体的能力,同时训练其手眼协调能力。

% 注意:玩具上绝不能有易掉落的活动金属物、小纽扣等。要让婴儿呈躺卧状,但也不要长时间保持这种姿势,避免脊柱弯曲。同时注意让婴儿拿东西时,学会把大拇指和其他四指分开。

% %ux4e5dux52a8ux4f5cux80fdux529bux8badux7ec3-1}{%
% \subsection{九、动作能力训练}%ux4e5dux52a8ux4f5cux80fdux529bux8badux7ec3-1}}

% %ux76f4ux7acb}{%
% \subsubsection{1. 直立}%ux76f4ux7acb}}

% 方法:两手扶着孩子腋下,让他站在你的大腿上,保持直立的姿势,并扶着小儿双腿跳动,每日反复练习几次,促进平衡感知觉的协调发展。

% 目的:训练直立能力,为走打下基础。

% 注意:直立时间不宜过长,防止累着婴儿。

% %ux9760ux5750}{%
% \subsubsection{2. 靠坐}%ux9760ux5750}}

% 方法:将小孩放在有扶手的沙发上或小椅上,让小孩靠坐着玩,或者家长给予一定的支撑,让小儿练习坐,支撑力量可逐渐减少。

% 目的:训练靠坐能力。

% 注意:每日可连续数次,每次10分钟。

% %ux5a74ux5e7cux513fux5065ux8eabux64cd-1}{%
% \subsubsection{3.
% 婴幼儿健身操}%ux5a74ux5e7cux513fux5065ux8eabux64cd-1}}

% 为了孩子的健康,要坚持给孩子锻炼身体。做操和室外晒太阳,这都是必不可少的锻炼方法。这个月除了坚持以前学过的几套动作外,可以增加两个动作。

% 两腿轮流屈伸运动:婴儿仰卧,两腿伸直,母亲用两手轻轻握住婴儿脚腕,推左腿屈至腹部,然后还原,再推右腿屈至腹部,然后放下还原,连续做2遍。

% 下肢放松运动:婴儿仰卧两腿伸直,母亲用两手轻轻握住婴儿脚腕,轻抬腿成45度,然后还原,连续做2遍。

% \begin{quote}
% 育儿小百科

% 婴幼儿体形美的训练方法

% 不要过早地让小儿学爬、坐、走,不要让婴儿头部过度地后倾,背部过度弯曲。因为新生儿的脊柱完全是直的,3-4月抬头时出现颈部脊柱的前凸,6-9月会坐时胸部的脊柱出现后凹,
% 12\textasciitilde18月开始学走路时腰部的脊柱出现前凸。如不注意很有可能造成畸形,如头后倾、背部过度弯曲,走路呈挺胸状等不雅的姿势。不要让小儿的摇篮长期处于一个固定不变的位置。由于光线、音响总是来自一个方向,小儿眼睛一直注视着这个方向,容易形成斜视。

% 小儿垫过多的尿布,将会使两侧大腿外旋,长大成人后走路就有可能呈鸭步态。

% 处于襁褓中的婴儿,必须把肢体摆放端正,捆扎的松紧度必须合适。否则,有可能造成终身畸形。

% 维生素D缺乏的小儿,在治愈之前,应避免过度的活动。不然的话,有可能形成或加重O型腿或X型腿。
% \end{quote}

% %ux7b2cux4e03ux828267ux4e2aux6708ux5a74ux513fux7684ux65e9ux671fux57f9ux80b2}{%
% \subsection{07第七节6〜7个月婴儿的早期培育}%ux7b2cux4e03ux828267ux4e2aux6708ux5a74ux513fux7684ux65e9ux671fux57f9ux80b2}}

% %ux4e006ux4e2aux6708ux5a74ux513fux7684ux667aux80fdux53d1ux80b2ux6c34ux5e73}{%
% \subsection{一、6个月婴儿的智能发育水平}%ux4e006ux4e2aux6708ux5a74ux513fux7684ux667aux80fdux53d1ux80b2ux6c34ux5e73}}

% 大运动:仰卧位自行翻身到俯卧位。

% 精细动作:能用双手把纸撕破,会耙弄到桌上一块方木。

% 适应能力:两手同时拿住2块方木,玩具失落有找的动作,但不一定找到失落的玩具。

% 语言:叫名字,会转头寻找呼唤的人。

% 社交行为:会自己吃饼干,会和大人玩``藏猫猫''游戏。

% %ux4e8cux751fux6d3bux81eaux7406ux80fdux529bux7684ux8badux7ec3}{%
% \subsection{二、生活自理能力的训练}%ux4e8cux751fux6d3bux81eaux7406ux80fdux529bux7684ux8badux7ec3}}

% 在小儿吃奶时,训练他能自己用双手扶奶瓶吃奶,家长要赞扬和鼓励孩子。还可边喝边配上儿歌,如``牛奶甜,牛奶香,喝了牛奶长得棒''等。喝奶间可以给孩子一块软质饼干,放在手里,鼓励他自拿自吃。

% 吃饭时,婴儿有可能来夺勺子,这正是学用勺子吃饭的好机会,开始时他还分不清凸凹面,快到1岁时就会装满勺子自己吃。学用勺子的方法如下:

% \begin{enumerate}
% \def\labelenumi{\arabic{enumi}.}
% \item
%   从2把勺到1把勺:先让婴儿右手持勺学吃,妈妈用另一把勺喂饭。
% \item
%   可左右手用勺:开始婴儿持勺不分左右手,没有必要非要纠正他,两手同时并用可促进左右大脑发育。
% \item
%   要有耐心:婴儿开始用勺子不够熟练,会弄得手、脸、衣服到处都是饭,甚至摔碎碗或杯子。这时不要大声呵骂,更不能因此让他失去学习的机会。
% \end{enumerate}

% %ux4e09ux5988ux5988ux6559ux513fux6b4c-1}{%
% \subsection{三、妈妈教儿歌}%ux4e09ux5988ux5988ux6559ux513fux6b4c-1}}

% 吃青菜

% 宝宝乖,宝宝乖,宝宝喜欢吃青菜,绿菠菜,翠黄瓜,青萝卜,嫩白菜,多吃青菜身体好,多吃青菜长得快。

% 吃豆豆

% 吃豆豆,长肉肉,不吃豆豆精瘦瘦。

% 相思(唐)王维\\
% 红豆生南国,春来发几枝?愿君多采撷,此物最相思。

% %ux56dbux60c5ux611fux8badux7ec3ux6e38ux620f}{%
% \subsection{四、情感训练游戏}%ux56dbux60c5ux611fux8badux7ec3ux6e38ux620f}}

% %ux4f20ux9012ux79efux6728}{%
% \subsubsection{1,传递积木}%ux4f20ux9012ux79efux6728}}

% 方法:婴儿坐在床上,妈妈给他一块积木,等他拿住后,再向同一只手递另一块积木,看他是否将原来的一块积木传递到另一只手后,来拿这一块积木。如果他将手中的积木扔掉再拿新积木,就要教他先换手再拿新的。

% 目的:训练手与上肢肌肉动作,提高用过去积累的知识解决新问题的能力。

% 注意:积木要保持清洁,坚持\textbf{经常擦洗}。

% %ux5750ux98deux8239}{%
% \subsubsection{2.坐``飞船''}%ux5750ux98deux8239}}

% \hspace{0pt}\includegraphics[width=2.41958in,height=2.68531in]{media/rId1856.png}\hspace{0pt}

% 方法:在婴儿情绪愉快时,爸爸与婴儿面对面,扶婴儿腋下站立,然后把婴儿往上举过自己的头顶,反复几次。也可把婴儿从爸爸身体的左侧向右上方举,再从右侧往左上方举,反复几次,边举边说:``(婴儿的名字)坐飞船,飞船开喽!''

% 目的:增近父子感情,加强游戏能力。

% 注意:不能将婴儿抛过头顶再接住,这样会增加婴儿患脑震荡的几率。

% %ux4e94ux8bb0ux5fc6ux80fdux529bux8badux7ec3}{%
% \subsection{五、记忆能力训练}%ux4e94ux8bb0ux5fc6ux80fdux529bux8badux7ec3}}

% %ux5bfbux627eux73a9ux5177}{%
% \subsubsection{1. 寻找玩具}%ux5bfbux627eux73a9ux5177}}

% 方法:让孩子看着把玩具小狗放在桌上,用手绢盖上,大人问:``小狗狗呢?''
% 孩子可能懂得被手绢盖着,用手扯开。如不懂,大人可帮他把手靠近手绢,让他拉开见到小狗。要多次训练,逐渐学会,一问便扯开手绢。以后当孩子面用碗把小玩具扣上,再问,是否知道是在碗下面而揭开,再反复训练。

% 目的:动动小脑筋,锻炼记忆力和判断力。

% 注意:玩具要经常更换。在用碗时要用带把手的喝水碗(杯),孩子不断揭碗,还能促进手指小肌肉的锻炼,增强手指力量。

% %ux4eceux8eb2ux907fux5230ux63a5ux53d7ux751fux4eba}{%
% \subsubsection{2.
% 从躲避到接受生人}%ux4eceux8eb2ux907fux5230ux63a5ux53d7ux751fux4eba}}

% 方法:来客人时,母亲抱宝宝去迎接客人,暂时不让客人接近宝宝,让宝宝有机会观察客人的说话和举止。适应一会儿后,母亲再抱宝宝接近客人,这时只让客人同母亲对话,偶尔看宝宝笑笑,不接触宝宝,使宝宝放松。告别时只要求宝宝表示``再见'',客人并不接触宝宝。第二或第三次再见面时,客人可拿个小玩具递给宝宝,如果宝宝表示高兴,客人把手伸向宝宝,看宝宝是否愿意让客人抱一会儿。客人抱宝宝时,母亲一定不要离开,使宝宝感到可以随时回到母亲怀抱。有过这种经历,宝宝就会从躲避到接受生人。

% 目的:要让宝宝从躲避转变为接受生人,要容许宝宝自己去观察探索。母亲要理解宝宝保护自己的意识,同时要逐渐让他能接受生人。

% 注意:循循善诱,不可强迫。

% %ux516dux542cux89c9ux80fdux529bux8badux7ec3-1}{%
% \subsection{六、听觉能力训练}%ux516dux542cux89c9ux80fdux529bux8badux7ec3-1}}

% %ux62c9ux4e00ux62c9ux4e01ux4e1cux54cd}{%
% \subsubsection{1.
% 拉一拉,丁东响}%ux62c9ux4e00ux62c9ux4e01ux4e1cux54cd}}

% 方法:在宝宝的床栏杆上,在其手可碰到的地方,挂一些漂亮、会发声的玩具。妈妈可拉着他的手去触拉玩具,让他体会拉拉线,玩具会动,还会听到丁东丁东响的乐趣。如此反复多次,孩子就会自己去触拉玩具,成功后,孩子会咧嘴咯咯大笑,且乐此不疲。

% 目的:训练听觉能力。

% 注意:为训练婴儿听力,还可以把表贴在婴儿耳边,并说:``嘀嗒嘀嗒......''接着把表贴在自己耳边说:``嘀嗒嘀嗒在哪儿呢?''如果婴儿表示出要把表放在耳边听的要求就可以了。这时,可让婴儿多听会儿,再说:``给妈妈听嘀嗒。''如果孩子把表送到你耳边,就说明他懂了。

% %ux6d4bux542cux529b}{%
% \subsubsection{2.测听力}%ux6d4bux542cux529b}}

% 方法:让宝宝坐在你的腿上,和墙壁的距离不应少于120厘米,并请另一个大人站在宝宝一侧与其耳朵齐高,但宝宝看不见的位置。

% 6个月大的宝宝,测试者应站在离他45厘米以外的地方。利用下列的顺序在宝宝耳朵高度处发出声音:

% \begin{enumerate}
% \def\labelenumi{\arabic{enumi}.}
% \item
%   利用你的声音发出低频率及高频率的声音。
% \item
%   摇动会发出声响的玩具。
% \item
%   以汤匙敲打杯子。
% \item
%   搓揉卫生纸。
% \item
%   摇动摇铃。
% \end{enumerate}

% 如果宝宝对声音没有反应,等两秒钟后再试,试过3次之后,如果还没有反应则继续做下一项测试。

% 目的:训练婴儿对各种物品发出的声音的辨别能力。

% 注意:敲打玩具时不要过于用力,会吓到婴儿。

% %ux4e03ux611fux89c9ux80fdux529bux8badux7ec3}{%
% \subsection{七、感觉能力训练}%ux4e03ux611fux89c9ux80fdux529bux8badux7ec3}}

% %ux89c2ux96e8}{%
% \subsubsection{1. 观雨}%ux89c2ux96e8}}

% \hspace{0pt}\includegraphics[width=2.47552in,height=3.16084in]{media/rId1872.png}\hspace{0pt}

% 方法:下雨时,可抱(扶)宝宝在窗前或阳台上,引导宝宝观看下雨的情景,听下雨的声音同时反复说:``嘀嗒嘀嗒,下小雨了,沙沙沙沙,小雨沙沙。''若下大雨,则说:``吧嗒吧嗒,下大雨啦,哗哗哗哗,大雨哗哗。''然后再唱一曲《下雨》给孩子听,加深印象。

% 目的:让婴儿熟悉雨的声音、感觉。

% 注意:不要让婴儿淋到雨。

% %ux96eaux767dux7684ux4e16ux754c}{%
% \subsubsection{2. 雪白的世界}%ux96eaux767dux7684ux4e16ux754c}}

% 方法:冬天白雪飘飘的时候,父母可带宝宝外出欣赏美丽的雪景,让宝宝看一看漫天飞舞的雪花,摸一摸堆积在一起的雪团,让宝宝领略一下这银白色苍茫大地的气派。当欣赏雪景的时候,父母可用语言与宝宝交流:``下雪了,雪花飘下来了,房顶上变白了,地上也变成白色的了,真漂亮。''

% 目的:使婴儿知道雪的感觉,认识雪。

% 注意:观雪景的时间不可过长,小心冻着婴儿。

% %ux516bux5a74ux5e7cux513fux773cux4fddux5065ux64cd}{%
% \subsection{八、婴幼儿眼保健操}%ux516bux5a74ux5e7cux513fux773cux4fddux5065ux64cd}}

% %ux63c9ux592aux9633ux7a74}{%
% \subsubsection{1. 揉太阳穴}%ux63c9ux592aux9633ux7a74}}

% 位置:太阳穴在眉梢与外眼角之间向后移约3厘米处,手指可摸到凹陷。

% 方法:四八拍,一八、三八拍向里揉,二八、四八拍向外揉。

% \hspace{0pt}\includegraphics[width=1.8042in,height=1.83217in]{media/rId1878.png}\hspace{0pt}

% %ux63c9ux6512ux7af9ux7a74}{%
% \subsubsection{2. 揉攒竹穴}%ux63c9ux6512ux7af9ux7a74}}

% 位置:攒竹穴在眉头端处。

% 方法:四八拍,一八、三八拍向里揉,二八、四八拍向外揉。

% \hspace{0pt}\includegraphics[width=1.93007in,height=1.87413in]{media/rId1882.png}\hspace{0pt}

% %ux634bux6cd5}{%
% \subsubsection{3. 捋法}%ux634bux6cd5}}

% 此法是按摩手法。

% 位置:眼部周围。

% 方法:共捋四次,捋的速度要均匀。

% \hspace{0pt}\includegraphics[width=2.11189in,height=1.86014in]{media/rId1886.png}\hspace{0pt}

% %ux6309ux63c9ux775bux660eux7a74}{%
% \subsubsection{4. 按揉睛明穴}%ux6309ux63c9ux775bux660eux7a74}}

% 位置:睛明穴在内眼角处,手指可摸到皮肤下有隆起的结。

% 方法:四八拍,一八、三八拍向里揉,二八、四八拍向外揉抒。

% \hspace{0pt}\includegraphics[width=1.9021in,height=1.8042in]{media/rId1890.png}\hspace{0pt}

% %ux63c9ux56dbux767dux7a74}{%
% \subsubsection{5. 揉四白穴}%ux63c9ux56dbux767dux7a74}}

% 位置:四白穴在眼下3厘米处,直对膜仁。

% 方法:四八拍,一八、三八拍向里揉,二八、四八拍向外揉。

% \hspace{0pt}\includegraphics[width=1.77622in,height=1.91608in]{media/rId1894.png}\hspace{0pt}

% %ux63c9ux5934ux7ef4ux7a74}{%
% \subsubsection{6. 揉头维穴}%ux63c9ux5934ux7ef4ux7a74}}

% 位置:头维穴在额角进入头发边界1.5厘米处。

% 方法:四八拍,一八、三八拍向里揉,二八、四八拍向外揉。

% \hspace{0pt}\includegraphics[width=1.94406in,height=1.9021in]{media/rId1898.png}\hspace{0pt}

% %ux63c9ux98ceux6c60ux7a74}{%
% \subsubsection{7. 揉风池穴}%ux63c9ux98ceux6c60ux7a74}}

% 位置:风池穴在颈后两旁凹陷中。

% 方法:四八拍,一八、三八拍向里揉,二八、四八拍向外揉。

% \hspace{0pt}\includegraphics[width=2in,height=2.05594in]{media/rId1902.png}\hspace{0pt}

% %ux70b9ux538bux5408ux8c37ux7a74}{%
% \subsubsection{8. 点压合谷穴}%ux70b9ux538bux5408ux8c37ux7a74}}

% 位置:合谷穴在虎口处第一掌骨与第二掌骨之间,靠近第二掌骨边缘,用手指压迫时感到酸沉。

% 方法:四八拍,一八、二八拍右手压左手合谷,三八、四八拍左手压右手合谷。

% \hspace{0pt}\includegraphics[width=1.83217in,height=2.22378in]{media/rId1906.png}\hspace{0pt}

% %ux6309ux5149ux660eux7a74}{%
% \subsubsection{9. 按光明穴}%ux6309ux5149ux660eux7a74}}

% 位置:光明穴在小腿外侧胫骨前缘,距足外踝约16厘米处。

% 方法:四八拍,一八、三八拍向里按揉,二八、四八拍外按揉。

% %ux8fdcux773a}{%
% \subsubsection{10. 远眺}%ux8fdcux773a}}

% 两眼向正前方看,在远处找一个比较大的目标,注视时两眼要放松,不可用力细看。远眺时间大约四八拍,也可多看几分钟。

% %ux7b2cux516bux828278ux4e2aux6708ux5a74ux513fux7684ux65e9ux671fux57f9ux80b2}{%
% \subsection{08第八节7〜8个月婴儿的早期培育}%ux7b2cux516bux828278ux4e2aux6708ux5a74ux513fux7684ux65e9ux671fux57f9ux80b2}}

% %ux4e007ux4e2aux6708ux5a74ux513fux7684ux667aux80fdux53d1ux80b2ux6c34ux5e73}{%
% \subsection{一、7个月婴儿的智能发育水平}%ux4e007ux4e2aux6708ux5a74ux513fux7684ux667aux80fdux53d1ux80b2ux6c34ux5e73}}

% 7个月是婴儿智能发育的关键年龄,其行为模式出现了飞跃性变化。

% 大运动:无须用手支撑,能独坐10分钟以上。

% 精细动作:能用全掌把弄到小丸,能自取一块方木后,再取一块。

% 适应能力:能将第一块方木换到另一只手,再去拿第二块方木,能伸手去够远处的玩具。

% 语言:会发``da-da''、``ma-ma''音节。

% 社交行为:对镜中影像有拍打、亲吻和微笑等表情,能认生人。

% %ux4e8cux6ce8ux610fux8badux7ec3ux5b9dux5b9dux624bux7684ux673aux80fd}{%
% \subsection{二、注意训练宝宝手的机能}%ux4e8cux6ce8ux610fux8badux7ec3ux5b9dux5b9dux624bux7684ux673aux80fd}}

% \hspace{0pt}\includegraphics[width=2.33566in,height=3.14685in]{media/rId1915.png}\hspace{0pt}

% 婴儿一过6个月,手的动作会变得更加灵活,已经可以用手抓起东西往嘴里放,也许他要显耀自己的能力,不管什么东西,只要能抓到手就喜欢送到嘴里,有些父母会担心婴儿吃进不干净的东西,阻止婴儿这样做,这是不科学的。婴儿发育到一定阶段就会出现一定的动作,这代表着他的进步,他能将东西往嘴里送,这就意味着他已在为日后自食打下良好的基础,若禁止婴儿用手抓东西吃,可能会打击他们日后学习自己吃饭的积极性。因此,父母应该从积极的方面采取措施,可以把婴儿的手洗干净,让他抓些像饼干、水果片等``指捏食品'',不仅可以训练他手的技能,还能摩擦牙床,以缓解长牙时牙床的刺痛。饼干、水果片通常是这个月龄婴儿最先用手捏起来吃的食物,他会把这些东西放在嘴里吸,也会用牙床咬,经过一番辛苦,能吃进去一部分,另一部分会沾到手上、脸上、头发上和周围的物品上,父母最好\textbf{由他去},不必计较这些小节,重要的是让婴儿体会到自食的乐趣。

% %ux4e09ux57f9ux517bux5a74ux513fux61c2ux9053ux7406}{%
% \subsection{三、培养婴儿懂道理}%ux4e09ux57f9ux517bux5a74ux513fux61c2ux9053ux7406}}

% 7个月婴儿已经知道控制自己的行为。这时,凡是他的合理要求,家长都应该满足他,而对于他的不合理要求,不论他如何哭闹,也不能答应他。比如,他要扭动电视机的按钮,要玩一玩电灯的开关......家长就要板起面孔,向他摆手,严肃地告诉他``不行''。关键的不是怕电视机坏了和电灯绳断了,而是要使孩子节制自己的行为,知道有些事可以去做,而另一些事不可以去做,家长要使孩子\textbf{从小养成讲道理的习惯},以免长大后成为无法无天的小霸王。

% %ux56dbux751fux6d3bux81eaux7406ux80fdux529bux8badux7ec3}{%
% \subsection{四、生活自理能力训练}%ux56dbux751fux6d3bux81eaux7406ux80fdux529bux8badux7ec3}}

% 喝水:训练小儿从盛了水的杯中喝水。

% 交往:继续让小儿多与同伴交往,帮助他克服怯生、焦虑的情绪,引导他正确地表达情感。与同伴玩,是宝宝学习语言、交际能力、培养良好素质的重要途径。

% 良好习惯:训练小儿养成安静入睡、高兴洗脸的习惯,养成定时、定地点大小便的好习惯,学会蹲便盆,大便前出声或做出使劲的表示。

% %ux4e94ux5988ux5988ux6559ux513fux6b4c-1}{%
% \subsection{五、妈妈教儿歌}%ux4e94ux5988ux5988ux6559ux513fux6b4c-1}}

% 肥皂

% 肥皂肥皂,像块糕糕,轻轻擦擦,变成白泡泡。

% 外婆桥

% 摇、摇、摇,摇到外婆桥,外婆真爱我,叫我好宝宝。

% 小木床

% 小木床,四方方,不用妈妈唱,宝宝自己睡床上。闭上眼入梦乡,不用奶奶哄。梦里上天逛一逛。

% %ux516dux667aux529bux57f9ux80b2ux6e38ux620f}{%
% \subsection{六、智力培育游戏}%ux516dux667aux529bux57f9ux80b2ux6e38ux620f}}

% %ux8fa8ux989cux8272}{%
% \subsubsection{1.辨颜色}%ux8fa8ux989cux8272}}

% \hspace{0pt}\includegraphics[width=2.53147in,height=2.6014in]{media/rId1922.png}\hspace{0pt}

% 方法:把玩具堆在一起,妈妈把红色的玩具一样一样拣出来,每拣一样给婴儿看一下说``红色的''。再把其他颜色的玩具照样分开堆。然后把玩具重新混在一起,叫婴儿分。如果婴儿一时还不能很好地做这件事,妈妈可以协助或引导婴儿,帮助他完成这个游戏。

% 目的:训练婴儿的色彩感和思考能力。

% 注意:色彩要单一化,不可复杂。

% %ux89c2ux5bdfux673aux52a8ux73a9ux5177}{%
% \subsubsection{2.观察机动玩具}%ux89c2ux5bdfux673aux52a8ux73a9ux5177}}

% 方法:把电动的或上足发条的机械玩具表演给婴儿看,大人配合玩具特点配以丰富的语言,如:``嘎一嘎,唐老鸭'',``小火车开了,呜呜------''。

% 目的:训练感知觉,培养观察力、注意力,理解简单的因果关系和语言。

% 注意:机动玩具的结构宜简单。

% %ux4e03ux8bedux8a00ux80fdux529bux8badux7ec3}{%
% \subsection{七、语言能力训练}%ux4e03ux8bedux8a00ux80fdux529bux8badux7ec3}}

% %ux61c2ux5f97ux4e0d}{%
% \subsubsection{1.懂得``不''}%ux61c2ux5f97ux4e0d}}

% 方法:妈妈指着热水杯对孩子严肃地说:``烫,不要动!''同时拉着孩子的手轻轻触摸杯子,然后把他的手离开物品,或轻轻拍打他的手,示意他停止动作。对小孩不该拿的东西要明确地说``不'',使其懂得``不''的意义。

% 目的:在理解``不''的基础上,增强对语言的理解能力。

% 注意:还要懂得大人的摇头摆手也表示``不''。

% %ux542cux53e3ux4ee4ux628aux73a9ux5177ux5012ux624b}{%
% \subsubsection{2.听口令把玩具倒手}%ux542cux53e3ux4ee4ux628aux73a9ux5177ux5012ux624b}}

% 方法:在学会玩具倒手的基础上,先给孩子一个玩具,让他用右手拿,再给他一块饼干,告诉他``倒手,倒手'',做对了,亲亲他,并奖励他。

% 目的:让小儿练习在口语指导下把玩具倒手,学会两手并用。

% 注意:奖励不一定是食品,只要是孩子喜欢的任一物品都可以。

% %ux516bux793eux4ea4ux80fdux529bux8badux7ec3}{%
% \subsection{八、社交能力训练}%ux516bux793eux4ea4ux80fdux529bux8badux7ec3}}

% %ux5b66ux4e60ux6325ux624bux62f1ux624bux52a8ux4f5c}{%
% \subsubsection{1学习挥手、拱手动作}%ux5b66ux4e60ux6325ux624bux62f1ux624bux52a8ux4f5c}}

% \hspace{0pt}\includegraphics[width=2.97902in,height=2.67133in]{media/rId1931.png}\hspace{0pt}

% 方法:经常将孩子右手举起,并不断挥动,让孩子学习``再见''动作。大人离家时要对孩子挥手,并说``再见'',反复练习。在孩子情绪好时,帮助孩子将两手握拳对起,然后不断摇动,学做谢谢动作。

% 目的:提高语言理解的能力。

% 注意:每次给孩子食品或玩具时,先让他拱手表示谢谢,然后再给他。

% %ux7ec3ux4e60ux722cux884c}{%
% \subsubsection{2.练习爬行}%ux7ec3ux4e60ux722cux884c}}

% 方法:孩子俯卧位,放在铺有毯子的地板上,家长将孩子喜欢的玩具放在前方,鼓励宝宝用力向前爬行,去取玩具。必要时家长可用手轻推孩子的脚掌协助。孩子会越爬越显得自如,扩大了视野,提高了观察力和思维能力。

% 目的:爬行是代表婴儿智能发展的重要动作之一,通过爬行训练全身肌肉,扩大视野和提高脑的统合能力。

% 注意:家长时刻在旁保护婴儿别跌落受伤。

% %ux4e5dux60c5ux611fux57f9ux80b2ux8badux7ec3}{%
% \subsection{九、情感培育训练}%ux4e5dux60c5ux611fux57f9ux80b2ux8badux7ec3}}

% %ux9a91ux5927ux9a6c}{%
% \subsubsection{1.骑大马}%ux9a91ux5927ux9a6c}}

% 方法:让婴儿面对妈妈骑在妈妈膝头,妈妈将双腿有节奏地上下颤动,一边颠一边说:``骑大马,呱咕哒,一跑跑到外婆家,见了外婆问声好,外婆对我笑哈哈。''

% 目的:培养婴儿的语感和节奏感,还可以引起婴儿欢乐的情绪。

% 注意:可反复地玩,婴儿很感兴趣。

% %ux53ebux540dux513f}{%
% \subsubsection{2.叫名儿}%ux53ebux540dux513f}}

% 方法:用相同的语调叫婴儿的名字和其他人的名字。看是否在叫到婴儿的名字时,他能转过头来,露出微笑,表示明白了。孩子如能准确听出自己的名字来,妈妈要说:``噢!对啦!你就是X
% X,X
% X真聪明!''之类的话,同时把婴儿抱起来,贴贴他的小脸。如果婴儿对叫声没有反应,就要耐心地反复地告诉他:``\,``X
% X,你就是X X呀。''

% 目的:训练婴儿对特定语言的反应,让婴儿知道自己是谁。

% 注意:婴儿心情不愉快时不要进行该游戏。

% %ux5341ux6a21ux4effux80fdux529bux8badux7ec3}{%
% \subsection{十、模仿能力训练}%ux5341ux6a21ux4effux80fdux529bux8badux7ec3}}

% %ux62cdux62cdux624bux70b9ux70b9ux5934}{%
% \subsubsection{1.拍拍手、点点头}%ux62cdux62cdux624bux70b9ux70b9ux5934}}

% \hspace{0pt}\includegraphics[width=2.43357in,height=2.68531in]{media/rId1940.png}\hspace{0pt}

% 方法:与婴儿对面而坐,先握住他的两只小手,边拍边对他说``拍拍手'';然后不要握他的手,你边拍边有节奏地对他说``拍拍手'',教他模仿,``点点头''亦如此。

% 目的:提高理解语言与模仿的能力。

% 注意:每天应不断重复的学习。

% %ux6d17ux6fa1ux73a9ux6c34}{%
% \subsubsection{2.洗澡玩水}%ux6d17ux6fa1ux73a9ux6c34}}

% 方法:把婴儿放进盆里坐着,给他一只吹气小鸭子边洗边玩,洗完澡后坐在盆中央,大人拿着婴儿的两只胳膊或一人扶着婴儿腋下,一人握着婴儿的双脚,边拍打水边念儿歌:``小小鸭子嘎嘎叫,走起路来摇呀摇,一摇摇到小河里,高高兴兴洗个澡。''

% 目的:熟悉水,提高感知能力,培养愉快情绪。

% 注意:不要在洗澡期间离开,以免婴儿溺水,发生危险。

% %ux5341ux4e00ux89c6ux89c9ux80fdux529bux8badux7ec3}{%
% \subsection{十一、视觉能力训练}%ux5341ux4e00ux89c6ux89c9ux80fdux529bux8badux7ec3}}

% %ux73a9ux73a9ux5076}{%
% \subsubsection{1.玩玩偶}%ux73a9ux73a9ux5076}}

% 方法:

% \begin{enumerate}
% \def\labelenumi{\arabic{enumi}.}
% \item
%   准备一双大号的白色袜子,你可以轻松地将手伸进去,
% \item
%   用粗头的马克笔在袜子的趾尖部位画上眼睛、眉毛、鼻子、耳朵,沿着脚跟部位的弧线画出嘴巴,并在褶皱处画出红色的舌头,
% \item
%   让宝宝坐在你温暖的怀抱里
% \item
%   将玩偶套在你的手上,对着宝宝唱歌、念儿歌或以玩偶的口气和宝宝说说话,另一只手上再套一个玩偶,让两只玩偶对话或是做游戏,宝宝将会更感兴趣。
% \end{enumerate}

% 用宝宝的袜子为宝宝做个小玩偶。游戏时,把小玩偶套在宝宝的小手上,同时,也可以套上大玩偶,这样的交流方式多有趣啊!

% 目的:让宝宝增强视觉敏锐度及语言和倾听能力。

% 注意:也可以在袜子上缝出眼睛、绒球鼻子、红布嘴巴等。

% %ux5bfbux627eux8d34ux7eb8}{%
% \subsubsection{2.寻找贴纸}%ux5bfbux627eux8d34ux7eb8}}

% 方法:

% \begin{enumerate}
% \def\labelenumi{\arabic{enumi}.}
% \item
%   脱去宝宝的衣服,只留尿布即可;
% \item
%   让宝宝靠坐在婴儿椅上,如果他已经学会坐,就让他坐在地板上;
% \item
%   坐在宝宝对面,并在附近放一些彩色小贴纸,
% \item
%   先给宝宝看其中一张贴纸,然后将它粘在宝宝身体任一部位上,不要让宝宝知道贴纸跑到哪儿去了,你可以先把贴纸藏在选定位置后,再将贴纸贴上;
% \item
%   贴好后,问宝宝:``贴纸在哪里?''
% \item
%   开始在宝宝身体上寻找贴纸,检查他的脸部,说:``没一有,不在这里。''
%   检查他的手臂,说:``没------有,不在这里。''继续检查,直到发现贴纸,然后说:``噢,在这里!''同时让宝宝看看贴纸在他身上;
% \item
%   换另一张贴纸,再和宝宝玩一次,把贴纸贴在宝宝不同的身体部位上,
% \item
%   可以让宝宝自己找贴纸,如果必要的话,给宝宝一点提示。
% \end{enumerate}

% 目的:培养宝宝的注意力、观察力及视觉搜寻能力。

% 注意:贴纸很小而且容易吞食,所以看好宝宝。

% %ux5341ux4e8cux542cux89c9ux80fdux529bux8badux7ec3}{%
% \subsection{十二、听觉能力训练}%ux5341ux4e8cux542cux89c9ux80fdux529bux8badux7ec3}}

% %ux8fb9ux8bf4ux8fb9ux4e22}{%
% \subsubsection{1.边说边丢}%ux8fb9ux8bf4ux8fb9ux4e22}}

% 方法:父母拿起丢在地上的玩具,边说``砰'',边把玩具丢入袋中或箱子里,然后扶着孩子的手,让他学着做。不久,只要说声``砰'',孩子就会把玩具丢入箱中。

% 目的:训练婴儿听觉能力。

% 注意:根据孩子对声音感兴趣的特点,可教孩子说些象声词。

% %ux542cux97f3ux627eux7269-1}{%
% \subsubsection{2.听音找物}%ux542cux97f3ux627eux7269-1}}

% 方法:给小儿看形象逼真的玩具和图片,告诉他名称并逗引他用眼睛去找,用手去指。如看到室内红红绿绿的气球后,家长说:``宝宝,球球在哪里?''
% 他听后会抬头去找,用手去指。诸如此类反复练习,可促进小儿听、视觉和动作协调发展。

% 目的:训练小儿听觉、视觉和动作的协调。

% 注意:玩具和图片要有益于婴儿的身心健康。

% %ux5341ux4e09ux52a8ux4f5cux80fdux529bux8badux7ec3}{%
% \subsection{十三、动作能力训练}%ux5341ux4e09ux52a8ux4f5cux80fdux529bux8badux7ec3}}

% %ux7ec3ux4e60ux72ecux5750}{%
% \subsubsection{1.练习独坐}%ux7ec3ux4e60ux72ecux5750}}

% 方法:将孩子置于座位,平坐硬床上,不用成人去支撑,使坐姿日趋平稳,达到独坐自如。

% 在继续做完八节被动体操的基础上,如孩子能够适应,试着练习少量的自动体操,促进全身肌肉和关节活动的发展。

% 目的:孩子已能独坐,继续锻炼颈、背、腰的肌肉力量。

% 注意:训练时间不宜过长。

% %ux7ffbux8eabux53d6ux7269}{%
% \subsubsection{2.翻身取物}%ux7ffbux8eabux53d6ux7269}}

% \hspace{0pt}\includegraphics[width=2.53147in,height=2.16783in]{media/rId1965.png}\hspace{0pt}

% 方法:让小儿平卧,将鲜艳带响的玩具放在孩子的一侧摇响,引逗孩子去取时家长将小儿的胳臂轻轻推向有玩具的一方,帮助小儿翻身,抓住玩具。

% 目的:练习翻身变更体位。

% 注意:在此基础上可以逐步练习连续翻滚。

% %ux5a74ux5e7cux513fux5065ux8eabux64cd-2}{%
% \subsubsection{3.婴幼儿健身操}%ux5a74ux5e7cux513fux5065ux8eabux64cd-2}}

% 腰部运动:婴儿仰卧两腿伸直,家长右手托住孩子腰部,左手按住两脚踝部,右手用力将孩子腰部轻轻托起。托起腰部时,应注意孩子的头部不要离开床面,用力不要过猛。

% %ux7b2cux4e5dux828289ux4e2aux6708ux5a74ux513fux7684ux65e9ux671fux57f9ux80b2}{%
% \subsection{09第九节8〜9个月婴儿的早期培育}%ux7b2cux4e5dux828289ux4e2aux6708ux5a74ux513fux7684ux65e9ux671fux57f9ux80b2}}

% %ux4e008ux4e2aux6708ux5a74ux513fux7684ux667aux80fdux53d1ux80b2ux6c34ux5e73}{%
% \subsection{一、8个月婴儿的智能发育水平}%ux4e008ux4e2aux6708ux5a74ux513fux7684ux667aux80fdux53d1ux80b2ux6c34ux5e73}}

% 大运动:双手扶栏杆可站立5分钟以上。

% 精细动作:能用拇指和其他指捏住小丸,有要取第三块方木的表示,不一定取到。

% 适应能力:用手持续追逐玩具,力图拿到;有意识地摇铃。

% 语言:模仿弹舌或咳嗽的声。

% 社交行为:对大人训斥或赞扬,有委曲或兴奋的不同表情。

% %ux4e8cux8bedux8a00ux80fdux529bux8badux7ec3-2}{%
% \subsection{二、语言能力训练}%ux4e8cux8bedux8a00ux80fdux529bux8badux7ec3-2}}

% %ux8bb2ux6545ux4e8b}{%
% \subsubsection{1 讲故事}%ux8bb2ux6545ux4e8b}}

% \hspace{0pt}\includegraphics[width=2.55944in,height=3.02098in]{media/rId1973.png}\hspace{0pt}

% 方法:妈妈可给孩子买一些构图简单、色彩鲜艳、故事情节单一、内容有趣的婴儿画册。在孩子有兴趣时,一边翻看、指点画册上的图像,一边用清晰而缓慢的语调给他讲故事。同一故事应当反复地讲。

% 例如:小蝌蚪找妈妈

% 小蝌蚪一出世就没见过它妈妈,它看别的小朋友都跟在妈妈身后玩耍,很羡慕。它想:我也要去找妈妈。于是,它游啊游啊,碰见一条鲤鱼,它就大叫:``妈妈,妈妈!''鲤鱼说:``我不是你妈妈。''小蝌蚪又游啊游,碰见一条泥鳅,它又大叫:``妈妈,妈妈!''泥鳅说:``我不是你妈妈。''小蝌蚪伤心地哭了。这时一只大青蛙跳下来,对小蝌蚪说,``小蝌蚪,我就是你妈妈。''小蝌蚪说:``我为什么和你长得不一样?''青蛙说:``等你长大了,就会长出四条腿,尾巴也没有了,就和妈妈一个样了。''小蝌蚪高兴地说:``哦,原来我会变成一只小青蛙。''

% 目的:给孩子讲故事,是促进其语言发展与智力开发的好办法,无论孩子是否能够听懂,妈妈一有时间都应绘声绘色地讲给孩子听,培养孩子爱听故事、对图书感兴趣的习惯。

% 注意:孩子实在不肯和妈妈一道看画册、听故事,妈妈也不必急躁,可以过一段时间后再试试。

% %ux8bedux8a00ux52a8ux4f5cux7ec3ux4e60}{%
% \subsubsection{2
% 语言动作练习}%ux8bedux8a00ux52a8ux4f5cux7ec3ux4e60}}

% 方法:在拿孩子熟悉的物品时,边说边问:``宝宝要不要饼干?''\,``宝宝要不要小熊?''让他用手推开或皱眉表示不喜欢,用伸手、点头、谢谢表示喜欢,表示要。

% 目的:训练婴儿理解语言能力。

% 注意:如果婴儿一时说不出来,家长\textbf{可适当做些提示}。

% %ux4e09ux793eux4ea4ux80fdux529bux8badux7ec3-1}{%
% \subsection{三、社交能力训练}%ux4e09ux793eux4ea4ux80fdux529bux8badux7ec3-1}}

% %ux8c22ux8c22ux518dux89c1}{%
% \subsubsection{1. ``谢谢''、``再见''}%ux8c22ux8c22ux518dux89c1}}

% 方法:爸爸给婴儿玩具或东西吃时,妈妈在一旁要讲``谢谢'',并要引导婴儿模仿点头或鞠躬的动作以表示``谢谢''。当家里有人要出门时,妈妈一面说``再见'',一面挥动婴儿的小手,向要走的人表示``再见''。反复训练,使他一听到``谢谢''就鞠躬或点头,一听到``再见''就挥手。

% 目的:掌握语言,发展动作,培养礼貌习惯。

% 注意:不可半途而废,必须经常训练。

% %ux4ea4ux670bux53cb}{%
% \subsubsection{2 交朋友}%ux4ea4ux670bux53cb}}

% 方法:户外活动时,可抱着婴儿和别的母亲抱着的婴儿相互接触,看一看或摸一摸别的婴儿,或在别人面前表演一下婴儿的本领,或观看别的婴儿的本领。也可让婴儿和其他同龄婴儿在铺有席子的地上互相追随爬着玩,或抓推滚着的小皮球玩,或和大一些的婴儿在一起玩。看他是否更喜欢和较大的婴儿在一起玩。

% 目的:锻炼社会交往能力。

% 注意:如果婴儿出现抓别人脸或抢别人的玩具等行为时,要制止他。

% %ux56dbux60c5ux611fux57f9ux80b2ux8badux7ec3}{%
% \subsection{四、情感培育训练}%ux56dbux60c5ux611fux57f9ux80b2ux8badux7ec3}}

% %ux6311ux7ef3ux5b50}{%
% \subsubsection{1 挑绳子}%ux6311ux7ef3ux5b50}}

% 方法:妈妈抱婴儿坐在桌前,桌上放着两根绳子,一根绳系着玩具,一根没有,让婴儿自己选择拉哪根绳。经过多次练习,孩子可以分辨出哪条绳上有玩具,哪条绳上没有玩具,并很快地把玩具拉过来玩。

% 目的:初步训练婴儿分辨事物的能力。

% 注意:渐渐地可增加绳子和玩具的数量。

% %ux78b0ux78b0ux5934}{%
% \subsubsection{2.碰碰头}%ux78b0ux78b0ux5934}}

% 方法:面对着你的婴儿扶着他的腋下,用自己的额部轻轻地触及婴儿的额部,并亲切愉快地呼唤他的名字,说:``碰碰头。''训练多次后,当你头稍向前倾时,他就会主动把头凑过来,并露出高兴的笑容。

% 目的:促进语言与动作的联系,引起愉快情绪。

% 注意:在此基础上还可教些其他动作,如亲一下妈妈,亲一下爸爸等。

% %ux4e94ux6280ux80fdux8badux7ec3ux6e38ux620f}{%
% \subsection{五、技能训练游戏}%ux4e94ux6280ux80fdux8badux7ec3ux6e38ux620f}}

% %ux4eceux751fux6d3bux4e2dux5b66}{%
% \subsubsection{1 从生活中学}%ux4eceux751fux6d3bux4e2dux5b66}}

% 方法:生活中的情景是丰富多彩和千变万化的,要引导和鼓励孩子观察和体验日常生活中所发生的一切,提高适应能力。比如,他观察到大人是用杯子饮水,经过一段时问他便会在大人给他端住杯子的条件下,咕噜咕噜的喝水;他观察到大人用拍手表示``欢迎'',用挥手表示``再见'',过一段时间,他也会模仿出这些动作。诸如此类,有很多内容,家长应留意在生活中寻找引导和培养孩子的机会。

% 目的:从生活中观察和学习各方面的有关知识。

% 注意:一定要把教语言和认识物体结合起来,反复教孩子认识他熟悉并喜爱的日常用品和玩具等名称。

% %ux4e8cux6234ux5e3dux5b50}{%
% \subsubsection{二、戴帽子}%ux4e8cux6234ux5e3dux5b50}}

% 方法:准备形状不同的帽子,如小布帽、毛线帽、军帽、皮帽、太阳帽、纸帽等,把婴儿抱在大镜子前给他戴上一顶帽子说``帽子''。玩一会儿把帽子摘下再戴上另一顶,还说``帽子''。依此类推。

% 目的:掌握语言,促使思维萌芽,形成概念。

% 注意:逐渐使婴儿知道尽管这些玩意儿大小、形状、颜色不同,但都是帽子,可以戴在头上。

% %ux516dux89c6ux89c9ux80fdux529bux8badux7ec3-1}{%
% \subsection{六、视觉能力训练}%ux516dux89c6ux89c9ux80fdux529bux8badux7ec3-1}}

% %ux8ba4ux8bc6ux4e94ux5b98}{%
% \subsubsection{1 认识五官}%ux8ba4ux8bc6ux4e94ux5b98}}

% \hspace{0pt}\includegraphics[width=2.58741in,height=2.29371in]{media/rId1988.png}\hspace{0pt}

% 方法:画出五官图片,并在图片上写上相应的字,如在画好的鼻子上写一个大大的``鼻''字。先教孩子指图说``鼻子'',再指自己的``鼻子'',再指字说``鼻子''。多次重复之后,孩子懂得图和字都是鼻子,当大人指图或字时,看看孩子是否指自己的鼻子。用同样方法孩子可以学认眼、耳、嘴、舌等字。

% 目的:训练视觉能力。

% 注意:孩子在学认五官的字之后,父母应经常调换字的先后次序和位置,反复训练,使孩子逐渐地学认身体其他部位的名称,如手、头、脚等。

% %ux7f8eux4e3dux7684ux661fux7a7a}{%
% \subsubsection{2 美丽的星空}%ux7f8eux4e3dux7684ux661fux7a7a}}

% \hspace{0pt}\includegraphics[width=2.88112in,height=2.96503in]{media/rId1992.png}\hspace{0pt}

% 方法:繁星满天的晚上,父母可与孩子来到室外,一起观察星空。这时,父母可告诉孩子``这是月亮''、``这是星星'',并拿着孩子的手数星星。然后,父母可给孩子唱一首关于星星的歌。

% 目的:训练婴儿视觉能力。

% 注意:不宜在冬夜进行此种训练。

% %ux4e03ux542cux89c9ux80fdux529bux8badux7ec3-1}{%
% \subsection{七、听觉能力训练}%ux4e03ux542cux89c9ux80fdux529bux8badux7ec3-1}}

% %ux9505ux7897ux74e2ux76c6ux4ea4ux54cdux66f2}{%
% \subsubsection{1
% 锅碗瓢盆交响曲}%ux9505ux7897ux74e2ux76c6ux4ea4ux54cdux66f2}}

% 方法:生活中,可引导婴儿在能发声的物体上有节奏地拍拍、敲敲、碰碰。可敲各种器皿,譬如用筷子敲敲盆子、碗、酒瓶、瓷盆等,可拍桌子、盆子、皮球等,也可让积木相碰、瓶子相碰、锅勺相碰等,同样可发出有节奏的声音。

% 目的:训练听觉能力。

% 注意:两物相碰或敲击时,声音不能过大,以防变成噪声,月龄小的孩子以父母辅助为主,月龄大的孩子以自己玩为主。

% %ux5b66ux53d1ux97f3}{%
% \subsubsection{2 学发音}%ux5b66ux53d1ux97f3}}

% 方法:举起婴儿的小手,放在母亲嘴唇上,发一个长音``啊一啊一''\,``呜一呜一''等重复音节,把婴儿的手从母亲嘴唇上挪开,和孩子面对面,用丰富的表情重复上面的音节,逗引婴儿注意母亲的口形,分别以低音高音继续上面的音节;重复几次再加一些新的重复音节``ba-ba,ma-ma''。

% 目的:这是为了发展婴儿的语言能力、听力和模仿力。

% 注意:每发一个重复音节应停顿一下,给孩子模仿的机会,可以把孩子抱到镜子前,练习模仿。

% %ux516bux611fux89c9ux80fdux529bux8badux7ec3-2}{%
% \subsection{八、感觉能力训练}%ux516bux611fux89c9ux80fdux529bux8badux7ec3-2}}

% %ux5403ux6c34ux679c}{%
% \subsubsection{吃水果}%ux5403ux6c34ux679c}}

% \hspace{0pt}\includegraphics[width=2.83916in,height=2.16783in]{media/rId2000.png}\hspace{0pt}

% 方法:夏天有许多水果,当孩子吃水果时,父母要边喂孩子,边用语言描绘。如吃葡萄:``这是葡萄,熟透了,你看紫莹莹的,宝宝尝一尝,葡萄很甜,啊呜一口吃掉了!''还有苹果、梨等,要经常用语言伴随动作来描述。

% 目的:培养婴儿的感知觉。

% 注意:当孩子见过尝过许多水果之后,再拿图片告诉他这是什么、什么颜色的、味道如何。继续给小儿抚摸、亲吻,如配合儿歌或音乐的拍子,握着小儿的手,教他拍手,按音乐节奏模仿小鸟飞,跳动身体。

% %ux4e5dux624bux773cux534fux8c03ux80fdux529bux8badux7ec3}{%
% \subsection{九、手眼协调能力训练}%ux4e5dux624bux773cux534fux8c03ux80fdux529bux8badux7ec3}}

% %ux634fux7cd6ux4e38}{%
% \subsubsection{1、捏糖丸}%ux634fux7cd6ux4e38}}

% 方法:让婴儿坐在你的腿上,两肘搁在桌面上,在桌上的盘子里放一个有盖的透明的杯子,里面装有彩色糖丸,先摇动杯子发出柔和的响声,并让婴儿看到糖丸在杯中跳动以激起婴儿玩的兴趣,再打开盖子(让他发现糖丸),把糖丸倒在盘子里,告诉他``这是糖''。边说边示范把一粒糖丸从盘里拣起放进杯子里(要用``慢镜头''),放进几粒后,让他用拇指和食指捏起小丸,再放进杯子里。开始你可以手把手教他,稍熟练后让他自己把糖丸放进杯子后再加盖摇一摇,发出有趣的声音。

% 目的:提高拇指、食指配合捏物的灵活性和手眼协调能力。

% 注意:玩时一定要有大人认真照看,避免婴儿吞食糖丸或发生呛噎、窒息。

% %ux62feux7269}{%
% \subsubsection{2、拾物}%ux62feux7269}}

% 方法:把婴儿放在床上,大人在后面用双手分别抱着婴儿的胸、腹部及膝部,把婴儿感兴趣的玩具放在婴儿前面床上,用语言逗引婴儿弯腰去捡玩具,捡到玩具后再直起身,反复多次训练。等婴儿学会扶站后,可将婴儿扶站在有栏杆的小床边,让婴儿一只手扶栏杆,如没有小床,大人可抓住婴儿一只手,使婴儿站稳,在婴儿脚边放一个玩具,帮助婴儿弯下腰用另一只手捡身边的玩具。

% 目的:训练手眼协调能力和婴儿弯曲及直立身体。

% 注意:捡到玩具后,大人可用语言或行动给婴儿一点表扬,如说``宝宝真能干''或亲吻一下婴儿,婴儿就会很愉快地再次去捡玩具。

% %ux5341ux52a8ux4f5cux80fdux529bux8badux7ec3}{%
% \subsection{十、动作能力训练}%ux5341ux52a8ux4f5cux80fdux529bux8badux7ec3}}

% %ux7ad9ux7acbux8fd0ux52a8}{%
% \subsubsection{1、站立运动}%ux7ad9ux7acbux8fd0ux52a8}}

% 方法:让婴儿抓住妈妈的大拇指,妈妈轻轻地把他从卧位拉到坐位,然后再拉他慢慢站起。每天练习几次,增强肩、胸的活动能力,等婴儿能站后,可以在床栏上挂些玩具,吸引婴儿站起来取玩具,妈妈在旁边帮忙和照顾,另外,可以让婴儿在床上站好,从旁边或前边轻轻地推他一下,使他失去平衡,再用另一只手准备扶住婴儿,避免他跌倒。

% 目的:训练婴儿站的能力和平衡能力。

% 注意:站立时间不宜过长。

% %ux8ffdux76aeux7403}{%
% \subsubsection{2. 追皮球}%ux8ffdux76aeux7403}}

% 方法:妈妈把皮球从床的这一边滚到床的另一边,或者把小鸭子从床的这边拉到那边,引导婴儿爬过去把皮球或鸭子抱起来,用手拍打大皮球或小鸭子。

% 目的:训练婴儿熟练爬行动作,活动全身各部位肌肉。

% 注意:不要太靠近床边,防止婴儿摔下床。

% %ux5a74ux5e7cux513fux5065ux8eabux64cd---ux817fux90e8ux8fd0ux52a8}{%
% \subsubsection{3. 婴幼儿健身操 -
% 腿部运动}%ux5a74ux5e7cux513fux5065ux8eabux64cd---ux817fux90e8ux8fd0ux52a8}}

% 让宝宝成俯卧状态,家长在孩子后面,两手捶住其手腕部,扶着宝宝跪起,然后再扶其站直。重复两遍。应该注意,当宝宝跪直后应该尽量让他自己用力站起来。

% %ux7b2cux5341ux82829-10ux4e2aux6708ux5a74ux513fux7684ux65e9ux671fux57f9ux80b2}{%
% \subsection{10第十节9 \textasciitilde{}
% 10个月婴儿的早期培育}%ux7b2cux5341ux82829-10ux4e2aux6708ux5a74ux513fux7684ux65e9ux671fux57f9ux80b2}}

% %ux4e009ux4e2aux6708ux5a74ux513fux7684ux667aux80fdux53d1ux80b2ux6c34ux5e73}{%
% \subsection{一、9个月婴儿的智能发育水平}%ux4e009ux4e2aux6708ux5a74ux513fux7684ux667aux80fdux53d1ux80b2ux6c34ux5e73}}

% 大运动:会爬,爬时仅用手脚(或手、膝)动作,躯干抬高,腹离床面;拉双手能向前走3步以上。

% 精细动作:能用拇指和食指捏住小丸。

% 适应能力:方木放入杯中后,能自行将方木从杯中取出;能用一只手中的方木明确地击打另一只手中的方木。

% 语言:能有``欢迎''和``再见''的表示。

% 社交行为:对不要之物(如食物或玩具)有``不要''(摇头或推开)的表示。

% %ux4e8cux57f9ux517bux5a74ux513fux7684ux826fux597dux54c1ux8d28}{%
% \subsection{二、培养婴儿的良好品质}%ux4e8cux57f9ux517bux5a74ux513fux7684ux826fux597dux54c1ux8d28}}

% 现在的独生子女都是父母的掌上明珠,但如果孩子有不合理要求时,家长应该拒绝他,绝不能看到他一哭一闹,心软了就对他让步而迁就他。要知道迁就会使孩子养成任性的习惯,越迁就,孩子越任性,长大之后就难以纠正了。比如,他玩玩具烦了,想要玩大人的眼镜,就要明确地告诉他:``这不是玩具,不能给你玩。''不管他如何哭闹,都不要去理睬他,等他闹过了,再和他讲道理。如果家长因为他哭闹而妥协的话,以后凡是没有达到孩子要求的,他就会以更加拼命的哭闹来达到目的。这样放任的结果是害了孩子。

% %ux4e09ux57f9ux517bux5a74ux513fux6b23ux8d4fux5927ux81eaux7136}{%
% \subsection{三、培养婴儿欣赏大自然}%ux4e09ux57f9ux517bux5a74ux513fux6b23ux8d4fux5927ux81eaux7136}}

% \hspace{0pt}\includegraphics[width=2.43357in,height=2.6014in]{media/rId2015.png}\hspace{0pt}

% 这个时期的婴儿心理活动的发展很快,出现了认生、害羞、兴奋等各种情绪反应。家长除了给婴儿各种玩具以及和他逗乐外,还应让孩子到大自然中去,让自然界的各种动植物、自然景观给婴儿以良好的感官刺激,使婴儿得到心理的安宁与美的享受,培养婴儿稳定、美好的情感,为以后良好的性格形成奠定基础。

% 家长可以带婴儿到公园去,看看公园各种颜色鲜艳的花朵、各种动物、小桥流水。具有色彩的或处于动态的自然景色特别能引起婴儿的注意,如飞舞的彩蝶、蜻蜓,在水中游动的各种色彩斑斓的金鱼,婴儿常常看得目不转睛,呈现出愉悦的表情。还可以让婴儿看太阳、月亮,看彩虹,看下雨,看飘扬的雪花等自然景观,这些活动可以促进孩子感知觉的发展,有益于婴儿的身心健康与智能发育。

% %ux56dbux751fux6d3bux81eaux7406ux80fdux529bux7684ux8badux7ec3-1}{%
% \subsection{四、生活自理能力的训练}%ux56dbux751fux6d3bux81eaux7406ux80fdux529bux7684ux8badux7ec3-1}}

% 大小便坐盆:训练宝宝养成大小便坐盆的习惯。此时小儿尚不能完全主动表示,可在宝宝有便意时定地点、定时协助他坐盆。

% 配合穿衣:给宝宝穿衣服时要告诉他``伸手''、``举手''、``抬腿''等,让他用动作配合穿衣、穿裤。如果他还未听懂就用手去示范协助。经常表扬他的合作,以后他就会主动伸臂入袖、伸腿穿裤。

% %ux4e94ux5988ux5988ux6559ux513fux6b4c-2}{%
% \subsection{五、妈妈教儿歌}%ux4e94ux5988ux5988ux6559ux513fux6b4c-2}}

% 我有一双小小手

% 我有一双小小手,一只左来一只右。小小手,小小手,一共十个手指头。我有一双小小手,能洗脸来能漱口。会穿衣,会梳头,自己事情自己做。

% 小白兔

% 小白兔,白又白,两只耳朵竖起来,爱吃萝卜爱吃菜,蹦蹦跳跳真可爱。

% 虫虫飞

% 虫虫飞,虫虫飞,飞到南山喝露水,露水喝不到,回来吃青草。

% %ux516dux8bedux8a00ux80fdux529bux8badux7ec3-1}{%
% \subsection{六、语言能力训练}%ux516dux8bedux8a00ux80fdux529bux8badux7ec3-1}}

% \hspace{0pt}\includegraphics[width=2.92308in,height=2.44755in]{media/rId2021.png}\hspace{0pt}

% %ux5b66ux8bcdux97f3ux8bcdux4e49}{%
% \subsubsection{1 学词音、词义}%ux5b66ux8bcdux97f3ux8bcdux4e49}}

% 方法:9个月的孩子不但要教他听懂词音,而且该教他听懂词义,家长要训练孩子把一些词和常用物体联系起来,因为这时小儿虽然还不会说话,但是已经会用动作来回答大人说的话了。比如,家长可以指着电灯告诉孩子说:``这是电灯。''然后再问他:``电灯在哪?''他就会转向电灯方向,或用手指着电灯,同时可能会发出声音。这虽然还不是语言,但对小儿发音器官是一个很好的锻炼。

% 目的:为模仿说话打下基础。

% 注意:家长还可以联系吃、喝、拿、给、尿、娃娃、皮球、小兔、狗等跟孩子说简单的词语。

% %ux5ff5ux513fux6b4cux8bb2ux6545ux4e8bux770bux56feux4e66}{%
% \subsubsection{2
% 念儿歌,讲故事,看图书}%ux5ff5ux513fux6b4cux8bb2ux6545ux4e8bux770bux56feux4e66}}

% 方法:每晚睡前给宝宝读一个简短、朗朗上口的故事,最好一字不差,一个记住了,再换别的以便加深宝宝的印象和记忆。

% 图书对宝宝来说是一种能打开合上的、能学说话的玩具,因此宝宝非常喜欢大人陪着他看图书,听大人给他讲书中的故事。图书画面要清楚,色彩要鲜艳,图像要大,文字大而对话简短生动,并多次重复出现,便于宝宝模仿。每天坚持念儿歌、讲故事、看图书,并采取有问有答的方式讲述图书中的故事,耳濡目染,宝宝就会对图书越来越感兴趣,对宝宝学习语言很有帮助。喜欢读书,对他一生具有重要的意义。

% 目的:帮助婴儿学习语言。

% 注意:这么大孩子的注意力集中时间很短,一般几十秒到1分钟。只要孩子有兴趣,很高兴,就与他一起念儿歌、讲故事,看图书,否则就没有意义了。

% %ux4e03ux793eux4ea4ux80fdux529bux8badux7ec3-1}{%
% \subsection{七、社交能力训练}%ux4e03ux793eux4ea4ux80fdux529bux8badux7ec3-1}}

% %ux6a21ux4effux5927ux4ebaux52a8ux4f5c}{%
% \subsubsection{1
% 模仿大人动作}%ux6a21ux4effux5927ux4ebaux52a8ux4f5c}}

% 方法:小儿在注视大人动作的基础上开始用成套动作来表演儿歌。父母要先设计好全套动作并配上相应的儿歌或短语,每次动作都要一样,包括拍手、摇头、身体扭动、踏脚或特殊手势示范动作,孩子很快就能学会而且能单独表演。

% 目的:训练表演与模仿能力。

% 注意:学习时每做对一种都要表扬鼓励。

% %ux542cux7238ux7238ux5988ux5988ux7684ux8bdd}{%
% \subsubsection{2
% 听爸爸妈妈的话}%ux542cux7238ux7238ux5988ux5988ux7684ux8bdd}}

% \hspace{0pt}\includegraphics[width=2.93706in,height=2.44755in]{media/rId2028.png}\hspace{0pt}

% 方法:爸爸、妈妈可事先准备一些宝宝熟悉的物品,比如几样玩具一汽车、布娃娃、皮球、摇铃等,准备几样日用品一小板凳、勺子、小塑料碗等;几样食物------香蕉、苹果、煮熟的鸡蛋等一些东西。游戏进行前,妈妈可和宝宝坐在一起,爸爸拿起一样东西,比如说玩具汽车,妈妈就说``玩具汽车'',加深宝宝的认识。再拿起一个香蕉,妈妈对宝宝说:``香蕉,这是香蕉。''这样,让宝宝明白每一样物体分别都是什么。然后进入游戏的第二步,爸爸挑几样东西分散放在屋内的各个地方。妈妈问宝宝:``宝宝找一找,宝宝的玩具汽车在哪里?''宝宝就会用眼睛去寻找妈妈问的东西。如此进行,妈妈把每样物品都问一遍。然后进入游戏第三步,爸爸把所有的玩具、用品都放在一起,妈妈对宝宝说:``宝宝,去把玩具汽车拿过来。''妈妈可协助宝宝进行第一次寻找,然后妈妈接着再说:``宝宝,去把香蕉拿过来。''游戏继续进行。

% 目的:训练宝宝能够根据父母的要求用眼睛寻找大人所用的东西,并伸手去拿父母所要的东西;训练宝宝的观察力。

% 注意:初次进行该游戏时,选用的物品应是宝宝极为熟悉的物品,随着游戏次数的增加,让幼儿认识的物品可日趋复杂。

% 以上游戏三步可同时进行,也可每次只进行游戏的一个部分。但是,必须是\textbf{在宝宝熟悉上一步的前提下才可进行}。游戏可选择一些闲余时间进行。

% %ux516bux60c5ux611fux57f9ux80b2ux8badux7ec3}{%
% \subsection{八、情感培育训练}%ux516bux60c5ux611fux57f9ux80b2ux8badux7ec3}}

% %ux53d6ux5a03ux86d9}{%
% \subsubsection{1、取娃蛙}%ux53d6ux5a03ux86d9}}

% 方法:妈妈当着婴儿的面,用纸把一个布娃娃包起来,然后交给婴儿说:``娃娃哪儿去了?宝宝,把娃娃找出来!''婴儿会翻弄纸包,把纸撕破,最终看见娃娃出现了,婴儿会非常开心。然后妈妈再用另一张纸把娃娃包好,然后又慢慢打开纸包,把娃娃拿出来,多次重复这一动作,给婴儿看,最后让婴儿学会不撕破纸,就能取出娃娃。

% 目的:进一步提高婴儿手的活动能力和理解语言的能力。

% 注意:要在婴儿情绪好时做此游戏。

% %ux63a8ux4e0dux5012ux7fc1}{%
% \subsubsection{2、推不倒翁}%ux63a8ux4e0dux5012ux7fc1}}

% 方法:取一只会响的不倒翁教婴儿推动,在学习中让他观察,体会推的重,摇的时间长,推的轻,摇的时间短。

% 目的:意识到自己的力量,认识自己与客观物体之间的关系,形成自我意识。

% 注意:可以把不倒翁经常放在婴儿身边,让其推玩。

% %ux5c0fux5c0fux6307ux6325ux5bb6}{%
% \subsubsection{3、小小``指挥家''}%ux5c0fux5c0fux6307ux6325ux5bb6}}

% \hspace{0pt}\includegraphics[width=2.46154in,height=2.99301in]{media/rId2035.png}\hspace{0pt}

% 方法:选择一首节奏鲜明、有强弱变化的音乐播放。婴儿坐在你的腿上,你从他背后握住他的前臂,说:``指挥!''然后合着音乐的节奏拍手,并随着音乐的强弱变化手臂动作幅度的大小,当乐曲停止时指挥动作同时停止,逐渐使婴儿能配合你的动作节奏。以后每当放音乐时,你一说指挥,他就能有节奏地挥动手臂。

% 目的:训练节奏感,理解动作与音乐的配合。

% 注意:时刻观察婴儿,当表现出不愿再玩时要及时停止。

% %ux4e5dux89c6ux89c9ux80fdux529bux8badux7ec3}{%
% \subsection{九、视觉能力训练}%ux4e5dux89c6ux89c9ux80fdux529bux8badux7ec3}}

% %ux6bd4ux4e00ux6bd4ux770bux4e00ux770b}{%
% \subsubsection{1、比一比,看一看}%ux6bd4ux4e00ux6bd4ux770bux4e00ux770b}}

% 方法:事先妈妈应先准备3个大小不同的杯子。这3个杯子分别为大中小3号,而且应该用3只有把、较轻的塑料杯子。游戏开始时,妈妈和宝宝一起坐在地上,在宝宝的背后垫一个枕头,防止宝宝坐不稳而摔倒。妈妈先把3个杯子按大中小的顺序依次摆开。对宝宝说:``宝宝看,这是杯子,是我们喝水用的杯子。这个最大,这个最小,这个不大也不小。今天我们用这3个杯子来做游戏,宝宝要仔细看妈妈怎么做。''妈妈将最大的杯子杯口朝下放在地板上,再将次大的杯子倒着放在最大杯子的上面,最后把小杯子倒放在顶端。让宝宝看清楚后,妈妈把杯子推倒,重新再堆一次。让宝宝用手推倒,激励起宝宝参与的积极性,鼓励宝宝模仿妈妈的动作进行游戏。

% 目的:训练宝宝的注视观察能力,鼓励宝宝进行模仿;使宝宝明白同样的物体可以有不同的表现方法。

% 注意:妈妈也可以将杯子按次序套在一起,让宝宝寻找套在大杯子中的中号杯和小号杯。

% %ux56deux5f52ux4f4d}{%
% \subsubsection{2. 回归位}%ux56deux5f52ux4f4d}}

% 方法:游戏前,妈妈先准备好一堆塑料球,一部分是红色,一部分是黄色。再准备两个纸盒子,一个是红色,一个是黄色。游戏让爸爸、妈妈和宝宝共同参与。爸爸、妈妈、宝宝都坐在地板上,爸爸独自一个坐在一边,妈妈搂着宝宝坐在相对的另一边,爸爸、妈妈的距离控制在2米左右。在爸爸的前面放上黄色的盒子,在妈妈和宝宝的前面放上红色的盒子,在中间放上红色的和黄色的小球。由爸爸先做示范,爸爸走到中间,一手拿起一个彩球,然后把手中红色的小球放在红色的盒子里,把黄色的小球放进黄色的盒子里,如此重复几次。然后在妈妈的指导下让宝宝来做这个动作。最初宝宝可能不熟练,但多做几次就可以让宝宝独自去进行。

% 目的:训练宝宝的观察能力,训练宝宝通过观察模仿的能力,训练宝宝手与上肢的运动能力,增强宝宝视觉、运动协调能力。

% 注意:红色的盒子、黄色的盒子都要大而且口一定要大,便于宝宝把塑料球投进去,爸爸的示范动作一定要仔细、缓慢,让宝宝看得清清楚楚,如果宝宝动作发展较好,可让宝宝坐在爸爸、妈妈的中间,移近盒子的距离,让宝宝把相同颜色的彩球扔进相应颜色的盒子里。但游戏不能太长,否则会造成宝宝疲劳。

% %ux5341ux542cux89c9ux80fdux529bux8badux7ec3}{%
% \subsection{十、听觉能力训练}%ux5341ux542cux89c9ux80fdux529bux8badux7ec3}}

% %ux4e01ux5f53ux54eaux513fux53bbux4e86}{%
% \subsubsection{1.
% 丁当哪儿去了}%ux4e01ux5f53ux54eaux513fux53bbux4e86}}

% 方法:游戏前,妈妈先准备摇铃一个,干净的手绢一条。游戏时,妈妈让宝宝坐在床上,爸爸在幼儿身后扶着。妈妈手拿摇铃在宝宝眼前有节奏地摇着,用来吸引宝宝的注意力。当孩子注视摇铃的时候,妈妈突然用手绢将摇铃的一大部分盖住,并且微笑着对宝宝说:``咦,奇怪呀!宝宝的摇铃哪儿去了?宝宝找一找摇铃在哪里?''宝宝的眼睛就会去搜寻刚才还存在的摇铃。很快,宝宝的眼睛就会注视着盖着手絹的摇铃,这时,家长不要急于去拿掉摇铃的手绢,而是让宝宝稍微多注视一段时间,提高宝宝的注意时间。随后,妈妈把手絹拿开,说:``啊!原来宝宝的摇铃在这里。宝宝找到了,宝宝真聪明。''游戏继续进行。妈妈再次摇铃,再次遮盖......

% 目的:发展宝宝的观察力,使宝宝通过观察寻找当面藏起来的东西,发展宝宝的听觉,培养宝宝的辨音能力及节奏感,训练其注意力;引发宝宝用眼睛去找寻自己喜爱的东西,并伸手去拿,发展宝宝的记忆力。

% 注意:妈妈摇铃应有节奏,向左边两下,右边两下。而且摇的时候,最好也有\textbf{节奏}地摇摇头,\textbf{方向}和节奏与摇铃相同。这样,极易引起宝宝的共鸣,吸引幼儿的兴趣。游戏进行中,让宝宝寻找被手絹遮住的摇铃,宝宝在注视被手绢遮住的摇铃时,妈妈一定要掌握好宝宝注视的时间。时间太长宝宝注意力就会转移,时间太短达不到训练的目的;游戏进行的时间长短由妈妈根据宝宝的厌倦程度来决定。

% %ux73a9ux8fc7ux5bb6ux5bb6}{%
% \subsubsection{2 玩过家家}%ux73a9ux8fc7ux5bb6ux5bb6}}

% 方法:妈妈事先准备一个玩具娃娃,这个玩具娃娃要比较精细,即玩具娃娃的头发可梳可扎,眼睛要会动,玩具娃娃的衣服可以脱下、穿上,玩具娃娃有袜子、鞋子等,使得游戏时使用方便。再准备一套玩具餐具。游戏开始时,爸爸、妈妈一边说话一边玩过家家,让宝宝在旁边看着。爸爸、妈妈很精细、很缓慢地做每一个动作,比如说给娃娃穿衣服、系扣子,给娃娃穿袜子、穿鞋子,妈妈给娃娃扎头发。然后用玩具餐具给娃娃喂饭。喂完饭,妈妈对宝宝说:``宝宝,爸爸、妈妈给娃娃喂完了饭,现在娃娃要出去玩了,请宝宝给娃娃换身衣服,我们带娃娃出去玩。''于是,爸爸把娃娃的衣服脱掉,拿出一身衣服给宝宝,宝宝就会根据自己的观察将爸爸、妈妈的动作重复再做一遍。

% 目的:培养宝宝的听力、注意力、观察力、动手能力。

% 注意:该游戏每次进行时,可只做一部分动作,比如只让宝宝观察爸爸、妈妈如何给娃娃穿衣服,还可以继续其他的动作;具体内容的多少要根据宝宝的发展情况来决定。爸爸、妈妈在做示范动作给宝宝看时,一定要慢,动作清楚,便于宝宝观察和模仿。

% %ux5341ux4e00ux611fux89c9ux80fdux529bux8badux7ec3}{%
% \subsection{十一、感觉能力训练}%ux5341ux4e00ux611fux89c9ux80fdux529bux8badux7ec3}}

% %ux73a9ux73a9ux5177}{%
% \subsubsection{1 玩玩具}%ux73a9ux73a9ux5177}}

% \hspace{0pt}\includegraphics[width=2.71329in,height=2.30769in]{media/rId2046.png}\hspace{0pt}

% 方法:妈妈要事先准备一些玩具,一类是电动玩具,如电动小汽车、电动小火车、电动飞机等;另一类是上发条的玩具,如玩具鸭子、玩具兔子、玩具青蛙......游戏开始时,妈妈不要把所有的玩具都堆在宝宝面前,而是一个个拿出来让宝宝看。比如妈妈先拿出一个鸭子的发条玩具,上足发条,把鸭子放到地上,鸭子有节奏地向前走去,妈妈则以丰富的语言配合玩具的特点来刺激宝宝:``嘎,嘎,唐老鸭,嘎,嘎,唐老鸭......''或是``呜一呜,小火车开走了......''

% 目的:训练宝宝的感知觉;培养宝宝的观察力、注意力,使宝宝延长注意的时间;发展宝宝的语言能力,让宝宝学会理解简单的因果关系的语言;发展宝宝的专注力。

% 注意:妈妈应一个一个出示玩具,否则玩具太多,容易分散宝宝的注意力;妈妈的语言要生动,前后的语言因果关系要明确,便于宝宝理解。

% %ux5947ux5999ux7684ux7535ux89c6}{%
% \subsubsection{2 奇妙的电视}%ux5947ux5999ux7684ux7535ux89c6}}

% \hspace{0pt}\includegraphics[width=3.07692in,height=2.57343in]{media/rId2050.png}\hspace{0pt}

% 方法:妈妈把宝宝抱到电视前,对宝宝说:``宝宝,今天妈妈让你看一个很好玩的东西。这是我们家的电视机。''揭开电视机罩子,让宝宝看到整个的电视机。妈妈说:``我们来打开电视机,看看电视机里都有些什么?''妈妈打开电视机开关,出现丰富多彩的电视画面和悦耳的声音,会引起宝宝极大的兴趣。妈妈把宝宝抱到距离电视约2米远的地方,让宝宝看上4〜5分钟电视,看的同时,妈妈可用简单的语言对宝宝解释电视画面内容。关掉电视以后,妈妈可对宝宝说一些有关电视的话,诸如:``电视可好看了,有宝宝喜欢的大汽车、大飞机、小猴子......宝宝以后可以经常看电视......''。\\
% 有选择性地让宝宝看一些电视节目,比如《七巧板》、《动画城》、《动物世界》等等。宝宝也许对这些内容不理解,但是丰富的色彩、活泼的形象却极易吸引宝宝的注意。有的宝宝则很容易表现出极强的专注力。

% 目的:让宝宝初步接触现代媒体;发展宝宝的感知能力,刺激宝宝的视听觉;培养宝宝的注意力,加强宝宝注意时间长短的培养,培养宝宝一定的专注力,使宝宝对图像、声音感兴趣。

% 注意:在初次让宝宝全面地了解电视前,应让宝宝有过看电视的经验,使宝宝不至于因电视打开突然出现的画面和声音而受到惊吓,要让宝宝养成良好的看电视的习惯。从内容上,要选择一些形式新颖的宝宝节目,不能让宝宝看战斗、恐怖电视,从时间上看,宝宝每次看电视时间不应超过10分钟,而且,每天在固定的时间内让宝宝看电视,另外,距离电视应在2米以外,以保护宝宝视力。

% %ux5341ux4e8cux52a8ux4f5cux80fdux529bux8badux7ec3}{%
% \subsubsection{十二、动作能力训练}%ux5341ux4e8cux52a8ux4f5cux80fdux529bux8badux7ec3}}

% %ux6edaux7b52}{%
% \subsubsection{1. 滚筒}%ux6edaux7b52}}

% 方法:将圆柱体的滚筒(饮料瓶代替也可)放在地上,让宝宝用两只手推动它向前滚动,待他熟练后,再让他用一只手推动滚筒,并把它滚到指定地点。

% 目的:训练手指能力,并在戏耍中逐渐建立起圆柱体物体能滚动的概念。

% 注意:做对了,给予鼓励。

% %ux5f15ux5bfcux5a74ux513fux5b66ux8d70ux8def}{%
% \subsubsection{2.
% 引导婴儿学走路}%ux5f15ux5bfcux5a74ux513fux5b66ux8d70ux8def}}

% 方法:要给婴儿创造条件,使他早日独立开步走,可以给他穿上布底鞋,衣着轻暖;再给他一辆小推车,让他在平整但不光滑的地面上推着向前学步而行,感知周围的世界。有时他会把小椅子放倒,推着向前走。一旦他发现这个新玩意儿可以推着走路,就会高兴地一刻不停地在屋里推来推去。你还可以给他特制一个巨型积木-个结实的大纸盒(30厘米X
% 30厘米X
% 40厘米左右),贴上有趣易懂的彩色图画,既可以围着爬、扶着站、推着走,又能学图画中的内容。

% 促使婴儿早日开步走的方法很多,如扶着婴儿腋下走;用长的浴巾从婴儿的胸前穿过两侧腋下,在后面轻轻地拽着让他向前行走,待他能站稳后,使他在妈妈和爸爸之间跨出1\textasciitilde2步,逐渐增大迈步的距离等等。

% \hspace{0pt}\includegraphics[width=2.51748in,height=3.07692in]{media/rId2056.png}\hspace{0pt}

% 要注意方法,每次练习时间不宜长,但练习次数可逐渐增加。要循序渐进,从轻扶双手、扶单手到独站,最后独自行走几步。

% 目的:学会站立和开步走是婴儿身心发展中的一大进步,动作的发展使婴儿大开眼界,增长见识,促进了心理发育。

% 注意:防止意外是最重要的,学步的周围应当没有泮手泮脚的物件,没有桌椅的尖角,没有电、火、热水,柜子抽屉里没危险品等。如果婴儿不小心跌倒在地,也不要一副担心的样子,马上就去扶、抱,这时,你应当用亲切的语言鼓励他自己爬起来继续练习,这是最初的意志磨炼。

% %ux5a74ux5e7cux513fux5065ux8eabux64cd-3}{%
% \subsubsection{3.
% 婴幼儿健身操}%ux5a74ux5e7cux513fux5065ux8eabux64cd-3}}

% 宝宝俯卧,两肘支撑身体,家长在孩子身后用双手握住宝宝的两只小腿,将双腿提起约30度。此时应让宝宝双手用力支撑,并向上抬头。可锻炼臂力和颈肌。

% %ux5a74ux5e7cux513fux4e3bux88abux52a8ux64cd}{%
% \subsubsection{4.
% 婴幼儿主被动操}%ux5a74ux5e7cux513fux4e3bux88abux52a8ux64cd}}

% 婴儿主被动操是在成人的适当扶持下,加入婴儿的部分主动动作完成的。婴儿主被动操的动作主要有锻炼四肢肌肉关节的上、下肢运动,锻炼腹肌、腰肌以及脊柱的桥形运动、拾物运动,为站立和行走作准备的立起、扶腋步行、双脚跳跃等动作。用于7\textasciitilde12个月的婴儿。这个时期的婴儿,已经有了初步的自主活动的能力,能自由转动头部,自己翻身,独坐片刻,双下肢已能负重,并上下跳动。婴儿每天进行主被动操的训练,可活动全身的肌肉关节,为爬行、站立和行走打下基础。

% %ux7b2cux5341ux4e00ux82821011ux4e2aux6708ux5a74ux513fux7684ux65e9ux671fux57f9ux80b2}{%
% \subsection{11第十一节10〜11个月婴儿的早期培育}%ux7b2cux5341ux4e00ux82821011ux4e2aux6708ux5a74ux513fux7684ux65e9ux671fux57f9ux80b2}}

% %ux4e0010ux4e2aux6708ux5a74ux513fux7684ux667aux80fdux53d1ux80b2ux6c34ux5e73}{%
% \subsection{一、10个月婴儿的智能发育水平}%ux4e0010ux4e2aux6708ux5a74ux513fux7684ux667aux80fdux53d1ux80b2ux6c34ux5e73}}

% 这个时期又是婴幼儿智能发育的\textbf{关键}年龄,在行为模式的发展方面将出现一些\textbf{飞跃性变化},也是积极开发和培养小儿智能的重要时机,其主要的行为模式表现如下:

% 大运动:小儿能自己拉住栏杆站起,身体完全直立;能扶住栏杆走3步以上,一边走,一边移手。

% 精细动作:能用拇指和食指的指端捏住小丸,动作熟练。

% 适应能力:能主动拿掉杯子取出藏在下面的方木玩具;能明确地寻找盒内的木珠。

% 语言:会模仿大人发1-2个字音,如``爸爸''、``妈妈''、``拿''、``走''等。

% 社交行为:懂得常见人及物的名称,会用眼注视所说的人或物。

% %ux4e8cux5a74ux513fux7684ux4e2aux6027ux5f00ux59cbux53d1ux5c55}{%
% \subsection{二、婴儿的个性开始发展}%ux4e8cux5a74ux513fux7684ux4e2aux6027ux5f00ux59cbux53d1ux5c55}}

% 10个月的婴儿已显出个体特征的某些倾向性,一些婴儿表现得活泼,而有的稳重,有的灵活,有的呆板。例如,有的婴儿不让别人抢走他手中的玩具或吃的东西,显得很``自私'';有的婴儿见别人有什么玩具就想要什么玩具;有的非常大方,把自己的东西送给别人,与别人一起分享;有的婴儿整天不声不响,任人摆布;而有的则不让别人碰一下,遇到生人就显示出戒备并啼哭。对成人的逗引,不同的婴儿则表现出不同的反应,如有的报以友好的微笑;有的则绷着脸,不理不睬;有的见人就打,还大喊大叫,以打人为乐。以上这些都说明了在婴儿期末婴儿就显示出了个性倾向,但这也不是固定不变的。成人对婴儿的表现要区别对待,好的行为要加以表扬,如点头微笑,拍手叫好等;而不好的行为则要表示出不满意,并说``这样做不好''、``这样做不是好婴儿''等。10个月至1岁的婴儿已经具有较强的模仿能力,为了使婴儿形成良好的个性,成人的榜样作用是十分重要的。良好的榜样、和睦的家庭气氛是形成婴儿良好个性的重要基础。

% %ux4e09ux751fux6d3bux81eaux7406ux80fdux529bux8badux7ec3}{%
% \subsection{三、生活自理能力训练}%ux4e09ux751fux6d3bux81eaux7406ux80fdux529bux8badux7ec3}}

% 捧杯喝水:鼓励宝宝自己捧杯喝水,由洒漏渐渐熟练到不洒漏,大人应放手让宝宝做。

% 穿脱衣服:穿脱衣服时教小儿配合,如穿上衣时知道把胳臂伸入袖内。

% %ux56dbux5988ux5988ux6559ux513fux6b4c}{%
% \subsection{四、妈妈教儿歌}%ux56dbux5988ux5988ux6559ux513fux6b4c}}

% 小八哥

% 小八哥,真好看,爱说话,嘴儿甜,看你来了问``您好!''看你走了说``再见!''小八哥,语言美,人人见了都喜欢。

% 穿袜子

% 比一比,量一量,两只袜子一个样,哪只穿左脚?哪只穿右脚?婴儿不会穿,不知怎么好。

% 数数歌

% 一二三,爬上山,四五六,翻跟头,七八九,拍皮球,伸出两只手,十个手指头。

% %ux4e94ux8bedux8a00ux80fdux529bux8badux7ec3}{%
% \subsection{五、语言能力训练}%ux4e94ux8bedux8a00ux80fdux529bux8badux7ec3}}

% %ux6559ux8bfbux5b57ux8bcd}{%
% \subsubsection{1 教读字词}%ux6559ux8bfbux5b57ux8bcd}}

% 方法:星期一至星期六每天教婴儿说一个字(词),如果有条件用普通话和英语交替着教(也可用其他外国语)。一天之中多次教读。星期天把6个字(词)复习几遍。例如:1妈妈
% mother 2爸爸 father 3脸 face 4眼睛 eye 5鼻子 nose 6嘴巴 mouth;

% 目的:对婴儿进行语言启蒙教育。

% 注意:教读时尽量结合实物或动作等。

% 现在只需每天教婴儿读,让婴儿熟悉这些字(词),不强求婴儿能念、能认。

% %ux8bb2ux6545ux4e8bux4e00ux5219}{%
% \subsubsection{讲故事一则}%ux8bb2ux6545ux4e8bux4e00ux5219}}

% 方法:妈妈每天给婴儿讲一个故事,故事内容要短小、精焊,可\textbf{重复讲一周},例如:母鸡妈妈孵了一群小鸡。每天,小鸡娃儿们跟在鸡妈妈身后找虫吃,跟妈妈一起玩游戏。母鸡妈妈找到一条大虫子,它叫着它的婴儿们:``咯咯咯,快来呀,这儿有一条大虫子。''小鸡娃儿们跑过来,一齐说:``吼吼叽,妈妈先吃,我们再吃。''

% 目的:训练婴儿语言能力。

% 注意:用普通话为婴儿讲故事。

% %ux516dux793eux4ea4ux80fdux529bux8badux7ec3}{%
% \subsection{六、社交能力训练}%ux516dux793eux4ea4ux80fdux529bux8badux7ec3}}

% %ux6ed1ux7a3dux53d8ux8138ux672f}{%
% \subsubsection{1 滑稽变脸术}%ux6ed1ux7a3dux53d8ux8138ux672f}}

% 方法:

% 1 找一面稍大的镜子,梳妆镜、橱柜镜或立于桌上的镜子都可以;

% 2 抱着宝宝坐在镜子前;

% 3 取下约30厘米长的透明胶带;

% 4
% 对着镜子扮个鬼脸,然后用胶带把你的这个表情粘住,胶带可以使你的嘴巴扭曲、眉毛上扬、鼻子变平、眼皮下垂;

% 5 说些有趣的事来配合你的表情;

% 6 教会宝宝撕下你脸上的胶带;

% 7 再扮个鬼脸,用胶带把这个表情留住;

% 8 撕下胶带后,和宝宝一起欢笑。

% 也可以在宝宝的脸上或手臂上等处粘胶带,再帮宝宝将胶带撕掉。

% 目的:培养宝宝的交流和沟通能力。

% 注意:你的鬼脸要让宝宝感到有趣,别吓到他;不要让宝宝吞下胶带,如果把胶带粘在宝宝脸上,不要贴住他的眼睛、鼻子和嘴巴,撕下胶带的时候动作要轻柔。

% %ux5bfbux627eux5c0fux7403}{%
% \subsubsection{2 寻找小球}%ux5bfbux627eux5c0fux7403}}

% 方法:用一个边长30厘米左右(正方形、长方形均可)的包装纸箱,上面开一个大约10厘米X
% 10厘米的洞。在右下角另剪一个边长为5厘米的等边三角形出口,让宝宝从大洞投入一个小球,叫他摇动纸箱使小球从边角出口处漏出。告诉宝宝从大洞里看看,哪一头亮就向哪边摇;宝宝起初会乱摇,后来他学会不必摇,让箱子斜着放,小球自然会滚出来。

% 目的:让婴儿学会解决问题的办法。

% 注意:洞的边缘要整齐\textbf{平滑},防止刮伤婴儿。

% %ux4e03ux60c5ux611fux57f9ux80b2ux8badux7ec3}{%
% \subsection{七、情感培育训练}%ux4e03ux60c5ux611fux57f9ux80b2ux8badux7ec3}}

% %ux8bfbux5e03ux4e66}{%
% \subsubsection{1、读``布书''}%ux8bfbux5e03ux4e66}}

% 方法:你可精心地挑选一些上面印有各种各样图案的手帕,或买一些可贴在童装上的各种小动物布贴,或用彩色碎布做成动物、日常用品、房屋、风景、交通工具等几种图案,把\textbf{尼龙粘扣}粘在手帕上,再把这些手帕粘在一起,一本``布书''就制成了。这本书里应当内容丰富,又富有情趣,可以有娃娃、水果、小动物、动画故事等,最好结合实物进行学习。

% 目的:培养对图书的兴趣,提高认识能力。

% 注意:可以将``布书''放在洗衣机中洗涤甩干,防止细菌滋生。

% %ux9012ux4e1cux897fux7ed9ux522bux4eba}{%
% \subsubsection{2、递东西给别人}%ux9012ux4e1cux897fux7ed9ux522bux4eba}}

% 方法:让宝宝从盘子内拿一个橘子给爸爸,拿一个给妈妈,自己再拿一个。有时宝宝舍不得把第一个分给别人,可以把次序倒过来,先自己拿一个,然后再分给别人。有过多次练习后,可以递一个给爷爷,再递一个给奶奶,最后让宝宝递东西给客人;经常让宝宝给客人递食物就会养成与人分享东西的好习惯。

% 目的:练习递东西给别人,一来学会与别人分享,养成不自私的习惯;二来学会给人递东西是当助手的基本功,以后大人做事时能与大人配合,学会当助手。

% 注意:所递的东西一定要安全,对宝宝没有危险。

% %ux516bux6570ux5b66ux80fdux529bux8badux7ec3}{%
% \subsection{八、数学能力训练}%ux516bux6570ux5b66ux80fdux529bux8badux7ec3}}

% %ux73a9ux5957ux73af}{%
% \subsubsection{1、玩套环}%ux73a9ux5957ux73af}}

% 方法:把一支铅笔插进一块橡皮泥或一个硬纸盒里用透明胶带固定,做成一个套环用的``柱子''。用铁丝拧3个直径为10厘米的环,每个环用不同颜色的布缠好,再用针线固定一圈。给宝宝示范将环套在``柱子''上,边套边数``1个、2个、3个'',套完后再一个个数着取出来,让婴儿学着自己动手。【笔太危险,用圆柱体的,安全的吧】

% 目的:训练手眼协调能力,数学启蒙。

% 注意:用针线固定的圈要圆。

% %ux533aux522b123}{%
% \subsubsection{2、区别1、2、3}%ux533aux522b123}}

% 方法:在婴儿的注视下,用一张16开的纸包上1块糖果,打开,再包上,引导他打开纸把糖果找出来,当他打开后,你就说``1块'',并把糖果给他作为奖励。当着婴儿的面另取4只一样的糖果,边说``这是1块,这是3块'',边用两张纸分别包上1块和3块,再打开让他注视两边的糖果各5秒钟后包上(两包的位置不要变),要求他把两包糖果都打开,看他要哪一包。玩过几次后,如果他总是要3个的一包,说明他能区别``1''和``3''。然后,你再包上2块和3块,看他是否还要3块,如果是,说明能区别``2''与``3''。

% 目的:提高注意力、记忆力和手的技巧,诱发简单数概念的萌芽。

% 注意:不要每打开一个包都把糖果给婴儿吃,那样会对婴儿的牙齿不利。

% %ux4e5dux6280ux80fdux6c34ux5e73ux8badux7ec3}{%
% \subsection{九、技能水平训练}%ux4e5dux6280ux80fdux6c34ux5e73ux8badux7ec3}}

% %ux62cdux624bux62cdux624b}{%
% \subsubsection{1. 拍手拍手}%ux62cdux624bux62cdux624b}}

% 方法:

% \begin{enumerate}
% \def\labelenumi{\arabic{enumi}.}
% \item
%   妈妈和宝宝面对面坐着,妈妈说:``请宝宝仔细看,请宝宝仔细听,看妈妈做什么,听妈妈做什么。''
% \item
%   妈妈拍手并且有节奏、有规律,节奏应为:X、X、XXX,X、X、XXX;
% \item
%   让宝宝仔细看认真听,反复进行几次宝宝就会掌握其中的规律,和妈妈一起拍手;
% \item
%   游戏进行一段时间后,妈妈可把拍手的节奏变得更为复杂一些。
% \end{enumerate}

% 目的:训练宝宝的注意力、观察力,使宝宝能够较长时间地集中注意力,训练宝宝的模仿能力,培养宝宝的动手能力。

% 注意:家长应对宝宝的错误及时纠正。

% %ux56dbux65b9ux7684ux548cux5706ux7684}{%
% \subsubsection{2.
% 四方的和圆的}%ux56dbux65b9ux7684ux548cux5706ux7684}}

% 方法:

% \begin{enumerate}
% \def\labelenumi{\arabic{enumi}.}
% \item
%   爸爸给宝宝两块积木(两块方形的积木),一个塑料球(积木要比塑料球小一些),教幼儿把一块积木搭在另一块上,再试着把塑料球搭在第二块积木上,宝宝尝试几次,但塑料球总是掉下来,滚到一边去了,这时,爸爸再给宝宝一块方积木,让宝宝搭上去,这次没有掉下,宝宝成功了,
% \item
%   爸爸给宝宝一根小棒和一个小皮球,看看宝宝是否知道用小棒推着皮球滚动,然后拿走皮球,给宝宝换来另一样东西(比如一个罐头盒、一个易拉罐),看宝宝是否会用小棒推着罐头盒或易拉罐滚动。
% \end{enumerate}

% \hspace{0pt}\includegraphics[width=2.71329in,height=2.25175in]{media/rId2087.png}\hspace{0pt}

% 目的:训练宝宝的观察力,让宝宝能够通过观察明白圆的东西可以滚动,训练宝宝小肌肉的运动,训练宝宝手指的灵活性,让宝宝逐渐理解物体与物体特性之间的关系。

% 注意:家长不要急于教宝宝玩,而是要观察宝宝、启发宝宝自己去做。游戏结束,可由家长对游戏进行总结,以加深宝宝的理解。

% %ux5341ux542cux89c9ux80fdux529bux8badux7ec3-1}{%
% \subsection{十、听觉能力训练}%ux5341ux542cux89c9ux80fdux529bux8badux7ec3-1}}

% %ux542cux97f3ux4e50ux53d6ux653eux7269}{%
% \subsubsection{1
% 听音乐取放物}%ux542cux97f3ux4e50ux53d6ux653eux7269}}

% 方法:播放一段活泼、欢快的音乐,将积木和盒子放于孩子面前,让孩子随音乐将积木从盒子中一一取出,再一一放入盒中。开始孩子的动作可能比较笨拙、缓慢,听到比较欢快、活泼的音乐,会慢慢灵活一点。

% 目的:训练婴儿对音乐的感知能力。

% 注意:可就地取材,父母敲击某物,让孩子听声音取放物。

% %ux533aux5206ux566aux58f0ux4e0eux97f3ux4e50}{%
% \subsubsection{2
% 区分噪声与音乐}%ux533aux5206ux566aux58f0ux4e0eux97f3ux4e50}}

% 方法:父母拿积木敲桌子,示意孩子``这是不好听的声音'',并对着孩子皱皱眉头,父母轻弹木琴,让孩子倾听,告诉孩子``这是好听的声音'',并对着孩子笑笑。播放(或用实物玩具弄出)一种轰隆隆的声音,再告诉孩子``这是不好听的声音'',并皱眉,播放一小段音乐,告诉孩子``这是好听的音乐'',并对孩子笑笑。然后,弄响发出噪声与音乐的物体或放录音,让孩子倾听,父母用皱眉或微笑给孩子以暗示。

% 目的:在游戏中分辨噪声与音乐,锻炼听力。

% 注意:用木琴弹出的音乐要柔美、悦耳。

% %ux5341ux4e00ux611fux89c9ux80fdux529bux8badux7ec3-1}{%
% \subsection{十一、感觉能力训练}%ux5341ux4e00ux611fux89c9ux80fdux529bux8badux7ec3-1}}

% %ux73a9ux6c34}{%
% \subsubsection{1 玩水}%ux73a9ux6c34}}

% 方法:准备一盆温水,把一些塑料小碗、小瓶、大盒盖或一块海绵、鹅卵石、吹塑小动物等放在盆里,教宝宝将水倒来倒去,把漂在水上的玩具推来推去地玩。

% 目的:锻炼小肌肉动作和眼手协调能力。

% 注意:春秋天可在洗澡前卷起宝宝的袖子玩,夏天可在户外阴凉处让孩子尽情地在水盆中玩。

% %ux6253ux5f00ux5957ux676fux76d6}{%
% \subsubsection{2 打开套杯盖}%ux6253ux5f00ux5957ux676fux76d6}}

% 方法:拿一只带盖的塑料茶杯放在孩子面前,向他示范打开盖,再合上盖的动作,然后让孩子练习只用大拇指与食指将杯盖掀起,再盖上,反复练习。用塑料套杯或套碗,让宝宝模仿大人一个一个套。

% 目的:促进宝宝空间知觉的发展。

% 注意:婴儿做对了就要称赞他。

% %ux7b2cux5341ux4e8cux828211-12ux4e2aux6708ux5a74ux513fux7684ux65e9ux671fux57f9ux80b2}{%
% \subsection{12第十二节11 \textasciitilde{}
% 12个月婴儿的早期培育}%ux7b2cux5341ux4e8cux828211-12ux4e2aux6708ux5a74ux513fux7684ux65e9ux671fux57f9ux80b2}}

% %ux4e0011ux4e2aux6708ux5a74ux513fux7684ux667aux80fdux53d1ux80b2ux6c34ux5e73}{%
% \subsection{一、11个月婴儿的智能发育水平}%ux4e0011ux4e2aux6708ux5a74ux513fux7684ux667aux80fdux53d1ux80b2ux6c34ux5e73}}

% 大运动:能一手扶栏蹲下,用另一只手捡取玩具,并能再站起来;能独自站立2秒钟以上。

% 精细动作:能有意识地打开包方木的纸,寻找方木。将方木拿到手里。

% 适应能力:能有意识地将方木放入杯中,能模仿在桌面上推动玩具小车。

% 语言:能有意识地并正确地发出相应的字音,以表示一个动作(如"拿"),一个人(如"姨")或一件物(如"鸡")。

% 社交行为:家长说"不动"或"不拿"后,会停止拿取玩具的动作。

% %ux4e8cux638cux63e1ux5a74ux5e7cux513fux667aux529bux53d1ux80b2ux7684ux5173ux952eux671f}{%
% \subsection{二、掌握婴幼儿智力发育的关键期}%ux4e8cux638cux63e1ux5a74ux5e7cux513fux667aux529bux53d1ux80b2ux7684ux5173ux952eux671f}}

% 人的一生是由婴儿、少年、青年、中年、老年一个阶段一个阶段的量变到质变而形成的动态体系,就智力发育而言,从智力发育规律可以知道,婴儿期、少年期、青年期的智力发展最快,变化最大,而且在一定程度上制约着一个人一生的智力发展水平,可以说,它们是人生智力发育的关键期。国内外的研究表明:2-5岁是教孩子怎样做到守规矩的关键期,4岁以前,是形象视觉发展的关键期;3-5岁是发展音乐能力的关键期;4-5岁是开始学习书面语言的关键期,5岁左右是数的概念形成的关键期,5-6岁是掌握汉语词汇能力的关键期;3-8岁是学习外国语的关键期,3岁左右和青少年时期是培养独立性的关键期,6-16岁是身体锻炼最有成效的时期,其中6\textasciitilde7岁是速度、灵敏度发育的关键期,等等。

% %ux4e09ux751fux6d3bux81eaux7406ux80fdux529bux7684ux8badux7ec3}{%
% \subsection{三、生活自理能力的训练}%ux4e09ux751fux6d3bux81eaux7406ux80fdux529bux7684ux8badux7ec3}}

% 培养良好的进餐习惯:从小给小儿一个\textbf{固定的坐位},让他养成安静坐着吃饭的好习惯。

% 学习用勺子:用一个玩具勺子在玩具碗内学习盛起小球、枣、药丸等。有了这种练习,孩子渐渐懂得用勺子的凹面将枣或小球盛入,放到另一个小碗内。母亲表扬孩子"真能干",为以后孩子自己吃饭打好基础。

% %ux56dbux5988ux5988ux6559ux513fux6b4c-1}{%
% \subsection{四、妈妈教儿歌}%ux56dbux5988ux5988ux6559ux513fux6b4c-1}}

% 小白兔

% 一只白兔长得美,两只耳朵三瓣嘴。前腿短,后腿长,蹦蹦跳跳四条腿。

% 五指歌

% 一二三四五。上山打老虎。老虎打不到,打到小松鼠。松鼠有几只?让我数一数。数来又数去,一二三四五。

% %ux4e94ux8bedux8a00ux80fdux529bux8badux7ec3-1}{%
% \subsection{五、语言能力训练}%ux4e94ux8bedux8a00ux80fdux529bux8badux7ec3-1}}

% %ux7528ux4e00ux4e2aux97f3ux8868ux793aux8981ux6c42}{%
% \subsubsection{1.
% 用一个音表示要求}%ux7528ux4e00ux4e2aux97f3ux8868ux793aux8981ux6c42}}

% 方法:宝宝经常是用一音表示他的各种意思和要求。如"妈妈走"的"走"可以代表妈妈、妈妈走啦、去上街、自己走等意思,要鼓励孩子说出来,并做好翻译员;还要诱导孩子联想、比较,比如:宝宝说"球"时,你可把各种颜色大小的球一个一个拿出来,告诉孩子这是"红球",那是"绿球"等,或这是"大球",那是"小球"等。

% 目的:训练婴儿语言能力。

% 注意:婴儿不高兴时,不要强迫其进行训练。

% %ux5b66ux62bcux97f5}{%
% \subsubsection{2. 学``押韵''}%ux5b66ux62bcux97f5}}

% 方法:选一首你最常教婴儿念的儿歌,而且每句最后一个押韵的词要容易发音,如``小娃娃,甜嘴巴,喊妈妈,喊爸爸,喊得奶奶笑掉牙......''念时,故意加重每句最后一个字的语气,并将前面的字拉长,念成``小娃一娃'',以强调最后那个押韵的字。你紧接着说:``宝宝,说`娃'!''然后你再念一遍``小娃一''故意不说出``娃''字,等着他说出。这样反复进行,使他逐渐能跟着你把最后一个押韵的词都说出来。关于``关键期''的研究,现在处在探索阶段,上面列举的也不是绝对的,应该说,处于不同条件下的孩子以及处在相同条件下的不同儿童是存在着差别的。

% 目的:提高语言表达能力。

% 注意:家长发音要准确、到位。

% %ux8bb2ux6545ux4e8bux4e00ux5219-1}{%
% \subsubsection{3. 讲故事一则}%ux8bb2ux6545ux4e8bux4e00ux5219-1}}

% 小松鼠断奶

% 高高的山上有一棵松树,松树上的树洞是松鼠的家。松鼠妈妈生了一只胖嘟嘟的小松鼠,已经快满周岁了,该断奶了,吃松果果了。

% 小松鼠要在晚上吃奶,吃奶时闭着眼睛。一天晚上,妈妈给小松鼠吃葡萄,一边问小松鼠:``奶好吃吗?''\,``真好吃!又甜又酸。''妈妈又说:``你长得这么大了,该断奶了。''
% ``不行不行!我就爱喝又甜又酸的奶。''妈妈乐呵呵地说:``好吧!那就天天晚上给你吃葡萄。''小松鼠睁大眼睛一看,哇,原来妈妈喂它的是一串晶莹的葡萄,它吱吱吱地笑着说:``妈妈!妈妈!我也觉得好奇呀,怎么从妈妈的奶里吃出核儿来了?''

% 教读英语字词

% \begin{enumerate}
% \def\labelenumi{\arabic{enumi}.}
% \item
%   我 - I
% \item
%   你 - you
% \item
%   他 - he
% \item
%   花 - flower
% \item
%   衣服 - clothes
% \item
%   鞋 - shoes
% \item
%   鸡蛋 - egg
% \item
%   香蕉 - banana
% \item
%   裤子 - trousers
% \item
%   脚 - foot
% \item
%   猫 - cat
% \item
%   球 - ball
% \item
%   爷爷 - grandfather
% \item
%   奶奶 - grandmother
% \item
%   哥哥 - brother
% \item
%   姐姐 - sister
% \item
%   叔叔 - uncle
% \item
%   阿姨 - auntie
% \item
%   梨子 - pear
% \item
%   公共汽车 - bus
% \item
%   猴子 - monkey
% \item
%   猪 - pig
% \item
%   帽子 - cap
% \item
%   门 - door
% \end{enumerate}

% %ux516dux793eux4ea4ux80fdux529bux8badux7ec3-1}{%
% \subsection{六、社交能力训练}%ux516dux793eux4ea4ux80fdux529bux8badux7ec3-1}}

% %ux968fux58f0ux821eux52a8-1}{%
% \subsubsection{1 随声舞动}%ux968fux58f0ux821eux52a8-1}}

% 方法:经常给宝宝听节奏明快的婴儿音乐或给他念押韵的儿歌,让他随声点头、拍手;也可用手扶着他的两只胳臂,左右摇身,多次重复后,他能随音乐的节奏做简单的动作。

% 目的:训练音乐与动作的协调能力。

% 注意:音乐速度宜慢不宜快。

% %ux5e73ux884cux6e38ux620f}{%
% \subsubsection{2 平行游戏}%ux5e73ux884cux6e38ux620f}}

% \hspace{0pt}\includegraphics[width=2.62937in,height=3.18881in]{media/rId2132.png}\hspace{0pt}

% 方法:让小儿与小伙伴、家长一起玩。找出相同玩具同小朋友一块玩,培养小儿愉快的情绪。学步的小儿如在一起各拉各的玩具车学走,能互相模仿,互不侵犯,加快独走进程。

% 目的:训练社会交往能力。

% 注意:同时教导婴儿与小伙伴们和睦相处。

% %ux4e03ux667aux529bux57f9ux80b2ux6e38ux620f}{%
% \subsection{七、智力培育游戏}%ux4e03ux667aux529bux57f9ux80b2ux6e38ux620f}}

% %ux5c3aux5b50ux8fc7ux5939ux7f1d}{%
% \subsubsection{1 尺子过夹缝}%ux5c3aux5b50ux8fc7ux5939ux7f1d}}

% 方法:让婴儿站在藤椅后面(椅靠背上有几处大约6厘米的空隙),使他的手指能够自由地在空当中间出入。母亲在椅子上竖直地(妈妈自己在前边用手不时地固定``水平''的位置)放好一把长尺(或是一块长形木头),然后叫孩子从椅子后面通过空当把尺子拿过去。婴儿抓住尺子,但不知道应该把尺子竖过来才能通过空当。当宝宝怎么也拿不出尺子时,妈妈再把尺子放竖,让婴儿通过空当,很容易地取出尺子。

% 目的:该游戏可以在婴儿脑子里形成一系列的连锁思维,使他初步掌握关于空间位置要互相适应的道理。

% 注意:第2次、第3次就可以换上别的长形玩具(宽度要能通过空当),让婴儿自己动一下小脑筋取出来。

% %ux81eaux5236ux753bux518c}{%
% \subsubsection{2 自制画册}%ux81eaux5236ux753bux518c}}

% 方法:把一些构图简单、色彩艳丽的画面从旧杂志上剪下来,装订好,制成一本``婴儿画册''给婴儿看。看时,每张画面可停留7\textasciitilde8秒钟,并配以简要的介绍。如:``这是一座漂亮的房子,房子前面的花园里开着很多美丽的花。''\,``这是一只大花狗,它正在啃骨头吃呢!''家长可以经常和婴儿一道翻着这本自制的画册,等到婴儿对画册中的内容非常熟悉以后,可让他按照父母的指令去翻找画册。如妈妈说:``大花狗在哪儿呀,你给妈妈找一找。''如果婴儿不知所措,妈妈可握着婴儿的小手翻到大花狗那一页说:``原来大花狗在这儿呢!''

% 目的:有目的地发展婴儿的注意力和观察力,并通过简短、清晰的语言与画面的有机结合,给予婴儿良好的语言刺激。

% 注意:如果婴儿实在不想看画册,这一游戏可推迟进行。

% %ux516bux7acbux4f53ux80fdux529bux7ec3ux4e60}{%
% \subsection{八、立体能力练习}%ux516bux7acbux4f53ux80fdux529bux7ec3ux4e60}}

% 小棍够玩具

% 方法:在和婴儿玩滚皮球的游戏时,妈妈故意将小球滚到婴儿能看到但用手够不着的地方,然后给他一支细长的纸棍,看他能否用棍够玩具,如果妈妈给他示范,他就会照着做。

% 目的:认识物体与物体之间的关系,初步尝试使用``工具''。

% 注意:不要苛求婴儿能熟练地把玩具取出来,他只要能用棍子碰到玩具就很不错了。

% %ux4e5dux60c5ux611fux57f9ux517bux8badux7ec3}{%
% \subsection{九、情感培养训练}%ux4e5dux60c5ux611fux57f9ux517bux8badux7ec3}}

% \begin{enumerate}
% \def\labelenumi{\arabic{enumi}.}
% \item
%   盖盖子
% \end{enumerate}

% 方法:准备一只杯子和大、中、小三只盖子,其中只有一个盖子是正好盖在杯子上的。先教婴儿盖杯子的动作,然后再把三只盖子都给他,叫他``看用哪个盖子能把杯子盖好''。婴儿在反复盖上取下后,终于选中了合适的那个时,妈妈要给予鼓励。

% 目的:让婴儿掌握物体之间以及物体特性之间的最简单联系,启发他最初的思维活动。

% 注意:杯子最好是塑料制成的,防止摔碎伤到婴儿。

% \begin{enumerate}
% \def\labelenumi{\arabic{enumi}.}
% \setcounter{enumi}{1}
% \item
%   认识``灯''
% \end{enumerate}

% 方法:教孩子认识各种各样的灯。它们的大小、形状、颜色、所在位置都是不同的,如台灯、吊灯、壁灯、红灯、绿灯、日光灯等。不论你指哪盏灯,都应该说``这是灯'',并将灯打开再关上,使孩子认识灯的共同特点。

% 目的:运用词的概括作用发展思维,提高对语言的理解力。

% 注意:训练一段时间后,问孩子:``灯呢?''启发他指出所有的灯。

% %ux5341ux6570ux5b66ux80fdux529bux8badux7ec3}{%
% \subsection{十、数学能力训练}%ux5341ux6570ux5b66ux80fdux529bux8badux7ec3}}

% \begin{enumerate}
% \def\labelenumi{\arabic{enumi}.}
% \item
%   知道大小
% \end{enumerate}

% \hspace{0pt}\includegraphics[width=2.7972in,height=2.57343in]{media/rId2145.png}\hspace{0pt}

% 方法:将孩子抱在桌前,盘子里放着大、小两种饼干,家长拿起大的饼干,给孩子看,同时告诉他``这是大的'';接着再拿一块小的饼干给孩子,同时说``这是小的''。经过几次训练后,家长可以向孩子发出``拿一块大的饼干''的要求,看他能否拿对,如拿对了,可给他以示鼓励,接着再向孩子发出``拿一块小的饼干给我'',观察他是否能拿对,如拿的正确也要给以鼓励。孩子很快就学会分辨大和小,再用玩具或日常用品分别进行类似训练,以进一步巩固大和小的概念。

% 目的:通过大小的练习,培养对比能力。

% 注意:同理,还可以进行``上和下''、``前和后''的训练。

% \begin{enumerate}
% \def\labelenumi{\arabic{enumi}.}
% \setcounter{enumi}{1}
% \item
%   听数数
% \end{enumerate}

% 方法:在妈妈抱着孩子上下楼梯或扶着他学走路时,妈妈要有节奏地从1数到10给他听,也可在他玩积木时,帮他给积木排队数数。每天至少3次,让他慢慢掌握数目的顺序。

% 目的:熟悉数字大小的顺序,为发展数学概念奠定基础。

% 注意:开始只能数10以内数字。

% %ux5341ux4e00ux89c6ux89c9ux80fdux529bux8badux7ec3-1}{%
% \subsection{十一、视觉能力训练}%ux5341ux4e00ux89c6ux89c9ux80fdux529bux8badux7ec3-1}}

% %ux6edaux7b52ux8ba4ux7269}{%
% \subsubsection{1 滚筒认物}%ux6edaux7b52ux8ba4ux7269}}

% 方法:在滚筒里放进一些塑料小球、小瓶盖、小积木等,盖好放倒,使其滚来滚去发出声响,也可让孩子用手推动它向前滚动,问孩子:``是什么声音?里面有什么?''打开筒,让孩子把东西一件件拿出来辨认。

% 目的:训练视觉能力。

% 注意:东西不可太多,开始最好用孩子非常熟悉的玩具。

% %ux8ba4ux624bux6307}{%
% \subsubsection{2 认手指}%ux8ba4ux624bux6307}}

% 方法:妈妈抓住孩子的手指,让他五指分开,再握拳,再分开,让他的5个手指逐个伸屈,并告诉他每个手指的名称。当妈妈说``伸拇指,屈拇指''时,自己先举出拇指,让孩子模仿。

% 目的:训练眼与手的协调能力。

% 注意:玩的过程中可教孩子读儿歌,小手小手乖乖,两只小手拍拍。

% %ux5341ux4e8c-ux542cux89c9ux80fdux529bux8badux7ec3}{%
% \subsection{十二
% 听觉能力训练}%ux5341ux4e8c-ux542cux89c9ux80fdux529bux8badux7ec3}}

% %ux901bux4e50ux5668ux5e97}{%
% \subsubsection{1 逛乐器店}%ux901bux4e50ux5668ux5e97}}

% 方法:父母带孩子逛乐器商店,感受乐器店的音乐气氛,让孩子看一看,摸一摸,听一听,并感受几种乐器音色,如钢琴、笛子、提琴等。

% 目的:训练听觉能力。

% 注意:在孩子观察过程中,父母应及时将乐器名称告诉孩子。

% %ux8fa8ux522bux58f0ux97f3}{%
% \subsubsection{2 辨别声音}%ux8fa8ux522bux58f0ux97f3}}

% \hspace{0pt}\includegraphics[width=3.14685in,height=1.76224in]{media/rId2154.png}\hspace{0pt}

% 方法:让孩子用筷子敲玻璃杯、瓶子、饭碗和饭盒,听一听各发出什么声音,然后记住声音。让孩子背过身去,由妈妈敲,让孩子猜是哪个容器发出的声音。如果猜对了,换另一种容器继续猜。将4个玻璃杯分别装入不等量水,让孩子敲,并记住声音。然后背过身去,由组织者敲,让孩子猜出是哪个杯子发出的声音。

% 目的:训练听觉能力。

% 注意:猜对后要及时给予鼓励。

% %ux5341ux4e09ux611fux89c9ux80fdux529bux8badux7ec3}{%
% \subsection{十三、感觉能力训练}%ux5341ux4e09ux611fux89c9ux80fdux529bux8badux7ec3}}

% %ux53d8ux8272ux7684ux4e16ux754c}{%
% \subsubsection{1 变色的世界}%ux53d8ux8272ux7684ux4e16ux754c}}

% 方法:父母准备一个万花筒或几块不间颜色的透明塑料、玻璃,晴天的时候带孩子到院子里做游戏,把万花筒或彩色玻璃放在孩子的眼睛前,使其仰起头朝光线好的方向看去,孩子会看到不同色彩的世界。

% 目的:训练感觉能力。

% 注意:此游戏应注意选择光滑无棱角的塑料、玻璃,注意安全,并且玩的时间不宜过长。

% %ux5c0fux72d7ux6709ux4ec0ux4e48}{%
% \subsubsection{2 小狗有什么}%ux5c0fux72d7ux6709ux4ec0ux4e48}}

% 方法:先让孩子看图片,告诉他这是小狗,父母边模仿小狗叫声边说:``小狗有尾巴,有腿,有尖耳朵,也有眼睛和嘴巴,它的鼻子最灵,用鼻子去闻找肉骨头吃。''还可带孩子到街上观察小狗,帮助孩子指出狗的基本特征。教孩子念儿歌:``小花狗,带铃铛,爱吃骨头汪汪汪。''

% 目的:训练语言与实际相结合的能力。

% 注意:不要接近狗,远看即可。

% %ux5341ux56dbux52a8ux4f5cux80fdux529bux8badux7ec3}{%
% \subsection{十四、动作能力训练}%ux5341ux56dbux52a8ux4f5cux80fdux529bux8badux7ec3}}

% 1、会不会倒退走

% 方法:画一条线,让孩子在上面直走或横着走,不过孩子最喜欢玩的还是倒着走。一开始他会小心翼翼地一边走一边回头看。妈妈可以陪他玩,看谁先走到终点。

% 目的:训练动作能力。

% 注意:场地要平整,无大的障碍物。

% 2、踢罐子

% 方法:对宝宝来说,一只脚支撑着体重且维持平衡,另一只脚抬高踢东西的确相当困难。妈妈可先扶住他的脚,从踢的动作开始,再一步步进展到站着踢,边走边踢。

% 目的:训练眼、脚协调能力。

% 注意:要有耐心,不可急于求成,更不要说``你真笨''、``你不行''等刺伤孩子自尊心的话。

% 3、爸爸走,我也走

% 方法:爸爸双脚稍分开站立,宝宝面对爸爸,双脚踩在爸爸脚背上,双手抱着爸爸腿。爸爸往前走,宝宝随之向后退,爸爸向后退,宝宝随之向前行。

% 目的:锻炼行走能力。宝宝行走能力发展和其他动作发展一样,经历着既有连续性又有阶段性的发展过程。这个游戏在于进一步锻炼宝宝双手、双腿动作的协调性、随意性和灵活性。

% 注意:爸爸的速度不要快,以免宝宝跟不上摔倒。

% %ux7b2cux516dux7bc7-ux5a74ux5e7cux513fux7684ux75beux75c5ux9632ux6cbb}{%
% \section{6第六篇
% 婴幼儿的疾病防治}%ux7b2cux516dux7bc7-ux5a74ux5e7cux513fux7684ux75beux75c5ux9632ux6cbb}}

% 第一节呼吸系统疾病

% 第二节消化系统疾病

% %ux7b2cux4e00ux8282ux547cux5438ux7cfbux7edfux75beux75c5}{%
% \subsection{01第一节呼吸系统疾病}%ux7b2cux4e00ux8282ux547cux5438ux7cfbux7edfux75beux75c5}}

% 呼吸道系统分为上、下呼吸道两部分。上呼吸道包括鼻、鼻窦、咽、耳咽管和喉;下呼吸道包括气管、支气管和肺泡。常见疾病有上、下呼吸道急、慢性炎症和过敏性疾病等,其中以急性呼吸道感染最常见。上呼吸道感染(上感)是小儿最常见的疾病。气管炎或支气管炎也很多见。由于小儿呼吸道黏膜薄嫩、血管多,局部和全身抗病能力差,因此很容易患病。

% %ux4e00ux6162ux6027ux9f3bux708e}{%
% \subsection{一、慢性鼻炎}%ux4e00ux6162ux6027ux9f3bux708e}}

% 本病是鼻黏膜的慢性炎症性病变。常见的原因有急性鼻炎反复发作迁延、花粉或其他原因过敏、慢性鼻窦炎经常流脓刺激等。

% %ux60a3ux513fux8868ux73b0}{%
% \subsubsection{1. 患儿表现}%ux60a3ux513fux8868ux73b0}}

% \begin{enumerate}
% \def\labelenumi{\arabic{enumi}.}
% \item
%   鼻子不通气,小婴儿不愿吃奶,或吃奶过程中突然停止,然后开始哭闹。
% \item
%   鼻咽部发痒,打喷嚏,流清水样鼻涕。
% \item
%   说话声音变浊,带有鼻音。
% \item
%   鼻腔检查时可见黏膜充血,鼻甲肥厚或萎缩。
% \end{enumerate}

% %ux6cbbux7597ux529eux6cd5}{%
% \subsubsection{2 治疗办法}%ux6cbbux7597ux529eux6cd5}}

% 去除病因,如治疗鼻窦炎、鼻息肉等。

% \begin{enumerate}
% \def\labelenumi{\arabic{enumi}.}
% \item
%   1\%麻黄素滴鼻。
% \item
%   针灸治疗(迎香、合谷、列缺、风池等)。
% \item
%   中药治疗。
% \end{enumerate}

% %ux62a4ux7406ux8981ux70b9}{%
% \subsubsection{3 护理要点}%ux62a4ux7406ux8981ux70b9}}

% \begin{enumerate}
% \def\labelenumi{\arabic{enumi}.}
% \item
%   加强身体锻炼。
% \item
%   加强营养。
% \item
%   多晒太阳。
% \end{enumerate}

% %ux9884ux9632ux63aaux65bd}{%
% \subsubsection{4 预防措施}%ux9884ux9632ux63aaux65bd}}

% \begin{enumerate}
% \def\labelenumi{\arabic{enumi}.}
% \item
%   积极治疗急性鼻炎,防止转为慢性。
% \item
%   对过敏体质的孩子,要加强锻炼,增强机体抵抗力,尽量避免接触能引起过敏的物质,如杨花、柳絮、花粉等。
% \end{enumerate}

% %ux4e8cux6162ux6027ux9f3bux7aa6ux708e}{%
% \subsection{二、慢性鼻窦炎}%ux4e8cux6162ux6027ux9f3bux7aa6ux708e}}

% 本病是位于鼻腔周围与鼻腔相通的骨性腔隙(即鼻窦)内黏膜的慢性炎症病。大多数是由于急性鼻窦炎反复发作引起。

% %ux60a3ux513fux8868ux73b0-1}{%
% \subsubsection{1 患儿表现}%ux60a3ux513fux8868ux73b0-1}}

% \begin{enumerate}
% \def\labelenumi{\arabic{enumi}.}
% \item
%   鼻子不通气,甚至张口呼吸。
% \item
%   流脓鼻涕(黄色)。
% \item
%   感觉前额、太阳穴附近或后枕部胀痛。
% \item
%   入睡前咳嗽,可有发热,食欲差和体重不增。
% \item
%   鼻旁或前额部压之疼痛,鼻内有脓性分泌物。
% \item
%   鼻窦X线片显示鼻窦内有炎症性阴影。
% \end{enumerate}

% %ux6cbbux7597ux529eux6cd5-1}{%
% \subsubsection{2 治疗办法}%ux6cbbux7597ux529eux6cd5-1}}

% \begin{enumerate}
% \def\labelenumi{\arabic{enumi}.}
% \item
%   1\%麻黄素点鼻,每日3\textasciitilde4次。
% \item
%   青霉素治疗或用其他抗菌素治疗2周。
% \item
%   必要时手术治疗。
% \item
%   理疗或中药、针灸治疗。
% \end{enumerate}

% %ux62a4ux7406ux8981ux70b9-1}{%
% \subsubsection{3 护理要点}%ux62a4ux7406ux8981ux70b9-1}}

% 保持鼻腔通畅。

% %ux9884ux9632ux63aaux65bd-1}{%
% \subsubsection{4.预防措施}%ux9884ux9632ux63aaux65bd-1}}

% (1)积极治疗急性鼻窦炎,防止转为慢性。

% (2)积极治疗可能引起鼻窦炎的其他疾病。

% %ux4e09ux6162ux6027ux54bdux708e}{%
% \subsection{三、慢性咽炎}%ux4e09ux6162ux6027ux54bdux708e}}

% 本病是咽部的慢性炎症,常继发于鼻、鼻窦、扁桃体等的慢性炎症。

% %ux60a3ux513fux8868ux73b0-2}{%
% \subsubsection{1. 患儿表现}%ux60a3ux513fux8868ux73b0-2}}

% (1)自觉咽部不适、发干或有异物感。

% (2)经常有清嗓子的动作,经常``吭、喀''。

% (3)重者发生咳嗽,咳时面部涨红、成串咳嗽,但无痰,受冷风或其他烟雾刺激时更明显。

% (4)颔下淋巴结可能肿大。

% (5)检査时咽后壁发红,有滤泡增生,并可见脓性分泌物。

% (6)急性发作时伴有发烧。

% %ux6cbbux7597ux65b9ux6cd5}{%
% \subsubsection{2 治疗方法}%ux6cbbux7597ux65b9ux6cd5}}

% (1)1\%碘甘油咽部涂抹。

% (2)局部理疗。

% (3)针灸疗法。

% (4)甘露饮口服。

% (5)碘含片、抗菌素含片含用。

% %ux62a4ux7406ux8981ux70b9-2}{%
% \subsubsection{3 护理要点}%ux62a4ux7406ux8981ux70b9-2}}

% 1 加强营养,多晒太阳,呼吸新鲜空气。

% 2 锻炼身体,增强身体抵抗力。

% %ux56dbux5148ux5929ux6027ux5589ux9e23}{%
% \subsection{四、先天性喉鸣}%ux56dbux5148ux5929ux6027ux5589ux9e23}}

% 本病是由于喉软骨软化而导致吸气时困难的一种疾病。常见的原因是气管软骨发育缺陷或某些物质(肿物、肿大淋巴结等)长期压迫气管或支气管所致。

% %ux60a3ux513fux8868ux73b0-3}{%
% \subsubsection{1 患儿表现}%ux60a3ux513fux8868ux73b0-3}}

% (1)出生后不久出现吸气声,可轻可重,持续存在,亦可间歇出现,安静、睡觉后可缓解,睡觉后更明显。

% (2)吸气时喉头软骨下陷,肋间隙亦凹向里面。

% (3)合并感染时出现面色、皮肤发青,嗓子里有``呼噜''声。

% (4)可合并鸡胸、漏斗胸等佝偻病体征。

% (5)2岁后症状消失。

% %ux6cbbux7597ux529eux6cd5-2}{%
% \subsubsection{2 治疗办法}%ux6cbbux7597ux529eux6cd5-2}}

% 1 服用补钙片和维生素D。

% 2 解除气管腔外的压迫因素,占位性病变手术治疗,肿大淋巴结等消炎治疗。

% %ux62a4ux7406ux8981ux70b9-3}{%
% \subsubsection{3 护理要点}%ux62a4ux7406ux8981ux70b9-3}}

% 1 保持室内空气新鲜,阳光充足,温度20°C左右,湿度60\%左右。

% 2 喂奶时不要过急,用奶嘴喂者,奶头孔不要过大以免呛入气管内造成吸入性肺炎。

% 3 随时保持呼吸道通畅,及时清除口鼻等处的分泌物。

% 4 多抱孩子到室外晒太阳。

% 5 \textbf{冬天时要注意保温,少串门},防止发生呼吸道感染,使症状加重。

% %ux9884ux9632ux63aaux65bd-2}{%
% \subsubsection{4 预防措施}%ux9884ux9632ux63aaux65bd-2}}

% (1)母亲怀孕期间多吃含钙质丰富的饮食,如鱼、排骨汤、鸡蛋等。

% (2)母亲怀孕时如有四肢酸麻或手脚抽筋立即吃钙片和维生素D,多在户外散步、晒太阳。

% %ux4e94ux6025ux6027ux611fux67d3ux54bdux5589ux708e}{%
% \subsection{五、急性感染咽喉炎}%ux4e94ux6025ux6027ux611fux67d3ux54bdux5589ux708e}}

% 本病是喉黏膜的弥漫性炎症,常发生于1\textasciitilde3岁小儿。常在上感、麻疹、肺炎等疾病病程中并发,主要由细菌感染引起。

% %ux60a3ux513fux8868ux73b0-4}{%
% \subsubsection{1 患儿表现}%ux60a3ux513fux8868ux73b0-4}}

% (1)吸气时喉头软骨下陷,嗓子里有声。

% (2)声音嘶哑。

% (3)犬吠样咳嗽,咳嗽时脸涨得通红。

% (4)发烧。

% (5)吸气困难,夜间重,白天轻。

% (6)喉充血水肿,肺呼吸音减弱或消失,心率快时呼吸困难,出现面色、口唇、指(趾)发青,打人咬人,恐惧,出汗,甚至突然窒息而死亡。

% %ux6cbbux7597ux529eux6cd5-3}{%
% \subsubsection{2 治疗办法}%ux6cbbux7597ux529eux6cd5-3}}

% 1 用青霉素或红霉素控制感染。\\
% 2 用激素减轻喉部水肿。\\
% 3 吸气。\\
% 4 镇静。\\
% 5 吸痰,雾化吸入糜蛋白酶、抗菌素和激素。\\
% 6 严重时气管切开。

% %ux62a4ux7406ux8981ux70b9-4}{%
% \subsubsection{3 护理要点}%ux62a4ux7406ux8981ux70b9-4}}

% 1
% 半卧位安静休息,避免哭闹或多说话,保持室内空气新鲜、湿润,减少对呼吸道的刺激。\\
% 2 多饮水、喝牛奶、吃稀粥、吃面条等,喂时要注意不要过急,以免哈入气管。\\
% 3 小儿脸色发青、出大汗、烦躁不安时要立即就诊,准备气管切开。

% %ux9884ux9632ux63aaux65bd-3}{%
% \subsubsection{4 预防措施}%ux9884ux9632ux63aaux65bd-3}}

% 冬春时节少带孩子去公共场所,以防呼吸道传染病。

% %ux516dux6025ux6027ux4e0aux547cux5438ux9053ux611fux67d3}{%
% \subsection{六、急性上呼吸道感染}%ux516dux6025ux6027ux4e0aux547cux5438ux9053ux611fux67d3}}

% 简称``上感'',是上呼吸道的感染性疾病。如果上呼吸道的某一部位炎症表现突出就分别称为急性咽喉炎、扁桃体炎、急性鼻咽炎(感冒等)最常见的是病毒感染,其次是细菌感染。

% %ux60a3ux513fux8868ux73b0-5}{%
% \subsubsection{1 患儿表现}%ux60a3ux513fux8868ux73b0-5}}

% (1)发烧,体温38-39°C,甚至40°C,可以引起突然抽搐,1-2天反复,长者可持续1周。\\
% (2) 精神不振,乏力或哭闹不安,食欲不好。\\
% (3)流清鼻涕,鼻子不通气,打喷嚏,嗓子发痒,吞咽时疼痛,可有轻微声音嘶哑和咳嗽。\\
% (4) 嗓子红,可见小泡、扁桃腺肿大并可见有脓点。

% %ux6cbbux7597ux529eux6cd5-4}{%
% \subsubsection{2 治疗办法}%ux6cbbux7597ux529eux6cd5-4}}

% 1 头部用湿毛巾冷敷,用酒精或白酒擦浴及口服退热药等方法降温。\\
% 2 滴鼻净滴鼻(婴儿慎用)\\
% 3 细菌感染可用复方新诺明、青霉素消炎。\\
% 4 用大青叶、板蓝根冲剂以及乂0银翘散等治疗。\\
% 5 咳重时用小儿止咳糖浆或其他止咳药。\\
% 6 可试用病毒灵或用干扰素肌注。

% %ux62a4ux7406ux8981ux70b9-5}{%
% \subsubsection{3 护理要点}%ux62a4ux7406ux8981ux70b9-5}}

% 1 适当休息,经常变换体位。\\
% 2 多饮水,多吃水果,喂流食(奶类、果汁等)或半流食(米粥、面条)\\
% 3 保持室内空气新鲜,开门窗时应避免穿堂风。\\
% 4 室温18\textasciitilde20°C,湿度60\%左右,屋地上要勤洒水。\\
% 5 年长儿外出时要戴口罩。\\
% 6
% 密切观察是否有其他异常情况(如耳道流水或流脓,皮肤有无疹子等)随时就诊治疗。

% %ux9884ux9632ux63aaux65bd-4}{%
% \subsubsection{4 预防措施}%ux9884ux9632ux63aaux65bd-4}}

% 1 多锻炼身体,参加户外活动,晒太阳,增强机体抗病能力。\\
% 2
% 婴儿最好母乳喂养,按时增加辅食,幼儿不要偏食,以防营养不良和佝偻病,减少上感的发生机会。\\
% 3 气候变化时要随时增减衣服。\\
% 4 搞好环境卫生,在呼吸道疾病多发时节(冬春)不带孩子去公共场所。

% %ux4e03ux6025ux6027ux652fux6c14ux7ba1ux708e}{%
% \subsection{七、急性支气管炎}%ux4e03ux6025ux6027ux652fux6c14ux7ba1ux708e}}

% 本病是支气管黏膜的炎症,常继发于上感之后,是一些传染病的常见合并症,是由于细菌、病毒感染或吸入有毒气体所致。

% %ux60a3ux513fux8868ux73b0-6}{%
% \subsubsection{1 患儿表现}%ux60a3ux513fux8868ux73b0-6}}

% 1 断断续续咳嗽,嗓子里有呼噜声,可有黏痰或脓痰。\\
% 2 发烧,体温38.5°C左右,2\textasciitilde4日体温恢复正常,也可能不发热。\\
% 3 大喘气时有时前胸痛。\\
% 4 肺呼吸音粗糙,有时能听到易变的湿罗音。\\
% 5 X光片肺纹理稍强。

% %ux6cbbux7597ux529eux6cd5-5}{%
% \subsubsection{2 治疗办法}%ux6cbbux7597ux529eux6cd5-5}}

% 1 考虑细菌感染时用青霉素或其他消炎药。\\
% 2 干咳影响休息时适当应用止咳药,如止咳七号、止咳糖浆等。\\
% 3 体温高时用物理或药物方法降温。

% %ux62a4ux7406ux8981ux70b9-6}{%
% \subsubsection{3 护理要点}%ux62a4ux7406ux8981ux70b9-6}}

% 1 适当休息和活动。\\
% 2 保持室内空气新鲜,保持60\%的湿度。\\
% 3 少食多餐,以免因咳嗽引起呕吐。\\
% 4 经常变换体位、拍背帮助咳痰。

% %ux9884ux9632ux63aaux65bd-5}{%
% \subsection{4 预防措施}%ux9884ux9632ux63aaux65bd-5}}

% 1 多做户外活动,多晒太阳。\\
% 2 加强营养,增加身体抵抗力。

% %ux516bux80baux708eux7403ux83ccux6027ux80baux708e}{%
% \subsection{八、肺炎球菌性肺炎}%ux516bux80baux708eux7403ux83ccux6027ux80baux708e}}

% 本病系由肺炎球菌所引起的肺部炎症病变,是婴幼儿时期最常见的一种肺炎。北方冬春季、南方夏季多见。常由上呼吸道感染或支气管炎向下蔓延或继发于麻疹、百日咳之后。

% %ux60a3ux513fux8868ux73b0-7}{%
% \subsubsection{1. 患儿表现}%ux60a3ux513fux8868ux73b0-7}}

% \begin{enumerate}
% \def\labelenumi{\arabic{enumi}.}
% \item
%   发烧、体温没有规律或一日之间温差很大,也可高热持续不降。
% \item
%   病初干咳,冷风刺激后加重,病情最重时咳嗽减轻,肺炎要好时咳又重,并有痰咳出。
% \item
%   呼吸快,喘气时鼻翼扇动。
% \item
%   周口、鼻沟、指(趾)端发青。
% \item
%   病后几天肺内可听到中小水泡音。
% \item
%   严重时哭闹不安,面色苍白或绀青,尿很少或无尿,面部、下肢浮肿,昏睡不醒,抽搐,不省人事,腹胀,吐血,排黑便,四肢冰凉,脉摸不清,心跳快,心音纯,肝大。
% \item
%   X线片肺内有小片状阴影。
% \end{enumerate}

% %ux6cbbux7597ux529eux6cd5-6}{%
% \subsubsection{2. 治疗办法}%ux6cbbux7597ux529eux6cd5-6}}

% \begin{enumerate}
% \def\labelenumi{\arabic{enumi}.}
% \item
%   青霉素、红霉素等抗菌素治疗,体温正常后5\textasciitilde7天,肺部检查正常时停药。
% \item
%   咳嗽严重时用止咳药,发烧在38.5°C以上及时用解热药或用冷枕、湿敷、酒精擦浴等方法降温。
% \item
%   吸氧、镇静、纠正心衰、利尿、降颅压(并发脑病者)输血或输血浆、应用激素等根据具体情况而定。
% \end{enumerate}

% %ux62a4ux7406ux8981ux70b9-7}{%
% \subsubsection{3. 护理要点}%ux62a4ux7406ux8981ux70b9-7}}

% \begin{enumerate}
% \def\labelenumi{\arabic{enumi}.}
% \item
%   保持室内空气新鲜,室温18\textasciitilde20°C,湿度60\%。
% \item
%   经常翻身拍背,减少肺淤血,促进炎症吸收。\textbf{及时消除鼻痂和鼻内分泌物}(鼻涕)。
% \item
%   小婴儿喂奶时要抬高头部或抱起喂奶,以防呛咳。无力吸奶者要用小匙或滴管喂奶。大孩要吃牛奶、面条、米粥等,少食多餐,多喝水,多吃水果。
% \item
%   腹胀明显时可用\textbf{热毛巾腹部热敷}或用手轻轻\textbf{按揉}腹部促进肠蠕动和排气。
% \item
%   孩子哭闹不安,脸部发青不重者,可抱孩子去走廊吸新鲜空气。
% \item
%   病程迁延,体温正常者可去理疗,促进肺部湿啰音吸收。
% \end{enumerate}

% %ux9884ux9632ux63aaux65bd-6}{%
% \subsubsection{4. 预防措施}%ux9884ux9632ux63aaux65bd-6}}

% \begin{enumerate}
% \def\labelenumi{\arabic{enumi}.}
% \item
%   婴儿要及时添加辅食,培养良好的饮食和卫生习惯,多晒太阳,防止佝偻病及营养不良。
% \item
%   从小锻炼身体,经常开窗通风及到户外活动或在户外睡眠,增强机体对环境、温度变化的适应力。
% \item
%   呼吸道疾病多发季节,少带孩子去公共场所。
% \item
%   及时治疗能并发严重肺炎的呼吸道传染病(麻疹、百日咳、流感等)。
% \item
%   一旦发生肺炎,要积极治疗,防止发生严重的并发症(心衰、脑病、脓胸等)。
% \end{enumerate}

% %ux4e5dux547cux5438ux9053ux5408ux80deux75c5ux6bd2ux6027ux80baux708e}{%
% \subsection{九、呼吸道合胞病毒性肺炎}%ux4e5dux547cux5438ux9053ux5408ux80deux75c5ux6bd2ux6027ux80baux708e}}

% 由呼吸道合胞病毒引起的肺部炎症。多见于2岁内儿童,特别是6月内的小儿。

% %ux60a3ux513fux8868ux73b0-8}{%
% \subsubsection{1. 患儿表现}%ux60a3ux513fux8868ux73b0-8}}

% \begin{enumerate}
% \def\labelenumi{\arabic{enumi}.}
% \item
%   突然喘憋,呼气费力,呼气时呻吟和鼻翼扇动,很远的地方都能听到``呼噜''声。
% \item
%   哭闹不安,面色苍白或发青,往往喘稍缓解后小儿立刻坐起来玩。
% \item
%   咳嗽不重,发烧不明显(偶尔也有高烧)
% \item
%   肺内有明显的哮鸣音。
% \item
%   喘重可引起心力衰竭及中毒性脑病。
% \item
%   血中白细胞数正常或偏低。
% \item
%   X光片见肺内小点片状薄阴影及肺气肿改变,轻症1-2周,重者4-6周消失。
% \end{enumerate}

% %ux6cbbux7597ux529eux6cd5-7}{%
% \subsubsection{2. 治疗办法}%ux6cbbux7597ux529eux6cd5-7}}

% \begin{enumerate}
% \def\labelenumi{\arabic{enumi}.}
% \item
%   冬眠灵(氯丙嗪)、非那根(异丙嗪)联合应用平喘,或用水合氯醛、安定、鲁米那、维生素K、小剂量碳酸氢钠等平喘止咳,喘重时吸氧。
% \item
%   将小儿抱到走廊或其他空气新鲜的地方呼吸新鲜空气。
% \item
%   应用抗菌素防治继发细菌感染。
% \item
%   后期(恢复期)可做胸部理疗,促进炎症吸收。
% \end{enumerate}

% %ux62a4ux7406ux8981ux70b9-8}{%
% \subsubsection{3. 护理要点}%ux62a4ux7406ux8981ux70b9-8}}

% 参见肺炎球菌性肺炎的护理要点。

% %ux9884ux9632ux63aaux65bd-7}{%
% \subsubsection{4. 预防措施}%ux9884ux9632ux63aaux65bd-7}}

% \begin{enumerate}
% \def\labelenumi{\arabic{enumi}.}
% \item
%   增强机体抵抗力,多晒太阳,多去户外活动。
% \item
%   避免与病儿接触,防止交叉感染。
% \end{enumerate}

% %ux5341ux652fux6c14ux7ba1ux54eeux5598}{%
% \subsection{十、支气管哮喘}%ux5341ux652fux6c14ux7ba1ux54eeux5598}}

% 本病是反复发作的呼吸道变态反应性疾病,当吸入过敏源(尘土、花粉等)或呼吸道感染后可引起哮喘发作。发作多有季节性,春秋季多,本人多有过敏史,家中有类似病人。

% %ux60a3ux513fux8868ux73b0-9}{%
% \subsubsection{1 患儿表现}%ux60a3ux513fux8868ux73b0-9}}

% \begin{enumerate}
% \def\labelenumi{\arabic{enumi}.}
% \item
%   突然呼吸费力、躁动不安、不能平躺、耸肩喘息、面色苍白、口唇发青、鼻翼扇动、表情恐惧、出汗及剧咳。
% \item
%   肺呼吸音弱,能听到哮鸣音及啰音。
% \item
%   发作持续24个小时以上(哮喘持续状态)可出现心力衰竭表现。
% \item
%   哮喘发作多在夜间或清晨,时常从睡中憋醒。
% \item
%   肺部X光片肺透亮度明显增加或融合成圆的透光区。
% \item
%   血中白细胞数,多数患儿可增高,嗜酸粒细胞多增多。
% \end{enumerate}

% %ux6cbbux7597ux529eux6cd5-8}{%
% \subsubsection{2 治疗办法}%ux6cbbux7597ux529eux6cd5-8}}

% \begin{enumerate}
% \def\labelenumi{\arabic{enumi}.}
% \item
%   氨茶碱静点或稀释后静推。
% \item
%   吸氧。
% \item
%   应用激素抗炎抗过敏。
% \item
%   舒喘灵雾化吸入。
% \item
%   镇静祛痰。
% \item
%   合并感染时应用抗菌素。
% \end{enumerate}

% %ux62a4ux7406ux8981ux70b9-9}{%
% \subsubsection{3 护理要点}%ux62a4ux7406ux8981ux70b9-9}}

% \begin{enumerate}
% \def\labelenumi{\arabic{enumi}.}
% \item
%   居室要空气流通和新鲜,无烟雾、煤气、油漆等刺激性物质,不用带皮毛的东西,如鸭绒被(服)等。
% \item
%   半卧位(后背用棉被等垫起)以利呼吸。
% \item
%   吃清淡饮食,每次吃的不宜过饱,避免食用易引起过敏的食物(如海味、鱼、蛋等)。
% \item
%   平时注意孩子发作前的症状,一旦鼻发痒流涕、打喷嚏、咳嗽、胸闷要立即就诊。
% \end{enumerate}

% %ux9884ux9632ux63aaux65bd-8}{%
% \subsubsection{4 预防措施}%ux9884ux9632ux63aaux65bd-8}}

% \begin{enumerate}
% \def\labelenumi{\arabic{enumi}.}
% \item
%   有条件者可移居其他地方(易地疗法),常可减轻发作或治愈。
% \item
%   加强身体锻炼,防止受潮受凉。
% \item
%   注射哮喘菌苗。
% \item
%   到医院查找过敏原后采用脱敏疗法。
% \end{enumerate}

% %ux7b2cux4e8cux8282ux6d88ux5316ux7cfbux7edfux75beux75c5}{%
% \subsection{02第二节消化系统疾病}%ux7b2cux4e8cux8282ux6d88ux5316ux7cfbux7edfux75beux75c5}}

% 消化系统包括口腔、咽、食管、胃、十二指肠、肝、胆、胰、小肠(空肠和回肠)、大肠(结肠和直肠)以及肛门。消化道各个部分都可发生疾病,常见的如口腔炎、消化道溃疡、腹泻和先天性消化道畸形等。当某一部分发生疾患时,就可能影响到食物的摄取、消化,营养成分的吸收,废物的排出等当中的一个或者几个环节。

% %ux4e00ux9f8bux9f7f}{%
% \subsection{一、龋齿}%ux4e00ux9f8bux9f7f}}

% 龋齿是牙齿硬组织逐渐被破坏的一种疾病,是小儿的常见病、多发病,换牙前5岁左右和换牙后15岁左右多发。常见的原因有细菌的破坏、食物中的碳水化合物和糖被细菌作用产生酸性物质的腐蚀、牙齿的发育异常(牙沟过深、钙化不良)和睡液异常等。

% \begin{enumerate}
% \def\labelenumi{\arabic{enumi}.}
% \item
%   患儿表现
% \end{enumerate}

% (1)牙齿上出现褐色或黑褐色斑点或斑块,逐渐加重,牙齿被逐渐破坏,最后仅留下残缺不全的牙茬。

% (2)喝凉水或吃甜酸的东西时牙痛,特别是吃热东西时疼的更重,甚至吸凉气时牙也痛。起初疼的不明显,以后疼痛逐渐加重,早期除去上述刺激后,疼痛会立即消失,但到后期即便除去刺激因素后也要疼一段时间才能好。

% \begin{enumerate}
% \def\labelenumi{\arabic{enumi}.}
% \setcounter{enumi}{1}
% \item
%   治疗办法
% \end{enumerate}

% (1)清除龋坏组织并清洗消毒,用填充材料填补,同时恢复牙齿缺损的外形。

% (2)没有龋洞形成的可用药物治疗(氟化双胺银、氟化钠糊剂等)。

% \begin{enumerate}
% \def\labelenumi{\arabic{enumi}.}
% \setcounter{enumi}{2}
% \item
%   预防措施
% \end{enumerate}

% (1)有目的地增加牙齿中的氟素,改变牙齿的表面结构,增加牙齿的抗龋性,常用的方法有\textbf{自来水氟化}及用\textbf{含氟牙膏}刷牙或氟溶液漱口等。

% (2)从小养成良好的卫生习惯,睡前刷牙,饭后漱口。掌握正确的刷牙方法:上牙往下刷、下牙往上刷,各处都要刷到。

% (3)提倡多吃粗食,吃时要细嚼,这样既能增强牙周组织的韧性,又能摩擦牙齿咬面,使牙沟变浅。

% (4)少吃零食和糖果糕点,睡前最好不要吃糖,吃糖后一定要漱口或刷牙。

% %ux4e8cux6025ux6027ux6e83ux75a1ux6027ux53e3ux8154ux708e}{%
% \subsection{二、急性溃疡性口腔炎}%ux4e8cux6025ux6027ux6e83ux75a1ux6027ux53e3ux8154ux708e}}

% 急性溃疡性口腔炎是口腔内细菌引起的口腔黏膜急性炎症。小儿多见,小婴儿发病较重。

% \begin{enumerate}
% \def\labelenumi{\arabic{enumi}.}
% \item
%   患儿表现
% \end{enumerate}

% (1)发烧,体温可高达39\textasciitilde40°C。

% (2)不愿吃奶、哭闹,大孩子可诉说口内疼痛。

% (3)小婴儿不停地淌口水。

% (4)口腔黏膜发红,舌头、颊部、唇内侧上牙堂等处可见大小不等的溃疡面。

% (5)颌部可以摸到肿大的肿块。

% \begin{enumerate}
% \def\labelenumi{\arabic{enumi}.}
% \setcounter{enumi}{1}
% \item
%   治疗办法
% \end{enumerate}

% (1)用磺胺药或青霉素等消炎抗感染。

% (2)吃维生素(B1、B2、B12、C等)。

% (3)高烧时用各种方法降温。

% (4)用生理盐水或淡食盐水漱口或清洗口腔,并涂抹紫药水、冰硼散(油)、锡类散或2.5\%金霉素-鱼肝油软膏。

% (5)嘴唇上有溃疡时忌用紫药水涂抹,可涂用磺胺软膏或其他抗菌素软膏,以免结痂后继续形成溃疡。

% \begin{enumerate}
% \def\labelenumi{\arabic{enumi}.}
% \setcounter{enumi}{2}
% \item
%   护理要点
% \end{enumerate}

% (1)每天清洗口腔,保持其清洁,有溃疡时可用双氧水或1:2000高锰酸钾液蘸棉签擦洗溃疡面,然后用盐水冲洗干净并涂上各类抗菌素软膏。

% (2)多喂水或果汁等饮料,以保持口腔黏膜湿润,防止口腔内细菌繁殖。食物宜稀且不要太热,以免引起疼痛。

% \begin{enumerate}
% \def\labelenumi{\arabic{enumi}.}
% \setcounter{enumi}{3}
% \item
%   预防措施
% \end{enumerate}

% (1)平时注意保持口腔清洁,减少细菌生长机会。

% (2)加强身体锻炼,增加机体抵抗力。

% (3)奶瓶、奶头、玩具等要保持清洁,定期消毒。

% %ux4e09ux538cux98df}{%
% \subsection{三、厌食}%ux4e09ux538cux98df}}

% 厌食是指较长时间的食欲减退或消失,是消化道功能紊乱的表现之一。局部或全身疾病影响到消化系统的功能,或中枢神经系统的调节失去平衡都可能导致厌食。

% %ux60a3ux513fux8868ux73b0-10}{%
% \subsubsection{1. 患儿表现}%ux60a3ux513fux8868ux73b0-10}}

% (1)不思饭菜,经常腹部不适,一见肥肉或其他油腻食物就恶心、想吐,如眼白珠黄,可能是患有肝脏疾病的征象。

% (2)经常无规律进食,进食后不久或下次饭前上腹疼痛,有时夜间甚至疼醒,爱打饱隔,反酸水,这可能是患有消化道溃疡的表现。

% (3)不爱吃东西,经常肚子不好,不是大便干燥就是泻肚,这常提示有慢性肠炎、肠结核(或全身其他部位结核)等疾病的可能,经常给孩子吃油腻食物、巧克力、糖果及其他零食,这时的食欲不好多是由于饮食习惯不良造成的。

% (4)食欲不好伴情绪低下,多与情绪变化有关。

% (5)气候、室内温度过高伴食欲低下多与神经调节有关。

% (6)经常吃淡菜伴食欲不好,可能是低盐引起的。

% %ux6cbbux7597ux529eux6cd5-9}{%
% \subsubsection{2. 治疗办法}%ux6cbbux7597ux529eux6cd5-9}}

% (1)治疗原发疾病。

% (2)中医治疗:食欲低下、恶心呕吐、手心脚心热、睡眠不好、腹泻、腹胀、舌苔黄白腻,可服保和丸。面黄肌瘦,精神倦怠乏力,溏便,唇舌较淡,舌无苔或少苔,脉细弱无力,可用理中汤随症加减。

% (3)针灸疗法。

% (4)捏脊疗法。

% %ux62a4ux7406ux8981ux70b9-10}{%
% \subsubsection{3. 护理要点}%ux62a4ux7406ux8981ux70b9-10}}

% (1)改掉不正常的饮食规律。

% (2)建立有规律的生活方式。

% %ux56dbux6d88ux5316ux9053ux6e83ux75a1}{%
% \subsection{四、消化道溃疡}%ux56dbux6d88ux5316ux9053ux6e83ux75a1}}

% 本病是指消化道黏膜因各种原因造成坏死、脱落留下缺损的一组症候群。胃溃疡常发生于小婴儿,十二指肠溃疡多发生于年长儿。常见的原因是胃酸分泌过多。

% %ux60a3ux513fux8868ux73b0-11}{%
% \subsubsection{1. 患儿表现}%ux60a3ux513fux8868ux73b0-11}}

% (1) 食欲不振、消瘦。

% (2)
% 脐周或上腹部疼痛,尤以饭前、夜间疼痛(十二指肠溃疡)或饭后疼痛(胃溃疡)更甚。

% (3) 呕吐食物、酸水甚至血液,经常打饱隔,大便干燥甚至排柏油样黑便。

% (4) 上腹部或剑突下(心口窝)触之不适,或有疼痛。

% (5)
% 突然腹痛明显,呕吐腹胀加重,上腹部触摸时很硬,提示可能出现穿孔合并腹膜炎。

% (6) 胃肠道钡餐透视可发现胃或十二指肠内有溃疡。

% %ux6cbbux7597ux529eux6cd5-10}{%
% \subsubsection{2. 治疗办法}%ux6cbbux7597ux529eux6cd5-10}}

% (1) 服用止血促进溃疡愈合的药物,如甲氰咪胍、胃必治、胃乐新、胃得乐等。

% (2) 出现穿孔腹膜炎者要手术治疗。

% (3) 出血多时可应用云南白药、三七粉、止血粉等口服,必要时需输血。

% (4) 用中药小建中汤治疗。

% %ux62a4ux7406ux8981ux70b9-11}{%
% \subsubsection{3. 护理要点}%ux62a4ux7406ux8981ux70b9-11}}

% (1) 疼痛剧烈或出现并发症时要卧床休息。

% (2)
% 要吃柔软、容易消化、含蛋白高、热量高、维生素多的饮食,像牛奶、蒸鸡蛋糕、豆浆等,少食多餐,以不断地中和稀释胃酸。

% (3)
% 忌食辛辣、酸、冷硬食品,大孩子不要抽烟、喝酒、喝浓茶或咖啡,鱼肉不要吃得过多,以免刺激胃液分泌过多。

% (4) 保持大便通畅。

% (5) 做好精神护理,坚持规律治疗,两餐间服抗酸药。

% (6) 一般服药治疗6-8周,大多数疗效都很显著。

% %ux9884ux9632ux63aaux65bd-9}{%
% \subsubsection{4. 预防措施}%ux9884ux9632ux63aaux65bd-9}}

% (1) 加强身体锻炼,增强机体抗病能力。

% (2) 避免多吃辛辣(辣椒、大蒜)、酸、冷、硬食物。

% (3) 建立有规律的饮食和生活习惯。

% %ux4e94ux86d4ux866bux6027ux80a0ux6897ux963b}{%
% \subsection{五、蛔虫性肠梗阻}%ux4e94ux86d4ux866bux6027ux80a0ux6897ux963b}}

% 由于各种因素刺激蛔虫,使蛔虫聚集成团阻塞肠腔和发生肠扭转而造成的梗阻现象。最常见的原因是驱虫药物剂量不足,造成蛔虫骚动,使之聚集扭转成团。多为不全梗阻,可以自行缓解。

% %ux60a3ux513fux8868ux73b0-12}{%
% \subsubsection{1. 患儿表现}%ux60a3ux513fux8868ux73b0-12}}

% (1) 阵发性哭闹不安。

% (2) 呕吐,有时吐出蛔虫。

% (3)
% 腹胀气,在腹部可摸到大小不等的包块,呈肠形块状,脐周围多见。用手按时高低不平,有轻度活动感,并常能摸到粗大的麻绳样索状物,严重时腹壁强直、拒按,压时明显疼痛。

% (4) 时间长可排血便(肠坏死)。

% %ux6cbbux7597ux529eux6cd5-11}{%
% \subsubsection{2. 治疗办法}%ux6cbbux7597ux529eux6cd5-11}}

% (1) 驱虫药治疗,如用塔糖、肠虫清、驱蛔灵等,或应用\textbf{氧气驱虫}。

% (2) 腹胀、呕吐频繁用阿托品解痉。

% (3) 脱水酸中毒时补液和应用碱性药。

% (4) 有腹壁强直、压痛、排血便或用上述方法无效时应手术取出蛔虫,解除梗阻。

% %ux62a4ux7406ux8981ux70b9-12}{%
% \subsubsection{3. 护理要点}%ux62a4ux7406ux8981ux70b9-12}}

% (1)
% 一旦发现小儿哭闹不安,阵发性发作,伴有呕吐,同时腹部有包块,大小不一,应考虑到蛔虫性肠梗阻,可用手轻揉腹部或到医院就诊。

% (2) 呕吐是要将头扭向一侧,防止呕吐物呛入气管发生窒息。

% (3) 孩子腹痛时不要乱吃止痛药,以免误诊。

% %ux9884ux9632ux63aaux65bd-10}{%
% \subsubsection{4. 预防措施}%ux9884ux9632ux63aaux65bd-10}}

% (1) 讲究卫生,饭前便后洗手。

% (2) 驱虫药要在医生指导下服用。

% %ux516dux80c6ux9053ux86d4ux866bux75c7}{%
% \subsection{六、胆道蛔虫症}%ux516dux80c6ux9053ux86d4ux866bux75c7}}

% 本病是指蛔虫由于各种原因的刺激而逆行到胆道内继而引起一系列异常表现。常见的原因有高烧、饥饿、刺激性药物以及驱虫药剂量不足等。

% %ux60a3ux513fux8868ux73b0-13}{%
% \subsubsection{1. 患儿表现}%ux60a3ux513fux8868ux73b0-13}}

% (1)
% 突然上腹部剧烈疼痛,像扭动一样,同时有一种锐利的东西向上顶钻感,呻吟叫喊,坐卧不安,爬床滚地,汗流满面,表情异常紧张痛苦。小婴儿用手抓上腹部或要父母揉搓腹部。

% (2) 一阵一阵地发作,间歇期小儿可玩耍如常。

% (3) 可有呕吐,甚至吐出蛔虫。

% (4) 剑突下(心口窝)或稍偏右按压痛明显。

% %ux6cbbux7597ux529eux6cd5-12}{%
% \subsubsection{2. 治疗办法}%ux6cbbux7597ux529eux6cd5-12}}

% (1) 度冷丁和阿托品解痉镇痛。

% (2)
% 乌梅丸(汤)、食醋炖服(60\textasciitilde120毫升)、10\%硫酸镁、胆道驱虫汤(槟榔、苦楝及使君肉、枳壳、木香)等驱虫治疗。

% (3) 抗菌素控制感染。

% 上述方法无效或出现并发症(严重胆道感染、肝脓肿,胰腺炎等)应手术治疗(取虫)。

% %ux62a4ux7406ux8981ux70b9-13}{%
% \subsubsection{3. 护理要点}%ux62a4ux7406ux8981ux70b9-13}}

% (1) 防止孩子在哭闹不安发作时出现外伤。

% (2) 减少各种对孩子的不良刺激。

% (3) 防止呕吐物呛入气管发生窒息。

% %ux9884ux9632ux63aaux65bd-11}{%
% \subsubsection{4. 预防措施}%ux9884ux9632ux63aaux65bd-11}}

% (1) 讲究卫生,饭前便后洗手,减少肠道蛔虫发生率。

% (2) 驱虫要在医生指导下进行。

% (3) 保证孩子营养供应,防止饥饿造成胆道蛔虫症。

% (4) 高烧时要采取各种方法降温。

% %ux4e03ux6025ux6027ux9611ux5c3eux708e}{%
% \subsection{七、急性阑尾炎}%ux4e03ux6025ux6027ux9611ux5c3eux708e}}

% 本病是由各种因素造成的阑尾急性炎症。多见于较大的儿童,发病与细菌感染、阑尾腔梗阻或神经反射等因素有关。

% %ux60a3ux513fux8868ux73b0-14}{%
% \subsubsection{1. 患儿表现}%ux60a3ux513fux8868ux73b0-14}}

% (1)
% 脐周或上腹部疼痛,数小时以后转移到右下腹,多为持续性的钝痛,并且发作性地加重。

% (2) 喜右侧卧位,双腿稍屈,也可是仰卧位,一旦选择好位置后,就不喜欢再动。

% (3) 恶心、呕吐,早期轻微,后期严重。

% (4) 体温增高,脉搏增快。

% (5) 右下腹紧张拒按,撒手后疼痛更明显。

% (6) 血白细胞增高。

% (7) 右下腹穿刺,可能抽出脓性液体(已穿孔时)。

% %ux6cbbux7597ux529eux6cd5-13}{%
% \subsubsection{2. 治疗办法}%ux6cbbux7597ux529eux6cd5-13}}

% (1) 早期可以试用中药。

% (2) 上述方法无效需手术切除阑尾。严重时(已穿孔化脓者)要早期手术引流。

% (3) 针灸和西药消炎治疗,发病3天以上,病情逐渐稳定的可继续治疗,不需手术。

% %ux62a4ux7406ux8981ux70b9-14}{%
% \subsubsection{3. 护理要点}%ux62a4ux7406ux8981ux70b9-14}}

% (1)
% 术后要禁食。病情比较轻,苏醒后12小时即可喝水,以后逐渐吃些米汤、菜汤、果汁等,3天以后可以吃面条、稀粥,逐渐过渡到正常饮食。严重病例,禁食时间稍长。

% (2) 术后腹胀,可以用热毛巾做腹部热敷和胃肠减压。

% (3) 鼓励孩子做深呼吸和咳出气管内分泌物,防止肺部感染。

% (4) 鼓励孩子早晨起床做轻微活动,有利于病情恢复。

% (5) 有腹膜炎者,最好采取半坐卧位,有利于呼吸及腹腔脓肿的引流。

% (6) 注意手术切口护理,防止继发感染。

% %ux516bux6025ux6027ux8179ux819cux708e}{%
% \subsection{八、急性腹膜炎}%ux516bux6025ux6027ux8179ux819cux708e}}

% 本病是腹膜的急性感染。原发性腹膜炎大多数是细菌通过血液进入腹腔。常见的细菌是链球菌、肺炎球菌、葡萄球菌和大肠杆菌等。

% %ux60a3ux513fux8868ux73b0-15}{%
% \subsubsection{1. 患儿表现}%ux60a3ux513fux8868ux73b0-15}}

% (1) 突然出现较剧烈腹痛,难以忍受。

% (2) 频繁呕吐,呕出食物残渣或黄绿色液体。

% (3) 高烧,体温可高达40°C。

% (4) 面色苍白、烦躁不安,对外界反应迟钝,脉搏快而细弱。

% (5) 腹胀,全腹部有压痛、拒按,有腹水表现,肠鸣音消失。

% (6) 血白细胞明显增高。

% (7) 继发穿孔者,腹部透视有气体。

% (8) 腹部穿刺抽出脓性液体。

% %ux6cbbux7597ux529eux6cd5-14}{%
% \subsubsection{2. 治疗办法}%ux6cbbux7597ux529eux6cd5-14}}

% (1) 用有效抗菌素消炎。

% (2) 纠正脱水及电解质紊乱(静脉补液)。

% (3) 输血或血浆。

% (4) 大量应用维生素类。

% (5) 胃肠减压,以减轻腹胀,使胃肠道得到休息。

% (6) 诊断不明和腹腔内有大量积脓时,要手术吸尽脓汁,切除坏死组织或修补穿孔。

% %ux62a4ux7406ux8981ux70b9-15}{%
% \subsubsection{3. 护理要点}%ux62a4ux7406ux8981ux70b9-15}}

% (1) 卧床休息。

% (2) 先禁食,后给予流食、半流食、少渣饮食逐渐过渡到正常饮食。

% (3) 高热时用各种方法降温。

% (4) 保持呼吸道通畅,及时清除分泌物,防止窒息和肺部并发症的发生。

% (5) 胃肠减压时要注意保持引流管通畅。

% %ux4e5dux6025ux6027ux80a0ux7cfbux819cux6dcbux5df4ux7ed3ux708e}{%
% \subsection{九、急性肠系膜淋巴结炎}%ux4e5dux6025ux6027ux80a0ux7cfbux819cux6dcbux5df4ux7ed3ux708e}}

% 本病是因为肠系膜淋巴结感染而引起的炎症病变。多见于7岁以下儿童,常常在急性上呼吸道感染过程中并发,或者是继发于肠道感染之后。

% %ux60a3ux513fux8868ux73b0-16}{%
% \subsubsection{1. 患儿表现}%ux60a3ux513fux8868ux73b0-16}}

% (1) 腹痛,说不清确切的位置,甚至问哪哪疼。\\
% (2) 发烧,开始体温很高,但很快就恢复正常。\\
% (3) 呕吐,但次数不多。\\
% (4) 可以有腹泻或便秘。\\
% (5)
% 全腹摸之不适,但说不出确切的压痛点,可以在腹部摸到小结节样物质(肿大淋巴结)。\\
% (6) 血中白细胞一般不高。\\
% (7) 常有流涕、咳嗽、嗓子疼的症状。

% %ux6cbbux7597ux529eux6cd5-15}{%
% \subsubsection{2. 治疗办法}%ux6cbbux7597ux529eux6cd5-15}}

% 主要是消炎治疗,应用有效的抗菌素以后症状会逐渐减轻。

% %ux62a4ux7406ux8981ux70b9-16}{%
% \subsubsection{3. 护理要点}%ux62a4ux7406ux8981ux70b9-16}}

% (1) 注意休息,保证足够的营养和水分。\\
% (2) 不要轻易地应用止痛药,以免贻误诊断。\\
% (3)
% 腹痛剧烈,孩子哭闹不安,可以用热毛巾腹部热敷,或者是应用孩子感兴趣的东西分散其注意力。\\
% (4) 发热明显的,用各种方法降温。

% %ux524dux8a00}{%
% \section{前言}%ux524dux8a00}}

% 对于新时期初为父母者来说,面临的一切都是新鲜而又令人手足无措的新课题。新生儿有哪些发育特点,新生儿该如何喂养、如何护理、如何进行早期培育,婴儿每个月的发育特点有哪些变化,婴幼儿的营养特点有哪些、如何进行早期培育、如何进行饮食指导、如何进行智力开发、如何培养孩子的良好习惯等,都让初为父母者格外关心。

% 《新育儿百科》汇集了诸多专家的研究成果和专业精华,内容极为丰富详尽。全书以时间为顺序,从母亲的孕前准备一直到婴儿的生长发育、科学喂养、日常保健、婴幼儿心理、学前启智以及常见病的防治、小儿用药等,对数百个专题进行了科学、细致的研究分析和阐述,对婴幼儿养育具有很强的指导性和实践价值。此外,书内还别具匠心地穿插了许多栩栩如生的情趣图片,使本书的内容更加形象地呈现于读者面前。可以说本书是集科学性、实用性、趣味性于一体的百科全书;是新婚夫妇、年轻父母的必读之书;也是妇幼保健工作者的良师益友。

% 王琪,现就职于首都医科大学附属北京妇产医院围生医学科,主任医师、教授。从事围生医学工作20余年。对于孕前保健、孕期保健、高危妊娠监测及管理、优生优育咨询、妊娠高血压疾病、妊娠期糖尿病、产科危重急症的诊断和治疗,具有较高的专业水平和丰富的临床经验。已发表专业论文10多篇,参与撰写生殖健康、孕期保健类科普书籍20余本。

% 现任北京市孕产期保健技术专家指导组成员,北京市危重孕产妇及高危围生儿救治专家,中华医学会、北京医学会医疗事故鉴定专家库成员,全国妊娠高血压疾病学组成员。

% 图书在版编目(CIP)数据


% 定义了一个命令 \mshowc,该命令接收一个参数(计数器名),并将 \setcounter{计数器名}{计数器当前值} 的形式展示在文档中。
\newcommand{\mshowc}[1]{\noindent\tt\string\setcounter\{#1\}\{\arabic{#1}\}\newline}
\mshowc{chapter}
\mshowc{section}
\mshowc{subsection}
\mshowc{subsection}
\mshowc{page}



 



% %第二节
% \section{新生儿的生长发育和智力评估}
% %第三节
% \section{婴儿期各阶段的生长发育特点}
% %第四节
% \section{幼儿期的生长发育特点}
 
% %第二篇
% \chapter{婴幼儿的喂养指导}
% %第三篇
% \chapter{幼儿的营养方案}
% %第四篇
% \chapter{婴幼儿的日常护理}
% %第五篇
% \chapter{婴幼儿的早期数育}
% %第六篇
% \chapter{婴幼儿的疾病防治}


% %%%sumBgColor%%%%%%%%%%%%%%%%%%%%%%%%%%%%%%%%%%%%%%%%%%%%%%%%%%%%%%%%%%%%%%%%%
% %abstractlist
% \definecolor{sumBgColor}{RGB}{236, 236, 211}
% \definecolor{sumTxtColor}{RGB}{167, 33, 22}
% \newenvironment{itemizeSum}{%
% \begin{itemize}}{\end{itemize}}
% \tcolorboxenvironment{itemizeSum}{
% %before skip和after skip选项分别表示在环境前后添加6pt的垂直空白。
% % before skip=6pt,after skip=6pt,
% breakable,%
% opacityframe=0.0,colback=sumBgColor,colupper=sumTxtColor,grow to left by=2em,before upper={\textbf{This chapter covers\\本章涵盖了}}
% }
% %%%%%%%%%%%%%%%%%%%%%%%%%%%%%%%%%%%%%%%%%%%%%%%%%%%%%%%%%%%%%%%%%%%%

% \newtcolorbox[blend into=figures]{myfigure}[2][]{float=htb,capture=hbox,
% title={#2},every float=\centering,#1}


% 
\end{document}
%cd /Volumes/RamDisk/ &&  xelatex --output-directory=/Volumes/RamDisk/ -synctex=1 -shell-escape  /Users/virhuiai/hlProjects/Latex-Typesetting-Hub/练习书本排版/iText_in_Action_Second_Edition/iText_in_Action_Second_Edition.tex