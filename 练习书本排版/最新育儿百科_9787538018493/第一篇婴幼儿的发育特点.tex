% 第一篇婴幼儿的发育特点.tex

%第一篇婴幼儿的发育特点
\chapter{婴幼儿的发育特点}

%第一节
\section{婴幼儿年龄分期}

婴幼儿时期是体格发育和神经智能发育最旺盛和最迅速的时期。随着身体各个器官不断发育长大,生理功能逐步成熟,身心几乎每月甚至每天都在不断发生变化,各个时期的解剖、生理和病理特点又都是极不相同的。为了便于了解和区分不同年龄婴幼儿生长发育的特点和规律,我们将婴幼儿分为新生儿期、婴儿期和幼儿期三个时期,现简单介绍如下。

% 一、
\subsection[新生儿期]{新生儿期-\small 从出生到满28天}

从出生到满28天,这个时期的宝宝被称为新生儿。新生儿从母体温暖安静的小环境中突然进入到寒冷嘈杂的大环境中,是十月怀胎后的第一场人生考验,对于宝宝来说需要全身心地投入和机体的动员以适应外部世界。由于内外环境的突然巨变,小宝宝机体内尚未建立完整和健全的调节能力和适应能力,特别容易发生心、脑、肺及全身性疾病而导致死亡。在许多发达国家都建立了围生期保健网和新生儿监护中心,大大降低了新生儿的死亡率。在我国由于地区不同,其医疗卫生条件、经济状况及地理环境等的不同而差异非常大,如北京、上海等大城市婴儿的死亡率就明显低于农村。

经专家研究发现,新生儿不仅具有视听能力,还有惊人的模仿和记忆能力,早期开发新生儿大脑功能已经得到越来越多的专家和学者注意,并得到了广泛的社会关注。在新生儿医学方面,已经逐步由单一的体格发育、营养发育的研究转为智力潜能开发的研究。


% 二、
\subsection[婴儿期]{婴儿期-\small 28天到1岁}

婴儿期是指从生后28天到1岁的时期,这个时期的宝宝称为婴儿。因为这个时期的宝宝主要以\textbf{乳制品}为主要食物,也有人称为乳儿期。这个时期宝宝体格发育和智能发育的速度是最快的,几乎每天都有新的变化令父母惊奇和高兴。宝宝的体重可增加2\textasciitilde4倍,身长增加大约是出生时的50\%,脑重量的发育从400克达到1000克左右,各个器官已经具有初步的调节功能和适应能力。宝宝从每天单一枯燥的仰卧位姿势发展到能够随意爬行,独自站立和迈步等运动形式。已经学会用手势表达自己的愿望,并发出单音节音表示喜怒哀乐。此阶段是大脑早期开发的最佳时期,家长必须重视。

婴儿期最好采用\textbf{母乳}喂养,母乳中的大量\textbf{抗体}和从\textbf{母体输送给宝宝的抗体}足够宝宝维持出生后\textbf{半年}的抗病能力,而半年后则需要依靠宝宝自身的抵抗力和预防接种来抵御各种疾病。婴儿期的合理喂养和疾病预防是家长至关重要的理解项目。

% 三、
\subsection[幼儿期]{幼儿期-\small 1\textasciitilde3岁}

幼儿期是指1\textasciitilde3岁的时期,这个时期内的宝宝被称为幼儿。在这期间宝宝的体格发育速度明显减慢,但脑神经的发育仍在快速发展,到2岁时宝宝的脑细胞数量已经增殖完毕。通过与外界的接触和交流,宝宝的认知能力,语言和思维能力,分析和判断能力明显增强。因此,幼儿期增加与外界的交流,接触大量的与人类有关的事物是开发宝宝智力的关键。

幼儿期宝宝的饮食从单纯母乳喂养逐渐过渡到与成人相似的普通膳食。\textbf{合理}的食品营养搭配,防止营养摄入不平衡而出现营养物质的缺乏和过剩是本阶段的重点。正确引导宝宝\textbf{建立良好的饮食习惯},才能为今后体格发育和智能发育打下良好基础。

\subsection{婴幼儿生长发育特点}

大自然中任何事物的发生和发展都有其一定的规律,人类的生长发育也同样遵循自然发展的规律。

%1
\subsubsection{器官发育顺序和速度}

人类自怀胎开始首先发育的是大脑和神经系统,其次是胸腹内脏器官和肢体骨骼系统。在胎儿形成过程中,胎儿大脑发育最早,神经管在怀孕3个月发育成型,出生时大脑有$350\sim{}400$克,$6$个月时增加到$600$克左右,而到$1$岁时已经增加至$1000$克左右(成人约为$1400$克),约为成人脑重量的$70\%$,而1岁宝宝的体重仅为成人的$15\%\sim{}20\%$。这足以说明人类发育初期大脑的发育速度是非常惊人的,这也是婴幼儿早期智力开发的重要依据之一。

肢体骨骼发育的年龄越小,发育的速度越快,以体重为例,出生第一年增长最快,为出生时的$2\sim{}4$倍,$2$岁以后增长速度明显减慢,每年增加2千克左右。在出生后的$2\sim{}3$年时间里,婴儿大脑发育速度明显快于体格发育,到3岁以后大脑细胞已经增殖完毕,宝宝已经具备了基本的运动、语言和思维能力,而体格发育并未达到相适应的程度。因此,有一些父母会有这样的感觉,``这么小的孩子怎么会说出这么成熟的话''。其实,这正是人类发育中的正常现象,千万不要误认为自己的宝宝比别人的宝宝聪明而放松教育。【\textbf{不要迷之自信},是正常】


%2
\subsubsection{运动和智能发育特点}%\label{ux8fd0ux52a8ux548cux667aux80fdux53d1ux80b2ux7279ux70b9}}

\begin{description}
\item[由上到下]
婴儿从头到脚的顺序开始生长发育。比如,3个月时会趴着抬起头,4个月时抬起头的同时还能带动胸部抬高,5个月时可以用胳臂支撑翻过身来,6个月会坐,10个月能站立,1岁会主动用脚迈步等。一系列运动发育就是\textbf{由颈部逐渐发展到上肢},又进一步发展到下肢的延伸过程。

\item[由近到远]
从身体的中心部位开始发展到四肢的过程。例如,新生儿时听到声音或朝有光亮的方向转头,这是颈背部肌肉在运动的结果,$1\sim{}2$个月看人或物时会扭动肩部和腰部来试图与人交流,$3\sim{}4$个月用手臂够喜欢的物品,$5\sim{}6$个月时会主动用手掌抓握东西,$7\sim{}8$个月时用大拇指和其他手指抓住小东西等。这些都是从身体中心的大肌肉逐渐发展延伸到四肢末端小肌肉的过程表现。

\item[由粗到细]
最典型的例子就是由粗大的动作发展到精细的动作过程。宝宝开始只能用手大把抓东西,不知不觉中就能拇指分开捡起床面或桌面上散落的小颗粒物品。年轻的父母看到宝宝这些以前尚未出现的动作时会情不自禁地说``我们的宝宝又进步了''。宝宝眼睛的分辨过程也是这样,刚出生的宝宝只能看到人面部的轮廓,到2个月时已经能清楚辨认人的面部,3个月时又进一步能区分母亲与其他亲人的面孔。宝宝由粗糙的大体视野发展到细致的局部视野过程其实就是一个由粗到细的观察过程。【渐渐迭代,先框架,后细节\ldots】

\item[从简单到复杂]
在宝宝的运动、语言发育过程和认识事物过程中处处体现了这种发展规律。比如从牙牙学语到会背诵诗歌,用笔乱画到有意识地画出直线和圆圈,再发展到能画出漂亮的图形等。

\item[从低级到高级]
这是人类与动物的最大区别所在。动物能听到自然界的声音,但是它们不会运用声音创造出表达完整意思的语言。而人类则不同,从仅能听到声音发展到会模仿声音,并创造出最优美最动听的能够表达任何意愿的语言。从仅仅会使用简单的工具到会制造复杂的工具,直到现在人类发明的电子信息技术,都是一部人类由低级到高级的发展史。
\end{description}

%3
\subsubsection{体格和智能发育的个体差异}%\label{ux4f53ux683cux548cux667aux80fdux53d1ux80b2ux7684ux4e2aux4f53ux5deeux5f02}}

世间每一件事物都是不尽相同的,万物生长各有其不同的特点。虽然都是人类,但由于基因的不同,人与人之间有着千差万别,这就是个体差异。个体差异受到遗传基因、居住环境、营养状况及教育程度等诸多方面因素的影响。例如,身高方面,北方人普遍比南方人偏高,这与遗传和居住环境有很大的关系。再比如,具有某些``天赋''的人,像音乐神童莫扎特这样的音乐天才毕竟只是凤毛麟角,这些人的智商和一般人有区别,不能盲目效仿。有些父母望子成龙,从宝宝很小就开始拼命加码,希望孩子能够成为莫扎特式的人物。然而,结果往往不仅没有达到预期的目的,反而给宝宝幼小的心灵造成了很大的伤害,甚至影响了孩子的终生。因此,对于自己的宝宝的体格发育和智能发育,一定要结合孩子的兴趣、家庭和生活居住环境等多种因素来培养和教育,使宝宝在德智体三方面得到均衡发展。


% \begin{quote}
% gpt:

% 个体差异主要表现在以下几个方面:

% \begin{itemize}
% \item
% 发育速度:即便是同一种群体,每个个体的发育速度也会有差异。有的孩子可能会早早学会走路,而有的孩子可能会稍微晚一些。这并不意味着发展晚的孩子就比发展早的孩子差,只是每个人的发展速度不同而已。
% \item
% 特长和优点:每个人都有自己的特长和优点,这是由于遗传和环境因素共同作用的结果。例如,有的孩子可能有很好的语言能力,而有的孩子可能在数学或者绘画上有独到之处。
% \item
% 性格特质:每个人的性格都是独一无二的,这也是个体差异的一部分。有的孩子可能生性活泼开朗,善于交朋友;而有的孩子可能内向腼腆,喜欢独处。
% \item
% 学习方式:人们的学习方式也各不相同。有的人通过观察和实践来学习,而有的人则通过阅读和理论学习。理解和接受新知识的方式也会因人而异。
% \end{itemize}

% 这些个体差异是非常正常的,我们应该尊重并欣赏这些差异,因为正是这些差异使我们每个人都是独一无二的。在教育和培养孩子的过程中,我们应该注意到并尊重他们的个体差异,鼓励他们充分发挥自己的特长和优点,同时帮助他们改进弱点和不足。
% \end{quote}



%五、
\subsection{影响婴幼儿生长和发育的因素}

%1
\subsubsection{遗传因素}%

婴幼儿生长发育的特征,如皮肤和头发的颜色、身高、体重、相貌、性格等均受到父母双方遗传因素的影响。遗传性疾病,如染色体异常(先天愚型)等和代谢性疾病(苯丙酮尿症等)对孩子的生长发育均有显著的影响。


\subsubsection{性别因素}%2 

男孩和女孩在婴幼儿时期的生长发育特点是有很大区别的。男孩比较调皮,不守规矩,注意力不容易集中,因而男孩的模仿能力发育一般较女孩的发育慢一些,如语言发育,背诵诗歌,学唱儿歌等。男孩比较好动,肌肉骨骼发育相对旺盛,在运动发育方面,一部分男孩可能要比女孩快一些。从体格发育情况来看,刚刚出生的男宝宝平均体重会比女宝宝重一些,身长也略微长一些,以后整个婴幼儿期都是这样一种发育规律。因此,在评价婴幼儿的体格和智能发育水平时,一定要参照男孩和女孩的不同标准进行判断。


\subsubsection{营养因素}%3. 

营养因素对于婴幼儿来讲非常重要。营养是人类生命物质的基础,是婴幼儿身体健康必不可少的关键条件。营养充足可以使机体发育达到最佳状态,尤其对大脑的发育更是如此。从怀孕期到生后两周岁的这段时间里,供给足够的营养物质会促进脑细胞的生长发育。怀孕期营养不良的胎儿出生时不仅身材矮小、体重轻,而且还会影响脑神经的正常发育,严重者可导致脑神经损害后遗症。生后第1\textasciitilde2年如果出现营养不良,就会影响婴幼儿脑细胞的进一步增殖和增大,引起脑细胞数量的匮乏和细胞体积的缩小,使婴幼儿生长发育受到严重影响。


\subsubsection{家庭环境和社会环境}%4.

在一个家庭中,父母亲的教育程度,兄弟姐妹之间的亲情,家庭的氛围对宝宝的生长发育和心理发育是非常重要的。良好的生活环境和充满爱心的正确引导会使宝宝的体格和智力潜能得到最佳的发展。周围环境和宝宝所处的时代也很重要。


\subsubsection{怀孕时机与孕期情况}%5. 

受孕时机最好选择在父母双方生理状态最好和心理情况最稳定的时候。受孕前后,如果过度饮酒或抽烟都会影响宝宝的顺利出生。怀孕期间,孕妇的家庭环境、周围生活环境、营养状况、心理情绪及有无疾病等都会对即将出生的宝宝有影响。怀孕早期如果叶酸缺乏可导致胎儿神经管畸形;病毒感染可引起先天性心脏病、脑发育畸形和早产;服用某些有毒药物、有害化学物质侵袭及放射线辐射等都可以影响胎儿或出生后的宝宝生长发育。精神状态对宝宝的影响也是很大的,抑郁的母亲可能使自己的宝宝性格发生异常;精神有障碍的母亲的宝宝往往胆小懦弱,缺乏自信,同时容易出现精神异常等。


\subsubsection{疾病}%6. 

许多先天性疾病都会影响宝宝的智能发展。先天性心脏病不仅影响宝宝的身体发育,长期缺氧还会影响宝宝的大脑发育,使宝宝反应迟钝,接受事物能力差;某些染色体异常、畸形或先天性代谢性疾病可导致宝宝大脑发育迟缓,智力低下,终身不能独立生活。后天的营养也是至关重要的,缺钙导致佝倭病、营养不良性贫血、微量元素的缺乏等,都会给宝宝的体格和智力发育带来不利的影响。

‍% 2第二节
\section{新生儿的生长发育和智力评估}

 
% 一、
\subsection{新生儿的范畴和主要发育特点}

新生儿是指出生后28天以内的婴儿,这一时期又称之为新生儿期。胎儿离开母体进入到人类自然生活环境中,由于机体生命器官发育不成熟,免疫力低,对生活环境适应能力差,因此很容易出现各种各样的疾病而导致死亡。

% 1.
\subsubsection{新生儿分为以下几种:}%ux65b0ux751fux513fux5206ux4e3aux4ee5ux4e0bux51e0ux79cd}}

% 将以下内容用描述环境表示
% \begin{enumerate}
% \item
%   足月儿:是指出生时胎龄满$37\sim{}42$周,体重大于2500克的活产婴儿。
% \item
%   早产儿:是指胎龄不足37周的活产婴儿。与母亲妊娠期间合并疾病、外伤、受刺激、劳累及生殖器畸形等有关,有一些先天遗传性疾病、染色体疾病等胎儿畸形也会引起早产。
% \item
%   低出生体重儿:出生时体重小于2500克,大多数是早产儿。
% \item
%   巨大儿:指的是出生体重超过4,000克的婴儿,包括正常和疾病儿。
% \end{enumerate}

\begin{description}
\item[足月儿] 是指出生时胎龄满$37\sim{}42$周,体重大于2500克的活产婴儿。
\item[早产儿] 是指胎龄不足37周的活产婴儿。与母亲妊娠期间合并疾病、外伤、受刺激、劳累及生殖器畸形等有关,有一些先天遗传性疾病、染色体疾病等胎儿畸形也会引起早产。
\item[低出生体重儿] 出生时体重小于2500克,大多数是早产儿。
\item[巨大儿] 指的是出生体重超过4,000克的婴儿,包括正常和疾病儿。
\end{description}



% 2.
\subsubsection{新生儿的发育特点}

刚出生的新生儿浑身沾满了黄白色的\textbf{胎脂},呼吸不均匀,经擦拭后可见略显青紫的皮肤,经过1\textasciitilde2天机体自身的调节和与外界空气的接触之后皮肤转变为粉红色。婴儿皮肤表层很薄,但皮下脂肪丰富,因而皮肤显得娇嫩、柔软和富有弹性,容易受到损伤而感染。

新生儿头比较大,头长为身长的1/4左右。刚出生时头部因分娩时受到产道的挤压,会出现头顶部的肿胀(血肿或产瘤)。在头顶部可摸到一块没有骨头的区域,称之为囟\marginpar{\huge\PinyinTi{囟}}门,这是由于该处的头颅骨尚未连接到一起所致。有的婴儿鼻子尖部可见黄白色小点状物,是由于皮脂分泌过多而堆积所致。眼球很少转动,呈定视状,眉毛较稀疏。

胸廓比较窄小,呈圆柱形,不论男婴还是女婴,刚出生时两侧乳腺都显得有些肿胀,有些婴儿还会流出少许白色乳汁样液体,不必处理,几天后自行消失。腹部比较膨隆,脐带部有残端断痕,有时会有很少量的渗血,均属正常现象。

男婴的睾丸大小不等,表面有皱褶,阴囊内可摸到睾丸时表明睾丸已降至阴囊,有时候睾丸停留在腹股沟区或者根本摸不到,这时就需要定期到医院检查,以免延误治疗。女婴大阴唇发育良好,能覆盖小阴唇及阴蒂,可有少许分泌物流出。



% 二、
\subsection{新生儿体格发育的主要指标}

% \hspace{0pt}\includegraphics[width=3.07692in,height=2.02797in]{media/rId48.png}\hspace{0pt}
 
% 1. 
\subsubsection{体重}


% (1)
% 出生体重:这是反映新生儿体格和营养发育的主要指标。最近全国调查结果显示,平均出生体重男婴为3.3±0.4千克,女婴为3.2±0.4千克,与世界卫生组织的参考值基本一致。体重小于2.5千克者可能是由于胎龄不够、早产所致,或由于怀孕期间疾病引起宫内发育迟缓,导致胎儿营养不良;体重大于4.0千克属于巨大儿,多系母亲患有糖尿病,或者因怀孕期母亲饮食过度导致宝宝发育过快引起宝宝肥胖所致。

% (2)
% 体重增长速度:出生第1个月体重增长很快,平均为\texttt{800\textasciitilde{}1000}\hspace{0pt}克,出生后前3个月平均每月增长\texttt{700\textasciitilde{}800}\hspace{0pt}克,平均每天增长\texttt{20\textasciitilde{}30}\hspace{0pt}克。如果体重增长接近此数值,说明孩子喂养充足,营养合适。如果体重增长过慢或不增长,排除患病因素外,多数是由于喂养不足或腹泻所致。新生儿体重测量多选用电子秤,比较准确方便,或者选用婴儿磅秤,测试时尽量脱掉外衣,只穿内衣及带尿布。

\begin{description}
\item[出生体重] 这是反映新生儿体格和营养发育的主要指标。最近全国调查结果显示,平均出生体重男婴为3.3±0.4千克,女婴为3.2±0.4千克,与世界卫生组织的参考值基本一致。体重小于2.5千克者可能是由于胎龄不够、早产所致,或由于怀孕期间疾病引起宫内发育迟缓,导致胎儿营养不良;体重大于4.0千克属于巨大儿,多系母亲患有糖尿病,或者因怀孕期母亲饮食过度导致宝宝发育过快引起宝宝肥胖所致。

\item[体重增长速度] 出生第1个月体重增长很快,平均为\texttt{800\textasciitilde{}1000}\hspace{0pt}克,出生后前3个月平均每月增长\texttt{700\textasciitilde{}800}\hspace{0pt}克,平均每天增长\texttt{20\textasciitilde{}30}\hspace{0pt}克。如果体重增长接近此数值,说明孩子喂养充足,营养合适。如果体重增长过慢或不增长,排除患病因素外,多数是由于喂养不足或腹泻所致。新生儿体重测量多选用电子秤,比较准确方便,或者选用婴儿磅秤,测试时尽量脱掉外衣,只穿内衣及带尿布。
\end{description}

\subsubsection{身高}%2. 

身高是反映骨骼发育情况的重要指标,正常新生儿的身高大约为50cm,如果测量的身高远低于这个数值,可能是早产或者存在其他的发育问题。身高的测量应当在婴儿\textbf{平躺并且放松}的情况下进行,测量的标准是从婴儿的头顶到脚底。

% 3. 
\subsubsection{头围}

头围是反映大脑和颅骨的发育程度。因胎儿时期脑发育较快,故出生时头围较大,可达\texttt{33\textasciitilde{}34}\hspace{0pt}厘米,1岁时达46厘米,2岁时达48厘米。头围测量值在2岁以内最有临床意义。头围过小提示脑发育不良,头围过大提示脑积水。

%三、
\subsection{新生儿的神经反射}

对刚出生不久的宝宝给予一定的刺激后,会出现一些反射性动作,医学上称之为原始反射,这些原始反射一般在出生后\texttt{3\textasciitilde{}4}\hspace{0pt}个月自行消失。

% \begin{enumerate}
% \item
%   吸吮反射:嘴唇接触到乳头或奶嘴时,自动张开嘴进行吸吮。
% \item
%   觅食反射:用手指或奶嘴等物轻轻接触宝宝面颊,宝宝会自动转向同侧面颊,并呈张口寻找状态。
% \item
%   握持反射:当手掌部接触到某件东西时会自动握紧,不易松开。
% \item
%   拥抱反射:突然巨大的声响后,双手张开,呈拥抱状态。
% \item
%   踏步反射:抱起宝宝让其站立时,反射性地向前迈出\texttt{1\textasciitilde{}2}\hspace{0pt}步。
% \end{enumerate}

\begin{description}
\item[吸吮反射] 嘴唇接触到乳头或奶嘴时,自动张开嘴进行吸吮。
\item[觅食反射] 用手指或奶嘴等物轻轻接触宝宝面颊,宝宝会自动转向同侧面颊,并呈张口寻找状态。
\item[握持反射] 当手掌部接触到某件东西时会自动握紧,不易松开。
\item[拥抱反射] 突然巨大的声响后,双手张开,呈拥抱状态。
\item[踏步反射] 抱起宝宝让其站立时,反射性地向前迈出\texttt{1\textasciitilde{}2}\hspace{0pt}步。
\end{description}

以上这些神经反射对判定新生儿神经系统是否正常非常有用。这些反射在出生后\texttt{3\textasciitilde{}4}\hspace{0pt}个月逐渐消失,如果出生后4个月仍未消失可视为异常,应及时到医院求治。



%四、
\subsection{新生儿的觉醒和睡眠状态}

一般认为,出生1个月以内的新生儿大部分时间都处于睡眠状态(每天睡眠时间可达$18\sim{}22$小时)。获得足够的食物,又没有消化方面问题的宝宝,仅在喂奶前和喂奶后的短暂时间内保持清醒状态,其余大部分时间都在睡眠。但有一些宝宝白天睡眠时间较短,睁眼的时间相对多一些,家长因此很担心由于睡眠时间短会影响宝宝的身体发育。其实这种担心是多余的。只要夜间睡眠好,宝宝本身也无不适感,可认为正常。这种睡眠时间短的宝宝可能与父母睡眠时间的长短(\textbf{遗传}因素)有关。不论是上述哪一种情况,只要宝宝食欲正常,体重增长良好,家长不必多虑,您的宝宝会自己调整睡眠时间的。一般宝宝的睡眠形式有两种:

%1.
\subsubsection{安静睡眠状态(深睡眠)}

宝宝睡眠非常安静,脸部、四肢均呈放松状态,偶尔在声音的刺激下有惊跳动作,一些宝宝出现嘴角的摆动,呼吸非常均匀,偶有鼻鼾声。在这个阶段,宝宝处在完全休息状态。

%2.
\subsubsection{活动睡眠状态(浅睡眠)}

整个睡眠过程不安静,眼虽然呈闭合状态,但可见到眼球在眼睑下快速运动,偶尔短暂的睁开眼睛,四肢和躯体有一些活动,脸上常显出可笑的表情,如微笑和皱眉。有时出现吸吮动作或咽喉动作,轻微的声响就可引发惊跳动作,有时突然啼哭。从安静睡眠到活动睡眠是一个睡眠周期,时间各占一半,一个周期持续%
$0.5\sim{}1$%
个小时。所以,新生儿每天有%
$18\sim{}20$%
个睡眠周期,在这期间中有%\texttt{9\textasciitilde{}10}\hspace{0pt}
$9\sim{}10$
个小时是浅睡眠状态,难怪有些家长看着孩子睡觉不踏实。

\begin{mybox}[colback=yellow]{专家讲堂}
\subsubsection*{新生儿惊跳不可怕}
% \parindent=2em
浅睡眠时,一些极其轻微的声音都可引发宝宝的突然惊跳,使家长认为孩子受了某种惊吓,急忙用手按压四肢或抱起宝宝,扰乱了宝宝的正常睡眠。一些宝宝浅睡眠时突然啼哭可能与做梦有关,一些家长担心可能是饿了,急忙抱起喂奶,而这时宝宝不希望有人打扰,因而更大声啼哭,将奶头塞进宝宝嘴里遭到拒绝,许多年轻的父母见此情景愈发着急,进一步采取各种方式去安抚宝宝,其实正是这些不必要的关怀使宝宝感到极度的厌烦,只有反复大声啼哭表示抗议,有些家长因此误以为孩子得了什么病,急忙抱到医院,到医院时才发现孩子已经安然入睡。

由此可见,新生儿在浅睡眠时出现肢体的部分运动和轻声抽泣,以及突然啼哭均属正常,这只是在睡眠周期中发生的现象,不用急于抱起来或者喂奶,\textbf{只要轻轻用手掌拍一拍宝宝的腹部,或者轻轻抓住宝宝的小手},很快宝宝就会安静下来,重新进入睡眠状态。
\end{mybox}

%五、
\subsection{新生儿看的能力}

尚未分娩的胎儿对光就有感觉,许多母亲会发现刚出生不久的孩子对光非常敏感。例如,用手电光照孩子的眼睛,孩子就会出现皱眉、突然闭眼。夜里突然打开灯,孩子会从睡眠中惊醒。

许多科学研究证明,新生儿是有视觉的。但是,这种视觉与成人视觉不同。首先,\textbf{新生儿都是近视},他们看东西的最佳距离在20厘米左右。其次,他们调节视焦距能力不成熟,把太远或太近的东西均看成模糊影。因此,要想使新生儿能非常清楚地看清某件东西时,应把视物放置在距离眼睛约\textbf{20厘米}的位置,这种能力一直到出生后\texttt{3\textasciitilde{}4}\hspace{0pt}个月时才会改变。

新生儿的视觉有以下特点:

\begin{description}

\item
[清醒状态下看东西]新生儿看东西一定要处于清醒状态,虽然这种状态很短暂,通常在吃奶后1小时左右。
\item
[喜欢红颜色]宝宝喜欢鲜\textbf{艳}的东西,最开始认识的颜色是红色,以后逐渐认识黄色、蓝色等。因此,视物最好以红色为宜。
\item
[视物放在眼前20厘米左右的位置]通常这种训练的做法是买一个鲜红色的、圆形或方形玩具,放置在距孩子眼睛20厘米的位置,当孩子注视你所提供的玩具时,左右和上下轻轻移动,孩子的眼睛就会跟随移动,这就证明孩子已经能看见东西了。可以每日训练多次。但要记住,新生儿不仅能看见而且还能记住。所以,最好\textbf{经常更换玩具},使孩子永远保持一种新奇的反应。
\end{description}

特别能引起小宝宝兴趣的是人脸,最喜欢看的还是妈妈的脸,不仅如此,宝宝还能分辨出母亲脸上的变化。如果戴上口罩或眼镜,孩子就会非常频繁地注视母亲的脸,当婴儿发现妈妈与以前不一样时就会显得很烦躁,如不好好吃奶、睡觉不踏实或睡眠时间减少等。

新生儿早期教育训练中,\textbf{视觉的练习}是非常重要的。通过眼睛看到的东西可以刺激脑细胞活动,\textbf{促进脑智力的发育}。


% 六、
\subsection{新生儿听的能力}%
\marginpar{11}%20

现已证明,刚出生的新生儿已经建立了非常完整的听觉系统。在孩子的耳边放一些比较\textbf{柔和的音乐},他们就会非常安静,甚至还出现面部表情的变化,如微笑等。有的妈妈在宝宝清醒时,摇晃小铃铛或者玩具棒,宝宝就会以某些方式来表示他们听到声音了,如皱眉、眨眼、张嘴、扭动身体等。宝宝非常喜欢\textbf{柔和、缓慢、单纯的声音},厌烦尖利噪声,这些不和谐之音会引发宝宝的突然躁动不安或哭闹。

新生儿的听力还有一个非常重要的功能,就是能够分辨声音来源的方向。有专家做实验,在距离孩子耳旁%\texttt{10\textasciitilde{}15}\hspace{0pt}
$10\sim{}15$%
厘米处轻轻摇动带有``咯咯''声音的小塑料盒,就会发现宝宝开始转动眼睛寻找声音的方向,通常转向声音发出的方向,同时还用眼睛寻找。这项试验通过视觉和听觉相结合,说明新生儿具有良好的眼耳协调能力。

新生儿从出生的头几天起,在所有的声音中似乎更喜欢倾听人类的声音,尤其是母亲的声音。但在早期,他们尚不能分辨父亲与其他人的声音,这\textbf{可能}是由于\textbf{母亲声调较高},男人声调低不易区别所致。有些哭闹的新生儿听到母亲的声音后立刻就安静下来,同时还会寻找自己母亲的脸。国外有人曾做过试验,当新生儿听到自己母亲声音而看到其他母亲的脸,或者看到自己母亲的脸而听到其他母亲的声音时,他们就会表现出非常不安的样子。只有当同时听到自己母亲的声音和看到自己母亲的脸时,他们才会变得非常安静。

总之,孩子从一出生就已经具备了很完善的视觉和听觉功能,父母应该细心捕捉新生儿的这种能力,及时发掘和引导。经过一段时间的努力,就会发现您的宝宝出现了令人惊喜的进步。

% 七、
\subsection{新生儿的模仿力和记忆能力}

%1.
\subsubsection{新生儿的模仿力}

刚出生的婴儿就已经具有很强的模仿力,最有名的就是``伸舌试验''。当宝宝处于清醒状态时,让宝宝的脸与大人的脸相距20厘米,并让孩子直接注视大人的脸。大人尽可能地伸出舌头,慢慢重复做伸舌动作,每20秒1次,共%
$6\sim{}8$%
% \texttt{6\textasciitilde{}8}\hspace{0pt}
次,然后停止。如果宝宝一直看着你的脸,首先会在嘴里移动自己的舌头,大约半分钟,宝宝就会模仿大人将舌头伸出嘴外。有趣的是,婴儿不仅能够记住整个伸舌过程,还能记住是谁做的伸舌动作。

有人做了这样的试验:首先由一个人对刚出生不久的新生儿反复做伸舌动作,待宝宝学会之后,让宝宝注视几个人的面孔,这里面包括对宝宝反复做伸舌动作的人。令人惊奇的是,宝宝见到别人时,没有嘴和舌的特殊动作,唯独见到这位反复伸舌的人时,不管这个人表情如何,宝宝都会伸出自己的舌头,令人忍俊不禁。这说明新生儿不仅具有模仿能力,还有准确的记忆能力。

% %ux65b0ux751fux513fux7684ux8bb0ux5fc6ux529b}{%
\subsubsection{%2.
新生儿的记忆力}%ux65b0ux751fux513fux7684ux8bb0ux5fc6ux529b}}

新生儿具有记忆能力的最好证明就是``母语分辨试验''。具体是这样做的:把父母都是说英语的新生儿放进摇篮,然后将一个橡皮奶嘴放进婴儿的嘴里。这个橡皮奶嘴连接着一台计算机,能够记录婴儿吸吮的频率和强度。研究人员让婴儿听英语和菲律宾语。当婴儿听到以前从未听过的菲律宾语时,其反应很微弱。然而,当听到熟悉的英语,即父母的语言时,婴儿的反应明显变化,吸吮的强度和频率迅速提高。当再次切换到菲律宾语时,婴儿的吸吮动作明显减少。这个试验充分证明婴儿在母亲的腹中就已经具有记忆能力。有记忆就有交流,父母与刚出生的小宝宝交流的越多,宝宝的记忆内容就会越多,记忆功能也会越强。由此看出,与宝宝交流的重要性。

再大一些的新生儿还能记住所看到的东西,如床头的彩图或者玩具。一开始他们会注视很长时间,然后注视的时间逐渐缩短,似乎已经厌烦了。这时,如果换一样新的东西,他们又会重新表现出好奇的样子。这说明新生儿对已经看过的图像和玩具具有记忆的能力。宝宝只需几天就能够记住妈妈的面孔。当妈妈突然戴上眼镜时,新生儿就会好奇地注视自己的妈妈。这些表现告诉我们,新生儿不仅会看东西,还能记住看到的形象,具有更高一级的脑功能。
