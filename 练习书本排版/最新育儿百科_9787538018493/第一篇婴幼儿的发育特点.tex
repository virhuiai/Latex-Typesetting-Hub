% 第一篇婴幼儿的发育特点.tex

%第一篇婴幼儿的发育特点
\chapter{婴幼儿的发育特点}

%第一节
\section{婴幼儿年龄分期}

婴幼儿时期是体格发育和神经智能发育最旺盛和最迅速的时期。随着身体各个器官不断发育长大,生理功能逐步成熟,身心几乎每月甚至每天都在不断发生变化,各个时期的解剖、生理和病理特点又都是极不相同的。为了便于了解和区分不同年龄婴幼儿生长发育的特点和规律,我们将婴幼儿分为新生儿期、婴儿期和幼儿期三个时期,现简单介绍如下。

% 一、
\subsection[新生儿期]{新生儿期-\small 从出生到满28天}

从出生到满28天,这个时期的宝宝被称为新生儿。新生儿从母体温暖安静的小环境中突然进入到寒冷嘈杂的大环境中,是十月怀胎后的第一场人生考验,对于宝宝来说需要全身心地投入和机体的动员以适应外部世界。由于内外环境的突然巨变,小宝宝机体内尚未建立完整和健全的调节能力和适应能力,特别容易发生心、脑、肺及全身性疾病而导致死亡。在许多发达国家都建立了围生期保健网和新生儿监护中心,大大降低了新生儿的死亡率。在我国由于地区不同,其医疗卫生条件、经济状况及地理环境等的不同而差异非常大,如北京、上海等大城市婴儿的死亡率就明显低于农村。

经专家研究发现,新生儿不仅具有视听能力,还有惊人的模仿和记忆能力,早期开发新生儿大脑功能已经得到越来越多的专家和学者注意,并得到了广泛的社会关注。在新生儿医学方面,已经逐步由单一的体格发育、营养发育的研究转为智力潜能开发的研究。


% 二、
\subsection[婴儿期]{婴儿期-\small 28天到1岁}

婴儿期是指从生后28天到1岁的时期,这个时期的宝宝称为婴儿。因为这个时期的宝宝主要以\textbf{乳制品}为主要食物,也有人称为乳儿期。这个时期宝宝体格发育和智能发育的速度是最快的,几乎每天都有新的变化令父母惊奇和高兴。宝宝的体重可增加2\textasciitilde4倍,身长增加大约是出生时的50\%,脑重量的发育从400克达到1000克左右,各个器官已经具有初步的调节功能和适应能力。宝宝从每天单一枯燥的仰卧位姿势发展到能够随意爬行,独自站立和迈步等运动形式。已经学会用手势表达自己的愿望,并发出单音节音表示喜怒哀乐。此阶段是大脑早期开发的最佳时期,家长必须重视。

婴儿期最好采用\textbf{母乳}喂养,母乳中的大量\textbf{抗体}和从\textbf{母体输送给宝宝的抗体}足够宝宝维持出生后\textbf{半年}的抗病能力,而半年后则需要依靠宝宝自身的抵抗力和预防接种来抵御各种疾病。婴儿期的合理喂养和疾病预防是家长至关重要的理解项目。

% 三、
\subsection[幼儿期]{幼儿期-\small 1\textasciitilde3岁}

幼儿期是指1\textasciitilde3岁的时期,这个时期内的宝宝被称为幼儿。在这期间宝宝的体格发育速度明显减慢,但脑神经的发育仍在快速发展,到2岁时宝宝的脑细胞数量已经增殖完毕。通过与外界的接触和交流,宝宝的认知能力,语言和思维能力,分析和判断能力明显增强。因此,幼儿期增加与外界的交流,接触大量的与人类有关的事物是开发宝宝智力的关键。

幼儿期宝宝的饮食从单纯母乳喂养逐渐过渡到与成人相似的普通膳食。\textbf{合理}的食品营养搭配,防止营养摄入不平衡而出现营养物质的缺乏和过剩是本阶段的重点。正确引导宝宝\textbf{建立良好的饮食习惯},才能为今后体格发育和智能发育打下良好基础。

\subsection{婴幼儿生长发育特点}

大自然中任何事物的发生和发展都有其一定的规律,人类的生长发育也同样遵循自然发展的规律。

%1
\subsubsection{器官发育顺序和速度}

人类自怀胎开始首先发育的是大脑和神经系统,其次是胸腹内脏器官和肢体骨骼系统。在胎儿形成过程中,胎儿大脑发育最早,神经管在怀孕3个月发育成型,出生时大脑有$350\sim{}400$克,$6$个月时增加到$600$克左右,而到$1$岁时已经增加至$1000$克左右(成人约为$1400$克),约为成人脑重量的$70\%$,而1岁宝宝的体重仅为成人的$15\%\sim{}20\%$。这足以说明人类发育初期大脑的发育速度是非常惊人的,这也是婴幼儿早期智力开发的重要依据之一。

肢体骨骼发育的年龄越小,发育的速度越快,以体重为例,出生第一年增长最快,为出生时的$2\sim{}4$倍,$2$岁以后增长速度明显减慢,每年增加2千克左右。在出生后的$2\sim{}3$年时间里,婴儿大脑发育速度明显快于体格发育,到3岁以后大脑细胞已经增殖完毕,宝宝已经具备了基本的运动、语言和思维能力,而体格发育并未达到相适应的程度。因此,有一些父母会有这样的感觉,``这么小的孩子怎么会说出这么成熟的话''。其实,这正是人类发育中的正常现象,千万不要误认为自己的宝宝比别人的宝宝聪明而放松教育。【\textbf{不要迷之自信},是正常】


%2
\subsubsection{运动和智能发育特点}%\label{ux8fd0ux52a8ux548cux667aux80fdux53d1ux80b2ux7279ux70b9}}

\begin{description}
\item[由上到下]
婴儿从头到脚的顺序开始生长发育。比如,3个月时会趴着抬起头,4个月时抬起头的同时还能带动胸部抬高,5个月时可以用胳臂支撑翻过身来,6个月会坐,10个月能站立,1岁会主动用脚迈步等。一系列运动发育就是\textbf{由颈部逐渐发展到上肢},又进一步发展到下肢的延伸过程。

\item[由近到远]
从身体的中心部位开始发展到四肢的过程。例如,新生儿时听到声音或朝有光亮的方向转头,这是颈背部肌肉在运动的结果,$1\sim{}2$个月看人或物时会扭动肩部和腰部来试图与人交流,$3\sim{}4$个月用手臂够喜欢的物品,$5\sim{}6$个月时会主动用手掌抓握东西,$7\sim{}8$个月时用大拇指和其他手指抓住小东西等。这些都是从身体中心的大肌肉逐渐发展延伸到四肢末端小肌肉的过程表现。

\item[由粗到细]
最典型的例子就是由粗大的动作发展到精细的动作过程。宝宝开始只能用手大把抓东西,不知不觉中就能拇指分开捡起床面或桌面上散落的小颗粒物品。年轻的父母看到宝宝这些以前尚未出现的动作时会情不自禁地说``我们的宝宝又进步了''。宝宝眼睛的分辨过程也是这样,刚出生的宝宝只能看到人面部的轮廓,到2个月时已经能清楚辨认人的面部,3个月时又进一步能区分母亲与其他亲人的面孔。宝宝由粗糙的大体视野发展到细致的局部视野过程其实就是一个由粗到细的观察过程。【渐渐迭代,先框架,后细节\ldots】

\item[从简单到复杂]
在宝宝的运动、语言发育过程和认识事物过程中处处体现了这种发展规律。比如从牙牙学语到会背诵诗歌,用笔乱画到有意识地画出直线和圆圈,再发展到能画出漂亮的图形等。

\item[从低级到高级]
这是人类与动物的最大区别所在。动物能听到自然界的声音,但是它们不会运用声音创造出表达完整意思的语言。而人类则不同,从仅能听到声音发展到会模仿声音,并创造出最优美最动听的能够表达任何意愿的语言。从仅仅会使用简单的工具到会制造复杂的工具,直到现在人类发明的电子信息技术,都是一部人类由低级到高级的发展史。
\end{description}

%3
\subsubsection{体格和智能发育的个体差异}%\label{ux4f53ux683cux548cux667aux80fdux53d1ux80b2ux7684ux4e2aux4f53ux5deeux5f02}}

世间每一件事物都是不尽相同的,万物生长各有其不同的特点。虽然都是人类,但由于基因的不同,人与人之间有着千差万别,这就是个体差异。个体差异受到遗传基因、居住环境、营养状况及教育程度等诸多方面因素的影响。例如,身高方面,北方人普遍比南方人偏高,这与遗传和居住环境有很大的关系。再比如,具有某些``天赋''的人,像音乐神童莫扎特这样的音乐天才毕竟只是凤毛麟角,这些人的智商和一般人有区别,不能盲目效仿。有些父母望子成龙,从宝宝很小就开始拼命加码,希望孩子能够成为莫扎特式的人物。然而,结果往往不仅没有达到预期的目的,反而给宝宝幼小的心灵造成了很大的伤害,甚至影响了孩子的终生。因此,对于自己的宝宝的体格发育和智能发育,一定要结合孩子的兴趣、家庭和生活居住环境等多种因素来培养和教育,使宝宝在德智体三方面得到均衡发展。

