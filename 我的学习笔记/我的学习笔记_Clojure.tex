\PassOptionsToPackage{no-math}{fontspec}%禁用了使用fontspec宏包中的数学字体功能。
\PassOptionsToPackage{AutoFakeBold=true,AutoFakeSlant=true}{xeCJK}%让xeCJK宏包自动产生伪粗体和伪斜体效果。

\documentclass[a4paper,landscape]{book}
\usepackage[heading=true
,scheme=chinese%中文方案
,fontset=none%不使用默认的字体设置
,space=auto%自动调整中英文间距
]{ctex}
\setCJKmainfont{FangZhengShuSong-GBK-1.ttf}[Path=/Users/virhuiai/hlProjects/Latex-Typesetting-Hub/font/方正/]%设置文本的中文有衬线字体
\setCJKsansfont{FangZhengHeiTi-GBK-1.ttf}[Path=/Users/virhuiai/hlProjects/Latex-Typesetting-Hub/font/方正/]%设置文本的中文无衬线字体为
\setCJKmonofont{FangZhengFangSong-GBK-1.ttf}[Path=/Users/virhuiai/hlProjects/Latex-Typesetting-Hub/font/方正/] %设置文本的中文等宽字体 
\usepackage[left=15mm,right=15mm,top=15mm,bottom=15mm]{geometry}
\usepackage{parskip}

\usepackage[all]{tcolorbox}
\usepackage{paralist}
\usepackage{pdfcolparcolumns}
\usepackage{zebra-goodies}

\usepackage{tikz}
\usetikzlibrary{tikzmark}
\begin{document}

% 定义打开和关闭设置临时计数器命令的功能
\newcommand\counterSetTmpOpen{\providecommand{\counterSetTmp}{}\renewcommand\counterSetTmp[2]{\setcounter{##1}{##2}}}
\newcommand\counterSetTmpClose{\providecommand{\counterSetTmp}{}\renewcommand\counterSetTmp[2]{}}
\counterSetTmpOpen
% \counterSetTmpClose

%%%%%%%%%%%%%%%%%%%%%%%%%%%%%%%%%%%%%%%%%%%%%%%%%%%%%%%%%%%%%%%%%%%%%%%%
% \newcounter{Emp}[chapter]
% \renewcommand{\theEmp}{\thechapter.\arabic{Emp}}
% \newcommand{\EX}{\par%
% {\bf 例~}%
% \refstepcounter{Emp}{\bf\theEmp}\hspace{1em}}


% \input{我的学习笔记_Clojure/

% CLOJURE_FOR_THE_BRAVE_AND_TRUE

\parindent=0pt

% 支持音标的字体
\newfontfamily\fontGentiumPlus{GentiumPlus}[Path=/Users/virhuiai/hlProjects/Latex-Typesetting-Hub/font/免费商用英文/支持音标-GentiumPlus-6.200/,
Extension=.ttf,
UprightFont=*-Regular ,
BoldFont=*-Bold ,
ItalicFont=*-Italic,
BoldItalicFont = *-BoldItalic
]
\newfontfamily\fontGentiumBookPlus{GentiumBookPlus}[Path=/Users/virhuiai/hlProjects/Latex-Typesetting-Hub/font/免费商用英文/支持音标-GentiumPlus-6.200/,
Extension=.ttf,
UprightFont=*-Regular ,
BoldFont=*-Bold ,
ItalicFont=*-Italic,
BoldItalicFont = *-BoldItalic
]




%% v8
% \newlength{\csh一音标一单词宽度}
% \newlength{\csh一音标一单词高度}
% \newcommand{\csh一音标}[2]{%
% \settowidth{\csh一音标一单词宽度}{#1~}%
% \settototalheight{\csh一音标一单词高度}{#1~}%
% \tikzmarknode{#1}{#1}%
% \begin{tikzpicture}[remember picture,overlay]
% \node [red,above=-0.8ex] at (#1.north) {\small\fontGentiumPlus#2};
% \end{tikzpicture}
% }

% v9 使用了 LaTeX 的计数器机制来确保每个 \tikzmarknode 的标签都是唯一的。
% v10 让单词和音标的左边界对齐,你需要调整音标的位置以便它与单词的左边界对齐。要做到这一点,你可以使用 TikZ 的 anchor 属性来指定音标文本框的右下角作为定位点,这样就可以确保音标的左边界与单词的左边界对齐。
\newcounter{csh一音标一计数器}
\newlength{\csh一音标一单词宽度}
\newlength{\csh一音标一单词高度}
\newcommand{\csh一音标}[2]{%
\settowidth{\csh一音标一单词宽度}{#1~}%
\settototalheight{\csh一音标一单词高度}{#1~}%
\tikzmarknode{#1\thecsh一音标一计数器}{#1}%
\begin{tikzpicture}[remember picture,overlay]%
\node [red,anchor=south west] at ([yshift=-0.5em] #1\thecsh一音标一计数器.north west) {\hspace{-1ex}\small\fontGentiumPlus#2};
\end{tikzpicture}\stepcounter{csh一音标一计数器}}



 
% Here is some text with a tikzmark \tikzmarknode{a}{here}.

% \begin{tikzpicture}[remember picture,overlay]
%     \node [red,above=0pt] at (a.north) {\small $s\le 3$};
%     \end{tikzpicture}



% \begin{parcolumns}[colwidths={1=0.45\textwidth}]{2}% 设置为两栏排版
% \colchunk{学习终极语言并成为更好的程序员。}
% \colchunk{learn the ultimate language and become a better programmer}
% \end{parcolumns}

\chapter{Introduction}




\begin{parcolumns}[colwidths={1=0.4\textwidth}]{2}% 设置为两栏排版
\colchunk{在你的内心深处,你一直知道你注定要学习
Clojure。每次你举起键盘,对难以理解的类层次结构感到痛苦;每次你在夜晚醒着,因为一个由突变引发的
Heisenbug
而令你的亲人痛苦;每次一个竞态条件使你拔出更多的你已经越来越少的头发,你内心的某个秘密部分总是知道
\emph{肯定有更好的办法}。}

\colchunk{%
% , crying out in
% \csh一音标{anguish}{/ˈæŋɡwɪʃ/ 痛苦} over an \csh一音标{incomprehensible}{/ˌɪnkɑːmprɪˈhɛnsɪbəl/ 难以理解的} class \csh一音标{hierarchy}{/ˈhaɪəˌrɑːrki/ 层次结构}; every time you lay
% awake at night, \csh一音标{disturbing}{/dɪˈstɜːrbɪŋ/ 令人不安的} your loved ones with \csh一音标{sobs}{/sɑːbz/ 抽泣} over a
% \csh一音标{mutation}{/mjuːˈteɪʃən/ 突变} \csh一音标{heisenbug}{/ˈhaɪzənbʌɡ/ 海森堡虫\footnotemark[1]}; every time a \csh一音标{race condition}{/reɪs kənˈdɪʃən/ 竞态条件\footnotemark[2]} caused you to
% \csh一音标{pull out}{/pʊl aʊt/ 拔出} more of your \csh一音标{everdwindling}{/ˈɛvər dwɪndlɪŋ/ 不断减少的} hair, some secret part of you has
% known that \emph{there has to be a better way}.
Deep in your innermost \csh一音标{being}{/ˈbiːɪŋ/ 存在}, you've always known you were \csh一音标{destined}{/ˈdɛstɪnd/ 注定的} to
learn Clojure. Every time you held your keyboard \csh一音标{aloft}{/əˈlɔːft/ 在空中,向上}, crying out in
anguish over an incomprehensible class hierarchy; every time you lay
awake at night, disturbing your loved ones with sobs over a
mutation-induced heisenbug; every time a race condition caused you to
pull out more of your ever-dwindling hair, some secret part of you has
known that \emph{there has to be a better way}.
% Deep in your innermost being, you've always known you were destined to
% learn Clojure. Every time you held your keyboard aloft, crying out in
% anguish over an incomprehensible class hierarchy; every time you lay
% awake at night, disturbing your loved ones with sobs over a
% mutation-induced heisenbug; every time a race condition caused you to
% pull out more of your ever-dwindling hair, some secret part of you has
% known that \emph{there has to be a better way}.
}
\footnotetext[1]{一种在尝试调试时会改变行为的计算机bug}%todo 后面去那边文档添加
\footnotetext[2]{计算机科学中关于资源争夺的问题}

\end{parcolumns}

% \zebranewnote{音标}{red}
% \音标{\fontGentiumPlus /ˈdɛstɪnd/ 注定的}


% origin

\end{document}

% fontGentiumPlus

% \begin{parcolumns}{2}% 设置为两栏排版
% \colchunk{左侧文本}
% \colchunk{右侧文本}
% \end{parcolumns}


% \begin{parcolumns}{2}% 设置为两栏排版
% \colchunk{左侧文本}
% \colchunk{右侧文本}
% \end{parcolumns}


% \begin{parcolumns}{2}% 设置为两栏排版
% \colchunk{左侧文本}
% \colchunk{右侧文本}
% \end{parcolumns}


% \begin{parcolumns}{2}% 设置为两栏排版
% \colchunk{左侧文本}
% \colchunk{右侧文本}
% \end{parcolumns}


% \begin{parcolumns}{2}% 设置为两栏排版
% \colchunk{左侧文本}
% \colchunk{右侧文本}
% \end{parcolumns}


% \begin{parcolumns}{2}% 设置为两栏排版
% \colchunk{左侧文本}
% \colchunk{右侧文本}
% \end{parcolumns}


% \begin{parcolumns}{2}% 设置为两栏排版
% \colchunk{左侧文本}
% \colchunk{右侧文本}
% \end{parcolumns}


% \begin{parcolumns}{2}% 设置为两栏排版
% \colchunk{左侧文本}
% \colchunk{右侧文本}
% \end{parcolumns}


% \begin{parcolumns}{2}% 设置为两栏排版
% \colchunk{左侧文本}
% \colchunk{右侧文本}
% \end{parcolumns}


% \begin{parcolumns}{2}% 设置为两栏排版
% \colchunk{左侧文本}
% \colchunk{右侧文本}
% \end{parcolumns}


% \begin{parcolumns}{2}% 设置为两栏排版
% \colchunk{左侧文本}
% \colchunk{右侧文本}
% \end{parcolumns}


% \begin{parcolumns}{2}% 设置为两栏排版
% \colchunk{左侧文本}
% \colchunk{右侧文本}
% \end{parcolumns}


% \begin{parcolumns}{2}% 设置为两栏排版
% \colchunk{左侧文本}
% \colchunk{右侧文本}
% \end{parcolumns}


% \begin{parcolumns}{2}% 设置为两栏排版
% \colchunk{左侧文本}
% \colchunk{右侧文本}
% \end{parcolumns}


% \begin{parcolumns}{2}% 设置为两栏排版
% \colchunk{左侧文本}
% \colchunk{右侧文本}
% \end{parcolumns}



%mypdf 我的学习笔记_XeLaTeX

% 我的学习笔记_Clojure.tex

% 在 parcolumns 环境中,可以使用 \parchunk1宏在两个或多个列中并列地排 版文本。也可以包括普通文本。

% \colchunk[2]{...}

% 将左侧列稍 微放大:colwidths={1=.55\linewidth}

% 三列,选项nofirstindent=true:
% \begin{parcolumns}[nofirstindent]{3}