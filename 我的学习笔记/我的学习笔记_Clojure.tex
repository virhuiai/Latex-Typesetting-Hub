\PassOptionsToPackage{no-math}{fontspec}%禁用了使用fontspec宏包中的数学字体功能。
\PassOptionsToPackage{AutoFakeBold=true,AutoFakeSlant=true}{xeCJK}%让xeCJK宏包自动产生伪粗体和伪斜体效果。

\documentclass[a4paper,landscape]{book}
\usepackage[heading=true
,scheme=chinese%中文方案
,fontset=none%不使用默认的字体设置
,space=auto%自动调整中英文间距
]{ctex}
\setCJKmainfont{FangZhengShuSong-GBK-1.ttf}[Path=/Users/virhuiai/hlProjects/Latex-Typesetting-Hub/font/方正/]%设置文本的中文有衬线字体
\setCJKsansfont{FangZhengHeiTi-GBK-1.ttf}[Path=/Users/virhuiai/hlProjects/Latex-Typesetting-Hub/font/方正/]%设置文本的中文无衬线字体为
\setCJKmonofont{FangZhengFangSong-GBK-1.ttf}[Path=/Users/virhuiai/hlProjects/Latex-Typesetting-Hub/font/方正/] %设置文本的中文等宽字体 
\usepackage[left=15mm,right=15mm,top=15mm,bottom=15mm]{geometry}

\usepackage[all]{tcolorbox}
\usepackage{paralist}
\usepackage{pdfcolparcolumns}
\begin{document}

% 定义打开和关闭设置临时计数器命令的功能
\newcommand\counterSetTmpOpen{\providecommand{\counterSetTmp}{}\renewcommand\counterSetTmp[2]{\setcounter{##1}{##2}}}
\newcommand\counterSetTmpClose{\providecommand{\counterSetTmp}{}\renewcommand\counterSetTmp[2]{}}
\counterSetTmpOpen
% \counterSetTmpClose

%%%%%%%%%%%%%%%%%%%%%%%%%%%%%%%%%%%%%%%%%%%%%%%%%%%%%%%%%%%%%%%%%%%%%%%%
% \newcounter{Emp}[chapter]
% \renewcommand{\theEmp}{\thechapter.\arabic{Emp}}
% \newcommand{\EX}{\par%
% {\bf 例~}%
% \refstepcounter{Emp}{\bf\theEmp}\hspace{1em}}


% \input{我的学习笔记_Clojure/

% CLOJURE_FOR_THE_BRAVE_AND_TRUE

\parindent=0pt

% \begin{parcolumns}[colwidths={1=5em}]{3}% 设置为两栏排版
\begin{parcolumns}[colwidths={1=0.45\textwidth}]{2}% 设置为两栏排版
\colchunk{学习终极语言并成为更好的程序员。}
\colchunk{learn the ultimate language and become a better programmer}

\end{parcolumns}

\end{document}

\begin{parcolumns}{2}% 设置为两栏排版
    \colchunk{左侧文本}
    \colchunk{右侧文本}
    \end{parcolumns}


%mypdf 我的学习笔记_XeLaTeX

% 我的学习笔记_Clojure.tex

% 在 parcolumns 环境中,可以使用 \parchunk1宏在两个或多个列中并列地排 版文本。也可以包括普通文本。

% \colchunk[2]{...}

% 将左侧列稍 微放大:colwidths={1=.55\linewidth}

% 三列,选项nofirstindent=true:
% \begin{parcolumns}[nofirstindent]{3}