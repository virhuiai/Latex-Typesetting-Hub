\PassOptionsToPackage{no-math}{fontspec}%禁用了使用fontspec宏包中的数学字体功能。
\PassOptionsToPackage{AutoFakeBold=true,AutoFakeSlant=true}{xeCJK}%让xeCJK宏包自动产生伪粗体和伪斜体效果。

\documentclass[a4paper,landscape]{book}
\usepackage[heading=true
,scheme=chinese%中文方案
,fontset=none%不使用默认的字体设置
,space=auto%自动调整中英文间距
]{ctex}
\setCJKmainfont{FangZhengShuSong-GBK-1.ttf}[Path=/Users/virhuiai/hlProjects/Latex-Typesetting-Hub/font/方正/]%设置文本的中文有衬线字体
\setCJKsansfont{FangZhengHeiTi-GBK-1.ttf}[Path=/Users/virhuiai/hlProjects/Latex-Typesetting-Hub/font/方正/]%设置文本的中文无衬线字体为
\setCJKmonofont{FangZhengFangSong-GBK-1.ttf}[Path=/Users/virhuiai/hlProjects/Latex-Typesetting-Hub/font/方正/] %设置文本的中文等宽字体 
% \setCJKfamilyfont{fontKai}{LXGWWenKai-Regular.ttf}[Path=/Users/virhuiai/hlProjects/Latex-Typesetting-Hub/font/霞鹜文楷/]
\setCJKfamilyfont{fontKai}{FangZhengKaiTi-GBK-1.ttf}[Path=/Users/virhuiai/hlProjects/Latex-Typesetting-Hub/font/方正/]
\newcommand\fontKai{\CJKfamily{fontKai}}


\usepackage[left=15mm,right=15mm,top=15mm,bottom=15mm]{geometry}
\usepackage{parskip}

\usepackage[all]{tcolorbox}
\usepackage{paralist}
\usepackage{pdfcolparcolumns}
\usepackage{zebra-goodies}

\usepackage{tikz}
\usetikzlibrary{tikzmark}
\usepackage{xspace}
\begin{document}

% 定义打开和关闭设置临时计数器命令的功能
\newcommand\counterSetTmpOpen{\providecommand{\counterSetTmp}{}\renewcommand\counterSetTmp[2]{\setcounter{##1}{##2}}}
\newcommand\counterSetTmpClose{\providecommand{\counterSetTmp}{}\renewcommand\counterSetTmp[2]{}}
\counterSetTmpOpen
% \counterSetTmpClose

%%%%%%%%%%%%%%%%%%%%%%%%%%%%%%%%%%%%%%%%%%%%%%%%%%%%%%%%%%%%%%%%%%%%%%%%
% \newcounter{Emp}[chapter]
% \renewcommand{\theEmp}{\thechapter.\arabic{Emp}}
% \newcommand{\EX}{\par%
% {\bf 例~}%
% \refstepcounter{Emp}{\bf\theEmp}\hspace{1em}}


% \input{我的学习笔记_Clojure/

% CLOJURE_FOR_THE_BRAVE_AND_TRUE

\parindent=0pt

% 支持音标的字体
\newfontfamily\fontGentiumPlus{GentiumPlus}[Path=/Users/virhuiai/hlProjects/Latex-Typesetting-Hub/font/免费商用英文/支持音标-GentiumPlus-6.200/,
Extension=.ttf,
UprightFont=*-Regular ,
BoldFont=*-Bold ,
ItalicFont=*-Italic,
BoldItalicFont = *-BoldItalic
]
\newfontfamily\fontGentiumBookPlus{GentiumBookPlus}[Path=/Users/virhuiai/hlProjects/Latex-Typesetting-Hub/font/免费商用英文/支持音标-GentiumPlus-6.200/,
Extension=.ttf,
UprightFont=*-Regular ,
BoldFont=*-Bold ,
ItalicFont=*-Italic,
BoldItalicFont = *-BoldItalic
]

%%%%%%%%%%%%%%%%%%%%%%%%% csh一音标/BEGIN %%%%%%%%%%%%%%%%%%%%%%%%%
\newcounter{csh一音标一计数器}
\newlength{\csh一音标一单词宽度}
\newlength{\csh一音标一单词高度}
\newcommand{\csh一音标}[2]{%
\tikzmarknode{#1\thecsh一音标一计数器}{\MakeLowercase{#1}\xspace}%
\begin{tikzpicture}[remember picture,overlay]%
\node [red,anchor=south west] at ([yshift=-0.45em] #1\thecsh一音标一计数器.north west) {\hspace{-1ex}\small\fontGentiumPlus#2};
\end{tikzpicture}\stepcounter{csh一音标一计数器}}
%%%%%%%%%%%%%%%%%%%%%%%%% csh一音标/END %%%%%%%%%%%%%%%%%%%%%%%%%

\newcommand\parcolumnsLeft{0.4\textwidth}
% \begin{parcolumns}[colwidths={1=0.4\textwidth}]{2}% 设置为两栏排版

\begin{parcolumns}[colwidths={1=\parcolumnsLeft}]{2}% 设置为两栏排版
\colchunk{在你的内心深处,你一直知道你注定要学习
Clojure。每次你举起键盘,对难以理解的类层次结构感到痛苦;每次你在夜晚醒着,因为一个由突变引发的
Heisenbug
而令你的亲人痛苦;每次一个竞态条件使你拔出更多的你已经越来越少的头发,你内心的某个秘密部分总是知道{\tt 肯定有更好的办法}。}
\colchunk{Deep in your \csh一音标{Innermost}{/ˈɪnərmoʊst/}%:最内部的
\ \csh一音标{being}{/ˈbiːɪŋ/ 存在}, you've always known you were \csh一音标{destined}{/ˈdɛstɪnd/ 注定的} to
learn Clojure. Every time you held your keyboard \csh一音标{aloft}{/əˈlɔːft/ 在空中,向上}, crying out in
\csh一音标{Anguish}{/ˈæŋɡwɪʃ/:痛苦} over an \csh一音标{Incomprehensible}{/ˌɪnkɑːmprɪˈhɛnsɪbəl/:难以理解的} class \csh一音标{Hierarchy}{/ˈhaɪəˌrɑːrki/:层次结构}; every time you lay
awake at night, \csh一音标{Disturbing}{/dɪˈstɜːrbɪŋ/:令人不安的} your loved ones with \csh一音标{Sobs}{/sɑːbz/:抽泣} over a
\csh一音标{Mutation}{/mjuːˈteɪʃən/:突变}-induced \csh一音标{heisenbug}{/ˈhaɪzənbʌɡ/ 海森堡虫\footnotemark[1]}; every time a \csh一音标{race condition}{/reɪs kənˈdɪʃən/ 竞态条件\footnotemark[2]} caused you to
pull out more of your \csh一音标{everdwindling}{/ˈɛvər dwɪndlɪŋ/ 不断减少的} hair, some secret part of you has
known that \emph{there has to be a better way}.
% Deep in your innermost being, you've always known you were destined to
% learn Clojure. Every time you held your keyboard aloft, crying out in
% anguish over an incomprehensible class hierarchy; every time you lay
% awake at night, disturbing your loved ones with sobs over a
% mutation-induced heisenbug; every time a race condition caused you to
% pull out more of your ever-dwindling hair, some secret part of you has
% known that \emph{there has to be a better way}.
}
\footnotetext[1]{一种在尝试调试时会改变行为的计算机bug}%todo 后面去那边文档添加
\footnotetext[2]{计算机科学中关于资源争夺的问题}
\end{parcolumns}


\begin{parcolumns}[colwidths={1=\parcolumnsLeft}]{2}% 设置为两栏排版
\colchunk{现在,终于,你面前的这个教材将会将你与你一直渴望的编程语言结合在一起。}
\colchunk{Now, at long last, the \csh一音标{Instructional}{/ɪnˈstrʌkʃənəl/}%:教学的
\ \csh一音标{Material}{/məˈtɪriəl/} %:材料,资料
you have in front of your
face will \csh一音标{Unite}{/juːˈnaɪt/:联合,结合} you with the programming language you've been \csh一音标{Longing}{/ˈlɔːŋɪŋ/:渴望}
for.}
\end{parcolumns}


\section[学习新的编程语言:穿越四个迷宫的旅程]{学习新的编程语言:穿越四个迷宫的旅程\\Learning a New Programming Language: A Journey Through the Four Labyrinths}

\begin{parcolumns}[colwidths={1=\parcolumnsLeft}]{2}% 设置为两栏排版
\colchunk{为了最大限度地掌握 Clojure,你必须克服学习新语言所面临的四个挑战:}
\colchunk{To \csh一音标{Wield}{/wiːld/:掌握,使用} Clojure to its fullest, you'll need to find your way through
the four \csh一音标{Labyrinths}{/ˈlæbərɪnθs/:迷宫} that face every programmer learning a new language:}
\end{parcolumns}

\begin{parcolumns}[colwidths={1=\parcolumnsLeft}]{2}% 设置为两栏排版
\colchunk{
\begin{description}
    \item [工具森林] 一个友好且高效的编程环境使得尝试你的想法变得简单。你将会学习如何设置你的环境。
    \item[语言之山] 当你攀登这座山时,你会学会 Clojure
的语法、语义和数据结构的知识。你将会学习如何使用最强大的编程工具之一,宏,以及如何使用
Clojure 的并发构造来简化你的生活。
\item[人工制品之洞]
在其深处,你将学习构建、运行和分发你自己的程序,以及如何使用代码库。你还将了解
Clojure 与 Java 虚拟机(JVM)的关系。
\item[心态的云堡] 在其稀薄的空气中,你将会了解 Lisp
和函数式编程的意义和方法。你将Clojure 中渗透的简洁哲学,以及如何像一个
Clojure 程序员那样解决问题。
\end{description}    
}

\colchunk{
    \begin{description}
        \item [The Forest of Tooling] A friendly and \csh一音标{Efficient}{/ɪˈfɪʃənt/:高效的} programming
environment makes it easy to try your ideas. You'll learn how to set up
your environment.
        \item[The Mountain of Language] As you \csh一音标{Ascend}{/əˈsɛnd/:攀登}, you'll \csh一音标{Gain}{/ɡeɪn/:获取} knowledge
of Clojure's syntax, \csh一音标{Semantics}{/sɪˈmæntɪks/:语义}, and data structures. You'll learn how to
use one of the \csh一音标{Mightiest}{/ˈmaɪtɪəst/:最强大的} programming tools, the macro, and learn how to
simplify your life with Clojure's \csh一音标{Concurrency}{/kənˈkɜːrənsi/:并发} constructs.
\item[The Cave of Artifacts] In its depths you'll learn to build, run,
and \csh一音标{Distribute}{/dɪˈstrɪbjuːt/:分发} your own programs, and how to use code libraries. You'll
also learn Clojure's relationship to the Java Virtual Machine (JVM).
\item[The Cloud Castle of Mindset] In its \csh一音标{Rarefied}{/ˈrɛrɪfaɪd/:稀薄的} air, you'll come to
know the why and how of Lisp and functional programming. You'll learn
about the \csh一音标{Philosophy}{/fɪˈlɑːsəfi/:哲学} of simplicity that \csh一音标{Permeates}{/ˈpɜːrmiːts/:渗透} Clojure, and how to
solve problems like a Clojurist.
    \end{description}
}
\end{parcolumns}


这段话介绍了 Clojure 编程语言独特的心态方式。在 Clojure
中,程序员应该专注于\textbf{问题的本质},而不是关注实现细节。Clojure
也强调简洁和优雅的代码风格。

\begin{parcolumns}[colwidths={1=\parcolumnsLeft}]{2}% 设置为两栏排版
\colchunk{学习 Clojure
需要付出努力,但这本书将让学习过程充满乐趣,而不是枯燥乏味。这是因为这本书遵循以下三个准则:}
\colchunk{Make no mistake, you will work. But this book will make the work feel
\csh一音标{exhilarating}{/ɪgˈzɪləreɪtɪŋ/ 令人兴奋的}, not \csh一音标{exhausting}{/ɪgˈzɔːstɪŋ/ 使人疲惫的,耗尽的}. That's because this book follows three
\csh一音标{guidelines}{/ˈɡaɪdlaɪnz/ 准则}:}
\end{parcolumns}

\begin{parcolumns}[colwidths={1=\parcolumnsLeft}]{2}% 设置为两栏排版
\colchunk{
    \begin{itemize}
        \item 它采取了``甜品先行''的方法,为你提供开发工具和语言细节,使你能\textbf{立即开始实践真实的程序}。
        \item 它假定你对JVM、函数式编程或Lisp没有任何经验。它详细介绍了这些主题,所以当你构建和运行Clojure程序时,你会对自己正在做的事情感到自信。
        \item 它避免使用现实世界的例子,而选择更有趣的练习,例如攻击霍比特人和追踪闪闪发光的吸血鬼。
    \end{itemize}
}
\colchunk{
    \begin{itemize}
        \item It takes the \csh一音标{dessert}{/dɪˈzɜːrt/ 甜点}-first \csh一音标{approach}{/əˈprəʊtʃ/ 方法,方式}, giving you the development tools
        and language details you need to start playing with real programs
        immediately.
        \item It assumes zero \csh一音标{experience}{/ɪkˈspɪəriəns/ 经验} with the JVM, functional programming, or
Lisp. It covers these topics in detail so you'll feel \csh一音标{confident}{/ˈkɒnfɪdənt/ 自信的} about
what you're doing when you build and run Clojure programs.
\item It \csh一音标{eschews}{/ɪˈʃuːz/ 避开,放弃} \emph{real-world} examples in \csh一音标{favor}{/ˈfeɪvər/ 支持,喜爱} of more interesting
exercises, like \emph{\csh一音标{assaulting}{/əˈsɔːltɪŋ/ 攻击} hobbits} and \emph{tracking \csh一音标{glittery}{/ˈɡlɪtəri/ 闪闪发光的~/ˈvæmpaɪərz/ 吸血鬼}
vampires}.
    \end{itemize}
}
\end{parcolumns}


\begin{parcolumns}[colwidths={1=\parcolumnsLeft}]{2}% 设置为两栏排版
\colchunk{到最后,你将能够使用 Clojure,一种世界上最令人兴奋和有趣的编程语言!}
\colchunk{By the end, you'll be able to use Clojure, one of the most exciting and
fun programming languages in \csh一音标{existence}{/ɪɡˈzɪstəns/ 存在,现存}!}
\end{parcolumns}

\section[本书的组织结构]{本书的组织结构\\How This Book Is Organized}

\begin{parcolumns}[colwidths={1=\parcolumnsLeft}]{2}% 设置为两栏排版
\colchunk{为了更好地引导你这个勇敢的初出茅庐的Clojure学者完成你的崇高使命,本书被分为三部分。}
\colchunk{This book is split into three parts to better guide you through your
\csh一音标{valiant}{/ˈvæliənt/} \csh一音标{quest}{~/kwɛst/ 探索}, brave \csh一音标{fledgling}{/ˈflɛdʒlɪŋ/ 初出茅庐的} Clojurist.}
\end{parcolumns}


\subsection[第一部分:环境设置]{第一部分:环境设置\hfill Environment Setup}

\begin{parcolumns}[colwidths={1=\parcolumnsLeft}]{2}% 设置为两栏排版
\colchunk{为了保持动力并有效地学习,你需要实际编写代码并构建可执行文件。这些章节将带你快速浏览你需要的工具以便轻松编写程序。这样你就可以专注于学习Clojure,而不是把时间浪费在调整环境上。}
\colchunk{To stay \csh一音标{motivated}{/ˈmoʊtɪveɪtɪd/ 有动力的} and learn \csh一音标{efficiently}{/ɪˈfɪʃəntli/ 有效地}, you need to actually write code
and build \csh一音标{executables}{/ɪɡˈzɛkjutəbəlz/ 可执行文件}. These chapters take you on a quick \csh一音标{tour}{/tʊər/ 游览,导览} of the
tools you'll need to easily write programs. That way you can focus on
learning Clojure, not \csh一音标{fiddling}{/ˈfɪdlɪŋ/ 浪费时间,摆弄} with your environment.}
\end{parcolumns}


\textbf{第一章:构建,运行和REPL \hfill Chapter 1: Building, Running, and the REPL}

\begin{parcolumns}[colwidths={1=\parcolumnsLeft}]{2}% 设置为两栏排版
\colchunk{让一个真实的程序运行起来有着某种强大和激励人的力量。一旦你能做到这一点,你就可以自由地实验,并且你实际上可以分享你的工作!}
\colchunk{There's something powerful and \csh一音标{motivating}{/ˈmoʊtɪveɪtɪŋ/ 激励人的} about getting a real program
running. Once you can do that, you're free to \csh一音标{experiment}{/ɪkˈspɛrɪmənt/ 实验}, and you can
actually share your work!}
\end{parcolumns}

\begin{parcolumns}[colwidths={1=\parcolumnsLeft}]{2}% 设置为两栏排版
\colchunk{在这个短小的章节中,你将花费一小部分时间来熟悉一种快速构建和运行Clojure程序的方法。你将学习如何在正在运行的Clojure过程中使用读取-求值-打印循环(REPL)来实验代码。这将使你的反馈循环更加紧密,并帮助你更有效地学习。}
\colchunk{In this short chapter, you'll \csh一音标{invest}{/ɪnˈvɛst/ 投入,花费} a small amount of time to become
familiar with a quick way to build and run Clojure programs. You'll
learn how to experiment with code in a running Clojure process using a
\csh一音标{read-eval-print}{/ˈriːd ˌiːvæl ˈprɪnt/ 一种简单的,交互式的编程环境} loop (REPL). This will \csh一音标{tighten}{/ˈtaɪtən/ 紧缩} your feedback loop and
help you learn more efficiently.}
\end{parcolumns}


\textbf{第二章:如何使用Emacs,一款出色的Clojure编辑器\hfill Chapter 2: How to Use Emacs, an Excellent Clojure Editor}


\begin{parcolumns}[colwidths={1=\parcolumnsLeft}]{2}% 设置为两栏排版
\colchunk{快速的反馈循环对于学习至关重要。在这一章中,我从头开始介绍Emacs,以保证你有一个高效的Emacs/Clojure工作流。}
\colchunk{A quick feedback loop is \csh一音标{crucial}{/ˈkruːʃəl/ 至关重要的} for learning. In this chapter, I cover
Emacs from the \csh一音标{ground}{/ɡraʊnd/ 基础} up to \csh一音标{guarantee}{/ˌɡærənˈtiː/ 保证} you have an efficient
Emacs/Clojure \csh一音标{workflow}{/ˈwɜːrkfloʊ/ 工作流程}.}
\end{parcolumns}

%%%% todo 以下未标音标

\subsection[第二部分:语言基础]{第二部分:语言基础\hfill Part II: Language Fundamentals}

\begin{parcolumns}[colwidths={1=\parcolumnsLeft}]{2}% 设置为两栏排版
\colchunk{这些章节为你打下了坚实的基础,以便继续学习 Clojure。你将从学习 Clojure 的基础知识(语法、语义和数据结构)开始,这样你就可以“做事情”。然后,你将退一步,深入研究 Clojure 中最常用的函数,并学习如何以“函数式编程”思维方式使用它们解决问题。}
\colchunk{These chapters give you a solid foundation on which to continue learning Clojure. You'll start by learning Clojure's basics (syntax, semantics, and data structures) so you can \emph{do things}. Then you'll take a step back to examine Clojure's most used functions in detail and learn how to solve problems with them using the \emph{functional programming} mindset.}
\end{parcolumns}

\textbf{第三章:做事情:Clojure 简明教程\hfill Chapter 3: Do Things: A Clojure Crash Course}

\begin{parcolumns}[colwidths={1=\parcolumnsLeft}]{2}% 设置为两栏排版
\colchunk{这是你真正深入学习 Clojure 的地方。在这里,你需要关上窗户,因为你会大声喊叫:“哇哦,太棒了!”直到你翻到了书的索引页。}
\colchunk{This is where you'll start to really dig into Clojure. It's also where you'll need to close your windows because you'll start shouting, ``\emph{HOLY MOLEY THAT'S SPIFFY!}'' at the top of your lungs and won't stop until you've hit this book's index. }
\end{parcolumns}

\begin{parcolumns}[colwidths={1=\parcolumnsLeft}]{2}% 设置为两栏排版
\colchunk{无疑,你已经听说过 Clojure 强大的并发支持和其他惊人的特性,但 Clojure 最显著的特点是它是一种 Lisp 语言。你将探索这个 Lisp 核心,它由两部分组成:函数和数据。}
\colchunk{You've undoubtedly heard of Clojure's awesome concurrency support and other stupendous features, but Clojure's most salient characteristic is that it is a Lisp. You'll explore this Lisp core, which is composed of two parts: functions and data.}
\end{parcolumns}

\textbf{第四章:深入核心函数\hfill Chapter 4: Core Functions in Depth}

\begin{parcolumns}[colwidths={1=\parcolumnsLeft}]{2}% 设置为两栏排版
    \colchunk{在这一章中,你将学习一些 Clojure 的基本概念。这将使你能够阅读你以前未使用过的函数的文档,并理解在你尝试使用它们时发生的情况。}
    \colchunk{In this chapter, you'll learn about a couple of Clojure's underlying concepts. This will give you the grounding you need to read the documentation for functions you haven't used before and to understand what's happening when you try them. }
\end{parcolumns}

\begin{parcolumns}[colwidths={1=\parcolumnsLeft}]{2}% 设置为两栏排版
    \colchunk{你还将看到你最常用的函数的用法示例。这将为你编写自己的代码和阅读和学习他人项目打下坚实基础。还记得我提到过追踪闪闪发光的吸血鬼吗?你将在本章中做到这一点(除非你已经在业余时间做过了)。}
    \colchunk{You'll also see usage examples of the functions you'll be reaching for the most. This will give you a solid foundation for writing your own code and for reading and learning from other people's projects. And remember how I mentioned tracking glittery vampires? You'll do that in this chapter (unless you already do it in your spare time).}
\end{parcolumns}


\textbf{第五章:函数式编程\hfill Chapter 5: Functional Programming}

\begin{parcolumns}[colwidths={1=\parcolumnsLeft}]{2}% 设置为两栏排版
    \colchunk{在本章中,你将把你对函数和数据结构的具体经验与新的思维方式相结合:函数式编程思维方式。你将通过构建一款风靡全国的最新热门游戏“Peg Thing”来展示你的知识。}
    \colchunk{In this chapter, you’ll take your concrete experience with functions and data structures and integrate it with a new mindset: the functional programming mindset. You’ll show off your knowledge by constructing the hottest new game that’s sweeping the nation: Peg Thing!}
\end{parcolumns}

\textbf{第六章:组织你的项目:图书管理员的故事 \hfill Chapter 6: Organizing Your Project: A Librarian's Tale}

\begin{parcolumns}[colwidths={1=\parcolumnsLeft}]{2}% 设置为两栏排版
    \colchunk{本章讲解了命名空间是什么以及如何使用它们来组织你的代码。我不想透露太多,但它还涉及到一个国际奶酪小偷。}
    \colchunk{This chapter explains what namespaces are and how to use them to organize your code. I don't want to give away too much, but it also involves an international cheese thief.}
\end{parcolumns}

\textbf{第七章:Clojure 炼金术:读取、求值和宏\hfill Chapter 7: Clojure Alchemy: Reading, Evaluation, and Macros}

\begin{parcolumns}[colwidths={1=\parcolumnsLeft}]{2}% 设置为两栏排版
    \colchunk{在本章中,我们将回顾一下 Clojure 如何运行你的代码。这将使你真正理解 Clojure 的工作原理以及它与其他非 Lisp 语言的区别。在建立了这个结构之后,我将介绍宏,这是一种最强大的工具之一。}
    \colchunk{In this chapter, we'll take a step back and describe how Clojure runs your code. This will give you the conceptual structure you need to truly understand how Clojure works and how it's different from other, non-Lisp languages. With this structure in place, I'll introduce the macro, one of the most powerful tools in existence.}
\end{parcolumns}


\textbf{第八章:编写宏\hfill Chapter 8: Writing Macros}

\begin{parcolumns}[colwidths={1=\parcolumnsLeft}]{2}% 设置为两栏排版
    \colchunk{本章详细讨论了如何编写宏,从基本示例到复杂示例逐步深入。最后,你将戴上想象的帽子,假装你经营一家在线魔药商店,并使用宏来验证客户订单。}
    \colchunk{This chapter thoroughly examines how to write macros, starting with basic examples and advancing to more complex ones. You'll close by donning your make-believe cap, pretending that you run an online potion store and using macros to validate customer orders.}
\end{parcolumns}

\subsection{Part III: 高级主题\\Part III: Advanced Topic}

\begin{parcolumns}[colwidths={1=\parcolumnsLeft}]{2}% 设置为两栏排版
    \colchunk{这些章节涵盖了 Clojure 的一些有趣的主题:并发、Java 互操作性和抽象。虽然你可以在不理解这些工具和概念的情况下编写程序,但它们在智力上具有回报,并且作为程序员,它们赋予你巨大的力量。人们说学习 Clojure 使你成为一个更好的程序员的原因之一就是它使这些章节涉及的概念易于理解并且实用。}
    \colchunk{These chapters cover Clojure's extra-fun topics: concurrency, Java interop, and abstraction. Although you can write programs without understanding these tools and concepts, they're intellectually rewarding and give you tremendous power as a programmer. One of the reasons people say that learning Clojure makes you a better programmer is that it makes the concepts covered in these chapters easy to understand and practical to use.}
\end{parcolumns}


\textbf{第九章:并发和并行编程的神圣艺术\hfill Chapter 9: The Sacred Art of Concurrent and Parallel Programming}

\begin{parcolumns}[colwidths={1=\parcolumnsLeft}]{2}% 设置为两栏排版
\colchunk{在本章中,您将学习并发和并行的概念,以及它们为什么重要。您将了解在编写并行程序时将面临的挑战,以及Clojure的设计如何帮助减轻这些挑战。您将使用future、delay和promise安全地编写并行程序。}
\colchunk{In this chapter, you'll learn what concurrency and parallelism are and
why they matter. You'll learn about the challenges you'll face when
writing parallel programs and about how Clojure's design helps to
mitigate them. You'll use futures, delays, and promises to safely write
parallel programs.}
\end{parcolumns}

\textbf{第十章:Clojure 的形而上学:Atoms、Refs、Vars 和 Cuddle Zombies\hfill Chapter 10: Clojure Metaphysics: Atoms, Refs, Vars, and Cuddle Zombies}


\begin{parcolumns}[colwidths={1=\parcolumnsLeft}]{2}% 设置为两栏排版
    \colchunk{本章将详细介绍 Clojure 处理状态的方法以及如何简化并发编程。你将学习如何使用 atoms、refs 和 vars 这三种构造来管理状态,并学习如何使用 pmap 进行无状态的并行计算。当然,还会有可爱的拥抱僵尸。}
    \colchunk{This chapter goes into great detail about Clojure's approach to managing state and how that simplifies concurrent programming. You'll learn how to use atoms, refs, and vars, three constructs for managing state, and you'll learn how to do stateless parallel computation with pmap. And there will be cuddle zombies.}
\end{parcolumns}

\textbf{第十一章:用 core.async 掌握并发进程\hfill Chapter 11: Mastering Concurrent Processes with core.async}

\begin{parcolumns}[colwidths={1=\parcolumnsLeft}]{2}% 设置为两栏排版
    \colchunk{在本章中,你将思考这样一个观点:宇宙中的一切都是热狗自动售货机。也就是说,你将学习如何使用 core.async 库,以通道为基础建模独立运行的进程系统,并通过通道进行通信。}
    \colchunk{In this chapter, you'll ponder the idea that everything in the universe is a hot dog vending machine. By which I mean you'll learn how to model systems of independently running processes that communicate with each other over channels using the core.async library.}
\end{parcolumns}

\textbf{第十二章:与 JVM 一起工作\hfill Chapter 12: Working with the JVM}

\begin{parcolumns}[colwidths={1=\parcolumnsLeft}]{2}% 设置为两栏排版
    \colchunk{本章是一个短语书和对 Java 之地文化的介绍的结合体。它给出了 JVM 是什么,它如何运行程序以及如何为其编译程序的概述。它还简要介绍了经常使用的 Java 类和方法,并解释了如何从 Clojure 与之交互。更重要的是,它向你展示了如何思考和理解 Java,以便将任何 Java 库整合到你的 Clojure 程序中。}
    \colchunk{This chapter is like a cross between a phrasebook and cultural introduction to the Land of Java. It gives you an overview of what the JVM is, how it runs programs, and how to compile programs for it. It also gives you a brief tour of frequently used Java classes and methods, and explains how to interact with them from Clojure. More than that, it shows you how to think about and understand Java so you can incorporate any Java library into your Clojure program.}
\end{parcolumns}

\textbf{第十三章:通过 Multimethods、Protocols 和 Records 创建和扩展抽象\hfill Chapter 13: Creating and Extending Abstractions with Multimethods, Protocols, and Records}

\begin{parcolumns}[colwidths={1=\parcolumnsLeft}]{2}% 设置为两栏排版
    \colchunk{在第四章中,你了解到 Clojure 是以抽象为基础进行编写的。本章是创建和实现自己的抽象世界的介绍。你将学习多方法、协议和记录的基础知识。}
    \colchunk{In Chapter 4, you learn that Clojure is written in terms of abstractions. This chapter serves as an introduction to the world of creating and implementing your own abstractions. You'll learn the basics of multimethods, protocols, and records.}
\end{parcolumns}

\textbf{附录 A:使用 Leiningen 构建和开发\hfill Appendix A: Building and Developing with Leiningen}

\begin{parcolumns}[colwidths={1=\parcolumnsLeft}]{2}% 设置为两栏排版
    \colchunk{本附录澄清了使用 Leiningen 进行工作的一些细节,比如 Maven 是什么以及如何确定 Java 库版本号以便使用它们。}
    \colchunk{This appendix clarifies some of the finer points of working with Leiningen, like what Maven is and how to figure out the version numbers of Java libraries so that you can use them.}
\end{parcolumns}

\textbf{附录 B:Boot,高级 Clojure 构建框架\hfill  Appendix B: Boot, the Fancy Clojure Build Framework}

\begin{parcolumns}[colwidths={1=\parcolumnsLeft}]{2}% 设置为两栏排版
    \colchunk{Boot 是 Leiningen 的一种替代方案,提供相同的功能,但额外提供了更易于扩展和编写可组合任务的优点。本附录解释了 Boot 的基本概念,并指导你编写第一个任务。}
    \colchunk{Boot is an alternative to Leiningen that provides the same functionality, but with the added bonus that it's easier to extend and write composable tasks. This appendix explains Boot's underlying concepts and guides you through writing your first tasks.}
\end{parcolumns}


\section{代码\hfill The Code}

\begin{parcolumns}[colwidths={1=\parcolumnsLeft}]{2}% 设置为两栏排版
    \colchunk{你可以从书上的网站下载所有源代码,网址为 \emph{http://www.nostarch.com/clojure/}。代码按章节进行组织。}
    \colchunk{You can download all the source code from the book at \emph{http://www.nostarch.com/clojure/}. The code is organized by chapter.}
\end{parcolumns}

\begin{parcolumns}[colwidths={1=\parcolumnsLeft}]{2}% 设置为两栏排版
    \colchunk{第一章描述了运行 Clojure 代码的不同方式,包括如何使用 REPL。我建议你在遇到示例时使用 REPL 运行大多数示例,特别是在第三章到第八章之间。这将帮助你习惯编写和理解 Lisp 代码,并帮助你记住你所学到的一切。但对于较长的示例,最好将代码写入文件,然后在 REPL 中运行你编写的代码。}
    \colchunk{Chapter 1 describes the different ways that you can run Clojure code, including how to use a REPL. I recommend running most of the examples in the REPL as you encounter them, especially in Chapters 3 through 8. This will help you get used to writing and understanding Lisp code, and it will help you retain everything you're learning. But for the examples that are long, it's best to write your code to a file, and then run the code you wrote in a REPL.}
\end{parcolumns}

\section{启程!\hfill The Journey Begins!}

\begin{parcolumns}[colwidths={1=\parcolumnsLeft}]{2}% 设置为两栏排版
    \colchunk{准备好了吗,勇敢的读者?准备好迎接真正的命运了吗?拿出你最好的括号对吧:你即将踏上终身的旅程!}
    \colchunk{Are you ready, brave reader? Are you ready to meet your true destiny? Grab your best pair of parentheses: you're about to embark on the journey of a lifetime!}
\end{parcolumns} 
%%%%%%%%%%%%%%%%%%%%%%%%%%%%
 

\end{document}



% fontGentiumPlus

% \begin{parcolumns}[colwidths={1=\parcolumnsLeft}]{2}% 设置为两栏排版
% \colchunk{左侧文本}
% \colchunk{右侧文本}
% \end{parcolumns}


% \begin{parcolumns}[colwidths={1=\parcolumnsLeft}]{2}% 设置为两栏排版
% \colchunk{左侧文本}
% \colchunk{右侧文本}
% \end{parcolumns}


% \begin{parcolumns}[colwidths={1=\parcolumnsLeft}]{2}% 设置为两栏排版
% \colchunk{左侧文本}
% \colchunk{右侧文本}
% \end{parcolumns}


% \begin{parcolumns}[colwidths={1=\parcolumnsLeft}]{2}% 设置为两栏排版
% \colchunk{左侧文本}
% \colchunk{右侧文本}
% \end{parcolumns}


% \begin{parcolumns}[colwidths={1=\parcolumnsLeft}]{2}% 设置为两栏排版
% \colchunk{左侧文本}
% \colchunk{右侧文本}
% \end{parcolumns}


% \begin{parcolumns}[colwidths={1=\parcolumnsLeft}]{2}% 设置为两栏排版
% \colchunk{左侧文本}
% \colchunk{右侧文本}
% \end{parcolumns}


% \begin{parcolumns}[colwidths={1=\parcolumnsLeft}]{2}% 设置为两栏排版
% \colchunk{左侧文本}
% \colchunk{右侧文本}
% \end{parcolumns}


% \begin{parcolumns}[colwidths={1=\parcolumnsLeft}]{2}% 设置为两栏排版
% \colchunk{左侧文本}
% \colchunk{右侧文本}
% \end{parcolumns}


% \begin{parcolumns}[colwidths={1=\parcolumnsLeft}]{2}% 设置为两栏排版
% \colchunk{左侧文本}
% \colchunk{右侧文本}
% \end{parcolumns}


% \begin{parcolumns}[colwidths={1=\parcolumnsLeft}]{2}% 设置为两栏排版
% \colchunk{左侧文本}
% \colchunk{右侧文本}
% \end{parcolumns}


% \begin{parcolumns}[colwidths={1=\parcolumnsLeft}]{2}% 设置为两栏排版
% \colchunk{左侧文本}
% \colchunk{右侧文本}
% \end{parcolumns}


% \begin{parcolumns}[colwidths={1=\parcolumnsLeft}]{2}% 设置为两栏排版
% \colchunk{左侧文本}
% \colchunk{右侧文本}
% \end{parcolumns}



%mypdf 我的学习笔记_XeLaTeX

% 我的学习笔记_Clojure.tex

% 在 parcolumns 环境中,可以使用 \parchunk1宏在两个或多个列中并列地排 版文本。也可以包括普通文本。

% \colchunk[2]{...}

% 将左侧列稍 微放大:colwidths={1=.55\linewidth}

% 三列,选项nofirstindent=true:
% \begin{parcolumns}[nofirstindent]{3}