\PassOptionsToPackage{no-math}{fontspec}%禁用了使用fontspec宏包中的数学字体功能。
\PassOptionsToPackage{AutoFakeBold=true,AutoFakeSlant=true}{xeCJK}%让xeCJK宏包自动产生伪粗体和伪斜体效果。

\documentclass[a4paper,landscape]{book}
\usepackage[heading=true
,scheme=chinese%中文方案
,fontset=none%不使用默认的字体设置
,space=auto%自动调整中英文间距
]{ctex}
\setCJKmainfont{FangZhengShuSong-GBK-1.ttf}[Path=/Users/virhuiai/hlProjects/Latex-Typesetting-Hub/font/方正/]%设置文本的中文有衬线字体
\setCJKsansfont{FangZhengHeiTi-GBK-1.ttf}[Path=/Users/virhuiai/hlProjects/Latex-Typesetting-Hub/font/方正/]%设置文本的中文无衬线字体为
\setCJKmonofont{FangZhengFangSong-GBK-1.ttf}[Path=/Users/virhuiai/hlProjects/Latex-Typesetting-Hub/font/方正/] %设置文本的中文等宽字体 
% \setCJKfamilyfont{fontKai}{LXGWWenKai-Regular.ttf}[Path=/Users/virhuiai/hlProjects/Latex-Typesetting-Hub/font/霞鹜文楷/]
\setCJKfamilyfont{fontKai}{FangZhengKaiTi-GBK-1.ttf}[Path=/Users/virhuiai/hlProjects/Latex-Typesetting-Hub/font/方正/]
\newcommand\fontKai{\CJKfamily{fontKai}}


\usepackage[left=15mm,right=15mm,top=15mm,bottom=15mm]{geometry}
\usepackage{parskip}

\usepackage[all]{tcolorbox}
\usepackage{paralist}
\usepackage{pdfcolparcolumns}
\usepackage{zebra-goodies}

\usepackage{tikz}
\usetikzlibrary{tikzmark}
\usepackage{xspace}
\begin{document}

% 定义打开和关闭设置临时计数器命令的功能
\newcommand\counterSetTmpOpen{\providecommand{\counterSetTmp}{}\renewcommand\counterSetTmp[2]{\setcounter{##1}{##2}}}
\newcommand\counterSetTmpClose{\providecommand{\counterSetTmp}{}\renewcommand\counterSetTmp[2]{}}
\counterSetTmpOpen
% \counterSetTmpClose

%%%%%%%%%%%%%%%%%%%%%%%%%%%%%%%%%%%%%%%%%%%%%%%%%%%%%%%%%%%%%%%%%%%%%%%%
% \newcounter{Emp}[chapter]
% \renewcommand{\theEmp}{\thechapter.\arabic{Emp}}
% \newcommand{\EX}{\par%
% {\bf 例~}%
% \refstepcounter{Emp}{\bf\theEmp}\hspace{1em}}


% \input{我的学习笔记_Clojure/

% CLOJURE_FOR_THE_BRAVE_AND_TRUE

\parindent=0pt

% 支持音标的字体
\newfontfamily\fontGentiumPlus{GentiumPlus}[Path=/Users/virhuiai/hlProjects/Latex-Typesetting-Hub/font/免费商用英文/支持音标-GentiumPlus-6.200/,
Extension=.ttf,
UprightFont=*-Regular ,
BoldFont=*-Bold ,
ItalicFont=*-Italic,
BoldItalicFont = *-BoldItalic
]
\newfontfamily\fontGentiumBookPlus{GentiumBookPlus}[Path=/Users/virhuiai/hlProjects/Latex-Typesetting-Hub/font/免费商用英文/支持音标-GentiumPlus-6.200/,
Extension=.ttf,
UprightFont=*-Regular ,
BoldFont=*-Bold ,
ItalicFont=*-Italic,
BoldItalicFont = *-BoldItalic
]

%%%%%%%%%%%%%%%%%%%%%%%%% csh一音标/BEGIN %%%%%%%%%%%%%%%%%%%%%%%%%
\newcounter{csh一音标一计数器}
\newlength{\csh一音标一单词宽度}
\newlength{\csh一音标一单词高度}
\newcommand{\csh一音标}[2]{%
\tikzmarknode{#1\thecsh一音标一计数器}{\MakeLowercase{#1}\xspace}%
\begin{tikzpicture}[remember picture,overlay]%
\node [red,anchor=south west] at ([yshift=-0.45em] #1\thecsh一音标一计数器.north west) {\hspace{-1ex}\small\fontGentiumPlus#2};
\end{tikzpicture}\stepcounter{csh一音标一计数器}}
%%%%%%%%%%%%%%%%%%%%%%%%% csh一音标/END %%%%%%%%%%%%%%%%%%%%%%%%%

\newcommand\parcolumnsLeft{0.4\textwidth}
% \begin{parcolumns}[colwidths={1=0.4\textwidth}]{2}% 设置为两栏排版

\chapter{Introduction}
% \begin{parcolumns}[colwidths={1=\parcolumnsLeft}]{2}% 设置为两栏排版
\colchunk{在你的内心深处,你一直知道你注定要学习
Clojure。每次你举起键盘,对难以理解的类层次结构感到痛苦;每次你在夜晚醒着,因为一个由突变引发的
Heisenbug
而令你的亲人痛苦;每次一个竞态条件使你拔出更多的你已经越来越少的头发,你内心的某个秘密部分总是知道{\tt 肯定有更好的办法}。}
\colchunk{Deep in your \csh一音标{Innermost}{/ˈɪnərmoʊst/}%:最内部的
\ \csh一音标{being}{/ˈbiːɪŋ/ 存在}, you've always known you were \csh一音标{destined}{/ˈdɛstɪnd/ 注定的} to
learn Clojure. Every time you held your keyboard \csh一音标{aloft}{/əˈlɔːft/ 在空中,向上}, crying out in
\csh一音标{Anguish}{/ˈæŋɡwɪʃ/:痛苦} over an \csh一音标{Incomprehensible}{/ˌɪnkɑːmprɪˈhɛnsɪbəl/:难以理解的} class \csh一音标{Hierarchy}{/ˈhaɪəˌrɑːrki/:层次结构}; every time you lay
awake at night, \csh一音标{Disturbing}{/dɪˈstɜːrbɪŋ/:令人不安的} your loved ones with \csh一音标{Sobs}{/sɑːbz/:抽泣} over a
\csh一音标{Mutation}{/mjuːˈteɪʃən/:突变}-induced \csh一音标{heisenbug}{/ˈhaɪzənbʌɡ/ 海森堡虫\footnotemark[1]}; every time a \csh一音标{race condition}{/reɪs kənˈdɪʃən/ 竞态条件\footnotemark[2]} caused you to
pull out more of your \csh一音标{everdwindling}{/ˈɛvər dwɪndlɪŋ/ 不断减少的} hair, some secret part of you has
known that \emph{there has to be a better way}.
% Deep in your innermost being, you've always known you were destined to
% learn Clojure. Every time you held your keyboard aloft, crying out in
% anguish over an incomprehensible class hierarchy; every time you lay
% awake at night, disturbing your loved ones with sobs over a
% mutation-induced heisenbug; every time a race condition caused you to
% pull out more of your ever-dwindling hair, some secret part of you has
% known that \emph{there has to be a better way}.
}
\footnotetext[1]{一种在尝试调试时会改变行为的计算机bug}%todo 后面去那边文档添加
\footnotetext[2]{计算机科学中关于资源争夺的问题}
\end{parcolumns}


\begin{parcolumns}[colwidths={1=\parcolumnsLeft}]{2}% 设置为两栏排版
\colchunk{现在,终于,你面前的这个教材将会将你与你一直渴望的编程语言结合在一起。}
\colchunk{Now, at long last, the \csh一音标{Instructional}{/ɪnˈstrʌkʃənəl/}%:教学的
\ \csh一音标{Material}{/məˈtɪriəl/} %:材料,资料
you have in front of your
face will \csh一音标{Unite}{/juːˈnaɪt/:联合,结合} you with the programming language you've been \csh一音标{Longing}{/ˈlɔːŋɪŋ/:渴望}
for.}
\end{parcolumns}

 



% \section[学习新的编程语言:穿越四个迷宫的旅程]{学习新的编程语言:穿越四个迷宫的旅程\\Learning a New Programming Language: A Journey Through the Four Labyrinths}

% \begin{parcolumns}[colwidths={1=\parcolumnsLeft}]{2}% 设置为两栏排版
% \colchunk{为了最大限度地掌握 Clojure,你必须克服学习新语言所面临的四个挑战:}
% \colchunk{To \csh一音标{Wield}{/wiːld/:掌握,使用} Clojure to its fullest, you'll need to find your way through
% the four \csh一音标{Labyrinths}{/ˈlæbərɪnθs/:迷宫} that face every programmer learning a new language:}
% \end{parcolumns}



\begin{parcolumns}[colwidths={1=\parcolumnsLeft}]{2}% 设置为两栏排版
\colchunk{
\begin{description}
    \item [工具森林] 一个友好且高效的编程环境使得尝试你的想法变得简单。你将会学习如何设置你的环境。
    \item[语言之山] 当你攀登这座山时,你会学会 Clojure
的语法、语义和数据结构的知识。你将会学习如何使用最强大的编程工具之一,宏,以及如何使用
Clojure 的并发构造来简化你的生活。
\item[人工制品之洞]
在其深处,你将学习构建、运行和分发你自己的程序,以及如何使用代码库。你还将了解
Clojure 与 Java 虚拟机(JVM)的关系。
\item[心态的云堡] 在其稀薄的空气中,你将会了解 Lisp
和函数式编程的意义和方法。你将Clojure 中渗透的简洁哲学,以及如何像一个
Clojure 程序员那样解决问题。
\end{description}    
}

\colchunk{
    \begin{description}
        \item [The Forest of Tooling] A friendly and \csh一音标{Efficient}{/ɪˈfɪʃənt/:高效的} programming
environment makes it easy to try your ideas. You'll learn how to set up
your environment.
        \item[The Mountain of Language] As you \csh一音标{Ascend}{/əˈsɛnd/:攀登}, you'll \csh一音标{Gain}{/ɡeɪn/:获取} knowledge
of Clojure's syntax, \csh一音标{Semantics}{/sɪˈmæntɪks/:语义}, and data structures. You'll learn how to
use one of the \csh一音标{Mightiest}{/ˈmaɪtɪəst/:最强大的} programming tools, the macro, and learn how to
simplify your life with Clojure's \csh一音标{Concurrency}{/kənˈkɜːrənsi/:并发} constructs.
\item[The Cave of Artifacts] In its depths you'll learn to build, run,
and \csh一音标{Distribute}{/dɪˈstrɪbjuːt/:分发} your own programs, and how to use code libraries. You'll
also learn Clojure's relationship to the Java Virtual Machine (JVM).
\item[The Cloud Castle of Mindset] In its \csh一音标{Rarefied}{/ˈrɛrɪfaɪd/:稀薄的} air, you'll come to
know the why and how of Lisp and functional programming. You'll learn
about the \csh一音标{Philosophy}{/fɪˈlɑːsəfi/:哲学} of simplicity that \csh一音标{Permeates}{/ˈpɜːrmiːts/:渗透} Clojure, and how to
solve problems like a Clojurist.
    \end{description}
}
\end{parcolumns}


\end{document}

% fontGentiumPlus




% \begin{parcolumns}[colwidths={1=\parcolumnsLeft}]{2}% 设置为两栏排版
% \colchunk{左侧文本}
% \colchunk{右侧文本}
% \end{parcolumns}


% \begin{parcolumns}[colwidths={1=\parcolumnsLeft}]{2}% 设置为两栏排版
% \colchunk{左侧文本}
% \colchunk{右侧文本}
% \end{parcolumns}


% \begin{parcolumns}[colwidths={1=\parcolumnsLeft}]{2}% 设置为两栏排版
% \colchunk{左侧文本}
% \colchunk{右侧文本}
% \end{parcolumns}


% \begin{parcolumns}[colwidths={1=\parcolumnsLeft}]{2}% 设置为两栏排版
% \colchunk{左侧文本}
% \colchunk{右侧文本}
% \end{parcolumns}


% \begin{parcolumns}[colwidths={1=\parcolumnsLeft}]{2}% 设置为两栏排版
% \colchunk{左侧文本}
% \colchunk{右侧文本}
% \end{parcolumns}


% \begin{parcolumns}[colwidths={1=\parcolumnsLeft}]{2}% 设置为两栏排版
% \colchunk{左侧文本}
% \colchunk{右侧文本}
% \end{parcolumns}


% \begin{parcolumns}[colwidths={1=\parcolumnsLeft}]{2}% 设置为两栏排版
% \colchunk{左侧文本}
% \colchunk{右侧文本}
% \end{parcolumns}


% \begin{parcolumns}[colwidths={1=\parcolumnsLeft}]{2}% 设置为两栏排版
% \colchunk{左侧文本}
% \colchunk{右侧文本}
% \end{parcolumns}


% \begin{parcolumns}[colwidths={1=\parcolumnsLeft}]{2}% 设置为两栏排版
% \colchunk{左侧文本}
% \colchunk{右侧文本}
% \end{parcolumns}


% \begin{parcolumns}[colwidths={1=\parcolumnsLeft}]{2}% 设置为两栏排版
% \colchunk{左侧文本}
% \colchunk{右侧文本}
% \end{parcolumns}


% \begin{parcolumns}[colwidths={1=\parcolumnsLeft}]{2}% 设置为两栏排版
% \colchunk{左侧文本}
% \colchunk{右侧文本}
% \end{parcolumns}


% \begin{parcolumns}[colwidths={1=\parcolumnsLeft}]{2}% 设置为两栏排版
% \colchunk{左侧文本}
% \colchunk{右侧文本}
% \end{parcolumns}


% \begin{parcolumns}[colwidths={1=\parcolumnsLeft}]{2}% 设置为两栏排版
% \colchunk{左侧文本}
% \colchunk{右侧文本}
% \end{parcolumns}


% \begin{parcolumns}[colwidths={1=\parcolumnsLeft}]{2}% 设置为两栏排版
% \colchunk{左侧文本}
% \colchunk{右侧文本}
% \end{parcolumns}



%mypdf 我的学习笔记_XeLaTeX

% 我的学习笔记_Clojure.tex

% 在 parcolumns 环境中,可以使用 \parchunk1宏在两个或多个列中并列地排 版文本。也可以包括普通文本。

% \colchunk[2]{...}

% 将左侧列稍 微放大:colwidths={1=.55\linewidth}

% 三列,选项nofirstindent=true:
% \begin{parcolumns}[nofirstindent]{3}