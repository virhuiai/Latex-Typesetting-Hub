\columnratio{0.55}
\begin{paracol}{2}
\switchcolumn[0]*%%%%%%%
\section{Fallthrough Attributes}
\switchcolumn
\section{透传 Attributes}
\switchcolumn[0]*%%%%%%%
\begin{quote}
This page assumes you've already read the
\href{https://vuejs.org/guide/essentials/component-basics.html}{Components
Basics}. Read that first if you are new to components.
\end{quote}
\switchcolumn
\begin{quote}
此章节假设你已经看过了\href{https://cn.vuejs.org/guide/essentials/component-basics.html}{组件基础}。若你还不了解组件是什么,请先阅读该章节。
\end{quote}
\switchcolumn[0]*%%%%%%%
\subsection{Attribute Inheritance}
\switchcolumn
\subsection{Attributes 继承}
\switchcolumn[0]*%%%%%%%
A "fallthrough attribute" is an attribute or \texttt{v-on} event
listener that is passed to a component, but is not explicitly declared
in the receiving component's
\href{https://vuejs.org/guide/components/props.html}{props} or
\href{https://vuejs.org/guide/components/events.html\#declaring-emitted-events}{emits}.
Common examples of this include \texttt{class}, \texttt{style}, and
\texttt{id} attributes.
\switchcolumn
``透传 attribute''指的是传递给一个组件,却没有被该组件声明为
\href{https://cn.vuejs.org/guide/components/props.html}{props} 或
\href{https://cn.vuejs.org/guide/components/events.html\#defining-custom-events}{emits}
的 attribute 或者 \texttt{v-on} 事件监听器。最常见的例子就是
\texttt{class}、\texttt{style} 和 \texttt{id}。
\switchcolumn[0]*%%%%%%%
When a component renders a single root element, fallthrough attributes
will be automatically added to the root element's attributes. For
example, given a \texttt{\textless{}MyButton\textgreater{}} component
with the following template:
\switchcolumn
当一个组件以单个元素为根作渲染时,透传的 attribute
会自动被添加到根元素上。举例来说,假如我们有一个
\texttt{\textless{}MyButton\textgreater{}} 组件,它的模板长这样:
\switchcolumn[0]*%%%%%%%
\begin{codeHtml}
<!-- <MyButton> 的模板 -->
<button>click me</button>
\end{codeHtml}
\switchcolumn
\begin{codeHtml}
<!-- <MyButton> 的模板 -->
<button>click me</button>
\end{codeHtml}
\switchcolumn[0]*%%%%%%%
And a parent using this component with:
\switchcolumn
一个父组件使用了这个组件,并且传入了 \texttt{class}:
\switchcolumn[0]*%%%%%%%
\begin{codeHtml}
<MyButton class="large" />
\end{codeHtml}
\switchcolumn
\begin{codeHtml}
<MyButton class="large" />
\end{codeHtml}
\switchcolumn[0]*%%%%%%%
The final rendered DOM would be:
\switchcolumn
最后渲染出的 DOM 结果是:
\switchcolumn[0]*%%%%%%%
\begin{codeHtml}
<button class="large">click me</button>
\end{codeHtml}
\switchcolumn
\begin{codeHtml}
<button class="large">click me</button>
\end{codeHtml}
\switchcolumn[0]*%%%%%%%
Here, \texttt{\textless{}MyButton\textgreater{}} did not declare
\texttt{class} as an accepted prop. Therefore, \texttt{class} is treated
as a fallthrough attribute and automatically added to
\texttt{\textless{}MyButton\textgreater{}}'s root element.
\switchcolumn
这里,\texttt{\textless{}MyButton\textgreater{}} 并没有将 \texttt{class}
声明为一个它所接受的 prop,所以 \texttt{class} 被视作透传
attribute,自动透传到了 \texttt{\textless{}MyButton\textgreater{}}
的根元素上。
\switchcolumn[0]*%%%%%%%
\subsubsection{class and style Merging}
\switchcolumn
\subsubsection{对 class 和 style 的合并}
\switchcolumn[0]*%%%%%%%
If the child component's root element already has existing
\texttt{class} or \texttt{style} attributes, it will be merged with the
\texttt{class} and \texttt{style} values that are inherited from the
parent. Suppose we change the template of
\texttt{\textless{}MyButton\textgreater{}} in the previous example to:
\switchcolumn
如果一个子组件的根元素已经有了 \texttt{class} 或 \texttt{style}
attribute,它会和从父组件上继承的值合并。如果我们将之前的
\texttt{\textless{}MyButton\textgreater{}} 组件的模板改成这样:
\switchcolumn[0]*%%%%%%%
\begin{codeHtml}
<!-- <MyButton> 的模板 -->
<button class="btn">click me</button>
\end{codeHtml}
\switchcolumn
\begin{codeHtml}
<!-- <MyButton> 的模板 -->
<button class="btn">click me</button>
\end{codeHtml}
\switchcolumn[0]*%%%%%%%
Then the final rendered DOM would now become:
\switchcolumn
则最后渲染出的 DOM 结果会变成:
\switchcolumn[0]*%%%%%%%
\begin{codeHtml}
<button class="btn large">click me</button>
\end{codeHtml}
\switchcolumn
\begin{codeHtml}
<button class="btn large">click me</button>
\end{codeHtml}
\end{paracol}

\columnratio{0.55}
\begin{paracol}{2}

\switchcolumn[0]*%%%%%%%
\subsubsection{v-on Listener Inheritance}
\switchcolumn
\subsubsection{v-on 监听器继承}
\switchcolumn[0]*%%%%%%%
The same rule applies to \texttt{v-on} event listeners:
\switchcolumn
同样的规则也适用于 \texttt{v-on} 事件监听器:
\switchcolumn[0]*%%%%%%%
\begin{codeHtml}
<MyButton @click="onClick" />
\end{codeHtml}
\switchcolumn
\begin{codeHtml}
<MyButton @click="onClick" />
\end{codeHtml}
\switchcolumn[0]*%%%%%%%
The \texttt{click} listener will be added to the root element of
\texttt{\textless{}MyButton\textgreater{}}, i.e. the native
\texttt{\textless{}button\textgreater{}} element. When the native
\texttt{\textless{}button\textgreater{}} is clicked, it will trigger the
\texttt{onClick} method of the parent component. If the native
\texttt{\textless{}button\textgreater{}} already has a \texttt{click}
listener bound with \texttt{v-on}, then both listeners will trigger.
\switchcolumn
\texttt{click} 监听器会被添加到
\texttt{\textless{}MyButton\textgreater{}} 的根元素,即那个原生的
\texttt{\textless{}button\textgreater{}} 元素之上。当原生的
\texttt{\textless{}button\textgreater{}} 被点击,会触发父组件的
\texttt{onClick} 方法。同样的,如果原生 \texttt{button} 元素自身也通过
\texttt{v-on}
绑定了一个事件监听器,则这个监听器和从父组件继承的监听器都会被触发。
\end{paracol}


\columnratio{0.55}
\begin{paracol}{2}
\switchcolumn[0]*%%%%%%%
\subsubsection{Nested Component Inheritance}
\switchcolumn
\subsubsection{深层组件继承}
\switchcolumn[0]*%%%%%%%
If a component renders another component as its root node, for example,
we refactored \texttt{\textless{}MyButton\textgreater{}} to render a
\texttt{\textless{}BaseButton\textgreater{}} as its root:
\switchcolumn
有些情况下一个组件会在根节点上渲染另一个组件。例如,我们重构一下
\texttt{\textless{}MyButton\textgreater{}},让它在根节点上渲染
\texttt{\textless{}BaseButton\textgreater{}}:
\switchcolumn[0]*%%%%%%%
\begin{codeHtml}
<!-- <MyButton/> 的模板,只是渲染另一个组件 -->
<BaseButton />
\end{codeHtml}
\switchcolumn
\begin{codeHtml}
<!-- <MyButton/> 的模板,只是渲染另一个组件 -->
<BaseButton />
\end{codeHtml}
\switchcolumn[0]*%%%%%%%
Then the fallthrough attributes received by
\texttt{\textless{}MyButton\textgreater{}} will be automatically
forwarded to \texttt{\textless{}BaseButton\textgreater{}}.
\switchcolumn
此时 \texttt{\textless{}MyButton\textgreater{}} 接收的透传 attribute
会直接继续传给 \texttt{\textless{}BaseButton\textgreater{}}。
\switchcolumn[0]*%%%%%%%
Note that:
\switchcolumn
请注意:
\switchcolumn[0]*%%%%%%%
\begin{enumerate}
\def\labelenumi{\arabic{enumi}.}
\item
  Forwarded attributes do not include any attributes that are declared
  as props, or \texttt{v-on} listeners of declared events by
  \texttt{\textless{}MyButton\textgreater{}} - in other words, the
  declared props and listeners have been "consumed" by
  \texttt{\textless{}MyButton\textgreater{}}.
\item
  Forwarded attributes may be accepted as props by
  \texttt{\textless{}BaseButton\textgreater{}}, if declared by it.
\end{enumerate}
\switchcolumn
\begin{enumerate}
\def\labelenumi{\arabic{enumi}.}
\item
  透传的 attribute 不会包含 \texttt{\textless{}MyButton\textgreater{}}
  上声明过的 props 或是针对 \texttt{emits} 声明事件的 \texttt{v-on}
  侦听函数,换句话说,声明过的 props 和侦听函数被
  \texttt{\textless{}MyButton\textgreater{}}``消费''了。
\item
  透传的 attribute 若符合声明,也可以作为 props 传入
  \texttt{\textless{}BaseButton\textgreater{}}。
\end{enumerate}
\end{paracol}

\columnratio{0.55}
\begin{paracol}{2}

\switchcolumn[0]*%%%%%%%
\subsection{Disabling Attribute Inheritance}
\switchcolumn
\subsection{禁用 Attributes 继承}
\switchcolumn[0]*%%%%%%%
If you do \textbf{not} want a component to automatically inherit
attributes, you can set \texttt{inheritAttrs:\ false} in the component's
options.
\switchcolumn
如果你\textbf{不想要}一个组件自动地继承
attribute,你可以在组件选项中设置 \texttt{inheritAttrs:\ false}。
\switchcolumn[0]*%%%%%%%
Since 3.3 you can also use
\href{https://vuejs.org/api/sfc-script-setup.html\#defineoptions}{\texttt{defineOptions}}
directly in \texttt{\textless{}script\ setup\textgreater{}}:
\switchcolumn
从 3.3 开始你也可以直接在
\texttt{\textless{}script\ setup\textgreater{}} 中使用
\href{https://cn.vuejs.org/api/sfc-script-setup.html\#defineoptions}{\texttt{defineOptions}}:
\switchcolumn[0]*%%%%%%%
\begin{codeHtml}
<script setup>
defineOptions({
    inheritAttrs: false
})
// ...setup 逻辑
</script>
\end{codeHtml}
\switchcolumn
\begin{codeHtml}
<script setup>
defineOptions({
    inheritAttrs: false
})
// ...setup 逻辑
</script>
\end{codeHtml}
\switchcolumn[0]*%%%%%%%
The common scenario for disabling attribute inheritance is when
attributes need to be applied to other elements besides the root node.
By setting the \texttt{inheritAttrs} option to \texttt{false}, you can
take full control over where the fallthrough attributes should be
applied.
\switchcolumn
最常见的需要禁用 attribute 继承的场景就是 attribute
需要应用在根节点以外的其他元素上。通过设置 \texttt{inheritAttrs} 选项为
\texttt{false},你可以完全控制透传进来的 attribute 被如何使用。
\switchcolumn[0]*%%%%%%%
These fallthrough attributes can be accessed directly in template
expressions as \texttt{\$attrs}:
\switchcolumn
这些透传进来的 attribute 可以在模板的表达式中直接用 \texttt{\$attrs}
访问到。
\switchcolumn[0]*%%%%%%%
\begin{codeHtml}
<span>Fallthrough attribute: {{ $attrs }}</span>
\end{codeHtml}
\switchcolumn
\begin{codeHtml}
<span>Fallthrough attribute: {{ $attrs }}</span>
\end{codeHtml}
\switchcolumn[0]*%%%%%%%
The \texttt{\$attrs} object includes all attributes that are not
declared by the component's \texttt{props} or \texttt{emits} options
(e.g., \texttt{class}, \texttt{style}, \texttt{v-on} listeners, etc.).
\switchcolumn
这个 \texttt{\$attrs} 对象包含了除组件所声明的 \texttt{props} 和
\texttt{emits} 之外的所有其他 attribute,例如
\texttt{class},\texttt{style},\texttt{v-on} 监听器等等。
\switchcolumn[0]*%%%%%%%
Some notes:
\switchcolumn
有几点需要注意:
\switchcolumn[0]*%%%%%%%
\begin{itemize}
\item
  Unlike props, fallthrough attributes preserve their original casing in
  JavaScript, so an attribute like \texttt{foo-bar} needs to be accessed
  as \texttt{\$attrs{[}\textquotesingle{}foo-bar\textquotesingle{}{]}}.
\item
  A \texttt{v-on} event listener like \texttt{@click} will be exposed on
  the object as a function under \texttt{\$attrs.onClick}.
\end{itemize}
\switchcolumn
\begin{itemize}
\item
  和 props 有所不同,透传 attributes 在 JavaScript
  中保留了它们原始的大小写,所以像 \texttt{foo-bar} 这样的一个 attribute
  需要通过
  \texttt{\$attrs{[}\textquotesingle{}foo-bar\textquotesingle{}{]}}
  来访问。
\item
  像 \texttt{@click} 这样的一个 \texttt{v-on}
  事件监听器将在此对象下被暴露为一个函数 \texttt{\$attrs.onClick}。
\end{itemize}
\switchcolumn[0]*%%%%%%%
Using our \texttt{\textless{}MyButton\textgreater{}} component example
from the
\href{https://vuejs.org/guide/components/attrs.html\#attribute-inheritance}{previous
section} - sometimes we may need to wrap the actual
\texttt{\textless{}button\textgreater{}} element with an extra
\texttt{\textless{}div\textgreater{}} for styling purposes:
\switchcolumn
现在我们要再次使用一下\href{https://cn.vuejs.org/guide/components/attrs.html\#attribute-inheritance}{之前小节}中的
\texttt{\textless{}MyButton\textgreater{}}
组件例子。有时候我们可能为了样式,需要在
\texttt{\textless{}button\textgreater{}} 元素外包装一层
\texttt{\textless{}div\textgreater{}}:
\switchcolumn[0]*%%%%%%%
\begin{codeHtml}
<div class="btn-wrapper">
  <button class="btn">click me</button>
</div>
\end{codeHtml}
\switchcolumn
\begin{codeHtml}
<div class="btn-wrapper">
  <button class="btn">click me</button>
</div>
\end{codeHtml}
\switchcolumn[0]*%%%%%%%
We want all fallthrough attributes like \texttt{class} and \texttt{v-on}
listeners to be applied to the inner
\texttt{\textless{}button\textgreater{}}, not the outer
\texttt{\textless{}div\textgreater{}}. We can achieve this with
\texttt{inheritAttrs:\ false} and \texttt{v-bind="\$attrs"}:
\switchcolumn
我们想要所有像 \texttt{class} 和 \texttt{v-on} 监听器这样的透传
attribute 都应用在内部的 \texttt{\textless{}button\textgreater{}}
上而不是外层的 \texttt{\textless{}div\textgreater{}}
上。我们可以通过设定 \texttt{inheritAttrs:\ false} 和使用
\texttt{v-bind="\$attrs"} 来实现:
\switchcolumn[0]*%%%%%%%
\begin{codeHtml}
<div class="btn-wrapper">
  <button class="btn" v-bind="$attrs">click me</button>
</div>
\end{codeHtml}
\switchcolumn
\begin{codeHtml}
<div class="btn-wrapper">
  <button class="btn" v-bind="$attrs">click me</button>
</div>
\end{codeHtml}
\switchcolumn[0]*%%%%%%%
Remember that
\href{https://vuejs.org/guide/essentials/template-syntax.html\#dynamically-binding-multiple-attributes}{\texttt{v-bind}
without an argument} binds all the properties of an object as attributes
of the target element.
\switchcolumn
小提示:\href{https://cn.vuejs.org/guide/essentials/template-syntax.html\#dynamically-binding-multiple-attributes}{没有参数的
\texttt{v-bind}} 会将一个对象的所有属性都作为 attribute
应用到目标元素上。
\end{paracol}

\columnratio{0.55}
\begin{paracol}{2}

\switchcolumn[0]*%%%%%%%
\subsection{Attribute Inheritance on Multiple Root Nodes}
\switchcolumn
\subsection{多根节点的 Attributes 继承}
\switchcolumn[0]*%%%%%%%
Unlike components with a single root node, components with multiple root
nodes do not have an automatic attribute fallthrough behavior. If
\texttt{\$attrs} are not bound explicitly, a runtime warning will be
issued.
\switchcolumn
和单根节点组件有所不同,有着多个根节点的组件没有自动 attribute
透传行为。如果 \texttt{\$attrs} 没有被显式绑定,将会抛出一个运行时警告。
\switchcolumn[0]*%%%%%%%
\begin{codeHtml}
<CustomLayout id="custom-layout" @click="changeValue" />
\end{codeHtml}
\switchcolumn
\begin{codeHtml}
<CustomLayout id="custom-layout" @click="changeValue" />
\end{codeHtml}
\switchcolumn[0]*%%%%%%%
If \texttt{\textless{}CustomLayout\textgreater{}} has the following
multi-root template, there will be a warning because Vue cannot be sure
where to apply the fallthrough attributes:
\switchcolumn
如果 \texttt{\textless{}CustomLayout\textgreater{}}
有下面这样的多根节点模板,由于 Vue 不知道要将 attribute
透传到哪里,所以会抛出一个警告。
\switchcolumn[0]*%%%%%%%
\begin{codeHtml}
<header>...</header>
<main>...</main>
<footer>...</footer>
\end{codeHtml}
\switchcolumn
\begin{codeHtml}
<header>...</header>
<main>...</main>
<footer>...</footer>
\end{codeHtml}
\switchcolumn[0]*%%%%%%%
The warning will be suppressed if \texttt{\$attrs} is explicitly bound:
\switchcolumn
如果 \texttt{\$attrs} 被显式绑定,则不会有警告:
\switchcolumn[0]*%%%%%%%
\begin{codeHtml}
<header>...</header>
<main v-bind="$attrs">...</main>
<footer>...</footer>
\end{codeHtml}
\switchcolumn
\begin{codeHtml}
<header>...</header>
<main v-bind="$attrs">...</main>
<footer>...</footer>
\end{codeHtml}
\switchcolumn[0]*%%%%%%%
\subsection{Accessing Fallthrough Attributes in JavaScript}
\switchcolumn
\subsection{在 JavaScript 中访问透传 Attributes}
\switchcolumn[0]*%%%%%%%
If needed, you can access a component's fallthrough attributes in
\texttt{\textless{}script\ setup\textgreater{}} using the
\texttt{useAttrs()} API:
\switchcolumn
如果需要,你可以在 \texttt{\textless{}script\ setup\textgreater{}}
中使用 \texttt{useAttrs()} API 来访问一个组件的所有透传 attribute:
\switchcolumn[0]*%%%%%%%
\begin{codeHtml}
<script setup>
import { useAttrs } from 'vue'
const attrs = useAttrs()
</script>
\end{codeHtml}
\switchcolumn
\begin{codeHtml}
<script setup>
import { useAttrs } from 'vue'
const attrs = useAttrs()
</script>
\end{codeHtml}
\switchcolumn[0]*%%%%%%%
If not using \texttt{\textless{}script\ setup\textgreater{}},
\texttt{attrs} will be exposed as a property of the \texttt{setup()}
context:
\switchcolumn
如果没有使用
\texttt{\textless{}script\ setup\textgreater{}},\texttt{attrs} 会作为
\texttt{setup()} 上下文对象的一个属性暴露:
\switchcolumn[0]*%%%%%%%
\begin{codeJs}
export default {
  setup(props, ctx) {
    // 透传 attribute 被暴露为 ctx.attrs
    console.log(ctx.attrs)
  }
}
\end{codeJs}
\switchcolumn
\begin{codeJs}
export default {
  setup(props, ctx) {
    // 透传 attribute 被暴露为 ctx.attrs
    console.log(ctx.attrs)
  }
}
\end{codeJs}
\switchcolumn[0]*%%%%%%%
Note that although the \texttt{attrs} object here always reflects the
latest fallthrough attributes, it isn't reactive (for performance
reasons). You cannot use watchers to observe its changes. If you need
reactivity, use a prop. Alternatively, you can use \texttt{onUpdated()}
to perform side effects with the latest \texttt{attrs} on each update.
\switchcolumn
需要注意的是,虽然这里的 \texttt{attrs} 对象总是反映为最新的透传
attribute,但它并不是响应式的
(考虑到性能因素)。你不能通过侦听器去监听它的变化。如果你需要响应性,可以使用
prop。或者你也可以使用 \texttt{onUpdated()} 使得在每次更新时结合最新的
\texttt{attrs} 执行副作用。
\end{paracol}
