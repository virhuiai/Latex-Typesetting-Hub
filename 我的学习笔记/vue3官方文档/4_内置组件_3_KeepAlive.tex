\columnratio{0.55}
\begin{paracol}{2}
 
\switchcolumn[0]*%%%%%%%
\section{KeepAlive}
\switchcolumn
\section{KeepAlive}
\switchcolumn[0]*%%%%%%%
\texttt{\textless{}KeepAlive\textgreater{}} is a built-in component that
allows us to conditionally cache component instances when dynamically
switching between multiple components.
\switchcolumn
\texttt{\textless{}KeepAlive\textgreater{}}
是一个内置组件,它的功能是在多个组件间动态切换时缓存被移除的组件实例。
\switchcolumn[0]*%%%%%%%
\subsection{Basic Usage}
\switchcolumn
\subsection{基本使用}
\switchcolumn[0]*%%%%%%%
In the Component Basics chapter, we introduced the syntax for
\href{https://vuejs.org/guide/essentials/component-basics.html\#dynamic-components}{Dynamic
Components}, using the \texttt{\textless{}component\textgreater{}}
special element:
\switchcolumn
在组件基础章节中,我们已经介绍了通过特殊的
\texttt{\textless{}component\textgreater{}}
元素来实现\href{https://cn.vuejs.org/guide/essentials/component-basics.html\#dynamic-components}{动态组件}的用法: 
\switchcolumn[0]*%%%%%%%
\begin{codeHtml}
<component :is="activeComponent" />
\end{codeHtml}
\switchcolumn
\begin{codeHtml}
<component :is="activeComponent" />
\end{codeHtml}
\switchcolumn[0]*%%%%%%%
By default, an active component instance will be unmounted when
switching away from it. This will cause any changed state it holds to be
lost. When this component is displayed again, a new instance will be
created with only the initial state.
\switchcolumn
默认情况下,一个组件实例在被替换掉后会被销毁。这会导致它丢失其中所有已变化的状态------当这个组件再一次被显示时,会创建一个只带有初始状态的新实例。
\switchcolumn[0]*%%%%%%%
In the example below, we have two stateful components - A contains a
counter, while B contains a message synced with an input via
\texttt{v-model}. Try updating the state of one of them, switch away,
and then switch back to it:
\switchcolumn
在下面的例子中,你会看到两个有状态的组件------A 有一个计数器,而 B
有一个通过 \texttt{v-model} 同步 input
框输入内容的文字展示。尝试先更改一下任意一个组件的状态,然后切走,再切回来:
\switchcolumn[0]*%%%%%%%
You'll notice that when switched back, the previous changed state would
have been reset.
\switchcolumn
你会发现在切回来之后,之前已更改的状态都被重置了。 
\switchcolumn[0]*%%%%%%%
Creating fresh component instance on switch is normally useful behavior,
but in this case, we'd really like the two component instances to be
preserved even when they are inactive. To solve this problem, we can
wrap our dynamic component with the
\texttt{\textless{}KeepAlive\textgreater{}} built-in component:
\switchcolumn
在切换时创建新的组件实例通常是有意义的,但在这个例子中,我们的确想要组件能在被``切走''的时候保留它们的状态。要解决这个问题,我们可以用
\texttt{\textless{}KeepAlive\textgreater{}}
内置组件将这些动态组件包装起来:
\switchcolumn[0]*%%%%%%%
\begin{codeHtml}
<!-- 非活跃的组件将会被缓存! -->
<KeepAlive>
  <component :is="activeComponent" />
</KeepAlive>
\end{codeHtml}
\switchcolumn
\begin{codeHtml}
<!-- 非活跃的组件将会被缓存! -->
<KeepAlive>
  <component :is="activeComponent" />
</KeepAlive>
\end{codeHtml}
\switchcolumn[0]*%%%%%%%
Now, the state will be persisted across component switches:
\switchcolumn
现在,在组件切换时状态也能被保留了:
\switchcolumn[0]*%%%%%%%
\href{https://play.vuejs.org/\#eNqtUsFOwzAM/RWrl4IGC+cqq2h3RFw495K12YhIk6hJi1DVf8dJSllBaAJxi+2XZz8/j0lhzHboeZIl1NadMA4sd73JKyVaozsHI9hnJqV+feJHmODY6RZS/JEuiL1uTTEXtiREnnINKFeAcgZUqtbKOqj7ruPKwe6s2VVguq4UJXEynAkDx1sjmeMYAdBGDFBLZu2uShre6ioJeaxIduAyp0KZ3oF7MxwRHWsEQmC4bXXDJWbmxpjLBiZ7DwptMUFyKCiJNP/BWUbO8gvnA+emkGKIgkKqRrRWfh+Z8MIWwpySpfbxn6wJKMGV4IuSs0UlN1HVJae7bxYvBuk+2IOIq7sLnph8P9u5DJv5VfpWWLaGqTzwZTCOM/M0IaMvBMihd04ruK+lqF/8Ajxms8EFbCiJxR8khsP6ncQosLWnWV6a/kUf2nqu75Fby04chA0iPftaYryhz6NBRLjdtajpHZTWPio=}{Try
it in the Playground}
\switchcolumn
\href{https://play.vuejs.org/\#eNqtUsFOwzAM/RWrl4IGC+cqq2h3RFw495K12YhIk6hJi1DVf8dJSllBaAJxi+2XZz8/j0lhzHboeZIl1NadMA4sd73JKyVaozsHI9hnJqV+feJHmODY6RZS/JEuiL1uTTEXtiREnnINKFeAcgZUqtbKOqj7ruPKwe6s2VVguq4UJXEynAkDx1sjmeMYAdBGDFBLZu2uShre6ioJeaxIduAyp0KZ3oF7MxwRHWsEQmC4bXXDJWbmxpjLBiZ7DwptMUFyKCiJNP/BWUbO8gvnA+emkGKIgkKqRrRWfh+Z8MIWwpySpfbxn6wJKMGV4IuSs0UlN1HVJae7bxYvBuk+2IOIq7sLnph8P9u5DJv5VfpWWLaGqTzwZTCOM/M0IaMvBMihd04ruK+lqF/8Ajxms8EFbCiJxR8khsP6ncQosLWnWV6a/kUf2nqu75Fby04chA0iPftaYryhz6NBRLjdtajpHZTWPio=}{在演练场中尝试一下}
\switchcolumn[0]*%%%%%%%
\begin{vueQuote}{TIP}
When used in
\href{https://vuejs.org/guide/essentials/component-basics.html\#in-dom-template-parsing-caveats}{in-DOM
templates}, it should be referenced as
\texttt{\textless{}keep-alive\textgreater{}}.
\end{vueQuote} 
\switchcolumn
\begin{vueQuote}{TIP}
在
\href{https://cn.vuejs.org/guide/essentials/component-basics.html\#in-dom-template-parsing-caveats}{DOM
内模板}中使用时,它应该被写为
\texttt{\textless{}keep-alive\textgreater{}}。
\end{vueQuote} 
\switchcolumn[0]*%%%%%%%
\subsection{Include / Exclude}
\switchcolumn
\subsection{包含/排除}
\switchcolumn[0]*%%%%%%%
By default, \texttt{\textless{}KeepAlive\textgreater{}} will cache any
component instance inside. We can customize this behavior via the
\texttt{include} and \texttt{exclude} props. Both props can be a
comma-delimited string, a \texttt{RegExp}, or an array containing either
types:
\switchcolumn
\texttt{\textless{}KeepAlive\textgreater{}}
默认会缓存内部的所有组件实例,但我们可以通过 \texttt{include} 和
\texttt{exclude} prop 来定制该行为。这两个 prop
的值都可以是一个以英文逗号分隔的字符串、一个正则表达式,或是包含这两种类型的一个数组:
\switchcolumn[0]*%%%%%%%
\begin{codeHtml}
<!-- 以英文逗号分隔的字符串 -->
<KeepAlive include="a,b">
    <component :is="view" />
</KeepAlive>
<!-- 正则表达式 (需使用 `v-bind`) -->
<KeepAlive :include="/a|b/">
    <component :is="view" />
</KeepAlive>
<!-- 数组 (需使用 `v-bind`) -->
<KeepAlive :include="['a', 'b']">
    <component :is="view" />
</KeepAlive>
\end{codeHtml}
\switchcolumn
\begin{codeHtml}
<!-- 以英文逗号分隔的字符串 -->
<KeepAlive include="a,b">
  <component :is="view" />
</KeepAlive>
<!-- 正则表达式 (需使用 `v-bind`) -->
<KeepAlive :include="/a|b/">
  <component :is="view" />
</KeepAlive>
<!-- 数组 (需使用 `v-bind`) -->
<KeepAlive :include="['a', 'b']">
  <component :is="view" />
</KeepAlive>
\end{codeHtml}
\switchcolumn[0]*%%%%%%%
The match is checked against the component's
\href{https://vuejs.org/api/options-misc.html\#name}{\texttt{name}}
option, so components that need to be conditionally cached by
\texttt{KeepAlive} must explicitly declare a \texttt{name} option.
\switchcolumn
它会根据组件的
\href{https://cn.vuejs.org/api/options-misc.html\#name}{\texttt{name}}
选项进行匹配,所以组件如果想要条件性地被 \texttt{KeepAlive}
缓存,就必须显式声明一个 \texttt{name} 选项。
\switchcolumn[0]*%%%%%%%
\begin{vueQuote}{TIP}
Since version 3.2.34, a single-file component using
\texttt{\textless{}script\ setup\textgreater{}} will automatically infer
its \texttt{name} option based on the filename, removing the need to
manually declare the name.
\end{vueQuote} 
\switchcolumn
\begin{vueQuote}{TIP}
在 3.2.34 或以上的版本中,使用
\texttt{\textless{}script\ setup\textgreater{}}
的单文件组件会自动根据文件名生成对应的 \texttt{name}
选项,无需再手动声明。
\end{vueQuote} 
\switchcolumn[0]*%%%%%%%
\subsection{Max Cached Instances}
\switchcolumn
\subsection{最大缓存实例数}
\switchcolumn[0]*%%%%%%%
We can limit the maximum number of component instances that can be
cached via the \texttt{max} prop. When \texttt{max} is specified,
\texttt{\textless{}KeepAlive\textgreater{}} behaves like an
\href{https://en.wikipedia.org/wiki/Cache_replacement_policies\#Least_recently_used_(LRU)}{LRU
cache}: if the number of cached instances is about to exceed the
specified max count, the least recently accessed cached instance will be
destroyed to make room for the new one.
\switchcolumn
我们可以通过传入 \texttt{max} prop
来限制可被缓存的最大组件实例数。\texttt{\textless{}KeepAlive\textgreater{}}
的行为在指定了 \texttt{max} 后类似一个
\href{https://en.wikipedia.org/wiki/Cache_replacement_policies\#Least_recently_used_(LRU)}{LRU
缓存}:如果缓存的实例数量即将超过指定的那个最大数量,则最久没有被访问的缓存实例将被销毁,以便为新的实例腾出空间。
\switchcolumn[0]*%%%%%%%
\begin{codeHtml}
<KeepAlive :max="10">
  <component :is="activeComponent" />
</KeepAlive>
\end{codeHtml}
\switchcolumn
\begin{codeHtml}
<KeepAlive :max="10">
  <component :is="activeComponent" />
</KeepAlive>
\end{codeHtml}
\switchcolumn[0]*%%%%%%%
\subsection{Lifecycle of Cached Instance}
\switchcolumn
\subsection{缓存实例的生命周期}
\switchcolumn[0]*%%%%%%%
When a component instance is removed from the DOM but is part of a
component tree cached by \texttt{\textless{}KeepAlive\textgreater{}}, it
goes into a \textbf{deactivated} state instead of being unmounted. When
a component instance is inserted into the DOM as part of a cached tree,
it is \textbf{activated}.
\switchcolumn
当一个组件实例从 DOM 上移除但因为被
\texttt{\textless{}KeepAlive\textgreater{}}
缓存而仍作为组件树的一部分时,它将变为\textbf{不活跃}状态而不是被卸载。当一个组件实例作为缓存树的一部分插入到
DOM 中时,它将重新\textbf{被激活}。
\switchcolumn[0]*%%%%%%%
A kept-alive component can register lifecycle hooks for these two states
using
\href{https://vuejs.org/api/composition-api-lifecycle.html\#onactivated}{\texttt{onActivated()}}
and
\href{https://vuejs.org/api/composition-api-lifecycle.html\#ondeactivated}{\texttt{onDeactivated()}}:
\switchcolumn
一个持续存在的组件可以通过
\href{https://cn.vuejs.org/api/composition-api-lifecycle.html\#onactivated}{\texttt{onActivated()}}
和
\href{https://cn.vuejs.org/api/composition-api-lifecycle.html\#ondeactivated}{\texttt{onDeactivated()}}
注册相应的两个状态的生命周期钩子:
\switchcolumn[0]*%%%%%%%
\begin{codeHtml}
<script setup>
import { onActivated, onDeactivated } from 'vue'
onActivated(() => {
  // 调用时机为首次挂载
  // 以及每次从缓存中被重新插入时
})
onDeactivated(() => {
  // 在从 DOM 上移除、进入缓存
  // 以及组件卸载时调用
})
</script>
\end{codeHtml}
\switchcolumn
\begin{codeHtml}
<script setup>
import { onActivated, onDeactivated } from 'vue'
onActivated(() => {
  // 调用时机为首次挂载
  // 以及每次从缓存中被重新插入时
})
onDeactivated(() => {
  // 在从 DOM 上移除、进入缓存
  // 以及组件卸载时调用
})
</script>
\end{codeHtml}
\switchcolumn[0]*%%%%%%%
Note that:
\switchcolumn
请注意:
\switchcolumn[0]*%%%%%%%
\begin{itemize}
\item
  \texttt{onActivated} is also called on mount, and
  \texttt{onDeactivated} on unmount.
\item
  Both hooks work for not only the root component cached by
  \texttt{\textless{}KeepAlive\textgreater{}}, but also descendant
  components in the cached tree.
\end{itemize}
\switchcolumn
\begin{itemize}
\item
  \texttt{onActivated} 在组件挂载时也会调用,并且 \texttt{onDeactivated}
  在组件卸载时也会调用。
\item
  这两个钩子不仅适用于 \texttt{\textless{}KeepAlive\textgreater{}}
  缓存的根组件,也适用于缓存树中的后代组件。
\end{itemize}
\switchcolumn[0]*%%%%%%%
\begin{center}\rule{0.5\linewidth}{0.5pt}\end{center}
\switchcolumn
\begin{center}\rule{0.5\linewidth}{0.5pt}\end{center}
\switchcolumn[0]*%%%%%%%
\textbf{Related}
\switchcolumn
\textbf{参考}
\switchcolumn[0]*%%%%%%%
\begin{itemize}
\item
  \href{https://vuejs.org/api/built-in-components.html\#keepalive}{``
  API reference}
\end{itemize}
\switchcolumn
\begin{itemize}
\item
  \href{https://cn.vuejs.org/api/built-in-components.html\#keepalive}{``
  API 参考}
\end{itemize}
\end{paracol}
