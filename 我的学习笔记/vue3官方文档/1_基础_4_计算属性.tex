
\columnratio{0.55}
\begin{paracol}{2}

\switchcolumn[0]*%%%%%%%
\section{Computed Properties}
\switchcolumn
\section{计算属性}
\switchcolumn[0]*%%%%%%%
\subsection{Basic Example}
\switchcolumn
\subsection{基础示例}
\switchcolumn[0]*%%%%%%%
In-template expressions are very convenient, but they are meant for
simple operations. Putting too much logic in your templates can make
them bloated and hard to maintain. For example, if we have an object
with a nested array:
\switchcolumn
模板中的表达式虽然方便,但也只能用来做简单的操作。如果在模板中写太多逻辑,会让模板变得臃肿,难以维护。比如说,我们有这样一个包含嵌套数组的对象:
\switchcolumn[0]*%%%%%%%
\begin{codeJs}
const author = reactive({
    name: 'John Doe',
    books: [
        'Vue 2 - Advanced Guide',
        'Vue 3 - Basic Guide',
        'Vue 4 - The Mystery'
    ]
})
\end{codeJs}
\switchcolumn
\begin{codeJs}
const author = reactive({
    name: 'John Doe',
    books: [
        'Vue 2 - Advanced Guide',
        'Vue 3 - Basic Guide',
        'Vue 4 - The Mystery'
    ]
})
\end{codeJs}
\switchcolumn[0]*%%%%%%%
And we want to display different messages depending on if
\texttt{author} already has some books or not:
\switchcolumn
我们想根据 \texttt{author} 是否已有一些书籍来展示不同的信息:
\switchcolumn[0]*%%%%%%%
\begin{codeHtml}
<p>Has published books:</p>
<span>{{ author.books.length > 0 ? 'Yes' : 'No' }}</span>
\end{codeHtml}
\switchcolumn
\begin{codeHtml}
<p>Has published books:</p>
<span>{{ author.books.length > 0 ? 'Yes' : 'No' }}</span>
\end{codeHtml}
\switchcolumn[0]*%%%%%%%
At this point, the template is getting a bit cluttered. We have to look
at it for a second before realizing that it performs a calculation
depending on \texttt{author.books}. More importantly, we probably don't
want to repeat ourselves if we need to include this calculation in the
template more than once.
\switchcolumn
这里的模板看起来有些复杂。我们必须认真看好一会儿才能明白它的计算依赖于
\texttt{author.books}。更重要的是,如果在模板中需要不止一次这样的计算,我们可不想将这样的代码在模板里重复好多遍。
\switchcolumn[0]*%%%%%%%
That's why for complex logic that includes reactive data, it is
recommended to use a \textbf{computed property}. Here's the same
example, refactored:
\switchcolumn
因此我们推荐使用\textbf{计算属性}来描述依赖响应式状态的复杂逻辑。这是重构后的示例:
\switchcolumn[0]*%%%%%%%
\begin{codeHtml}
<script setup>
import { reactive, computed } from 'vue'

const author = reactive({
    name: 'John Doe',
    books: [
    'Vue 2 - Advanced Guide',
    'Vue 3 - Basic Guide',
    'Vue 4 - The Mystery'
    ]
})

// 一个计算属性 ref
const publishedBooksMessage = computed(() => {
    return author.books.length > 0 ? 'Yes' : 'No'
})
</script>

<template>
    <p>Has published books:</p>
    <span>{{ publishedBooksMessage }}</span>
</template>
\end{codeHtml}
\switchcolumn
\begin{codeHtml}
<script setup>
import { reactive, computed } from 'vue'

const author = reactive({
    name: 'John Doe',
    books: [
    'Vue 2 - Advanced Guide',
    'Vue 3 - Basic Guide',
    'Vue 4 - The Mystery'
    ]
})

// 一个计算属性 ref
const publishedBooksMessage = computed(() => {
    return author.books.length > 0 ? 'Yes' : 'No'
})
</script>

<template>
    <p>Has published books:</p>
    <span>{{ publishedBooksMessage }}</span>
</template>
\end{codeHtml}
\switchcolumn[0]*%%%%%%%
\href{https://play.vuejs.org/\#eNp1kE9Lw0AQxb/KI5dtoTainkoaaREUoZ5EEONhm0ybYLO77J9CCfnuzta0vdjbzr6Zeb95XbIwZroPlMySzJW2MR6OfDB5oZrWaOvRwZIsfbOnCUrdmuCpQo+N1S0ET4pCFarUynnI4GttMT9PjLpCAUq2NIN41bXCkyYxiZ9rrX/cDF/xDYiPQLjDDRbVXqqSHZ5DUw2tg3zP8lK6pvxHe2DtvSasDs6TPTAT8F2ofhzh0hTygm5pc+I1Yb1rXE3VMsKsyDm5JcY/9Y5GY8xzHI+wnIpVw4nTI/10R2rra+S4xSPEJzkBvvNNs310ztK/RDlLLjy1Zic9cQVkJn+R7gIwxJGlMXiWnZEq77orhH3Pq2NH9DjvTfpfSBSbmA==}{Try
it in the Playground}
\switchcolumn
\href{https://play.vuejs.org/\#eNp1kE9Lw0AQxb/KI5dtoTainkoaaREUoZ5EEONhm0ybYLO77J9CCfnuzta0vdjbzr6Zeb95XbIwZroPlMySzJW2MR6OfDB5oZrWaOvRwZIsfbOnCUrdmuCpQo+N1S0ET4pCFarUynnI4GttMT9PjLpCAUq2NIN41bXCkyYxiZ9rrX/cDF/xDYiPQLjDDRbVXqqSHZ5DUw2tg3zP8lK6pvxHe2DtvSasDs6TPTAT8F2ofhzh0hTygm5pc+I1Yb1rXE3VMsKsyDm5JcY/9Y5GY8xzHI+wnIpVw4nTI/10R2rra+S4xSPEJzkBvvNNs310ztK/RDlLLjy1Zic9cQVkJn+R7gIwxJGlMXiWnZEq77orhH3Pq2NH9DjvTfpfSBSbmA==}{在演练场中尝试一下}


\switchcolumn[0]*%%%%%%%
Here we have declared a computed property
\texttt{publishedBooksMessage}. The \texttt{computed()} function expects
to be passed a getter function, and the returned value is a
\textbf{computed ref}. Similar to normal refs, you can access the
computed result as \texttt{publishedBooksMessage.value}. Computed refs
are also auto-unwrapped in templates so you can reference them without
\texttt{.value} in template expressions.
\switchcolumn
我们在这里定义了一个计算属性
\texttt{publishedBooksMessage}。\texttt{computed()} 方法期望接收一个
getter 函数,返回值为一个\textbf{计算属性 ref}。和其他一般的 ref
类似,你可以通过 \texttt{publishedBooksMessage.value}
访问计算结果。计算属性 ref
也会在模板中自动解包,因此在模板表达式中引用时无需添加 \texttt{.value}。
\switchcolumn[0]*%%%%%%%
A computed property automatically tracks its reactive dependencies. Vue
is aware that the computation of \texttt{publishedBooksMessage} depends
on \texttt{author.books}, so it will update any bindings that depend on
\texttt{publishedBooksMessage} when \texttt{author.books} changes.
\switchcolumn
Vue 的计算属性会自动追踪响应式依赖。它会检测到
\texttt{publishedBooksMessage} 依赖于 \texttt{author.books},所以当
\texttt{author.books} 改变时,任何依赖于 \texttt{publishedBooksMessage}
的绑定都会同时更新。
\switchcolumn[0]*%%%%%%%
See also:
\href{https://vuejs.org/guide/typescript/composition-api.html\#typing-computed}{Typing
Computed}
\switchcolumn
也可参考:\href{https://cn.vuejs.org/guide/typescript/composition-api.html\#typing-computed}{为计算属性标注类型}
\end{paracol}



\end{document}
%%%%%%%]*%%%%%%%]*%%%%%%%]*%%%%%%%]*%%%%%%%]*%%%%%%%]*%%%%%%%]*%%%%%%%]*%%%%%%%]*%%%%%%%
\switchcolumn[0]*%%%%%%%
\begin{vueQuote}{}
\end{vueQuote} 
\switchcolumn
\begin{vueQuote}{}
\end{vueQuote} 

\switchcolumn[0]*%%%%%%%
\begin{codeJs}

\end{codeJs}  
\switchcolumn
\begin{codeJs}

\end{codeJs}  


\switchcolumn[0]*%%%%%%%
\begin{codeHtml}

\end{codeHtml}  
\switchcolumn
\begin{codeHtml}

\end{codeHtml}  