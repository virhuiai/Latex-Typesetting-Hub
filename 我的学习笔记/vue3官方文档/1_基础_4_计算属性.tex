
\columnratio{0.55}
\begin{paracol}{2}

\switchcolumn[0]*%%%%%%%
\section{Computed Properties}
\switchcolumn
\section{计算属性}
\switchcolumn[0]*%%%%%%%
\subsection{Basic Example}
\switchcolumn
\subsection{基础示例}
\switchcolumn[0]*%%%%%%%
In-template expressions are very convenient, but they are meant for
simple operations. Putting too much logic in your templates can make
them bloated and hard to maintain. For example, if we have an object
with a nested array:
\switchcolumn
模板中的表达式虽然方便,但也只能用来做简单的操作。如果在模板中写太多逻辑,会让模板变得臃肿,难以维护。比如说,我们有这样一个包含嵌套数组的对象:
\switchcolumn[0]*%%%%%%%
\begin{codeJs}
const author = reactive({
    name: 'John Doe',
    books: [
        'Vue 2 - Advanced Guide',
        'Vue 3 - Basic Guide',
        'Vue 4 - The Mystery'
    ]
})
\end{codeJs}
\switchcolumn
\begin{codeJs}
const author = reactive({
    name: 'John Doe',
    books: [
        'Vue 2 - Advanced Guide',
        'Vue 3 - Basic Guide',
        'Vue 4 - The Mystery'
    ]
})
\end{codeJs}
\switchcolumn[0]*%%%%%%%
And we want to display different messages depending on if
\texttt{author} already has some books or not:
\switchcolumn
我们想根据 \texttt{author} 是否已有一些书籍来展示不同的信息:
\switchcolumn[0]*%%%%%%%
\begin{codeHtml}
<p>Has published books:</p>
<span>{{ author.books.length > 0 ? 'Yes' : 'No' }}</span>
\end{codeHtml}
\switchcolumn
\begin{codeHtml}
<p>Has published books:</p>
<span>{{ author.books.length > 0 ? 'Yes' : 'No' }}</span>
\end{codeHtml}
\switchcolumn[0]*%%%%%%%
At this point, the template is getting a bit cluttered. We have to look
at it for a second before realizing that it performs a calculation
depending on \texttt{author.books}. More importantly, we probably don't
want to repeat ourselves if we need to include this calculation in the
template more than once.
\switchcolumn
这里的模板看起来有些复杂。我们必须认真看好一会儿才能明白它的计算依赖于
\texttt{author.books}。更重要的是,如果在模板中需要不止一次这样的计算,我们可不想将这样的代码在模板里重复好多遍。
\switchcolumn[0]*%%%%%%%
That's why for complex logic that includes reactive data, it is
recommended to use a \textbf{computed property}. Here's the same
example, refactored:
\switchcolumn
因此我们推荐使用\textbf{计算属性}来描述依赖响应式状态的复杂逻辑。这是重构后的示例:
\switchcolumn[0]*%%%%%%%
\begin{codeHtml}
<script setup>
import { reactive, computed } from 'vue'

const author = reactive({
    name: 'John Doe',
    books: [
    'Vue 2 - Advanced Guide',
    'Vue 3 - Basic Guide',
    'Vue 4 - The Mystery'
    ]
})

// 一个计算属性 ref
const publishedBooksMessage = computed(() => {
    return author.books.length > 0 ? 'Yes' : 'No'
})
</script>

<template>
    <p>Has published books:</p>
    <span>{{ publishedBooksMessage }}</span>
</template>
\end{codeHtml}
\switchcolumn
\begin{codeHtml}
<script setup>
import { reactive, computed } from 'vue'

const author = reactive({
    name: 'John Doe',
    books: [
    'Vue 2 - Advanced Guide',
    'Vue 3 - Basic Guide',
    'Vue 4 - The Mystery'
    ]
})

// 一个计算属性 ref
const publishedBooksMessage = computed(() => {
    return author.books.length > 0 ? 'Yes' : 'No'
})
</script>

<template>
    <p>Has published books:</p>
    <span>{{ publishedBooksMessage }}</span>
</template>
\end{codeHtml}
\switchcolumn[0]*%%%%%%%
\href{https://play.vuejs.org/\#eNp1kE9Lw0AQxb/KI5dtoTainkoaaREUoZ5EEONhm0ybYLO77J9CCfnuzta0vdjbzr6Zeb95XbIwZroPlMySzJW2MR6OfDB5oZrWaOvRwZIsfbOnCUrdmuCpQo+N1S0ET4pCFarUynnI4GttMT9PjLpCAUq2NIN41bXCkyYxiZ9rrX/cDF/xDYiPQLjDDRbVXqqSHZ5DUw2tg3zP8lK6pvxHe2DtvSasDs6TPTAT8F2ofhzh0hTygm5pc+I1Yb1rXE3VMsKsyDm5JcY/9Y5GY8xzHI+wnIpVw4nTI/10R2rra+S4xSPEJzkBvvNNs310ztK/RDlLLjy1Zic9cQVkJn+R7gIwxJGlMXiWnZEq77orhH3Pq2NH9DjvTfpfSBSbmA==}{Try
it in the Playground}
\switchcolumn
\href{https://play.vuejs.org/\#eNp1kE9Lw0AQxb/KI5dtoTainkoaaREUoZ5EEONhm0ybYLO77J9CCfnuzta0vdjbzr6Zeb95XbIwZroPlMySzJW2MR6OfDB5oZrWaOvRwZIsfbOnCUrdmuCpQo+N1S0ET4pCFarUynnI4GttMT9PjLpCAUq2NIN41bXCkyYxiZ9rrX/cDF/xDYiPQLjDDRbVXqqSHZ5DUw2tg3zP8lK6pvxHe2DtvSasDs6TPTAT8F2ofhzh0hTygm5pc+I1Yb1rXE3VMsKsyDm5JcY/9Y5GY8xzHI+wnIpVw4nTI/10R2rra+S4xSPEJzkBvvNNs310ztK/RDlLLjy1Zic9cQVkJn+R7gIwxJGlMXiWnZEq77orhH3Pq2NH9DjvTfpfSBSbmA==}{在演练场中尝试一下}


\switchcolumn[0]*%%%%%%%
Here we have declared a computed property
\texttt{publishedBooksMessage}. The \texttt{computed()} function expects
to be passed a getter function, and the returned value is a
\textbf{computed ref}. Similar to normal refs, you can access the
computed result as \texttt{publishedBooksMessage.value}. Computed refs
are also auto-unwrapped in templates so you can reference them without
\texttt{.value} in template expressions.
\switchcolumn
我们在这里定义了一个计算属性
\texttt{publishedBooksMessage}。\texttt{computed()} 方法期望接收一个
getter 函数,返回值为一个\textbf{计算属性 ref}。和其他一般的 ref
类似,你可以通过 \texttt{publishedBooksMessage.value}
访问计算结果。计算属性 ref
也会在模板中自动解包,因此在模板表达式中引用时无需添加 \texttt{.value}。
\switchcolumn[0]*%%%%%%%
A computed property automatically tracks its reactive dependencies. Vue
is aware that the computation of \texttt{publishedBooksMessage} depends
on \texttt{author.books}, so it will update any bindings that depend on
\texttt{publishedBooksMessage} when \texttt{author.books} changes.
\switchcolumn
Vue 的计算属性会自动追踪响应式依赖。它会检测到
\texttt{publishedBooksMessage} 依赖于 \texttt{author.books},所以任何依赖于 \texttt{publishedBooksMessage}
的绑定,都会在 \texttt{author.books} 改变时同时更新。
\switchcolumn[0]*%%%%%%%
See also:
\href{https://vuejs.org/guide/typescript/composition-api.html\#typing-computed}{Typing
Computed}
\switchcolumn
也可参考:\href{https://cn.vuejs.org/guide/typescript/composition-api.html\#typing-computed}{为计算属性标注类型}
\end{paracol}

\columnratio{0.55}
\begin{paracol}{2}

\switchcolumn[0]*%%%%%%%
\subsection{Computed Caching vs. Methods}
\switchcolumn
\subsection{计算属性缓存 vs 方法}
\switchcolumn[0]*%%%%%%%
You may have noticed we can achieve the same result by invoking a method
in the expression:
\switchcolumn
你可能注意到我们在表达式中像这样调用一个函数也会获得和计算属性相同的结果:
\switchcolumn[0]*%%%%%%%
\begin{codeHtml}
<p>{{ calculateBooksMessage() }}</p>
\end{codeHtml}
\switchcolumn
\begin{codeHtml}
<p>{{ calculateBooksMessage() }}</p>
\end{codeHtml}
\switchcolumn[0]*%%%%%%%
\begin{codeJs}
// in component
function calculateBooksMessage() {
    return author.books.length > 0 ? 'Yes' : 'No'
}
\end{codeJs}
\switchcolumn
\begin{codeJs}
// 组件中
function calculateBooksMessage() {
    return author.books.length > 0 ? 'Yes' : 'No'
}
\end{codeJs}
\switchcolumn[0]*%%%%%%%
Instead of a computed property, we can define the same function as a
method. For the end result, the two approaches are indeed exactly the
same. However, the difference is that \textbf{computed properties are
cached based on their reactive dependencies.} A computed property will
only re-evaluate when some of its reactive dependencies have changed.
This means as long as \texttt{author.books} has not changed, multiple
access to \texttt{publishedBooksMessage} will immediately return the
previously computed result without having to run the getter function
again.
\switchcolumn
若我们将同样的函数定义为一个方法而不是计算属性,两种方式在结果上确实是完全相同的,然而,不同之处在于\textbf{计算属性值会基于其响应式依赖被缓存}。一个计算属性仅会在其响应式依赖更新时才重新计算。这意味着只要
\texttt{author.books} 不改变,无论多少次访问
\texttt{publishedBooksMessage}
都会立即返回先前的计算结果,而不用重复执行 getter 函数。
\switchcolumn[0]*%%%%%%%
This also means the following computed property will never update,
because \texttt{Date.now()} is not a reactive dependency:
\switchcolumn
这也解释了为什么下面的计算属性永远不会更新,因为 \texttt{Date.now()}
并不是一个响应式依赖:
\switchcolumn[0]*%%%%%%%
\begin{codeJs}
const now = computed(() => Date.now())
\end{codeJs}
\switchcolumn
\begin{codeJs}
const now = computed(() => Date.now())
\end{codeJs}
\switchcolumn[0]*%%%%%%%
In comparison, a method invocation will \textbf{always} run the function
whenever a re-render happens.
\switchcolumn
相比之下,方法调用\textbf{总是}会在重渲染发生时再次执行函数。
\switchcolumn[0]*%%%%%%%
Why do we need caching? Imagine we have an expensive computed property
\texttt{list}, which requires looping through a huge array and doing a
lot of computations. Then we may have other computed properties that in
turn depend on \texttt{list}. Without caching, we would be executing
\texttt{list}'s getter many more times than necessary! In cases where
you do not want caching, use a method call instead.
\switchcolumn
为什么需要缓存呢?想象一下我们有一个非常耗性能的计算属性
\texttt{list},需要循环一个巨大的数组并做许多计算逻辑,并且可能也有其他计算属性依赖于
\texttt{list}。没有缓存的话,我们会重复执行非常多次 \texttt{list} 的
getter,然而这实际上没有必要!如果你确定不需要缓存,那么也可以使用方法调用。
\end{paracol}

\columnratio{0.55}
\begin{paracol}{2}
\switchcolumn[0]*%%%%%%%
\subsection{Writable Computed}
\switchcolumn
\subsection{可写计算属性}
\switchcolumn[0]*%%%%%%%
Computed properties are by default getter-only. If you attempt to assign
a new value to a computed property, you will receive a runtime warning.
In the rare cases where you need a "writable" computed property, you can
create one by providing both a getter and a setter:
\switchcolumn
计算属性默认是只读的。当你尝试修改一个计算属性时,你会收到一个运行时警告。只在某些特殊场景中你可能才需要用到``可写''的属性,你可以通过同时提供
getter 和 setter 来创建:

\switchcolumn[0]*%%%%%%%
\begin{codeHtml}
<script setup>
import { ref, computed } from 'vue'

const firstName = ref('John')
const lastName = ref('Doe')

const fullName = computed({
    // getter
    get() {
    return firstName.value + ' ' + lastName.value
    },
    // setter
    set(newValue) {
    // Note: we are using destructuring assignment syntax here.
    [firstName.value, lastName.value] = newValue.split(' ')
    }
})
</script>
\end{codeHtml}
\switchcolumn
\begin{codeHtml}
<script setup>
import { ref, computed } from 'vue'

const firstName = ref('John')
const lastName = ref('Doe')

const fullName = computed({
    // getter
    get() {
    return firstName.value + ' ' + lastName.value
    },
    // setter
    set(newValue) {
    // 注意:我们这里使用的是解构赋值语法
    [firstName.value, lastName.value] = newValue.split(' ')
    }
})
</script>
\end{codeHtml}

\switchcolumn[0]*%%%%%%%
Now when you run
\texttt{fullName.value\ =\ \textquotesingle{}John\ Doe\textquotesingle{}},
the setter will be invoked and \texttt{firstName} and \texttt{lastName}
will be updated accordingly.
\switchcolumn
现在当你再运行
\texttt{fullName.value\ =\ \textquotesingle{}John\ Doe\textquotesingle{}}
时,setter 会被调用而 \texttt{firstName} 和 \texttt{lastName}
会随之更新。
\switchcolumn[0]*%%%%%%%
\subsection{Best Practices}
\switchcolumn
\subsection{最佳实践}
\switchcolumn[0]*%%%%%%%
\subsubsection{Getters should be side-effect free}
\switchcolumn
\subsubsection{Getter 不应有副作用}
\switchcolumn[0]*%%%%%%%
It is important to remember that computed getter functions should only
perform pure computation and be free of side effects. For example,
\textbf{don't make async requests or mutate the DOM inside a computed
getter!} Think of a computed property as declaratively describing how to
derive a value based on other values - its only responsibility should be
computing and returning that value. Later in the guide we will discuss
how we can perform side effects in reaction to state changes with
\href{https://vuejs.org/guide/essentials/watchers.html}{watchers}.
\switchcolumn
计算属性的 getter
应只做计算而没有任何其他的副作用,这一点非常重要,请务必牢记。举例来说,\textbf{不要在
getter 中做异步请求或者更改
DOM}!一个计算属性的声明中描述的是如何根据其他值派生一个值。因此 getter
的职责应该仅为计算和返回该值。在之后的指引中我们会讨论如何使用\href{https://cn.vuejs.org/guide/essentials/watchers.html}{侦听器}根据其他响应式状态的变更来创建副作用。
\switchcolumn[0]*%%%%%%%
\subsubsection{Avoid mutating computed value}
\switchcolumn
\subsubsection{避免直接修改计算属性值}
\switchcolumn[0]*%%%%%%%
The returned value from a computed property is derived state. Think of
it as a temporary snapshot - every time the source state changes, a new
snapshot is created. It does not make sense to mutate a snapshot, so a
computed return value should be treated as read-only and never be
mutated - instead, update the source state it depends on to trigger new
computations.
\switchcolumn
从计算属性返回的值是派生状态。可以把它看作是一个``临时快照'',每当源状态发生变化时,就会创建一个新的快照。更改快照是没有意义的,因此计算属性的返回值应该被视为只读的,并且永远不应该被更改------应该更新它所依赖的源状态以触发新的计算。
\end{paracol}