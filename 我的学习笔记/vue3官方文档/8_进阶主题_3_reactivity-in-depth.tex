
\columnratio{0.55}
\begin{paracol}{2} 
 
\switchcolumn[0]*%%%%%%%
\section{Reactivity in Depth}
\switchcolumn
\section{深入响应式系统}
\switchcolumn[0]*%%%%%%%
One of Vue's most distinctive features is the unobtrusive reactivity
system. Component state consists of reactive JavaScript objects. When
you modify them, the view updates. It makes state management simple and
intuitive, but it's also important to understand how it works to avoid
some common gotchas. In this section, we are going to dig into some of
the lower-level details of Vue's reactivity system.
\switchcolumn
Vue 最标志性的功能就是其低侵入性的响应式系统。组件状态都是由响应式的
JavaScript
对象组成的。当更改它们时,视图会随即自动更新。这让状态管理更加简单直观,但理解它是如何工作的也是很重要的,这可以帮助我们避免一些常见的陷阱。在本节中,我们将深入研究
Vue 响应性系统的一些底层细节。
\switchcolumn[0]*%%%%%%%
\subsection{What is Reactivity?}
\switchcolumn
\subsection{什么是响应性}
\switchcolumn[0]*%%%%%%%
This term comes up in programming quite a bit these days, but what do
people mean when they say it? Reactivity is a programming paradigm that
allows us to adjust to changes in a declarative manner. The canonical
example that people usually show, because it's a great one, is an Excel
spreadsheet:
\switchcolumn
这个术语在今天的各种编程讨论中经常出现,但人们说它的时候究竟是想表达什么意思呢?本质上,响应性是一种可以使我们声明式地处理变化的编程范式。一个经常被拿来当作典型例子的用例即是
Excel 表格:
\switchcolumn[0]*%%%%%%%
Here cell A2 is defined via a formula of \texttt{=\ A0\ +\ A1} (you can
click on A2 to view or edit the formula), so the spreadsheet gives us 3.
No surprises there. But if you update A0 or A1, you'll notice that A2
automagically updates too.
\switchcolumn
这里单元格 A2 中的值是通过公式 \texttt{=\ A0\ +\ A1} 来定义的 (你可以在
A2 上点击来查看或编辑该公式),因此最终得到的值为
3,正如所料。但如果你试着更改 A0 或 A1,你会注意到 A2 也随即自动更新了。
\switchcolumn[0]*%%%%%%%
JavaScript doesn't usually work like this. If we were to write something
comparable in JavaScript:
\switchcolumn
而 JavaScript 默认并不是这样的。如果我们用 JavaScript 写类似的逻辑:
\switchcolumn[0]*%%%%%%%
\begin{codeJs}
let A0 = 1
let A1 = 2
let A2 = A0 + A1
console.log(A2) // 3
A0 = 2
console.log(A2) // 仍然是 3
\end{codeJs}
\switchcolumn
\begin{codeJs}
let A0 = 1
let A1 = 2
let A2 = A0 + A1
console.log(A2) // 3
A0 = 2
console.log(A2) // 仍然是 3
\end{codeJs}
\switchcolumn[0]*%%%%%%%
When we mutate \texttt{A0}, \texttt{A2} does not change automatically.
\switchcolumn
当我们更改 \texttt{A0} 后,\texttt{A2} 不会自动更新。
\switchcolumn[0]*%%%%%%%
So how would we do this in JavaScript? First, in order to re-run the
code that updates \texttt{A2}, let's wrap it in a function:
\switchcolumn
那么我们如何在 JavaScript
中做到这一点呢?首先,为了能重新运行计算的代码来更新
\texttt{A2},我们需要将其包装为一个函数:
\switchcolumn[0]*%%%%%%%
\begin{codeJs}
let A2
function update() {
  A2 = A0 + A1
}
\end{codeJs}
\switchcolumn
\begin{codeJs}
let A2
function update() {
  A2 = A0 + A1
}
\end{codeJs}
\switchcolumn[0]*%%%%%%%
Then, we need to define a few terms:
\switchcolumn
然后,我们需要定义几个术语:
\switchcolumn[0]*%%%%%%%
\begin{itemize}
\item
  The \texttt{update()} function produces a \textbf{side effect}, or
  \textbf{effect} for short, because it modifies the state of the
  program.
\item
  \texttt{A0} and \texttt{A1} are considered \textbf{dependencies} of
  the effect, as their values are used to perform the effect. The effect
  is said to be a \textbf{subscriber} to its dependencies.
\end{itemize}
\switchcolumn
\begin{itemize}
\item
  这个 \texttt{update()}
  函数会产生一个\textbf{副作用},或者就简称为\textbf{作用}
  (effect),因为它会更改程序里的状态。
\item
  \texttt{A0} 和 \texttt{A1} 被视为这个作用的\textbf{依赖}
  (dependency),因为它们的值被用来执行这个作用。因此这次作用也可以说是一个它依赖的\textbf{订阅者}
  (subscriber)。
\end{itemize}
\switchcolumn[0]*%%%%%%%
What we need is a magic function that can invoke \texttt{update()} (the
\textbf{effect}) whenever \texttt{A0} or \texttt{A1} (the
\textbf{dependencies}) change:
\switchcolumn
我们需要一个魔法函数,能够在 \texttt{A0} 或 \texttt{A1}
(这两个\textbf{依赖}) 变化时调用 \texttt{update()} (产生\textbf{作用})。
\switchcolumn[0]*%%%%%%%
\begin{codeJs}
whenDepsChange(update)
\end{codeJs}
\switchcolumn
\begin{codeJs}
whenDepsChange(update)
\end{codeJs}
\switchcolumn[0]*%%%%%%%
This \texttt{whenDepsChange()} function has the following tasks:
\switchcolumn
这个 \texttt{whenDepsChange()} 函数有如下的任务:
\switchcolumn[0]*%%%%%%%
\begin{enumerate}
\item
  Track when a variable is read. E.g. when evaluating the expression
  \texttt{A0\ +\ A1}, both \texttt{A0} and \texttt{A1} are read.
\item
  If a variable is read when there is a currently running effect, make
  that effect a subscriber to that variable. E.g. because \texttt{A0}
  and \texttt{A1} are read when \texttt{update()} is being executed,
  \texttt{update()} becomes a subscriber to both \texttt{A0} and
  \texttt{A1} after the first call.
\item
  Detect when a variable is mutated. E.g. when \texttt{A0} is assigned a
  new value, notify all its subscriber effects to re-run.
\end{enumerate}
\switchcolumn
\begin{enumerate}
\item
  当一个变量被读取时进行追踪。例如我们执行了表达式 \texttt{A0\ +\ A1}
  的计算,则 \texttt{A0} 和 \texttt{A1} 都被读取到了。
\item
  如果一个变量在当前运行的副作用中被读取了,就将该副作用设为此变量的一个订阅者。例如由于
  \texttt{A0} 和 \texttt{A1} 在 \texttt{update()} 执行时被访问到了,则
  \texttt{update()} 需要在第一次调用之后成为 \texttt{A0} 和 \texttt{A1}
  的订阅者。
\item
  探测一个变量的变化。例如当我们给 \texttt{A0}
  赋了一个新的值后,应该通知其所有订阅了的副作用重新执行。
\end{enumerate}
\end{paracol}



\columnratio{0.55}
\begin{paracol}{2} 
 
\switchcolumn[0]*%%%%%%%
\subsection{How Reactivity Works in Vue}
\switchcolumn
\subsection{Vue 中的响应性是如何工作的}
\switchcolumn[0]*%%%%%%%
We can't really track the reading and writing of local variables like in
the example. There's just no mechanism for doing that in vanilla
JavaScript. What we \textbf{can} do though, is intercept the reading and
writing of \textbf{object properties}.
\switchcolumn
我们无法直接追踪对上述示例中局部变量的读写,原生 JavaScript
没有提供任何机制能做到这一点。\textbf{但是},我们是可以追踪\textbf{对象属性}的读写的。
\switchcolumn[0]*%%%%%%%
There are two ways of intercepting property access in JavaScript:
\href{https://developer.mozilla.org/en-US/docs/Web/JavaScript/Reference/Functions/get}{getter}
/
\href{https://developer.mozilla.org/en-US/docs/Web/JavaScript/Reference/Functions/set}{setters}
and
\href{https://developer.mozilla.org/en-US/docs/Web/JavaScript/Reference/Global_Objects/Proxy}{Proxies}.
Vue 2 used getter / setters exclusively due to browser support
limitations. In Vue 3, Proxies are used for reactive objects and getter
/ setters are used for refs. Here's some pseudo-code that illustrates
how they work:
\switchcolumn
在 JavaScript 中有两种劫持 property
访问的方式:\href{https://developer.mozilla.org/en-US/docs/Web/JavaScript/Reference/Functions/get}{getter}
/
\href{https://developer.mozilla.org/en-US/docs/Web/JavaScript/Reference/Functions/set}{setters}
和
\href{https://developer.mozilla.org/en-US/docs/Web/JavaScript/Reference/Global_Objects/Proxy}{Proxies}。Vue
2 使用 getter / setters 完全是出于支持旧版本浏览器的限制。而在 Vue 3
中则使用了 Proxy 来创建响应式对象,仅将 getter / setter 用于
ref。下面的伪代码将会说明它们是如何工作的:
\switchcolumn[0]*%%%%%%%
\begin{codeJs}
function reactive(obj) {
  return new Proxy(obj, {
    get(target, key) {
      track(target, key)
      return target[key]
    },
    set(target, key, value) {
      target[key] = value
      trigger(target, key)
    }
  })
}
function ref(value) {
  const refObject = {
    get value() {
      track(refObject, 'value')
      return value
    },
    set value(newValue) {
      value = newValue
      trigger(refObject, 'value')
    }
  }
  return refObject
}
\end{codeJs}
\switchcolumn
\begin{codeJs}
function reactive(obj) {
  return new Proxy(obj, {
    get(target, key) {
      track(target, key)
      return target[key]
    },
    set(target, key, value) {
      target[key] = value
      trigger(target, key)
    }
  })
}
function ref(value) {
  const refObject = {
    get value() {
      track(refObject, 'value')
      return value
    },
    set value(newValue) {
      value = newValue
      trigger(refObject, 'value')
    }
  }
  return refObject
}
\end{codeJs}
\switchcolumn[0]*%%%%%%%
\begin{vueQuote}{TIP}
Code snippets here and below are meant to explain the core concepts in
the simplest form possible, so many details are omitted, and edge cases
ignored.
\end{vueQuote} 
\switchcolumn
\begin{vueQuote}{TIP}
这里和下面的代码片段皆旨在以最简单的形式解释核心概念,因此省略了许多细节和边界情况。
\end{vueQuote} 
\switchcolumn[0]*%%%%%%%
This explains a few
\href{https://vuejs.org/guide/essentials/reactivity-fundamentals.html\#limitations-of-reactive}{limitations
of reactive objects} that we have discussed in the fundamentals section:
\switchcolumn
以上代码解释了我们在基础章节部分讨论过的一些
\href{https://cn.vuejs.org/guide/essentials/reactivity-fundamentals.html\#limitations-of-reactive}{\texttt{reactive()}
的局限性}:
\switchcolumn[0]*%%%%%%%
\begin{itemize}
\item
  When you assign or destructure a reactive object's property to a local
  variable, accessing or assigning to that variable is non-reactive
  because it no longer triggers the get / set proxy traps on the source
  object. Note this "disconnect" only affects the variable binding - if
  the variable points to a non-primitive value such as an object,
  mutating the object would still be reactive.
\item
  The returned proxy from \texttt{reactive()}, although behaving just
  like the original, has a different identity if we compare it to the
  original using the \texttt{===} operator.
\end{itemize}
\switchcolumn
\begin{itemize}
\item
  当你将一个响应式对象的属性赋值或解构到一个本地变量时,访问或赋值该变量是非响应式的,因为它将不再触发源对象上的
  get / set
  代理。注意这种``断开''只影响变量绑定------如果变量指向一个对象之类的非原始值,那么对该对象的修改仍然是响应式的。
\item
  从 \texttt{reactive()}
  返回的代理尽管行为上表现得像原始对象,但我们通过使用 \texttt{===}
  运算符还是能够比较出它们的不同。
\end{itemize}
\switchcolumn[0]*%%%%%%%
Inside \texttt{track()}, we check whether there is a currently running
effect. If there is one, we lookup the subscriber effects (stored in a
Set) for the property being tracked, and add the effect to the Set:
\switchcolumn
在 \texttt{track()}
内部,我们会检查当前是否有正在运行的副作用。如果有,我们会查找到一个存储了所有追踪了该属性的订阅者的
Set,然后将当前这个副作用作为新订阅者添加到该 Set 中。
\switchcolumn[0]*%%%%%%%
\begin{codeJs}
// 这会在一个副作用就要运行之前被设置
// 我们会在后面处理它
let activeEffect
function track(target, key) {
  if (activeEffect) {
    const effects = getSubscribersForProperty(target, key)
    effects.add(activeEffect)
  }
}
\end{codeJs}
\switchcolumn
\begin{codeJs}
// 这会在一个副作用就要运行之前被设置
// 我们会在后面处理它
let activeEffect
function track(target, key) {
  if (activeEffect) {
    const effects = getSubscribersForProperty(target, key)
    effects.add(activeEffect)
  }
}
\end{codeJs}
\switchcolumn[0]*%%%%%%%
Effect subscriptions are stored in a global
\texttt{WeakMap\textless{}target,\ Map\textless{}key,\ Set\textless{}effect\textgreater{}\textgreater{}\textgreater{}}
data structure. If no subscribing effects Set was found for a property
(tracked for the first time), it will be created. This is what the
\texttt{getSubscribersForProperty()} function does, in short. For
simplicity, we will skip its details.
\switchcolumn
副作用订阅将被存储在一个全局的
\texttt{WeakMap\textless{}target,\ Map\textless{}key,\ Set\textless{}effect\textgreater{}\textgreater{}\textgreater{}}
数据结构中。如果在第一次追踪时没有找到对相应属性订阅的副作用集合,它将会在这里新建。这就是
\texttt{getSubscribersForProperty()}
函数所做的事。为了简化描述,我们跳过了它其中的细节。
\switchcolumn[0]*%%%%%%%
Inside \texttt{trigger()}, we again lookup the subscriber effects for
the property. But this time we invoke them instead:
\switchcolumn
在 \texttt{trigger()}
之中,我们会再查找到该属性的所有订阅副作用。但这一次我们需要执行它们:
\switchcolumn[0]*%%%%%%%
\begin{codeJs}
function trigger(target, key) {
  const effects = getSubscribersForProperty(target, key)
  effects.forEach((effect) => effect())
}
\end{codeJs}
\switchcolumn
\begin{codeJs}
function trigger(target, key) {
  const effects = getSubscribersForProperty(target, key)
  effects.forEach((effect) => effect())
}
\end{codeJs}
\switchcolumn[0]*%%%%%%%
Now let's circle back to the \texttt{whenDepsChange()} function:
\switchcolumn
现在让我们回到 \texttt{whenDepsChange()} 函数中:
\switchcolumn[0]*%%%%%%%
\begin{codeJs}
function whenDepsChange(update) {
  const effect = () => {
    activeEffect = effect
    update()
    activeEffect = null
  }
  effect()
}
\end{codeJs}
\switchcolumn
\begin{codeJs}
function whenDepsChange(update) {
  const effect = () => {
    activeEffect = effect
    update()
    activeEffect = null
  }
  effect()
}
\end{codeJs}
\switchcolumn[0]*%%%%%%%
It wraps the raw \texttt{update} function in an effect that sets itself
as the current active effect before running the actual update. This
enables \texttt{track()} calls during the update to locate the current
active effect.
\switchcolumn
它将原本的 \texttt{update}
函数包装在了一个副作用函数中。在运行实际的更新之前,这个外部函数会将自己设为当前活跃的副作用。这使得在更新期间的
\texttt{track()} 调用都能定位到这个当前活跃的副作用。
\switchcolumn[0]*%%%%%%%
At this point, we have created an effect that automatically tracks its
dependencies, and re-runs whenever a dependency changes. We call this a
\textbf{Reactive Effect}.
\switchcolumn
此时,我们已经创建了一个能自动跟踪其依赖的副作用,它会在任意依赖被改动时重新运行。我们称其为\textbf{响应式副作用}。
\switchcolumn[0]*%%%%%%%
Vue provides an API that allows you to create reactive effects:
\href{https://vuejs.org/api/reactivity-core.html\#watcheffect}{\texttt{watchEffect()}}.
In fact, you may have noticed that it works pretty similarly to the
magical \texttt{whenDepsChange()} in the example. We can now rework the
original example using actual Vue APIs:
\switchcolumn
Vue 提供了一个 API 来让你创建响应式副作用
\href{https://cn.vuejs.org/api/reactivity-core.html\#watcheffect}{\texttt{watchEffect()}}。事实上,你会发现它的使用方式和我们上面示例中说的魔法函数
\texttt{whenDepsChange()} 非常相似。我们可以用真正的 Vue API
改写上面的例子:
\switchcolumn[0]*%%%%%%%
\begin{codeJs}
import { ref, watchEffect } from 'vue'
const A0 = ref(0)
const A1 = ref(1)
const A2 = ref()
watchEffect(() => {
  // 追踪 A0 和 A1
  A2.value = A0.value + A1.value
})
// 将触发副作用
A0.value = 2
\end{codeJs}
\switchcolumn
\begin{codeJs}
import { ref, watchEffect } from 'vue'
const A0 = ref(0)
const A1 = ref(1)
const A2 = ref()
watchEffect(() => {
  // 追踪 A0 和 A1
  A2.value = A0.value + A1.value
})
// 将触发副作用
A0.value = 2
\end{codeJs}
\switchcolumn[0]*%%%%%%%
Using a reactive effect to mutate a ref isn't the most interesting use
case - in fact, using a computed property makes it more declarative:
\switchcolumn
使用一个响应式副作用来更改一个 ref
并不是最优解,事实上使用计算属性会更直观简洁:
\switchcolumn[0]*%%%%%%%
\begin{codeJs}
import { ref, computed } from 'vue'
const A0 = ref(0)
const A1 = ref(1)
const A2 = computed(() => A0.value + A1.value)
A0.value = 2
\end{codeJs}
\switchcolumn
\begin{codeJs}
import { ref, computed } from 'vue'
const A0 = ref(0)
const A1 = ref(1)
const A2 = computed(() => A0.value + A1.value)
A0.value = 2
\end{codeJs}
\switchcolumn[0]*%%%%%%%
Internally, \texttt{computed} manages its invalidation and
re-computation using a reactive effect.
\switchcolumn
在内部,\texttt{computed} 会使用响应式副作用来管理失效与重新计算的过程。
\switchcolumn[0]*%%%%%%%
So what's an example of a common and useful reactive effect? Well,
updating the DOM! We can implement simple "reactive rendering" like
this:
\switchcolumn
那么,常见的响应式副作用的用例是什么呢?自然是更新
DOM!我们可以像下面这样实现一个简单的``响应式渲染'':
\switchcolumn[0]*%%%%%%%
\begin{codeJs}
import { ref, watchEffect } from 'vue'
const count = ref(0)
watchEffect(() => {
  document.body.innerHTML = `计数:${count.value}`
})
// 更新 DOM
count.value++
\end{codeJs}
\switchcolumn
\begin{codeJs}
import { ref, watchEffect } from 'vue'
const count = ref(0)
watchEffect(() => {
  document.body.innerHTML = `计数:${count.value}`
})
// 更新 DOM
count.value++
\end{codeJs}
\switchcolumn[0]*%%%%%%%
In fact, this is pretty close to how a Vue component keeps the state and
the DOM in sync - each component instance creates a reactive effect to
render and update the DOM. Of course, Vue components use much more
efficient ways to update the DOM than \texttt{innerHTML}. This is
discussed in
\href{https://vuejs.org/guide/extras/rendering-mechanism.html}{Rendering
Mechanism}.
\switchcolumn
实际上,这与 Vue 组件保持状态和 DOM
同步的方式非常接近------每个组件实例创建一个响应式副作用来渲染和更新
DOM。当然,Vue 组件使用了比 \texttt{innerHTML} 更高效的方式来更新
DOM。这会在\href{https://cn.vuejs.org/guide/extras/rendering-mechanism.html}{渲染机制}一章中详细介绍。
\end{paracol}


\columnratio{0.55}
\begin{paracol}{2} 
 
\switchcolumn[0]*%%%%%%%
\subsection{Runtime vs. Compile-time Reactivity}
\switchcolumn
\subsection{运行时 vs. 编译时响应性}
\switchcolumn[0]*%%%%%%%
Vue's reactivity system is primarily runtime-based: the tracking and
triggering are all performed while the code is running directly in the
browser. The pros of runtime reactivity are that it can work without a
build step, and there are fewer edge cases. On the other hand, this
makes it constrained by the syntax limitations of JavaScript, leading to
the need of value containers like Vue refs.
\switchcolumn
Vue
的响应式系统基本是基于运行时的。追踪和触发都是在浏览器中运行时进行的。运行时响应性的优点是,它可以在没有构建步骤的情况下工作,而且边界情况较少。另一方面,这使得它受到了
JavaScript 语法的制约,导致需要使用一些例如 Vue ref 这样的值的容器。
\switchcolumn[0]*%%%%%%%
Some frameworks, such as \href{https://svelte.dev/}{Svelte}, choose to
overcome such limitations by implementing reactivity during compilation.
It analyzes and transforms the code in order to simulate reactivity. The
compilation step allows the framework to alter the semantics of
JavaScript itself - for example, implicitly injecting code that performs
dependency analysis and effect triggering around access to locally
defined variables. The downside is that such transforms require a build
step, and altering JavaScript semantics is essentially creating a
language that looks like JavaScript but compiles into something else.
\switchcolumn
一些框架,如
\href{https://svelte.dev/}{Svelte},选择通过编译时实现响应性来克服这种限制。它对代码进行分析和转换,以模拟响应性。该编译步骤允许框架改变
JavaScript
本身的语义------例如,隐式地注入执行依赖性分析的代码,以及围绕对本地定义的变量的访问进行作用触发。这样做的缺点是,该转换需要一个构建步骤,而改变
JavaScript 的语义实质上是在创造一种新语言,看起来像 JavaScript
但编译出来的东西是另外一回事。
\switchcolumn[0]*%%%%%%%
The Vue team did explore this direction via an experimental feature
called
\href{https://vuejs.org/guide/extras/reactivity-transform.html}{Reactivity
Transform}, but in the end we have decided that it would not be a good
fit for the project due to
\href{https://github.com/vuejs/rfcs/discussions/369\#discussioncomment-5059028}{the
reasoning here}.
\switchcolumn
Vue
团队确实曾通过一个名为\href{https://cn.vuejs.org/guide/extras/reactivity-transform.html}{响应性语法糖}的实验性功能来探索这个方向,但最后由于\href{https://github.com/vuejs/rfcs/discussions/369\#discussioncomment-5059028}{这个原因},我们认为它不适合这个项目。
\switchcolumn[0]*%%%%%%%
\subsection{Reactivity Debugging}
\switchcolumn
\subsection{响应性调试}
\switchcolumn[0]*%%%%%%%
It's great that Vue's reactivity system automatically tracks
dependencies, but in some cases we may want to figure out exactly what
is being tracked, or what is causing a component to re-render.
\switchcolumn
Vue
的响应性系统可以自动跟踪依赖关系,但在某些情况下,我们可能希望确切地知道正在跟踪什么,或者是什么导致了组件重新渲染。
\end{paracol}



\columnratio{0.55}
\begin{paracol}{2} 
 
\switchcolumn[0]*%%%%%%%
\subsubsection{Component Debugging Hooks}
\switchcolumn
\subsubsection{组件调试钩子}
\switchcolumn[0]*%%%%%%%
We can debug what dependencies are used during a component's render and
which dependency is triggering an update using the
\texttt{onRenderTracked} and \texttt{onRenderTriggered} lifecycle hooks.
Both hooks will receive a debugger event which contains information on
the dependency in question. It is recommended to place a
\texttt{debugger} statement in the callbacks to interactively inspect
the dependency:
\switchcolumn
我们可以在一个组件渲染时使用 \texttt{onRenderTracked}
生命周期钩子来调试查看哪些依赖正在被使用,或是用
\texttt{onRenderTriggered}
来确定哪个依赖正在触发更新。这些钩子都会收到一个调试事件,其中包含了触发相关事件的依赖的信息。推荐在回调中放置一个
\texttt{debugger} 语句,使你可以在开发者工具中交互式地查看依赖:
\switchcolumn[0]*%%%%%%%
\begin{codeHtml}
<script setup>
import { onRenderTracked, onRenderTriggered } from 'vue'
onRenderTracked((event) => {
  debugger
})
onRenderTriggered((event) => {
  debugger
})
</script>
\end{codeHtml}
\switchcolumn
\begin{codeHtml}
<script setup>
import { onRenderTracked, onRenderTriggered } from 'vue'
onRenderTracked((event) => {
  debugger
})
onRenderTriggered((event) => {
  debugger
})
</script>
\end{codeHtml}
\switchcolumn[0]*%%%%%%%
\begin{vueQuote}{TIP}
Component debug hooks only work in development mode.
\end{vueQuote} 
\switchcolumn
\begin{vueQuote}{TIP}
组件调试钩子仅会在开发模式下工作
\end{vueQuote} 
\switchcolumn[0]*%%%%%%%
The debug event objects have the following type:
\switchcolumn
调试事件对象有如下的类型定义:
\switchcolumn[0]*%%%%%%%
\begin{codeTs}
type DebuggerEvent = {
  effect: ReactiveEffect
  target: object
  type:
    | TrackOpTypes /* 'get' | 'has' | 'iterate' */
    | TriggerOpTypes /* 'set' | 'add' | 'delete' | 'clear' */
  key: any
  newValue?: any
  oldValue?: any
  oldTarget?: Map<any, any> | Set<any>
}
\end{codeTs}
\switchcolumn
\begin{codeTs}
type DebuggerEvent = {
  effect: ReactiveEffect
  target: object
  type:
    | TrackOpTypes /* 'get' | 'has' | 'iterate' */
    | TriggerOpTypes /* 'set' | 'add' | 'delete' | 'clear' */
  key: any
  newValue?: any
  oldValue?: any
  oldTarget?: Map<any, any> | Set<any>
}
\end{codeTs}
\switchcolumn[0]*%%%%%%%
\subsubsection{Computed Debugging}
\switchcolumn
\subsubsection{计算属性调试}
\switchcolumn[0]*%%%%%%%
We can debug computed properties by passing \texttt{computed()} a second
options object with \texttt{onTrack} and \texttt{onTrigger} callbacks:
\switchcolumn
我们可以向 \texttt{computed()} 传入第二个参数,是一个包含了
\texttt{onTrack} 和 \texttt{onTrigger} 两个回调函数的对象:
\switchcolumn[0]*%%%%%%%
\begin{itemize}
\item
  \texttt{onTrack} will be called when a reactive property or ref is
  tracked as a dependency.
\item
  \texttt{onTrigger} will be called when the watcher callback is
  triggered by the mutation of a dependency.
\end{itemize}
\switchcolumn
\begin{itemize}
\item
  \texttt{onTrack} 将在响应属性或引用作为依赖项被跟踪时被调用。
\item
  \texttt{onTrigger} 将在侦听器回调被依赖项的变更触发时被调用。
\end{itemize}
\switchcolumn[0]*%%%%%%%
Both callbacks will receive debugger events in the
\href{https://vuejs.org/guide/extras/reactivity-in-depth.html\#debugger-event}{same
format} as component debug hooks:
\switchcolumn
这两个回调都会作为组件调试的钩子,接受\href{https://cn.vuejs.org/guide/extras/reactivity-in-depth.html\#debugger-event}{相同格式}的调试事件:
\switchcolumn[0]*%%%%%%%
\begin{codeJs}
const plusOne = computed(() => count.value + 1, {
  onTrack(e) {
    // 当 count.value 被追踪为依赖时触发
    debugger
  },
  onTrigger(e) {
    // 当 count.value 被更改时触发
    debugger
  }
})
// 访问 plusOne,会触发 onTrack
console.log(plusOne.value)
// 更改 count.value,应该会触发 onTrigger
count.value++
\end{codeJs}
\switchcolumn
\begin{codeJs}
const plusOne = computed(() => count.value + 1, {
  onTrack(e) {
    // 当 count.value 被追踪为依赖时触发
    debugger
  },
  onTrigger(e) {
    // 当 count.value 被更改时触发
    debugger
  }
})
// 访问 plusOne,会触发 onTrack
console.log(plusOne.value)
// 更改 count.value,应该会触发 onTrigger
count.value++
\end{codeJs}
\switchcolumn[0]*%%%%%%%
\begin{vueQuote}{TIP}
\texttt{onTrack} and \texttt{onTrigger} computed options only work in
development mode.
\end{vueQuote} 
\switchcolumn
\begin{vueQuote}{TIP}
计算属性的 \texttt{onTrack} 和 \texttt{onTrigger}
选项仅会在开发模式下工作。
\end{vueQuote} 
\end{paracol}



\columnratio{0.55}
\begin{paracol}{2} 
 
\switchcolumn[0]*%%%%%%%
\subsubsection{Watcher Debugging}
\switchcolumn
\subsubsection{侦听器调试}
\switchcolumn[0]*%%%%%%%
Similar to \texttt{computed()}, watchers also support the
\texttt{onTrack} and \texttt{onTrigger} options:
\switchcolumn
和 \texttt{computed()} 类似,侦听器也支持 \texttt{onTrack} 和
\texttt{onTrigger} 选项:

\end{paracol}

\columnratio{0.55}
\begin{paracol}{2}
\switchcolumn[0]*%%%%%%%
\begin{codeJs}
watch(source, callback, {
  onTrack(e) {
    debugger
  },
  onTrigger(e) {
    debugger
  }
})
watchEffect(callback, {
  onTrack(e) {
    debugger
  },
  onTrigger(e) {
    debugger
  }
})
\end{codeJs}
\switchcolumn
\begin{codeJs}
watch(source, callback, {
  onTrack(e) {
    debugger
  },
  onTrigger(e) {
    debugger
  }
})
watchEffect(callback, {
  onTrack(e) {
    debugger
  },
  onTrigger(e) {
    debugger
  }
})
\end{codeJs}
\end{paracol}

\columnratio{0.55}
\begin{paracol}{2}
\switchcolumn[0]*%%%%%%%
~\begin{vueQuote}{TIP}
\texttt{onTrack} and \texttt{onTrigger} watcher options only work in
development mode.
\end{vueQuote}
\switchcolumn
~\begin{vueQuote}{TIP}
侦听器的 \texttt{onTrack} 和 \texttt{onTrigger}
选项仅会在开发模式下工作。
\end{vueQuote} 
\switchcolumn[0]*%%%%%%%
\subsection{Integration with External State Systems}
\switchcolumn
\subsection{与外部状态系统集成}
\switchcolumn[0]*%%%%%%%
Vue's reactivity system works by deeply converting plain JavaScript
objects into reactive proxies. The deep conversion can be unnecessary or
sometimes unwanted when integrating with external state management
systems (e.g. if an external solution also uses Proxies).
\switchcolumn
Vue 的响应性系统是通过深度转换普通 JavaScript
对象为响应式代理来实现的。这种深度转换在一些情况下是不必要的,在和一些外部状态管理系统集成时,甚至是需要避免的
(例如,当一个外部的解决方案也用了 Proxy 时)。
\switchcolumn[0]*%%%%%%%
The general idea of integrating Vue's reactivity system with an external
state management solution is to hold the external state in a
\href{https://vuejs.org/api/reactivity-advanced.html\#shallowref}{\texttt{shallowRef}}.
A shallow ref is only reactive when its \texttt{.value} property is
accessed - the inner value is left intact. When the external state
changes, replace the ref value to trigger updates.
\switchcolumn
将 Vue
的响应性系统与外部状态管理方案集成的大致思路是:将外部状态放在一个
\href{https://cn.vuejs.org/api/reactivity-advanced.html\#shallowref}{\texttt{shallowRef}}
中。一个浅层的 ref 中只有它的 \texttt{.value}
属性本身被访问时才是有响应性的,而不关心它内部的值。当外部状态改变时,替换此
ref 的 \texttt{.value} 才会触发更新。
\switchcolumn[0]*%%%%%%%
\subsubsection{Immutable Data}
\switchcolumn
\subsubsection{不可变数据}
\switchcolumn[0]*%%%%%%%
If you are implementing an undo / redo feature, you likely want to take
a snapshot of the application's state on every user edit. However, Vue's
mutable reactivity system isn't best suited for this if the state tree
is large, because serializing the entire state object on every update
can be expensive in terms of both CPU and memory costs.
\switchcolumn
如果你正在实现一个撤销/重做的功能,你可能想要对用户编辑时应用的状态进行快照记录。然而,如果状态树很大的话,Vue
的可变响应性系统没法很好地处理这种情况,因为在每次更新时都序列化整个状态对象对
CPU 和内存开销来说都是非常昂贵的。
\switchcolumn[0]*%%%%%%%
\href{https://en.wikipedia.org/wiki/Persistent_data_structure}{Immutable
data structures} solve this by never mutating the state objects -
instead, it creates new objects that share the same, unchanged parts
with old ones. There are different ways of using immutable data in
JavaScript, but we recommend using
\href{https://immerjs.github.io/immer/}{Immer} with Vue because it
allows you to use immutable data while keeping the more ergonomic,
mutable syntax.
\switchcolumn
\href{https://en.wikipedia.org/wiki/Persistent_data_structure}{不可变数据结构}通过永不更改状态对象来解决这个问题。与
Vue 不同的是,它会创建一个新对象,保留旧的对象未发生改变的一部分。在
JavaScript 中有多种不同的方式来使用不可变数据,但我们推荐使用
\href{https://immerjs.github.io/immer/}{Immer} 搭配
Vue,因为它使你可以在保持原有直观、可变的语法的同时,使用不可变数据。
\switchcolumn[0]*%%%%%%%
We can integrate Immer with Vue via a simple composable:
\switchcolumn
我们可以通过一个简单的组合式函数来集成 Immer:
\switchcolumn[0]*%%%%%%%
\begin{codeJs}
import produce from 'immer'
import { shallowRef } from 'vue'
export function useImmer(baseState) {
  const state = shallowRef(baseState)
  const update = (updater) => {
    state.value = produce(state.value, updater)
  }
  return [state, update]
}
\end{codeJs}
\switchcolumn
\begin{codeJs}
import produce from 'immer'
import { shallowRef } from 'vue'
export function useImmer(baseState) {
  const state = shallowRef(baseState)
  const update = (updater) => {
    state.value = produce(state.value, updater)
  }
  return [state, update]
}
\end{codeJs}
\switchcolumn[0]*%%%%%%%
\href{https://play.vuejs.org/\#eNplU8Fu2zAM/RXOlzpAYu82zEu67lhgpw3bJcrBs5VYqywJkpxmMPzvoyjZNRodbJF84iOppzH7ZkxxHXhWZXvXWGE8OO4H88iU6I22HkYYHH/ue25hgrPVPTwUpQh28dc9MAXAVKOV83AUnvduC4Npa8+fg3GCw3I8PwbwGD64vPCSV8Cy77y2Cn4PnGXbFGu1wpC36EPHRO67c78cD6fgVfgOiOB9gnMtXczA1GnDFFPnQTVeaAVeXy6SSsyFavltE/OvKs+pGTg8zsxkHwl9KgIBtvbhzkl0yIWU+zIOFEeJBgKNxORoAewHSX/cSQHX3VnbA8vyMXa3pfqxb0i1CRXZWZb6w1U1snYOT40JvQ4+NVI0Lxi865NliTisMRHChOVSNaUUscCSKtyXq7LRdP6fDNvYPw3G85vftbzRtg6TrUAKxXe+s3q4dF/mQdC5bJtFTe362qB4tELVURKWAthhNc87+OhSw2V33htXleWgzMulaHQfFfj0ufhYfCpb4XySJHc9Zv7a63aQqKh0+xNRR8kiZ1K2sYhqeBI1xVHPi+xdV0upX3/w8yJ8fCiIYIrfCLPIaZH4n9rxnx7nlQQVH4YLHpTLW8YV8A0W1Ye4PO7sZiU/ylFca4mSP8yl5yvv/O4sZcSmw8/iW8bXdSTcjDiFgUz/AcH6WZQ=}{Try
it in the Playground}
\switchcolumn
\href{https://play.vuejs.org/\#eNplU8Fu2zAM/RXOlzpAYu82zEu67lhgpw3bJcrBs5VYqywJkpxmMPzvoyjZNRodbJF84iOppzH7ZkxxHXhWZXvXWGE8OO4H88iU6I22HkYYHH/ue25hgrPVPTwUpQh28dc9MAXAVKOV83AUnvduC4Npa8+fg3GCw3I8PwbwGD64vPCSV8Cy77y2Cn4PnGXbFGu1wpC36EPHRO67c78cD6fgVfgOiOB9gnMtXczA1GnDFFPnQTVeaAVeXy6SSsyFavltE/OvKs+pGTg8zsxkHwl9KgIBtvbhzkl0yIWU+zIOFEeJBgKNxORoAewHSX/cSQHX3VnbA8vyMXa3pfqxb0i1CRXZWZb6w1U1snYOT40JvQ4+NVI0Lxi865NliTisMRHChOVSNaUUscCSKtyXq7LRdP6fDNvYPw3G85vftbzRtg6TrUAKxXe+s3q4dF/mQdC5bJtFTe362qB4tELVURKWAthhNc87+OhSw2V33htXleWgzMulaHQfFfj0ufhYfCpb4XySJHc9Zv7a63aQqKh0+xNRR8kiZ1K2sYhqeBI1xVHPi+xdV0upX3/w8yJ8fCiIYIrfCLPIaZH4n9rxnx7nlQQVH4YLHpTLW8YV8A0W1Ye4PO7sZiU/ylFca4mSP8yl5yvv/O4sZcSmw8/iW8bXdSTcjDiFgUz/AcH6WZQ=}{在演练场中尝试一下}
\end{paracol}



\columnratio{0.55}
\begin{paracol}{2} 
 
\switchcolumn[0]*%%%%%%%
\subsubsection{State Machines}
\switchcolumn
\subsubsection{状态机}
\switchcolumn[0]*%%%%%%%
\href{https://en.wikipedia.org/wiki/Finite-state_machine}{State Machine}
is a model for describing all the possible states an application can be
in, and all the possible ways it can transition from one state to
another. While it may be overkill for simple components, it can help
make complex state flows more robust and manageable.
\switchcolumn
\href{https://en.wikipedia.org/wiki/Finite-state_machine}{状态机}是一种数据模型,用于描述应用可能处于的所有可能状态,以及从一种状态转换到另一种状态的所有可能方式。虽然对于简单的组件来说,这可能有些小题大做了,但它的确可以使得复杂的状态流更加健壮和易于管理。
\switchcolumn[0]*%%%%%%%
One of the most popular state machine implementations in JavaScript is
\href{https://xstate.js.org/}{XState}. Here's a composable that
integrates with it:
\switchcolumn
\href{https://xstate.js.org/}{XState} 是 JavaScript
中一个比较常用的状态机实现方案。这里是集成它的一个例子:
\switchcolumn[0]*%%%%%%%
\begin{codeJs}
import { createMachine, interpret } from 'xstate'
import { shallowRef } from 'vue'
export function useMachine(options) {
  const machine = createMachine(options)
  const state = shallowRef(machine.initialState)
  const service = interpret(machine)
    .onTransition((newState) => (state.value = newState))
    .start()
  const send = (event) => service.send(event)
  return [state, send]
}
\end{codeJs}
\switchcolumn
\begin{codeJs}
import { createMachine, interpret } from 'xstate'
import { shallowRef } from 'vue'
export function useMachine(options) {
  const machine = createMachine(options)
  const state = shallowRef(machine.initialState)
  const service = interpret(machine)
    .onTransition((newState) => (state.value = newState))
    .start()
  const send = (event) => service.send(event)
  return [state, send]
}
\end{codeJs}
\switchcolumn[0]*%%%%%%%
\href{https://play.vuejs.org/\#eNp1U81unDAQfpWRL7DSFqqqUiXEJumhyqVVpDa3ugcKZtcJjC1syEqId8/YBu/uIRcEM9/P/DGz71pn0yhYwUpTD1JbMMKO+o6j7LUaLMwwGvGrqk8SBSzQDqqHJMv7EMleTMIRgGOt0Fj4a2xlxZ5EsPkHhytuOjucbApIrDoeO5HsfQCllVVHUYlVbeW0xr2OKcCzHCwkKQAK3fP56fHx5w/irSyqbfFMgA+h0cKBHZYey45jmYfeqWv6sKLXHbnTF0D5f7RWITzUnaxfD5y5ztIkSCY7zjwKYJ5DyVlf2fokTMrZ5sbZDu6Bs6e25QwK94b0svgKyjwYkEyZR2e2Z2H8n/pK04wV0oL8KEjWJwxncTicnb23C3F2slabIs9H1K/HrFZ9HrIPX7Mv37LPuTC5xEacSfa+V83YEW+bBfleFkuW8QbqQZDEuso9rcOKQQ/CxosIHnQLkWJOVdept9+ijSA6NEJwFGePaUekAdFwr65EaRcxu9BbOKq1JDqnmzIi9oL0RRDu4p1u/ayH9schrhlimGTtOLGnjeJRAJnC56FCQ3SFaYriLWjA4Q7SsPOp6kYnEXMbldKDTW/ssCFgKiaB1kusBWT+rkLYjQiAKhkHvP2j3IqWd5iMQ+M=}{Try
it in the Playground}
\switchcolumn
\href{https://play.vuejs.org/\#eNp1U81unDAQfpWRL7DSFqqqUiXEJumhyqVVpDa3ugcKZtcJjC1syEqId8/YBu/uIRcEM9/P/DGz71pn0yhYwUpTD1JbMMKO+o6j7LUaLMwwGvGrqk8SBSzQDqqHJMv7EMleTMIRgGOt0Fj4a2xlxZ5EsPkHhytuOjucbApIrDoeO5HsfQCllVVHUYlVbeW0xr2OKcCzHCwkKQAK3fP56fHx5w/irSyqbfFMgA+h0cKBHZYey45jmYfeqWv6sKLXHbnTF0D5f7RWITzUnaxfD5y5ztIkSCY7zjwKYJ5DyVlf2fokTMrZ5sbZDu6Bs6e25QwK94b0svgKyjwYkEyZR2e2Z2H8n/pK04wV0oL8KEjWJwxncTicnb23C3F2slabIs9H1K/HrFZ9HrIPX7Mv37LPuTC5xEacSfa+V83YEW+bBfleFkuW8QbqQZDEuso9rcOKQQ/CxosIHnQLkWJOVdept9+ijSA6NEJwFGePaUekAdFwr65EaRcxu9BbOKq1JDqnmzIi9oL0RRDu4p1u/ayH9schrhlimGTtOLGnjeJRAJnC56FCQ3SFaYriLWjA4Q7SsPOp6kYnEXMbldKDTW/ssCFgKiaB1kusBWT+rkLYjQiAKhkHvP2j3IqWd5iMQ+M=}{在演练场中尝试一下}
\switchcolumn[0]*%%%%%%%
\subsubsection{RxJS}
\switchcolumn
\subsubsection{RxJS}
\switchcolumn[0]*%%%%%%%
\href{https://rxjs.dev/}{RxJS} is a library for working with
asynchronous event streams. The \href{https://vueuse.org/}{VueUse}
library provides the
\href{https://vueuse.org/rxjs/readme.html}{\texttt{@vueuse/rxjs}} add-on
for connecting RxJS streams with Vue's reactivity system.
\switchcolumn
\href{https://rxjs.dev/}{RxJS}
是一个用于处理异步事件流的库。\href{https://vueuse.org/}{VueUse}
库提供了
\href{https://vueuse.org/rxjs/readme.html}{\texttt{@vueuse/rxjs}}
扩展来支持连接 RxJS 流与 Vue 的响应性系统。
\switchcolumn[0]*%%%%%%%
\subsection{Connection to Signals}
\switchcolumn
\subsection{与信号 (signal) 的联系}
\switchcolumn[0]*%%%%%%%
Quite a few other frameworks have introduced reactivity primitives
similar to refs from Vue's Composition API, under the term "signals":
\switchcolumn
很多其他框架已经引入了与 Vue 组合式 API 中的 ref
类似的响应性基础类型,并称之为``信号'':
\switchcolumn[0]*%%%%%%%
\begin{itemize}
\item
  \href{https://www.solidjs.com/docs/latest/api\#createsignal}{Solid
  Signals}
\item
  \href{https://github.com/angular/angular/discussions/49090}{Angular
  Signals}
\item
  \href{https://preactjs.com/guide/v10/signals/}{Preact Signals}
\item
  \href{https://qwik.builder.io/docs/components/state/\#usesignal}{Qwik
  Signals}
\end{itemize}
\switchcolumn
\begin{itemize}
\item
  \href{https://www.solidjs.com/docs/latest/api\#createsignal}{Solid
  信号}
\item
  \href{https://github.com/angular/angular/discussions/49090}{Angular
  信号}
\item
  \href{https://preactjs.com/guide/v10/signals/}{Preact 信号}
\item
  \href{https://qwik.builder.io/docs/components/state/\#usesignal}{Qwik
  信号}
\end{itemize}
\switchcolumn[0]*%%%%%%%
Fundamentally, signals are the same kind of reactivity primitive as Vue
refs. It's a value container that provides dependency tracking on
access, and side-effect triggering on mutation. This
reactivity-primitive-based paradigm isn't a particularly new concept in
the frontend world: it dates back to implementations like
\href{https://knockoutjs.com/documentation/observables.html}{Knockout
observables} and \href{https://docs.meteor.com/api/tracker.html}{Meteor
Tracker} from more than a decade ago. Vue Options API and the React
state management library \href{https://mobx.js.org/}{MobX} are also
based on the same principles, but hide the primitives behind object
properties.
\switchcolumn
从根本上说,信号是与 Vue 中的 ref
相同的响应性基础类型。它是一个在访问时跟踪依赖、在变更时触发副作用的值容器。这种基于响应性基础类型的范式在前端领域并不是一个特别新的概念:它可以追溯到十多年前的
\href{https://knockoutjs.com/documentation/observables.html}{Knockout
observables} 和 \href{https://docs.meteor.com/api/tracker.html}{Meteor
Tracker} 等实现。Vue 的选项式 API 和 React 的状态管理库
\href{https://mobx.js.org/}{MobX}
也是基于同样的原则,只不过将基础类型这部分隐藏在了对象属性背后。
\switchcolumn[0]*%%%%%%%
Although not a necessary trait for something to qualify as signals,
today the concept is often discussed alongside the rendering model where
updates are performed through fine-grained subscriptions. Due to the use
of Virtual DOM, Vue currently
\href{https://vuejs.org/guide/extras/rendering-mechanism.html\#compiler-informed-virtual-dom}{relies
on compilers to achieve similar optimizations}. However, we are also
exploring a new Solid-inspired compilation strategy (Vapor Mode) that
does not rely on Virtual DOM and takes more advantage of Vue's built-in
reactivity system.
\switchcolumn
虽然这并不是信号的必要特征,但如今这个概念经常与细粒度订阅和更新的渲染模型一起讨论。由于使用了虚拟
DOM,Vue
目前\href{https://cn.vuejs.org/guide/extras/rendering-mechanism.html\#compiler-informed-virtual-dom}{依靠编译器来实现类似的优化}。然而,我们也在探索一种新的受
Solid 启发的编译策略 (Vapor Mode),它不依赖于虚拟 DOM,而是更多地利用
Vue 的内置响应性系统。
\switchcolumn[0]*%%%%%%%
\subsubsection{API Design Trade-Offs}
\switchcolumn
\subsubsection{API 设计权衡}
\switchcolumn[0]*%%%%%%%
The design of Preact and Qwik's signals are very similar to Vue's
\href{https://vuejs.org/api/reactivity-advanced.html\#shallowref}{shallowRef}:
all three provide a mutable interface via the \texttt{.value} property.
We will focus the discussion on Solid and Angular signals.
\switchcolumn
Preact 和 Qwik 的信号设计与 Vue 的
\href{https://cn.vuejs.org/api/reactivity-advanced.html\#shallowref}{shallowRef}
非常相似:三者都通过 \texttt{.value}
属性提供了一个更改接口。我们将重点讨论 Solid 和 Angular 的信号。
\switchcolumn[0]*%%%%%%%
\paragraph{Solid Signals}
\switchcolumn
\paragraph{Solid Signals}
\switchcolumn[0]*%%%%%%%
Solid's \texttt{createSignal()} API design emphasizes read / write
segregation. Signals are exposed as a read-only getter and a separate
setter:
\switchcolumn
Solid 的 \texttt{createSignal()} API
设计强调了读/写隔离。信号通过一个只读的 getter 和另一个单独的 setter
暴露:
\switchcolumn[0]*%%%%%%%
\begin{codeJs}
const [count, setCount] = createSignal(0)
count() // 访问值
setCount(1) // 更新值
\end{codeJs}
\switchcolumn
\begin{codeJs}
const [count, setCount] = createSignal(0)
count() // 访问值
setCount(1) // 更新值
\end{codeJs}
\switchcolumn[0]*%%%%%%%
Notice how the \texttt{count} signal can be passed down without the
setter. This ensures that the state can never be mutated unless the
setter is also explicitly exposed. Whether this safety guarantee
justifies the more verbose syntax could be subject to the requirement of
the project and personal taste - but in case you prefer this API style,
you can easily replicate it in Vue:
\switchcolumn
注意到 \texttt{count} 信号在没有 setter 的情况也能传递。这就保证了除非
setter
也被明确暴露,否则状态永远不会被改变。这种更冗长的语法带来的安全保证的合理性取决于项目的要求和个人品味------但如果你喜欢这种
API 风格,可以轻易地在 Vue 中复制它:
\switchcolumn[0]*%%%%%%%
\begin{codeJs}
import { shallowRef, triggerRef } from 'vue'
export function createSignal(value, options) {
  const r = shallowRef(value)
  const get = () => r.value
  const set = (v) => {
    r.value = typeof v === 'function' ? v(r.value) : v
    if (options?.equals === false) triggerRef(r)
  }
  return [get, set]
}
\end{codeJs}
\switchcolumn
\begin{codeJs}
import { shallowRef, triggerRef } from 'vue'
export function createSignal(value, options) {
  const r = shallowRef(value)
  const get = () => r.value
  const set = (v) => {
    r.value = typeof v === 'function' ? v(r.value) : v
    if (options?.equals === false) triggerRef(r)
  }
  return [get, set]
}
\end{codeJs}
\switchcolumn[0]*%%%%%%%
\href{https://play.vuejs.org/\#eNpdUk1TgzAQ/Ss7uQAjgr12oNXxH+ix9IAYaDQkMV/qMPx3N6G0Uy9Msu/tvn2PTORJqcI7SrakMp1myoKh1qldI9iopLYwQadpa+krG0TLYYZeyxGSojSSs/d7E8vFh0ka0YhOCmPh0EknbB4mPYfTEeqbIelD1oiqXPRQCS+WjoojAW8A1Wmzm1A39KYZzHNVYiUib85aKeCx46z7rBuySqQe6h14uINN1pDIBWACVUcqbGwtl17EqvIiR3LyzwcmcXFuTi3n8vuF9jlYzYaBajxfMsDcomv6E/m9E51luN2NV99yR3OQKkAmgykss+SkMZerxMLEZFZ4oBYJGAA600VEryAaD6CPaJwJKwnr9ldR2WMedV1Dsi6WwB58emZlsAV/zqmH9LzfvqBfruUmNvZ4QN7VearjenP4aHwmWsABt4x/+tiImcx/z27Jqw==}{Try
it in the Playground}
\switchcolumn
\href{https://play.vuejs.org/\#eNpdUk1TgzAQ/Ss7uQAjgr12oNXxH+ix9IAYaDQkMV/qMPx3N6G0Uy9Msu/tvn2PTORJqcI7SrakMp1myoKh1qldI9iopLYwQadpa+krG0TLYYZeyxGSojSSs/d7E8vFh0ka0YhOCmPh0EknbB4mPYfTEeqbIelD1oiqXPRQCS+WjoojAW8A1Wmzm1A39KYZzHNVYiUib85aKeCx46z7rBuySqQe6h14uINN1pDIBWACVUcqbGwtl17EqvIiR3LyzwcmcXFuTi3n8vuF9jlYzYaBajxfMsDcomv6E/m9E51luN2NV99yR3OQKkAmgykss+SkMZerxMLEZFZ4oBYJGAA600VEryAaD6CPaJwJKwnr9ldR2WMedV1Dsi6WwB58emZlsAV/zqmH9LzfvqBfruUmNvZ4QN7VearjenP4aHwmWsABt4x/+tiImcx/z27Jqw==}{在演练场中尝试一下}
\switchcolumn[0]*%%%%%%%
\paragraph{Angular Signals}
\switchcolumn
\paragraph{Angular 信号}
\switchcolumn[0]*%%%%%%%
Angular is undergoing some fundamental changes by foregoing
dirty-checking and introducing its own implementation of a reactivity
primitive. The Angular Signal API looks like this:
\switchcolumn
Angular
正在经历一些底层的变化,它放弃了脏检查,并引入了自己的响应性基础类型实现。Angular
的信号 API 看起来像这样:
\switchcolumn[0]*%%%%%%%
\begin{codeJs}
const count = signal(0)
count() // 访问值
count.set(1) //设置值
count.update((v) => v + 1) // 通过前值更新
// 对具有相同身份的深层对象进行更改
const state = signal({ count: 0 })
state.mutate((o) => {
  o.count++
})
\end{codeJs}
\switchcolumn
\begin{codeJs}
const count = signal(0)
count() // 访问值
count.set(1) //设置值
count.update((v) => v + 1) // 通过前值更新
// 对具有相同身份的深层对象进行更改
const state = signal({ count: 0 })
state.mutate((o) => {
  o.count++
})
\end{codeJs}
\switchcolumn[0]*%%%%%%%
Again, we can easily replicate the API in Vue:
\switchcolumn
同样,我们可以轻易地在 Vue 中复制这个 API:
\switchcolumn[0]*%%%%%%%
\begin{codeJs}
import { shallowRef, triggerRef } from 'vue'
export function signal(initialValue) {
  const r = shallowRef(initialValue)
  const s = () => r.value
  s.set = (value) => {
    r.value = value
  }
  s.update = (updater) => {
    r.value = updater(r.value)
  }
  s.mutate = (mutator) => {
    mutator(r.value)
    triggerRef(r)
  }
  return s
}
\end{codeJs}
\switchcolumn
\begin{codeJs}
import { shallowRef, triggerRef } from 'vue'
export function signal(initialValue) {
  const r = shallowRef(initialValue)
  const s = () => r.value
  s.set = (value) => {
    r.value = value
  }
  s.update = (updater) => {
    r.value = updater(r.value)
  }
  s.mutate = (mutator) => {
    mutator(r.value)
    triggerRef(r)
  }
  return s
}
\end{codeJs}
\switchcolumn[0]*%%%%%%%
\href{https://play.vuejs.org/\#eNp9UslOwzAQ/ZVRLiRQEsqxpBUIvoADp0goTd3U4DiWl4AU5d8ZL3E3iZtn5r1Z3vOYvAiRD4Ykq6RUjaRCgyLaiE3FaSd6qWEERVteswU0fSeMJjuYYC/7Dm7youatYbW895D8S91UvOJNz5VGuOEa1oGePmRzYdebLSNYmRumaQbrjSfg8xYeEVsWfh/cBANNOsFqTTACKA/LzavrTtUKxjEyp6kssDZj3vygAPJjL1Bbo3XP4blhtPleV4nrlBuxw1npYLca4A6WWZU4PADljSQd4drRC8//rxfKaW+f+ZJg4oJbFvG8ZJFcaYreHL041Iz1P+9kvwAtadsS6d7Rm1rB55VRaLAzhvy6NnvDG01x1WAN5VTTmn3UzJAMRrudd0pa++LEc9wRpRDlHZT5YGu2pOzhWHAJWxvnakxOHufFxqx/4MxzcEinIYP+eV5ntOe5Rx94IYjopxOZUhnIEr+4xPMrjuG1LPFftkTj5DNJGhwYBZ4BJz3DV56FmJLpD1DrKXU=}{Try
it in the Playground}
\switchcolumn
\href{https://play.vuejs.org/\#eNp9UslOwzAQ/ZVRLiRQEsqxpBUIvoADp0goTd3U4DiWl4AU5d8ZL3E3iZtn5r1Z3vOYvAiRD4Ykq6RUjaRCgyLaiE3FaSd6qWEERVteswU0fSeMJjuYYC/7Dm7youatYbW895D8S91UvOJNz5VGuOEa1oGePmRzYdebLSNYmRumaQbrjSfg8xYeEVsWfh/cBANNOsFqTTACKA/LzavrTtUKxjEyp6kssDZj3vygAPJjL1Bbo3XP4blhtPleV4nrlBuxw1npYLca4A6WWZU4PADljSQd4drRC8//rxfKaW+f+ZJg4oJbFvG8ZJFcaYreHL041Iz1P+9kvwAtadsS6d7Rm1rB55VRaLAzhvy6NnvDG01x1WAN5VTTmn3UzJAMRrudd0pa++LEc9wRpRDlHZT5YGu2pOzhWHAJWxvnakxOHufFxqx/4MxzcEinIYP+eV5ntOe5Rx94IYjopxOZUhnIEr+4xPMrjuG1LPFftkTj5DNJGhwYBZ4BJz3DV56FmJLpD1DrKXU=}{在演练场中尝试一下}
\switchcolumn[0]*%%%%%%%
Compared to Vue refs, Solid and Angular's getter-based API style provide
some interesting trade-offs when used in Vue components:
\switchcolumn
与 Vue 的 ref 相比,Solid 和 Angular 基于 getter 的 API 风格在 Vue
组件中使用时提供了一些有趣的权衡:
\switchcolumn[0]*%%%%%%%
\begin{itemize}
\item
  \texttt{()} is slightly less verbose than \texttt{.value}, but
  updating the value is more verbose.
\item
  There is no ref-unwrapping: accessing values always require
  \texttt{()}. This makes value access consistent everywhere. This also
  means you can pass raw signals down as component props.
\end{itemize}
\switchcolumn
\begin{itemize}
\item
  \texttt{()} 比 \texttt{.value} 略微省事,但更新值却更冗长;
\item
  没有 ref 解包:总是需要通过 \texttt{()}
  来访问值。这使得值的访问在任何地方都是一致的。这也意味着你可以将原始信号作为组件的参数传递下去。
\end{itemize}
\switchcolumn[0]*%%%%%%%
Whether these API styles suit you is to some extent subjective. Our goal
here is to demonstrate the underlying similarity and trade-offs between
these different API designs. We also want to show that Vue is flexible:
you are not really locked into the existing APIs. Should it be
necessary, you can create your own reactivity primitive API to suit more
specific needs.
\switchcolumn
这些 API
风格是否适合你,在某种程度上是主观的。我们在这里的目标是展示这些不同的
API 设计之间的基本相似性和取舍。我们还想说明 Vue
是灵活的:你并没有真正被限定在现有的 API
中。如有必要,你可以创建你自己的响应性基础 API,以满足更多的具体需求。
\end{paracol}

 