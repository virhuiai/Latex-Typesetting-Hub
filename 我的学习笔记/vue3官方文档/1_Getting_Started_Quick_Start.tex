\columnratio{0.55}
\begin{paracol}{2}
\switchcolumn[0]*%%%%%%%
\section{Quick Start}
\switchcolumn
\section{快速上手}
\switchcolumn[0]*%%%%%%%
\subsection{Try Vue Online}
\switchcolumn
\subsection{线上尝试 Vue}
\switchcolumn[0]*%%%%%%%
\begin{itemize}
\item
    To quickly get a taste of Vue, you can try it directly in our
    \href{https://play.vuejs.org/\#eNo9jcEKwjAMhl/lt5fpQYfXUQfefAMvvRQbddC1pUuHUPrudg4HIcmXjyRZXEM4zYlEJ+T0iEPgXjn6BB8Zhp46WUZWDjCa9f6w9kAkTtH9CRinV4fmRtZ63H20Ztesqiylphqy3R5UYBqD1UyVAPk+9zkvV1CKbCv9poMLiTEfR2/IXpSoXomqZLtti/IFwVtA9A==}{Playground}.
\item
    If you prefer a plain HTML setup without any build steps, you can use
    this \href{https://jsfiddle.net/yyx990803/2ke1ab0z/}{JSFiddle} as your
    starting point.
\item
    If you are already familiar with Node.js and the concept of build
    tools, you can also try a complete build setup right within your
    browser on \href{https://vite.new/vue}{StackBlitz}.
\end{itemize}
\switchcolumn
\begin{itemize}
\item
    想要快速体验
    Vue,你可以直接试试我们的\href{https://play.vuejs.org/\#eNo9jcEKwjAMhl/lt5fpQYfXUQfefAMvvRQbddC1pUuHUPrudg4HIcmXjyRZXEM4zYlEJ+T0iEPgXjn6BB8Zhp46WUZWDjCa9f6w9kAkTtH9CRinV4fmRtZ63H20Ztesqiylphqy3R5UYBqD1UyVAPk+9zkvV1CKbCv9poMLiTEfR2/IXpSoXomqZLtti/IFwVtA9A==}{演练场}。
\item
    如果你更喜欢不用任何构建的原始 HTML,可以使用
    \href{https://jsfiddle.net/yyx990803/2ke1ab0z/}{JSFiddle} 入门。
\item
    如果你已经比较熟悉 Node.js 和构建工具等概念,还可以直接在浏览器中打开
    \href{https://vite.new/vue}{StackBlitz} 来尝试完整的构建设置。
\end{itemize}
\switchcolumn[0]*%%%%%%%
\subsection{Creating a Vue Application}
\switchcolumn
\subsection{创建一个 Vue 应用}
\switchcolumn[0]*%%%%%%%
\begin{vueQuote}{Prerequisites}
\begin{itemize}
\item
    Familiarity with the command line
\item
    Install \href{https://nodejs.org/}{Node.js} version 16.0 or higher
\end{itemize}        
\end{vueQuote}    
\switchcolumn
\begin{vueQuote}{前提条件}
\begin{itemize}
\item
    熟悉命令行
\item
    已安装 16.0 或更高版本的 \href{https://nodejs.org/}{Node.js}
\end{itemize}
\end{vueQuote}    
\switchcolumn[0]*%%%%%%%
In this section we will introduce how to scaffold a Vue
\href{https://vuejs.org/guide/extras/ways-of-using-vue.html\#single-page-application-spa}{Single
Page Application} on your local machine. The created project will be
using a build setup based on \href{https://vitejs.dev/}{Vite} and allow
us to use Vue
\href{https://vuejs.org/guide/scaling-up/sfc.html}{Single-File
Components} (SFCs).
\switchcolumn
在本节中,我们将介绍如何在本地搭建 Vue
\href{https://cn.vuejs.org/guide/extras/ways-of-using-vue.html\#single-page-application-spa}{单页应用}。创建的项目将使用基于
\href{https://vitejs.dev/}{Vite} 的构建设置,并允许我们使用 Vue
的\href{https://cn.vuejs.org/guide/scaling-up/sfc.html}{单文件组件}
(SFC)。
\switchcolumn[0]*%%%%%%%
Make sure you have an up-to-date version of
\href{https://nodejs.org/}{Node.js} installed and your current working
directory is the one where you intend to create a project. Run the
following command in your command line (without the
\texttt{\textgreater{}} sign): 
\switchcolumn
确保你安装了最新版本的
\href{https://nodejs.org/}{Node.js},并且你的当前工作目录正是打算创建项目的目录。在命令行中运行以下命令
(不要带上 \texttt{\textgreater{}} 符号):
\switchcolumn[0]*%%%%%%%
\begin{codeShell}
npm create vue@latest
\end{codeShell}
\switchcolumn
\begin{codeShell}
npm create vue@latest
\end{codeShell}
\switchcolumn[0]*%%%%%%%
This command will install and execute
\href{https://github.com/vuejs/create-vue}{create-vue}, the official Vue
project scaffolding tool. You will be presented with prompts for several
optional features such as TypeScript and testing support:
\switchcolumn
这一指令将会安装并执行
\href{https://github.com/vuejs/create-vue}{create-vue},它是 Vue
官方的项目脚手架工具。你将会看到一些诸如 TypeScript
和测试支持之类的可选功能提示:
\switchcolumn[0]*%%%%%%%
\begin{codeConsole*}{escapeinside=||}
|\checkmark| Project name: … <your-project-name>
|\checkmark| Add TypeScript? … No / Yes
|\checkmark| Add JSX Support? … No / Yes
|\checkmark| Add Vue Router for Single Page Application development? … No / Yes
|\checkmark| Add Pinia for state management? … No / Yes
|\checkmark| Add Vitest for Unit testing? … No / Yes
|\checkmark| Add an End-to-End Testing Solution? … No / Cypress / Playwright
|\checkmark| Add ESLint for code quality? … No / Yes
|\checkmark| Add Prettier for code formatting? … No / Yes

Scaffolding project in ./<your-project-name>...
Done.
\end{codeConsole*}
\switchcolumn
\begin{codeConsole*}{escapeinside=||}
|\checkmark| Project name: … <your-project-name>
|\checkmark| Add TypeScript? … No / Yes
|\checkmark| Add JSX Support? … No / Yes
|\checkmark| Add Vue Router for Single Page Application development? … No / Yes
|\checkmark| Add Pinia for state management? … No / Yes
|\checkmark| Add Vitest for Unit testing? … No / Yes
|\checkmark| Add an End-to-End Testing Solution? … No / Cypress / Playwright
|\checkmark| Add ESLint for code quality? … No / Yes
|\checkmark| Add Prettier for code formatting? … No / Yes

Scaffolding project in ./<your-project-name>...
Done.
\end{codeConsole*} 
\switchcolumn[0]*%%%%%%%
If you are unsure about an option, simply choose \texttt{No} by hitting
enter for now. Once the project is created, follow the instructions to
install dependencies and start the dev server:
\switchcolumn
如果不确定是否要开启某个功能,你可以直接按下回车键选择
\texttt{No}。在项目被创建后,通过以下步骤安装依赖并启动开发服务器:
\switchcolumn[0]*%%%%%%%
~\begin{codeShellMul}
cd <your-project-name>
npm install
npm run dev
\end{codeShellMul}
\switchcolumn
~\begin{codeShellMul}
cd <your-project-name>
npm install
npm run dev
\end{codeShellMul}
\switchcolumn[0]*%%%%%%%
You should now have your first Vue project running! Note that the
example components in the generated project are written using the
\href{https://vuejs.org/guide/introduction.html\#composition-api}{Composition
API} and \texttt{\textless{}script\ setup\textgreater{}}, rather than
the
\href{https://vuejs.org/guide/introduction.html\#options-api}{Options
API}. Here are some additional tips:
\switchcolumn
你现在应该已经运行起来了你的第一个 Vue
项目!请注意,生成的项目中的示例组件使用的是\href{https://cn.vuejs.org/guide/introduction.html\#composition-api}{组合式
API} 和
\texttt{\textless{}script\ setup\textgreater{}},而非\href{https://cn.vuejs.org/guide/introduction.html\#options-api}{选项式
API}。下面是一些补充提示:
\switchcolumn[0]*%%%%%%%
\begin{itemize}
    \item
      The recommended IDE setup is
      \href{https://code.visualstudio.com/}{Visual Studio Code} +
      \href{https://marketplace.visualstudio.com/items?itemName=Vue.volar}{Volar
      extension}. If you use other editors, check out the
      \href{https://vuejs.org/guide/scaling-up/tooling.html\#ide-support}{IDE
      support section}.
    \item
      More tooling details, including integration with backend frameworks,
      are discussed in the
      \href{https://vuejs.org/guide/scaling-up/tooling.html}{Tooling Guide}.
    \item
      To learn more about the underlying build tool Vite, check out the
      \href{https://vitejs.dev/}{Vite docs}.
    \item
      If you choose to use TypeScript, check out the
      \href{https://vuejs.org/guide/typescript/overview.html}{TypeScript
      Usage Guide}.
    \end{itemize}
\switchcolumn
\begin{itemize}
    \item
      推荐的 IDE 配置是 \href{https://code.visualstudio.com/}{Visual Studio
      Code} +
      \href{https://marketplace.visualstudio.com/items?itemName=Vue.volar}{Volar
      扩展}。如果使用其他编辑器,参考
      \href{https://cn.vuejs.org/guide/scaling-up/tooling.html\#ide-support}{IDE
      支持章节}。
    \item
      更多工具细节,包括与后端框架的整合,我们会在\href{https://cn.vuejs.org/guide/scaling-up/tooling.html}{工具链指南}进行讨论。
    \item
      要了解构建工具 Vite 更多背后的细节,请查看
      \href{https://cn.vitejs.dev/}{Vite 文档}。
    \item
      如果你选择使用 TypeScript,请阅读
      \href{https://cn.vuejs.org/guide/typescript/overview.html}{TypeScript
      使用指南}。
    \end{itemize}
\switchcolumn[0]*%%%%%%%
When you are ready to ship your app to production, run the following:
\switchcolumn
当你准备将应用发布到生产环境时,请运行:
\switchcolumn[0]*%%%%%%%
~\begin{codeShellMul}
npm run build
\end{codeShellMul}
\switchcolumn
~\begin{codeShellMul}
npm run build
\end{codeShellMul}
\switchcolumn[0]*%%%%%%%
This will create a production-ready build of your app in the project's
\texttt{./dist} directory. Check out the
\href{https://vuejs.org/guide/best-practices/production-deployment.html}{Production
Deployment Guide} to learn more about shipping your app to production.
\switchcolumn
此命令会在 \texttt{./dist}
文件夹中为你的应用创建一个生产环境的构建版本。关于将应用上线生产环境的更多内容,请阅读\href{https://cn.vuejs.org/guide/best-practices/production-deployment.html}{生产环境部署指南}。
\switchcolumn[0]*%%%%%%%
\href{https://vuejs.org/guide/quick-start.html\#next-steps}{Next Steps
\textgreater{}}
\switchcolumn
\href{https://cn.vuejs.org/guide/quick-start.html\#next-steps}{下一步\textgreater{}}
\switchcolumn[0]*%%%%%%%
\subsection{Using Vue from CDN}
\switchcolumn
\subsection{通过 CDN 使用 Vue}
\switchcolumn[0]*%%%%%%%
You can use Vue directly from a CDN via a script tag:
\begin{codeHtml}
<script src="https://unpkg.com/vue@3/dist/vue.global.js"></script>
\end{codeHtml}  
\switchcolumn
你可以借助 script 标签直接通过 CDN 来使用 Vue:
\begin{codeHtml}
<script src="https://unpkg.com/vue@3/dist/vue.global.js"></script>
\end{codeHtml}  
\switchcolumn[0]*%%%%%%%
Here we are using \href{https://unpkg.com/}{unpkg}, but you can also use
any CDN that serves npm packages, for example
\href{https://www.jsdelivr.com/package/npm/vue}{jsdelivr} or
\href{https://cdnjs.com/libraries/vue}{cdnjs}. Of course, you can also
download this file and serve it yourself.
\switchcolumn
这里我们使用了 \href{https://unpkg.com/}{unpkg},但你也可以使用任何提供
npm 包服务的 CDN,例如
\href{https://www.jsdelivr.com/package/npm/vue}{jsdelivr} 或
\href{https://cdnjs.com/libraries/vue}{cdnjs}。当然,你也可以下载此文件并自行提供服务。
\switchcolumn[0]*%%%%%%%
When using Vue from a CDN, there is no "build step" involved. This makes
the setup a lot simpler, and is suitable for enhancing static HTML or
integrating with a backend framework. However, you won't be able to use
the Single-File Component (SFC) syntax.
\switchcolumn
通过 CDN 使用 Vue
时,不涉及``构建步骤''。这使得设置更加简单,并且可以用于增强静态的 HTML
或与后端框架集成。但是,你将无法使用单文件组件 (SFC) 语法。
\switchcolumn[0]*%%%%%%%
\subsubsection{Using the Global Build}
\switchcolumn
\subsubsection{使用全局构建版本}
\switchcolumn[0]*%%%%%%%
The above link loads the \emph{global build} of Vue, where all top-level
APIs are exposed as properties on the global \texttt{Vue} object. Here
is a full example using the global build:
\switchcolumn
上面的链接使用了\emph{全局构建版本}的 Vue,该版本的所有顶层 API
都以属性的形式暴露在了全局的 \texttt{Vue}
对象上。这里有一个使用全局构建版本的例子:
\switchcolumn[0]*%%%%%%%
\begin{codeHtml}
<script src="https://unpkg.com/vue@3/dist/vue.global.js"></script>

<div id="app">{{ message }}</div>

<script>
    const { createApp, ref } = Vue

    createApp({
    setup() {
        const message = ref('Hello vue!')
        return {
        message
        }
    }
    }).mount('#app')
</script>
\end{codeHtml}  
\switchcolumn
\begin{codeHtml}
<script src="https://unpkg.com/vue@3/dist/vue.global.js"></script>

<div id="app">{{ message }}</div>

<script>
    const { createApp, ref } = Vue

    createApp({
    setup() {
        const message = ref('Hello vue!')
        return {
        message
        }
    }
    }).mount('#app')
</script>
\end{codeHtml}
\switchcolumn[0]*%%%%%%%
\href{https://codepen.io/vuejs-examples/pen/eYQpQEG}{Codepen demo}
\switchcolumn
\href{https://codepen.io/vuejs-examples/pen/eYQpQEG}{Codepen 示例}
\switchcolumn[0]*%%%%%%%
\begin{vueQuote}{TIP}
Many of the examples for Composition API throughout the guide will be
using the \texttt{\textless{}script\ setup\textgreater{}} syntax, which
requires build tools. If you intend to use Composition API without a
build step, consult the usage of the
\href{https://vuejs.org/api/composition-api-setup.html}{\texttt{setup()}
option}.
\end{vueQuote}
\switchcolumn
\begin{vueQuote}{TIP}
本指南中许多关于组合式 API 的例子将使用
\texttt{\textless{}script\ setup\textgreater{}}
语法,这需要构建工具。如果你打算在没有构建步骤的情况下使用组合式
API,请参考
\href{https://cn.vuejs.org/api/composition-api-setup.html}{\texttt{setup()}
选项}的用法。
\end{vueQuote}
\switchcolumn[0]*%%%%%%%
\subsubsection{Using the ES Module Build}
\switchcolumn
\subsubsection{使用 ES 模块构建版本}
\switchcolumn[0]*%%%%%%%
Throughout the rest of the documentation, we will be primarily using
\href{https://developer.mozilla.org/en-US/docs/Web/JavaScript/Guide/Modules}{ES
modules} syntax. Most modern browsers now support ES modules natively,
so we can use Vue from a CDN via native ES modules like this:
\switchcolumn
在本文档的其余部分我们使用的主要是
\href{https://developer.mozilla.org/zh-CN/docs/Web/JavaScript/Guide/Modules}{ES
模块}语法。现代浏览器大多都已原生支持 ES 模块。因此我们可以像这样通过
CDN 以及原生 ES 模块使用 Vue:
\switchcolumn[0]*%%%%%%%
\begin{codeHtml}
<div id="app">{{ message }}</div>

<script type="module">
    import { createApp, ref } from 'https://unpkg.com/vue@3/dist/vue.esm-browser.js'

    createApp({
    setup() {
        const message = ref('Hello Vue!')
        return {
        message
        }
    }
    }).mount('#app')
</script>
\end{codeHtml}  
\switchcolumn
\begin{codeHtml}
<div id="app">{{ message }}</div>

<script type="module">
    import { createApp, ref } from 'https://unpkg.com/vue@3/dist/vue.esm-browser.js'

    createApp({
    setup() {
        const message = ref('Hello Vue!')
        return {
        message
        }
    }
    }).mount('#app')
</script>
\end{codeHtml}  
\switchcolumn[0]*%%%%%%%
Notice that we are using
\texttt{\textless{}script\ type="module"\textgreater{}}, and the
imported CDN URL is pointing to the \textbf{ES modules build} of Vue
instead.
\switchcolumn
注意我们使用了
\texttt{\textless{}script\ type="module"\textgreater{}},且导入的 CDN
URL 指向的是 Vue 的 \textbf{ES 模块构建版本}。
\switchcolumn[0]*%%%%%%%
\href{https://codepen.io/vuejs-examples/pen/MWzazEv}{Codepen demo}
\switchcolumn
\href{https://codepen.io/vuejs-examples/pen/MWzazEv}{Codepen 示例}
\switchcolumn[0]*%%%%%%%
\subsubsection{Enabling Import maps}
\switchcolumn
\subsubsection{启用 Import maps}
\switchcolumn[0]*%%%%%%%
In the above example, we are importing from the full CDN URL, but in the
rest of the documentation you will see code like this:
\begin{codeJs}
import { createApp } from 'vue'
\end{codeJs}  
\switchcolumn
在上面的示例中,我们使用了完整的 CDN URL
来导入,但在文档的其余部分中,你将看到如下代码:
\begin{codeJs}
import { createApp } from 'vue'
\end{codeJs}  
\switchcolumn[0]*%%%%%%%
We can teach the browser where to locate the \texttt{vue} import by
using \href{https://caniuse.com/import-maps}{Import Maps}:
\switchcolumn
我们可以使用\href{https://caniuse.com/import-maps}{导入映射表 (Import
Maps)} 来告诉浏览器如何定位到导入的 \texttt{vue}:
\switchcolumn[0]*%%%%%%%
\begin{codeHtml}
<script type="importmap">
{
    "imports": {
    "vue": "https://unpkg.com/vue@3/dist/vue.esm-browser.js"
    }
}
</script>

<div id="app">{{ message }}</div>

<script type="module">
import { createApp, ref } from 'vue'

createApp({
    setup() {
    const message = ref('Hello Vue!')
    return {
        message
    }
    }
}).mount('#app')
</script>
\end{codeHtml}  
\switchcolumn
\begin{codeHtml}
<script type="importmap">
{
    "imports": {
    "vue": "https://unpkg.com/vue@3/dist/vue.esm-browser.js"
    }
}
</script>

<div id="app">{{ message }}</div>

<script type="module">
import { createApp, ref } from 'vue'

createApp({
    setup() {
    const message = ref('Hello Vue!')
    return {
        message
    }
    }
}).mount('#app')
</script>
\end{codeHtml}  
\switchcolumn[0]*%%%%%%%
\href{https://codepen.io/vuejs-examples/pen/YzRyRYM}{Codepen demo}
\switchcolumn
\href{https://codepen.io/vuejs-examples/pen/YzRyRYM}{Codepen demo}

\switchcolumn[0]*%%%%%%%
You can also add entries for other dependencies to the import map - but
make sure they point to the ES modules version of the library you intend
to use.
\switchcolumn 
你也可以在映射表中添加其他的依赖------但请务必确保你使用的是该库的 ES
模块版本。

\switchcolumn[0]*%%%%%%%
\begin{vueQuote}{Import Maps Browser Support}
Import Maps is a relatively new browser feature. Make sure to use a
browser within its \href{https://caniuse.com/import-maps}{support
range}. In particular, it is only supported in Safari 16.4+.
\end{vueQuote}
\begin{vueQuoteWarn}{Notes on Production Use}
The examples so far are using the development build of Vue - if you
intend to use Vue from a CDN in production, make sure to check out the
\href{https://vuejs.org/guide/best-practices/production-deployment.html\#without-build-tools}{Production
Deployment Guide}.
\end{vueQuoteWarn}
\switchcolumn
\begin{vueQuote}{导入映射表的浏览器支持情况}
导入映射表是一个相对较新的浏览器功能。请确保使用其\href{https://caniuse.com/import-maps}{支持范围}内的浏览器。请注意,只有
Safari 16.4 以上版本支持。
\end{vueQuote}
\begin{vueQuoteWarn}{生产环境中的注意事项}
到目前为止示例中使用的都是 Vue
的开发构建版本------如果你打算在生产中通过 CDN 使用
Vue,请务必查看\href{https://cn.vuejs.org/guide/best-practices/production-deployment.html\#without-build-tools}{生产环境部署指南}。
\end{vueQuoteWarn}
\switchcolumn[0]*%%%%%%%
\subsubsection{Splitting Up the Modules}
\switchcolumn
\subsubsection{拆分模块}
\switchcolumn[0]*%%%%%%%
As we dive deeper into the guide, we may need to split our code into
separate JavaScript files so that they are easier to manage. For
example:
\switchcolumn
随着对这份指南的逐步深入,我们可能需要将代码分割成单独的 JavaScript
文件,以便更容易管理。例如:
\switchcolumn[0]*%%%%%%%
\begin{codeHtml}
<!-- index.html -->
<div id="app"></div>

<script type="module">
    import { createApp } from 'vue'
    import MyComponent from './my-component.js'

    createApp(MyComponent).mount('#app')
</script>
\end{codeHtml}  

\begin{codeJs}
// my-component.js
import { ref } from 'vue'
export default {
    setup() {
    const count = ref(0)
    return { count }
    },
    template: `<div>count is {{ count }}</div>`
}
\end{codeJs}  
\switchcolumn
\begin{codeHtml}
<!-- index.html -->
<div id="app"></div>

<script type="module">
    import { createApp } from 'vue'
    import MyComponent from './my-component.js'

    createApp(MyComponent).mount('#app')
</script>
\end{codeHtml}  

\begin{codeJs}
// my-component.js
import { ref } from 'vue'
export default {
    setup() {
    const count = ref(0)
    return { count }
    },
    template: `<div>count is {{ count }}</div>`
}
\end{codeJs}  

\switchcolumn[0]*%%%%%%%
If you directly open the above \texttt{index.html} in your browser, you
will find that it throws an error because ES modules cannot work over
the \texttt{file://} protocol, which is the protocol the browser uses
when you open a local file.
\switchcolumn
如果直接在浏览器中打开了上面的
\texttt{index.html},你会发现它抛出了一个错误,因为 ES 模块不能通过
\texttt{file://}
协议工作,也即是当你打开一个本地文件时浏览器使用的协议。
\switchcolumn[0]*%%%%%%%
Due to security reasons, ES modules can only work over the
\texttt{http://} protocol, which is what the browsers use when opening
pages on the web. In order for ES modules to work on our local machine,
we need to serve the \texttt{index.html} over the \texttt{http://}
protocol, with a local HTTP server.
\switchcolumn
由于安全原因,ES 模块只能通过 \texttt{http://}
协议工作,也即是浏览器在打开网页时使用的协议。为了使 ES
模块在我们的本地机器上工作,我们需要使用本地的 HTTP 服务器,通过
\texttt{http://} 协议来提供 \texttt{index.html}。
\switchcolumn[0]*%%%%%%%
To start a local HTTP server, first make sure you have
\href{https://nodejs.org/en/}{Node.js} installed, then run
\texttt{npx\ serve} from the command line in the same directory where
your HTML file is. You can also use any other HTTP server that can serve
static files with the correct MIME types.
\switchcolumn
要启动一个本地的 HTTP 服务器,请先安装
\href{https://nodejs.org/zh/}{Node.js},然后通过命令行在 HTML
文件所在文件夹下运行
\texttt{npx\ serve}。你也可以使用其他任何可以基于正确的 MIME
类型服务静态文件的 HTTP 服务器。
\switchcolumn[0]*%%%%%%%
You may have noticed that the imported component's template is inlined
as a JavaScript string. If you are using VSCode, you can install the
\href{https://marketplace.visualstudio.com/items?itemName=Tobermory.es6-string-html}{es6-string-html}
extension and prefix the strings with a \texttt{/*html*/} comment to get
syntax highlighting for them.
\switchcolumn
可能你也注意到了,这里导入的组件模板是内联的 JavaScript
字符串。如果你正在使用 VSCode,你可以安装
\href{https://marketplace.visualstudio.com/items?itemName=Tobermory.es6-string-html}{es6-string-html}
扩展,然后在字符串前加上一个前缀注释 \texttt{/*html*/} 以高亮语法。
\switchcolumn[0]*%%%%%%%
\subsection{Next Steps}
\switchcolumn
\subsection{下一步}
\switchcolumn[0]*%%%%%%%
If you skipped the
\href{https://vuejs.org/guide/introduction.html}{Introduction}, we
strongly recommend reading it before moving on to the rest of the
documentation.
\switchcolumn
如果你尚未阅读\href{https://cn.vuejs.org/guide/introduction.html}{简介},我们强烈推荐你在移步到后续文档之前返回去阅读一下。
\switchcolumn[0]*%%%%%%%
\href{https://vuejs.org/guide/essentials/application.html}{Continue with
the GuideThe guide walks you through every aspect of the framework in
full detail.}\href{https://vuejs.org/tutorial/}{Try the TutorialFor
those who prefer learning things
hands-on.}\href{https://vuejs.org/examples/}{Check out the
ExamplesExplore examples of core features and common UI tasks.}
\switchcolumn
\href{https://cn.vuejs.org/guide/essentials/application.html}{继续阅读该指南该指南会带你深入了解框架所有方面的细节。}\href{https://cn.vuejs.org/tutorial/}{尝试互动教程适合喜欢边动手边学的读者。}\href{https://cn.vuejs.org/examples/}{查看示例浏览核心功能和常见用户界面的示例。}
% \switchcolumn[0]*%%%%%%%
% \href{https://github.com/vuejs/docs/edit/main/src/guide/quick-start.md}{Edit
% this page on GitHub}
% \switchcolumn
% \href{https://github.com/vuejs-translations/docs-zh-cn/edit/main/src/guide/quick-start.md}{在
% GitHub 上编辑此页}
\end{paracol}
