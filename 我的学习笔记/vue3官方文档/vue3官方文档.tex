% vue3官方文档.tex
\PassOptionsToPackage{no-math}{fontspec}%禁用了使用fontspec宏包中的数学字体功能。
\PassOptionsToPackage{AutoFakeBold=true,AutoFakeSlant=true}{xeCJK}%让xeCJK宏包自动产生伪粗体和伪斜体效果。

\documentclass[oneside]{book}
\usepackage[heading=true
,scheme=chinese%中文方案
,fontset=none%不使用默认的字体设置
,space=auto%自动调整中英文间距
]{ctex}
\setCJKmainfont{FangZhengShuSong-GBK-1.ttf}[Path=/Users/virhuiai/hlProjects/Latex-Typesetting-Hub/font/方正/]%设置文本的中文有衬线字体
\setCJKsansfont{FangZhengHeiTi-GBK-1.ttf}[Path=/Users/virhuiai/hlProjects/Latex-Typesetting-Hub/font/方正/]%设置文本的中文无衬线字体为
\setCJKmonofont{FangZhengFangSong-GBK-1.ttf}[Path=/Users/virhuiai/hlProjects/Latex-Typesetting-Hub/font/方正/] %设置文本的中文等宽字体 

\setCJKfamilyfont{fontFangSong}{FangZhengFangSong-GBK-1.ttf}[Path=/Users/virhuiai/hlProjects/Latex-Typesetting-Hub/font/方正/]
\setCJKfamilyfont{fontKai}{FangZhengKaiTi-GBK-1.ttf}[Path=/Users/virhuiai/hlProjects/Latex-Typesetting-Hub/font/方正/]
\newcommand\fontKai{\CJKfamily{fontKai}}

% 支持音标的字体
\newfontfamily\fontGentiumPlus{GentiumPlus}[Path=/Users/virhuiai/hlProjects/Latex-Typesetting-Hub/font/免费商用英文/支持音标-GentiumPlus-6.200/,
Extension=.ttf,
UprightFont=*-Regular ,
BoldFont=*-Bold ,
ItalicFont=*-Italic,
BoldItalicFont = *-BoldItalic
]
\newfontfamily\fontGentiumBookPlus{GentiumBookPlus}[Path=/Users/virhuiai/hlProjects/Latex-Typesetting-Hub/font/免费商用英文/支持音标-GentiumPlus-6.200/,
Extension=.ttf,
UprightFont=*-Regular ,
BoldFont=*-Bold ,
ItalicFont=*-Italic,
BoldItalicFont = *-BoldItalic
]


\usepackage[a3paper,landscape]{geometry}
\usepackage{paracol}
\usepackage[all]{tcolorbox}
\usepackage{parskip}
\usepackage{calc,pifont}\newcounter{带圈文字}\newcommand\带圈文字[1]{\protect\setcounter{带圈文字}{171+#1}\protect\ding{\value{带圈文字}}}
% \带圈文字{1}
% \带圈文字{2}\end{document}
\usepackage{amssymb}%\checkmark
%如果你想在LaTeX中输入"✔"符号,你可以使用amssymb宏包提供的\checkmark命令。
% 在常见的Unicode字符集中,"✔"的编码为U+2714。这个字符可以在大多数现代字体和字符集中正确显示。
\parindent=0pt

% \usepackage{graphicx}
% \makeatletter
% \def\fps@figure{htbp}
% \makeatother
\begin{document}

\newminted[codeJs]{js}{frame=single,label={js}}
\newminted[codeHtml]{html}{frame=single,label={html}}
\newminted[codeVue]{html}{frame=single,label={vue}}
\tcbset{
    csh shell/.style={
    skin=bicolor,
    colback=black,colupper=green,colframe=yellow!75!black,
    fontupper=\tt,
    before upper=\textcolor{red}{\small\ttfamily\bfseries virhuiai \%>~}
    } 
}
\newtcolorbox{codeShell}{csh shell} 
\newtcblisting{codeShellMul}{colback=black,colupper=white,colframe=yellow!75!black, listing only,listing options={style=tcblatex,language=sh},
every listing line={\textcolor{red}{\small\ttfamily\bfseries virhuiai \$> }}}

\tcolorboxenvironment{quote}{blanker, borderline west={1mm}{0pt}{gray}}
 
% \tcbset{
%     csh console/.style={
%     skin=bicolor,
%     colback=black,colupper=green,colframe=yellow!75!black,
%     fontupper=\tt,
%     % before upper=\textcolor{red}{\small\ttfamily\bfseries virhuiai \%>~}
%     } 
% }
% \newtcolorbox{consoleCode}{csh console}
% 需要指定字体 \setCJKfamilyfont{fontFangSong}{FangZhengFangSong-GBK-1.ttf}[Path=/Users/virhuiai/hlProjects/Latex-Typesetting-Hub/font/方正/]
\newminted[codeConsole]{console}{frame=single,fontfamily=fontFangSong} 
%todo 符号 

\definecolor{vueQuoteBg}{RGB}{249, 249, 249}
\definecolor{vueQuoteFrame}{RGB}{101, 181, 135}
\definecolor{vueQuoteFrameWarn}{RGB}{247, 197, 72}


% \begin{codeVue}
% \end{codeVue}


\newtcolorbox{vueQuote}[2][]{colback=vueQuoteBg, colframe=vueQuoteFrame,fonttitle=\bfseries, enhanced,coltitle=black,
% attach boxed title to top center={yshift=-2mm},
attach title to upper={\par},
title={%\makebox[0pt]{\textcircled{i}\quad}
#2},#1} 
\newtcolorbox{vueQuoteWarn}[2][]{colback=vueQuoteBg, colframe=vueQuoteFrameWarn,fonttitle=\bfseries, enhanced,coltitle=black,
% attach boxed title to top center={yshift=-2mm},
attach title to upper={\par},
title={%\makebox[0pt]{\textcircled{i}\quad}
#2},#1} 
 



% \chapter{Getting Started\hfill 开始}
% 
% [Introduction](https://vuejs.org/guide/introduction)[Quick Start](https://vuejs.org/guide/quick-start)
% [简介](https://cn.vuejs.org/guide/introduction.html)[快速上手](https://cn.vuejs.org/guide/quick-start.html)



\columnratio{0.55}
\begin{paracol}{2}
% \chapter{Getting Started}
% \switchcolumn
% \chapter{开始}
\switchcolumn[0]*%%%%%%%
\section{introduction}
\switchcolumn
\section{介绍}
\switchcolumn[0]*%%%%%%%
\begin{vueQuote}{You are reading the documentation for Vue 3!}
\begin{itemize}
    \item
      Vue 2 support will end on Dec 31, 2023. Learn more about
      \href{https://v2.vuejs.org/lts/}{Vue 2 Extended LTS}.
    \item
      Vue 2 documentation has been moved to
      \href{https://v2.vuejs.org/}{v2.vuejs.org}.
    \item
      Upgrading from Vue 2? Check out the
      \href{https://v3-migration.vuejs.org/}{Migration Guide}.
    \end{itemize}
\end{vueQuote}
\switchcolumn
\begin{vueQuote}{你正在阅读的是 Vue 3 的文档!}
\begin{itemize}
\item
    Vue 2 将于 2023 年 12 月 31 日停止维护。详见
    \href{https://v2.vuejs.org/lts/}{Vue 2 延长 LTS}。
\item
    Vue 2 中文文档已迁移至
    \href{https://v2.cn.vuejs.org/}{v2.cn.vuejs.org}。
\item
    想从 Vue 2
    升级?请参考\href{https://v3-migration.vuejs.org/}{迁移指南}。
\end{itemize}
\end{vueQuote}
\switchcolumn[0]*%%%%%%%
\subsection{What is Vue?}
\switchcolumn
\subsection{什么是 Vue?}
\switchcolumn[0]*%%%%%%%
Vue (pronounced {\fontGentiumPlus /vjuː/}, like \textbf{view}) is a JavaScript framework
for building user interfaces. It builds on top of standard HTML, CSS,
and JavaScript and provides a declarative and component-based
programming model that helps you efficiently develop user interfaces, be
they simple or complex.
\switchcolumn
Vue (发音为 {\fontGentiumPlus /vjuː/},类似 \textbf{view}) 是用于构建用户界面的
JavaScript 框架。它基于标准 HTML、CSS 和 JavaScript
构建,并提供了一套声明式的、组件化的编程模型,助你高效地开发用户界面。无论是简单还是复杂的界面,Vue
都可以胜任。
\switchcolumn[0]*%%%%%%%
Here is a minimal example:
\switchcolumn
下面是一个最基本的示例:
\switchcolumn[0]*%%%%%%%
\begin{codeJs*}{label={js}}
import { createApp, ref } from 'vue'

createApp({
  setup() {
    return {
      count: ref(0)
    }
  }
}).mount('#app')
\end{codeJs*}
\switchcolumn
\begin{codeJs*}{label={js:UMD浏览器引用JS方式}}
const {createApp,ref} = Vue;

createApp({
    setup() {
    return {
        count: ref(0)
    }
    }
}).mount('#app')
\end{codeJs*}
\switchcolumn[0]*%%%%%%%
\begin{codeHtml*}{label={template}}
<div id="app">
    <button @click="count++">
        Count is: {{ count }}
    </button>
</div>
\end{codeHtml*}
\switchcolumn
\begin{codeHtml*}{label={template}}
<div id="app">
    <button @click="count++">
        Count is: {{ count }}
    </button>
</div>
\end{codeHtml*}
\switchcolumn[0]*%%%%%%%
The above example demonstrates the two core features of Vue:
\switchcolumn
上面的示例展示了 Vue 的两个核心功能:
\end{paracol}

\columnratio{0.55}
\begin{itemize}
\begin{paracol}{2}
\item
\textbf{Declarative Rendering}: Vue extends standard HTML with a
template syntax that allows us to declaratively describe HTML output
based on JavaScript state.
\switchcolumn
\item
\textbf{声明式渲染}:Vue 基于标准 HTML
拓展了一套模板语法,使得我们可以声明式地描述最终输出的 HTML 和
JavaScript 状态之间的关系。
\switchcolumn[0]*%%%%%%%
\item
\textbf{Reactivity}: Vue automatically tracks JavaScript state changes
and efficiently updates the DOM when changes happen.
\switchcolumn
\item
\textbf{响应性}:Vue 会自动跟踪 JavaScript
状态并在其发生变化时响应式地更新 DOM。
\end{paracol}
\end{itemize}


\columnratio{0.55}
\begin{paracol}{2}
\switchcolumn[0]*%%%%%%%
You may already have questions - don't worry. We will cover every little
detail in the rest of the documentation. For now, please read along so
you can have a high-level understanding of what Vue offers.
\switchcolumn
你可能已经有了些疑问------先别急,在后续的文档中我们会详细介绍每一个细节。现在,请继续看下去,以确保你对
Vue 作为一个框架到底提供了什么有一个宏观的了解。
\switchcolumn[0]*%%%%%%%
\begin{vueQuote}
{Prerequisites}
The rest of the documentation assumes basic familiarity with HTML, CSS,
and JavaScript. If you are totally new to frontend development, it might
not be the best idea to jump right into a framework as your first step -
grasp the basics and then come back! You can check your knowledge level
with
\href{https://developer.mozilla.org/en-US/docs/Web/JavaScript/A_re-introduction_to_JavaScript}{this
JavaScript overview}. Prior experience with other frameworks helps, but
is not required.
\end{vueQuote}
\switchcolumn

\begin{vueQuote}{预备知识}
文档接下来的内容会假设你对 HTML、CSS 和 JavaScript
已经基本熟悉。如果你对前端开发完全陌生,最好不要直接从一个框架开始进行入门学习------最好是掌握了基础知识再回到这里。你可以通过这篇
\href{https://developer.mozilla.org/zh-CN/docs/Web/JavaScript/A_re-introduction_to_JavaScript}{JavaScript
概述}来检验你的 JavaScript
知识水平。如果之前有其他框架的经验会很有帮助,但也不是必须的。
\end{vueQuote}
\switchcolumn[0]*%%%%%%%
\subsection{The Progressive Framework}
\switchcolumn
\subsection{渐进式框架}
\switchcolumn[0]*%%%%%%%
Vue is a framework and ecosystem that covers most of the common features
needed in frontend development. But the web is extremely diverse - the
things we build on the web may vary drastically in form and scale. With
that in mind, Vue is designed to be flexible and incrementally
adoptable. Depending on your use case, Vue can be used in different
ways:
\switchcolumn
Vue 是一个框架,也是一个生态。其功能覆盖了大部分前端开发常见的需求。但
Web 世界是十分多样化的,不同的开发者在 Web
上构建的东西可能在形式和规模上会有很大的不同。考虑到这一点,Vue
的设计非常注重灵活性和``可以被逐步集成''这个特点。根据你的需求场景,你可以用不同的方式使用
Vue:
\switchcolumn[0]*%%%%%%%
\begin{itemize}
\item
    Enhancing static HTML without a build step
\item
    Embedding as Web Components on any page
\item
    Single-Page Application (SPA)
\item
    Fullstack / Server-Side Rendering (SSR)
\item
    Jamstack / Static Site Generation (SSG)
\item
    Targeting desktop, mobile, WebGL, and even the terminal
\end{itemize}
\switchcolumn
\begin{itemize}
\item
    无需构建步骤,渐进式增强静态的 HTML
\item
    在任何页面中作为 Web Components 嵌入
\item
    单页应用 (SPA)
\item
    全栈 / 服务端渲染 (SSR)
\item
    Jamstack / 静态站点生成 (SSG)
\item
    开发桌面端、移动端、WebGL,甚至是命令行终端中的界面
\end{itemize}
\switchcolumn[0]*%%%%%%%
If you find these concepts intimidating, don't worry! The tutorial and
guide only require basic HTML and JavaScript knowledge, and you should
be able to follow along without being an expert in any of these.
\switchcolumn
如果你是初学者,可能会觉得这些概念有些复杂。别担心!理解教程和指南的内容只需要具备基础的
HTML 和 JavaScript 知识。即使你不是这些方面的专家,也能够跟得上。
\switchcolumn[0]*%%%%%%%
If you are an experienced developer interested in how to best integrate
Vue into your stack, or you are curious about what these terms mean, we
discuss them in more detail in
\href{https://vuejs.org/guide/extras/ways-of-using-vue}{Ways of Using
Vue}.
\switchcolumn
如果你是有经验的开发者,希望了解如何以最合适的方式在项目中引入
Vue,或者是对上述的这些概念感到好奇,我们在\href{https://cn.vuejs.org/guide/extras/ways-of-using-vue.html}{使用
Vue 的多种方式}中讨论了有关它们的更多细节。
\switchcolumn[0]*%%%%%%%
Despite the flexibility, the core knowledge about how Vue works is
shared across all these use cases. Even if you are just a beginner now,
the knowledge gained along the way will stay useful as you grow to
tackle more ambitious goals in the future. If you are a veteran, you can
pick the optimal way to leverage Vue based on the problems you are
trying to solve, while retaining the same productivity. This is why we
call Vue "The Progressive Framework": it's a framework that can grow
with you and adapt to your needs.
\switchcolumn
无论再怎么灵活,Vue
的核心知识在所有这些用例中都是通用的。即使你现在只是一个初学者,随着你的不断成长,到未来有能力实现更复杂的项目时,这一路上获得的知识依然会适用。如果你已经是一个老手,你可以根据实际场景来选择使用
Vue 的最佳方式,在各种场景下都可以保持同样的开发效率。这就是为什么我们将
Vue 称为``渐进式框架'':它是一个可以与你共同成长、适应你不同需求的框架。
\switchcolumn[0]*%%%%%%%
\subsection{Single-File Components}
\switchcolumn
\subsection{单文件组件}
\switchcolumn[0]*%%%%%%%
In most build-tool-enabled Vue projects, we author Vue components using
an HTML-like file format called \textbf{Single-File Component} (also
known as \texttt{*.vue} files, abbreviated as \textbf{SFC}). A Vue SFC,
as the name suggests, encapsulates the component's logic (JavaScript),
template (HTML), and styles (CSS) in a single file. Here's the previous
example, written in SFC format:
\switchcolumn
在大多数启用了构建工具的 Vue 项目中,我们可以使用一种类似 HTML
格式的文件来书写 Vue 组件,它被称为\textbf{单文件组件} (也被称为
\texttt{*.vue} 文件,英文 Single-File Components,缩写为
\textbf{SFC})。顾名思义,Vue 的单文件组件会将一个组件的逻辑
(JavaScript),模板 (HTML) 和样式 (CSS)
封装在同一个文件里。下面我们将用单文件组件的格式重写上面的计数器示例:
\switchcolumn[0]*%%%%%%%
\begin{codeVue}
<script setup>
import { ref } from 'vue'
const count = ref(0)
</script>

<template>
    <button @click="count++">Count is: {{ count }}</button>
</template>

<style scoped>
button {
    font-weight: bold;
}
</style>
\end{codeVue}
\switchcolumn
\begin{codeVue}
<script setup>
import { ref } from 'vue'
const count = ref(0)
</script>

<template>
    <button @click="count++">Count is: {{ count }}</button>
</template>

<style scoped>
button {
    font-weight: bold;
}
</style>
\end{codeVue}
\switchcolumn[0]*%%%%%%%
SFC is a defining feature of Vue and is the recommended way to author
Vue components \textbf{if} your use case warrants a build setup. You can
learn more about the \href{https://vuejs.org/guide/scaling-up/sfc}{how
and why of SFC} in its dedicated section - but for now, just know that
Vue will handle all the build tools setup for you.
\switchcolumn
单文件组件是 Vue
的标志性功能。如果你的用例需要进行构建,我们推荐用它来编写 Vue
组件。你可以在后续相关章节里了解更多关于\href{https://cn.vuejs.org/guide/scaling-up/sfc.html}{单文件组件的用法及用途}。但你暂时只需要知道
Vue 会帮忙处理所有这些构建工具的配置就好。
\switchcolumn[0]*%%%%%%%
\subsection{API Styles}
\switchcolumn
\subsection{API 风格}
\switchcolumn[0]*%%%%%%%
Vue components can be authored in two different API styles:\textbf{Options API} and \textbf{Composition API}.
\switchcolumn
Vue 的组件可以按两种不同的风格书写:\textbf{选项式 API} 和\textbf{组合式
API}。
\switchcolumn[0]*%%%%%%%
\subsubsection{Options API}
\switchcolumn
\subsubsection{选项式 API (Options API)}
\switchcolumn[0]*%%%%%%%
With Options API, we define a component's logic using an object of
options such as \texttt{data}, \texttt{methods}, and \texttt{mounted}.
Properties defined by options are exposed on \texttt{this} inside
functions, which points to the component instance:
\switchcolumn
使用选项式 API,我们可以用包含多个选项的对象来描述组件的逻辑,例如
\texttt{data}、\texttt{methods} 和
\texttt{mounted}。选项所定义的属性都会暴露在函数内部的 \texttt{this}
上,它会指向当前的组件实例。
\switchcolumn[0]*%%%%%%%
\begin{codeVue}
    <script>
    export default {
      // Properties returned from data() become reactive state
      // and will be exposed on `this`.
      data() {
        return {
          count: 0
        }
      },
    
      // Methods are functions that mutate state and trigger updates.
      // They can be bound as event handlers in templates.
      methods: {
        increment() {
          this.count++
        }
      },
    
      // Lifecycle hooks are called at different stages
      // of a component's lifecycle.
      // This function will be called when the component is mounted.
      mounted() {
        console.log(`The initial count is ${this.count}.`)
      }
    }
    </script>
    
    <template>
      <button @click="increment">Count is: {{ count }}</button>
    </template>
\end{codeVue}
\switchcolumn
\begin{codeVue}
    <script>
    export default {
      // data() 返回的属性将会成为响应式的状态
      // 并且暴露在 `this` 上
      data() {
        return {
          count: 0
        }
      },
    
      // methods 是一些用来更改状态与触发更新的函数
      // 它们可以在模板中作为事件处理器绑定
      methods: {
        increment() {
          this.count++
        }
      },
    
      // 生命周期钩子会在组件生命周期的各个不同阶段被调用
      // 例如这个函数就会在组件挂载完成后被调用
      mounted() {
        console.log(`The initial count is ${this.count}.`)
      }
    }
    </script>
    
    <template>
      <button @click="increment">Count is: {{ count }}</button>
    </template>
\end{codeVue}
\switchcolumn[0]*%%%%%%%$
\subsubsection{Composition API}
\switchcolumn
\subsubsection{组合式 API (Composition API)}
\switchcolumn[0]*%%%%%%%
With Composition API, we define a component's logic using imported API
functions. In SFCs, Composition API is typically used with
\href{https://vuejs.org/api/sfc-script-setup}{``}. The \texttt{setup}
attribute is a hint that makes Vue perform compile-time transforms that
allow us to use Composition API with less boilerplate. For example,
imports and top-level variables / functions declared in
\texttt{\textless{}script\ setup\textgreater{}} are directly usable in
the template.
\switchcolumn
通过组合式 API,我们可以使用导入的 API
函数来描述组件逻辑。在单文件组件中,组合式 API 通常会与
\href{https://cn.vuejs.org/api/sfc-script-setup.html}{``} 搭配使用。这个
\texttt{setup} attribute 是一个标识,告诉 Vue
需要在编译时进行一些处理,让我们可以更简洁地使用组合式
API。比如,\texttt{\textless{}script\ setup\textgreater{}}
中的导入和顶层变量/函数都能够在模板中直接使用。
\switchcolumn[0]*%%%%%%%
Here is the same component, with the exact same template, but using
Composition API and \texttt{\textless{}script\ setup\textgreater{}}
instead:
\switchcolumn
下面是使用了组合式 API 与
\texttt{\textless{}script\ setup\textgreater{}}
改造后和上面的模板完全一样的组件:
\switchcolumn[0]*%%%%%%%
\begin{codeVue}
    <script setup>
    import { ref, onMounted } from 'vue'
    
    // reactive state
    const count = ref(0)
    
    // functions that mutate state and trigger updates
    function increment() {
      count.value++
    }
    
    // lifecycle hooks
    onMounted(() => {
      console.log(`The initial count is ${count.value}.`)
    })
    </script>
    
    <template>
      <button @click="increment">Count is: {{ count }}</button>
    </template>
\end{codeVue}    
\switchcolumn
\begin{codeVue}    
    <script setup>
    import { ref, onMounted } from 'vue'
    
    // 响应式状态
    const count = ref(0)
    
    // 用来修改状态、触发更新的函数
    function increment() {
      count.value++
    }
    
    // 生命周期钩子
    onMounted(() => {
      console.log(`The initial count is ${count.value}.`)
    })
    </script>
    
    <template>
      <button @click="increment">Count is: {{ count }}</button>
    </template>
\end{codeVue}    
\switchcolumn[0]*%%%%%%%
\subsubsection{Which to Choose?}
\switchcolumn
\subsubsection{该选哪一个?}
\switchcolumn[0]*%%%%%%%
Both API styles are fully capable of covering common use cases. They are
different interfaces powered by the exact same underlying system. In
fact, the Options API is implemented on top of the Composition API! The
fundamental concepts and knowledge about Vue are shared across the two
styles.
\switchcolumn
两种 API
风格都能够覆盖大部分的应用场景。它们只是同一个底层系统所提供的两套不同的接口。实际上,选项式
API 是在组合式 API 的基础上实现的!关于 Vue
的基础概念和知识在它们之间都是通用的。
\switchcolumn[0]*%%%%%%%
The Options API is centered around the concept of a "component instance"
(\texttt{this} as seen in the example), which typically aligns better
with a class-based mental model for users coming from OOP language
backgrounds. It is also more beginner-friendly by abstracting away the
reactivity details and enforcing code organization via option groups.
\switchcolumn
选项式 API 以``组件实例''的概念为中心 (即上述例子中的
\texttt{this}),对于有面向对象语言背景的用户来说,这通常与基于类的心智模型更为一致。同时,它将响应性相关的细节抽象出来,并强制按照选项来组织代码,从而对初学者而言更为友好。
\switchcolumn[0]*%%%%%%%
The Composition API is centered around declaring reactive state
variables directly in a function scope and composing state from multiple
functions together to handle complexity. It is more free-form and
requires an understanding of how reactivity works in Vue to be used
effectively. In return, its flexibility enables more powerful patterns
for organizing and reusing logic.
\switchcolumn
组合式 API
的核心思想是直接在函数作用域内定义响应式状态变量,并将从多个函数中得到的状态组合起来处理复杂问题。这种形式更加自由,也需要你对
Vue
的响应式系统有更深的理解才能高效使用。相应的,它的灵活性也使得组织和重用逻辑的模式变得更加强大。
\switchcolumn[0]*%%%%%%%
You can learn more about the comparison between the two styles and the
potential benefits of Composition API in the
\href{https://vuejs.org/guide/extras/composition-api-faq}{Composition
API FAQ}.
\switchcolumn
在\href{https://cn.vuejs.org/guide/extras/composition-api-faq.html}{组合式
API FAQ} 章节中,你可以了解更多关于这两种 API 风格的对比以及组合式 API
所带来的潜在收益。
\switchcolumn[0]*%%%%%%%
If you are new to Vue, here's our general recommendation:
\switchcolumn
如果你是使用 Vue 的新手,这里是我们的大致建议:
\switchcolumn[0]*%%%%%%%
\begin{itemize}
\item
For learning purposes, go with the style that looks easier to
understand to you. Again, most of the core concepts are shared between
the two styles. You can always pick up the other style later.
\item
For production use:

\begin{itemize}
\item
Go with Options API if you are not using build tools, or plan to use
Vue primarily in low-complexity scenarios, e.g. progressive
enhancement.
\item
Go with Composition API + Single-File Components if you plan to
build full applications with Vue.
\end{itemize}
\end{itemize}
\switchcolumn
\begin{itemize}
\item
在学习的过程中,推荐采用更易于自己理解的风格。再强调一下,大部分的核心概念在这两种风格之间都是通用的。熟悉了一种风格以后,你也能够很快地理解另一种风格。
\item
在生产项目中:

\begin{itemize}
\item
当你不需要使用构建工具,或者打算主要在低复杂度的场景中使用
Vue,例如渐进增强的应用场景,推荐采用选项式 API。
\item
当你打算用 Vue 构建完整的单页应用,推荐采用组合式 API + 单文件组件。
\end{itemize}
\end{itemize}
\switchcolumn[0]*%%%%%%%
You don't have to commit to only one style during the learning phase.
The rest of the documentation will provide code samples in both styles
where applicable, and you can toggle between them at any time using the
\textbf{API Preference switches} at the top of the left sidebar.
\switchcolumn
在学习阶段,你不必只固守一种风格。在接下来的文档中我们会为你提供一系列两种风格的代码供你参考,你可以随时通过左上角的
\textbf{API 风格偏好}来做切换。
\switchcolumn[0]*%%%%%%%
\subsection{Still Got Questions?}
Check out our \href{https://vuejs.org/about/faq}{FAQ}.
\switchcolumn
\subsection{还有其他问题?}
请查看我们的 \href{https://cn.vuejs.org/about/faq.html}{FAQ}。
\switchcolumn[0]*%%%%%%%
\subsection{Pick Your Learning Path}
\switchcolumn
\subsection{选择你的学习路径}
\switchcolumn[0]*%%%%%%%
Different developers have different learning styles. Feel free to pick a
learning path that suits your preference - although we do recommend
going over all of the content, if possible!
\switchcolumn
不同的开发者有不同的学习方式。尽管在可能的情况下,我们推荐你通读所有内容,但你还是可以自由地选择一种自己喜欢的学习路径!
\switchcolumn[0]*%%%%%%%
\href{https://vuejs.org/tutorial/}{Try the TutorialFor those who prefer
learning things
hands-on.}\href{https://vuejs.org/guide/quick-start}{Read the GuideThe
guide walks you through every aspect of the framework in full
detail.}\href{https://vuejs.org/examples/}{Check out the ExamplesExplore
examples of core features and common UI tasks.}
\switchcolumn
\href{https://cn.vuejs.org/tutorial/}{尝试互动教程适合喜欢边动手边学的读者。}\href{https://cn.vuejs.org/guide/quick-start.html}{继续阅读该指南该指南会带你深入了解框架所有方面的细节。}\href{https://cn.vuejs.org/examples/}{查看示例浏览核心功能和常见用户界面的示例。}
% \switchcolumn[0]*%%%%%%%
% \href{https://github.com/vuejs/docs/edit/main/src/guide/introduction.md}{Edit
% this page on GitHub}
% \switchcolumn
% \href{https://github.com/vuejs-translations/docs-zh-cn/edit/main/src/guide/introduction.md}{在
% GitHub 上编辑此页}
\end{paracol}



 
% \columnratio{0.55}
\begin{paracol}{2}
\switchcolumn[0]*%%%%%%%
\section{Quick Start}
\switchcolumn
\section{快速上手}
\switchcolumn[0]*%%%%%%%
\subsection{Try Vue Online}
\switchcolumn
\subsection{线上尝试 Vue}
\switchcolumn[0]*%%%%%%%
\begin{itemize}
\item
    To quickly get a taste of Vue, you can try it directly in our
    \href{https://play.vuejs.org/\#eNo9jcEKwjAMhl/lt5fpQYfXUQfefAMvvRQbddC1pUuHUPrudg4HIcmXjyRZXEM4zYlEJ+T0iEPgXjn6BB8Zhp46WUZWDjCa9f6w9kAkTtH9CRinV4fmRtZ63H20Ztesqiylphqy3R5UYBqD1UyVAPk+9zkvV1CKbCv9poMLiTEfR2/IXpSoXomqZLtti/IFwVtA9A==}{Playground}.
\item
    If you prefer a plain HTML setup without any build steps, you can use
    this \href{https://jsfiddle.net/yyx990803/2ke1ab0z/}{JSFiddle} as your
    starting point.
\item
    If you are already familiar with Node.js and the concept of build
    tools, you can also try a complete build setup right within your
    browser on \href{https://vite.new/vue}{StackBlitz}.
\end{itemize}
\switchcolumn
\begin{itemize}
\item
    想要快速体验
    Vue,你可以直接试试我们的\href{https://play.vuejs.org/\#eNo9jcEKwjAMhl/lt5fpQYfXUQfefAMvvRQbddC1pUuHUPrudg4HIcmXjyRZXEM4zYlEJ+T0iEPgXjn6BB8Zhp46WUZWDjCa9f6w9kAkTtH9CRinV4fmRtZ63H20Ztesqiylphqy3R5UYBqD1UyVAPk+9zkvV1CKbCv9poMLiTEfR2/IXpSoXomqZLtti/IFwVtA9A==}{演练场}。
\item
    如果你更喜欢不用任何构建的原始 HTML,可以使用
    \href{https://jsfiddle.net/yyx990803/2ke1ab0z/}{JSFiddle} 入门。
\item
    如果你已经比较熟悉 Node.js 和构建工具等概念,还可以直接在浏览器中打开
    \href{https://vite.new/vue}{StackBlitz} 来尝试完整的构建设置。
\end{itemize}
\switchcolumn[0]*%%%%%%%
\subsection{Creating a Vue Application}
\switchcolumn
\subsection{创建一个 Vue 应用}
\switchcolumn[0]*%%%%%%%
\begin{vueQuote}{Prerequisites}
\begin{itemize}
\item
    Familiarity with the command line
\item
    Install \href{https://nodejs.org/}{Node.js} version 16.0 or higher
\end{itemize}        
\end{vueQuote}    
\switchcolumn
\begin{vueQuote}{前提条件}
\begin{itemize}
\item
    熟悉命令行
\item
    已安装 16.0 或更高版本的 \href{https://nodejs.org/}{Node.js}
\end{itemize}
\end{vueQuote}    
\switchcolumn[0]*%%%%%%%
In this section we will introduce how to scaffold a Vue
\href{https://vuejs.org/guide/extras/ways-of-using-vue.html\#single-page-application-spa}{Single
Page Application} on your local machine. The created project will be
using a build setup based on \href{https://vitejs.dev/}{Vite} and allow
us to use Vue
\href{https://vuejs.org/guide/scaling-up/sfc.html}{Single-File
Components} (SFCs).
\switchcolumn
在本节中,我们将介绍如何在本地搭建 Vue
\href{https://cn.vuejs.org/guide/extras/ways-of-using-vue.html\#single-page-application-spa}{单页应用}。创建的项目将使用基于
\href{https://vitejs.dev/}{Vite} 的构建设置,并允许我们使用 Vue
的\href{https://cn.vuejs.org/guide/scaling-up/sfc.html}{单文件组件}
(SFC)。
\switchcolumn[0]*%%%%%%%
Make sure you have an up-to-date version of
\href{https://nodejs.org/}{Node.js} installed and your current working
directory is the one where you intend to create a project. Run the
following command in your command line (without the
\texttt{\textgreater{}} sign): 
\switchcolumn
确保你安装了最新版本的
\href{https://nodejs.org/}{Node.js},并且你的当前工作目录正是打算创建项目的目录。在命令行中运行以下命令
(不要带上 \texttt{\textgreater{}} 符号):
\switchcolumn[0]*%%%%%%%
\begin{codeShell}
npm create vue@latest
\end{codeShell}
\switchcolumn
\begin{codeShell}
npm create vue@latest
\end{codeShell}
\switchcolumn[0]*%%%%%%%
This command will install and execute
\href{https://github.com/vuejs/create-vue}{create-vue}, the official Vue
project scaffolding tool. You will be presented with prompts for several
optional features such as TypeScript and testing support:
\switchcolumn
这一指令将会安装并执行
\href{https://github.com/vuejs/create-vue}{create-vue},它是 Vue
官方的项目脚手架工具。你将会看到一些诸如 TypeScript
和测试支持之类的可选功能提示:
\switchcolumn[0]*%%%%%%%
\begin{codeConsole*}{escapeinside=||}
|\checkmark| Project name: … <your-project-name>
|\checkmark| Add TypeScript? … No / Yes
|\checkmark| Add JSX Support? … No / Yes
|\checkmark| Add Vue Router for Single Page Application development? … No / Yes
|\checkmark| Add Pinia for state management? … No / Yes
|\checkmark| Add Vitest for Unit testing? … No / Yes
|\checkmark| Add an End-to-End Testing Solution? … No / Cypress / Playwright
|\checkmark| Add ESLint for code quality? … No / Yes
|\checkmark| Add Prettier for code formatting? … No / Yes

Scaffolding project in ./<your-project-name>...
Done.
\end{codeConsole*}
\switchcolumn
\begin{codeConsole*}{escapeinside=||}
|\checkmark| Project name: … <your-project-name>
|\checkmark| Add TypeScript? … No / Yes
|\checkmark| Add JSX Support? … No / Yes
|\checkmark| Add Vue Router for Single Page Application development? … No / Yes
|\checkmark| Add Pinia for state management? … No / Yes
|\checkmark| Add Vitest for Unit testing? … No / Yes
|\checkmark| Add an End-to-End Testing Solution? … No / Cypress / Playwright
|\checkmark| Add ESLint for code quality? … No / Yes
|\checkmark| Add Prettier for code formatting? … No / Yes

Scaffolding project in ./<your-project-name>...
Done.
\end{codeConsole*} 
\switchcolumn[0]*%%%%%%%
If you are unsure about an option, simply choose \texttt{No} by hitting
enter for now. Once the project is created, follow the instructions to
install dependencies and start the dev server:
\switchcolumn
如果不确定是否要开启某个功能,你可以直接按下回车键选择
\texttt{No}。在项目被创建后,通过以下步骤安装依赖并启动开发服务器:
\switchcolumn[0]*%%%%%%%
\begin{codeShellMul}
cd <your-project-name>
npm install
npm run dev
\end{codeShellMul}
\switchcolumn
\begin{codeShellMul}
cd <your-project-name>
npm install
npm run dev
\end{codeShellMul}
\switchcolumn[0]*%%%%%%%
You should now have your first Vue project running! Note that the
example components in the generated project are written using the
\href{https://vuejs.org/guide/introduction.html\#composition-api}{Composition
API} and \texttt{\textless{}script\ setup\textgreater{}}, rather than
the
\href{https://vuejs.org/guide/introduction.html\#options-api}{Options
API}. Here are some additional tips:
\switchcolumn
你现在应该已经运行起来了你的第一个 Vue
项目!请注意,生成的项目中的示例组件使用的是\href{https://cn.vuejs.org/guide/introduction.html\#composition-api}{组合式
API} 和
\texttt{\textless{}script\ setup\textgreater{}},而非\href{https://cn.vuejs.org/guide/introduction.html\#options-api}{选项式
API}。下面是一些补充提示:
\switchcolumn[0]*%%%%%%%
\begin{itemize}
    \item
      The recommended IDE setup is
      \href{https://code.visualstudio.com/}{Visual Studio Code} +
      \href{https://marketplace.visualstudio.com/items?itemName=Vue.volar}{Volar
      extension}. If you use other editors, check out the
      \href{https://vuejs.org/guide/scaling-up/tooling.html\#ide-support}{IDE
      support section}.
    \item
      More tooling details, including integration with backend frameworks,
      are discussed in the
      \href{https://vuejs.org/guide/scaling-up/tooling.html}{Tooling Guide}.
    \item
      To learn more about the underlying build tool Vite, check out the
      \href{https://vitejs.dev/}{Vite docs}.
    \item
      If you choose to use TypeScript, check out the
      \href{https://vuejs.org/guide/typescript/overview.html}{TypeScript
      Usage Guide}.
    \end{itemize}
\switchcolumn
\begin{itemize}
    \item
      推荐的 IDE 配置是 \href{https://code.visualstudio.com/}{Visual Studio
      Code} +
      \href{https://marketplace.visualstudio.com/items?itemName=Vue.volar}{Volar
      扩展}。如果使用其他编辑器,参考
      \href{https://cn.vuejs.org/guide/scaling-up/tooling.html\#ide-support}{IDE
      支持章节}。
    \item
      更多工具细节,包括与后端框架的整合,我们会在\href{https://cn.vuejs.org/guide/scaling-up/tooling.html}{工具链指南}进行讨论。
    \item
      要了解构建工具 Vite 更多背后的细节,请查看
      \href{https://cn.vitejs.dev/}{Vite 文档}。
    \item
      如果你选择使用 TypeScript,请阅读
      \href{https://cn.vuejs.org/guide/typescript/overview.html}{TypeScript
      使用指南}。
    \end{itemize}
\switchcolumn[0]*%%%%%%%
When you are ready to ship your app to production, run the following:
\switchcolumn
当你准备将应用发布到生产环境时,请运行:
\switchcolumn[0]*%%%%%%%
\begin{codeShellMul}
npm run build
\end{codeShellMul}
\switchcolumn
\begin{codeShellMul}
npm run build
\end{codeShellMul}
\switchcolumn[0]*%%%%%%%
This will create a production-ready build of your app in the project's
\texttt{./dist} directory. Check out the
\href{https://vuejs.org/guide/best-practices/production-deployment.html}{Production
Deployment Guide} to learn more about shipping your app to production.
\switchcolumn
此命令会在 \texttt{./dist}
文件夹中为你的应用创建一个生产环境的构建版本。关于将应用上线生产环境的更多内容,请阅读\href{https://cn.vuejs.org/guide/best-practices/production-deployment.html}{生产环境部署指南}。
\switchcolumn[0]*%%%%%%%
\href{https://vuejs.org/guide/quick-start.html\#next-steps}{Next Steps
\textgreater{}}
\switchcolumn
\href{https://cn.vuejs.org/guide/quick-start.html\#next-steps}{下一步\textgreater{}}
\switchcolumn[0]*%%%%%%%

\switchcolumn

\switchcolumn[0]*%%%%%%%

\switchcolumn

\switchcolumn[0]*%%%%%%%

\switchcolumn

\switchcolumn[0]*%%%%%%%

\switchcolumn

\switchcolumn[0]*%%%%%%%

\switchcolumn

\switchcolumn[0]*%%%%%%%

\switchcolumn

\switchcolumn[0]*%%%%%%%

\switchcolumn

\switchcolumn[0]*%%%%%%%

\switchcolumn

\switchcolumn[0]*%%%%%%%

\switchcolumn

\switchcolumn[0]*%%%%%%%

\switchcolumn

\switchcolumn[0]*%%%%%%%

\switchcolumn

\switchcolumn[0]*%%%%%%%

\switchcolumn

\switchcolumn[0]*%%%%%%%

\switchcolumn

\switchcolumn[0]*%%%%%%%

\switchcolumn

\switchcolumn[0]*%%%%%%%

\switchcolumn

\switchcolumn[0]*%%%%%%%

\switchcolumn

\switchcolumn[0]*%%%%%%%

\switchcolumn

\switchcolumn[0]*%%%%%%%

\switchcolumn

\switchcolumn[0]*%%%%%%%

\switchcolumn

\switchcolumn[0]*%%%%%%%

\switchcolumn


\switchcolumn[0]*%%%%%%%

\switchcolumn


\switchcolumn[0]*%%%%%%%

\switchcolumn

\switchcolumn[0]*%%%%%%%

\switchcolumn


\switchcolumn[0]*%%%%%%%

\switchcolumn

\switchcolumn[0]*%%%%%%%

\switchcolumn


\switchcolumn[0]*%%%%%%%

\switchcolumn

\switchcolumn[0]*%%%%%%%

\switchcolumn


\switchcolumn[0]*%%%%%%%

\switchcolumn

\switchcolumn[0]*%%%%%%%

\switchcolumn


\switchcolumn[0]*%%%%%%%

\switchcolumn


\switchcolumn[0]*%%%%%%%

\switchcolumn

\switchcolumn[0]*%%%%%%%

\switchcolumn


\switchcolumn[0]*%%%%%%%

\switchcolumn

\switchcolumn[0]*%%%%%%%

\switchcolumn


\switchcolumn[0]*%%%%%%%

\switchcolumn

\switchcolumn[0]*%%%%%%%

\switchcolumn


\switchcolumn[0]*%%%%%%%

\switchcolumn


\end{paracol}
 

\chapter{Essentials\hfill 基础}
% \columnratio{0.55}
\begin{paracol}{2}
\switchcolumn[0]*%%%%%%%
\section{Creating a Vue Application}
\switchcolumn
\section{创建一个 Vue 应用}
\switchcolumn[0]*%%%%%%%
\subsection{The application instance}
\switchcolumn
\subsection{应用实例}
\switchcolumn[0]*%%%%%%%
Every Vue application starts by creating a new \textbf{application
instance} with the
\href{https://vuejs.org/api/application.html\#createapp}{\texttt{createApp}}
function:
\switchcolumn
每个 Vue 应用都是通过
\href{https://cn.vuejs.org/api/application.html\#createapp}{\texttt{createApp}}
函数创建一个新的 \textbf{应用实例}:
\switchcolumn[0]*%%%%%%%
\begin{codeJs}
import { createApp } from 'vue'

const app = createApp({
    /* root component options */
})
\end{codeJs}
\switchcolumn
\begin{codeJs}
import { createApp } from 'vue'

const app = createApp({
    /* root component options */
})
\end{codeJs}
\switchcolumn[0]*%%%%%%%
\subsection{The Root Component}
\switchcolumn
\subsection{根组件}
\switchcolumn[0]*%%%%%%%
The object we are passing into \texttt{createApp} is in fact a
component. Every app requires a "root component" that can contain other
components as its children.
\switchcolumn
我们传入 \texttt{createApp}
的对象实际上是一个组件,每个应用都需要一个``根组件'',其他组件将作为其子组件。
\switchcolumn[0]*%%%%%%%
If you are using Single-File Components, we typically import the root
component from another file:
\switchcolumn
如果你使用的是单文件组件,我们可以直接从另一个文件中导入根组件。
\switchcolumn[0]*%%%%%%%
\begin{codeJs}
import { createApp } from 'vue'
// import the root component App from a single-file component.
import App from './App.vue'

const app = createApp(App)
\end{codeJs}
\switchcolumn
\begin{codeJs}
import { createApp } from 'vue'
// import the root component App from a single-file component.
import App from './App.vue'

const app = createApp(App)
\end{codeJs}
\switchcolumn[0]*%%%%%%%
While many examples in this guide only need a single component, most
real applications are organized into a tree of nested, reusable
components. For example, a Todo application's component tree might look
like this:
\switchcolumn
虽然本指南中的许多示例只需要一个组件,但大多数真实的应用都是由一棵嵌套的、可重用的组件树组成的。例如,一个待办事项
(Todos) 应用的组件树可能是这样的:

\end{paracol}

% \columnratio{0.55}
\begin{paracol}{2}
\switchcolumn[0]*%%%%%%%
\section{Template Syntax}
\switchcolumn
\section{模板语法}
\switchcolumn[0]*%%%%%%%
Vue uses an HTML-based template syntax that allows you to declaratively
bind the rendered DOM to the underlying component instance's data. All
Vue templates are syntactically valid HTML that can be parsed by
spec-compliant browsers and HTML parsers.
\switchcolumn
Vue 使用一种基于 HTML
的模板语法,使我们能够声明式地将其组件实例的数据绑定到呈现的 DOM
上。所有的 Vue 模板都是语法层面合法的 HTML,可以被符合规范的浏览器和
HTML 解析器解析。
\switchcolumn[0]*%%%%%%%
Under the hood, Vue compiles the templates into highly-optimized
JavaScript code. Combined with the reactivity system, Vue can
intelligently figure out the minimal number of components to re-render
and apply the minimal amount of DOM manipulations when the app state
changes.
\switchcolumn
在底层机制中,Vue 会将模板编译成高度优化的 JavaScript
代码。结合响应式系统,当应用状态变更时,Vue
能够智能地推导出需要重新渲染的组件的最少数量,并应用最少的 DOM 操作。
\switchcolumn[0]*%%%%%%%
If you are familiar with Virtual DOM concepts and prefer the raw power
of JavaScript, you can also
\href{https://vuejs.org/guide/extras/render-function.html}{directly
write render functions} instead of templates, with optional JSX support.
However, do note that they do not enjoy the same level of compile-time
optimizations as templates.
\switchcolumn
如果你对虚拟 DOM 的概念比较熟悉,并且偏好直接使用
JavaScript,你也可以结合可选的 JSX
支持\href{https://cn.vuejs.org/guide/extras/render-function.html}{直接手写渲染函数}而不采用模板。但请注意,这将不会享受到和模板同等级别的编译时优化。
\switchcolumn[0]*%%%%%%%
\subsection{Text Interpolation}
\switchcolumn
\subsection{文本插值}
\switchcolumn[0]*%%%%%%%
The most basic form of data binding is text interpolation using the
"Mustache" syntax (double curly braces):
\switchcolumn
最基本的数据绑定形式是文本插值,它使用的是``Mustache''语法
(即双大括号):
\switchcolumn[0]*%%%%%%%
\begin{codeHtml*}{label=template}
<span>Message: {{ msg }}</span>
\end{codeHtml*}  
\switchcolumn
\begin{codeHtml*}{label=template}
<span>Message: {{ msg }}</span>
\end{codeHtml*}  

\switchcolumn[0]*%%%%%%%
The mustache tag will be replaced with the value of the \texttt{msg}
property
\href{https://vuejs.org/guide/essentials/reactivity-fundamentals.html\#declaring-reactive-state}{from
the corresponding component instance}. It will also be updated whenever
the \texttt{msg} property changes.
\switchcolumn
双大括号标签会被替换为\href{https://cn.vuejs.org/guide/essentials/reactivity-fundamentals.html\#declaring-reactive-state}{相应组件实例中}
\texttt{msg} 属性的值。同时每次 \texttt{msg} 属性更改时它也会同步更新。
\switchcolumn[0]*%%%%%%%
\subsection{Raw HTML}
\switchcolumn
\subsection{原始 HTML}
\switchcolumn[0]*%%%%%%%
The double mustaches interpret the data as plain text, not HTML. In
order to output real HTML, you will need to use the
\href{https://vuejs.org/api/built-in-directives.html\#v-html}{\texttt{v-html}
directive}:
\switchcolumn
双大括号会将数据解释为纯文本,而不是 HTML。若想插入 HTML,你需要使用
\href{https://cn.vuejs.org/api/built-in-directives.html\#v-html}{\texttt{v-html}
指令}:
\switchcolumn[0]*%%%%%%%
\begin{codeHtml}
<p>Using text interpolation: {{ rawHtml }}</p>
<p>Using v-html directive: <span v-html="rawHtml"></span></p>
\end{codeHtml}  
\switchcolumn
\begin{codeHtml}
<p>Using text interpolation: {{ rawHtml }}</p>
<p>Using v-html directive: <span v-html="rawHtml"></span></p>
\end{codeHtml}  
\switchcolumn[0]*%%%%%%%
\begin{vueQuote}{result}
Using text interpolation: This should be red.\\
Using v-html directive: {\textcolor{red}{This should be red.}}
\end{vueQuote}
\switchcolumn
\begin{vueQuote}{结果}
Using text interpolation: This should be red.\\
Using v-html directive: {\textcolor{red}{This should be red.}}
\end{vueQuote}

\switchcolumn[0]*%%%%%%%
Here we're encountering something new. The \texttt{v-html} attribute
you're seeing is called a \textbf{directive}. Directives are prefixed
with \texttt{v-} to indicate that they are special attributes provided
by Vue, and as you may have guessed, they apply special reactive
behavior to the rendered DOM. Here, we're basically saying "keep this
element's inner HTML up-to-date with the \texttt{rawHtml} property on
the current active instance."
\switchcolumn
这里我们遇到了一个新的概念。这里看到的 \texttt{v-html} attribute
被称为一个\textbf{指令}。指令由 \texttt{v-} 作为前缀,表明它们是一些由
Vue 提供的特殊 attribute,你可能已经猜到了,它们将为渲染的 DOM
应用特殊的响应式行为。这里我们做的事情简单来说就是:在当前组件实例上,将此元素的
innerHTML 与 \texttt{rawHtml} 属性保持同步。
\switchcolumn[0]*%%%%%%%
The contents of the \texttt{span} will be replaced with the value of the
\texttt{rawHtml} property, interpreted as plain HTML - data bindings are
ignored. Note that you cannot use \texttt{v-html} to compose template
partials, because Vue is not a string-based templating engine. Instead,
components are preferred as the fundamental unit for UI reuse and
composition.
\switchcolumn
\texttt{span} 的内容将会被替换为 \texttt{rawHtml} 属性的值,插值为纯
HTML------数据绑定将会被忽略。注意,你不能使用 \texttt{v-html}
来拼接组合模板,因为 Vue 不是一个基于字符串的模板引擎。在使用 Vue
时,应当使用组件作为 UI 重用和组合的基本单元。
\switchcolumn[0]*%%%%%%%
\begin{vueQuoteWarn}{Security Warning}
Dynamically rendering arbitrary HTML on your website can be very
dangerous because it can easily lead to
\href{https://en.wikipedia.org/wiki/Cross-site_scripting}{XSS
vulnerabilities}. Only use \texttt{v-html} on trusted content and
\textbf{never} on user-provided content.
\end{vueQuoteWarn}
\switchcolumn
\begin{vueQuoteWarn}{安全警告}
在网站上动态渲染任意 HTML 是非常危险的,因为这非常容易造成
\href{https://zh.wikipedia.org/wiki/跨網站指令碼}{XSS
漏洞}。请仅在内容安全可信时再使用
\texttt{v-html},并且\textbf{永远不要}使用用户提供的 HTML 内容。
\end{vueQuoteWarn}
\switchcolumn[0]*%%%%%%%
\subsection{Attribute Bindings}
\switchcolumn
\subsection{Attribute 绑定}
\switchcolumn[0]*%%%%%%%
Mustaches cannot be used inside HTML attributes. Instead, use a
\href{https://vuejs.org/api/built-in-directives.html\#v-bind}{\texttt{v-bind}
directive}:
\switchcolumn
双大括号不能在 HTML attributes 中使用。想要响应式地绑定一个
attribute,应该使用
\href{https://cn.vuejs.org/api/built-in-directives.html\#v-bind}{\texttt{v-bind}
指令}:

\switchcolumn[0]*%%%%%%%
\begin{codeHtml}
<div v-bind:id="dynamicId"></div>
\end{codeHtml}  
\switchcolumn
\begin{codeHtml}
<div v-bind:id="dynamicId"></div>
\end{codeHtml}  
\switchcolumn[0]*%%%%%%%
The \texttt{v-bind} directive instructs Vue to keep the element's
\texttt{id} attribute in sync with the component's \texttt{dynamicId}
property. If the bound value is \texttt{null} or \texttt{undefined},
then the attribute will be removed from the rendered element.
\switchcolumn
\texttt{v-bind} 指令指示 Vue 将元素的 \texttt{id} attribute 与组件的
\texttt{dynamicId} 属性保持一致。如果绑定的值是 \texttt{null} 或者
\texttt{undefined},那么该 attribute 将会从渲染的元素上移除。
\switchcolumn[0]*%%%%%%%
\subsubsection{Shorthand}
\switchcolumn
\subsubsection{简写}
\switchcolumn[0]*%%%%%%%
Because \texttt{v-bind} is so commonly used, it has a dedicated
shorthand syntax:
\switchcolumn
因为 \texttt{v-bind} 非常常用,我们提供了特定的简写语法:


\switchcolumn[0]*%%%%%%%
\begin{codeHtml}
<div :id="dynamicId"></div>
\end{codeHtml}  
\switchcolumn
\begin{codeHtml}
<div :id="dynamicId"></div>
\end{codeHtml}  
\switchcolumn[0]*%%%%%%%
Attributes that start with \texttt{:} may look a bit different from
normal HTML, but it is in fact a valid character for attribute names and
all Vue-supported browsers can parse it correctly. In addition, they do
not appear in the final rendered markup. The shorthand syntax is
optional, but you will likely appreciate it when you learn more about
its usage later.
\switchcolumn
开头为 \texttt{:} 的 attribute 可能和一般的 HTML attribute
看起来不太一样,但它的确是合法的 attribute 名称字符,并且所有支持 Vue
的浏览器都能正确解析它。此外,他们不会出现在最终渲染的 DOM
中。简写语法是可选的,但相信在你了解了它更多的用处后,你应该会更喜欢它。
\switchcolumn[0]*%%%%%%%
\begin{quote}
For the rest of the guide, we will be using the shorthand syntax in code
examples, as that's the most common usage for Vue developers.
\end{quote}
\switchcolumn
\begin{quote}
接下来的指引中,我们都将在示例中使用简写语法,因为这是在实际开发中更常见的用法。
\end{quote}
\switchcolumn[0]*%%%%%%%
\subsubsection{Boolean Attributes}
\switchcolumn
\subsubsection{布尔型 Attribute}
\switchcolumn[0]*%%%%%%%
\href{https://html.spec.whatwg.org/multipage/common-microsyntaxes.html\#boolean-attributes}{Boolean
attributes} are attributes that can indicate true / false values by
their presence on an element. For example,
\href{https://developer.mozilla.org/en-US/docs/Web/HTML/Attributes/disabled}{\texttt{disabled}}
is one of the most commonly used boolean attributes.
\switchcolumn
\href{https://developer.mozilla.org/zh-CN/docs/Web/HTML/Attributes\#布尔值属性}{布尔型
attribute} 依据 true / false 值来决定 attribute
是否应该存在于该元素上。\href{https://developer.mozilla.org/en-US/docs/Web/HTML/Attributes/disabled}{\texttt{disabled}}
就是最常见的例子之一。
\switchcolumn[0]*%%%%%%%
\texttt{v-bind} works a bit differently in this case:
\switchcolumn
\texttt{v-bind} 在这种场景下的行为略有不同:
\end{paracol}

% \columnratio{0.55}
\begin{paracol}{2}
\switchcolumn[0]*%%%%%%%
\section{Reactivity Fundamentals}
\switchcolumn
\section{响应式基础}
\switchcolumn[0]*%%%%%%%
\begin{vueQuote}{API Preference}
This page and many other chapters later in the guide contain different
content for the Options API and the Composition API. Your current
preference is Composition API. You can toggle between the API styles
using the "API Preference" switches at the top of the left sidebar.
\end{vueQuote} 
\switchcolumn
\begin{vueQuote}{API 参考}
本页和后面很多页面中都分别包含了选项式 API 和组合式 API
的示例代码。现在你选择的是 组合式 API。你可以使用左侧侧边栏顶部的 ``API
风格偏好'' 开关在 API 风格之间切换。
\end{vueQuote} 
\switchcolumn[0]*%%%%%%%
\subsection{Declaring Reactive State}
\switchcolumn
\subsection{声明响应式状态}
\switchcolumn[0]*%%%%%%%
\subsubsection{ref()}
\switchcolumn
\subsubsection{ref()}
\switchcolumn[0]*%%%%%%%
In Composition API, the recommended way to declare reactive state is
using the
\href{https://vuejs.org/api/reactivity-core.html\#ref}{\texttt{ref()}}
function:
\switchcolumn
在组合式 API 中,推荐使用
\href{https://cn.vuejs.org/api/reactivity-core.html\#ref}{\texttt{ref()}}
函数来声明响应式状态:
\switchcolumn[0]*%%%%%%%
\begin{codeJs}
import { ref } from 'vue'

const count = ref(0)
\end{codeJs}
\switchcolumn
\begin{codeJs}
import { ref } from 'vue'

const count = ref(0)
\end{codeJs}
\switchcolumn[0]*%%%%%%%
\texttt{ref()} takes the argument and returns it wrapped within a ref
object with a \texttt{.value} property:
\switchcolumn
\texttt{ref()} 接收参数,并将其包裹在一个带有 \texttt{.value} 属性的 ref
对象中返回:
\switchcolumn[0]*%%%%%%%
\begin{codeJs}
const count = ref(0)

console.log(count) // { value: 0 }
console.log(count.value) // 0

count.value++
console.log(count.value) // 1
\end{codeJs}
\switchcolumn
\begin{codeJs}
const count = ref(0)

console.log(count) // { value: 0 }
console.log(count.value) // 0

count.value++
console.log(count.value) // 1
\end{codeJs}

\switchcolumn[0]*%%%%%%%
\begin{quote}
See also:
\href{https://vuejs.org/guide/typescript/composition-api.html\#typing-ref}{Typing
Refs}
\end{quote}
\switchcolumn
\begin{quote}
参考:\href{https://cn.vuejs.org/guide/typescript/composition-api.html\#typing-ref}{为
refs 标注类型}
\end{quote}
\switchcolumn[0]*%%%%%%%
To access refs in a component's template, declare and return them from a
component's \texttt{setup()} function:
\switchcolumn
要在组件模板中访问 ref,请从组件的 \texttt{setup()}
函数中声明并返回它们:
\switchcolumn[0]*%%%%%%%
\begin{codeJs}
import { ref } from 'vue'

export default {
    // `setup` 是一个特殊的钩子,专门用于组合式 API。
    setup() {
    const count = ref(0)

    // 将 ref 暴露给模板
    return {
        count
    }
    }
}
\end{codeJs}
\switchcolumn
\begin{codeJs}
import { ref } from 'vue'

export default {
    // `setup` 是一个特殊的钩子,专门用于组合式 API。
    setup() {
    const count = ref(0)

    // 将 ref 暴露给模板
    return {
        count
    }
    }
}
\end{codeJs}
\switchcolumn[0]*%%%%%%%
\begin{codeHtml}
<div>{{ count }}</div>
\end{codeHtml}  
\switchcolumn
\begin{codeHtml}
<div>{{ count }}</div>
\end{codeHtml}  

\switchcolumn[0]*%%%%%%%
Notice that we did \textbf{not} need to append \texttt{.value} when
using the ref in the template. For convenience, refs are automatically
unwrapped when used inside templates (with a few
\href{https://vuejs.org/guide/essentials/reactivity-fundamentals.html\#caveat-when-unwrapping-in-templates}{caveats}).
\switchcolumn
注意,在模板中使用 ref 时,我们\textbf{不}需要附加
\texttt{.value}。为了方便起见,当在模板中使用时,ref 会自动解包
(有一些\href{https://cn.vuejs.org/guide/essentials/reactivity-fundamentals.html\#caveat-when-unwrapping-in-templates}{注意事项})。
\switchcolumn[0]*%%%%%%%
You can also mutate a ref directly in event handlers:
\switchcolumn
你也可以直接在事件监听器中改变一个 ref:
\switchcolumn[0]*%%%%%%%
\begin{codeHtml}
<button @click="count++">
{{ count }}
</button>
\end{codeHtml}  
\switchcolumn
\begin{codeHtml}
<button @click="count++">
{{ count }}
</button>
\end{codeHtml}  

\switchcolumn[0]*%%%%%%%
For more complex logic, we can declare functions that mutate refs in the
same scope and expose them as methods alongside the state:
\switchcolumn
对于更复杂的逻辑,我们可以在同一作用域内声明更改 ref
的函数,并将它们作为方法与状态一起公开:
\switchcolumn[0]*%%%%%%%
\begin{codeJs}
import { ref } from 'vue'

export default {
    setup() {
    const count = ref(0)

    function increment() {
        // .value is needed in JavaScript
        count.value++
    }

    // don't forget to expose the function as well.
    return {
        count,
        increment
    }
    }
}
\end{codeJs}
\switchcolumn
\begin{codeJs}
import { ref } from 'vue'

export default {
    setup() {
    const count = ref(0)

    function increment() {
        // 在 JavaScript 中需要 .value
        count.value++
    }

    // 不要忘记同时暴露 increment 函数
    return {
        count,
        increment
    }
    }
}
\end{codeJs}

\end{paracol}



\end{document}]%%%%%%%]*%%%%%%%]*%%%%%%%]*%%%%%%%]*%%%%%%%]*%%%%%%%]*%%%%%%%]*%%%%%%%]*%%%%%%%]*%%%%%%%
\switchcolumn[0]*%%%%%%%
\begin{vueQuote}{}
\end{vueQuote} 
\switchcolumn
\begin{vueQuote}{}
\end{vueQuote} 




\switchcolumn[0]*%%%%%%%
\begin{codeHtml}

\end{codeHtml}  
\switchcolumn
\begin{codeHtml}

\end{codeHtml}  
% 
\columnratio{0.55}
\begin{paracol}{2}

\switchcolumn[0]*%%%%%%%
\section{Computed Properties}
\switchcolumn
\section{计算属性}
\switchcolumn[0]*%%%%%%%
\subsection{Basic Example}
\switchcolumn
\subsection{基础示例}
\switchcolumn[0]*%%%%%%%
In-template expressions are very convenient, but they are meant for
simple operations. Putting too much logic in your templates can make
them bloated and hard to maintain. For example, if we have an object
with a nested array:
\switchcolumn
模板中的表达式虽然方便,但也只能用来做简单的操作。如果在模板中写太多逻辑,会让模板变得臃肿,难以维护。比如说,我们有这样一个包含嵌套数组的对象:
\switchcolumn[0]*%%%%%%%
\begin{codeJs}
const author = reactive({
    name: 'John Doe',
    books: [
        'Vue 2 - Advanced Guide',
        'Vue 3 - Basic Guide',
        'Vue 4 - The Mystery'
    ]
})
\end{codeJs}
\switchcolumn
\begin{codeJs}
const author = reactive({
    name: 'John Doe',
    books: [
        'Vue 2 - Advanced Guide',
        'Vue 3 - Basic Guide',
        'Vue 4 - The Mystery'
    ]
})
\end{codeJs}
\switchcolumn[0]*%%%%%%%
And we want to display different messages depending on if
\texttt{author} already has some books or not:
\switchcolumn
我们想根据 \texttt{author} 是否已有一些书籍来展示不同的信息:
\switchcolumn[0]*%%%%%%%
\begin{codeHtml}
<p>Has published books:</p>
<span>{{ author.books.length > 0 ? 'Yes' : 'No' }}</span>
\end{codeHtml}
\switchcolumn
\begin{codeHtml}
<p>Has published books:</p>
<span>{{ author.books.length > 0 ? 'Yes' : 'No' }}</span>
\end{codeHtml}
\switchcolumn[0]*%%%%%%%
At this point, the template is getting a bit cluttered. We have to look
at it for a second before realizing that it performs a calculation
depending on \texttt{author.books}. More importantly, we probably don't
want to repeat ourselves if we need to include this calculation in the
template more than once.
\switchcolumn
这里的模板看起来有些复杂。我们必须认真看好一会儿才能明白它的计算依赖于
\texttt{author.books}。更重要的是,如果在模板中需要不止一次这样的计算,我们可不想将这样的代码在模板里重复好多遍。
\switchcolumn[0]*%%%%%%%
That's why for complex logic that includes reactive data, it is
recommended to use a \textbf{computed property}. Here's the same
example, refactored:
\switchcolumn
因此我们推荐使用\textbf{计算属性}来描述依赖响应式状态的复杂逻辑。这是重构后的示例:
\switchcolumn[0]*%%%%%%%
\begin{codeHtml}
<script setup>
import { reactive, computed } from 'vue'

const author = reactive({
    name: 'John Doe',
    books: [
    'Vue 2 - Advanced Guide',
    'Vue 3 - Basic Guide',
    'Vue 4 - The Mystery'
    ]
})

// 一个计算属性 ref
const publishedBooksMessage = computed(() => {
    return author.books.length > 0 ? 'Yes' : 'No'
})
</script>

<template>
    <p>Has published books:</p>
    <span>{{ publishedBooksMessage }}</span>
</template>
\end{codeHtml}
\switchcolumn
\begin{codeHtml}
<script setup>
import { reactive, computed } from 'vue'

const author = reactive({
    name: 'John Doe',
    books: [
    'Vue 2 - Advanced Guide',
    'Vue 3 - Basic Guide',
    'Vue 4 - The Mystery'
    ]
})

// 一个计算属性 ref
const publishedBooksMessage = computed(() => {
    return author.books.length > 0 ? 'Yes' : 'No'
})
</script>

<template>
    <p>Has published books:</p>
    <span>{{ publishedBooksMessage }}</span>
</template>
\end{codeHtml}
\switchcolumn[0]*%%%%%%%
\href{https://play.vuejs.org/\#eNp1kE9Lw0AQxb/KI5dtoTainkoaaREUoZ5EEONhm0ybYLO77J9CCfnuzta0vdjbzr6Zeb95XbIwZroPlMySzJW2MR6OfDB5oZrWaOvRwZIsfbOnCUrdmuCpQo+N1S0ET4pCFarUynnI4GttMT9PjLpCAUq2NIN41bXCkyYxiZ9rrX/cDF/xDYiPQLjDDRbVXqqSHZ5DUw2tg3zP8lK6pvxHe2DtvSasDs6TPTAT8F2ofhzh0hTygm5pc+I1Yb1rXE3VMsKsyDm5JcY/9Y5GY8xzHI+wnIpVw4nTI/10R2rra+S4xSPEJzkBvvNNs310ztK/RDlLLjy1Zic9cQVkJn+R7gIwxJGlMXiWnZEq77orhH3Pq2NH9DjvTfpfSBSbmA==}{Try
it in the Playground}
\switchcolumn
\href{https://play.vuejs.org/\#eNp1kE9Lw0AQxb/KI5dtoTainkoaaREUoZ5EEONhm0ybYLO77J9CCfnuzta0vdjbzr6Zeb95XbIwZroPlMySzJW2MR6OfDB5oZrWaOvRwZIsfbOnCUrdmuCpQo+N1S0ET4pCFarUynnI4GttMT9PjLpCAUq2NIN41bXCkyYxiZ9rrX/cDF/xDYiPQLjDDRbVXqqSHZ5DUw2tg3zP8lK6pvxHe2DtvSasDs6TPTAT8F2ofhzh0hTygm5pc+I1Yb1rXE3VMsKsyDm5JcY/9Y5GY8xzHI+wnIpVw4nTI/10R2rra+S4xSPEJzkBvvNNs310ztK/RDlLLjy1Zic9cQVkJn+R7gIwxJGlMXiWnZEq77orhH3Pq2NH9DjvTfpfSBSbmA==}{在演练场中尝试一下}


\switchcolumn[0]*%%%%%%%
Here we have declared a computed property
\texttt{publishedBooksMessage}. The \texttt{computed()} function expects
to be passed a getter function, and the returned value is a
\textbf{computed ref}. Similar to normal refs, you can access the
computed result as \texttt{publishedBooksMessage.value}. Computed refs
are also auto-unwrapped in templates so you can reference them without
\texttt{.value} in template expressions.
\switchcolumn
我们在这里定义了一个计算属性
\texttt{publishedBooksMessage}。\texttt{computed()} 方法期望接收一个
getter 函数,返回值为一个\textbf{计算属性 ref}。和其他一般的 ref
类似,你可以通过 \texttt{publishedBooksMessage.value}
访问计算结果。计算属性 ref
也会在模板中自动解包,因此在模板表达式中引用时无需添加 \texttt{.value}。
\switchcolumn[0]*%%%%%%%
A computed property automatically tracks its reactive dependencies. Vue
is aware that the computation of \texttt{publishedBooksMessage} depends
on \texttt{author.books}, so it will update any bindings that depend on
\texttt{publishedBooksMessage} when \texttt{author.books} changes.
\switchcolumn
Vue 的计算属性会自动追踪响应式依赖。它会检测到
\texttt{publishedBooksMessage} 依赖于 \texttt{author.books},所以任何依赖于 \texttt{publishedBooksMessage}
的绑定,都会在 \texttt{author.books} 改变时同时更新。
\switchcolumn[0]*%%%%%%%
See also:
\href{https://vuejs.org/guide/typescript/composition-api.html\#typing-computed}{Typing
Computed}
\switchcolumn
也可参考:\href{https://cn.vuejs.org/guide/typescript/composition-api.html\#typing-computed}{为计算属性标注类型}
\end{paracol}

\columnratio{0.55}
\begin{paracol}{2}

\switchcolumn[0]*%%%%%%%
\subsection{Computed Caching vs. Methods}
\switchcolumn
\subsection{计算属性缓存 vs 方法}
\switchcolumn[0]*%%%%%%%
You may have noticed we can achieve the same result by invoking a method
in the expression:
\switchcolumn
你可能注意到我们在表达式中像这样调用一个函数也会获得和计算属性相同的结果:
\switchcolumn[0]*%%%%%%%
\begin{codeHtml}
<p>{{ calculateBooksMessage() }}</p>
\end{codeHtml}
\switchcolumn
\begin{codeHtml}
<p>{{ calculateBooksMessage() }}</p>
\end{codeHtml}
\switchcolumn[0]*%%%%%%%
\begin{codeJs}
// in component
function calculateBooksMessage() {
    return author.books.length > 0 ? 'Yes' : 'No'
}
\end{codeJs}
\switchcolumn
\begin{codeJs}
// 组件中
function calculateBooksMessage() {
    return author.books.length > 0 ? 'Yes' : 'No'
}
\end{codeJs}
\switchcolumn[0]*%%%%%%%
Instead of a computed property, we can define the same function as a
method. For the end result, the two approaches are indeed exactly the
same. However, the difference is that \textbf{computed properties are
cached based on their reactive dependencies.} A computed property will
only re-evaluate when some of its reactive dependencies have changed.
This means as long as \texttt{author.books} has not changed, multiple
access to \texttt{publishedBooksMessage} will immediately return the
previously computed result without having to run the getter function
again.
\switchcolumn
若我们将同样的函数定义为一个方法而不是计算属性,两种方式在结果上确实是完全相同的,然而,不同之处在于\textbf{计算属性值会基于其响应式依赖被缓存}。一个计算属性仅会在其响应式依赖更新时才重新计算。这意味着只要
\texttt{author.books} 不改变,无论多少次访问
\texttt{publishedBooksMessage}
都会立即返回先前的计算结果,而不用重复执行 getter 函数。
\switchcolumn[0]*%%%%%%%
This also means the following computed property will never update,
because \texttt{Date.now()} is not a reactive dependency:
\switchcolumn
这也解释了为什么下面的计算属性永远不会更新,因为 \texttt{Date.now()}
并不是一个响应式依赖:
\switchcolumn[0]*%%%%%%%
\begin{codeJs}
const now = computed(() => Date.now())
\end{codeJs}
\switchcolumn
\begin{codeJs}
const now = computed(() => Date.now())
\end{codeJs}
\switchcolumn[0]*%%%%%%%
In comparison, a method invocation will \textbf{always} run the function
whenever a re-render happens.
\switchcolumn
相比之下,方法调用\textbf{总是}会在重渲染发生时再次执行函数。
\switchcolumn[0]*%%%%%%%
Why do we need caching? Imagine we have an expensive computed property
\texttt{list}, which requires looping through a huge array and doing a
lot of computations. Then we may have other computed properties that in
turn depend on \texttt{list}. Without caching, we would be executing
\texttt{list}'s getter many more times than necessary! In cases where
you do not want caching, use a method call instead.
\switchcolumn
为什么需要缓存呢?想象一下我们有一个非常耗性能的计算属性
\texttt{list},需要循环一个巨大的数组并做许多计算逻辑,并且可能也有其他计算属性依赖于
\texttt{list}。没有缓存的话,我们会重复执行非常多次 \texttt{list} 的
getter,然而这实际上没有必要!如果你确定不需要缓存,那么也可以使用方法调用。
\end{paracol}

\columnratio{0.55}
\begin{paracol}{2}
\switchcolumn[0]*%%%%%%%
\subsection{Writable Computed}
\switchcolumn
\subsection{可写计算属性}
\switchcolumn[0]*%%%%%%%
Computed properties are by default getter-only. If you attempt to assign
a new value to a computed property, you will receive a runtime warning.
In the rare cases where you need a "writable" computed property, you can
create one by providing both a getter and a setter:
\switchcolumn
计算属性默认是只读的。当你尝试修改一个计算属性时,你会收到一个运行时警告。只在某些特殊场景中你可能才需要用到``可写''的属性,你可以通过同时提供
getter 和 setter 来创建:

\switchcolumn[0]*%%%%%%%
\begin{codeHtml}
<script setup>
import { ref, computed } from 'vue'

const firstName = ref('John')
const lastName = ref('Doe')

const fullName = computed({
    // getter
    get() {
    return firstName.value + ' ' + lastName.value
    },
    // setter
    set(newValue) {
    // Note: we are using destructuring assignment syntax here.
    [firstName.value, lastName.value] = newValue.split(' ')
    }
})
</script>
\end{codeHtml}
\switchcolumn
\begin{codeHtml}
<script setup>
import { ref, computed } from 'vue'

const firstName = ref('John')
const lastName = ref('Doe')

const fullName = computed({
    // getter
    get() {
    return firstName.value + ' ' + lastName.value
    },
    // setter
    set(newValue) {
    // 注意:我们这里使用的是解构赋值语法
    [firstName.value, lastName.value] = newValue.split(' ')
    }
})
</script>
\end{codeHtml}

\switchcolumn[0]*%%%%%%%
Now when you run
\texttt{fullName.value\ =\ \textquotesingle{}John\ Doe\textquotesingle{}},
the setter will be invoked and \texttt{firstName} and \texttt{lastName}
will be updated accordingly.
\switchcolumn
现在当你再运行
\texttt{fullName.value\ =\ \textquotesingle{}John\ Doe\textquotesingle{}}
时,setter 会被调用而 \texttt{firstName} 和 \texttt{lastName}
会随之更新。
\switchcolumn[0]*%%%%%%%
\subsection{Best Practices}
\switchcolumn
\subsection{最佳实践}
\switchcolumn[0]*%%%%%%%
\subsubsection{Getters should be side-effect free}
\switchcolumn
\subsubsection{Getter 不应有副作用}
\switchcolumn[0]*%%%%%%%
It is important to remember that computed getter functions should only
perform pure computation and be free of side effects. For example,
\textbf{don't make async requests or mutate the DOM inside a computed
getter!} Think of a computed property as declaratively describing how to
derive a value based on other values - its only responsibility should be
computing and returning that value. Later in the guide we will discuss
how we can perform side effects in reaction to state changes with
\href{https://vuejs.org/guide/essentials/watchers.html}{watchers}.
\switchcolumn
计算属性的 getter
应只做计算而没有任何其他的副作用,这一点非常重要,请务必牢记。举例来说,\textbf{不要在
getter 中做异步请求或者更改
DOM}!一个计算属性的声明中描述的是如何根据其他值派生一个值。因此 getter
的职责应该仅为计算和返回该值。在之后的指引中我们会讨论如何使用\href{https://cn.vuejs.org/guide/essentials/watchers.html}{侦听器}根据其他响应式状态的变更来创建副作用。
\switchcolumn[0]*%%%%%%%
\subsubsection{Avoid mutating computed value}
\switchcolumn
\subsubsection{避免直接修改计算属性值}
\switchcolumn[0]*%%%%%%%
The returned value from a computed property is derived state. Think of
it as a temporary snapshot - every time the source state changes, a new
snapshot is created. It does not make sense to mutate a snapshot, so a
computed return value should be treated as read-only and never be
mutated - instead, update the source state it depends on to trigger new
computations.
\switchcolumn
从计算属性返回的值是派生状态。可以把它看作是一个``临时快照'',每当源状态发生变化时,就会创建一个新的快照。更改快照是没有意义的,因此计算属性的返回值应该被视为只读的,并且永远不应该被更改------应该更新它所依赖的源状态以触发新的计算。
\end{paracol}
\columnratio{0.55}
\begin{paracol}{2}

\switchcolumn[0]*%%%%%%%
\section{Class and Style Bindings}
\switchcolumn
\section{Class 与 Style 绑定}
\switchcolumn[0]*%%%%%%%
A common need for data binding is manipulating an element's class list
and inline styles. Since \texttt{class} and \texttt{style} are both
attributes, we can use \texttt{v-bind} to assign them a string value
dynamically, much like with other attributes. However, trying to
generate those values using string concatenation can be annoying and
error-prone. For this reason, Vue provides special enhancements when
\texttt{v-bind} is used with \texttt{class} and \texttt{style}. In
addition to strings, the expressions can also evaluate to objects or
arrays.
\switchcolumn
数据绑定的一个常见需求场景是操纵元素的 CSS class 列表和内联样式。因为
\texttt{class} 和 \texttt{style} 都是 attribute,我们可以和其他
attribute 一样使用 \texttt{v-bind}
将它们和动态的字符串绑定。但是,在处理比较复杂的绑定时,通过拼接生成字符串是麻烦且易出错的。因此,Vue
专门为 \texttt{class} 和 \texttt{style} 的 \texttt{v-bind}
用法提供了特殊的功能增强。除了字符串外,表达式的值也可以是对象或数组。
\switchcolumn[0]*%%%%%%%
\subsection{Binding HTML Classes}
\switchcolumn
\subsection{绑定 HTML class}
\switchcolumn[0]*%%%%%%%
\subsubsection{Binding to Objects}
\switchcolumn
\subsubsection{绑定对象}
\switchcolumn[0]*%%%%%%%
We can pass an object to \texttt{:class} (short for
\texttt{v-bind:class}) to dynamically toggle classes:
\switchcolumn
我们可以给 \texttt{:class} (\texttt{v-bind:class} 的缩写)
传递一个对象来动态切换 class:
\switchcolumn[0]*%%%%%%%
\begin{codeHtml}
<div :class="{ active: isActive }"></div>
\end{codeHtml}
\switchcolumn
\begin{codeHtml}
<div :class="{ active: isActive }"></div>
\end{codeHtml}

\switchcolumn[0]*%%%%%%%
The above syntax means the presence of the \texttt{active} class will be
determined by the
\href{https://developer.mozilla.org/en-US/docs/Glossary/Truthy}{truthiness}
of the data property \texttt{isActive}.
\switchcolumn
上面的语法表示 \texttt{active} 是否存在取决于数据属性 \texttt{isActive}
的\href{https://developer.mozilla.org/en-US/docs/Glossary/Truthy}{真假值}。
\switchcolumn[0]*%%%%%%%
You can have multiple classes toggled by having more fields in the
object. In addition, the \texttt{:class} directive can also co-exist
with the plain \texttt{class} attribute. So given the following state:
\switchcolumn
你可以在对象中写多个字段来操作多个 class。此外,\texttt{:class}
指令也可以和一般的 \texttt{class} attribute
共存。举例来说,下面这样的状态:
\switchcolumn[0]*%%%%%%%
\begin{codeJs}
const isActive = ref(true)
const hasError = ref(false)
\end{codeJs}
\switchcolumn
\begin{codeJs}
const isActive = ref(true)
const hasError = ref(false)
\end{codeJs}
\switchcolumn[0]*%%%%%%%
And the following template:
\switchcolumn
配合以下模板:
\switchcolumn[0]*%%%%%%%
\begin{codeHtml}
<div
  class="static"
  :class="{ active: isActive, 'text-danger': hasError }"
></div>
\end{codeHtml}
\switchcolumn
\begin{codeHtml}
<div
  class="static"
  :class="{ active: isActive, 'text-danger': hasError }"
></div>
\end{codeHtml}
\switchcolumn[0]*%%%%%%%
It will render:
\switchcolumn
渲染的结果会是:
\switchcolumn[0]*%%%%%%%
\begin{codeHtml}
<div class="static active"></div>
\end{codeHtml}
\switchcolumn
\begin{codeHtml}
<div class="static active"></div>
\end{codeHtml}
\switchcolumn[0]*%%%%%%%
When \texttt{isActive} or \texttt{hasError} changes, the class list will
be updated accordingly. For example, if \texttt{hasError} becomes
\texttt{true}, the class list will become
\texttt{"static\ active\ text-danger"}.
\switchcolumn
当 \texttt{isActive} 或者 \texttt{hasError} 改变时,class
列表会随之更新。举例来说,如果 \texttt{hasError} 变为
\texttt{true},class 列表也会变成
\texttt{"static\ active\ text-danger"}。
\switchcolumn[0]*%%%%%%%
The bound object doesn't have to be inline:
\switchcolumn
绑定的对象并不一定需要写成内联字面量的形式,也可以直接绑定一个对象:

\switchcolumn[0]*%%%%%%%
\begin{codeJs}
const classObject = reactive({
  active: true,
  'text-danger': false
})
\end{codeJs}
\switchcolumn
\begin{codeJs}
const classObject = reactive({
  active: true,
  'text-danger': false
})
\end{codeJs}
\switchcolumn[0]*%%%%%%%
\begin{codeHtml}
<div :class="classObject"></div>
\end{codeHtml}
\switchcolumn
\begin{codeHtml}
<div :class="classObject"></div>
\end{codeHtml}
\switchcolumn[0]*%%%%%%%
This will render:
\switchcolumn
这将渲染:
\switchcolumn[0]*%%%%%%%
\begin{codeHtml}
<div class="active"></div>
\end{codeHtml}
\switchcolumn
\begin{codeHtml}
<div class="active"></div>
\end{codeHtml}
\switchcolumn[0]*%%%%%%%
We can also bind to a
\href{https://vuejs.org/guide/essentials/computed.html}{computed
property} that returns an object. This is a common and powerful pattern:
\switchcolumn
我们也可以绑定一个返回对象的\href{https://cn.vuejs.org/guide/essentials/computed.html}{计算属性}。这是一个常见且很有用的技巧:
\switchcolumn[0]*%%%%%%%
\begin{codeJs}
const isActive = ref(true)
const error = ref(null)

const classObject = computed(() => ({
    active: isActive.value && !error.value,
    'text-danger': error.value && error.value.type === 'fatal'
}))
\end{codeJs}
\switchcolumn
\begin{codeJs}
const isActive = ref(true)
const error = ref(null)

const classObject = computed(() => ({
    active: isActive.value && !error.value,
    'text-danger': error.value && error.value.type === 'fatal'
}))
\end{codeJs}

\switchcolumn[0]*%%%%%%%
\begin{codeHtml}
<div :class="classObject"></div>
\end{codeHtml}
\switchcolumn
\begin{codeHtml}
<div :class="classObject"></div>
\end{codeHtml}
\end{paracol}
 
\columnratio{0.55}
\begin{paracol}{2}
\switchcolumn[0]*%%%%%%%
\subsubsection{Binding to Arrays}
\switchcolumn
\subsubsection{绑定数组}
\switchcolumn[0]*%%%%%%%
We can bind \texttt{:class} to an array to apply a list of classes:
\switchcolumn
我们可以给 \texttt{:class} 绑定一个数组来渲染多个 CSS class:
\switchcolumn[0]*%%%%%%%
\begin{codeJs}
const activeClass = ref('active')
const errorClass = ref('text-danger')
\end{codeJs}
\switchcolumn
\begin{codeJs}
const activeClass = ref('active')
const errorClass = ref('text-danger')
\end{codeJs}

\switchcolumn[0]*%%%%%%%
\begin{codeHtml}
<div :class="[activeClass, errorClass]"></div>
\end{codeHtml}
\switchcolumn
\begin{codeHtml}
<div :class="[activeClass, errorClass]"></div>
\end{codeHtml}
\switchcolumn[0]*%%%%%%%
Which will render:
\switchcolumn
渲染的结果是:
\switchcolumn[0]*%%%%%%%
\begin{codeHtml}
<div class="active text-danger"></div>
\end{codeHtml}
\switchcolumn
\begin{codeHtml}
<div class="active text-danger"></div>
\end{codeHtml}
\switchcolumn[0]*%%%%%%%
If you would like to also toggle a class in the list conditionally, you
can do it with a ternary expression:
\switchcolumn
如果你也想在数组中有条件地渲染某个 class,你可以使用三元表达式:
\switchcolumn[0]*%%%%%%%
\begin{codeHtml}
<div :class="[isActive ? activeClass : '', errorClass]"></div>
\end{codeHtml}
\switchcolumn
\begin{codeHtml}
<div :class="[isActive ? activeClass : '', errorClass]"></div>
\end{codeHtml}
\switchcolumn[0]*%%%%%%%
This will always apply \texttt{errorClass}, but \texttt{activeClass}
will only be applied when \texttt{isActive} is truthy.
\switchcolumn
\texttt{errorClass} 会一直存在,但 \texttt{activeClass} 只会在
\texttt{isActive} 为真时才存在。
\switchcolumn[0]*%%%%%%%
However, this can be a bit verbose if you have multiple conditional
classes. That's why it's also possible to use the object syntax inside
the array syntax:
\switchcolumn
然而,这可能在有多个依赖条件的 class
时会有些冗长。因此也可以在数组中嵌套对象:
\switchcolumn[0]*%%%%%%%
\begin{codeHtml}
<div :class="[{ active: isActive }, errorClass]"></div>
\end{codeHtml}
\switchcolumn
\begin{codeHtml}
<div :class="[{ active: isActive }, errorClass]"></div>
\end{codeHtml}
\switchcolumn[0]*%%%%%%%
\subsubsection{With Components}
\switchcolumn
\subsubsection{在组件上使用}
\switchcolumn[0]*%%%%%%%
\begin{quote}
This section assumes knowledge of
\href{https://vuejs.org/guide/essentials/component-basics.html}{Components}.
Feel free to skip it and come back later.
\end{quote}
\switchcolumn
\begin{quote}
本节假设你已经有
\href{https://cn.vuejs.org/guide/essentials/component-basics.html}{Vue
组件}的知识基础。如果没有,你也可以暂时跳过,以后再阅读。
\end{quote}
\switchcolumn[0]*%%%%%%%
When you use the \texttt{class} attribute on a component with a single
root element, those classes will be added to the component's root
element and merged with any existing class already on it.
\switchcolumn
对于只有一个根元素的组件,当你使用了 \texttt{class} attribute 时,这些
class 会被添加到根元素上并与该元素上已有的 class 合并。
\switchcolumn[0]*%%%%%%%
For example, if we have a component named \texttt{MyComponent} with the
following template:
\switchcolumn
举例来说,如果你声明了一个组件名叫 \texttt{MyComponent},模板如下:
\switchcolumn[0]*%%%%%%%
\begin{codeHtml}
<!-- child component template -->
<p class="foo bar">Hi!</p>
\end{codeHtml}
\switchcolumn
\begin{codeHtml}
<!-- 子组件模板 -->
<p class="foo bar">Hi!</p>
\end{codeHtml}
\switchcolumn[0]*%%%%%%%
Then add some classes when using it:
\switchcolumn
在使用时添加一些 class:
\switchcolumn[0]*%%%%%%%
\begin{codeHtml}
<!-- when using the component -->
<MyComponent class="baz boo" />
\end{codeHtml}
\switchcolumn
\begin{codeHtml}
<!-- 在使用组件时 -->
<MyComponent class="baz boo" />
\end{codeHtml}
\switchcolumn[0]*%%%%%%%
The rendered HTML will be:
\switchcolumn
渲染出的 HTML 为:

\switchcolumn[0]*%%%%%%%
\begin{codeHtml}
<p class="foo bar baz boo">Hi!</p>
\end{codeHtml}
\switchcolumn
\begin{codeHtml}
<p class="foo bar baz boo">Hi!</p>
\end{codeHtml}
\switchcolumn[0]*%%%%%%%
The same is true for class bindings:
\switchcolumn
Class 的绑定也是同样的:
\switchcolumn[0]*%%%%%%%
\begin{codeHtml}
<MyComponent :class="{ active: isActive }" />
\end{codeHtml}
\switchcolumn
\begin{codeHtml}
<MyComponent :class="{ active: isActive }" />
\end{codeHtml}
\switchcolumn[0]*%%%%%%%
When \texttt{isActive} is truthy, the rendered HTML will be:
\switchcolumn
当 \texttt{isActive} 为真时,被渲染的 HTML 会是:
\switchcolumn[0]*%%%%%%%
\begin{codeHtml}
<p class="foo bar active">Hi!</p>
\end{codeHtml}
\switchcolumn
\begin{codeHtml}
<p class="foo bar active">Hi!</p>
\end{codeHtml}
\switchcolumn[0]*%%%%%%%
If your component has multiple root elements, you would need to define
which element will receive this class. You can do this using the
\texttt{\$attrs} component property:
\switchcolumn
如果你的组件有多个根元素,你将需要指定哪个根元素来接收这个
class。你可以通过组件的 \texttt{\$attrs} 属性来实现指定:
\switchcolumn[0]*%%%%%%%
\begin{codeHtml}
<!-- MyComponent template using $attrs -->
<p :class="$attrs.class">Hi!</p>
<span>This is a child component</span>
\end{codeHtml}
\switchcolumn
\begin{codeHtml}
<!-- MyComponent 模板使用 $attrs 时 -->
<p :class="$attrs.class">Hi!</p>
<span>This is a child component</span>
\end{codeHtml}
\switchcolumn[0]*%%%%%%%
\begin{codeHtml}
<MyComponent class="baz" />
\end{codeHtml}
\switchcolumn
\begin{codeHtml}
<MyComponent class="baz" />
\end{codeHtml}
\switchcolumn[0]*%%%%%%%
Will render:
\switchcolumn
这将被渲染为:
\switchcolumn[0]*%%%%%%%
\begin{codeHtml}
<p class="baz">Hi!</p>
<span>This is a child component</span>
\end{codeHtml}
\switchcolumn
\begin{codeHtml}
<p class="baz">Hi!</p>
<span>This is a child component</span>
\end{codeHtml}
\switchcolumn[0]*%%%%%%%
You can learn more about component attribute inheritance in
\href{https://vuejs.org/guide/components/attrs.html}{Fallthrough
Attributes} section.
\switchcolumn
你可以在\href{https://cn.vuejs.org/guide/components/attrs.html}{透传
Attribute} 一章中了解更多组件的 attribute 继承的细节。
\end{paracol}

\columnratio{0.55}
\begin{paracol}{2}

\switchcolumn[0]*%%%%%%%
\subsection{Binding Inline Styles}
\switchcolumn
\subsection{绑定内联样式}
\switchcolumn[0]*%%%%%%%
\subsubsection{Binding to Objects}
\switchcolumn
\subsubsection{绑定对象}
\switchcolumn[0]*%%%%%%%
\texttt{:style} supports binding to JavaScript object values - it
corresponds to an
\href{https://developer.mozilla.org/en-US/docs/Web/API/HTMLElement/style}{HTML
element's \texttt{style} property}:
\switchcolumn
\texttt{:style} 支持绑定 JavaScript 对象值,对应的是
\href{https://developer.mozilla.org/en-US/docs/Web/API/HTMLElement/style}{HTML
元素的 \texttt{style} 属性}:
\switchcolumn[0]*%%%%%%%
\begin{codeJs}
const activeColor = ref('red')
const fontSize = ref(30)
\end{codeJs}
\switchcolumn
\begin{codeJs}
const activeColor = ref('red')
const fontSize = ref(30)
\end{codeJs}
\switchcolumn[0]*%%%%%%%
\begin{codeHtml}
<div :style="{ color: activeColor, fontSize: fontSize + 'px' }"></div>
\end{codeHtml}
\switchcolumn
\begin{codeHtml}
<div :style="{ color: activeColor, fontSize: fontSize + 'px' }"></div>
\end{codeHtml}
\switchcolumn[0]*%%%%%%%
Although camelCase keys are recommended, \texttt{:style} also supports
kebab-cased CSS property keys (corresponds to how they are used in
actual CSS) - for example:
\switchcolumn
尽管推荐使用 camelCase,但 \texttt{:style} 也支持 kebab-cased 形式的 CSS
属性 key (对应其 CSS 中的实际名称),例如:
\switchcolumn[0]*%%%%%%%
\begin{codeHtml}
<div :style="{ 'font-size': fontSize + 'px' }"></div>
\end{codeHtml}
\switchcolumn
\begin{codeHtml}
<div :style="{ 'font-size': fontSize + 'px' }"></div>
\end{codeHtml}
\switchcolumn[0]*%%%%%%%
It is often a good idea to bind to a style object directly so that the
template is cleaner:
\switchcolumn
直接绑定一个样式对象通常是一个好主意,这样可以使模板更加简洁:
\switchcolumn[0]*%%%%%%%
\begin{codeJs}
const styleObject = reactive({
    color: 'red',
    fontSize: '13px'
})
\end{codeJs}
\switchcolumn
\begin{codeJs}
const styleObject = reactive({
    color: 'red',
    fontSize: '13px'
})
\end{codeJs}
\switchcolumn[0]*%%%%%%%
\begin{codeHtml}
<div :style="styleObject"></div>
\end{codeHtml}
\switchcolumn
\begin{codeHtml}
<div :style="styleObject"></div>
\end{codeHtml}
\switchcolumn[0]*%%%%%%%
Again, object style binding is often used in conjunction with computed
properties that return objects.
\switchcolumn
同样的,如果样式对象需要更复杂的逻辑,也可以使用返回样式对象的计算属性。


\switchcolumn[0]*%%%%%%%
\subsubsection{Binding to Arrays}
\switchcolumn
\subsubsection{绑定数组}
\switchcolumn[0]*%%%%%%%
We can bind \texttt{:style} to an array of multiple style objects. These
objects will be merged and applied to the same element:
\switchcolumn
我们还可以给 \texttt{:style}
绑定一个包含多个样式对象的数组。这些对象会被合并后渲染到同一元素上:
\switchcolumn[0]*%%%%%%%
\begin{codeHtml}
<div :style="[baseStyles, overridingStyles]"></div>
\end{codeHtml}
\switchcolumn
\begin{codeHtml}
<div :style="[baseStyles, overridingStyles]"></div>
\end{codeHtml}
\switchcolumn[0]*%%%%%%%
\subsubsection{Auto-prefixing}
\switchcolumn
\subsubsection{自动前缀}
\switchcolumn[0]*%%%%%%%
When you use a CSS property that requires a
\href{https://developer.mozilla.org/en-US/docs/Glossary/Vendor_Prefix}{vendor
prefix} in \texttt{:style}, Vue will automatically add the appropriate
prefix. Vue does this by checking at runtime to see which style
properties are supported in the current browser. If the browser doesn't
support a particular property then various prefixed variants will be
tested to try to find one that is supported.
\switchcolumn
当你在 \texttt{:style}
中使用了需要\href{https://developer.mozilla.org/en-US/docs/Glossary/Vendor_Prefix}{浏览器特殊前缀}的
CSS 属性时,Vue 会自动为他们加上相应的前缀。Vue
是在运行时检查该属性是否支持在当前浏览器中使用。如果浏览器不支持某个属性,那么将尝试加上各个浏览器特殊前缀,以找到哪一个是被支持的。
\switchcolumn[0]*%%%%%%%
\subsubsection{Multiple Values}
\switchcolumn
\subsubsection{样式多值}
\switchcolumn[0]*%%%%%%%
You can provide an array of multiple (prefixed) values to a style
property, for example:
\switchcolumn
你可以对一个样式属性提供多个 (不同前缀的) 值,举例来说:
\switchcolumn[0]*%%%%%%%
\begin{codeHtml}
<div :style="{ display: ['-webkit-box', '-ms-flexbox', 'flex'] }"></div>
\end{codeHtml}
\switchcolumn
\begin{codeHtml}
<div :style="{ display: ['-webkit-box', '-ms-flexbox', 'flex'] }"></div>
\end{codeHtml}
\switchcolumn[0]*%%%%%%%
This will only render the last value in the array which the browser
supports. In this example, it will render \texttt{display:\ flex} for
browsers that support the unprefixed version of flexbox.
\switchcolumn
数组仅会渲染浏览器支持的最后一个值。在这个示例中,在支持不需要特别前缀的浏览器中都会渲染为
\texttt{display:\ flex}。
\end{paracol}



  




\end{document}
\begin{codeHtml}

\end{codeHtml}  

\begin{codeJs}
\end{codeJs}

\begin{codeVue}

\end{codeVue}
%%%%%%%
\begin{vueQuoteWarn}
\end{vueQuoteWarn}

\begin{vueQuote}
\end{vueQuote}



\end{document}]%%%%%%%]*%%%%%%%]*%%%%%%%]*%%%%%%%]*%%%%%%%]*%%%%%%%]*%%%%%%%]*%%%%%%%]*%%%%%%%]*%%%%%%%
\switchcolumn[0]*%%%%%%%
\begin{vueQuote}{}
\end{vueQuote} 
\switchcolumn
\begin{vueQuote}{}
\end{vueQuote} 


\switchcolumn[0]*%%%%%%%
\begin{codeHtml}

\end{codeHtml}  
\switchcolumn
\begin{codeHtml}

\end{codeHtml}  