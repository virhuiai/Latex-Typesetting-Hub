% vue3官方文档.tex
\PassOptionsToPackage{no-math}{fontspec}%禁用了使用fontspec宏包中的数学字体功能。
\PassOptionsToPackage{AutoFakeBold=true,AutoFakeSlant=true}{xeCJK}%让xeCJK宏包自动产生伪粗体和伪斜体效果。

\documentclass{book}
\usepackage[heading=true
,scheme=chinese%中文方案
,fontset=none%不使用默认的字体设置
,space=auto%自动调整中英文间距
]{ctex}
\setCJKmainfont{FangZhengShuSong-GBK-1.ttf}[Path=/Users/virhuiai/hlProjects/Latex-Typesetting-Hub/font/方正/]%设置文本的中文有衬线字体
\setCJKsansfont{FangZhengHeiTi-GBK-1.ttf}[Path=/Users/virhuiai/hlProjects/Latex-Typesetting-Hub/font/方正/]%设置文本的中文无衬线字体为
\setCJKmonofont{FangZhengFangSong-GBK-1.ttf}[Path=/Users/virhuiai/hlProjects/Latex-Typesetting-Hub/font/方正/] %设置文本的中文等宽字体 

\setCJKfamilyfont{fontFangSong}{FangZhengFangSong-GBK-1.ttf}[Path=/Users/virhuiai/hlProjects/Latex-Typesetting-Hub/font/方正/]
\setCJKfamilyfont{fontKai}{FangZhengKaiTi-GBK-1.ttf}[Path=/Users/virhuiai/hlProjects/Latex-Typesetting-Hub/font/方正/]
\newcommand\fontKai{\CJKfamily{fontKai}}

% 支持音标的字体
\newfontfamily\fontGentiumPlus{GentiumPlus}[Path=/Users/virhuiai/hlProjects/Latex-Typesetting-Hub/font/免费商用英文/支持音标-GentiumPlus-6.200/,
Extension=.ttf,
UprightFont=*-Regular ,
BoldFont=*-Bold ,
ItalicFont=*-Italic,
BoldItalicFont = *-BoldItalic
]
\newfontfamily\fontGentiumBookPlus{GentiumBookPlus}[Path=/Users/virhuiai/hlProjects/Latex-Typesetting-Hub/font/免费商用英文/支持音标-GentiumPlus-6.200/,
Extension=.ttf,
UprightFont=*-Regular ,
BoldFont=*-Bold ,
ItalicFont=*-Italic,
BoldItalicFont = *-BoldItalic
]


\usepackage[a3paper,landscape]{geometry}
\usepackage{paracol}
\usepackage[all]{tcolorbox}
\usepackage{parskip}
\usepackage{calc,pifont}\newcounter{带圈文字}\newcommand\带圈文字[1]{\protect\setcounter{带圈文字}{171+#1}\protect\ding{\value{带圈文字}}}
% \带圈文字{1}
% \带圈文字{2}\end{document}
\usepackage{amssymb}%\checkmark
%如果你想在LaTeX中输入"✔"符号,你可以使用amssymb宏包提供的\checkmark命令。
% 在常见的Unicode字符集中,"✔"的编码为U+2714。这个字符可以在大多数现代字体和字符集中正确显示。
\parindent=0pt
\begin{document}

\newminted[codeJs]{js}{frame=single,label={js}}
\newminted[codeHtml]{html}{frame=single,label={html}}
\newminted[codeVue]{html}{frame=single,label={vue}}
\tcbset{
    csh shell/.style={
    skin=bicolor,
    colback=black,colupper=green,colframe=yellow!75!black,
    fontupper=\tt,
    before upper=\textcolor{red}{\small\ttfamily\bfseries virhuiai \%>~}
    } 
}
\newtcolorbox{codeShell}{csh shell} 
\newtcblisting{codeShellMul}{colback=black,colupper=white,colframe=yellow!75!black, listing only,listing options={style=tcblatex,language=sh},
every listing line={\textcolor{red}{\small\ttfamily\bfseries virhuiai \$> }}}

\tcolorboxenvironment{quote}{blanker, borderline west={1mm}{0pt}{gray}}
 
% \tcbset{
%     csh console/.style={
%     skin=bicolor,
%     colback=black,colupper=green,colframe=yellow!75!black,
%     fontupper=\tt,
%     % before upper=\textcolor{red}{\small\ttfamily\bfseries virhuiai \%>~}
%     } 
% }
% \newtcolorbox{consoleCode}{csh console}
% 需要指定字体 \setCJKfamilyfont{fontFangSong}{FangZhengFangSong-GBK-1.ttf}[Path=/Users/virhuiai/hlProjects/Latex-Typesetting-Hub/font/方正/]
\newminted[codeConsole]{console}{frame=single,fontfamily=fontFangSong} 
%todo 符号 

\definecolor{vueQuoteBg}{RGB}{249, 249, 249}
\definecolor{vueQuoteFrame}{RGB}{101, 181, 135}
\definecolor{vueQuoteFrameWarn}{RGB}{247, 197, 72}


% \begin{codeVue}
% \end{codeVue}


\newtcolorbox{vueQuote}[2][]{colback=vueQuoteBg, colframe=vueQuoteFrame,fonttitle=\bfseries, enhanced,coltitle=black,
% attach boxed title to top center={yshift=-2mm},
attach title to upper={\par},
title={%\makebox[0pt]{\textcircled{i}\quad}
#2},#1} 
\newtcolorbox{vueQuoteWarn}[2][]{colback=vueQuoteBg, colframe=vueQuoteFrameWarn,fonttitle=\bfseries, enhanced,coltitle=black,
% attach boxed title to top center={yshift=-2mm},
attach title to upper={\par},
title={%\makebox[0pt]{\textcircled{i}\quad}
#2},#1} 
 



% \chapter{Getting Started\hfill 开始}
% 
% [Introduction](https://vuejs.org/guide/introduction)[Quick Start](https://vuejs.org/guide/quick-start)
% [简介](https://cn.vuejs.org/guide/introduction.html)[快速上手](https://cn.vuejs.org/guide/quick-start.html)



\columnratio{0.55}
\begin{paracol}{2}
% \chapter{Getting Started}
% \switchcolumn
% \chapter{开始}
\switchcolumn[0]*%%%%%%%
\section{introduction}
\switchcolumn
\section{介绍}
\switchcolumn[0]*%%%%%%%
\begin{vueQuote}{You are reading the documentation for Vue 3!}
\begin{itemize}
    \item
      Vue 2 support will end on Dec 31, 2023. Learn more about
      \href{https://v2.vuejs.org/lts/}{Vue 2 Extended LTS}.
    \item
      Vue 2 documentation has been moved to
      \href{https://v2.vuejs.org/}{v2.vuejs.org}.
    \item
      Upgrading from Vue 2? Check out the
      \href{https://v3-migration.vuejs.org/}{Migration Guide}.
    \end{itemize}
\end{vueQuote}
\switchcolumn
\begin{vueQuote}{你正在阅读的是 Vue 3 的文档!}
\begin{itemize}
\item
    Vue 2 将于 2023 年 12 月 31 日停止维护。详见
    \href{https://v2.vuejs.org/lts/}{Vue 2 延长 LTS}。
\item
    Vue 2 中文文档已迁移至
    \href{https://v2.cn.vuejs.org/}{v2.cn.vuejs.org}。
\item
    想从 Vue 2
    升级?请参考\href{https://v3-migration.vuejs.org/}{迁移指南}。
\end{itemize}
\end{vueQuote}
\switchcolumn[0]*%%%%%%%
\subsection{What is Vue?}
\switchcolumn
\subsection{什么是 Vue?}
\switchcolumn[0]*%%%%%%%
Vue (pronounced {\fontGentiumPlus /vjuː/}, like \textbf{view}) is a JavaScript framework
for building user interfaces. It builds on top of standard HTML, CSS,
and JavaScript and provides a declarative and component-based
programming model that helps you efficiently develop user interfaces, be
they simple or complex.
\switchcolumn
Vue (发音为 {\fontGentiumPlus /vjuː/},类似 \textbf{view}) 是用于构建用户界面的
JavaScript 框架。它基于标准 HTML、CSS 和 JavaScript
构建,并提供了一套声明式的、组件化的编程模型,助你高效地开发用户界面。无论是简单还是复杂的界面,Vue
都可以胜任。
\switchcolumn[0]*%%%%%%%
Here is a minimal example:
\switchcolumn
下面是一个最基本的示例:
\switchcolumn[0]*%%%%%%%
\begin{codeJs*}{label={js}}
import { createApp, ref } from 'vue'

createApp({
  setup() {
    return {
      count: ref(0)
    }
  }
}).mount('#app')
\end{codeJs*}
\switchcolumn
\begin{codeJs*}{label={js:UMD浏览器引用JS方式}}
const {createApp,ref} = Vue;

createApp({
    setup() {
    return {
        count: ref(0)
    }
    }
}).mount('#app')
\end{codeJs*}
\switchcolumn[0]*%%%%%%%
\begin{codeHtml*}{label={template}}
<div id="app">
    <button @click="count++">
        Count is: {{ count }}
    </button>
</div>
\end{codeHtml*}
\switchcolumn
\begin{codeHtml*}{label={template}}
<div id="app">
    <button @click="count++">
        Count is: {{ count }}
    </button>
</div>
\end{codeHtml*}
\switchcolumn[0]*%%%%%%%
The above example demonstrates the two core features of Vue:
\switchcolumn
上面的示例展示了 Vue 的两个核心功能:
\end{paracol}

\columnratio{0.55}
\begin{itemize}
\begin{paracol}{2}
\item
\textbf{Declarative Rendering}: Vue extends standard HTML with a
template syntax that allows us to declaratively describe HTML output
based on JavaScript state.
\switchcolumn
\item
\textbf{声明式渲染}:Vue 基于标准 HTML
拓展了一套模板语法,使得我们可以声明式地描述最终输出的 HTML 和
JavaScript 状态之间的关系。
\switchcolumn[0]*%%%%%%%
\item
\textbf{Reactivity}: Vue automatically tracks JavaScript state changes
and efficiently updates the DOM when changes happen.
\switchcolumn
\item
\textbf{响应性}:Vue 会自动跟踪 JavaScript
状态并在其发生变化时响应式地更新 DOM。
\end{paracol}
\end{itemize}


\columnratio{0.55}
\begin{paracol}{2}
\switchcolumn[0]*%%%%%%%
You may already have questions - don't worry. We will cover every little
detail in the rest of the documentation. For now, please read along so
you can have a high-level understanding of what Vue offers.
\switchcolumn
你可能已经有了些疑问------先别急,在后续的文档中我们会详细介绍每一个细节。现在,请继续看下去,以确保你对
Vue 作为一个框架到底提供了什么有一个宏观的了解。
\switchcolumn[0]*%%%%%%%
\begin{vueQuote}
{Prerequisites}
The rest of the documentation assumes basic familiarity with HTML, CSS,
and JavaScript. If you are totally new to frontend development, it might
not be the best idea to jump right into a framework as your first step -
grasp the basics and then come back! You can check your knowledge level
with
\href{https://developer.mozilla.org/en-US/docs/Web/JavaScript/A_re-introduction_to_JavaScript}{this
JavaScript overview}. Prior experience with other frameworks helps, but
is not required.
\end{vueQuote}
\switchcolumn

\begin{vueQuote}{预备知识}
文档接下来的内容会假设你对 HTML、CSS 和 JavaScript
已经基本熟悉。如果你对前端开发完全陌生,最好不要直接从一个框架开始进行入门学习------最好是掌握了基础知识再回到这里。你可以通过这篇
\href{https://developer.mozilla.org/zh-CN/docs/Web/JavaScript/A_re-introduction_to_JavaScript}{JavaScript
概述}来检验你的 JavaScript
知识水平。如果之前有其他框架的经验会很有帮助,但也不是必须的。
\end{vueQuote}
\switchcolumn[0]*%%%%%%%
\subsection{The Progressive Framework}
\switchcolumn
\subsection{渐进式框架}
\switchcolumn[0]*%%%%%%%
Vue is a framework and ecosystem that covers most of the common features
needed in frontend development. But the web is extremely diverse - the
things we build on the web may vary drastically in form and scale. With
that in mind, Vue is designed to be flexible and incrementally
adoptable. Depending on your use case, Vue can be used in different
ways:
\switchcolumn
Vue 是一个框架,也是一个生态。其功能覆盖了大部分前端开发常见的需求。但
Web 世界是十分多样化的,不同的开发者在 Web
上构建的东西可能在形式和规模上会有很大的不同。考虑到这一点,Vue
的设计非常注重灵活性和``可以被逐步集成''这个特点。根据你的需求场景,你可以用不同的方式使用
Vue:
\switchcolumn[0]*%%%%%%%
\begin{itemize}
\item
    Enhancing static HTML without a build step
\item
    Embedding as Web Components on any page
\item
    Single-Page Application (SPA)
\item
    Fullstack / Server-Side Rendering (SSR)
\item
    Jamstack / Static Site Generation (SSG)
\item
    Targeting desktop, mobile, WebGL, and even the terminal
\end{itemize}
\switchcolumn
\begin{itemize}
\item
    无需构建步骤,渐进式增强静态的 HTML
\item
    在任何页面中作为 Web Components 嵌入
\item
    单页应用 (SPA)
\item
    全栈 / 服务端渲染 (SSR)
\item
    Jamstack / 静态站点生成 (SSG)
\item
    开发桌面端、移动端、WebGL,甚至是命令行终端中的界面
\end{itemize}
\switchcolumn[0]*%%%%%%%
If you find these concepts intimidating, don't worry! The tutorial and
guide only require basic HTML and JavaScript knowledge, and you should
be able to follow along without being an expert in any of these.
\switchcolumn
如果你是初学者,可能会觉得这些概念有些复杂。别担心!理解教程和指南的内容只需要具备基础的
HTML 和 JavaScript 知识。即使你不是这些方面的专家,也能够跟得上。
\switchcolumn[0]*%%%%%%%
If you are an experienced developer interested in how to best integrate
Vue into your stack, or you are curious about what these terms mean, we
discuss them in more detail in
\href{https://vuejs.org/guide/extras/ways-of-using-vue}{Ways of Using
Vue}.
\switchcolumn
如果你是有经验的开发者,希望了解如何以最合适的方式在项目中引入
Vue,或者是对上述的这些概念感到好奇,我们在\href{https://cn.vuejs.org/guide/extras/ways-of-using-vue.html}{使用
Vue 的多种方式}中讨论了有关它们的更多细节。
\switchcolumn[0]*%%%%%%%
Despite the flexibility, the core knowledge about how Vue works is
shared across all these use cases. Even if you are just a beginner now,
the knowledge gained along the way will stay useful as you grow to
tackle more ambitious goals in the future. If you are a veteran, you can
pick the optimal way to leverage Vue based on the problems you are
trying to solve, while retaining the same productivity. This is why we
call Vue "The Progressive Framework": it's a framework that can grow
with you and adapt to your needs.
\switchcolumn
无论再怎么灵活,Vue
的核心知识在所有这些用例中都是通用的。即使你现在只是一个初学者,随着你的不断成长,到未来有能力实现更复杂的项目时,这一路上获得的知识依然会适用。如果你已经是一个老手,你可以根据实际场景来选择使用
Vue 的最佳方式,在各种场景下都可以保持同样的开发效率。这就是为什么我们将
Vue 称为``渐进式框架'':它是一个可以与你共同成长、适应你不同需求的框架。
\switchcolumn[0]*%%%%%%%
\subsection{Single-File Components}
\switchcolumn
\subsection{单文件组件}
\switchcolumn[0]*%%%%%%%
In most build-tool-enabled Vue projects, we author Vue components using
an HTML-like file format called \textbf{Single-File Component} (also
known as \texttt{*.vue} files, abbreviated as \textbf{SFC}). A Vue SFC,
as the name suggests, encapsulates the component's logic (JavaScript),
template (HTML), and styles (CSS) in a single file. Here's the previous
example, written in SFC format:
\switchcolumn
在大多数启用了构建工具的 Vue 项目中,我们可以使用一种类似 HTML
格式的文件来书写 Vue 组件,它被称为\textbf{单文件组件} (也被称为
\texttt{*.vue} 文件,英文 Single-File Components,缩写为
\textbf{SFC})。顾名思义,Vue 的单文件组件会将一个组件的逻辑
(JavaScript),模板 (HTML) 和样式 (CSS)
封装在同一个文件里。下面我们将用单文件组件的格式重写上面的计数器示例:
\switchcolumn[0]*%%%%%%%
\begin{codeVue}
<script setup>
import { ref } from 'vue'
const count = ref(0)
</script>

<template>
    <button @click="count++">Count is: {{ count }}</button>
</template>

<style scoped>
button {
    font-weight: bold;
}
</style>
\end{codeVue}
\switchcolumn
\begin{codeVue}
<script setup>
import { ref } from 'vue'
const count = ref(0)
</script>

<template>
    <button @click="count++">Count is: {{ count }}</button>
</template>

<style scoped>
button {
    font-weight: bold;
}
</style>
\end{codeVue}
\switchcolumn[0]*%%%%%%%
SFC is a defining feature of Vue and is the recommended way to author
Vue components \textbf{if} your use case warrants a build setup. You can
learn more about the \href{https://vuejs.org/guide/scaling-up/sfc}{how
and why of SFC} in its dedicated section - but for now, just know that
Vue will handle all the build tools setup for you.
\switchcolumn
单文件组件是 Vue
的标志性功能。如果你的用例需要进行构建,我们推荐用它来编写 Vue
组件。你可以在后续相关章节里了解更多关于\href{https://cn.vuejs.org/guide/scaling-up/sfc.html}{单文件组件的用法及用途}。但你暂时只需要知道
Vue 会帮忙处理所有这些构建工具的配置就好。
\switchcolumn[0]*%%%%%%%
\subsection{API Styles}
\switchcolumn
\subsection{API 风格}
\switchcolumn[0]*%%%%%%%
Vue components can be authored in two different API styles:\textbf{Options API} and \textbf{Composition API}.
\switchcolumn
Vue 的组件可以按两种不同的风格书写:\textbf{选项式 API} 和\textbf{组合式
API}。
\switchcolumn[0]*%%%%%%%
\subsubsection{Options API}
\switchcolumn
\subsubsection{选项式 API (Options API)}
\switchcolumn[0]*%%%%%%%
With Options API, we define a component's logic using an object of
options such as \texttt{data}, \texttt{methods}, and \texttt{mounted}.
Properties defined by options are exposed on \texttt{this} inside
functions, which points to the component instance:
\switchcolumn
使用选项式 API,我们可以用包含多个选项的对象来描述组件的逻辑,例如
\texttt{data}、\texttt{methods} 和
\texttt{mounted}。选项所定义的属性都会暴露在函数内部的 \texttt{this}
上,它会指向当前的组件实例。
\switchcolumn[0]*%%%%%%%
\begin{codeVue}
    <script>
    export default {
      // Properties returned from data() become reactive state
      // and will be exposed on `this`.
      data() {
        return {
          count: 0
        }
      },
    
      // Methods are functions that mutate state and trigger updates.
      // They can be bound as event handlers in templates.
      methods: {
        increment() {
          this.count++
        }
      },
    
      // Lifecycle hooks are called at different stages
      // of a component's lifecycle.
      // This function will be called when the component is mounted.
      mounted() {
        console.log(`The initial count is ${this.count}.`)
      }
    }
    </script>
    
    <template>
      <button @click="increment">Count is: {{ count }}</button>
    </template>
\end{codeVue}
\switchcolumn
\begin{codeVue}
    <script>
    export default {
      // data() 返回的属性将会成为响应式的状态
      // 并且暴露在 `this` 上
      data() {
        return {
          count: 0
        }
      },
    
      // methods 是一些用来更改状态与触发更新的函数
      // 它们可以在模板中作为事件处理器绑定
      methods: {
        increment() {
          this.count++
        }
      },
    
      // 生命周期钩子会在组件生命周期的各个不同阶段被调用
      // 例如这个函数就会在组件挂载完成后被调用
      mounted() {
        console.log(`The initial count is ${this.count}.`)
      }
    }
    </script>
    
    <template>
      <button @click="increment">Count is: {{ count }}</button>
    </template>
\end{codeVue}
\switchcolumn[0]*%%%%%%%$
\subsubsection{Composition API}
\switchcolumn
\subsubsection{组合式 API (Composition API)}
\switchcolumn[0]*%%%%%%%
With Composition API, we define a component's logic using imported API
functions. In SFCs, Composition API is typically used with
\href{https://vuejs.org/api/sfc-script-setup}{``}. The \texttt{setup}
attribute is a hint that makes Vue perform compile-time transforms that
allow us to use Composition API with less boilerplate. For example,
imports and top-level variables / functions declared in
\texttt{\textless{}script\ setup\textgreater{}} are directly usable in
the template.
\switchcolumn
通过组合式 API,我们可以使用导入的 API
函数来描述组件逻辑。在单文件组件中,组合式 API 通常会与
\href{https://cn.vuejs.org/api/sfc-script-setup.html}{``} 搭配使用。这个
\texttt{setup} attribute 是一个标识,告诉 Vue
需要在编译时进行一些处理,让我们可以更简洁地使用组合式
API。比如,\texttt{\textless{}script\ setup\textgreater{}}
中的导入和顶层变量/函数都能够在模板中直接使用。
\switchcolumn[0]*%%%%%%%
Here is the same component, with the exact same template, but using
Composition API and \texttt{\textless{}script\ setup\textgreater{}}
instead:
\switchcolumn
下面是使用了组合式 API 与
\texttt{\textless{}script\ setup\textgreater{}}
改造后和上面的模板完全一样的组件:
\switchcolumn[0]*%%%%%%%
\begin{codeVue}
    <script setup>
    import { ref, onMounted } from 'vue'
    
    // reactive state
    const count = ref(0)
    
    // functions that mutate state and trigger updates
    function increment() {
      count.value++
    }
    
    // lifecycle hooks
    onMounted(() => {
      console.log(`The initial count is ${count.value}.`)
    })
    </script>
    
    <template>
      <button @click="increment">Count is: {{ count }}</button>
    </template>
\end{codeVue}    
\switchcolumn
\begin{codeVue}    
    <script setup>
    import { ref, onMounted } from 'vue'
    
    // 响应式状态
    const count = ref(0)
    
    // 用来修改状态、触发更新的函数
    function increment() {
      count.value++
    }
    
    // 生命周期钩子
    onMounted(() => {
      console.log(`The initial count is ${count.value}.`)
    })
    </script>
    
    <template>
      <button @click="increment">Count is: {{ count }}</button>
    </template>
\end{codeVue}    
\switchcolumn[0]*%%%%%%%
\subsubsection{Which to Choose?}
\switchcolumn
\subsubsection{该选哪一个?}
\switchcolumn[0]*%%%%%%%
Both API styles are fully capable of covering common use cases. They are
different interfaces powered by the exact same underlying system. In
fact, the Options API is implemented on top of the Composition API! The
fundamental concepts and knowledge about Vue are shared across the two
styles.
\switchcolumn
两种 API
风格都能够覆盖大部分的应用场景。它们只是同一个底层系统所提供的两套不同的接口。实际上,选项式
API 是在组合式 API 的基础上实现的!关于 Vue
的基础概念和知识在它们之间都是通用的。
\switchcolumn[0]*%%%%%%%
The Options API is centered around the concept of a "component instance"
(\texttt{this} as seen in the example), which typically aligns better
with a class-based mental model for users coming from OOP language
backgrounds. It is also more beginner-friendly by abstracting away the
reactivity details and enforcing code organization via option groups.
\switchcolumn
选项式 API 以``组件实例''的概念为中心 (即上述例子中的
\texttt{this}),对于有面向对象语言背景的用户来说,这通常与基于类的心智模型更为一致。同时,它将响应性相关的细节抽象出来,并强制按照选项来组织代码,从而对初学者而言更为友好。
\switchcolumn[0]*%%%%%%%
The Composition API is centered around declaring reactive state
variables directly in a function scope and composing state from multiple
functions together to handle complexity. It is more free-form and
requires an understanding of how reactivity works in Vue to be used
effectively. In return, its flexibility enables more powerful patterns
for organizing and reusing logic.
\switchcolumn
组合式 API
的核心思想是直接在函数作用域内定义响应式状态变量,并将从多个函数中得到的状态组合起来处理复杂问题。这种形式更加自由,也需要你对
Vue
的响应式系统有更深的理解才能高效使用。相应的,它的灵活性也使得组织和重用逻辑的模式变得更加强大。
\switchcolumn[0]*%%%%%%%
You can learn more about the comparison between the two styles and the
potential benefits of Composition API in the
\href{https://vuejs.org/guide/extras/composition-api-faq}{Composition
API FAQ}.
\switchcolumn
在\href{https://cn.vuejs.org/guide/extras/composition-api-faq.html}{组合式
API FAQ} 章节中,你可以了解更多关于这两种 API 风格的对比以及组合式 API
所带来的潜在收益。
\switchcolumn[0]*%%%%%%%
If you are new to Vue, here's our general recommendation:
\switchcolumn
如果你是使用 Vue 的新手,这里是我们的大致建议:
\switchcolumn[0]*%%%%%%%
\begin{itemize}
\item
For learning purposes, go with the style that looks easier to
understand to you. Again, most of the core concepts are shared between
the two styles. You can always pick up the other style later.
\item
For production use:

\begin{itemize}
\item
Go with Options API if you are not using build tools, or plan to use
Vue primarily in low-complexity scenarios, e.g. progressive
enhancement.
\item
Go with Composition API + Single-File Components if you plan to
build full applications with Vue.
\end{itemize}
\end{itemize}
\switchcolumn
\begin{itemize}
\item
在学习的过程中,推荐采用更易于自己理解的风格。再强调一下,大部分的核心概念在这两种风格之间都是通用的。熟悉了一种风格以后,你也能够很快地理解另一种风格。
\item
在生产项目中:

\begin{itemize}
\item
当你不需要使用构建工具,或者打算主要在低复杂度的场景中使用
Vue,例如渐进增强的应用场景,推荐采用选项式 API。
\item
当你打算用 Vue 构建完整的单页应用,推荐采用组合式 API + 单文件组件。
\end{itemize}
\end{itemize}
\switchcolumn[0]*%%%%%%%
You don't have to commit to only one style during the learning phase.
The rest of the documentation will provide code samples in both styles
where applicable, and you can toggle between them at any time using the
\textbf{API Preference switches} at the top of the left sidebar.
\switchcolumn
在学习阶段,你不必只固守一种风格。在接下来的文档中我们会为你提供一系列两种风格的代码供你参考,你可以随时通过左上角的
\textbf{API 风格偏好}来做切换。
\switchcolumn[0]*%%%%%%%
\subsection{Still Got Questions?}
Check out our \href{https://vuejs.org/about/faq}{FAQ}.
\switchcolumn
\subsection{还有其他问题?}
请查看我们的 \href{https://cn.vuejs.org/about/faq.html}{FAQ}。
\switchcolumn[0]*%%%%%%%
\subsection{Pick Your Learning Path}
\switchcolumn
\subsection{选择你的学习路径}
\switchcolumn[0]*%%%%%%%
Different developers have different learning styles. Feel free to pick a
learning path that suits your preference - although we do recommend
going over all of the content, if possible!
\switchcolumn
不同的开发者有不同的学习方式。尽管在可能的情况下,我们推荐你通读所有内容,但你还是可以自由地选择一种自己喜欢的学习路径!
\switchcolumn[0]*%%%%%%%
\href{https://vuejs.org/tutorial/}{Try the TutorialFor those who prefer
learning things
hands-on.}\href{https://vuejs.org/guide/quick-start}{Read the GuideThe
guide walks you through every aspect of the framework in full
detail.}\href{https://vuejs.org/examples/}{Check out the ExamplesExplore
examples of core features and common UI tasks.}
\switchcolumn
\href{https://cn.vuejs.org/tutorial/}{尝试互动教程适合喜欢边动手边学的读者。}\href{https://cn.vuejs.org/guide/quick-start.html}{继续阅读该指南该指南会带你深入了解框架所有方面的细节。}\href{https://cn.vuejs.org/examples/}{查看示例浏览核心功能和常见用户界面的示例。}
% \switchcolumn[0]*%%%%%%%
% \href{https://github.com/vuejs/docs/edit/main/src/guide/introduction.md}{Edit
% this page on GitHub}
% \switchcolumn
% \href{https://github.com/vuejs-translations/docs-zh-cn/edit/main/src/guide/introduction.md}{在
% GitHub 上编辑此页}
\end{paracol}



 
% \columnratio{0.55}
\begin{paracol}{2}
\switchcolumn[0]*%%%%%%%
\section{Quick Start}
\switchcolumn
\section{快速上手}
\switchcolumn[0]*%%%%%%%
\subsection{Try Vue Online}
\switchcolumn
\subsection{线上尝试 Vue}
\switchcolumn[0]*%%%%%%%
\begin{itemize}
\item
    To quickly get a taste of Vue, you can try it directly in our
    \href{https://play.vuejs.org/\#eNo9jcEKwjAMhl/lt5fpQYfXUQfefAMvvRQbddC1pUuHUPrudg4HIcmXjyRZXEM4zYlEJ+T0iEPgXjn6BB8Zhp46WUZWDjCa9f6w9kAkTtH9CRinV4fmRtZ63H20Ztesqiylphqy3R5UYBqD1UyVAPk+9zkvV1CKbCv9poMLiTEfR2/IXpSoXomqZLtti/IFwVtA9A==}{Playground}.
\item
    If you prefer a plain HTML setup without any build steps, you can use
    this \href{https://jsfiddle.net/yyx990803/2ke1ab0z/}{JSFiddle} as your
    starting point.
\item
    If you are already familiar with Node.js and the concept of build
    tools, you can also try a complete build setup right within your
    browser on \href{https://vite.new/vue}{StackBlitz}.
\end{itemize}
\switchcolumn
\begin{itemize}
\item
    想要快速体验
    Vue,你可以直接试试我们的\href{https://play.vuejs.org/\#eNo9jcEKwjAMhl/lt5fpQYfXUQfefAMvvRQbddC1pUuHUPrudg4HIcmXjyRZXEM4zYlEJ+T0iEPgXjn6BB8Zhp46WUZWDjCa9f6w9kAkTtH9CRinV4fmRtZ63H20Ztesqiylphqy3R5UYBqD1UyVAPk+9zkvV1CKbCv9poMLiTEfR2/IXpSoXomqZLtti/IFwVtA9A==}{演练场}。
\item
    如果你更喜欢不用任何构建的原始 HTML,可以使用
    \href{https://jsfiddle.net/yyx990803/2ke1ab0z/}{JSFiddle} 入门。
\item
    如果你已经比较熟悉 Node.js 和构建工具等概念,还可以直接在浏览器中打开
    \href{https://vite.new/vue}{StackBlitz} 来尝试完整的构建设置。
\end{itemize}
\switchcolumn[0]*%%%%%%%
\subsection{Creating a Vue Application}
\switchcolumn
\subsection{创建一个 Vue 应用}
\switchcolumn[0]*%%%%%%%
\begin{vueQuote}{Prerequisites}
\begin{itemize}
\item
    Familiarity with the command line
\item
    Install \href{https://nodejs.org/}{Node.js} version 16.0 or higher
\end{itemize}        
\end{vueQuote}    
\switchcolumn
\begin{vueQuote}{前提条件}
\begin{itemize}
\item
    熟悉命令行
\item
    已安装 16.0 或更高版本的 \href{https://nodejs.org/}{Node.js}
\end{itemize}
\end{vueQuote}    
\switchcolumn[0]*%%%%%%%
In this section we will introduce how to scaffold a Vue
\href{https://vuejs.org/guide/extras/ways-of-using-vue.html\#single-page-application-spa}{Single
Page Application} on your local machine. The created project will be
using a build setup based on \href{https://vitejs.dev/}{Vite} and allow
us to use Vue
\href{https://vuejs.org/guide/scaling-up/sfc.html}{Single-File
Components} (SFCs).
\switchcolumn
在本节中,我们将介绍如何在本地搭建 Vue
\href{https://cn.vuejs.org/guide/extras/ways-of-using-vue.html\#single-page-application-spa}{单页应用}。创建的项目将使用基于
\href{https://vitejs.dev/}{Vite} 的构建设置,并允许我们使用 Vue
的\href{https://cn.vuejs.org/guide/scaling-up/sfc.html}{单文件组件}
(SFC)。
\switchcolumn[0]*%%%%%%%
Make sure you have an up-to-date version of
\href{https://nodejs.org/}{Node.js} installed and your current working
directory is the one where you intend to create a project. Run the
following command in your command line (without the
\texttt{\textgreater{}} sign): 
\switchcolumn
确保你安装了最新版本的
\href{https://nodejs.org/}{Node.js},并且你的当前工作目录正是打算创建项目的目录。在命令行中运行以下命令
(不要带上 \texttt{\textgreater{}} 符号):
\switchcolumn[0]*%%%%%%%
\begin{codeShell}
npm create vue@latest
\end{codeShell}
\switchcolumn
\begin{codeShell}
npm create vue@latest
\end{codeShell}
\switchcolumn[0]*%%%%%%%
This command will install and execute
\href{https://github.com/vuejs/create-vue}{create-vue}, the official Vue
project scaffolding tool. You will be presented with prompts for several
optional features such as TypeScript and testing support:
\switchcolumn
这一指令将会安装并执行
\href{https://github.com/vuejs/create-vue}{create-vue},它是 Vue
官方的项目脚手架工具。你将会看到一些诸如 TypeScript
和测试支持之类的可选功能提示:
\switchcolumn[0]*%%%%%%%
\begin{codeConsole*}{escapeinside=||}
|\checkmark| Project name: … <your-project-name>
|\checkmark| Add TypeScript? … No / Yes
|\checkmark| Add JSX Support? … No / Yes
|\checkmark| Add Vue Router for Single Page Application development? … No / Yes
|\checkmark| Add Pinia for state management? … No / Yes
|\checkmark| Add Vitest for Unit testing? … No / Yes
|\checkmark| Add an End-to-End Testing Solution? … No / Cypress / Playwright
|\checkmark| Add ESLint for code quality? … No / Yes
|\checkmark| Add Prettier for code formatting? … No / Yes

Scaffolding project in ./<your-project-name>...
Done.
\end{codeConsole*}
\switchcolumn
\begin{codeConsole*}{escapeinside=||}
|\checkmark| Project name: … <your-project-name>
|\checkmark| Add TypeScript? … No / Yes
|\checkmark| Add JSX Support? … No / Yes
|\checkmark| Add Vue Router for Single Page Application development? … No / Yes
|\checkmark| Add Pinia for state management? … No / Yes
|\checkmark| Add Vitest for Unit testing? … No / Yes
|\checkmark| Add an End-to-End Testing Solution? … No / Cypress / Playwright
|\checkmark| Add ESLint for code quality? … No / Yes
|\checkmark| Add Prettier for code formatting? … No / Yes

Scaffolding project in ./<your-project-name>...
Done.
\end{codeConsole*} 
\switchcolumn[0]*%%%%%%%
If you are unsure about an option, simply choose \texttt{No} by hitting
enter for now. Once the project is created, follow the instructions to
install dependencies and start the dev server:
\switchcolumn
如果不确定是否要开启某个功能,你可以直接按下回车键选择
\texttt{No}。在项目被创建后,通过以下步骤安装依赖并启动开发服务器:
\switchcolumn[0]*%%%%%%%
\begin{codeShellMul}
cd <your-project-name>
npm install
npm run dev
\end{codeShellMul}
\switchcolumn
\begin{codeShellMul}
cd <your-project-name>
npm install
npm run dev
\end{codeShellMul}
\switchcolumn[0]*%%%%%%%
You should now have your first Vue project running! Note that the
example components in the generated project are written using the
\href{https://vuejs.org/guide/introduction.html\#composition-api}{Composition
API} and \texttt{\textless{}script\ setup\textgreater{}}, rather than
the
\href{https://vuejs.org/guide/introduction.html\#options-api}{Options
API}. Here are some additional tips:
\switchcolumn
你现在应该已经运行起来了你的第一个 Vue
项目!请注意,生成的项目中的示例组件使用的是\href{https://cn.vuejs.org/guide/introduction.html\#composition-api}{组合式
API} 和
\texttt{\textless{}script\ setup\textgreater{}},而非\href{https://cn.vuejs.org/guide/introduction.html\#options-api}{选项式
API}。下面是一些补充提示:
\switchcolumn[0]*%%%%%%%
\begin{itemize}
    \item
      The recommended IDE setup is
      \href{https://code.visualstudio.com/}{Visual Studio Code} +
      \href{https://marketplace.visualstudio.com/items?itemName=Vue.volar}{Volar
      extension}. If you use other editors, check out the
      \href{https://vuejs.org/guide/scaling-up/tooling.html\#ide-support}{IDE
      support section}.
    \item
      More tooling details, including integration with backend frameworks,
      are discussed in the
      \href{https://vuejs.org/guide/scaling-up/tooling.html}{Tooling Guide}.
    \item
      To learn more about the underlying build tool Vite, check out the
      \href{https://vitejs.dev/}{Vite docs}.
    \item
      If you choose to use TypeScript, check out the
      \href{https://vuejs.org/guide/typescript/overview.html}{TypeScript
      Usage Guide}.
    \end{itemize}
\switchcolumn
\begin{itemize}
    \item
      推荐的 IDE 配置是 \href{https://code.visualstudio.com/}{Visual Studio
      Code} +
      \href{https://marketplace.visualstudio.com/items?itemName=Vue.volar}{Volar
      扩展}。如果使用其他编辑器,参考
      \href{https://cn.vuejs.org/guide/scaling-up/tooling.html\#ide-support}{IDE
      支持章节}。
    \item
      更多工具细节,包括与后端框架的整合,我们会在\href{https://cn.vuejs.org/guide/scaling-up/tooling.html}{工具链指南}进行讨论。
    \item
      要了解构建工具 Vite 更多背后的细节,请查看
      \href{https://cn.vitejs.dev/}{Vite 文档}。
    \item
      如果你选择使用 TypeScript,请阅读
      \href{https://cn.vuejs.org/guide/typescript/overview.html}{TypeScript
      使用指南}。
    \end{itemize}
\switchcolumn[0]*%%%%%%%
When you are ready to ship your app to production, run the following:
\switchcolumn
当你准备将应用发布到生产环境时,请运行:
\switchcolumn[0]*%%%%%%%
\begin{codeShellMul}
npm run build
\end{codeShellMul}
\switchcolumn
\begin{codeShellMul}
npm run build
\end{codeShellMul}
\switchcolumn[0]*%%%%%%%
This will create a production-ready build of your app in the project's
\texttt{./dist} directory. Check out the
\href{https://vuejs.org/guide/best-practices/production-deployment.html}{Production
Deployment Guide} to learn more about shipping your app to production.
\switchcolumn
此命令会在 \texttt{./dist}
文件夹中为你的应用创建一个生产环境的构建版本。关于将应用上线生产环境的更多内容,请阅读\href{https://cn.vuejs.org/guide/best-practices/production-deployment.html}{生产环境部署指南}。
\switchcolumn[0]*%%%%%%%
\href{https://vuejs.org/guide/quick-start.html\#next-steps}{Next Steps
\textgreater{}}
\switchcolumn
\href{https://cn.vuejs.org/guide/quick-start.html\#next-steps}{下一步\textgreater{}}
\switchcolumn[0]*%%%%%%%

\switchcolumn

\switchcolumn[0]*%%%%%%%

\switchcolumn

\switchcolumn[0]*%%%%%%%

\switchcolumn

\switchcolumn[0]*%%%%%%%

\switchcolumn

\switchcolumn[0]*%%%%%%%

\switchcolumn

\switchcolumn[0]*%%%%%%%

\switchcolumn

\switchcolumn[0]*%%%%%%%

\switchcolumn

\switchcolumn[0]*%%%%%%%

\switchcolumn

\switchcolumn[0]*%%%%%%%

\switchcolumn

\switchcolumn[0]*%%%%%%%

\switchcolumn

\switchcolumn[0]*%%%%%%%

\switchcolumn

\switchcolumn[0]*%%%%%%%

\switchcolumn

\switchcolumn[0]*%%%%%%%

\switchcolumn

\switchcolumn[0]*%%%%%%%

\switchcolumn

\switchcolumn[0]*%%%%%%%

\switchcolumn

\switchcolumn[0]*%%%%%%%

\switchcolumn

\switchcolumn[0]*%%%%%%%

\switchcolumn

\switchcolumn[0]*%%%%%%%

\switchcolumn

\switchcolumn[0]*%%%%%%%

\switchcolumn

\switchcolumn[0]*%%%%%%%

\switchcolumn


\switchcolumn[0]*%%%%%%%

\switchcolumn


\switchcolumn[0]*%%%%%%%

\switchcolumn

\switchcolumn[0]*%%%%%%%

\switchcolumn


\switchcolumn[0]*%%%%%%%

\switchcolumn

\switchcolumn[0]*%%%%%%%

\switchcolumn


\switchcolumn[0]*%%%%%%%

\switchcolumn

\switchcolumn[0]*%%%%%%%

\switchcolumn


\switchcolumn[0]*%%%%%%%

\switchcolumn

\switchcolumn[0]*%%%%%%%

\switchcolumn


\switchcolumn[0]*%%%%%%%

\switchcolumn


\switchcolumn[0]*%%%%%%%

\switchcolumn

\switchcolumn[0]*%%%%%%%

\switchcolumn


\switchcolumn[0]*%%%%%%%

\switchcolumn

\switchcolumn[0]*%%%%%%%

\switchcolumn


\switchcolumn[0]*%%%%%%%

\switchcolumn

\switchcolumn[0]*%%%%%%%

\switchcolumn


\switchcolumn[0]*%%%%%%%

\switchcolumn


\end{paracol}
 

\chapter{Essentials\hfill 基础}
% \columnratio{0.55}
\begin{paracol}{2}
\switchcolumn[0]*%%%%%%%
\section{Creating a Vue Application}
\switchcolumn
\section{创建一个 Vue 应用}
\switchcolumn[0]*%%%%%%%
\subsection{The application instance}
\switchcolumn
\subsection{应用实例}
\switchcolumn[0]*%%%%%%%
Every Vue application starts by creating a new \textbf{application
instance} with the
\href{https://vuejs.org/api/application.html\#createapp}{\texttt{createApp}}
function:
\switchcolumn
每个 Vue 应用都是通过
\href{https://cn.vuejs.org/api/application.html\#createapp}{\texttt{createApp}}
函数创建一个新的 \textbf{应用实例}:
\switchcolumn[0]*%%%%%%%
\begin{codeJs}
import { createApp } from 'vue'

const app = createApp({
    /* root component options */
})
\end{codeJs}
\switchcolumn
\begin{codeJs}
import { createApp } from 'vue'

const app = createApp({
    /* root component options */
})
\end{codeJs}
\switchcolumn[0]*%%%%%%%
\subsection{The Root Component}
\switchcolumn
\subsection{根组件}
\switchcolumn[0]*%%%%%%%
The object we are passing into \texttt{createApp} is in fact a
component. Every app requires a "root component" that can contain other
components as its children.
\switchcolumn
我们传入 \texttt{createApp}
的对象实际上是一个组件,每个应用都需要一个``根组件'',其他组件将作为其子组件。
\switchcolumn[0]*%%%%%%%
If you are using Single-File Components, we typically import the root
component from another file:
\switchcolumn
如果你使用的是单文件组件,我们可以直接从另一个文件中导入根组件。
\switchcolumn[0]*%%%%%%%
\begin{codeJs}
import { createApp } from 'vue'
// import the root component App from a single-file component.
import App from './App.vue'

const app = createApp(App)
\end{codeJs}
\switchcolumn
\begin{codeJs}
import { createApp } from 'vue'
// import the root component App from a single-file component.
import App from './App.vue'

const app = createApp(App)
\end{codeJs}
\switchcolumn[0]*%%%%%%%
While many examples in this guide only need a single component, most
real applications are organized into a tree of nested, reusable
components. For example, a Todo application's component tree might look
like this:
\switchcolumn
虽然本指南中的许多示例只需要一个组件,但大多数真实的应用都是由一棵嵌套的、可重用的组件树组成的。例如,一个待办事项
(Todos) 应用的组件树可能是这样的:

\end{paracol}

\columnratio{0.55}
\begin{paracol}{2}
\switchcolumn[0]*%%%%%%%
\section{Template Syntax}
\switchcolumn
\section{模板语法}
\switchcolumn[0]*%%%%%%%
Vue uses an HTML-based template syntax that allows you to declaratively
bind the rendered DOM to the underlying component instance's data. All
Vue templates are syntactically valid HTML that can be parsed by
spec-compliant browsers and HTML parsers.
\switchcolumn
Vue 使用一种基于 HTML
的模板语法,使我们能够声明式地将其组件实例的数据绑定到呈现的 DOM
上。所有的 Vue 模板都是语法层面合法的 HTML,可以被符合规范的浏览器和
HTML 解析器解析。
\switchcolumn[0]*%%%%%%%
Under the hood, Vue compiles the templates into highly-optimized
JavaScript code. Combined with the reactivity system, Vue can
intelligently figure out the minimal number of components to re-render
and apply the minimal amount of DOM manipulations when the app state
changes.
\switchcolumn
在底层机制中,Vue 会将模板编译成高度优化的 JavaScript
代码。结合响应式系统,当应用状态变更时,Vue
能够智能地推导出需要重新渲染的组件的最少数量,并应用最少的 DOM 操作。
\switchcolumn[0]*%%%%%%%
If you are familiar with Virtual DOM concepts and prefer the raw power
of JavaScript, you can also
\href{https://vuejs.org/guide/extras/render-function.html}{directly
write render functions} instead of templates, with optional JSX support.
However, do note that they do not enjoy the same level of compile-time
optimizations as templates.
\switchcolumn
如果你对虚拟 DOM 的概念比较熟悉,并且偏好直接使用
JavaScript,你也可以结合可选的 JSX
支持\href{https://cn.vuejs.org/guide/extras/render-function.html}{直接手写渲染函数}而不采用模板。但请注意,这将不会享受到和模板同等级别的编译时优化。
\switchcolumn[0]*%%%%%%%
\subsection{Text Interpolation}
\switchcolumn
\subsection{文本插值}
\switchcolumn[0]*%%%%%%%
The most basic form of data binding is text interpolation using the
"Mustache" syntax (double curly braces):
\switchcolumn
最基本的数据绑定形式是文本插值,它使用的是``Mustache''语法
(即双大括号):
\switchcolumn[0]*%%%%%%%
\begin{codeHtml*}{label=template}
<span>Message: {{ msg }}</span>
\end{codeHtml*}  
\switchcolumn
\begin{codeHtml*}{label=template}
<span>Message: {{ msg }}</span>
\end{codeHtml*}  

\switchcolumn[0]*%%%%%%%
The mustache tag will be replaced with the value of the \texttt{msg}
property
\href{https://vuejs.org/guide/essentials/reactivity-fundamentals.html\#declaring-reactive-state}{from
the corresponding component instance}. It will also be updated whenever
the \texttt{msg} property changes.
\switchcolumn
双大括号标签会被替换为\href{https://cn.vuejs.org/guide/essentials/reactivity-fundamentals.html\#declaring-reactive-state}{相应组件实例中}
\texttt{msg} 属性的值。同时每次 \texttt{msg} 属性更改时它也会同步更新。
\switchcolumn[0]*%%%%%%%
\subsection{Raw HTML}
\switchcolumn
\subsection{原始 HTML}
\switchcolumn[0]*%%%%%%%
The double mustaches interpret the data as plain text, not HTML. In
order to output real HTML, you will need to use the
\href{https://vuejs.org/api/built-in-directives.html\#v-html}{\texttt{v-html}
directive}:
\switchcolumn
双大括号会将数据解释为纯文本,而不是 HTML。若想插入 HTML,你需要使用
\href{https://cn.vuejs.org/api/built-in-directives.html\#v-html}{\texttt{v-html}
指令}:
\switchcolumn[0]*%%%%%%%
\begin{codeHtml}
<p>Using text interpolation: {{ rawHtml }}</p>
<p>Using v-html directive: <span v-html="rawHtml"></span></p>
\end{codeHtml}  
\switchcolumn
\begin{codeHtml}
<p>Using text interpolation: {{ rawHtml }}</p>
<p>Using v-html directive: <span v-html="rawHtml"></span></p>
\end{codeHtml}  
\switchcolumn[0]*%%%%%%%
\begin{vueQuote}{result}
Using text interpolation: This should be red.\\
Using v-html directive: {\textcolor{red}{This should be red.}}
\end{vueQuote}
\switchcolumn
\begin{vueQuote}{结果}
Using text interpolation: This should be red.\\
Using v-html directive: {\textcolor{red}{This should be red.}}
\end{vueQuote}

\switchcolumn[0]*%%%%%%%
Here we're encountering something new. The \texttt{v-html} attribute
you're seeing is called a \textbf{directive}. Directives are prefixed
with \texttt{v-} to indicate that they are special attributes provided
by Vue, and as you may have guessed, they apply special reactive
behavior to the rendered DOM. Here, we're basically saying "keep this
element's inner HTML up-to-date with the \texttt{rawHtml} property on
the current active instance."
\switchcolumn
这里我们遇到了一个新的概念。这里看到的 \texttt{v-html} attribute
被称为一个\textbf{指令}。指令由 \texttt{v-} 作为前缀,表明它们是一些由
Vue 提供的特殊 attribute,你可能已经猜到了,它们将为渲染的 DOM
应用特殊的响应式行为。这里我们做的事情简单来说就是:在当前组件实例上,将此元素的
innerHTML 与 \texttt{rawHtml} 属性保持同步。
\switchcolumn[0]*%%%%%%%
The contents of the \texttt{span} will be replaced with the value of the
\texttt{rawHtml} property, interpreted as plain HTML - data bindings are
ignored. Note that you cannot use \texttt{v-html} to compose template
partials, because Vue is not a string-based templating engine. Instead,
components are preferred as the fundamental unit for UI reuse and
composition.
\switchcolumn
\texttt{span} 的内容将会被替换为 \texttt{rawHtml} 属性的值,插值为纯
HTML------数据绑定将会被忽略。注意,你不能使用 \texttt{v-html}
来拼接组合模板,因为 Vue 不是一个基于字符串的模板引擎。在使用 Vue
时,应当使用组件作为 UI 重用和组合的基本单元。
\switchcolumn[0]*%%%%%%%
\begin{vueQuoteWarn}{Security Warning}
Dynamically rendering arbitrary HTML on your website can be very
dangerous because it can easily lead to
\href{https://en.wikipedia.org/wiki/Cross-site_scripting}{XSS
vulnerabilities}. Only use \texttt{v-html} on trusted content and
\textbf{never} on user-provided content.
\end{vueQuoteWarn}
\switchcolumn
\begin{vueQuoteWarn}{安全警告}
在网站上动态渲染任意 HTML 是非常危险的,因为这非常容易造成
\href{https://zh.wikipedia.org/wiki/跨網站指令碼}{XSS
漏洞}。请仅在内容安全可信时再使用
\texttt{v-html},并且\textbf{永远不要}使用用户提供的 HTML 内容。
\end{vueQuoteWarn}
\switchcolumn[0]*%%%%%%%
\subsection{Attribute Bindings}
\switchcolumn
\subsection{Attribute 绑定}
\switchcolumn[0]*%%%%%%%
Mustaches cannot be used inside HTML attributes. Instead, use a
\href{https://vuejs.org/api/built-in-directives.html\#v-bind}{\texttt{v-bind}
directive}:
\switchcolumn
双大括号不能在 HTML attributes 中使用。想要响应式地绑定一个
attribute,应该使用
\href{https://cn.vuejs.org/api/built-in-directives.html\#v-bind}{\texttt{v-bind}
指令}:

\switchcolumn[0]*%%%%%%%
\begin{codeHtml}
<div v-bind:id="dynamicId"></div>
\end{codeHtml}  
\switchcolumn
\begin{codeHtml}
<div v-bind:id="dynamicId"></div>
\end{codeHtml}  
\switchcolumn[0]*%%%%%%%
The \texttt{v-bind} directive instructs Vue to keep the element's
\texttt{id} attribute in sync with the component's \texttt{dynamicId}
property. If the bound value is \texttt{null} or \texttt{undefined},
then the attribute will be removed from the rendered element.
\switchcolumn
\texttt{v-bind} 指令指示 Vue 将元素的 \texttt{id} attribute 与组件的
\texttt{dynamicId} 属性保持一致。如果绑定的值是 \texttt{null} 或者
\texttt{undefined},那么该 attribute 将会从渲染的元素上移除。
\switchcolumn[0]*%%%%%%%
\subsubsection{Shorthand}
\switchcolumn
\subsubsection{简写}
\switchcolumn[0]*%%%%%%%
Because \texttt{v-bind} is so commonly used, it has a dedicated
shorthand syntax:
\switchcolumn
因为 \texttt{v-bind} 非常常用,我们提供了特定的简写语法:


\switchcolumn[0]*%%%%%%%
\begin{codeHtml}
<div :id="dynamicId"></div>
\end{codeHtml}  
\switchcolumn
\begin{codeHtml}
<div :id="dynamicId"></div>
\end{codeHtml}  
\switchcolumn[0]*%%%%%%%
Attributes that start with \texttt{:} may look a bit different from
normal HTML, but it is in fact a valid character for attribute names and
all Vue-supported browsers can parse it correctly. In addition, they do
not appear in the final rendered markup. The shorthand syntax is
optional, but you will likely appreciate it when you learn more about
its usage later.
\switchcolumn
开头为 \texttt{:} 的 attribute 可能和一般的 HTML attribute
看起来不太一样,但它的确是合法的 attribute 名称字符,并且所有支持 Vue
的浏览器都能正确解析它。此外,他们不会出现在最终渲染的 DOM
中。简写语法是可选的,但相信在你了解了它更多的用处后,你应该会更喜欢它。
\switchcolumn[0]*%%%%%%%
\begin{quote}
For the rest of the guide, we will be using the shorthand syntax in code
examples, as that's the most common usage for Vue developers.
\end{quote}
\switchcolumn
\begin{quote}
接下来的指引中,我们都将在示例中使用简写语法,因为这是在实际开发中更常见的用法。
\end{quote}
\switchcolumn[0]*%%%%%%%
\subsubsection{Boolean Attributes}
\switchcolumn
\subsubsection{布尔型 Attribute}
\switchcolumn[0]*%%%%%%%
\href{https://html.spec.whatwg.org/multipage/common-microsyntaxes.html\#boolean-attributes}{Boolean
attributes} are attributes that can indicate true / false values by
their presence on an element. For example,
\href{https://developer.mozilla.org/en-US/docs/Web/HTML/Attributes/disabled}{\texttt{disabled}}
is one of the most commonly used boolean attributes.
\switchcolumn
\href{https://developer.mozilla.org/zh-CN/docs/Web/HTML/Attributes\#布尔值属性}{布尔型
attribute} 依据 true / false 值来决定 attribute
是否应该存在于该元素上。\href{https://developer.mozilla.org/en-US/docs/Web/HTML/Attributes/disabled}{\texttt{disabled}}
就是最常见的例子之一。
\switchcolumn[0]*%%%%%%%
\texttt{v-bind} works a bit differently in this case:
\switchcolumn
\texttt{v-bind} 在这种场景下的行为略有不同:
\end{paracol}

 
% %% todo 带圈文字的处理,代码中的带圈文字可以改成引用...
% % % \part{Environment Setup}% \textbf{Part I}\\
% \textbf{第一部分}\\
% \textbf{环境设置}

\chapter{Building, Running, and the REPL}
% 构建、运行和REPL
\columnratio{0.55}
\begin{paracol}{2}
\switchcolumn[0]*
In this chapter, you'll invest a small amount of time up front to get
familiar with a quick, foolproof way to build and run Clojure programs.
It feels great to get a real program running. Reaching that milestone
frees you up to experiment, share your work, and gloat to your
colleagues who are still using last decade's languages. This will help
keep you motivated!
\switchcolumn
在本章中,您将花费一小部分时间来熟悉一种快速、可靠的构建和运行Clojure程序的方法。让一个真正的程序运行起来感觉很棒。达到这个里程碑将使您能够自由地进行实验,分享您的工作,并向那些仍在使用上个十年语言的同事炫耀。这将有助于保持您的动力!
\switchcolumn[0]*
You'll also learn how to instantly run code within a running Clojure
process using a \emph{Read-Eval-Print Loop (REPL)}, which allows you to
quickly test your understanding of the language and learn more
efficiently.
\switchcolumn
您还将学习如何使用\emph{Read-Eval-Print Loop (REPL)}在运行中的Clojure进程中立即运行代码,这样可以快速测试您对语言的理解并更有效地学习。
\switchcolumn[0]*
But first, I'll briefly introduce Clojure. Next, I'll cover Leiningen,
the de facto standard build tool for Clojure. By the end of the chapter,
you'll know how to do the following:
\switchcolumn
但首先,我将简要介绍Clojure。接下来,我将介绍Leiningen,Clojure的事实上的标准构建工具。在本章结束时,您将知道如何执行以下操作:

\begin{itemize}
\switchcolumn[0]*
\item Create a new Clojure project with Leiningen
\switchcolumn
\item 使用Leiningen创建一个新的Clojure项目
\switchcolumn[0]*
\item Build the project to create an executable JAR file
\switchcolumn
\item 构建项目以创建可执行的JAR文件
\switchcolumn[0]*
\item Execute the JAR file
\switchcolumn
\item 执行JAR文件
\switchcolumn[0]*
\item Execute code in a Clojure REPL
\switchcolumn
\item 在Clojure REPL中执行代码
\end{itemize}
\switchcolumn[0]*
\section{First Things First: What Is Clojure?}
\switchcolumn
\section{首先要明确的是:Clojure是什么?}
\switchcolumn[0]*
Clojure was forged in a mythic volcano by Rich Hickey. Using an alloy of
Lisp, functional programming, and a lock of his own epic hair, he
crafted a language that's delightful yet powerful. Its Lisp heritage
gives you the power to write code more expressively than is possible in
most non-Lisp languages, and its distinct take on functional programming
will sharpen your thinking as a programmer. Plus, Clojure gives you
better tools for tackling complex domains (like concurrent programming)
that are traditionally known to drive developers into years of therapy.
\switchcolumn
Clojure是由Rich Hickey在一个神话般的火山中锻造而成的。他使用了Lisp、函数式编程和他自己史诗般的头发的一缕合金,创造了一种既令人愉悦又强大的语言。它的Lisp传统使您能够以比大多数非Lisp语言更富有表现力的方式编写代码,并且它对函数式编程的独特理解将增强您作为程序员的思维能力。此外,Clojure为您提供了更好的工具来处理复杂的领域(如并发编程),而这些领域通常被认为会让开发人员陷入多年的治疗中。
\switchcolumn[0]*
When talking about Clojure, though, it's important to keep in mind the
distinction between the Clojure language and the Clojure compiler. The
Clojure language is a Lisp dialect with a functional emphasis whose
syntax and semantics are independent of any implementation. The compiler
is an executable JAR file, \emph{clojure.jar}, which takes code written
in the Clojure language and compiles it to Java Virtual Machine (JVM)
bytecode. You'll see \emph{Clojure} used to refer to both the language
and the compiler, which can be confusing if you're not aware that
they're separate things. But now that you're aware, you'll be fine.
\switchcolumn
然而,在谈论Clojure时,重要的是要区分Clojure语言和Clojure编译器之间的区别。Clojure语言是一个带有函数式强调的Lisp方言,其语法和语义与任何实现无关。编译器是一个可执行的JAR文件\emph{clojure.jar},它将用Clojure语言编写的代码编译为Java虚拟机(JVM)字节码。您会看到\emph{Clojure}一词既用于指代语言,也用于指代编译器,如果您不知道它们是分开的,这可能会让您感到困惑。但是现在您知道了,一切都会没事的。
\switchcolumn[0]*
This distinction is necessary because, unlike most programming languages
like Ruby, Python, C, and a bazillion others, Clojure is a \emph{hosted
language}. Clojure programs are executed within a JVM and rely on the
JVM for core features like threading and garbage collection. Clojure
also targets JavaScript and the Microsoft Common Language Runtime (CLR),
but this book only focuses on the JVM implementation.
\switchcolumn
之所以需要这种区分,是因为与大多数编程语言(如Ruby、Python、C等等)不同,Clojure是一种\emph{托管语言}。Clojure程序在JVM内执行,并依赖JVM提供核心功能,如线程和垃圾回收。Clojure还支持JavaScript和Microsoft Common Language Runtime (CLR),但本书只关注JVM实现。
\switchcolumn[0]*
We'll explore the relationship between Clojure and the JVM more later
on, but for now the main concepts you need to understand are these:
\switchcolumn
我们将在后面更详细地探讨Clojure与JVM之间的关系,但目前您需要了解的主要概念是:
\begin{itemize}
\switchcolumn[0]*
\item JVM processes execute Java bytecode.
\switchcolumn
\item JVM进程执行Java字节码。
\switchcolumn[0]*
\item Usually, the Java Compiler produces Java bytecode from Java source
code.
\switchcolumn
\item 通常,Java编译器会从Java源代码生成Java字节码。
\switchcolumn[0]*
\item JAR files are collections of Java bytecode.
\switchcolumn
JAR文件是Java字节码的集合。
\switchcolumn[0]*
\item Java programs are usually distributed as JAR files.
\switchcolumn
\item Java程序通常以JAR文件的形式分发。
\switchcolumn[0]*
\item The Java program \emph{clojure.jar} reads Clojure source code and
produces Java bytecode.
\switchcolumn
\item Java程序\emph{clojure.jar}读取Clojure源代码并生成Java字节码。
\switchcolumn[0]*
\item That Java bytecode is then executed by the same JVM process already
running \emph{clojure.jar}.
\switchcolumn
\item 然后,同一JVM进程中已经运行的\emph{clojure.jar}执行该Java字节码。
\end{itemize}
\switchcolumn[0]*
Clojure continues to evolve. As of this writing, it's at version 1.7.0,
and development is going strong. If you're reading this book in the far
future and Clojure has a higher version number, don't worry! This book
covers Clojure's fundamentals, which shouldn't change from one version
to the next. There's no need for your robot butler to return this book
to the bookstore.
\switchcolumn
Clojure不断发展。截至本书编写时,它的版本为1.7.0,并且开发工作正在进行中。如果你在未来的某个时间阅读本书,而Clojure的版本号更高,不要担心!本书涵盖了Clojure的基础知识,这些知识不应该随着版本变化而改变。你的机器人管家没有必要把本书退还给书店。
\switchcolumn[0]*
Now that you know what Clojure is, let's actually build a freakin'
Clojure program!
\switchcolumn
现在你知道Clojure是什么了,让我们来实际构建一个Clojure程序吧!
\switchcolumn[0]*
\section{Leiningen}
\switchcolumn
\section{Leiningen}
\switchcolumn[0]*
These days, most Clojurists use Leiningen to build and manage their
projects. You can read a full description of Leiningen in
\href{javascript:void(0)}{Appendix A}, but for now we'll focus on using
it for four tasks:
\switchcolumn
现在大多数Clojure开发人员使用Leiningen来构建和管理他们的项目。你可以在\href{javascript:void(0)}{附录A}中阅读关于Leiningen的完整描述,但现在我们将重点介绍使用Leiningen执行以下四个任务:
\begin{enumerate}
\switchcolumn[0]*
\item  Creating a new Clojure project
\switchcolumn
\item  创建一个新的Clojure项目
\switchcolumn[0]*
\item  Running the Clojure project
\switchcolumn
\item  运行Clojure项目
\switchcolumn[0]*
\item  Building the Clojure project
\switchcolumn
\item  构建Clojure项目
\switchcolumn[0]*
\item  Using the REPL
\switchcolumn
\item  使用REPL
\end{enumerate}

\switchcolumn[0]*
Before continuing, make sure you have Java version 1.6 or later
installed. You can check your version by running java -version in your
terminal, and download the latest Java Runtime Environment (JRE)\footnote{from
\emph{http://www.oracle.com/technetwork/java/javase/downloads/index.html}}.
Then, install Leiningen using the instructions on the Leiningen home
page at \emph{http://leiningen.org/} (Windows users, note there's a
Windows installer). When you install Leiningen, it automatically
downloads the Clojure compiler, \emph{clojure.jar}.
\switchcolumn
在继续之前,请确保你已经安装了Java版本1.6或更高版本。你可以在终端中运行java -version命令来检查你的版本,并下载\footnote{从\emph{http://www.oracle.com/technetwork/java/javase/downloads/index.html}}最新的Java Runtime Environment (JRE)。然后,按照\emph{http://leiningen.org/}主页上的说明安装Leiningen(Windows用户请注意,有一个Windows安装程序)。当你安装Leiningen时,它会自动下载Clojure编译器\emph{clojure.jar}。
\switchcolumn[0]*
\subsection{Creating a New Clojure Project}
\switchcolumn
\subsection{创建一个新的Clojure项目}
\switchcolumn[0]*
Creating a new Clojure project is very simple. A single Leiningen
command creates a project skeleton. Later, you'll learn how to do tasks
like incorporate Clojure libraries, but for now, these instructions will
enable you to execute the code you write.
\switchcolumn
创建一个新的Clojure项目非常简单。一个Leiningen命令就可以创建一个项目骨架。稍后,你将学习如何执行诸如整合Clojure库之类的任务,但现在这些说明将使你能够执行你编写的代码。
\switchcolumn[0]*
Go ahead and create your first Clojure project by typing the following
in your terminal:
\switchcolumn
在终端中输入以下命令,创建你的第一个Clojure项目:
\switchcolumn[0]*
\begin{verbatim}
lein new app clojure-noob
\end{verbatim}
\switchcolumn
\begin{verbatim}
lein new app clojure-noob
\end{verbatim}
\switchcolumn[0]*
This command should create a directory structure that looks similar to
this (it's okay if there are some differences):
\switchcolumn
这个命令应该会创建一个类似下面这样的目录结构(如果有一些差异也没关系):
\switchcolumn[0]*
\begin{verbatim}
| .gitignore
| doc
| | intro.md
➊ | project.clj
| README.md
➋ | resources
| src

| | clojure_noob
➌ | | | core.clj
➍ | test
| | clojure_noob
| | | core_test.clj
\end{verbatim}
\switchcolumn
\begin{verbatim}
| .gitignore
| doc
| | intro.md
➊ | project.clj
| README.md
➋ | resources
| src

| | clojure_noob
➌ | | | core.clj
➍ | test
| | clojure_noob
| | | core_test.clj
\end{verbatim}
\switchcolumn[0]*
This project skeleton isn't inherently special or Clojure-y. It's just a
convention used by Leiningen. You'll be using Leiningen to build and run
Clojure apps, and Leiningen expects your app to have this structure. The
first file of note is \emph{project.clj} at ➊, which is a configuration
file for Leiningen. It helps Leiningen answer such questions as ``What
dependencies does this project have?'' and ``When this Clojure program
runs, what function should run first?'' In general, you'll save your
your source code in \emph{src/\textless project\_name\textgreater{}}. In
this case, the file \emph{src/clojure\_noob/core.clj} at ➌ is where
you'll be writing your Clojure code for a while. The \emph{test}
directory at ➍ obviously contains tests, and \emph{resources} at ➋ is
where you store assets like images.
\switchcolumn
这个项目骨架本身并没有什么特别的或者特定于Clojure的东西。它只是Leiningen使用的一种约定。你将使用Leiningen来构建和运行Clojure应用程序,而Leiningen希望你的应用程序具有这种结构。第一个要注意的文件是位于➊处的\emph{project.clj},这是Leiningen的配置文件。它帮助Leiningen回答诸如“这个项目有哪些依赖项?”和“当这个Clojure程序运行时,哪个函数应该首先运行?”这样的问题。通常情况下,你会将源代码保存在\emph{src/\textless project\_name\textgreater{}}中。在这个例子中,文件\emph{src/clojure\_noob/core.clj}位于➌处,你将在其中写入你的Clojure代码。➍处的\emph{test}目录显然包含测试代码,而\emph{resources}目录位于➋处,用于存储像图像之类的资源文件。
\switchcolumn[0]*
\subsection{Running the Clojure Project}
\switchcolumn
\subsection{运行Clojure项目}
\switchcolumn[0]*
Now let's actually run the project. Open
\emph{src/clojure\_noob/core.clj} in your favorite editor. You should
see this:
\switchcolumn
现在让我们实际运行这个项目。在你喜欢的编辑器中打开\emph{src/clojure\_noob/core.clj}。你应该会看到以下内容:
\switchcolumn[0]*
\begin{verbatim}
➊ (ns clojure-noob.core
        (:gen-class))
➋ (defn -main
        "I don't do a whole lot...yet."
        [& args]
➌   (println "Hello, World!"))
\end{verbatim}
\switchcolumn
\begin{verbatim}
➊ (ns clojure-noob.core
        (:gen-class))
➋ (defn -main
        "I don't do a whole lot...yet."
        [& args]
➌   (println "Hello, World!"))
\end{verbatim}

\switchcolumn[0]*
The lines at ➊ declare a namespace, which you don't need to worry about
right now. The -main function at ➋ is the \emph{entry point} to your
program, a topic that is covered in \href{javascript:void(0)}{Appendix
A}. For now, replace the text "Hello, World!" at ➌ with "I'm a little
teapot!". The full line should read (println "I'm a little teapot!")).
\switchcolumn
➊处的代码声明了一个命名空间,你现在不需要担心它。➋处的-main函数是你程序的\emph{入口点},这个主题在\href{javascript:void(0)}{附录A}中有介绍。现在,将➌处的文本“Hello, World!”替换为“I'm a little teapot!”。完整的行应该是(println "I'm a little teapot!")。
\switchcolumn[0]*
Next, navigate to the \emph{clojure\_noob} directory in your terminal
and enter:
\begin{verbatim}
lein run
\end{verbatim}
\switchcolumn
接下来,在终端中导航到\emph{clojure\_noob}目录,并输入以下命令:
\begin{verbatim}
lein run
\end{verbatim}
\switchcolumn[0]*
You should see the output "I'm a little teapot!" Congratulations, little
teapot, you wrote and executed a program!
\switchcolumn
如果一切顺利,你将在终端中看到输出的消息:“I'm a little teapot!”
\switchcolumn[0]*
You'll learn more about what's actually happening in the program as you
read through the book, but for now all you need to know is that you
created a function, -main, and that function runs when you execute lein
run at the command line.
\switchcolumn
当您阅读本书时,您将更多地了解程序中实际发生的情况,但现在您只需要知道您创建了一个函数,-main,并且当您在命令行上执行lein run时,该函数将运行。
\switchcolumn[0]*
\subsection{Building the Clojure Project}
\switchcolumn
\subsection{构建Clojure项目}
\switchcolumn[0]*
Using lein run is great for trying out your code, but what if you want
to share your work with people who don't have Leiningen installed? To do
that, you can create a stand-alone file that anyone with Java installed
(which is basically everyone) can execute. To create the file, run this:
\begin{verbatim}
lein uberjar
\end{verbatim}
\switchcolumn
使用lein run非常适合尝试你的代码,但是如果你想与没有安装Leiningen的人分享你的工作呢?为了实现这一点,你可以创建一个独立的文件,任何安装了Java(基本上是每个人)的人都可以执行。要创建该文件,请运行以下命令:
\begin{verbatim}
lein uberjar
\end{verbatim}
\switchcolumn[0]*
This command creates the file \emph{target/uberjar/clojure-noob-0.1.0
-SNAPSHOT-standalone.jar}. You can make Java execute it by running this:
\begin{verbatim}
java -jar target/uberjar/clojure-noob-0.1.0-SNAPSHOT-standalone.jar
\end{verbatim}
\switchcolumn
该命令会创建文件\emph{target/uberjar/clojure-noob-0.1.0-SNAPSHOT-standalone.jar}。你可以通过运行以下命令让Java执行该文件:
\begin{verbatim}
java -jar target/uberjar/clojure-noob-0.1.0-SNAPSHOT-standalone.jar
\end{verbatim}
\switchcolumn[0]*
Look at that! The file \emph{target/uberjar/clojure-noob-0.1.0-SNAPSHOT
-standalone.jar} is your new, award-winning Clojure program, which you
can distribute and run on almost any platform.
\switchcolumn
看到了吗!文件\emph{target/uberjar/clojure-noob-0.1.0-SNAPSHOT-standalone.jar}就是你的新的、屡获殊荣的Clojure程序,你可以在几乎任何平台上分发和运行它。
\switchcolumn[0]*
You now have all the basic details you need to build, run, and
distribute (very) basic Clojure programs. In later chapters, you'll
learn more details about what Leiningen is doing when you run the
preceding commands, gaining a complete understanding of Clojure's
relationship to the JVM and how you can run production code.
\switchcolumn
现在你已经掌握了构建、运行和分发(非常)基本的Clojure程序所需的基本细节。在后面的章节中,你将了解在运行前面的命令时Leiningen在做什么,从而完全理解Clojure与JVM的关系以及如何运行生产代码。
\switchcolumn[0]*
Before we move on to \href{javascript:void(0)}{Chapter 2} and discuss
the wonder and glory of Emacs, let's go over another important tool: the
REPL.
\switchcolumn
在我们继续讨论Emacs的奇迹和荣耀之前,让我们再介绍一个重要的工具:REPL。

\switchcolumn[0]*
\subsection{Using the REPL}
\switchcolumn
\subsection{使用REPL}
\switchcolumn[0]*
The REPL is a tool for experimenting with code. It allows you to
interact with a running program and quickly try out ideas. It does this
by presenting you with a prompt where you can enter code. It then
\emph{reads} your input, \emph{evaluates} it, \emph{prints} the result,
and \emph{loops}, presenting you with a prompt again.
\switchcolumn
REPL是一个用于实验代码的工具。它允许你与正在运行的程序进行交互,并快速尝试想法。它通过向你提供一个提示符,你可以在其中输入代码来实现这一点。然后,它\emph{读取}你的输入,\emph{评估}它,\emph{打印}结果,并\emph{循环},然后再次向你提供一个提示符。
\switchcolumn[0]*
This process enables a quick feedback cycle that isn't possible in most
other languages. I strongly recommend that you use it frequently because
you'll be able to quickly check your understanding of Clojure as you
learn. Besides that, REPL development is an essential part of the Lisp
experience, and you'd really be missing out if you didn't use it.
\switchcolumn
这个过程实现了一个快速的反馈循环,在大多数其他语言中是不可能的。我强烈建议你经常使用它,因为在学习过程中,你可以快速检查你对Clojure的理解。除此之外,REPL开发是Lisp体验的一个重要组成部分,如果你不使用它,你会错过很多东西。
\switchcolumn[0]*
To start a REPL, run this:
\begin{verbatim}
lein repl
\end{verbatim}
\switchcolumn
要启动REPL,请运行以下命令:
\begin{verbatim}
lein repl
\end{verbatim}
\switchcolumn[0]*
The output should look like this:
\begin{verbatim}
nREPL server started on port 28925
REPL-y 0.1.10
Clojure 1.7.0
    Exit: Control+D or (exit) or (quit)
Commands: (user/help)
    Docs: (doc function-name-here)
            (find-doc "part-of-name-here")

    Source: (source function-name-here)
            (user/sourcery function-name-here)
    Javadoc: (javadoc java-object-or-class-here)
Examples from clojuredocs.org: [clojuredocs or cdoc]
            (user/clojuredocs name-here)
            (user/clojuredocs "ns-here" "name-here")
clojure-noob.core=>
\end{verbatim}
\switchcolumn
输出应该如下所示:
\begin{verbatim}
nREPL server started on port 28925
REPL-y 0.1.10
Clojure 1.7.0
    Exit: Control+D or (exit) or (quit)
Commands: (user/help)
    Docs: (doc function-name-here)
            (find-doc "part-of-name-here")

    Source: (source function-name-here)
            (user/sourcery function-name-here)
    Javadoc: (javadoc java-object-or-class-here)
Examples from clojuredocs.org: [clojuredocs or cdoc]
            (user/clojuredocs name-here)
            (user/clojuredocs "ns-here" "name-here")
clojure-noob.core=>
\end{verbatim}
\switchcolumn[0]*
The last line, clojure-noob.core=\textgreater, tells you that you're in
the clojure -noob.core namespace. You'll learn about namespaces later,
but for now notice that the namespace basically matches the name of your
\emph{src/clojure\_noob/core.clj} file. Also, notice that the REPL shows
the version as \emph{Clojure 1.7.0}, but as mentioned earlier,
everything will work okay no matter which version you use.
\switchcolumn
最后一行,\verb|clojure-noob.core=>|,告诉你当前位于clojure-noob.core命名空间。稍后你会学习有关命名空间的知识,但现在要注意的是命名空间基本上与你的\emph{src/clojure\_noob/core.clj}文件的名称匹配。另外,请注意REPL显示的版本为\emph{Clojure 1.7.0},但正如之前提到的,无论使用哪个版本都可以正常工作。
\switchcolumn[0]*
The prompt also indicates that your code is loaded in the REPL, and you
can execute the functions that are defined. Right now only one function,
-main, is defined. Go ahead and execute it now:
\begin{verbatim}
clojure-noob.core=> (-main)
I'm a little teapot!
nil
\end{verbatim}
\switchcolumn
提示还表示你的代码已加载到REPL中,并且可以执行定义的函数。目前只定义了一个函数-main。现在尝试执行它:
\begin{verbatim}
clojure-noob.core=> (-main)
I'm a little teapot!
nil
\end{verbatim}
\switchcolumn[0]*
Well done! You just used the REPL to evaluate a function call. Try a few
more basic Clojure functions:
\begin{verbatim}
clojure-noob.core=> (+ 1 2 3 4)
10
clojure-noob.core=> (* 1 2 3 4)
24
clojure-noob.core=> (first [1 2 3 4])
1
\end{verbatim}
\switchcolumn
干得好!你刚刚使用REPL评估了一个函数调用。尝试一些更基本的Clojure函数:
\begin{verbatim}
clojure-noob.core=> (+ 1 2 3 4)
10
clojure-noob.core=> (* 1 2 3 4)
24
clojure-noob.core=> (first [1 2 3 4])
1
\end{verbatim}
\switchcolumn[0]*
Awesome! You added some numbers, multiplied some numbers, and took the
first element from a vector. You also had your first encounter with
weird Lisp syntax! All Lisps, Clojure included, employ \emph{prefix
notation}, meaning that the operator always comes first in an
expression. If you're unsure about what that means, don't worry. You'll
learn all about Clojure's syntax soon.
\switchcolumn
太棒了!你进行了加法运算、乘法运算,并从向量中取出了第一个元素。你还第一次遇到了奇怪的Lisp语法!所有的Lisp语言,包括Clojure,在表达式中都使用\emph{前缀表示法},这意味着运算符始终位于表达式的第一位。如果你对此不确定,不用担心。你很快就会学习所有有关Clojure语法的知识。
\switchcolumn[0]*
Conceptually, the REPL is similar to Secure Shell (SSH). In the same way
that you can use SSH to interact with a remote server, the Clojure REPL
allows you to interact with a running Clojure process. This feature can
be very powerful because you can even attach a REPL to a live production
app and modify your program as it runs. For now, though, you'll be using
the REPL to build your knowledge of Clojure syntax and semantics.
\switchcolumn
从概念上讲,REPL类似于安全外壳(SSH)。就像你可以使用SSH与远程服务器进行交互一样,Clojure REPL允许你与运行中的Clojure进程进行交互。这个功能非常强大,因为你甚至可以将REPL连接到实时生产应用程序,并在程序运行时修改它。不过,现在你将使用REPL来建立对Clojure语法和语义的了解。
\switchcolumn[0]*
One more note: going forward, this book will present code without REPL
prompts, but please do try the code! Here's an example:
\begin{verbatim}
(do (println "no prompt here!")
    (+ 1 3))

; => no prompt here!
; => 4
\end{verbatim}
\switchcolumn
还有一点需要注意:今后,本书将呈现不带REPL提示的代码,但请确实尝试运行代码!这是一个例子:
\begin{verbatim}
(do (println "no prompt here!")
    (+ 1 3))

; => no prompt here!
; => 4
\end{verbatim}
\switchcolumn[0]*
When you see code snippets like this, lines that begin with ;
=\textgreater{} indicate the output of the code being run. In this case,
the text no prompt here should be printed, and the return value of the
code is 4.
\switchcolumn
当你看到像这样的代码片段时,以 ;=\textgreater{} 开头的行表示代码的输出结果。在这种情况下,应该打印出文本“这里没有提示!”,并且代码的返回值是4。

\switchcolumn[0]*
\section{Clojure Editors}
\switchcolumn
\section{Clojure编辑器}
\switchcolumn[0]*
At this point you should have the basic knowledge you need to begin
learning the Clojure language without having to fuss with an editor or
integrated development environment (IDE). But if you do want a good
tutorial on a powerful editor, \href{javascript:void(0)}{Chapter 2}
covers Emacs, the most popular editor among Clojurists. You absolutely
do not need to use Emacs for Clojure development, but Emacs offers tight
integration with the Clojure REPL and is well-suited to writing Lisp
code. What's most important, however, is that you use whatever works for
you.
\switchcolumn
到目前为止,你应该已经掌握了开始学习Clojure语言所需的基本知识,无需纠结于编辑器或集成开发环境(IDE)。但是,如果你确实想要一个关于强大编辑器的好教程,\href{javascript:void(0)}{第2章}介绍了Emacs,这是Clojure程序员中最流行的编辑器。你绝对不需要使用Emacs进行Clojure开发,但Emacs与Clojure REPL紧密集成,非常适合编写Lisp代码。然而,最重要的是使用适合自己的工具。
\switchcolumn[0]*
If Emacs isn't your cup of tea, here are some resources for setting up
other text editors and IDEs for Clojure development:
\switchcolumn
如果你不喜欢Emacs,下面是一些为Clojure开发设置其他文本编辑器和IDE的资源:

\begin{itemize}
\switchcolumn[0]*
\item This YouTube video will show you how to set up Sublime Text 2 for
Clojure development: \emph{http://www.youtube.com/watch?v=wBl0rYXQdGg/}.
\switchcolumn
\item  这个YouTube视频将向你展示如何在SublimeText 2上进行Clojure开发的设置:\emph{http://www.youtube.com/watch?v=wBl0rYXQdGg}。
\switchcolumn[0]*
\item Vim has good tools for Clojure development. This article is a good
starting point:
\emph{http://mybuddymichael.com/writings/writing-clojure-with-vim-in-2013.html}.
\switchcolumn
\item Vim有很好的Clojure开发工具。这篇文章是一个很好的起点:\emph{http://mybuddymichael.com/writings/writing-clojure-with-vim-in-2013.html}。
\switchcolumn[0]*
\item Counterclockwise is a highly recommended Eclipse plug-in:
\emph{https://github.com/laurentpetit/ccw/wiki/GoogleCodeHome}.
\switchcolumn
\item Counterclockwise是一个高度推荐的Eclipse插件:\emph{https://github.com/laurentpetit/ccw/wiki/GoogleCodeHome}。
\switchcolumn[0]*
\item Cursive Clojure is the recommended IDE for those who use IntelliJ:
\emph{https://cursiveclojure.com/}.
\switchcolumn
\item Cursive Clojure是那些使用IntelliJ的人推荐的IDE:\emph{https://cursiveclojure.com/}。
\switchcolumn[0]*
\item Nightcode is a simple, free IDE written in Clojure:
\emph{https://github.com/oakes/Nightcode/}.
\switchcolumn
\item Nightcode是一个简单的、免费的Clojure IDE:\emph{https://github.com/oakes/Nightcode/}。
\end{itemize}
\switchcolumn[0]*
\section{Summary}
\switchcolumn
\section{总结}
\switchcolumn[0]*
I'm so proud of you, little teapot. You've run your first Clojure
program! Not only that, but you've become acquainted with the REPL, one
of the most important tools for developing Clojure software. Amazing! It
brings to mind the immortal lines from ``Long Live'' by one of my
personal heroes:
\switchcolumn
小茶壶,我为你感到骄傲。你运行了你的第一个Clojure程序!不仅如此,你还熟悉了REPL,这是开发Clojure软件最重要的工具之一。太棒了!这让我想起了我个人英雄之一的泰勒·斯威夫特(Taylor Swift)在《Long Live》中永恒的诗句:
\switchcolumn[0]*
\begin{quote}
You held your head like a hero\\
On a history book page\\
It was the end of a decade\\
But the start of an age\\

\hfill ---Taylor Swift

Bravo!
\end{quote}
\switchcolumn
\begin{quote}
你像一个英雄一样昂首挺胸,\\
就如历史书页上的一页。\\
那是一个十年的结束,\\
却是一个新时代的开始。\\

\hfill ---泰勒·斯威夫特

太棒了!
\end{quote}
\end{paracol}

% %todo chapter2

% % Part II
% % Language Fundamentals

 

% % \backmatter
% \setcounter{chapter}{13}
% % \chapter{A Building and Developing with Leiningen}
\columnratio{0.55}
\begin{paracol}{2}
Writing software in any language involves generating \emph{artifacts},
which are executable files or library packages that are meant to be
deployed or shared. It also involves managing dependent artifacts, also
called \emph{dependencies}, by ensuring that they're loaded into the
project you're building. The most popular tool among Clojurists for
managing artifacts is Leiningen, and this appendix will show you how to
use it. You'll also learn how to use Leiningen to totally enhancify your
development experience with \emph{plug-ins}.
\switchcolumn
使用任何语言编写软件都涉及生成\emph{构件},这些构件是可执行文件或库包,用于部署或共享。同时,还需要管理依赖构件,也称为\emph{依赖项},以确保它们被加载到正在构建的项目中。在Clojure开发者中,最流行的管理构件的工具是Leiningen,本附录将向您展示如何使用它。您还将学习如何使用Leiningen通过\emph{插件}来完善您的开发体验。
%%%%%%%%%%%%%%%%%%%%%
\switchcolumn[0]*
\section{The Artifact Ecosystem}
\switchcolumn
\section{构件生态系统}
%%%%%%%%%%%%%%%%%%%%%
\switchcolumn[0]*
Because Clojure is hosted on the Java Virtual Machine (JVM), Clojure
artifacts are distributed as JAR files (covered in
\href{javascript:void(0)}{Chapter 12}). Java land already has an entire
artifact ecosystem for handling JAR files, and Clojure uses it.
\emph{Artifact ecosystem} isn't an official programming term; I use it
to refer to the suite of tools, resources, and conventions used to
identify and distribute artifacts. Java's ecosystem grew up around the
Maven build tool, and because Clojure uses this ecosystem, you'll often
see references to Maven. Maven is a huge tool that can perform all kinds
of wacky project management tasks. Thankfully, you don't need to get
your PhD in Mavenology to be an effective Clojurist. The only feature
you need to know is that Maven specifies a pattern for identifying
artifacts that Clojure projects adhere to, and it also specifies how to
host these artifacts in Maven \emph{repositories}, which are just
servers that store artifacts for distribution.
\switchcolumn
由于Clojure运行在Java虚拟机(JVM)上,因此Clojure构件以JAR文件的形式进行分发(在第12章中介绍)。Java领域已经有了一个完整的用于处理JAR文件的构件生态系统,并且Clojure也在使用它。\emph{构件生态系统}不是一个官方的编程术语;我用它来指代用于识别和分发构件的一套工具、资源和约定。Java的生态系统是围绕Maven构建工具发展起来的,因为Clojure使用了这个生态系统,您经常会看到对Maven的引用。Maven是一个功能强大的工具,可以执行各种奇怪的项目管理任务。但幸运的是,您不需要在Maven学院获得博士学位才能成为高效的Clojure开发者。您只需要知道Maven规定了一种用于识别构件的模式,Clojure项目遵循这种模式,并且规定了如何将这些构件托管在Maven\emph{仓库}中,这些仓库只是用于存储构件以进行分发的服务器。
%%%%%%%%%%%%%%%%%%%%%
\switchcolumn[0]*
\subsection{Identification}
\switchcolumn
\subsection{识别}
%%%%%%%%%%%%%%%%%%%%%
\switchcolumn[0]*
Maven artifacts need a \emph{group ID}, an \emph{artifact ID}, and a
\emph{version}. You can specify these for your project in the
\emph{project.clj} file. Here's what the first line of
\emph{project.clj} looks like for the clojure-noob project you created
in \href{javascript:void(0)}{Chapter 1}:
\begin{verbatim}
(defproject clojure-noob "0.1.0-SNAPSHOT"
\end{verbatim}
\switchcolumn
Maven构件需要一个\emph{组ID}、一个\emph{构件ID}和一个\emph{版本号}。您可以在\emph{project.clj}文件中为您的项目指定这些信息。以下是在\href{javascript:void(0)}{第1章}中创建的clojure-noob项目的\emph{project.clj}文件的第一行代码:
\begin{verbatim}
(defproject clojure-noob "0.1.0-SNAPSHOT"
\end{verbatim}
\switchcolumn[0]*
clojure-noob is both the group ID and the artifact ID of your project,
and ``0.1.0-SNAPSHOT'' is its version. In general, versions are permanent;
if you deploy an artifact with version 0.1.0 to a repository, you can't
make changes to the artifact and deploy it using the same version
number. You'll need to change the version number. (Many programmers like
the Semantic Versioning system, which you can read about at
\emph{http://semver.org/.}) If you want to indicate that the version is
a work in progress and you plan to keep updating it, you can append
-SNAPSHOT to your version number.
\switchcolumn
clojure-noob是您项目的组ID和构件ID,``0.1.0-SNAPSHOT'' 是其版本号。一般来说,版本是永久的;如果您使用版本号0.1.0将构件部署到存储库中,您不能对构件进行更改并使用相同的版本号再次部署。您需要更改版本号(许多程序员喜欢语义化版本控制系统,您可以在http://semver.org/上阅读相关信息)。如果您想表明版本是一个正在进行中的工作,并且计划继续更新它,您可以在版本号后面添加-SNAPSHOT。
\switchcolumn[0]*
If you want your group ID to be different from your artifact ID, you can
separate the two with a slash, like so:
\begin{verbatim}
(defproject group-id/artifact-id "0.1.0-SNAPSHOT"
\end{verbatim}
\switchcolumn
如果您希望组ID与构件ID不同,您可以使用斜杠将两者分开,如下所示:
\begin{verbatim}
(defproject group-id/artifact-id "0.1.0-SNAPSHOT"
\end{verbatim}
\switchcolumn[0]*
Often, developers will use their company name or their GitHub username
as the group ID.
\switchcolumn
通常,开发人员会使用公司名称或GitHub用户名作为组ID。
\switchcolumn[0]*
\subsubsection{Dependencies}
\switchcolumn
\subsubsection{依赖项}
\switchcolumn[0]*
Your \emph{project.clj} file also includes a line that looks like this,
which lists your project's dependencies:
\begin{verbatim}
:dependencies [[org.clojure/clojure "1.7.0"]]
\end{verbatim}
\switchcolumn
您的project.clj文件还包括以下行,其中列出了您项目的依赖项:
\begin{verbatim}
:dependencies [[org.clojure/clojure "1.7.0"]]
\end{verbatim}
\switchcolumn[0]*
If you want to use a library, add it to this dependency vector using the
same naming schema that you use to name your project. For example, if
you want to easily work with dates and times, you could add the clj-time
library, like this:
\begin{verbatim}
:dependencies [[org.clojure/clojure "1.7.0"]
                [clj-time "0.9.0"]]
\end{verbatim}
\switchcolumn
如果您想使用一个库,可以将其添加到此依赖项向量中,使用与命名项目相同的命名模式。例如,如果您想要轻松处理日期和时间,可以添加clj-time库,如下所示:
\begin{verbatim}
:dependencies [[org.clojure/clojure "1.7.0"]
                [clj-time "0.9.0"]]
\end{verbatim}
\switchcolumn[0]*
The next time you start your project, either by running it or by
starting a REPL, Leiningen will automatically download clj-time and make
it available within your project.
\switchcolumn
下次启动项目时,无论是运行还是启动REPL,Leiningen都会自动下载clj-time,并使其在您的项目中可用。
\switchcolumn[0]*
The Clojure community has created a multitude of useful libraries, and a
good place to look for them is the Clojure Toolbox at
\emph{http://www.clojure-toolbox.com}, which categorizes projects
according to their purpose. Nearly every Clojure library provides its
identifier at the top of its README, making it easy for you to figure
out how to add it to your Leiningen dependencies.
\switchcolumn
Clojure社区创建了大量有用的库,一个好的查找库的地方是Clojure Toolbox网站(http://www.clojure-toolbox.com),该网站根据库的用途对项目进行分类。几乎每个Clojure库都在其README的顶部提供了其标识符,让您很容易找出如何将其添加到Leiningen的依赖项中。
\switchcolumn[0]*
Sometimes you might want to use a Java library, but the identifier isn't
as readily available. If you want to add Apache Commons Email, for
example, you have to search online until you find a web page that
contains something like this:
\begin{verbatim}
<dependency>
    <groupId>org.apache.commons</groupId>
    <artifactId>commons-email</artifactId>
    <version>1.3.3</version>
</dependency>
\end{verbatim}
\switchcolumn
有时,您可能想使用一个Java库,但标识符不容易找到。例如,如果您想要添加Apache Commons Email库,您必须在网上搜索,直到找到一个包含以下内容的网页:
\begin{verbatim}
<dependency>
    <groupId>org.apache.commons</groupId>
    <artifactId>commons-email</artifactId>
    <version>1.3.3</version>
</dependency>
\end{verbatim}
\switchcolumn[0]*
This XML is how Java projects communicate their Maven identifier. To add
it your Clojure project, you'd change your :dependencies vector so it
looks like this:
\begin{verbatim}
:dependencies [[org.clojure/clojure "1.7.0"]
                [clj-time "0.9.0"]
                [org.apache.commons/commons-email "1.3.3"]]
\end{verbatim}
\switchcolumn
这个XML是Java项目用来传达它们的Maven标识符的方式。要将其添加到您的Clojure项目中,您需要更改:dependencies向量,使其如下所示:
\begin{verbatim}
:dependencies [[org.clojure/clojure "1.7.0"]
                [clj-time "0.9.0"]
                [org.apache.commons/commons-email "1.3.3"]]
\end{verbatim}
\switchcolumn[0]*
The main Clojure repository is Clojars (\emph{https://clojars.org/}),
and the main Java repository is The Central Repository
(\emph{http://search.maven.org/}), which is often referred to as just
\emph{Central} in the same way that San Francisco residents refer to San
Francisco as \emph{the city}. You can use these sites to find libraries
and their identifiers.
\switchcolumn
主要的Clojure仓库是Clojars(https://clojars.org/),主要的Java仓库是中央仓库(http://search.maven.org/),通常被称为“中央仓库”,就像旧金山居民称旧金山为“这个城市”一样。您可以使用这些网站来查找库和它们的标识符。
\switchcolumn[0]*
To deploy your own projects to Clojars, all you have to do is create an
account there and run lein deploy clojars in your project. This task
generates everything necessary for a Maven artifact to be stored in a
repository, including a POM file (which I won't go into) and a JAR file.
Then it uploads them to Clojars.
\switchcolumn
要将自己的项目部署到Clojars,您只需在该网站上创建一个帐户,并在项目中运行“lein deploy clojars”命令。这个任务会生成一切必要的东西,用于将Maven构件存储在仓库中,包括一个POM文件(我不会详细介绍)和一个JAR文件。然后将它们上传到Clojars。
\switchcolumn[0]*
\subsubsection{Plug-Ins}
\switchcolumn
\subsubsection{插件}
\switchcolumn[0]*
Leiningen lets you use \emph{plug-ins}, which are libraries that help
you when you're writing code. For example, the Eastwood plug-in is a
Clojure lint tool; it identifies poorly written code. You'll usually
want to specify your plug-ins in the file
\emph{\$HOME/.lein/profiles.clj}. To add Eastwood, you'd change
\emph{profiles.clj} to look like this:
\begin{verbatim}
{:user {:plugins [[jonase/eastwood "0.2.1"]] }}
\end{verbatim}
\switchcolumn
Leiningen允许您使用插件,这些插件是帮助您编写代码的库。例如,Eastwood插件是一个Clojure代码检查工具,它可以识别编写不良的代码。通常,您会在文件``\emph{\$HOME/.lein/profiles.clj}''中指定您的插件。要添加Eastwood,您可以将“profiles.clj”更改为以下内容:
\begin{verbatim}
{:user {:plugins [[jonase/eastwood "0.2.1"]] }}
\end{verbatim} 
\switchcolumn[0]*
This enables an eastwood Leiningen task for all your projects, which you
can run with lein eastwood at the project's root.

Leiningen's GitHub project page has excellent documentation on how to
use profiles and plug-ins, and it includes a handy list of plug-ins.
\switchcolumn
这样,您就可以在项目的根目录中使用“lein eastwood”命令运行eastwood Leiningen任务。

Leiningen的GitHub项目页面提供了关于如何使用配置文件和插件的优秀文档,并且还包括一个方便的插件列表。
\switchcolumn[0]*
\subsubsection{Summary}
\switchcolumn
\subsubsection{小结}
\switchcolumn[0]*
This appendix focused on the aspects of project management that are
important but that are difficult to find out about, like what Maven is
and Clojure's relationship to it. It showed you how to use Leiningen to
name your project, specify dependencies, and deploy to Clojars.
Leiningen offers a lot of functionality for software development tasks
that don't involve actually writing your code. If you want to find out
more, check out the Leiningen tutorial online at\\
\emph{https://github.com/technomancy/leiningen/blob/stable/doc/TUTORIAL.md/}.
\switchcolumn
本附录重点介绍了项目管理的重要方面,例如Maven是什么以及Clojure与Maven的关系。它向您展示了如何使用Leiningen为您的项目命名,指定依赖项并部署到Clojars。Leiningen提供了许多与软件开发任务相关的功能,而不仅仅是编写代码。如果您想了解更多信息,请查看Leiningen在线教程,网址为\\
\emph{https://github.com/technomancy/leiningen/blob/stable/doc/TUTORIAL.md/}。
\end{paracol}

% \setcounter{chapter}{14}
% % \chapter{Boot, the Fancy Clojure Build Framework}

\columnratio{0.55}
\begin{paracol}{2}
\switchcolumn[0]*
Boot is an alternative to Leiningen that provides the same
functionality. Leiningen's more popular (as of the summer of 2015), but
I personally like to work with Boot because it's easier to extend. This
appendix explains Boot's underlying concepts and guides you through
writing your first Boot tasks. If you're interested in using Boot to
build projects right this second, check out its GitHub README
(\emph{https://github.com/boot-clj/boot/}) and its wiki
(\emph{https://github.com/boot-clj/boot/wiki/}).
\switchcolumn
Boot是Leiningen的替代品,提供了相同的功能。虽然Leiningen更受欢迎(截至2015年夏季),但我个人更喜欢使用Boot,因为它更容易扩展。本附录解释了Boot的基本概念,并指导您编写第一个Boot任务。如果您有兴趣立即使用Boot构建项目,请查看其GitHub README(https://github.com/boot-clj/boot/)和其Wiki(https://github.com/boot-clj/boot/wiki/)。
\switchcolumn[0]*
\begin{quote}
\textbf{NOTE}:
\emph{As of this writing, Boot has limited support for Windows. The Boot
team welcomes contributions!}
\end{quote}
\switchcolumn
\begin{quote}
注意:截至本文写作时,Boot对Windows的支持有限。Boot团队欢迎贡献!
\end{quote}
\switchcolumn[0]*
\section{Boot's Abstractions}
\switchcolumn
\section{Boot的抽象}
\switchcolumn[0]*
Created by Micha Niskin and Alan Dipert, Boot is a fun and powerful
addition to the Clojure tooling landscape. On the surface, it's a
convenient way to build Clojure applications and run Clojure tasks from
the command line. Dig a little deeper and you'll see that Boot is like
the Lisped-up lovechild of Git and Unix in that it provides abstractions
that make it more pleasant to write code that exists at the intersection
of your operating system and your application.
\switchcolumn
由Micha Niskin和Alan Dipert创建,Boot是Clojure工具生态系统中有趣且强大的补充。从表面上看,它是一种方便的方式来构建Clojure应用程序并从命令行运行Clojure任务。深入挖掘,您会发现Boot就像Git和Unix的结合体,它提供了一些抽象,使得编写处于操作系统和应用程序交集处的代码更加愉快。
\switchcolumn[0]*
Unix provides abstractions that we're all familiar with to the point
where we take them for granted. (Would it kill you to take your computer
out to a nice restaurant once in a while?) The process abstraction lets
you reason about programs as isolated units of logic that can be easily
composed into a stream-processing pipeline through the STDIN and STDOUT
file descriptors. These abstractions make certain kinds of operations,
like text processing, very straightforward.
\switchcolumn
Unix提供了我们都熟悉的抽象,以至于我们认为它们理所当然。 (难道你不能偶尔带电脑去好餐厅吗?)进程抽象使您可以将程序视为独立的逻辑单元,通过STDIN和STDOUT文件描述符轻松组合成流处理管道。这些抽象使得某些操作(如文本处理)变得非常简单。
\switchcolumn[0]*
Similarly, Boot provides abstractions that make it easy to compose
independent operations into the kinds of complex, coordinated operations
that build tools end up doing, like converting ClojureScript into
JavaScript. Boot's task abstraction lets you easily define units of
logic that communicate through \emph{filesets}. The fileset abstraction
keeps track of the evolving build context and provides a well-defined,
reliable method of task coordination.
\switchcolumn
类似地,Boot提供了一些抽象,使得将独立操作组合成构建工具常常需要的复杂协调操作变得容易,比如将ClojureScript转换为JavaScript。Boot的任务抽象使您可以轻松定义通过文件集相互通信的逻辑单元。文件集抽象跟踪不断演变的构建上下文,并提供了一种明确定义的、可靠的任务协调方法。
\switchcolumn[0]*
That's a lot of high-level description, which hopefully has hooked your
attention. But I would be ashamed to leave you with a plateful of
metaphors. Oh no, dear reader, that was only the appetizer. Throughout
the rest of this appendix, you'll learn how to build your own Boot
tasks. Along the way, you'll discover that build tools can actually have
a conceptual foundation.
\switchcolumn
这是很多高层描述,希望能吸引您的注意。但是,如果我只给你上了一盘隐喻,我会感到羞愧的。不,亲爱的读者,那只是开胃菜。在本附录的其余部分,您将学习如何构建自己的Boot任务。在此过程中,您将发现构建工具实际上可以有一个概念基础。
\switchcolumn[0]*
\section{Tasks}
\switchcolumn
\section{任务}
\switchcolumn[0]*
Like make, rake, grunt, and other build tools of yore, Boot lets you
define tasks. \emph{Tasks} are named operations that take command line
options dispatched by some intermediary program (make, rake, Boot).
\switchcolumn
像make、rake、grunt和其他早期的构建工具一样,Boot允许您定义任务。任务是命名的操作,它们接受由某个中介程序(如make、rake、Boot)分派的命令行选项。
\switchcolumn[0]*
Boot provides the dispatching program, \emph{boot}, and a Clojure
library that makes it easy for you to define named operations and their
command line options with the deftask macro. To see what all the fuss is
about, let's create your first task. Normally, programming tutorials
encourage you to write code that prints ``Hello World,'' but I like my
examples to have real-world utility, so your task is to print ``My pants
are on fire!'' This information is objectively more useful. First,
install Boot; then create a new directory named \emph{boot-walkthrough},
navigate to that directory, create a file named \emph{build.boot,} and
write this:
\begin{verbatim}
(deftask fire
    "Prints 'My pants are on fire!'"
    []
    (println "My pants are on fire!"))
\end{verbatim}
\switchcolumn
Boot提供了分派程序boot和一个Clojure库,使用deftask宏可以轻松定义命名操作及其命令行选项。为了了解到底是怎么回事,让我们创建您的第一个任务。通常,编程教程鼓励您编写打印“Hello World”的代码,但我喜欢我的示例具有实际的用途,所以您的任务是打印“我的裤子着火了!”这个信息是客观上更有用的。首先,安装Boot;然后创建一个名为boot-walkthrough的新目录,进入该目录,创建一个名为build.boot的文件,并编写以下内容:
\begin{verbatim}
(deftask fire
    "Prints 'My pants are on fire!'"
    []
    (println "My pants are on fire!"))
\end{verbatim}
\switchcolumn[0]*
Now run this task from the command line with boot fire; you should see
the message you wrote printed to your terminal. This task demonstrates
two out of the three task components: the task is named (fire), and it's
dispatched by boot. This is super cool. You've essentially created a
Clojure shell script, stand-alone Clojure code that you can run from the
command line with ease. No \emph{project.clj}, directory structure, or
namespaces needed!
\switchcolumn
现在在命令行中使用boot fire运行此任务;您应该看到您编写的消息打印到终端上。这个任务演示了三个任务组件中的两个:任务有一个名称(fire),并且由boot分派。这非常酷。您实际上创建了一个Clojure shell脚本,一个独立的Clojure代码,可以轻松地从命令行运行。不需要project.clj、目录结构或命名空间!
\switchcolumn[0]*
Let's extend the example to demonstrate how you'd write command line
options:
\begin{verbatim}
(deftask fire
    "Announces that something is on fire"
    [t thing     THING str "The thing that's on fire"
    p pluralize       bool "Whether to pluralize"]
    (let [verb (if pluralize "are" "is")]
    (println "My" thing verb "on fire!")))
\end{verbatim}
\switchcolumn
让我们扩展示例以演示如何编写命令行选项:
\begin{verbatim}
(deftask fire
    "Announces that something is on fire"
    [t thing     THING str "The thing that's on fire"
    p pluralize       bool "Whether to pluralize"]
    (let [verb (if pluralize "are" "is")]
    (println "My" thing verb "on fire!")))
\end{verbatim}
\switchcolumn[0]*
Try running the task like so:
\begin{verbatim}
boot fire -t heart
# => My heart is on fire!


boot fire -t logs -p
# => My logs are on fire!
\end{verbatim}
\switchcolumn
尝试以以下方式运行任务:
\begin{verbatim}
boot fire -t heart
# => My heart is on fire!


boot fire -t logs -p
# => My logs are on fire!
\end{verbatim}
\switchcolumn[0]*
In the first instance, either you're newly in love or you need to be
rushed to the emergency room. In the second, you are a Boy Scout
awkwardly expressing your excitement over meeting the requirements for a
merit badge. In both instances, you were able to easily specify options
for the task.
\switchcolumn
在第一个示例中,要么您新恋爱了,要么您需要赶紧去急诊室。在第二个示例中,您是一个笨拙地表达自己对满足获得勋章要求的兴奋的童子军。在这两种情况下,您都可以轻松地为任务指定选项。
\switchcolumn[0]*
This refinement of the fire task introduced two command line options,
thing and pluralize. Both options are defined using a
\emph{domain-specific language (DSL)}. DSLs are their own topic, but
briefly, the term refers to mini-languages that you can use within a
larger program to write compact, expressive code for narrowly defined
domains (like defining options).
\switchcolumn
这个改进的fire任务引入了两个命令行选项thing和pluralize。这两个选项使用特定领域语言(DSL)进行定义。DSL是一个独立的主题,但简而言之,该术语指的是您可以在较大程序中使用的迷你语言,用于在狭义定义的领域(如定义选项)中编写紧凑、表达力强的代码。
\switchcolumn[0]*
In the option thing, t specifies its short name, and thing specifies its
long name. THING is a bit complicated, and I'll get to it in a second.
str specifies the option's type, and Boot uses that to validate the
argument and convert it. "The thing that's on fire" is the documentation
for the option. You can view a task's documentation in the terminal with
boot task-name -h:
\begin{verbatim}
boot fire -h
# Announces that something is on fire
#
# Options:
#   -h, --help        Print this help info.
#   -t, --thing THING Set the thing that's on fire to THING.
#   -p, --pluralize   Whether to pluralize
\end{verbatim}
\switchcolumn
在选项thing中,t指定了其短名称,thing指定了其长名称。THING有点复杂,我一会儿会解释。str指定了选项的类型,Boot使用它来验证参数并进行转换。"着火的东西"是该选项的文档。您可以在终端中使用boot任务名称 -h查看任务的文档。
\begin{verbatim}
boot fire -h
# Announces that something is on fire
#
# Options:
#   -h, --help        Print this help info.
#   -t, --thing THING Set the thing that's on fire to THING.
#   -p, --pluralize   Whether to pluralize
\end{verbatim}
\switchcolumn[0]*
Pretty groovy! Boot makes it very easy to write code that's meant to be
invoked from the command line.
\switchcolumn
非常棒!Boot使得编写可以从命令行调用的代码变得非常容易。
\switchcolumn[0]*
Now, let's look at THING. THING is an \emph{optarg}, and it indicates
that this option expects an argument. You don't have to include an
optarg when you're defining an option (notice that the pluralize option
has no optarg). The optarg doesn't have to correspond to the full name
of the option; you could replace THING with BILLY\_JOEL or whatever you
want and the task would work the same. You can also designate complex
options using the optarg. (Visit
\emph{https://github.com/boot-clj/boot/wiki/Task-Options-DSL\#complex-options}
for Boot's documentation on the subject.) Basically, complex options
allow you to specify that option arguments should be treated as maps,
sets, vectors, or even nested collections. It's pretty powerful.
\switchcolumn
现在,让我们来看看THING。THING是一个\emph{optarg},它表示该选项需要一个参数。在定义选项时,你不必包含optarg(注意到复数化选项没有optarg)。optarg不必对应于选项的全名;你可以将THING替换为BILLY\_JOEL或其他任何你想要的名称,任务将工作得一样好。你还可以使用optarg来指定复杂的选项。(请访问
\emph{https://github.com/boot-clj/boot/wiki/Task-Options-DSL\#complex-options}
,查看关于Boot的文档。)基本上,复杂选项允许你将选项参数处理为映射、集合、向量,甚至是嵌套集合。这非常强大。
\switchcolumn[0]*
Boot provides you with all the tools you could ask for to build command
line interfaces with Clojure. And you've only just started learning
about it!
\switchcolumn
Boot为你提供了构建Clojure命令行界面所需的所有工具。而你现在只是刚刚开始学习它!
\switchcolumn[0]*
\section{REPL}
\switchcolumn
\section{REPL}
\switchcolumn[0]*
Boot comes with a number of useful built-in tasks, including a REPL
task. Run boot repl to fire up that puppy. The Boot REPL is similar to
Leiningen's in that it handles loading your project code so you can play
around with it. You might not think this applies to the project you've
been writing because you've only written tasks, but you can actually run
tasks in the REPL (I've omitted the boot.user=\textgreater{} prompt).
You can specify options using a string:
\switchcolumn
Boot提供了许多有用的内置任务,包括一个REPL任务。运行boot repl来启动这个任务。Boot REPL与Leiningen的REPL类似,它会加载你的项目代码,这样你就可以对其进行操作。你可能认为这与你正在编写的项目无关,因为你只编写了任务,但实际上你可以在REPL中运行任务(我省略了boot.user=\textgreater{}提示符)。你可以使用字符串来指定选项:
\switchcolumn[0]*
\begin{verbatim}
(fire "-t" "NBA Jam guy")
; My NBA Jam guy is on fire!
; => nil
\end{verbatim}
\switchcolumn
\begin{verbatim}
(fire "-t" "NBA Jam guy")
; My NBA Jam guy is on fire!
; => nil
\end{verbatim}
\switchcolumn[0]*
Notice that the option's value comes right after the option.

You can also specify an option using a keyword:
\switchcolumn
注意选项的值紧跟在选项后面。

你还可以使用关键字来指定选项:
\switchcolumn[0]*
\begin{verbatim}
(fire :thing "NBA Jam guy")
; My NBA Jam guy is on fire!
; => nil
\end{verbatim}
\switchcolumn
\begin{verbatim}
(fire :thing "NBA Jam guy")
; My NBA Jam guy is on fire!
; => nil
\end{verbatim}
\switchcolumn[0]*
You can also combine options:
\switchcolumn
你还可以组合选项:
\switchcolumn[0]*
\begin{verbatim}
(fire "-p" "-t" "NBA Jam guys")
; My NBA Jam guys are on fire!
; => nil

(fire :pluralize true :thing "NBA Jam guys")
; My NBA Jam guys are on fire!
; => nil
\end{verbatim}
\switchcolumn
\begin{verbatim}
(fire "-p" "-t" "NBA Jam guys")
; My NBA Jam guys are on fire!
; => nil

(fire :pluralize true :thing "NBA Jam guys")
; My NBA Jam guys are on fire!
; => nil
\end{verbatim}
\switchcolumn[0]*
And of course, you can use deftask in the REPL as well---it's just
Clojure, after all. The takeaway is that Boot lets you interact with
your tasks as Clojure functions, because that's what they are.
\switchcolumn
当然,你也可以在REPL中使用deftask——毕竟它就是Clojure。关键是,Boot允许你将任务作为Clojure函数与之交互,因为它们本质上就是函数。
\switchcolumn[0]*
\section{Composition and Coordination}
\switchcolumn
\section{组合和协调}
\switchcolumn[0]*
If what you've seen so far was all that Boot had to offer, it'd be a
pretty swell tool, but it wouldn't be very different from other build
tools. One feature that sets Boot apart is how it lets you compose
tasks. For comparison's sake, here's an example Rake invocation (Rake is
the premier Ruby build tool):
\switchcolumn
如果到目前为止你所见到的就是Boot所提供的全部功能,那么它将是一个非常棒的工具,但并不与其他构建工具有多大区别。Boot的一个与众不同之处在于它允许你组合任务。为了进行比较,这里是一个Rake调用的例子(Rake是最好的Ruby构建工具):
\switchcolumn[0]*
\begin{verbatim}
rake db:create db:migrate db:seed
\end{verbatim}
\switchcolumn
\begin{verbatim}
rake db:create db:migrate db:seed
\end{verbatim}
\switchcolumn[0]*
This code will create a database, run migrations on it, and populate it
with seed data when run in a Rails project. However, worth noting is
that Rake doesn't provide any way for these tasks to communicate with
each other. Specifying multiple tasks is just a convenience, saving you
from having to run rake db:create; rake db:migrate; rake db:seed. If you
want to access the result of Task A within Task B, the build tool
doesn't help you; you have to manage that coordination yourself.
Usually, you'll do this by shoving the result of Task A into a special
place on the filesystem and then making sure Task B reads that special
place. This looks like programming with mutable, global variables, and
it's just as brittle.
\switchcolumn
当在Rails项目中运行时,此代码将创建一个数据库,在其上运行迁移,并用种子数据填充它。然而值得注意的是,Rake不提供任何方式让这些任务相互通信。指定多个任务只是一种方便,省去了运行rake db:create; rake db:migrate; rake db:seed的麻烦。如果你想在任务A中访问任务A的结果,构建工具并没有帮助你;你必须自己管理协调。通常,你会将任务A的结果放入文件系统的一个特殊位置,然后确保任务B读取该特殊位置。这看起来就像是使用可变的全局变量进行编程,同样脆弱。
\switchcolumn[0]*
\subsubsection{Handlers and Middleware}
\switchcolumn
\subsubsection{处理器和中间件}
\switchcolumn[0]*
Boot addresses this task communication problem by treating tasks as
\emph{middleware factories}. If you're familiar with Ring, Boot's tasks
work very similarly, so feel free to skip to
``\href{javascript:void(0)}{Tasks Are Middleware Factories}'' on
\href{javascript:void(0)}{page 287}. If you're not familiar with the
concept of middleware, allow me to explain! \emph{Middleware} refers to
a set of \emph{conventions} that programmers adhere to so they can
flexibly create domain-specific function pipelines. That's pretty dense,
so let's un-dense it. I'll discuss the \emph{flexible} part in this
section and cover \emph{domain-specific} in
``\href{javascript:void(0)}{Filesets}'' on
\href{javascript:void(0)}{page 288}.
\switchcolumn
Boot通过将任务视为\emph{中间件工厂}来解决这个任务通信问题。如果你熟悉Ring,Boot的任务工作方式非常相似,所以可以直接跳到第287页的``任务是中间件工厂''部分。如果你对中间件的概念不熟悉,让我来解释一下!\emph{中间件}是指程序员遵循的一组\emph{约定},以便能够灵活地创建特定于领域的函数流水线。这听起来有点复杂,让我们来简化一下。在本节中,我将讨论灵活性的部分,并在第288页的``文件集''中讨论\emph{特定于领域}的部分。
\switchcolumn[0]*
To understand how the middleware approach differs from run-of-the-mill
function composition, here's an example of composing everyday functions:
\begin{verbatim}
(def strinc (comp str inc))
(strinc 3)
; => "4"
\end{verbatim}
\switchcolumn
为了理解中间件方法与普通函数组合的不同之处,这里有一个组合常规函数的示例:
\begin{verbatim}
(def strinc (comp str inc))
(strinc 3)
; => "4"
\end{verbatim}
\switchcolumn[0]*
There's nothing interesting about this function composition. In fact,
this function composition is so unremarkable that it strains my
abilities as a writer to actually say anything about it. There are two
functions, each does its own thing, and now they've been composed into
one. Whoop-dee-doo!
\switchcolumn
这个函数组合并没有什么特别的。实际上,这个函数组合是如此平凡,以至于作为作者实际上无法对其说出任何有意义的话。有两个函数,每个函数都做自己的事情,现在它们被组合成一个函数。太棒了!
\switchcolumn[0]*
Middleware introduces an extra step to function composition, giving you
more flexibility in defining your function pipeline. Suppose, in the
preceding example, that you wanted to return "I don't like the number X"
for arbitrary numbers but return a string-ified number for everything
else. Here's how you could do that:
\switchcolumn
中间件在函数组合中引入了一个额外的步骤,使您能够更灵活地定义函数流水线。假设在前面的示例中,您想要对任意数字返回"I don't like the number X",但对其他所有内容返回字符串化的数字。您可以这样做:
\switchcolumn[0]*
\begin{verbatim}
(defn whiney-str
    [rejects]
    {:pre [(set? rejects)]}
    (fn [x]
    (if (rejects x)
        (str "I don't like " x)
        (str x))))

(def whiney-strinc (comp (whiney-str #{2}) inc))
(whiney-strinc 1)
; => "I don't like 2"
\end{verbatim}
\switchcolumn
\begin{verbatim}
(defn whiney-str
    [rejects]
    {:pre [(set? rejects)]}
    (fn [x]
    (if (rejects x)
        (str "I don't like " x)
        (str x))))

(def whiney-strinc (comp (whiney-str #{2}) inc))
(whiney-strinc 1)
; => "I don't like 2"
\end{verbatim}
\switchcolumn[0]*
Now let's take it one step further. What if you want to decide whether
or not to call inc in the first place? \href{javascript:void(0)}{Listing
B-1} shows how you could do that:
\begin{verbatim}
(defn whiney-middleware
    [next-handler rejects]
    {:pre [(set? rejects)]}
    (fn [x]
    (if (= x 1)
        "I'm not going to bother doing anything to that"
        (let [y (next-handler x)]
            (if (rejects y)
            (str "I don't like " y)
            (str y))))))

(def whiney-strinc (whiney-middleware inc #{2}))
(whiney-strinc 1)
; => "I don't like 2"
\end{verbatim}
\emph{Listing B-1: The middleware approach to function composition lets
you introduce choice}
\switchcolumn
现在让我们再进一步。如果您想要决定是否调用inc呢?\href{javascript:void(0)}{清单B-1}展示了如何实现:
\begin{verbatim}
(defn whiney-middleware
    [next-handler rejects]
    {:pre [(set? rejects)]}
    (fn [x]
    (if (= x 1)
        "I'm not going to bother doing anything to that"
        (let [y (next-handler x)]
            (if (rejects y)
            (str "I don't like " y)
            (str y))))))

(def whiney-strinc (whiney-middleware inc #{2}))
(whiney-strinc 1)
; => "I don't like 2"
\end{verbatim}
\emph{清单B-1:中间件方法可以引入选择}
\switchcolumn[0]*
Here, instead of using comp to create your function pipeline, you pass
the next function in the pipeline as the first argument to the
middleware function. In this case, you're passing inc as the first
argument to whiney-middleware as next-handler. whiney-middleware then
returns an anonymous function that closes over inc and has the ability
to choose whether to call it or not. You can see this choice at ➊.
\switchcolumn
在这里,您不是使用comp创建函数流水线,而是将下一个函数作为中间件函数的第一个参数传递。在这种情况下,您将inc作为下一个处理程序作为第一个参数传递给whiney-middleware作为next-handler。whiney-middleware然后返回一个匿名函数,该函数闭包inc并具有选择是否调用它的能力。您可以在➊处看到这个选择。
\switchcolumn[0]*
We say that a middleware takes a handler as its first argument and
returns a handler. In this example, whiney-middleware takes a handler as
its first argument, inc, and it returns another handler, the anonymous
function with x as its only argument. Middleware can also take extra
arguments, like rejects, that act as configuration. The result is that
the handler returned by the middleware can behave more flexibly (thanks
to configuration), and it has more control over the function pipeline
(because it can choose whether or not to call the next handler).
\switchcolumn
我们说中间件将处理程序作为其第一个参数并返回处理程序。在这个示例中,whiney-middleware将处理程序作为其第一个参数,即inc,并返回另一个处理程序,即只有x作为唯一参数的匿名函数。中间件还可以接受额外的参数,比如rejects,它们充当配置。结果是,由中间件返回的处理程序可以更灵活地行为(由于配置),并且它对函数流水线有更多的控制。


\end{paracol}

\end{document}

\paragraph{任务是中间件工厂}\label{header-n111}}

%todo 以后有空可以研究下
% % 
% % 



\end{document}
\begin{codeHtml}

\end{codeHtml}  

\begin{codeJs}
%
\end{codeJs}

\begin{codeHVue}

\end{codeHVue}
%%%%%%%
\begin{vueQuoteWarn}
\end{vueQuoteWarn}

\begin{vueQuote}
\end{vueQuote}









\switchcolumn[0]*%%%%%%%

\switchcolumn

\switchcolumn[0]*%%%%%%%

\switchcolumn

\switchcolumn[0]*%%%%%%%

\switchcolumn

\switchcolumn[0]*%%%%%%%

\switchcolumn

\switchcolumn[0]*%%%%%%%

\switchcolumn

\switchcolumn[0]*%%%%%%%

\switchcolumn

\switchcolumn[0]*%%%%%%%

\switchcolumn

\switchcolumn[0]*%%%%%%%

\switchcolumn

\switchcolumn[0]*%%%%%%%

\switchcolumn

\switchcolumn[0]*%%%%%%%

\switchcolumn

\switchcolumn[0]*%%%%%%%

\switchcolumn

\switchcolumn[0]*%%%%%%%

\switchcolumn

\switchcolumn[0]*%%%%%%%

\switchcolumn

\switchcolumn[0]*%%%%%%%

\switchcolumn

\switchcolumn[0]*%%%%%%%

\switchcolumn

\switchcolumn[0]*%%%%%%%

\switchcolumn

\switchcolumn[0]*%%%%%%%

\switchcolumn

\switchcolumn[0]*%%%%%%%

\switchcolumn

\switchcolumn[0]*%%%%%%%

\switchcolumn

\switchcolumn[0]*%%%%%%%

\switchcolumn

\switchcolumn[0]*%%%%%%%

\switchcolumn

\switchcolumn[0]*%%%%%%%

\switchcolumn

\switchcolumn[0]*%%%%%%%

\switchcolumn

\switchcolumn[0]*%%%%%%%

\switchcolumn

\switchcolumn[0]*%%%%%%%

\switchcolumn

\switchcolumn[0]*%%%%%%%

\switchcolumn

\switchcolumn[0]*%%%%%%%

\switchcolumn

\switchcolumn[0]*%%%%%%%

\switchcolumn

\switchcolumn[0]*%%%%%%%

\switchcolumn

\switchcolumn[0]*%%%%%%%

\switchcolumn

\switchcolumn[0]*%%%%%%%

\switchcolumn

\switchcolumn[0]*%%%%%%%

\switchcolumn

\switchcolumn[0]*%%%%%%%

\switchcolumn

\switchcolumn[0]*%%%%%%%

\switchcolumn

\switchcolumn[0]*%%%%%%%

\switchcolumn

\switchcolumn[0]*%%%%%%%

\switchcolumn

\switchcolumn[0]*%%%%%%%

\switchcolumn

\switchcolumn[0]*%%%%%%%

\switchcolumn


\switchcolumn[0]*%%%%%%%

\switchcolumn


\switchcolumn[0]*%%%%%%%

\switchcolumn

\switchcolumn[0]*%%%%%%%

\switchcolumn


\switchcolumn[0]*%%%%%%%

\switchcolumn

\switchcolumn[0]*%%%%%%%

\switchcolumn


\switchcolumn[0]*%%%%%%%

\switchcolumn

\switchcolumn[0]*%%%%%%%

\switchcolumn


\switchcolumn[0]*%%%%%%%

\switchcolumn

\switchcolumn[0]*%%%%%%%

\switchcolumn


\switchcolumn[0]*%%%%%%%

\switchcolumn


\switchcolumn[0]*%%%%%%%

\switchcolumn

\switchcolumn[0]*%%%%%%%

\switchcolumn


\switchcolumn[0]*%%%%%%%

\switchcolumn

\switchcolumn[0]*%%%%%%%

\switchcolumn


\switchcolumn[0]*%%%%%%%

\switchcolumn

\switchcolumn[0]*%%%%%%%

\switchcolumn


\switchcolumn[0]*%%%%%%%

\switchcolumn

