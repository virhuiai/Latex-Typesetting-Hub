% vue3官方文档.tex
\PassOptionsToPackage{no-math}{fontspec}%禁用了使用fontspec宏包中的数学字体功能。
\PassOptionsToPackage{AutoFakeBold=true,AutoFakeSlant=true}{xeCJK}%让xeCJK宏包自动产生伪粗体和伪斜体效果。

\documentclass{book}
\usepackage[heading=true
,scheme=chinese%中文方案
,fontset=none%不使用默认的字体设置
,space=auto%自动调整中英文间距
]{ctex}
\setCJKmainfont{FangZhengShuSong-GBK-1.ttf}[Path=/Users/virhuiai/hlProjects/Latex-Typesetting-Hub/font/方正/]%设置文本的中文有衬线字体
\setCJKsansfont{FangZhengHeiTi-GBK-1.ttf}[Path=/Users/virhuiai/hlProjects/Latex-Typesetting-Hub/font/方正/]%设置文本的中文无衬线字体为
\setCJKmonofont{FangZhengFangSong-GBK-1.ttf}[Path=/Users/virhuiai/hlProjects/Latex-Typesetting-Hub/font/方正/] %设置文本的中文等宽字体 

\setCJKfamilyfont{fontFangSong}{FangZhengFangSong-GBK-1.ttf}[Path=/Users/virhuiai/hlProjects/Latex-Typesetting-Hub/font/方正/]
\setCJKfamilyfont{fontKai}{FangZhengKaiTi-GBK-1.ttf}[Path=/Users/virhuiai/hlProjects/Latex-Typesetting-Hub/font/方正/]
\newcommand\fontKai{\CJKfamily{fontKai}}

% 支持音标的字体
\newfontfamily\fontGentiumPlus{GentiumPlus}[Path=/Users/virhuiai/hlProjects/Latex-Typesetting-Hub/font/免费商用英文/支持音标-GentiumPlus-6.200/,
Extension=.ttf,
UprightFont=*-Regular ,
BoldFont=*-Bold ,
ItalicFont=*-Italic,
BoldItalicFont = *-BoldItalic
]
\newfontfamily\fontGentiumBookPlus{GentiumBookPlus}[Path=/Users/virhuiai/hlProjects/Latex-Typesetting-Hub/font/免费商用英文/支持音标-GentiumPlus-6.200/,
Extension=.ttf,
UprightFont=*-Regular ,
BoldFont=*-Bold ,
ItalicFont=*-Italic,
BoldItalicFont = *-BoldItalic
]


\usepackage[a3paper,landscape]{geometry}
\usepackage{paracol}
\usepackage[all]{tcolorbox}
\usepackage{parskip}
\usepackage{calc,pifont}\newcounter{带圈文字}\newcommand\带圈文字[1]{\protect\setcounter{带圈文字}{171+#1}\protect\ding{\value{带圈文字}}}
% \带圈文字{1}
% \带圈文字{2}\end{document}
\usepackage{amssymb}%\checkmark
%如果你想在LaTeX中输入"✔"符号,你可以使用amssymb宏包提供的\checkmark命令。
% 在常见的Unicode字符集中,"✔"的编码为U+2714。这个字符可以在大多数现代字体和字符集中正确显示。
\parindent=0pt

% \usepackage{graphicx}
% \makeatletter
% \def\fps@figure{htbp}
% \makeatother
\begin{document}

\newminted[codeJs]{js}{frame=single,label={js}}
\newminted[codeHtml]{html}{frame=single,label={html}}
\newminted[codeVue]{html}{frame=single,label={vue}}
\tcbset{
    csh shell/.style={
    skin=bicolor,
    colback=black,colupper=green,colframe=yellow!75!black,
    fontupper=\tt,
    before upper=\textcolor{red}{\small\ttfamily\bfseries virhuiai \%>~}
    } 
}
\newtcolorbox{codeShell}{csh shell} 
\newtcblisting{codeShellMul}{colback=black,colupper=white,colframe=yellow!75!black, listing only,listing options={style=tcblatex,language=sh},
every listing line={\textcolor{red}{\small\ttfamily\bfseries virhuiai \$> }}}

\tcolorboxenvironment{quote}{blanker, borderline west={1mm}{0pt}{gray}}
 
% \tcbset{
%     csh console/.style={
%     skin=bicolor,
%     colback=black,colupper=green,colframe=yellow!75!black,
%     fontupper=\tt,
%     % before upper=\textcolor{red}{\small\ttfamily\bfseries virhuiai \%>~}
%     } 
% }
% \newtcolorbox{consoleCode}{csh console}
% 需要指定字体 \setCJKfamilyfont{fontFangSong}{FangZhengFangSong-GBK-1.ttf}[Path=/Users/virhuiai/hlProjects/Latex-Typesetting-Hub/font/方正/]
\newminted[codeConsole]{console}{frame=single,fontfamily=fontFangSong} 
%todo 符号 

\definecolor{vueQuoteBg}{RGB}{249, 249, 249}
\definecolor{vueQuoteFrame}{RGB}{101, 181, 135}
\definecolor{vueQuoteFrameWarn}{RGB}{247, 197, 72}


% \begin{codeVue}
% \end{codeVue}


\newtcolorbox{vueQuote}[2][]{colback=vueQuoteBg, colframe=vueQuoteFrame,fonttitle=\bfseries, enhanced,coltitle=black,
% attach boxed title to top center={yshift=-2mm},
attach title to upper={\par},
title={%\makebox[0pt]{\textcircled{i}\quad}
#2},#1} 
\newtcolorbox{vueQuoteWarn}[2][]{colback=vueQuoteBg, colframe=vueQuoteFrameWarn,fonttitle=\bfseries, enhanced,coltitle=black,
% attach boxed title to top center={yshift=-2mm},
attach title to upper={\par},
title={%\makebox[0pt]{\textcircled{i}\quad}
#2},#1} 
 



% \chapter{Getting Started\hfill 开始}
% \input{1_Getting_Started_introduction.tex} 
% \columnratio{0.55}
\begin{paracol}{2}
\switchcolumn[0]*%%%%%%%
\section{Quick Start}
\switchcolumn
\section{快速上手}
\switchcolumn[0]*%%%%%%%
\subsection{Try Vue Online}
\switchcolumn
\subsection{线上尝试 Vue}
\switchcolumn[0]*%%%%%%%
\begin{itemize}
\item
    To quickly get a taste of Vue, you can try it directly in our
    \href{https://play.vuejs.org/\#eNo9jcEKwjAMhl/lt5fpQYfXUQfefAMvvRQbddC1pUuHUPrudg4HIcmXjyRZXEM4zYlEJ+T0iEPgXjn6BB8Zhp46WUZWDjCa9f6w9kAkTtH9CRinV4fmRtZ63H20Ztesqiylphqy3R5UYBqD1UyVAPk+9zkvV1CKbCv9poMLiTEfR2/IXpSoXomqZLtti/IFwVtA9A==}{Playground}.
\item
    If you prefer a plain HTML setup without any build steps, you can use
    this \href{https://jsfiddle.net/yyx990803/2ke1ab0z/}{JSFiddle} as your
    starting point.
\item
    If you are already familiar with Node.js and the concept of build
    tools, you can also try a complete build setup right within your
    browser on \href{https://vite.new/vue}{StackBlitz}.
\end{itemize}
\switchcolumn
\begin{itemize}
\item
    想要快速体验
    Vue,你可以直接试试我们的\href{https://play.vuejs.org/\#eNo9jcEKwjAMhl/lt5fpQYfXUQfefAMvvRQbddC1pUuHUPrudg4HIcmXjyRZXEM4zYlEJ+T0iEPgXjn6BB8Zhp46WUZWDjCa9f6w9kAkTtH9CRinV4fmRtZ63H20Ztesqiylphqy3R5UYBqD1UyVAPk+9zkvV1CKbCv9poMLiTEfR2/IXpSoXomqZLtti/IFwVtA9A==}{演练场}。
\item
    如果你更喜欢不用任何构建的原始 HTML,可以使用
    \href{https://jsfiddle.net/yyx990803/2ke1ab0z/}{JSFiddle} 入门。
\item
    如果你已经比较熟悉 Node.js 和构建工具等概念,还可以直接在浏览器中打开
    \href{https://vite.new/vue}{StackBlitz} 来尝试完整的构建设置。
\end{itemize}
\switchcolumn[0]*%%%%%%%
\subsection{Creating a Vue Application}
\switchcolumn
\subsection{创建一个 Vue 应用}
\switchcolumn[0]*%%%%%%%
\begin{vueQuote}{Prerequisites}
\begin{itemize}
\item
    Familiarity with the command line
\item
    Install \href{https://nodejs.org/}{Node.js} version 16.0 or higher
\end{itemize}        
\end{vueQuote}    
\switchcolumn
\begin{vueQuote}{前提条件}
\begin{itemize}
\item
    熟悉命令行
\item
    已安装 16.0 或更高版本的 \href{https://nodejs.org/}{Node.js}
\end{itemize}
\end{vueQuote}    
\switchcolumn[0]*%%%%%%%
In this section we will introduce how to scaffold a Vue
\href{https://vuejs.org/guide/extras/ways-of-using-vue.html\#single-page-application-spa}{Single
Page Application} on your local machine. The created project will be
using a build setup based on \href{https://vitejs.dev/}{Vite} and allow
us to use Vue
\href{https://vuejs.org/guide/scaling-up/sfc.html}{Single-File
Components} (SFCs).
\switchcolumn
在本节中,我们将介绍如何在本地搭建 Vue
\href{https://cn.vuejs.org/guide/extras/ways-of-using-vue.html\#single-page-application-spa}{单页应用}。创建的项目将使用基于
\href{https://vitejs.dev/}{Vite} 的构建设置,并允许我们使用 Vue
的\href{https://cn.vuejs.org/guide/scaling-up/sfc.html}{单文件组件}
(SFC)。
\switchcolumn[0]*%%%%%%%
Make sure you have an up-to-date version of
\href{https://nodejs.org/}{Node.js} installed and your current working
directory is the one where you intend to create a project. Run the
following command in your command line (without the
\texttt{\textgreater{}} sign): 
\switchcolumn
确保你安装了最新版本的
\href{https://nodejs.org/}{Node.js},并且你的当前工作目录正是打算创建项目的目录。在命令行中运行以下命令
(不要带上 \texttt{\textgreater{}} 符号):
\switchcolumn[0]*%%%%%%%
\begin{codeShell}
npm create vue@latest
\end{codeShell}
\switchcolumn
\begin{codeShell}
npm create vue@latest
\end{codeShell}
\switchcolumn[0]*%%%%%%%
This command will install and execute
\href{https://github.com/vuejs/create-vue}{create-vue}, the official Vue
project scaffolding tool. You will be presented with prompts for several
optional features such as TypeScript and testing support:
\switchcolumn
这一指令将会安装并执行
\href{https://github.com/vuejs/create-vue}{create-vue},它是 Vue
官方的项目脚手架工具。你将会看到一些诸如 TypeScript
和测试支持之类的可选功能提示:
\switchcolumn[0]*%%%%%%%
\begin{codeConsole*}{escapeinside=||}
|\checkmark| Project name: … <your-project-name>
|\checkmark| Add TypeScript? … No / Yes
|\checkmark| Add JSX Support? … No / Yes
|\checkmark| Add Vue Router for Single Page Application development? … No / Yes
|\checkmark| Add Pinia for state management? … No / Yes
|\checkmark| Add Vitest for Unit testing? … No / Yes
|\checkmark| Add an End-to-End Testing Solution? … No / Cypress / Playwright
|\checkmark| Add ESLint for code quality? … No / Yes
|\checkmark| Add Prettier for code formatting? … No / Yes

Scaffolding project in ./<your-project-name>...
Done.
\end{codeConsole*}
\switchcolumn
\begin{codeConsole}

\end{codeConsole}    
\switchcolumn[0]*%%%%%%%

\switchcolumn

\switchcolumn[0]*%%%%%%%

\switchcolumn

\switchcolumn[0]*%%%%%%%

\switchcolumn

\switchcolumn[0]*%%%%%%%

\switchcolumn

\switchcolumn[0]*%%%%%%%

\switchcolumn

\switchcolumn[0]*%%%%%%%

\switchcolumn

\switchcolumn[0]*%%%%%%%

\switchcolumn

\switchcolumn[0]*%%%%%%%

\switchcolumn

\switchcolumn[0]*%%%%%%%

\switchcolumn

\switchcolumn[0]*%%%%%%%

\switchcolumn

\switchcolumn[0]*%%%%%%%

\switchcolumn

\switchcolumn[0]*%%%%%%%

\switchcolumn

\switchcolumn[0]*%%%%%%%

\switchcolumn

\switchcolumn[0]*%%%%%%%

\switchcolumn

\switchcolumn[0]*%%%%%%%

\switchcolumn

\switchcolumn[0]*%%%%%%%

\switchcolumn

\switchcolumn[0]*%%%%%%%

\switchcolumn

\switchcolumn[0]*%%%%%%%

\switchcolumn

\switchcolumn[0]*%%%%%%%

\switchcolumn

\switchcolumn[0]*%%%%%%%

\switchcolumn

\switchcolumn[0]*%%%%%%%

\switchcolumn

\switchcolumn[0]*%%%%%%%

\switchcolumn

\switchcolumn[0]*%%%%%%%

\switchcolumn

\switchcolumn[0]*%%%%%%%

\switchcolumn

\switchcolumn[0]*%%%%%%%

\switchcolumn

\switchcolumn[0]*%%%%%%%

\switchcolumn

\switchcolumn[0]*%%%%%%%

\switchcolumn

\switchcolumn[0]*%%%%%%%

\switchcolumn


\switchcolumn[0]*%%%%%%%

\switchcolumn


\switchcolumn[0]*%%%%%%%

\switchcolumn

\switchcolumn[0]*%%%%%%%

\switchcolumn


\switchcolumn[0]*%%%%%%%

\switchcolumn

\switchcolumn[0]*%%%%%%%

\switchcolumn


\switchcolumn[0]*%%%%%%%

\switchcolumn

\switchcolumn[0]*%%%%%%%

\switchcolumn


\switchcolumn[0]*%%%%%%%

\switchcolumn

\switchcolumn[0]*%%%%%%%

\switchcolumn


\switchcolumn[0]*%%%%%%%

\switchcolumn


\switchcolumn[0]*%%%%%%%

\switchcolumn

\switchcolumn[0]*%%%%%%%

\switchcolumn


\switchcolumn[0]*%%%%%%%

\switchcolumn

\switchcolumn[0]*%%%%%%%

\switchcolumn


\switchcolumn[0]*%%%%%%%

\switchcolumn

\switchcolumn[0]*%%%%%%%

\switchcolumn


\switchcolumn[0]*%%%%%%%

\switchcolumn


\end{paracol}
 

\chapter{Essentials\hfill 基础}
% \columnratio{0.55}
\begin{paracol}{2}
\switchcolumn[0]*%%%%%%%
\section{Creating a Vue Application}
\switchcolumn
\section{创建一个 Vue 应用}
\switchcolumn[0]*%%%%%%%
\subsection{The application instance}
\switchcolumn
\subsection{应用实例}
\switchcolumn[0]*%%%%%%%
Every Vue application starts by creating a new \textbf{application
instance} with the
\href{https://vuejs.org/api/application.html\#createapp}{\texttt{createApp}}
function:
\switchcolumn
每个 Vue 应用都是通过
\href{https://cn.vuejs.org/api/application.html\#createapp}{\texttt{createApp}}
函数创建一个新的 \textbf{应用实例}:
\switchcolumn[0]*%%%%%%%
\begin{codeJs}
import { createApp } from 'vue'

const app = createApp({
    /* root component options */
})
\end{codeJs}
\switchcolumn
\begin{codeJs}
import { createApp } from 'vue'

const app = createApp({
    /* root component options */
})
\end{codeJs}
\switchcolumn[0]*%%%%%%%
\subsection{The Root Component}
\switchcolumn
\subsection{根组件}
\switchcolumn[0]*%%%%%%%
The object we are passing into \texttt{createApp} is in fact a
component. Every app requires a "root component" that can contain other
components as its children.
\switchcolumn
我们传入 \texttt{createApp}
的对象实际上是一个组件,每个应用都需要一个``根组件'',其他组件将作为其子组件。
\switchcolumn[0]*%%%%%%%
If you are using Single-File Components, we typically import the root
component from another file:
\switchcolumn
如果你使用的是单文件组件,我们可以直接从另一个文件中导入根组件。
\switchcolumn[0]*%%%%%%%
\begin{codeJs}
import { createApp } from 'vue'
// import the root component App from a single-file component.
import App from './App.vue'

const app = createApp(App)
\end{codeJs}
\switchcolumn
\begin{codeJs}
import { createApp } from 'vue'
// import the root component App from a single-file component.
import App from './App.vue'

const app = createApp(App)
\end{codeJs}
\switchcolumn[0]*%%%%%%%
While many examples in this guide only need a single component, most
real applications are organized into a tree of nested, reusable
components. For example, a Todo application's component tree might look
like this:
\switchcolumn
虽然本指南中的许多示例只需要一个组件,但大多数真实的应用都是由一棵嵌套的、可重用的组件树组成的。例如,一个待办事项
(Todos) 应用的组件树可能是这样的:
\switchcolumn[0]*%%%%%%%
\begin{verbatim}
App (root component)
├─ TodoList
│  └─ TodoItem
│     ├─ TodoDeleteButton
│     └─ TodoEditButton
└─ TodoFooter
    ├─ TodoClearButton
    └─ TodoStatistics
\end{verbatim}
\switchcolumn
\begin{verbatim}
App (root component)
├─ TodoList
│  └─ TodoItem
│     ├─ TodoDeleteButton
│     └─ TodoEditButton
└─ TodoFooter
    ├─ TodoClearButton
    └─ TodoStatistics
\end{verbatim}
\switchcolumn[0]*%%%%%%%
In later sections of the guide, we will discuss how to define and
compose multiple components together. Before that, we will focus on what
happens inside a single component.
\switchcolumn
我们会在指南的后续章节中讨论如何定义和组合多个组件。在那之前,我们得先关注一个组件内到底发生了什么。
\switchcolumn[0]*%%%%%%%
\subsection{Mounting the App}
\switchcolumn
\subsection{挂载应用}
\switchcolumn[0]*%%%%%%%
An application instance won't render anything until its
\texttt{.mount()} method is called. It expects a "container" argument,
which can either be an actual DOM element or a selector string:
\switchcolumn
应用实例必须在调用了 \texttt{.mount()}
方法后才会渲染出来。该方法接收一个``容器''参数,可以是一个实际的 DOM
元素或是一个 CSS 选择器字符串:
\switchcolumn[0]*%%%%%%%
\begin{codeHtml}
<div id="app"></div>
\end{codeHtml}
\begin{codeJs}
app.mount('#app')
\end{codeJs}
\switchcolumn
\begin{codeHtml}
<div id="app"></div>
\end{codeHtml}
\begin{codeJs}
app.mount('#app')
\end{codeJs}
\switchcolumn[0]*%%%%%%%
The content of the app's root component will be rendered inside the
container element. The container element itself is not considered part
of the app.
\switchcolumn
应用根组件的内容将会被渲染在容器元素里面。容器元素自己将\textbf{不会}被视为应用的一部分。
%%%%%
\switchcolumn[0]*%%%%%%%
The \texttt{.mount()} method should always be called after all app
configurations and asset registrations are done. Also note that its
return value, unlike the asset registration methods, is the root
component instance instead of the application instance.
\switchcolumn
\texttt{.mount()}
方法应该始终在整个应用配置和资源注册完成后被调用。同时请注意,不同于其他资源注册方法,它的返回值是根组件实例而非应用实例。
\switchcolumn[0]*%%%%%%%
\subsubsection{In-DOM Root Component Template}
\switchcolumn
\subsubsection{DOM 中的根组件模板}
\switchcolumn[0]*%%%%%%%
The template for the root component is usually part of the component
itself, but it is also possible to provide the template separately by
writing it directly inside the mount container:
\switchcolumn
根组件的模板通常是组件本身的一部分,但也可以直接通过在挂载容器内编写模板来单独提供:

\switchcolumn[0]*%%%%%%%
\begin{codeHtml}
<div id="app">
<button @click="count++">{{ count }}</button>
</div>
\end{codeHtml}
\switchcolumn
\begin{codeHtml}
<div id="app">
<button @click="count++">{{ count }}</button>
</div>
\end{codeHtml}


\switchcolumn[0]*%%%%%%%
\begin{codeJs}
import { createApp } from 'vue'

const app = createApp({
    data() {
    return {
        count: 0
    }
    }
})

app.mount('#app')
\end{codeJs}
\switchcolumn
\begin{codeJs}
import { createApp } from 'vue'

const app = createApp({
    data() {
    return {
        count: 0
    }
    }
})

app.mount('#app')
\end{codeJs}


\switchcolumn[0]*%%%%%%%
Vue will automatically use the container's \texttt{innerHTML} as the
template if the root component does not already have a \texttt{template}
option.
\switchcolumn
当根组件没有设置 \texttt{template} 选项时,Vue 将自动使用容器的
\texttt{innerHTML} 作为模板。
\switchcolumn[0]*%%%%%%%
In-DOM templates are often used in applications that are
\href{https://vuejs.org/guide/quick-start.html\#using-vue-from-cdn}{using
Vue without a build step}. They can also be used in conjunction with
server-side frameworks, where the root template might be generated
dynamically by the server.
\switchcolumn
DOM
内模板通常用于\href{https://cn.vuejs.org/guide/quick-start.html\#using-vue-from-cdn}{无构建步骤}的
Vue
应用程序。它们也可以与服务器端框架一起使用,其中根模板可能是由服务器动态生成的。
\switchcolumn[0]*%%%%%%%
\subsection{App Configurations}
\switchcolumn
\subsection{应用配置}
\switchcolumn[0]*%%%%%%%
The application instance exposes a \texttt{.config} object that allows
us to configure a few app-level options, for example, defining an
app-level error handler that captures errors from all descendant
components:
\switchcolumn
应用实例会暴露一个 \texttt{.config}
对象允许我们配置一些应用级的选项,例如定义一个应用级的错误处理器,用来捕获所有子组件上的错误:

\switchcolumn[0]*%%%%%%%
\begin{codeJs}
app.config.errorHandler = (err) => {
/* handle error */
}
\end{codeJs}
\switchcolumn
\begin{codeJs}
    app.config.errorHandler = (err) => {
/* handle error */
}
\end{codeJs}

\switchcolumn[0]*%%%%%%%
The application instance also provides a few methods for registering
app-scoped assets. For example, registering a component:
\switchcolumn
应用实例还提供了一些方法来注册应用范围内可用的资源,例如注册一个组件:
\switchcolumn[0]*%%%%%%%
\begin{codeJs}
app.component('TodoDeleteButton', TodoDeleteButton)
\end{codeJs}
\switchcolumn
\begin{codeJs}
app.component('TodoDeleteButton', TodoDeleteButton)
\end{codeJs}

\switchcolumn[0]*%%%%%%%
This makes the \texttt{TodoDeleteButton} available for use anywhere in
our app. We will discuss registration for components and other types of
assets in later sections of the guide. You can also browse the full list
of application instance APIs in its
\href{https://vuejs.org/api/application.html}{API reference}.
\switchcolumn
这使得 \texttt{TodoDeleteButton}
在应用的任何地方都是可用的。我们会在指南的后续章节中讨论关于组件和其他资源的注册。你也可以在
\href{https://cn.vuejs.org/api/application.html}{API 参考}中浏览应用实例
API 的完整列表。
\switchcolumn[0]*%%%%%%%
Make sure to apply all app configurations before mounting the app!
\switchcolumn
确保在挂载应用实例之前完成所有应用配置!
\switchcolumn[0]*%%%%%%%
\subsection{Multiple application instances}
\switchcolumn
\subsection{多个应用实例}
\switchcolumn[0]*%%%%%%%
You are not limited to a single application instance on the same page.
The \texttt{createApp} API allows multiple Vue applications to co-exist
on the same page, each with its own scope for configuration and global
assets:
\switchcolumn
应用实例并不只限于一个。\texttt{createApp} API
允许你在同一个页面中创建多个共存的 Vue
应用,而且每个应用都拥有自己的用于配置和全局资源的作用域。

\switchcolumn[0]*%%%%%%%
\begin{codeJs}
const app1 = createApp({
    /* ... */
})
app1.mount('#container-1')

const app2 = createApp({
    /* ... */
})
app2.mount('#container-2')
\end{codeJs}
\switchcolumn
\begin{codeJs}
const app1 = createApp({
    /* ... */
})
app1.mount('#container-1')

const app2 = createApp({
    /* ... */
})
app2.mount('#container-2')
\end{codeJs}

\switchcolumn[0]*%%%%%%%
If you are using Vue to enhance server-rendered HTML and only need Vue
to control specific parts of a large page, avoid mounting a single Vue
application instance on the entire page. Instead, create multiple small
application instances and mount them on the elements they are
responsible for.
\switchcolumn
如果你正在使用 Vue 来增强服务端渲染 HTML,并且只想要 Vue
去控制一个大型页面中特殊的一小部分,应避免将一个单独的 Vue
应用实例挂载到整个页面上,而是应该创建多个小的应用实例,将它们分别挂载到所需的元素上去。
\end{paracol}

\columnratio{0.55}
\begin{paracol}{2}
\switchcolumn[0]*%%%%%%%
\section{Template Syntax}
\switchcolumn
\section{模板语法}
\switchcolumn[0]*%%%%%%%
Vue uses an HTML-based template syntax that allows you to declaratively
bind the rendered DOM to the underlying component instance's data. All
Vue templates are syntactically valid HTML that can be parsed by
spec-compliant browsers and HTML parsers.
\switchcolumn
Vue 使用一种基于 HTML
的模板语法,使我们能够声明式地将其组件实例的数据绑定到呈现的 DOM
上。所有的 Vue 模板都是语法层面合法的 HTML,可以被符合规范的浏览器和
HTML 解析器解析。
\switchcolumn[0]*%%%%%%%
Under the hood, Vue compiles the templates into highly-optimized
JavaScript code. Combined with the reactivity system, Vue can
intelligently figure out the minimal number of components to re-render
and apply the minimal amount of DOM manipulations when the app state
changes.
\switchcolumn
在底层机制中,Vue 会将模板编译成高度优化的 JavaScript
代码。结合响应式系统,当应用状态变更时,Vue
能够智能地推导出需要重新渲染的组件的最少数量,并应用最少的 DOM 操作。
\switchcolumn[0]*%%%%%%%
If you are familiar with Virtual DOM concepts and prefer the raw power
of JavaScript, you can also
\href{https://vuejs.org/guide/extras/render-function.html}{directly
write render functions} instead of templates, with optional JSX support.
However, do note that they do not enjoy the same level of compile-time
optimizations as templates.
\switchcolumn
如果你对虚拟 DOM 的概念比较熟悉,并且偏好直接使用
JavaScript,你也可以结合可选的 JSX
支持\href{https://cn.vuejs.org/guide/extras/render-function.html}{直接手写渲染函数}而不采用模板。但请注意,这将不会享受到和模板同等级别的编译时优化。
\switchcolumn[0]*%%%%%%%
\subsection{Text Interpolation}
\switchcolumn
\subsection{文本插值}
\switchcolumn[0]*%%%%%%%
The most basic form of data binding is text interpolation using the
"Mustache" syntax (double curly braces):
\switchcolumn
最基本的数据绑定形式是文本插值,它使用的是``Mustache''语法
(即双大括号):
\switchcolumn[0]*%%%%%%%
\begin{codeHtml*}{label=template}
<span>Message: {{ msg }}</span>
\end{codeHtml*}  
\switchcolumn
\begin{codeHtml*}{label=template}
<span>Message: {{ msg }}</span>
\end{codeHtml*}  

\switchcolumn[0]*%%%%%%%
The mustache tag will be replaced with the value of the \texttt{msg}
property
\href{https://vuejs.org/guide/essentials/reactivity-fundamentals.html\#declaring-reactive-state}{from
the corresponding component instance}. It will also be updated whenever
the \texttt{msg} property changes.
\switchcolumn
双大括号标签会被替换为\href{https://cn.vuejs.org/guide/essentials/reactivity-fundamentals.html\#declaring-reactive-state}{相应组件实例中}
\texttt{msg} 属性的值。同时每次 \texttt{msg} 属性更改时它也会同步更新。
\switchcolumn[0]*%%%%%%%
\subsection{Raw HTML}
\switchcolumn
\subsection{原始 HTML}
\switchcolumn[0]*%%%%%%%
The double mustaches interpret the data as plain text, not HTML. In
order to output real HTML, you will need to use the
\href{https://vuejs.org/api/built-in-directives.html\#v-html}{\texttt{v-html}
directive}:
\switchcolumn
双大括号会将数据解释为纯文本,而不是 HTML。若想插入 HTML,你需要使用
\href{https://cn.vuejs.org/api/built-in-directives.html\#v-html}{\texttt{v-html}
指令}:
\switchcolumn[0]*%%%%%%%
\begin{codeHtml}
<p>Using text interpolation: {{ rawHtml }}</p>
<p>Using v-html directive: <span v-html="rawHtml"></span></p>
\end{codeHtml}  
\switchcolumn
\begin{codeHtml}
<p>Using text interpolation: {{ rawHtml }}</p>
<p>Using v-html directive: <span v-html="rawHtml"></span></p>
\end{codeHtml}  
\switchcolumn[0]*%%%%%%%
\begin{vueQuote}{result}
Using text interpolation: This should be red.\\
Using v-html directive: {\textcolor{red}{This should be red.}}
\end{vueQuote}
\switchcolumn
\begin{vueQuote}{结果}
Using text interpolation: This should be red.\\
Using v-html directive: {\textcolor{red}{This should be red.}}
\end{vueQuote}

\switchcolumn[0]*%%%%%%%
Here we're encountering something new. The \texttt{v-html} attribute
you're seeing is called a \textbf{directive}. Directives are prefixed
with \texttt{v-} to indicate that they are special attributes provided
by Vue, and as you may have guessed, they apply special reactive
behavior to the rendered DOM. Here, we're basically saying "keep this
element's inner HTML up-to-date with the \texttt{rawHtml} property on
the current active instance."
\switchcolumn
这里我们遇到了一个新的概念。这里看到的 \texttt{v-html} attribute
被称为一个\textbf{指令}。指令由 \texttt{v-} 作为前缀,表明它们是一些由
Vue 提供的特殊 attribute,你可能已经猜到了,它们将为渲染的 DOM
应用特殊的响应式行为。这里我们做的事情简单来说就是:在当前组件实例上,将此元素的
innerHTML 与 \texttt{rawHtml} 属性保持同步。
\switchcolumn[0]*%%%%%%%
The contents of the \texttt{span} will be replaced with the value of the
\texttt{rawHtml} property, interpreted as plain HTML - data bindings are
ignored. Note that you cannot use \texttt{v-html} to compose template
partials, because Vue is not a string-based templating engine. Instead,
components are preferred as the fundamental unit for UI reuse and
composition.
\switchcolumn
\texttt{span} 的内容将会被替换为 \texttt{rawHtml} 属性的值,插值为纯
HTML------数据绑定将会被忽略。注意,你不能使用 \texttt{v-html}
来拼接组合模板,因为 Vue 不是一个基于字符串的模板引擎。在使用 Vue
时,应当使用组件作为 UI 重用和组合的基本单元。
\switchcolumn[0]*%%%%%%%
\begin{vueQuoteWarn}{Security Warning}
Dynamically rendering arbitrary HTML on your website can be very
dangerous because it can easily lead to
\href{https://en.wikipedia.org/wiki/Cross-site_scripting}{XSS
vulnerabilities}. Only use \texttt{v-html} on trusted content and
\textbf{never} on user-provided content.
\end{vueQuoteWarn}
\switchcolumn
\begin{vueQuoteWarn}{安全警告}
在网站上动态渲染任意 HTML 是非常危险的,因为这非常容易造成
\href{https://zh.wikipedia.org/wiki/跨網站指令碼}{XSS
漏洞}。请仅在内容安全可信时再使用
\texttt{v-html},并且\textbf{永远不要}使用用户提供的 HTML 内容。
\end{vueQuoteWarn}
\switchcolumn[0]*%%%%%%%
\subsection{Attribute Bindings}
\switchcolumn
\subsection{Attribute 绑定}
\switchcolumn[0]*%%%%%%%
Mustaches cannot be used inside HTML attributes. Instead, use a
\href{https://vuejs.org/api/built-in-directives.html\#v-bind}{\texttt{v-bind}
directive}:
\switchcolumn
双大括号不能在 HTML attributes 中使用。想要响应式地绑定一个
attribute,应该使用
\href{https://cn.vuejs.org/api/built-in-directives.html\#v-bind}{\texttt{v-bind}
指令}:

\switchcolumn[0]*%%%%%%%
\begin{codeHtml}
<div v-bind:id="dynamicId"></div>
\end{codeHtml}  
\switchcolumn
\begin{codeHtml}
<div v-bind:id="dynamicId"></div>
\end{codeHtml}  
\switchcolumn[0]*%%%%%%%
The \texttt{v-bind} directive instructs Vue to keep the element's
\texttt{id} attribute in sync with the component's \texttt{dynamicId}
property. If the bound value is \texttt{null} or \texttt{undefined},
then the attribute will be removed from the rendered element.
\switchcolumn
\texttt{v-bind} 指令指示 Vue 将元素的 \texttt{id} attribute 与组件的
\texttt{dynamicId} 属性保持一致。如果绑定的值是 \texttt{null} 或者
\texttt{undefined},那么该 attribute 将会从渲染的元素上移除。
\switchcolumn[0]*%%%%%%%
\subsubsection{Shorthand}
\switchcolumn
\subsubsection{简写}
\switchcolumn[0]*%%%%%%%
Because \texttt{v-bind} is so commonly used, it has a dedicated
shorthand syntax:
\switchcolumn
因为 \texttt{v-bind} 非常常用,我们提供了特定的简写语法:


\switchcolumn[0]*%%%%%%%
\begin{codeHtml}
<div :id="dynamicId"></div>
\end{codeHtml}  
\switchcolumn
\begin{codeHtml}
<div :id="dynamicId"></div>
\end{codeHtml}  
\switchcolumn[0]*%%%%%%%
Attributes that start with \texttt{:} may look a bit different from
normal HTML, but it is in fact a valid character for attribute names and
all Vue-supported browsers can parse it correctly. In addition, they do
not appear in the final rendered markup. The shorthand syntax is
optional, but you will likely appreciate it when you learn more about
its usage later.
\switchcolumn
开头为 \texttt{:} 的 attribute 可能和一般的 HTML attribute
看起来不太一样,但它的确是合法的 attribute 名称字符,并且所有支持 Vue
的浏览器都能正确解析它。此外,他们不会出现在最终渲染的 DOM
中。简写语法是可选的,但相信在你了解了它更多的用处后,你应该会更喜欢它。
\switchcolumn[0]*%%%%%%%
\begin{quote}
For the rest of the guide, we will be using the shorthand syntax in code
examples, as that's the most common usage for Vue developers.
\end{quote}
\switchcolumn
\begin{quote}
接下来的指引中,我们都将在示例中使用简写语法,因为这是在实际开发中更常见的用法。
\end{quote}
\switchcolumn[0]*%%%%%%%
\subsubsection{Boolean Attributes}
\switchcolumn
\subsubsection{布尔型 Attribute}
\switchcolumn[0]*%%%%%%%
\href{https://html.spec.whatwg.org/multipage/common-microsyntaxes.html\#boolean-attributes}{Boolean
attributes} are attributes that can indicate true / false values by
their presence on an element. For example,
\href{https://developer.mozilla.org/en-US/docs/Web/HTML/Attributes/disabled}{\texttt{disabled}}
is one of the most commonly used boolean attributes.
\switchcolumn
\href{https://developer.mozilla.org/zh-CN/docs/Web/HTML/Attributes\#布尔值属性}{布尔型
attribute} 依据 true / false 值来决定 attribute
是否应该存在于该元素上。\href{https://developer.mozilla.org/en-US/docs/Web/HTML/Attributes/disabled}{\texttt{disabled}}
就是最常见的例子之一。
\switchcolumn[0]*%%%%%%%
\texttt{v-bind} works a bit differently in this case:
\switchcolumn
\texttt{v-bind} 在这种场景下的行为略有不同:
\end{paracol}

 
% %% todo 带圈文字的处理,代码中的带圈文字可以改成引用...
% % % \part{Environment Setup}% \textbf{Part I}\\
% \textbf{第一部分}\\
% \textbf{环境设置}

\chapter{Building, Running, and the REPL}
% 构建、运行和REPL
\columnratio{0.55}
\begin{paracol}{2}
\switchcolumn[0]*
In this chapter, you'll invest a small amount of time up front to get
familiar with a quick, foolproof way to build and run Clojure programs.
It feels great to get a real program running. Reaching that milestone
frees you up to experiment, share your work, and gloat to your
colleagues who are still using last decade's languages. This will help
keep you motivated!
\switchcolumn
在本章中,您将花费一小部分时间来熟悉一种快速、可靠的构建和运行Clojure程序的方法。让一个真正的程序运行起来感觉很棒。达到这个里程碑将使您能够自由地进行实验,分享您的工作,并向那些仍在使用上个十年语言的同事炫耀。这将有助于保持您的动力!
\switchcolumn[0]*
You'll also learn how to instantly run code within a running Clojure
process using a \emph{Read-Eval-Print Loop (REPL)}, which allows you to
quickly test your understanding of the language and learn more
efficiently.
\switchcolumn
您还将学习如何使用\emph{Read-Eval-Print Loop (REPL)}在运行中的Clojure进程中立即运行代码,这样可以快速测试您对语言的理解并更有效地学习。
\switchcolumn[0]*
But first, I'll briefly introduce Clojure. Next, I'll cover Leiningen,
the de facto standard build tool for Clojure. By the end of the chapter,
you'll know how to do the following:
\switchcolumn
但首先,我将简要介绍Clojure。接下来,我将介绍Leiningen,Clojure的事实上的标准构建工具。在本章结束时,您将知道如何执行以下操作:

\begin{itemize}
\switchcolumn[0]*
\item Create a new Clojure project with Leiningen
\switchcolumn
\item 使用Leiningen创建一个新的Clojure项目
\switchcolumn[0]*
\item Build the project to create an executable JAR file
\switchcolumn
\item 构建项目以创建可执行的JAR文件
\switchcolumn[0]*
\item Execute the JAR file
\switchcolumn
\item 执行JAR文件
\switchcolumn[0]*
\item Execute code in a Clojure REPL
\switchcolumn
\item 在Clojure REPL中执行代码
\end{itemize}
\switchcolumn[0]*
\section{First Things First: What Is Clojure?}
\switchcolumn
\section{首先要明确的是:Clojure是什么?}
\switchcolumn[0]*
Clojure was forged in a mythic volcano by Rich Hickey. Using an alloy of
Lisp, functional programming, and a lock of his own epic hair, he
crafted a language that's delightful yet powerful. Its Lisp heritage
gives you the power to write code more expressively than is possible in
most non-Lisp languages, and its distinct take on functional programming
will sharpen your thinking as a programmer. Plus, Clojure gives you
better tools for tackling complex domains (like concurrent programming)
that are traditionally known to drive developers into years of therapy.
\switchcolumn
Clojure是由Rich Hickey在一个神话般的火山中锻造而成的。他使用了Lisp、函数式编程和他自己史诗般的头发的一缕合金,创造了一种既令人愉悦又强大的语言。它的Lisp传统使您能够以比大多数非Lisp语言更富有表现力的方式编写代码,并且它对函数式编程的独特理解将增强您作为程序员的思维能力。此外,Clojure为您提供了更好的工具来处理复杂的领域(如并发编程),而这些领域通常被认为会让开发人员陷入多年的治疗中。
\switchcolumn[0]*
When talking about Clojure, though, it's important to keep in mind the
distinction between the Clojure language and the Clojure compiler. The
Clojure language is a Lisp dialect with a functional emphasis whose
syntax and semantics are independent of any implementation. The compiler
is an executable JAR file, \emph{clojure.jar}, which takes code written
in the Clojure language and compiles it to Java Virtual Machine (JVM)
bytecode. You'll see \emph{Clojure} used to refer to both the language
and the compiler, which can be confusing if you're not aware that
they're separate things. But now that you're aware, you'll be fine.
\switchcolumn
然而,在谈论Clojure时,重要的是要区分Clojure语言和Clojure编译器之间的区别。Clojure语言是一个带有函数式强调的Lisp方言,其语法和语义与任何实现无关。编译器是一个可执行的JAR文件\emph{clojure.jar},它将用Clojure语言编写的代码编译为Java虚拟机(JVM)字节码。您会看到\emph{Clojure}一词既用于指代语言,也用于指代编译器,如果您不知道它们是分开的,这可能会让您感到困惑。但是现在您知道了,一切都会没事的。
\switchcolumn[0]*
This distinction is necessary because, unlike most programming languages
like Ruby, Python, C, and a bazillion others, Clojure is a \emph{hosted
language}. Clojure programs are executed within a JVM and rely on the
JVM for core features like threading and garbage collection. Clojure
also targets JavaScript and the Microsoft Common Language Runtime (CLR),
but this book only focuses on the JVM implementation.
\switchcolumn
之所以需要这种区分,是因为与大多数编程语言(如Ruby、Python、C等等)不同,Clojure是一种\emph{托管语言}。Clojure程序在JVM内执行,并依赖JVM提供核心功能,如线程和垃圾回收。Clojure还支持JavaScript和Microsoft Common Language Runtime (CLR),但本书只关注JVM实现。
\switchcolumn[0]*
We'll explore the relationship between Clojure and the JVM more later
on, but for now the main concepts you need to understand are these:
\switchcolumn
我们将在后面更详细地探讨Clojure与JVM之间的关系,但目前您需要了解的主要概念是:
\begin{itemize}
\switchcolumn[0]*
\item JVM processes execute Java bytecode.
\switchcolumn
\item JVM进程执行Java字节码。
\switchcolumn[0]*
\item Usually, the Java Compiler produces Java bytecode from Java source
code.
\switchcolumn
\item 通常,Java编译器会从Java源代码生成Java字节码。
\switchcolumn[0]*
\item JAR files are collections of Java bytecode.
\switchcolumn
JAR文件是Java字节码的集合。
\switchcolumn[0]*
\item Java programs are usually distributed as JAR files.
\switchcolumn
\item Java程序通常以JAR文件的形式分发。
\switchcolumn[0]*
\item The Java program \emph{clojure.jar} reads Clojure source code and
produces Java bytecode.
\switchcolumn
\item Java程序\emph{clojure.jar}读取Clojure源代码并生成Java字节码。
\switchcolumn[0]*
\item That Java bytecode is then executed by the same JVM process already
running \emph{clojure.jar}.
\switchcolumn
\item 然后,同一JVM进程中已经运行的\emph{clojure.jar}执行该Java字节码。
\end{itemize}
\switchcolumn[0]*
Clojure continues to evolve. As of this writing, it's at version 1.7.0,
and development is going strong. If you're reading this book in the far
future and Clojure has a higher version number, don't worry! This book
covers Clojure's fundamentals, which shouldn't change from one version
to the next. There's no need for your robot butler to return this book
to the bookstore.
\switchcolumn
Clojure不断发展。截至本书编写时,它的版本为1.7.0,并且开发工作正在进行中。如果你在未来的某个时间阅读本书,而Clojure的版本号更高,不要担心!本书涵盖了Clojure的基础知识,这些知识不应该随着版本变化而改变。你的机器人管家没有必要把本书退还给书店。
\switchcolumn[0]*
Now that you know what Clojure is, let's actually build a freakin'
Clojure program!
\switchcolumn
现在你知道Clojure是什么了,让我们来实际构建一个Clojure程序吧!
\switchcolumn[0]*
\section{Leiningen}
\switchcolumn
\section{Leiningen}
\switchcolumn[0]*
These days, most Clojurists use Leiningen to build and manage their
projects. You can read a full description of Leiningen in
\href{javascript:void(0)}{Appendix A}, but for now we'll focus on using
it for four tasks:
\switchcolumn
现在大多数Clojure开发人员使用Leiningen来构建和管理他们的项目。你可以在\href{javascript:void(0)}{附录A}中阅读关于Leiningen的完整描述,但现在我们将重点介绍使用Leiningen执行以下四个任务:
\begin{enumerate}
\switchcolumn[0]*
\item  Creating a new Clojure project
\switchcolumn
\item  创建一个新的Clojure项目
\switchcolumn[0]*
\item  Running the Clojure project
\switchcolumn
\item  运行Clojure项目
\switchcolumn[0]*
\item  Building the Clojure project
\switchcolumn
\item  构建Clojure项目
\switchcolumn[0]*
\item  Using the REPL
\switchcolumn
\item  使用REPL
\end{enumerate}

\switchcolumn[0]*
Before continuing, make sure you have Java version 1.6 or later
installed. You can check your version by running java -version in your
terminal, and download the latest Java Runtime Environment (JRE)\footnote{from
\emph{http://www.oracle.com/technetwork/java/javase/downloads/index.html}}.
Then, install Leiningen using the instructions on the Leiningen home
page at \emph{http://leiningen.org/} (Windows users, note there's a
Windows installer). When you install Leiningen, it automatically
downloads the Clojure compiler, \emph{clojure.jar}.
\switchcolumn
在继续之前,请确保你已经安装了Java版本1.6或更高版本。你可以在终端中运行java -version命令来检查你的版本,并下载\footnote{从\emph{http://www.oracle.com/technetwork/java/javase/downloads/index.html}}最新的Java Runtime Environment (JRE)。然后,按照\emph{http://leiningen.org/}主页上的说明安装Leiningen(Windows用户请注意,有一个Windows安装程序)。当你安装Leiningen时,它会自动下载Clojure编译器\emph{clojure.jar}。
\switchcolumn[0]*
\subsection{Creating a New Clojure Project}
\switchcolumn
\subsection{创建一个新的Clojure项目}
\switchcolumn[0]*
Creating a new Clojure project is very simple. A single Leiningen
command creates a project skeleton. Later, you'll learn how to do tasks
like incorporate Clojure libraries, but for now, these instructions will
enable you to execute the code you write.
\switchcolumn
创建一个新的Clojure项目非常简单。一个Leiningen命令就可以创建一个项目骨架。稍后,你将学习如何执行诸如整合Clojure库之类的任务,但现在这些说明将使你能够执行你编写的代码。
\switchcolumn[0]*
Go ahead and create your first Clojure project by typing the following
in your terminal:
\switchcolumn
在终端中输入以下命令,创建你的第一个Clojure项目:
\switchcolumn[0]*

\switchcolumn

\switchcolumn[0]*

\switchcolumn

\switchcolumn[0]*

\switchcolumn

\switchcolumn[0]*

\switchcolumn

\switchcolumn[0]*

\switchcolumn

\switchcolumn[0]*

\switchcolumn

\switchcolumn[0]*

\switchcolumn


\switchcolumn[0]*

\switchcolumn


\switchcolumn[0]*

\switchcolumn

\switchcolumn[0]*

\switchcolumn


\switchcolumn[0]*

\switchcolumn

\switchcolumn[0]*

\switchcolumn


\switchcolumn[0]*

\switchcolumn

\switchcolumn[0]*

\switchcolumn


\switchcolumn[0]*

\switchcolumn

\switchcolumn[0]*

\switchcolumn


\switchcolumn[0]*

\switchcolumn


\switchcolumn[0]*

\switchcolumn

\switchcolumn[0]*

\switchcolumn


\switchcolumn[0]*

\switchcolumn

\switchcolumn[0]*

\switchcolumn


\switchcolumn[0]*

\switchcolumn

\switchcolumn[0]*

\switchcolumn


\switchcolumn[0]*

\switchcolumn

\end{paracol}

% %todo chapter2

% % Part II
% % Language Fundamentals

 

% % \backmatter
% \setcounter{chapter}{13}
% % \chapter{A Building and Developing with Leiningen}
\columnratio{0.55}
\begin{paracol}{2}
Writing software in any language involves generating \emph{artifacts},
which are executable files or library packages that are meant to be
deployed or shared. It also involves managing dependent artifacts, also
called \emph{dependencies}, by ensuring that they're loaded into the
project you're building. The most popular tool among Clojurists for
managing artifacts is Leiningen, and this appendix will show you how to
use it. You'll also learn how to use Leiningen to totally enhancify your
development experience with \emph{plug-ins}.
\switchcolumn
使用任何语言编写软件都涉及生成\emph{构件},这些构件是可执行文件或库包,用于部署或共享。同时,还需要管理依赖构件,也称为\emph{依赖项},以确保它们被加载到正在构建的项目中。在Clojure开发者中,最流行的管理构件的工具是Leiningen,本附录将向您展示如何使用它。您还将学习如何使用Leiningen通过\emph{插件}来完善您的开发体验。
%%%%%%%%%%%%%%%%%%%%%
\switchcolumn[0]*
\section{The Artifact Ecosystem}
\switchcolumn
\section{构件生态系统}
%%%%%%%%%%%%%%%%%%%%%
\switchcolumn[0]*
Because Clojure is hosted on the Java Virtual Machine (JVM), Clojure
artifacts are distributed as JAR files (covered in
\href{javascript:void(0)}{Chapter 12}). Java land already has an entire
artifact ecosystem for handling JAR files, and Clojure uses it.
\emph{Artifact ecosystem} isn't an official programming term; I use it
to refer to the suite of tools, resources, and conventions used to
identify and distribute artifacts. Java's ecosystem grew up around the
Maven build tool, and because Clojure uses this ecosystem, you'll often
see references to Maven. Maven is a huge tool that can perform all kinds
of wacky project management tasks. Thankfully, you don't need to get
your PhD in Mavenology to be an effective Clojurist. The only feature
you need to know is that Maven specifies a pattern for identifying
artifacts that Clojure projects adhere to, and it also specifies how to
host these artifacts in Maven \emph{repositories}, which are just
servers that store artifacts for distribution.
\switchcolumn
由于Clojure运行在Java虚拟机(JVM)上,因此Clojure构件以JAR文件的形式进行分发(在第12章中介绍)。Java领域已经有了一个完整的用于处理JAR文件的构件生态系统,并且Clojure也在使用它。\emph{构件生态系统}不是一个官方的编程术语;我用它来指代用于识别和分发构件的一套工具、资源和约定。Java的生态系统是围绕Maven构建工具发展起来的,因为Clojure使用了这个生态系统,您经常会看到对Maven的引用。Maven是一个功能强大的工具,可以执行各种奇怪的项目管理任务。但幸运的是,您不需要在Maven学院获得博士学位才能成为高效的Clojure开发者。您只需要知道Maven规定了一种用于识别构件的模式,Clojure项目遵循这种模式,并且规定了如何将这些构件托管在Maven\emph{仓库}中,这些仓库只是用于存储构件以进行分发的服务器。
%%%%%%%%%%%%%%%%%%%%%
\switchcolumn[0]*
\subsection{Identification}
\switchcolumn
\subsection{识别}
%%%%%%%%%%%%%%%%%%%%%
\switchcolumn[0]*
Maven artifacts need a \emph{group ID}, an \emph{artifact ID}, and a
\emph{version}. You can specify these for your project in the
\emph{project.clj} file. Here's what the first line of
\emph{project.clj} looks like for the clojure-noob project you created
in \href{javascript:void(0)}{Chapter 1}:
\begin{verbatim}
(defproject clojure-noob "0.1.0-SNAPSHOT"
\end{verbatim}
\switchcolumn
Maven构件需要一个\emph{组ID}、一个\emph{构件ID}和一个\emph{版本号}。您可以在\emph{project.clj}文件中为您的项目指定这些信息。以下是在\href{javascript:void(0)}{第1章}中创建的clojure-noob项目的\emph{project.clj}文件的第一行代码:
\begin{verbatim}
(defproject clojure-noob "0.1.0-SNAPSHOT"
\end{verbatim}
\switchcolumn[0]*
clojure-noob is both the group ID and the artifact ID of your project,
and ``0.1.0-SNAPSHOT'' is its version. In general, versions are permanent;
if you deploy an artifact with version 0.1.0 to a repository, you can't
make changes to the artifact and deploy it using the same version
number. You'll need to change the version number. (Many programmers like
the Semantic Versioning system, which you can read about at
\emph{http://semver.org/.}) If you want to indicate that the version is
a work in progress and you plan to keep updating it, you can append
-SNAPSHOT to your version number.
\switchcolumn
clojure-noob是您项目的组ID和构件ID,``0.1.0-SNAPSHOT'' 是其版本号。一般来说,版本是永久的;如果您使用版本号0.1.0将构件部署到存储库中,您不能对构件进行更改并使用相同的版本号再次部署。您需要更改版本号(许多程序员喜欢语义化版本控制系统,您可以在http://semver.org/上阅读相关信息)。如果您想表明版本是一个正在进行中的工作,并且计划继续更新它,您可以在版本号后面添加-SNAPSHOT。
\switchcolumn[0]*
If you want your group ID to be different from your artifact ID, you can
separate the two with a slash, like so:
\begin{verbatim}
(defproject group-id/artifact-id "0.1.0-SNAPSHOT"
\end{verbatim}
\switchcolumn
如果您希望组ID与构件ID不同,您可以使用斜杠将两者分开,如下所示:
\begin{verbatim}
(defproject group-id/artifact-id "0.1.0-SNAPSHOT"
\end{verbatim}
\switchcolumn[0]*
Often, developers will use their company name or their GitHub username
as the group ID.
\switchcolumn
通常,开发人员会使用公司名称或GitHub用户名作为组ID。
\switchcolumn[0]*
\subsubsection{Dependencies}
\switchcolumn
\subsubsection{依赖项}
\switchcolumn[0]*
Your \emph{project.clj} file also includes a line that looks like this,
which lists your project's dependencies:
\begin{verbatim}
:dependencies [[org.clojure/clojure "1.7.0"]]
\end{verbatim}
\switchcolumn
您的project.clj文件还包括以下行,其中列出了您项目的依赖项:
\begin{verbatim}
:dependencies [[org.clojure/clojure "1.7.0"]]
\end{verbatim}
\switchcolumn[0]*
If you want to use a library, add it to this dependency vector using the
same naming schema that you use to name your project. For example, if
you want to easily work with dates and times, you could add the clj-time
library, like this:
\begin{verbatim}
:dependencies [[org.clojure/clojure "1.7.0"]
                [clj-time "0.9.0"]]
\end{verbatim}
\switchcolumn
如果您想使用一个库,可以将其添加到此依赖项向量中,使用与命名项目相同的命名模式。例如,如果您想要轻松处理日期和时间,可以添加clj-time库,如下所示:
\begin{verbatim}
:dependencies [[org.clojure/clojure "1.7.0"]
                [clj-time "0.9.0"]]
\end{verbatim}
\switchcolumn[0]*
The next time you start your project, either by running it or by
starting a REPL, Leiningen will automatically download clj-time and make
it available within your project.
\switchcolumn
下次启动项目时,无论是运行还是启动REPL,Leiningen都会自动下载clj-time,并使其在您的项目中可用。
\switchcolumn[0]*
The Clojure community has created a multitude of useful libraries, and a
good place to look for them is the Clojure Toolbox at
\emph{http://www.clojure-toolbox.com}, which categorizes projects
according to their purpose. Nearly every Clojure library provides its
identifier at the top of its README, making it easy for you to figure
out how to add it to your Leiningen dependencies.
\switchcolumn
Clojure社区创建了大量有用的库,一个好的查找库的地方是Clojure Toolbox网站(http://www.clojure-toolbox.com),该网站根据库的用途对项目进行分类。几乎每个Clojure库都在其README的顶部提供了其标识符,让您很容易找出如何将其添加到Leiningen的依赖项中。
\switchcolumn[0]*
Sometimes you might want to use a Java library, but the identifier isn't
as readily available. If you want to add Apache Commons Email, for
example, you have to search online until you find a web page that
contains something like this:
\begin{verbatim}
<dependency>
    <groupId>org.apache.commons</groupId>
    <artifactId>commons-email</artifactId>
    <version>1.3.3</version>
</dependency>
\end{verbatim}
\switchcolumn
有时,您可能想使用一个Java库,但标识符不容易找到。例如,如果您想要添加Apache Commons Email库,您必须在网上搜索,直到找到一个包含以下内容的网页:
\begin{verbatim}
<dependency>
    <groupId>org.apache.commons</groupId>
    <artifactId>commons-email</artifactId>
    <version>1.3.3</version>
</dependency>
\end{verbatim}
\switchcolumn[0]*
This XML is how Java projects communicate their Maven identifier. To add
it your Clojure project, you'd change your :dependencies vector so it
looks like this:
\begin{verbatim}
:dependencies [[org.clojure/clojure "1.7.0"]
                [clj-time "0.9.0"]
                [org.apache.commons/commons-email "1.3.3"]]
\end{verbatim}
\switchcolumn
这个XML是Java项目用来传达它们的Maven标识符的方式。要将其添加到您的Clojure项目中,您需要更改:dependencies向量,使其如下所示:
\begin{verbatim}
:dependencies [[org.clojure/clojure "1.7.0"]
                [clj-time "0.9.0"]
                [org.apache.commons/commons-email "1.3.3"]]
\end{verbatim}
\switchcolumn[0]*
The main Clojure repository is Clojars (\emph{https://clojars.org/}),
and the main Java repository is The Central Repository
(\emph{http://search.maven.org/}), which is often referred to as just
\emph{Central} in the same way that San Francisco residents refer to San
Francisco as \emph{the city}. You can use these sites to find libraries
and their identifiers.
\switchcolumn
主要的Clojure仓库是Clojars(https://clojars.org/),主要的Java仓库是中央仓库(http://search.maven.org/),通常被称为“中央仓库”,就像旧金山居民称旧金山为“这个城市”一样。您可以使用这些网站来查找库和它们的标识符。
\switchcolumn[0]*
To deploy your own projects to Clojars, all you have to do is create an
account there and run lein deploy clojars in your project. This task
generates everything necessary for a Maven artifact to be stored in a
repository, including a POM file (which I won't go into) and a JAR file.
Then it uploads them to Clojars.
\switchcolumn
要将自己的项目部署到Clojars,您只需在该网站上创建一个帐户,并在项目中运行“lein deploy clojars”命令。这个任务会生成一切必要的东西,用于将Maven构件存储在仓库中,包括一个POM文件(我不会详细介绍)和一个JAR文件。然后将它们上传到Clojars。
\switchcolumn[0]*
\subsubsection{Plug-Ins}
\switchcolumn
\subsubsection{插件}
\switchcolumn[0]*
Leiningen lets you use \emph{plug-ins}, which are libraries that help
you when you're writing code. For example, the Eastwood plug-in is a
Clojure lint tool; it identifies poorly written code. You'll usually
want to specify your plug-ins in the file
\emph{\$HOME/.lein/profiles.clj}. To add Eastwood, you'd change
\emph{profiles.clj} to look like this:
\begin{verbatim}
{:user {:plugins [[jonase/eastwood "0.2.1"]] }}
\end{verbatim}
\switchcolumn
Leiningen允许您使用插件,这些插件是帮助您编写代码的库。例如,Eastwood插件是一个Clojure代码检查工具,它可以识别编写不良的代码。通常,您会在文件``\emph{\$HOME/.lein/profiles.clj}''中指定您的插件。要添加Eastwood,您可以将“profiles.clj”更改为以下内容:
\begin{verbatim}
{:user {:plugins [[jonase/eastwood "0.2.1"]] }}
\end{verbatim} 
\switchcolumn[0]*
This enables an eastwood Leiningen task for all your projects, which you
can run with lein eastwood at the project's root.

Leiningen's GitHub project page has excellent documentation on how to
use profiles and plug-ins, and it includes a handy list of plug-ins.
\switchcolumn
这样,您就可以在项目的根目录中使用“lein eastwood”命令运行eastwood Leiningen任务。

Leiningen的GitHub项目页面提供了关于如何使用配置文件和插件的优秀文档,并且还包括一个方便的插件列表。
\switchcolumn[0]*
\subsubsection{Summary}
\switchcolumn
\subsubsection{小结}
\switchcolumn[0]*
This appendix focused on the aspects of project management that are
important but that are difficult to find out about, like what Maven is
and Clojure's relationship to it. It showed you how to use Leiningen to
name your project, specify dependencies, and deploy to Clojars.
Leiningen offers a lot of functionality for software development tasks
that don't involve actually writing your code. If you want to find out
more, check out the Leiningen tutorial online at\\
\emph{https://github.com/technomancy/leiningen/blob/stable/doc/TUTORIAL.md/}.
\switchcolumn
本附录重点介绍了项目管理的重要方面,例如Maven是什么以及Clojure与Maven的关系。它向您展示了如何使用Leiningen为您的项目命名,指定依赖项并部署到Clojars。Leiningen提供了许多与软件开发任务相关的功能,而不仅仅是编写代码。如果您想了解更多信息,请查看Leiningen在线教程,网址为\\
\emph{https://github.com/technomancy/leiningen/blob/stable/doc/TUTORIAL.md/}。
\end{paracol}

% \setcounter{chapter}{14}
% % \chapter{Boot, the Fancy Clojure Build Framework}

\columnratio{0.55}
\begin{paracol}{2}
\switchcolumn[0]*
Boot is an alternative to Leiningen that provides the same
functionality. Leiningen's more popular (as of the summer of 2015), but
I personally like to work with Boot because it's easier to extend. This
appendix explains Boot's underlying concepts and guides you through
writing your first Boot tasks. If you're interested in using Boot to
build projects right this second, check out its GitHub README
(\emph{https://github.com/boot-clj/boot/}) and its wiki
(\emph{https://github.com/boot-clj/boot/wiki/}).
\switchcolumn
Boot是Leiningen的替代品,提供了相同的功能。虽然Leiningen更受欢迎(截至2015年夏季),但我个人更喜欢使用Boot,因为它更容易扩展。本附录解释了Boot的基本概念,并指导您编写第一个Boot任务。如果您有兴趣立即使用Boot构建项目,请查看其GitHub README(https://github.com/boot-clj/boot/)和其Wiki(https://github.com/boot-clj/boot/wiki/)。
\switchcolumn[0]*
\begin{quote}
\textbf{NOTE}:
\emph{As of this writing, Boot has limited support for Windows. The Boot
team welcomes contributions!}
\end{quote}
\switchcolumn
\begin{quote}
注意:截至本文写作时,Boot对Windows的支持有限。Boot团队欢迎贡献!
\end{quote}
\switchcolumn[0]*
\section{Boot's Abstractions}
\switchcolumn
\section{Boot的抽象}
\switchcolumn[0]*
Created by Micha Niskin and Alan Dipert, Boot is a fun and powerful
addition to the Clojure tooling landscape. On the surface, it's a
convenient way to build Clojure applications and run Clojure tasks from
the command line. Dig a little deeper and you'll see that Boot is like
the Lisped-up lovechild of Git and Unix in that it provides abstractions
that make it more pleasant to write code that exists at the intersection
of your operating system and your application.
\switchcolumn
由Micha Niskin和Alan Dipert创建,Boot是Clojure工具生态系统中有趣且强大的补充。从表面上看,它是一种方便的方式来构建Clojure应用程序并从命令行运行Clojure任务。深入挖掘,您会发现Boot就像Git和Unix的结合体,它提供了一些抽象,使得编写处于操作系统和应用程序交集处的代码更加愉快。
\switchcolumn[0]*
Unix provides abstractions that we're all familiar with to the point
where we take them for granted. (Would it kill you to take your computer
out to a nice restaurant once in a while?) The process abstraction lets
you reason about programs as isolated units of logic that can be easily
composed into a stream-processing pipeline through the STDIN and STDOUT
file descriptors. These abstractions make certain kinds of operations,
like text processing, very straightforward.
\switchcolumn
Unix提供了我们都熟悉的抽象,以至于我们认为它们理所当然。 (难道你不能偶尔带电脑去好餐厅吗?)进程抽象使您可以将程序视为独立的逻辑单元,通过STDIN和STDOUT文件描述符轻松组合成流处理管道。这些抽象使得某些操作(如文本处理)变得非常简单。
\switchcolumn[0]*
Similarly, Boot provides abstractions that make it easy to compose
independent operations into the kinds of complex, coordinated operations
that build tools end up doing, like converting ClojureScript into
JavaScript. Boot's task abstraction lets you easily define units of
logic that communicate through \emph{filesets}. The fileset abstraction
keeps track of the evolving build context and provides a well-defined,
reliable method of task coordination.
\switchcolumn
类似地,Boot提供了一些抽象,使得将独立操作组合成构建工具常常需要的复杂协调操作变得容易,比如将ClojureScript转换为JavaScript。Boot的任务抽象使您可以轻松定义通过文件集相互通信的逻辑单元。文件集抽象跟踪不断演变的构建上下文,并提供了一种明确定义的、可靠的任务协调方法。
\switchcolumn[0]*
That's a lot of high-level description, which hopefully has hooked your
attention. But I would be ashamed to leave you with a plateful of
metaphors. Oh no, dear reader, that was only the appetizer. Throughout
the rest of this appendix, you'll learn how to build your own Boot
tasks. Along the way, you'll discover that build tools can actually have
a conceptual foundation.
\switchcolumn
这是很多高层描述,希望能吸引您的注意。但是,如果我只给你上了一盘隐喻,我会感到羞愧的。不,亲爱的读者,那只是开胃菜。在本附录的其余部分,您将学习如何构建自己的Boot任务。在此过程中,您将发现构建工具实际上可以有一个概念基础。
\switchcolumn[0]*
\section{Tasks}
\switchcolumn
\section{任务}
\switchcolumn[0]*
Like make, rake, grunt, and other build tools of yore, Boot lets you
define tasks. \emph{Tasks} are named operations that take command line
options dispatched by some intermediary program (make, rake, Boot).
\switchcolumn
像make、rake、grunt和其他早期的构建工具一样,Boot允许您定义任务。任务是命名的操作,它们接受由某个中介程序(如make、rake、Boot)分派的命令行选项。
\switchcolumn[0]*
Boot provides the dispatching program, \emph{boot}, and a Clojure
library that makes it easy for you to define named operations and their
command line options with the deftask macro. To see what all the fuss is
about, let's create your first task. Normally, programming tutorials
encourage you to write code that prints ``Hello World,'' but I like my
examples to have real-world utility, so your task is to print ``My pants
are on fire!'' This information is objectively more useful. First,
install Boot; then create a new directory named \emph{boot-walkthrough},
navigate to that directory, create a file named \emph{build.boot,} and
write this:
\begin{verbatim}
(deftask fire
    "Prints 'My pants are on fire!'"
    []
    (println "My pants are on fire!"))
\end{verbatim}
\switchcolumn
Boot提供了分派程序boot和一个Clojure库,使用deftask宏可以轻松定义命名操作及其命令行选项。为了了解到底是怎么回事,让我们创建您的第一个任务。通常,编程教程鼓励您编写打印“Hello World”的代码,但我喜欢我的示例具有实际的用途,所以您的任务是打印“我的裤子着火了!”这个信息是客观上更有用的。首先,安装Boot;然后创建一个名为boot-walkthrough的新目录,进入该目录,创建一个名为build.boot的文件,并编写以下内容:
\begin{verbatim}
(deftask fire
    "Prints 'My pants are on fire!'"
    []
    (println "My pants are on fire!"))
\end{verbatim}
\switchcolumn[0]*
Now run this task from the command line with boot fire; you should see
the message you wrote printed to your terminal. This task demonstrates
two out of the three task components: the task is named (fire), and it's
dispatched by boot. This is super cool. You've essentially created a
Clojure shell script, stand-alone Clojure code that you can run from the
command line with ease. No \emph{project.clj}, directory structure, or
namespaces needed!
\switchcolumn
现在在命令行中使用boot fire运行此任务;您应该看到您编写的消息打印到终端上。这个任务演示了三个任务组件中的两个:任务有一个名称(fire),并且由boot分派。这非常酷。您实际上创建了一个Clojure shell脚本,一个独立的Clojure代码,可以轻松地从命令行运行。不需要project.clj、目录结构或命名空间!
\switchcolumn[0]*
Let's extend the example to demonstrate how you'd write command line
options:
\begin{verbatim}
(deftask fire
    "Announces that something is on fire"
    [t thing     THING str "The thing that's on fire"
    p pluralize       bool "Whether to pluralize"]
    (let [verb (if pluralize "are" "is")]
    (println "My" thing verb "on fire!")))
\end{verbatim}
\switchcolumn
让我们扩展示例以演示如何编写命令行选项:
\begin{verbatim}
(deftask fire
    "Announces that something is on fire"
    [t thing     THING str "The thing that's on fire"
    p pluralize       bool "Whether to pluralize"]
    (let [verb (if pluralize "are" "is")]
    (println "My" thing verb "on fire!")))
\end{verbatim}
\switchcolumn[0]*
Try running the task like so:
\begin{verbatim}
boot fire -t heart
# => My heart is on fire!


boot fire -t logs -p
# => My logs are on fire!
\end{verbatim}
\switchcolumn
尝试以以下方式运行任务:
\begin{verbatim}
boot fire -t heart
# => My heart is on fire!


boot fire -t logs -p
# => My logs are on fire!
\end{verbatim}
\switchcolumn[0]*
In the first instance, either you're newly in love or you need to be
rushed to the emergency room. In the second, you are a Boy Scout
awkwardly expressing your excitement over meeting the requirements for a
merit badge. In both instances, you were able to easily specify options
for the task.
\switchcolumn
在第一个示例中,要么您新恋爱了,要么您需要赶紧去急诊室。在第二个示例中,您是一个笨拙地表达自己对满足获得勋章要求的兴奋的童子军。在这两种情况下,您都可以轻松地为任务指定选项。
\switchcolumn[0]*
This refinement of the fire task introduced two command line options,
thing and pluralize. Both options are defined using a
\emph{domain-specific language (DSL)}. DSLs are their own topic, but
briefly, the term refers to mini-languages that you can use within a
larger program to write compact, expressive code for narrowly defined
domains (like defining options).
\switchcolumn
这个改进的fire任务引入了两个命令行选项thing和pluralize。这两个选项使用特定领域语言(DSL)进行定义。DSL是一个独立的主题,但简而言之,该术语指的是您可以在较大程序中使用的迷你语言,用于在狭义定义的领域(如定义选项)中编写紧凑、表达力强的代码。
\switchcolumn[0]*
In the option thing, t specifies its short name, and thing specifies its
long name. THING is a bit complicated, and I'll get to it in a second.
str specifies the option's type, and Boot uses that to validate the
argument and convert it. "The thing that's on fire" is the documentation
for the option. You can view a task's documentation in the terminal with
boot task-name -h:
\begin{verbatim}
boot fire -h
# Announces that something is on fire
#
# Options:
#   -h, --help        Print this help info.
#   -t, --thing THING Set the thing that's on fire to THING.
#   -p, --pluralize   Whether to pluralize
\end{verbatim}
\switchcolumn
在选项thing中,t指定了其短名称,thing指定了其长名称。THING有点复杂,我一会儿会解释。str指定了选项的类型,Boot使用它来验证参数并进行转换。"着火的东西"是该选项的文档。您可以在终端中使用boot任务名称 -h查看任务的文档。
\begin{verbatim}
boot fire -h
# Announces that something is on fire
#
# Options:
#   -h, --help        Print this help info.
#   -t, --thing THING Set the thing that's on fire to THING.
#   -p, --pluralize   Whether to pluralize
\end{verbatim}
\switchcolumn[0]*
Pretty groovy! Boot makes it very easy to write code that's meant to be
invoked from the command line.
\switchcolumn
非常棒!Boot使得编写可以从命令行调用的代码变得非常容易。
\switchcolumn[0]*
Now, let's look at THING. THING is an \emph{optarg}, and it indicates
that this option expects an argument. You don't have to include an
optarg when you're defining an option (notice that the pluralize option
has no optarg). The optarg doesn't have to correspond to the full name
of the option; you could replace THING with BILLY\_JOEL or whatever you
want and the task would work the same. You can also designate complex
options using the optarg. (Visit
\emph{https://github.com/boot-clj/boot/wiki/Task-Options-DSL\#complex-options}
for Boot's documentation on the subject.) Basically, complex options
allow you to specify that option arguments should be treated as maps,
sets, vectors, or even nested collections. It's pretty powerful.
\switchcolumn
现在,让我们来看看THING。THING是一个\emph{optarg},它表示该选项需要一个参数。在定义选项时,你不必包含optarg(注意到复数化选项没有optarg)。optarg不必对应于选项的全名;你可以将THING替换为BILLY\_JOEL或其他任何你想要的名称,任务将工作得一样好。你还可以使用optarg来指定复杂的选项。(请访问
\emph{https://github.com/boot-clj/boot/wiki/Task-Options-DSL\#complex-options}
,查看关于Boot的文档。)基本上,复杂选项允许你将选项参数处理为映射、集合、向量,甚至是嵌套集合。这非常强大。
\switchcolumn[0]*
Boot provides you with all the tools you could ask for to build command
line interfaces with Clojure. And you've only just started learning
about it!
\switchcolumn
Boot为你提供了构建Clojure命令行界面所需的所有工具。而你现在只是刚刚开始学习它!
\switchcolumn[0]*
\section{REPL}
\switchcolumn
\section{REPL}
\switchcolumn[0]*
Boot comes with a number of useful built-in tasks, including a REPL
task. Run boot repl to fire up that puppy. The Boot REPL is similar to
Leiningen's in that it handles loading your project code so you can play
around with it. You might not think this applies to the project you've
been writing because you've only written tasks, but you can actually run
tasks in the REPL (I've omitted the boot.user=\textgreater{} prompt).
You can specify options using a string:
\switchcolumn
Boot提供了许多有用的内置任务,包括一个REPL任务。运行boot repl来启动这个任务。Boot REPL与Leiningen的REPL类似,它会加载你的项目代码,这样你就可以对其进行操作。你可能认为这与你正在编写的项目无关,因为你只编写了任务,但实际上你可以在REPL中运行任务(我省略了boot.user=\textgreater{}提示符)。你可以使用字符串来指定选项:
\switchcolumn[0]*
\begin{verbatim}
(fire "-t" "NBA Jam guy")
; My NBA Jam guy is on fire!
; => nil
\end{verbatim}
\switchcolumn
\begin{verbatim}
(fire "-t" "NBA Jam guy")
; My NBA Jam guy is on fire!
; => nil
\end{verbatim}
\switchcolumn[0]*
Notice that the option's value comes right after the option.

You can also specify an option using a keyword:
\switchcolumn
注意选项的值紧跟在选项后面。

你还可以使用关键字来指定选项:
\switchcolumn[0]*
\begin{verbatim}
(fire :thing "NBA Jam guy")
; My NBA Jam guy is on fire!
; => nil
\end{verbatim}
\switchcolumn
\begin{verbatim}
(fire :thing "NBA Jam guy")
; My NBA Jam guy is on fire!
; => nil
\end{verbatim}
\switchcolumn[0]*
You can also combine options:
\switchcolumn
你还可以组合选项:
\switchcolumn[0]*
\begin{verbatim}
(fire "-p" "-t" "NBA Jam guys")
; My NBA Jam guys are on fire!
; => nil

(fire :pluralize true :thing "NBA Jam guys")
; My NBA Jam guys are on fire!
; => nil
\end{verbatim}
\switchcolumn
\begin{verbatim}
(fire "-p" "-t" "NBA Jam guys")
; My NBA Jam guys are on fire!
; => nil

(fire :pluralize true :thing "NBA Jam guys")
; My NBA Jam guys are on fire!
; => nil
\end{verbatim}
\switchcolumn[0]*
And of course, you can use deftask in the REPL as well---it's just
Clojure, after all. The takeaway is that Boot lets you interact with
your tasks as Clojure functions, because that's what they are.
\switchcolumn
当然,你也可以在REPL中使用deftask——毕竟它就是Clojure。关键是,Boot允许你将任务作为Clojure函数与之交互,因为它们本质上就是函数。
\switchcolumn[0]*
\section{Composition and Coordination}
\switchcolumn
\section{组合和协调}
\switchcolumn[0]*
If what you've seen so far was all that Boot had to offer, it'd be a
pretty swell tool, but it wouldn't be very different from other build
tools. One feature that sets Boot apart is how it lets you compose
tasks. For comparison's sake, here's an example Rake invocation (Rake is
the premier Ruby build tool):
\switchcolumn
如果到目前为止你所见到的就是Boot所提供的全部功能,那么它将是一个非常棒的工具,但并不与其他构建工具有多大区别。Boot的一个与众不同之处在于它允许你组合任务。为了进行比较,这里是一个Rake调用的例子(Rake是最好的Ruby构建工具):
\switchcolumn[0]*
\begin{verbatim}
rake db:create db:migrate db:seed
\end{verbatim}
\switchcolumn
\begin{verbatim}
rake db:create db:migrate db:seed
\end{verbatim}
\switchcolumn[0]*
This code will create a database, run migrations on it, and populate it
with seed data when run in a Rails project. However, worth noting is
that Rake doesn't provide any way for these tasks to communicate with
each other. Specifying multiple tasks is just a convenience, saving you
from having to run rake db:create; rake db:migrate; rake db:seed. If you
want to access the result of Task A within Task B, the build tool
doesn't help you; you have to manage that coordination yourself.
Usually, you'll do this by shoving the result of Task A into a special
place on the filesystem and then making sure Task B reads that special
place. This looks like programming with mutable, global variables, and
it's just as brittle.
\switchcolumn
当在Rails项目中运行时,此代码将创建一个数据库,在其上运行迁移,并用种子数据填充它。然而值得注意的是,Rake不提供任何方式让这些任务相互通信。指定多个任务只是一种方便,省去了运行rake db:create; rake db:migrate; rake db:seed的麻烦。如果你想在任务A中访问任务A的结果,构建工具并没有帮助你;你必须自己管理协调。通常,你会将任务A的结果放入文件系统的一个特殊位置,然后确保任务B读取该特殊位置。这看起来就像是使用可变的全局变量进行编程,同样脆弱。
\switchcolumn[0]*
\subsubsection{Handlers and Middleware}
\switchcolumn
\subsubsection{处理器和中间件}
\switchcolumn[0]*
Boot addresses this task communication problem by treating tasks as
\emph{middleware factories}. If you're familiar with Ring, Boot's tasks
work very similarly, so feel free to skip to
``\href{javascript:void(0)}{Tasks Are Middleware Factories}'' on
\href{javascript:void(0)}{page 287}. If you're not familiar with the
concept of middleware, allow me to explain! \emph{Middleware} refers to
a set of \emph{conventions} that programmers adhere to so they can
flexibly create domain-specific function pipelines. That's pretty dense,
so let's un-dense it. I'll discuss the \emph{flexible} part in this
section and cover \emph{domain-specific} in
``\href{javascript:void(0)}{Filesets}'' on
\href{javascript:void(0)}{page 288}.
\switchcolumn
Boot通过将任务视为\emph{中间件工厂}来解决这个任务通信问题。如果你熟悉Ring,Boot的任务工作方式非常相似,所以可以直接跳到第287页的``任务是中间件工厂''部分。如果你对中间件的概念不熟悉,让我来解释一下!\emph{中间件}是指程序员遵循的一组\emph{约定},以便能够灵活地创建特定于领域的函数流水线。这听起来有点复杂,让我们来简化一下。在本节中,我将讨论灵活性的部分,并在第288页的``文件集''中讨论\emph{特定于领域}的部分。
\switchcolumn[0]*
To understand how the middleware approach differs from run-of-the-mill
function composition, here's an example of composing everyday functions:
\begin{verbatim}
(def strinc (comp str inc))
(strinc 3)
; => "4"
\end{verbatim}
\switchcolumn
为了理解中间件方法与普通函数组合的不同之处,这里有一个组合常规函数的示例:
\begin{verbatim}
(def strinc (comp str inc))
(strinc 3)
; => "4"
\end{verbatim}
\switchcolumn[0]*
There's nothing interesting about this function composition. In fact,
this function composition is so unremarkable that it strains my
abilities as a writer to actually say anything about it. There are two
functions, each does its own thing, and now they've been composed into
one. Whoop-dee-doo!
\switchcolumn
这个函数组合并没有什么特别的。实际上,这个函数组合是如此平凡,以至于作为作者实际上无法对其说出任何有意义的话。有两个函数,每个函数都做自己的事情,现在它们被组合成一个函数。太棒了!
\switchcolumn[0]*
Middleware introduces an extra step to function composition, giving you
more flexibility in defining your function pipeline. Suppose, in the
preceding example, that you wanted to return "I don't like the number X"
for arbitrary numbers but return a string-ified number for everything
else. Here's how you could do that:
\switchcolumn
中间件在函数组合中引入了一个额外的步骤,使您能够更灵活地定义函数流水线。假设在前面的示例中,您想要对任意数字返回"I don't like the number X",但对其他所有内容返回字符串化的数字。您可以这样做:
\switchcolumn[0]*
\begin{verbatim}
(defn whiney-str
    [rejects]
    {:pre [(set? rejects)]}
    (fn [x]
    (if (rejects x)
        (str "I don't like " x)
        (str x))))

(def whiney-strinc (comp (whiney-str #{2}) inc))
(whiney-strinc 1)
; => "I don't like 2"
\end{verbatim}
\switchcolumn
\begin{verbatim}
(defn whiney-str
    [rejects]
    {:pre [(set? rejects)]}
    (fn [x]
    (if (rejects x)
        (str "I don't like " x)
        (str x))))

(def whiney-strinc (comp (whiney-str #{2}) inc))
(whiney-strinc 1)
; => "I don't like 2"
\end{verbatim}
\switchcolumn[0]*
Now let's take it one step further. What if you want to decide whether
or not to call inc in the first place? \href{javascript:void(0)}{Listing
B-1} shows how you could do that:
\begin{verbatim}
(defn whiney-middleware
    [next-handler rejects]
    {:pre [(set? rejects)]}
    (fn [x]
    (if (= x 1)
        "I'm not going to bother doing anything to that"
        (let [y (next-handler x)]
            (if (rejects y)
            (str "I don't like " y)
            (str y))))))

(def whiney-strinc (whiney-middleware inc #{2}))
(whiney-strinc 1)
; => "I don't like 2"
\end{verbatim}
\emph{Listing B-1: The middleware approach to function composition lets
you introduce choice}
\switchcolumn
现在让我们再进一步。如果您想要决定是否调用inc呢?\href{javascript:void(0)}{清单B-1}展示了如何实现:
\begin{verbatim}
(defn whiney-middleware
    [next-handler rejects]
    {:pre [(set? rejects)]}
    (fn [x]
    (if (= x 1)
        "I'm not going to bother doing anything to that"
        (let [y (next-handler x)]
            (if (rejects y)
            (str "I don't like " y)
            (str y))))))

(def whiney-strinc (whiney-middleware inc #{2}))
(whiney-strinc 1)
; => "I don't like 2"
\end{verbatim}
\emph{清单B-1:中间件方法可以引入选择}
\switchcolumn[0]*
Here, instead of using comp to create your function pipeline, you pass
the next function in the pipeline as the first argument to the
middleware function. In this case, you're passing inc as the first
argument to whiney-middleware as next-handler. whiney-middleware then
returns an anonymous function that closes over inc and has the ability
to choose whether to call it or not. You can see this choice at ➊.
\switchcolumn
在这里,您不是使用comp创建函数流水线,而是将下一个函数作为中间件函数的第一个参数传递。在这种情况下,您将inc作为下一个处理程序作为第一个参数传递给whiney-middleware作为next-handler。whiney-middleware然后返回一个匿名函数,该函数闭包inc并具有选择是否调用它的能力。您可以在➊处看到这个选择。
\switchcolumn[0]*
We say that a middleware takes a handler as its first argument and
returns a handler. In this example, whiney-middleware takes a handler as
its first argument, inc, and it returns another handler, the anonymous
function with x as its only argument. Middleware can also take extra
arguments, like rejects, that act as configuration. The result is that
the handler returned by the middleware can behave more flexibly (thanks
to configuration), and it has more control over the function pipeline
(because it can choose whether or not to call the next handler).
\switchcolumn
我们说中间件将处理程序作为其第一个参数并返回处理程序。在这个示例中,whiney-middleware将处理程序作为其第一个参数,即inc,并返回另一个处理程序,即只有x作为唯一参数的匿名函数。中间件还可以接受额外的参数,比如rejects,它们充当配置。结果是,由中间件返回的处理程序可以更灵活地行为(由于配置),并且它对函数流水线有更多的控制。


\switchcolumn[0]*
\subsubsection{Tasks Are Middleware Factories}
\switchcolumn
\subsubsection{任务是中间件工厂}
\switchcolumn[0]*
Boot takes this pattern of making function composition more flexible one
step further by separating middleware configuration from handler
creation. First, you create a function that takes \emph{n} configuration
arguments. This is the \emph{middleware factory}, and it returns a
middleware function. The middleware function expects one argument, the
next handler, and it returns a handler, just like in the preceding
example. Here's a whiney middleware factory:
\begin{verbatim}
(defn whiney-middleware-factory
    [rejects]
    {:pre [(set? rejects)]}
    (fn [next-handler]
    (fn [x]
        (if (rejects x)
        (str "I don't like " x)
        (str x)))))

(def whiney-strinc (whiney-middleware-factory #{2}))
(whiney-strinc 1)
; => "I don't like 2"
\end{verbatim}
\switchcolumn
Boot将使函数组合更加灵活的模式进一步发展,通过将中间件配置与处理程序创建分离。首先,您创建一个接受\emph{n}个配置参数的函数。这就是\emph{中间件工厂},它返回一个中间件函数。中间件函数期望一个参数,即下一个处理程序,并返回一个处理程序,就像前面的示例一样。这是一个whiney中间件工厂的例子:
\begin{verbatim}
(defn whiney-middleware-factory
    [rejects]
    {:pre [(set? rejects)]}
    (fn [next-handler]
    (fn [x]
        (if (rejects x)
        (str "I don't like " x)
        (str x)))))

(def whiney-strinc (whiney-middleware-factory #{2}))
(whiney-strinc 1)
; => "I don't like 2"
\end{verbatim}
\switchcolumn[0]*
As you can see, this code is nearly identical to
\href{javascript:void(0)}{Listing B-1}. The change is that the topmost
function, whiney-middleware-factory, now only accepts one argument,
rejects. It returns an anonymous function, the middleware, which expects
one argument, a handler. The rest of the code is the same.
\switchcolumn
正如您所看到的,这段代码与\href{javascript:void(0)}{B-1清单}几乎完全相同。变化在于最上面的函数whiney-middleware-factory现在只接受一个参数rejects。它返回一个匿名函数,即中间件,该中间件期望一个参数handler。其余的代码都是相同的。
\switchcolumn[0]*
In Boot, tasks can act as middleware factories. To show this, let's
split the fire task into two tasks: what and fire (see
\href{javascript:void(0)}{Listing B-2}). what lets you specify an object
and whether it's plural, and fire announces that it's on fire. This is
great modular software engineering because it allows you to add other
tasks, like gnomes, to announce that a thing is being overrun with
gnomes, which is just as objectively useful. (As an exercise, try
creating the gnome task. It should compose with the what task, just as
fire does.)
\switchcolumn
在Boot中,任务可以充当中间件工厂。为了展示这一点,让我们将fire任务拆分为两个任务:what和fire(参见\href{javascript:void(0)}{B-2清单})。what允许您指定一个对象以及它是否为复数形式,而fire则宣布该对象着火。这是一种很好的模块化软件工程,因为它允许您添加其他任务,比如gnomes任务,用于宣布某个事物被gnomes入侵,这同样是客观有用的。(作为练习,尝试创建gnomes任务。它应该与what任务组合,就像fire任务一样。)
\switchcolumn[0]*
\begin{verbatim}
(deftask what
    "Specify a thing"
    [t thing     THING str  "An object"
    p pluralize       bool "Whether to pluralize"]
    (fn middleware [next-handler]
➊     (fn handler [fileset]
        (next-handler (merge fileset {:thing thing :pluralize pluralize})))))
(deftask fire
    "Announce a thing is on fire"
    []
    (fn middleware [next-handler]
➋     (fn handler [fileset]
        (let [verb (if (:pluralize fileset) "are" "is")]

        (println "My" (:thing fileset) verb "on fire!")
        fileset))))
\end{verbatim}
\switchcolumn
\begin{verbatim}
(deftask what
    "Specify a thing"
    [t thing     THING str  "An object"
    p pluralize       bool "Whether to pluralize"]
    (fn middleware [next-handler]
➊     (fn handler [fileset]
        (next-handler (merge fileset {:thing thing :pluralize pluralize})))))
(deftask fire
    "Announce a thing is on fire"
    []
    (fn middleware [next-handler]
➋     (fn handler [fileset]
        (let [verb (if (:pluralize fileset) "are" "is")]

        (println "My" (:thing fileset) verb "on fire!")
        fileset))))
\end{verbatim}
\switchcolumn[0]*
\emph{Listing B-2: The full code for composable Boot tasks that announce
something's on fire}
\switchcolumn
\emph{清单B-2:可组合的Boot任务的完整代码,用于宣布某个事物着火}
\switchcolumn[0]*
Here's how you'd run this on the command line:
\switchcolumn
以下是如何在命令行中运行它的方式:
\switchcolumn[0]*
\begin{verbatim}
boot what -t "pants" -p – fire
\end{verbatim}
\switchcolumn
\begin{verbatim}
boot what -t "pants" -p – fire
\end{verbatim}
\switchcolumn[0]*
And here's how you'd run it in the REPL:
\switchcolumn
以下是在REPL中运行它的方式:
\switchcolumn[0]*
\begin{verbatim}
(boot (what :thing "pants" :pluralize true) (fire))
\end{verbatim}
\switchcolumn
\begin{verbatim}
(boot (what :thing "pants" :pluralize true) (fire))
\end{verbatim}

\switchcolumn[0]*
Wait a minute, what's that boot call doing there? And what's with
fileset at ➊ and ➋? In Micha's words, ``The boot macro takes care of
setup and cleanup (creating the initial fileset, stopping servers
started by tasks, things like that). Tasks are functions, so you can
call them directly, but if they use the fileset, they will fail unless
you call them via the boot macro.'' Let's take a closer look at
filesets.
\switchcolumn
等一下,那个boot调用在那里做什么?而且在➊和➋处的fileset是什么意思?用Micha的话来说,``boot宏负责设置和清理(创建初始的fileset,停止任务启动的服务器等等)。任务是函数,所以您可以直接调用它们,但如果它们使用fileset,除非通过boot宏调用它们,否则它们将失败。''让我们更仔细地看一下fileset。
\switchcolumn[0]*
\section{Filesets}
\switchcolumn
\section{文件集}
\switchcolumn[0]*
Earlier I mentioned that middleware is for creating
\emph{domain-specific} function pipelines. All that means is that each
handler expects to receive domain-specific data and returns
domain-specific data. With Ring, for example, each handler expects to
receive a request map representing the HTTP request, which might look
something like this:
\begin{verbatim}
{:server-port 80
    :request-method :get
    :scheme :http}
\end{verbatim}
\switchcolumn
前面我提到中间件用于创建\emph{领域特定}函数管道。这只是意味着每个处理程序都期望接收领域特定的数据并返回领域特定的数据。例如,在Ring中,每个处理程序都期望接收代表HTTP请求的请求映射,可能如下所示:
\begin{verbatim}
{:server-port 80
    :request-method :get
    :scheme :http}
\end{verbatim}
\switchcolumn[0]*
Each handler can choose to modify this request map in some way before
passing it on to the next handler, say, by adding a :params key with a
nice Clojure map of all query string and POST parameters. Ring handlers
return a \emph{response map}, which consists of the keys :status,
:headers, and :body, and once again each handler can transform this data
in some way before returning it to its parent handler.
\switchcolumn
每个处理程序可以选择在将请求映射传递给下一个处理程序之前以某种方式修改它,例如通过添加具有所有查询字符串和POST参数的漂亮Clojure映射的:params键。Ring处理程序返回一个\emph{响应映射},其中包含键:status、headers和body,同样,每个处理程序可以在将数据返回给其父处理程序之前以某种方式转换此数据。
\switchcolumn[0]*
In Boot, each handler receives and returns a \emph{fileset}. The fileset
abstraction lets you treat files on your filesystem as immutable data,
and this is a great innovation for build tools because building projects
is so file-centric. For example, your project might need to place
temporary, intermediary files on the filesystem. Usually, with most
build tools, these files get placed in some specially named place, say,
\emph{project/target/tmp}. The problem with this is that
\emph{project/target/tmp} is effectively a global variable, and other
tasks can accidentally muck it up.
\switchcolumn
在Boot中,每个处理程序接收和返回一个\emph{文件集}。文件集抽象允许您将文件系统上的文件视为不可变数据,这对于构建工具来说是一个很好的创新,因为构建项目与文件密切相关。例如,您的项目可能需要将临时中间文件放置在文件系统上。通常情况下,在大多数构建工具中,这些文件会被放置在某个特定命名的位置,比如\emph{project/target/tmp}。使用这种方式的问题在于\emph{project/target/tmp}实际上是一个全局变量,其他任务可能会意外地破坏它。
\switchcolumn[0]*
Boot's fileset abstraction solves this problem by adding a layer of
indirection on top of the filesystem. Let's say Task A creates File X
and tells the fileset to store it. Behind the scenes, the fileset stores
the file in an anonymous, temporary directory. The fileset then gets
passed to Task B, and Task B modifies File X and asks the fileset to
store the result. Behind the scenes, a new file, File Y, is created and
stored, but File X remains untouched. In Task B, an updated fileset is
returned. This is the equivalent of doing assoc-in with a map: Task A
can still access the original fileset and the files it references.
\switchcolumn
Boot的文件集抽象通过在文件系统之上添加一层间接性来解决了这个问题。假设任务A创建了文件X并告诉文件集将其存储起来。在幕后,文件集将文件存储在一个匿名的临时目录中。然后将文件集传递给任务B,任务B修改文件X并要求文件集存储结果。在幕后,创建并存储了一个新文件Y,但文件X保持不变。在任务B中,返回了一个更新后的文件集。这相当于对具有map的assoc-in操作:任务A仍然可以访问原始文件集和它引用的文件。
\switchcolumn[0]*
And you didn't even use any of this cool file management functionality
in the what and fire tasks in \href{javascript:void(0)}{Listing B-2}!
Nevertheless, when Boot composes tasks, it expects handlers to receive
and return fileset records. Therefore, to convey your data across tasks,
you sneakily added it to the fileset record using (merge fileset
\{:thing thing :pluralize pluralize\}).
\switchcolumn
你甚至没有使用\href{javascript:void(0)}{图例B-2}中的what和fire任务中的任何这些酷炫的文件管理功能!然而,当Boot组合任务时,它期望处理程序接收和返回文件集记录。因此,为了在任务之间传递数据,你偷偷地使用(merge fileset {:thing thing :pluralize pluralize})将数据添加到文件集记录中。
\switchcolumn[0]*
Although that covers the basic concept of a middleware factory, you'll
need to learn a bit more to take full advantage of filesets. The
mechanics of working with filesets are all explained in the fileset wiki
(\emph{https://github.com/boot-clj/boot/wiki/Filesets}). In the
meantime, I hope this information gave you a good conceptual overview!
\switchcolumn
虽然这只涵盖了中间件工厂的基本概念,但你还需要学习更多知识,以充分利用文件集。关于如何使用文件集的机制在文件集维基(\emph{https://github.com/boot-clj/boot/wiki/Filesets})中都有详细解释。与此同时,希望这些信息能给你提供一个良好的概念概述!
\switchcolumn[0]*
\section{Next Steps}
\switchcolumn
\section{下一步}

\switchcolumn[0]*
The point of this appendix was to explain the concepts behind Boot.
However, Boot also has a bunch of other functions, like set-env! and
task-options!, that make your programming life easier when you're
actually using it. It offers amazing magical features, like providing
classpath isolation so you can run multiple projects using one JVM, and
letting you add new dependencies to your project without having to
restart your REPL. If Boot tickles your fancy, check out its README for
more information on real-world usage. Also, its wiki provides top-notch
documentation.
\switchcolumn
本附录的目的是解释Boot背后的概念。然而,Boot还有很多其他功能,如set-env!和task-options!,在你实际使用它时可以让你的编程生活更轻松。它提供了一些神奇的特性,比如提供类路径隔离,这样你就可以在一个JVM上运行多个项目,而且可以在不重新启动REPL的情况下添加新的依赖项到你的项目中。如果Boot引起了你的兴趣,请查阅它的README以获取有关实际用法的更多信息。此外,它的维基提供了一流的文档。
\end{paracol}
%todo 以后有空可以研究下
% % 
% % 



\end{document}
\begin{codeHtml}

\end{codeHtml}  

\begin{codeJs}
\end{codeJs}

\begin{codeHVue}

\end{codeHVue}
%%%%%%%
\begin{vueQuoteWarn}
\end{vueQuoteWarn}

\begin{vueQuote}
\end{vueQuote}









\switchcolumn[0]*%%%%%%%

\switchcolumn

\switchcolumn[0]*%%%%%%%

\switchcolumn

\switchcolumn[0]*%%%%%%%

\switchcolumn

\switchcolumn[0]*%%%%%%%

\switchcolumn

\switchcolumn[0]*%%%%%%%

\switchcolumn

\switchcolumn[0]*%%%%%%%

\switchcolumn

\switchcolumn[0]*%%%%%%%

\switchcolumn

\switchcolumn[0]*%%%%%%%

\switchcolumn

\switchcolumn[0]*%%%%%%%

\switchcolumn

\switchcolumn[0]*%%%%%%%

\switchcolumn

\switchcolumn[0]*%%%%%%%

\switchcolumn

\switchcolumn[0]*%%%%%%%

\switchcolumn

\switchcolumn[0]*%%%%%%%

\switchcolumn

\switchcolumn[0]*%%%%%%%

\switchcolumn

\switchcolumn[0]*%%%%%%%

\switchcolumn

\switchcolumn[0]*%%%%%%%

\switchcolumn

\switchcolumn[0]*%%%%%%%

\switchcolumn

\switchcolumn[0]*%%%%%%%

\switchcolumn

\switchcolumn[0]*%%%%%%%

\switchcolumn

\switchcolumn[0]*%%%%%%%

\switchcolumn

\switchcolumn[0]*%%%%%%%

\switchcolumn

\switchcolumn[0]*%%%%%%%

\switchcolumn

\switchcolumn[0]*%%%%%%%

\switchcolumn

\switchcolumn[0]*%%%%%%%

\switchcolumn

\switchcolumn[0]*%%%%%%%

\switchcolumn

\switchcolumn[0]*%%%%%%%

\switchcolumn

\switchcolumn[0]*%%%%%%%

\switchcolumn

\switchcolumn[0]*%%%%%%%

\switchcolumn

\switchcolumn[0]*%%%%%%%

\switchcolumn

\switchcolumn[0]*%%%%%%%

\switchcolumn

\switchcolumn[0]*%%%%%%%

\switchcolumn

\switchcolumn[0]*%%%%%%%

\switchcolumn

\switchcolumn[0]*%%%%%%%

\switchcolumn

\switchcolumn[0]*%%%%%%%

\switchcolumn

\switchcolumn[0]*%%%%%%%

\switchcolumn

\switchcolumn[0]*%%%%%%%

\switchcolumn

\switchcolumn[0]*%%%%%%%

\switchcolumn

\switchcolumn[0]*%%%%%%%

\switchcolumn


\switchcolumn[0]*%%%%%%%

\switchcolumn


\switchcolumn[0]*%%%%%%%

\switchcolumn

\switchcolumn[0]*%%%%%%%

\switchcolumn


\switchcolumn[0]*%%%%%%%

\switchcolumn

\switchcolumn[0]*%%%%%%%

\switchcolumn


\switchcolumn[0]*%%%%%%%

\switchcolumn

\switchcolumn[0]*%%%%%%%

\switchcolumn


\switchcolumn[0]*%%%%%%%

\switchcolumn

\switchcolumn[0]*%%%%%%%

\switchcolumn


\switchcolumn[0]*%%%%%%%

\switchcolumn


\switchcolumn[0]*%%%%%%%

\switchcolumn

\switchcolumn[0]*%%%%%%%

\switchcolumn


\switchcolumn[0]*%%%%%%%

\switchcolumn

\switchcolumn[0]*%%%%%%%

\switchcolumn


\switchcolumn[0]*%%%%%%%

\switchcolumn

\switchcolumn[0]*%%%%%%%

\switchcolumn


\switchcolumn[0]*%%%%%%%

\switchcolumn

