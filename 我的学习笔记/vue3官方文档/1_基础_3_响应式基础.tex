\columnratio{0.55}
\begin{paracol}{2}
\switchcolumn[0]*%%%%%%%
\section{Reactivity Fundamentals}
\switchcolumn
\section{响应式基础}
\switchcolumn[0]*%%%%%%%
\begin{vueQuote}{API Preference}
This page and many other chapters later in the guide contain different
content for the Options API and the Composition API. Your current
preference is Composition API. You can toggle between the API styles
using the "API Preference" switches at the top of the left sidebar.
\end{vueQuote} 
\switchcolumn
\begin{vueQuote}{API 参考}
本页和后面很多页面中都分别包含了选项式 API 和组合式 API
的示例代码。现在你选择的是 组合式 API。你可以使用左侧侧边栏顶部的 ``API
风格偏好'' 开关在 API 风格之间切换。
\end{vueQuote} 
\switchcolumn[0]*%%%%%%%
\subsection{Declaring Reactive State}
\switchcolumn
\subsection{声明响应式状态}
\switchcolumn[0]*%%%%%%%
\subsubsection{ref()}
\switchcolumn
\subsubsection{ref()}
\switchcolumn[0]*%%%%%%%
In Composition API, the recommended way to declare reactive state is
using the
\href{https://vuejs.org/api/reactivity-core.html\#ref}{\texttt{ref()}}
function:
\switchcolumn
在组合式 API 中,推荐使用
\href{https://cn.vuejs.org/api/reactivity-core.html\#ref}{\texttt{ref()}}
函数来声明响应式状态:
\switchcolumn[0]*%%%%%%%
\begin{codeJs}
import { ref } from 'vue'

const count = ref(0)
\end{codeJs}
\switchcolumn
\begin{codeJs}
import { ref } from 'vue'

const count = ref(0)
\end{codeJs}
\switchcolumn[0]*%%%%%%%
\texttt{ref()} takes the argument and returns it wrapped within a ref
object with a \texttt{.value} property:
\switchcolumn
\texttt{ref()} 接收参数,并将其包裹在一个带有 \texttt{.value} 属性的 ref
对象中返回:
\switchcolumn[0]*%%%%%%%
\begin{codeJs}
const count = ref(0)

console.log(count) // { value: 0 }
console.log(count.value) // 0

count.value++
console.log(count.value) // 1
\end{codeJs}
\switchcolumn
\begin{codeJs}
const count = ref(0)

console.log(count) // { value: 0 }
console.log(count.value) // 0

count.value++
console.log(count.value) // 1
\end{codeJs}

\switchcolumn[0]*%%%%%%%
\begin{quote}
See also:
\href{https://vuejs.org/guide/typescript/composition-api.html\#typing-ref}{Typing
Refs}
\end{quote}
\switchcolumn
\begin{quote}
参考:\href{https://cn.vuejs.org/guide/typescript/composition-api.html\#typing-ref}{为
refs 标注类型}
\end{quote}
\switchcolumn[0]*%%%%%%%
To access refs in a component's template, declare and return them from a
component's \texttt{setup()} function:
\switchcolumn
要在组件模板中访问 ref,请从组件的 \texttt{setup()}
函数中声明并返回它们:
\switchcolumn[0]*%%%%%%%
\begin{codeJs}
import { ref } from 'vue'

export default {
    // `setup` 是一个特殊的钩子,专门用于组合式 API。
    setup() {
    const count = ref(0)

    // 将 ref 暴露给模板
    return {
        count
    }
    }
}
\end{codeJs}
\switchcolumn
\begin{codeJs}
import { ref } from 'vue'

export default {
    // `setup` 是一个特殊的钩子,专门用于组合式 API。
    setup() {
    const count = ref(0)

    // 将 ref 暴露给模板
    return {
        count
    }
    }
}
\end{codeJs}
\switchcolumn[0]*%%%%%%%
\begin{codeHtml}
<div>{{ count }}</div>
\end{codeHtml}  
\switchcolumn
\begin{codeHtml}
<div>{{ count }}</div>
\end{codeHtml}  

\switchcolumn[0]*%%%%%%%
Notice that we did \textbf{not} need to append \texttt{.value} when
using the ref in the template. For convenience, refs are automatically
unwrapped when used inside templates (with a few
\href{https://vuejs.org/guide/essentials/reactivity-fundamentals.html\#caveat-when-unwrapping-in-templates}{caveats}).
\switchcolumn
注意,在模板中使用 ref 时,我们\textbf{不}需要附加
\texttt{.value}。为了方便起见,当在模板中使用时,ref 会自动解包
(有一些\href{https://cn.vuejs.org/guide/essentials/reactivity-fundamentals.html\#caveat-when-unwrapping-in-templates}{注意事项})。
\switchcolumn[0]*%%%%%%%
You can also mutate a ref directly in event handlers:
\switchcolumn
你也可以直接在事件监听器中改变一个 ref:
\switchcolumn[0]*%%%%%%%
\begin{codeHtml}
<button @click="count++">
{{ count }}
</button>
\end{codeHtml}  
\switchcolumn
\begin{codeHtml}
<button @click="count++">
{{ count }}
</button>
\end{codeHtml}  

\switchcolumn[0]*%%%%%%%
For more complex logic, we can declare functions that mutate refs in the
same scope and expose them as methods alongside the state:
\switchcolumn
对于更复杂的逻辑,我们可以在同一作用域内声明更改 ref
的函数,并将它们作为方法与状态一起公开:
\switchcolumn[0]*%%%%%%%
\begin{codeJs}
import { ref } from 'vue'

export default {
    setup() {
    const count = ref(0)

    function increment() {
        // .value is needed in JavaScript
        count.value++
    }

    // don't forget to expose the function as well.
    return {
        count,
        increment
    }
    }
}
\end{codeJs}
\switchcolumn
\begin{codeJs}
import { ref } from 'vue'

export default {
    setup() {
    const count = ref(0)

    function increment() {
        // 在 JavaScript 中需要 .value
        count.value++
    }

    // 不要忘记同时暴露 increment 函数
    return {
        count,
        increment
    }
    }
}
\end{codeJs}

\switchcolumn[0]*%%%%%%%
Exposed methods can then be used as event handlers:
\switchcolumn
然后,暴露的方法可以被用作事件监听器:
\switchcolumn[0]*%%%%%%%
\begin{codeHtml}
<button @click="increment">
{{ count }}
</button>
\end{codeHtml}  
\switchcolumn
\begin{codeHtml}
<button @click="increment">
{{ count }}
</button>
\end{codeHtml}  
\switchcolumn[0]*%%%%%%%
Here's the example live on
\href{https://codepen.io/vuejs-examples/pen/WNYbaqo}{Codepen}, without
using any build tools.
\switchcolumn
这里是 \href{https://codepen.io/vuejs-examples/pen/WNYbaqo}{Codepen}
上的例子,没有使用任何构建工具。
\switchcolumn[0]*%%%%%%%
\subsubsection{\textless script setup\textgreater{}}
\switchcolumn
\subsubsection{\textless script setup\textgreater{}}
\switchcolumn[0]*%%%%%%%
Manually exposing state and methods via \texttt{setup()} can be verbose.
Luckily, it can be avoided when using
\href{https://vuejs.org/guide/scaling-up/sfc.html}{Single-File
Components (SFCs)}. We can simplify the usage with
\texttt{\textless{}script\ setup\textgreater{}}:
\switchcolumn
在 \texttt{setup()}
函数中手动暴露大量的状态和方法非常繁琐。幸运的是,我们可以通过使用\href{https://cn.vuejs.org/guide/scaling-up/sfc.html}{单文件组件
(SFC)} 来避免这种情况。我们可以使用
\texttt{\textless{}script\ setup\textgreater{}} 来大幅度地简化代码:
\switchcolumn[0]*%%%%%%%
\begin{codeVue}
<script setup>
import { ref } from 'vue'

const count = ref(0)

function increment() {
    count.value++
}
</script>

<template>
    <button @click="increment">
    {{ count }}
    </button>
</template>
\end{codeVue}
\switchcolumn
\begin{codeVue}
<script setup>
import { ref } from 'vue'

const count = ref(0)

function increment() {
    count.value++
}
</script>

<template>
    <button @click="increment">
    {{ count }}
    </button>
</template>
\end{codeVue}
\switchcolumn[0]*%%%%%%%
\href{https://play.vuejs.org/\#eNo9jUEKgzAQRa8yZKMiaNcllvYe2dgwQqiZhDhxE3L3jrW4/DPvv1/UK8Zhz6juSm82uciwIef4MOR8DImhQMIFKiwpeGgEbQwZsoE2BhsyMUwH0d66475ksuwCgSOb0CNx20ExBCc77POase8NVUN6PBdlSwKjj+vMKAlAvzOzWJ52dfYzGXXpjPoBAKX856uopDGeFfnq8XKp+gWq4FAi}{Try
it in the Playground}
\switchcolumn
\href{https://play.vuejs.org/\#eNo9jUEKgzAQRa8yZKMiaNcllvYe2dgwQqiZhDhxE3L3jrW4/DPvv1/UK8Zhz6juSm82uciwIef4MOR8DImhQMIFKiwpeGgEbQwZsoE2BhsyMUwH0d66475ksuwCgSOb0CNx20ExBCc77POase8NVUN6PBdlSwKjj+vMKAlAvzOzWJ52dfYzGXXpjPoBAKX856uopDGeFfnq8XKp+gWq4FAi}{在演练场中尝试一下}
\switchcolumn[0]*%%%%%%%
Top-level imports, variables and functions declared in
\texttt{\textless{}script\ setup\textgreater{}} are automatically usable
in the template of the same component. Think of the template as a
JavaScript function declared in the same scope - it naturally has access
to everything declared alongside it.
\switchcolumn
\texttt{\textless{}script\ setup\textgreater{}} 中的顶层的导入、声明的变量和函数可在同一组件的模板中直接使用。你可以理解为模板是在同一作用域内声明的一个 JavaScript 函数——它自然可以访问与它一起声明的所有内容。
\switchcolumn[0]*%%%%%%%
\begin{vueQuote}{TIP}
For the rest of the guide, we will be primarily using SFC +
\texttt{\textless{}script\ setup\textgreater{}} syntax for the
Composition API code examples, as that is the most common usage for Vue
developers.

If you are not using SFC, you can still use Composition API with the
\href{https://vuejs.org/api/composition-api-setup.html}{\texttt{setup()}}
option.
\end{vueQuote} 
\switchcolumn
\begin{vueQuote}{TIP}
在指南的后续章节中,我们基本上都会在组合式 API 示例中使用单文件组件 +
\texttt{\textless{}script\ setup\textgreater{}} 的语法,因为大多数 Vue
开发者都会这样使用。

如果你没有使用单文件组件,你仍然可以在
\href{https://cn.vuejs.org/api/composition-api-setup.html}{\texttt{setup()}}
选项中使用组合式 API。
\end{vueQuote} 

\switchcolumn[0]*%%%%%%%
\subsubsection{Why Refs?}
\switchcolumn
\subsubsection{为什么要使用 ref?}
\switchcolumn[0]*%%%%%%%
You might be wondering why we need refs with the \texttt{.value} instead
of plain variables. To explain that, we will need to briefly discuss how
Vue's reactivity system works.
\switchcolumn
你可能会好奇:为什么我们需要使用带有 \texttt{.value} 的
ref,而不是普通的变量?为了解释这一点,我们需要简单地讨论一下 Vue
的响应式系统是如何工作的。
\switchcolumn[0]*%%%%%%%
When you use a ref in a template, and change the ref's value later, Vue
automatically detects the change and updates the DOM accordingly. This
is made possible with a dependency-tracking based reactivity system.
When a component is rendered for the first time, Vue \textbf{tracks}
every ref that was used during the render. Later on, when a ref is
mutated, it will \textbf{trigger} a re-render for components that are
tracking it.
\switchcolumn
当你在模板中使用了一个 ref,然后改变了这个 ref 的值时,Vue
会自动检测到这个变化,并且相应地更新
DOM。这是通过一个基于依赖追踪的响应式系统实现的。当一个组件首次渲染时,Vue
会\textbf{追踪}在渲染过程中使用的每一个 ref。然后,当一个 ref
被修改时,它会\textbf{触发}追踪它的组件的一次重新渲染。
\switchcolumn[0]*%%%%%%%
In standard JavaScript, there is no way to detect the access or mutation
of plain variables. However, we can intercept the get and set operations
of an object's properties using getter and setter methods.
\switchcolumn
在标准的 JavaScript
中,检测普通变量的访问或修改是行不通的。然而,我们可以通过 getter 和
setter 方法来拦截对象属性的 get 和 set 操作。
\switchcolumn[0]*%%%%%%%
The \texttt{.value} property gives Vue the opportunity to detect when a
ref has been accessed or mutated. Under the hood, Vue performs the
tracking in its getter, and performs triggering in its setter.
Conceptually, you can think of a ref as an object that looks like this:
\switchcolumn
该 \texttt{.value} 属性给予了 Vue 一个机会来检测 ref
何时被访问或修改。在其内部,Vue 在它的 getter 中执行追踪,在它的 setter
中执行触发。从概念上讲,你可以将 ref 看作是一个像这样的对象:
\switchcolumn[0]*%%%%%%%
\begin{codeJs}
// pseudo code, not actual implementation
const myRef = {
    _value: 0,
    get value() {
    track()
    return this._value
    },
    set value(newValue) {
    this._value = newValue
    trigger()
    }
}
\end{codeJs}
\switchcolumn
\begin{codeJs}
// 伪代码,不是真正的实现
const myRef = {
    _value: 0,
    get value() {
    track()
    return this._value
    },
    set value(newValue) {
    this._value = newValue
    trigger()
    }
}
\end{codeJs}

\switchcolumn[0]*%%%%%%%
Another nice trait of refs is that unlike plain variables, you can pass
refs into functions while retaining access to the latest value and the
reactivity connection. This is particularly useful when refactoring
complex logic into reusable code.
\switchcolumn
另一个 ref 的好处是,与普通变量不同,你可以将 ref
传递给函数,同时保留对最新值和响应式连接的访问。当将复杂的逻辑重构为可重用的代码时,这将非常有用。
\switchcolumn[0]*%%%%%%%
The reactivity system is discussed in more details in the
\href{https://vuejs.org/guide/extras/reactivity-in-depth.html}{Reactivity
in Depth} section.
\switchcolumn
该响应性系统在\href{https://cn.vuejs.org/guide/extras/reactivity-in-depth.html}{深入响应式原理}章节中有更详细的讨论。
\switchcolumn[0]*%%%%%%%
\subsubsection{Deep Reactivity}
\switchcolumn
\subsubsection{深层响应性}
\switchcolumn[0]*%%%%%%%
Refs can hold any value type, including deeply nested objects, arrays,
or JavaScript built-in data structures like \texttt{Map}.
\switchcolumn
Ref 可以持有任何类型的值,包括深层嵌套的对象、数组或者 JavaScript
内置的数据结构,比如 \texttt{Map}。
\switchcolumn[0]*%%%%%%%
A ref will make its value deeply reactive. This means you can expect
changes to be detected even when you mutate nested objects or arrays:
\switchcolumn
Ref
会使它的值具有深层响应性。这意味着即使改变嵌套对象或数组时,变化也会被检测到:
\switchcolumn[0]*%%%%%%%
\begin{codeHtml}
import { ref } from 'vue'

const obj = ref({
    nested: { count: 0 },
    arr: ['foo', 'bar']
})

function mutateDeeply() {
    // these will work as expected.
    obj.value.nested.count++
    obj.value.arr.push('baz')
}
\end{codeHtml}  
\switchcolumn
\begin{codeHtml}
import { ref } from 'vue'

const obj = ref({
    nested: { count: 0 },
    arr: ['foo', 'bar']
})

function mutateDeeply() {
    // 以下都会按照期望工作
    obj.value.nested.count++
    obj.value.arr.push('baz')
}
\end{codeHtml}  

\switchcolumn[0]*%%%%%%%
Non-primitive values are turned into reactive proxies via
\href{https://vuejs.org/guide/essentials/reactivity-fundamentals.html\#reactive}{\texttt{reactive()}},
which is discussed below.
\switchcolumn
非原始值将通过
\href{https://cn.vuejs.org/guide/essentials/reactivity-fundamentals.html\#reactive}{\texttt{reactive()}}
转换为响应式代理,该函数将在后面讨论。
\switchcolumn[0]*%%%%%%%
It is also possible to opt-out of deep reactivity with
\href{https://vuejs.org/api/reactivity-advanced.html\#shallowref}{shallow
refs}. For shallow refs, only \texttt{.value} access is tracked for
reactivity. Shallow refs can be used for optimizing performance by
avoiding the observation cost of large objects, or in cases where the
inner state is managed by an external library.
\switchcolumn
也可以通过
\href{https://cn.vuejs.org/api/reactivity-advanced.html\#shallowref}{shallow
ref} 来放弃深层响应性。对于浅层 ref,只有 \texttt{.value}
的访问会被追踪。浅层 ref
可以用于避免对大型数据的响应性开销来优化性能、或者有外部库管理其内部状态的情况。
\switchcolumn[0]*%%%%%%%
Further reading:
\switchcolumn
阅读更多:
\switchcolumn[0]*%%%%%%%
\begin{itemize}
\item
  \href{https://vuejs.org/guide/best-practices/performance.html\#reduce-reactivity-overhead-for-large-immutable-structures}{Reduce
  Reactivity Overhead for Large Immutable Structures}
\item
  \href{https://vuejs.org/guide/extras/reactivity-in-depth.html\#integration-with-external-state-systems}{Integration
  with External State Systems}
\end{itemize}
\switchcolumn
\begin{itemize}
\item
  \href{https://cn.vuejs.org/guide/best-practices/performance.html\#reduce-reactivity-overhead-for-large-immutable-structures}{减少大型不可变数据的响应性开销}
\item
  \href{https://cn.vuejs.org/guide/extras/reactivity-in-depth.html\#integration-with-external-state-systems}{与外部状态系统集成}
\end{itemize}
\switchcolumn[0]*%%%%%%%
\subsubsection{DOM Update Timing}
\switchcolumn
\subsubsection{DOM 更新时机}
\switchcolumn[0]*%%%%%%%
When you mutate reactive state, the DOM is updated automatically.
However, it should be noted that the DOM updates are not applied
synchronously. Instead, Vue buffers them until the "next tick" in the
update cycle to ensure that each component updates only once no matter
how many state changes you have made.
\switchcolumn
当你修改了响应式状态时,DOM 会被自动更新。但是需要注意的是,DOM
更新不是同步的。Vue 会在``next
tick''更新周期中缓冲所有状态的修改,以确保不管你进行了多少次状态修改,每个组件都只会被更新一次。
\switchcolumn[0]*%%%%%%%
To wait for the DOM update to complete after a state change, you can use
the \href{https://vuejs.org/api/general.html\#nexttick}{nextTick()}
global API:
\switchcolumn
要等待 DOM 更新完成后再执行额外的代码,可以使用
\href{https://cn.vuejs.org/api/general.html\#nexttick}{nextTick()} 全局
API:
\switchcolumn[0]*%%%%%%%
\begin{codeJs}
import { nextTick } from 'vue'

async function increment() {
    count.value++
    await nextTick()
    // Now the DOM is updated
}
\end{codeJs}
\switchcolumn
\begin{codeJs}
import { nextTick } from 'vue'

async function increment() {
    count.value++
    await nextTick()
    // 现在 DOM 已经更新了
}
\end{codeJs}

\switchcolumn[0]*%%%%%%%
\subsection{reactive()}
\switchcolumn
\subsection{reactive()}
\switchcolumn[0]*%%%%%%%
There is another way to declare reactive state, with the
\texttt{reactive()} API. Unlike a ref which wraps the inner value in a
special object, \texttt{reactive()} makes an object itself reactive:
\switchcolumn
还有另一种声明响应式状态的方式,即使用 \texttt{reactive()}
API。与将内部值包装在特殊对象中的 ref 不同,\texttt{reactive()}
将使对象本身具有响应性:
\switchcolumn[0]*%%%%%%%
\begin{codeJs}
import { reactive } from 'vue'

const state = reactive({ count: 0 })
\end{codeJs}  
\switchcolumn
\begin{codeJs}
import { reactive } from 'vue'

const state = reactive({ count: 0 })
\end{codeJs}

\switchcolumn[0]*%%%%%%%
\subsection{reactive()}
\switchcolumn
\subsection{reactive()}
\switchcolumn[0]*%%%%%%%
There is another way to declare reactive state, with the
\texttt{reactive()} API. Unlike a ref which wraps the inner value in a
special object, \texttt{reactive()} makes an object itself reactive:
\switchcolumn
还有另一种声明响应式状态的方式,即使用 \texttt{reactive()}
API。与将内部值包装在特殊对象中的 ref 不同,\texttt{reactive()}
将使对象本身具有响应性:
\switchcolumn[0]*%%%%%%%
\begin{codeJs}
import { reactive } from 'vue'

const state = reactive({ count: 0 })
\end{codeJs}
\switchcolumn
\begin{codeJs}
import { reactive } from 'vue'

const state = reactive({ count: 0 })
\end{codeJs}

\switchcolumn[0]*%%%%%%%
\begin{quote}
See also:
\href{https://vuejs.org/guide/typescript/composition-api.html\#typing-reactive}{Typing
Reactive}
\end{quote}
\switchcolumn
\begin{quote}
参考:\href{https://cn.vuejs.org/guide/typescript/composition-api.html\#typing-reactive}{为
\texttt{reactive()} 标注类型}
\end{quote}
\switchcolumn[0]*%%%%%%%
Usage in template:
\switchcolumn
在模板中使用:
\switchcolumn[0]*%%%%%%%
\begin{codeHtml}
<button @click="state.count++">
{{ state.count }}
</button>
\end{codeHtml}
\switchcolumn
\begin{codeHtml}
<button @click="state.count++">
{{ state.count }} 
</button>
\end{codeHtml}
\switchcolumn[0]*%%%%%%%
Reactive objects are
\href{https://developer.mozilla.org/en-US/docs/Web/JavaScript/Reference/Global_Objects/Proxy}{JavaScript
Proxies} and behave just like normal objects. The difference is that Vue
is able to intercept the access and mutation of all properties of a
reactive object for reactivity tracking and triggering.
\switchcolumn
响应式对象是
\href{https://developer.mozilla.org/en-US/docs/Web/JavaScript/Reference/Global_Objects/Proxy}{JavaScript
代理},其行为就和普通对象一样。不同的是,Vue
能够拦截对响应式对象所有属性的访问和修改,以便进行依赖追踪和触发更新。
\switchcolumn[0]*%%%%%%%
\texttt{reactive()} converts the object deeply: nested objects are also
wrapped with \texttt{reactive()} when accessed. It is also called by
\texttt{ref()} internally when the ref value is an object. Similar to
shallow refs, there is also the
\href{https://vuejs.org/api/reactivity-advanced.html\#shallowreactive}{\texttt{shallowReactive()}}
API for opting-out of deep reactivity.
\switchcolumn
\texttt{reactive()} 将深层地转换对象:当访问嵌套对象时,它们也会被
\texttt{reactive()} 包装。当 ref 的值是一个对象时,\texttt{ref()}
也会在内部调用它。与浅层 ref 类似,这里也有一个
\href{https://cn.vuejs.org/api/reactivity-advanced.html\#shallowreactive}{\texttt{shallowReactive()}}
API 可以选择退出深层响应性。
\switchcolumn[0]*%%%%%%%
\subsubsection{Reactive Proxy vs. Original}
\switchcolumn
\subsubsection{Reactive Proxy vs. Original}
\switchcolumn[0]*%%%%%%%
It is important to note that the returned value from \texttt{reactive()}
is a
\href{https://developer.mozilla.org/en-US/docs/Web/JavaScript/Reference/Global_Objects/Proxy}{Proxy}
of the original object, which is not equal to the original object:
\switchcolumn
值得注意的是,\texttt{reactive()} 返回的是一个原始对象的
\href{https://developer.mozilla.org/en-US/docs/Web/JavaScript/Reference/Global_Objects/Proxy}{Proxy},它和原始对象是不相等的:
\switchcolumn[0]*%%%%%%%
\begin{codeJs}
const raw = {}
const proxy = reactive(raw)

// proxy is NOT equal to the original.
console.log(proxy === raw) // false    
\end{codeJs}
\switchcolumn
\begin{codeJs}
const raw = {}
const proxy = reactive(raw)

// 代理对象和原始对象不是全等的
console.log(proxy === raw) // false
\end{codeJs} 

\switchcolumn[0]*%%%%%%%
Only the proxy is reactive - mutating the original object will not
trigger updates. Therefore, the best practice when working with Vue's
reactivity system is to \textbf{exclusively use the proxied versions of
your state}.
\switchcolumn
只有代理对象是响应式的,更改原始对象不会触发更新。因此,使用 Vue
的响应式系统的最佳实践是 \textbf{仅使用你声明对象的代理版本}。
\switchcolumn[0]*%%%%%%%
To ensure consistent access to the proxy, calling \texttt{reactive()} on
the same object always returns the same proxy, and calling
\texttt{reactive()} on an existing proxy also returns that same proxy:
\switchcolumn
为保证访问代理的一致性,对同一个原始对象调用 \texttt{reactive()}
会总是返回同样的代理对象,而对一个已存在的代理对象调用
\texttt{reactive()} 会返回其本身:
\switchcolumn[0]*%%%%%%%
\begin{codeJs}
// calling reactive() on the same object returns the same proxy
console.log(reactive(raw) === proxy) // true

// calling reactive() on a proxy returns itself
console.log(reactive(proxy) === proxy) // true
\end{codeJs}
\switchcolumn
\begin{codeJs}
// 在同一个对象上调用 reactive() 会返回相同的代理
console.log(reactive(raw) === proxy) // true

// 在一个代理上调用 reactive() 会返回它自己
console.log(reactive(proxy) === proxy) // true
\end{codeJs}
\switchcolumn[0]*%%%%%%%
This rule applies to nested objects as well. Due to deep reactivity,
nested objects inside a reactive object are also proxies:
\switchcolumn
这个规则对嵌套对象也适用。依靠深层响应性,响应式对象内的嵌套对象依然是代理:
\switchcolumn[0]*%%%%%%%
\begin{codeJs}
const proxy = reactive({})

const raw = {}
proxy.nested = raw

console.log(proxy.nested === raw) // false
\end{codeJs}
\switchcolumn
\begin{codeJs}
const proxy = reactive({})

const raw = {}
proxy.nested = raw

console.log(proxy.nested === raw) // false
\end{codeJs}

\end{paracol}



\end{document}]%%%%%%%]*%%%%%%%]*%%%%%%%]*%%%%%%%]*%%%%%%%]*%%%%%%%]*%%%%%%%]*%%%%%%%]*%%%%%%%]*%%%%%%%
\switchcolumn[0]*%%%%%%%
\begin{vueQuote}{}
\end{vueQuote} 
\switchcolumn
\begin{vueQuote}{}
\end{vueQuote} 

\switchcolumn[0]*%%%%%%%
\begin{codeJs}

\end{codeJs}  
\switchcolumn
\begin{codeJs}

\end{codeJs}  


\switchcolumn[0]*%%%%%%%
\begin{codeHtml}

\end{codeHtml}  
\switchcolumn
\begin{codeHtml}

\end{codeHtml}  