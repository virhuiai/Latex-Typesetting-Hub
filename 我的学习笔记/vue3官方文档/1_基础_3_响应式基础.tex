\columnratio{0.55}
\begin{paracol}{2}
\switchcolumn[0]*%%%%%%%
\section{Reactivity Fundamentals}
\switchcolumn
\section{响应式基础}
\switchcolumn[0]*%%%%%%%
\begin{vueQuote}{API Preference}
This page and many other chapters later in the guide contain different
content for the Options API and the Composition API. Your current
preference is Composition API. You can toggle between the API styles
using the "API Preference" switches at the top of the left sidebar.
\end{vueQuote} 
\switchcolumn
\begin{vueQuote}{API 参考}
本页和后面很多页面中都分别包含了选项式 API 和组合式 API
的示例代码。现在你选择的是 组合式 API。你可以使用左侧侧边栏顶部的 ``API
风格偏好'' 开关在 API 风格之间切换。
\end{vueQuote} 
\switchcolumn[0]*%%%%%%%
\subsection{Declaring Reactive State}
\switchcolumn
\subsection{声明响应式状态}
\switchcolumn[0]*%%%%%%%
\subsubsection{ref()}
\switchcolumn
\subsubsection{ref()}
\switchcolumn[0]*%%%%%%%
In Composition API, the recommended way to declare reactive state is
using the
\href{https://vuejs.org/api/reactivity-core.html\#ref}{\texttt{ref()}}
function:
\switchcolumn
在组合式 API 中,推荐使用
\href{https://cn.vuejs.org/api/reactivity-core.html\#ref}{\texttt{ref()}}
函数来声明响应式状态:
\switchcolumn[0]*%%%%%%%
\begin{codeJs}
import { ref } from 'vue'

const count = ref(0)
\end{codeJs}
\switchcolumn
\begin{codeJs}
import { ref } from 'vue'

const count = ref(0)
\end{codeJs}
\switchcolumn[0]*%%%%%%%
\texttt{ref()} takes the argument and returns it wrapped within a ref
object with a \texttt{.value} property:
\switchcolumn
\texttt{ref()} 接收参数,并将其包裹在一个带有 \texttt{.value} 属性的 ref
对象中返回:
\switchcolumn[0]*%%%%%%%
\begin{codeJs}
const count = ref(0)

console.log(count) // { value: 0 }
console.log(count.value) // 0

count.value++
console.log(count.value) // 1
\end{codeJs}
\switchcolumn
\begin{codeJs}
const count = ref(0)

console.log(count) // { value: 0 }
console.log(count.value) // 0

count.value++
console.log(count.value) // 1
\end{codeJs}

\switchcolumn[0]*%%%%%%%
\begin{quote}
See also:
\href{https://vuejs.org/guide/typescript/composition-api.html\#typing-ref}{Typing
Refs}
\end{quote}
\switchcolumn
\begin{quote}
参考:\href{https://cn.vuejs.org/guide/typescript/composition-api.html\#typing-ref}{为
refs 标注类型}
\end{quote}
\switchcolumn[0]*%%%%%%%
To access refs in a component's template, declare and return them from a
component's \texttt{setup()} function:
\switchcolumn
要在组件模板中访问 ref,请从组件的 \texttt{setup()}
函数中声明并返回它们:
\switchcolumn[0]*%%%%%%%
\begin{codeJs}
import { ref } from 'vue'

export default {
    // `setup` 是一个特殊的钩子,专门用于组合式 API。
    setup() {
    const count = ref(0)

    // 将 ref 暴露给模板
    return {
        count
    }
    }
}
\end{codeJs}
\switchcolumn
\begin{codeJs}
import { ref } from 'vue'

export default {
    // `setup` 是一个特殊的钩子,专门用于组合式 API。
    setup() {
    const count = ref(0)

    // 将 ref 暴露给模板
    return {
        count
    }
    }
}
\end{codeJs}
\switchcolumn[0]*%%%%%%%
\begin{codeHtml}
<div>{{ count }}</div>
\end{codeHtml}  
\switchcolumn
\begin{codeHtml}
<div>{{ count }}</div>
\end{codeHtml}  

\switchcolumn[0]*%%%%%%%
Notice that we did \textbf{not} need to append \texttt{.value} when
using the ref in the template. For convenience, refs are automatically
unwrapped when used inside templates (with a few
\href{https://vuejs.org/guide/essentials/reactivity-fundamentals.html\#caveat-when-unwrapping-in-templates}{caveats}).
\switchcolumn
注意,在模板中使用 ref 时,我们\textbf{不}需要附加
\texttt{.value}。为了方便起见,当在模板中使用时,ref 会自动解包
(有一些\href{https://cn.vuejs.org/guide/essentials/reactivity-fundamentals.html\#caveat-when-unwrapping-in-templates}{注意事项})。
\switchcolumn[0]*%%%%%%%
You can also mutate a ref directly in event handlers:
\switchcolumn
你也可以直接在事件监听器中改变一个 ref:
\switchcolumn[0]*%%%%%%%
\begin{codeHtml}
<button @click="count++">
{{ count }}
</button>
\end{codeHtml}  
\switchcolumn
\begin{codeHtml}
<button @click="count++">
{{ count }}
</button>
\end{codeHtml}  

\switchcolumn[0]*%%%%%%%
For more complex logic, we can declare functions that mutate refs in the
same scope and expose them as methods alongside the state:
\switchcolumn
对于更复杂的逻辑,我们可以在同一作用域内声明更改 ref
的函数,并将它们作为方法与状态一起公开:
\switchcolumn[0]*%%%%%%%
\begin{codeJs}
import { ref } from 'vue'

export default {
    setup() {
    const count = ref(0)

    function increment() {
        // .value is needed in JavaScript
        count.value++
    }

    // don't forget to expose the function as well.
    return {
        count,
        increment
    }
    }
}
\end{codeJs}
\switchcolumn
\begin{codeJs}
import { ref } from 'vue'

export default {
    setup() {
    const count = ref(0)

    function increment() {
        // 在 JavaScript 中需要 .value
        count.value++
    }

    // 不要忘记同时暴露 increment 函数
    return {
        count,
        increment
    }
    }
}
\end{codeJs}

\end{paracol}



\end{document}]%%%%%%%]*%%%%%%%]*%%%%%%%]*%%%%%%%]*%%%%%%%]*%%%%%%%]*%%%%%%%]*%%%%%%%]*%%%%%%%]*%%%%%%%
\switchcolumn[0]*%%%%%%%
\begin{vueQuote}{}
\end{vueQuote} 
\switchcolumn
\begin{vueQuote}{}
\end{vueQuote} 




\switchcolumn[0]*%%%%%%%
\begin{codeHtml}

\end{codeHtml}  
\switchcolumn
\begin{codeHtml}

\end{codeHtml}  