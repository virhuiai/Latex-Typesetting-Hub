\columnratio{0.55}
\begin{paracol}{2}
\switchcolumn[0]*%%%%%%%
\section{List Rendering}
\switchcolumn
\section{列表渲染}
\switchcolumn[0]*%%%%%%%
\subsection{v-for}
\switchcolumn
\subsection{v-for}
\switchcolumn[0]*%%%%%%%
We can use the \texttt{v-for} directive to render a list of items based
on an array. The \texttt{v-for} directive requires a special syntax in
the form of \texttt{item\ in\ items}, where \texttt{items} is the source
data array and \texttt{item} is an \textbf{alias} for the array element
being iterated on:
\switchcolumn
我们可以使用 \texttt{v-for}
指令基于一个数组来渲染一个列表。\texttt{v-for} 指令的值需要使用
\texttt{item\ in\ items} 形式的特殊语法,其中 \texttt{items}
是源数据的数组,而 \texttt{item} 是迭代项的\textbf{别名}:

\switchcolumn[0]*%%%%%%%
\begin{codeJs}
const items = ref([{ message: 'Foo' }, { message: 'Bar' }])
\end{codeJs}
\switchcolumn
\begin{codeJs}
const items = ref([{ message: 'Foo' }, { message: 'Bar' }])
\end{codeJs}
\switchcolumn[0]*%%%%%%%
\begin{codeHtml}
<li v-for="item in items">
  {{ item.message }}
</li>
\end{codeHtml}
\switchcolumn
\begin{codeHtml}
<li v-for="item in items">
  {{ item.message }}
</li>
\end{codeHtml}
\switchcolumn[0]*%%%%%%%
Inside the \texttt{v-for} scope, template expressions have access to all
parent scope properties. In addition, \texttt{v-for} also supports an
optional second alias for the index of the current item:
\switchcolumn
在 \texttt{v-for}
块中可以完整地访问父作用域内的属性和变量。\texttt{v-for}
也支持使用可选的第二个参数表示当前项的位置索引。
\switchcolumn[0]*%%%%%%%
\begin{codeJs}
const parentMessage = ref('Parent')
const items = ref([{ message: 'Foo' }, { message: 'Bar' }])
\end{codeJs}
\switchcolumn
\begin{codeJs}
const parentMessage = ref('Parent')
const items = ref([{ message: 'Foo' }, { message: 'Bar' }])
\end{codeJs}
\switchcolumn[0]*%%%%%%%
\begin{codeHtml}
<li v-for="(item, index) in items">
  {{ parentMessage }} - {{ index }} - {{ item.message }}
</li>
\end{codeHtml}
\switchcolumn
\begin{codeHtml}
<li v-for="(item, index) in items">
  {{ parentMessage }} - {{ index }} - {{ item.message }}
</li>
\end{codeHtml}
\switchcolumn[0]*%%%%%%%
\href{https://play.vuejs.org/\#eNpdTsuqwjAQ/ZVDNlFQu5d64bpwJ7g3LopOJdAmIRlFCPl3p60PcDWcM+eV1X8Iq/uN1FrV6RxtYCTiW/gzzvbBR0ZGpBYFbfQ9tEi1ccadvUuM0ERyvKeUmithMyhn+jCSev4WWaY+vZ7HjH5Sr6F33muUhTR8uW0ThTuJua6mPbJEgGSErmEaENedxX3Z+rgxajbEL2DdhR5zOVOdUSIEDOf8M7IULCHsaPgiMa1eK4QcS6rOSkhdfapVeQLQEWnH}{Try
it in the Playground}
\switchcolumn
\href{https://play.vuejs.org/\#eNpdTsuqwjAQ/ZVDNlFQu5d64bpwJ7g3LopOJdAmIRlFCPl3p60PcDWcM+eV1X8Iq/uN1FrV6RxtYCTiW/gzzvbBR0ZGpBYFbfQ9tEi1ccadvUuM0ERyvKeUmithMyhn+jCSev4WWaY+vZ7HjH5Sr6F33muUhTR8uW0ThTuJua6mPbJEgGSErmEaENedxX3Z+rgxajbEL2DdhR5zOVOdUSIEDOf8M7IULCHsaPgiMa1eK4QcS6rOSkhdfapVeQLQEWnH}{在演练场中尝试一下}
\switchcolumn[0]*%%%%%%%
The variable scoping of \texttt{v-for} is similar to the following
JavaScript:
\switchcolumn
\texttt{v-for} 变量的作用域和下面的 JavaScript 代码很类似:
\switchcolumn[0]*%%%%%%%
\begin{codeJs}
const parentMessage = 'Parent'
const items = [
  /* ... */
]
items.forEach((item, index) => {
  // 可以访问外层的 `parentMessage`
  // 而 `item` 和 `index` 只在这个作用域可用
  console.log(parentMessage, item.message, index)
})
\end{codeJs}
\switchcolumn
\begin{codeJs}
const parentMessage = 'Parent'
const items = [
  /* ... */
]
items.forEach((item, index) => {
  // 可以访问外层的 `parentMessage`
  // 而 `item` 和 `index` 只在这个作用域可用
  console.log(parentMessage, item.message, index)
})
\end{codeJs}
\switchcolumn[0]*%%%%%%%
Notice how the \texttt{v-for} value matches the function signature of
the \texttt{forEach} callback. In fact, you can use destructuring on the
\texttt{v-for} item alias similar to destructuring function arguments:
\switchcolumn
注意 \texttt{v-for} 是如何对应 \texttt{forEach}
回调的函数签名的。实际上,你也可以在定义 \texttt{v-for}
的变量别名时使用解构,和解构函数参数类似:
\switchcolumn[0]*%%%%%%%
\begin{codeHtml}
<li v-for="{ message } in items">
  {{ message }}
</li>
<!-- 有 index 索引时 -->
<li v-for="({ message }, index) in items">
  {{ message }} {{ index }}
</li>
\end{codeHtml}
\switchcolumn
\begin{codeHtml}
<li v-for="{ message } in items">
  {{ message }}
</li>
<!-- 有 index 索引时 -->
<li v-for="({ message }, index) in items">
  {{ message }} {{ index }}
</li>
\end{codeHtml}

\switchcolumn[0]*%%%%%%%
For nested \texttt{v-for}, scoping also works similar to nested
functions. Each \texttt{v-for} scope has access to parent scopes:
\switchcolumn
对于多层嵌套的
\texttt{v-for},作用域的工作方式和函数的作用域很类似。每个
\texttt{v-for} 作用域都可以访问到父级作用域:
\switchcolumn[0]*%%%%%%%
\begin{codeHtml}
<li v-for="item in items">
  <span v-for="childItem in item.children">
    {{ item.message }} {{ childItem }}
  </span>
</li>
\end{codeHtml}
\switchcolumn
\begin{codeHtml}
<li v-for="item in items">
  <span v-for="childItem in item.children">
    {{ item.message }} {{ childItem }}
  </span>
</li>
\end{codeHtml}
\switchcolumn[0]*%%%%%%%
You can also use \texttt{of} as the delimiter instead of \texttt{in}, so
that it is closer to JavaScript's syntax for iterators:
\switchcolumn
你也可以使用 \texttt{of} 作为分隔符来替代 \texttt{in},这更接近
JavaScript 的迭代器语法:
\switchcolumn[0]*%%%%%%%
\begin{codeHtml}
<div v-for="item of items"></div>
\end{codeHtml}
\switchcolumn
\begin{codeHtml}
<div v-for="item of items"></div>
\end{codeHtml}
\end{paracol}

\columnratio{0.55}
\begin{paracol}{2}
\switchcolumn[0]*%%%%%%%
\subsection{v-for with an Object}
\switchcolumn
\subsection{v-for 与对象}
\switchcolumn[0]*%%%%%%%
You can also use \texttt{v-for} to iterate through the properties of an
object. The iteration order will be based on the result of calling
\texttt{Object.keys()} on the object:
\switchcolumn
你也可以使用 \texttt{v-for}
来遍历一个对象的所有属性。遍历的顺序会基于对该对象调用
\texttt{Object.keys()} 的返回值来决定。
\switchcolumn[0]*%%%%%%%
\begin{codeJs}
const myObject = reactive({
    title: 'How to do lists in Vue',
    author: 'Jane Doe',
    publishedAt: '2016-04-10'
})
\end{codeJs}
\switchcolumn
\begin{codeJs}
const myObject = reactive({
    title: 'How to do lists in Vue',
    author: 'Jane Doe',
    publishedAt: '2016-04-10'
})
\end{codeJs}
\switchcolumn[0]*%%%%%%%
\begin{codeHtml}
<ul>
    <li v-for="value in myObject">
    {{ value }}
    </li>
</ul>
\end{codeHtml}
\switchcolumn
\begin{codeHtml}
<ul>
    <li v-for="value in myObject">
    {{ value }}
    </li>
</ul>
\end{codeHtml}
\switchcolumn[0]*%%%%%%%
You can also provide a second alias for the property's name (a.k.a.
key):
\switchcolumn
可以通过提供第二个参数表示属性名 (例如 key):
\switchcolumn[0]*%%%%%%%
\begin{codeHtml}
<li v-for="(value, key) in myObject">
    {{ key }}: {{ value }}
</li>
\end{codeHtml}
\switchcolumn
\begin{codeHtml}
<li v-for="(value, key) in myObject">
    {{ key }}: {{ value }}
</li>
\end{codeHtml}
\switchcolumn[0]*%%%%%%%
And another for the index:
\switchcolumn
第三个参数表示位置索引:
\switchcolumn[0]*%%%%%%%
\begin{codeHtml}
<li v-for="(value, key, index) in myObject">
    {{ index }}. {{ key }}: {{ value }}
</li>
\end{codeHtml}
\switchcolumn
\begin{codeHtml}
<li v-for="(value, key, index) in myObject">
    {{ index }}. {{ key }}: {{ value }}
</li>
\end{codeHtml}
\switchcolumn[0]*%%%%%%%
\href{https://play.vuejs.org/\#eNo9jjFvgzAQhf/KE0sSCQKpqg7IqRSpQ9WlWycvBC6KW2NbcKaNEP+9B7Tx4nt33917Y3IKYT9ESspE9XVnAqMnjuFZO9MG3zFGdFTVbAbChEvnW2yE32inXe1dz2hv7+dPqhnHO7kdtQPYsKUSm1f/DfZoPKzpuYdx+JAL6cxUka++E+itcoQX/9cO8SzslZoTy+yhODxlxWN2KMR22mmn8jWrpBTB1AZbMc2KVbTyQ56yBkN28d1RJ9uhspFSfNEtFf+GfnZzjP/oOll2NQPjuM4xTftZyIaU5VwuN0SsqMqtWZxUvliq/J4jmX4BTCp08A==}{Try
it in the Playground}
\switchcolumn
\href{https://play.vuejs.org/\#eNo9jjFvgzAQhf/KE0sSCQKpqg7IqRSpQ9WlWycvBC6KW2NbcKaNEP+9B7Tx4nt33917Y3IKYT9ESspE9XVnAqMnjuFZO9MG3zFGdFTVbAbChEvnW2yE32inXe1dz2hv7+dPqhnHO7kdtQPYsKUSm1f/DfZoPKzpuYdx+JAL6cxUka++E+itcoQX/9cO8SzslZoTy+yhODxlxWN2KMR22mmn8jWrpBTB1AZbMc2KVbTyQ56yBkN28d1RJ9uhspFSfNEtFf+GfnZzjP/oOll2NQPjuM4xTftZyIaU5VwuN0SsqMqtWZxUvliq/J4jmX4BTCp08A==}{在演练场中尝试一下}
\end{paracol}

\columnratio{0.55}
\begin{paracol}{2}
\switchcolumn[0]*%%%%%%%
\subsection{v-for with a Range}
\switchcolumn
\subsection{在 v-for 里使用范围值}
\switchcolumn[0]*%%%%%%%
\texttt{v-for} can also take an integer. In this case it will repeat the
template that many times, based on a range of \texttt{1...n}.
\switchcolumn
\texttt{v-for} 可以直接接受一个整数值。在这种用例中,会将该模板基于
\texttt{1...n} 的取值范围重复多次。
\switchcolumn[0]*%%%%%%%
\begin{codeHtml}
<span v-for="n in 10">{{ n }}</span>
\end{codeHtml}
\switchcolumn
\begin{codeHtml}
<span v-for="n in 10">{{ n }}</span>
\end{codeHtml}
\switchcolumn[0]*%%%%%%%
Note here \texttt{n} starts with an initial value of \texttt{1} instead
of \texttt{0}.
\switchcolumn
注意此处 \texttt{n} 的初值是从 \texttt{1} 开始而非 \texttt{0}。


\switchcolumn[0]*%%%%%%%
\subsection{v-for on \textless template\textgreater{}}
\switchcolumn
\subsection{\textless template\textgreater{} 上的 v-for}
\switchcolumn[0]*%%%%%%%
Similar to template \texttt{v-if}, you can also use a
\texttt{\textless{}template\textgreater{}} tag with \texttt{v-for} to
render a block of multiple elements. For example:
\switchcolumn
与模板上的 \texttt{v-if} 类似,你也可以在
\texttt{\textless{}template\textgreater{}} 标签上使用 \texttt{v-for}
来渲染一个包含多个元素的块。例如:
\switchcolumn[0]*%%%%%%%
\begin{codeHtml}
<ul>
  <template v-for="item in items">
    <li>{{ item.msg }}</li>
    <li class="divider" role="presentation"></li>
  </template>
</ul>
\end{codeHtml}
\switchcolumn
\begin{codeHtml}
<ul>
  <template v-for="item in items">
    <li>{{ item.msg }}</li>
    <li class="divider" role="presentation"></li>
  </template>
</ul>
\end{codeHtml}
\switchcolumn[0]*%%%%%%%
\subsection{v-for with v-if}
\switchcolumn
\subsection{v-for 与 v-if}
\switchcolumn[0]*%%%%%%%
\begin{vueQuoteWarn}{Note}
It's \textbf{not} recommended to use \texttt{v-if} and \texttt{v-for} on
the same element due to implicit precedence. Refer to
\href{https://vuejs.org/style-guide/rules-essential.html\#avoid-v-if-with-v-for}{style
guide} for details.
\end{vueQuoteWarn}
\switchcolumn
\begin{vueQuoteWarn}{注意}
同时使用 \texttt{v-if} 和 \texttt{v-for}
是\textbf{不推荐的},因为这样二者的优先级不明显。请转阅\href{https://cn.vuejs.org/style-guide/rules-essential.html\#avoid-v-if-with-v-for}{风格指南}查看更多细节。
\end{vueQuoteWarn}


\switchcolumn[0]*%%%%%%%
When they exist on the same node, \texttt{v-if} has a higher priority
than \texttt{v-for}. That means the \texttt{v-if} condition will not
have access to variables from the scope of the \texttt{v-for}:
\switchcolumn
当它们同时存在于一个节点上时,\texttt{v-if} 比 \texttt{v-for}
的优先级更高。这意味着 \texttt{v-if} 的条件将无法访问到 \texttt{v-for}
作用域内定义的变量别名:
\switchcolumn[0]*%%%%%%%
\begin{codeHtml}
<!--
 这会抛出一个错误,因为属性 todo 此时
 没有在该实例上定义
-->
<li v-for="todo in todos" v-if="!todo.isComplete">
  {{ todo.name }}
</li>
\end{codeHtml}
\switchcolumn
\begin{codeHtml}
<!--
 这会抛出一个错误,因为属性 todo 此时
 没有在该实例上定义
-->
<li v-for="todo in todos" v-if="!todo.isComplete">
  {{ todo.name }}
</li>
\end{codeHtml}
\switchcolumn[0]*%%%%%%%
This can be fixed by moving \texttt{v-for} to a wrapping
\texttt{\textless{}template\textgreater{}} tag (which is also more
explicit):
\switchcolumn
在外新包装一层 \texttt{\textless{}template\textgreater{}} 再在其上使用
\texttt{v-for} 可以解决这个问题 (这也更加明显易读):
\switchcolumn[0]*%%%%%%%
\begin{codeHtml}
<template v-for="todo in todos">
  <li v-if="!todo.isComplete">
    {{ todo.name }}
  </li>
</template>
\end{codeHtml}
\switchcolumn
\begin{codeHtml}
<template v-for="todo in todos">
  <li v-if="!todo.isComplete">
    {{ todo.name }}
  </li>
</template>
\end{codeHtml}
\end{paracol}

\columnratio{0.55}
\begin{paracol}{2}

\switchcolumn[0]*%%%%%%%
\subsection{Maintaining State with key}
\switchcolumn
\subsection{通过 key 管理状态}
\switchcolumn[0]*%%%%%%%
When Vue is updating a list of elements rendered with \texttt{v-for}, by
default it uses an "in-place patch" strategy. If the order of the data
items has changed, instead of moving the DOM elements to match the order
of the items, Vue will patch each element in-place and make sure it
reflects what should be rendered at that particular index.
\switchcolumn
Vue 默认按照``就地更新''的策略来更新通过 \texttt{v-for}
渲染的元素列表。当数据项的顺序改变时,Vue 不会随之移动 DOM
元素的顺序,而是就地更新每个元素,确保它们在原本指定的索引位置上渲染。
\switchcolumn[0]*%%%%%%%
This default mode is efficient, but \textbf{only suitable when your list
render output does not rely on child component state or temporary DOM
state (e.g. form input values)}.
\switchcolumn
默认模式是高效的,但\textbf{只适用于列表渲染输出的结果不依赖子组件状态或者临时
DOM 状态 (例如表单输入值) 的情况}。
\switchcolumn[0]*%%%%%%%
To give Vue a hint so that it can track each node's identity, and thus
reuse and reorder existing elements, you need to provide a unique
\texttt{key} attribute for each item:
\switchcolumn
为了给 Vue
一个提示,以便它可以跟踪每个节点的标识,从而重用和重新排序现有的元素,你需要为每个元素对应的块提供一个唯一的
\texttt{key} attribute:
\switchcolumn[0]*%%%%%%%
\begin{codeHtml}
<div v-for="item in items" :key="item.id">
    <!-- 内容 -->
</div>
\end{codeHtml}
\switchcolumn
\begin{codeHtml}
<div v-for="item in items" :key="item.id">
    <!-- 内容 -->
</div>
\end{codeHtml}
\switchcolumn[0]*%%%%%%%
When using \texttt{\textless{}template\ v-for\textgreater{}}, the
\texttt{key} should be placed on the
\texttt{\textless{}template\textgreater{}} container:
\switchcolumn
当你使用 \texttt{\textless{}template\ v-for\textgreater{}}
时,\texttt{key} 应该被放置在这个
\texttt{\textless{}template\textgreater{}} 容器上:
\switchcolumn[0]*%%%%%%%
\begin{codeHtml}
<template v-for="todo in todos" :key="todo.name">
    <li>{{ todo.name }}</li>
</template>
\end{codeHtml}
\switchcolumn
\begin{codeHtml}
<template v-for="todo in todos" :key="todo.name">
    <li>{{ todo.name }}</li>
</template>
\end{codeHtml}
\switchcolumn[0]*%%%%%%%
\begin{vueQuote}{Note}
\texttt{key} here is a special attribute being bound with
\texttt{v-bind}. It should not be confused with the property key
variable when
\href{https://vuejs.org/guide/essentials/list.html\#v-for-with-an-object}{using
\texttt{v-for} with an object}.
\end{vueQuote}
\switchcolumn
\begin{vueQuote}{注意}
\texttt{key} 在这里是一个通过 \texttt{v-bind} 绑定的特殊
attribute。请不要和\href{https://cn.vuejs.org/guide/essentials/list.html\#v-for-with-an-object}{在
\texttt{v-for} 中使用对象}里所提到的对象属性名相混淆。
\end{vueQuote}


\switchcolumn[0]*%%%%%%%
\href{https://vuejs.org/style-guide/rules-essential.html\#use-keyed-v-for}{It
is recommended} to provide a \texttt{key} attribute with \texttt{v-for}
whenever possible, unless the iterated DOM content is simple (i.e.
contains no components or stateful DOM elements), or you are
intentionally relying on the default behavior for performance gains.
\switchcolumn
\href{https://cn.vuejs.org/style-guide/rules-essential.html\#use-keyed-v-for}{推荐}在任何可行的时候为
\texttt{v-for} 提供一个 \texttt{key} attribute,除非所迭代的 DOM
内容非常简单 (例如:不包含组件或有状态的 DOM
元素),或者你想有意采用默认行为来提高性能。
\switchcolumn[0]*%%%%%%%
The \texttt{key} binding expects primitive values - i.e. strings and
numbers. Do not use objects as \texttt{v-for} keys. For detailed usage
of the \texttt{key} attribute, please see the
\href{https://vuejs.org/api/built-in-special-attributes.html\#key}{\texttt{key}
API documentation}.
\switchcolumn
\texttt{key} 绑定的值期望是一个基础类型的值,例如字符串或 number
类型。不要用对象作为 \texttt{v-for} 的 key。关于 \texttt{key} attribute
的更多用途细节,请参阅
\href{https://cn.vuejs.org/api/built-in-special-attributes.html\#key}{\texttt{key}
API 文档}。
\end{paracol}



