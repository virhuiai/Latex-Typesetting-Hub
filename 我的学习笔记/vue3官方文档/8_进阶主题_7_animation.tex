
\columnratio{0.55}
\begin{paracol}{2} 
 
\switchcolumn[0]*%%%%%%%
\section{Animation Techniques}
\switchcolumn
\section{动画技巧}
\switchcolumn[0]*%%%%%%%
Vue provides the
\href{https://vuejs.org/guide/built-ins/transition.html}{``} and
\href{https://vuejs.org/guide/built-ins/transition-group.html}{``}
components for handling enter / leave and list transitions. However,
there are many other ways of using animations on the web, even in a Vue
application. Here we will discuss a few additional techniques.
\switchcolumn
Vue 提供了
\href{https://cn.vuejs.org/guide/built-ins/transition.html}{``} 和
\href{https://cn.vuejs.org/guide/built-ins/transition-group.html}{``}
组件来处理元素进入、离开和列表顺序变化的过渡效果。但除此之外,还有许多其他制作网页动画的方式在
Vue 应用中也适用。这里我们会探讨一些额外的技巧。
\switchcolumn[0]*%%%%%%%
\subsection{Class-based Animations}
\switchcolumn
\subsection{基于 CSS class 的动画}
\switchcolumn[0]*%%%%%%%
For elements that are not entering / leaving the DOM, we can trigger
animations by dynamically adding a CSS class:
\switchcolumn
对于那些不是正在进入或离开 DOM 的元素,我们可以通过给它们动态添加 CSS
class 来触发动画:
\switchcolumn[0]*%%%%%%%
\begin{codeJs}
const disabled = ref(false)
function warnDisabled() {
  disabled.value = true
  setTimeout(() => {
    disabled.value = false
  }, 1500)
}
\end{codeJs}
\switchcolumn
\begin{codeJs}
const disabled = ref(false)
function warnDisabled() {
  disabled.value = true
  setTimeout(() => {
    disabled.value = false
  }, 1500)
}
\end{codeJs}
\switchcolumn[0]*%%%%%%%
\begin{codeHtml}
<div :class="{ shake: disabled }">
  <button @click="warnDisabled">Click me</button>
  <span v-if="disabled">This feature is disabled!</span>
</div>
\end{codeHtml}
\switchcolumn
\begin{codeHtml}
<div :class="{ shake: disabled }">
  <button @click="warnDisabled">Click me</button>
  <span v-if="disabled">This feature is disabled!</span>
</div>
\end{codeHtml}
\switchcolumn[0]*%%%%%%%
\begin{codeCss}
.shake {
  animation: shake 0.82s cubic-bezier(0.36, 0.07, 0.19, 0.97) both;
  transform: translate3d(0, 0, 0);
}
@keyframes shake {
  10%,
  90% {
    transform: translate3d(-1px, 0, 0);
  }
  20%,
  80% {
    transform: translate3d(2px, 0, 0);
  }
  30%,
  50%,
  70% {
    transform: translate3d(-4px, 0, 0);
  }
  40%,
  60% {
    transform: translate3d(4px, 0, 0);
  }
}
\end{codeCss}
\switchcolumn
\begin{codeCss}
.shake {
  animation: shake 0.82s cubic-bezier(0.36, 0.07, 0.19, 0.97) both;
  transform: translate3d(0, 0, 0);
}
@keyframes shake {
  10%,
  90% {
    transform: translate3d(-1px, 0, 0);
  }
  20%,
  80% {
    transform: translate3d(2px, 0, 0);
  }
  30%,
  50%,
  70% {
    transform: translate3d(-4px, 0, 0);
  }
  40%,
  60% {
    transform: translate3d(4px, 0, 0);
  }
}
\end{codeCss}
\switchcolumn[0]*%%%%%%%
\subsection{State-driven Animations}
\switchcolumn
\subsection{状态驱动的动画}
\switchcolumn[0]*%%%%%%%
Some transition effects can be applied by interpolating values, for
instance by binding a style to an element while an interaction occurs.
Take this example for instance:
\switchcolumn
有些过渡效果可以通过动态插值来实现,比如在交互时动态地给元素绑定样式。看下面这个例子:
\switchcolumn[0]*%%%%%%%
\begin{codeJs}
const x = ref(0)
function onMousemove(e) {
  x.value = e.clientX
}
\end{codeJs}
\switchcolumn
\begin{codeJs}
const x = ref(0)
function onMousemove(e) {
  x.value = e.clientX
}
\end{codeJs}
\switchcolumn[0]*%%%%%%%
\begin{codeHtml}
<div
  @mousemove="onMousemove"
  :style="{ backgroundColor: `hsl(${x}, 80%, 50%)` }"
  class="movearea"
>
  <p>Move your mouse across this div...</p>
  <p>x: {{ x }}</p>
</div>
\end{codeHtml}
\switchcolumn
\begin{codeHtml}
<div
  @mousemove="onMousemove"
  :style="{ backgroundColor: `hsl(${x}, 80%, 50%)` }"
  class="movearea"
>
  <p>Move your mouse across this div...</p>
  <p>x: {{ x }}</p>
</div>
\end{codeHtml}
\switchcolumn[0]*%%%%%%%
\begin{codeCss}
.movearea {
  transition: 0.3s background-color ease;
}
\end{codeCss}
\switchcolumn
\begin{codeCss}
.movearea {
  transition: 0.3s background-color ease;
}
\end{codeCss}
\switchcolumn[0]*%%%%%%%
In addition to color, you can also use style bindings to animate
transform, width, or height. You can even animate SVG paths using spring
physics - after all, they are all attribute data bindings:
\switchcolumn
除了颜色外,你还可以使用样式绑定 CSS
transform、宽度或高度。你甚至可以通过运用弹性物理模拟为 SVG
添加动画,毕竟它们也只是 attribute 的数据绑定:
\switchcolumn[0]*%%%%%%%
\href{https://play.vuejs.org/\#eNqlVmtv2zYU/SsXboE6mC3bSdwVmpM9MAz9sAIdsA8b5gGhRUrWKpEESTl2DP/3HZKSbbkuUKBA4Ij3ce65D15pP/hZ62TTiEE6WNjMlNqRFa7Rj0tZ1loZR3sygmWu3IgRZarWjROcDpQbVdMbeL45WvKdZHWZ2VbXHZP/LGyWMlPSOloLxoV5L8pi7eiBZrdTr6uEo9L+alhRlLKAPGeVFZ2PdQzwD6CyTWk6oh1+6dBpM2g6isNgch4jWPeCHm4u2Xxkbg2QLrvh8IYeHmm/lARg1xhJTx+moyn9fnfr//nf1/tzzMMfr/dZsj1AnCW7Azhe6J+W8jwsfp2Q7qOypSuV/ELsaMt3Xp3saNxL48yA1Vp4DFg+ojA/0i2ldH/GPqAROcOkzZWpU3oKzxVzYui5wnPS4hz09gZsydc3Us4bidqCZWiD79FQ3ERMgagiydZMFoL/qZpsLSziX4r+mf4LRmgn9ZvsTBOEATjZBjDNCvHXSeiTj8K/wadHR8lv5ZLT8MSnhUFVA5PeyHxHw5YZutCyRW2C9alJLc+jya5n8VmXZsm865MP6hEugp61pevIRUOUDjVouX8hoWsXy8uPF5ThBvtRiGKQGXVPX3OdzozdTov0hGu1QdAR8cYwTzil76e4vrkpA/+Ubt+Fe+ydQzljhotJ3ETYQTg4UWs/qDgRLXi5aQtWMWsflgPuU2OrSiwHUfFTrRoruHqW0B5H9qh1fgyC+Ko6ONdqI6CNA9b3vK4KXo0OiLElfS8h+TVdcKsEC5B9bcgW+dpNcUx1BR09l9ytccASwmkdeoDj/C2OrRPctN9oqQ962nAwz8uqguzV3W/z2S9zOCwm3rILNkG07hmFPgaOGDD3/OiDWEygvWbY7jVESq3bVT6ti1UHkPeiqtQJontaTM46jWMAIJspLTgkybHRcaxXLPtUGNVIPs5UpUxKr/I8/yGo1HZs1wwj4N8T93pLs7f4McWKYdv5F4j/S2bzm2AeKpr6ra63QRCLium87yRouskrj7cuORcyCGtmcKfgCCtijVNBqttEUyxfJIN3UhA7sXVjVpUFFBnqIUwQ56jO2JYvuDUzED3JnlsOd9NpEGJQzNgPSwahVDKirpRBY8aG8bKxKb0LCLhByaqIVTqxYSurKrxhIhulUZrwWIkciPH5ZVxKLvw7toWJjccFT9o2XqL2cjy6z2IlGOe4/rFAp8aUL0HYUoeoF6t78/U72nUEXwvnRYqFuxX153VbqYpHYEy1nySM0GA0iF8q45ppfJUoia+eEG/ZKuxykHZbE6vl9AHj5bgHzmmbTiYZl4n9tNMYwYSLzaRn2O2xweF/7cIdbA==}{Source
code}
\switchcolumn
\href{https://play.vuejs.org/\#eNqlVmtv2zYU/SsXboE6mC3bSdwVmpM9MAz9sAIdsA8b5gGhRUrWKpEESTl2DP/3HZKSbbkuUKBA4Ij3ce65D15pP/hZ62TTiEE6WNjMlNqRFa7Rj0tZ1loZR3sygmWu3IgRZarWjROcDpQbVdMbeL45WvKdZHWZ2VbXHZP/LGyWMlPSOloLxoV5L8pi7eiBZrdTr6uEo9L+alhRlLKAPGeVFZ2PdQzwD6CyTWk6oh1+6dBpM2g6isNgch4jWPeCHm4u2Xxkbg2QLrvh8IYeHmm/lARg1xhJTx+moyn9fnfr//nf1/tzzMMfr/dZsj1AnCW7Azhe6J+W8jwsfp2Q7qOypSuV/ELsaMt3Xp3saNxL48yA1Vp4DFg+ojA/0i2ldH/GPqAROcOkzZWpU3oKzxVzYui5wnPS4hz09gZsydc3Us4bidqCZWiD79FQ3ERMgagiydZMFoL/qZpsLSziX4r+mf4LRmgn9ZvsTBOEATjZBjDNCvHXSeiTj8K/wadHR8lv5ZLT8MSnhUFVA5PeyHxHw5YZutCyRW2C9alJLc+jya5n8VmXZsm865MP6hEugp61pevIRUOUDjVouX8hoWsXy8uPF5ThBvtRiGKQGXVPX3OdzozdTov0hGu1QdAR8cYwTzil76e4vrkpA/+Ubt+Fe+ydQzljhotJ3ETYQTg4UWs/qDgRLXi5aQtWMWsflgPuU2OrSiwHUfFTrRoruHqW0B5H9qh1fgyC+Ko6ONdqI6CNA9b3vK4KXo0OiLElfS8h+TVdcKsEC5B9bcgW+dpNcUx1BR09l9ytccASwmkdeoDj/C2OrRPctN9oqQ962nAwz8uqguzV3W/z2S9zOCwm3rILNkG07hmFPgaOGDD3/OiDWEygvWbY7jVESq3bVT6ti1UHkPeiqtQJontaTM46jWMAIJspLTgkybHRcaxXLPtUGNVIPs5UpUxKr/I8/yGo1HZs1wwj4N8T93pLs7f4McWKYdv5F4j/S2bzm2AeKpr6ra63QRCLium87yRouskrj7cuORcyCGtmcKfgCCtijVNBqttEUyxfJIN3UhA7sXVjVpUFFBnqIUwQ56jO2JYvuDUzED3JnlsOd9NpEGJQzNgPSwahVDKirpRBY8aG8bKxKb0LCLhByaqIVTqxYSurKrxhIhulUZrwWIkciPH5ZVxKLvw7toWJjccFT9o2XqL2cjy6z2IlGOe4/rFAp8aUL0HYUoeoF6t78/U72nUEXwvnRYqFuxX153VbqYpHYEy1nySM0GA0iF8q45ppfJUoia+eEG/ZKuxykHZbE6vl9AHj5bgHzmmbTiYZl4n9tNMYwYSLzaRn2O2xweF/7cIdbA==}{Source
code}
\end{paracol}



\columnratio{0.55}
\begin{paracol}{2} 
\switchcolumn[0]*%%%%%%%
\subsection{Animating with Watchers}
\switchcolumn
\subsection{基于侦听器的动画}
\switchcolumn[0]*%%%%%%%
With some creativity, we can use watchers to animate anything based on
some numerical state. For example, we can animate the number itself:
\switchcolumn
通过发挥一些创意,我们可以基于一些数字状态,配合侦听器给任何东西加上动画。例如,我们可以将数字本身变成动画:
\switchcolumn[0]*%%%%%%%
\begin{codeJs}
import { ref, reactive, watch } from 'vue'
import gsap from 'gsap'
const number = ref(0)
const tweened = reactive({
  number: 0
})
watch(number, (n) => {
  gsap.to(tweened, { duration: 0.5, number: Number(n) || 0 })
})
\end{codeJs}
\switchcolumn
\begin{codeJs}
import { ref, reactive, watch } from 'vue'
import gsap from 'gsap'
const number = ref(0)
const tweened = reactive({
  number: 0
})
watch(number, (n) => {
  gsap.to(tweened, { duration: 0.5, number: Number(n) || 0 })
})
\end{codeJs}
\switchcolumn[0]*%%%%%%%
\begin{codeHtml}
Type a number: <input v-model.number="number" />
<p>{{ tweened.number.toFixed(0) }}</p>
\end{codeHtml}
\switchcolumn
\begin{codeHtml}
Type a number: <input v-model.number="number" />
<p>{{ tweened.number.toFixed(0) }}</p>
\end{codeHtml}
\switchcolumn[0]*%%%%%%%
\href{https://play.vuejs.org/\#eNpNUstygzAM/BWNLyEzBDKd6YWSdHrpsacefSGgJG7xY7BImhL+vTKv9ILllXYlr+jEm3PJpUWRidyXjXIEHql1e2mUdrYh6KDBY8yfoiR1wRiuBZVn6OHYWA0r5q6W2pMv3ISHkBPSlNZ4AtPqAzawC2LRdj3DdEU0WA34qB910sBUnsFWmp6LpRmaRo9UHMLIrGG3h4EBQ/OEbDRpxjx51TYFKWtYKHmOF9WP4Qzs+x22EDoA9NLwmaejC/x+vhBqVxeEfAPIK3WBsi6830lRobZSDDjA580hFIt8roxrCS4bbSuskxFmzhhIAenEy92id1CnzZzfd91szETmZ72rH6zYOej7PA3rYXrKE3GUp//m5KunWx3C5CE6enS0hjZXVKczZXCwdfWyoF79YgZPqBliJ9iGSUTEYlzuRrO9X94a/lUGNTklvBTZvAMpwhYCIMWZyPksTVvjvk9JaXUacq9sSlujFJPnvej/AElH3FQ=}{Try
it in the Playground}

\switchcolumn
\href{https://play.vuejs.org/\#eNpNUstygzAM/BWNLyEzBDKd6YWSdHrpsacefSGgJG7xY7BImhL+vTKv9ILllXYlr+jEm3PJpUWRidyXjXIEHql1e2mUdrYh6KDBY8yfoiR1wRiuBZVn6OHYWA0r5q6W2pMv3ISHkBPSlNZ4AtPqAzawC2LRdj3DdEU0WA34qB910sBUnsFWmp6LpRmaRo9UHMLIrGG3h4EBQ/OEbDRpxjx51TYFKWtYKHmOF9WP4Qzs+x22EDoA9NLwmaejC/x+vhBqVxeEfAPIK3WBsi6830lRobZSDDjA580hFIt8roxrCS4bbSuskxFmzhhIAenEy92id1CnzZzfd91szETmZ72rH6zYOej7PA3rYXrKE3GUp//m5KunWx3C5CE6enS0hjZXVKczZXCwdfWyoF79YgZPqBliJ9iGSUTEYlzuRrO9X94a/lUGNTklvBTZvAMpwhYCIMWZyPksTVvjvk9JaXUacq9sSlujFJPnvej/AElH3FQ=}{在演练场中尝试一下}

\end{paracol}


