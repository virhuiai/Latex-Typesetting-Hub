\columnratio{0.55}
\begin{paracol}{2}

\switchcolumn[0]*%%%%%%%
\section{Custom Directives}
\switchcolumn
\section{自定义指令}
\switchcolumn[0]*%%%%%%%
In addition to the default set of directives shipped in core (like
\texttt{v-model} or \texttt{v-show}), Vue also allows you to register
your own custom directives.
\switchcolumn
\subsection{介绍}
\switchcolumn[0]*%%%%%%%
\subsection{Introduction}
\switchcolumn
除了 Vue 内置的一系列指令 (比如 \texttt{v-model} 或 \texttt{v-show})
之外,Vue 还允许你注册自定义的指令 (Custom Directives)。 
\switchcolumn[0]*%%%%%%%
We have introduced two forms of code reuse in Vue:
\href{https://vuejs.org/guide/essentials/component-basics.html}{components}
and
\href{https://vuejs.org/guide/reusability/composables.html}{composables}.
Components are the main building blocks, while composables are focused
on reusing stateful logic. Custom directives, on the other hand, are
mainly intended for reusing logic that involves low-level DOM access on
plain elements.
\switchcolumn
我们已经介绍了两种在 Vue
中重用代码的方式:\href{https://cn.vuejs.org/guide/essentials/component-basics.html}{组件}和\href{https://cn.vuejs.org/guide/reusability/composables.html}{组合式函数}。组件是主要的构建模块,而组合式函数则侧重于有状态的逻辑。另一方面,自定义指令主要是为了重用涉及普通元素的底层
DOM 访问的逻辑。
\switchcolumn[0]*%%%%%%%
A custom directive is defined as an object containing lifecycle hooks
similar to those of a component. The hooks receive the element the
directive is bound to. Here is an example of a directive that focuses an
input when the element is inserted into the DOM by Vue:
\switchcolumn
一个自定义指令由一个包含类似组件生命周期钩子的对象来定义。钩子函数会接收到指令所绑定元素作为其参数。下面是一个自定义指令的例子,当一个
input 元素被 Vue 插入到 DOM 中后,它会被自动聚焦:
\switchcolumn[0]*%%%%%%%
\begin{codeHtml}
<script setup>
// 在模板中启用 v-focus
const vFocus = {
  mounted: (el) => el.focus()
}
</script>
<template>
  <input v-focus />
</template>
\end{codeHtml}
\switchcolumn
\begin{codeHtml}
<script setup>
// 在模板中启用 v-focus
const vFocus = {
  mounted: (el) => el.focus()
}
</script>
<template>
  <input v-focus />
</template>
\end{codeHtml}
\switchcolumn[0]*%%%%%%%
Assuming you haven't clicked elsewhere on the page, the input above
should be auto-focused. This directive is more useful than the
\texttt{autofocus} attribute because it works not just on page load - it
also works when the element is dynamically inserted by Vue.
\switchcolumn
假设你还未点击页面中的其他地方,那么上面这个 input
元素应该会被自动聚焦。该指令比 \texttt{autofocus} attribute
更有用,因为它不仅仅可以在页面加载完成后生效,还可以在 Vue
动态插入元素后生效。
\switchcolumn[0]*%%%%%%%
In \texttt{\textless{}script\ setup\textgreater{}}, any camelCase
variable that starts with the \texttt{v} prefix can be used as a custom
directive. In the example above, \texttt{vFocus} can be used in the
template as \texttt{v-focus}.
\switchcolumn
在 \texttt{\textless{}script\ setup\textgreater{}} 中,任何以 \texttt{v}
开头的驼峰式命名的变量都可以被用作一个自定义指令。在上面的例子中,\texttt{vFocus}
即可以在模板中以 \texttt{v-focus} 的形式使用。
\switchcolumn[0]*%%%%%%%
If not using \texttt{\textless{}script\ setup\textgreater{}}, custom
directives can be registered using the \texttt{directives} option:
\switchcolumn
在没有使用 \texttt{\textless{}script\ setup\textgreater{}}
的情况下,自定义指令需要通过 \texttt{directives} 选项注册: 
\switchcolumn[0]*%%%%%%%
\begin{codeJs}
export default {
  setup() {
    /*...*/
  },
  directives: {
    // 在模板中启用 v-focus
    focus: {
      /* ... */
    }
  }
}
\end{codeJs}
\switchcolumn
\begin{codeJs}
export default {
  setup() {
    /*...*/
  },
  directives: {
    // 在模板中启用 v-focus
    focus: {
      /* ... */
    }
  }
}
\end{codeJs}
\switchcolumn[0]*%%%%%%%
It is also common to globally register custom directives at the app
level:
\switchcolumn
将一个自定义指令全局注册到应用层级也是一种常见的做法:
\switchcolumn[0]*%%%%%%%
\begin{codeJs}
const app = createApp({})
// 使 v-focus 在所有组件中都可用
app.directive('focus', {
  /* ... */
})
\end{codeJs}
\switchcolumn
\begin{codeJs}
const app = createApp({})
// 使 v-focus 在所有组件中都可用
app.directive('focus', {
  /* ... */
})
\end{codeJs}
\switchcolumn[0]*%%%%%%%
\begin{vueQuote}{TIP}
Custom directives should only be used when the desired functionality can
only be achieved via direct DOM manipulation. Prefer declarative
templating using built-in directives such as \texttt{v-bind} when
possible because they are more efficient and server-rendering friendly.
\end{vueQuote} 
\switchcolumn
\begin{vueQuote}{TIP}
只有当所需功能只能通过直接的 DOM
操作来实现时,才应该使用自定义指令。其他情况下应该尽可能地使用
\texttt{v-bind}
这样的内置指令来声明式地使用模板,这样更高效,也对服务端渲染更友好。
\end{vueQuote} 
\end{paracol}

\columnratio{0.55}
\begin{paracol}{2}
 
\switchcolumn[0]*%%%%%%%
\subsection{Directive Hooks}
\switchcolumn
\subsection{指令钩子}
\switchcolumn[0]*%%%%%%%
A directive definition object can provide several hook functions (all
optional):
\switchcolumn
一个指令的定义对象可以提供几种钩子函数 (都是可选的):
\switchcolumn[0]*%%%%%%%
\begin{codeJs}
const myDirective = {
  // 在绑定元素的 attribute 前
  // 或事件监听器应用前调用
  created(el, binding, vnode, prevVnode) {
    // 下面会介绍各个参数的细节
  },
  // 在元素被插入到 DOM 前调用
  beforeMount(el, binding, vnode, prevVnode) {},
  // 在绑定元素的父组件
  // 及他自己的所有子节点都挂载完成后调用
  mounted(el, binding, vnode, prevVnode) {},
  // 绑定元素的父组件更新前调用
  beforeUpdate(el, binding, vnode, prevVnode) {},
  // 在绑定元素的父组件
  // 及他自己的所有子节点都更新后调用
  updated(el, binding, vnode, prevVnode) {},
  // 绑定元素的父组件卸载前调用
  beforeUnmount(el, binding, vnode, prevVnode) {},
  // 绑定元素的父组件卸载后调用
  unmounted(el, binding, vnode, prevVnode) {}
}
\end{codeJs}
\switchcolumn
\begin{codeJs}
const myDirective = {
  // 在绑定元素的 attribute 前
  // 或事件监听器应用前调用
  created(el, binding, vnode, prevVnode) {
    // 下面会介绍各个参数的细节
  },
  // 在元素被插入到 DOM 前调用
  beforeMount(el, binding, vnode, prevVnode) {},
  // 在绑定元素的父组件
  // 及他自己的所有子节点都挂载完成后调用
  mounted(el, binding, vnode, prevVnode) {},
  // 绑定元素的父组件更新前调用
  beforeUpdate(el, binding, vnode, prevVnode) {},
  // 在绑定元素的父组件
  // 及他自己的所有子节点都更新后调用
  updated(el, binding, vnode, prevVnode) {},
  // 绑定元素的父组件卸载前调用
  beforeUnmount(el, binding, vnode, prevVnode) {},
  // 绑定元素的父组件卸载后调用
  unmounted(el, binding, vnode, prevVnode) {}
}
\end{codeJs}
\switchcolumn[0]*%%%%%%%
\subsubsection{Hook Arguments}
\switchcolumn
\subsubsection{钩子参数}
\switchcolumn[0]*%%%%%%%
Directive hooks are passed these arguments:
\switchcolumn
指令的钩子会传递以下几种参数:
\switchcolumn[0]*%%%%%%%
\begin{itemize}
\item
  \texttt{el}: the element the directive is bound to. This can be used
  to directly manipulate the DOM.
\item
  \texttt{binding}: an object containing the following properties.
  \begin{itemize}
    \item
      \texttt{value}: The value passed to the directive. For example in
      \texttt{v-my-directive="1\ +\ 1"}, the value would be \texttt{2}.
    \item
      \texttt{oldValue}: The previous value, only available in
      \texttt{beforeUpdate} and \texttt{updated}. It is available whether
      or not the value has changed.
    \item
      \texttt{arg}: The argument passed to the directive, if any. For
      example in \texttt{v-my-directive:foo}, the arg would be
      \texttt{"foo"}.
    \item
      \texttt{modifiers}: An object containing modifiers, if any. For
      example in \texttt{v-my-directive.foo.bar}, the modifiers object
      would be \texttt{\{\ foo:\ true,\ bar:\ true\ \}}.
    \item
      \texttt{instance}: The instance of the component where the directive
      is used.
    \item
      \texttt{dir}: the directive definition object.
    \end{itemize}
  \item
    \texttt{vnode}: the underlying VNode representing the bound element.
  \item
    \texttt{prevNode}: the VNode representing the bound element from the
    previous render. Only available in the \texttt{beforeUpdate} and
    \texttt{updated} hooks.
  \end{itemize}
\switchcolumn
\begin{itemize}
\item
  \texttt{el}:指令绑定到的元素。这可以用于直接操作 DOM。
\item
  \texttt{binding}:一个对象,包含以下属性。
  \begin{itemize}
    \item
      \texttt{value}:传递给指令的值。例如在
      \texttt{v-my-directive="1\ +\ 1"} 中,值是 \texttt{2}。
    \item
      \texttt{oldValue}:之前的值,仅在 \texttt{beforeUpdate} 和
      \texttt{updated} 中可用。无论值是否更改,它都可用。
    \item
      \texttt{arg}:传递给指令的参数 (如果有的话)。例如在
      \texttt{v-my-directive:foo} 中,参数是 \texttt{"foo"}。
    \item
      \texttt{modifiers}:一个包含修饰符的对象 (如果有的话)。例如在
      \texttt{v-my-directive.foo.bar} 中,修饰符对象是
      \texttt{\{\ foo:\ true,\ bar:\ true\ \}}。
    \item
      \texttt{instance}:使用该指令的组件实例。
    \item
      \texttt{dir}:指令的定义对象。
    \end{itemize}
  \item
    \texttt{vnode}:代表绑定元素的底层 VNode。
  \item
    \texttt{prevNode}:代表之前的渲染中指令所绑定元素的 VNode。仅在
    \texttt{beforeUpdate} 和 \texttt{updated} 钩子中可用。
  \end{itemize}
\end{paracol}

\columnratio{0.55}
\begin{paracol}{2}
 
\switchcolumn[0]*%%%%%%%
As an example, consider the following directive usage:
\switchcolumn
举例来说,像下面这样使用指令:
\switchcolumn[0]*%%%%%%%
\begin{codeHtml}
<div v-example:foo.bar="baz">
\end{codeHtml}
\switchcolumn
\begin{codeHtml}
<div v-example:foo.bar="baz">
\end{codeHtml}
\switchcolumn[0]*%%%%%%%
The \texttt{binding} argument would be an object in the shape of:
\switchcolumn
\texttt{binding} 参数会是一个这样的对象:
\switchcolumn[0]*%%%%%%%
\begin{codeJs}
{
  arg: 'foo',
  modifiers: { bar: true },
  value: /* `baz` 的值 */,
  oldValue: /* 上一次更新时 `baz` 的值 */
}
\end{codeJs}
\switchcolumn
\begin{codeJs}
{
  arg: 'foo',
  modifiers: { bar: true },
  value: /* `baz` 的值 */,
  oldValue: /* 上一次更新时 `baz` 的值 */
}
\end{codeJs}
\switchcolumn[0]*%%%%%%%
Similar to built-in directives, custom directive arguments can be
dynamic. For example:
\switchcolumn
和内置指令类似,自定义指令的参数也可以是动态的。举例来说:
\switchcolumn[0]*%%%%%%%
\begin{codeHtml}
<div v-example:[arg]="value"></div>
\end{codeHtml}
\switchcolumn
\begin{codeHtml}
<div v-example:[arg]="value"></div>
\end{codeHtml}
\switchcolumn[0]*%%%%%%%
Here the directive argument will be reactively updated based on
\texttt{arg} property in our component state.
\switchcolumn
这里指令的参数会基于组件的 \texttt{arg} 数据属性响应式地更新。
\switchcolumn[0]*%%%%%%%
\begin{vueQuote}{Note}
Apart from \texttt{el}, you should treat these arguments as read-only
and never modify them. If you need to share information across hooks, it
is recommended to do so through element's
\href{https://developer.mozilla.org/en-US/docs/Web/API/HTMLElement/dataset}{dataset}.
\end{vueQuote} 
\switchcolumn
\begin{vueQuote}{Note}
除了 \texttt{el}
外,其他参数都是只读的,不要更改它们。若你需要在不同的钩子间共享信息,推荐通过元素的
\href{https://developer.mozilla.org/en-US/docs/Web/API/HTMLElement/dataset}{dataset}
attribute 实现。
\end{vueQuote} 
\end{paracol}

\columnratio{0.55}
\begin{paracol}{2}
 
\switchcolumn[0]*%%%%%%%
\subsection{Function Shorthand}
\switchcolumn
\subsection{简化形式}
\switchcolumn[0]*%%%%%%%
It's common for a custom directive to have the same behavior for
\texttt{mounted} and \texttt{updated}, with no need for the other hooks.
In such cases we can define the directive as a function:
\switchcolumn
对于自定义指令来说,一个很常见的情况是仅仅需要在 \texttt{mounted} 和
\texttt{updated}
上实现相同的行为,除此之外并不需要其他钩子。这种情况下我们可以直接用一个函数来定义指令,如下所示:
\switchcolumn[0]*%%%%%%%
\begin{codeHtml}
<div v-color="color"></div>
\end{codeHtml}
\switchcolumn
\begin{codeHtml}
<div v-color="color"></div>
\end{codeHtml}
\switchcolumn[0]*%%%%%%%
\begin{codeJs}
app.directive('color', (el, binding) => {
  // 这会在 `mounted` 和 `updated` 时都调用
  el.style.color = binding.value
})
\end{codeJs}
\switchcolumn
\begin{codeJs}
app.directive('color', (el, binding) => {
  // 这会在 `mounted` 和 `updated` 时都调用
  el.style.color = binding.value
})
\end{codeJs}
\switchcolumn[0]*%%%%%%%
\subsection{Object Literals}
\switchcolumn
\subsection{对象字面量}
\switchcolumn[0]*%%%%%%%
If your directive needs multiple values, you can also pass in a
JavaScript object literal. Remember, directives can take any valid
JavaScript expression.
\switchcolumn
如果你的指令需要多个值,你可以向它传递一个 JavaScript
对象字面量。别忘了,指令也可以接收任何合法的 JavaScript 表达式。
\switchcolumn[0]*%%%%%%%
\begin{codeHtml}
<div v-demo="{ color: 'white', text: 'hello!' }"></div>
\end{codeHtml}
\switchcolumn
\begin{codeHtml}
<div v-demo="{ color: 'white', text: 'hello!' }"></div>
\end{codeHtml}
\switchcolumn[0]*%%%%%%%
\begin{codeJs}
app.directive('demo', (el, binding) => {
  console.log(binding.value.color) // => "white"
  console.log(binding.value.text) // => "hello!"
})
\end{codeJs}
\switchcolumn
\begin{codeJs}
app.directive('demo', (el, binding) => {
  console.log(binding.value.color) // => "white"
  console.log(binding.value.text) // => "hello!"
})
\end{codeJs}
 
\switchcolumn[0]*%%%%%%%
\subsection{Usage on Components}
\switchcolumn
\subsection{在组件上使用}
\switchcolumn[0]*%%%%%%%
When used on components, custom directives will always apply to a
component's root node, similar to
\href{https://vuejs.org/guide/components/attrs.html}{Fallthrough
Attributes}.
\switchcolumn
当在组件上使用自定义指令时,它会始终应用于组件的根节点,和\href{https://cn.vuejs.org/guide/components/attrs.html}{透传
attributes} 类似。
\switchcolumn[0]*%%%%%%%
\begin{codeHtml}
<MyComponent v-demo="test" />
\end{codeHtml}
\switchcolumn
\begin{codeHtml}
<MyComponent v-demo="test" />
\end{codeHtml}
\switchcolumn[0]*%%%%%%%
\begin{codeHtml}
<!-- MyComponent 的模板 -->
<div> <!-- v-demo 指令会被应用在此处 -->
  <span>My component content</span>
</div>
\end{codeHtml}
\switchcolumn
\begin{codeHtml}
<!-- MyComponent 的模板 -->
<div> <!-- v-demo 指令会被应用在此处 -->
  <span>My component content</span>
</div>
\end{codeHtml}
\switchcolumn[0]*%%%%%%%
Note that components can potentially have more than one root node. When
applied to a multi-root component, a directive will be ignored and a
warning will be thrown. Unlike attributes, directives can't be passed to
a different element with \texttt{v-bind="\$attrs"}. In general, it is
\textbf{not} recommended to use custom directives on components.
\switchcolumn
需要注意的是组件可能含有多个根节点。当应用到一个多根组件时,指令将会被忽略且抛出一个警告。和
attribute 不同,指令不能通过 \texttt{v-bind="\$attrs"}
来传递给一个不同的元素。总的来说,\textbf{不}推荐在组件上使用自定义指令。
\end{paracol}

