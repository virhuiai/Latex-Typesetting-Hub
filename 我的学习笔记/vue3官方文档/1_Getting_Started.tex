
% [Introduction](https://vuejs.org/guide/introduction)[Quick Start](https://vuejs.org/guide/quick-start)
% [简介](https://cn.vuejs.org/guide/introduction.html)[快速上手](https://cn.vuejs.org/guide/quick-start.html)
\columnratio{0.55}
\begin{paracol}{2}
\section{Getting Started}
\switchcolumn
\section{开始}
\switchcolumn[0]*%%%%%%%
\begin{vueQuote}{You are reading the documentation for Vue 3!}
\begin{itemize}
    \item
      Vue 2 support will end on Dec 31, 2023. Learn more about
      \href{https://v2.vuejs.org/lts/}{Vue 2 Extended LTS}.
    \item
      Vue 2 documentation has been moved to
      \href{https://v2.vuejs.org/}{v2.vuejs.org}.
    \item
      Upgrading from Vue 2? Check out the
      \href{https://v3-migration.vuejs.org/}{Migration Guide}.
    \end{itemize}
\end{vueQuote}
\switchcolumn
\begin{vueQuote}{你正在阅读的是 Vue 3 的文档!}
\begin{itemize}
\item
    Vue 2 将于 2023 年 12 月 31 日停止维护。详见
    \href{https://v2.vuejs.org/lts/}{Vue 2 延长 LTS}。
\item
    Vue 2 中文文档已迁移至
    \href{https://v2.cn.vuejs.org/}{v2.cn.vuejs.org}。
\item
    想从 Vue 2
    升级?请参考\href{https://v3-migration.vuejs.org/}{迁移指南}。
\end{itemize}
\end{vueQuote}
\switchcolumn[0]*%%%%%%%
\subsection{What is Vue?}
\switchcolumn
\subsection{什么是 Vue?}
\switchcolumn[0]*%%%%%%%
Vue (pronounced {\fontGentiumPlus /vjuː/}, like \textbf{view}) is a JavaScript framework
for building user interfaces. It builds on top of standard HTML, CSS,
and JavaScript and provides a declarative and component-based
programming model that helps you efficiently develop user interfaces, be
they simple or complex.
\switchcolumn
Vue (发音为 {\fontGentiumPlus /vjuː/},类似 \textbf{view}) 是用于构建用户界面的
JavaScript 框架。它基于标准 HTML、CSS 和 JavaScript
构建,并提供了一套声明式的、组件化的编程模型,助你高效地开发用户界面。无论是简单还是复杂的界面,Vue
都可以胜任。
\switchcolumn[0]*%%%%%%%
Here is a minimal example:
\switchcolumn
下面是一个最基本的示例:
\switchcolumn[0]*%%%%%%%
\begin{codeJs*}{label={js}}
import { createApp, ref } from 'vue'

createApp({
  setup() {
    return {
      count: ref(0)
    }
  }
}).mount('#app')
\end{codeJs*}
\switchcolumn
\begin{codeJs*}{label={js:UMD浏览器引用JS方式}}
const {createApp,ref} = Vue;

createApp({
    setup() {
    return {
        count: ref(0)
    }
    }
}).mount('#app')
\end{codeJs*}
\switchcolumn[0]*%%%%%%%
\begin{codeHtml*}{label={template}}
<div id="app">
    <button @click="count++">
        Count is: {{ count }}
    </button>
</div>
\end{codeHtml*}
\switchcolumn
\begin{codeHtml*}{label={template}}
<div id="app">
    <button @click="count++">
        Count is: {{ count }}
    </button>
</div>
\end{codeHtml*}
\switchcolumn[0]*%%%%%%%
The above example demonstrates the two core features of Vue:
\switchcolumn
上面的示例展示了 Vue 的两个核心功能:
\end{paracol}

\columnratio{0.55}
\begin{itemize}
\begin{paracol}{2}
\item
\textbf{Declarative Rendering}: Vue extends standard HTML with a
template syntax that allows us to declaratively describe HTML output
based on JavaScript state.
\switchcolumn
\item
\textbf{声明式渲染}:Vue 基于标准 HTML
拓展了一套模板语法,使得我们可以声明式地描述最终输出的 HTML 和
JavaScript 状态之间的关系。
\switchcolumn[0]*%%%%%%%
\item
\textbf{Reactivity}: Vue automatically tracks JavaScript state changes
and efficiently updates the DOM when changes happen.
\switchcolumn
\item
\textbf{响应性}:Vue 会自动跟踪 JavaScript
状态并在其发生变化时响应式地更新 DOM。
\end{paracol}
\end{itemize}


\columnratio{0.55}
\begin{paracol}{2}
\switchcolumn[0]*%%%%%%%
You may already have questions - don't worry. We will cover every little
detail in the rest of the documentation. For now, please read along so
you can have a high-level understanding of what Vue offers.
\switchcolumn
你可能已经有了些疑问------先别急,在后续的文档中我们会详细介绍每一个细节。现在,请继续看下去,以确保你对
Vue 作为一个框架到底提供了什么有一个宏观的了解。
\switchcolumn[0]*%%%%%%%
\begin{vueQuote}
{Prerequisites}
The rest of the documentation assumes basic familiarity with HTML, CSS,
and JavaScript. If you are totally new to frontend development, it might
not be the best idea to jump right into a framework as your first step -
grasp the basics and then come back! You can check your knowledge level
with
\href{https://developer.mozilla.org/en-US/docs/Web/JavaScript/A_re-introduction_to_JavaScript}{this
JavaScript overview}. Prior experience with other frameworks helps, but
is not required.
\end{vueQuote}
\switchcolumn

\begin{vueQuote}{预备知识}
文档接下来的内容会假设你对 HTML、CSS 和 JavaScript
已经基本熟悉。如果你对前端开发完全陌生,最好不要直接从一个框架开始进行入门学习------最好是掌握了基础知识再回到这里。你可以通过这篇
\href{https://developer.mozilla.org/zh-CN/docs/Web/JavaScript/A_re-introduction_to_JavaScript}{JavaScript
概述}来检验你的 JavaScript
知识水平。如果之前有其他框架的经验会很有帮助,但也不是必须的。
\end{vueQuote}
\switchcolumn[0]*%%%%%%%
\subsection{The Progressive Framework}
\switchcolumn
\subsection{渐进式框架}
\switchcolumn[0]*%%%%%%%
Vue is a framework and ecosystem that covers most of the common features
needed in frontend development. But the web is extremely diverse - the
things we build on the web may vary drastically in form and scale. With
that in mind, Vue is designed to be flexible and incrementally
adoptable. Depending on your use case, Vue can be used in different
ways:
\switchcolumn
Vue 是一个框架,也是一个生态。其功能覆盖了大部分前端开发常见的需求。但
Web 世界是十分多样化的,不同的开发者在 Web
上构建的东西可能在形式和规模上会有很大的不同。考虑到这一点,Vue
的设计非常注重灵活性和``可以被逐步集成''这个特点。根据你的需求场景,你可以用不同的方式使用
Vue:
\switchcolumn[0]*%%%%%%%
\begin{itemize}
\item
    Enhancing static HTML without a build step
\item
    Embedding as Web Components on any page
\item
    Single-Page Application (SPA)
\item
    Fullstack / Server-Side Rendering (SSR)
\item
    Jamstack / Static Site Generation (SSG)
\item
    Targeting desktop, mobile, WebGL, and even the terminal
\end{itemize}
\switchcolumn
\begin{itemize}
\item
    无需构建步骤,渐进式增强静态的 HTML
\item
    在任何页面中作为 Web Components 嵌入
\item
    单页应用 (SPA)
\item
    全栈 / 服务端渲染 (SSR)
\item
    Jamstack / 静态站点生成 (SSG)
\item
    开发桌面端、移动端、WebGL,甚至是命令行终端中的界面
\end{itemize}
\switchcolumn[0]*%%%%%%%
If you find these concepts intimidating, don't worry! The tutorial and
guide only require basic HTML and JavaScript knowledge, and you should
be able to follow along without being an expert in any of these.
\switchcolumn
如果你是初学者,可能会觉得这些概念有些复杂。别担心!理解教程和指南的内容只需要具备基础的
HTML 和 JavaScript 知识。即使你不是这些方面的专家,也能够跟得上。
\switchcolumn[0]*%%%%%%%
If you are an experienced developer interested in how to best integrate
Vue into your stack, or you are curious about what these terms mean, we
discuss them in more detail in
\href{https://vuejs.org/guide/extras/ways-of-using-vue}{Ways of Using
Vue}.
\switchcolumn
如果你是有经验的开发者,希望了解如何以最合适的方式在项目中引入
Vue,或者是对上述的这些概念感到好奇,我们在\href{https://cn.vuejs.org/guide/extras/ways-of-using-vue.html}{使用
Vue 的多种方式}中讨论了有关它们的更多细节。
\switchcolumn[0]*%%%%%%%
Despite the flexibility, the core knowledge about how Vue works is
shared across all these use cases. Even if you are just a beginner now,
the knowledge gained along the way will stay useful as you grow to
tackle more ambitious goals in the future. If you are a veteran, you can
pick the optimal way to leverage Vue based on the problems you are
trying to solve, while retaining the same productivity. This is why we
call Vue "The Progressive Framework": it's a framework that can grow
with you and adapt to your needs.
\switchcolumn
无论再怎么灵活,Vue
的核心知识在所有这些用例中都是通用的。即使你现在只是一个初学者,随着你的不断成长,到未来有能力实现更复杂的项目时,这一路上获得的知识依然会适用。如果你已经是一个老手,你可以根据实际场景来选择使用
Vue 的最佳方式,在各种场景下都可以保持同样的开发效率。这就是为什么我们将
Vue 称为``渐进式框架'':它是一个可以与你共同成长、适应你不同需求的框架。
\switchcolumn[0]*%%%%%%%
\subsection{Single-File Components}
\switchcolumn
\subsection{单文件组件}
\switchcolumn[0]*%%%%%%%
In most build-tool-enabled Vue projects, we author Vue components using
an HTML-like file format called \textbf{Single-File Component} (also
known as \texttt{*.vue} files, abbreviated as \textbf{SFC}). A Vue SFC,
as the name suggests, encapsulates the component's logic (JavaScript),
template (HTML), and styles (CSS) in a single file. Here's the previous
example, written in SFC format:
\switchcolumn
在大多数启用了构建工具的 Vue 项目中,我们可以使用一种类似 HTML
格式的文件来书写 Vue 组件,它被称为\textbf{单文件组件} (也被称为
\texttt{*.vue} 文件,英文 Single-File Components,缩写为
\textbf{SFC})。顾名思义,Vue 的单文件组件会将一个组件的逻辑
(JavaScript),模板 (HTML) 和样式 (CSS)
封装在同一个文件里。下面我们将用单文件组件的格式重写上面的计数器示例:
\switchcolumn[0]*%%%%%%%
\begin{codeVue}
<script setup>
import { ref } from 'vue'
const count = ref(0)
</script>

<template>
    <button @click="count++">Count is: {{ count }}</button>
</template>

<style scoped>
button {
    font-weight: bold;
}
</style>
\end{codeVue}
\switchcolumn
\begin{codeVue}
<script setup>
import { ref } from 'vue'
const count = ref(0)
</script>

<template>
    <button @click="count++">Count is: {{ count }}</button>
</template>

<style scoped>
button {
    font-weight: bold;
}
</style>
\end{codeVue}
\switchcolumn[0]*%%%%%%%
SFC is a defining feature of Vue and is the recommended way to author
Vue components \textbf{if} your use case warrants a build setup. You can
learn more about the \href{https://vuejs.org/guide/scaling-up/sfc}{how
and why of SFC} in its dedicated section - but for now, just know that
Vue will handle all the build tools setup for you.
\switchcolumn
单文件组件是 Vue
的标志性功能。如果你的用例需要进行构建,我们推荐用它来编写 Vue
组件。你可以在后续相关章节里了解更多关于\href{https://cn.vuejs.org/guide/scaling-up/sfc.html}{单文件组件的用法及用途}。但你暂时只需要知道
Vue 会帮忙处理所有这些构建工具的配置就好。
\switchcolumn[0]*%%%%%%%
\subsection{API Styles}
\switchcolumn
\subsection{API 风格}
\switchcolumn[0]*%%%%%%%
Vue components can be authored in two different API styles:\textbf{Options API} and \textbf{Composition API}.
\switchcolumn
Vue 的组件可以按两种不同的风格书写:\textbf{选项式 API} 和\textbf{组合式
API}。
\switchcolumn[0]*%%%%%%%
\subsubsection{Options API}
\switchcolumn
\subsubsection{选项式 API (Options API)}
\switchcolumn[0]*%%%%%%%
With Options API, we define a component's logic using an object of
options such as \texttt{data}, \texttt{methods}, and \texttt{mounted}.
Properties defined by options are exposed on \texttt{this} inside
functions, which points to the component instance:
\switchcolumn
使用选项式 API,我们可以用包含多个选项的对象来描述组件的逻辑,例如
\texttt{data}、\texttt{methods} 和
\texttt{mounted}。选项所定义的属性都会暴露在函数内部的 \texttt{this}
上,它会指向当前的组件实例。
\switchcolumn[0]*%%%%%%%
\begin{codeVue}
    <script>
    export default {
      // Properties returned from data() become reactive state
      // and will be exposed on `this`.
      data() {
        return {
          count: 0
        }
      },
    
      // Methods are functions that mutate state and trigger updates.
      // They can be bound as event handlers in templates.
      methods: {
        increment() {
          this.count++
        }
      },
    
      // Lifecycle hooks are called at different stages
      // of a component's lifecycle.
      // This function will be called when the component is mounted.
      mounted() {
        console.log(`The initial count is ${this.count}.`)
      }
    }
    </script>
    
    <template>
      <button @click="increment">Count is: {{ count }}</button>
    </template>
\end{codeVue}
\switchcolumn
\begin{codeVue}
    <script>
    export default {
      // data() 返回的属性将会成为响应式的状态
      // 并且暴露在 `this` 上
      data() {
        return {
          count: 0
        }
      },
    
      // methods 是一些用来更改状态与触发更新的函数
      // 它们可以在模板中作为事件处理器绑定
      methods: {
        increment() {
          this.count++
        }
      },
    
      // 生命周期钩子会在组件生命周期的各个不同阶段被调用
      // 例如这个函数就会在组件挂载完成后被调用
      mounted() {
        console.log(`The initial count is ${this.count}.`)
      }
    }
    </script>
    
    <template>
      <button @click="increment">Count is: {{ count }}</button>
    </template>
\end{codeVue}
\switchcolumn[0]*%%%%%%%$
\subsubsection{Composition API}
\switchcolumn
\subsubsection{组合式 API (Composition API)}
\switchcolumn[0]*%%%%%%%
With Composition API, we define a component's logic using imported API
functions. In SFCs, Composition API is typically used with
\href{https://vuejs.org/api/sfc-script-setup}{``}. The \texttt{setup}
attribute is a hint that makes Vue perform compile-time transforms that
allow us to use Composition API with less boilerplate. For example,
imports and top-level variables / functions declared in
\texttt{\textless{}script\ setup\textgreater{}} are directly usable in
the template.
\switchcolumn
通过组合式 API,我们可以使用导入的 API
函数来描述组件逻辑。在单文件组件中,组合式 API 通常会与
\href{https://cn.vuejs.org/api/sfc-script-setup.html}{``} 搭配使用。这个
\texttt{setup} attribute 是一个标识,告诉 Vue
需要在编译时进行一些处理,让我们可以更简洁地使用组合式
API。比如,\texttt{\textless{}script\ setup\textgreater{}}
中的导入和顶层变量/函数都能够在模板中直接使用。
\switchcolumn[0]*%%%%%%%
Here is the same component, with the exact same template, but using
Composition API and \texttt{\textless{}script\ setup\textgreater{}}
instead:
\switchcolumn
下面是使用了组合式 API 与
\texttt{\textless{}script\ setup\textgreater{}}
改造后和上面的模板完全一样的组件:
\switchcolumn[0]*%%%%%%%
\begin{codeVue}
    <script setup>
    import { ref, onMounted } from 'vue'
    
    // reactive state
    const count = ref(0)
    
    // functions that mutate state and trigger updates
    function increment() {
      count.value++
    }
    
    // lifecycle hooks
    onMounted(() => {
      console.log(`The initial count is ${count.value}.`)
    })
    </script>
    
    <template>
      <button @click="increment">Count is: {{ count }}</button>
    </template>
\end{codeVue}    
\switchcolumn
\begin{codeVue}    
    <script setup>
    import { ref, onMounted } from 'vue'
    
    // 响应式状态
    const count = ref(0)
    
    // 用来修改状态、触发更新的函数
    function increment() {
      count.value++
    }
    
    // 生命周期钩子
    onMounted(() => {
      console.log(`The initial count is ${count.value}.`)
    })
    </script>
    
    <template>
      <button @click="increment">Count is: {{ count }}</button>
    </template>
\end{codeVue}    
\switchcolumn[0]*%%%%%%%
\subsubsection{Which to Choose?}
\switchcolumn
\subsubsection{该选哪一个?}
\switchcolumn[0]*%%%%%%%
Both API styles are fully capable of covering common use cases. They are
different interfaces powered by the exact same underlying system. In
fact, the Options API is implemented on top of the Composition API! The
fundamental concepts and knowledge about Vue are shared across the two
styles.
\switchcolumn
两种 API
风格都能够覆盖大部分的应用场景。它们只是同一个底层系统所提供的两套不同的接口。实际上,选项式
API 是在组合式 API 的基础上实现的!关于 Vue
的基础概念和知识在它们之间都是通用的。
\switchcolumn[0]*%%%%%%%
The Options API is centered around the concept of a "component instance"
(\texttt{this} as seen in the example), which typically aligns better
with a class-based mental model for users coming from OOP language
backgrounds. It is also more beginner-friendly by abstracting away the
reactivity details and enforcing code organization via option groups.
\switchcolumn
选项式 API 以``组件实例''的概念为中心 (即上述例子中的
\texttt{this}),对于有面向对象语言背景的用户来说,这通常与基于类的心智模型更为一致。同时,它将响应性相关的细节抽象出来,并强制按照选项来组织代码,从而对初学者而言更为友好。
\switchcolumn[0]*%%%%%%%
The Composition API is centered around declaring reactive state
variables directly in a function scope and composing state from multiple
functions together to handle complexity. It is more free-form and
requires an understanding of how reactivity works in Vue to be used
effectively. In return, its flexibility enables more powerful patterns
for organizing and reusing logic.
\switchcolumn
组合式 API
的核心思想是直接在函数作用域内定义响应式状态变量,并将从多个函数中得到的状态组合起来处理复杂问题。这种形式更加自由,也需要你对
Vue
的响应式系统有更深的理解才能高效使用。相应的,它的灵活性也使得组织和重用逻辑的模式变得更加强大。
\switchcolumn[0]*%%%%%%%
You can learn more about the comparison between the two styles and the
potential benefits of Composition API in the
\href{https://vuejs.org/guide/extras/composition-api-faq}{Composition
API FAQ}.
\switchcolumn
在\href{https://cn.vuejs.org/guide/extras/composition-api-faq.html}{组合式
API FAQ} 章节中,你可以了解更多关于这两种 API 风格的对比以及组合式 API
所带来的潜在收益。
\switchcolumn[0]*%%%%%%%
If you are new to Vue, here's our general recommendation:
\switchcolumn
如果你是使用 Vue 的新手,这里是我们的大致建议:
\switchcolumn[0]*%%%%%%%
\begin{itemize}
\item
For learning purposes, go with the style that looks easier to
understand to you. Again, most of the core concepts are shared between
the two styles. You can always pick up the other style later.
\item
For production use:

\begin{itemize}
\item
Go with Options API if you are not using build tools, or plan to use
Vue primarily in low-complexity scenarios, e.g. progressive
enhancement.
\item
Go with Composition API + Single-File Components if you plan to
build full applications with Vue.
\end{itemize}
\end{itemize}
\switchcolumn
\begin{itemize}
\item
在学习的过程中,推荐采用更易于自己理解的风格。再强调一下,大部分的核心概念在这两种风格之间都是通用的。熟悉了一种风格以后,你也能够很快地理解另一种风格。
\item
在生产项目中:

\begin{itemize}
\item
当你不需要使用构建工具,或者打算主要在低复杂度的场景中使用
Vue,例如渐进增强的应用场景,推荐采用选项式 API。
\item
当你打算用 Vue 构建完整的单页应用,推荐采用组合式 API + 单文件组件。
\end{itemize}
\end{itemize}
\switchcolumn[0]*%%%%%%%
You don't have to commit to only one style during the learning phase.
The rest of the documentation will provide code samples in both styles
where applicable, and you can toggle between them at any time using the
\textbf{API Preference switches} at the top of the left sidebar.
\switchcolumn
在学习阶段,你不必只固守一种风格。在接下来的文档中我们会为你提供一系列两种风格的代码供你参考,你可以随时通过左上角的
\textbf{API 风格偏好}来做切换。
\switchcolumn[0]*%%%%%%%
\subsection{Still Got Questions?}
Check out our \href{https://vuejs.org/about/faq}{FAQ}.
\switchcolumn
\subsection{还有其他问题?}
请查看我们的 \href{https://cn.vuejs.org/about/faq.html}{FAQ}。
\switchcolumn[0]*%%%%%%%
\subsection{Pick Your Learning Path}
\switchcolumn
\subsection{选择你的学习路径}
\switchcolumn[0]*%%%%%%%
Different developers have different learning styles. Feel free to pick a
learning path that suits your preference - although we do recommend
going over all of the content, if possible!
\switchcolumn
不同的开发者有不同的学习方式。尽管在可能的情况下,我们推荐你通读所有内容,但你还是可以自由地选择一种自己喜欢的学习路径!
\switchcolumn[0]*%%%%%%%
\href{https://vuejs.org/tutorial/}{Try the TutorialFor those who prefer
learning things
hands-on.}\href{https://vuejs.org/guide/quick-start}{Read the GuideThe
guide walks you through every aspect of the framework in full
detail.}\href{https://vuejs.org/examples/}{Check out the ExamplesExplore
examples of core features and common UI tasks.}
\switchcolumn
\href{https://cn.vuejs.org/tutorial/}{尝试互动教程适合喜欢边动手边学的读者。}\href{https://cn.vuejs.org/guide/quick-start.html}{继续阅读该指南该指南会带你深入了解框架所有方面的细节。}\href{https://cn.vuejs.org/examples/}{查看示例浏览核心功能和常见用户界面的示例。}
% \switchcolumn[0]*%%%%%%%
% \href{https://github.com/vuejs/docs/edit/main/src/guide/introduction.md}{Edit
% this page on GitHub}
% \switchcolumn
% \href{https://github.com/vuejs-translations/docs-zh-cn/edit/main/src/guide/introduction.md}{在
% GitHub 上编辑此页}
\end{paracol}



