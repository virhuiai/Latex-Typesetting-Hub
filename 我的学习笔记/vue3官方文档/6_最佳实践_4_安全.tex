
\columnratio{0.55}
\begin{paracol}{2} 
\switchcolumn[0]*%%%%%%%
\section{Security}
\switchcolumn
\section{安全}
\switchcolumn[0]*%%%%%%%
\subsection{Reporting Vulnerabilities}
\switchcolumn
\subsection{报告漏洞}
\switchcolumn[0]*%%%%%%%
When a vulnerability is reported, it immediately becomes our top
concern, with a full-time contributor dropping everything to work on it.
To report a vulnerability, please email
\href{mailto:security@vuejs.org}{\nolinkurl{security@vuejs.org}}.
\switchcolumn
当一个漏洞被上报时,它会立刻成为我们最关心的问题,会有全职的贡献者暂时搁置其他所有任务来解决这个问题。如需报告漏洞,请发送电子邮件至
\href{mailto:security@vuejs.org}{\nolinkurl{security@vuejs.org}}。
\switchcolumn[0]*%%%%%%%
While the discovery of new vulnerabilities is rare, we also recommend
always using the latest versions of Vue and its official companion
libraries to ensure your application remains as secure as possible.
\switchcolumn
虽然很少发现新的漏洞,但我们仍建议始终使用最新版本的 Vue
及其官方配套库,以确保你的应用尽可能地安全。
\switchcolumn[0]*%%%%%%%
\subsection{Rule No.1: Never Use Non-trusted Templates}
\switchcolumn
\subsection{首要规则:不要使用无法信赖的模板}
\switchcolumn[0]*%%%%%%%
The most fundamental security rule when using Vue is \textbf{never use
non-trusted content as your component template}. Doing so is equivalent
to allowing arbitrary JavaScript execution in your application - and
worse, could lead to server breaches if the code is executed during
server-side rendering. An example of such usage:
\switchcolumn
使用 Vue
时最基本的安全规则就是\textbf{不要将无法信赖的内容作为你的组件模板}。使用无法信赖的模板相当于允许任意的
JavaScript
在你的应用中执行。更糟糕的是,如果在服务端渲染时执行了这些代码,可能会导致服务器被攻击。举例来说:
\switchcolumn[0]*%%%%%%%
\begin{codeJs}
Vue.createApp({
  template: `<div>` + userProvidedString + `</div>` // 永远不要这样做!
}).mount('#app')
\end{codeJs}
\switchcolumn
\begin{codeJs}
Vue.createApp({
  template: `<div>` + userProvidedString + `</div>` // 永远不要这样做!
}).mount('#app')
\end{codeJs}
\switchcolumn[0]*%%%%%%%
Vue templates are compiled into JavaScript, and expressions inside
templates will be executed as part of the rendering process. Although
the expressions are evaluated against a specific rendering context, due
to the complexity of potential global execution environments, it is
impractical for a framework like Vue to completely shield you from
potential malicious code execution without incurring unrealistic
performance overhead. The most straightforward way to avoid this
category of problems altogether is to make sure the contents of your Vue
templates are always trusted and entirely controlled by you.
\switchcolumn
Vue 模板会被编译成
JavaScript,而模板内的表达式将作为渲染过程的一部分被执行。尽管这些表达式在特定的渲染环境中执行,但由于全局执行环境的复杂性,Vue
作为一个开发框架,要在性能开销合理的前提下完全避免潜在的恶意代码执行是不现实的。避免这类问题最直接的方法是确保你的
Vue 模板始终是可信的,并且完全由你控制。
\end{paracol}



\columnratio{0.55}
\begin{paracol}{2} 
\switchcolumn[0]*%%%%%%%
\subsection{What Vue Does to Protect You}
\switchcolumn
\subsection{Vue 自身的安全机制}
\switchcolumn[0]*%%%%%%%
\subsubsection{HTML content}
\switchcolumn
\subsubsection{HTML 内容}
\switchcolumn[0]*%%%%%%%
Whether using templates or render functions, content is automatically
escaped. That means in this template:
\switchcolumn
无论是使用模板还是渲染函数,内容都是自动转义的。这意味着在这个模板中:
\switchcolumn[0]*%%%%%%%
\begin{codeHtml}
<h1>{{ userProvidedString }}</h1>
\end{codeHtml}
\switchcolumn
\begin{codeHtml}
<h1>{{ userProvidedString }}</h1>
\end{codeHtml}
\switchcolumn[0]*%%%%%%%
if \texttt{userProvidedString} contained:
\switchcolumn
如果 \texttt{userProvidedString} 包含了:
\switchcolumn[0]*%%%%%%%
\begin{codeJs}
'<script>alert("hi")</script>'
\end{codeJs}
\switchcolumn
\begin{codeJs}
'<script>alert("hi")</script>'
\end{codeJs}
\switchcolumn[0]*%%%%%%%
then it would be escaped to the following HTML:
\switchcolumn
那么它将被转义为如下的 HTML:
\switchcolumn[0]*%%%%%%%
\begin{codeHtml}
&lt;script&gt;alert(&quot;hi&quot;)&lt;/script&gt;
\end{codeHtml}
\switchcolumn
\begin{codeHtml}
&lt;script&gt;alert(&quot;hi&quot;)&lt;/script&gt;
\end{codeHtml}
\switchcolumn[0]*%%%%%%%
thus preventing the script injection. This escaping is done using native
browser APIs, like \texttt{textContent}, so a vulnerability can only
exist if the browser itself is vulnerable.
\switchcolumn
从而防止脚本注入。这种转义是使用 \texttt{textContent} 这样的浏览器原生
API 完成的,所以只有当浏览器本身存在漏洞时,才会存在漏洞。
\switchcolumn[0]*%%%%%%%
\subsubsection{Attribute bindings}
\switchcolumn
\subsubsection{Attribute 绑定}
\switchcolumn[0]*%%%%%%%
Similarly, dynamic attribute bindings are also automatically escaped.
That means in this template:
\switchcolumn
同样地,动态 attribute 的绑定也会被自动转义。这意味着在这个模板中:
\switchcolumn[0]*%%%%%%%
\begin{codeHtml}
<h1 :title="userProvidedString">
  hello
</h1>
\end{codeHtml}
\switchcolumn
\begin{codeHtml}
<h1 :title="userProvidedString">
  hello
</h1>
\end{codeHtml}
\switchcolumn[0]*%%%%%%%
if \texttt{userProvidedString} contained:
\switchcolumn
如果 \texttt{userProvidedString} 包含了:
\switchcolumn[0]*%%%%%%%
\begin{codeJs}
'" onclick="alert(\'hi\')'
\end{codeJs}
\switchcolumn
\begin{codeJs}
'" onclick="alert(\'hi\')'
\end{codeJs}
\switchcolumn[0]*%%%%%%%
then it would be escaped to the following HTML:
\switchcolumn
那么它将被转义为如下的 HTML:
\switchcolumn[0]*%%%%%%%
\begin{codeHtml}
&quot; onclick=&quot;alert('hi')
\end{codeHtml}
\switchcolumn
\begin{codeHtml}
&quot; onclick=&quot;alert('hi')
\end{codeHtml}
\switchcolumn[0]*%%%%%%%
thus preventing the close of the \texttt{title} attribute to inject new,
arbitrary HTML. This escaping is done using native browser APIs, like
\texttt{setAttribute}, so a vulnerability can only exist if the browser
itself is vulnerable.
\switchcolumn
从而防止在 \texttt{title} attribute 解析时,注入任意的
HTML。这种转义是使用 \texttt{setAttribute} 这样的浏览器原生 API
完成的,所以只有当浏览器本身存在漏洞时,才会存在漏洞。
\switchcolumn[0]*%%%%%%%
\subsection{Potential Dangers}
\switchcolumn
\subsection{潜在的危险}
\switchcolumn[0]*%%%%%%%
In any web application, allowing unsanitized, user-provided content to
be executed as HTML, CSS, or JavaScript is potentially dangerous, so it
should be avoided wherever possible. There are times when some risk may
be acceptable, though.
\switchcolumn
在任何 Web 应用中,允许以 HTML、CSS 或 JavaScript
形式执行未经无害化处理的、用户提供的内容都有潜在的安全隐患,因此这应尽可能避免。不过,有时候一些风险或许是可以接受的。
\switchcolumn[0]*%%%%%%%
For example, services like CodePen and JSFiddle allow user-provided
content to be executed, but it's in a context where this is expected and
sandboxed to some extent inside iframes. In the cases when an important
feature inherently requires some level of vulnerability, it's up to your
team to weigh the importance of the feature against the worst-case
scenarios the vulnerability enables.
\switchcolumn
例如,像 CodePen 和 JSFiddle 这样的服务允许执行用户提供的内容,但这是在
iframe
这样一个可预期的沙盒环境中。当一个重要的功能本身会伴随某种程度的漏洞时,就需要你自行权衡该功能的重要性和该漏洞所带来的最坏情况。
\end{paracol}


\columnratio{0.55}
\begin{paracol}{2} 
 
\switchcolumn[0]*%%%%%%%
\subsubsection{HTML Injection}
\switchcolumn
\subsubsection{注入 HTML}
\switchcolumn[0]*%%%%%%%
As you learned earlier, Vue automatically escapes HTML content,
preventing you from accidentally injecting executable HTML into your
application. However, \textbf{in cases where you know the HTML is safe},
you can explicitly render HTML content:
\switchcolumn
我们现在已经知道 Vue 会自动转义 HTML 内容,防止你意外地将可执行的 HTML
注入到你的应用中。然而,\textbf{在你知道 HTML
安全的情况下},你还是可以显式地渲染 HTML 内容。
\switchcolumn[0]*%%%%%%%
\begin{itemize}
\item
  Using a template:
\begin{codeHtml}
<div v-html="userProvidedHtml"></div>
\end{codeHtml}
\item
  Using a render function:
\begin{codeJs}
h('div', {
  innerHTML: this.userProvidedHtml
})
\end{codeJs}
\item
  Using a render function with JSX:
\begin{codeHtml}
<div innerHTML={this.userProvidedHtml}></div>
\end{codeHtml}
\end{itemize}
\switchcolumn
\begin{itemize}
\item
  使用模板:
\begin{codeHtml}
<div v-html="userProvidedHtml"></div>
\end{codeHtml}
\item
  使用渲染函数:
\begin{codeJs}
h('div', {
  innerHTML: this.userProvidedHtml
})
\end{codeJs}
\item
  以 JSX 形式使用渲染函数:
\begin{codeHtml}
<div innerHTML={this.userProvidedHtml}></div>
\end{codeHtml}
\end{itemize}
\switchcolumn[0]*%%%%%%%
~
\begin{vueQuoteWarn}{WARNING}
User-provided HTML can never be considered 100\% safe unless it's in a
sandboxed iframe or in a part of the app where only the user who wrote
that HTML can ever be exposed to it. Additionally, allowing users to
write their own Vue templates brings similar dangers.
\end{vueQuoteWarn}
\switchcolumn~
\begin{vueQuoteWarn}{警告}
用户提供的 HTML 永远不能被认为是 100\% 安全的,除非它在 iframe
这样的沙盒环境中,或者该 HTML 只会被该用户看到。此外,允许用户编写自己的
Vue 模板也会带来类似的危险。
\end{vueQuoteWarn}
\end{paracol}



\columnratio{0.55}
\begin{paracol}{2} 
 
\switchcolumn[0]*%%%%%%%
\subsubsection{URL Injection}
\switchcolumn
\subsubsection{URL 注入}
\switchcolumn[0]*%%%%%%%
In a URL like this:
\switchcolumn
在这样一个使用 URL 的场景中:
\switchcolumn[0]*%%%%%%%
\begin{codeHtml}
<a :href="userProvidedUrl">
  click me
</a>
\end{codeHtml}
\switchcolumn
\begin{codeHtml}
<a :href="userProvidedUrl">
  click me
</a>
\end{codeHtml}
\switchcolumn[0]*%%%%%%%
There's a potential security issue if the URL has not been "sanitized"
to prevent JavaScript execution using \texttt{javascript:}. There are
libraries such as
\href{https://www.npmjs.com/package/@braintree/sanitize-url}{sanitize-url}
to help with this, but note: if you're ever doing URL sanitization on
the frontend, you already have a security issue. \textbf{User-provided
URLs should always be sanitized by your backend before even being saved
to a database.} Then the problem is avoided for \emph{every} client
connecting to your API, including native mobile apps. Also note that
even with sanitized URLs, Vue cannot help you guarantee that they lead
to safe destinations.
\switchcolumn
如果这个 URL 允许通过 \texttt{javascript:} 执行
JavaScript,即没有进行无害化处理,那么就会有一些潜在的安全问题。可以使用一些库来解决此类问题,比如
\href{https://www.npmjs.com/package/@braintree/sanitize-url}{sanitize-url},但请注意:如果你发现你需要在前端做
URL
无害化处理,那你的应用已经存在一个更严重的安全问题了。\textbf{任何用户提供的
URL 在被保存到数据库之前都应该先在后端做无害化处理}。这样,连接到你 API
的\emph{每一个}客户端都可以避免这个问题,包括原生移动应用。另外,即使是经过无害化处理的
URL,Vue 也不能保证它们指向安全的目的地。
\switchcolumn[0]*%%%%%%%
\subsubsection{Style Injection}
\switchcolumn
\subsubsection{样式注入}
\switchcolumn[0]*%%%%%%%
Looking at this example:
\switchcolumn
我们来看这样一个例子:
\switchcolumn[0]*%%%%%%%
\begin{codeHtml}
<a
  :href="sanitizedUrl"
  :style="userProvidedStyles"
>
  click me
</a>
\end{codeHtml}
\switchcolumn
\begin{codeHtml}
<a
  :href="sanitizedUrl"
  :style="userProvidedStyles"
>
  click me
</a>
\end{codeHtml}
\switchcolumn[0]*%%%%%%%
Let's assume that \texttt{sanitizedUrl} has been sanitized, so that it's
definitely a real URL and not JavaScript. With the
\texttt{userProvidedStyles}, malicious users could still provide CSS to
"click jack", e.g. styling the link into a transparent box over the "Log
in" button. Then if \texttt{https://user-controlled-website.com/} is
built to resemble the login page of your application, they might have
just captured a user's real login information.
\switchcolumn
我们假设 \texttt{sanitizedUrl} 已进行无害化处理,它是一个正常 URL 而非
JavaScript。然而,由于 \texttt{userProvidedStyles}
的存在,恶意用户仍然能利用 CSS
进行``点击劫持'',例如,可以在``登录''按钮上方覆盖一个透明的链接。如果用户控制的页面
\texttt{https://user-controlled-website.com/}
专门仿造了你应用的登录页,那么他们就有可能捕获用户的真实登录信息。
\switchcolumn[0]*%%%%%%%
You may be able to imagine how allowing user-provided content for a
\texttt{\textless{}style\textgreater{}} element would create an even
greater vulnerability, giving that user full control over how to style
the entire page. That's why Vue prevents rendering of style tags inside
templates, such as:
\switchcolumn
你可以想象,如果允许在 \texttt{\textless{}style\textgreater{}}
元素中插入用户提供的内容,会造成更大的漏洞,因为这使得用户能控制整个页面的样式。因此
Vue 阻止了在模板中像这样渲染 style 标签:
\switchcolumn[0]*%%%%%%%
\begin{codeHtml}
<style>{{ userProvidedStyles }}</style>
\end{codeHtml}
\switchcolumn
\begin{codeHtml}
<style>{{ userProvidedStyles }}</style>
\end{codeHtml}
\switchcolumn[0]*%%%%%%%
To keep your users fully safe from clickjacking, we recommend only
allowing full control over CSS inside a sandboxed iframe. Alternatively,
when providing user control through a style binding, we recommend using
its
\href{https://vuejs.org/guide/essentials/class-and-style.html\#binding-to-objects-1}{object
syntax} and only allowing users to provide values for specific
properties it's safe for them to control, like this:
\switchcolumn
为了避免用户的点击被劫持,我们建议仅在沙盒环境的 iframe 中允许用户控制
CSS。或者,当用户控制样式绑定时,我们建议使用其\href{https://cn.vuejs.org/guide/essentials/class-and-style.html\#object-syntax-2}{对象值形式}并仅允许用户提供能够安全控制的、特定的属性,就像这样:
\switchcolumn[0]*%%%%%%%
\begin{codeHtml}
<a
  :href="sanitizedUrl"
  :style="{
    color: userProvidedColor,
    background: userProvidedBackground
  }"
>
  click me
</a>
\end{codeHtml}
\switchcolumn
\begin{codeHtml}
<a
  :href="sanitizedUrl"
  :style="{
    color: userProvidedColor,
    background: userProvidedBackground
  }"
>
  click me
</a>
\end{codeHtml}
\end{paracol}



\columnratio{0.55}
\begin{paracol}{2} 
 
\switchcolumn[0]*%%%%%%%
\subsubsection{JavaScript Injection}
\switchcolumn
\subsubsection{JavaScript 注入}
\switchcolumn[0]*%%%%%%%
We strongly discourage ever rendering a
\texttt{\textless{}script\textgreater{}} element with Vue, since
templates and render functions should never have side effects. However,
this isn't the only way to include strings that would be evaluated as
JavaScript at runtime.
\switchcolumn
我们强烈建议任何时候都不要在 Vue 中渲染
\texttt{\textless{}script\textgreater{}},因为模板和渲染函数不应有其他副作用。但是,渲染
\texttt{\textless{}script\textgreater{}} 并不是插入在运行时执行的
JavaScript 字符串的唯一方法。
\switchcolumn[0]*%%%%%%%
Every HTML element has attributes with values accepting strings of
JavaScript, such as \texttt{onclick}, \texttt{onfocus}, and
\texttt{onmouseenter}. Binding user-provided JavaScript to any of these
event attributes is a potential security risk, so it should be avoided.
\switchcolumn
每个 HTML 元素都有能接受字符串形式 JavaScript 的 attribute,例如
\texttt{onclick}、\texttt{onfocus} 和
\texttt{onmouseenter}。绑定任何用户提供的 JavaScript 给这些事件
attribute 都具有潜在风险,因此需要避免这么做。
\switchcolumn[0]*%%%%%%%
\begin{vueQuoteWarn}{WARNING}
User-provided JavaScript can never be considered 100\% safe unless it's
in a sandboxed iframe or in a part of the app where only the user who
wrote that JavaScript can ever be exposed to it.
\end{vueQuoteWarn}
\switchcolumn
\begin{vueQuoteWarn}{警告}
用户提供的 JavaScript 永远不能被认为是 100\% 安全的,除非它在 iframe
这样的沙盒环境中,或者该段代码只会在该用户登录的页面上被执行。
\end{vueQuoteWarn}
\switchcolumn[0]*%%%%%%%
Sometimes we receive vulnerability reports on how it's possible to do
cross-site scripting (XSS) in Vue templates. In general, we do not
consider such cases to be actual vulnerabilities because there's no
practical way to protect developers from the two scenarios that would
allow XSS:
\switchcolumn
有时我们会收到漏洞报告,说在 Vue 模板中可以进行跨站脚本攻击
(XSS)。一般来说,我们不认为这种情况是真正的漏洞,因为没有切实可行的方法,能够在以下两种场景中保护开发者不受
XSS 的影响。
\switchcolumn[0]*%%%%%%%
\begin{enumerate}
\item
  The developer is explicitly asking Vue to render user-provided,
  unsanitized content as Vue templates. This is inherently unsafe, and
  there's no way for Vue to know the origin.
\item
  The developer is mounting Vue to an entire HTML page which happens to
  contain server-rendered and user-provided content. This is
  fundamentally the same problem as \#1, but sometimes devs may do it
  without realizing it. This can lead to possible vulnerabilities where
  the attacker provides HTML which is safe as plain HTML but unsafe as a
  Vue template. The best practice is to \textbf{never mount Vue on nodes
  that may contain server-rendered and user-provided content}.
\end{enumerate}
\switchcolumn
\begin{enumerate}
\item
  开发者显式地将用户提供的、未经无害化处理的内容作为 Vue
  模板渲染。这本身就是不安全的,Vue 也无从溯源。
\item
  开发者将 Vue 挂载到可能包含服务端渲染或用户提供内容的 HTML
  页面上,这与 \#1
  的问题基本相同,但有时开发者可能会不知不觉地这样做。攻击者提供的 HTML
  可能在普通 HTML 中是安全的,但在 Vue
  模板中是不安全的,这就会导致漏洞。最佳实践是:\textbf{不要将 Vue
  挂载到可能包含服务端渲染或用户提供内容的 DOM 节点上}。
\end{enumerate}
\end{paracol}



\columnratio{0.55}
\begin{paracol}{2} 
 
\switchcolumn[0]*%%%%%%%
\subsection{Best Practices}
\switchcolumn
\subsection{最佳实践}
\switchcolumn[0]*%%%%%%%
The general rule is that if you allow unsanitized, user-provided content
to be executed (as either HTML, JavaScript, or even CSS), you might open
yourself up to attacks. This advice actually holds true whether using
Vue, another framework, or even no framework.
\switchcolumn
最基本的规则就是只要你允许执行未经无害化处理的、用户提供的内容 (无论是
HTML、JavaScript 还是 CSS),你就可能面临攻击。无论是使用
Vue、其他框架,或是不使用框架,道理都是一样的。
\switchcolumn[0]*%%%%%%%
Beyond the recommendations made above for
\href{https://vuejs.org/guide/best-practices/security.html\#potential-dangers}{Potential
Dangers}, we also recommend familiarizing yourself with these resources:
\switchcolumn
除了上面为处理\href{https://cn.vuejs.org/guide/best-practices/security.html\#potential-dangers}{潜在危险}提供的建议,我们也建议你熟读下面这些资源:
\switchcolumn[0]*%%%%%%%
\begin{itemize}
\item
  \href{https://html5sec.org/}{HTML5 Security Cheat Sheet}
\item
  \href{https://cheatsheetseries.owasp.org/cheatsheets/Cross_Site_Scripting_Prevention_Cheat_Sheet.html}{OWASP's
  Cross Site Scripting (XSS) Prevention Cheat Sheet}
\end{itemize}
\switchcolumn
\begin{itemize}
\item
  \href{https://html5sec.org/}{HTML5 安全手册}
\item
  \href{https://cheatsheetseries.owasp.org/cheatsheets/Cross_Site_Scripting_Prevention_Cheat_Sheet.html}{OWASP
  的跨站脚本攻击 (XSS) 防护手册}
\end{itemize}
\switchcolumn[0]*%%%%%%%
Then use what you learn to also review the source code of your
dependencies for potentially dangerous patterns, if any of them include
3rd-party components or otherwise influence what's rendered to the DOM.
\switchcolumn
接着你可以利用学到的知识,来审查依赖项的源代码,看看是否有潜在的危险,防止它们中的任何一个以第三方组件或其他方式影响
DOM 渲染的内容。
\switchcolumn[0]*%%%%%%%
\subsection{Backend Coordination}
\switchcolumn
\subsection{后端协调}
\switchcolumn[0]*%%%%%%%
HTTP security vulnerabilities, such as cross-site request forgery
(CSRF/XSRF) and cross-site script inclusion (XSSI), are primarily
addressed on the backend, so they aren't a concern of Vue's. However,
it's still a good idea to communicate with your backend team to learn
how to best interact with their API, e.g., by submitting CSRF tokens
with form submissions.
\switchcolumn
类似跨站请求伪造 (CSRF/XSRF) 和跨站脚本引入 (XSSI) 这样的 HTTP
安全漏洞,主要由后端负责处理,因此它们不是 Vue
职责范围内的问题。但是,你应该与后端团队保持沟通,了解如何更好地与后端
API 进行交互,例如,在提交表单时附带 CSRF 令牌。
\switchcolumn[0]*%%%%%%%
\subsection{Server-Side Rendering (SSR)}
\switchcolumn
\subsection{服务端渲染 (SSR)}
\switchcolumn[0]*%%%%%%%
There are some additional security concerns when using SSR, so make sure
to follow the best practices outlined throughout
\href{https://vuejs.org/guide/scaling-up/ssr.html}{our SSR
documentation} to avoid vulnerabilities.
\switchcolumn
在使用 SSR 时还有一些其他的安全注意事项,因此请确保遵循我们的
\href{https://cn.vuejs.org/guide/scaling-up/ssr.html}{SSR
文档}给出的最佳实践来避免产生漏洞。 
\end{paracol}


\columnratio{0.55}
\begin{paracol}{2} 

\end{paracol}



\columnratio{0.55}
\begin{paracol}{2} 

\end{paracol}



\columnratio{0.55}
\begin{paracol}{2} 

\end{paracol}


\columnratio{0.55}
\begin{paracol}{2} 

\end{paracol}



\columnratio{0.55}
\begin{paracol}{2} 

\end{paracol}