\columnratio{0.55}
\begin{paracol}{2}
 
\switchcolumn[0]*%%%%%%%
\section{Async Components}
\switchcolumn
\section{异步组件}
\switchcolumn[0]*%%%%%%%
\subsection{Basic Usage}
\switchcolumn
\subsection{基本用法}
\switchcolumn[0]*%%%%%%%
In large applications, we may need to divide the app into smaller chunks
and only load a component from the server when it's needed. To make that
possible, Vue has a
\href{https://vuejs.org/api/general.html\#defineasynccomponent}{\texttt{defineAsyncComponent}}
function:
\switchcolumn
在大型项目中,我们可能需要拆分应用为更小的块,并仅在需要时再从服务器加载相关组件。Vue
提供了
\href{https://cn.vuejs.org/api/general.html\#defineasynccomponent}{\texttt{defineAsyncComponent}}
方法来实现此功能:
\switchcolumn[0]*%%%%%%%
\begin{codeJs}
import { defineAsyncComponent } from 'vue'
const AsyncComp = defineAsyncComponent(() => {
  return new Promise((resolve, reject) => {
    // ...从服务器获取组件
    resolve(/* 获取到的组件 */)
  })
})
// ... 像使用其他一般组件一样使用 `AsyncComp`
\end{codeJs}
\switchcolumn
\begin{codeJs}
import { defineAsyncComponent } from 'vue'
const AsyncComp = defineAsyncComponent(() => {
  return new Promise((resolve, reject) => {
    // ...从服务器获取组件
    resolve(/* 获取到的组件 */)
  })
})
// ... 像使用其他一般组件一样使用 `AsyncComp`
\end{codeJs}
\switchcolumn[0]*%%%%%%%
As you can see, \texttt{defineAsyncComponent} accepts a loader function
that returns a Promise. The Promise's \texttt{resolve} callback should
be called when you have retrieved your component definition from the
server. You can also call \texttt{reject(reason)} to indicate the load
has failed.
\switchcolumn
如你所见,\texttt{defineAsyncComponent} 方法接收一个返回 Promise
的加载函数。这个 Promise 的 \texttt{resolve}
回调方法应该在从服务器获得组件定义时调用。你也可以调用
\texttt{reject(reason)} 表明加载失败。
\switchcolumn[0]*%%%%%%%
\href{https://developer.mozilla.org/en-US/docs/Web/JavaScript/Reference/Operators/import}{ES
module dynamic import} also returns a Promise, so most of the time we
will use it in combination with \texttt{defineAsyncComponent}. Bundlers
like Vite and webpack also support the syntax (and will use it as bundle
split points), so we can use it to import Vue SFCs:
\switchcolumn
\href{https://developer.mozilla.org/en-US/docs/Web/JavaScript/Reference/Operators/import}{ES
模块动态导入}也会返回一个 Promise,所以多数情况下我们会将它和
\texttt{defineAsyncComponent} 搭配使用。类似 Vite 和 Webpack
这样的构建工具也支持此语法
(并且会将它们作为打包时的代码分割点),因此我们也可以用它来导入 Vue
单文件组件:
\switchcolumn[0]*%%%%%%%
\begin{codeJs}
import { defineAsyncComponent } from 'vue'
const AsyncComp = defineAsyncComponent(() =>
  import('./components/MyComponent.vue')
)
\end{codeJs}
\switchcolumn
\begin{codeJs}
import { defineAsyncComponent } from 'vue'
const AsyncComp = defineAsyncComponent(() =>
  import('./components/MyComponent.vue')
)
\end{codeJs}
\switchcolumn[0]*%%%%%%%
The resulting \texttt{AsyncComp} is a wrapper component that only calls
the loader function when it is actually rendered on the page. In
addition, it will pass along any props and slots to the inner component,
so you can use the async wrapper to seamlessly replace the original
component while achieving lazy loading.
\switchcolumn
最后得到的 \texttt{AsyncComp}
是一个外层包装过的组件,仅在页面需要它渲染时才会调用加载内部实际组件的函数。它会将接收到的
props
和插槽传给内部组件,所以你可以使用这个异步的包装组件无缝地替换原始组件,同时实现延迟加载。
\switchcolumn[0]*%%%%%%%
As with normal components, async components can be
\href{https://vuejs.org/guide/components/registration.html\#global-registration}{registered
globally} using \texttt{app.component()}:
\switchcolumn
与普通组件一样,异步组件可以使用 \texttt{app.component()}
\href{https://cn.vuejs.org/guide/components/registration.html\#global-registration}{全局注册}:
\switchcolumn[0]*%%%%%%%
\begin{codeJs}
app.component('MyComponent', defineAsyncComponent(() =>
  import('./components/MyComponent.vue')
))
\end{codeJs}
\switchcolumn
\begin{codeJs}
app.component('MyComponent', defineAsyncComponent(() =>
  import('./components/MyComponent.vue')
))
\end{codeJs}
\switchcolumn[0]*%%%%%%%
They can also be defined directly inside their parent component:
\switchcolumn
也可以直接在父组件中直接定义它们:
\switchcolumn[0]*%%%%%%%
\begin{codeHtml}
<script setup>
import { defineAsyncComponent } from 'vue'
const AdminPage = defineAsyncComponent(() =>
  import('./components/AdminPageComponent.vue')
)
</script>
<template>
  <AdminPage />
</template>
\end{codeHtml}
\switchcolumn
\begin{codeHtml}
<script setup>
import { defineAsyncComponent } from 'vue'
const AdminPage = defineAsyncComponent(() =>
  import('./components/AdminPageComponent.vue')
)
</script>
<template>
  <AdminPage />
</template>
\end{codeHtml}
\end{paracol}

\columnratio{0.55}
\begin{paracol}{2}
 
\switchcolumn[0]*%%%%%%%
\subsection{Loading and Error States}
\switchcolumn
\subsection{加载与错误状态}
\switchcolumn[0]*%%%%%%%
Asynchronous operations inevitably involve loading and error states -
\texttt{defineAsyncComponent()} supports handling these states via
advanced options:
\switchcolumn
异步操作不可避免地会涉及到加载和错误状态,因此
\texttt{defineAsyncComponent()} 也支持在高级选项中处理这些状态:
\switchcolumn[0]*%%%%%%%
\begin{codeJs}
const AsyncComp = defineAsyncComponent({
  // 加载函数
  loader: () => import('./Foo.vue'),
  // 加载异步组件时使用的组件
  loadingComponent: LoadingComponent,
  // 展示加载组件前的延迟时间,默认为 200ms
  delay: 200,
  // 加载失败后展示的组件
  errorComponent: ErrorComponent,
  // 如果提供了一个 timeout 时间限制,并超时了
  // 也会显示这里配置的报错组件,默认值是:Infinity
  timeout: 3000
})
\end{codeJs}
\switchcolumn
\begin{codeJs}
const AsyncComp = defineAsyncComponent({
  // 加载函数
  loader: () => import('./Foo.vue'),
  // 加载异步组件时使用的组件
  loadingComponent: LoadingComponent,
  // 展示加载组件前的延迟时间,默认为 200ms
  delay: 200,
  // 加载失败后展示的组件
  errorComponent: ErrorComponent,
  // 如果提供了一个 timeout 时间限制,并超时了
  // 也会显示这里配置的报错组件,默认值是:Infinity
  timeout: 3000
})
\end{codeJs}
\switchcolumn[0]*%%%%%%%
If a loading component is provided, it will be displayed first while the
inner component is being loaded. There is a default 200ms delay before
the loading component is shown - this is because on fast networks, an
instant loading state may get replaced too fast and end up looking like
a flicker.
\switchcolumn
如果提供了一个加载组件,它将在内部组件加载时先行显示。在加载组件显示之前有一个默认的
200ms
延迟------这是因为在网络状况较好时,加载完成得很快,加载组件和最终组件之间的替换太快可能产生闪烁,反而影响用户感受。
\switchcolumn[0]*%%%%%%%
If an error component is provided, it will be displayed when the Promise
returned by the loader function is rejected. You can also specify a
timeout to show the error component when the request is taking too long.
\switchcolumn
如果提供了一个报错组件,则它会在加载器函数返回的 Promise
抛错时被渲染。你还可以指定一个超时时间,在请求耗时超过指定时间时也会渲染报错组件。
\end{paracol}

\columnratio{0.55}
\begin{paracol}{2}
 
\switchcolumn[0]*%%%%%%%
\subsection{Using with Suspense}
\switchcolumn
\subsection{搭配 Suspense 使用}
\switchcolumn[0]*%%%%%%%
Async components can be used with the
\texttt{\textless{}Suspense\textgreater{}} built-in component. The
interaction between \texttt{\textless{}Suspense\textgreater{}} and async
components is documented in the
\href{https://vuejs.org/guide/built-ins/suspense.html}{dedicated chapter
for ``}.
\switchcolumn
异步组件可以搭配内置的 \texttt{\textless{}Suspense\textgreater{}}
组件一起使用,若想了解 \texttt{\textless{}Suspense\textgreater{}}
和异步组件之间交互,请参阅
\href{https://cn.vuejs.org/guide/built-ins/suspense.html}{{\tt <Suspense>}} 章节。
\end{paracol}
 