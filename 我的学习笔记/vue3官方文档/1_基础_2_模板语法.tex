\columnratio{0.55}
\begin{paracol}{2}
\switchcolumn[0]*%%%%%%%
\section{Template Syntax}
\switchcolumn
\section{模板语法}
\switchcolumn[0]*%%%%%%%
Vue uses an HTML-based template syntax that allows you to declaratively
bind the rendered DOM to the underlying component instance's data. All
Vue templates are syntactically valid HTML that can be parsed by
spec-compliant browsers and HTML parsers.
\switchcolumn
Vue 使用一种基于 HTML
的模板语法,使我们能够声明式地将其组件实例的数据绑定到呈现的 DOM
上。所有的 Vue 模板都是语法层面合法的 HTML,可以被符合规范的浏览器和
HTML 解析器解析。
\switchcolumn[0]*%%%%%%%
Under the hood, Vue compiles the templates into highly-optimized
JavaScript code. Combined with the reactivity system, Vue can
intelligently figure out the minimal number of components to re-render
and apply the minimal amount of DOM manipulations when the app state
changes.
\switchcolumn
在底层机制中,Vue 会将模板编译成高度优化的 JavaScript
代码。结合响应式系统,当应用状态变更时,Vue
能够智能地推导出需要重新渲染的组件的最少数量,并应用最少的 DOM 操作。
\switchcolumn[0]*%%%%%%%
If you are familiar with Virtual DOM concepts and prefer the raw power
of JavaScript, you can also
\href{https://vuejs.org/guide/extras/render-function.html}{directly
write render functions} instead of templates, with optional JSX support.
However, do note that they do not enjoy the same level of compile-time
optimizations as templates.
\switchcolumn
如果你对虚拟 DOM 的概念比较熟悉,并且偏好直接使用
JavaScript,你也可以结合可选的 JSX
支持\href{https://cn.vuejs.org/guide/extras/render-function.html}{直接手写渲染函数}而不采用模板。但请注意,这将不会享受到和模板同等级别的编译时优化。
\switchcolumn[0]*%%%%%%%
\subsection{Text Interpolation}
\switchcolumn
\subsection{文本插值}
\switchcolumn[0]*%%%%%%%
The most basic form of data binding is text interpolation using the
"Mustache" syntax (double curly braces):
\switchcolumn
最基本的数据绑定形式是文本插值,它使用的是``Mustache''语法
(即双大括号):
\switchcolumn[0]*%%%%%%%
\begin{codeHtml*}{label=template}
<span>Message: {{ msg }}</span>
\end{codeHtml*}  
\switchcolumn
\begin{codeHtml*}{label=template}
<span>Message: {{ msg }}</span>
\end{codeHtml*}  

\switchcolumn[0]*%%%%%%%
The mustache tag will be replaced with the value of the \texttt{msg}
property
\href{https://vuejs.org/guide/essentials/reactivity-fundamentals.html\#declaring-reactive-state}{from
the corresponding component instance}. It will also be updated whenever
the \texttt{msg} property changes.
\switchcolumn
双大括号标签会被替换为\href{https://cn.vuejs.org/guide/essentials/reactivity-fundamentals.html\#declaring-reactive-state}{相应组件实例中}
\texttt{msg} 属性的值。同时每次 \texttt{msg} 属性更改时它也会同步更新。
\switchcolumn[0]*%%%%%%%
\subsection{Raw HTML}
\switchcolumn
\subsection{原始 HTML}
\switchcolumn[0]*%%%%%%%
The double mustaches interpret the data as plain text, not HTML. In
order to output real HTML, you will need to use the
\href{https://vuejs.org/api/built-in-directives.html\#v-html}{\texttt{v-html}
directive}:
\switchcolumn
双大括号会将数据解释为纯文本,而不是 HTML。若想插入 HTML,你需要使用
\href{https://cn.vuejs.org/api/built-in-directives.html\#v-html}{\texttt{v-html}
指令}:
\switchcolumn[0]*%%%%%%%
\begin{codeHtml}
<p>Using text interpolation: {{ rawHtml }}</p>
<p>Using v-html directive: <span v-html="rawHtml"></span></p>
\end{codeHtml}  
\switchcolumn
\begin{codeHtml}
<p>Using text interpolation: {{ rawHtml }}</p>
<p>Using v-html directive: <span v-html="rawHtml"></span></p>
\end{codeHtml}  
\switchcolumn[0]*%%%%%%%
\begin{vueQuote}{result}
Using text interpolation: This should be red.\\
Using v-html directive: {\textcolor{red}{This should be red.}}
\end{vueQuote}
\switchcolumn
\begin{vueQuote}{结果}
Using text interpolation: This should be red.\\
Using v-html directive: {\textcolor{red}{This should be red.}}
\end{vueQuote}

\switchcolumn[0]*%%%%%%%
Here we're encountering something new. The \texttt{v-html} attribute
you're seeing is called a \textbf{directive}. Directives are prefixed
with \texttt{v-} to indicate that they are special attributes provided
by Vue, and as you may have guessed, they apply special reactive
behavior to the rendered DOM. Here, we're basically saying "keep this
element's inner HTML up-to-date with the \texttt{rawHtml} property on
the current active instance."
\switchcolumn
这里我们遇到了一个新的概念。这里看到的 \texttt{v-html} attribute
被称为一个\textbf{指令}。指令由 \texttt{v-} 作为前缀,表明它们是一些由
Vue 提供的特殊 attribute,你可能已经猜到了,它们将为渲染的 DOM
应用特殊的响应式行为。这里我们做的事情简单来说就是:在当前组件实例上,将此元素的
innerHTML 与 \texttt{rawHtml} 属性保持同步。
\switchcolumn[0]*%%%%%%%
The contents of the \texttt{span} will be replaced with the value of the
\texttt{rawHtml} property, interpreted as plain HTML - data bindings are
ignored. Note that you cannot use \texttt{v-html} to compose template
partials, because Vue is not a string-based templating engine. Instead,
components are preferred as the fundamental unit for UI reuse and
composition.
\switchcolumn
\texttt{span} 的内容将会被替换为 \texttt{rawHtml} 属性的值,插值为纯
HTML------数据绑定将会被忽略。注意,你不能使用 \texttt{v-html}
来拼接组合模板,因为 Vue 不是一个基于字符串的模板引擎。在使用 Vue
时,应当使用组件作为 UI 重用和组合的基本单元。
\switchcolumn[0]*%%%%%%%
\begin{vueQuoteWarn}{Security Warning}
Dynamically rendering arbitrary HTML on your website can be very
dangerous because it can easily lead to
\href{https://en.wikipedia.org/wiki/Cross-site_scripting}{XSS
vulnerabilities}. Only use \texttt{v-html} on trusted content and
\textbf{never} on user-provided content.
\end{vueQuoteWarn}
\switchcolumn
\begin{vueQuoteWarn}{安全警告}
在网站上动态渲染任意 HTML 是非常危险的,因为这非常容易造成
\href{https://zh.wikipedia.org/wiki/跨網站指令碼}{XSS
漏洞}。请仅在内容安全可信时再使用
\texttt{v-html},并且\textbf{永远不要}使用用户提供的 HTML 内容。
\end{vueQuoteWarn}
\switchcolumn[0]*%%%%%%%
\subsection{Attribute Bindings}
\switchcolumn
\subsection{Attribute 绑定}
\switchcolumn[0]*%%%%%%%
Mustaches cannot be used inside HTML attributes. Instead, use a
\href{https://vuejs.org/api/built-in-directives.html\#v-bind}{\texttt{v-bind}
directive}:
\switchcolumn
双大括号不能在 HTML attributes 中使用。想要响应式地绑定一个
attribute,应该使用
\href{https://cn.vuejs.org/api/built-in-directives.html\#v-bind}{\texttt{v-bind}
指令}:

\switchcolumn[0]*%%%%%%%
\begin{codeHtml}
<div v-bind:id="dynamicId"></div>
\end{codeHtml}  
\switchcolumn
\begin{codeHtml}
<div v-bind:id="dynamicId"></div>
\end{codeHtml}  
\switchcolumn[0]*%%%%%%%
The \texttt{v-bind} directive instructs Vue to keep the element's
\texttt{id} attribute in sync with the component's \texttt{dynamicId}
property. If the bound value is \texttt{null} or \texttt{undefined},
then the attribute will be removed from the rendered element.
\switchcolumn
\texttt{v-bind} 指令指示 Vue 将元素的 \texttt{id} attribute 与组件的
\texttt{dynamicId} 属性保持一致。如果绑定的值是 \texttt{null} 或者
\texttt{undefined},那么该 attribute 将会从渲染的元素上移除。
\switchcolumn[0]*%%%%%%%
\subsubsection{Shorthand}
\switchcolumn
\subsubsection{简写}
\switchcolumn[0]*%%%%%%%
Because \texttt{v-bind} is so commonly used, it has a dedicated
shorthand syntax:
\switchcolumn
因为 \texttt{v-bind} 非常常用,我们提供了特定的简写语法:


\switchcolumn[0]*%%%%%%%
\begin{codeHtml}
<div :id="dynamicId"></div>
\end{codeHtml}  
\switchcolumn
\begin{codeHtml}
<div :id="dynamicId"></div>
\end{codeHtml}  
\switchcolumn[0]*%%%%%%%
Attributes that start with \texttt{:} may look a bit different from
normal HTML, but it is in fact a valid character for attribute names and
all Vue-supported browsers can parse it correctly. In addition, they do
not appear in the final rendered markup. The shorthand syntax is
optional, but you will likely appreciate it when you learn more about
its usage later.
\switchcolumn
开头为 \texttt{:} 的 attribute 可能和一般的 HTML attribute
看起来不太一样,但它的确是合法的 attribute 名称字符,并且所有支持 Vue
的浏览器都能正确解析它。此外,他们不会出现在最终渲染的 DOM
中。简写语法是可选的,但相信在你了解了它更多的用处后,你应该会更喜欢它。
\switchcolumn[0]*%%%%%%%
\begin{quote}
For the rest of the guide, we will be using the shorthand syntax in code
examples, as that's the most common usage for Vue developers.
\end{quote}
\switchcolumn
\begin{quote}
接下来的指引中,我们都将在示例中使用简写语法,因为这是在实际开发中更常见的用法。
\end{quote}
\switchcolumn[0]*%%%%%%%
\subsubsection{Boolean Attributes}
\switchcolumn
\subsubsection{布尔型 Attribute}
\switchcolumn[0]*%%%%%%%
\href{https://html.spec.whatwg.org/multipage/common-microsyntaxes.html\#boolean-attributes}{Boolean
attributes} are attributes that can indicate true / false values by
their presence on an element. For example,
\href{https://developer.mozilla.org/en-US/docs/Web/HTML/Attributes/disabled}{\texttt{disabled}}
is one of the most commonly used boolean attributes.
\switchcolumn
\href{https://developer.mozilla.org/zh-CN/docs/Web/HTML/Attributes\#布尔值属性}{布尔型
attribute} 依据 true / false 值来决定 attribute
是否应该存在于该元素上。\href{https://developer.mozilla.org/en-US/docs/Web/HTML/Attributes/disabled}{\texttt{disabled}}
就是最常见的例子之一。
\switchcolumn[0]*%%%%%%%
\texttt{v-bind} works a bit differently in this case:
\switchcolumn
\texttt{v-bind} 在这种场景下的行为略有不同:
\end{paracol}
