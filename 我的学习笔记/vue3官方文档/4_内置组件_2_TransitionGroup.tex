\columnratio{0.55}
\begin{paracol}{2}
 
\switchcolumn[0]*%%%%%%%
\section{TransitionGroup}
\switchcolumn
\section{TransitionGroup}
\switchcolumn[0]*%%%%%%%
\texttt{\textless{}TransitionGroup\textgreater{}} is a built-in
component designed for animating the insertion, removal, and order
change of elements or components that are rendered in a list.
\switchcolumn
\texttt{\textless{}TransitionGroup\textgreater{}} 是一个内置组件,用于对
\texttt{v-for} 列表中的元素或组件的插入、移除和顺序改变添加动画效果。
\switchcolumn[0]*%%%%%%%
\subsection{Differences from \textless Transition\textgreater{}}
\switchcolumn
\subsection{和 \textless Transition\textgreater{} 的区别}
\switchcolumn[0]*%%%%%%%
\texttt{\textless{}TransitionGroup\textgreater{}} supports the same
props, CSS transition classes, and JavaScript hook listeners as
\texttt{\textless{}Transition\textgreater{}}, with the following
differences:
\switchcolumn
\texttt{\textless{}TransitionGroup\textgreater{}} 支持和
\texttt{\textless{}Transition\textgreater{}} 基本相同的 props、CSS 过渡
class 和 JavaScript 钩子监听器,但有以下几点区别:
\switchcolumn[0]*%%%%%%%
\begin{itemize}
\item
  By default, it doesn't render a wrapper element. But you can specify
  an element to be rendered with the \texttt{tag} prop.
\item
  \href{https://vuejs.org/guide/built-ins/transition.html\#transition-modes}{Transition
  modes} are not available, because we are no longer alternating between
  mutually exclusive elements.
\item
  Elements inside are \textbf{always required} to have a unique
  \texttt{key} attribute.
\item
  CSS transition classes will be applied to individual elements in the
  list, \textbf{not} to the group / container itself.
\end{itemize}
\switchcolumn
\begin{itemize}
\item
  默认情况下,它不会渲染一个容器元素。但你可以通过传入 \texttt{tag} prop
  来指定一个元素作为容器元素来渲染。
\item
  \href{https://cn.vuejs.org/guide/built-ins/transition.html\#transition-modes}{过渡模式}在这里不可用,因为我们不再是在互斥的元素之间进行切换。
\item
  列表中的每个元素都\textbf{必须}有一个独一无二的 \texttt{key}
  attribute。
\item
  CSS 过渡 class 会被应用在列表内的元素上,\textbf{而不是}容器元素上。
\end{itemize}
\switchcolumn[0]*%%%%%%%
\begin{vueQuote}{TIP}
When used in
\href{https://vuejs.org/guide/essentials/component-basics.html\#in-dom-template-parsing-caveats}{in-DOM
templates}, it should be referenced as
\texttt{\textless{}transition-group\textgreater{}}.
\end{vueQuote} 
\switchcolumn
\begin{vueQuote}{TIP}
当在
\href{https://cn.vuejs.org/guide/essentials/component-basics.html\#in-dom-template-parsing-caveats}{DOM
内模板}中使用时,组件名需要写为
\texttt{\textless{}transition-group\textgreater{}}。
\end{vueQuote} 
\end{paracol}

\columnratio{0.55}
\begin{paracol}{2}
 
\switchcolumn[0]*%%%%%%%
\subsection{Enter / Leave Transitions}
\switchcolumn
\subsection{进入 / 离开动画}
\switchcolumn[0]*%%%%%%%
Here is an example of applying enter / leave transitions to a
\texttt{v-for} list using
\texttt{\textless{}TransitionGroup\textgreater{}}:
\switchcolumn
这里是 \texttt{\textless{}TransitionGroup\textgreater{}} 对一个
\texttt{v-for} 列表添加进入 / 离开动画的示例:
\switchcolumn[0]*%%%%%%%
\begin{codeHtml}
<TransitionGroup name="list" tag="ul">
  <li v-for="item in items" :key="item">
    {{ item }}
  </li>
</TransitionGroup>
\end{codeHtml}
\switchcolumn
\begin{codeHtml}
<TransitionGroup name="list" tag="ul">
  <li v-for="item in items" :key="item">
    {{ item }}
  </li>
</TransitionGroup>
\end{codeHtml}
\switchcolumn[0]*%%%%%%%
\begin{codeCss}
.list-enter-active,
.list-leave-active {
  transition: all 0.5s ease;
}
.list-enter-from,
.list-leave-to {
  opacity: 0;
  transform: translateX(30px);
}
\end{codeCss}
\switchcolumn
\begin{codeCss}
.list-enter-active,
.list-leave-active {
  transition: all 0.5s ease;
}
.list-enter-from,
.list-leave-to {
  opacity: 0;
  transform: translateX(30px);
}
\end{codeCss}
\end{paracol}

\columnratio{0.55}
\begin{paracol}{2}
 
\switchcolumn[0]*%%%%%%%
\subsection{Move Transitions}
\switchcolumn
\subsection{移动动画}
\switchcolumn[0]*%%%%%%%
The above demo has some obvious flaws: when an item is inserted or
removed, its surrounding items instantly "jump" into place instead of
moving smoothly. We can fix this by adding a few additional CSS rules:
\switchcolumn
上面的示例有一些明显的缺陷:当某一项被插入或移除时,它周围的元素会立即发生``跳跃''而不是平稳地移动。我们可以通过添加一些额外的
CSS 规则来解决这个问题:
\switchcolumn[0]*%%%%%%%
\begin{codeCss}
.list-move, /* 对移动中的元素应用的过渡 */
.list-enter-active,
.list-leave-active {
  transition: all 0.5s ease;
}
.list-enter-from,
.list-leave-to {
  opacity: 0;
  transform: translateX(30px);
}
/* 确保将离开的元素从布局流中删除
  以便能够正确地计算移动的动画。 */
.list-leave-active {
  position: absolute;
}
\end{codeCss}
\switchcolumn
\begin{codeCss}
.list-move, /* 对移动中的元素应用的过渡 */
.list-enter-active,
.list-leave-active {
  transition: all 0.5s ease;
}
.list-enter-from,
.list-leave-to {
  opacity: 0;
  transform: translateX(30px);
}
/* 确保将离开的元素从布局流中删除
  以便能够正确地计算移动的动画。 */
.list-leave-active {
  position: absolute;
}
\end{codeCss}
\switchcolumn[0]*%%%%%%%
Now it looks much better - even animating smoothly when the whole list
is shuffled:
\switchcolumn
现在它看起来好多了,甚至对整个列表执行洗牌的动画也都非常流畅:
\switchcolumn[0]*%%%%%%%
\href{https://vuejs.org/examples/\#list-transition}{Full Example}
\switchcolumn
\href{https://cn.vuejs.org/examples/\#list-transition}{完整的示例}
\end{paracol}

\columnratio{0.55}
\begin{paracol}{2}
 
\switchcolumn[0]*%%%%%%%
\subsection{Staggering List Transitions}
\switchcolumn
\subsection{渐进延迟列表动画}
\switchcolumn[0]*%%%%%%%
By communicating with JavaScript transitions through data attributes,
it's also possible to stagger transitions in a list. First, we render
the index of an item as a data attribute on the DOM element:
\switchcolumn
通过在 JavaScript 钩子中读取元素的 data
attribute,我们可以实现带渐进延迟的列表动画。首先,我们把每一个元素的索引渲染为该元素上的一个
data attribute:
\switchcolumn[0]*%%%%%%%
\begin{codeHtml}
<TransitionGroup
  tag="ul"
  :css="false"
  @before-enter="onBeforeEnter"
  @enter="onEnter"
  @leave="onLeave"
>
  <li
    v-for="(item, index) in computedList"
    :key="item.msg"
    :data-index="index"
  >
    {{ item.msg }}
  </li>
</TransitionGroup>
\end{codeHtml}
\switchcolumn
\begin{codeHtml}
<TransitionGroup
  tag="ul"
  :css="false"
  @before-enter="onBeforeEnter"
  @enter="onEnter"
  @leave="onLeave"
>
  <li
    v-for="(item, index) in computedList"
    :key="item.msg"
    :data-index="index"
  >
    {{ item.msg }}
  </li>
</TransitionGroup>
\end{codeHtml}
\switchcolumn[0]*%%%%%%%
Then, in JavaScript hooks, we animate the element with a delay based on
the data attribute. This example is using the
\href{https://greensock.com/}{GreenSock library} to perform the
animation:
\switchcolumn
接着,在 JavaScript 钩子中,我们基于当前元素的 data attribute
对该元素的进场动画添加一个延迟。以下是一个基于
\href{https://greensock.com/}{GreenSock library} 的动画示例:
\switchcolumn[0]*%%%%%%%
\begin{codeJs}
function onEnter(el, done) {
  gsap.to(el, {
    opacity: 1,
    height: '1.6em',
    delay: el.dataset.index * 0.15,
    onComplete: done
  })
}
\end{codeJs}
\switchcolumn
\begin{codeJs}
function onEnter(el, done) {
  gsap.to(el, {
    opacity: 1,
    height: '1.6em',
    delay: el.dataset.index * 0.15,
    onComplete: done
  })
}
\end{codeJs}
\switchcolumn[0]*%%%%%%%
\href{https://play.vuejs.org/\#eNqlVMuu0zAQ/ZVRNklRm7QLWETtBW4FSFCxYkdYmGSSmjp28KNQVfl3xk7SFyvEponPGc+cOTPNOXrbdenRYZRHa1Nq3lkwaF33VEjedkpbOIPGeg6lajtnsYIeaq1aiOlSfAlqDOtG3L8SUchSSWNBcPrZwNdCAqVqTZND/KxdibBDjKGf3xIfWXngCNs9k4/Udu/KA3xWWnPz1zW0sOOP6CcnG3jv9ImIQn67SvrpUJ9IE/WVxPHsSkw97gbN0zFJZrB5grNPrskcLUNXac2FRZ0k3GIbIvxLSsVTq3bqF+otM5jMUi5L4So0SSicHplwOKOyfShdO1lariQo+Yy10vhO+qwoZkNFFKmxJ4Gp6ljJrRe+vMP3yJu910swNXqXcco1h0pJHDP6CZHEAAcAYMydwypYCDAkJRdX6Sts4xGtUDAKotIVs9Scpd4q/A0vYJmuXo5BSm7JOIEW81DVo77VR207ZEf8F23LB23T+X9VrbNh82nn6UAz7ASzSCeANZe0AnBctIqqbIoojLCIIBvoL5pJw31DH7Ry3VDKsoYinSii4ZyXxhBQM2Fwwt58D7NeoB8QkXfDvwRd2XtceOsCHkwc8KCINAk+vADJppQUFjZ0DsGVGT3uFn1KSjoPeKLoaYtvCO/rIlz3vH9O5FiU/nXny/pDT6YGKZngg0/Zg1GErrMbp6N5NHxJFi3N/4dRkj5IYf5ULxCmiPJpI4rIr4kHimhvbWfyLHOyOzQpNZZ57jXNy4nRGFLTR/0fWBqe7w==}{Full
Example in the Playground}
\switchcolumn
\href{https://play.vuejs.org/\#eNqlVMuu0zAQ/ZVRNklRm7QLWETtBW4FSFCxYkdYmGSSmjp28KNQVfl3xk7SFyvEponPGc+cOTPNOXrbdenRYZRHa1Nq3lkwaF33VEjedkpbOIPGeg6lajtnsYIeaq1aiOlSfAlqDOtG3L8SUchSSWNBcPrZwNdCAqVqTZND/KxdibBDjKGf3xIfWXngCNs9k4/Udu/KA3xWWnPz1zW0sOOP6CcnG3jv9ImIQn67SvrpUJ9IE/WVxPHsSkw97gbN0zFJZrB5grNPrskcLUNXac2FRZ0k3GIbIvxLSsVTq3bqF+otM5jMUi5L4So0SSicHplwOKOyfShdO1lariQo+Yy10vhO+qwoZkNFFKmxJ4Gp6ljJrRe+vMP3yJu910swNXqXcco1h0pJHDP6CZHEAAcAYMydwypYCDAkJRdX6Sts4xGtUDAKotIVs9Scpd4q/A0vYJmuXo5BSm7JOIEW81DVo77VR207ZEf8F23LB23T+X9VrbNh82nn6UAz7ASzSCeANZe0AnBctIqqbIoojLCIIBvoL5pJw31DH7Ry3VDKsoYinSii4ZyXxhBQM2Fwwt58D7NeoB8QkXfDvwRd2XtceOsCHkwc8KCINAk+vADJppQUFjZ0DsGVGT3uFn1KSjoPeKLoaYtvCO/rIlz3vH9O5FiU/nXny/pDT6YGKZngg0/Zg1GErrMbp6N5NHxJFi3N/4dRkj5IYf5ULxCmiPJpI4rIr4kHimhvbWfyLHOyOzQpNZZ57jXNy4nRGFLTR/0fWBqe7w==}{在演练场中查看完整示例}
\switchcolumn[0]*%%%%%%%
\begin{center}\rule{0.5\linewidth}{0.5pt}\end{center}
\switchcolumn
\begin{center}\rule{0.5\linewidth}{0.5pt}\end{center}
\switchcolumn[0]*%%%%%%%
\textbf{Related}
\switchcolumn
\textbf{参考}
\switchcolumn[0]*%%%%%%%
\begin{itemize}
\item
  \href{https://vuejs.org/api/built-in-components.html\#transitiongroup}{``
  API reference}
\end{itemize}
\switchcolumn
\begin{itemize}
\item
  \href{https://cn.vuejs.org/api/built-in-components.html\#transitiongroup}{``
  API 参考}
\end{itemize}
\end{paracol}

