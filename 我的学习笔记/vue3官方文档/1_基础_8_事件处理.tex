\columnratio{0.55}
\begin{paracol}{2}

\switchcolumn[0]*%%%%%%%
\section{Event Handling}
\switchcolumn
\section{事件处理}
\switchcolumn[0]*%%%%%%%
\subsection{Listening to Events}
\switchcolumn
\subsection{监听事件}
\switchcolumn[0]*%%%%%%%
We can use the \texttt{v-on} directive, which we typically shorten to
the \texttt{@} symbol, to listen to DOM events and run some JavaScript
when they're triggered. The usage would be \texttt{v-on:click="handler"}
or with the shortcut, \texttt{@click="handler"}.
\switchcolumn
我们可以使用 \texttt{v-on} 指令 (简写为 \texttt{@}) 来监听 DOM
事件,并在事件触发时执行对应的
JavaScript。用法:\texttt{v-on:click="handler"} 或
\texttt{@click="handler"}。
\switchcolumn[0]*%%%%%%%
The handler value can be one of the following:
\switchcolumn
事件处理器 (handler) 的值可以是:
\switchcolumn[0]*%%%%%%%
\begin{enumerate}
\def\labelenumi{\arabic{enumi}.}
\item
    \textbf{Inline handlers:} Inline JavaScript to be executed when the
    event is triggered (similar to the native \texttt{onclick} attribute).
\item
    \textbf{Method handlers:} A property name or path that points to a
    method defined on the component.
\end{enumerate}
\switchcolumn
\begin{enumerate}
\def\labelenumi{\arabic{enumi}.}
\item
    \textbf{内联事件处理器}:事件被触发时执行的内联 JavaScript 语句 (与
    \texttt{onclick} 类似)。
\item
    \textbf{方法事件处理器}:一个指向组件上定义的方法的属性名或是路径。
\end{enumerate}
\switchcolumn[0]*%%%%%%%
\subsection{Inline Handlers}
\switchcolumn
\subsection{内联事件处理器}
\switchcolumn[0]*%%%%%%%
Inline handlers are typically used in simple cases, for example:
\switchcolumn
内联事件处理器通常用于简单场景,例如:

\switchcolumn[0]*%%%%%%%
\begin{codeJs}
const count = ref(0)
\end{codeJs}
\switchcolumn
\begin{codeJs}
const count = ref(0)
\end{codeJs}
\switchcolumn[0]*%%%%%%%
\begin{codeHtml}
<button @click="count++">Add 1</button>
<p>Count is: {{ count }}</p>
\end{codeHtml}
\switchcolumn
\begin{codeHtml}
<button @click="count++">Add 1</button>
<p>Count is: {{ count }}</p>
\end{codeHtml}
\switchcolumn[0]*%%%%%%%
\href{https://play.vuejs.org/\#eNo9jssKgzAURH/lko0tgrbbEqX+Q5fZaLxiqHmQ3LgJ+fdqFZcD58xMYp1z1RqRvRgP0itHEJCia4VR2llPkMDjBBkmbzUUG1oII4y0JhBIGw2hh2Znbo+7MLw+WjZ/C4TaLT3hnogPkcgaeMtFyW8j2GmXpWBtN47w5PWBHLhrPzPCKfWDXRHmPsCAaOBfgSOkdH3IGUhpDBWv9/e8vsZZ/gFFhFJN}{Try
it in the Playground}
\switchcolumn
\href{https://play.vuejs.org/\#eNo9jssKgzAURH/lko0tgrbbEqX+Q5fZaLxiqHmQ3LgJ+fdqFZcD58xMYp1z1RqRvRgP0itHEJCia4VR2llPkMDjBBkmbzUUG1oII4y0JhBIGw2hh2Znbo+7MLw+WjZ/C4TaLT3hnogPkcgaeMtFyW8j2GmXpWBtN47w5PWBHLhrPzPCKfWDXRHmPsCAaOBfgSOkdH3IGUhpDBWv9/e8vsZZ/gFFhFJN}{在演练场中尝试一下}
\switchcolumn[0]*%%%%%%%
\subsection{Method Handlers}
\switchcolumn
\subsection{方法事件处理器}
\switchcolumn[0]*%%%%%%%
The logic for many event handlers will be more complex though, and
likely isn't feasible with inline handlers. That's why \texttt{v-on} can
also accept the name or path of a component method you'd like to call.
\switchcolumn
随着事件处理器的逻辑变得愈发复杂,内联代码方式变得不够灵活。因此
\texttt{v-on} 也可以接受一个方法名或对某个方法的调用。

\switchcolumn[0]*%%%%%%%
For example:
\switchcolumn
举例来说:
\switchcolumn[0]*%%%%%%%
\begin{codeJs}
const name = ref('Vue.js')
function greet(event) {
  alert(`Hello ${name.value}!`)
  // `event` 是 DOM 原生事件
  if (event) {
    alert(event.target.tagName)
  }
}
\end{codeJs}
\switchcolumn
\begin{codeJs}
const name = ref('Vue.js')
function greet(event) {
  alert(`Hello ${name.value}!`)
  // `event` 是 DOM 原生事件
  if (event) {
    alert(event.target.tagName)
  }
}
\end{codeJs}
\switchcolumn[0]*%%%%%%%
\begin{codeHtml}
<!-- `greet` 是上面定义过的方法名 -->
<button @click="greet">Greet</button>
\end{codeHtml}
\switchcolumn
\begin{codeHtml}
<!-- `greet` 是上面定义过的方法名 -->
<button @click="greet">Greet</button>
\end{codeHtml}
\switchcolumn[0]*%%%%%%%
\href{https://play.vuejs.org/\#eNpVj0FLxDAQhf/KMwjtXtq7dBcFQS/qzVMOrWFao2kSkkkvpf/dJIuCEBgm771vZnbx4H23JRJ3YogqaM+IxMlfpNWrd4GxI9CMA3NwK5psbaSVVjkbGXZaCediaJv3RN1XbE5FnZNVrJ3FEoi4pY0sn7BLC0yGArfjMxnjcLsXQrdNJtFxM+Ys0PcYa2CEjuBPylNYb4THtxdUobj0jH/YX3D963gKC5WyvGZ+xR7S5jf01yPzeblhWr2ZmErHw0dizivfK6PV91mKursUl6dSh/4qZ+vQ/+XE8QODonDi}{Try
it in the Playground}
\switchcolumn
\href{https://play.vuejs.org/\#eNpVj0FLxDAQhf/KMwjtXtq7dBcFQS/qzVMOrWFao2kSkkkvpf/dJIuCEBgm771vZnbx4H23JRJ3YogqaM+IxMlfpNWrd4GxI9CMA3NwK5psbaSVVjkbGXZaCediaJv3RN1XbE5FnZNVrJ3FEoi4pY0sn7BLC0yGArfjMxnjcLsXQrdNJtFxM+Ys0PcYa2CEjuBPylNYb4THtxdUobj0jH/YX3D963gKC5WyvGZ+xR7S5jf01yPzeblhWr2ZmErHw0dizivfK6PV91mKursUl6dSh/4qZ+vQ/+XE8QODonDi}{在演练场中尝试一下}
\switchcolumn[0]*%%%%%%%
A method handler automatically receives the native DOM Event object that
triggers it - in the example above, we are able to access the element
dispatching the event via \texttt{event.target.tagName}.
\switchcolumn
方法事件处理器会自动接收原生 DOM
事件并触发执行。在上面的例子中,我们能够通过被触发事件的
\texttt{event.target.tagName} 访问到该 DOM 元素。
\switchcolumn[0]*%%%%%%%
See also:
\href{https://vuejs.org/guide/typescript/composition-api.html\#typing-event-handlers}{Typing
Event Handlers}
\switchcolumn
你也可以看看\href{https://cn.vuejs.org/guide/typescript/composition-api.html\#typing-event-handlers}{为事件处理器标注类型}这一章了解更多。
\end{paracol}

\columnratio{0.55}
\begin{paracol}{2}

\switchcolumn[0]*%%%%%%%
\subsubsection{Method vs. Inline Detection}
\switchcolumn
\subsubsection{方法与内联事件判断}
\switchcolumn[0]*%%%%%%%
The template compiler detects method handlers by checking whether the
\texttt{v-on} value string is a valid JavaScript identifier or property
access path. For example, \texttt{foo}, \texttt{foo.bar} and
\texttt{foo{[}\textquotesingle{}bar\textquotesingle{}{]}} are treated as
method handlers, while \texttt{foo()} and \texttt{count++} are treated
as inline handlers.
\switchcolumn
模板编译器会通过检查 \texttt{v-on} 的值是否是合法的 JavaScript
标识符或属性访问路径来断定是何种形式的事件处理器。举例来说,\texttt{foo}、\texttt{foo.bar}
和 \texttt{foo{[}\textquotesingle{}bar\textquotesingle{}{]}}
会被视为方法事件处理器,而 \texttt{foo()} 和 \texttt{count++}
会被视为内联事件处理器。
\switchcolumn[0]*%%%%%%%
\subsection{Calling Methods in Inline Handlers}
\switchcolumn
\subsection{在内联处理器中调用方法}
\switchcolumn[0]*%%%%%%%
Instead of binding directly to a method name, we can also call methods
in an inline handler. This allows us to pass the method custom arguments
instead of the native event:
\switchcolumn
除了直接绑定方法名,你还可以在内联事件处理器中调用方法。这允许我们向方法传入自定义参数以代替原生事件:
\switchcolumn[0]*%%%%%%%
\begin{codeJs}
function say(message) {
    alert(message)
}
\end{codeJs}
\switchcolumn
\begin{codeJs}
function say(message) {
    alert(message)
}
\end{codeJs}
\switchcolumn[0]*%%%%%%%
\begin{codeHtml}
<button @click="say('hello')">Say hello</button>
<button @click="say('bye')">Say bye</button>
\end{codeHtml}
\switchcolumn
\begin{codeHtml}
<button @click="say('hello')">Say hello</button>
<button @click="say('bye')">Say bye</button>
\end{codeHtml}

\switchcolumn[0]*%%%%%%%
\href{https://play.vuejs.org/\#eNp9jTEOwjAMRa8SeSld6I5CBWdg9ZJGBiJSN2ocpKjq3UmpFDGx+Vn//b/ANYTjOxGcQEc7uyAqkqTQI98TW3ETq2jyYaQYzYNatSArZTzNUn/IK7Ludr2IBYTG4I3QRqKHJFJ6LtY7+zojbIXNk7yfmhahv5msvqS7PfnHGjJVp9w/hu7qKKwfEd1NSg==}{Try
it in the Playground}
\switchcolumn
\href{https://play.vuejs.org/\#eNp9jTEOwjAMRa8SeSld6I5CBWdg9ZJGBiJSN2ocpKjq3UmpFDGx+Vn//b/ANYTjOxGcQEc7uyAqkqTQI98TW3ETq2jyYaQYzYNatSArZTzNUn/IK7Ludr2IBYTG4I3QRqKHJFJ6LtY7+zojbIXNk7yfmhahv5msvqS7PfnHGjJVp9w/hu7qKKwfEd1NSg==}{在演练场中尝试一下}
\end{paracol}

\columnratio{0.55}
\begin{paracol}{2}

\switchcolumn[0]*%%%%%%%
\subsection{Accessing Event Argument in Inline Handlers}
\switchcolumn
\subsection{在内联事件处理器中访问事件参数}
\switchcolumn[0]*%%%%%%%
Sometimes we also need to access the original DOM event in an inline
handler. You can pass it into a method using the special
\texttt{\$event} variable, or use an inline arrow function:
\switchcolumn
有时我们需要在内联事件处理器中访问原生 DOM
事件。你可以向该处理器方法传入一个特殊的 \texttt{\$event}
变量,或者使用内联箭头函数:
\switchcolumn[0]*%%%%%%%
\begin{codeHtml}
<!-- 使用特殊的 $event 变量 -->
<button @click="warn('Form cannot be submitted yet.', $event)">
    Submit
</button>
<!-- 使用内联箭头函数 -->
<button @click="(event) => warn('Form cannot be submitted yet.', event)">
    Submit
</button>
\end{codeHtml}
\switchcolumn
\begin{codeHtml}
<!-- 使用特殊的 $event 变量 -->
<button @click="warn('Form cannot be submitted yet.', $event)">
    Submit
</button>
<!-- 使用内联箭头函数 -->
<button @click="(event) => warn('Form cannot be submitted yet.', event)">
    Submit
</button>
\end{codeHtml}
\switchcolumn[0]*%%%%%%%
\begin{codeJs}
function warn(message, event) {
    // 这里可以访问原生事件
    if (event) {
    event.preventDefault()
    }
    alert(message)
}
\end{codeJs}
\switchcolumn
\begin{codeJs}
function warn(message, event) {
    // 这里可以访问原生事件
    if (event) {
    event.preventDefault()
    }
    alert(message)
}
\end{codeJs}
\end{paracol}

\columnratio{0.55}
\begin{paracol}{2}

\switchcolumn[0]*%%%%%%%
\subsection{Event Modifiers}
\switchcolumn
\subsection{事件修饰符}
\switchcolumn[0]*%%%%%%%
It is a very common need to call \texttt{event.preventDefault()} or
\texttt{event.stopPropagation()} inside event handlers. Although we can
do this easily inside methods, it would be better if the methods can be
purely about data logic rather than having to deal with DOM event
details.
\switchcolumn
在处理事件时调用 \texttt{event.preventDefault()} 或
\texttt{event.stopPropagation()}
是很常见的。尽管我们可以直接在方法内调用,但如果方法能更专注于数据逻辑而不用去处理
DOM 事件的细节会更好。
\switchcolumn[0]*%%%%%%%
To address this problem, Vue provides \textbf{event modifiers} for
\texttt{v-on}. Recall that modifiers are directive postfixes denoted by
a dot.
\switchcolumn
为解决这一问题,Vue 为 \texttt{v-on}
提供了\textbf{事件修饰符}。修饰符是用 \texttt{.}
表示的指令后缀,包含以下这些:
\switchcolumn[0]*%%%%%%%
\begin{itemize}
\item
    \texttt{.stop}
\item
    \texttt{.prevent}
\item
    \texttt{.self}
\item
    \texttt{.capture}
\item
    \texttt{.once}
\item
    \texttt{.passive}
\end{itemize}
\switchcolumn
\begin{itemize}
\item
    \texttt{.stop}
\item
    \texttt{.prevent}
\item
    \texttt{.self}
\item
    \texttt{.capture}
\item
    \texttt{.once}
\item
    \texttt{.passive}
\end{itemize}


\switchcolumn[0]*%%%%%%%
\begin{codeHtml}
<!-- 单击事件将停止传递 -->
<a @click.stop="doThis"></a>
<!-- 提交事件将不再重新加载页面 -->
<form @submit.prevent="onSubmit"></form>
<!-- 修饰语可以使用链式书写 -->
<a @click.stop.prevent="doThat"></a>
<!-- 也可以只有修饰符 -->
<form @submit.prevent></form>
<!-- 仅当 event.target 是元素本身时才会触发事件处理器 -->
<!-- 例如:事件处理器不来自子元素 -->
<div @click.self="doThat">...</div>
\end{codeHtml}
\switchcolumn
\begin{codeHtml}
<!-- 单击事件将停止传递 -->
<a @click.stop="doThis"></a>
<!-- 提交事件将不再重新加载页面 -->
<form @submit.prevent="onSubmit"></form>
<!-- 修饰语可以使用链式书写 -->
<a @click.stop.prevent="doThat"></a>
<!-- 也可以只有修饰符 -->
<form @submit.prevent></form>
<!-- 仅当 event.target 是元素本身时才会触发事件处理器 -->
<!-- 例如:事件处理器不来自子元素 -->
<div @click.self="doThat">...</div>
\end{codeHtml}
\switchcolumn[0]*%%%%%%%
\begin{vueQuote}{TIP}
Order matters when using modifiers because the relevant code is
generated in the same order. Therefore using
\texttt{@click.prevent.self} will prevent \textbf{click's default action
on the element itself and its children}, while
\texttt{@click.self.prevent} will only prevent click's default action on
the element itself.
\end{vueQuote}
\switchcolumn
\begin{vueQuote}{TIP}
使用修饰符时需要注意调用顺序,因为相关代码是以相同的顺序生成的。因此使用
\texttt{@click.prevent.self}
会阻止\textbf{元素及其子元素的所有点击事件的默认行为},而
\texttt{@click.self.prevent} 则只会阻止对元素本身的点击事件的默认行为。
\end{vueQuote}
\switchcolumn[0]*%%%%%%%
The \texttt{.capture}, \texttt{.once}, and \texttt{.passive} modifiers
mirror the
\href{https://developer.mozilla.org/en-US/docs/Web/API/EventTarget/addEventListener\#options}{options
of the native \texttt{addEventListener} method}:
\switchcolumn
\texttt{.capture}、\texttt{.once} 和 \texttt{.passive}
修饰符与\href{https://developer.mozilla.org/zh-CN/docs/Web/API/EventTarget/addEventListener\#options}{原生
\texttt{addEventListener} 事件}相对应:
\switchcolumn[0]*%%%%%%%
\begin{codeHtml}
<!-- 添加事件监听器时,使用 `capture` 捕获模式 -->
<!-- 例如:指向内部元素的事件,在被内部元素处理前,先被外部处理 -->
<div @click.capture="doThis">...</div>
<!-- 点击事件最多被触发一次 -->
<a @click.once="doThis"></a>
<!-- 滚动事件的默认行为 (scrolling) 将立即发生而非等待 `onScroll` 完成 -->
<!-- 以防其中包含 `event.preventDefault()` -->
<div @scroll.passive="onScroll">...</div>
\end{codeHtml}
\switchcolumn
\begin{codeHtml}
<!-- 添加事件监听器时,使用 `capture` 捕获模式 -->
<!-- 例如:指向内部元素的事件,在被内部元素处理前,先被外部处理 -->
<div @click.capture="doThis">...</div>
<!-- 点击事件最多被触发一次 -->
<a @click.once="doThis"></a>
<!-- 滚动事件的默认行为 (scrolling) 将立即发生而非等待 `onScroll` 完成 -->
<!-- 以防其中包含 `event.preventDefault()` -->
<div @scroll.passive="onScroll">...</div>
\end{codeHtml}


\switchcolumn[0]*%%%%%%%
The \texttt{.passive} modifier is typically used with touch event
listeners for
\href{https://developer.mozilla.org/en-US/docs/Web/API/EventTarget/addEventListener\#improving_scroll_performance_using_passive_listeners}{improving
performance on mobile devices}.
\switchcolumn
\texttt{.passive}
修饰符一般用于触摸事件的监听器,可以用来\href{https://developer.mozilla.org/zh-CN/docs/Web/API/EventTarget/addEventListener\#使用_passive_改善滚屏性能}{改善移动端设备的滚屏性能}。
\switchcolumn[0]*%%%%%%%
\begin{vueQuote}{TIP}
Do not use \texttt{.passive} and \texttt{.prevent} together, because
\texttt{.passive} already indicates to the browser that you \emph{do
not} intend to prevent the event's default behavior, and you will likely
see a warning from the browser if you do so.
\end{vueQuote}
\switchcolumn
\begin{vueQuote}{TIP}
请勿同时使用 \texttt{.passive} 和 \texttt{.prevent},因为
\texttt{.passive}
已经向浏览器表明了你\emph{不想}阻止事件的默认行为。如果你这么做了,则
\texttt{.prevent} 会被忽略,并且浏览器会抛出警告。
\end{vueQuote}
\end{paracol}

\columnratio{0.55}
\begin{paracol}{2}

\switchcolumn[0]*%%%%%%%
\subsection{Key Modifiers}
\switchcolumn
\subsection{按键修饰符}
\switchcolumn[0]*%%%%%%%
When listening for keyboard events, we often need to check for specific
keys. Vue allows adding key modifiers for \texttt{v-on} or \texttt{@}
when listening for key events:
\switchcolumn
在监听键盘事件时,我们经常需要检查特定的按键。Vue 允许在 \texttt{v-on}
或 \texttt{@} 监听按键事件时添加按键修饰符。
\switchcolumn[0]*%%%%%%%
\begin{codeHtml}
<!-- 仅在 `key` 为 `Enter` 时调用 `submit` -->
<input @keyup.enter="submit" />
\end{codeHtml}
\switchcolumn
\begin{codeHtml}
<!-- 仅在 `key` 为 `Enter` 时调用 `submit` -->
<input @keyup.enter="submit" />
\end{codeHtml}
\switchcolumn[0]*%%%%%%%
You can directly use any valid key names exposed via
\href{https://developer.mozilla.org/en-US/docs/Web/API/KeyboardEvent/key/Key_Values}{\texttt{KeyboardEvent.key}}
as modifiers by converting them to kebab-case.
\switchcolumn
你可以直接使用
\href{https://developer.mozilla.org/zh-CN/docs/Web/API/KeyboardEvent/key/Key_Values}{\texttt{KeyboardEvent.key}}
暴露的按键名称作为修饰符,但需要转为 kebab-case 形式。
\switchcolumn[0]*%%%%%%%
\begin{codeHtml}
<input @keyup.page-down="onPageDown" />
\end{codeHtml}
\switchcolumn
\begin{codeHtml}
<input @keyup.page-down="onPageDown" />
\end{codeHtml}
\switchcolumn[0]*%%%%%%%
In the above example, the handler will only be called if
\texttt{\$event.key} is equal to
\texttt{\textquotesingle{}PageDown\textquotesingle{}}.
\switchcolumn
在上面的例子中,仅会在 \texttt{\$event.key} 为
\texttt{\textquotesingle{}PageDown\textquotesingle{}} 时调用事件处理。
\end{paracol}

\columnratio{0.55}
\begin{paracol}{2}

\switchcolumn[0]*%%%%%%%
\subsubsection{Key Aliases}
\switchcolumn
\subsubsection{按键别名}
\switchcolumn[0]*%%%%%%%
Vue provides aliases for the most commonly used keys:
\switchcolumn
Vue 为一些常用的按键提供了别名:
\switchcolumn[0]*%%%%%%%
\begin{itemize}
\item
    \texttt{.enter}
\item
    \texttt{.tab}
\item
    \texttt{.delete} (captures both "Delete" and "Backspace" keys)
\item
    \texttt{.esc}
\item
    \texttt{.space}
\item
    \texttt{.up}
\item
    \texttt{.down}
\item
    \texttt{.left}
\item
    \texttt{.right}
\end{itemize}
\switchcolumn
\begin{itemize}
\item
    \texttt{.enter}
\item
    \texttt{.tab}
\item
    \texttt{.delete} (捕获``Delete''和``Backspace''两个按键)
\item
    \texttt{.esc}
\item
    \texttt{.space}
\item
    \texttt{.up}
\item
    \texttt{.down}
\item
    \texttt{.left}
\item
    \texttt{.right}
\end{itemize}
\end{paracol}

\columnratio{0.55}
\begin{paracol}{2}

\switchcolumn[0]*%%%%%%%
\subsubsection{System Modifier Keys}
\switchcolumn
\subsubsection{系统按键修饰符}
\switchcolumn[0]*%%%%%%%
You can use the following modifiers to trigger mouse or keyboard event
listeners only when the corresponding modifier key is pressed:
\switchcolumn
你可以使用以下系统按键修饰符来触发鼠标或键盘事件监听器,只有当按键被按下时才会触发。
\switchcolumn[0]*%%%%%%%
\begin{itemize}
\item
    \texttt{.ctrl}
\item
    \texttt{.alt}
\item
    \texttt{.shift}
\item
    \texttt{.meta}
\end{itemize}
\switchcolumn
\begin{itemize}
\item
    \texttt{.ctrl}
\item
    \texttt{.alt}
\item
    \texttt{.shift}
\item
    \texttt{.meta}
\end{itemize}
\switchcolumn[0]*%%%%%%%
\begin{vueQuote}{Note}
On Macintosh keyboards, meta is the command key (⌘). On Windows
keyboards, meta is the Windows key (⊞). On Sun Microsystems keyboards,
meta is marked as a solid diamond (◆). On certain keyboards,
specifically MIT and Lisp machine keyboards and successors, such as the
Knight keyboard, space-cadet keyboard, meta is labeled ``META''. On
Symbolics keyboards, meta is labeled ``META'' or ``Meta''.
\end{vueQuote}
\switchcolumn
\begin{vueQuote}{注意}
在 Mac 键盘上,meta 是 Command 键 (⌘)。在 Windows 键盘上,meta 键是
Windows 键 (⊞)。在 Sun 微机系统键盘上,meta 是钻石键
(◆)。在某些键盘上,特别是 MIT 和 Lisp 机器的键盘及其后代版本的键盘,如
Knight 键盘,space-cadet 键盘,meta 都被标记为``META''。在 Symbolics
键盘上,meta 也被标识为``META''或``Meta''。
\end{vueQuote}
\switchcolumn[0]*%%%%%%%
For example:
\switchcolumn
举例来说:
\switchcolumn[0]*%%%%%%%
\begin{codeHtml}
<!-- Alt + Enter -->
<input @keyup.alt.enter="clear" />
<!-- Ctrl + 点击 -->
<div @click.ctrl="doSomething">Do something</div>
\end{codeHtml}
\switchcolumn
\begin{codeHtml}
<!-- Alt + Enter -->
<input @keyup.alt.enter="clear" />
<!-- Ctrl + 点击 -->
<div @click.ctrl="doSomething">Do something</div>
\end{codeHtml}
\switchcolumn[0]*%%%%%%%
\begin{vueQuote}{TIP}
Note that modifier keys are different from regular keys and when used
with \texttt{keyup} events, they have to be pressed when the event is
emitted. In other words, \texttt{keyup.ctrl} will only trigger if you
release a key while holding down \texttt{ctrl}. It won't trigger if you
release the \texttt{ctrl} key alone.
\end{vueQuote}
\switchcolumn
\begin{vueQuote}{TIP}
请注意,系统按键修饰符和常规按键不同。与 \texttt{keyup}
事件一起使用时,该按键必须在事件发出时处于按下状态。换句话说,\texttt{keyup.ctrl}
只会在你仍然按住 \texttt{ctrl} 但松开了另一个键时被触发。若你单独松开
\texttt{ctrl} 键将不会触发。
\end{vueQuote}


\switchcolumn[0]*%%%%%%%
\subsubsection{.exact Modifier}
\switchcolumn
\subsubsection{.exact 修饰符}
\switchcolumn[0]*%%%%%%%
The \texttt{.exact} modifier allows control of the exact combination of
system modifiers needed to trigger an event.
\switchcolumn
\texttt{.exact}
修饰符允许控制触发一个事件所需的确定组合的系统按键修饰符。
\switchcolumn[0]*%%%%%%%
\begin{codeHtml}
<!-- 当按下 Ctrl 时,即使同时按下 Alt 或 Shift 也会触发 -->
<button @click.ctrl="onClick">A</button>
<!-- 仅当按下 Ctrl 且未按任何其他键时才会触发 -->
<button @click.ctrl.exact="onCtrlClick">A</button>
<!-- 仅当没有按下任何系统按键时触发 -->
<button @click.exact="onClick">A</button>
\end{codeHtml}
\switchcolumn
\begin{codeHtml}
<!-- 当按下 Ctrl 时,即使同时按下 Alt 或 Shift 也会触发 -->
<button @click.ctrl="onClick">A</button>
<!-- 仅当按下 Ctrl 且未按任何其他键时才会触发 -->
<button @click.ctrl.exact="onCtrlClick">A</button>
<!-- 仅当没有按下任何系统按键时触发 -->
<button @click.exact="onClick">A</button>
\end{codeHtml} 
\end{paracol}

\columnratio{0.55}
\begin{paracol}{2}
\switchcolumn[0]*%%%%%%%
\subsection{Mouse Button Modifiers}
\switchcolumn
\subsection{鼠标按键修饰符}
\switchcolumn[0]*%%%%%%%
\begin{itemize}
\item
    \texttt{.left}
\item
    \texttt{.right}
\item
    \texttt{.middle}
\end{itemize}
\switchcolumn
\begin{itemize}
\item
    \texttt{.left}
\item
    \texttt{.right}
\item
    \texttt{.middle}
\end{itemize}
\switchcolumn[0]*%%%%%%%
These modifiers restrict the handler to events triggered by a specific
mouse button.
\switchcolumn
这些修饰符将处理程序限定为由特定鼠标按键触发的事件。
\end{paracol}
