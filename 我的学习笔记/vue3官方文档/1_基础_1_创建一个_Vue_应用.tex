\columnratio{0.55}
\begin{paracol}{2}
\switchcolumn[0]*%%%%%%%
\section{Creating a Vue Application}
\switchcolumn
\section{创建一个 Vue 应用}
\switchcolumn[0]*%%%%%%%
\subsection{The application instance}
\switchcolumn
\subsection{应用实例}
\switchcolumn[0]*%%%%%%%
Every Vue application starts by creating a new \textbf{application
instance} with the
\href{https://vuejs.org/api/application.html\#createapp}{\texttt{createApp}}
function:
\switchcolumn
每个 Vue 应用都是通过
\href{https://cn.vuejs.org/api/application.html\#createapp}{\texttt{createApp}}
函数创建一个新的 \textbf{应用实例}:
\switchcolumn[0]*%%%%%%%
\begin{codeJs}
import { createApp } from 'vue'

const app = createApp({
    /* root component options */
})
\end{codeJs}
\switchcolumn
\begin{codeJs}
import { createApp } from 'vue'

const app = createApp({
    /* root component options */
})
\end{codeJs}
\switchcolumn[0]*%%%%%%%
\subsection{The Root Component}
\switchcolumn
\subsection{根组件}
\switchcolumn[0]*%%%%%%%
The object we are passing into \texttt{createApp} is in fact a
component. Every app requires a "root component" that can contain other
components as its children.
\switchcolumn
我们传入 \texttt{createApp}
的对象实际上是一个组件,每个应用都需要一个``根组件'',其他组件将作为其子组件。
\switchcolumn[0]*%%%%%%%
If you are using Single-File Components, we typically import the root
component from another file:
\switchcolumn
如果你使用的是单文件组件,我们可以直接从另一个文件中导入根组件。
\switchcolumn[0]*%%%%%%%
\begin{codeJs}
import { createApp } from 'vue'
// import the root component App from a single-file component.
import App from './App.vue'

const app = createApp(App)
\end{codeJs}
\switchcolumn
\begin{codeJs}
import { createApp } from 'vue'
// import the root component App from a single-file component.
import App from './App.vue'

const app = createApp(App)
\end{codeJs}
\switchcolumn[0]*%%%%%%%
While many examples in this guide only need a single component, most
real applications are organized into a tree of nested, reusable
components. For example, a Todo application's component tree might look
like this:
\switchcolumn
虽然本指南中的许多示例只需要一个组件,但大多数真实的应用都是由一棵嵌套的、可重用的组件树组成的。例如,一个待办事项
(Todos) 应用的组件树可能是这样的:
\switchcolumn[0]*%%%%%%%
\begin{verbatim}
App (root component)
├─ TodoList
│  └─ TodoItem
│     ├─ TodoDeleteButton
│     └─ TodoEditButton
└─ TodoFooter
    ├─ TodoClearButton
    └─ TodoStatistics
\end{verbatim}
\switchcolumn
\begin{verbatim}
App (root component)
├─ TodoList
│  └─ TodoItem
│     ├─ TodoDeleteButton
│     └─ TodoEditButton
└─ TodoFooter
    ├─ TodoClearButton
    └─ TodoStatistics
\end{verbatim}
\switchcolumn[0]*%%%%%%%
In later sections of the guide, we will discuss how to define and
compose multiple components together. Before that, we will focus on what
happens inside a single component.
\switchcolumn
我们会在指南的后续章节中讨论如何定义和组合多个组件。在那之前,我们得先关注一个组件内到底发生了什么。
\switchcolumn[0]*%%%%%%%
\subsection{Mounting the App}
\switchcolumn
\subsection{挂载应用}
\switchcolumn[0]*%%%%%%%
An application instance won't render anything until its
\texttt{.mount()} method is called. It expects a "container" argument,
which can either be an actual DOM element or a selector string:
\switchcolumn
应用实例必须在调用了 \texttt{.mount()}
方法后才会渲染出来。该方法接收一个``容器''参数,可以是一个实际的 DOM
元素或是一个 CSS 选择器字符串:
\switchcolumn[0]*%%%%%%%
\begin{codeHtml}
<div id="app"></div>
\end{codeHtml}
\begin{codeJs}
app.mount('#app')
\end{codeJs}
\switchcolumn
\begin{codeHtml}
<div id="app"></div>
\end{codeHtml}
\begin{codeJs}
app.mount('#app')
\end{codeJs}
\switchcolumn[0]*%%%%%%%
The content of the app's root component will be rendered inside the
container element. The container element itself is not considered part
of the app.
\switchcolumn
应用根组件的内容将会被渲染在容器元素里面。容器元素自己将\textbf{不会}被视为应用的一部分。
%%%%%
\switchcolumn[0]*%%%%%%%
The \texttt{.mount()} method should always be called after all app
configurations and asset registrations are done. Also note that its
return value, unlike the asset registration methods, is the root
component instance instead of the application instance.
\switchcolumn
\texttt{.mount()}
方法应该始终在整个应用配置和资源注册完成后被调用。同时请注意,不同于其他资源注册方法,它的返回值是根组件实例而非应用实例。
\switchcolumn[0]*%%%%%%%
\subsubsection{In-DOM Root Component Template}
\switchcolumn
\subsubsection{DOM 中的根组件模板}
\switchcolumn[0]*%%%%%%%
The template for the root component is usually part of the component
itself, but it is also possible to provide the template separately by
writing it directly inside the mount container:
\switchcolumn
根组件的模板通常是组件本身的一部分,但也可以直接通过在挂载容器内编写模板来单独提供:

\switchcolumn[0]*%%%%%%%
\begin{codeHtml}
<div id="app">
<button @click="count++">{{ count }}</button>
</div>
\end{codeHtml}
\switchcolumn
\begin{codeHtml}
<div id="app">
<button @click="count++">{{ count }}</button>
</div>
\end{codeHtml}


\switchcolumn[0]*%%%%%%%
\begin{codeJs}
import { createApp } from 'vue'

const app = createApp({
    data() {
    return {
        count: 0
    }
    }
})

app.mount('#app')
\end{codeJs}
\switchcolumn
\begin{codeJs}
import { createApp } from 'vue'

const app = createApp({
    data() {
    return {
        count: 0
    }
    }
})

app.mount('#app')
\end{codeJs}


\switchcolumn[0]*%%%%%%%
Vue will automatically use the container's \texttt{innerHTML} as the
template if the root component does not already have a \texttt{template}
option.
\switchcolumn
当根组件没有设置 \texttt{template} 选项时,Vue 将自动使用容器的
\texttt{innerHTML} 作为模板。
\switchcolumn[0]*%%%%%%%
In-DOM templates are often used in applications that are
\href{https://vuejs.org/guide/quick-start.html\#using-vue-from-cdn}{using
Vue without a build step}. They can also be used in conjunction with
server-side frameworks, where the root template might be generated
dynamically by the server.
\switchcolumn
DOM
内模板通常用于\href{https://cn.vuejs.org/guide/quick-start.html\#using-vue-from-cdn}{无构建步骤}的
Vue
应用程序。它们也可以与服务器端框架一起使用,其中根模板可能是由服务器动态生成的。
\switchcolumn[0]*%%%%%%%
\subsection{App Configurations}
\switchcolumn
\subsection{应用配置}
\switchcolumn[0]*%%%%%%%
The application instance exposes a \texttt{.config} object that allows
us to configure a few app-level options, for example, defining an
app-level error handler that captures errors from all descendant
components:
\switchcolumn
应用实例会暴露一个 \texttt{.config}
对象允许我们配置一些应用级的选项,例如定义一个应用级的错误处理器,用来捕获所有子组件上的错误:

\switchcolumn[0]*%%%%%%%
\begin{codeJs}
app.config.errorHandler = (err) => {
/* handle error */
}
\end{codeJs}
\switchcolumn
\begin{codeJs}
    app.config.errorHandler = (err) => {
/* handle error */
}
\end{codeJs}

\switchcolumn[0]*%%%%%%%
The application instance also provides a few methods for registering
app-scoped assets. For example, registering a component:
\switchcolumn
应用实例还提供了一些方法来注册应用范围内可用的资源,例如注册一个组件:
\switchcolumn[0]*%%%%%%%
\begin{codeJs}
app.component('TodoDeleteButton', TodoDeleteButton)
\end{codeJs}
\switchcolumn
\begin{codeJs}
app.component('TodoDeleteButton', TodoDeleteButton)
\end{codeJs}

\switchcolumn[0]*%%%%%%%
This makes the \texttt{TodoDeleteButton} available for use anywhere in
our app. We will discuss registration for components and other types of
assets in later sections of the guide. You can also browse the full list
of application instance APIs in its
\href{https://vuejs.org/api/application.html}{API reference}.
\switchcolumn
这使得 \texttt{TodoDeleteButton}
在应用的任何地方都是可用的。我们会在指南的后续章节中讨论关于组件和其他资源的注册。你也可以在
\href{https://cn.vuejs.org/api/application.html}{API 参考}中浏览应用实例
API 的完整列表。
\switchcolumn[0]*%%%%%%%
Make sure to apply all app configurations before mounting the app!
\switchcolumn
确保在挂载应用实例之前完成所有应用配置!
\switchcolumn[0]*%%%%%%%
\subsection{Multiple application instances}
\switchcolumn
\subsection{多个应用实例}
\switchcolumn[0]*%%%%%%%
You are not limited to a single application instance on the same page.
The \texttt{createApp} API allows multiple Vue applications to co-exist
on the same page, each with its own scope for configuration and global
assets:
\switchcolumn
应用实例并不只限于一个。\texttt{createApp} API
允许你在同一个页面中创建多个共存的 Vue
应用,而且每个应用都拥有自己的用于配置和全局资源的作用域。

\end{paracol}
