
\columnratio{0.55}
\begin{paracol}{2} 

\switchcolumn[0]*%%%%%%%
\section{Production Deployment}
\switchcolumn
\section{生产部署}
\switchcolumn[0]*%%%%%%%
\subsection{Development vs. Production}
\switchcolumn
\subsection{开发环境 vs. 生产环境}
\switchcolumn[0]*%%%%%%%
During development, Vue provides a number of features to improve the
development experience:
\switchcolumn
在开发过程中,Vue 提供了许多功能来提升开发体验:  
\switchcolumn[0]*%%%%%%%
\begin{itemize}
\item
  Warning for common errors and pitfalls
\item
  Props / events validation
\item
  \href{https://vuejs.org/guide/extras/reactivity-in-depth.html\#reactivity-debugging}{Reactivity
  debugging hooks}
\item
  Devtools integration
\end{itemize}
\switchcolumn
\begin{itemize}
\item
  对常见错误和隐患的警告
\item
  对组件 props / 自定义事件的校验
\item
  \href{https://cn.vuejs.org/guide/extras/reactivity-in-depth.html\#reactivity-debugging}{响应性调试钩子}
\item
  开发工具集成
\end{itemize}
\switchcolumn[0]*%%%%%%%
However, these features become useless in production. Some of the
warning checks can also incur a small amount of performance overhead.
When deploying to production, we should drop all the unused,
development-only code branches for smaller payload size and better
performance.
\switchcolumn
然而,这些功能在生产环境中并不会被使用,一些警告检查也会产生少量的性能开销。当部署到生产环境中时,我们应该移除所有未使用的、仅用于开发环境的代码分支,来获得更小的包体积和更好的性能。
\switchcolumn[0]*%%%%%%%
\subsection{Without Build Tools}
\switchcolumn
\subsection{不使用构建工具}
\switchcolumn[0]*%%%%%%%
If you are using Vue without a build tool by loading it from a CDN or
self-hosted script, make sure to use the production build (dist files
that end in \texttt{.prod.js}) when deploying to production. Production
builds are pre-minified with all development-only code branches removed.
\switchcolumn
如果你没有使用任何构建工具,而是从 CDN 或其他源来加载
Vue,请确保在部署时使用的是生产环境版本(以 \texttt{.prod.js}
结尾的构建文件)。生产环境版本会被最小化,并移除了所有仅用于开发环境的代码分支。
\switchcolumn[0]*%%%%%%%
\begin{itemize}
\item
  If using global build (accessing via the \texttt{Vue} global): use
  \texttt{vue.global.prod.js}.
\item
  If using ESM build (accessing via native ESM imports): use
  \texttt{vue.esm-browser.prod.js}.
\end{itemize}
\switchcolumn
\begin{itemize}
\item
  如果需要使用全局变量版本(通过 \texttt{Vue} 全局变量访问):请使用
  \texttt{vue.global.prod.js}。
\item
  如果需要 ESM 版本(通过原生 ESM 导入访问):请使用
  \texttt{vue.esm-browser.prod.js}。
\end{itemize}
\switchcolumn[0]*%%%%%%%
Consult the
\href{https://github.com/vuejs/core/tree/main/packages/vue\#which-dist-file-to-use}{dist
file guide} for more details.
\switchcolumn
更多细节请参考\href{https://github.com/vuejs/core/tree/main/packages/vue\#which-dist-file-to-use}{构建文件指南}。
\end{paracol}



\columnratio{0.55}
\begin{paracol}{2} 
 
\switchcolumn[0]*%%%%%%%
\subsection{With Build Tools}
\switchcolumn
\subsection{使用构建工具}
\switchcolumn[0]*%%%%%%%
Projects scaffolded via \texttt{create-vue} (based on Vite) or Vue CLI
(based on webpack) are pre-configured for production builds.
\switchcolumn
通过 \texttt{create-vue}(基于 Vite)或是 Vue CLI(基于
webpack)搭建的项目都已经预先做好了针对生产环境的配置。
\switchcolumn[0]*%%%%%%%
If using a custom setup, make sure that:
\switchcolumn
如果使用了自定义的构建,请确保:
\switchcolumn[0]*%%%%%%%
\begin{enumerate}
\item
  \texttt{vue} resolves to \texttt{vue.runtime.esm-bundler.js}.
\item
  The
  \href{https://github.com/vuejs/core/tree/main/packages/vue\#bundler-build-feature-flags}{compile
  time feature flags} are properly configured.
\item
  \texttt{process.env.NODE\_ENV} is replaced with \texttt{"production"}
  during build.
\end{enumerate}
\switchcolumn
\begin{enumerate}
\item
  \texttt{vue} 被解析为 \texttt{vue.runtime.esm-bundler.js}。
\item
  \href{https://github.com/vuejs/core/tree/main/packages/vue\#bundler-build-feature-flags}{编译时功能标记}已被正确配置。
\item
  \texttt{process.env.NODE\_ENV} 会在构建时被替换为
  \texttt{"production"}。
\end{enumerate}
\switchcolumn[0]*%%%%%%%
Additional references:
\switchcolumn
其他参考:
\switchcolumn[0]*%%%%%%%
\begin{itemize}
\item
  \href{https://vitejs.dev/guide/build.html}{Vite production build
  guide}
\item
  \href{https://vitejs.dev/guide/static-deploy.html}{Vite deployment
  guide}
\item
  \href{https://cli.vuejs.org/guide/deployment.html}{Vue CLI deployment
  guide}
\end{itemize}
\switchcolumn
\begin{itemize}
\item
  \href{https://cn.vitejs.dev/guide/build.html}{Vite 生产环境指南}
\item
  \href{https://cn.vitejs.dev/guide/static-deploy.html}{Vite 部署指南}
\item
  \href{https://cli.vuejs.org/zh/guide/deployment.html}{Vue CLI
  部署指南}
\end{itemize}
\switchcolumn[0]*%%%%%%%
\subsection{Tracking Runtime Errors}
\switchcolumn
\subsection{追踪运行时错误}
\switchcolumn[0]*%%%%%%%
The
\href{https://vuejs.org/api/application.html\#app-config-errorhandler}{app-level
error handler} can be used to report errors to tracking services:
\switchcolumn
\href{https://cn.vuejs.org/api/application.html\#app-config-errorhandler}{应用级错误处理}
可以用来向追踪服务报告错误:
\switchcolumn[0]*%%%%%%%
\begin{codeJs}
import { createApp } from 'vue'
const app = createApp(...)
app.config.errorHandler = (err, instance, info) => {
  // 向追踪服务报告错误
}
\end{codeJs}
\switchcolumn
\begin{codeJs}
import { createApp } from 'vue'
const app = createApp(...)
app.config.errorHandler = (err, instance, info) => {
  // 向追踪服务报告错误
}
\end{codeJs}
\switchcolumn[0]*%%%%%%%
Services such as
\href{https://docs.sentry.io/platforms/javascript/guides/vue/}{Sentry}
and \href{https://docs.bugsnag.com/platforms/javascript/vue/}{Bugsnag}
also provide official integrations for Vue.
\switchcolumn
诸如
\href{https://docs.sentry.io/platforms/javascript/guides/vue/}{Sentry}
和 \href{https://docs.bugsnag.com/platforms/javascript/vue/}{Bugsnag}
等服务也为 Vue 提供了官方集成。
\end{paracol}

 