\columnratio{0.55}
\begin{paracol}{2}

\switchcolumn[0]*%%%%%%%
\section{Class and Style Bindings}
\switchcolumn
\section{Class 与 Style 绑定}
\switchcolumn[0]*%%%%%%%
A common need for data binding is manipulating an element's class list
and inline styles. Since \texttt{class} and \texttt{style} are both
attributes, we can use \texttt{v-bind} to assign them a string value
dynamically, much like with other attributes. However, trying to
generate those values using string concatenation can be annoying and
error-prone. For this reason, Vue provides special enhancements when
\texttt{v-bind} is used with \texttt{class} and \texttt{style}. In
addition to strings, the expressions can also evaluate to objects or
arrays.
\switchcolumn
数据绑定的一个常见需求场景是操纵元素的 CSS class 列表和内联样式。因为
\texttt{class} 和 \texttt{style} 都是 attribute,我们可以和其他
attribute 一样使用 \texttt{v-bind}
将它们和动态的字符串绑定。但是,在处理比较复杂的绑定时,通过拼接生成字符串是麻烦且易出错的。因此,Vue
专门为 \texttt{class} 和 \texttt{style} 的 \texttt{v-bind}
用法提供了特殊的功能增强。除了字符串外,表达式的值也可以是对象或数组。
\switchcolumn[0]*%%%%%%%
\subsection{Binding HTML Classes}
\switchcolumn
\subsection{绑定 HTML class}
\switchcolumn[0]*%%%%%%%
\subsubsection{Binding to Objects}
\switchcolumn
\subsubsection{绑定对象}
\switchcolumn[0]*%%%%%%%
We can pass an object to \texttt{:class} (short for
\texttt{v-bind:class}) to dynamically toggle classes:
\switchcolumn
我们可以给 \texttt{:class} (\texttt{v-bind:class} 的缩写)
传递一个对象来动态切换 class:
\switchcolumn[0]*%%%%%%%
\begin{codeHtml}
<div :class="{ active: isActive }"></div>
\end{codeHtml}
\switchcolumn
\begin{codeHtml}
<div :class="{ active: isActive }"></div>
\end{codeHtml}

\switchcolumn[0]*%%%%%%%
The above syntax means the presence of the \texttt{active} class will be
determined by the
\href{https://developer.mozilla.org/en-US/docs/Glossary/Truthy}{truthiness}
of the data property \texttt{isActive}.
\switchcolumn
上面的语法表示 \texttt{active} 是否存在取决于数据属性 \texttt{isActive}
的\href{https://developer.mozilla.org/en-US/docs/Glossary/Truthy}{真假值}。
\switchcolumn[0]*%%%%%%%
You can have multiple classes toggled by having more fields in the
object. In addition, the \texttt{:class} directive can also co-exist
with the plain \texttt{class} attribute. So given the following state:
\switchcolumn
你可以在对象中写多个字段来操作多个 class。此外,\texttt{:class}
指令也可以和一般的 \texttt{class} attribute
共存。举例来说,下面这样的状态:
\switchcolumn[0]*%%%%%%%
\begin{codeJs}
const isActive = ref(true)
const hasError = ref(false)
\end{codeJs}
\switchcolumn
\begin{codeJs}
const isActive = ref(true)
const hasError = ref(false)
\end{codeJs}
\switchcolumn[0]*%%%%%%%
And the following template:
\switchcolumn
配合以下模板:
\switchcolumn[0]*%%%%%%%
\begin{codeHtml}
<div
  class="static"
  :class="{ active: isActive, 'text-danger': hasError }"
></div>
\end{codeHtml}
\switchcolumn
\begin{codeHtml}
<div
  class="static"
  :class="{ active: isActive, 'text-danger': hasError }"
></div>
\end{codeHtml}
\switchcolumn[0]*%%%%%%%
It will render:
\switchcolumn
渲染的结果会是:
\switchcolumn[0]*%%%%%%%
\begin{codeHtml}
<div class="static active"></div>
\end{codeHtml}
\switchcolumn
\begin{codeHtml}
<div class="static active"></div>
\end{codeHtml}
\switchcolumn[0]*%%%%%%%
When \texttt{isActive} or \texttt{hasError} changes, the class list will
be updated accordingly. For example, if \texttt{hasError} becomes
\texttt{true}, the class list will become
\texttt{"static\ active\ text-danger"}.
\switchcolumn
当 \texttt{isActive} 或者 \texttt{hasError} 改变时,class
列表会随之更新。举例来说,如果 \texttt{hasError} 变为
\texttt{true},class 列表也会变成
\texttt{"static\ active\ text-danger"}。
\switchcolumn[0]*%%%%%%%
The bound object doesn't have to be inline:
\switchcolumn
绑定的对象并不一定需要写成内联字面量的形式,也可以直接绑定一个对象:

\switchcolumn[0]*%%%%%%%
\begin{codeJs}
const classObject = reactive({
  active: true,
  'text-danger': false
})
\end{codeJs}
\switchcolumn
\begin{codeJs}
const classObject = reactive({
  active: true,
  'text-danger': false
})
\end{codeJs}
\switchcolumn[0]*%%%%%%%
\begin{codeHtml}
<div :class="classObject"></div>
\end{codeHtml}
\switchcolumn
\begin{codeHtml}
<div :class="classObject"></div>
\end{codeHtml}
\switchcolumn[0]*%%%%%%%
This will render:
\switchcolumn
这将渲染:
\switchcolumn[0]*%%%%%%%
\begin{codeHtml}
<div class="active"></div>
\end{codeHtml}
\switchcolumn
\begin{codeHtml}
<div class="active"></div>
\end{codeHtml}
\switchcolumn[0]*%%%%%%%
We can also bind to a
\href{https://vuejs.org/guide/essentials/computed.html}{computed
property} that returns an object. This is a common and powerful pattern:
\switchcolumn
我们也可以绑定一个返回对象的\href{https://cn.vuejs.org/guide/essentials/computed.html}{计算属性}。这是一个常见且很有用的技巧:
\switchcolumn[0]*%%%%%%%
\begin{codeJs}
const isActive = ref(true)
const error = ref(null)

const classObject = computed(() => ({
    active: isActive.value && !error.value,
    'text-danger': error.value && error.value.type === 'fatal'
}))
\end{codeJs}
\switchcolumn
\begin{codeJs}
const isActive = ref(true)
const error = ref(null)

const classObject = computed(() => ({
    active: isActive.value && !error.value,
    'text-danger': error.value && error.value.type === 'fatal'
}))
\end{codeJs}

\switchcolumn[0]*%%%%%%%
\begin{codeHtml}
<div :class="classObject"></div>
\end{codeHtml}
\switchcolumn
\begin{codeHtml}
<div :class="classObject"></div>
\end{codeHtml}
\end{paracol}
 
\columnratio{0.55}
\begin{paracol}{2}
\switchcolumn[0]*%%%%%%%
\subsubsection{Binding to Arrays}
\switchcolumn
\subsubsection{绑定数组}
\switchcolumn[0]*%%%%%%%
We can bind \texttt{:class} to an array to apply a list of classes:
\switchcolumn
我们可以给 \texttt{:class} 绑定一个数组来渲染多个 CSS class:
\switchcolumn[0]*%%%%%%%
\begin{codeJs}
const activeClass = ref('active')
const errorClass = ref('text-danger')
\end{codeJs}
\switchcolumn
\begin{codeJs}
const activeClass = ref('active')
const errorClass = ref('text-danger')
\end{codeJs}

\switchcolumn[0]*%%%%%%%
\begin{codeHtml}
<div :class="[activeClass, errorClass]"></div>
\end{codeHtml}
\switchcolumn
\begin{codeHtml}
<div :class="[activeClass, errorClass]"></div>
\end{codeHtml}
\switchcolumn[0]*%%%%%%%
Which will render:
\switchcolumn
渲染的结果是:
\switchcolumn[0]*%%%%%%%
\begin{codeHtml}
<div class="active text-danger"></div>
\end{codeHtml}
\switchcolumn
\begin{codeHtml}
<div class="active text-danger"></div>
\end{codeHtml}
\switchcolumn[0]*%%%%%%%
If you would like to also toggle a class in the list conditionally, you
can do it with a ternary expression:
\switchcolumn
如果你也想在数组中有条件地渲染某个 class,你可以使用三元表达式:
\switchcolumn[0]*%%%%%%%
\begin{codeHtml}
<div :class="[isActive ? activeClass : '', errorClass]"></div>
\end{codeHtml}
\switchcolumn
\begin{codeHtml}
<div :class="[isActive ? activeClass : '', errorClass]"></div>
\end{codeHtml}
\switchcolumn[0]*%%%%%%%
This will always apply \texttt{errorClass}, but \texttt{activeClass}
will only be applied when \texttt{isActive} is truthy.
\switchcolumn
\texttt{errorClass} 会一直存在,但 \texttt{activeClass} 只会在
\texttt{isActive} 为真时才存在。
\switchcolumn[0]*%%%%%%%
However, this can be a bit verbose if you have multiple conditional
classes. That's why it's also possible to use the object syntax inside
the array syntax:
\switchcolumn
然而,这可能在有多个依赖条件的 class
时会有些冗长。因此也可以在数组中嵌套对象:
\switchcolumn[0]*%%%%%%%
\begin{codeHtml}
<div :class="[{ active: isActive }, errorClass]"></div>
\end{codeHtml}
\switchcolumn
\begin{codeHtml}
<div :class="[{ active: isActive }, errorClass]"></div>
\end{codeHtml}
\switchcolumn[0]*%%%%%%%
\subsubsection{With Components}
\switchcolumn
\subsubsection{在组件上使用}
\switchcolumn[0]*%%%%%%%
\begin{quote}
This section assumes knowledge of
\href{https://vuejs.org/guide/essentials/component-basics.html}{Components}.
Feel free to skip it and come back later.
\end{quote}
\switchcolumn
\begin{quote}
本节假设你已经有
\href{https://cn.vuejs.org/guide/essentials/component-basics.html}{Vue
组件}的知识基础。如果没有,你也可以暂时跳过,以后再阅读。
\end{quote}
\switchcolumn[0]*%%%%%%%
When you use the \texttt{class} attribute on a component with a single
root element, those classes will be added to the component's root
element and merged with any existing class already on it.
\switchcolumn
对于只有一个根元素的组件,当你使用了 \texttt{class} attribute 时,这些
class 会被添加到根元素上并与该元素上已有的 class 合并。
\switchcolumn[0]*%%%%%%%
For example, if we have a component named \texttt{MyComponent} with the
following template:
\switchcolumn
举例来说,如果你声明了一个组件名叫 \texttt{MyComponent},模板如下:
\switchcolumn[0]*%%%%%%%
\begin{codeHtml}
<!-- child component template -->
<p class="foo bar">Hi!</p>
\end{codeHtml}
\switchcolumn
\begin{codeHtml}
<!-- 子组件模板 -->
<p class="foo bar">Hi!</p>
\end{codeHtml}
\switchcolumn[0]*%%%%%%%
Then add some classes when using it:
\switchcolumn
在使用时添加一些 class:
\switchcolumn[0]*%%%%%%%
\begin{codeHtml}
<!-- when using the component -->
<MyComponent class="baz boo" />
\end{codeHtml}
\switchcolumn
\begin{codeHtml}
<!-- 在使用组件时 -->
<MyComponent class="baz boo" />
\end{codeHtml}
\switchcolumn[0]*%%%%%%%
The rendered HTML will be:
\switchcolumn
渲染出的 HTML 为:

\switchcolumn[0]*%%%%%%%
\begin{codeHtml}
<p class="foo bar baz boo">Hi!</p>
\end{codeHtml}
\switchcolumn
\begin{codeHtml}
<p class="foo bar baz boo">Hi!</p>
\end{codeHtml}
\switchcolumn[0]*%%%%%%%
The same is true for class bindings:
\switchcolumn
Class 的绑定也是同样的:
\switchcolumn[0]*%%%%%%%
\begin{codeHtml}
<MyComponent :class="{ active: isActive }" />
\end{codeHtml}
\switchcolumn
\begin{codeHtml}
<MyComponent :class="{ active: isActive }" />
\end{codeHtml}
\switchcolumn[0]*%%%%%%%
When \texttt{isActive} is truthy, the rendered HTML will be:
\switchcolumn
当 \texttt{isActive} 为真时,被渲染的 HTML 会是:
\switchcolumn[0]*%%%%%%%
\begin{codeHtml}
<p class="foo bar active">Hi!</p>
\end{codeHtml}
\switchcolumn
\begin{codeHtml}
<p class="foo bar active">Hi!</p>
\end{codeHtml}
\switchcolumn[0]*%%%%%%%
If your component has multiple root elements, you would need to define
which element will receive this class. You can do this using the
\texttt{\$attrs} component property:
\switchcolumn
如果你的组件有多个根元素,你将需要指定哪个根元素来接收这个
class。你可以通过组件的 \texttt{\$attrs} 属性来实现指定:
\switchcolumn[0]*%%%%%%%
\begin{codeHtml}
<!-- MyComponent template using $attrs -->
<p :class="$attrs.class">Hi!</p>
<span>This is a child component</span>
\end{codeHtml}
\switchcolumn
\begin{codeHtml}
<!-- MyComponent 模板使用 $attrs 时 -->
<p :class="$attrs.class">Hi!</p>
<span>This is a child component</span>
\end{codeHtml}
\switchcolumn[0]*%%%%%%%
\begin{codeHtml}
<MyComponent class="baz" />
\end{codeHtml}
\switchcolumn
\begin{codeHtml}
<MyComponent class="baz" />
\end{codeHtml}
\switchcolumn[0]*%%%%%%%
Will render:
\switchcolumn
这将被渲染为:
\switchcolumn[0]*%%%%%%%
\begin{codeHtml}
<p class="baz">Hi!</p>
<span>This is a child component</span>
\end{codeHtml}
\switchcolumn
\begin{codeHtml}
<p class="baz">Hi!</p>
<span>This is a child component</span>
\end{codeHtml}
\switchcolumn[0]*%%%%%%%
You can learn more about component attribute inheritance in
\href{https://vuejs.org/guide/components/attrs.html}{Fallthrough
Attributes} section.
\switchcolumn
你可以在\href{https://cn.vuejs.org/guide/components/attrs.html}{透传
Attribute} 一章中了解更多组件的 attribute 继承的细节。
\end{paracol}

\columnratio{0.55}
\begin{paracol}{2}

\switchcolumn[0]*%%%%%%%
\subsection{Binding Inline Styles}
\switchcolumn
\subsection{绑定内联样式}
\switchcolumn[0]*%%%%%%%
\subsubsection{Binding to Objects}
\switchcolumn
\subsubsection{绑定对象}
\switchcolumn[0]*%%%%%%%
\texttt{:style} supports binding to JavaScript object values - it
corresponds to an
\href{https://developer.mozilla.org/en-US/docs/Web/API/HTMLElement/style}{HTML
element's \texttt{style} property}:
\switchcolumn
\texttt{:style} 支持绑定 JavaScript 对象值,对应的是
\href{https://developer.mozilla.org/en-US/docs/Web/API/HTMLElement/style}{HTML
元素的 \texttt{style} 属性}:
\switchcolumn[0]*%%%%%%%
\begin{codeJs}
const activeColor = ref('red')
const fontSize = ref(30)
\end{codeJs}
\switchcolumn
\begin{codeJs}
const activeColor = ref('red')
const fontSize = ref(30)
\end{codeJs}
\switchcolumn[0]*%%%%%%%
\begin{codeHtml}
<div :style="{ color: activeColor, fontSize: fontSize + 'px' }"></div>
\end{codeHtml}
\switchcolumn
\begin{codeHtml}
<div :style="{ color: activeColor, fontSize: fontSize + 'px' }"></div>
\end{codeHtml}
\switchcolumn[0]*%%%%%%%
Although camelCase keys are recommended, \texttt{:style} also supports
kebab-cased CSS property keys (corresponds to how they are used in
actual CSS) - for example:
\switchcolumn
尽管推荐使用 camelCase,但 \texttt{:style} 也支持 kebab-cased 形式的 CSS
属性 key (对应其 CSS 中的实际名称),例如:
\switchcolumn[0]*%%%%%%%
\begin{codeHtml}
<div :style="{ 'font-size': fontSize + 'px' }"></div>
\end{codeHtml}
\switchcolumn
\begin{codeHtml}
<div :style="{ 'font-size': fontSize + 'px' }"></div>
\end{codeHtml}
\switchcolumn[0]*%%%%%%%
It is often a good idea to bind to a style object directly so that the
template is cleaner:
\switchcolumn
直接绑定一个样式对象通常是一个好主意,这样可以使模板更加简洁:
\switchcolumn[0]*%%%%%%%
\begin{codeJs}
const styleObject = reactive({
    color: 'red',
    fontSize: '13px'
})
\end{codeJs}
\switchcolumn
\begin{codeJs}
const styleObject = reactive({
    color: 'red',
    fontSize: '13px'
})
\end{codeJs}
\switchcolumn[0]*%%%%%%%
\begin{codeHtml}
<div :style="styleObject"></div>
\end{codeHtml}
\switchcolumn
\begin{codeHtml}
<div :style="styleObject"></div>
\end{codeHtml}
\switchcolumn[0]*%%%%%%%
Again, object style binding is often used in conjunction with computed
properties that return objects.
\switchcolumn
同样的,如果样式对象需要更复杂的逻辑,也可以使用返回样式对象的计算属性。


\switchcolumn[0]*%%%%%%%
\subsubsection{Binding to Arrays}
\switchcolumn
\subsubsection{绑定数组}
\switchcolumn[0]*%%%%%%%
We can bind \texttt{:style} to an array of multiple style objects. These
objects will be merged and applied to the same element:
\switchcolumn
我们还可以给 \texttt{:style}
绑定一个包含多个样式对象的数组。这些对象会被合并后渲染到同一元素上:
\switchcolumn[0]*%%%%%%%
\begin{codeHtml}
<div :style="[baseStyles, overridingStyles]"></div>
\end{codeHtml}
\switchcolumn
\begin{codeHtml}
<div :style="[baseStyles, overridingStyles]"></div>
\end{codeHtml}
\switchcolumn[0]*%%%%%%%
\subsubsection{Auto-prefixing}
\switchcolumn
\subsubsection{自动前缀}
\switchcolumn[0]*%%%%%%%
When you use a CSS property that requires a
\href{https://developer.mozilla.org/en-US/docs/Glossary/Vendor_Prefix}{vendor
prefix} in \texttt{:style}, Vue will automatically add the appropriate
prefix. Vue does this by checking at runtime to see which style
properties are supported in the current browser. If the browser doesn't
support a particular property then various prefixed variants will be
tested to try to find one that is supported.
\switchcolumn
当你在 \texttt{:style}
中使用了需要\href{https://developer.mozilla.org/en-US/docs/Glossary/Vendor_Prefix}{浏览器特殊前缀}的
CSS 属性时,Vue 会自动为他们加上相应的前缀。Vue
是在运行时检查该属性是否支持在当前浏览器中使用。如果浏览器不支持某个属性,那么将尝试加上各个浏览器特殊前缀,以找到哪一个是被支持的。
\switchcolumn[0]*%%%%%%%
\subsubsection{Multiple Values}
\switchcolumn
\subsubsection{样式多值}
\switchcolumn[0]*%%%%%%%
You can provide an array of multiple (prefixed) values to a style
property, for example:
\switchcolumn
你可以对一个样式属性提供多个 (不同前缀的) 值,举例来说:
\switchcolumn[0]*%%%%%%%
\begin{codeHtml}
<div :style="{ display: ['-webkit-box', '-ms-flexbox', 'flex'] }"></div>
\end{codeHtml}
\switchcolumn
\begin{codeHtml}
<div :style="{ display: ['-webkit-box', '-ms-flexbox', 'flex'] }"></div>
\end{codeHtml}
\switchcolumn[0]*%%%%%%%
This will only render the last value in the array which the browser
supports. In this example, it will render \texttt{display:\ flex} for
browsers that support the unprefixed version of flexbox.
\switchcolumn
数组仅会渲染浏览器支持的最后一个值。在这个示例中,在支持不需要特别前缀的浏览器中都会渲染为
\texttt{display:\ flex}。
\end{paracol}

