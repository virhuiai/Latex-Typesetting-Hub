
\columnratio{0.55}
\begin{paracol}{2} 
\switchcolumn[0]*%%%%%%%
\section{Accessibility}
\switchcolumn
\section{无障碍访问}  
\switchcolumn[0]*%%%%%%%
Web accessibility (also known as a11y) refers to the practice of
creating websites that can be used by anyone --- be that a person with a
disability, a slow connection, outdated or broken hardware or simply
someone in an unfavorable environment. For example, adding subtitles to
a video would help both your deaf and hard-of-hearing users and your
users who are in a loud environment and can't hear their phone.
Similarly, making sure your text isn't too low contrast will help both
your low-vision users and your users who are trying to use their phone
in bright sunlight.
\switchcolumn
Web 无障碍访问 (也称为 a11y)
是指创建可供任何人使用的网站的做法------无论是身患某种障碍、通过慢速的网络连接访问、使用老旧或损坏的硬件,还是仅处于某种不方便的环境。例如,在视频中添加字幕可以帮助失聪、有听力障碍或身处嘈杂环境而听不到手机的用户。同样地,确保文字样式没有处于太低的对比度,可以对低视力用户和在明亮的强光下使用手机的用户都有所帮助。
\switchcolumn[0]*%%%%%%%
Ready to start but aren't sure where?
\switchcolumn
你是否已经准备开始却又无从下手?
\switchcolumn[0]*%%%%%%%
Checkout the
\href{https://www.w3.org/WAI/planning-and-managing/}{Planning and
managing web accessibility guide} provided by
\href{https://www.w3.org/}{World Wide Web Consortium (W3C)}
\switchcolumn
请先阅读由\href{https://www.w3.org/}{万维网联盟 (W3C)} 提供的
\href{https://www.w3.org/WAI/planning-and-managing/}{Web
无障碍访问的规划和管理}。
\end{paracol}



\columnratio{0.55}
\begin{paracol}{2} 
 
\switchcolumn[0]*%%%%%%%
\subsection{Skip link}
\switchcolumn
\subsection{跳过链接}
\switchcolumn[0]*%%%%%%%
You should add a link at the top of each page that goes directly to the
main content area so users can skip content that is repeated on multiple
Web pages.
\switchcolumn
你应该在每个页面的顶部添加一个直接指向主内容区域的链接,这样用户就可以跳过在多个网页上重复的内容。
\switchcolumn[0]*%%%%%%%
Typically this is done on the top of \texttt{App.vue} as it will be the
first focusable element on all your pages:
\switchcolumn
通常这个链接会放在 \texttt{App.vue}
的顶部,这样它就会是所有页面上的第一个可聚焦元素:
\switchcolumn[0]*%%%%%%%
\begin{codeHtml}
<ul class="skip-links">
  <li>
    <a href="#main" ref="skipLink" class="skip-link">Skip to main content</a>
  </li>
</ul>
\end{codeHtml}
\switchcolumn
\begin{codeHtml}
<ul class="skip-links">
  <li>
    <a href="#main" ref="skipLink" class="skip-link">Skip to main content</a>
  </li>
</ul>
\end{codeHtml}
\switchcolumn[0]*%%%%%%%
To hide the link unless it is focused, you can add the following style:
\switchcolumn
若想在非聚焦状态下隐藏该链接,可以添加以下样式:
\switchcolumn[0]*%%%%%%%
\begin{codeCss}
.skip-link {
  white-space: nowrap;
  margin: 1em auto;
  top: 0;
  position: fixed;
  left: 50%;
  margin-left: -72px;
  opacity: 0;
}
.skip-link:focus {
  opacity: 1;
  background-color: white;
  padding: 0.5em;
  border: 1px solid black;
}
\end{codeCss}
\switchcolumn
\begin{codeCss}
.skip-link {
  white-space: nowrap;
  margin: 1em auto;
  top: 0;
  position: fixed;
  left: 50%;
  margin-left: -72px;
  opacity: 0;
}
.skip-link:focus {
  opacity: 1;
  background-color: white;
  padding: 0.5em;
  border: 1px solid black;
}
\end{codeCss}
\switchcolumn[0]*%%%%%%%
Once a user changes route, bring focus back to the skip link. This can
be achieved by calling focus on the skip link's template ref (assuming
usage of \texttt{vue-router}):
\switchcolumn
一旦用户改变路由,请将焦点放回到这个``跳过''链接。通过如下方式聚焦``跳过''链接的模板引用
(假设使用了 \texttt{vue-router}) 即可实现:
\switchcolumn[0]*%%%%%%%
\begin{codeHtml}
<script setup>
import { ref, watch } from 'vue'
import { useRoute } from 'vue-router'
const route = useRoute()
const skipLink = ref()
watch(
  () => route.path,
  () => {
    skipLink.value.focus()
  }
)
</script>
\end{codeHtml}
\switchcolumn
\begin{codeHtml}
<script setup>
import { ref, watch } from 'vue'
import { useRoute } from 'vue-router'
const route = useRoute()
const skipLink = ref()
watch(
  () => route.path,
  () => {
    skipLink.value.focus()
  }
)
</script>
\end{codeHtml}
\switchcolumn[0]*%%%%%%%
\href{https://www.w3.org/WAI/WCAG21/Techniques/general/G1.html}{Read
documentation on skip link to main content}
\switchcolumn
\href{https://www.w3.org/WAI/WCAG21/Techniques/general/G1.html}{阅读关于跳过链接到主要内容的文档}
\end{paracol}



\columnratio{0.55}
\begin{paracol}{2} 
 
\switchcolumn[0]*%%%%%%%
\subsection{Content Structure}
\switchcolumn
\subsection{内容结构}
\switchcolumn[0]*%%%%%%%
One of the most important pieces of accessibility is making sure that
design can support accessible implementation. Design should consider not
only color contrast, font selection, text sizing, and language, but also
how the content is structured in the application.
\switchcolumn
确保设计可以支持易于访问的实现是无障碍访问最重要的部分之一。设计不仅要考虑颜色对比度、字体选择、文本大小和语言,还要考虑应用中的内容是如何组织的。
\switchcolumn[0]*%%%%%%%
\subsubsection{Headings}
\switchcolumn
\subsubsection{标题}
\switchcolumn[0]*%%%%%%%
Users can navigate an application through headings. Having descriptive
headings for every section of your application makes it easier for users
to predict the content of each section. When it comes to headings, there
are a couple of recommended accessibility practices:
\switchcolumn
用户可以通过标题在应用中进行导航。为应用的每个部分设置描述性标题,这可以让用户更容易地预测每个部分的内容。说到标题,有几个推荐的无障碍访问实践:
\switchcolumn[0]*%%%%%%%
\begin{itemize}
\item
  Nest headings in their ranking order:
  \texttt{\textless{}h1\textgreater{}} -
  \texttt{\textless{}h6\textgreater{}}
\item
  Don't skip headings within a section
\item
  Use actual heading tags instead of styling text to give the visual
  appearance of headings
\end{itemize}
\switchcolumn
\begin{itemize}
\item
  按级别顺序嵌套标题:\texttt{\textless{}h1\textgreater{}} -
  \texttt{\textless{}h6\textgreater{}}
\item
  不要在一个章节内跳跃标题的级别
\item
  使用实际的标题标记,而不是通过对文本设置样式以提供视觉上的标题
\end{itemize}
\switchcolumn[0]*%%%%%%%
\href{https://www.w3.org/TR/UNDERSTANDING-WCAG20/navigation-mechanisms-descriptive.html}{Read
more about headings}
\switchcolumn
\href{https://www.w3.org/TR/UNDERSTANDING-WCAG20/navigation-mechanisms-descriptive.html}{阅读更多有关标题的信息}
\switchcolumn[0]*%%%%%%%
\begin{codeHtml}
<main role="main" aria-labelledby="main-title">
  <h1 id="main-title">Main title</h1>
  <section aria-labelledby="section-title-1">
    <h2 id="section-title-1"> Section Title </h2>
    <h3>Section Subtitle</h3>
    <!-- 内容 -->
  </section>
  <section aria-labelledby="section-title-2">
    <h2 id="section-title-2"> Section Title </h2>
    <h3>Section Subtitle</h3>
    <!-- 内容 -->
    <h3>Section Subtitle</h3>
    <!-- 内容 -->
  </section>
</main>
\end{codeHtml}
\switchcolumn
\begin{codeHtml}
<main role="main" aria-labelledby="main-title">
  <h1 id="main-title">Main title</h1>
  <section aria-labelledby="section-title-1">
    <h2 id="section-title-1"> Section Title </h2>
    <h3>Section Subtitle</h3>
    <!-- 内容 -->
  </section>
  <section aria-labelledby="section-title-2">
    <h2 id="section-title-2"> Section Title </h2>
    <h3>Section Subtitle</h3>
    <!-- 内容 -->
    <h3>Section Subtitle</h3>
    <!-- 内容 -->
  </section>
</main>
\end{codeHtml}
\switchcolumn[0]*%%%%%%%
\subsubsection{Landmarks}
\switchcolumn
\subsubsection{Landmarks}
\switchcolumn[0]*%%%%%%%
\href{https://developer.mozilla.org/en-US/docs/Web/Accessibility/ARIA/Roles/landmark_role}{Landmarks}
provide programmatic access to sections within an application. Users who
rely on assistive technology can navigate to each section of the
application and skip over content. You can use
\href{https://developer.mozilla.org/en-US/docs/Web/Accessibility/ARIA/Roles}{ARIA
roles} to help you achieve this.
\switchcolumn
\href{https://developer.mozilla.org/en-US/docs/Web/Accessibility/ARIA/Roles/landmark_role}{Landmark}
会为应用中的章节提供访问规划。依赖辅助技术的用户可以跳过内容直接导航到应用的每个部分。你可以使用
\href{https://developer.mozilla.org/en-US/docs/Web/Accessibility/ARIA/Roles}{ARIA
role} 帮助你实现这个目标。
\end{paracol}



\columnratio{0.55}
\begin{paracol}{2} 

\end{paracol}


\columnratio{0.55}
\begin{paracol}{2} 

\end{paracol}



\columnratio{0.55}
\begin{paracol}{2} 

\end{paracol}



\columnratio{0.55}
\begin{paracol}{2} 

\end{paracol}


\columnratio{0.55}
\begin{paracol}{2} 

\end{paracol}



\columnratio{0.55}
\begin{paracol}{2} 

\end{paracol}



\columnratio{0.55}
\begin{paracol}{2} 

\end{paracol}



\columnratio{0.55}
\begin{paracol}{2} 

\end{paracol}


\columnratio{0.55}
\begin{paracol}{2} 

\end{paracol}



\columnratio{0.55}
\begin{paracol}{2} 

\end{paracol}



\columnratio{0.55}
\begin{paracol}{2} 

\end{paracol}


\columnratio{0.55}
\begin{paracol}{2} 

\end{paracol}



\columnratio{0.55}
\begin{paracol}{2} 

\end{paracol}


\columnratio{0.55}
\begin{paracol}{2} 

\end{paracol}



\columnratio{0.55}
\begin{paracol}{2} 

\end{paracol}


\columnratio{0.55}
\begin{paracol}{2} 

\end{paracol}



\columnratio{0.55}
\begin{paracol}{2} 

\end{paracol}



\columnratio{0.55}
\begin{paracol}{2} 

\end{paracol}



\columnratio{0.55}
\begin{paracol}{2} 

\end{paracol}


\columnratio{0.55}
\begin{paracol}{2} 

\end{paracol}



\columnratio{0.55}
\begin{paracol}{2} 

\end{paracol}



\columnratio{0.55}
\begin{paracol}{2} 

\end{paracol}


\columnratio{0.55}
\begin{paracol}{2} 

\end{paracol}



\columnratio{0.55}
\begin{paracol}{2} 

\end{paracol}