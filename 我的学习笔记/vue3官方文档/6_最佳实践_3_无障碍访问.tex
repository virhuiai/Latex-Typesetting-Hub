
\columnratio{0.55}
\begin{paracol}{2} 
\switchcolumn[0]*%%%%%%%
\section{Accessibility}
\switchcolumn
\section{无障碍访问}  
\switchcolumn[0]*%%%%%%%
Web accessibility (also known as a11y) refers to the practice of
creating websites that can be used by anyone --- be that a person with a
disability, a slow connection, outdated or broken hardware or simply
someone in an unfavorable environment. For example, adding subtitles to
a video would help both your deaf and hard-of-hearing users and your
users who are in a loud environment and can't hear their phone.
Similarly, making sure your text isn't too low contrast will help both
your low-vision users and your users who are trying to use their phone
in bright sunlight.
\switchcolumn
Web 无障碍访问 (也称为 a11y)
是指创建可供任何人使用的网站的做法------无论是身患某种障碍、通过慢速的网络连接访问、使用老旧或损坏的硬件,还是仅处于某种不方便的环境。例如,在视频中添加字幕可以帮助失聪、有听力障碍或身处嘈杂环境而听不到手机的用户。同样地,确保文字样式没有处于太低的对比度,可以对低视力用户和在明亮的强光下使用手机的用户都有所帮助。
\switchcolumn[0]*%%%%%%%
Ready to start but aren't sure where?
\switchcolumn
你是否已经准备开始却又无从下手?
\switchcolumn[0]*%%%%%%%
Checkout the
\href{https://www.w3.org/WAI/planning-and-managing/}{Planning and
managing web accessibility guide} provided by
\href{https://www.w3.org/}{World Wide Web Consortium (W3C)}
\switchcolumn
请先阅读由\href{https://www.w3.org/}{万维网联盟 (W3C)} 提供的
\href{https://www.w3.org/WAI/planning-and-managing/}{Web
无障碍访问的规划和管理}。
\end{paracol}



\columnratio{0.55}
\begin{paracol}{2} 
 
\switchcolumn[0]*%%%%%%%
\subsection{Skip link}
\switchcolumn
\subsection{跳过链接}
\switchcolumn[0]*%%%%%%%
You should add a link at the top of each page that goes directly to the
main content area so users can skip content that is repeated on multiple
Web pages.
\switchcolumn
你应该在每个页面的顶部添加一个直接指向主内容区域的链接,这样用户就可以跳过在多个网页上重复的内容。
\switchcolumn[0]*%%%%%%%
Typically this is done on the top of \texttt{App.vue} as it will be the
first focusable element on all your pages:
\switchcolumn
通常这个链接会放在 \texttt{App.vue}
的顶部,这样它就会是所有页面上的第一个可聚焦元素:
\switchcolumn[0]*%%%%%%%
\begin{codeHtml}
<ul class="skip-links">
  <li>
    <a href="#main" ref="skipLink" class="skip-link">Skip to main content</a>
  </li>
</ul>
\end{codeHtml}
\switchcolumn
\begin{codeHtml}
<ul class="skip-links">
  <li>
    <a href="#main" ref="skipLink" class="skip-link">Skip to main content</a>
  </li>
</ul>
\end{codeHtml}
\switchcolumn[0]*%%%%%%%
To hide the link unless it is focused, you can add the following style:
\switchcolumn
若想在非聚焦状态下隐藏该链接,可以添加以下样式:
\switchcolumn[0]*%%%%%%%
\begin{codeCss}
.skip-link {
  white-space: nowrap;
  margin: 1em auto;
  top: 0;
  position: fixed;
  left: 50%;
  margin-left: -72px;
  opacity: 0;
}
.skip-link:focus {
  opacity: 1;
  background-color: white;
  padding: 0.5em;
  border: 1px solid black;
}
\end{codeCss}
\switchcolumn
\begin{codeCss}
.skip-link {
  white-space: nowrap;
  margin: 1em auto;
  top: 0;
  position: fixed;
  left: 50%;
  margin-left: -72px;
  opacity: 0;
}
.skip-link:focus {
  opacity: 1;
  background-color: white;
  padding: 0.5em;
  border: 1px solid black;
}
\end{codeCss}
\switchcolumn[0]*%%%%%%%
Once a user changes route, bring focus back to the skip link. This can
be achieved by calling focus on the skip link's template ref (assuming
usage of \texttt{vue-router}):
\switchcolumn
一旦用户改变路由,请将焦点放回到这个``跳过''链接。通过如下方式聚焦``跳过''链接的模板引用
(假设使用了 \texttt{vue-router}) 即可实现:
\switchcolumn[0]*%%%%%%%
\begin{codeHtml}
<script setup>
import { ref, watch } from 'vue'
import { useRoute } from 'vue-router'
const route = useRoute()
const skipLink = ref()
watch(
  () => route.path,
  () => {
    skipLink.value.focus()
  }
)
</script>
\end{codeHtml}
\switchcolumn
\begin{codeHtml}
<script setup>
import { ref, watch } from 'vue'
import { useRoute } from 'vue-router'
const route = useRoute()
const skipLink = ref()
watch(
  () => route.path,
  () => {
    skipLink.value.focus()
  }
)
</script>
\end{codeHtml}
\switchcolumn[0]*%%%%%%%
\href{https://www.w3.org/WAI/WCAG21/Techniques/general/G1.html}{Read
documentation on skip link to main content}
\switchcolumn
\href{https://www.w3.org/WAI/WCAG21/Techniques/general/G1.html}{阅读关于跳过链接到主要内容的文档}
\end{paracol}



\columnratio{0.55}
\begin{paracol}{2} 
 
\switchcolumn[0]*%%%%%%%
\subsection{Content Structure}
\switchcolumn
\subsection{内容结构}
\switchcolumn[0]*%%%%%%%
One of the most important pieces of accessibility is making sure that
design can support accessible implementation. Design should consider not
only color contrast, font selection, text sizing, and language, but also
how the content is structured in the application.
\switchcolumn
确保设计可以支持易于访问的实现是无障碍访问最重要的部分之一。设计不仅要考虑颜色对比度、字体选择、文本大小和语言,还要考虑应用中的内容是如何组织的。
\switchcolumn[0]*%%%%%%%
\subsubsection{Headings}
\switchcolumn
\subsubsection{标题}
\switchcolumn[0]*%%%%%%%
Users can navigate an application through headings. Having descriptive
headings for every section of your application makes it easier for users
to predict the content of each section. When it comes to headings, there
are a couple of recommended accessibility practices:
\switchcolumn
用户可以通过标题在应用中进行导航。为应用的每个部分设置描述性标题,这可以让用户更容易地预测每个部分的内容。说到标题,有几个推荐的无障碍访问实践:
\switchcolumn[0]*%%%%%%%
\begin{itemize}
\item
  Nest headings in their ranking order:
  \texttt{\textless{}h1\textgreater{}} -
  \texttt{\textless{}h6\textgreater{}}
\item
  Don't skip headings within a section
\item
  Use actual heading tags instead of styling text to give the visual
  appearance of headings
\end{itemize}
\switchcolumn
\begin{itemize}
\item
  按级别顺序嵌套标题:\texttt{\textless{}h1\textgreater{}} -
  \texttt{\textless{}h6\textgreater{}}
\item
  不要在一个章节内跳跃标题的级别
\item
  使用实际的标题标记,而不是通过对文本设置样式以提供视觉上的标题
\end{itemize}
\switchcolumn[0]*%%%%%%%
\href{https://www.w3.org/TR/UNDERSTANDING-WCAG20/navigation-mechanisms-descriptive.html}{Read
more about headings}
\switchcolumn
\href{https://www.w3.org/TR/UNDERSTANDING-WCAG20/navigation-mechanisms-descriptive.html}{阅读更多有关标题的信息}
\switchcolumn[0]*%%%%%%%
\begin{codeHtml}
<main role="main" aria-labelledby="main-title">
  <h1 id="main-title">Main title</h1>
  <section aria-labelledby="section-title-1">
    <h2 id="section-title-1"> Section Title </h2>
    <h3>Section Subtitle</h3>
    <!-- 内容 -->
  </section>
  <section aria-labelledby="section-title-2">
    <h2 id="section-title-2"> Section Title </h2>
    <h3>Section Subtitle</h3>
    <!-- 内容 -->
    <h3>Section Subtitle</h3>
    <!-- 内容 -->
  </section>
</main>
\end{codeHtml}
\switchcolumn
\begin{codeHtml}
<main role="main" aria-labelledby="main-title">
  <h1 id="main-title">Main title</h1>
  <section aria-labelledby="section-title-1">
    <h2 id="section-title-1"> Section Title </h2>
    <h3>Section Subtitle</h3>
    <!-- 内容 -->
  </section>
  <section aria-labelledby="section-title-2">
    <h2 id="section-title-2"> Section Title </h2>
    <h3>Section Subtitle</h3>
    <!-- 内容 -->
    <h3>Section Subtitle</h3>
    <!-- 内容 -->
  </section>
</main>
\end{codeHtml}
\switchcolumn[0]*%%%%%%%
\subsubsection{Landmarks}
\switchcolumn
\subsubsection{Landmarks}
\switchcolumn[0]*%%%%%%%
\href{https://developer.mozilla.org/en-US/docs/Web/Accessibility/ARIA/Roles/landmark_role}{Landmarks}
provide programmatic access to sections within an application. Users who
rely on assistive technology can navigate to each section of the
application and skip over content. You can use
\href{https://developer.mozilla.org/en-US/docs/Web/Accessibility/ARIA/Roles}{ARIA
roles} to help you achieve this.
\switchcolumn
\href{https://developer.mozilla.org/en-US/docs/Web/Accessibility/ARIA/Roles/landmark_role}{Landmark}
会为应用中的章节提供访问规划。依赖辅助技术的用户可以跳过内容直接导航到应用的每个部分。你可以使用
\href{https://developer.mozilla.org/en-US/docs/Web/Accessibility/ARIA/Roles}{ARIA
role} 帮助你实现这个目标。
\end{paracol}


\begingroup
\LTleft=0pt \LTright=0pt
\begin{longtable}{@{\extracolsep{\fill}}|p{0.15\textwidth}|p{0.15\textwidth}|p{0.3\textwidth}|p{0.3\textwidth}|}
\hline
HTML&ARIA Role&地标的目的&Landmark Purpose\\\hline
header       & \verb|role="banner"|        & 主标题:页面的标题& Prime heading: title of the page\\\hline
nav          & \verb|role="navigation"|    & 适合用作文档或相关文档导航的链接集合 &Collection of links suitable for use when navigating the document or related documents  \\\hline
main         & \verb|role="main"|          & 文档的主体或中心内容  &The main or central content of the document. \\\hline
footer       & \verb|role="contentinfo"|   & 关于父级文档的信息:脚注/版权/隐私声明链接&Information about the parent document: footnotes/copyrights/links to privacy statement \\\hline
aside        & \verb|role="complementary"| & 用来支持主内容,同时其自身的内容是相对独立且有意义的 &Supports the main content, yet is separated and meaningful on its own content  \\\hline
*无对应元素*      & \verb|role="search"|        & 该章节包含整个应用的搜索功能 &This section contains the search functionality for the application\\hline
form         & \verb|role="form"|          & 表单相关元素的集合 &Collection of form-associated elements                                       \\\hline
section      & \verb|role="region"|        & 相关的且用户可能会导航至此的内容。必须为该元素提供 label&Content that is relevant and that users will likely want to navigate to. Label must be provided for this element \\\hline
\end{longtable}
\endgroup


\columnratio{0.55}
\begin{paracol}{2} 
 
\switchcolumn[0]*%%%%%%%
\begin{vueQuote}{TIP}
It is recommended to use landmark HTML elements with redundant landmark
role attributes in order to maximize compatibility with legacy
\href{https://caniuse.com/\#feat=html5semantic}{browsers that don't
support HTML5 semantic elements}.
\end{vueQuote} 
\switchcolumn
\begin{vueQuote}{TIP}
建议同时使用 landmark HTML 元素和 role
属性,以最大程度地兼容\href{https://caniuse.com/\#feat=html5semantic}{不支持
HTML5 语义元素的传统浏览器}。
\end{vueQuote} 
\switchcolumn[0]*%%%%%%%
\href{https://www.w3.org/TR/wai-aria-1.2/\#landmark_roles}{Read more
about landmarks}
\switchcolumn
\href{https://www.w3.org/TR/wai-aria-1.2/\#landmark_roles}{阅读更多有关标题的细节}
\switchcolumn[0]*%%%%%%%
\subsection{Semantic Forms}
\switchcolumn
\subsection{语义化表单}
\switchcolumn[0]*%%%%%%%
When creating a form, you can use the following elements:
\texttt{\textless{}form\textgreater{}},
\texttt{\textless{}label\textgreater{}},
\texttt{\textless{}input\textgreater{}},
\texttt{\textless{}textarea\textgreater{}}, and
\texttt{\textless{}button\textgreater{}}
\switchcolumn
当创建一个表单,你可能使用到以下几个元素:\texttt{\textless{}form\textgreater{}}、\texttt{\textless{}label\textgreater{}}、\texttt{\textless{}input\textgreater{}}、\texttt{\textless{}textarea\textgreater{}}
和 \texttt{\textless{}button\textgreater{}}。
\switchcolumn[0]*%%%%%%%
Labels are typically placed on top or to the left of the form fields:
\switchcolumn
标签通常放置在表格字段的顶部或左侧:
\switchcolumn[0]*%%%%%%%
\begin{codeHtml}
<form action="/dataCollectionLocation" method="post" autocomplete="on">
  <div v-for="item in formItems" :key="item.id" class="form-item">
    <label :for="item.id">{{ item.label }}: </label>
    <input
      :type="item.type"
      :id="item.id"
      :name="item.id"
      v-model="item.value"
    />
  </div>
  <button type="submit">Submit</button>
</form>
\end{codeHtml}
\switchcolumn
\begin{codeHtml}
<form action="/dataCollectionLocation" method="post" autocomplete="on">
  <div v-for="item in formItems" :key="item.id" class="form-item">
    <label :for="item.id">{{ item.label }}: </label>
    <input
      :type="item.type"
      :id="item.id"
      :name="item.id"
      v-model="item.value"
    />
  </div>
  <button type="submit">Submit</button>
</form>
\end{codeHtml}
\switchcolumn[0]*%%%%%%%
Notice how you can include
\texttt{autocomplete=\textquotesingle{}on\textquotesingle{}} on the form
element and it will apply to all inputs in your form. You can also set
different
\href{https://developer.mozilla.org/en-US/docs/Web/HTML/Attributes/autocomplete}{values
for autocomplete attribute} for each input.
\switchcolumn
请注意这里我们是如何在表单元素中引入
\texttt{autocomplete=\textquotesingle{}on\textquotesingle{}}
的,它将应用于表单中的所有 input 框。你也可以为每个 input 框都设置不同的
\href{https://developer.mozilla.org/en-US/docs/Web/HTML/Attributes/autocomplete}{autocomplete
attribute 的值}。
\switchcolumn[0]*%%%%%%%
\subsubsection{Labels}
\switchcolumn
\subsubsection{标签}
\switchcolumn[0]*%%%%%%%
Provide labels to describe the purpose of all form control; linking
\texttt{for} and \texttt{id}:
\switchcolumn
提供标签来描述所有表单控件的用途;使 \texttt{for} 和 \texttt{id}
链接起来:
\switchcolumn[0]*%%%%%%%
\begin{codeHtml}
<label for="name">Name</label>
<input type="text" name="name" id="name" v-model="name" />
\end{codeHtml}
\switchcolumn
\begin{codeHtml}
<label for="name">Name</label>
<input type="text" name="name" id="name" v-model="name" />
\end{codeHtml}
\switchcolumn[0]*%%%%%%%
If you inspect this element in your chrome developer tools and open the
Accessibility tab inside the Elements tab, you will see how the input
gets its name from the label:
\switchcolumn
如果你在 chrome 开发工具中检查这个元素,并打开 Elements 选项卡中的
Accessibility 选项卡,你将看到输入是如何从标签中获取其名称的:
\end{paracol}

\begin{center} 
\includegraphics{./img/AccessibleLabelChromeDevTools.8d40e8fa.png} 
\end{center}
    

\columnratio{0.55}
\begin{paracol}{2} 
 
\switchcolumn[0]*%%%%%%%
\begin{vueQuoteWarn}{Warning:}
Though you might have seen labels wrapping the input fields like this:
\begin{codeHtml}
<label>
  Name:
  <input type="text" name="name" id="name" v-model="name" />
</label>
\end{codeHtml}
Explicitly setting the labels with a matching id is better supported by
assistive technology.
\end{vueQuoteWarn}
\switchcolumn
\begin{vueQuoteWarn}{警告:}
你可能还见过这样的包装 input 框的标签:
\begin{codeHtml}
<label>
  Name:
  <input type="text" name="name" id="name" v-model="name" />
</label>
\end{codeHtml}
但我们仍建议你显式地为 input 元素设置 id
相匹配的标签,以更好地实现无障碍访问。
\end{vueQuoteWarn}
\switchcolumn[0]*%%%%%%%
\paragraph{aria-label}
\switchcolumn
\paragraph{aria-label}
\switchcolumn[0]*%%%%%%%
You can also give the input an accessible name with
\href{https://developer.mozilla.org/en-US/docs/Web/Accessibility/ARIA/Attributes/aria-label}{\texttt{aria-label}}.
\switchcolumn
你也可以为 input 框配置一个带有
\href{https://developer.mozilla.org/en-US/docs/Web/Accessibility/ARIA/Attributes/aria-label}{\texttt{aria-label}}
的无障碍访问名。
\switchcolumn[0]*%%%%%%%
\begin{codeHtml}
<label for="name">Name</label>
<input
  type="text"
  name="name"
  id="name"
  v-model="name"
  :aria-label="nameLabel"
/>
\end{codeHtml}
\switchcolumn
\begin{codeHtml}
<label for="name">Name</label>
<input
  type="text"
  name="name"
  id="name"
  v-model="name"
  :aria-label="nameLabel"
/>
\end{codeHtml}
\switchcolumn[0]*%%%%%%%
Feel free to inspect this element in Chrome DevTools to see how the
accessible name has changed:
\switchcolumn
在 Chrome DevTools 中审查此元素,查看无障碍名称是如何更改的:
\end{paracol}

\begin{center} 
\includegraphics{./img/AccessibleARIAlabelDevTools.2b376a03.png} 
\end{center}
    
 
\columnratio{0.55}
\begin{paracol}{2} 
 
\switchcolumn[0]*%%%%%%%
\paragraph{aria-labelledby}
\switchcolumn
\paragraph{aria-labelledby}
\switchcolumn[0]*%%%%%%%
Using
\href{https://developer.mozilla.org/en-US/docs/Web/Accessibility/ARIA/Attributes/aria-labelledby}{\texttt{aria-labelledby}}
is similar to \texttt{aria-label} except it is used if the label text is
visible on screen. It is paired to other elements by their \texttt{id}
and you can link multiple \texttt{id}s:
\switchcolumn
使用
\href{https://developer.mozilla.org/en-US/docs/Web/Accessibility/ARIA/Attributes/aria-labelledby}{\texttt{aria-labelledby}}
类似于 \texttt{aria-label},除非标签文本在屏幕上可见。它通过 \texttt{id}
与其他元素配对,你可以链接多个 \texttt{id}:
\switchcolumn[0]*%%%%%%%
\begin{codeHtml}
<form
  class="demo"
  action="/dataCollectionLocation"
  method="post"
  autocomplete="on"
>
  <h1 id="billing">Billing</h1>
  <div class="form-item">
    <label for="name">Name:</label>
    <input
      type="text"
      name="name"
      id="name"
      v-model="name"
      aria-labelledby="billing name"
    />
  </div>
  <button type="submit">Submit</button>
</form>
\end{codeHtml}
\switchcolumn
\begin{codeHtml}
<form
  class="demo"
  action="/dataCollectionLocation"
  method="post"
  autocomplete="on"
>
  <h1 id="billing">Billing</h1>
  <div class="form-item">
    <label for="name">Name:</label>
    <input
      type="text"
      name="name"
      id="name"
      v-model="name"
      aria-labelledby="billing name"
    />
  </div>
  <button type="submit">Submit</button>
</form>
\end{codeHtml}
\end{paracol}

\begin{center} 
\includegraphics{./img/AccessibleARIAlabelledbyDevTools.56ced2ed.png} 
\end{center}
    

\columnratio{0.55}
\begin{paracol}{2} 
 
\switchcolumn[0]*%%%%%%%
\paragraph{aria-describedby}
\switchcolumn
\paragraph{aria-describedby}
\switchcolumn[0]*%%%%%%%
\href{https://developer.mozilla.org/en-US/docs/Web/Accessibility/ARIA/Attributes/aria-describedby}{aria-describedby}
is used the same way as \texttt{aria-labelledby} except provides a
description with additional information that the user might need. This
can be used to describe the criteria for any input:
\switchcolumn
\href{https://developer.mozilla.org/en-US/docs/Web/Accessibility/ARIA/Attributes/aria-describedby}{aria-describedby}
的用法与 \texttt{aria-labelledby}
相同,它提供了一条用户可能需要的附加描述信息。这可用于描述任何输入的标准:
\switchcolumn[0]*%%%%%%%
\begin{codeHtml}
<form
  class="demo"
  action="/dataCollectionLocation"
  method="post"
  autocomplete="on"
>
  <h1 id="billing">Billing</h1>
  <div class="form-item">
    <label for="name">Full Name:</label>
    <input
      type="text"
      name="name"
      id="name"
      v-model="name"
      aria-labelledby="billing name"
      aria-describedby="nameDescription"
    />
    <p id="nameDescription">Please provide first and last name.</p>
  </div>
  <button type="submit">Submit</button>
</form>
\end{codeHtml}
\switchcolumn
\begin{codeHtml}
<form
  class="demo"
  action="/dataCollectionLocation"
  method="post"
  autocomplete="on"
>
  <h1 id="billing">Billing</h1>
  <div class="form-item">
    <label for="name">Full Name:</label>
    <input
      type="text"
      name="name"
      id="name"
      v-model="name"
      aria-labelledby="billing name"
      aria-describedby="nameDescription"
    />
    <p id="nameDescription">Please provide first and last name.</p>
  </div>
  <button type="submit">Submit</button>
</form>
\end{codeHtml}
\switchcolumn[0]*%%%%%%%
You can see the description by inspecting Chrome DevTools:
\switchcolumn
你可以通过使用 Chrome 开发工具来查看说明:
\end{paracol}

\begin{center} 
\includegraphics{./img/AccessibleARIAdescribedby.659c43af.png} 
\end{center}
    

\columnratio{0.55}
\begin{paracol}{2} 
  
\switchcolumn[0]*%%%%%%%
\subsubsection{Placeholder}
\switchcolumn
\subsubsection{占位符}
\switchcolumn[0]*%%%%%%%
Avoid using placeholders as they can confuse many users.
\switchcolumn
避免使用占位符,因为它们可能会使许多用户感到困惑。
\switchcolumn[0]*%%%%%%%
One of the issues with placeholders is that they don't meet the
\href{https://www.w3.org/WAI/WCAG21/Understanding/contrast-minimum.html}{color
contrast criteria} by default; fixing the color contrast makes the
placeholder look like pre-populated data in the input fields. Looking at
the following example, you can see that the Last Name placeholder which
meets the color contrast criteria looks like pre-populated data:
\switchcolumn
占位符的缺陷之一是默认情况下它们不符合\href{https://www.w3.org/WAI/WCAG21/Understanding/contrast-minimum.html}{颜色对比度标准};应当修改其颜色,让它看起来像是预先填入
input
框中的数据一样。查看以下示例,可以看到满足颜色对比度条件的姓氏占位符看起来像预填充的数据:
\end{paracol}

\begin{center} 
\includegraphics{./img/AccessiblePlaceholder.5dfc1226.png} 
\end{center}
    

\columnratio{0.55}
\begin{paracol}{2} 
\switchcolumn[0]*%%%%%%%
\begin{codeHtml}
<form
  class="demo"
  action="/dataCollectionLocation"
  method="post"
  autocomplete="on"
>
  <div v-for="item in formItems" :key="item.id" class="form-item">
    <label :for="item.id">{{ item.label }}: </label>
    <input
      type="text"
      :id="item.id"
      :name="item.id"
      v-model="item.value"
      :placeholder="item.placeholder"
    />
  </div>
  <button type="submit">Submit</button>
</form>
\end{codeHtml}
\switchcolumn
\begin{codeHtml}
<form
  class="demo"
  action="/dataCollectionLocation"
  method="post"
  autocomplete="on"
>
  <div v-for="item in formItems" :key="item.id" class="form-item">
    <label :for="item.id">{{ item.label }}: </label>
    <input
      type="text"
      :id="item.id"
      :name="item.id"
      v-model="item.value"
      :placeholder="item.placeholder"
    />
  </div>
  <button type="submit">Submit</button>
</form>
\end{codeHtml}
\switchcolumn[0]*%%%%%%%
\begin{codeCss}
    /* https://www.w3schools.com/howto/howto_css_placeholder.asp */
    #lastName::placeholder {
      /* Chrome, Firefox, Opera, Safari 10.1+ */
      color: black;
      opacity: 1; /* Firefox */
    }
    #lastName:-ms-input-placeholder {
      /* Internet Explorer 10-11 */
      color: black;
    }
    #lastName::-ms-input-placeholder {
      /* Microsoft Edge */
      color: black;
    }
    \end{codeCss}
\switchcolumn
\begin{codeCss}
/* https://www.w3schools.com/howto/howto_css_placeholder.asp */
#lastName::placeholder {
  /* Chrome, Firefox, Opera, Safari 10.1+ */
  color: black;
  opacity: 1; /* Firefox */
}
#lastName:-ms-input-placeholder {
  /* Internet Explorer 10-11 */
  color: black;
}
#lastName::-ms-input-placeholder {
  /* Microsoft Edge */
  color: black;
}
\end{codeCss}
\switchcolumn[0]*%%%%%%%
It is best to provide all the information the user needs to fill out
forms outside any inputs.
\switchcolumn
最好在表单外提供所有用户需要填写输入的信息。
\end{paracol}


\columnratio{0.55}
\begin{paracol}{2} 
 
\switchcolumn[0]*%%%%%%%
\subsubsection{Instructions}
\switchcolumn
\subsubsection{用法说明}
\switchcolumn[0]*%%%%%%%
When adding instructions for your input fields, make sure to link it
correctly to the input. You can provide additional instructions and bind
multiple ids inside an
\href{https://developer.mozilla.org/en-US/docs/Web/Accessibility/ARIA/Attributes/aria-labelledby}{\texttt{aria-labelledby}}.
This allows for more flexible design.
\switchcolumn
添加用法说明时,请确保将其正确链接到目标 input
框。你可以提供附加用法说明并在
\href{https://developer.mozilla.org/en-US/docs/Web/Accessibility/ARIA/Attributes/aria-labelledby}{\texttt{aria-labelledby}}
内绑定多个 id。这可以使设计更加灵活。
\switchcolumn[0]*%%%%%%%
\begin{codeHtml}
<fieldset>
  <legend>Using aria-labelledby</legend>
  <label id="date-label" for="date">Current Date:</label>
  <input
    type="date"
    name="date"
    id="date"
    aria-labelledby="date-label date-instructions"
  />
  <p id="date-instructions">MM/DD/YYYY</p>
</fieldset>
\end{codeHtml}
\switchcolumn
\begin{codeHtml}
<fieldset>
  <legend>Using aria-labelledby</legend>
  <label id="date-label" for="date">Current Date:</label>
  <input
    type="date"
    name="date"
    id="date"
    aria-labelledby="date-label date-instructions"
  />
  <p id="date-instructions">MM/DD/YYYY</p>
</fieldset>
\end{codeHtml}
\switchcolumn[0]*%%%%%%%
Alternatively, you can attach the instructions to the input with
\href{https://developer.mozilla.org/en-US/docs/Web/Accessibility/ARIA/Attributes/aria-describedby}{\texttt{aria-describedby}}:
\switchcolumn
或者,你可以通过
\href{https://developer.mozilla.org/en-US/docs/Web/Accessibility/ARIA/Attributes/aria-labelledby}{\texttt{aria-describedby}}
将用法说明附加到 input 框上。
\switchcolumn[0]*%%%%%%%
\begin{codeHtml}
<fieldset>
  <legend>Using aria-describedby</legend>
  <label id="dob" for="dob">Date of Birth:</label>
  <input type="date" name="dob" id="dob" aria-describedby="dob-instructions" />
  <p id="dob-instructions">MM/DD/YYYY</p>
</fieldset>
\end{codeHtml}
\switchcolumn
\begin{codeHtml}
<fieldset>
  <legend>Using aria-describedby</legend>
  <label id="dob" for="dob">Date of Birth:</label>
  <input type="date" name="dob" id="dob" aria-describedby="dob-instructions" />
  <p id="dob-instructions">MM/DD/YYYY</p>
</fieldset>
\end{codeHtml}
\switchcolumn[0]*%%%%%%%
\subsubsection{Hiding Content}
\switchcolumn
\subsubsection{隐藏内容}
\switchcolumn[0]*%%%%%%%
Usually it is not recommended to visually hide labels, even if the input
has an accessible name. However, if the functionality of the input can
be understood with surrounding content, then we can hide the visual
label.
\switchcolumn
通常,即使 input
框具有无障碍的名称,也不建议在视觉上隐藏标签。但是,如果可以借助周围的内容来理解输入的功能,那么我们也可以隐藏视觉标签。
\switchcolumn[0]*%%%%%%%
Let's look at this search field:
\switchcolumn
让我们看看这个搜索框:
\switchcolumn[0]*%%%%%%%
\begin{codeHtml}
<form role="search">
  <label for="search" class="hidden-visually">Search: </label>
  <input type="text" name="search" id="search" v-model="search" />
  <button type="submit">Search</button>
</form>
\end{codeHtml}
\switchcolumn
\begin{codeHtml}
<form role="search">
  <label for="search" class="hidden-visually">Search: </label>
  <input type="text" name="search" id="search" v-model="search" />
  <button type="submit">Search</button>
</form>
\end{codeHtml}
\switchcolumn[0]*%%%%%%%
We can do this because the search button will help visual users identify
the purpose of the input field.
\switchcolumn
现在,只要视力情况良好,用户可以就能通过按钮的内容识别出该 input
框的目的。
\switchcolumn[0]*%%%%%%%
We can use CSS to visually hide elements but keep them available for
assistive technology:
\switchcolumn
此时我们可以使用 CSS 从视觉上隐藏元素,同时也不会影响到无障碍访问:
\switchcolumn[0]*%%%%%%%
\begin{codeHtml}
.hidden-visually {
  position: absolute;
  overflow: hidden;
  white-space: nowrap;
  margin: 0;
  padding: 0;
  height: 1px;
  width: 1px;
  clip: rect(0 0 0 0);
  clip-path: inset(100%);
}
\end{codeHtml}
\switchcolumn
\begin{codeCss}
.hidden-visually {
  position: absolute;
  overflow: hidden;
  white-space: nowrap;
  margin: 0;
  padding: 0;
  height: 1px;
  width: 1px;
  clip: rect(0 0 0 0);
  clip-path: inset(100%);
}
\end{codeCss}
\switchcolumn[0]*%%%%%%%
\paragraph{aria-hidden="true"}
\switchcolumn
\paragraph{aria-hidden="true"}
\switchcolumn[0]*%%%%%%%
Adding \texttt{aria-hidden="true"} will hide the element from assistive
technology but leave it visually available for other users. Do not use
it on focusable elements, purely on decorative, duplicated or offscreen
content.
\switchcolumn
添加 \texttt{aria-hidden="true"}
在无障碍访问时被隐藏,但对其他可视用户仍然是可见的。不要在可聚焦的元素上使用它,请只在装饰性的、重复的或屏幕外的内容上使用它。
\switchcolumn[0]*%%%%%%%
\begin{codeHtml}
<p>This is not hidden from screen readers.</p>
<p aria-hidden="true">This is hidden from screen readers.</p>
\end{codeHtml}
\switchcolumn
\begin{codeHtml}
<p>This is not hidden from screen readers.</p>
<p aria-hidden="true">This is hidden from screen readers.</p>
\end{codeHtml}
\end{paracol}


\columnratio{0.55}
\begin{paracol}{2} 
 
\switchcolumn[0]*%%%%%%%
\subsubsection{Buttons}
\switchcolumn
\subsubsection{按钮}
\switchcolumn[0]*%%%%%%%
When using buttons inside a form, you must set the type to prevent
submitting the form. You can also use an input to create buttons:
\switchcolumn
在表单中使用按钮时,必须设置类型以防止提交表单。 你也可以使用一个 input
元素来创建按钮:
\switchcolumn[0]*%%%%%%%
\begin{codeHtml}
<form action="/dataCollectionLocation" method="post" autocomplete="on">
  <!-- 按钮 -->
  <button type="button">Cancel</button>
  <button type="submit">Submit</button>
  <!-- 输入按钮 -->
  <input type="button" value="Cancel" />
  <input type="submit" value="Submit" />
</form>
\end{codeHtml}
\switchcolumn
\begin{codeHtml}
<form action="/dataCollectionLocation" method="post" autocomplete="on">
  <!-- 按钮 -->
  <button type="button">Cancel</button>
  <button type="submit">Submit</button>
  <!-- 输入按钮 -->
  <input type="button" value="Cancel" />
  <input type="submit" value="Submit" />
</form>
\end{codeHtml}
\switchcolumn[0]*%%%%%%%
\subsubsection{Functional Images}
\switchcolumn
\subsubsection{功能图片}
\switchcolumn[0]*%%%%%%%
You can use this technique to create functional images.
\switchcolumn
你可以使用这种方式来创建一个带有功能的图片。
\switchcolumn[0]*%%%%%%%
\begin{itemize}
\item
  Input fields
  \begin{itemize}
  \item
    These images will act as a submit type button on forms
  \end{itemize}
\begin{codeHtml}
<form role="search">
  <label for="search" class="hidden-visually">Search: </label>
  <input type="text" name="search" id="search" v-model="search" />
  <input
    type="image"
    class="btnImg"
    src="https://img.icons8.com/search"
    alt="Search"
  />
</form>
\end{codeHtml}
\item
  Icons
\end{itemize}
\begin{codeHtml}
<form role="search">
  <label for="searchIcon" class="hidden-visually">Search: </label>
  <input type="text" name="searchIcon" id="searchIcon" v-model="searchIcon" />
  <button type="submit">
    <i class="fas fa-search" aria-hidden="true"></i>
    <span class="hidden-visually">Search</span>
  </button>
</form>
\end{codeHtml}
\switchcolumn
\begin{itemize}
\item
  input 框
  \begin{itemize}
  \item
    这些图片会像一个类型为 submit 的表单按钮一样
  \end{itemize}
\begin{codeHtml}
<form role="search">
  <label for="search" class="hidden-visually">Search: </label>
  <input type="text" name="search" id="search" v-model="search" />
  <input
    type="image"
    class="btnImg"
    src="https://img.icons8.com/search"
    alt="Search"
  />
</form>
\end{codeHtml}
\item
  图标
\end{itemize}
\begin{codeHtml}
<form role="search">
  <label for="searchIcon" class="hidden-visually">Search: </label>
  <input type="text" name="searchIcon" id="searchIcon" v-model="searchIcon" />
  <button type="submit">
    <i class="fas fa-search" aria-hidden="true"></i>
    <span class="hidden-visually">Search</span>
  </button>
</form>
\end{codeHtml}
\end{paracol}



\columnratio{0.55}
\begin{paracol}{2} 
 
\switchcolumn[0]*%%%%%%%
\subsection{Standards}
\switchcolumn
\subsection{规范}
\switchcolumn[0]*%%%%%%%
The World Wide Web Consortium (W3C) Web Accessibility Initiative (WAI)
develops web accessibility standards for the different components:
\switchcolumn
万维网联盟 (W3C) Web 无障碍访问倡议 (WAI) 为不同的组件制定了 Web
无障碍性标准:
\switchcolumn[0]*%%%%%%%
\begin{itemize}
\item
  User Agent Accessibility Guidelines (UAAG)
  \begin{itemize}
  \item
    web browsers and media players, including some aspects of assistive
    technologies
  \end{itemize}
\item
  Authoring Tool Accessibility Guidelines (ATAG)
  \begin{itemize}
  \item
    authoring tools
  \end{itemize}
\item
  Web Content Accessibility Guidelines (WCAG)
  \begin{itemize}
  \item
    web content - used by developers, authoring tools, and accessibility
    evaluation tools
  \end{itemize}
\end{itemize}
\switchcolumn
\begin{itemize}
\item
  用户代理无障碍访问指南 (UAAG)
  \begin{itemize}
  \item
    浏览器和媒体查询,包括一些其他方面的辅助技术
  \end{itemize}
\item
  创作工具无障碍访问指南 (ATAG)
  \begin{itemize}
  \item
    创作工具
  \end{itemize}
\item
  Web 内容无障碍访问指南 (WCAG)
  \begin{itemize}
  \item
    网站内容 - 由开发者、创作工具和无障碍访问评估工具使用。
  \end{itemize}
\end{itemize}
\end{paracol}



\columnratio{0.55}
\begin{paracol}{2} 
 
\switchcolumn[0]*%%%%%%%
\subsubsection{Web Content Accessibility Guidelines (WCAG)}
\switchcolumn
\subsubsection{网络内容无障碍指南 (WCAG)}
\switchcolumn[0]*%%%%%%%
\href{https://www.w3.org/TR/WCAG21/}{WCAG 2.1} extends on
\href{https://www.w3.org/TR/WCAG20/}{WCAG 2.0} and allows implementation
of new technologies by addressing changes to the web. The W3C encourages
use of the most current version of WCAG when developing or updating Web
accessibility policies.
\switchcolumn
\href{https://www.w3.org/TR/WCAG21/}{WCAG 2.1} 继承自
\href{https://www.w3.org/TR/WCAG20/}{WCAG 2.0},接纳 Web
演进过程中的新技术。W3C 鼓励在开发或更新 Web 无障碍访问策略时使用 WCAG
的最新版本。
\end{paracol}


\columnratio{0.55}
\begin{paracol}{2} 
 
\switchcolumn[0]*%%%%%%%
\paragraph{WCAG 2.1 Four Main Guiding Principles (abbreviated as POUR):}
\switchcolumn
\paragraph{WCAG 2.1 四大指导原则 (缩写 POUR):}
\switchcolumn[0]*%%%%%%%
\begin{itemize}
\item
  Perceivable
  \begin{itemize}
  \item
    Users must be able to perceive the information being presented
  \end{itemize}
\item
  Operable
  \begin{itemize}
  \item
    Interface forms, controls, and navigation are operable
  \end{itemize}
\item
  Understandable
  \begin{itemize}
  \item
    Information and the operation of user interface must be
    understandable to all users
  \end{itemize}
\item
  Robust
  \begin{itemize}
  \item
    Users must be able to access the content as technologies advance
  \end{itemize}
\end{itemize}
\switchcolumn
\begin{itemize}
\item
  可感知性
  \begin{itemize}
  \item
    用户必须能够感知所渲染的信息
  \end{itemize}
\item
  可操作性
  \begin{itemize}
  \item
    表单界面,控件和导航是可操作的
  \end{itemize}
\item
  可理解性
  \begin{itemize}
  \item
    信息和用户界面的操作必须为所有用户所理解
  \end{itemize}
\item
  健壮性
  \begin{itemize}
  \item
    随着技术的进步,用户必须能够访问内容
  \end{itemize}
\end{itemize}
\switchcolumn[0]*%%%%%%%
\paragraph{Web Accessibility Initiative -- Accessible Rich Internet Applications (WAI-ARIA)}
\switchcolumn
\paragraph{Web 无障碍倡议 -- 无障碍访问丰富的互联网应用 (WAI-ARIA)}
\switchcolumn[0]*%%%%%%%
W3C's WAI-ARIA provides guidance on how to build dynamic content and
advanced user interface controls.
\switchcolumn
W3C 的 WAI-ARIA 为如何构建动态内容和高阶用户界面控件提供了指导。
\switchcolumn[0]*%%%%%%%
\begin{itemize}
\item
  \href{https://www.w3.org/TR/wai-aria-1.2/}{Accessible Rich Internet
  Applications (WAI-ARIA) 1.2}
\item
  \href{https://www.w3.org/TR/wai-aria-practices-1.2/}{WAI-ARIA
  Authoring Practices 1.2}
\end{itemize}
\switchcolumn
\begin{itemize}
\item
  \href{https://www.w3.org/TR/wai-aria-1.2/}{可便捷访问的丰富互联网应用
  (WAI-ARIA) 1.2}
\item
  \href{https://www.w3.org/TR/wai-aria-practices-1.2/}{WAI-ARIA 实践
  1.2}
\end{itemize}
\switchcolumn[0]*%%%%%%%
\subsection{Resources}
\switchcolumn
\subsection{资源}
\switchcolumn[0]*%%%%%%%
\subsubsection{Documentation}
\switchcolumn
\subsubsection{文档}
\switchcolumn[0]*%%%%%%%
\begin{itemize}
\item
  \href{https://www.w3.org/TR/WCAG20/}{WCAG 2.0}
\item
  \href{https://www.w3.org/TR/WCAG21/}{WCAG 2.1}
\item
  \href{https://www.w3.org/TR/wai-aria-1.2/}{Accessible Rich Internet
  Applications (WAI-ARIA) 1.2}
\item
  \href{https://www.w3.org/TR/wai-aria-practices-1.2/}{WAI-ARIA
  Authoring Practices 1.2}
\end{itemize}
\switchcolumn
\begin{itemize}
\item
  \href{https://www.w3.org/TR/WCAG20/}{WCAG 2.0}
\item
  \href{https://www.w3.org/TR/WCAG21/}{WCAG 2.1}
\item
  \href{https://www.w3.org/TR/wai-aria-1.2/}{Accessible Rich Internet
  Applications (WAI-ARIA) 1.2}
\item
  \href{https://www.w3.org/TR/wai-aria-practices-1.2/}{WAI-ARIA
  Authoring Practices 1.2}
\end{itemize}
\switchcolumn[0]*%%%%%%%
\subsubsection{Assistive Technologies}
\switchcolumn
\subsubsection{辅助技术}
\switchcolumn[0]*%%%%%%%
\begin{itemize}
\item
  Screen Readers
  \begin{itemize}
  \item
    \href{https://www.nvaccess.org/download/}{NVDA}
  \item
    \href{https://www.apple.com/accessibility/mac/vision/}{VoiceOver}
  \item
    \href{https://www.freedomscientific.com/products/software/jaws/?utm_term=jaws\%20screen\%20reader\&utm_source=adwords\&utm_campaign=All+Products\&utm_medium=ppc\&hsa_tgt=kwd-394361346638\&hsa_cam=200218713\&hsa_ad=296201131673\&hsa_kw=jaws\%20screen\%20reader\&hsa_grp=52663682111\&hsa_net=adwords\&hsa_mt=e\&hsa_src=g\&hsa_acc=1684996396\&hsa_ver=3\&gclid=Cj0KCQjwnv71BRCOARIsAIkxW9HXKQ6kKNQD0q8a_1TXSJXnIuUyb65KJeTWmtS6BH96-5he9dsNq6oaAh6UEALw_wcB}{JAWS}
  \item
    \href{https://chrome.google.com/webstore/detail/chromevox-classic-extensi/kgejglhpjiefppelpmljglcjbhoiplfn?hl=en}{ChromeVox}
  \end{itemize}
\item
  Zooming Tools
  \begin{itemize}
  \item
    \href{https://www.freedomscientific.com/products/software/magic/}{MAGic}
  \item
    \href{https://www.zoomtext.com/}{ZoomText}
  \item
    \href{https://support.microsoft.com/en-us/help/11542/windows-use-magnifier-to-make-things-easier-to-see}{Magnifier}
  \end{itemize}
\end{itemize}
\switchcolumn
\begin{itemize}
\item
  屏幕助读器
  \begin{itemize}
  \item
    \href{https://www.nvaccess.org/download/}{NVDA}
  \item
    \href{https://www.apple.com/accessibility/mac/vision/}{VoiceOver}
  \item
    \href{https://www.freedomscientific.com/products/software/jaws/?utm_term=jaws\%20screen\%20reader\&utm_source=adwords\&utm_campaign=All+Products\&utm_medium=ppc\&hsa_tgt=kwd-394361346638\&hsa_cam=200218713\&hsa_ad=296201131673\&hsa_kw=jaws\%20screen\%20reader\&hsa_grp=52663682111\&hsa_net=adwords\&hsa_mt=e\&hsa_src=g\&hsa_acc=1684996396\&hsa_ver=3\&gclid=Cj0KCQjwnv71BRCOARIsAIkxW9HXKQ6kKNQD0q8a_1TXSJXnIuUyb65KJeTWmtS6BH96-5he9dsNq6oaAh6UEALw_wcB}{JAWS}
  \item
    \href{https://chrome.google.com/webstore/detail/chromevox-classic-extensi/kgejglhpjiefppelpmljglcjbhoiplfn?hl=en}{ChromeVox}
  \end{itemize}
\item
  缩放工具
  \begin{itemize}
  \item
    \href{https://www.freedomscientific.com/products/software/magic/}{MAGic}
  \item
    \href{https://www.zoomtext.com/}{ZoomText}
  \item
    \href{https://support.microsoft.com/en-us/help/11542/windows-use-magnifier-to-make-things-easier-to-see}{Magnifier}
  \end{itemize}
\end{itemize}
\switchcolumn[0]*%%%%%%%
\subsubsection{Testing}
\switchcolumn
\subsubsection{测试}
\switchcolumn[0]*%%%%%%%
\begin{itemize}
\item
  Automated Tools
  \begin{itemize}
  \item
    \href{https://chrome.google.com/webstore/detail/lighthouse/blipmdconlkpinefehnmjammfjpmpbjk}{Lighthouse}
  \item
    \href{https://chrome.google.com/webstore/detail/wave-evaluation-tool/jbbplnpkjmmeebjpijfedlgcdilocofh}{WAVE}
  \item
    \href{https://chrome.google.com/webstore/detail/arc-toolkit/chdkkkccnlfncngelccgbgfmjebmkmce?hl=en-US}{ARC
    Toolkit}
  \end{itemize}
\item
  Color Tools
  \begin{itemize}
  \item
    \href{https://webaim.org/resources/contrastchecker/}{WebAim Color
    Contrast}
  \item
    \href{https://webaim.org/resources/linkcontrastchecker}{WebAim Link
    Color Contrast}
  \end{itemize}
\item
  Other Helpful Tools
  \begin{itemize}
  \item
    \href{https://chrome.google.com/webstore/detail/headingsmap/flbjommegcjonpdmenkdiocclhjacmbi?hl=en…}{HeadingMap}
  \item
    \href{https://colororacle.org/}{Color Oracle}
  \item
    \href{https://chrome.google.com/webstore/detail/nerdefocus/lpfiljldhgjecfepfljnbjnbjfhennpd?hl=en-US…}{NerdeFocus}
  \item
    \href{https://chrome.google.com/webstore/detail/visual-aria/lhbmajchkkmakajkjenkchhnhbadmhmk?hl=en-US}{Visual
    Aria}
  \item
    \href{https://chrome.google.com/webstore/detail/silktide-website-accessib/okcpiimdfkpkjcbihbmhppldhiebhhaf?hl=en-US}{Silktide
    Website Accessibility Simulator}
  \end{itemize}
\end{itemize}
\switchcolumn
\begin{itemize}
\item
  自动化相关的工具
  \begin{itemize}
  \item
    \href{https://chrome.google.com/webstore/detail/lighthouse/blipmdconlkpinefehnmjammfjpmpbjk}{Lighthouse}
  \item
    \href{https://chrome.google.com/webstore/detail/wave-evaluation-tool/jbbplnpkjmmeebjpijfedlgcdilocofh}{WAVE}
  \item
    \href{https://chrome.google.com/webstore/detail/arc-toolkit/chdkkkccnlfncngelccgbgfmjebmkmce?hl=en-US}{ARC
    Toolkit}
  \end{itemize}
\item
  颜色相关的工具
  \begin{itemize}
  \item
    \href{https://webaim.org/resources/contrastchecker/}{WebAim Color
    Contrast}
  \item
    \href{https://webaim.org/resources/linkcontrastchecker}{WebAim Link
    Color Contrast}
  \end{itemize}
\item
  其他有用的工具
  \begin{itemize}
  \item
    \href{https://chrome.google.com/webstore/detail/headingsmap/flbjommegcjonpdmenkdiocclhjacmbi?hl=en…}{HeadingMap}
  \item
    \href{https://colororacle.org/}{Color Oracle}
  \item
    \href{https://chrome.google.com/webstore/detail/nerdefocus/lpfiljldhgjecfepfljnbjnbjfhennpd?hl=en-US…}{NerdeFocus}
  \item
    \href{https://chrome.google.com/webstore/detail/visual-aria/lhbmajchkkmakajkjenkchhnhbadmhmk?hl=en-US}{Visual
    Aria}
  \item
    \href{https://chrome.google.com/webstore/detail/silktide-website-accessib/okcpiimdfkpkjcbihbmhppldhiebhhaf?hl=en-US}{Silktide
    Website Accessibility Simulator}
  \end{itemize}
\end{itemize}
\end{paracol}



\columnratio{0.55}
\begin{paracol}{2} 
 
\switchcolumn[0]*%%%%%%%
\subsubsection{Users}
\switchcolumn
\subsubsection{用户}
\switchcolumn[0]*%%%%%%%
The World Health Organization estimates that 15\% of the world's
population has some form of disability, 2-4\% of them severely so. That
is an estimated 1 billion people worldwide; making people with
disabilities the largest minority group in the world.
\switchcolumn
世界卫生组织估计,全世界 15\% 的人口患有某种形式的残疾,其中约 2 - 4\%
的人严重残疾。估计全世界有 10 亿残障人士,他们是世界上最大的少数群体。
\switchcolumn[0]*%%%%%%%
There are a huge range of disabilities, which can be divided roughly
into four categories:
\switchcolumn
残疾的种类繁多,大致可分为以下四类:
\switchcolumn[0]*%%%%%%%
\begin{itemize}
\item
  \emph{\href{https://webaim.org/articles/visual/}{Visual}} - These
  users can benefit from the use of screen readers, screen
  magnification, controlling screen contrast, or braille display.
\item
  \emph{\href{https://webaim.org/articles/auditory/}{Auditory}} - These
  users can benefit from captioning, transcripts or sign language video.
\item
  \emph{\href{https://webaim.org/articles/motor/}{Motor}} - These users
  can benefit from a range of
  \href{https://webaim.org/articles/motor/assistive}{assistive
  technologies for motor impairments}: voice recognition software, eye
  tracking, single-switch access, head wand, sip and puff switch,
  oversized trackball mouse, adaptive keyboard or other assistive
  technologies.
\item
  \emph{\href{https://webaim.org/articles/cognitive/}{Cognitive}} -
  These users can benefit from supplemental media, structural
  organization of content, clear and simple writing.
\end{itemize}
\switchcolumn
\begin{itemize}
\item
  \emph{\href{https://webaim.org/articles/visual/}{视觉}} -
  可以为这些用户提供屏幕助读器、屏幕缩放、控制屏幕对比度或盲文显示等帮助。
\item
  \emph{\href{https://webaim.org/articles/auditory/}{听觉}} -
  可以为这些用户提供视频字幕、文字记录或手语视频。
\item
  \emph{\href{https://webaim.org/articles/motor/}{运动能力}} -
  可以为这些用户提供一系列\href{https://webaim.org/articles/motor/assistive}{运动障碍辅助技术}中:比如语音识别软件、眼球跟踪、单刀式开关、超大轨迹球鼠标、自适应键盘等等。
\item
  \emph{\href{https://webaim.org/articles/cognitive/}{认知能力}} -
  可以为这些用户提供补充媒体、更清晰和简单、更结构化的内容。
\end{itemize}
\switchcolumn[0]*%%%%%%%
Check out the following links from WebAim to understand from users:
\switchcolumn
你可以查看以下来自 WebAim 的链接,更深入地了解这些用户的需求:
\switchcolumn[0]*%%%%%%%
\begin{itemize}
\item
  \href{https://www.w3.org/WAI/perspective-videos/}{Web Accessibility
  Perspectives: Explore the Impact and Benefits for Everyone}
\item
  \href{https://www.w3.org/WAI/people-use-web/user-stories/}{Stories of
  Web Users}
\end{itemize}
\switchcolumn
\begin{itemize}
\item
  \href{https://www.w3.org/WAI/perspective-videos/}{Web
  无障碍愿景:探索改变 \& 人人受益}
\item
  \href{https://www.w3.org/WAI/people-use-web/user-stories/}{Web
  用户的故事}
\end{itemize}
\switchcolumn[0]*%%%%%%%
\href{https://github.com/vuejs/docs/edit/main/src/guide/best-practices/accessibility.md}{Edit
this page on GitHub}
\switchcolumn
\href{https://github.com/vuejs-translations/docs-zh-cn/edit/main/src/guide/best-practices/accessibility.md}{在
GitHub 上编辑此页}
\end{paracol}

