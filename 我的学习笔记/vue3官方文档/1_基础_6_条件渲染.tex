\columnratio{0.55}
\begin{paracol}{2}

\switchcolumn[0]*%%%%%%%
\section{Conditional Rendering}
\switchcolumn
\section{条件渲染}
\switchcolumn[0]*%%%%%%%
\subsection{v-if}
\switchcolumn
\subsection{v-if}
\switchcolumn[0]*%%%%%%%
The directive \texttt{v-if} is used to conditionally render a block. The
block will only be rendered if the directive's expression returns a
truthy value.
\switchcolumn
\texttt{v-if}
指令用于条件性地渲染一块内容。这块内容只会在指令的表达式返回真值时才被渲染。
\switchcolumn[0]*%%%%%%%
\begin{codeHtml}
<h1 v-if="awesome">Vue is awesome!</h1>
\end{codeHtml}
\switchcolumn
\begin{codeHtml}
<h1 v-if="awesome">Vue is awesome!</h1>
\end{codeHtml}
\switchcolumn[0]*%%%%%%%
\subsection{v-else}
\switchcolumn
\subsection{v-else}
\switchcolumn[0]*%%%%%%%
You can use the \texttt{v-else} directive to indicate an "else block"
for \texttt{v-if}:
\switchcolumn
你也可以使用 \texttt{v-else} 为 \texttt{v-if} 添加一个``else 区块''。
\switchcolumn[0]*%%%%%%%
\begin{codeHtml}
<button @click="awesome = !awesome">Toggle</button>
<h1 v-if="awesome">Vue is awesome!</h1>
<h1 v-else>Oh no 😢</h1>
\end{codeHtml}
\switchcolumn
\begin{codeHtml}
<button @click="awesome = !awesome">Toggle</button>
<h1 v-if="awesome">Vue is awesome!</h1>
<h1 v-else>Oh no 😢</h1>
\end{codeHtml}

\switchcolumn[0]*%%%%%%%
\href{https://play.vuejs.org/\#eNpFjkEOgjAQRa8ydIMulLA1hegJ3LnqBskAjdA27RQXhHu4M/GEHsEiKLv5mfdf/sBOxux7j+zAuCutNAQOyZtcKNkZbQkGsFjBCJXVHcQBjYUSqtTKERR3dLpDyCZmQ9bjViiezKKgCIGwM21BGBIAv3oireBYtrK8ZYKtgmg5BctJ13WLPJnhr0YQb1Lod7JaS4G8eATpfjMinjTphC8wtg7zcwNKw/v5eC1fnvwnsfEDwaha7w==}{Try
it in the Playground}
\switchcolumn
\href{https://play.vuejs.org/\#eNpFjkEOgjAQRa8ydIMulLA1hegJ3LnqBskAjdA27RQXhHu4M/GEHsEiKLv5mfdf/sBOxux7j+zAuCutNAQOyZtcKNkZbQkGsFjBCJXVHcQBjYUSqtTKERR3dLpDyCZmQ9bjViiezKKgCIGwM21BGBIAv3oireBYtrK8ZYKtgmg5BctJ13WLPJnhr0YQb1Lod7JaS4G8eATpfjMinjTphC8wtg7zcwNKw/v5eC1fnvwnsfEDwaha7w==}{在演练场中尝试一下}
\switchcolumn[0]*%%%%%%%
A \texttt{v-else} element must immediately follow a \texttt{v-if} or a
\texttt{v-else-if} element - otherwise it will not be recognized.
\switchcolumn
一个 \texttt{v-else} 元素必须跟在一个 \texttt{v-if} 或者
\texttt{v-else-if} 元素后面,否则它将不会被识别。
\switchcolumn[0]*%%%%%%%
\subsection{v-else-if}
\switchcolumn
\subsection{v-else-if}
\switchcolumn[0]*%%%%%%%
The \texttt{v-else-if}, as the name suggests, serves as an "else if
block" for \texttt{v-if}. It can also be chained multiple times:
\switchcolumn
顾名思义,\texttt{v-else-if} 提供的是相应于 \texttt{v-if} 的``else if
区块''。它可以连续多次重复使用:
\switchcolumn[0]*%%%%%%%
\begin{codeHtml}
<div v-if="type === 'A'">
  A
</div>
<div v-else-if="type === 'B'">
  B
</div>
<div v-else-if="type === 'C'">
  C
</div>
<div v-else>
  Not A/B/C
</div>
\end{codeHtml}
\switchcolumn
\begin{codeHtml}
<div v-if="type === 'A'">
  A
</div>
<div v-else-if="type === 'B'">
  B
</div>
<div v-else-if="type === 'C'">
  C
</div>
<div v-else>
  Not A/B/C
</div>
\end{codeHtml}
\switchcolumn[0]*%%%%%%%
Similar to \texttt{v-else}, a \texttt{v-else-if} element must
immediately follow a \texttt{v-if} or a \texttt{v-else-if} element.
\switchcolumn
和 \texttt{v-else} 类似,一个使用 \texttt{v-else-if}
的元素必须紧跟在一个 \texttt{v-if} 或一个 \texttt{v-else-if} 元素后面。

\switchcolumn[0]*%%%%%%%
\subsection{v-if on \textless template\textgreater{}}
\switchcolumn
\subsection{\textless template\textgreater{} 上的 v-if}
\switchcolumn[0]*%%%%%%%
Because \texttt{v-if} is a directive, it has to be attached to a single
element. But what if we want to toggle more than one element? In this
case we can use \texttt{v-if} on a
\texttt{\textless{}template\textgreater{}} element, which serves as an
invisible wrapper. The final rendered result will not include the
\texttt{\textless{}template\textgreater{}} element.
\switchcolumn
因为 \texttt{v-if}
是一个指令,他必须依附于某个元素。但如果我们想要切换不止一个元素呢?在这种情况下我们可以在一个
\texttt{\textless{}template\textgreater{}} 元素上使用
\texttt{v-if},这只是一个不可见的包装器元素,最后渲染的结果并不会包含这个
\texttt{\textless{}template\textgreater{}} 元素。
\switchcolumn[0]*%%%%%%%
\begin{codeHtml}
<template v-if="ok">
  <h1>Title</h1>
  <p>Paragraph 1</p>
  <p>Paragraph 2</p>
</template>
\end{codeHtml}
\switchcolumn
\begin{codeHtml}
<template v-if="ok">
  <h1>Title</h1>
  <p>Paragraph 1</p>
  <p>Paragraph 2</p>
</template>
\end{codeHtml}
\switchcolumn[0]*%%%%%%%
\texttt{v-else} and \texttt{v-else-if} can also be used on
\texttt{\textless{}template\textgreater{}}.
\switchcolumn
\texttt{v-else} 和 \texttt{v-else-if} 也可以在
\texttt{\textless{}template\textgreater{}} 上使用。
\switchcolumn[0]*%%%%%%%
\subsection{v-show}
\switchcolumn
\subsection{v-show}
\switchcolumn[0]*%%%%%%%
Another option for conditionally displaying an element is the
\texttt{v-show} directive. The usage is largely the same:
\switchcolumn
另一个可以用来按条件显示一个元素的指令是
\texttt{v-show}。其用法基本一样:
\switchcolumn[0]*%%%%%%%
\begin{codeHtml}
<h1 v-show="ok">Hello!</h1>
\end{codeHtml}
\switchcolumn
\begin{codeHtml}
<h1 v-show="ok">Hello!</h1>
\end{codeHtml}
\switchcolumn[0]*%%%%%%%
The difference is that an element with \texttt{v-show} will always be
rendered and remain in the DOM; \texttt{v-show} only toggles the
\texttt{display} CSS property of the element.
\switchcolumn
不同之处在于 \texttt{v-show} 会在 DOM 渲染中保留该元素;\texttt{v-show}
仅切换了该元素上名为 \texttt{display} 的 CSS 属性。
\switchcolumn[0]*%%%%%%%
\texttt{v-show} doesn't support the
\texttt{\textless{}template\textgreater{}} element, nor does it work
with \texttt{v-else}.
\switchcolumn
\texttt{v-show} 不支持在 \texttt{\textless{}template\textgreater{}}
元素上使用,也不能和 \texttt{v-else} 搭配使用。
\switchcolumn[0]*%%%%%%%
\subsection{v-if vs. v-show}
\switchcolumn
\subsection{v-if vs. v-show}
\switchcolumn[0]*%%%%%%%
\texttt{v-if} is "real" conditional rendering because it ensures that
event listeners and child components inside the conditional block are
properly destroyed and re-created during toggles.
\switchcolumn
\texttt{v-if}
是``真实的''按条件渲染,因为它确保了在切换时,条件区块内的事件监听器和子组件都会被销毁与重建。
\switchcolumn[0]*%%%%%%%
\texttt{v-if} is also \textbf{lazy}: if the condition is false on
initial render, it will not do anything - the conditional block won't be
rendered until the condition becomes true for the first time.
\switchcolumn
\texttt{v-if} 也是\textbf{惰性}的:如果在初次渲染时条件值为
false,则不会做任何事。条件区块只有当条件首次变为 true 时才被渲染。
\switchcolumn[0]*%%%%%%%
In comparison, \texttt{v-show} is much simpler - the element is always
rendered regardless of initial condition, with CSS-based toggling.
\switchcolumn
相比之下,\texttt{v-show}
简单许多,元素无论初始条件如何,始终会被渲染,只有 CSS \texttt{display}
属性会被切换。
\switchcolumn[0]*%%%%%%%
Generally speaking, \texttt{v-if} has higher toggle costs while
\texttt{v-show} has higher initial render costs. So prefer
\texttt{v-show} if you need to toggle something very often, and prefer
\texttt{v-if} if the condition is unlikely to change at runtime.
\switchcolumn
总的来说,\texttt{v-if} 有更高的切换开销,而 \texttt{v-show}
有更高的初始渲染开销。因此,如果需要频繁切换,则使用 \texttt{v-show}
较好;如果在运行时绑定条件很少改变,则 \texttt{v-if} 会更合适。


\end{paracol}

\columnratio{0.55}
\begin{paracol}{2}
\switchcolumn[0]*%%%%%%%
\subsection{v-if with v-for}
\switchcolumn
\subsection{v-if 和 v-for}
\switchcolumn[0]*%%%%%%%
\begin{vueQuoteWarn}{Note}
It's \textbf{not} recommended to use \texttt{v-if} and \texttt{v-for} on
the same element due to implicit precedence. Refer to
\href{https://vuejs.org/style-guide/rules-essential.html\#avoid-v-if-with-v-for}{style
guide} for details.    
\end{vueQuoteWarn}
\switchcolumn
\begin{vueQuoteWarn}{警告}
同时使用 \texttt{v-if} 和 \texttt{v-for}
是\textbf{不推荐的},因为这样二者的优先级不明显。请查看\href{https://cn.vuejs.org/style-guide/rules-essential.html\#avoid-v-if-with-v-for}{风格指南}获得更多信息。
\end{vueQuoteWarn}
\switchcolumn[0]*%%%%%%%
When \texttt{v-if} and \texttt{v-for} are both used on the same element,
\texttt{v-if} will be evaluated first. See the
\href{https://vuejs.org/guide/essentials/list.html\#v-for-with-v-if}{list
rendering guide} for details.
\switchcolumn
当 \texttt{v-if} 和 \texttt{v-for}
同时存在于一个元素上的时候,\texttt{v-if}
会首先被执行。请查看\href{https://cn.vuejs.org/guide/essentials/list.html\#v-for-with-v-if}{列表渲染指南}获取更多细节。
\end{paracol}
