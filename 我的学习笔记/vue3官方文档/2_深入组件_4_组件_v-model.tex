\columnratio{0.55}
\begin{paracol}{2}
\switchcolumn[0]*%%%%%%%
\section{Component v-model}
\switchcolumn
\section{组件 v-model}
\switchcolumn[0]*%%%%%%%
\texttt{v-model} can be used on a component to implement a two-way
binding.
\switchcolumn
\texttt{v-model} 可以在组件上使用以实现双向绑定。
\switchcolumn[0]*%%%%%%%
First let's revisit how \texttt{v-model} is used on a native element:
\switchcolumn
首先让我们回忆一下 \texttt{v-model} 在原生元素上的用法:
\switchcolumn[0]*%%%%%%%
\begin{codeHtml}
<input v-model="searchText" />
\end{codeHtml}
\switchcolumn
\begin{codeHtml}
<input v-model="searchText" />
\end{codeHtml}

\switchcolumn[0]*%%%%%%%
Under the hood, the template compiler expands \texttt{v-model} to the
more verbose equivalent for us. So the above code does the same as the
following:
\switchcolumn
在代码背后,模板编译器会对 \texttt{v-model}
进行更冗长的等价展开。因此上面的代码其实等价于下面这段:
\switchcolumn[0]*%%%%%%%
\begin{codeHtml}
<input
  :value="searchText"
  @input="searchText = $event.target.value"
/>
\end{codeHtml}
\switchcolumn
\begin{codeHtml}
<input
  :value="searchText"
  @input="searchText = $event.target.value"
/>
\end{codeHtml}
\switchcolumn[0]*%%%%%%%
When used on a component, \texttt{v-model} instead expands to this:
\switchcolumn
而当使用在一个组件上时,\texttt{v-model} 会被展开为如下的形式:
\switchcolumn[0]*%%%%%%%
\begin{codeHtml}
<CustomInput
  :model-value="searchText"
  @update:model-value="newValue => searchText = newValue"
/>
\end{codeHtml}
\switchcolumn
\begin{codeHtml}
<CustomInput
  :model-value="searchText"
  @update:model-value="newValue => searchText = newValue"
/>
\end{codeHtml}

\switchcolumn[0]*%%%%%%%
For this to actually work though, the
\texttt{\textless{}CustomInput\textgreater{}} component must do two
things:
\switchcolumn
要让这个例子实际工作起来,\texttt{\textless{}CustomInput\textgreater{}}
组件内部需要做两件事:
\switchcolumn[0]*%%%%%%%
\begin{enumerate}
\item
  Bind the \texttt{value} attribute of a native
  \texttt{\textless{}input\textgreater{}} element to the
  \texttt{modelValue} prop
\item
  When a native \texttt{input} event is triggered, emit an
  \texttt{update:modelValue} custom event with the new value
\end{enumerate}
\switchcolumn
\begin{enumerate}
\item
  将内部原生 \texttt{\textless{}input\textgreater{}} 元素的
  \texttt{value} attribute 绑定到 \texttt{modelValue} prop
\item
  当原生的 \texttt{input} 事件触发时,触发一个携带了新值的
  \texttt{update:modelValue} 自定义事件
\end{enumerate}
\switchcolumn[0]*%%%%%%%
Here's that in action:
\switchcolumn
这里是相应的代码:
\switchcolumn[0]*%%%%%%%
\begin{codeHtml}
<!-- CustomInput.vue -->
<script setup>
defineProps(['modelValue'])
defineEmits(['update:modelValue'])
</script>
<template>
  <input
    :value="modelValue"
    @input="$emit('update:modelValue', $event.target.value)"
  />
</template>
\end{codeHtml}
\switchcolumn
\begin{codeHtml}
<!-- CustomInput.vue -->
<script setup>
defineProps(['modelValue'])
defineEmits(['update:modelValue'])
</script>
<template>
  <input
    :value="modelValue"
    @input="$emit('update:modelValue', $event.target.value)"
  />
</template>
\end{codeHtml}
\switchcolumn[0]*%%%%%%%
Now \texttt{v-model} should work perfectly with this component:
\switchcolumn
现在 \texttt{v-model} 可以在这个组件上正常工作了:
\switchcolumn[0]*%%%%%%%
\begin{codeHtml}
<CustomInput v-model="searchText" />
\end{codeHtml}
\switchcolumn
\begin{codeHtml}
<CustomInput v-model="searchText" />
\end{codeHtml}
\switchcolumn[0]*%%%%%%%
\href{https://play.vuejs.org/\#eNp9j81qwzAQhF9lEQE7kNp344SW0kNvPfVS9WDidSrQH9LKF+N37yoOxoSQm7QzO9/sJN68r8aEohFtPAflCSJS8idplfEuEEwQcIAZhuAMFGwtVuk9RXLm0/pEN7mqN7Ocy2YAac/ORgKDMXYXhGOOLIs/1NoVe2nbekEzlD+ExuuOkH8A7ZYxvhjXoz5KcUuSAuoTTNOaPM85bU0QB3HX58GdPQ7K4ldwPpY/xZXw3Wmu/svVFvHDKMpi8j3HNneeZ/VVBucXQDPmjVx+XZdikV6vNpZ2yKTyAecAOxzRUkVduCCfkqf7Zb9m1Pbo+R9ZkqZn}{Try
it in the Playground}
\switchcolumn
\href{https://play.vuejs.org/\#eNp9j81qwzAQhF9lEQE7kNp344SW0kNvPfVS9WDidSrQH9LKF+N37yoOxoSQm7QzO9/sJN68r8aEohFtPAflCSJS8idplfEuEEwQcIAZhuAMFGwtVuk9RXLm0/pEN7mqN7Ocy2YAac/ORgKDMXYXhGOOLIs/1NoVe2nbekEzlD+ExuuOkH8A7ZYxvhjXoz5KcUuSAuoTTNOaPM85bU0QB3HX58GdPQ7K4ldwPpY/xZXw3Wmu/svVFvHDKMpi8j3HNneeZ/VVBucXQDPmjVx+XZdikV6vNpZ2yKTyAecAOxzRUkVduCCfkqf7Zb9m1Pbo+R9ZkqZn}{在演练场中尝试一下}
\switchcolumn[0]*%%%%%%%
Another way of implementing \texttt{v-model} within this component is to
use a writable \texttt{computed} property with both a getter and a
setter. The \texttt{get} method should return the \texttt{modelValue}
property and the \texttt{set} method should emit the corresponding
event:
\switchcolumn
另一种在组件内实现 \texttt{v-model} 的方式是使用一个可写的,同时具有
getter 和 setter 的 \texttt{computed} 属性。\texttt{get} 方法需返回
\texttt{modelValue} prop,而 \texttt{set} 方法需触发相应的事件:
\switchcolumn[0]*%%%%%%%
\begin{codeHtml}
<!-- CustomInput.vue -->
<script setup>
import { computed } from 'vue'
const props = defineProps(['modelValue'])
const emit = defineEmits(['update:modelValue'])
const value = computed({
  get() {
    return props.modelValue
  },
  set(value) {
    emit('update:modelValue', value)
  }
})
</script>
<template>
  <input v-model="value" />
</template>
\end{codeHtml}
\switchcolumn
\begin{codeHtml}
<!-- CustomInput.vue -->
<script setup>
import { computed } from 'vue'
const props = defineProps(['modelValue'])
const emit = defineEmits(['update:modelValue'])
const value = computed({
  get() {
    return props.modelValue
  },
  set(value) {
    emit('update:modelValue', value)
  }
})
</script>
<template>
  <input v-model="value" />
</template>
\end{codeHtml}
\end{paracol}

\columnratio{0.55}
\begin{paracol}{2}

\switchcolumn[0]*%%%%%%%
\subsection{v-model arguments}
\switchcolumn
\subsection{v-model 的参数}
\switchcolumn[0]*%%%%%%%
By default, \texttt{v-model} on a component uses \texttt{modelValue} as
the prop and \texttt{update:modelValue} as the event. We can modify
these names passing an argument to \texttt{v-model}:
\switchcolumn
默认情况下,\texttt{v-model} 在组件上都是使用 \texttt{modelValue} 作为
prop,并以 \texttt{update:modelValue} 作为对应的事件。我们可以通过给
\texttt{v-model} 指定一个参数来更改这些名字:
\switchcolumn[0]*%%%%%%%
\begin{codeHtml}
<MyComponent v-model:title="bookTitle" />
\end{codeHtml}
\switchcolumn
\begin{codeHtml}
<MyComponent v-model:title="bookTitle" />
\end{codeHtml}
\switchcolumn[0]*%%%%%%%
In this case, the child component should expect a \texttt{title} prop
and emit an \texttt{update:title} event to update the parent value:
\switchcolumn
在这个例子中,子组件应声明一个 \texttt{title} prop,并通过触发
\texttt{update:title} 事件更新父组件值:
\switchcolumn[0]*%%%%%%%
\begin{codeHtml}
<!-- MyComponent.vue -->
<script setup>
defineProps(['title'])
defineEmits(['update:title'])
</script>
<template>
  <input
    type="text"
    :value="title"
    @input="$emit('update:title', $event.target.value)"
  />
</template>
\end{codeHtml}
\switchcolumn
\begin{codeHtml}
<!-- MyComponent.vue -->
<script setup>
defineProps(['title'])
defineEmits(['update:title'])
</script>
<template>
  <input
    type="text"
    :value="title"
    @input="$emit('update:title', $event.target.value)"
  />
</template>
\end{codeHtml}
\switchcolumn[0]*%%%%%%%
\href{https://play.vuejs.org/\#eNp9kE1rwzAMhv+KMIW00DXsGtKyMXYc7D7vEBplM8QfOHJoCfnvk+1QsjJ2svVKevRKk3h27jAGFJWoh7NXjmBACu4kjdLOeoIJPHYwQ+ethoJLi1vq7fpi+WfQ0JI+lCstcrkYQJqzNQMBKeoRjhG4LcYHbVvsofFfQUcCXhrteix20tRl9sIuOCBkvSHkCKD+fjxN04Ka57rkOOlrMwu7SlVHKdIrBZRcWpc3ntiLO7t/nKHFThl899YN248ikYpP9pj1V60o6sG1TMwDU/q/FZRxgeIPgK4uGcQLSZGlamz6sHKd1afUxOoGeeT298A9bHCMKxBfE3mTSNjl1vud5x8qNa76}{Try
it in the Playground}
\switchcolumn
\href{https://play.vuejs.org/\#eNp9kE1rwzAMhv+KMIW00DXsGtKyMXYc7D7vEBplM8QfOHJoCfnvk+1QsjJ2svVKevRKk3h27jAGFJWoh7NXjmBACu4kjdLOeoIJPHYwQ+ethoJLi1vq7fpi+WfQ0JI+lCstcrkYQJqzNQMBKeoRjhG4LcYHbVvsofFfQUcCXhrteix20tRl9sIuOCBkvSHkCKD+fjxN04Ka57rkOOlrMwu7SlVHKdIrBZRcWpc3ntiLO7t/nKHFThl899YN248ikYpP9pj1V60o6sG1TMwDU/q/FZRxgeIPgK4uGcQLSZGlamz6sHKd1afUxOoGeeT298A9bHCMKxBfE3mTSNjl1vud5x8qNa76}{在演练场中尝试一下}
\end{paracol}

\columnratio{0.55}
\begin{paracol}{2}
\switchcolumn[0]*%%%%%%%
\subsection{Multiple v-model bindings}
\switchcolumn
\subsection{多个 v-model 绑定}
\switchcolumn[0]*%%%%%%%
By leveraging the ability to target a particular prop and event as we
learned before with
\href{https://vuejs.org/guide/components/v-model.html\#v-model-arguments}{\texttt{v-model}
arguments}, we can now create multiple \texttt{v-model} bindings on a
single component instance.
\switchcolumn
利用刚才在
\href{https://cn.vuejs.org/guide/components/v-model.html\#v-model-arguments}{\texttt{v-model}
参数}小节中学到的指定参数与事件名的技巧,我们可以在单个组件实例上创建多个
\texttt{v-model} 双向绑定。
\switchcolumn[0]*%%%%%%%
Each \texttt{v-model} will sync to a different prop, without the need
for extra options in the component:
\switchcolumn
组件上的每一个 \texttt{v-model} 都会同步不同的 prop,而无需额外的选项:
\switchcolumn[0]*%%%%%%%
\begin{codeHtml}
<UserName
  v-model:first-name="first"
  v-model:last-name="last"
/>
\end{codeHtml}
\switchcolumn
\begin{codeHtml}
<UserName
  v-model:first-name="first"
  v-model:last-name="last"
/>
\end{codeHtml}
\switchcolumn[0]*%%%%%%%
\begin{codeHtml}
<script setup>
defineProps({
  firstName: String,
  lastName: String
})
defineEmits(['update:firstName', 'update:lastName'])
</script>
<template>
  <input
    type="text"
    :value="firstName"
    @input="$emit('update:firstName', $event.target.value)"
  />
  <input
    type="text"
    :value="lastName"
    @input="$emit('update:lastName', $event.target.value)"
  />
</template>
\end{codeHtml}
\switchcolumn
\begin{codeHtml}
<script setup>
defineProps({
  firstName: String,
  lastName: String
})
defineEmits(['update:firstName', 'update:lastName'])
</script>
<template>
  <input
    type="text"
    :value="firstName"
    @input="$emit('update:firstName', $event.target.value)"
  />
  <input
    type="text"
    :value="lastName"
    @input="$emit('update:lastName', $event.target.value)"
  />
</template>
\end{codeHtml}
\switchcolumn[0]*%%%%%%%
\href{https://play.vuejs.org/\#eNqNUc1qwzAMfhVjCk6hTdg1pGWD7bLDGIydlh1Cq7SGxDaOEjaC332yU6cdFNpLsPRJ348y8idj0qEHnvOi21lpkHWAvdmWSrZGW2Qjs1Azx2qrWyZoVMzQZwf2rWrhhKVZbHhGGivVTqsOWS0tfTeeKBGv+qjEMkJNdUaeNXigyCYjZIEKhNY0FQJVjBXHh+04nvicY/QOBM4VGUFhJHrwBWPDutV7aPKwslbU35Q8FCX/P+GJ4oB/T3hGpEU2m+ArfpnxytX2UEsF71abLhk9QxDzCzn7QCvVYeW7XuGyWSpH0eP6SyuxS75Eb/akOpn302LFYi8SiO8bJ5PK9DhFxV/j0yH8zOnzoWr6+SbhbifkMSwSsgByk1zzsoABFKZY2QNgGpiW57Pdrx2z3JCeI99Svvxh7g8muf2x}{Try
it in the Playground}
\switchcolumn
\href{https://play.vuejs.org/\#eNqNUc1qwzAMfhVjCk6hTdg1pGWD7bLDGIydlh1Cq7SGxDaOEjaC332yU6cdFNpLsPRJ348y8idj0qEHnvOi21lpkHWAvdmWSrZGW2Qjs1Azx2qrWyZoVMzQZwf2rWrhhKVZbHhGGivVTqsOWS0tfTeeKBGv+qjEMkJNdUaeNXigyCYjZIEKhNY0FQJVjBXHh+04nvicY/QOBM4VGUFhJHrwBWPDutV7aPKwslbU35Q8FCX/P+GJ4oB/T3hGpEU2m+ArfpnxytX2UEsF71abLhk9QxDzCzn7QCvVYeW7XuGyWSpH0eP6SyuxS75Eb/akOpn302LFYi8SiO8bJ5PK9DhFxV/j0yH8zOnzoWr6+SbhbifkMSwSsgByk1zzsoABFKZY2QNgGpiW57Pdrx2z3JCeI99Svvxh7g8muf2x}{在演练场中尝试一下}
\end{paracol}

\columnratio{0.55}
\begin{paracol}{2}

\switchcolumn[0]*%%%%%%%
\subsection{Handling v-model modifiers}
\switchcolumn
\subsection{处理 v-model 修饰符}
\switchcolumn[0]*%%%%%%%
When we were learning about form input bindings, we saw that
\texttt{v-model} has
\href{https://vuejs.org/guide/essentials/forms.html\#modifiers}{built-in
modifiers} - \texttt{.trim}, \texttt{.number} and \texttt{.lazy}. In
some cases, you might also want the \texttt{v-model} on your custom
input component to support custom modifiers.
\switchcolumn
在学习输入绑定时,我们知道了 \texttt{v-model}
有一些\href{https://cn.vuejs.org/guide/essentials/forms.html\#modifiers}{内置的修饰符},例如
\texttt{.trim},\texttt{.number} 和
\texttt{.lazy}。在某些场景下,你可能想要一个自定义组件的
\texttt{v-model} 支持自定义的修饰符。
\switchcolumn[0]*%%%%%%%
Let's create an example custom modifier, \texttt{capitalize}, that
capitalizes the first letter of the string provided by the
\texttt{v-model} binding:
\switchcolumn
我们来创建一个自定义的修饰符 \texttt{capitalize},它会自动将
\texttt{v-model} 绑定输入的字符串值第一个字母转为大写:
\switchcolumn[0]*%%%%%%%
\begin{codeHtml}
<MyComponent v-model.capitalize="myText" />
\end{codeHtml}
\switchcolumn
\begin{codeHtml}
<MyComponent v-model.capitalize="myText" />
\end{codeHtml}
\switchcolumn[0]*%%%%%%%
Modifiers added to a component \texttt{v-model} will be provided to the
component via the \texttt{modelModifiers} prop. In the below example, we
have created a component that contains a \texttt{modelModifiers} prop
that defaults to an empty object:
\switchcolumn
组件的 \texttt{v-model} 上所添加的修饰符,可以通过
\texttt{modelModifiers} prop 在组件内访问到。在下面的组件中,我们声明了
\texttt{modelModifiers} 这个 prop,它的默认值是一个空对象:
\switchcolumn[0]*%%%%%%%
\begin{codeHtml}
<script setup>
const props = defineProps({
  modelValue: String,
  modelModifiers: { default: () => ({}) }
})
defineEmits(['update:modelValue'])
console.log(props.modelModifiers) // { capitalize: true }
</script>
<template>
  <input
    type="text"
    :value="modelValue"
    @input="$emit('update:modelValue', $event.target.value)"
  />
</template>
\end{codeHtml}
\switchcolumn
\begin{codeHtml}
<script setup>
const props = defineProps({
  modelValue: String,
  modelModifiers: { default: () => ({}) }
})
defineEmits(['update:modelValue'])
console.log(props.modelModifiers) // { capitalize: true }
</script>
<template>
  <input
    type="text"
    :value="modelValue"
    @input="$emit('update:modelValue', $event.target.value)"
  />
</template>
\end{codeHtml}
\switchcolumn[0]*%%%%%%%
Notice the component's \texttt{modelModifiers} prop contains
\texttt{capitalize} and its value is \texttt{true} - due to it being set
on the \texttt{v-model} binding \texttt{v-model.capitalize="myText"}.
\switchcolumn
注意这里组件的 \texttt{modelModifiers} prop 包含了 \texttt{capitalize}
且其值为 \texttt{true},因为它在模板中的 \texttt{v-model} 绑定
\texttt{v-model.capitalize="myText"} 上被使用了。
\switchcolumn[0]*%%%%%%%
Now that we have our prop set up, we can check the
\texttt{modelModifiers} object keys and write a handler to change the
emitted value. In the code below we will capitalize the string whenever
the \texttt{\textless{}input\ /\textgreater{}} element fires an
\texttt{input} event.
\switchcolumn
有了这个 prop,我们就可以检查 \texttt{modelModifiers}
对象的键,并编写一个处理函数来改变抛出的值。在下面的代码里,我们就是在每次
\texttt{\textless{}input\ /\textgreater{}} 元素触发 \texttt{input}
事件时将值的首字母大写:
\switchcolumn[0]*%%%%%%%
\begin{codeHtml}
<script setup>
const props = defineProps({
  modelValue: String,
  modelModifiers: { default: () => ({}) }
})
const emit = defineEmits(['update:modelValue'])
function emitValue(e) {
  let value = e.target.value
  if (props.modelModifiers.capitalize) {
    value = value.charAt(0).toUpperCase() + value.slice(1)
  }
  emit('update:modelValue', value)
}
</script>
<template>
  <input type="text" :value="modelValue" @input="emitValue" />
</template>
\end{codeHtml}
\switchcolumn
\begin{codeHtml}
<script setup>
const props = defineProps({
  modelValue: String,
  modelModifiers: { default: () => ({}) }
})
const emit = defineEmits(['update:modelValue'])
function emitValue(e) {
  let value = e.target.value
  if (props.modelModifiers.capitalize) {
    value = value.charAt(0).toUpperCase() + value.slice(1)
  }
  emit('update:modelValue', value)
}
</script>
<template>
  <input type="text" :value="modelValue" @input="emitValue" />
</template>
\end{codeHtml}
\switchcolumn[0]*%%%%%%%
\href{https://play.vuejs.org/\#eNp9Us1Og0AQfpUJF5ZYqV4JNTaNxyYmVi/igdCh3QR2N7tDIza8u7NLpdU0nmB+v5/ZY7Q0Jj10GGVR7iorDYFD6sxDoWRrtCU4gsUaBqitbiHm1ngqrfuV5j+Fik7ldH6R83u5GaBQlVaOoO03+Emw8BtFHCeFyucjKMNxQNiapiTkCGCzlw6kMh1BVRpJZSO/0AEe0Pa0l2oHve6AYdBmvj+/ZHO4bfUWm/Q8uSiiEb6IYM4A+XxCi2bRH9ZX3BgVGKuNYwFbrKXCZx+Jo0cPcG9l02EGL2SZ3mxKr/VW1hKty9hMniy7hjIQCSweQByHBIZCDWzGDwi20ps0Yjxx4MR73Jktc83OOPFHGKk7VZHUKkyFgsAEAqcG2Qif4WWYUml3yOp8wldlDSLISX+TvPDstAemLeGbVvvSLkncJSnpV2PQrkqHLOfmVHeNrFDcMz3w0iBQE1cUzMYBbuS2f55CPj4D6o0/I41HzMKsP+u0kLOPoZWzkx1X7j18A8s0DEY=}{Try
it in the Playground}
\switchcolumn
\href{https://play.vuejs.org/\#eNp9Us1Og0AQfpUJF5ZYqV4JNTaNxyYmVi/igdCh3QR2N7tDIza8u7NLpdU0nmB+v5/ZY7Q0Jj10GGVR7iorDYFD6sxDoWRrtCU4gsUaBqitbiHm1ngqrfuV5j+Fik7ldH6R83u5GaBQlVaOoO03+Emw8BtFHCeFyucjKMNxQNiapiTkCGCzlw6kMh1BVRpJZSO/0AEe0Pa0l2oHve6AYdBmvj+/ZHO4bfUWm/Q8uSiiEb6IYM4A+XxCi2bRH9ZX3BgVGKuNYwFbrKXCZx+Jo0cPcG9l02EGL2SZ3mxKr/VW1hKty9hMniy7hjIQCSweQByHBIZCDWzGDwi20ps0Yjxx4MR73Jktc83OOPFHGKk7VZHUKkyFgsAEAqcG2Qif4WWYUml3yOp8wldlDSLISX+TvPDstAemLeGbVvvSLkncJSnpV2PQrkqHLOfmVHeNrFDcMz3w0iBQE1cUzMYBbuS2f55CPj4D6o0/I41HzMKsP+u0kLOPoZWzkx1X7j18A8s0DEY=}{在演练场中尝试一下}
\end{paracol}

\columnratio{0.55}
\begin{paracol}{2}

\switchcolumn[0]*%%%%%%%
\subsubsection{Modifiers for v-model with arguments}
\switchcolumn
\subsubsection{带参数的 v-model 修饰符}
\switchcolumn[0]*%%%%%%%
For \texttt{v-model} bindings with both argument and modifiers, the
generated prop name will be \texttt{arg\ +\ "Modifiers"}. For example:
\switchcolumn
对于又有参数又有修饰符的 \texttt{v-model} 绑定,生成的 prop 名将是
\texttt{arg\ +\ "Modifiers"}。举例来说:
\switchcolumn[0]*%%%%%%%
\begin{codeHtml}
<MyComponent v-model:title.capitalize="myText">
\end{codeHtml}
\switchcolumn
\begin{codeHtml}
<MyComponent v-model:title.capitalize="myText">
\end{codeHtml}
\switchcolumn[0]*%%%%%%%
The corresponding declarations should be:
\switchcolumn
相应的声明应该是:
\switchcolumn[0]*%%%%%%%
\begin{codeJs}
const props = defineProps(['title', 'titleModifiers'])
defineEmits(['update:title'])
console.log(props.titleModifiers) // { capitalize: true }
\end{codeJs}
\switchcolumn
\begin{codeJs}
const props = defineProps(['title', 'titleModifiers'])
defineEmits(['update:title'])
console.log(props.titleModifiers) // { capitalize: true }
\end{codeJs}
\switchcolumn[0]*%%%%%%%
Here's another example of using modifiers with multiple \texttt{v-model}
with different arguments:
\switchcolumn
这里是另一个例子,展示了如何在使用多个不同参数的 \texttt{v-model}
时使用修饰符:
\switchcolumn[0]*%%%%%%%
\begin{codeHtml}
<UserName
  v-model:first-name.capitalize="first"
  v-model:last-name.uppercase="last"
/>
\end{codeHtml}
\switchcolumn
\begin{codeHtml}
<UserName
  v-model:first-name.capitalize="first"
  v-model:last-name.uppercase="last"
/>
\end{codeHtml}
\switchcolumn[0]*%%%%%%%
\begin{codeHtml}
<script setup>
const props = defineProps({
  firstName: String,
  lastName: String,
  firstNameModifiers: { default: () => ({}) },
  lastNameModifiers: { default: () => ({}) }
})
defineEmits(['update:firstName', 'update:lastName'])
console.log(props.firstNameModifiers) // { capitalize: true }
console.log(props.lastNameModifiers) // { uppercase: true}
</script>
\end{codeHtml}
\switchcolumn
\begin{codeHtml}
<script setup>
const props = defineProps({
  firstName: String,
  lastName: String,
  firstNameModifiers: { default: () => ({}) },
  lastNameModifiers: { default: () => ({}) }
})
defineEmits(['update:firstName', 'update:lastName'])
console.log(props.firstNameModifiers) // { capitalize: true }
console.log(props.lastNameModifiers) // { uppercase: true}
</script>
\end{codeHtml}
\end{paracol} 