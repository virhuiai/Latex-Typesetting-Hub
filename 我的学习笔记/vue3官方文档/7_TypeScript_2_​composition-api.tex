\columnratio{0.55}
\begin{paracol}{2} 
 
\switchcolumn[0]*%%%%%%%
\section{TypeScript with Composition API}
\switchcolumn
\section{TypeScript 与组合式 API}
\switchcolumn[0]*%%%%%%%
\begin{quote}
This page assumes you've already read the overview on
\href{https://vuejs.org/guide/typescript/overview.html}{Using Vue with
TypeScript}.
\end{quote}
\switchcolumn
\begin{quote}
这一章假设你已经阅读了\href{https://cn.vuejs.org/guide/typescript/overview.html}{搭配
TypeScript 使用 Vue} 的概览。
\end{quote}
\end{paracol}


\columnratio{0.55}
\begin{paracol}{2} 
 
\switchcolumn[0]*%%%%%%%
\subsection{Typing Component Props}
\switchcolumn
\subsection{为组件的 props 标注类型}
\switchcolumn[0]*%%%%%%%
\subsubsection{Using \textless script setup\textgreater{}}
\switchcolumn
\subsubsection{使用 \textless script setup\textgreater{}}
\switchcolumn[0]*%%%%%%%
When using \texttt{\textless{}script\ setup\textgreater{}}, the
\texttt{defineProps()} macro supports inferring the props types based on
its argument:
\switchcolumn
当使用 \texttt{\textless{}script\ setup\textgreater{}}
时,\texttt{defineProps()} 宏函数支持从它的参数中推导类型:
\switchcolumn[0]*%%%%%%%
\begin{codeHtml}
<script setup lang="ts">
const props = defineProps({
  foo: { type: String, required: true },
  bar: Number
})
props.foo // string
props.bar // number | undefined
</script>
\end{codeHtml}
\switchcolumn
\begin{codeHtml}
<script setup lang="ts">
const props = defineProps({
  foo: { type: String, required: true },
  bar: Number
})
props.foo // string
props.bar // number | undefined
</script>
\end{codeHtml}
\switchcolumn[0]*%%%%%%%
This is called "runtime declaration", because the argument passed to
\texttt{defineProps()} will be used as the runtime \texttt{props}
option.
\switchcolumn
这被称之为``运行时声明'',因为传递给 \texttt{defineProps()}
的参数会作为运行时的 \texttt{props} 选项使用。
\switchcolumn[0]*%%%%%%%
However, it is usually more straightforward to define props with pure
types via a generic type argument:
\switchcolumn
然而,通过泛型参数来定义 props 的类型通常更直接:
\switchcolumn[0]*%%%%%%%
\begin{codeHtml}
<script setup lang="ts">
const props = defineProps<{
  foo: string
  bar?: number
}>()
</script>
\end{codeHtml}
\switchcolumn
\begin{codeHtml}
<script setup lang="ts">
const props = defineProps<{
  foo: string
  bar?: number
}>()
</script>
\end{codeHtml}
\switchcolumn[0]*%%%%%%%
This is called "type-based declaration". The compiler will try to do its
best to infer the equivalent runtime options based on the type argument.
In this case, our second example compiles into the exact same runtime
options as the first example.
\switchcolumn
这被称之为``基于类型的声明''。编译器会尽可能地尝试根据类型参数推导出等价的运行时选项。在这种场景下,我们第二个例子中编译出的运行时选项和第一个是完全一致的。
\switchcolumn[0]*%%%%%%%
You can use either type-based declaration OR runtime declaration, but
you cannot use both at the same time.
\switchcolumn
基于类型的声明或者运行时声明可以择一使用,但是不能同时使用。
\switchcolumn[0]*%%%%%%%
We can also move the props types into a separate interface:
\switchcolumn
我们也可以将 props 的类型移入一个单独的接口中:
\switchcolumn[0]*%%%%%%%
\begin{codeHtml}
<script setup lang="ts">
interface Props {
  foo: string
  bar?: number
}
const props = defineProps<Props>()
</script>
\end{codeHtml}
\switchcolumn
\begin{codeHtml}
<script setup lang="ts">
interface Props {
  foo: string
  bar?: number
}
const props = defineProps<Props>()
</script>
\end{codeHtml}
\switchcolumn[0]*%%%%%%%
\paragraph{Syntax Limitations}
\switchcolumn
\paragraph{语法限制}
\switchcolumn[0]*%%%%%%%
In version 3.2 and below, the generic type parameter for
\texttt{defineProps()} were limited to a type literal or a reference to
a local interface.
\switchcolumn
在 3.2 及以下版本中,\texttt{defineProps()}
的泛型类型参数仅限于类型文字或对本地接口的引用。
\end{paracol}



\columnratio{0.55}
\begin{paracol}{2} 
 
\switchcolumn[0]*%%%%%%%
This limitation has been resolved in 3.3. The latest version of Vue
supports referencing imported and a limited set of complex types in the
type parameter position. However, because the type to runtime conversion
is still AST-based, some complex types that require actual type
analysis, e.g. conditional types, are not supported. You can use
conditional types for the type of a single prop, but not the entire
props object.
\switchcolumn
这个限制在 3.3 中得到了解决。最新版本的 Vue
支持在类型参数位置引用导入和有限的复杂类型。但是,由于类型到运行时转换仍然基于
AST,一些需要实际类型分析的复杂类型,例如条件类型,还未支持。您可以使用条件类型来指定单个
prop 的类型,但不能用于整个 props 对象的类型。
\switchcolumn[0]*%%%%%%%
\subsubsection{Props Default Values}
\switchcolumn
\subsubsection{Props 解构默认值}
\switchcolumn[0]*%%%%%%%
When using type-based declaration, we lose the ability to declare
default values for the props. This can be resolved by the
\texttt{withDefaults} compiler macro:
\begin{codeTs}
export interface Props {
  msg?: string
  labels?: string[]
}
const props = withDefaults(defineProps<Props>(), {
  msg: 'hello',
  labels: () => ['one', 'two'] 
})
\end{codeTs}
\switchcolumn
当使用基于类型的声明时,我们失去了为 props 声明默认值的能力。这可以通过
\texttt{withDefaults} 编译器宏解决:
\begin{codeTs}
    export interface Props {
      msg?: string
      labels?: string[]
    }
    const props = withDefaults(defineProps<Props>(), {
      msg: 'hello',
      labels: () => ['one', 'two']
    })
    \end{codeTs}
\end{paracol}



\columnratio{0.55}
\begin{paracol}{2} 

\end{paracol}


\columnratio{0.55}
\begin{paracol}{2} 

\end{paracol}



\columnratio{0.55}
\begin{paracol}{2} 

\end{paracol}