\chapter{Boot, the Fancy Clojure Build Framework}

\columnratio{0.55}
\begin{paracol}{2}
\switchcolumn[0]*
Boot is an alternative to Leiningen that provides the same
functionality. Leiningen's more popular (as of the summer of 2015), but
I personally like to work with Boot because it's easier to extend. This
appendix explains Boot's underlying concepts and guides you through
writing your first Boot tasks. If you're interested in using Boot to
build projects right this second, check out its GitHub README
(\emph{https://github.com/boot-clj/boot/}) and its wiki
(\emph{https://github.com/boot-clj/boot/wiki/}).
\switchcolumn
Boot是Leiningen的替代品,提供了相同的功能。虽然Leiningen更受欢迎(截至2015年夏季),但我个人更喜欢使用Boot,因为它更容易扩展。本附录解释了Boot的基本概念,并指导您编写第一个Boot任务。如果您有兴趣立即使用Boot构建项目,请查看其GitHub README(https://github.com/boot-clj/boot/)和其Wiki(https://github.com/boot-clj/boot/wiki/)。
\switchcolumn[0]*
\begin{quote}
\textbf{NOTE}:
\emph{As of this writing, Boot has limited support for Windows. The Boot
team welcomes contributions!}
\end{quote}
\switchcolumn
\begin{quote}
注意:截至本文写作时,Boot对Windows的支持有限。Boot团队欢迎贡献!
\end{quote}
\switchcolumn[0]*
\section{Boot's Abstractions}
\switchcolumn
\section{Boot的抽象}
\switchcolumn[0]*
Created by Micha Niskin and Alan Dipert, Boot is a fun and powerful
addition to the Clojure tooling landscape. On the surface, it's a
convenient way to build Clojure applications and run Clojure tasks from
the command line. Dig a little deeper and you'll see that Boot is like
the Lisped-up lovechild of Git and Unix in that it provides abstractions
that make it more pleasant to write code that exists at the intersection
of your operating system and your application.
\switchcolumn
由Micha Niskin和Alan Dipert创建,Boot是Clojure工具生态系统中有趣且强大的补充。从表面上看,它是一种方便的方式来构建Clojure应用程序并从命令行运行Clojure任务。深入挖掘,您会发现Boot就像Git和Unix的结合体,它提供了一些抽象,使得编写处于操作系统和应用程序交集处的代码更加愉快。
\switchcolumn[0]*
Unix provides abstractions that we're all familiar with to the point
where we take them for granted. (Would it kill you to take your computer
out to a nice restaurant once in a while?) The process abstraction lets
you reason about programs as isolated units of logic that can be easily
composed into a stream-processing pipeline through the STDIN and STDOUT
file descriptors. These abstractions make certain kinds of operations,
like text processing, very straightforward.
\switchcolumn
Unix提供了我们都熟悉的抽象,以至于我们认为它们理所当然。 (难道你不能偶尔带电脑去好餐厅吗?)进程抽象使您可以将程序视为独立的逻辑单元,通过STDIN和STDOUT文件描述符轻松组合成流处理管道。这些抽象使得某些操作(如文本处理)变得非常简单。
\switchcolumn[0]*
Similarly, Boot provides abstractions that make it easy to compose
independent operations into the kinds of complex, coordinated operations
that build tools end up doing, like converting ClojureScript into
JavaScript. Boot's task abstraction lets you easily define units of
logic that communicate through \emph{filesets}. The fileset abstraction
keeps track of the evolving build context and provides a well-defined,
reliable method of task coordination.
\switchcolumn
类似地,Boot提供了一些抽象,使得将独立操作组合成构建工具常常需要的复杂协调操作变得容易,比如将ClojureScript转换为JavaScript。Boot的任务抽象使您可以轻松定义通过文件集相互通信的逻辑单元。文件集抽象跟踪不断演变的构建上下文,并提供了一种明确定义的、可靠的任务协调方法。
\switchcolumn[0]*
That's a lot of high-level description, which hopefully has hooked your
attention. But I would be ashamed to leave you with a plateful of
metaphors. Oh no, dear reader, that was only the appetizer. Throughout
the rest of this appendix, you'll learn how to build your own Boot
tasks. Along the way, you'll discover that build tools can actually have
a conceptual foundation.
\switchcolumn
这是很多高层描述,希望能吸引您的注意。但是,如果我只给你上了一盘隐喻,我会感到羞愧的。不,亲爱的读者,那只是开胃菜。在本附录的其余部分,您将学习如何构建自己的Boot任务。在此过程中,您将发现构建工具实际上可以有一个概念基础。
\switchcolumn[0]*
\section{Tasks}
\switchcolumn
\section{任务}
\switchcolumn[0]*
Like make, rake, grunt, and other build tools of yore, Boot lets you
define tasks. \emph{Tasks} are named operations that take command line
options dispatched by some intermediary program (make, rake, Boot).
\switchcolumn
像make、rake、grunt和其他早期的构建工具一样,Boot允许您定义任务。任务是命名的操作,它们接受由某个中介程序(如make、rake、Boot)分派的命令行选项。
\switchcolumn[0]*
Boot provides the dispatching program, \emph{boot}, and a Clojure
library that makes it easy for you to define named operations and their
command line options with the deftask macro. To see what all the fuss is
about, let's create your first task. Normally, programming tutorials
encourage you to write code that prints ``Hello World,'' but I like my
examples to have real-world utility, so your task is to print ``My pants
are on fire!'' This information is objectively more useful. First,
install Boot; then create a new directory named \emph{boot-walkthrough},
navigate to that directory, create a file named \emph{build.boot,} and
write this:
\begin{verbatim}
(deftask fire
    "Prints 'My pants are on fire!'"
    []
    (println "My pants are on fire!"))
\end{verbatim}
\switchcolumn
Boot提供了分派程序boot和一个Clojure库,使用deftask宏可以轻松定义命名操作及其命令行选项。为了了解到底是怎么回事,让我们创建您的第一个任务。通常,编程教程鼓励您编写打印“Hello World”的代码,但我喜欢我的示例具有实际的用途,所以您的任务是打印“我的裤子着火了!”这个信息是客观上更有用的。首先,安装Boot;然后创建一个名为boot-walkthrough的新目录,进入该目录,创建一个名为build.boot的文件,并编写以下内容:
\begin{verbatim}
(deftask fire
    "Prints 'My pants are on fire!'"
    []
    (println "My pants are on fire!"))
\end{verbatim}
\switchcolumn[0]*
Now run this task from the command line with boot fire; you should see
the message you wrote printed to your terminal. This task demonstrates
two out of the three task components: the task is named (fire), and it's
dispatched by boot. This is super cool. You've essentially created a
Clojure shell script, stand-alone Clojure code that you can run from the
command line with ease. No \emph{project.clj}, directory structure, or
namespaces needed!
\switchcolumn
现在在命令行中使用boot fire运行此任务;您应该看到您编写的消息打印到终端上。这个任务演示了三个任务组件中的两个:任务有一个名称(fire),并且由boot分派。这非常酷。您实际上创建了一个Clojure shell脚本,一个独立的Clojure代码,可以轻松地从命令行运行。不需要project.clj、目录结构或命名空间!
\switchcolumn[0]*
Let's extend the example to demonstrate how you'd write command line
options:
\begin{verbatim}
(deftask fire
    "Announces that something is on fire"
    [t thing     THING str "The thing that's on fire"
    p pluralize       bool "Whether to pluralize"]
    (let [verb (if pluralize "are" "is")]
    (println "My" thing verb "on fire!")))
\end{verbatim}
\switchcolumn
让我们扩展示例以演示如何编写命令行选项:
\begin{verbatim}
(deftask fire
    "Announces that something is on fire"
    [t thing     THING str "The thing that's on fire"
    p pluralize       bool "Whether to pluralize"]
    (let [verb (if pluralize "are" "is")]
    (println "My" thing verb "on fire!")))
\end{verbatim}
\switchcolumn[0]*
Try running the task like so:
\begin{verbatim}
boot fire -t heart
# => My heart is on fire!


boot fire -t logs -p
# => My logs are on fire!
\end{verbatim}
\switchcolumn
尝试以以下方式运行任务:
\begin{verbatim}
boot fire -t heart
# => My heart is on fire!


boot fire -t logs -p
# => My logs are on fire!
\end{verbatim}
\switchcolumn[0]*
In the first instance, either you're newly in love or you need to be
rushed to the emergency room. In the second, you are a Boy Scout
awkwardly expressing your excitement over meeting the requirements for a
merit badge. In both instances, you were able to easily specify options
for the task.
\switchcolumn
在第一个示例中,要么您新恋爱了,要么您需要赶紧去急诊室。在第二个示例中,您是一个笨拙地表达自己对满足获得勋章要求的兴奋的童子军。在这两种情况下,您都可以轻松地为任务指定选项。
\switchcolumn[0]*
This refinement of the fire task introduced two command line options,
thing and pluralize. Both options are defined using a
\emph{domain-specific language (DSL)}. DSLs are their own topic, but
briefly, the term refers to mini-languages that you can use within a
larger program to write compact, expressive code for narrowly defined
domains (like defining options).
\switchcolumn
这个改进的fire任务引入了两个命令行选项thing和pluralize。这两个选项使用特定领域语言(DSL)进行定义。DSL是一个独立的主题,但简而言之,该术语指的是您可以在较大程序中使用的迷你语言,用于在狭义定义的领域(如定义选项)中编写紧凑、表达力强的代码。
\switchcolumn[0]*
In the option thing, t specifies its short name, and thing specifies its
long name. THING is a bit complicated, and I'll get to it in a second.
str specifies the option's type, and Boot uses that to validate the
argument and convert it. "The thing that's on fire" is the documentation
for the option. You can view a task's documentation in the terminal with
boot task-name -h:
\begin{verbatim}
boot fire -h
# Announces that something is on fire
#
# Options:
#   -h, --help        Print this help info.
#   -t, --thing THING Set the thing that's on fire to THING.
#   -p, --pluralize   Whether to pluralize
\end{verbatim}
\switchcolumn
在选项thing中,t指定了其短名称,thing指定了其长名称。THING有点复杂,我一会儿会解释。str指定了选项的类型,Boot使用它来验证参数并进行转换。"着火的东西"是该选项的文档。您可以在终端中使用boot任务名称 -h查看任务的文档。
\begin{verbatim}
boot fire -h
# Announces that something is on fire
#
# Options:
#   -h, --help        Print this help info.
#   -t, --thing THING Set the thing that's on fire to THING.
#   -p, --pluralize   Whether to pluralize
\end{verbatim}
\switchcolumn[0]*
Pretty groovy! Boot makes it very easy to write code that's meant to be
invoked from the command line.
\switchcolumn
非常棒!Boot使得编写可以从命令行调用的代码变得非常容易。
\switchcolumn[0]*
Now, let's look at THING. THING is an \emph{optarg}, and it indicates
that this option expects an argument. You don't have to include an
optarg when you're defining an option (notice that the pluralize option
has no optarg). The optarg doesn't have to correspond to the full name
of the option; you could replace THING with BILLY\_JOEL or whatever you
want and the task would work the same. You can also designate complex
options using the optarg. (Visit
\emph{https://github.com/boot-clj/boot/wiki/Task-Options-DSL\#complex-options}
for Boot's documentation on the subject.) Basically, complex options
allow you to specify that option arguments should be treated as maps,
sets, vectors, or even nested collections. It's pretty powerful.
\switchcolumn
现在,让我们来看看THING。THING是一个\emph{optarg},它表示该选项需要一个参数。在定义选项时,你不必包含optarg(注意到复数化选项没有optarg)。optarg不必对应于选项的全名;你可以将THING替换为BILLY\_JOEL或其他任何你想要的名称,任务将工作得一样好。你还可以使用optarg来指定复杂的选项。(请访问
\emph{https://github.com/boot-clj/boot/wiki/Task-Options-DSL\#complex-options}
,查看关于Boot的文档。)基本上,复杂选项允许你将选项参数处理为映射、集合、向量,甚至是嵌套集合。这非常强大。
\switchcolumn[0]*
Boot provides you with all the tools you could ask for to build command
line interfaces with Clojure. And you've only just started learning
about it!
\switchcolumn
Boot为你提供了构建Clojure命令行界面所需的所有工具。而你现在只是刚刚开始学习它!
\switchcolumn[0]*
\section{REPL}
\switchcolumn
\section{REPL}
\switchcolumn[0]*
Boot comes with a number of useful built-in tasks, including a REPL
task. Run boot repl to fire up that puppy. The Boot REPL is similar to
Leiningen's in that it handles loading your project code so you can play
around with it. You might not think this applies to the project you've
been writing because you've only written tasks, but you can actually run
tasks in the REPL (I've omitted the boot.user=\textgreater{} prompt).
You can specify options using a string:
\switchcolumn
Boot提供了许多有用的内置任务,包括一个REPL任务。运行boot repl来启动这个任务。Boot REPL与Leiningen的REPL类似,它会加载你的项目代码,这样你就可以对其进行操作。你可能认为这与你正在编写的项目无关,因为你只编写了任务,但实际上你可以在REPL中运行任务(我省略了boot.user=\textgreater{}提示符)。你可以使用字符串来指定选项:
\switchcolumn[0]*
\begin{verbatim}
(fire "-t" "NBA Jam guy")
; My NBA Jam guy is on fire!
; => nil
\end{verbatim}
\switchcolumn
\begin{verbatim}
(fire "-t" "NBA Jam guy")
; My NBA Jam guy is on fire!
; => nil
\end{verbatim}
\switchcolumn[0]*
Notice that the option's value comes right after the option.

You can also specify an option using a keyword:
\switchcolumn
注意选项的值紧跟在选项后面。

你还可以使用关键字来指定选项:
\switchcolumn[0]*
\begin{verbatim}
(fire :thing "NBA Jam guy")
; My NBA Jam guy is on fire!
; => nil
\end{verbatim}
\switchcolumn
\begin{verbatim}
(fire :thing "NBA Jam guy")
; My NBA Jam guy is on fire!
; => nil
\end{verbatim}
\switchcolumn[0]*
You can also combine options:
\switchcolumn
你还可以组合选项:
\switchcolumn[0]*
\begin{verbatim}
(fire "-p" "-t" "NBA Jam guys")
; My NBA Jam guys are on fire!
; => nil

(fire :pluralize true :thing "NBA Jam guys")
; My NBA Jam guys are on fire!
; => nil
\end{verbatim}
\switchcolumn
\begin{verbatim}
(fire "-p" "-t" "NBA Jam guys")
; My NBA Jam guys are on fire!
; => nil

(fire :pluralize true :thing "NBA Jam guys")
; My NBA Jam guys are on fire!
; => nil
\end{verbatim}
\switchcolumn[0]*
And of course, you can use deftask in the REPL as well---it's just
Clojure, after all. The takeaway is that Boot lets you interact with
your tasks as Clojure functions, because that's what they are.
\switchcolumn
当然,你也可以在REPL中使用deftask——毕竟它就是Clojure。关键是,Boot允许你将任务作为Clojure函数与之交互,因为它们本质上就是函数。
\switchcolumn[0]*
\section{Composition and Coordination}
\switchcolumn
\section{组合和协调}
\switchcolumn[0]*
If what you've seen so far was all that Boot had to offer, it'd be a
pretty swell tool, but it wouldn't be very different from other build
tools. One feature that sets Boot apart is how it lets you compose
tasks. For comparison's sake, here's an example Rake invocation (Rake is
the premier Ruby build tool):
\switchcolumn
如果到目前为止你所见到的就是Boot所提供的全部功能,那么它将是一个非常棒的工具,但并不与其他构建工具有多大区别。Boot的一个与众不同之处在于它允许你组合任务。为了进行比较,这里是一个Rake调用的例子(Rake是最好的Ruby构建工具):
\switchcolumn[0]*
\begin{verbatim}
rake db:create db:migrate db:seed
\end{verbatim}
\switchcolumn
\begin{verbatim}
rake db:create db:migrate db:seed
\end{verbatim}
\switchcolumn[0]*
This code will create a database, run migrations on it, and populate it
with seed data when run in a Rails project. However, worth noting is
that Rake doesn't provide any way for these tasks to communicate with
each other. Specifying multiple tasks is just a convenience, saving you
from having to run rake db:create; rake db:migrate; rake db:seed. If you
want to access the result of Task A within Task B, the build tool
doesn't help you; you have to manage that coordination yourself.
Usually, you'll do this by shoving the result of Task A into a special
place on the filesystem and then making sure Task B reads that special
place. This looks like programming with mutable, global variables, and
it's just as brittle.
\switchcolumn
当在Rails项目中运行时,此代码将创建一个数据库,在其上运行迁移,并用种子数据填充它。然而值得注意的是,Rake不提供任何方式让这些任务相互通信。指定多个任务只是一种方便,省去了运行rake db:create; rake db:migrate; rake db:seed的麻烦。如果你想在任务A中访问任务A的结果,构建工具并没有帮助你;你必须自己管理协调。通常,你会将任务A的结果放入文件系统的一个特殊位置,然后确保任务B读取该特殊位置。这看起来就像是使用可变的全局变量进行编程,同样脆弱。
\switchcolumn[0]*

\switchcolumn


\switchcolumn[0]*

\switchcolumn


\switchcolumn[0]*

\switchcolumn

\switchcolumn[0]*

\switchcolumn


\switchcolumn[0]*

\switchcolumn

\switchcolumn[0]*

\switchcolumn


\switchcolumn[0]*

\switchcolumn

\switchcolumn[0]*

\switchcolumn


\switchcolumn[0]*

\switchcolumn

\end{paracol}

\end{document}

