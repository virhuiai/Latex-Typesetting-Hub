\chapter{Boot, the Fancy Clojure Build Framework}

\columnratio{0.55}
\begin{paracol}{2}
\switchcolumn[0]*
Boot is an alternative to Leiningen that provides the same
functionality. Leiningen's more popular (as of the summer of 2015), but
I personally like to work with Boot because it's easier to extend. This
appendix explains Boot's underlying concepts and guides you through
writing your first Boot tasks. If you're interested in using Boot to
build projects right this second, check out its GitHub README
(\emph{https://github.com/boot-clj/boot/}) and its wiki
(\emph{https://github.com/boot-clj/boot/wiki/}).
\switchcolumn
Boot是Leiningen的替代品,提供了相同的功能。虽然Leiningen更受欢迎(截至2015年夏季),但我个人更喜欢使用Boot,因为它更容易扩展。本附录解释了Boot的基本概念,并指导您编写第一个Boot任务。如果您有兴趣立即使用Boot构建项目,请查看其GitHub README(https://github.com/boot-clj/boot/)和其Wiki(https://github.com/boot-clj/boot/wiki/)。
\switchcolumn[0]*
\begin{quote}
\textbf{NOTE}:
\emph{As of this writing, Boot has limited support for Windows. The Boot
team welcomes contributions!}
\end{quote}
\switchcolumn
\begin{quote}
注意:截至本文写作时,Boot对Windows的支持有限。Boot团队欢迎贡献!
\end{quote}
\switchcolumn[0]*
\section{Boot's Abstractions}
\switchcolumn
\section{Boot的抽象}
\switchcolumn[0]*
Created by Micha Niskin and Alan Dipert, Boot is a fun and powerful
addition to the Clojure tooling landscape. On the surface, it's a
convenient way to build Clojure applications and run Clojure tasks from
the command line. Dig a little deeper and you'll see that Boot is like
the Lisped-up lovechild of Git and Unix in that it provides abstractions
that make it more pleasant to write code that exists at the intersection
of your operating system and your application.
\switchcolumn
由Micha Niskin和Alan Dipert创建,Boot是Clojure工具生态系统中有趣且强大的补充。从表面上看,它是一种方便的方式来构建Clojure应用程序并从命令行运行Clojure任务。深入挖掘,您会发现Boot就像Git和Unix的结合体,它提供了一些抽象,使得编写处于操作系统和应用程序交集处的代码更加愉快。
\switchcolumn[0]*
Unix provides abstractions that we're all familiar with to the point
where we take them for granted. (Would it kill you to take your computer
out to a nice restaurant once in a while?) The process abstraction lets
you reason about programs as isolated units of logic that can be easily
composed into a stream-processing pipeline through the STDIN and STDOUT
file descriptors. These abstractions make certain kinds of operations,
like text processing, very straightforward.
\switchcolumn
Unix提供了我们都熟悉的抽象,以至于我们认为它们理所当然。 (难道你不能偶尔带电脑去好餐厅吗?)进程抽象使您可以将程序视为独立的逻辑单元,通过STDIN和STDOUT文件描述符轻松组合成流处理管道。这些抽象使得某些操作(如文本处理)变得非常简单。
\switchcolumn[0]*
Similarly, Boot provides abstractions that make it easy to compose
independent operations into the kinds of complex, coordinated operations
that build tools end up doing, like converting ClojureScript into
JavaScript. Boot's task abstraction lets you easily define units of
logic that communicate through \emph{filesets}. The fileset abstraction
keeps track of the evolving build context and provides a well-defined,
reliable method of task coordination.
\switchcolumn
类似地,Boot提供了一些抽象,使得将独立操作组合成构建工具常常需要的复杂协调操作变得容易,比如将ClojureScript转换为JavaScript。Boot的任务抽象使您可以轻松定义通过文件集相互通信的逻辑单元。文件集抽象跟踪不断演变的构建上下文,并提供了一种明确定义的、可靠的任务协调方法。
\switchcolumn[0]*
That's a lot of high-level description, which hopefully has hooked your
attention. But I would be ashamed to leave you with a plateful of
metaphors. Oh no, dear reader, that was only the appetizer. Throughout
the rest of this appendix, you'll learn how to build your own Boot
tasks. Along the way, you'll discover that build tools can actually have
a conceptual foundation.
\switchcolumn
这是很多高层描述,希望能吸引您的注意。但是,如果我只给你上了一盘隐喻,我会感到羞愧的。不,亲爱的读者,那只是开胃菜。在本附录的其余部分,您将学习如何构建自己的Boot任务。在此过程中,您将发现构建工具实际上可以有一个概念基础。
\switchcolumn[0]*

\switchcolumn

\switchcolumn[0]*

\switchcolumn


\switchcolumn[0]*

\switchcolumn

\end{paracol}

\end{document}

