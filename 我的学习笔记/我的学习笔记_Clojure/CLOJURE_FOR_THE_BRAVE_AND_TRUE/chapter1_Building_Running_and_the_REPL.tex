% \part{Environment Setup}% \textbf{Part I}\\
% \textbf{第一部分}\\
% \textbf{环境设置}

\chapter{Building, Running, and the REPL}
% 构建、运行和REPL
\columnratio{0.55}
\begin{paracol}{2}
\switchcolumn[0]*
In this chapter, you'll invest a small amount of time up front to get
familiar with a quick, foolproof way to build and run Clojure programs.
It feels great to get a real program running. Reaching that milestone
frees you up to experiment, share your work, and gloat to your
colleagues who are still using last decade's languages. This will help
keep you motivated!
\switchcolumn
在本章中,您将花费一小部分时间来熟悉一种快速、可靠的构建和运行Clojure程序的方法。让一个真正的程序运行起来感觉很棒。达到这个里程碑将使您能够自由地进行实验,分享您的工作,并向那些仍在使用上个十年语言的同事炫耀。这将有助于保持您的动力!
\switchcolumn[0]*
You'll also learn how to instantly run code within a running Clojure
process using a \emph{Read-Eval-Print Loop (REPL)}, which allows you to
quickly test your understanding of the language and learn more
efficiently.
\switchcolumn
您还将学习如何使用\emph{Read-Eval-Print Loop (REPL)}在运行中的Clojure进程中立即运行代码,这样可以快速测试您对语言的理解并更有效地学习。
\switchcolumn[0]*
But first, I'll briefly introduce Clojure. Next, I'll cover Leiningen,
the de facto standard build tool for Clojure. By the end of the chapter,
you'll know how to do the following:
\switchcolumn
但首先,我将简要介绍Clojure。接下来,我将介绍Leiningen,Clojure的事实上的标准构建工具。在本章结束时,您将知道如何执行以下操作:

\begin{itemize}
\switchcolumn[0]*
\item Create a new Clojure project with Leiningen
\switchcolumn
\item 使用Leiningen创建一个新的Clojure项目
\switchcolumn[0]*
\item Build the project to create an executable JAR file
\switchcolumn
\item 构建项目以创建可执行的JAR文件
\switchcolumn[0]*
\item Execute the JAR file
\switchcolumn
\item 执行JAR文件
\switchcolumn[0]*
\item Execute code in a Clojure REPL
\switchcolumn
\item 在Clojure REPL中执行代码
\end{itemize}
\switchcolumn[0]*
\section{First Things First: What Is Clojure?}
\switchcolumn
\section{首先要明确的是:Clojure是什么?}
\switchcolumn[0]*
Clojure was forged in a mythic volcano by Rich Hickey. Using an alloy of
Lisp, functional programming, and a lock of his own epic hair, he
crafted a language that's delightful yet powerful. Its Lisp heritage
gives you the power to write code more expressively than is possible in
most non-Lisp languages, and its distinct take on functional programming
will sharpen your thinking as a programmer. Plus, Clojure gives you
better tools for tackling complex domains (like concurrent programming)
that are traditionally known to drive developers into years of therapy.
\switchcolumn
Clojure是由Rich Hickey在一个神话般的火山中锻造而成的。他使用了Lisp、函数式编程和他自己史诗般的头发的一缕合金,创造了一种既令人愉悦又强大的语言。它的Lisp传统使您能够以比大多数非Lisp语言更富有表现力的方式编写代码,并且它对函数式编程的独特理解将增强您作为程序员的思维能力。此外,Clojure为您提供了更好的工具来处理复杂的领域(如并发编程),而这些领域通常被认为会让开发人员陷入多年的治疗中。
\switchcolumn[0]*
When talking about Clojure, though, it's important to keep in mind the
distinction between the Clojure language and the Clojure compiler. The
Clojure language is a Lisp dialect with a functional emphasis whose
syntax and semantics are independent of any implementation. The compiler
is an executable JAR file, \emph{clojure.jar}, which takes code written
in the Clojure language and compiles it to Java Virtual Machine (JVM)
bytecode. You'll see \emph{Clojure} used to refer to both the language
and the compiler, which can be confusing if you're not aware that
they're separate things. But now that you're aware, you'll be fine.
\switchcolumn
然而,在谈论Clojure时,重要的是要区分Clojure语言和Clojure编译器之间的区别。Clojure语言是一个带有函数式强调的Lisp方言,其语法和语义与任何实现无关。编译器是一个可执行的JAR文件\emph{clojure.jar},它将用Clojure语言编写的代码编译为Java虚拟机(JVM)字节码。您会看到\emph{Clojure}一词既用于指代语言,也用于指代编译器,如果您不知道它们是分开的,这可能会让您感到困惑。但是现在您知道了,一切都会没事的。
\switchcolumn[0]*
This distinction is necessary because, unlike most programming languages
like Ruby, Python, C, and a bazillion others, Clojure is a \emph{hosted
language}. Clojure programs are executed within a JVM and rely on the
JVM for core features like threading and garbage collection. Clojure
also targets JavaScript and the Microsoft Common Language Runtime (CLR),
but this book only focuses on the JVM implementation.
\switchcolumn
之所以需要这种区分,是因为与大多数编程语言(如Ruby、Python、C等等)不同,Clojure是一种\emph{托管语言}。Clojure程序在JVM内执行,并依赖JVM提供核心功能,如线程和垃圾回收。Clojure还支持JavaScript和Microsoft Common Language Runtime (CLR),但本书只关注JVM实现。
\switchcolumn[0]*
We'll explore the relationship between Clojure and the JVM more later
on, but for now the main concepts you need to understand are these:
\switchcolumn
我们将在后面更详细地探讨Clojure与JVM之间的关系,但目前您需要了解的主要概念是:
\begin{itemize}
\switchcolumn[0]*
\item JVM processes execute Java bytecode.
\switchcolumn
\item JVM进程执行Java字节码。
\switchcolumn[0]*
\item Usually, the Java Compiler produces Java bytecode from Java source
code.
\switchcolumn
\item 通常,Java编译器会从Java源代码生成Java字节码。
\switchcolumn[0]*
\item JAR files are collections of Java bytecode.
\switchcolumn
JAR文件是Java字节码的集合。
\switchcolumn[0]*
\item Java programs are usually distributed as JAR files.
\switchcolumn
\item Java程序通常以JAR文件的形式分发。
\switchcolumn[0]*
\item The Java program \emph{clojure.jar} reads Clojure source code and
produces Java bytecode.
\switchcolumn
\item Java程序\emph{clojure.jar}读取Clojure源代码并生成Java字节码。
\switchcolumn[0]*
\item That Java bytecode is then executed by the same JVM process already
running \emph{clojure.jar}.
\switchcolumn
\item 然后,同一JVM进程中已经运行的\emph{clojure.jar}执行该Java字节码。
\end{itemize}
\switchcolumn[0]*

\switchcolumn

\switchcolumn[0]*

\switchcolumn

\switchcolumn[0]*

\switchcolumn

\switchcolumn[0]*

\switchcolumn

\switchcolumn[0]*

\switchcolumn

\switchcolumn[0]*

\switchcolumn

\switchcolumn[0]*

\switchcolumn

\switchcolumn[0]*

\switchcolumn

\switchcolumn[0]*

\switchcolumn

\switchcolumn[0]*

\switchcolumn

\switchcolumn[0]*

\switchcolumn

\switchcolumn[0]*

\switchcolumn

\switchcolumn[0]*

\switchcolumn

\switchcolumn[0]*

\switchcolumn

\switchcolumn[0]*

\switchcolumn

\switchcolumn[0]*

\switchcolumn

\switchcolumn[0]*

\switchcolumn

\switchcolumn[0]*

\switchcolumn

\switchcolumn[0]*

\switchcolumn


\switchcolumn[0]*

\switchcolumn


\switchcolumn[0]*

\switchcolumn

\switchcolumn[0]*

\switchcolumn


\switchcolumn[0]*

\switchcolumn

\switchcolumn[0]*

\switchcolumn


\switchcolumn[0]*

\switchcolumn

\switchcolumn[0]*

\switchcolumn


\switchcolumn[0]*

\switchcolumn

\switchcolumn[0]*

\switchcolumn


\switchcolumn[0]*

\switchcolumn


\switchcolumn[0]*

\switchcolumn

\switchcolumn[0]*

\switchcolumn


\switchcolumn[0]*

\switchcolumn

\switchcolumn[0]*

\switchcolumn


\switchcolumn[0]*

\switchcolumn

\switchcolumn[0]*

\switchcolumn


\switchcolumn[0]*

\switchcolumn

\end{paracol}
