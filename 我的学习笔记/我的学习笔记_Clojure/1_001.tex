\begin{parcolumns}[colwidths={1=\parcolumnsLeft}]{2}% 设置为两栏排版
\colchunk{在你的内心深处,你一直知道你注定要学习
Clojure。每次你举起键盘,对难以理解的类层次结构感到痛苦;每次你在夜晚醒着,因为一个由突变引发的
Heisenbug
而令你的亲人痛苦;每次一个竞态条件使你拔出更多的你已经越来越少的头发,你内心的某个秘密部分总是知道{\tt 肯定有更好的办法}。}
\colchunk{Deep in your \csh一音标{Innermost}{/ˈɪnərmoʊst/}%:最内部的
\ \csh一音标{being}{/ˈbiːɪŋ/ 存在}, you've always known you were \csh一音标{destined}{/ˈdɛstɪnd/ 注定的} to
learn Clojure. Every time you held your keyboard \csh一音标{aloft}{/əˈlɔːft/ 在空中,向上}, crying out in
\csh一音标{Anguish}{/ˈæŋɡwɪʃ/:痛苦} over an \csh一音标{Incomprehensible}{/ˌɪnkɑːmprɪˈhɛnsɪbəl/:难以理解的} class \csh一音标{Hierarchy}{/ˈhaɪəˌrɑːrki/:层次结构}; every time you lay
awake at night, \csh一音标{Disturbing}{/dɪˈstɜːrbɪŋ/:令人不安的} your loved ones with \csh一音标{Sobs}{/sɑːbz/:抽泣} over a
\csh一音标{Mutation}{/mjuːˈteɪʃən/:突变}-induced \csh一音标{heisenbug}{/ˈhaɪzənbʌɡ/ 海森堡虫\footnotemark[1]}; every time a \csh一音标{race condition}{/reɪs kənˈdɪʃən/ 竞态条件\footnotemark[2]} caused you to
pull out more of your \csh一音标{everdwindling}{/ˈɛvər dwɪndlɪŋ/ 不断减少的} hair, some secret part of you has
known that \emph{there has to be a better way}.
% Deep in your innermost being, you've always known you were destined to
% learn Clojure. Every time you held your keyboard aloft, crying out in
% anguish over an incomprehensible class hierarchy; every time you lay
% awake at night, disturbing your loved ones with sobs over a
% mutation-induced heisenbug; every time a race condition caused you to
% pull out more of your ever-dwindling hair, some secret part of you has
% known that \emph{there has to be a better way}.
}
\footnotetext[1]{一种在尝试调试时会改变行为的计算机bug}%todo 后面去那边文档添加
\footnotetext[2]{计算机科学中关于资源争夺的问题}
\end{parcolumns}


\begin{parcolumns}[colwidths={1=\parcolumnsLeft}]{2}% 设置为两栏排版
\colchunk{现在,终于,你面前的这个教材将会将你与你一直渴望的编程语言结合在一起。}
\colchunk{Now, at long last, the \csh一音标{Instructional}{/ɪnˈstrʌkʃənəl/}%:教学的
\ \csh一音标{Material}{/məˈtɪriəl/} %:材料,资料
you have in front of your
face will \csh一音标{Unite}{/juːˈnaɪt/:联合,结合} you with the programming language you've been \csh一音标{Longing}{/ˈlɔːŋɪŋ/:渴望}
for.}
\end{parcolumns}


\section[学习新的编程语言:穿越四个迷宫的旅程]{学习新的编程语言:穿越四个迷宫的旅程\\Learning a New Programming Language: A Journey Through the Four Labyrinths}

\begin{parcolumns}[colwidths={1=\parcolumnsLeft}]{2}% 设置为两栏排版
\colchunk{为了最大限度地掌握 Clojure,你必须克服学习新语言所面临的四个挑战:}
\colchunk{To \csh一音标{Wield}{/wiːld/:掌握,使用} Clojure to its fullest, you'll need to find your way through
the four \csh一音标{Labyrinths}{/ˈlæbərɪnθs/:迷宫} that face every programmer learning a new language:}
\end{parcolumns}

\begin{parcolumns}[colwidths={1=\parcolumnsLeft}]{2}% 设置为两栏排版
\colchunk{
\begin{description}
    \item [工具森林] 一个友好且高效的编程环境使得尝试你的想法变得简单。你将会学习如何设置你的环境。
    \item[语言之山] 当你攀登这座山时,你会学会 Clojure
的语法、语义和数据结构的知识。你将会学习如何使用最强大的编程工具之一,宏,以及如何使用
Clojure 的并发构造来简化你的生活。
\item[人工制品之洞]
在其深处,你将学习构建、运行和分发你自己的程序,以及如何使用代码库。你还将了解
Clojure 与 Java 虚拟机(JVM)的关系。
\item[心态的云堡] 在其稀薄的空气中,你将会了解 Lisp
和函数式编程的意义和方法。你将Clojure 中渗透的简洁哲学,以及如何像一个
Clojure 程序员那样解决问题。
\end{description}    
}

\colchunk{
    \begin{description}
        \item [The Forest of Tooling] A friendly and \csh一音标{Efficient}{/ɪˈfɪʃənt/:高效的} programming
environment makes it easy to try your ideas. You'll learn how to set up
your environment.
        \item[The Mountain of Language] As you \csh一音标{Ascend}{/əˈsɛnd/:攀登}, you'll \csh一音标{Gain}{/ɡeɪn/:获取} knowledge
of Clojure's syntax, \csh一音标{Semantics}{/sɪˈmæntɪks/:语义}, and data structures. You'll learn how to
use one of the \csh一音标{Mightiest}{/ˈmaɪtɪəst/:最强大的} programming tools, the macro, and learn how to
simplify your life with Clojure's \csh一音标{Concurrency}{/kənˈkɜːrənsi/:并发} constructs.
\item[The Cave of Artifacts] In its depths you'll learn to build, run,
and \csh一音标{Distribute}{/dɪˈstrɪbjuːt/:分发} your own programs, and how to use code libraries. You'll
also learn Clojure's relationship to the Java Virtual Machine (JVM).
\item[The Cloud Castle of Mindset] In its \csh一音标{Rarefied}{/ˈrɛrɪfaɪd/:稀薄的} air, you'll come to
know the why and how of Lisp and functional programming. You'll learn
about the \csh一音标{Philosophy}{/fɪˈlɑːsəfi/:哲学} of simplicity that \csh一音标{Permeates}{/ˈpɜːrmiːts/:渗透} Clojure, and how to
solve problems like a Clojurist.
    \end{description}
}
\end{parcolumns}


这段话介绍了 Clojure 编程语言独特的心态方式。在 Clojure
中,程序员应该专注于\textbf{问题的本质},而不是关注实现细节。Clojure
也强调简洁和优雅的代码风格。

\begin{parcolumns}[colwidths={1=\parcolumnsLeft}]{2}% 设置为两栏排版
\colchunk{学习 Clojure
需要付出努力,但这本书将让学习过程充满乐趣,而不是枯燥乏味。这是因为这本书遵循以下三个准则:}
\colchunk{Make no mistake, you will work. But this book will make the work feel
\csh一音标{exhilarating}{/ɪgˈzɪləreɪtɪŋ/ 令人兴奋的}, not \csh一音标{exhausting}{/ɪgˈzɔːstɪŋ/ 使人疲惫的,耗尽的}. That's because this book follows three
\csh一音标{guidelines}{/ˈɡaɪdlaɪnz/ 准则}:}
\end{parcolumns}

\begin{parcolumns}[colwidths={1=\parcolumnsLeft}]{2}% 设置为两栏排版
\colchunk{
    \begin{itemize}
        \item 它采取了``甜品先行''的方法,为你提供开发工具和语言细节,使你能\textbf{立即开始实践真实的程序}。
        \item 它假定你对JVM、函数式编程或Lisp没有任何经验。它详细介绍了这些主题,所以当你构建和运行Clojure程序时,你会对自己正在做的事情感到自信。
        \item 它避免使用现实世界的例子,而选择更有趣的练习,例如攻击霍比特人和追踪闪闪发光的吸血鬼。
    \end{itemize}
}
\colchunk{
    \begin{itemize}
        \item It takes the \csh一音标{dessert}{/dɪˈzɜːrt/ 甜点}-first \csh一音标{approach}{/əˈprəʊtʃ/ 方法,方式}, giving you the development tools
        and language details you need to start playing with real programs
        immediately.
        \item It assumes zero \csh一音标{experience}{/ɪkˈspɪəriəns/ 经验} with the JVM, functional programming, or
Lisp. It covers these topics in detail so you'll feel \csh一音标{confident}{/ˈkɒnfɪdənt/ 自信的} about
what you're doing when you build and run Clojure programs.
\item It \csh一音标{eschews}{/ɪˈʃuːz/ 避开,放弃} \emph{real-world} examples in \csh一音标{favor}{/ˈfeɪvər/ 支持,喜爱} of more interesting
exercises, like \emph{\csh一音标{assaulting}{/əˈsɔːltɪŋ/ 攻击} hobbits} and \emph{tracking \csh一音标{glittery}{/ˈɡlɪtəri/ 闪闪发光的~/ˈvæmpaɪərz/ 吸血鬼}
vampires}.
    \end{itemize}
}
\end{parcolumns}


\begin{parcolumns}[colwidths={1=\parcolumnsLeft}]{2}% 设置为两栏排版
\colchunk{到最后,你将能够使用 Clojure,一种世界上最令人兴奋和有趣的编程语言!}
\colchunk{By the end, you'll be able to use Clojure, one of the most exciting and
fun programming languages in \csh一音标{existence}{/ɪɡˈzɪstəns/ 存在,现存}!}
\end{parcolumns}

\section[本书的组织结构]{本书的组织结构\\How This Book Is Organized}

\begin{parcolumns}[colwidths={1=\parcolumnsLeft}]{2}% 设置为两栏排版
\colchunk{为了更好地引导你这个勇敢的初出茅庐的Clojure学者完成你的崇高使命,本书被分为三部分。}
\colchunk{This book is split into three parts to better guide you through your
\csh一音标{valiant}{/ˈvæliənt/} \csh一音标{quest}{~/kwɛst/ 探索}, brave \csh一音标{fledgling}{/ˈflɛdʒlɪŋ/ 初出茅庐的} Clojurist.}
\end{parcolumns}


\subsection[第一部分:环境设置]{第一部分:环境设置\hfill Environment Setup}

\begin{parcolumns}[colwidths={1=\parcolumnsLeft}]{2}% 设置为两栏排版
\colchunk{为了保持动力并有效地学习,你需要实际编写代码并构建可执行文件。这些章节将带你快速浏览你需要的工具以便轻松编写程序。这样你就可以专注于学习Clojure,而不是把时间浪费在调整环境上。}
\colchunk{To stay \csh一音标{motivated}{/ˈmoʊtɪveɪtɪd/ 有动力的} and learn \csh一音标{efficiently}{/ɪˈfɪʃəntli/ 有效地}, you need to actually write code
and build \csh一音标{executables}{/ɪɡˈzɛkjutəbəlz/ 可执行文件}. These chapters take you on a quick \csh一音标{tour}{/tʊər/ 游览,导览} of the
tools you'll need to easily write programs. That way you can focus on
learning Clojure, not \csh一音标{fiddling}{/ˈfɪdlɪŋ/ 浪费时间,摆弄} with your environment.}
\end{parcolumns}


\textbf{第一章:构建,运行和REPL \hfill Chapter 1: Building, Running, and the REPL}

\begin{parcolumns}[colwidths={1=\parcolumnsLeft}]{2}% 设置为两栏排版
\colchunk{让一个真实的程序运行起来有着某种强大和激励人的力量。一旦你能做到这一点,你就可以自由地实验,并且你实际上可以分享你的工作!}
\colchunk{There's something powerful and \csh一音标{motivating}{/ˈmoʊtɪveɪtɪŋ/ 激励人的} about getting a real program
running. Once you can do that, you're free to \csh一音标{experiment}{/ɪkˈspɛrɪmənt/ 实验}, and you can
actually share your work!}
\end{parcolumns}

\begin{parcolumns}[colwidths={1=\parcolumnsLeft}]{2}% 设置为两栏排版
\colchunk{在这个短小的章节中,你将花费一小部分时间来熟悉一种快速构建和运行Clojure程序的方法。你将学习如何在正在运行的Clojure过程中使用读取-求值-打印循环(REPL)来实验代码。这将使你的反馈循环更加紧密,并帮助你更有效地学习。}
\colchunk{In this short chapter, you'll \csh一音标{invest}{/ɪnˈvɛst/ 投入,花费} a small amount of time to become
familiar with a quick way to build and run Clojure programs. You'll
learn how to experiment with code in a running Clojure process using a
\csh一音标{read-eval-print}{/ˈriːd ˌiːvæl ˈprɪnt/ 一种简单的,交互式的编程环境} loop (REPL). This will \csh一音标{tighten}{/ˈtaɪtən/ 紧缩} your feedback loop and
help you learn more efficiently.}
\end{parcolumns}


\textbf{第二章:如何使用Emacs,一款出色的Clojure编辑器\hfill Chapter 2: How to Use Emacs, an Excellent Clojure Editor}


\begin{parcolumns}[colwidths={1=\parcolumnsLeft}]{2}% 设置为两栏排版
\colchunk{快速的反馈循环对于学习至关重要。在这一章中,我从头开始介绍Emacs,以保证你有一个高效的Emacs/Clojure工作流。}
\colchunk{A quick feedback loop is \csh一音标{crucial}{/ˈkruːʃəl/ 至关重要的} for learning. In this chapter, I cover
Emacs from the \csh一音标{ground}{/ɡraʊnd/ 基础} up to \csh一音标{guarantee}{/ˌɡærənˈtiː/ 保证} you have an efficient
Emacs/Clojure \csh一音标{workflow}{/ˈwɜːrkfloʊ/ 工作流程}.}
\end{parcolumns}

%%%% todo 以下未标音标

\subsection[第二部分:语言基础]{第二部分:语言基础\hfill Part II: Language Fundamentals}

\begin{parcolumns}[colwidths={1=\parcolumnsLeft}]{2}% 设置为两栏排版
\colchunk{这些章节为你打下了坚实的基础,以便继续学习 Clojure。你将从学习 Clojure 的基础知识(语法、语义和数据结构)开始,这样你就可以“做事情”。然后,你将退一步,深入研究 Clojure 中最常用的函数,并学习如何以“函数式编程”思维方式使用它们解决问题。}
\colchunk{These chapters give you a solid foundation on which to continue learning Clojure. You'll start by learning Clojure's basics (syntax, semantics, and data structures) so you can \emph{do things}. Then you'll take a step back to examine Clojure's most used functions in detail and learn how to solve problems with them using the \emph{functional programming} mindset.}
\end{parcolumns}

\textbf{第三章:做事情:Clojure 简明教程\hfill Chapter 3: Do Things: A Clojure Crash Course}

\begin{parcolumns}[colwidths={1=\parcolumnsLeft}]{2}% 设置为两栏排版
\colchunk{这是你真正深入学习 Clojure 的地方。在这里,你需要关上窗户,因为你会大声喊叫:“哇哦,太棒了!”直到你翻到了书的索引页。}
\colchunk{This is where you'll start to really dig into Clojure. It's also where you'll need to close your windows because you'll start shouting, ``\emph{HOLY MOLEY THAT'S SPIFFY!}'' at the top of your lungs and won't stop until you've hit this book's index. }
\end{parcolumns}

\begin{parcolumns}[colwidths={1=\parcolumnsLeft}]{2}% 设置为两栏排版
\colchunk{无疑,你已经听说过 Clojure 强大的并发支持和其他惊人的特性,但 Clojure 最显著的特点是它是一种 Lisp 语言。你将探索这个 Lisp 核心,它由两部分组成:函数和数据。}
\colchunk{You've undoubtedly heard of Clojure's awesome concurrency support and other stupendous features, but Clojure's most salient characteristic is that it is a Lisp. You'll explore this Lisp core, which is composed of two parts: functions and data.}
\end{parcolumns}

\textbf{第四章:深入核心函数\hfill Chapter 4: Core Functions in Depth}

\begin{parcolumns}[colwidths={1=\parcolumnsLeft}]{2}% 设置为两栏排版
    \colchunk{在这一章中,你将学习一些 Clojure 的基本概念。这将使你能够阅读你以前未使用过的函数的文档,并理解在你尝试使用它们时发生的情况。}
    \colchunk{In this chapter, you'll learn about a couple of Clojure's underlying concepts. This will give you the grounding you need to read the documentation for functions you haven't used before and to understand what's happening when you try them. }
\end{parcolumns}

\begin{parcolumns}[colwidths={1=\parcolumnsLeft}]{2}% 设置为两栏排版
    \colchunk{你还将看到你最常用的函数的用法示例。这将为你编写自己的代码和阅读和学习他人项目打下坚实基础。还记得我提到过追踪闪闪发光的吸血鬼吗?你将在本章中做到这一点(除非你已经在业余时间做过了)。}
    \colchunk{You'll also see usage examples of the functions you'll be reaching for the most. This will give you a solid foundation for writing your own code and for reading and learning from other people's projects. And remember how I mentioned tracking glittery vampires? You'll do that in this chapter (unless you already do it in your spare time).}
\end{parcolumns}