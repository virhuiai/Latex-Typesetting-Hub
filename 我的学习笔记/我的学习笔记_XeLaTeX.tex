\PassOptionsToPackage{no-math}{fontspec}%禁用了使用fontspec宏包中的数学字体功能。
\PassOptionsToPackage{AutoFakeBold=true,AutoFakeSlant=true}{xeCJK}%让xeCJK宏包自动产生伪粗体和伪斜体效果。

\documentclass{book}
\usepackage[heading=true
,scheme=chinese%中文方案
,fontset=none%不使用默认的字体设置
,space=auto%自动调整中英文间距
]{ctex}
\setCJKmainfont{FangZhengShuSong-GBK-1.ttf}[Path=/Users/virhuiai/hlProjects/Latex-Typesetting-Hub/font/方正/]%设置文本的中文有衬线字体
\setCJKsansfont{FangZhengHeiTi-GBK-1.ttf}[Path=/Users/virhuiai/hlProjects/Latex-Typesetting-Hub/font/方正/]%设置文本的中文无衬线字体为
\setCJKmonofont{FangZhengFangSong-GBK-1.ttf}[Path=/Users/virhuiai/hlProjects/Latex-Typesetting-Hub/font/方正/] %设置文本的中文等宽字体 

\usepackage[all]{tcolorbox}
\begin{document}

% 定义打开和关闭设置临时计数器命令的功能
\newcommand\counterSetTmpOpen{\providecommand{\counterSetTmp}{}\renewcommand\counterSetTmp[2]{\setcounter{##1}{##2}}}
\newcommand\counterSetTmpClose{\providecommand{\counterSetTmp}{}\renewcommand\counterSetTmp[2]{}}
% \counterSetTmpOpen
\counterSetTmpClose

\counterSetTmp{chapter}{6}%第7章ok v5
\chapter{列表}%-v5


\end{document}
%mypdf 我的学习笔记_XeLaTeX