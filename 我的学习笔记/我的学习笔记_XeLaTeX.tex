\PassOptionsToPackage{no-math}{fontspec}%禁用了使用fontspec宏包中的数学字体功能。
\PassOptionsToPackage{AutoFakeBold=true,AutoFakeSlant=true}{xeCJK}%让xeCJK宏包自动产生伪粗体和伪斜体效果。

\documentclass{book}
\usepackage[heading=true
,scheme=chinese%中文方案
,fontset=none%不使用默认的字体设置
,space=auto%自动调整中英文间距
]{ctex}
\setCJKmainfont{FangZhengShuSong-GBK-1.ttf}[Path=/Users/virhuiai/hlProjects/Latex-Typesetting-Hub/font/方正/]%设置文本的中文有衬线字体
\setCJKsansfont{FangZhengHeiTi-GBK-1.ttf}[Path=/Users/virhuiai/hlProjects/Latex-Typesetting-Hub/font/方正/]%设置文本的中文无衬线字体为
\setCJKmonofont{FangZhengFangSong-GBK-1.ttf}[Path=/Users/virhuiai/hlProjects/Latex-Typesetting-Hub/font/方正/] %设置文本的中文等宽字体 

\usepackage[all]{tcolorbox}
\usepackage{paralist}
\begin{document}

% 定义打开和关闭设置临时计数器命令的功能
\newcommand\counterSetTmpOpen{\providecommand{\counterSetTmp}{}\renewcommand\counterSetTmp[2]{\setcounter{##1}{##2}}}
\newcommand\counterSetTmpClose{\providecommand{\counterSetTmp}{}\renewcommand\counterSetTmp[2]{}}
\counterSetTmpOpen
% \counterSetTmpClose

\counterSetTmp{chapter}{6}%第7章ok v5
\chapter{列表}%-v5

列表就是将某一论述的内容分成若干个简短的条目,并按一定的顺序排列,以达到{\bf 简明扼要,醒目直观}的阅读效果。列表是论文写作的重要论述手段。

\LaTeX{\tiny 提供}%有3种
的标准列表环境:
\begin{compactitem}
  \item
  常规列表环境{itemize}%,{[ˈaɪtəmaɪz]逐条列记}
\item
  排序列表环境{enumerate}%,{[ɪˈnjuːməreɪt]列举}%[ɪ'njuːməreɪt] [美: ɪ'numəret] [英: ɪ'njuːməreɪt]
\item
  解说列表环境{description}%,{[dɪˈskrɪpʃn]描述}
\end{compactitem}

可使用相关的命令在全文或者局部文本中修改这3种标准列表环境的排版样式;
 
若调用paralist和mdwlist等列表宏包还可以得到具有更多排版样式的列表环境;

此外,LaTeX还提供有两个通用列表环境:
\begin{compactitem}
\item list
\item trivlist
\end{compactitem}
%latexdef -s 16KTemplateBook02.tex  list
%latexdef -s 16KTemplateBook02.tex trivlist
作者可使用它们自行创建新的列表环境或者其他用途的环境。

\end{document}
%mypdf 我的学习笔记_XeLaTeX