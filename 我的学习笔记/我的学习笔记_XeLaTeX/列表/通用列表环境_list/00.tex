\section{通用列表环境list}

% 前面已经介绍了多种列表环境,可以基本满足论文写作的需要,
如果对某个列表环境的排版样式不够满意而又难以修改,
可采用list通用列表环境,\index{environment}{列表!通用列表!list}%
上述各种列表环境都是用它创建的,其命令结构为:

\begin{minted}{latex}
\begin{list}{默认标号}{声明}
\item[标号] 条目1
\item[标号] 条目2
......
\end{list}
\end{minted}

其中各参数的说明如下。

% \begin{description}%todo优化下展示
% %{默认标号}
% \item[默认标号]%
% 在条目之前加入的标号,若条目命令~\verb|\item|~中没有给出可选参数标号的话。默认标号 可以空置也可以是一段文本,文本中可包含符号和命令。

% \item[声明]%
% 针对条目标号、条目字体和条目尺寸等列表样式的设置命令,其中对各种条目尺寸的设置要用到专用于通用列表环境的条目尺寸命令。
% \end{description}

\noindent \makebox[0pt][r]{\bf 默认标号\quad}在条目之前加入的标号,若条目命令~\verb|\item|~中没有给出可选参数标号的话。默认标号 可以空置也可以是一段文本,文本中可包含符号和命令。

\noindent \makebox[0pt][r]{\bf 声明\quad}针对条目标号、条目字体和条目尺寸等列表样式的设置命令,其中对各种条目尺寸的设置要用到专用于通用列表环境的条目尺寸命令。

