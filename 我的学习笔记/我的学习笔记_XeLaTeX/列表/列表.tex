\counterSetTmp{chapter}{6}%第7章ok v5
\chapter{列表}%-v5

列表就是将某一论述的内容分成若干个简短的条目,并按一定的顺序排列,以达到{\bf 简明扼要,醒目直观}的阅读效果。列表是论文写作的重要论述手段。

\LaTeX{\tiny 提供}%有3种
的标准列表环境:
\begin{compactitem}
  \item
  常规列表环境{itemize}%,{[ˈaɪtəmaɪz]逐条列记}
\item
  排序列表环境{enumerate}%,{[ɪˈnjuːməreɪt]列举}%[ɪ'njuːməreɪt] [美: ɪ'numəret] [英: ɪ'njuːməreɪt]
\item
  解说列表环境{description}%,{[dɪˈskrɪpʃn]描述}
\end{compactitem}

可使用相关的命令在全文或者局部文本中修改这3种标准列表环境的排版样式;
 
若调用paralist和mdwlist等列表宏包还可以得到具有更多排版样式的列表环境;

此外,LaTeX还提供有两个通用列表环境:
\begin{compactitem}
\item list
\item trivlist
\end{compactitem}
%latexdef -s 16KTemplateBook02.tex  list
%latexdef -s 16KTemplateBook02.tex trivlist
作者可使用它们自行创建新的列表环境或者其他用途的环境。