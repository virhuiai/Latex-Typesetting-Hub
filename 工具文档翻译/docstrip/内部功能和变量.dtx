% \section{Internal functions and variables\\内部功能和变量}
%
%    An important consideration for \LaTeX\ development is separating
%    out public and internal functions. Functions and variables which
%    are private to one module should not be used or modified by any
%    other module. As \TeX{} does not have any formal namespacing
%    system, this requires a convention for indicating which functions
%    in a code-level module are public and which are private. \\对于\LaTeX\ 开发来说,一个重要的考虑因素是分离公共和内部函数。私有于一个模块的函数和变量不应该被其他模块使用或修改。由于\TeX{}没有任何正式的命名空间系统,这需要一种约定来指示哪些代码级模块中的函数是公共的,哪些是私有的。
%
%    Using \ds\ allows internal functions to be indicated
%    using a `two part' system. Within the \texttt{.dtx} file,
%    internal functions may be indicated using |@@| in place of the
%    module name, for example \\使用 \ds\ 可以使用“两部分”系统指示内部函数。在 \texttt{.dtx} 文件中,可以使用 |@@| 代替模块名称来指示内部函数,例如
% \begin{verbatim}
%    \cs_new_protected:Npn \@@_some_function:nn #1#2
%      {
%        % Some code here
%      }
%    \tl_new:N \l_@@_internal_tl
% \end{verbatim}
%
%    To extract the code using \ds, the original `guard'
%    mechanism is extended by the introduction of the syntax
%    \texttt{\%<@@=\meta{module}>}. The \meta{module} name then
%    replaces the |@@| when the code is extracted, so that\\使用\ds 提取代码时,原始的“守卫”机制通过引入语法\texttt{\%<@@=\meta{module}>}得到了扩展。当代码被提取时,\meta{module}名称将替换|@@|。

% \begin{verbatim}
%   %<*package>
%   %<@@=foo>
%   \cs_new_protected:Npn \@@_some_function:nn #1#2
%      {
%        % Some code here
%      }
%   \tl_new:N \l_@@_internal_tl
%   %</package>
% \end{verbatim}
%    is extracted as\\被提取为
% \begin{verbatim}
%   \cs_new_protected:Npn \__foo_some_function:nn #1#2
%      {
%        % Some code here
%      }
%   \tl_new:N \l__foo_internal_tl
% \end{verbatim}
%    where the |__| indicates that the functions and variables are
%    internal to the \texttt{foo} module.\\其中 |__| 表示函数和变量是 \texttt{foo} 模块的内部内容。
%
%    Use |@@@@| to obtain |@@| in the output (|@@@@@| to get |@@@|).
%    For longer pieces of code the replacement can be completely
%    suppressed by giving an empty module name, namely using the
%    syntax \texttt{\%<@@=>}.\\使用|@@@@|来获得输出中的|@@|(使用|@@@@@|来获取|@@@|)。 对于较长的代码片段,可以通过使用空的模块名来完全禁止替换,即使用\texttt{\%<@@=>}语法。

