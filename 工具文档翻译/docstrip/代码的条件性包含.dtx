% \section{Conditional inclusion of code\\代码的条件性包含}
%
%    When you use the \ds{} program to strip comments out of
%    \TeX\ macro files you have the possibility to make more than one
%    stripped macro file from one documented file. This is achieved by
%    the support for optional code. The optional code is marked
%    in the documented file with a `guard'.\\当使用\ds{}程序从\TeX{}宏文件中去除注释时,您可以从一个文档文件生成多个去除注释的宏文件。这是通过可选代码的支持实现的。可选代码在文档文件中用“保护”标记标记。
%
%    A guard is a boolean expression that is enclosed in |<| and |>|.
%    It also {\em has\/} to follow the |%| at the beginning of the line.
%    For example:\\守卫是一个布尔表达式,被包含在 |<| 和 |>| 中。 它还必须在行首跟随 |%|。 例如:
%\begin{verbatim}
%    ...
%    %<bool>\TeX code
%    ...
%\end{verbatim}
%    In this example the line of code will be included in \meta{output}
%    if the option \texttt{bool} is present in the \meta{optionlist} of
%    the |\generateFile| command.\\在这个例子中,如果|\generateFile|命令的\meta{optionlist}中包含\texttt{bool}选项,那么这行代码将会被包含在\meta{output}中。
%
%    The syntax for the boolean expressions is:\\布尔表达式的语法为:
%
%\DeleteShortVerb\|
%    \begin{tabular}{lcl}
%    \meta{Expression} & $::=$ & \meta{Secondary}
%                            [\{\texttt{|}, \texttt{,}\}
%                            \meta{Secondary}]*\\
%    \meta{Secondary}    & $::=$ &
%                        \meta{Primary} [\texttt{\&}
%                         \meta{Primary}]*\\
%    \meta{Primary}  & $::=$ &
%                        \meta{Terminal} $|$ \texttt{!}\meta{Primary}
%                        $|$ \texttt{(}\meta{Expression}\texttt{)}\\
%    \end{tabular}
%
%    The \texttt{|} stands for disjunction, the \texttt{\&} stands for
%    conjunction and the \texttt{!}\ stands for negation. The
%    \meta{Terminal} is any sequence of letters and evaluates to
%    \meta{true} iff\footnote{iff stands for `if and only if'} it
%    occurs in the list of options that have to be included.\\\texttt{|} 代表析取,\texttt{\&} 代表合取,\texttt{!} 代表否定。 \meta{Terminal} 是任何字母序列,并在它出现在必须包含的选项列表中时评估为 \meta{true}。其中,\texttt{iff} 代表“当且仅当”。

%\MakeShortVerb\|
%
%    Two kinds of optional code are supported: one can either have
%    optional code that `fits' on one line of text, like the example
%    above, or one can have blocks of optional code.\\支持两种可选代码:一种是可以在一行文本上“适合”的可选代码,就像上面的例子一样,或者可以有一块可选代码。
%
%    To distinguish both kinds of optional code the `guard modifier'
%    has been introduced. The `guard modifier' is one character that
%    immediately follows the |<| of the guard. It can be either |*|
%    for the beginning of a block of code, or |/| for the end of a
%    block of code\footnote{To be compatible with the earlier version
%    of \ds{} also \texttt{+} and \texttt{-} are supported as `guard
%    modifiers'.  However, there is an incompatibility with the
%    earlier version since a line with a \texttt{+}-modified guard is
%    not included inside a block with a guard that evaluates to false,
%    in contrast to the previous behaviour.}.  The beginning and
%    ending guards for a block of code have to be on a line by
%    themselves.\\为了区分这两种可选代码,引入了“guard修饰符”。“guard修饰符”是紧跟在保护符的“|<|”后面的一个字符。它可以是“|*|”表示代码块的开始,也可以是“|/|”表示代码块的结束\footnote{为了与早期版本的\ds{}兼容,也支持“+”和“-”作为“guard修饰符”。然而,与早期版本不兼容的是,带有“+”修饰符的保护符的行不会被包含在一个求值为false的保护块中,与之前的行为相反。}。代码块的开始和结束保护符必须独占一行。
%
%    When a block of code is {\em not\/} included, any guards that occur
%    within that block are {\em not\/} evaluated.\\当一个代码块被排除时,该块中出现的任何保护条件都不会被评估。