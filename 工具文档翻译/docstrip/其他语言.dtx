% \section{Those other languages\\其他语言}
%    Since \TeX\ is an open system some of \TeX\ packages include
%    non-\TeX\ files. Some authors use \ds\ to generate PostScript
%    headers, shell scripts or programs in other languages. For them
%    the comments-stripping activity of \ds\ may cause some
%    trouble. This section describes how to produce non-\TeX\ files
%    with \ds\ effectively.\\由于 \TeX\ 是一个开放系统,一些 \TeX\ 包中包含了非 \TeX\ 文件。一些作者使用 \ds\ 生成 PostScript 头文件、Shell 脚本或其他语言的程序。对于他们来说,\ds\ 中的注释去除可能会造成一些麻烦。本节介绍如何有效地使用 \ds\ 生成非 \TeX\ 文件。

%
% \subsection{Stuff \ds\ puts in every file\\\ds\ 在每个文件中添加的内容}
%    First problem when producing files in ``other'' languages is that
%    \ds\ adds some bits to the beginning and end of every generated
%    file that may not fit with the syntax of the language in
%    question. So we'll study carefully what exactly goes where.\\在生成“其他”语言的文件时,第一个问题是\ds\ 会在每个生成的文件的开头和结尾添加一些内容,这些内容可能与使用的语言的语法不相符。因此,我们需要仔细研究这些内容的确切位置和含义。
%
%    The whole text put on beginning of file is kept in a macro defined
%    by |\declarepreamble|. Every line of input presented to
%    |\declarepreamble| is prepended with current value of
%    |\MetaPrefix|. Standard \ds\ header is inserted before your text,
%    and macros |\inFileName|, |\outFileName| and |\ReferenceLines|
%    are used as placeholders for information which will be filled in
%    later (specifically for each output file). Don't try to redefine
%    these macros. After\\整个文件开头放置的文本被保存在由 |\declarepreamble| 定义的宏中。每输入一行到 |\declarepreamble|,都会在该行前加上当前 |\MetaPrefix| 的值。标准的 \ds\ 头将在你的文本之前插入,并且宏 |\inFileName|、|\outFileName| 和 |\ReferenceLines| 被用作占位符,稍后会填充信息(具体针对每个输出文件)。不要试图重新定义这些宏。在此之后,
%\begin{verbatim}
%    \declarepreamble\foo
%    ____________________________
%    Package FOO for use with TeX
%    \endpreamble
%\end{verbatim}
%    macro |\foo| is defined as\\宏 |\foo| 被定义为
%\begin{verbatim}
%    %%^^J
%    %% This is file `\outFileName ',^^J
%    %% generated with the docstrip utility.^^J
%    \ReferenceLines^^J
%    %% ____________________________^^J
%    %% Package FOO for use with TeX.
%\end{verbatim}
%    You can play with it freely or even define it from scratch. To
%    embed the preamble in Adobe structured comments just use |\edef|:\\你可以自由地玩耍,甚至可以从零开始定义它。要将导言嵌入Adobe结构化注释中,只需使用|\edef|:

%\begin{verbatim}
%    \edef\foo{\perCent!PS-Adobe-3.0^^J%
%              \DoubleperCent\space Title: \outFileName^^J%
%              \foo^^J%
%              \DoubleperCent\space EndComments}
%\end{verbatim}
%    After that use |\usepreamble\foo| to select your new preamble.
%    Everything above works as well for postambles.\\然后使用 |\usepreamble\foo| 来选择你的新导言部分。以上所有内容同样适用于后导言部分。
%
%    You may also prevent \ds\ from adding anything to your file, and
%    put any language specific invocations directly in your code:\\您也可以防止 \ds\ 向您的文件中添加任何内容,并直接在代码中放置任何特定于语言的调用:
%\begin{verbatim}
%    \generate{\usepreamble\empty
%              \usepostamble\empty
%              \file{foo.ps}{\from{mypackage.dtx}{ps}}}
%\end{verbatim}
%    or alternatively |\nopreamble| and |\nopostamble|.\\或者用|\nopreamble|和|\nopostamble|代替。
%
% \subsection{Meta comments\\元注释}
%    You can change the prefix used for putting meta comments to
%    output files by redefining |\MetaPrefix|. Its default value is
%    |\DoubleperCent|. The preamble uses value of |\MetaPrefix|
%    current at time of |\declarepreamble| while meta comments in the
%    source file use value current at time of |\generate|. Note that
%    this means that you cannot produce concurrently two files using
%    different |\MetaPrefix|es.\\您可以通过重新定义|\MetaPrefix|来更改用于将元注释放入输出文件中的前缀。其默认值为|\DoubleperCent|。导言部分使用|\declarepreamble|时的|\MetaPrefix|值,而源文件中的元注释使用|\generate|时的当前值。请注意,这意味着您无法同时使用不同的|\MetaPrefix|生成两个文件。
%
% \subsection{Verbatim mode\\逐字模式}
%    If your programming language uses some construct that can
%    interfere badly with \ds\ (e.g., percent in column one) you may
%    need a way for preventing it from being stripped off. For that
%    purpose \ds\ features `verbatim mode'.\\如果您的编程语言使用一些可能会严重干扰 \ds\ 的结构(例如第一列中的百分号),则您可能需要一种防止其被剥离的方法。为此,\ds\ 提供了“逐字模式”功能。
%
%    A `Guard expression' of the form |%<<|\meta{END-TAG} marks
%    the start of a section that will be copied verbatim upto a line
%    containing only a percent in column 1 followed by \meta{END-TAG}.
%    You can select any \meta{END-TAG} you want, but note that spaces
%     count here. Example:\\形如 |%<<|\meta{END-TAG} 的 `Guard expression' 标志着一个节的开始,该节将被原样复制,直到第一列只包含一个百分号以及紧随其后的 \meta{END-TAG} 的行。您可以选择任何您想要的 \meta{END-TAG},但请注意空格的计数。例如:
%\begin{verbatim}
%    %<*myblock>
%    some stupid()
%       #computer<program>
%    %<<COMMENT
%    % These two lines are copied verbatim (including percents
%    %% even if \MetaPrefix is something different than %%).
%    %COMMENT
%       using*strange@programming<language>
%    %</myblock>
%\end{verbatim}
%    And the output is (when stripped with \texttt{myblock} defined):\\当使用定义的 \texttt{myblock} 去除空格后,输出结果为:
%\begin{verbatim}
%    some stupid()
%       #computer<program>
%    % These two lines are copied verbatim (including percents
%    %% even if \MetaPrefix is something different than %%).
%       using*strange@programming<language>
%\end{verbatim}