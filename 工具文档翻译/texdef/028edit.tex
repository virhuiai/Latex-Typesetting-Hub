\begin{docKeys}[
    doc no index,   %  no index entries for this example
    doc parameter = {~},
    ]
    {
    {
        doc name        = --edit,
        doc description = ,
    }
    }
    
    Opens the file holding the macro definition. Uses \verb|--Find| and \verb|--source|.
    If the source definition can not be found the definition is printed as normal instead. (L)
    
    打开保存宏定义的文件。使用 \verb|--Find| 和 \verb|--source|。 如果无法找到源定义,则正常打印定义。(L)

% \begin{commandshell} 
% latexdef -c '[a4paper]{book}' -v paperwidth
% latexdef -c '[letter]{book}'  -v paperwidth
% \end{commandshell}  
\end{docKeys}

\begin{docKeys}[
    doc no index,   %  no index entries for this example
    doc parameter = {~\meta{editor}},
    ]
    {
    {
        doc name        = --editor,
        doc description = ,
    }
    }
    
Can be used to set the used editor. If not used the environment variables \verb|TEXDEF_EDITOR|, \verb|EDITOR| and
\verb|SELECTED_EDITOR| are read in this order. If none of these are set a list of default
editors are tried.  The \verb|<editor>| string can include `\verb|%f|' for the filename, `\verb|%n|' for
the line number and `\verb|%%|' for a literal `\verb|%|'.  If no `\verb|%|' is used `\verb|+%n %f|' is added to
the given command.

可用于设置使用的编辑器。如果未使用,将按照此顺序读取环境变量\verb|TEXDEF_EDITOR|,\verb|EDITOR|和 \verb|SELECTED_EDITOR|。如果这些都未设置,则尝试使用默认编辑器列表。 \verb|<editor>|字符串可以包括`\verb|%f|'表示文件名,`\verb|%n|'表示行号,`\verb|%%|'表示文字`\verb|%|'。如果未使用`\verb|%|',则给定命令中添加`\verb|+%n %f|'。

% \begin{commandshell} 
% latexdef -c '[a4paper]{book}' -v paperwidth
% latexdef -c '[letter]{book}'  -v paperwidth
% \end{commandshell}  
\end{docKeys}
    
     